
\documentclass{article}

% Recommended, but optional, packages for figures and better typesetting:
\usepackage{microtype}
\usepackage{graphicx}
\usepackage{subfigure}
\usepackage{booktabs} % for professional tables
\usepackage{enumitem} % for customizing enumerate environments


% hyperref makes hyperlinks in the resulting PDF.
% If your build breaks (sometimes temporarily if a hyperlink spans a page)
% please comment out the following usepackage line and replace
\usepackage{hyperref}


% Attempt to make hyperref and algorithmic work together better:
\newcommand{\theHalgorithm}{\arabic{algorithm}}

% Use the following line for the initial blind version submitted for review:

% If accepted, instead use the following line for the camera-ready submission:
\usepackage{preprint}

% For theorems and such
\usepackage{amsmath, amssymb, mathtools, amsthm, bm}

% if you use cleveref..
%\usepackage[capitalize,noabbrev]{cleveref}

%%%%%%%%%%%%%%%%%%%%%%%%%%%%%%%%
% THEOREMS
%%%%%%%%%%%%%%%%%%%%%%%%%%%%%%%%
\theoremstyle{plain}
\newtheorem{theorem}{Theorem}[section]
\newtheorem{proposition}[theorem]{Proposition}
\newtheorem{lemma}[theorem]{Lemma}
\newtheorem{corollary}[theorem]{Corollary}
\theoremstyle{definition}
\newtheorem{definition}[theorem]{Definition}
\newtheorem{assumption}[theorem]{Assumption}
\theoremstyle{remark}
\newtheorem{remark}[theorem]{Remark}

% Todonotes is useful during development; simply uncomment the next line
%    and comment out the line below the next line to turn off comments
%\usepackage[disable,textsize=tiny]{todonotes}
%\usepackage[textsize=tiny]{todonotes}


% The \papertitle you define below is probably too long as a header.
% Therefore, a short form for the running title is supplied here:
\titlerunning{Scalable First-order Method for Certifying Optimal k-Sparse GLMs}


%%%%% NEW MATH DEFINITIONS %%%%%

% In approved list
% https://authors.acm.org/proceedings/production-information/accepted-latex-packages
% \usepackage{amsthm}
% \usepackage{amsfonts}
% \usepackage{mathtools}
% \usepackage{soul}
% \usepackage{bbm}

% Removed to avoid \Bbbk already defined error
% \usepackage{amssymb}

\usepackage{bm}
% \usepackage{multirow}
% \usepackage{wrapfig}
% \usepackage{rotating}
% \usepackage{colortbl}
%\usepackage{hyperref}


% \usepackage{amsmath}
% \usepackage{booktabs} 
% \usepackage{caption}
% \usepackage{tabularx}
% \usepackage{graphicx}

% Disallowed
% \usepackage{sidecap}
% \usepackage{multicol}
% \usepackage{numprint}
% \usepackage{mathrsfs}

\usepackage{multirow}

%Tables
\usepackage{rotating}
% \usepackage{array,tabularx,multirow,hhline,booktabs,colortbl}
% \usepackage[table]{xcolor}
%Hyperlinks
% \usepackage{xcolor, hyperref}
% \usepackage{xcolor, hyperref}

\newcommand{\modelname}{our model }


% \usepackage{natbib} 
% \setlength{\bibsep}{0.0pt}

% \newtheorem{theorem}{Theorem}[section]
% \newtheorem{lemma}[theorem]{Lemma}
% \newtheorem{corollary}[theorem]{Corollary}
% \newtheorem*{remark}{Remark}

\newtheoremstyle{named}{}{}{\itshape}{}{\bfseries}{.}{.5em}{#1 \thmnote{#3}}
\theoremstyle{named}
\newtheorem*{namedtheorem}{Theorem}
\newtheorem*{namedlemma}{Lemma}

\newenvironment{nalign}{
    \begin{equation}
    \begin{aligned}
}{
    \end{aligned}
    \end{equation}
    %%%\ignorespacesafterend
}


%%% macros
%text in equal
\newcommand{\diag}{\text{diag}}
\newcommand{\VI}{\text{VI}}
\newcommand{\MI}{\mathrm{I}}
\newcommand{\CLUB}{\MI_{\text{CLUB}}}
\newcommand{\vCLUB}{\MI_{\text{vCLUB}}}
\newcommand{\loout}{L$\bm{1}$Out}
%%--------------------

\newcommand{\tabincell}[2]{\begin{tabular}{@{} #1 @{}} #2 \end{tabular}}
\newcommand{\e}[1]{\times 10^{#1}}
\newcommand{\scrL}{\mathscr{L}}
\newcommand{\scrP}{\mathscr{P}}

\newcommand{\Enpy}{\text{H}}

\newcommand{\ba}{\bm{a}}
\newcommand{\bb}{\bm{b}}
\newcommand{\bc}{\bm{c}}
\newcommand{\bd}{\bm{d}}
\newcommand{\bE}{\bm{E}}
\newcommand{\bF}{\bm{F}}
\newcommand{\bq}{\bm{q}}
\newcommand{\bp}{\bm{p}}
\newcommand{\bu}{\bm{u}}
\newcommand{\bw}{\bm{w}}
\newcommand{\bv}{\bm{v}}
\newcommand{\bx}{\bm{x}}
\newcommand{\bg}{\bm{g}}
\newcommand{\bX}{\bm{X}}
\newcommand{\bQ}{\bm{Q}}
\newcommand{\bQlambda}{\bm{Q}_{\lambda}}
\newcommand{\bz}{\bm{z}}
\newcommand{\bs}{\bm{s}}
\newcommand{\bZ}{\bm{Z}}
\newcommand{\by}{\bm{y}}
\newcommand{\be}{\bm{e}}
\newcommand{\bl}{\bm{l}}
\newcommand{\bt}{\bm{t}}
\newcommand{\balpha}{\bm{\alpha}}
\newcommand{\btheta}{\bm{\theta}}

\newcommand{\btq}{\bm{\tilde q}}
\newcommand{\br}{\bm{r}}
\newcommand{\bI}{\bm{I}}
\newcommand{\bA}{\bm{A}}
\newcommand{\bB}{\bm{B}}
\newcommand{\btz}{\bm{\tilde z}}
\newcommand{\bdelta}{\bm{\delta}}
\newcommand{\bbeta}{\bm{\beta}}
\newcommand{\bBeta}{\bm{\Beta}}
\newcommand{\brho}{\bm{\rho}}
\newcommand{\Dbeta}{\Delta\beta}
\newcommand{\bEta}{\bm{\eta}}
\newcommand{\bgamma}{\bm{\gamma}}
\newcommand{\hatbgamma}{\hat{\bm{\gamma}}}
\newcommand{\bGamma}{\bm{\Gamma}}
\newcommand{\bmu}{\bm{\mu}}
\newcommand{\bnu}{\bm{\nu}}
\newcommand{\bpi}{\bm{\pi}}
\newcommand{\bTheta}{\bm{\Theta}}
\newcommand{\bSigma}{\bm{\Sigma}}
\newcommand{\bXi}{\bm{\Xi}}
\newcommand{\bzeta}{\bm{\zeta}}
\newcommand{\bpsi}{\bm{\psi}}
\newcommand{\bphi}{\bm{\phi}}
\newcommand{\bxi}{\bm{\xi}}

\newcommand{\tf}{{\tilde f}}
\newcommand{\tq}{{\tilde q}}
\newcommand{\tx}{{\tilde x}}
\newcommand{\ty}{{\tilde y}}
\newcommand{\tz}{{\tilde z}}
\newcommand{\txi}{{\tilde\xi}}
\newcommand{\tGamma}{{\tilde\Gamma}}
\newcommand{\tmu}{{\tilde\mu}}
\newcommand{\tnu}{{\tilde\nu}}

\newcommand{\hq}{{\hat q}}
\newcommand{\bhq}{\bm{\hat q}}

\newcommand{\bbR}{\mathbb{R}}
\newcommand{\bbS}{\mathbb{S}}
\newcommand{\bbE}{\mathbb{E}}
\newcommand{\bbZ}{\mathbb{Z}}
\newcommand{\bbP}{\mathbb{P}}
\newcommand{\bbM}{\mathbb{M}}
\newcommand{\bbN}{\mathbb{N}}
\newcommand{\bbW}{\mathbb{W}}

\newcommand{\calA}{\mathcal{A}}
\newcommand{\calB}{\mathcal{B}}
\newcommand{\calC}{\mathcal{C}}
\newcommand{\calD}{\mathcal{D}}
\newcommand{\calE}{\mathcal{E}}
\newcommand{\calF}{\mathcal{F}}
\newcommand{\calG}{\mathcal{G}}
\newcommand{\calH}{\mathcal{H}}
\newcommand{\calI}{\mathcal{I}}
\newcommand{\calJ}{\mathcal{J}}
\newcommand{\calK}{\mathcal{K}}
\newcommand{\calL}{\mathcal{L}}
\newcommand{\calM}{\mathcal{M}}
\newcommand{\calN}{\mathcal{N}}
\newcommand{\calO}{\mathcal{O}}
\newcommand{\calP}{\mathcal{P}}
\newcommand{\calQ}{\mathcal{Q}}
\newcommand{\calR}{\mathcal{R}}
\newcommand{\calS}{\mathcal{S}}
\newcommand{\calT}{\mathcal{T}}
\newcommand{\calV}{\mathcal{V}}
\newcommand{\calW}{\mathcal{W}}
\newcommand{\calX}{\mathcal{X}}
\newcommand{\calY}{\mathcal{Y}}
\newcommand{\calZ}{\mathcal{Z}}
\newcommand{\tcalZ}{{\tilde{\mathcal{Z}}}}
\newcommand{\tcalM}{\tilde{\calM}}


\newcommand{\ud}{\mathrm{d}}
\newcommand{\dd}{\mathop{}\,\mathrm{d}}
\newcommand{\Bern}{\mathrm{Bern}}
\newcommand{\dir}{\mathrm{Dir}}
\renewcommand{\div}{\mathrm{div}}
\newcommand{\Exp}{\mathrm{Exp}}
\newcommand{\grad}{\mathrm{grad}\,}
\newcommand{\gradn}{\mathrm{grad}}
\newcommand{\jac}{\mathrm{Jac}\,}

\newcommand{\id}{\mathrm{id}}
\newcommand{\lgpp}{\mathrm{log-perp}}
\newcommand{\prob}{\mathrm{Prob}}
\newcommand{\tr}{\mathrm{tr}}
\newcommand{\vmf}{\mathrm{vMF}}
\newcommand{\med}{\mathrm{med}}
\newcommand{\const}{\mathrm{const}}


\newcommand{\frX}{\mathfrak{X}}

\newcommand*{\rfrac}[2]{{}^{#1}\!/_{\!#2}}
\newcommand{\trs}{^{\top}}
\newcommand{\pdf}{p.d.f.~}%{\textit{pdf}}
\newcommand{\wrt}{w.r.t.~}%{\textit{wrt}}
%\newcommand{\trs}{^{\mathrm{T}}}
\newcommand{\asn}{\leftarrow}
\newcommand{\defas}{:=}

% \DeclareMathOperator*{\argmax}{arg\,max}
\DeclareMathOperator*{\argmax}{\arg\!\max}
% \DeclareMathOperator*{\argmin}{arg\,min}
\DeclareMathOperator*{\argmin}{\arg\!\min}



% Mark sections of captions for referring to divisions of figures
\newcommand{\figleft}{{\em (Left)}}
\newcommand{\figcenter}{{\em (Center)}}
\newcommand{\figright}{{\em (Right)}}
\newcommand{\figtop}{{\em (Top)}}
\newcommand{\figbottom}{{\em (Bottom)}}
\newcommand{\captiona}{{\em (a)}}
\newcommand{\captionb}{{\em (b)}}
\newcommand{\captionc}{{\em (c)}}
\newcommand{\captiond}{{\em (d)}}

% Highlight a newly defined term
\newcommand{\newterm}[1]{{\bf #1}}


% Figure reference, lower-case.
\def\figref#1{figure~\ref{#1}}
% Figure reference, capital. For start of sentence
\def\Figref#1{Figure~\ref{#1}}
\def\twofigref#1#2{figures \ref{#1} and \ref{#2}}
\def\quadfigref#1#2#3#4{figures \ref{#1}, \ref{#2}, \ref{#3} and \ref{#4}}
% Section reference, lower-case.
\def\secref#1{section~\ref{#1}}
% Section reference, capital.
\def\Secref#1{Section~\ref{#1}}
% Reference to two sections.
\def\twosecrefs#1#2{sections \ref{#1} and \ref{#2}}
% Reference to three sections.
\def\secrefs#1#2#3{sections \ref{#1}, \ref{#2} and \ref{#3}}
% Reference to an equation, lower-case.
% \def\eqref#1{equation~\ref{#1}}
% Reference to an equation, upper case
% \def\Eqref#1{Equation~\ref{#1}}
% A raw reference to an equation---avoid using if possible
\def\plaineqref#1{\ref{#1}}
% Reference to a chapter, lower-case.
\def\chapref#1{chapter~\ref{#1}}
% Reference to an equation, upper case.
\def\Chapref#1{Chapter~\ref{#1}}
% Reference to a range of chapters
\def\rangechapref#1#2{chapters\ref{#1}--\ref{#2}}
% Reference to an algorithm, lower-case.
\def\algref#1{algorithm~\ref{#1}}
% Reference to an algorithm, upper case.
\def\Algref#1{Algorithm~\ref{#1}}
\def\twoalgref#1#2{algorithms \ref{#1} and \ref{#2}}
\def\Twoalgref#1#2{Algorithms \ref{#1} and \ref{#2}}
% Reference to a part, lower case
\def\partref#1{part~\ref{#1}}
% Reference to a part, upper case
\def\Partref#1{Part~\ref{#1}}
\def\twopartref#1#2{parts \ref{#1} and \ref{#2}}

\def\ceil#1{\lceil #1 \rceil}
\def\floor#1{\lfloor #1 \rfloor}
\def\1{\bm{1}}
\newcommand{\train}{\mathcal{D}}
\newcommand{\valid}{\mathcal{D_{\mathrm{valid}}}}
\newcommand{\test}{\mathcal{D_{\mathrm{test}}}}

\def\eps{{\epsilon}}


% Random variables
\def\reta{{\textnormal{$\eta$}}}
% \def\ra{{\textnormal{a}}}
\def\rb{{\textnormal{b}}}
\def\rc{{\textnormal{c}}}
\def\rd{{\textnormal{d}}}
\def\re{{\textnormal{e}}}
\def\rf{{\textnormal{f}}}
\def\rg{{\textnormal{g}}}
\def\rh{{\textnormal{h}}}
\def\ri{{\textnormal{i}}}
\def\rj{{\textnormal{j}}}
\def\rk{{\textnormal{k}}}
\def\rl{{\textnormal{l}}}
% rm is already a command, just don't name any random variables m
\def\rn{{\textnormal{n}}}
\def\ro{{\textnormal{o}}}
\def\rp{{\textnormal{p}}}
\def\rq{{\textnormal{q}}}
\def\rr{{\textnormal{r}}}
\def\rs{{\textnormal{s}}}
\def\rt{{\textnormal{t}}}
\def\ru{{\textnormal{u}}}
\def\rv{{\textnormal{v}}}
\def\rw{{\textnormal{w}}}
\def\rx{{\textnormal{x}}}
\def\ry{{\textnormal{y}}}
\def\rz{{\textnormal{z}}}

% Random vectors
\def\rvepsilon{{\mathbf{\epsilon}}}
\def\rvtheta{{\mathbf{\theta}}}
\def\rva{{\mathbf{a}}}
\def\rvb{{\mathbf{b}}}
\def\rvc{{\mathbf{c}}}
\def\rvd{{\mathbf{d}}}
\def\rve{{\mathbf{e}}}
\def\rvf{{\mathbf{f}}}
\def\rvg{{\mathbf{g}}}
\def\rvh{{\mathbf{h}}}
\def\rvi{{\mathbf{i}}}
\def\rvI{{\mathbf{I}}}
\def\rvj{{\mathbf{j}}}
\def\rvk{{\mathbf{k}}}
\def\rvl{{\mathbf{l}}}
\def\rvm{{\mathbf{m}}}
\def\rvn{{\mathbf{n}}}
\def\rvo{{\mathbf{o}}}
\def\rvp{{\mathbf{p}}}
\def\rvq{{\mathbf{q}}}
\def\rvr{{\mathbf{r}}}
\def\rvs{{\mathbf{s}}}
\def\rvt{{\mathbf{t}}}
\def\rvu{{\mathbf{u}}}
\def\rvv{{\mathbf{v}}}
\def\rvw{{\mathbf{w}}}
\def\rvx{{\mathbf{x}}}
\def\rvy{{\mathbf{y}}}
\def\rvz{{\mathbf{z}}}

% Elements of random vectors
\def\erva{{\textnormal{a}}}
\def\ervb{{\textnormal{b}}}
\def\ervc{{\textnormal{c}}}
\def\ervd{{\textnormal{d}}}
\def\erve{{\textnormal{e}}}
\def\ervf{{\textnormal{f}}}
\def\ervg{{\textnormal{g}}}
\def\ervh{{\textnormal{h}}}
\def\ervi{{\textnormal{i}}}
\def\ervj{{\textnormal{j}}}
\def\ervk{{\textnormal{k}}}
\def\ervl{{\textnormal{l}}}
\def\ervm{{\textnormal{m}}}
\def\ervn{{\textnormal{n}}}
\def\ervo{{\textnormal{o}}}
\def\ervp{{\textnormal{p}}}
\def\ervq{{\textnormal{q}}}
\def\ervr{{\textnormal{r}}}
\def\ervs{{\textnormal{s}}}
\def\ervt{{\textnormal{t}}}
\def\ervu{{\textnormal{u}}}
\def\ervv{{\textnormal{v}}}
\def\ervw{{\textnormal{w}}}
\def\ervx{{\textnormal{x}}}
\def\ervy{{\textnormal{y}}}
\def\ervz{{\textnormal{z}}}

% Random matrices
\def\rmA{{\mathbf{A}}}
\def\rmB{{\mathbf{B}}}
\def\rmC{{\mathbf{C}}}
\def\rmD{{\mathbf{D}}}
\def\rmE{{\mathbf{E}}}
\def\rmF{{\mathbf{F}}}
\def\rmG{{\mathbf{G}}}
\def\rmH{{\mathbf{H}}}
\def\rmI{{\mathbf{I}}}
\def\rmJ{{\mathbf{J}}}
\def\rmK{{\mathbf{K}}}
\def\rmL{{\mathbf{L}}}
\def\rmM{{\mathbf{M}}}
\def\rmN{{\mathbf{N}}}
\def\rmO{{\mathbf{O}}}
\def\rmP{{\mathbf{P}}}
\def\rmQ{{\mathbf{Q}}}
\def\rmR{{\mathbf{R}}}
\def\rmS{{\mathbf{S}}}
\def\rmT{{\mathbf{T}}}
\def\rmU{{\mathbf{U}}}
\def\rmV{{\mathbf{V}}}
\def\rmW{{\mathbf{W}}}
\def\rmX{{\mathbf{X}}}
\def\rmY{{\mathbf{Y}}}
\def\rmZ{{\mathbf{Z}}}

% Elements of random matrices
\def\ermA{{\textnormal{A}}}
\def\ermB{{\textnormal{B}}}
\def\ermC{{\textnormal{C}}}
\def\ermD{{\textnormal{D}}}
\def\ermE{{\textnormal{E}}}
\def\ermF{{\textnormal{F}}}
\def\ermG{{\textnormal{G}}}
\def\ermH{{\textnormal{H}}}
\def\ermI{{\textnormal{I}}}
\def\ermJ{{\textnormal{J}}}
\def\ermK{{\textnormal{K}}}
\def\ermL{{\textnormal{L}}}
\def\ermM{{\textnormal{M}}}
\def\ermN{{\textnormal{N}}}
\def\ermO{{\textnormal{O}}}
\def\ermP{{\textnormal{P}}}
\def\ermQ{{\textnormal{Q}}}
\def\ermR{{\textnormal{R}}}
\def\ermS{{\textnormal{S}}}
\def\ermT{{\textnormal{T}}}
\def\ermU{{\textnormal{U}}}
\def\ermV{{\textnormal{V}}}
\def\ermW{{\textnormal{W}}}
\def\ermX{{\textnormal{X}}}
\def\ermY{{\textnormal{Y}}}
\def\ermZ{{\textnormal{Z}}}

% Vectors
\def\vzero{{\bm{0}}}
\def\vone{{\bm{1}}}
\def\vmu{{\bm{\mu}}}
\def\vtheta{{\bm{\theta}}}
\def\vsigma{{\bm{\sigma}}}
\def\va{{\bm{a}}}
\def\vb{{\bm{b}}}
\def\vc{{\bm{c}}}
\def\vd{{\bm{d}}}
\def\ve{{\bm{e}}}
\def\vf{{\bm{f}}}
\def\vg{{\bm{g}}}
\def\vh{{\bm{h}}}
\def\vi{{\bm{i}}}
\def\vj{{\bm{j}}}
\def\vk{{\bm{k}}}
\def\vl{{\bm{l}}}
\def\vm{{\bm{m}}}
\def\vn{{\bm{n}}}
\def\vo{{\bm{o}}}
\def\vp{{\bm{p}}}
\def\vq{{\bm{q}}}
\def\vr{{\bm{r}}}
\def\vs{{\bm{s}}}
\def\vt{{\bm{t}}}
\def\vu{{\bm{u}}}
\def\vv{{\bm{v}}}
\def\vw{{\bm{w}}}
\def\vx{{\bm{x}}}
\def\vy{{\bm{y}}}
\def\vz{{\bm{z}}}

% Elements of vectors
\def\evalpha{{\alpha}}
\def\evbeta{{\beta}}
\def\evepsilon{{\epsilon}}
\def\evlambda{{\lambda}}
\def\evomega{{\omega}}
\def\evmu{{\mu}}
\def\evpsi{{\psi}}
\def\evsigma{{\sigma}}
\def\evtheta{{\theta}}
\def\eva{{a}}
\def\evb{{b}}
\def\evc{{c}}
\def\evd{{d}}
\def\eve{{e}}
\def\evf{{f}}
\def\evg{{g}}
\def\evh{{h}}
\def\evi{{i}}
\def\evj{{j}}
\def\evk{{k}}
\def\evl{{l}}
\def\evm{{m}}
\def\evn{{n}}
\def\evo{{o}}
\def\evp{{p}}
\def\evq{{q}}
\def\evr{{r}}
\def\evs{{s}}
\def\evt{{t}}
\def\evu{{u}}
\def\evv{{v}}
\def\evw{{w}}
\def\evx{{x}}
\def\evy{{y}}
\def\evz{{z}}

% Matrix
\def\mA{{\bm{A}}}
\def\mB{{\bm{B}}}
\def\mC{{\bm{C}}}
\def\mD{{\bm{D}}}
\def\mE{{\bm{E}}}
\def\mF{{\bm{F}}}
\def\mG{{\bm{G}}}
\def\mH{{\bm{H}}}
\def\mI{{\bm{I}}}
\def\mJ{{\bm{J}}}
\def\mK{{\bm{K}}}
\def\mL{{\bm{L}}}
\def\mM{{\bm{M}}}
\def\mN{{\bm{N}}}
\def\mO{{\bm{O}}}
\def\mP{{\bm{P}}}
\def\mQ{{\bm{Q}}}
\def\mR{{\bm{R}}}
\def\mS{{\bm{S}}}
\def\mT{{\bm{T}}}
\def\mU{{\bm{U}}}
\def\mV{{\bm{V}}}
\def\mW{{\bm{W}}}
\def\mX{{\bm{X}}}
\def\mY{{\bm{Y}}}
\def\mZ{{\bm{Z}}}
\def\mBeta{{\bm{\beta}}}
\def\mPhi{{\bm{\Phi}}}
\def\mLambda{{\bm{\Lambda}}}
\def\mSigma{{\bm{\Sigma}}}

% Tensor
\DeclareMathAlphabet{\mathsfit}{\encodingdefault}{\sfdefault}{m}{sl}
\SetMathAlphabet{\mathsfit}{bold}{\encodingdefault}{\sfdefault}{bx}{n}
\newcommand{\tens}[1]{\bm{\mathsfit{#1}}}
\def\tA{{\tens{A}}}
\def\tB{{\tens{B}}}
\def\tC{{\tens{C}}}
\def\tD{{\tens{D}}}
\def\tE{{\tens{E}}}
\def\tF{{\tens{F}}}
\def\tG{{\tens{G}}}
\def\tH{{\tens{H}}}
\def\tI{{\tens{I}}}
\def\tJ{{\tens{J}}}
\def\tK{{\tens{K}}}
\def\tL{{\tens{L}}}
\def\tM{{\tens{M}}}
\def\tN{{\tens{N}}}
\def\tO{{\tens{O}}}
\def\tP{{\tens{P}}}
\def\tQ{{\tens{Q}}}
\def\tR{{\tens{R}}}
\def\tS{{\tens{S}}}
\def\tT{{\tens{T}}}
\def\tU{{\tens{U}}}
\def\tV{{\tens{V}}}
\def\tW{{\tens{W}}}
\def\tX{{\tens{X}}}
\def\tY{{\tens{Y}}}
\def\tZ{{\tens{Z}}}


% Graph
\def\gA{{\mathcal{A}}}
\def\gB{{\mathcal{B}}}
\def\gC{{\mathcal{C}}}
\def\gD{{\mathcal{D}}}
\def\gE{{\mathcal{E}}}
\def\gF{{\mathcal{F}}}
\def\gG{{\mathcal{G}}}
\def\gH{{\mathcal{H}}}
\def\gI{{\mathcal{I}}}
\def\gJ{{\mathcal{J}}}
\def\gK{{\mathcal{K}}}
\def\gL{{\mathcal{L}}}
\def\gLridge{{\mathcal{L}_{\rm ridge}}}
\def\gM{{\mathcal{M}}}
\def\gN{{\mathcal{N}}}
\def\gO{{\mathcal{O}}}
\def\gP{{\mathcal{P}}}
\def\gQ{{\mathcal{Q}}}
\def\gR{{\mathcal{R}}}
\def\gS{{\mathcal{S}}}
\def\gT{{\mathcal{T}}}
\def\gU{{\mathcal{U}}}
\def\gV{{\mathcal{V}}}
\def\gW{{\mathcal{W}}}
\def\gX{{\mathcal{X}}}
\def\gY{{\mathcal{Y}}}
\def\gZ{{\mathcal{Z}}}

% Sets
\def\sA{{\mathbb{A}}}
\def\sB{{\mathbb{B}}}
\def\sC{{\mathbb{C}}}
\def\sD{{\mathbb{D}}}
% Don't use a set called E, because this would be the same as our symbol
% for expectation.
\def\sF{{\mathbb{F}}}
\def\sG{{\mathbb{G}}}
\def\sH{{\mathbb{H}}}
\def\sI{{\mathbb{I}}}
\def\sJ{{\mathbb{J}}}
\def\sK{{\mathbb{K}}}
\def\sL{{\mathbb{L}}}
\def\sM{{\mathbb{M}}}
\def\sN{{\mathbb{N}}}
\def\sO{{\mathbb{O}}}
\def\sP{{\mathbb{P}}}
\def\sQ{{\mathbb{Q}}}
\def\sR{{\mathbb{R}}}
\def\sS{{\mathbb{S}}}
\def\sT{{\mathbb{T}}}
\def\sU{{\mathbb{U}}}
\def\sV{{\mathbb{V}}}
\def\sW{{\mathbb{W}}}
\def\sX{{\mathbb{X}}}
\def\sY{{\mathbb{Y}}}
\def\sZ{{\mathbb{Z}}}

% Entries of a matrix
\def\emLambda{{\Lambda}}
\def\emA{{A}}
\def\emB{{B}}
\def\emC{{C}}
\def\emD{{D}}
\def\emE{{E}}
\def\emF{{F}}
\def\emG{{G}}
\def\emH{{H}}
\def\emI{{I}}
\def\emJ{{J}}
\def\emK{{K}}
\def\emL{{L}}
\def\emM{{M}}
\def\emN{{N}}
\def\emO{{O}}
\def\emP{{P}}
\def\emQ{{Q}}
\def\emR{{R}}
\def\emS{{S}}
\def\emT{{T}}
\def\emU{{U}}
\def\emV{{V}}
\def\emW{{W}}
\def\emX{{X}}
\def\emY{{Y}}
\def\emZ{{Z}}
\def\emSigma{{\Sigma}}

% entries of a tensor
% Same font as tensor, without \bm wrapper
\newcommand{\etens}[1]{\mathsfit{#1}}
\def\etLambda{{\etens{\Lambda}}}
\def\etA{{\etens{A}}}
\def\etB{{\etens{B}}}
\def\etC{{\etens{C}}}
\def\etD{{\etens{D}}}
\def\etE{{\etens{E}}}
\def\etF{{\etens{F}}}
\def\etG{{\etens{G}}}
\def\etH{{\etens{H}}}
\def\etI{{\etens{I}}}
\def\etJ{{\etens{J}}}
\def\etK{{\etens{K}}}
\def\etL{{\etens{L}}}
\def\etM{{\etens{M}}}
\def\etN{{\etens{N}}}
\def\etO{{\etens{O}}}
\def\etP{{\etens{P}}}
\def\etQ{{\etens{Q}}}
\def\etR{{\etens{R}}}
\def\etS{{\etens{S}}}
\def\etT{{\etens{T}}}
\def\etU{{\etens{U}}}
\def\etV{{\etens{V}}}
\def\etW{{\etens{W}}}
\def\etX{{\etens{X}}}
\def\etY{{\etens{Y}}}
\def\etZ{{\etens{Z}}}
\def\ourmethod{BnBTree}



% The true underlying data generating distribution
\newcommand{\pdata}{p_{\rm{data}}}
% The empirical distribution defined by the training set
\newcommand{\ptrain}{\hat{p}_{\rm{data}}}
\newcommand{\Ptrain}{\hat{P}_{\rm{data}}}
% The model distribution
\newcommand{\pmodel}{p_{\rm{model}}}
\newcommand{\Pmodel}{P_{\rm{model}}}
\newcommand{\ptildemodel}{\tilde{p}_{\rm{model}}}
% Stochastic autoencoder distributions
\newcommand{\pencode}{p_{\rm{encoder}}}
\newcommand{\pdecode}{p_{\rm{decoder}}}
\newcommand{\precons}{p_{\rm{reconstruct}}}

\newcommand{\laplace}{\mathrm{Laplace}} % Laplace distribution

\newcommand{\E}{\mathbb{E}}
\newcommand{\Ls}{\mathcal{L}}
\newcommand{\R}{\mathbb{R}}
\newcommand{\emp}{\tilde{p}}
\newcommand{\lr}{\alpha}
\newcommand{\reg}{\lambda}
\newcommand{\rect}{\mathrm{rectifier}}
\newcommand{\softmax}{\mathrm{softmax}}
\newcommand{\sigmoid}{\sigma}
\newcommand{\softplus}{\zeta}
\newcommand{\KL}{\text{KL}}
\newcommand{\Var}{\mathrm{Var}}
\newcommand{\standarderror}{\mathrm{SE}}
\newcommand{\Cov}{\mathrm{Cov}}
% Wolfram Mathworld says $L^2$ is for function spaces and $\ell^2$ is for vectors
% But then they seem to use $L^2$ for vectors throughout the site, and so does
% wikipedia.
\newcommand{\normlzero}{L^0}
\newcommand{\normlone}{L^1}
\newcommand{\normltwo}{L^2}
\newcommand{\normlp}{L^p}
\newcommand{\normmax}{L^\infty}

\newcommand{\parents}{Pa} % See usage in notation.tex. Chosen to match Daphne's book.

% absolute value and norm
\def\vert#1{\lvert #1 \rvert}
\def\Vert#1{\lVert #1 \rVert}

% inner product
\newcommand{\innerProduct}[2]{\langle #1, #2 \rangle}
\newcommand{\largeInnerProduct}[2]{\left\langle #1, #2 \right\rangle}
\newcommand{\myfrac}[2]{\frac{#1}{#2}}

% \def\innnerProduct[2]{\langle #1, #2 \rangle}
% \def\largeInnnerProduct[2]{\left\langle #1, #2 \right\rangle}

\DeclareMathOperator{\sign}{sign}
\DeclareMathOperator{\Tr}{Tr}
\let\ab\allowbreak


\newcommand{\textssm}[1]{#1}

\newcommand{\modelfont}{\renewcommand*\familydefault{\sfdefault}\normalfont}
\newcommand{\prow}[0]{\quad\mathrel{\raisebox{-0.75ex}{\dots}}}
\newcommand{\risklabel}[0]{{\color{black}\textbf{RISK}}}
\newcommand{\scorelabel}[0]{{\color{black}\textbf{SCORE}}}

%\definecolor{guidecolor}{white}
\definecolor{predcolor}{gray}{0.95}
\definecolor{scorecolor}{gray}{0.95}
\definecolor{riskcolor}{gray}{0.95}
\definecolor{transparentcolor}{gray}{0.95}

\newcommand{\instruction}[2]{{\color{white}\phantom{\textbf{add points from rows {#1} to {#2}}}}}
\newcommand{\scoringsystem}[0]{\scriptsize\centering\renewcommand{\arraystretch}{1.35}\modelfont}
\newcommand{\risktable}[0]{\par\vspace{0.5em}\scriptsize\centering\renewcommand{\arraystretch}{1.25}\modelfont}
\newcommand{\predcell}[2]{\par\vspace{0.5em}\renewcommand{\arraystretch}{1.5}\modelfont%
\begin{tabular}{|>{\columncolor{predcolor}}{#1}|}\hline\small{\textbf{#2}}\\\hline\end{tabular}}

% algorithm package
\usepackage{algorithm}
\usepackage{algorithmic}
% \usepackage{algpseudocode}
\definecolor{ForestGreen}{rgb}{0.13, 0.55, 0.13} % http://latexcolor.com/
% algorithmic comment
% \renewcommand{\algorithmiccomment}[1]{\hfill$\triangleright$\textcolor{ForestGreen}{\textit{#1}}} %# 
\renewcommand{\algorithmiccomment}[1]{$\triangleright$\textcolor{ForestGreen}{\textit{#1}}} %# 
% equation commment
\usepackage{color} % Load color package                                                                     
% \newcommand{\myeqcomment}[1]{\eqcomment{\textcolor{red}{#1}}}
\newcommand{\eqcomment}[1]{\tag*{\textit{\# \textcolor{ForestGreen}{#1}}}}


% special spacing
\newcommand{\specialQuad}{\ \ }

%%%%% To create table of contents only for Appendix %%%%%%

\usepackage[toc,page,header]{appendix}
\usepackage{minitoc}

% Make the "Part I" text invisible
\renewcommand \thepart{}
\renewcommand \partname{}

%%%%%%%%%%%%%%%%%%%%%%
% new command for indicating which part to read and edit
\newcommand{\CynthiaBeginsEditHere}{\textcolor{red}{[05/19/2024] Please read the Abstract,  Introduction, Preliminary, and Methodology for now. }}

\newcommand{\CynthiaEndsEditHere}{\textcolor{red}{\CynthiaBeginsEditHere 
Other writing, essentially everything below, is not polished enough for editing.}}

\newcommand{\ToDo}[1]{\textcolor{red}{#1}}

\DeclareMathOperator{\cl}{cl}
\DeclareMathOperator{\conv}{conv}
\DeclareMathOperator{\TopSum}{TopSum}
\DeclareMathOperator{\prox}{prox}
\undef\st
\DeclareMathOperator{\st}{s.\!t.\!}
\DeclareMathOperator{\sgn}{sgn}

\begin{document}

\twocolumn[
\papertitle{Scalable First-order Method for Certifying Optimal k-Sparse GLMs}

% It is OKAY to include author information, even for blind
% submissions: the style file will automatically remove it for you
% unless you've provided the [accepted] option to the xxx
% package.

% List of affiliations: The first argument should be a (short)
% identifier you will use later to specify author affiliations
% Academic affiliations should list Department, University, City, Region, Country
% Industry affiliations should list Company, City, Region, Country

% You can specify symbols, otherwise they are numbered in order.
% Ideally, you should not use this facility. Affiliations will be numbered
% in order of appearance and this is the preferred way.

% \setsymbol{equal}{*}

\begin{authorlist}
\paperauthor{Jichang Liu}{yyy}
\paperauthor{Soroosh Shafiee}{yyy}
\paperauthor{Andrea Lodi}{comp}
\end{authorlist}

\affiliation{yyy}{School of Operations Research and Information Engineering, Cornell University, Ithaca, NY, USA}
\affiliation{comp}{Jacobs Technion-Cornell Institute, Cornell Tech and Technion–IIT, New York, NY, USA}

\displayemail{\{jiachang.liu, shafiee, al748\}@cornell.edu}


% You may provide any keywords that you
% find helpful for describing your paper; these are used to populate
% the "keywords" metadata in the PDF but will not be shown in the document
\keywords{Machine Learning, Optimization}

\vskip 0.3in
]

% this must go after the closing bracket ] following \twocolumn[ ...

% This command actually creates the footnote in the first column
% listing the affiliations and the copyright notice.
% The command takes one argument, which is text to display at the start of the footnote.
% The \xxxEqualContribution command is standard text for equal contribution.
% Remove it (just {}) if you do not need this facility.

\printAffiliationsAndNotice{}  % leave blank if no need to mention equal contribution
%\printAffiliationsAndNotice{\xxxEqualContribution} % otherwise use the standard text.

%%%%% To create table of contents only for Appendix %%%%%%

%\doparttoc % Tell to minitoc to generate a toc for the parts
%\faketableofcontents % Run a fake tableofcontents command for the partocs

\begin{abstract}\label{00_Abstract}
Research in the field of automated vehicles, or more generally cognitive cyber-physical systems that operate in the real world, is leading to increasingly complex systems. Among other things, artificial intelligence enables an ever-increasing degree of autonomy. In this context, the V-model, which has served for decades as a process reference model of the system development lifecycle is reaching its limits. To the contrary, innovative processes and frameworks have been developed that take into account the characteristics of emerging autonomous systems. To bridge the gap and merge the different methodologies, we present an extension of the V-model for iterative data-based development processes that harmonizes and formalizes the existing methods towards a generic framework. The iterative approach allows for seamless integration of continuous system refinement. While the data-based approach constitutes the consideration of data-based development processes and formalizes the use of synthetic and real world data. In this way, formalizing the process of development, verification, validation, and continuous integration contributes to ensuring the safety of emerging complex systems that incorporate AI. 
\end{abstract}


\begin{IEEEkeywords}
	Process Reference Model, V-Model, Continuous Integration, AI Systems, Autonomy Technology, Safety Assurance
\end{IEEEkeywords}


\section{Introduction}

Sparse generalized linear models (GLMs) are essential tools in machine learning (ML), widely applied in fields like healthcare, finance, engineering, and science. 
These models provide a flexible framework for capturing relationships between variables while ensuring interpretability, which is critical in high-stakes applications.
Recently, using the $\ell_0$ norm to induce sparsity has gained significant attention. This approach provides distinct advantages over traditional convex relaxation methods, such as replacing $\ell_0$ with $\ell_1$, particularly in cases involving highly correlated features.

In this paper, we aim to solve
\begin{align} \label{obj:original_sparse_problem}
    \begin{array}{cl}
        \min\limits_{\bbeta \in \R^p} & f(\bX \bbeta, \by) + \lambda_2 \lVert \bbeta \rVert_2^2 \\
        \st & \| \bbeta \|_\infty \leq M, ~ \lVert \bbeta \rVert_0 \leq k,
    \end{array}
\end{align}
where $\bX \in \R^{n \times p}$ and $\by \in \R^n$ denote the matrix of features and the vector of labels, respectively, while the parameter $M > 0$ can be either user-defined based on prior knowledge or estimated from the data~\citep{park2020subset}. The GLM loss function, denoted by $f : \R^n \times \R^n \to \R$, is assumed to be Lipschitz smooth, the parameter $k \in \mathbb N$ controls the number of nonzero coefficients, and $\lambda_2 > 0$ is a small Tikhonov regularization coefficient to address collinearity.
Alas, problem~\eqref{obj:original_sparse_problem} is NP-hard~\citep{natarajan1995sparse}. 
As a result, most existing methods rely on heuristics that deliver high-quality approximations but lack guarantees of optimality. 
This limitation is particularly problematic in high-stakes applications like healthcare, where ensuring accuracy, reliability, and safety is essential. 
Therefore, we emphasize the pursuit of certifiably optimal solutions.

A naive approach to solve~\eqref{obj:original_sparse_problem} to optimality is to reformulate it as a mixed-integer programming (MIP) problem and leverage commercial MIP solvers.
However, these solvers face significant scalability challenges, particularly with large datasets and nonlinear objectives. 
A major bottleneck arises from the need to compute tight lower bounds at each node of the branch-and-bound (BnB) tree, a critical component for efficient pruning and solver performance.
Existing methods for computing lower bounds typically rely on linear programming or conic optimization techniques.
However, these approaches either generate loose bounds that reduce pruning efficiency or result in high computational costs per iteration. 
Moreover, they are challenging to parallelize, which limits the potential to take advantage of modern hardware accelerators like GPUs.

To address these challenges, we propose a scalable first-order method for efficiently calculating lower bounds within the BnB framework.
We begin with a perspective reformulation of~\eqref{obj:original_sparse_problem} and derive its continuous relaxation.
The resulting formulation is then expressed as an unconstrained optimization problem, characterized by a convex composite objective function, which enables the application of the Fast Iterative Shrinkage-Thresholding Algorithm (FISTA), a well-known first-order method~\citep{beck2009fast}, to compute lower bounds.
The successful implementation of FISTA, however, relies on efficient computation of the proximal operator, which requires solving a second order cone program (SOCP) problem.
To the best of our knowledge, the efficient computation of this proximal operator has not been previously addressed in the literature. 
Therefore, we propose a customized pooled-adjacent-violation algorithm (PAVA) that evaluates the proximal operator exactly with log-linear time complexity, ensuring the scalability of our FISTA approach for large problem instances.
% Our method achieves three core objectives: a) fast convergence rate, b) low per-iteration computational complexity, and c) compatibility with GPU acceleration.
A major advantage of our approach is its computational efficiency, in which instead of solving costly linear systems, it only relies on matrix-vector multiplication, which is highly amenable to GPU acceleration.
This capability addresses a key limitation of existing approaches that struggle to parallelize their computations on modern hardware.

To accelerate the performance of the FISTA algorithm, we introduce a restart heuristic. 
This leads to an empirical linear convergence rate, a result not previously achieved by other first-order methods for this type of problem.
Empirically, our method demonstrates substantial speedups in computing dual bounds -- often by 1-2 orders of magnitude -- compared to existing techniques. 
These improvements significantly enhance the overall efficiency of the BnB process, enabling the certification of large-scale instances of~\eqref{obj:original_sparse_problem} that were previously intractable using commercial MIP solvers. All omitted proofs are provided in~\ref{appendix_sec:proofs}. Additional numerical results are reported in~\ref{appendix:numerical}.

\subsection{Contributions}
The key contributions of this paper are summarized below.
\begin{itemize}[label=$\diamond$,leftmargin=*]
    \item We propose a FISTA-based first-order method to enhance the scalability of solving~\eqref{obj:original_sparse_problem}, with a focus on efficient lower-bound computation within the BnB framework.
    \item The proximal operator in the FISTA method is computed using a customized PAVA that leverages hidden mathematical structures and enjoys log-linear time complexity, ensuring scalability for large-scale problems.
    \item Besides achieving fast convergence rates (via a restart strategy) and low per-iteration computational complexity, our method can be easily parallelized on GPUs, something not currently achievable by MIP methods.
    \item We validate the practical efficiency of our approach on both synthetic and real-world datasets, demonstrating substantial speedups in computing dual bounds and certifying optimal solutions for large-scale sparse GLMs.
    % highlighting its broader impact on optimization and machine learning.
\end{itemize}

\subsection{Related Works}
\label{sec:related_work}

\paragraph{MIP for ML.}
MIP has been successfully applied in
medical scoring systems~\citep{ustun2016supersparse, ustun2019learning, liu2022fasterrisk}, 
portfolio optimization \citep{bienstock1996computational,wei2022convex}, nonlinear identification systems~\citep{bertsimas2023learning, liu2024okridge},
decision trees~\citep{bertsimas2017optimal, hu2019optimal},
survival analysis~\citep{zhang2023optimal, liu2024fastsurvival},
hierarchical models~\citep{bertsimas2020sparse}, regression and classification models~\citep{atamturk2020safe, bertsimas2020sparse, bertsimas2020sparse1, bertsimas2020sparse2, hazimeh2020fast, xie2020scalable, atamturk2021sparse, dedieu2021learning, hazimeh2022sparse, liu2024okridge, guyard2024el0ps}, graphical models \citep{manzour2021integer, kucukyavuz2023consistent}, and outlier detection \citep{gomez2021outlier,gomez2023outlier}.
The primary focus of these works is on obtaining high-quality feasible solutions, with only a small subset addressing the certification of optimality.
Our work aims to contribute to this literature, with a strong focus on enhancing the computational scalability of certifying optimality for solving sparse GLM problems.

\paragraph{Perspective Formulations.} 
The application of perspective functions to derive convex relaxations for~\eqref{obj:original_sparse_problem} dates back to the seminal work of \citet{ceria1999convex}.
Perspective formulations have been developed for separable functions in \citep{gunluk2010perspective, xie2020scalable, wei2022ideal, bacci2019new, shafiee2024constrained} and for rank-one functions in \citep{atamturk2020supermodularity, wei2020convexification, wei2022ideal, han2021compact, shafiee2024constrained} under various conditions. 
Our work uses perspective formulations of separable functions that appear in~\eqref{obj:original_sparse_problem} as the Tikhonov regularization function. 

\paragraph{Lower Bound Calculation.}
A key aspect of certifying optimality in MIP problems is the efficient computation of tight lower bounds.
Commercial MIP solvers typically iteratively linearize the objective function using the celebrated outer approximation method~\citep{kelley1960cutting} (via cutting planes) and solve the resulting linear program~\citep{schrijver1998theory, wolsey2020integer}.
However, this approach often produce loose lower bounds, especially when high-quality linear cuts are not generated.
Alternatively, solvers may use conic convex relaxations and solve them with the interior-point method (IPM)~\citep{dikin1967iterative, renegar2001mathematical, nesterov1994interior}. 
While this approach often yields tighter lower bounds, IPM does not scale well due to its reliance on second-order information and because -- differently from the linear case -- effectively warm-starting IPMs is not possible.
Recent attempts are based on first-order methods, including subgradient descent~\citep{bertsimas2020sparse1}, ADMM~\citep{liu2024okridge}, and coordinate descent~\citep{hazimeh2022sparse}. Our work builds on this, offering faster convergence, low computational complexity, and significant GPU acceleration. We also observe that our proposed FISTA method achieves linear convergence rates empirically, a result not previously achieved by other first-order methods for this problem.


\paragraph{GPU Acceleration.}
Recently, there have been some promising works on using GPUs to accelerate continuous optimization problems, including linear programming~\citep{applegate2021practical, lu2023cupdlp}, quadratic programming~\citep{lu2023practical}, and semidefinite programming~\citep{han2024accelerating}.
%There are few works of using GPUs to solve discrete problems.
A natural way to leverage GPUs for discrete problems is by using GPU-based LPs within MIP solvers, as demonstrated by~\citet{de2024power} for solving clustering problems.
However, in \citep{de2024power}, the challenge is to approximate the original objective function with a potentially exponential number of cutting planes. %their proposed cutting-plane approach faces challenges when it requires iteratively generating an exponential number of cuts to approximate the original objective function. 
In contrast, we develop a customized FISTA method that directly handles the nonlinear objective function, while the computation can be easily parallelized since it only involves matrix-vector multiplication. 
Other first-order methods, such as ADMM~\citep{liu2024okridge} and coordinate descent~\citep{hazimeh2022sparse}, are unsuitable for GPUs: ADMM requires solving linear systems, while coordinate descent is inherently sequential.



\section{Problem Formulation}
In this preliminary section, we introduce some backgrounds on how to obtain a lower bound (which will be used for the branch-and-bound process to prune nodes) for the optimal value of Problem~\eqref{obj:original_sparse_problem} by solving an associated convex relaxation problem.
First, note that we can cast problem~\eqref{obj:original_sparse_problem}~as 
\begin{align}
    \label{epigraph:formulation}
    \min \left\{ \tau \,:\, (\tau, \bbeta, \bz) \in \mathcal S \right\},
\end{align}
where the extended feasible set is defined as
\begin{align}
    \label{eq:S}
    \mathcal{S} = \left\{ (\tau, \bbeta, \bz)  \;\middle|\;
    \begin{array}{l} 
        \| \bbeta \|_\infty \leq M, \\
        \bz \in \{0, 1\}^p, \, \mathbf{1}^T \bz \leq k, \\
        \beta_j ( 1 - z_j) = 0 ~~ \forall j \in [p] \\
        f(\bX \bbeta, \by) + \lambda_2 \| \bbeta \|_2^2 \leq \tau
    \end{array}
    \right\},
\end{align}
and $[p] = \{1, \dots, p \}$ stands for the set of all integers up to $p \in \mathbb N$.
Put it differently, each binary variable $z_j$ indicates whether a continuous variable $\beta_j$ is zero or not by requiring $\beta_j = 0$ when $z_j = 0$ and allowing $\beta_j$ to take any value when $z_j = 1$. 
Meanwhile, the objective function is linearized using the epigraph reformulation technique, which allows us to interpret the optimal value of~\eqref{epigraph:formulation} as the evaluation of the support function of $\mathcal S$ at $(\bm 0, \bm 0, 1)$.
By virtue of \citep[\S13]{rockafellar1970convex}, the optimal value of~\eqref{epigraph:formulation} remains unchanged if we replace $\mathcal S$ with $\cl \conv(\mathcal S)$, where $\cl \conv(\mathcal S)$ denotes the closed convex hull of $\mathcal S$. 
Alas, the exact description of $\cl \conv(\mathcal S)$ requires exponentially many (nonlinear) constraints, which leads to the NP-hardness of~\eqref{obj:original_sparse_problem}.

We thus explore other options for a convex relaxation of~\eqref{obj:original_sparse_problem}.
It turns out that a tractable convex hull can be obtained if the objective function only includes the Tikhonov regularization term $\| \bbeta \|_2^2$, using the perspective function.
The perspective function of the quadratic function $h(\beta) = \beta^2$ is $h^\pi(\beta, z) = \beta^2 / z$ if $z > 0$, $= 0$ if $\beta = z = 0$, and $= \infty$ otherwise.
For simplicity, we write $\beta^2/z$ instead of $h^\pi(\beta, z)$ even if $z = 0$. 
The following lemma provides an exact perspective formulation of the convex hull when $\mathcal S$ does not include $f(\bX \bbeta, \by)$. This result extends \citep[Lemma~6]{gunluk2010perspective} by incorporating sparsity constraints, while also extending \citep[Theorem~2]{shafiee2024constrained} to account for $\ell_\infty$ box constraint on $\bbeta$.
\begin{lemma}
    \label{lemma:equivalence_between_perspective_relaxation_and_convexification}
    The closed convex hull of the set
    \begin{align*}
        \left\{ (\tau, \bbeta, \bz) \middle|
        \begin{array}{l}
            \| \bbeta \|_\infty \leq M, \,  \\
            \bz \in \{0, 1\}^p, \, \mathbf{1}^T \bz \leq k, \\ \beta_j ( 1 - z_j) = 0 ~~ \forall j \in [p], \\
            \sum_{j \in [p]} \beta_j^2 \leq \tau
        \end{array}
        \right\}
    \end{align*}
    is given by the set
    \begin{align*}
        \left\{ (\tau, \bbeta, \bz)  \;\middle|\;
        \begin{array}{l} 
            -M z_j\leq \bbeta_j \leq M z_j ~ \forall j \in [p], \\
            \bz \in [0, 1]^p, \, \mathbf{1}^T \bz \leq k, \\
            \sum_{j \in [p]} \beta_j^2 / z_j \leq \tau
        \end{array}
        \right\}.
    \end{align*}
\end{lemma}
The convex hull formulation presented in Lemma~\ref{lemma:equivalence_between_perspective_relaxation_and_convexification} is a second-order conic set.
Specifically, the epigraph of the sum of perspective functions in the last line satisfies
\begin{align*}
    \sum_{j \in [p]} {\beta_j^2}/{z_j} \leq \tau \iff \exists \bt \in \R_+^p ~ \st ~ 
    \begin{cases}
        \bm 1^\top \bm t = \tau, \\
        \beta_j^2 \leq z_j t_j ~ \forall j \in [p],
    \end{cases}
\end{align*}
which is second order cone representable.
Motivated by Lemma~\ref{lemma:equivalence_between_perspective_relaxation_and_convexification}, we immediately see that the extended feasible set $\mathcal S$ defined in~\eqref{eq:S} admits the following perspective representation
\begin{align*}
% \label{eq:perspective:S}
    \mathcal{S} = \left\{ (\tau, \bbeta, \bz)  \;\middle|\;
    \begin{array}{l} 
        -M z_j\leq \bbeta_j \leq M z_j ~ j \in [p], \\
        \bz \in \{0, 1\}^p, \, \mathbf{1}^T \bz \leq k,  \\
        f(\bX \bbeta, \by) + \lambda_2 \sum_{j \in [p]} \beta_j^2 / z_j \leq \tau
    \end{array}
    \right\}.
\end{align*}
Plutting in this new perspective representation into Problem~\eqref{epigraph:formulation}, we can reformulate~\eqref{obj:original_sparse_problem} as follows
\begin{align}
    \label{obj:original_sparse_problem_perspective_formulation}
    P_{\text{MIP}}^\star = \left\{
    \begin{array}{cll}
        \min\limits_{\bbeta, \bz \in \R^p} & f(\bX \bbeta, \by) + \lambda_2 \sum_{j \in [p]} {\beta_j^2}/{z_j} \\[1ex]
        \text{\; s.t.} & \bz \in \{0, 1\}^p, \, \mathbf{1}^T \bz \leq k, \\[1ex]
        & -M z_j \leq \beta_j \leq M z_j ~ \forall j \in [p].
    \end{array}
    \right.
\end{align}
By relaxing the binary variables $z_j$ to the interval $[0, 1]$, we obtain the following strong convex relaxation of~\eqref{obj:original_sparse_problem_perspective_formulation}
\begin{align}
    \label{obj:original_sparse_problem_perspective_formulation_convex_relaxation}
    P_{\text{conv}}^\star = \left\{
    \begin{array}{cll}
        \min\limits_{\bbeta, \bz \in \R^p} & f(\bX \bbeta, \by) + \lambda_2 \sum_{j \in [p]} {\beta_j^2}/{z_j} \\[1ex]
        \text{\; s.t.} & \bz \in [0, 1]^p, \, \mathbf{1}^T \bz \leq k, \\[1ex]
        & -M z_j \leq \beta_j \leq M z_j ~ \forall j \in [p].
    \end{array}
    \right.
\end{align}
Although this is not the convex hull formulation due to the term $f(\bX \bbeta, \by)$, unlike in Lemma~\ref{lemma:equivalence_between_perspective_relaxation_and_convexification}, $P^\star_{\text{conv}}$ still provides a lower bound for Problem~\eqref{obj:original_sparse_problem}.


We can solve~\eqref{obj:original_sparse_problem_perspective_formulation_convex_relaxation} using standard conic optimization solvers like Mosek and Gurobi, which rely on IPMs for solving such subproblems in the BnB framework.
However, IPMs are computationally expensive and do not scale well for large datasets.
Alternatively, first-order conic solvers such as SCS~\cite{o2016conic}, based on ADMM, can be used.
While these methods are more scalable, they suffer from slow convergence rates and require solving linear systems at each iteration, which can also be computationally intensive for large instances. The main goal of the paper is to introduce an efficient and scalable first-order method to address these limitations.

\section{Methodology}
\label{sec:methodology}

We begin with reformulating~\eqref{obj:original_sparse_problem_perspective_formulation_convex_relaxation} as the following \textit{unconstrained} optimization problem
\begin{align}
    \label{obj:original_sparse_problem_convex_composite_reformulation}
    \min_{\bbeta} f(\bX \bbeta, \by) + 2 \lambda_2 \, g(\bbeta),
\end{align}
where the implicit function $g: \R^p \to \R \cup \{ \infty \}$ is defined~as
\begin{align}
    \label{eq:function_g_definition}
    g(\bbeta) = \left\{ 
    \begin{array}{cl}
        \min\limits_{\bz \in \R^p} & \frac{1}{2} \sum_{j \in [p]} \beta_j^2 / z_j \\[1ex]
        \st & \bz \in [0, 1]^p, \, \bm 1^\top \bz \leq k, \\
        & -M z_j \leq \beta_j \leq M z_j ~ \forall j \in [p].
    \end{array} 
    \right.
\end{align}
Here, we follow the standard convention that an infeasible minimization problem is assigned a value of $+\infty$. 
Note that $g$ is convex as convexity is preserved under partial minimization over a convex set~\citep[Theorem~5,3]{rockafellar1970convex}. 
Furthermore, as $f$ is assumed to be Lipschitz smooth and $g$ is non-smooth, problem~\eqref{obj:original_sparse_problem_convex_composite_reformulation} is an unconstrained optimization problem with a convex composite objective function. As such, it is amenable to be solved using the FISTA algorithm proposed in \citep{beck2009fast}. 

In the following, we first analyze the conjugate of $g$. We then propose an efficient numerical method to compute the proximal operator of $g^*$. This, in turn, enables us to compute the proximal operator of $g$, leading to an efficient implementation of the FISTA algorithm. To further enhance the performance of FISTA, we present an efficient approach to solve the minimization problem~\eqref{eq:function_g_definition}, which guides us in developing an effective restart procedure. Finally, we conclude this section by providing efficient lower bounds for each step of the BnB framework.

\subsection{Conjugate function $g^*$}
Recall that the conjugate of $g$ is defined as
\begin{align*}
    g^*(\bm \alpha) = \sup_{\bm \beta \in \R^p} ~ \bm \alpha^\top \bm \beta - g(\bm \beta).
\end{align*}
The following lemma gives a closed-form expression for $g^*$, where $\TopSum_k(\cdot)$ denotes the sum of the top $k$ largest elements, and $H_M: \R \to \R$ is the Huber loss function defined as
\begin{equation}
    H_M(\alpha_j) := \begin{cases}
        \frac{1}{2} \alpha_j^2 & \text{if } \vert{\alpha_j} \leq M \\
        M \vert{\alpha_j} - \frac{1}{2} M^2 & \text{if } \vert{\alpha_j} > M
    \end{cases}.
\end{equation}
For notational simplicity, we use the shorthand notation ${\bf H}_M(\balpha)$ to denote ${\bf H}_M(\balpha) = (H_M(\alpha_1), \dots, H_M(\alpha_p))$.

\begin{lemma}
    \label{lemma:fenchel_conjugate_of_g_closed_form_expression}
    The conjugate of $g$ is given by
    \begin{equation}
        g^*(\balpha) = \TopSum_k({\bf H}_M(\balpha)).
    \end{equation}
\end{lemma}
This closed-form expression enables us to compute the proximal of $g^*$. Note that while the proximal operators of both $\TopSum_k$ and ${\bf H}_M$ functions are known (see, for example, \citep[Examples~6.50 \& 6.54]{beck2017first}), the conjugate function $g^*$ is defined as the composition of these two functions.
Alas, there is no general formula to derive the proximal operator of a composition of two functions based on the proximal operators of the individual functions. In the next section, we will see how to bypass this compositional difficulty.


\subsection{Proximal operator of $g^*$}
Recall that the proximal operator of $g^*$ with weight parameter $\rho > 0$ is defined as
\begin{align}
    \label{eq:prox:g*}
    \prox_{\rho g^*}(\bm \mu) = \argmin_{\bm \alpha \in \R^p} ~ \frac{1}{2} \Vert{\bm \alpha - \bm \mu}_2^2 + \rho g^*(\bm \alpha).
\end{align}
The evaluation of $\prox_{\rho g^*}$ involves a minimization problem that can be reformulated as a convex SOCP problem. 
Generic solvers based on IPM and ADMM require solving systems of linear equations. This results in cubic time complexity per iteration, making them computationally expensive, particularly for large-scale problems. 
These methods also cannot return \emph{exact} solutions. 
The lack of exactness can affect the stability and reliability of the proximal operator, which is crucial for the convergence of the FISTA algorithm. 
Inspired by~\citep{busing2022monotone}, we present Algorithm~\ref{alg:PAVA_algorithm}, a customized pooled adjacent violators algorithm that provides an exact evaluation of $\prox_{\rho g^*}$ in linear time after performing a simple 1D sorting step.

\begin{theorem}
    \label{theorem:pava_algorithm_linear_time_complexity_and_exact_solution}
    For any $\bmu \in \R^p$, Algorithm~\ref{alg:PAVA_algorithm} returns the \textit{exact} evaluation of $\prox_{\rho g^*}(\bmu)$ in $\tilde {\mathcal O}(p)$.
\end{theorem}


\begin{algorithm}[tb]
\caption{Customized PAVA to solve $\text{prox}_{\rho g^*}(\bmu)$}
\label{alg:PAVA_algorithm}
\begin{flushleft}
\textbf{Input:} vector $\bmu$, scalar multiplier $\rho$, and threshold $M$ of the Huber loss function $H_M(\cdot)$\\
% \textbf{Output:} vector $\balpha^*=\text{prox}_{\rho g^*}(\bmu)$.
\end{flushleft}
\begin{algorithmic}[1]
    \STATE Initialize $\brho \in \mathbb{R}^n$ with $\rho_j = \rho$ if $j \in \{1, 2, ..., k\}$ and $\rho_j = 0$ otherwise.
    \STATE \COMMENT{Sort $\bmu$ such that $\vert{\mu_1} \geq \vert{\mu_2} \geq ... \geq \vert{\mu_p}$; $\bpi$ is the sorting order.}
    \STATE $\bmu, \bpi = \text{SpecialSort} (\bmu)$
    
    \STATE \COMMENT{STEP 1: Initialize a pool of $p$ blocks with start and end indices; each block initially has length equal to $1$}
    \STATE $\calJ = \{[1, 1], [2, 2], ..., [p, p]\}$

    \STATE \COMMENT{STEP 2: Initialize $\hat{\nu}_j$ in each block by ignoring the isotonic constraints}
    \FOR{$j=1, 2, \dots, p$} 
        \STATE $\hat{\nu}_j = \argmin_{\nu} \frac{1}{2} (\nu - \vert{\mu_j})^2 + \rho_j H_M(\nu)$ \label{alg_line:PAVA_algorithm_initialization_step}
    \ENDFOR
    
    \STATE \COMMENT{STEP 3: Whenever there is an isotonic constraint violation between two adjacent blocks, merge the two blocks by setting all values to be the minimizer of the objective function restricted to this merged block; use \textcolor{red}{Algorithm~\ref{alg:up_and_down_block_algorithm_for_merging_in_PAVA}} in~\ref{appendix_sec:proofs}}
    \WHILE{$\exists [a_1, a_2], [a_2+1, a_3] \in \calJ \text{ s.t. } \hat{\nu}_{a_1} < \hat{\nu}_{a_3}$}
        \STATE $\calJ = \calJ \setminus \{[a_1, a_2]\} \setminus \{[a_2+1, a_3]\} \cup \{[a_1, a_3]\}$
        \STATE $\hat{\nu}_{[a_1:a_3]} = \argmin\limits_{\nu} \sum\limits_{j=a_1}^{a_3} \left[ \frac{1}{2} (\nu - \vert{\mu_j})^2 + \rho_j H_M(\nu) \right]$ \label{alg_line:PAVA_algorithm_pooling_step}
    \ENDWHILE

    \STATE \COMMENT{Return $\hat{\bnu}$ with the inverse sorting order}
    \STATE \textbf{Return} $\text{sgn}(\bmu) \odot \bpi^{-1}(\hat{\bnu})$
    
\end{algorithmic}
\end{algorithm}

The proof relies on several auxiliary lemmas.
We start with the following lemma, which uncovers a close connection between the proximal operator of $g^*$ and the generalized isotonic regression problems.

\begin{lemma}
    \label{lemma:equivalence_between_proximal_operator_and_huber_isotonic_regression}
    For any $\bmu \in \R^p$, we have 
    $$\prox_{\rho g^*}(\bmu) = \sgn(\bmu) \odot \bnu^\star, $$ 
    where $\odot$ denotes the Hadamard (element-wise) product, $\bnu^\star$ is the unique solution of the following optimization problem
    \begin{align}
        \label{obj:KyFan_Huber_isotonic_regression}
        \begin{array}{cl}
            \min\limits_{\bnu \in \R^p} & \frac{1}{2} \sum_{j \in [p]} (\nu_j - \vert{\mu_j})^2 + \rho \sum_{j \in \calJ} H_M (\nu_j) \\[2ex]
            \st & \quad \nu_j \geq \nu_l \; \text{ if } \; \vert{\mu_j} \geq \vert{\mu_l} ~~ \forall j, l \in [p],
        \end{array} 
    \end{align}
    and $\calJ$ is the set of indices of the top $k$ largest elements of~$ \vert{\mu_j}, j \in [p]$. 
\end{lemma}
Problem~\eqref{obj:KyFan_Huber_isotonic_regression} replaces the $\TopSum_k$ in~\eqref{eq:prox:g*} from the conjugate function $g^*$ (as shown in Lemma~\ref{lemma:fenchel_conjugate_of_g_closed_form_expression}) with linear constraints.
%These constraints ensure that the elements of $\bnu$ maintain a non-decreasing order in magnitude.
While this may appear computationally complex, it actually converts the problem into an instance of isotonic regression~\citep{best1990active}. Such problems can be solved exactly in linear time after performing a simple sorting step.
The procedure is known as PAVA~\citep{busing2022monotone}. Specifically, Algorithm~\ref{alg:PAVA_algorithm} implements a customized PAVA variant designed to compute $\prox_{\rho g^*}$ exactly.
The following lemma shows that the vector generated by Algorithm~\ref{alg:PAVA_algorithm} is an exact solution to~\eqref{obj:KyFan_Huber_isotonic_regression}.
Intuitively, Algorithm~\ref{alg:PAVA_algorithm} iteratively merges adjacent blocks until no isotonic constraint violations remain, at which point the resulting vector is guaranteed to be the optimal solution to~\eqref{obj:KyFan_Huber_isotonic_regression}.

\begin{lemma}
    \label{lemma:PAVA_algorithm_exact_solution}
    The vector $\hat \bnu$ in Algorithm~\ref{alg:PAVA_algorithm} solves~\eqref{obj:KyFan_Huber_isotonic_regression} exactly.
\end{lemma}

Finally, the merging process in Algorithm~\ref{alg:PAVA_algorithm} can be executed efficiently. Intuitively, each element of $\bmu$ is visited at most twice; once during its initial processing and once when it is included in a merged block. This ensures that the process achieves a linear time complexity.

\begin{lemma}
    \label{lemma:PAVA_merging_linear_time_complexity}
    The merging step (lines 11-14) in Algorithm~\ref{alg:PAVA_algorithm} can be performed in linear time complexity $\mathcal O(p)$.
\end{lemma}
% \begin{proof}
%     The merging process in the PAVA algorithm involves iterating through the elements of the input vector $\bmu$ and merging adjacent blocks whenever an isotonic constraint violation is detected. Each element is visited at most twice: once when it is initially processed and once when it is part of a merged block. Therefore, the total number of operations is proportional to the number of elements, resulting in a linear time complexity, $O(n)$.
% \end{proof}

Armed with these lemmas, one can easily prove Theorem~\ref{theorem:pava_algorithm_linear_time_complexity_and_exact_solution}. Details are provided in~\ref{appendix_sec:proofs}.

\vspace{-2mm}
\subsection{FISTA algorithm with restart}
A critical computational step in FISTA is the efficient evaluation of the proximal operator of $g$. By the extended Moreau decomposition theorem~\citep[Theorem~6.45]{beck2017first}, for any weight parameter $\rho > 0$ and any point $\bmu \in \R^p$, the proximal operators of $g$ and $g^*$ satisfies
\begin{align}
    \label{eq:Moreaus_identity}
    \prox_{\rho^{-1} g}(\bmu) = \bmu - \rho^{-1} \, \prox_{\rho g^*} \left( \rho \bmu \right).
\end{align}
Hence, together with Theorem~\ref{theorem:pava_algorithm_linear_time_complexity_and_exact_solution}, we can compute exactly $\prox_{\rho^{-1} g}$ using Algorithm~\ref{alg:PAVA_algorithm} with log-linear time complexity. This enables an efficient implementation of the FISTA algorithm. 
Alas, the vanilla FISTA algorithm is prone to oscillatory behavior, which results in a sub-linear convergence rate of $\mathcal O(1/T^2)$ after $T$ iterations. 
In the following, we further accelerate the empirical convergence performance of the FISTA algorithm by incorporating a simple restart strategy based on the function value, originally proposed in~\citep{o2015adaptive}.

In simple terms, the restart strategy operates as follows: if the objective function increases during the iterative process, the momentum coefficient is reset to its initial value.
The effectiveness of the restart strategy hinges on the efficient computation of the loss function. This task essentially reduces to evaluating the implicit function $g$ defined in~\eqref{eq:function_g_definition}, which would involve solving a SOCP problem.
However, the value of $g$ can be computed efficiently by leveraging the majorization technique~\citep{kim2022convexification}, as shown in Algorithm~\ref{alg:compute_g_value_algorithm}.

\begin{algorithm}[tb]
    \caption{Algorithm to compute $g(\bbeta)$}
    \label{alg:compute_g_value_algorithm}
    \begin{flushleft}
    \textbf{Input:} vectors $\bbeta \in \R^p$ \\ %from Step 2 Line 8 in Algorithm~\ref{alg:main_algorithm}. 
    %\textbf{Output:} The value of $g(\bbeta)$
    \end{flushleft}
    \begin{algorithmic}[1]
        \STATE Initialize: $\bpsi = \boldsymbol{0} \in \mathbb{R}^k$
        \STATE Sort $\bbeta$ partially such that \\ \vspace{0.3em}\hspace*{2em}
        $\vert{\beta_1} \geq \vert{\beta_2} \geq ... \geq \vert{\beta_k} \geq \max\limits_{k+1, ..., p} \{ \vert{\beta_j} \}$
        % Let $\bs$ be the cumulative sum of the absolute values of sorted $\bbeta$.
        \STATE $s = \sum_{j=1}^p \vert{\beta_j}$
        \FOR{$j=1, 2, \dots, k$}
            \STATE $\overline{s} = s / (k - j + 1)$
            \STATE \textbf{if} $\overline{s} \geq \vert{\beta_j}$ \textbf{then} $\psi_{j:k} = \overline{s}$; break \textbf{else} $\psi_j = \vert{\beta_j}$
            \STATE $s = s - \vert{\beta_j}$
        \ENDFOR
        \STATE \textbf{return} $\sum_{j=1}^k \psi_j^2$
    \end{algorithmic}
\end{algorithm}


\begin{theorem}
    \label{theorem:compute_g_value_algorithm_correctness}
        For any $\bbeta \in \R^p$, Algorithm~\ref{alg:compute_g_value_algorithm} computes the exact value of $g(\bbeta)$, defined in~\eqref{eq:function_g_definition}, in $\mathcal O(p + p \log k)$.
\vspace{-3mm}
\end{theorem}
Theorem~\ref{theorem:compute_g_value_algorithm_correctness} guarantees that Algorithm~\ref{alg:compute_g_value_algorithm} can efficiently compute the value of $g(\bbeta)$, which is crucial for our value-based restart strategy to be effective in practice.
Empirically, we observe that the function value-based restart strategy can accelerate FISTA from the sub-linear convergence rate of \(O(1/T^2)\) to a linear convergence rate in many empirical results.
To the best of our knowledge, \textit{this is the first linear convergence result of using a first-order method in the MIP context} when calculating the lower bounds in the BnB tree.
The FISTA algorithm is summarized in Algorithm~\ref{alg:main_algorithm}.

\begin{algorithm}[!b]
    \caption{Main algorithm to solve problem \eqref{obj:original_sparse_problem_perspective_formulation_convex_relaxation}}
    \label{alg:main_algorithm}
    \begin{flushleft}
    \textbf{Input:} number of iterations $T$, coefficient $\lambda_2$ for the $\ell_2$ regularization, and step size $L$ (Lipschitz-continuity parameter of $\nabla F(\bbeta)$)  
    \end{flushleft}
    \begin{algorithmic}[1]
        \STATE Initialize: $\bbeta^0 = \mathbf{0}$, $\bbeta^1 = \mathbf{0}$, $\phi = 1$
        \STATE Let: $\rho = L / (2\lambda_2)$, $\calL^1 = f(\mathbf{\bbeta^1}, \by)$
        \FOR{$t=1, 2, 3, ..., T$}
            \STATE \COMMENT{Step 1: momentum acceleration} 
            \STATE $\bgamma^t = \bbeta^t + \frac{t}{t+3} (\bbeta^t - \bbeta^{t-1})$ \vspace{1.5mm}
            \STATE \COMMENT{Step 2: proximal gradient descent; use \textcolor{red}{Algorithm~\ref{alg:PAVA_algorithm}}} 
            \STATE $\bgamma^t = \bgamma^t - \frac{1}{L} \nabla F(\bgamma^t)$ \label{alg_line:gradient_descent} 
            \STATE $\bbeta^{t+1} = \bgamma^t - \rho^{-1} \text{prox}_{\rho g^*} (\rho \bgamma^t)$ \label{alg_line:proximal_step} \vspace{1.5mm} 
            \STATE \COMMENT{Step 3: restart; use \textcolor{red}{Algorithm~\ref{alg:compute_g_value_algorithm}}}
            \STATE $\calL^{t+1} = f(\bX \bbeta^{t+1}, \by) + 2 \lambda_2 g(\bbeta^{t+1})$
            \STATE \textbf{if} $\calL^{t+1} \geq \calL^{t}$ \textbf{then} $\phi = 1$ \textbf{else} $\phi = \phi + 1$
        \ENDFOR
        \STATE \textbf{return} $\bbeta^{T+1}$
    \end{algorithmic}
\end{algorithm}


\vspace{-3mm}
\subsection{Safe Lower Bounds for GLMs}
\label{subsec:safe_lower_bounds_for_glms}
We conclude this section by commenting on how to use Algorithm~\ref{alg:main_algorithm} in the BnB tree to prune nodes.
As an iterative algorithm, FISTA yields only an approximate solution $\hat{\bbeta}$ to~\eqref{obj:original_sparse_problem_perspective_formulation_convex_relaxation}. Consequently, while we can calculate the objective function for $\hat{\bbeta}$ efficiently, this value is not necessarily a lower bound of the original problem--only the optimal value of the relaxed problem~\eqref{obj:original_sparse_problem_perspective_formulation_convex_relaxation} serves as a guaranteed lower bound.
To get a safe lower bound, we rely on the weak duality theorem, in which for any proper, lower semi-continuous, and convex functions $F: \R^n \to \R \cup \{\infty\}$ and $G: \R^p \to \R \cup \{\infty\}$, we have
\begin{align*}
    \inf_{\bbeta \in \R^p} F(\bX \bbeta) + G(\bbeta) 
    &\geq \sup_{\bzeta \in \R^n} - F^*(-\bzeta) - G^*(\bX^\top \bzeta) \\ 
    &\geq - F^*(-\bzeta) - G^*(\bX^\top \bzeta) \,~ \forall \bzeta \in \R^n\!\!,
\end{align*}
where $F^*$ and $G^*$ denote the conjugates of $F$ and $G$, respectively, while the second inequality follows from the definition of the supremum operator. Letting $F(\bX \bbeta) = f(\bm X \bm \beta, \bm y)$, $G(\bbeta) = 2 \lambda_2 g(\bbeta)$ and $\bzeta = \nabla F(\bX \hat \bbeta)$, where $\hat \bbeta$ is the output of the FISTA Algorithm~\ref{alg:main_algorithm}, we arrive at the safe lower bound
\begin{align}
    \label{eq:fenchel_duality_theorem_F_y(Ax)+G(x)}
    P_{\text{MIP}}^\star \geq - F^*(-\hat{\bzeta}) - G^*(\bX^\top \hat{\bzeta}),
\end{align}
where the inequality follows from the relaxation bound $P_{\text{MIP}}^\star \geq P_{\text{conv}}^\star$ and the weak duality theorem.
For convenience, we provide a list of $F^*(\cdot)$ for different GLM loss functions in~\ref{appendix_sec:convex_conjugate_for_GLM_loss_functions}.
The readers are also referred to~\ref{appendix_sec:safe_lower_bound_more_discussions},  where we derive the safe lower bound for the linear regression problem with eigen-perspective relaxation as an example.

%\section{Related Work}
\noindent\textbf{Prompt engineering.} Instruction tuning aligns LLMs more closely with human instructions \cite{instructGPT,cross_task_generalization, ft_lm_are_zs_learners,general_language_assistant,super_natural_instructions, self_instruct}. Concurrently, numerous prompting strategies have been developed \cite{prompt_survey} and shown to consistently enhance the performance of instruction-tuned LLMs, such as in-context learning \cite{lm_are_few_shot_learner,rethink_demonstration} and chain-of-thought \cite{cot,lm_zero_shot_reasoner}. These prompting strategies are proven effective in multilingual tasks as well \cite{lm_few_shot_multilingual_learner,few_shot_learning_multilingual_lm, mgsm}.

\vspace{2pt}\noindent\textbf{Multilingual ICL.} 
For languages of templates, demonstrations and sample questions in native languages are conventionally inserted into a predefined English template \cite{few_shot_learning_multilingual_lm,polyglot_prompt,cross_lingual_prompting,not_all_language_are_created_equal,mega,plug}.  
\citet{roles_of_english} critiques this widespread misuse of English as the \textit{interface} language. \citet{cross_lingual_prompting,not_all_language_are_created_equal,plug} guide models to ``think'' and generate CoT in English, regardless of the input language, leading to improved performance for generation tasks compared to ``thinking'' in other language(s). \citet{sensitivity_prompt_design,impact_demonstration_multilingual_icl} highlight that models are sensitive to those templates.
For languages of demonstrations and test questions, \citet{mgsm,mega} conclude that in-language demonstrations outperform English demonstrations. \citet{do_llm_think_better_in_english,is_translation_all_you_need} suggest translating questions from LRLs into English can improve performance.


\vspace{-3mm}
\section{Experiments}
\vspace{-1mm}

We evaluate our proposed methods using both synthetic and real-world datasets to address three key empirical questions: \\[-2em]
\begin{itemize}[label=$\diamond$,leftmargin=*]
    \item How fast is our customized PAVA algorithm in evaluating $\prox_g$ compared to existing solvers?
    \item How fast is our proposed FISTA method in calculating the lower bounds compared to existing solvers?
    \item How fast is our customized BnB algorithm compared to existing solvers?
\end{itemize}
We implement our algorithms in python.

For baselines, we compare with the following state-of-the-art commercial and open-source SOCP solvers: Gurobi~\citep{gurobi}, MOSEK~\citep{mosek}, SCS~\citep{scs}, and Clarabel~\cite{Clarabel}, with the python package cvxpy~\cite{cvxpy} as the interface to these solvers.

\vspace{-2mm}
\subsection{How Fast Can We Evaluate $\text{prox}_{\rho^{-1} g}(\cdot)$?}

\begin{figure*}[!htb]
    % \vspace{-0.5em}
    \centering
    \includegraphics[width=0.85\textwidth]{sections/Plots/prox_comparison/prox_comparison.png}
    \vspace{-1em}
    \caption{Running time comparison of evaluating the proximal operators, for both $g$ (left) and $g^*$ (right).
    The baselines evaluate the proximal operators by directly solving the corresponding second-order conic problems (SOCP), respectively.}
    \label{fig:prox_comparison}
    \vspace{-3mm}
\end{figure*}

\begin{figure*}[!htb]
    \centering
    \includegraphics[width=0.85\textwidth]{sections/Plots/big_M_perturbation/convex_relaxation_comparison_n_p_ratio_1.0_M_2.0.png}
    \vspace{-1em}
    \caption{Running time comparison of solving Problem~\eqref{obj:original_sparse_problem_perspective_formulation_convex_relaxation}, the perspective relaxation of the original MIP problem.
    We set $M=2.0$, $\lambda_2=1.0$, and $n$-to-$p$ ratio to be 1. Gurobi cannot solve the cardinality constrained logistic regression problem.}
    \label{fig:solve_convex_relaxation_main_paper}
    \vspace{-5mm}
\end{figure*}

In this subsection, we demonstrate the computational efficiency of using our PAVA algorithm for evaluating the proximal operators.
We conduct the comparisons in two ways --- evaluating both a) the proximal operator of the original function $g$ and b) the proximal operator of its conjugate $g^*$.
Detailed experimental configurations, including parameter specifications and synthetic data generation process, are provided in Appendix~\ref{appendix:setup_for_evaluating_proximal_operators}.

The results shown in Figure~\ref{fig:prox_comparison} highlight the superiority of our method.
Our algorithm achieves a computational speedup of  approximately two orders of magnitude compared to conventional SOCP solvers.
This performance gain is largely due to our customized PAVA implementation in Algorithm~\ref{alg:PAVA_algorithm}.
For instance, in high-dimensional settings ($p=10^5$), baseline methods require several seconds to minutes to evaluate the proximal operators, whereas our approach completes the same task in 0.01 seconds.
Additionally, our method guarantees \textit{exact} solutions to the optimization problem, in contrast to the approximate solutions returned by the baselines.
This combination of precision and efficiency constitutes a critical advantage for our first-order optimization framework over generic conic programming solvers, as demonstrated in subsequent sections.

\vspace{-2mm}
\subsection{How Fast Can We Calculate the Lower Bound?}
\vspace{-1mm}

\begin{figure*}[!htb]
    \centering
    \includegraphics[width=0.95\textwidth]{sections/Plots/RestartedFISTA_linear_convergence_rate/convergence_comparison.png}
    \vspace{-2mm}
    \caption{Empirical convergence rate of our restarted FISTA (compared with PGD, the proximal gradient method, and FISTA) on solving the perspective relaxation in Problem~\eqref{obj:original_sparse_problem_perspective_formulation_convex_relaxation} with the logistic loss, $n=16000, p=16000, k=10, \rho=0.5, \lambda_2=1.0, \text{ and } M=2.0$. }
    \label{fig:RestartedFISTA_linear_convergence_rate}
    \vspace{-3mm}
\end{figure*}


% Please add the following required packages to your document preamble:
% \usepackage{multirow}
% \usepackage{graphicx}
\begin{table}[]
\centering
\vspace{-2mm}
\caption{GPU acceleration of our method on the linear regression task. Top and bottom rows correspond to the mean and standard deviation of running times (seconds).}
\vspace{2mm}
\label{tab:GPU_acceleration}
\resizebox{\columnwidth}{!}{%
\begin{tabular}{cccccc}
\toprule
p & 1k & 2k & 4k & 8k & 16k \\ \hline
\multirow{2}{*}{ours CPU} & 0.19 & 0.48 & 1.54 & 4.80 & 19.52 \\
 & (0.01) & (0.05) & (0.21) & (0.57) & (1.27) \\ \hline
\multirow{2}{*}{ours GPU} & 0.29 & 0.19 & 0.26 & 0.59 & 2.09 \\
 & (0.04) & (0.02) & (0.02) & (0.08) & (0.11)\\
 \bottomrule
\end{tabular}%
}
\vspace{-5mm}
\end{table}

% % Please add the following required packages to your document preamble:
% % \usepackage{graphicx}
% \begin{table*}[!ht]
% \centering
% \caption{}
% \label{tab:my-table}
% \resizebox{\textwidth}{!}{%
% \begin{tabular}{lcccccc}
% \toprule
%  & \multicolumn{2}{c}{ours} & \multicolumn{2}{c}{Gurobi} & \multicolumn{2}{c}{MOSEK} \\
%  & \multicolumn{1}{l}{time (s)} & \multicolumn{1}{l}{optimality gap (\%)} & \multicolumn{1}{l}{time (s)} & \multicolumn{1}{l}{optimality gap (\%)} & \multicolumn{1}{l}{time (s)} & \multicolumn{1}{l}{optimality gap (\%)} \\ \hline
% \begin{tabular}[c]{@{}l@{}}Linear Regression\\ Synthetic \\ (n=16000, p=16000)\end{tabular} & 57 & 0.0 & 3351 & - & 2148 & - \\ \hline
% \begin{tabular}[c]{@{}l@{}}Linear Regression\\ Cancer Drug Response\\ (n=822, p=2300)\end{tabular} & 47 & 0.0 & 1800 & 0.31 & 212 & 0.0 \\ \hline
% \begin{tabular}[c]{@{}l@{}}Logistic Regression\\ Synthetic\\ (n=16000, p=16000)\end{tabular} & 271 & 0.0 & N/A & N/A & 1800 & - \\ \hline
% \begin{tabular}[c]{@{}l@{}}Logistic Regression\\ Dorothea\\ (n=1150, p=91598)\end{tabular} & 62 & 0.0 & N/A & N/A & 600 & 0.0 \\
% \bottomrule
% \end{tabular}%
% }
% \end{table*}

% Please add the following required packages to your document preamble:
% \usepackage{multirow}
% \usepackage{graphicx}
\begin{table*}[]
\centering
\caption{Certifying optimality on large-scale and real-world datasets.}
\vspace{2mm}
\label{tab:my-table}
\resizebox{\textwidth}{!}{%
\begin{tabular}{llcccccc}
\toprule
 &  & \multicolumn{2}{c}{ours} & \multicolumn{2}{c}{Gurobi} & \multicolumn{2}{c}{MOSEK} \\
 &  & time (s) & opt. gap (\%) & time (s) & opt. gap (\%) & time (s) & opt. gap (\%) \\ \hline
\multirow{2}{*}{Linear Regression} & \begin{tabular}[c]{@{}l@{}}synthetic ($k=10, M=2$)\\ (n=16k, p=16k, seed=0)\end{tabular} & 79 & 0.0 & 1800 & - & 1915 & - \\ \cline{2-8}
 & \begin{tabular}[c]{@{}l@{}}Cancer Drug Response ($k=5, M=5$)\\ (n=822, p=2300)\end{tabular} & 41 & 0.0 & 1800 & 0.89 & 188 & 0.0 \\ \hline
\multirow{2}{*}{Logistic Regression} & \begin{tabular}[c]{@{}l@{}}Synthetic ($k=10, M=2$)\\ (n=16k, p=16k, seed=0)\end{tabular} & 626 & 0.0 & N/A & N/A & 2446 & - \\ \cline{2-8}
 & \begin{tabular}[c]{@{}l@{}}DOROTHEA ($k=15, M=2$)\\ (n=1150, p=91598)\end{tabular} & 91 & 0.0 & N/A & N/A & 634 & 0.0 \\
 \bottomrule
\end{tabular}%
}
% \vspace{-3mm}
\end{table*}
We next benchmark the computational speed and scalability of our method against the state-of-the-art solvers (Gurobi, MOSEK, SCS, and Clarabel) for solving the perspective relaxation of the original MIP problem.
Evaluations are performed on both linear and logistic regression tasks.


Experimental configurations are detailed in Appendix~\ref{appendix:setup_for_solving_the_perspective_relaxation}.
Additional perturbation studies, such as on the sample-to-feature ($n$-to-$p$) ratio, box constraint $M$, and $\ell_2$ regularization coefficient $\lambda_2$, are provided in Appendix~\ref{appendix:numerical_solve_convex_relaxation}.
All solvers are terminated upon achieving an optimality gap tolerance of $\epsilon=10^{-6}$ or exceeding a runtime limit of 1800 seconds.

The results, shown in Figure~\ref{fig:solve_convex_relaxation_main_paper}, demonstrates that our method outperforms the fastest conic solver (MOSEK) by over one order of magnitude.
For the largest tested instances ($n=16000$ and $p=16000$), our approach attains the target tolerance ($10^{-6}$) in under 100 seconds across regression and classification datasets, whereas most baselines fail to converge within the 1800-second threshold.

There are two factors driving this speedup.
First, our efficient proximal operator evaluation reduces per-iteration complexity.
Second, our efficient method to compute $g(\bbeta)$ (in Algorithm~\ref{alg:compute_g_value_algorithm}) exactly enables integration of the value-based restart technique within FISTA, significantly improving convergence.
Figure~\ref{fig:RestartedFISTA_linear_convergence_rate} illustrate this enhancement: while the proximal gradient algorithm (PGD) and FISTA exhibit sublinear convergence rates, FISTA with restarts achieves linear convergence on both dual loss and primal-dual gap metrics.
To the best of our knowledge, this marks the first empirical demonstration of linear convergence for a first-order method applied to solving the convex relaxation of this MIP class.

Finally, our method permits GPU acceleration because our most computationally intensive component is matrix-vector multiplications.
As shown in Table~\ref{tab:GPU_acceleration}, GPU implementation reduces runtime by an additional order of magnitude on high-dimensional instances.



\vspace{-0mm}
\subsection{How Fast Can We Certify Optimality?}
Finally, we demonstrate how our method's ability to compute tight lower bounds enables efficient optimality certification for large-scale datasets, outperforming state-of-the-art commericial MIP solvers.
Integrating our lower-bound computation into a minimalist branch-and-bound (BnB) framework, we prioritize node pruning via lower bound calculations while intentionally omitting advanced MIP heuristics (e.g., cutting planes, presolve routines) to evaluate the impact of our method.
Experimental configurations, including dataset descriptions and BnB implementation details, are provided in Appendix~\ref{appendix:setup_for_certifying_optimality}.
We benchmark our approach against Gurobi and MOSEK, reporting both runtime and final optimality gaps.


Results in Table~\ref{tab:my-table} show that our method certifies optimality for three of the four tested datasets within 2 minutes and the fourth within 10 minutes.
In contrast, Gurobi and MOSEK either exceed the time limit (1800 seconds) during the presolve stage or require significantly longer runtimes to achieve zero or small gaps.
Crucially, this efficiency stems from our efficient lower-bound computations and dynamic early termination criteria.
Specifically, we avoid waiting for full convergence by leveraging two key rules: 
(1) if the primal loss falls below the incumbent solution’s loss, we terminate early and proceed to branching; 
(2) if the dual loss exceeds the incumbent’s loss, we halt computation and prune the node immediately. This adaptive approach eliminates unnecessary iterations while ensuring we prune the search space effectively.








\vspace{-3.5mm}
\section{Conclusion}
\vspace{-2mm}

In summary, we introduce a first-order proximal algorithm to solve the perspective relaxation of cardinality-constrained GLM problems.
By leveraging the problem’s unique mathematical structure, we design a customized PAVA to efficiently evaluate the proximal operator, ensuring scalability to high-dimensional settings.
Further acceleration is achieved through an efficient value-based restart strategy and compatibility with GPUs, which collectively enhance convergence rates and computational speed.
Extensive empirical results demonstrate that our method outperforms state-of-the-art solvers by 1-2 orders of magnitude, establishing it as a practical, high-performance component for integration into next-generation MIP solvers.



\bibliographystyle{preprint}
\bibliography{references}


%%%%%%%%%%%%%%%%%%%%%%%%%%%%%%%%%%%%%%%%%%%%%%%%%%%%%%%%%%%%%%%%%%%%%%%%%%%%%%%
%%%%%%%%%%%%%%%%%%%%%%%%%%%%%%%%%%%%%%%%%%%%%%%%%%%%%%%%%%%%%%%%%%%%%%%%%%%%%%%
% APPENDIX
%%%%%%%%%%%%%%%%%%%%%%%%%%%%%%%%%%%%%%%%%%%%%%%%%%%%%%%%%%%%%%%%%%%%%%%%%%%%%%%
%%%%%%%%%%%%%%%%%%%%%%%%%%%%%%%%%%%%%%%%%%%%%%%%%%%%%%%%%%%%%%%%%%%%%%%%%%%%%%%
\newpage


\appendix

\section{Ice-breaker questions}
\label{appendix:ice-breaker}
Please go through these questions together, you do not need to answer them all.
\begin{itemize}
    \item Share [University Name] Introductions 
    \item What's the last TV show or movie you watched and enjoyed?
    \item Do you have any pets, and if not, what kind of pet would you like to have?
    \item If you could travel anywhere in the world, where would you go and why?
    \item What was your dream job as a kid?
    \item What's your favorite type of food, and why do you love it?
    \item What's a hobby or activity you enjoy doing in your free time?
    \item What kind of music do you like to listen to, and do you have any favorite artists?
    \item What's a skill or talent you wish you had, and why?
    \item What's the best piece of advice you've ever received, and did you follow it?
\end{itemize}










%%%%%%%%%%%%%%%%%%%%%%%%%%%%%%%%%%%%%%%%%%%%%%%%%%%%%%%%%%%%%%%%%%%%%%%%%%%%%%%
%%%%%%%%%%%%%%%%%%%%%%%%%%%%%%%%%%%%%%%%%%%%%%%%%%%%%%%%%%%%%%%%%%%%%%%%%%%%%%%


\end{document}


% This document was modified from the file originally made available by
% Pat Langley and Andrea Danyluk for xxx-2K. This version was created
% by Iain Murray in 2018, and modified by Alexandre Bouchard in
% 2019 and 2021 and by Csaba Szepesvari, Gang Niu and Sivan Sabato in 2022.
% Modified again in 2023 and 2024 by Sivan Sabato and Jonathan Scarlett.
% Previous contributors include Dan Roy, Lise Getoor and Tobias
% Scheffer, which was slightly modified from the 2010 version by
% Thorsten Joachims & Johannes Fuernkranz, slightly modified from the
% 2009 version by Kiri Wagstaff and Sam Roweis's 2008 version, which is
% slightly modified from Prasad Tadepalli's 2007 version which is a
% lightly changed version of the previous year's version by Andrew
% Moore, which was in turn edited from those of Kristian Kersting and
% Codrina Lauth. Alex Smola contributed to the algorithmic style files.
