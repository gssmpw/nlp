
\documentclass{article}

% Recommended, but optional, packages for figures and better typesetting:
\usepackage{microtype}
\usepackage{graphicx}
\usepackage{subfigure}
\usepackage{booktabs} % for professional tables
\usepackage{enumitem} % for customizing enumerate environments


% hyperref makes hyperlinks in the resulting PDF.
% If your build breaks (sometimes temporarily if a hyperlink spans a page)
% please comment out the following usepackage line and replace
\usepackage{hyperref}


% Attempt to make hyperref and algorithmic work together better:
\newcommand{\theHalgorithm}{\arabic{algorithm}}

% Use the following line for the initial blind version submitted for review:

% If accepted, instead use the following line for the camera-ready submission:
\usepackage{preprint}

% For theorems and such
\usepackage{amsmath, amssymb, mathtools, amsthm, bm}

% if you use cleveref..
%\usepackage[capitalize,noabbrev]{cleveref}

%%%%%%%%%%%%%%%%%%%%%%%%%%%%%%%%
% THEOREMS
%%%%%%%%%%%%%%%%%%%%%%%%%%%%%%%%
\theoremstyle{plain}
\newtheorem{theorem}{Theorem}[section]
\newtheorem{proposition}[theorem]{Proposition}
\newtheorem{lemma}[theorem]{Lemma}
\newtheorem{corollary}[theorem]{Corollary}
\theoremstyle{definition}
\newtheorem{definition}[theorem]{Definition}
\newtheorem{assumption}[theorem]{Assumption}
\theoremstyle{remark}
\newtheorem{remark}[theorem]{Remark}

% Todonotes is useful during development; simply uncomment the next line
%    and comment out the line below the next line to turn off comments
%\usepackage[disable,textsize=tiny]{todonotes}
%\usepackage[textsize=tiny]{todonotes}


% The \papertitle you define below is probably too long as a header.
% Therefore, a short form for the running title is supplied here:
\titlerunning{Scalable First-order Method for Certifying Optimal k-Sparse GLMs}


\newcommand{\eop}{{$\blacksquare$}}
\newcommand{\eod}{{${}$\\}}


\newcommand{\cI}{\mathcal{I}}
\newcommand{\A}{\mathcal A}
\newcommand{\anc}{\mathcal A}
\newcommand{\B}{{\mathrm B}}
\newcommand{\bB}{{\mathbb B}}
\newcommand{\cB}{{\mathcal B}}
\newcommand{\bI}{\mathbb{I}}
\newcommand{\R}{{\mathbb R}}
\newcommand{\N}{{\mathbb N}}
%\newcommand{\C}{{\mathbb C}}
\newcommand{\bP}{\mathbb{P}}
\newcommand{\cH}{{\cal H}}
\newcommand{\cA}{{\cal A}}
\newcommand{\cY}{{\cal Y}}
\newcommand{\cW}{{\cal W}}
\newcommand{\cX}{{\cal X}}
\newcommand{\cZ}{{\cal Z}}
\newcommand{\cN}{\mathcal{N}}
\newcommand{\ci}{{\mathfrak C}}
\newcommand{\bE}{\mathbf{E}}
\newcommand{\cS}{\mathcal{S}}
\newcommand{\cT}{\mathcal{T}}
\newcommand{\cM}{{\cal M}}
\newcommand{\cF}{{\cal F}}
\newcommand{\cG}{{\cal G}}
\newcommand{\cC}{{\cal C}}
\newcommand{\cP}{{\cal P}}
\newcommand{\cQ}{{\cal Q}}
\newcommand{\cL}{\mathcal{L}}
\newcommand{\<}{\langle}
\newcommand{\Z}{\mathbb{Z}}
\newcommand{\Ceq}{\stackrel{+}{=}}
\newcommand{\id}{{\bf I}}

\newcommand{\bx}{{\bf x}}
\newcommand{\bX}{{\bf X}}
\newcommand{\bY}{{\bf Y}}
\newcommand{\bZ}{{\bf Z}}
\newcommand{\bN}{{\bf N}}
\newcommand{\bV}{{\bf V}}
\newcommand{\cov}{\operatorname{Cov}}
\newcommand{\var}{\operatorname{Var}}
\newcommand{\dd}{\mathrm{d}}
\newcommand{\bz}{{\bf z}}
\newcommand{\by}{{\bf y}}
\newcommand{\bW}{{\bf W}}
\newcommand{\bC}{{\bf C}}
\newcommand{\An}{{An}}
\newcommand{\Adj}{\operatorname{Adj}}
\newcommand{\PA}{\operatorname{PA}}
\newcommand{\pa}{{pa}}

\newcommand{\MEQ}{\operatorname{MEQ}}
\newcommand{\CI}{\operatorname{CI}}

\newcommand\independent{\protect\mathpalette{\protect\independenT}{\perp}}
\def\independenT#1#2{\mathrel{\rlap{$#1#2$}\mkern2mu{#1#2}}}
\newcommand{\ind}{\independent}

\newcommand{\MD}{\operatorname{MD}}

\usepackage{etoolbox}
%\usepackage{mathabx}

\newtoggle{printcomments}
\toggletrue{printcomments} % Set the flag to true by default.
\togglefalse{printcomments} % Uncomment this to remove all comments in the doc

\newcommand\dominik[1]{%
  \iftoggle{printcomments}{%
    \textcolor{cyan}{Dominik: #1}%
  }{}%
}


\newcommand\philipp[1]{%
  \iftoggle{printcomments}{%
\textcolor{red}{Philipp: #1} % Notes for Leena
  }{}%
}


\newcommand\todo[1]{%
  \iftoggle{printcomments}{%
\textcolor{red}{#1} % Notes for Leena
  }{}%
}


\begin{document}

\twocolumn[
\papertitle{Scalable First-order Method for Certifying Optimal k-Sparse GLMs}

% It is OKAY to include author information, even for blind
% submissions: the style file will automatically remove it for you
% unless you've provided the [accepted] option to the xxx
% package.

% List of affiliations: The first argument should be a (short)
% identifier you will use later to specify author affiliations
% Academic affiliations should list Department, University, City, Region, Country
% Industry affiliations should list Company, City, Region, Country

% You can specify symbols, otherwise they are numbered in order.
% Ideally, you should not use this facility. Affiliations will be numbered
% in order of appearance and this is the preferred way.

% \setsymbol{equal}{*}

\begin{authorlist}
\paperauthor{Jichang Liu}{yyy}
\paperauthor{Soroosh Shafiee}{yyy}
\paperauthor{Andrea Lodi}{comp}
\end{authorlist}

\affiliation{yyy}{School of Operations Research and Information Engineering, Cornell University, Ithaca, NY, USA}
\affiliation{comp}{Jacobs Technion-Cornell Institute, Cornell Tech and Technion–IIT, New York, NY, USA}

\displayemail{\{jiachang.liu, shafiee, al748\}@cornell.edu}


% You may provide any keywords that you
% find helpful for describing your paper; these are used to populate
% the "keywords" metadata in the PDF but will not be shown in the document
\keywords{Machine Learning, Optimization}

\vskip 0.3in
]

% this must go after the closing bracket ] following \twocolumn[ ...

% This command actually creates the footnote in the first column
% listing the affiliations and the copyright notice.
% The command takes one argument, which is text to display at the start of the footnote.
% The \xxxEqualContribution command is standard text for equal contribution.
% Remove it (just {}) if you do not need this facility.

\printAffiliationsAndNotice{}  % leave blank if no need to mention equal contribution
%\printAffiliationsAndNotice{\xxxEqualContribution} % otherwise use the standard text.

%%%%% To create table of contents only for Appendix %%%%%%

%\doparttoc % Tell to minitoc to generate a toc for the parts
%\faketableofcontents % Run a fake tableofcontents command for the partocs

% 150words
% Replying to workplace emails that are typically long and require politeness is time-consuming and cognitively demanding.
% Replying to lengthy and polite workplace emails is often time-consuming and cognitively demanding.
% takes time to understand and reply
\red{Replying to formal emails is time-consuming and cognitively demanding, as it requires crafting polite phrasing and providing an adequate response to the sender's demands.}
% \red{Replying to formal emails, which often takes time to understand and require polite phrasing, is time-consuming and cognitively demanding.}
Although systems with Large Language Models (LLM) were designed to simplify the email replying process, users still need to provide detailed prompts to obtain the expected output.
Therefore, we proposed and evaluated an \red{LLM-powered question-and-answer (QA)-based approach} for users to reply to emails by answering a set of simple and short questions generated from the incoming email.
We developed a prototype system, \textit{ResQ}, and conducted controlled and field experiments with 12 and \red{8} participants.
Our results demonstrated that \red{the QA-based approach} improves the efficiency of replying to emails and reduces workload while maintaining email quality, compared to a conventional prompt-based approach that requires users to craft appropriate prompts to obtain email drafts.
We discuss how \red{the QA-based approach} influences the email reply process and interpersonal relationship dynamics, as well as the opportunities and challenges associated with using a QA-based approach in AI-mediated communication.

% original
% Replying to lengthy and polite workplace emails is often time-consuming and cognitively demanding.
% Although systems with Large Language Models were designed to simplify the email replying process, users still needed to provide detailed prompts to obtain the expected output.
% Therefore, we proposed and evaluated a question-and-answer-based approach for users to reply to emails by answering a set of simple and short questions generated from the incoming email.
% We developed a prototype system, \textit{ResQ}, and conducted both controlled and field experiments with 12 and 9 participants.
% Our results demonstrated that ResQ improves the efficiency of replying to emails and reduces workload while maintaining email quality compared to a conventional prompt-based approach that requires users to craft appropriate prompts to obtain email drafts.
% We discuss how ResQ influences the email reply process and interpersonal relationship dynamics, as well as the opportunities and challenges associated with using a QA-based approach in AI-mediated communication.

\section{Introduction}

Sparse generalized linear models (GLMs) are essential tools in machine learning (ML), widely applied in fields like healthcare, finance, engineering, and science. 
These models provide a flexible framework for capturing relationships between variables while ensuring interpretability, which is critical in high-stakes applications.
Recently, using the $\ell_0$ norm to induce sparsity has gained significant attention. This approach provides distinct advantages over traditional convex relaxation methods, such as replacing $\ell_0$ with $\ell_1$, particularly in cases involving highly correlated features.

In this paper, we aim to solve
\begin{align} \label{obj:original_sparse_problem}
    \begin{array}{cl}
        \min\limits_{\bbeta \in \R^p} & f(\bX \bbeta, \by) + \lambda_2 \lVert \bbeta \rVert_2^2 \\
        \st & \| \bbeta \|_\infty \leq M, ~ \lVert \bbeta \rVert_0 \leq k,
    \end{array}
\end{align}
where $\bX \in \R^{n \times p}$ and $\by \in \R^n$ denote the matrix of features and the vector of labels, respectively, while the parameter $M > 0$ can be either user-defined based on prior knowledge or estimated from the data~\citep{park2020subset}. The GLM loss function, denoted by $f : \R^n \times \R^n \to \R$, is assumed to be Lipschitz smooth, the parameter $k \in \mathbb N$ controls the number of nonzero coefficients, and $\lambda_2 > 0$ is a small Tikhonov regularization coefficient to address collinearity.
Alas, problem~\eqref{obj:original_sparse_problem} is NP-hard~\citep{natarajan1995sparse}. 
As a result, most existing methods rely on heuristics that deliver high-quality approximations but lack guarantees of optimality. 
This limitation is particularly problematic in high-stakes applications like healthcare, where ensuring accuracy, reliability, and safety is essential. 
Therefore, we emphasize the pursuit of certifiably optimal solutions.

A naive approach to solve~\eqref{obj:original_sparse_problem} to optimality is to reformulate it as a mixed-integer programming (MIP) problem and leverage commercial MIP solvers.
However, these solvers face significant scalability challenges, particularly with large datasets and nonlinear objectives. 
A major bottleneck arises from the need to compute tight lower bounds at each node of the branch-and-bound (BnB) tree, a critical component for efficient pruning and solver performance.
Existing methods for computing lower bounds typically rely on linear programming or conic optimization techniques.
However, these approaches either generate loose bounds that reduce pruning efficiency or result in high computational costs per iteration. 
Moreover, they are challenging to parallelize, which limits the potential to take advantage of modern hardware accelerators like GPUs.

To address these challenges, we propose a scalable first-order method for efficiently calculating lower bounds within the BnB framework.
We begin with a perspective reformulation of~\eqref{obj:original_sparse_problem} and derive its continuous relaxation.
The resulting formulation is then expressed as an unconstrained optimization problem, characterized by a convex composite objective function, which enables the application of the Fast Iterative Shrinkage-Thresholding Algorithm (FISTA), a well-known first-order method~\citep{beck2009fast}, to compute lower bounds.
The successful implementation of FISTA, however, relies on efficient computation of the proximal operator, which requires solving a second order cone program (SOCP) problem.
To the best of our knowledge, the efficient computation of this proximal operator has not been previously addressed in the literature. 
Therefore, we propose a customized pooled-adjacent-violation algorithm (PAVA) that evaluates the proximal operator exactly with log-linear time complexity, ensuring the scalability of our FISTA approach for large problem instances.
% Our method achieves three core objectives: a) fast convergence rate, b) low per-iteration computational complexity, and c) compatibility with GPU acceleration.
A major advantage of our approach is its computational efficiency, in which instead of solving costly linear systems, it only relies on matrix-vector multiplication, which is highly amenable to GPU acceleration.
This capability addresses a key limitation of existing approaches that struggle to parallelize their computations on modern hardware.

To accelerate the performance of the FISTA algorithm, we introduce a restart heuristic. 
This leads to an empirical linear convergence rate, a result not previously achieved by other first-order methods for this type of problem.
Empirically, our method demonstrates substantial speedups in computing dual bounds -- often by 1-2 orders of magnitude -- compared to existing techniques. 
These improvements significantly enhance the overall efficiency of the BnB process, enabling the certification of large-scale instances of~\eqref{obj:original_sparse_problem} that were previously intractable using commercial MIP solvers. All omitted proofs are provided in~\ref{appendix_sec:proofs}. Additional numerical results are reported in~\ref{appendix:numerical}.

\subsection{Contributions}
The key contributions of this paper are summarized below.
\begin{itemize}[label=$\diamond$,leftmargin=*]
    \item We propose a FISTA-based first-order method to enhance the scalability of solving~\eqref{obj:original_sparse_problem}, with a focus on efficient lower-bound computation within the BnB framework.
    \item The proximal operator in the FISTA method is computed using a customized PAVA that leverages hidden mathematical structures and enjoys log-linear time complexity, ensuring scalability for large-scale problems.
    \item Besides achieving fast convergence rates (via a restart strategy) and low per-iteration computational complexity, our method can be easily parallelized on GPUs, something not currently achievable by MIP methods.
    \item We validate the practical efficiency of our approach on both synthetic and real-world datasets, demonstrating substantial speedups in computing dual bounds and certifying optimal solutions for large-scale sparse GLMs.
    % highlighting its broader impact on optimization and machine learning.
\end{itemize}

\subsection{Related Works}
\label{sec:related_work}

\paragraph{MIP for ML.}
MIP has been successfully applied in
medical scoring systems~\citep{ustun2016supersparse, ustun2019learning, liu2022fasterrisk}, 
portfolio optimization \citep{bienstock1996computational,wei2022convex}, nonlinear identification systems~\citep{bertsimas2023learning, liu2024okridge},
decision trees~\citep{bertsimas2017optimal, hu2019optimal},
survival analysis~\citep{zhang2023optimal, liu2024fastsurvival},
hierarchical models~\citep{bertsimas2020sparse}, regression and classification models~\citep{atamturk2020safe, bertsimas2020sparse, bertsimas2020sparse1, bertsimas2020sparse2, hazimeh2020fast, xie2020scalable, atamturk2021sparse, dedieu2021learning, hazimeh2022sparse, liu2024okridge, guyard2024el0ps}, graphical models \citep{manzour2021integer, kucukyavuz2023consistent}, and outlier detection \citep{gomez2021outlier,gomez2023outlier}.
The primary focus of these works is on obtaining high-quality feasible solutions, with only a small subset addressing the certification of optimality.
Our work aims to contribute to this literature, with a strong focus on enhancing the computational scalability of certifying optimality for solving sparse GLM problems.

\paragraph{Perspective Formulations.} 
The application of perspective functions to derive convex relaxations for~\eqref{obj:original_sparse_problem} dates back to the seminal work of \citet{ceria1999convex}.
Perspective formulations have been developed for separable functions in \citep{gunluk2010perspective, xie2020scalable, wei2022ideal, bacci2019new, shafiee2024constrained} and for rank-one functions in \citep{atamturk2020supermodularity, wei2020convexification, wei2022ideal, han2021compact, shafiee2024constrained} under various conditions. 
Our work uses perspective formulations of separable functions that appear in~\eqref{obj:original_sparse_problem} as the Tikhonov regularization function. 

\paragraph{Lower Bound Calculation.}
A key aspect of certifying optimality in MIP problems is the efficient computation of tight lower bounds.
Commercial MIP solvers typically iteratively linearize the objective function using the celebrated outer approximation method~\citep{kelley1960cutting} (via cutting planes) and solve the resulting linear program~\citep{schrijver1998theory, wolsey2020integer}.
However, this approach often produce loose lower bounds, especially when high-quality linear cuts are not generated.
Alternatively, solvers may use conic convex relaxations and solve them with the interior-point method (IPM)~\citep{dikin1967iterative, renegar2001mathematical, nesterov1994interior}. 
While this approach often yields tighter lower bounds, IPM does not scale well due to its reliance on second-order information and because -- differently from the linear case -- effectively warm-starting IPMs is not possible.
Recent attempts are based on first-order methods, including subgradient descent~\citep{bertsimas2020sparse1}, ADMM~\citep{liu2024okridge}, and coordinate descent~\citep{hazimeh2022sparse}. Our work builds on this, offering faster convergence, low computational complexity, and significant GPU acceleration. We also observe that our proposed FISTA method achieves linear convergence rates empirically, a result not previously achieved by other first-order methods for this problem.


\paragraph{GPU Acceleration.}
Recently, there have been some promising works on using GPUs to accelerate continuous optimization problems, including linear programming~\citep{applegate2021practical, lu2023cupdlp}, quadratic programming~\citep{lu2023practical}, and semidefinite programming~\citep{han2024accelerating}.
%There are few works of using GPUs to solve discrete problems.
A natural way to leverage GPUs for discrete problems is by using GPU-based LPs within MIP solvers, as demonstrated by~\citet{de2024power} for solving clustering problems.
However, in \citep{de2024power}, the challenge is to approximate the original objective function with a potentially exponential number of cutting planes. %their proposed cutting-plane approach faces challenges when it requires iteratively generating an exponential number of cuts to approximate the original objective function. 
In contrast, we develop a customized FISTA method that directly handles the nonlinear objective function, while the computation can be easily parallelized since it only involves matrix-vector multiplication. 
Other first-order methods, such as ADMM~\citep{liu2024okridge} and coordinate descent~\citep{hazimeh2022sparse}, are unsuitable for GPUs: ADMM requires solving linear systems, while coordinate descent is inherently sequential.



\section{Problem Formulation}
In this preliminary section, we introduce some backgrounds on how to obtain a lower bound (which will be used for the branch-and-bound process to prune nodes) for the optimal value of Problem~\eqref{obj:original_sparse_problem} by solving an associated convex relaxation problem.
First, note that we can cast problem~\eqref{obj:original_sparse_problem}~as 
\begin{align}
    \label{epigraph:formulation}
    \min \left\{ \tau \,:\, (\tau, \bbeta, \bz) \in \mathcal S \right\},
\end{align}
where the extended feasible set is defined as
\begin{align}
    \label{eq:S}
    \mathcal{S} = \left\{ (\tau, \bbeta, \bz)  \;\middle|\;
    \begin{array}{l} 
        \| \bbeta \|_\infty \leq M, \\
        \bz \in \{0, 1\}^p, \, \mathbf{1}^T \bz \leq k, \\
        \beta_j ( 1 - z_j) = 0 ~~ \forall j \in [p] \\
        f(\bX \bbeta, \by) + \lambda_2 \| \bbeta \|_2^2 \leq \tau
    \end{array}
    \right\},
\end{align}
and $[p] = \{1, \dots, p \}$ stands for the set of all integers up to $p \in \mathbb N$.
Put it differently, each binary variable $z_j$ indicates whether a continuous variable $\beta_j$ is zero or not by requiring $\beta_j = 0$ when $z_j = 0$ and allowing $\beta_j$ to take any value when $z_j = 1$. 
Meanwhile, the objective function is linearized using the epigraph reformulation technique, which allows us to interpret the optimal value of~\eqref{epigraph:formulation} as the evaluation of the support function of $\mathcal S$ at $(\bm 0, \bm 0, 1)$.
By virtue of \citep[\S13]{rockafellar1970convex}, the optimal value of~\eqref{epigraph:formulation} remains unchanged if we replace $\mathcal S$ with $\cl \conv(\mathcal S)$, where $\cl \conv(\mathcal S)$ denotes the closed convex hull of $\mathcal S$. 
Alas, the exact description of $\cl \conv(\mathcal S)$ requires exponentially many (nonlinear) constraints, which leads to the NP-hardness of~\eqref{obj:original_sparse_problem}.

We thus explore other options for a convex relaxation of~\eqref{obj:original_sparse_problem}.
It turns out that a tractable convex hull can be obtained if the objective function only includes the Tikhonov regularization term $\| \bbeta \|_2^2$, using the perspective function.
The perspective function of the quadratic function $h(\beta) = \beta^2$ is $h^\pi(\beta, z) = \beta^2 / z$ if $z > 0$, $= 0$ if $\beta = z = 0$, and $= \infty$ otherwise.
For simplicity, we write $\beta^2/z$ instead of $h^\pi(\beta, z)$ even if $z = 0$. 
The following lemma provides an exact perspective formulation of the convex hull when $\mathcal S$ does not include $f(\bX \bbeta, \by)$. This result extends \citep[Lemma~6]{gunluk2010perspective} by incorporating sparsity constraints, while also extending \citep[Theorem~2]{shafiee2024constrained} to account for $\ell_\infty$ box constraint on $\bbeta$.
\begin{lemma}
    \label{lemma:equivalence_between_perspective_relaxation_and_convexification}
    The closed convex hull of the set
    \begin{align*}
        \left\{ (\tau, \bbeta, \bz) \middle|
        \begin{array}{l}
            \| \bbeta \|_\infty \leq M, \,  \\
            \bz \in \{0, 1\}^p, \, \mathbf{1}^T \bz \leq k, \\ \beta_j ( 1 - z_j) = 0 ~~ \forall j \in [p], \\
            \sum_{j \in [p]} \beta_j^2 \leq \tau
        \end{array}
        \right\}
    \end{align*}
    is given by the set
    \begin{align*}
        \left\{ (\tau, \bbeta, \bz)  \;\middle|\;
        \begin{array}{l} 
            -M z_j\leq \bbeta_j \leq M z_j ~ \forall j \in [p], \\
            \bz \in [0, 1]^p, \, \mathbf{1}^T \bz \leq k, \\
            \sum_{j \in [p]} \beta_j^2 / z_j \leq \tau
        \end{array}
        \right\}.
    \end{align*}
\end{lemma}
The convex hull formulation presented in Lemma~\ref{lemma:equivalence_between_perspective_relaxation_and_convexification} is a second-order conic set.
Specifically, the epigraph of the sum of perspective functions in the last line satisfies
\begin{align*}
    \sum_{j \in [p]} {\beta_j^2}/{z_j} \leq \tau \iff \exists \bt \in \R_+^p ~ \st ~ 
    \begin{cases}
        \bm 1^\top \bm t = \tau, \\
        \beta_j^2 \leq z_j t_j ~ \forall j \in [p],
    \end{cases}
\end{align*}
which is second order cone representable.
Motivated by Lemma~\ref{lemma:equivalence_between_perspective_relaxation_and_convexification}, we immediately see that the extended feasible set $\mathcal S$ defined in~\eqref{eq:S} admits the following perspective representation
\begin{align*}
% \label{eq:perspective:S}
    \mathcal{S} = \left\{ (\tau, \bbeta, \bz)  \;\middle|\;
    \begin{array}{l} 
        -M z_j\leq \bbeta_j \leq M z_j ~ j \in [p], \\
        \bz \in \{0, 1\}^p, \, \mathbf{1}^T \bz \leq k,  \\
        f(\bX \bbeta, \by) + \lambda_2 \sum_{j \in [p]} \beta_j^2 / z_j \leq \tau
    \end{array}
    \right\}.
\end{align*}
Plutting in this new perspective representation into Problem~\eqref{epigraph:formulation}, we can reformulate~\eqref{obj:original_sparse_problem} as follows
\begin{align}
    \label{obj:original_sparse_problem_perspective_formulation}
    P_{\text{MIP}}^\star = \left\{
    \begin{array}{cll}
        \min\limits_{\bbeta, \bz \in \R^p} & f(\bX \bbeta, \by) + \lambda_2 \sum_{j \in [p]} {\beta_j^2}/{z_j} \\[1ex]
        \text{\; s.t.} & \bz \in \{0, 1\}^p, \, \mathbf{1}^T \bz \leq k, \\[1ex]
        & -M z_j \leq \beta_j \leq M z_j ~ \forall j \in [p].
    \end{array}
    \right.
\end{align}
By relaxing the binary variables $z_j$ to the interval $[0, 1]$, we obtain the following strong convex relaxation of~\eqref{obj:original_sparse_problem_perspective_formulation}
\begin{align}
    \label{obj:original_sparse_problem_perspective_formulation_convex_relaxation}
    P_{\text{conv}}^\star = \left\{
    \begin{array}{cll}
        \min\limits_{\bbeta, \bz \in \R^p} & f(\bX \bbeta, \by) + \lambda_2 \sum_{j \in [p]} {\beta_j^2}/{z_j} \\[1ex]
        \text{\; s.t.} & \bz \in [0, 1]^p, \, \mathbf{1}^T \bz \leq k, \\[1ex]
        & -M z_j \leq \beta_j \leq M z_j ~ \forall j \in [p].
    \end{array}
    \right.
\end{align}
Although this is not the convex hull formulation due to the term $f(\bX \bbeta, \by)$, unlike in Lemma~\ref{lemma:equivalence_between_perspective_relaxation_and_convexification}, $P^\star_{\text{conv}}$ still provides a lower bound for Problem~\eqref{obj:original_sparse_problem}.


We can solve~\eqref{obj:original_sparse_problem_perspective_formulation_convex_relaxation} using standard conic optimization solvers like Mosek and Gurobi, which rely on IPMs for solving such subproblems in the BnB framework.
However, IPMs are computationally expensive and do not scale well for large datasets.
Alternatively, first-order conic solvers such as SCS~\cite{o2016conic}, based on ADMM, can be used.
While these methods are more scalable, they suffer from slow convergence rates and require solving linear systems at each iteration, which can also be computationally intensive for large instances. The main goal of the paper is to introduce an efficient and scalable first-order method to address these limitations.

\section{Iterative Data-based V-model}\label{03_Methodology}

Even though the existing development and V\&V frameworks for emerging complex systems, particularly automated vehicles, differ in their terminology, methodological specifics, and different emphasis, there is a general common sense that arises from the inherent nature of the engineering processes. Furthermore, while corresponding safety assurance is crucial for approval, it is highly dependent on the system and the environment, which is also referred to as ODD. Therefore, the overall methodology should be decoupled from the application-specific safety assurance, generalized across existing frameworks, and formalized along the transitions between simulation and the real world, while striving to support and enable safety argumentation along the resulting iterative process reference model in a general form. This requires, similar to the classical V-model, a certain level of abstraction for widespread applicability. Therefore, a detailed, fully-fledged safety case that leads to direct release cannot be provided. Consequently, the iterative data-based V-model is introduced to build on the generality of the classical V-model, bridging existing frameworks while simultaneously addressing the challenges of emerging systems and technologies in a sophisticated manner. Thereby, a particular emphasis is laid on cognitive cyber-physical systems of the real world that are safety critical to adress challenges of arising autonomous technologies.

\subsection{Fundamental Principles}
Adressing emerging autonomous technologies in a general manner highlights the criticality of a generic scneario-based approach \cite{riedmaier2020survey, elster2021fundamental, VVMOverall}, as corresponding database generation could be exhaustive and costly. Consequently, this would limit the suitability of the methodology towards solid technological applications that justify the efforts. Elsewise, a silent testing \cite{Tesla_shadow, templeton2019} approach for corresponding data acquisition, as scenario database equivalent, is also limited by the fact that dedicated systems and applications must already be operating on a large scale in the target environment. Beyond that, direct data collection in the real world might also be inappropriate, as it could lead to hazards and does not provide the necessary scale of datasets. As a consequence, the general methodology of the emerging autonomous systems is based on systematically generated data on a large scale through simulation. This observation is inline with the project series AI Family \cite{KIFamilie} that covers a range of sub-projects on AI assurance \cite{KIAbsicherungSynData}, transfer \& scaling \cite{KIDeltaSynData}, data tooling \cite{KIDataTooling}, and the hybridization of knowledge and data for automated driving applications \cite{KIWissen_D1_D4, KIWissen_D2}. Although the project series dealt with core topics such as the exploitation of simulation and real data, an overarching reference process has not been established. 

\subsection{General Methodology}\label{sec:methodo}
The proposed iterative data-based V-model builds on the VVM \cite{VVM} project by taking up the general V-model structure that is enhanced by a consistent alignment of the ODD along the V-stages. Beyond the application of automated driving, the ODD is interpreted as a generic concept that systematically defines the environment and context of the respective system. In addition, the iterative data-based V-model reflects Waymo's dynamic lifecycle approach \cite{favaro2023building}. In this way, the open world associated challenges as well as systematic gaps are addressed in a natural, iterative and continuous refinement manner. While the approach focuses on the product and provides guidance through the reference process, the chosen level of abstraction also allows for underlying process refinement along the safety argumentation. In addition, the iterative data-based V-model generalizes the scenario-based database approach. Furthermore, the iterative data-based V-model considers AI-specific development aspects and takes up ideas of the data engine \cite{karpathy_cvpr21} to increase efficiency. However, a central part remains a V\&V that is adapted to the challenge of the targeted use of simulation and the real world. Along a systematic consideration of the strengths of innovative frameworks while addressing existing limitations when applied to complex systems, as discussed in Section (\ref{sec:diff}). This results in an iterative data-based V-model, which is outlined in Figure \ref{fig:ours}. The individual aspects are described in more detail below. 

\textbf{Operational Design Domain:} The cycle initially starts with the definition of the ODD. Here, the ODD states an application independent foundation for defining the system's context, e.g. the environment and operating conditions. Besides that, the ODD represents the top-level target designation and belonging requirements specification. Moreover, the ODD provides guidance for the requirement definition of subordinate systems and functions. Thus, taking into account the system ODD independently of the granularity of the functionality or component to be developed ensures top-level aligment.

\textbf{Function-specific ODD:} Here, the specific functionality of the system, subsystem, or component to be developed is specified by means of a dedicated ODD. Thereby, following the top-down decomposition, the relevant top-level targets of the ODD are stringently broken down and transferred to the respective function-specific ODD. Moreover, specifics such as the desired behavior of the functionality that does not result directly from the top-level targets is included. Accordingly, the explicit function-specific ODD compactly defines the requirements specification while simultaneously minimizing specification uncertainties \cite{burton2023closing} through its alignment with the ODD.

\textbf{Data-specific ODD:} In this context, the previously outlined requirements specification is transfered into a data perspective. In this way, the data-specific ODD accounts for the fact that the availability, quality, and nature of the data can have a direct impact on how the system performs the task at hand. Thus, the definition of the data-specific ODD is responsible
for transforming the function-specific ODD requirements into the data-based representation. Making this transformation explicit within the framework should minimize the systematic gaps. In particular, the fact that development and V\&V are data-driven highlights the importance of a precise translation of the formally defined specifications and requirements into a data-specific representation.

\begin{figure*}[]
	\centering	
	\includegraphics[width=0.7\linewidth]{img/LRT_UL_X_3.png}
	\caption{The iterative data-based V-model, which formalizes and merges the various existing methods. The initial loop starts with the definition of the ODD. The explicit formalization of the process from the real world to simulation and back and the data-based characteristic address the challenges of complex systems that embrace AI. The iterative approach, on the other hand, addresses the challenges of open world complexity and offers continuous system and confidence improvement in an intuitive way.}
	\label{fig:ours}
\end{figure*} 

Consequently, the data-specific ODD, which is responsible for the syntactic transition into the data domain while preserving the semantics to meet required demands, constitutes an interface. In more detail, the data-specific ODD represents an handover area between system domain experts and funtion experts (e.g. AI experts), who use different terminologies (linguistic gap) and have different understanding (knowledge gap) due to different backgrounds (specialization gap). This entails the risk that the semantics of the requirements are not fully translated and usually leads to a specification gap. The framework aims to tackle and minimize this risk, which has already been described as specification uncertainties \cite{burton2023closing} that lead to specification insufficiencies \cite{burton2023addressing} and result in a semantic gap. The goal of the data-specific ODD is to enhance the efficiency and effectiveness of developing and ultimately safeguarding emerging complex systems that incorporate AI. This target-oriented addressing of the semantic gap can lead to overarching improved performance, reliability, and robustness.

\textbf{Architectural Design Domain / Data Design Domain:}  The two components of this stage address the transition to development. The architectural design domain defines a framework of possible architectures of the system to be developed on basis of the data-specific ODD. Thus, on the one hand, this step defines requirements for the architecture while on the other hand first assumptions of the design process that constitute from the data-specific ODD are made explicit. Thus, this stage represents a design abstrachtion layer for general descisions during the development and improvement processes.

In parallel, the definition of the data design domain transfers the requirements of the data-specific ODD into the requirements for
generating the data. However, the generation of data is not part of this stage. This deliberate separation of the definition of a design domain from the realization is an integral part of the framework and aims at the explicit separation, formulation, and documentation of requirements and assumptions while at the same time disclosing process related
uncertainties. In addition, this enables a decoupled validation of the design domains and the implementation via the continuous refinement process and indicates the need for action. For example, if deficiencies are identified during monitoring, it is possible to check whether the data design domain was valid and the problem arises from the incompleteness of the dataset in relation to it, or whether a refinement of the previous specifications and assumptions is necessary. 

\textbf{Architecture Definition / Sim. Dataset Generation:} This stage represents a deliberated engineering stage that focus on both the architecture setup as well as the simulation based data generation. The architectural definition selects a certain architectur within the corresponding architectural design domain and determines the specifics like parameters. 

The simulation data generation characterizes the systematic synthesis of data with respect to the data design domain. While the architecture selects a dedicated realization from the set of possible architectures, the creation of the dataset aims at completeness. In line with other frameworks like \cite{VVMOverall, favaro2023building, karpathy_cvpr21} and standards such as SOTIF \cite{iso21448}, the iterative data-based V-model refers to a complete coverage of the whole space by a sufficient decomposition through trigger constraints, yet in a data-based way. Moreover, the strategic generation of synthetic data offers the possibility of uncovering trigger conditions at an early stage and countering the challenge of "unknown unknowns". For example, logical inference based on existing trigger conditions can be used for this purpose.  

\textbf{System Development:}  This level represents the final implementation, which takes up and connects the previously decoupled paths of the architecture and data, and yields to a resulting system. Thereby, the system can consider individual dedicated (AI) models up to a system of systems, depending on the task. In addition, with regard to the overall process reference model, the system development phase completes the left leg of the V, which represents the design phase, and forms the intersection with the V\&V phase, the right leg of the V that is integral to the iterative data-based V-model.

\textbf{System Evaluation:} The first evaluation stage within the V\&V phase is represented by system evaluation. Here, the previously stipulated (functional) requirements are verified on the basis of the test split of the generated simulation data. Accordingly, this first evaluation stage can also be referred to as a virtual system testing. In more detail, this simulative performance evaluation is carried out using specified trigger conditions provided in terms of the dataset and assesses performance through corresponding indicators and acceptance criteria, which have to be derived from the individually customizable safety argumentation. 

\textbf{Real World Dataset Generation:} In this context, real-world data is generated. Data generation takes place in accordance with the data design domain. For safety reasons, the real world dataset represents a subset of the simulation dataset. Moreover, the effort in the real world can be limited for reasons of cost-effectiveness if this is permitted by the safety argument. As a consequence, the dedicated requirement for the scope of the real world dataset depends on the individual functionality to be developed and the corresponding safety argumentation and assurance. Therefore, a general specification cannot be provided by the process reference model.

\textbf{System Transfer:}  In order to counteract the gap between simulation and real world, the system transfer stage is introduced explicitly. The systems functionality can thus be adapted on the basis of the real data from the previous stage.

\textbf{Transfer Evaluation:} This second evaluation stage within the V\&V phase constitutes the transfer evaluation of the system. Here, the previously stipulated requirements are verified on the basis of the test split of the generated real world data. Consequently, this second evaluation stage can also be described as a simulation-based system test, which is carried out by means of real world data. The procedure as well as the performance indicators and acceptance criteria can be adopted from the first evaluation stage, the system evaluation, as the main difference is the changed origin of the data. Consequently, the desired functionality of the system and the associated quality and assessment criteria can be adopted.

\textbf{Open-Loop Evaluation (Silent Testing):}  This is the third evaluation stage, which marks the transition back to the real world. More precisely, the developed system is evaluated in the real world with the real input, whilst the system output is not applied in the real world. This open-loop evaluation thus represents a first step in the gradual return to the real world. In recent years, the open-loop evaluation has received increasing attention and can also be interpreted as silent testing \cite{wang2021online}, shadow mode, and virtual assessment of automation in field operation (VAAFO) \cite{wang2020reduction}.

\textbf{Closed-Loop Evaluation:} In the context of the fourth evaluation stage, the term closed-loop refers to the previous open-loop, which is now closed, meaning that the evaluation formalizes the system-based feedback into the real world. Here, the functionality is initially verified in the real world. Furthermore, this closed-loop evaluation phase is best viewed as a real world system test or, with regard to the application of automated driving, as a vehicle-in-the-loop (VIL). In the case of automated driving, this system evaluation can be carried out on the proving ground, for example. 

\textbf{Field Operation Evaluation:} The fifth and final evaluation phase is carried out in the real world on a larger scale than the previous one and aims to validate the system in the real world. It can be seen that V\&V, which was mostly considered jointly, is separate in the real world. While verification in the real world is possible on the proving ground, for example, validation requires further operation in the field. In principle, this kind of evaluation setting matches the consideration of on-the-road tests in automated driving applications.

\textbf{Deployment - System Operation \& Monitoring:} Once the system has successfully passed all evaluations, the acceptance criteria stipulated by the safety argumentation are fulfilled and the system can be depolyed. Continuous trust building arise in the course of system operation and ongoing monitoring. Furthermore, intelligent data harvesting can be performed during operation. In automated driving, for example, each individual vehicle can collect and select local data such that the cumulative gain can be used for subsequent refinement measures. 

\textbf{Detection of Deficiencies:} By means of the safety argumentation, corresponding acceptance criteria, and system specific performance indicators, deficiencies can be detected. If, e.g., an "unknown unknown" is detected at the overall level, which can have or has had catastrophic consequences, the approval of the systems must be withdrawn and the development and assurance process must cycled again from the beginning, taking into account the adapted requirements. While this is defined here in a structured way, we can see a very similar approach in practice today in the example of Cruise LCC in the USA, according to the incident of hitting a pedestrian \cite{equipmentrecallreport, NHTSARecall23E, NHTSARecallLetter}.

\textbf{Continuous Refinement:} In case of insufficencies, the overall loop is reinitiated by starting with a refinement of the ODD. A continuous transfer of the acquired findings back into the simulation and a step-by-step return to the real world is an integral part of the framework. Even in the long term, the use of synthetically generated data is targeted for safety, and efficiency reasons. In terms of trigger conditions, real world data might be limited to uncovered trigger conditions, while synthetic data allows the consideration of conceivable but unseen trigger conditions. Thus, the use of exclusively real data represents a subspace of imaginable trigger conditions and is therefore only effective in combination with synthetically augmented data. Accordingly, synthetic data can increase safety and efficiency in the development process. In general, moreover, the continuous refinement of the framework leverages the iterative nature of error analysis and specification adaptation, which increases safety and efficiency throughout the entire product lifecycle.

\textbf{Failure Handling:} Only the forward-looking transitions are indicated in Figure \ref{fig:ours}. Nevertheless, a large number of subordinate feedback flows are present, which have been omitted for the sake of clarity. Dealing with failed evaluations is of particular importance, which is why this is described in more general terms. 

On the one hand, an error can be caused by the design and realization of the functionality. On the other hand, it can be caused by gaps in the data used. Analyzing the error case can provide more information about the underlying cause. If an error case is caused by data and the data is within the data design domain, there is an error in the dataset generation. However, if the error case data is part of the data-specific ODD but not the data design domain, there is a gap in the specification of the data design domain. Otherwise, if the error case data is neither part of the data-specific ODD, the data-specific ODD definition itself must be adapted. An adaptation of a specification, e.g., the data-specific ODD or the data design domain, always requires stepping back to this level and revising the subsequent levels.

However, the functionality itself can also be the source of an error. If the selected function is within the architectural design domain, it is conceivable that the choice of architecture, although permissible, was not appropriate. If the realized functionality falls within the data-specific ODD but is not covered by the architectural design domain, the architectural design domain must be updated. If the function causes a behavior that does not correspond to the data-specific ODD but does correspond to the function-specific ODD, the data-specific ODD must be updated. Ultimately, gaps in the function-specific ODD can also lead to a subsequent error, which may require this specification to be updated. Furthermore, it is generally assumed that the assessments are conducted within the ODD. Otherwise, an ODD refinement is deemed to take place.

\textbf{Safety Argumentation Decoupling:} Along the process outlined above, the safety argumentation and safety assessment are customizable. While the VVM project \cite{VVMOverall} and Waymo's safety determination lifecycle \cite{favaro2023building} specify the absence of unreasonable risk in detail with respect to the system, this is omitted by the iterative data-based V-model, analogous to the classical V-model \cite{brohl1993v}. Nevertheless, it is evident that distinct evaluation stages throughout the V\&V process outlined above require the definition of acceptance criteria based on a system-specific safety argumentation. In this way, Waymo's case credibility assessment \cite{favaro2023building} is built-in by design, albeit with greater flexibility. This is due to the fact that the safety argumentation is not predefined. Therefore, to achieve the overall objectives, the framework enforces the creation, evaluation, and refinement of the safety argumentation. In particular, this customizability of the safety argumentation within the proposed framework implicitly leads to reasonableness, confidence, and coverage assessments along the interative refinement and the corresponding approval, similar to Waymo's case credibility assessment. Since the safety argumentation depends on the system as well as its environment and context, in other words specific on the ODD, and is also subject to application-specific standards and regulations, the safety argumentation is required to be decoupled and customizable to achive the desired generality of the methodology.


\subsection{Classification and Delimitation of the Methodology}\label{sec:classi}

The framework of the proposed iterative data-based V-model takes up existing further developments of the V-model \cite{VVMOverall} as well as innovative frameworks \cite{favaro2023building, karpathy_cvpr21} for handling complex systems. It addresses the characteristics of complex systems that integrate AI. This is illustrated by the use of data-based methods and the separate consideration of the architecture. In particular, complex systems are handled by means of dedicated stages and the formalized exploitation of simulation and real data. Central aspects from the AI Family project \cite{KIFamilie}, in particular the two sub-projects AI Delta Learning \cite{KIDeltaSynData} and AI Data Tooling \cite{KIDataTooling}, are thus taken up and formally integrated into an overall perspective. At the same time, the iterative data-based V-model offers the opportunity to take the results of the above-mentioned projects \cite{KIFamilie, KIDeltaSynData, KIDataTooling} into account. For instance, the stringent safety argumentation that was developed for a pedestrian detection AI \cite{KIAbsicherungSynMethoden, KIAbsicherungSynAbsicherung} can be considered. In addition, approaches for generating synthetic data \cite{KIAbsicherungSynData, KIDeltaSynData} can also be incorporated. This illustrates the generic and unifying character of the iterative data-based V-model framework. Moreover, the framework offers the possibility of using the emerging imaginative intelligence \cite{wang2024does, li2024sora} in the development process and is therefore also equipped for future developments.

Furthermore, the transition from a scenario-based approach to a general data-based approach opens up the broad applicability of the methodology. Additionally, the present approach enables scalability by the consideration of different levels of granularity of a system and, thus, also the system complexity. Thereby, the methodology addresses the architecture layer, the behavioral layer, as well as the in-service operational layer of Waymo \cite{karpathy_cvpr21} or the capability, engineering, and real world layer of the VVM project \cite{VVMAPerspectives}. Due to the claim of a generic framework and the necessity of individual performance measures and acceptance criteria, the framework does not claim to be a comprehensive framework for safety assessment and assurance. Nevertheless, Waymo's Case Credibility Assessment \cite{favaro2023building} approach is inherently integrated. This is due to the fact that the individual definition of the safety argumentation and thus the acceptance criteria and performance indicators require a suitability assessment, while the multiple assessments under increasingly realistic conditions require to address the coverage assessment. The iterative approach implies a process refinement within the structure defined by the framework. Hence, a native implementation of credibility \cite{koopman2019credible} can be envisioned. In particular, with increasing time and scale of the system, validation of the process and product takes place, thus addressing trust and credibility. This analysis of the proposed framework in relation to the improved V-model \cite{VVMOverall} of the VVM project, the safety determination lifecycle \cite{favaro2023building} of Waymo, and the data engine \cite{karpathy_cvpr21} of Tesla demonstrates that the iterative data-based V-model specifically combines the various methods and perspectives and formalizes them at the macro process level in order to maintain the generality of the classical V-model \cite{brohl1993v}.

While the framework does not explicitly address safety assessment and assurance, it opens up the possibility of data-driven functional safety assurance along with the potential to incorporate the results of current research from projects such as SUNRISE \cite{SUNRISE} and V4SAFETY \cite{V4SAFETY}. While allowing for statistical assurance of AI systems, the framework can also be applied to more traditional approaches. Ultimately, this leaves space for the use of different concepts. This is particularly important with regard to safety and explainability of AI systems, as research in this area is still raising open questions and different solutions are conceivable \cite{neto2022safety}. The future applicability of the process reference model is therefore addressed by the safety argumentation flexibility.

For an extended analysis of the iterative data-based V-model in relation to the previously analyzed process reference frameworks from Section \ref{02_New}, major characteristics are compared in Table \ref{tab:compare_frameworks}. Thereby, the advantages and disadvantages of the individual frameworks are illustrated in a compact and abstract manner while demonstrating that the iterative data-based V-model is able to unite the various frameworks except for the safety assessment, which is purposefully detached in accordance with the classical V-model. The individual criteria in Table \ref{tab:compare_frameworks} result from the analysis of the innovative development processes from Section \ref{02_New}, in particular from the discussion as well as the fundamental principles from Section \ref{03_Methodology}.

More specifically, the proposed framework extends the improved V-model specifically towards a continuous integration process that allows to respond to changes in the real world and address the open long-tail distribution challenge over time. Along with this standardization of the different frameworks, there is also a simplification. As discussed above, compared to VVM and Waymo, due to the chosen abstraction of the process, only one perspective is required to address different levels of system's granularity and complexity. Likewise, data-based processes, such as the one of Tesla's data engine, can be considered upfront, whilst a variety of system types, such as traditional or and mixed systems, can be considered in parallel. This is supplemented in Table \ref{tab:compare_frameworks}, which contains a more detailed assessment of the proposed framework in relation to previously investigated process reference frameworks. 

\newcommand\RotText[1]{\rotatebox{90}{\parbox{3.9cm}{\raggedright#1}}}

\begin{table}[]
	\centering
	\caption{Comparison of the previously analyzed process reference frameworks with the established iterative data-based V-model.}
	\begin{tabularx}{\linewidth}{l *{5}{>{\raggedright\arraybackslash}X}}
		\toprule
		& \RotText{Classical V-model} & \RotText{Improved V-model \quad \quad \quad \quad \quad  \tiny(VVM Project)} & \RotText{Safety Determination Lifecycle \tiny(Waymo)} & \RotText{Data Engine \quad \quad \quad \quad \quad \quad \quad \quad } & \RotText{Iterative data-based V-model \tiny(proposed)} \\
		\midrule
		Design phase & \cmark & \cmark & (\cmark) & \cmark & \cmark \\
		V\&V phase & \cmark & \cmark & \cmark & \cmark & \cmark \\
		Application independence & \cmark & \xmark & (\cmark) & (\cmark) & \cmark \\
		\midrule
		Generic across system granularities & \cmark & \xmark & \xmark & (\xmark) & \cmark \\
		Use of specific databases & \xmark & \cmark & (\cmark) & \cmark & \cmark \\
		Use of generic databases & \xmark & \xmark & (\xmark) & \xmark & \cmark \\
		Formalized simulation exploitation & \xmark & \xmark & \xmark & \xmark & \cmark \\
		\midrule
		Iterative product refinement & \xmark & \xmark & \cmark & \cmark & \cmark \\
		Iterative process refinement & \xmark & \xmark & \cmark & (\xmark) & (\cmark) \\
		Continuous trust building & \xmark & \xmark & \cmark & (\xmark) & \cmark \\
		\midrule
		System monitoring & \xmark & \xmark & \cmark & \cmark & \cmark \\
		Safety assessment (e.g. w.r.t. residual risk) & \xmark & \cmark & \cmark & \xmark & \xmark \\
		Safety argumentation customizability & \cmark & \xmark & \xmark & (\xmark) & \cmark \\
		\midrule
		Appropriate for traditional systems & \cmark & \cmark & \cmark & \xmark & \cmark \\
		Appropriate for AI systems & \xmark & (\cmark) & (\cmark) & \cmark & \cmark \\
		Appropriate for complex systems incl. AI & \xmark & (\cmark) & \cmark & (\cmark) & \cmark \\
		\midrule
		Overall generality and transferability & \cmark & (\xmark) & (\cmark) & (\cmark) & \cmark \\
		Overall suitability for emerging AI systems & \xmark & (\cmark) & (\cmark) & (\xmark) & \cmark \\
		\bottomrule
	\end{tabularx}%
	\label{tab:compare_frameworks}
\end{table}%

\subsection{Summary and Discussion of the Proposed Framework}

Consequently, it can be summarized that the iterative data-based V-model
\begin{itemize}
	\item represents a systematic update of the classical V-model,
	\item recognises the data-driven AI development and V\&V,
	\item generalizes across innovative development frameworks,
	\item thus formalizes a unifying process reference model.
\end{itemize}

Therefore, like the classical V-model, the iterative data-based V-model
\begin{itemize}
	\item decouples the process from specific safety assurance,
	\item unites multiple perspectives and approaches into a single,
	\item applies across different levels of system granularities,
	\item thus enables the desired generality and transferability.
\end{itemize}

In addition, the various advantages of the different innovative development processes are taken into account by the iterative data-based V-model, that 
\begin{itemize}
	\item considers system environment/context through the ODD,
	\item maps the open world in a managable (trigger) datasets,
	\item accounts for the prospective and retrospective view,
	\item enables iterative refinement throughout the lifecycle,
	\item thus harmonizes advantages of existing methodologies.
\end{itemize}

Furthermore, the data-based iterative V-model also addresses respective disadvantages of existing process reference models, as it
\begin{itemize}
	\item relaxes assumptions w.r.t. databases or hardware,
	\item formalizes the exploitation of simulation and real world,
	\item applies to traditional, AI-based, or mixed systems,
	\item represents an application-agnostic methodology,
	\item and overcomes limitations of existing frameworks.
\end{itemize}

The iterative data-based V-model harmonizes the respective advantages from Table \ref{tab:compare_frameworks_ad_disad} while counteracting the disadvantages from Table \ref{tab:compare_frameworks_ad_disad}. Thereby, the classical V-model is deliberately extended to the needs of complex systems incorporating AI. 

The proposed framework also has some implications. Particularly, while decoupling the framework from safety assessment and argumentation enables broad applicability, it does not inherently ensure the development of safe systems. Therefore, for safety-critical applications, individual safety assessments and arguments are necessary. However, the framework provides the desired flexibility and adaptability. Morover, among other things, the harmonization of different approaches has a specific implication on the formalization of the process, which is illustrated in Figure \ref{fig:ours}. This formalization explicitly addresses various data sources, from synthetic to real data. In particular, the systematic consideration of synthetic data facilitates the early uncovering of trigger conditions and counteracts the challenge of “unknown unknowns”. The framework thus enables greater efficiency and safety through the systematic consideration of synthetic data. Furthermore, the continuous refinement provided by the framework and thus the iterative nature of error analysis and specification adaptation throughout the entire product lifecycle ensures and increases safety.

Beyond that, the formalization also entails the separation in the depth of the design domain and design implementation as well as in the separation in the breadth of the architecture and the parameterizing data, in order to systematically close gaps in development phase. Overall, a unique feature of the framework is its ability to cover different types of systems, levels of granularity and complexity, and application areas from a unified perspective. The property to adapt to varying system levels and complexities is illustrated abstractly in the following.









%\section{Related Work}
\label{sec:related_work}

\paragraph{MIP for ML}
There have been some successful cases of using MIP for ML applications, including
medical scoring systems~\cite{ustun2016supersparse, ustun2019learning, liu2022fasterrisk}, 
identification of nonlinear dynamical systems~\cite{bertsimas2023learning, liu2024okridge},
decision trees~\cite{bertsimas2017optimal, hu2019optimal},
survival analysis~\cite{zhang2023optimal, liu2024fastsurvival},
hierarchical models~\cite{bertsimas2020sparse, shafiee2024constrained},
and regression models~\cite{xie2020scalable, bertsimas2020sparse1, bertsimas2020sparse2, hazimeh2020fast, atamturk2020safe, hazimeh2022sparse, liu2024okridge, guyard2024el0ps}.
% and low-rank models~\cite{}.
A large number of these works focus on obtaining high-quality feasible solutions.
Only a small portion of them focus on certifying optimality due to scalability issues.
However, for certain applications in high-stakes decision-making and scientific discovery, it is critical to certify optimality or quantify the optimality gap.
Our work aims to make a contribution in this direction, with great emphasis on improving the computational scalability of certifying optimality for a general class of ML problems --- the generalized linear models.

\paragraph{Lower Bound Calculation}
One crucial aspect of certifying optimality for large-scale datasets is to compute tight lower bounds efficiently.
For commercial MIP solvers, there are in general two approaches of calculating the lower bounds.
The first approach is to iteratively approximate the objective function with cutting planes~\cite{kelley1960cutting} and solve the related problem with linear programming~\cite{schrijver1998theory, wolsey2020integer}.
For the cutting-plane type method, the lower bound can be loose if we cannot generate many high-quality cutting planes to approximate the objective function accurately.
The second approach is to perform conic convex relaxation and solve the related problem using the interior-point method (IPM)~\cite{dikin1967iterative, renegar2001mathematical, nesterov1994interior}.
Although the lower bound is often tighter, IPM does not scale favorably because it is a second-order method.
Recently, there have been some attempts of using first-order methods to calculate the lower bound, including subgradient descent~\cite{bertsimas2020sparse1}, alternating direction method of multipliers (ADMM)~\cite{liu2024okridge}, and coordinate descent~\cite{hazimeh2022sparse}.
Our work also belongs to this line of research effort.
However, compared to previous works, ours enjoys fast convergence rate (\ToDo{first approach to achieve linear convergence rate?}), has low computational complexity per iteration, and can significantly benefit from running on GPUs.
We elaborate on the hardware aspect in more details.

% However, our work is much scalable 
% , with major differences on both convergence rate and algorithmic parameter selection.
% Previous methods have a sublinear convergence rate (at best with $O(1/k^2)$ if Nesterov acceleration is used) and may require substantial preprocessing time (for large-scale problems) to estimate the stepsizes.
% In contrast, our method has linear convergence rate and is parameter-free in the sense that the algorithm can automatically adjust the stepsizes adaptive to each data instance, thanks to the restart technique and adaptive estimation of step sizes.


% Customized schemes - first order methods
% -gradient descent - generic and slow in general
% ADMM - becomes expensive for complicated functions
% Coordinate descent - efficient, no restart
% $O(1/k)$ or $O(1/k^2)$ convergence rate
% Ours can achieve linear convergence rates, and leverage parallel computing.

\paragraph{GPU Acceleration}
Recently, there have been some promising works on using GPUs to accelerate continuous optimization, including linear programming~\cite{applegate2021practical, lu2023cupdlp}, quadratic programming (with only linear constraints)~\cite{lu2023practical}, and semidefinite programming~\cite{han2024accelerating}.
There are few works of using GPUs to solve discrete problems.
One natural way to take advantage of GPUs for discrete problems is to use the GPU-based LP inside MIP solvers, like what~\cite{de2024power} has done to solve the clustering problems.
However, this cutting-plane type approach still faces difficulties if we have to iteratively generate an exponential number of cutting planes to approximate the original objective function.
% Since our method can be easily parallelized on GPUs, it provides tremendous opportunities for large-scale computing.
In contrast, we develop a customized FISTA method to directly handle the nonlinear objective function.
The computation can be easily parallelized because we only require matrix-vector multiplication.
For other first-order methods such as  ADMM~\cite{liu2024okridge} and coordinate descent~\cite{hazimeh2022sparse} mentioned in the last subsection, they are unsuitable for GPUs.
The former requires solving linear systems while the latter is sequential in nature.


\section{Experiments}
\label{sec:experiment}
\begin{table}[t]
\sisetup{table-format = 2.2, round-mode = places, round-precision=2, round-pad = false}
\caption{Comparison of test accuracy between the DUAL score method and existing techniques using ResNet-18 on CIFAR-10 and CIFAR-100 datasets. Training the model on the full dataset achieves an average test accuracy of 95.30\% on CIFAR-10 and 78.91\% on CIFAR-100. The best result in each pruning ratio is highlighted in bold.}
\setlength{\tabcolsep}{3.1pt}
\centering
\resizebox{\linewidth}{!}{
\begin{tabular}{l c@{}cc@{}cc@{}cc@{}cc@{}c | c@{}cc@{}cc@{}cc@{}cc@{}c}
    \toprule
    \textbf{Dataset ($\rightarrow$)} & \multicolumn{10}{c}{\textbf{CIFAR10}} & \multicolumn{10}{c}{\textbf{CIFAR100}}\\
    \cmidrule(lr){2-21}
    
    \textbf{Pruning Rate ($\rightarrow$)} & \multicolumn{2}{c}{\textbf{30\%}} & \multicolumn{2}{c}{\textbf{50\%}} & \multicolumn{2}{c}{\textbf{70\%}} & \multicolumn{2}{c}{\textbf{80\%}} & \multicolumn{2}{c}{\textbf{90\%}} & \multicolumn{2}{c}{\textbf{30\%}} & \multicolumn{2}{c}{\textbf{50\%}} & \multicolumn{2}{c}{\textbf{70\%}} & \multicolumn{2}{c}{\textbf{80\%}} & \multicolumn{2}{c}{\textbf{90\%}} \\
    \midrule
    
    \textbf{Random} & 94.39 & {\scriptsize \num{ +-0.2275}} & 93.20 & {\scriptsize \num{ +-0.1188}} & 90.47 & {\scriptsize \num{ +-0.1678}} & 88.28 & {\scriptsize \num{ +-0.1731}} & 83.74 & {\scriptsize \num{ +-0.2051}} & 75.15 & {\scriptsize \num{ +-0.2825}} & 71.68 & {\scriptsize \num{ +-0.3065}} & 64.86 & {\scriptsize \num{ +-0.3939}} & 59.23 & {\scriptsize \num{ +-0.6189}} & 45.09 & {\scriptsize \num{ +-1.2610}} \\
    
    \textbf{Entropy} & 93.48  & {\scriptsize \num{ +-0.0566 }} & 92.47 & {\scriptsize \num{ +-0.1707}} & 89.54 & {\scriptsize \num{ +-0.1753}} & 88.53 & {\scriptsize \num{ +-0.1864}} & 82.57 & {\scriptsize \num{ +-0.3645}} & 75.20 & {\scriptsize \num{ +-0.2486 }} & 70.90 & {\scriptsize \num{ +-0.3464}} & 61.70 & {\scriptsize \num{ +-0.4669}} & 56.24 & {\scriptsize \num{ +-0.5082}} & 42.25 & {\scriptsize \num{ +-0.3915}} \\
    
    \textbf{Forgetting} & 95.48  & {\scriptsize \num{ +-0.1412 }} & 94.94 & {\scriptsize \num{ +-0.2116}} & 89.55 & {\scriptsize \num{ +-0.6456}} & 75.47 & {\scriptsize \num{ +-1.2726}} & 46.64 & {\scriptsize \num{ +-1.9039}} & 77.52 & {\scriptsize \num{ +-0.2585 }} & 70.93 & {\scriptsize \num{ +-0.3700}} & 49.66 & {\scriptsize \num{ +-0.1962}} & 39.09 & {\scriptsize \num{ +-0.4065}} & 26.87 & {\scriptsize \num{ +-0.7318}} \\
    
    \textbf{EL2N} & 95.44  & {\scriptsize \num{ +-0.0628 }} & 95.19 & {\scriptsize \num{ +-0.1134}} & 91.62 & {\scriptsize \num{ +-0.1397}} & 74.70 & {\scriptsize \num{ +-0.4523}} & 38.74 & {\scriptsize \num{ +-0.7506}} & 77.13 & {\scriptsize \num{ +-0.2348}} & 68.98 & {\scriptsize \num{ +-0.3539}} & 34.59 & {\scriptsize \num{ +-0.4824}} & 19.52 & {\scriptsize \num{ +-0.7925}} & 8.89 & {\scriptsize \num{ +-0.2774}} \\
    
    \textbf{AUM} & 90.62  & {\scriptsize \num{ +-0.0921 }} & 87.26 & {\scriptsize \num{ +-0.1128}} & 81.28 & {\scriptsize \num{ +-0.2564}} & 76.58 & {\scriptsize \num{ +-0.3458}} & 67.88 & {\scriptsize \num{ +-0.5275}} & 74.34 & {\scriptsize \num{ +-0.1424 }} & 69.57 & {\scriptsize \num{ +-0.2100}} & 61.12 & {\scriptsize \num{ +-0.2004}} & 55.80 & {\scriptsize \num{ +-0.3256}} & 45.00 & {\scriptsize \num{ +-0.3694}} \\
    
    \textbf{Moderate} & 94.26  & {\scriptsize \num{ +-0.0904 }} & 92.79 & {\scriptsize \num{ +-0.0856}} & 90.45 & {\scriptsize \num{ +-0.2110}} & 88.90 & {\scriptsize \num{ +-0.1684}} & 85.52 & {\scriptsize \num{ +-0.2906}} & 75.20  & {\scriptsize \num{ +-0.2486 }} & 70.90 & {\scriptsize \num{ +-0.3464}} & 61.70 & {\scriptsize \num{ +-0.4699}} & 56.24 & {\scriptsize \num{ +-0.5082}} & 42.25 & {\scriptsize \num{ +-0.3915}} \\
    
    \textbf{Dyn-Unc} & 95.49  & {\scriptsize \num{ +-0.2061 }} & \textbf{95.35} & {\scriptsize \num{ +-0.1205}} & 91.78 & {\scriptsize \num{ +-0.6516}} & 83.32 & {\scriptsize \num{ +-0.9391}} & 59.67 & {\scriptsize \num{ +-1.7929}} & 77.67  & {\scriptsize \num{ +-0.1381}} & 74.23 & {\scriptsize \num{ +-0.2214}} & 64.30 & {\scriptsize \num{ +-0.1333}} & 55.01 & {\scriptsize \num{ +-0.5465}} & 34.57 & {\scriptsize \num{ +-0.6920}} \\
    
    \textbf{TDDS} & 94.42  & {\scriptsize \num{ +-0.1252 }} & 93.11 & {\scriptsize \num{ +-0.1377}} & 91.02 & {\scriptsize \num{ +-0.1908}} & 88.25 & {\scriptsize \num{ +-0.2385}} & 82.49 & {\scriptsize \num{ +-0.2799}} & 75.02  & {\scriptsize \num{ +-0.3682}} & 71.80 & {\scriptsize \num{ +-0.3323}} & 64.61 & {\scriptsize \num{ +-0.2431}} & 59.88 & {\scriptsize \num{ +-0.2110}} & 47.93 & {\scriptsize \num{ +-0.2147}} \\
    
    \textbf{CCS} & 95.31  & {\scriptsize \num{ +-0.2238}} & 95.06 & {\scriptsize \num{ +-0.1547}} & 92.68 & {\scriptsize \num{ +-0.1704}} & 91.25 & {\scriptsize \num{ +-0.2073}} & 85.92 & {\scriptsize \num{ +-0.3901}} & 77.15 & {\scriptsize \num{ +-0.2816}} & 73.83 & {\scriptsize \num{ +-0.2073}} & 68.65 & {\scriptsize \num{ +-0.3130}} & 64.06 & {\scriptsize \num{ +-0.2084}} & 54.23 & {\scriptsize \num{ +-0.4813}} \\
    
    \textbf{D2} & 94.13  & {\scriptsize \num{ +-0.2033}} & 93.26 & {\scriptsize \num{ +-0.1623}} & 92.34 & {\scriptsize \num{ +-0.1786}} & 90.38 & {\scriptsize \num{ +-0.3376}} & 86.11 & {\scriptsize \num{ +-0.2072}} & 76.47  & {\scriptsize \num{ +-0.2934}} & 73.88 & {\scriptsize \num{ +-0.2780}} & 62.99 & {\scriptsize \num{ +-0.2775}} & 61.48 & {\scriptsize \num{ +-0.3361}} & 50.14 & {\scriptsize \num{ +-0.8951}} \\
    
    \midrule
    
    \textbf{DUAL} & 95.25  & {\scriptsize \num{ +-0.17 }} & 94.95 & {\scriptsize \num{ +-0.22 }} & 91.75 & {\scriptsize \num{ +-0.98 }} & 82.02 & {\scriptsize \num{ +-1.85 }} & 54.95 & {\scriptsize \num{ +-0.42 }} & 77.43 & {\scriptsize \num{ +-0.18 }} & 74.62 & {\scriptsize \num{ +-0.47 }} & 66.41 & {\scriptsize \num{ +-0.52 }} & 56.57 & {\scriptsize \num{ +-0.57 }} & 34.38 & {\scriptsize \num{ +-1.39 }} \\
    
    \textbf{DUAL+$\beta$ sampling} & \textbf{95.51}  & {\scriptsize \num{ +-0.0634}} & 95.23 & {\scriptsize \num{ +-0.0796}} & \textbf{93.04} & {\scriptsize \num{ +-0.4282}} & \textbf{91.42} & {\scriptsize \num{ +-0.352}} & \textbf{87.09} & {\scriptsize \num{ +-0.3599}} & \textbf{77.86}  & {\scriptsize \num{ +-0.1186}} & \textbf{74.66} & {\scriptsize \num{ +-0.1173}} & \textbf{69.25} & {\scriptsize \num{ +-0.2156}} & \textbf{64.76} & {\scriptsize \num{ +-0.2272}} & \textbf{54.54} & {\scriptsize \num{ +-0.0884}} \\
    
    \bottomrule
\end{tabular}
}
\label{tbl:main_cifar}
\end{table}

\subsection{Experimental Settings}
We assessed the performance of our proposed method in three key scenarios: image classification, image classification with noisy labels and corrupted images. In addition, we validate cross-architecture generalization on three-layer CNN, VGG-16~\citep{simonyan2015deepconvolutionalnetworkslargescale}, ResNet-18 and ResNet-50~\citep{he2015deepresiduallearningimage}.

\textbf{Hyperparameters.}
For training CIFAR-10 and CIFAR-100, we train ResNet-18 for 200 epochs with a batch size of 128. SGD optimizer with momentum of 0.9 and weight decay of 0.0005 is used. The learning rate is initialized as 0.1 and decays with the cosine annealing scheduler. As \citet{zhang2024spanning} show that smaller batch size boosts performance at high pruning rates, we also halved the batch size for 80\% pruning, and for 90\% we reduced it to one-fourth. For ImageNet-1k, ResNet-34 is trained for 90 epochs with a batch size of 256 across all pruning ratios. An SGD optimizer with a momentum of 0.9, a weight decay of 0.0001, and an initial learning rate of 0.1 is used, combined with a cosine annealing scheduler.

\textbf{Baselines.} The baselines considered in this study are listed as follows\footnote{Infomax~\citep{tan2025data} was excluded as it employs different base hyperparameters in the original paper compared to other baselines and does not provide publicly available code. See Appendix~\ref{Appendix_Technical_Details_of_Baselines} for more discussion.}: (1) Random; (2) Entropy~\citep{coleman2020selectionproxyefficientdata}; (3) Forgetting~\citep{toneva2018empirical}; (4) EL2N~\citep{gordon2021data}; (5) AUM~\citep{pleiss2020identifyingmislabeleddatausing}; (6) Moderate~\citep{xia2022moderate}; (7) Dyn-Unc~\citep{he2024large}; (8) TDDS~\citep{zhang2024spanning}; (9) CCS\mbox{~\citep{zheng2022coverage}}; and (10) $\mathbb{D}^2$~\citep{maharana2023d2}. To ensure a fair comparison, all methods were trained with the same base hyperparameters for training, and the best hyperparameters reported in their respective original works for scoring. Technical details are provided in the \cref{Appendix_Technical_Details_of_Baselines}. 


\begin{wrapfigure}[16]{R}{0.5\textwidth}
\vspace{-15pt}
    \begin{center}
        \includegraphics[width=0.9\linewidth]{Figures/total_computation_time.pdf}
    \end{center}
    \vspace{-8pt}
    \caption{Comparison in total time spent on CIFAR datasets.}
    \label{fig:total_time_consumed}
    % \vspace{-10pt}
\end{wrapfigure}
\vspace{3pt}
\subsection{Image Classification Benchmarks}
\cref{tbl:main_cifar} presents the test accuracy for image classification results on CIFAR-10 and CIFAR-100. Our pruning method consistently outperforms other baselines, particularly when combined with Beta sampling. While the DUAL score exhibits competitive performance in lower pruning ratios, its accuracy degrades with more aggressive pruning. Our Beta sampling effectively mitigates this performance drop.

Notably, the DUAL score only requires training a single model for \emph{only 30 epochs}, significantly reducing the computational cost. In contrast, the second-best methods, Dyn-Unc and CCS, rely on scores computed over a full 200-epoch training cycle, making them considerably less efficient. Even considering subset selection, score computation, and subset training, the total time remains less than a single full training run, as shown in Figure~\ref{fig:total_time_consumed}. Specifically, on CIFAR-10, our method achieves lossless pruning up to a 50\% pruning ratio while saving 35.5\% of total training time. 

\begin{wrapfigure}[20]{L}{0.5\textwidth}
\vspace{2mm}
    \centering
    \captionof{table}{\label{tab:imagenet_results} Comparison of test accuracy on ImageNet-1k. The model trained with the full dataset achieves 73.1\% test accuracy. The best result in each pruning ratio is highlighted in bold.}
    \small
    \setlength{\tabcolsep}{3pt}
    \begin{tabular}{lccccc}
        \toprule
        \textbf{Pruning Rate} & \textbf{30\%} & \textbf{50\%} & \textbf{70\%} & \textbf{80\%} & \textbf{90\%} \\
        \midrule
        \textbf{Random} & 72.2 & 70.3 & 66.7 & 62.5 & 52.3 \\
        \textbf{Entropy} & 72.3 & 70.8 & 64.0 & 55.8 & 39.0 \\
        \textbf{Forgetting} & 72.6 & 70.9 & 66.5 & 62.9 & 52.3 \\
        \textbf{EL2N} & 72.2 & 67.2 & 48.8 & 31.2 & 12.9 \\
        \textbf{AUM} & 72.5 & 66.6 & 40.4 & 21.1 & 9.9 \\
        \textbf{Moderate} & 72.0 & 70.3 & 65.9 & 61.3 & 52.1 \\
        \textbf{Dyn-Unc} & 70.9 & 68.3 & 63.5 & 59.1 & 49.0 \\
        \textbf{TDDS} & 70.5 & 66.8 & 59.4 & 54.4 & 46.0 \\
        \textbf{CCS} & 72.3 & 70.5 & 67.8 & 64.5 & 57.3 \\
        \textbf{D2} & 72.9 & 71.8 & 68.1 & 65.9 & 55.6 \\
        \midrule
        \textbf{DUAL} & 72.8 & 71.5 & 68.6 & 64.7 & 53.1 \\
        \textbf{DUAL+$\beta$ sampling} & \textbf{73.3} & \textbf{72.3} & \textbf{69.4} & \textbf{66.5} & \textbf{60.0} \\
        \bottomrule
    \end{tabular}
\end{wrapfigure}
\vspace{2pt}
We also evaluate our pruning method on the large-scale dataset, ImageNet-1k. The DUAL score is computed during training, specifically at epoch 60, which is 33\% earlier than the original train epoch used to compute scores for other baseline methods. As shown in \cref{tab:imagenet_results}, Dyn-Unc performs worse than random pruning across all pruning ratios, and we attribute this undesirable performance to its limited total training epochs (only 90), which is insufficient for Dyn-Unc to fully capture the training dynamics of each sample. In contrast, our DUAL score, combined with Beta sampling, outperforms all competitors while requiring the least computational cost. The DUAL score's ability to consider both training dynamics and the difficulty of examples enables it to effectively identify uncertain samples early in the training process, even when training dynamics are limited. Remarkably, for 90\% pruned Imagenet-1K, it maintains a test accuracy of 60.0\%, surpassing the previous state-of-the-art (SOTA) by a large margin.
\vspace{15pt}
\subsection{Experiments under More Realistic Scenarios}
\subsubsection{Label Noise and Image Corruption}
Data affected by label noise or image corruption are difficult and unnecessary samples that hinder model learning and degrade generalization performance. Therefore, filtering out these samples through data pruning is crucial. Most data pruning methods, however, either focus solely on selecting difficult samples based on example difficulty~\citep{gordon2021data, pleiss2020identifyingmislabeleddatausing, coleman2020selectionproxyefficientdata} or prioritize dataset diversity~\citep{zheng2022coverage, xia2022moderate}, making them unsuitable for effectively pruning such noisy and corrupted samples.

In contrast, methods that select uncertain samples while considering training dynamics, such as Forgetting \citep{toneva2018empirical} and Dyn-Unc~\citep{he2024large}, demonstrate robustness by pruning both the hardest and easiest samples, ultimately improving generalization performance, as illustrated in Figure~\ref{fig:label_noise_20_ratio}. However, since noisy samples tend to be memorized after useful samples are learned~\citep{arpit2017closer, jiang2020characterizing}, there is a possibility that those noisy samples may still be treated as uncertain in the later stages of training and thus be included in the selected subset.

The DUAL score aims to identify high-uncertainty samples early in training by considering both training dynamics and example difficulty. Noisy data, typically under-learned compared to other challenging samples during this phase, exhibit lower uncertainty (Figure~\ref{fig:label_noise_visualization}, Appendix~\ref{Appendix_labelnoise_experiments}). Consequently, our method effectively prunes these noisy samples.

To verify this, we evaluate our method by introducing a specific proportion of symmetric label noise~\citep{patrini2017making, xia2020robust, li2022selective} and applying five different types of image corruptions~\citep{wang2018iterative, hendrycks2019benchmarking, xia2021instance}. We use CIFAR-100 with ResNet-18 and Tiny-ImageNet with ResNet-34 for these experiments. On CIFAR-100, we test label noise and image corruption ratios of 20\%, 30\%, and 40\%. For Tiny-ImageNet, we use a 20\% ratio of label noise and image corruption. We prune the label noise-added dataset using a model trained for 50 epochs and the image-corrupted dataset with a model trained for 30 epochs using DUAL pruning—both significantly lower than the 200 epochs used by other methods. For detailed experimental settings, please refer to Appendix~\ref{Appendix_Technical_Details_of_Ours}.
\begin{figure*}[t]
    \centering
    \begin{subfigure}{0.325\textwidth}
        \centering
        \includegraphics[width=\textwidth]{Figures/label_noise20_ratio.pdf}
        \caption{\label{fig:label_noise_20_ratio}Pruned mislabeled data ratio}
    \end{subfigure}
    \hfill
    \begin{subfigure}{0.325\textwidth}
        \centering
        \includegraphics[width=\textwidth]{Figures/label_noise20_testacc.pdf}
        \caption{\label{fig:label_noise_20_testacc}Test accuracy under label noise}
    \end{subfigure}
    \hfill
    \begin{subfigure}{0.325\textwidth}
        \centering
        \includegraphics[width=\textwidth]{Figures/imagecorruption_20_testacc.pdf}
    \caption{\label{fig:imagecorruption_20_testacc}Test accuracy under image noise}
    \end{subfigure}
    \caption{\label{fig:labelnoise_main}The left figure shows the ratio of pruned mislabeled data under 20\% label noise on CIFAR-100 trained with ResNet-18. When label noise is 20\%, the optimal value (black dashed line) corresponds to pruning 100\% of mislabeled data at a 20\% pruning ratio. The middle and right figures depict test accuracy under 20\% label noise and 20\% image corruption, respectively. Our method effectively prunes mislabeled data near the optimal value while maintaining strong generalization performance. Results are averaged over five random seeds.}
    \vspace{-10pt}
\end{figure*}
As shown in Figure~\ref{fig:labelnoise_main}, the left plot demonstrates that DUAL pruning effectively removes mislabeled data at a ratio close to the optimal. Notably, when the pruning ratio is 10\%, nearly \emph{all pruned samples are mislabeled data}.
Consequently, as observed in Figure~\ref{fig:label_noise_20_testacc}, DUAL pruning leads to improved test accuracy compared to training on the full dataset, even up to a pruning ratio of 70\%. At lower pruning ratios, performance improves as mislabeled data are effectively removed, highlighting the advantage of our approach in handling label noise.
Similarly, for image corruption, our method prunes more corrupted data across all corruption rates compared to other methods, as shown in Figure~\ref{fig:imagecorruption_203040_ratio},~\ref{fig:imagecorruption_all} in Appendix~\ref{Appendix_imagecorruption_experiments}. As a result, this leads to higher test accuracy, as demonstrated in Figure~\ref{fig:imagecorruption_20_testacc}. 

Detailed results, including exact numerical values for different corruption rates and Tiny-ImageNet experiments, can be found in Appendix~\ref{Appendix_labelnoise_experiments} and \ref{Appendix_imagecorruption_experiments}.  Across all experiments, DUAL pruning consistently shows \emph{strong noise robustness} and outperforms other methods by a substantial margin.

\subsubsection{Cross-Architecture Generalization}
We also evaluate the ability to transfer scores across various model architectures. To be specific, if we can get high-quality example scores for pruning by using a simpler architecture than one for the training, our DUAL pruning would become even more efficient in time and computational cost. Therefore, we focus on the cross-architecture generalization from relatively small networks to larger ones with three-layer CNN, VGG-16, ResNet-18, and ResNet-50. Competitors are selected from each categorized group of the pruning approach: EL2N from difficulty-based, Dyn-Unc from uncertainty-based, and CCS from the geometry-based group. 

For instance, we get training dynamics from the ResNet-18 and then calculate the example scores. Then, we prune samples using scores calculated from ResNet-18, and train selected subsets on ResNet-50. The result with ResNet-18 and ResNet-50 is described in Table~\ref{tab:04-cross-arch-r18-r50}. Surprisingly, the coreset shows competitive performance to the baseline, where the baseline refers to the test accuracy after training a coreset constructed based on the score calculated from ResNet-50.
For all pruning cases, we observe that our methods reveal the highest performances. Specifically, when we prune 70\% and 90\% of the original dataset, we find that all other methods fail, showing worse test accuracies than random pruning. 


\begin{table}[t]
\caption{Cross-architecture generalization performance on CIFAR-100 from ResNet-18 to ResNet-50. We report an average of five runs. `R50 $\rightarrow$ R50' stands for score computation on ResNet-50, as a baseline.}
\label{tab:04-cross-arch-r18-r50}
\setlength{\tabcolsep}{3.1pt}
\centering
\begin{tabular}{lcccc}
    \toprule
    \multicolumn{1}{c}{} & \multicolumn{4}{c}{ResNet-18 $\rightarrow$ ResNet-50} \\
    \midrule
    Pruning Rate ($\rightarrow$) & 30\% & 50\% & 70\% & 90\% \\
    \hline
    \midrule
    Random & 74.47 \scriptsize{$\pm 0.67$} & 70.09 \scriptsize{$\pm 0.42$} & 60.06 \scriptsize{$\pm 0.99$} & 41.91 \scriptsize{$\pm 4.32 $} \\
    EL2N  & 76.42 \scriptsize{$\pm 1.00$} & 69.14 \scriptsize{$\pm 1.00$} & 45.16 \scriptsize{$\pm 3.21$} & 19.63 \scriptsize{$\pm 1.15 $} \\
    Dyn-Unc  & 77.31 \scriptsize{$\pm 0.34$} & 72.12 \scriptsize{$\pm 0.68$} & 59.38 \scriptsize{$\pm 2.35$} & 31.74 \scriptsize{$\pm 2.31 $} \\
    CCS  & 74.78 \scriptsize{$\pm 0.66$} & 69.98 \scriptsize{$\pm 1.18$} & 59.75 \scriptsize{$\pm 1.41$} & 41.54 \scriptsize{$\pm 3.94 $} \\
    \midrule
    DUAL   & \textbf{78.03} \scriptsize{$\pm 0.83$} & 72.82 \scriptsize{$\pm 1.46$} & 63.08 \scriptsize{$\pm 2.45$} & 33.65 \scriptsize{$\pm 2.92 $} \\
    DUAL +$\beta$  & 77.82 \scriptsize{$\pm 0.65$} & \textbf{73.98} \scriptsize{$\pm 0.62$} & \textbf{66.36} \scriptsize{$\pm 1.66$} & \textbf{49.90} \scriptsize{$\pm 2.56 $} \\
    \hline
    \midrule
    DUAL (R50$\rightarrow$R50)  & 77.82 \scriptsize{$\pm 0.64$} & 73.66 \scriptsize{$\pm 0.85$} & 52.12 \scriptsize{$\pm 2.73 $} & 26.13 \scriptsize{$\pm 1.96 $} \\
    DUAL (R50$\rightarrow$R50)+$\beta$  & 77.57 \scriptsize{$\pm 0.23$} & 73.44 \scriptsize{$\pm 0.87$} & 65.17 \scriptsize{$\pm 0.96$} & 47.63 \scriptsize{$\pm 2.47 $} \\
    \bottomrule
    \vspace{-10pt}
\end{tabular}
\end{table}

We also test the cross-architecture generalization performance with three-layer CNN, VGG-16, and ResNet-18 in Appendix~\ref{Appendix_cross_architecture}. 
Even for a simple model like three-layer CNN, we see our methods show consistent performance, as can be seen in Table~\ref{tab:cnn-to-resnet18} in Appendix~\ref{Appendix_cross_architecture}. This observation gives rise to an opportunity to develop some small proxy networks to get example difficulty with less computational cost. 
Transfer across models with similar capacities, $e.g.$ from VGG-16 to ResNet-18 and vice versa, also supports the verification of cross-architecture compatibility.   

\subsection{Ablation Studies}
\paragraph{Hyperparameter Analysis.}
\begin{wrapfigure}[10]{r}{0.5\textwidth}
\vspace{-5mm}
    \begin{center}
        \includegraphics[width=0.92\linewidth]{Figures/TestAcc_TC_varying.pdf}
    \end{center}
    \vspace{-5mm}
    \caption{\textbf{Left}: Varying T with $J=10$ and $c_\gD=4$. \textbf{Right}: Varying $c_\gD$ with $T=30$ and $J=10$.}
    \label{fig:TC_varying}
\end{wrapfigure}
Here, we investigate the robustness of our hyperparameters, $T$, $J$, and $c_\gD$. We fix $J$ across all experiments, as it has minimal impact on selection, indicating its robustness (Fig~\ref{fig:J_varying}, Appendix~\ref{Appendix_Experiments}). In Figure~\ref{fig:TC_varying}, we assess the robustness of $T$ by varying it from 20 to 200 on CIFAR-100. We find that while $T$ remains highly robust in earlier epochs, increasing $T$ degrades generalization performance. This is expected, as larger $T$ overemphasizes difficult samples due to our difficulty-aware selection.  Thus, pruning in earlier epochs (from 30 to 50) proves to be more effective and robust. For the $c_\gD$, we vary it from 3 to 7 and find robustness, especially in the aggressive pruning regime. All results are averaged across three runs.

\paragraph{Beta Sampling Analysis.}
Next, we study the impact of our proposed pruning-ratio-adaptive Beta sampling on existing score metrics. We apply our Beta sampling strategy to prior score-based methods, including Forgetting, EL2N, and Dyn-Unc, using the CIFAR10 and CIFAR100 datasets. By comparing our sampling approach with vanilla threshold pruning, which selects only the highest-scoring samples, we observe that prior score-based methods become remarkably comparable to random pruning after Beta sampling is adjusted (see \cref{tab:abl_beta_cifar10_100_90_main}).

\begin{table*}[t]
\caption{Comparison on CIFAR-10 and CIFAR-100 for $90\%$ pruning rate. 
We report average accuracy with five runs. The best performance is in bold in each column.}
\label{tab:abl_beta_cifar10_100_90_main}
\setlength{\tabcolsep}{4.5pt}
\centering
\begin{tabular}{lcc|cc}
    \toprule
    \multicolumn{1}{c}{} & \multicolumn{2}{c}{CIFAR-10} & \multicolumn{2}{c}{CIFAR-100} \\
    \midrule
    Method & Thresholding & $\beta$-Sampling & Thresholding & $\beta$-Sampling \\
    \midrule
    Random &  \textbf{83.74}\scriptsize{$\pm$0.21} & 83.31 (-0.43) \scriptsize{$\pm$0.14} & \textbf{45.09} \scriptsize{$\pm$1.26} & 51.76 (+6.67) \scriptsize{$\pm$0.25} \\
    EL2N &  38.74 \scriptsize{$\pm$0.75} & 87.00 (+48.26) \scriptsize{$\pm$0.45} & 8.89 \scriptsize{$\pm$0.28} & 53.97 (+45.08)  \scriptsize{$\pm$0.63}  \\
    Forgetting &  46.64 \scriptsize{$\pm$1.90} & 85.67 (+39.03) \scriptsize{$\pm$0.13} & 26.87 \scriptsize{$\pm$0.73} & 52.40 (+25.53) \scriptsize{$\pm$0.43} \\
    Dyn-Unc &  59.67 \scriptsize{$\pm$1.79} & 85.33 (+32.14) \scriptsize{$\pm$0.20} & 34.57 \scriptsize{$\pm$0.69} & 51.85 (+17.28) \scriptsize{$\pm$0.35}   \\
    \midrule
    Ours & 54.95 \scriptsize{$\pm$0.42} & \textbf{87.09} (+31.51) \scriptsize{$\pm$0.36} & 34.28 \scriptsize{$\pm$1.39} & \textbf{54.54} (+20.26) \scriptsize{$\pm$0.09}  \\
    \bottomrule
\end{tabular}
\end{table*}


Even adapted for random pruning, our Beta sampling proves to perform well. Notably, EL2N, which performs poorly on its own, becomes significantly more effective when combined with our sampling method. Similar improvements are also seen with Forgetting and Dyn-Unc scores. This is because our proposed Beta sampling enhances the diversity of selected samples in turn, especially when used with example difficulty-based methods. More results conducted for 80\% pruning cases are included in the \cref{Appendix_beta_samapling}.
    

\paragraph{Additional Analysis.}
In addition to the main results presented in this paper, we also conducted various experiments to further explore the effectiveness of our method. These additional results include an analysis of coreset performance under a time budget ($e.g.$ other score metrics are also computed by using training dynamics up to epoch 30) and Spearman rank correlation was calculated between individual scores and the averaged score across five runs to assess the consistency of scores for each sample. Furthermore, additional results in extreme cases, 30\% and 40\% of label noise and image corruption can be found in \cref{Appendix_Experiments}.


\section{Conclusion}\label{sec:conclusion}
%This work explores the impact of grid-connected and wireless measurement setups on capacitive human body communication, revealing significant differences in both channel \revise{gain} and frequency behavior. 
While conventional data acquisition setups are effective for quantifying the forward path loss, which depends on the conductive properties of the human body, they substantially alter the return path behavior by artificially modifying the capacitive coupling to earth ground.
Therefore, a wireless, wearable-sized data acquisition system is essential for quantitatively evaluating the full \ac{HBC} communication channel in a realistic environment with minimal measurement interference. 
To address this challenge, this work introduces \textit{BodySense}, an evaluation platform for human body communication that is fully wireless, compact enough for wearable applications, and designed for extendability.
To validate the proposed system, the measured channel gains of a classical, grid-connected setup and a wireless setup have been determined for distances of \qty{10}{\centi\meter}, \qty{30}{\centi\meter}, and \qty{50}{\centi\meter} between transmitter and receiver for a frequency range between \qty{4}{\mega\hertz} and \qty{64}{\mega\hertz}.
A comparison between the two scenarios yields an average overestimation of \qty{18.15}{\db} over all investigated distances for the classical case, highlighting the importance of evaluating capacitive \ac{HBC} in realistic conditions.
When comparing the energy consumption of capacitive \ac{HBC} with \ac{BLE}, we achieved results comparable to state-of-the-art \ac{BLE} frontends. 
This demonstrates its potential as a promising alternative to conventional \ac{RF} links, offering opportunities to further enhance the overall energy efficiency of wearable devices and move closer to the realization of battery-free, body-worn sensor nodes.



%This paper proposes \textit{Bodysense}, a fully wireless, wearable-sized system designed to accurately evaluate capacitive human body communication. Experimental evaluation has revealed significant differences in both channel loss and frequency behavior. This paper demonstrated that while conventional data acquisition setups are effective for quantifying the forward path loss, which depends on the conductive properties of the human body, they substantially alter the return path behavior by artificially modifying the capacitive coupling to earth ground. Thus, the proposed wearable-sized data acquisition system is essential for quantitatively evaluating the full \ac{HBC} communication channel in a realistic environment with minimal measurement interference. 
%To address this issue, this paper presents \textit{Bodysense}, a fully wireless, wearable-sized, and extendable evaluation platform for human body communication.
%To validate the proposed system, the measured channel gains of a classical, grid-connected setup and a wireless setup have been determined for distances of \qty{10}{\centi\meter}, \qty{30}{\centi\meter}, and \qty{50}{\centi\meter} between transmitter and receiver for a frequency range between \qty{4}{\mega\hertz} and \qty{64}{\mega\hertz}.
%A comparison between the two scenarios yields an average overestimation of \qty{18.15}{\db} over all investigated distances for the classical case, highlighting the importance of evaluating capacitive \ac{HBC} with a measurement setup that is similar or ideally identical to the envisaged use case.




\bibliographystyle{preprint}
\bibliography{references}


%%%%%%%%%%%%%%%%%%%%%%%%%%%%%%%%%%%%%%%%%%%%%%%%%%%%%%%%%%%%%%%%%%%%%%%%%%%%%%%
%%%%%%%%%%%%%%%%%%%%%%%%%%%%%%%%%%%%%%%%%%%%%%%%%%%%%%%%%%%%%%%%%%%%%%%%%%%%%%%
% APPENDIX
%%%%%%%%%%%%%%%%%%%%%%%%%%%%%%%%%%%%%%%%%%%%%%%%%%%%%%%%%%%%%%%%%%%%%%%%%%%%%%%
%%%%%%%%%%%%%%%%%%%%%%%%%%%%%%%%%%%%%%%%%%%%%%%%%%%%%%%%%%%%%%%%%%%%%%%%%%%%%%%
\newpage


\appendix
\onecolumn

\part{Appendix} 

\newcommand{\appendixnumberline}[1]{Appendix\space}

\renewcommand{\appendixname}{Appendix}
\renewcommand{\thesection}{\appendixname~\Alph{section}}
\renewcommand{\thesubsection}{\Alph{section}.\arabic{subsection}}

\section{Proofs}
\label{appendix_sec:proofs}
This section contains all omitted proofs in the paper.

\subsection{Proof of Lemma~\ref{lemma:equivalence_between_perspective_relaxation_and_convexification}}

\begin{namedlemma}
    [~\ref{lemma:equivalence_between_perspective_relaxation_and_convexification}]
    The closed convex hull of the set
    \begin{align*}
        \textstyle \left\{ (\tau, \bbeta, \bz) \middle|
        \| \bbeta \|_\infty \leq M, \, \bz \in \{0, 1\}^p, \, \mathbf{1}^\top \bz \leq k, \, \beta_j ( 1 - z_j) = 0 ~~ \forall j \in [p], \, \sum_{j \in [p]} \beta_j^2 \leq \tau \right\}
    \end{align*}
    is given by the set
    \begin{align*}
        \textstyle \left\{ (\tau, \bbeta, \bz)  \;\middle|\; -M z_j\leq \bbeta_j \leq M z_j ~ \forall j \in [p], \, \bz \in [0, 1]^p, \, \mathbf{1}^\top \bz \leq k, \, \sum_{j \in [p]} \beta_j^2 / z_j \leq \tau \right\}.
    \end{align*}
\end{namedlemma}

\begin{proof}
    Let $\mathcal T$ represent the first set mentioned in the statement of the lemma. Using the definition of the perspective function and applying the big-M formulation technique, we have
    \begin{align*}
        \textstyle \mathcal T = \left\{ (\tau, \bbeta, \bz)  \;\middle|\; -M z_j\leq \bbeta_j \leq M z_j ~ \forall j \in [p], \, \bz \in \{0, 1\}^p, \, \mathbf{1}^\top \bz \leq k, \, \sum_{j \in [p]} \beta_j^2 / z_j \leq \tau \right\}.
    \end{align*}
    As the epigraph of a perspective function constitutes a cone \citep[Lemma~1 \& 2]{shafiee2024constrained}, we may write $\mathcal T = \mathrm{Proj}_{(\tau, \bbeta, \bz)}(\overline {\mathcal T})$, where 
    \begin{align*}
        \textstyle \overline {\mathcal T} = \left\{ (\tau, \bbeta, \bt, \bz) \;\middle|\; \bm 1^\top \bt = \tau, \, \bz \in \{0, 1\}^p, \, \mathbf{1}^\top \bz \leq k, \, \bm A_j \begin{bmatrix} t_j \\ \beta_j \end{bmatrix} + \bm B_j z_j \in \mathbb K_j ~ \forall j \in [p] \right\}
    \end{align*}
    admits a mixed-binary conic representation with
    \begin{align*}
        \bm A = \begin{bmatrix} 1 & 0 \\ 0 & 1 \\ 0 & 0 \\ 0 & 1 \\ 0 & -1 \end{bmatrix}, \,
        \bm B = \begin{bmatrix} 0 \\ 0 \\ 0 \\ M \\ M \end{bmatrix}, \,
        \mathbb K_j = \mathbb L_+ \times \R_+ \times \R_+ \qquad \forall j \in [p].
    \end{align*}
    Here, $\mathbb L_+ \in \R^3$ denotes the rotated second order cone, that is, $\mathbb L_+ = \{ (t, \beta, z) \in \R_+ \times \R \times \R_+: \beta^2 \leq t z  \}$.
    Thus, using \citep[Lemma~4]{shafiee2024constrained}, the set $\overline{\mathcal T}$ satisfies all the requirements of \citep[Theorem~1]{shafiee2024constrained}, and therefore, its continuous relaxation gives the closed convex hull of $\overline{\mathcal T}$, that is,
    \begin{align*}
        \textstyle \cl \conv(\overline {\mathcal T}) = \left\{ (\tau, \bbeta, \bt, \bz) \;\middle|\; \bm 1^\top \bt = \tau, \, \bz \in [0, 1]^p, \, \mathbf{1}^\top \bz \leq k, \, \bm A_j \begin{bmatrix} t_j \\ \beta_j \end{bmatrix} + \bm B_j z_j \in \mathbb K_j ~ \forall j \in [p] \right\}.
    \end{align*}
    The prove concludes by applying Fourier-Motzkin elimination method to project out the variable $\bt$.
\end{proof} 

\begin{namedlemma}
    [~\ref{lemma:fenchel_conjugate_of_g_closed_form_expression}]
    The conjugate of $g$ is given by
    \begin{equation*}
        g^*(\balpha) = \TopSum_k({\bf H}_M(\balpha)).
    \end{equation*}
\end{namedlemma}

\begin{proof}
    Using the definition of the implicit function $g$ in~\eqref{eq:function_g_definition}, we have
    \begin{align}
        \label{eq:max:g*}
        g^*(\balpha) = \left\{
        \begin{array}{cl}
            \max & \balpha^\top \bbeta -  \frac{1}{2} \sum_{j \in [p]} {\beta_j^2}/{z_j} \\[1ex]
            \st & \bbeta \in \R^p, \, \bz \in [0, 1]^p, \, \bm 1^\top \bz \leq k, \\[1ex]
            & -M z_j \leq \beta_j \leq M z_j ~ \forall j \in [p]
        \end{array}
        \right.
    \end{align}
    For any fixed feasible $\bz$, the maximization problem over $\bbeta$ is a simple constrained quadratic problem, that can be solved analytically by the vector $\beta^\star$ whose $j$'th element is given by
    $\beta_j^\star = \sgn(\alpha_j) \min(\vert{\alpha_j}, M) z_j.$
    Substituting the optimizer $\beta^\star$, the objective function of the maximization problem in~\eqref{eq:max:g*} simplifies to
    \begin{align*}
        \balpha^\top \bbeta^\star - \frac{1}{2} \sum_{j \in [p]} {\beta_j^\star}^2 / z_j 
        &= \sum_{j \in [p]} \alpha_j \cdot \sgn(\alpha_j) \min(\vert{\alpha_j}, M) z_j - \frac{\left( \sgn\left( \alpha_j \right) \min\left(\vert{\alpha_j}, M \right) z_j \right)^2}{2z_j} \\
        &= \sum_{j \in [p]} ( \vert{\alpha_j} \min(\vert{\alpha_j}, M) - \frac{1}{2} \min(\alpha_j^2, M^2) ) z_j %\\
        % &= \begin{cases} \frac{1}{2} \alpha_j^2 z_j & \text{if } \vert{\alpha_j} \leq M  \\ \left( M \vert{\alpha_j} - \frac{1}{2} M^2 \right) z_j & \text{if } \vert{\alpha_j} > M
        % \end{cases} \\
        = H_M(\alpha_j) z_j,
    \end{align*}
    where the second equality holds as $\bz$ is a binary vector, and the last equality follows from the definition of the Huber loss function. We thus arrive at
    \begin{align*}
        g^*(\balpha) = \max_{\bz \in [0,1]^p} \left\{ \textstyle \sum_{j \in [p]} H_M (\alpha_j) z_j: \bm 1^\top \bz \leq k \right\} = \TopSum_k ({\mathbf{H}}_M(\balpha)).
    \end{align*}
    This completes the proof.
\end{proof}

\subsection{Proof of Lemma~\ref{lemma:equivalence_between_proximal_operator_and_huber_isotonic_regression}}

\begin{namedlemma}
    [~\ref{lemma:equivalence_between_proximal_operator_and_huber_isotonic_regression}]
    For any $\bmu \in \R^p$, we have 
    $$\prox_{\rho g^*}(\bmu) = \sgn(\bmu) \odot \bnu^\star, $$ 
    where $\odot$ denotes the Hadamard (element-wise) product, $\bnu^\star$ is the unique solution of the following optimization problem
    \begin{align}
        \label{A:obj:KyFan_Huber_isotonic_regression}
        \begin{array}{cl}
            \min\limits_{\bnu \in \R^p} & \frac{1}{2} \sum_{j \in [p]} (\nu_j - \vert{\mu_j})^2 + \rho \sum_{j \in \calJ} H_M (\nu_j) \\[2ex]
            \st & \quad \nu_j \geq \nu_l \; \text{ if } \; \vert{\mu_j} \geq \vert{\mu_l} ~~ \forall j, l \in [p],
        \end{array} 
    \end{align}
    and $\calJ$ is the set of indices of the top $k$ largest elements of~$ \vert{\mu_j}, j \in [p]$. 
\end{namedlemma}

\begin{proof}
    For simplicity, let $\balpha^\star = \prox_{\rho g^*}(\bmu)$, that is,
    \begin{align}
        \label{eq:alpha:star}
        \balpha^\star = \argmin_{\bm \alpha \in \R^p} ~ \frac{1}{2} \Vert{\bm \alpha - \bm \mu}_2^2 + \rho g^*(\bm \alpha).
    \end{align}
    We first show that $\sgn(\balpha^\star) = \sgn(\bmu)$ (step 1) and then establish that for every $j, l \in [p]$ with $\vert{\mu_j} \geq \vert{\mu_l}$, we have $\vert{\alpha_j^\star} \geq \vert{\alpha_l^\star}$ (step 2). We then conclude the proof using these observations.

    \begin{itemize}[label=$\diamond$,leftmargin=*]
        \item \textbf{Step 1.} We prove the sign-preserving property through a proof by contradiction. For the sake of contradiction, suppose that there exists some $j \in [p]$ such that $\sgn(\alpha_j^\star) \neq \sgn(\mu_j)$.
        Hence, we can construct a new $\balpha'$ by flipping the sign of $\alpha_j^\star$, i.e., $\alpha_j' = -\alpha_j^\star$, and keeping the rest of the elements the same as $\balpha^\star$.
        Now under the assumption that $\sgn(\alpha_j^\star) \neq \sgn(\mu_j)$, we have $\left\lvert{\alpha_j^\star - \mu_j}\right\rvert > \left\lvert{\lvert{\alpha_j^\star}\rvert - \lvert{\mu_j}\rvert}\right\rvert = \left\lvert{\alpha_j' - \mu_j}\right\rvert$, so the $j$-th term in the first summation of the objective function will decrease while everything else remains the same.
        This leads to a smaller objective value for $\balpha'$ than $\balpha^\star$, which contradicts the optimality of $\balpha^\star$.
        Thus, the claim follows.
        
        \item \textbf{Step 2.} We prove the relative magnitude-preserving property through a proof by contradiction. For the sake of contradiction, suppose that there exists some $j, l \in [p]$ such that $\vert{\mu_j} \geq \vert{\mu_l}$ but $\vert{\alpha_j^\star} < \vert{\alpha_l^\star}$.
        Then, we can construct a new $\balpha'$ by swapping $\alpha_j^\star$ and $\alpha_l^\star$, i.e., $\alpha_j' = \alpha_l^\star$ and $\alpha_l' = \alpha_j^\star$, and keeping the rest of the elements the same as $\balpha^\star$.
        Under the assumption that $\vert{\mu_j} \geq \vert{\mu_l}$ but $\vert{\alpha_j^\star} < \vert{\alpha_l^\star}$, we have $\left\lvert{\alpha_j^\star - \mu_j}\right\rvert + \left\lvert{\alpha_l^\star - \mu_l}\right\rvert > \left\lvert{\alpha_l^\star - \mu_j}\right\rvert + \left\lvert{\alpha_j^\star - \mu_l}\right\rvert =
        \left\lvert{\alpha_j' - \mu_j}\right\rvert + \left\lvert{\alpha_l' - \mu_l}\right\rvert$, so the sum of the $j$-th and $l$-th terms in the first summation of the objective function will decrease while everything else remains the same.
        This leads to a smaller objective value for $\balpha'$ than $\balpha^\star$, which contradicts the optimality of $\balpha^\star$. Thus, the claim~follows.
    \end{itemize}
    Using these two observations, we are ready to prove that $\balpha^\star = \sgn(\bmu) \odot \bnu^\star$.
    We first reparametrize the minimization problem~\eqref{eq:alpha:star} by substituting the decision variable $\balpha$ with a new variable $\bnu \in \R_+^p$ satisfying $\balpha = \sgn(\bmu) \odot \bnu$. By the sign-preserving property in step 1, it is easy to show the equivalence between the optimization problem in~\eqref{eq:alpha:star} and the following optimization problem
    \begin{align*}
        \min_{\bnu \in \R^p_+} ~ \textstyle \frac{1}{2} \sum_{j \in [p]} (\nu_j - \vert{\mu_j})^2 + \rho \TopSum_k \left( \mathbf{H}_M ( \bnu ) \right).
    \end{align*}
    By the relative magnitude-preserving property in step 2, we can further set the equivalence between the minimization problem in~\eqref{eq:alpha:star} and the following optimization problem
    \begin{align*}
        \begin{array}{cl}
            \displaystyle \min_{\bnu \in \R_+^p} & \frac{1}{2} \sum_{j \in [p]} (\nu_j - \vert{\mu_j})^2 + \rho \sum_{j \in \calJ} H_M (\nu_j), \\ 
            \st & \quad \nu_j \geq \nu_l \; \text{ if } \; \vert{\mu_j} \geq \vert{\mu_l}.
        \end{array} 
    \end{align*}
    Lastly, the nonnegative constraint on $\bnu$ can be removed as the second summation term in the objective function implies that $\nu_j \geq 0$. Thus, we have shown that any feasible point $\balpha$ in the minimization problem~\eqref{eq:alpha:star} can be reconstructed by any feasible point $\bnu$ in the minimization problem in the statement of lemma, while maintaining the same objective value. Hence, we may conclude that $\balpha^\star = \sgn(\bmu) \odot \bnu^\star$, as required.
\end{proof}

\subsection{Proof of Lemma~\ref{lemma:PAVA_algorithm_exact_solution}}






%Assuming that the input vector $\bmu$ has already been sorted so that the elements are in nonincreasing order in terms of their absolute values, the algorithm runs in linear time complexity, $O(p)$, where $p$ is the number of elements in the input vector $\bmu$.

\begin{namedlemma}
    [~\ref{lemma:PAVA_algorithm_exact_solution}]
    The vector $\hat \bnu$ in Algorithm~\ref{alg:PAVA_algorithm} solves~\eqref{obj:KyFan_Huber_isotonic_regression} exactly.
\end{namedlemma}

\begin{proof}
    The minimization problem~\eqref{obj:KyFan_Huber_isotonic_regression} is an instance of a generalized isotonic regression problem taking the form
    \begin{align}
        \label{obj:KyFan_Huber_isotonic_regression_rewritten_as_generalized_isotonic_regression}
        \min_{\bnu} \sum_{j=1}^{p} h_j(\nu_j) \quad \st \quad \nu_1 \geq \nu_2 \geq \cdots \geq \nu_J,
    \end{align}
    where $h_j(\nu) = \frac{1}{2} (\nu - \mu_j)^2 + \rho_j H_M(\nu)$, $\rho_j = \rho$ if $j \in \calJ$ and $\rho_j = 0$ otherwise, and the set $\calJ$ is the set of indices of top k largest elements of $\vert{\mu_j}$, as defined in the statement of Lemma~\ref{lemma:equivalence_between_proximal_operator_and_huber_isotonic_regression}.
    Thanks to~\cite{best2000minimizing,ahuja2001fast}, the optimizer of~\eqref{obj:KyFan_Huber_isotonic_regression_rewritten_as_generalized_isotonic_regression} satisfies two key properties: 
    \begin{itemize}[label=$\diamond$,leftmargin=*]
        \item \textbf{Property 1: Optimal solution for a merged block is single-valued.} 
        Suppose we have two adjacent blocks $[a_1, a_2]$ and $[a_2+1, a_3]$ such that the optimal solution of each block is single-valued, that is, the minimization problems
        \begin{align*}
            \left\{
            \begin{array}{cl}
                \min\limits_{\bnu_{a_1:a_2}} & \sum_{j=a_1}^{a_2} h_j(\nu_j) \\
                \st & \nu_{a_1} \geq \cdots \geq \nu_{a_2}
            \end{array}
            \right. \quad \text{and} \quad
            \left\{
            \begin{array}{cl}
                \min\limits_{\bnu_{a_2+1:a_3}} & \sum_{j=a_2+1}^{a_3} h_j(\nu_j) \\
                \st & \nu_{a_2+1} \geq \cdots \geq \nu_{a_3} \\
            \end{array}
            \right.
        \end{align*}
        are solved by $\bnu_{a_1:a_2}^\star$ and $\bnu_{a_2+1:a_3}^\star$ with $\nu_{a_1}^\star = \cdots = \nu_{a_2}^\star$ and $\nu_{a_2+1}^\star = \cdots = \nu_{a_3}^\star$, respectively.
        If $\nu_{a_1}^\star \leq \nu_{a_2+1}^\star$, then the optimal solution for the merged block $[a_1, a_3]$ is single-valued, that is, the minimization problem
        \begin{align*}
            \left\{
            \begin{array}{cl}
                \min\limits_{\bnu_{a_1:a_3}} & \sum_{j=a_1}^{a_3} h_j(\nu_j) \\
                \st & \nu_{a_1} \geq \cdots \geq \nu_{a_3}
            \end{array}
            \right.
        \end{align*}
        is solved by $\bnu_{a_1:a_3}^\star$ with $\nu_{a_1}^\star = \cdots = \nu_{a_3}^\star$.

        \item \textbf{Property 2: No isotonic constraint violation between single-valued blocks implies the solution is optimal.} Suppose that we have $s$ blocks $[a_1, a_2], [a_2+1, a_3], \ldots, [a_{s}+1, a_{s+1}]$ (with $a_1=1$ and $a_{s+1}=p$) such that the optimal solution for each block is single-valued, that is, $\nu^\star_{a_l+1} = \dots = \nu^\star_{a_{l+1}}$ for all $l \in [s]$. Then, if $\hat{\nu}_{a_1} \geq \hat{\nu}_{a_2+1} \geq \ldots \hat{\nu}_{a_{s}}$, then $\hat{\bnu}$ is the optimal solution to~\eqref{obj:KyFan_Huber_isotonic_regression_rewritten_as_generalized_isotonic_regression}.
    \end{itemize}
    
    Using these two properties, it is now easy to see why Algorithm~\ref{alg:PAVA_algorithm} returns the optimal solution. 
    We start by constructing blocks which have length 1.
    The initial value restricted to each block is optimal.
    Then, we iteratively merge adjacent blocks and update the values of $\nu_j$'s whenever there is a violation of the isotonic constraint.
    By the first property, the optimal solution for the merged block is single-valued.
    Therefore, we can compute the optimal solution for the merged block by solving a univariate optimization problem.
    We keep merging blocks until there is no isotonic constraint violation.
    When this happens, by construction, the solution for each block is single-valued and optimal.
    By the second property, the final vector $\hat{\bnu}$ is the optimal solution to~\eqref{obj:KyFan_Huber_isotonic_regression_rewritten_as_generalized_isotonic_regression}, as required.
\end{proof}

\subsection{Proof of Lemma~\ref{lemma:PAVA_merging_linear_time_complexity}}

\begin{namedlemma}
    [~\ref{lemma:PAVA_merging_linear_time_complexity}]
    The merging step (lines 11-14) in Algorithm~\ref{alg:PAVA_algorithm} can be performed in linear time complexity $\mathcal O(p)$.
\end{namedlemma}

\begin{proof}
A detailed implementation of line 11-14 (Step 3) of the PAVA Algorithm~\ref{alg:PAVA_algorithm} that achieves a linear time complexity is presented in Algorithm~\ref{alg:up_and_down_block_algorithm_for_merging_in_PAVA}. In the following, we first show that Algorithm~\ref{alg:up_and_down_block_algorithm_for_merging_in_PAVA} accomplishes the objective in lines 11-14 of Algorithm~\ref{alg:PAVA_algorithm}. We then establish that Algorithm~\ref{alg:up_and_down_block_algorithm_for_merging_in_PAVA} runs in linear time complexity.

\begin{algorithm}[hb]
    \caption{Up and Down Block Algorithm for Merging in PAVA}
    \label{alg:up_and_down_block_algorithm_for_merging_in_PAVA}
    \begin{flushleft}
    \textbf{Input:} vector $\bmu \in \mathbb{R}^p$, nonnegative weights $\brho \in \mathbb{R}_{+}^p$ ($\rho_{[1:k]}=\rho, \rho_{k+1:p}=0$), vector $\hat{\bnu}$ ($\hat{\nu}_j = \text{prox}_{\rho_j H_M}(\vert{\mu_j})$), integer $k \in \mathbb{N}$ (first $k$ elements subject to Huber penalty), and threshold $M > 0$ for the Huber loss function. %\\
    %\textbf{Output:} vector $\hat{\bnu} \in \mathbb{R}^p$, which is the optimal solution to Problem~\eqref{obj:KyFan_Huber_isotonic_regression}.\\
    \end{flushleft}
    \begin{algorithmic}[1]
        \STATE \COMMENT{Initialization for the first block}
        \STATE Initialize $b=1$, $P_1 = \rho_1$, $S_1 = \vert{\mu_1}$, $N_b=1$, $\nu_1$, $r_1 = 1$.
        \STATE $\nu_{\text{prev}} = \hat{\nu}_1$, $j=2$
        \WHILE{$j \leq n$}
            \STATE $b = b + 1$
            \STATE $P_b = \rho_j$, $S_b = \vert{\mu_j}$, $N_b=1$, $\nu = \hat{\nu}_j$
            \STATE \COMMENT{If the value for the current singleton block is greater that of the previous block (isotonic violation), merge the current block with the previous block}
            \IF{$\nu > v_{\text{prev}}$}
                \STATE $b = b - 1$
                \STATE $P_b = P_b + \rho_j$, \, $S_b = S_b + \vert{\mu_j}$, \, $N_b = N_b + 1$, \, $\nu = \text{prox}_{\frac{P_b}{N_b} H_{M}}(\frac{S_b}{N_b})$
                \STATE \COMMENT{Look forward: keep merging the current block with the next block if the isotonic violation persists}
                \WHILE{$j < n$ \AND $\nu \leq \hat{\nu}_j$}
                    \STATE $j = j + 1$
                    \STATE $P_b = P_b + \rho_j$, \, $S_b = S_b + \vert{\mu_j}$, \, $N_b = N_b + 1$, \, $\nu = \text{prox}_{\frac{P_b}{N_b} H_{M}}(\frac{S_b}{N_b})$
                \ENDWHILE
                \STATE \COMMENT{Look backward: keep merging the current block with the previous block if the isotonic violation persists}
                \WHILE{$b > 1$ \AND $\nu_{b-1} < \nu$}
                    \STATE $b = b - 1$
                    \STATE $P_b = P_b + P_{b+1}$, \, $S_b = S_b + S_{b+1}$, \, $N_b = N_b + N_{b+1}$, \, $\nu = \text{prox}_{\frac{P_b}{N_b} H_{M}}(\frac{S_b}{N_b})$
                \ENDWHILE
            \ENDIF
            \STATE \COMMENT{Save the current block's value and the index of the last element in the block}
            \STATE $\nu_b = \nu$, $r_b = j$
            \STATE \COMMENT{Start fresh on the next element}
            \STATE $\nu_{\text{prev}} = \nu$, $j = j + 1$
        \ENDWHILE
        \STATE \COMMENT{Modify the output vector to have the same new value for all elements in each block}
        \FOR{$l = 1, ..., b$}
            \STATE $\hat{\nu}_{[r_{l-1}+1:r_l]} = \nu_l$
        \ENDFOR
        \STATE \textbf{return} $\hat{\bnu}$
    \end{algorithmic}
\end{algorithm}

% \begin{algorithm}[H]
%     \caption{Modified PAVA with Huber Penalty for Nonincreasing Isotonic Regression}
%     \label{alg:up_and_down_block_algorithm_for_merging_in_PAVA}
%     \begin{flushleft}
%     \textbf{Input:} vector $\bmu \in \mathbb{R}^n$ (observations), nonnegative weights $w \in \mathbb{R}_{\ge 0}^n$, integer $k \in \mathbb{N}$ (first $k$ elements subject to Huber penalty), scalar $\rho > 0$ (Huber penalty coefficient), and threshold $M > 0$.\\
%     \textbf{Output:} vector $x \in \mathbb{R}^n$ (monotone, nonincreasing sequence approximating $y$).\\
%     \end{flushleft}
%     \begin{algorithmic}[1]
%         \STATE $M, P$
%         \STATE $y_1 \gets y,\; y_2 \gets y,\; w_1 \gets w,\; w_2 \gets w$
%         \FOR{$j = 0$ to $k-1$}
%             \STATE $w_1[j] \gets w[j] + \frac{\rho}{2}$
%             \STATE $y_1[j] \gets y[j] \cdot \frac{w[j]}{w_1[j]}$
%             \STATE $y_2[j] \gets y[j] - \frac{\rho M}{2w_2[j]}$
%         \ENDFOR
%         \STATE Initialize boolean array $\text{use\_y1}$ of length $n$:
%         \FOR{$j = 0$ to $n-1$}
%             \STATE $\text{use\_y1}[j] \gets (y_1[j] \leq M)$
%         \ENDFOR
    
%         \STATE Allocate arrays $x_1^{\text{block}}, w_1^{\text{block}}, x_2^{\text{block}}, w_2^{\text{block}}$, and $\text{use\_x1\_block}$ of length $n$
%         \STATE Allocate array $r$ of length $n+1$
%         \STATE $r[0] \gets -1,\; r[1] \gets 0$
%         \STATE $b \gets 1$ \COMMENT{Number of blocks}
    
%         \STATE $x_1^{\text{block}}[0] \gets y_1[0],\; w_1^{\text{block}}[0] \gets w_1[0]$
%         \STATE $x_2^{\text{block}}[0] \gets y_2[0],\; w_2^{\text{block}}[0] \gets w_2[0]$
%         \STATE $\text{use\_x1\_block}[0] \gets \text{use\_y1}[0]$
    
%         \STATE $j \gets 1$
%         \WHILE{$j < n$}
%             \STATE $b \gets b + 1$
    
%             \STATE \textbf{Compute current values:}
%             \IF{$\text{use\_y1}[j] = \text{True}$}
%                 \STATE $x_{\text{curr}} \gets y_1[j],\; w_{\text{curr}} \gets w_1[j]$
%             \ELSE
%                 \STATE $x_{\text{curr}} \gets \max(y_2[j], M),\; w_{\text{curr}} \gets w_2[j]$
%             \ENDIF
    
%             \STATE \textbf{Compute previous block values:}
%             \STATE $\ell \gets b - 2$ \COMMENT{Index of previous block}
%             \IF{$\text{use\_x1\_block}[\ell] = \text{True}$}
%                 \STATE $x_{\text{prev}} \gets x_1^{\text{block}}[\ell],\; w_{\text{prev}} \gets w_1^{\text{block}}[\ell]$
%             \ELSE
%                 \STATE $x_{\text{prev}} \gets \max(x_2^{\text{block}}[\ell], M),\; w_{\text{prev}} \gets w_2^{\text{block}}[\ell]$
%             \ENDIF
    
%             \STATE \textbf{Check for nonincreasing violation:} 
%             \IF{$x_{\text{prev}} < x_{\text{curr}}$}
%                 \STATE $b \gets b - 1$
%                 \STATE Merge current element with previous block:
    
%                 \STATE $S_1 \gets (w_1^{\text{block}}[b-1] \cdot x_1^{\text{block}}[b-1]) + (w_1[j] \cdot y_1[j])$
%                 \STATE $W_1 \gets w_1^{\text{block}}[b-1] + w_1[j]$
%                 \STATE $x_{1,\text{merged}} \gets S_1 / W_1$
    
%                 \STATE $S_2 \gets (w_2^{\text{block}}[b-1] \cdot x_2^{\text{block}}[b-1]) + (w_2[j] \cdot y_2[j])$
%                 \STATE $W_2 \gets w_2^{\text{block}}[b-1] + w_2[j]$
%                 \STATE $x_{2,\text{merged}} \gets S_2 / W_2$
    
%                 \STATE $\text{use\_x1\_merged} \gets (x_{1,\text{merged}} \leq M)$
    
%                 \COMMENT{k-up step: merge forward if violation persists}
%                 \WHILE{$j < n-1$ \AND $\bigl(x_{1,\text{merged}} \cdot \text{use\_x1\_merged} + (1-\text{use\_x1\_merged}) \cdot \max(x_{2,\text{merged}}, M)\bigr) \leq \bigl(y_1[j+1] \cdot \text{use\_y1}[j+1] + (1-\text{use\_y1}[j+1]) \cdot \max(y_2[j+1], M)\bigr)$}
%                     \STATE $j \gets j + 1$
%                     \STATE $S_1 \gets S_1 + w_1[j] \cdot y_1[j],\; W_1 \gets W_1 + w_1[j],\; x_{1,\text{merged}} \gets S_1 / W_1$
%                     \STATE $S_2 \gets S_2 + w_2[j] \cdot y_2[j],\; W_2 \gets W_2 + w_2[j],\; x_{2,\text{merged}} \gets S_2 / W_2$
%                     \STATE $\text{use\_x1\_merged} \gets (x_{1,\text{merged}} \leq M)$
%                 \ENDWHILE
    
%                 \COMMENT{k-down step: merge backward if violation persists}
%                 \WHILE{$b > 1$ \AND $\bigl(x_1^{\text{block}}[b-2] \cdot \text{use\_x1\_block}[b-2] + (1-\text{use\_x1\_block}[b-2]) \cdot \max(x_2^{\text{block}}[b-2], M)\bigr) < \bigl(x_{1,\text{merged}} \cdot \text{use\_x1\_merged} + (1-\text{use\_x1\_merged}) \cdot \max(x_{2,\text{merged}}, M)\bigr)$}
%                     \STATE $b \gets b - 1$
%                     \STATE $S_1 \gets S_1 + w_1^{\text{block}}[b-1] \cdot x_1^{\text{block}}[b-1],\; W_1 \gets W_1 + w_1^{\text{block}}[b-1],\; x_{1,\text{merged}} \gets S_1 / W_1$
%                     \STATE $S_2 \gets S_2 + w_2^{\text{block}}[b-1] \cdot x_2^{\text{block}}[b-1],\; W_2 \gets W_2 + w_2^{\text{block}}[b-1],\; x_{2,\text{merged}} \gets S_2 / W_2$
%                     \STATE $\text{use\_x1\_merged} \gets (x_{1,\text{merged}} \leq M)$
%                 \ENDWHILE
    
%                 \STATE $x_1^{\text{block}}[b-1] \gets x_{1,\text{merged}},\; w_1^{\text{block}}[b-1] \gets W_1$
%                 \STATE $x_2^{\text{block}}[b-1] \gets x_{2,\text{merged}},\; w_2^{\text{block}}[b-1] \gets W_2$
%                 \STATE $\text{use\_x1\_block}[b-1] \gets \text{use\_x1\_merged}$
%                 \STATE \textit{No violation}
%             \ELSE
%                 \STATE \COMMENT{No violation}
%                 \STATE $x_1^{\text{block}}[b-1] \gets y_1[j],\; w_1^{\text{block}}[b-1] \gets w_1[j]$
%                 \STATE $x_2^{\text{block}}[b-1] \gets y_2[j],\; w_2^{\text{block}}[b-1] \gets w_2[j]$
%                 \STATE $\text{use\_x1\_block}[b-1] \gets \text{use\_y1}[j]$
%             \ENDIF
    
%             \STATE $r[b] \gets j$
%             \STATE $j \gets j + 1$
%         \ENDWHILE
    
%         \COMMENT{Expand blocks to form final $x$}
%         \STATE $x \gets$ empty array of length $n$
%         \STATE $f \gets n-1$
    
%         \FOR{$\ell = b$ down to $1$}
%             \STATE $start\_idx \gets r[\ell-1] + 1$
%             \STATE $end\_idx \gets r[\ell]$
%             \IF{$\text{use\_x1\_block}[\ell-1] = \text{True}$}
%                 \STATE $block\_value \gets x_1^{\text{block}}[\ell-1]$
%             \ELSE
%                 \STATE $block\_value \gets \max(x_2^{\text{block}}[\ell-1], M)$
%             \ENDIF
%             \FOR{$idx = end\_idx$ down to $start\_idx$}
%                 \STATE $x[idx] \gets block\_value$
%             \ENDFOR
%             \STATE $f \gets start\_idx - 1$
%         \ENDFOR
    
%         \STATE \textbf{return} $x$
%     \end{algorithmic}
% \end{algorithm}


To prove the first claim, we show that the parameters $P_b, S_b,$ and $\nu_b$ amount to
\begin{align*}
    \textstyle
    P_b = \sum_{j \in \calB(b)} \rho_j, ~ 
    S_b = \sum_{j \in \calB(b)} \vert{\mu_j}, ~ 
    \nu_b = \prox_{\sum_{j \in \calB(b)} \rho_j H_M}(|\mu_j|)
\end{align*}
for each block index $b$, where $\calB(b)$ denoting the set of indices in the $b$'th block. It is easy to verify that Algorithm~\ref{alg:up_and_down_block_algorithm_for_merging_in_PAVA} recursively computes $P_b$ and $S_b$. Thus, we will focus on $\nu_b$.
Note that the computation of the proximal operator in $\nu_b$ is reduced to solving a univariate optimization problem for each $b$ and satisfies
\begin{align*}
    \nu_b =& \argmin_{v \in \R} \sum_{j \in \calB(b)} \left( \frac{1}{2} (v - \vert{\mu_j})^2 + \rho_j H_M(v) \right) \\
    %= & \argmin_{v} \sum_{j \in \calB(b)} \left( \frac{1}{2} (v^2 - 2v\vert{\mu_j} + \mu_j^2) + \rho_j H_M(v) \right) \\
    = & \argmin_{v} \sum_{j \in \calB(b)} \left( \frac{1}{2} v^2 - v\vert{\mu_j} + \rho_j H_M(v) \right) \\
    %= & \argmin_{v} \left( \sum_{j \in \calB(b)} \frac{1}{2} v^2 - \sum_{j \in \calB(b)} v\vert{\mu_j} + \sum_{j \in \calB(b)} \rho_j H_M(v) \right) \\
    %= & \argmin_{v} \left( N_b \frac{1}{2} v^2 - S_b \vert{\mu_j} + P_b H_M(v) \right) \\
    = & \argmin_{v} \left( \frac{1}{2} v^2 - \frac{S_b}{N_b} \vert{\mu_j} + \frac{P_b}{N_b} H_M(v) \right) 
    = \argmin_{v} \left( \frac{1}{2} \left( v - \frac{S_b}{N_b} \right)^2 + \frac{P_b}{N_b} H_M(v) \right) 
    = \prox_{\frac{P_b}{N_b} H_{M}}(\frac{S_b}{N_b}).
\end{align*}

Thus, Algorithm~\ref{alg:up_and_down_block_algorithm_for_merging_in_PAVA} merges two adjacent blocks if the isotonic violation persists, and the output of the proximal operator is the minimizer of the univariate function in the merged block.
This is exactly the same as the objective in lines 11-14 of Algorithm~\ref{alg:PAVA_algorithm}. Hence, the first claim follows.

To show that the algorithm runs in linear time, notice that in the while loop $j \leq p$ in Algorithm~\ref{alg:up_and_down_block_algorithm_for_merging_in_PAVA}, the variable $j$ is incremented by $1$ in each iteration, and the loop terminates when $j = p$.
Although there are two while loops inside the main while loop, the total number of iterations in the two inner while loops is at most $p$.
This is because we start with $p$ blocks, and each iteration of the inner while loops either merges two blocks forward or merges two blocks backward.
The total number of merging operations is at most $p-1$.
Thus, the total number of iterations in the while loop $j \leq p$ is at most $p$.
Lastly, since we can evaluate the proximal operator of the Huber loss function, $\mathbf{H}_M$, in constant time complexity, the total time complexity of Algorithm~\ref{alg:up_and_down_block_algorithm_for_merging_in_PAVA} is $O(p)$.
\end{proof}

\subsection{Proof of Theorem~\ref{theorem:pava_algorithm_linear_time_complexity_and_exact_solution}}

\begin{namedtheorem}
    [~\ref{theorem:pava_algorithm_linear_time_complexity_and_exact_solution}]
    For any $\bmu \in \R^p$, Algorithm~\ref{alg:PAVA_algorithm} returns the \textit{exact} evaluation of $\prox_{\rho g^*}(\bmu)$ in $\tilde {\mathcal O}(p)$.
\end{namedtheorem}

\begin{proof}
    By Lemmas~\ref{lemma:equivalence_between_proximal_operator_and_huber_isotonic_regression} and~\ref{lemma:PAVA_algorithm_exact_solution}, the output of Algorithm~\ref{alg:PAVA_algorithm} computes $\prox_{\rho g^*}$ exactly. 
    The linear time complexity statement also holds thanks to Lemma~\ref{lemma:PAVA_merging_linear_time_complexity}.
\end{proof}

\subsection{Proof of Theorem~\ref{theorem:compute_g_value_algorithm_correctness}}

\begin{namedtheorem}
    [~\ref{theorem:compute_g_value_algorithm_correctness}]
        For any $\bbeta \in \R^p$, Algorithm~\ref{alg:compute_g_value_algorithm} computes the exact value of $g(\bbeta)$, defined in~\eqref{eq:function_g_definition}, in $\mathcal O(p + p \log k)$.
\end{namedtheorem}

\begin{proof}
% Add proof content here

We first show that the algorithm correctly computes the value of $g(\bbeta)$ and then analyze its computational complexity. Define the mixed-binary set
\begin{align*}
    \calS_0 = \left\{ (t, \bbeta) \;\middle|\; \textstyle \frac{1}{2} \sum_{j \in [p]} \beta_j^2 \leq t, \, \|\bbeta \|_\infty \leq M, \, \|\bbeta \|_0 \leq k \right\}.
\end{align*}
Using the perspective and big-M reformulation techniques, the set $\calS_0$ admits the equivalent representation
\begin{align*}
    \calS_0 = \left\{ (t, \bbeta) \;\middle|\; \exists \bz \in \{0,1\}^p ~ \st ~ \textstyle \frac{1}{2} \sum_{j \in [p]} \beta_j^2 / z_j \leq t, \, \bm 1^\top \bz \leq k, \, -M z_j \leq \beta_j \leq M z_j ~~ \forall j \in [p] \right\}.
\end{align*}
Following the proof of Lemma~\ref{lemma:equivalence_between_perspective_relaxation_and_convexification}, one can show that the closed convex hull of $\calS_0$ is given by 
\begin{align*}
    \cl \conv(\calS_0) = \left\{ (t, \bbeta) \;\middle|\; \exists \bz \in [0,1]^p ~ \st ~ \textstyle \frac{1}{2} \sum_{j \in [p]} \beta_j^2 / z_j \leq t, \, \bm 1^\top \bz \leq k, \, -M z_j \leq \beta_j \leq M z_j ~~ \forall j \in [p] \right\}.
\end{align*}
Therefore, the implicit function $g$ can be written as the evaluation of the support function of $\cl\conv(\calS_0)$ at $(1, \bm 0)$, that is,
\begin{align}
    \label{eq:g:S0}
    g(\bbeta) = \min  \{ t : (t, \bbeta) \in \cl\conv(\calS_0) \}.
\end{align}
Notice that the set $\calS_0$ is sign- and permutation-invariants. Hence, by ~\citep[Theorem~4]{kim2022convexification}, its closed convex hull admits the following (different) lifted represenation
\begin{align}
    \label{eq:diff:conv}
    \cl \conv(\calS_0) = \left\{ (t, \bbeta) \;\middle|\; \exists \bphi \in \R^p ~ \st ~
    \begin{array}{l}
        \frac{1}{2} \sum_{j \in [p]} \phi_j^2 \leq t, \, \vert{\bbeta} \preceq_m \bphi, \\
        0 \leq \phi_k \leq \ldots \leq \phi_1 \leq M, \\
        \phi_{k+1} = \phi_{k+2} = \ldots = \phi_n = 0 
    \end{array}
    \right\},
\end{align}
where the absolute value operator $\vert{\cdot}$ is applied to a vector in an element-wise fashion, and the constraint $\vert{\bbeta} \preceq_m \bphi$ denotes that $\bphi$ majorizes $\vert{\bbeta}$, that is,
\begin{align*}
    \textstyle \vert{\bbeta} \preceq_m \bphi  \quad \iff \quad  \sum_{j \in [l]} \vert{\beta_j} \leq \sum_{j \in [l]} \phi_j \quad \forall l \in [p-1] \quad \text{and} \quad \sum_{j \in [p]} \phi_j = \sum_{j \in [p]} \vert{\beta_j}.
\end{align*}
Using this alternative convex hull description of $\calS_0$ in~\eqref{eq:diff:conv} and the implicit formulation~\eqref{eq:g:S0}, we may conclude that
\begin{align}
    g(\bbeta) = \min\limits_{\bphi \in \R^p}
    \textstyle \left\{ \frac{1}{2} \sum_{j \in [p]} \phi_j^2 :  \vert{\bbeta} \preceq_m \bphi, \, 0 \leq \phi_k \leq \ldots \leq \phi_1 \leq M, \, \phi_{k+1} = \phi_{k+2} = \ldots = \phi_n = 0
    \right\}. \label{appendix_obj:compute_g_value_majorization_formulation}
\end{align}
In the following we show that Algorithm~\ref{alg:compute_g_value_algorithm} can efficiently solve the minimization problem in~\eqref{appendix_obj:compute_g_value_majorization_formulation}. At the first iteration $j=1$ of the algorithm, we have
\begin{align*}
    \textstyle k \phi_1 \geq \sum_{j \in [k]} \phi_j = \sum_{j \in [p]} \phi_j \geq \sum_{j \in [p]} \vert{\beta_j} \quad \Rightarrow \quad \phi_1 \geq \frac{1}{k} \sum_{j \in [p]} \vert{\beta_j}.
\end{align*}
At the same time, we also need to satisfy $\vert{\beta_1} \leq \phi_1$ from the first majorization constraint. We now discuss two cases
\begin{itemize}[label=$\diamond$,leftmargin=*]
    \item \textbf{Case 1:} If $\frac{1}{k} \sum_{j \in [p]} \vert{\beta_j} \geq \vert{\beta_1}$, in order to solve the minimization problem in~\eqref{appendix_obj:compute_g_value_majorization_formulation}, we set $\phi_1 = \frac{1}{k} \sum_{j=1}^n \vert{\beta_j}$. Notice that $\phi_1 \leq M$ is automatically satisfied because $\phi_1 = \frac{1}{k} \sum_{j \in [p]} \vert{\beta_j} = \frac{1}{k} \sum_{j \in [p]} M z_j \leq M$. This leads to $\phi_2 = \ldots = \phi_k = \frac{1}{k} \sum_{j \in [p]} \vert{\beta_j}$.
    To see this, for the sake of contradition, assume that $\exists j \in \{2, \ldots, k\}$ such that $\phi_j < \frac{1}{k} \sum_{j \in [p]} \vert{\beta_j}$. 
    Since $\phi_j \leq \phi_1 = \frac{1}{k} \sum_{j \in [p]} \vert{\beta_j}$, we have $\sum_{j \in [k]} \phi_j < \sum_{j \in [k]} \frac{1}{k} \sum_{j \in [p]} \vert{\beta_j} = \sum_{j \in [p]} \vert{\beta_j}$, which contradicts the majorization constraint.

    \item \textbf{Case 2:} If $\frac{1}{k} \sum_{j \in [n]} \vert{\beta_j} < \vert{\beta_1}$, we can set $\phi_1 = \vert{\beta_1}$. Notice that $\phi_1 \leq M$ is automatically satisfied because $\vert{\beta_1} \leq M z_1 \leq M$.
    Then we are left with $k-1$ coefficients to set, and we can follow the same argument as we did for $j=1$ with slight difference that the majorization constraints are changed to
    \begin{align*}
        \textstyle
        \sum_{j=2}^l \phi_j \geq \sum_{j=2}^l \vert{\beta_j} \quad \forall l \in \{2, \ldots, p-1\} \quad \text{and} \quad \sum_{j=2}^p \phi_j = \sum_{j=2}^p \vert{\beta_j}.
    \end{align*}
\end{itemize}
We repeat this process until we set all $k$ coefficients $\phi_1, \ldots, \phi_k$, as implemented by Algorithm~\ref{alg:compute_g_value_algorithm}.
The output of the algorithm coincides with the optimal value of the minimization problem in~\eqref{appendix_obj:compute_g_value_majorization_formulation}. Hence, the first claim follows.

As for the complexity claim, it is easy to see that Algorithm~\ref{alg:compute_g_value_algorithm}.
only requires partial sorting step on Line 2, which has a complexity of $\mathcal O(p \log k)$. The summation step on Line 3 has a complexity of $\mathcal O(p)$. The for-loop step on Line 4-8 has a complexity of $\mathcal O(k)$, so does the final summation step on Line 9. Therefore, the overall computational complexity of Algorithm~\ref{alg:compute_g_value_algorithm} is $\mathcal O(p + p \log k)$. This concludes the proof.
\end{proof}

\newpage
\section{Experimental Setup Details}
\label{appendix:experimental_setup}

\subsection{Setup for Evaluating Proximal Operators}
\label{appendix:setup_for_evaluating_proximal_operators}
The synthetic data generation process is as follows.
We sample the input vector $\bgamma \in \bbR^p$ from the standard multivariate Gaussian distribution, $\bgamma \sim \calN(\mathbf{0}, \bI_p)$, where $\bI_p$ denotes the identity matrix with dimension $p$.
We vary the dimension $p \in \{2^0, 2^1, ..., 2^{10}\} \times 10^2$ and set the cardinality $k$ to be $10$, the box constraint $M$ to be $1.0$, and the weight parameter $\rho$  to be $1.0$.
We report the running time for evaluating these proximal operators.
To obtain the mean and standard deviation of the running time, we repeat each setting 5 times, each with a different random seed.

\subsection{Setup for Solving the Perspective Relaxation}
\label{appendix:setup_for_solving_the_perspective_relaxation}

We generate our synthetic datasets in the following procedure.
First, we sample each feature vector $\bx_i \in \bbR^p $ from a Gaussian distribution, $\bx_i \sim \calN(\mathbf{0}, \bSigma)$, where the covariance matrix has entries $\Sigma_{jl} = \sigma^{\vert{j-l}}$.
The variable $\sigma \in (0, 1)$ controls the features correlation: if we increase $\sigma$, feature columns in the design matrix $\bX$ become more correlated.
Throughout the experimental section, we set $\sigma=0.5$.
Next, we create the sparse coefficient vector $\bbeta^*$ with $k$ equally spaced nonzero entries, where $\beta^*_j = 1$ if $j \text{ mod } (p/k) = 0$ and $\beta^*_j = 0$ otherwise.
After these two steps, we build the prediction vector $\by$.
If our loss function is squared error loss (regression task), we set $y_i = \bx_i^T \bbeta^* + \epsilon_i$, where $\epsilon_i$ is a Gaussian random noise with $\epsilon_i \sim \calN(0, \frac{\Vert{\bX \bbeta^*}}{\text{SNR}})$, and $\text{SNR}$ stands for the signal-to-noise ratio.
In all our experiments, we choose $\text{SNR}=5$.
If our loss function is logistic loss (classification task), we set $y_i \sim Bern(\bx_i^T \bbeta^* + \epsilon_i)$, where $Bern(P)$ is a Bernoulli random variable with $\bbP(y_i = 1) = P$ and $\bbP(y_i = -1) = 1 - P$.
For this experiment, we vary the feature dimension $p \in \{1000, 2000, 4000, 8000, 16000\}$.
We control the sample size by using a parameter called $n$-to-$p$ ratio, or sample to feature ratio.
For the results in the main paper, we set $n$-to-$p$ ratio to be $1.0$, the box constraint $M$ to be $2$, the number of nonzero coefficients k (also the cardinality constraint) to be $10$, and $\ell_2$ regularization coefficient $\lambda_2$ to be $1.0$.
Again, we report and compare the running times, with means and standard deviations calculated based on 5 repeated simulations with different random seeds.

\subsection{Setup for Certifying Optimality}
\label{appendix:setup_for_certifying_optimality}

\paragraph{Datasets and Preprocessing}
We run on both synthetic and real-world datasets.
For the synthetic datasets, we run on the largest synthetic instances ($n=16000$ and $p=16000$).
For the real-world datasets, we use the dataset cancer drug response~\cite{liu2020deepcdr} for linear regression and DOROTHEA~\cite{asuncion2007uci} for logistic regression.

The cancer drug response dataset has 822 samples and orginally has 34674 features.
However, many feature only has a single value, so we prune all these features, which result in 2200 features.
The DOROTHEA dataset has 1950 samples and 100000 features.
After pruning redundant features, we have 91598 features.

For both the cancer drug response and DOROTHEA dataset, we center each feature to have mean $0$ and norm equal to $1$.

\paragraph{Choice of Hyperparameters}
For the cardinality constraint $k$, we set $k=10$ for both synthetic datasets.
For the cancer drug response dataset, we set $k=5$.
For DOROTHEA, we set $k=15$.
In practice, this choice can be made more judiciously by doing 5 fold cross validation with a heuristic sparse learning algorithm first.
However, since our emphasis here is simply to compare certification speed, we just pick a variety of $k$'s.

For the $\ell_2$ regularization coefficient, we set $\lambda_2=1$.
For the box constraint, we set $M=2$ for the synthetic datasets and DOROTHEA.
The infinity norm of the final optimal solution less than this value.
For the cancer drug response dataset, we set $M=5$, which is also bigger than the infinity norm of the final optimal solution.

\paragraph{Branch and Bound}
For our method, we write a customized branch-and-bound (BnB) framework.
We use Algorithm~\ref{alg:main_algorithm} to solve the relaxation at each node and use Equation~\eqref{eq:fenchel_duality_theorem_F_y(Ax)+G(x)} to calculate the safe lower bound to prune the search space.
To find feasible solutions, we use an effective approach called beamsearch~\cite{liu2022fasterrisk} from the existing literature.
For branching, we branch on the feature based on the best feasible solution found by the beamsearch algorithm at each node.
For the nonzero coefficients of this solution, we branch on the variable which would lead to the largest loss increase if the coefficient to $0$.
The intuition is that such a variable is important and should be branched early in the BnB framework.

\subsection{Computing Platforms}
When investigating how much GPU can accelerate our computation, we run the experiments with both CPU and GPU implementations on the Nvidia RTXA5000s.
For everything else, we run the experiments with the CPU implementation on AMD Milan with CPU speed 2.45 Ghz and 8 cores.

\section{Additional Numerical Results}
\label{appendix:numerical}

% \subsection{Perturbation Study regarding Proximal Operators}
% \label{appendix:numerical_proximal_operators}

% \subsubsection{Perturbation Study on $k$ Values}

% \subsubsection{Perturbation Study on $M$ Values}

% \subsubsection{Perturbation Study on $\rho$ Values}

\subsection{Perturbation Study regarding Solving the Perspective Relaxation}
\label{appendix:numerical_solve_convex_relaxation}

\subsubsection{Perturbation Study on $M$ Values}

\begin{figure*}[!ht]
    \centering
    \includegraphics[width=0.9\textwidth]{sections/Plots/big_M_perturbation/convex_relaxation_comparison_n_p_ratio_1.0_M_1.2.png}
    \caption{Solve the perspective relaxation in Problem~\eqref{obj:original_sparse_problem_perspective_formulation_convex_relaxation}.
    We set $M=1.2$, $\lambda_2=1.0$, $n$-to-$p$ ratio to be 1.}
    \label{fig:solve_convex_relaxation_M_1.2_lambda2_1.0_n_p_ratio_1.0}
\end{figure*}

\begin{figure*}[!ht]
    \centering
    \includegraphics[width=0.9\textwidth]{sections/Plots/big_M_perturbation/convex_relaxation_comparison_n_p_ratio_1.0_M_1.5.png}
    \caption{Solve the perspective relaxation in Problem~\eqref{obj:original_sparse_problem_perspective_formulation_convex_relaxation}.
    We set $M=1.5$, $\lambda_2=1.0$, $n$-to-$p$ ratio to be 1.}
    \label{fig:solve_convex_relaxation_M_1.5_lambda2_1.0_n_p_ratio_1.0}
\end{figure*}

\begin{figure*}[!ht]
    \centering
    \includegraphics[width=0.9\textwidth]{sections/Plots/big_M_perturbation/convex_relaxation_comparison_n_p_ratio_1.0_M_3.0.png}
    \caption{Solve the perspective relaxation in Problem~\eqref{obj:original_sparse_problem_perspective_formulation_convex_relaxation}.
    We set $M=3.0$, $\lambda_2=1.0$, $n$-to-$p$ ratio to be 1.}
    \label{fig:solve_convex_relaxation_M_3.0_lambda2_1.0_n_p_ratio_1.0}
\end{figure*}

\begin{figure*}[!ht]
    \centering
    \includegraphics[width=0.9\textwidth]{sections/Plots/big_M_perturbation/convex_relaxation_comparison_n_p_ratio_1.0_M_5.0.png}
    \caption{Solve the perspective relaxation in Problem~\eqref{obj:original_sparse_problem_perspective_formulation_convex_relaxation}.
    We set $M=5.0$, $\lambda_2=1.0$, $n$-to-$p$ ratio to be 1.}
    \label{fig:solve_convex_relaxation_M_5.0_lambda2_1.0_n_p_ratio_1.0}
\end{figure*}

\begin{figure*}[!ht]
    \centering
    \includegraphics[width=0.9\textwidth]{sections/Plots/big_M_perturbation/convex_relaxation_comparison_n_p_ratio_1.0_M_10.0.png}
    \caption{Solve the perspective relaxation in Problem~\eqref{obj:original_sparse_problem_perspective_formulation_convex_relaxation}.
    We set $M=10.0$, $\lambda_2=1.0$, $n$-to-$p$ ratio to be 1.}
    \label{fig:solve_convex_relaxation_M_10.0_lambda2_1.0_n_p_ratio_1.0}
\end{figure*}

\newpage

\subsubsection{Perturbation Study on $\lambda_2$ Values}


\begin{figure*}[!ht]
    \centering
    \includegraphics[width=0.9\textwidth]{sections/Plots/lambda2_perturbation/convex_relaxation_comparison_lambda2_0.1.png}
    \caption{Solve the perspective relaxation in Problem~\eqref{obj:original_sparse_problem_perspective_formulation_convex_relaxation}.
    We set $M=2.0$, $\lambda_2=0.1$, $n$-to-$p$ ratio to be 1.}
    \label{fig:solve_convex_relaxation_M_2.0_lambda2_0.1_n_p_ratio_1.0}
\end{figure*}

\begin{figure*}[!ht]
    \centering
    \includegraphics[width=0.9\textwidth]{sections/Plots/lambda2_perturbation/convex_relaxation_comparison_lambda2_10.0.png}
    \caption{Solve the perspective relaxation in Problem~\eqref{obj:original_sparse_problem_perspective_formulation_convex_relaxation}.
    We set $M=2.0$, $\lambda_2=10.0$, $n$-to-$p$ ratio to be 1.}
    \label{fig:solve_convex_relaxation_M_2.0_lambda2_10.0_n_p_ratio_1.0}
\end{figure*}

\newpage

\subsubsection{Perturbation Study on $n$-to-$p$ Ratios}


\begin{figure*}[!ht]
    \centering
    \includegraphics[width=0.9\textwidth]{sections/Plots/n_p_ratio_perturbation/convex_relaxation_comparison_n_p_ratio_0.1_M_2.0.png}
    \caption{Solve the perspective relaxation in Problem~\eqref{obj:original_sparse_problem_perspective_formulation_convex_relaxation}.
    We set $M=2.0$, $\lambda_2=1.0$, $n$-to-$p$ ratio to be 10.0.}
    \label{fig:solve_convex_relaxation_M_2.0_lambda2_1.0_n_p_ratio_10.0}
\end{figure*}


\begin{figure*}[!ht]
    \centering
    \includegraphics[width=0.9\textwidth]{sections/Plots/n_p_ratio_perturbation/convex_relaxation_comparison_n_p_ratio_10.0_M_2.0.png}
    \caption{Solve the perspective relaxation in Problem~\eqref{obj:original_sparse_problem_perspective_formulation_convex_relaxation}.
    We set $M=2.0$, $\lambda_2=1.0$, $n$-to-$p$ ratio to be 0.1.}
    \label{fig:solve_convex_relaxation_M_2.0_lambda2_1.0_n_p_ratio_0.1}
\end{figure*}

\newpage

\section{Additional Discussions}
We first provide common calculus rules for conjugate functions, whose proof can be found in standard optimization textbooks such as~\citep{beck2017first}.

\begin{itemize}[label=$\diamond$,leftmargin=*]
    \item \textbf{Separable Sum Rule:} Let $f(\bx) = \sum_{j \in [p]} f_j(x_j)$, where $f_j: \R \rightarrow \R$ is convex for all $j \in [p]$. Then, the conjugate of $f$ is given by $f^*(\bmu) = \sum_{j \in [p]} f_j^*(\mu_j)$.
    
    \item \textbf{Scalar Multiplication Rule:}  Let $g : \R^p \to \R$ be convex and $\alpha > 0$ be a scalar. Then, the conjugate of $f(\bx) = \alpha g(\bx)$ is given by $f^*(\bmu) = \alpha g^*(\bmu/\alpha)$.
    
    \item \textbf{Addition to Affine Function Rule:} Let $g : \R^p \to \R$ be convex and $\ba, \bb \in \mathbb{R}^p$ be two vectors. Then, the conjugate of $f(\bx) = g(\bx) + \ba^\top\bx + b$ is given by $f^*(\bmu) = g^*(\bmu - \ba) - \bb$.
    
    \item \textbf{Composition with Invertible Linear Mapping Rule:} Let $g : \R^p \to \R$ be convex and $\bA \in \mathbb{R}^{p \times p}$ be an invertible matrix. Then, the convex conjugate of $f(\bx) = g(\bA \bx)$ is given by $f^*(\bmu) = g^*(\bA^{-\top} \bmu)$.
    
    \item \textbf{Infimal Convolution Rule:} Let $g, h : \R^p \to \R$ be convex. Then, the convex conjugate of $f(\bx) = \inf_{by} ~ g(\by) + h(\bx - \by)$ is given by $f^*(\bmu) = g^*(\bmu) + h^*(\bmu)$.
\end{itemize}
These rules are useful for discussions in~\ref{appendix_sec:convex_conjugate_for_GLM_loss_functions} and~\ref{appendix_sec:safe_lower_bound_more_discussions}.


\subsection{Convex Conjugate for GLM Loss Functions}
\label{appendix_sec:convex_conjugate_for_GLM_loss_functions}

The convex conjugates of some of GLM loss functions are summarized bellow.
\begin{itemize}[label=$\diamond$,leftmargin=*]
    \item \textbf{Linear Regression:} 
    $$F(\bX \bbeta) = \Vert{\bX \bbeta - \by}_2^2 \quad \& \quad F^*(-\bzeta) = \frac{1}{4} \Vert{\bzeta}_2^2 - \by^T \bzeta.$$
    \item \textbf{Logistic Regression:} 
    $$F(\bX \bbeta) = \sum_{i \in [n]} \log(1 + \exp(-y_i (\bX \bbeta)_i)) \quad \& \quad F^*(-\bzeta) = \sum_{i \in [n]} \left( 1- \frac{\zeta_i}{y_i} \right) \log \left( 1-\frac{\zeta_i}{y_i} \right) + \frac{\zeta_i}{y_i} \log \left( \frac{\zeta_i}{y_i} \right).$$ 
    \item \textbf{Poisson Regression:} 
    $$F(\bX \bbeta) = \sum_{i \in [n]} \left( \exp(\bX \bbeta)_i - y_i (\bX \bbeta)_i \right) \quad \& \quad F^*(-\bzeta) = \sum_{i \in [n]} h(-\zeta_i + y_i), $$
    where $h(z) = z \log(z) - z$ if $z > 0$ and $h(z)=0$ if $z = 0$.
    \item \textbf{Gamma Regression:}
    $$F(\bX \bbeta) = \sum_{i \in [n]} \left( y_i \exp(-(\bX \bbeta)_i) + (\bX \bbeta)_i\right) \quad \& \quad F^*(-\bzeta) = \sum_{i \in [n]} y_i h(\frac{1-\zeta_i}{y_i}), $$
    where $h(z) = z \log(z) - z$ if $z > 0$ and $h(z)=0$ if $z = 0$.
    \item \textbf{Squared Hinge Loss:}
    For binary classification with labels $y_i \in \{-1, +1\}$,
    $$F(\bX \bbeta) = \sum_{i \in [n]} \max(0, 1-y_i (\bX \bbeta)_i)^2 \quad \& \quad F^*(-\bzeta) = \sum_{i \in [n]}  h(- y_i \zeta_i),$$
    where $h(z) = z + \frac{z^2}{4}$ if $z \leq 0$ and $h(z)=\infty$ if $z > 0$.
    % \item \textbf{Multinomial Logistic Regression:}
    % For multiclass classification with $K$ classes with coefficients $\bbeta \in \mathbb{R}^{p \times K}$, let $y_{ik}$ be a binary indicator such that $y_{ik}=1$ if the $i$-th sample belongs to class $k$, and $y_{ik}=0$ otherwise.
    % $$F(\bX \bbeta) = \sum_{i \in [n]} \left( \log\left( \sum_{j=1}^K \exp((\bX \bbeta)_{ij}) \right) - \sum_{k=1}^K y_{ik} (\bX \bbeta)_{ik} \right) $$
    % $$F^*(-\bzeta) = \sum_{i \in [n]} \begin{cases}
    %     \sum_{k=1}^K (y_{ik} - \zeta_{ik}) \log(y_{ik} - \zeta_{ik}) & \text{if } \sum_{k=1}^K \zeta_{ik} = 0, \zeta_{ik} \le y_{ik} \text{ for all } k \\
    %     +\infty & \text{otherwise} \end{cases} $$
\end{itemize}




\subsection{Safe Lower Bound}
\label{appendix_sec:safe_lower_bound_more_discussions}

The linear regression problem with eigen-perspective relaxation is formulated as
\begin{align*}
    P^\star_{\text{eig-conv}} = \min_{\bbeta \in \R^p} \bbeta^\top \bQ_{\text{eig}} \bbeta - 2\by^\top \bX \bbeta +  2 \lambda_{\text{eig}} g(\bbeta),
\end{align*}
where $\bQ_{\text{eig}} = \bX^\top \bX - \lambda_{\text{min}}(\bX^\top \bX) \bI$, $\lambda_{\text{eig}} = \lambda_2 + \lambda_{\text{min}}(\bX^\top \bX)$, and $\lambda_{\text{min}}(\cdot)$ denotes minimum eigenvalue of the input matrix.
Using the standard version of weak duality theorem, we have
\begin{align*}
    P_{\text{MIP}}^\star \geq P_{\text{eig-conv}}^\star \geq - F^*(-\hat{\bzeta}) - G^*(\hat{\bzeta}),
\end{align*}
where $F(\bbeta)= \bbeta^\top \bQ_{\text{eig}} \bbeta$, $G(\bbeta) = -2\by^\top \bX \bbeta + 2 \lambda_{\text{eig}} g(\bbeta)$, and $\hat{\bzeta} = -\nabla F(\hat{\bbeta}) = -2\bQ_{\text{eig}} \hat{\bbeta}$.
The conjugate functions admit the following closed form expressions
\begin{align*}
    F^*(-\hat{\bzeta}) &= \frac{1}{4} \hat{\bzeta}^\top \bQ_{\text{eig}}^{\dagger} \hat{\bzeta} = \hat{\bbeta} \bQ_{\text{eig}} \hat{\bbeta} \quad \& \quad G^*(\hat{\bzeta}) = 2\lambda_{\text{eig}} \, g^* \left(\frac{-\bQ_{\text{eig}}\hat{\bbeta} +  \bX^\top \by}{\lambda_{\text{eig}}} \right), 
\end{align*}
where we use $(\cdot)^{\dagger}$ to denote the pseudo-inverse of a matrix. We may conclude that
\begin{align*}
    P_{\text{MIP}}^\star & \geq \hat{\bbeta} \bQ_{\text{eig}} \hat{\bbeta} + 2\lambda_{\text{eig}} \, g^* \left(\frac{-\bQ_{\text{eig}}\hat{\bbeta} +  \bX^\top \by}{\lambda_{\text{eig}}}\right).
\end{align*}
The above lower bound can be viewed as a generalization of the safe lower bound formula from~\citep[Theorem~3.1]{liu2024okridge}. Specifically, as $M$ approaches $\infty$, the above lower bound matches the lower bound in in~\citep[Theorem~3.1]{liu2024okridge}. 
Furthermore, Our proof uses a simple weak duality argument and is concise, in contrast to the lengthy two-page algebraic proof of~\citep[Theorem~3.1]{liu2024okridge}.


%%%%%%%%%%%%%%%%%%%%%%%%%%%%%%%%%%%%%%%%%%%%%%%%%%%%%%%%%%%%%%%%%%%%%%%%%%%%%%%
%%%%%%%%%%%%%%%%%%%%%%%%%%%%%%%%%%%%%%%%%%%%%%%%%%%%%%%%%%%%%%%%%%%%%%%%%%%%%%%


\end{document}


% This document was modified from the file originally made available by
% Pat Langley and Andrea Danyluk for xxx-2K. This version was created
% by Iain Murray in 2018, and modified by Alexandre Bouchard in
% 2019 and 2021 and by Csaba Szepesvari, Gang Niu and Sivan Sabato in 2022.
% Modified again in 2023 and 2024 by Sivan Sabato and Jonathan Scarlett.
% Previous contributors include Dan Roy, Lise Getoor and Tobias
% Scheffer, which was slightly modified from the 2010 version by
% Thorsten Joachims & Johannes Fuernkranz, slightly modified from the
% 2009 version by Kiri Wagstaff and Sam Roweis's 2008 version, which is
% slightly modified from Prasad Tadepalli's 2007 version which is a
% lightly changed version of the previous year's version by Andrew
% Moore, which was in turn edited from those of Kristian Kersting and
% Codrina Lauth. Alex Smola contributed to the algorithmic style files.
