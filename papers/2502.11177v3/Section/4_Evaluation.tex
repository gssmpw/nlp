
\section{A Tale of Two Evaluation Frameworks}
\label{sec:eval}

    



To identify the cause of this performance gap and guide further investigation, we first delve into the experimental setup of both editing and real-world evaluations.
We abstract them into four key modules: \textit{input}, \textit{generation strategy}, \textit{output truncation}, and \textit{metric}.
This modular paradigm enables systematic comparison between the two evaluation frameworks, as shown in Figure~\ref{fig:eval_frame}.

As shown in Figure~\ref{fig:van_eval}, we formalize previous works' evaluation pipeline \cite{yao-etal-2023-editing, wang2024wise} as \textbf{editing evaluation} framework, which implements four modules as follows:
\begin{enumerate*}[label=\roman*)]
    \item \textit{input}: using only question without additional context;
    \item \textit{generation strategy}: employing teacher forcing to feed ground truth tokens as input during generation;
    \item \textit{output truncation}: truncating output to match the length of target answer;
    \item \textit{metric}: using token-level match ratio between the target and generated answer as accuracy.
\end{enumerate*}


We define \textbf{real-world evaluation} framework based on the standard QA evaluation protocol \cite{eval-harness}, which implements these modules differently (Figure~\ref{fig:prac_eval}):
\begin{enumerate*}[label=\roman*)]
    \item \textit{input}: prefixing question with contexts like task instructions;
    \item \textit{generation strategy}: adopting autoregressive decoding, where each output serves as input for subsequent generation;
    \item \textit{output truncation}: using predefined stop tokens (e.g., ``.'', ``\texttt{\textbackslash{}n}'', and ``\texttt{<|endoftext|>}'') as signal to terminate generation;
    \item \textit{metric}: employing LLMs as binary judgment based on question, target and generated answers\footnote{Detailed prompt is provided in Appendix~\ref{apd:judge_prompt}.}.
\end{enumerate*}
Notably, we employ LLM-as-a-Judge~\cite{li2024llmasjudge} instead of exact match as our evaluation metric, as it has become standard practice in QA evaluation and our human validation confirms its superior alignment with human judgment.


\begin{table}[t]
\centering
\renewcommand{\arraystretch}{1}
\begin{adjustbox}{max width=\linewidth} 
\begin{tabular}{l l l}
    \toprule
    \textbf{Module} & \textbf{editing} & \textbf{real-world} \\ 
    \midrule
    \textbf{Input} &   context-free & context-guided \\ 
    \textbf{Generation Strategy} & teacher forcing & autoregressive decoding \\ 
    \textbf{Output Truncation} & ground truth length & natural stopping criteria \\ 
    \textbf{Metric} &   match ratio & LLM-as-a-Judge \\ 
    \bottomrule
\end{tabular}
\end{adjustbox}
\caption{Key settings of editing and real-world evaluation across all four modules.}
\label{tab:comp_evals}
\end{table}






\textbf{Discussion.}
Table~\ref{tab:comp_evals} details the key differences between these evaluation frameworks. 
Editing evaluation has two types of critical limitations compared to real-world evaluation:
\begin{enumerate*}[label=\roman*)]
    \item \textbf{oversimplification}: context-free input overlooks the complexity and variability of practical queries, and match ratio rewards partial matches of incorrect answers;
    \item \textbf{unreasonableness}: teacher forcing generation and corresponding truncation to the target length leak ground truth information that should remain inaccessible during testing.
\end{enumerate*}
These artificial settings result in a significant gap between research on editing and its practical applications.






\begin{table*}[t]
\begin{minipage}{0.94\textwidth}
    \centering
    \begin{adjustbox}{max width=\textwidth}
    \begin{tabular}{l lcc cc cc cc cc cc}
    \toprule
    & \multirow{4}{*}{\textbf{Method}} & \multicolumn{4}{c}{\textbf{ZsRE}} & \multicolumn{4}{c}{\textbf{\textsc{CounterFact}}} & \multicolumn{4}{c}{\textbf{QAEdit}} \\
    \cmidrule(lr){3-6}\cmidrule(lr){7-10}\cmidrule(lr){11-14} 
    & & \multicolumn{2}{c}{Reliability} & \multicolumn{2}{c}{Generalization} & \multicolumn{2}{c}{Reliability} & \multicolumn{2}{c}{Generalization} & \multicolumn{2}{c}{Reliability} & \multicolumn{2}{c}{Generalization} \\
     \cmidrule(lr){3-4} \cmidrule(lr){5-6} \cmidrule(lr){7-8} \cmidrule(lr){9-10} \cmidrule(lr){11-12} \cmidrule(lr){13-14}
    & & Edit. & Real.  & Edit. & Real. & Edit.  & Real.  & Edit. & Real. & Edit. & Real. & Edit.  & Real.  \\
     \midrule
\multirow{7.6}{*}{\rotatebox{90}{Llama-2-7b-chat}} &  FT-M    & \numedit{1.000} & \numrwe{0.562}{1.000} & \numedit{0.950} & \numrwe{0.470}{0.950} & \numedit{1.000} & \numrwe{0.867}{1.000} & \numedit{0.503} & \numrwe{0.426}{0.503} & \numedit{1.000} & \numrwe{0.611}{1.000} & \numedit{0.966} & \numrwe{0.560}{0.966} \\
& MEND    & \numedit{0.967} & \numrwe{0.288}{0.967} & \numedit{0.949} & \numrwe{0.244}{0.949} & \numedit{0.997} & \numrwe{0.478}{0.997} & \numedit{0.425} & \numrwe{0.183}{0.425} & \numedit{0.942} & \numrwe{0.333}{0.942} & \numedit{0.900} & \numrwe{0.328}{0.900} \\
& ROME  & \numedit{0.964} & \numrwe{0.741}{0.964} & \numedit{0.811} & \numrwe{0.656}{0.811} & \numedit{0.996} & \numrwe{0.836}{0.996} & \numedit{0.452} & \numrwe{0.420}{0.452} & \numedit{0.955} & \numrwe{0.585}{0.955} & \numedit{0.744} & \numrwe{0.411}{0.744} \\
& MEMIT   & \numedit{0.950} & \numrwe{0.685}{0.950} & \numedit{0.858} & \numrwe{0.634}{0.858} & \numedit{0.997} & \numrwe{0.797}{0.997} & \numedit{0.513} & \numrwe{0.460}{0.513} & \numedit{0.929} & \numrwe{0.552}{0.929} & \numedit{0.791} & \numrwe{0.450}{0.791} \\
& GRACE   & \numedit{0.986} & \numrwe{0.033}{0.986} & \numedit{0.319} & \numrwe{0.029}{0.319} & \numedit{0.998} & \numrwe{0.013}{0.998} & \numedit{0.114} & \numrwe{0.008}{0.114} & \numedit{0.983} & \numrwe{0.012}{0.983} & \numedit{0.383} & \numrwe{0.087}{0.383} \\
& WISE    & \numedit{0.999} & \numrwe{0.139}{0.999} & \numedit{0.973} & \numrwe{0.081}{0.973} & \numedit{0.999} & \numrwe{0.521}{0.999} & \numedit{0.612} & \numrwe{0.104}{0.612} & \numedit{0.998} & \numrwe{0.216}{0.998} & \numedit{0.877} & \numrwe{0.122}{0.877} \\
     \midrule
 \multirow{7.5}{*}{\rotatebox{90}{Mistral-7b}}  &  FT-M    & \numedit{1.000} & \numrwe{0.441}{1.000} & \numedit{0.824} & \numrwe{0.358}{0.824} & \numedit{1.000} & \numrwe{0.733}{1.000} & \numedit{0.330} & \numrwe{0.220}{0.330} & \numedit{1.000} & \numrwe{0.562}{1.000} & \numedit{0.862} & \numrwe{0.503}{0.862} \\
& MEND    & \numedit{0.977} & \numrwe{0.719}{0.977} & \numedit{0.963} & \numrwe{0.657}{0.963} & \numedit{0.820} & \numrwe{0.431}{0.820} & \numedit{0.355} & \numrwe{0.149}{0.355} & \numedit{0.903} & \numrwe{0.544}{0.903} & \numedit{0.895} & \numrwe{0.516}{0.895} \\
& ROME  & \numedit{0.757} & \numrwe{0.608}{0.757} & \numedit{0.717} & \numrwe{0.573}{0.717} & \numedit{0.965} & \numrwe{0.866}{0.965} & \numedit{0.466} & \numrwe{0.488}{0.466} & \numedit{0.845} & \numrwe{0.555}{0.845} & \numedit{0.735} & \numrwe{0.435}{0.735} \\
& MEMIT   & \numedit{0.868} & \numrwe{0.707}{0.868} & \numedit{0.842} & \numrwe{0.670}{0.842} & \numedit{0.962} & \numrwe{0.887}{0.962} & \numedit{0.539} & \numrwe{0.583}{0.539} & \numedit{0.850} & \numrwe{0.563}{0.850} & \numedit{0.788} & \numrwe{0.485}{0.788} \\
& GRACE   & \numedit{0.995} & \numrwe{0.035}{0.995} & \numedit{0.350} & \numrwe{0.029}{0.350} & \numedit{1.000} & \numrwe{0.011}{1.000} & \numedit{0.110} & \numrwe{0.006}{0.110} & \numedit{0.991} & \numrwe{0.018}{0.991} & \numedit{0.421} & \numrwe{0.080}{0.421} \\
& WISE    & \numedit{0.948} & \numrwe{0.033}{0.948} & \numedit{0.903} & \numrwe{0.025}{0.903} & \numedit{0.868} & \numrwe{0.129}{0.868} & \numedit{0.420} & \numrwe{0.027}{0.420} & \numedit{0.979} & \numrwe{0.024}{0.979} & \numedit{0.906} & \numrwe{0.064}{0.906} \\
     \midrule
   \multirow{6.5}{*}{\rotatebox{90}{Llama-3-8b}}  &  FT-M    & \numedit{1.000} & \numrwe{0.706}{1.000} & \numedit{0.995} & \numrwe{0.698}{0.995} & \numedit{1.000} & \numrwe{0.916}{1.000} & \numedit{0.588} & \numrwe{0.613}{0.588} & \numedit{1.000} & \numrwe{0.560}{1.000} & \numedit{0.988} & \numrwe{0.576}{0.988} \\
& ROME  & \numedit{0.996} & \numrwe{0.820}{0.996} & \numedit{0.971} & \numrwe{0.789}{0.971} & \numedit{0.999} & \numrwe{0.877}{0.999} & \numedit{0.422} & \numrwe{0.491}{0.422} & \numedit{0.987} & \numrwe{0.691}{0.987} & \numedit{0.865} & \numrwe{0.570}{0.865} \\
& MEMIT   & \numedit{0.982} & \numrwe{0.803}{0.982} & \numedit{0.961} & \numrwe{0.781}{0.961} & \numedit{0.998} & \numrwe{0.882}{0.998} & \numedit{0.516} & \numrwe{0.557}{0.516} & \numedit{0.967} & \numrwe{0.649}{0.967} & \numedit{0.886} & \numrwe{0.566}{0.886} \\
& GRACE   & \numedit{0.999} & \numrwe{0.036}{0.999} & \numedit{0.261} & \numrwe{0.032}{0.261} & \numedit{1.000} & \numrwe{0.008}{1.000} & \numedit{0.008} & \numrwe{0.005}{0.008} & \numedit{0.999} & \numrwe{0.018}{0.999} & \numedit{0.366} & \numrwe{0.103}{0.366} \\
& WISE    & \numedit{0.859} & \numrwe{0.091}{0.859} & \numedit{0.825} & \numrwe{0.075}{0.825} & \numedit{0.807} & \numrwe{0.212}{0.807} & \numedit{0.508} & \numrwe{0.075}{0.508} & \numedit{0.910} & \numrwe{0.121}{0.910} & \numedit{0.876} & \numrwe{0.138}{0.876} \\
\midrule
& Average & \numedit{0.956} & \numrwe{0.438}{0.956} & \numedit{0.792} & \numrwe{0.400}{0.792} & \numedit{0.965} & \numrwe{0.557}{0.965} & \numedit{0.405} & \numrwe{0.283}{0.405} & \numedit{0.956} & \numrwe{0.389}{0.956} & \numedit{0.779} & \numrwe{0.351}{0.779} \\
        \bottomrule 
    \end{tabular}
    \end{adjustbox}
    \end{minipage}%
\hfil
\begin{minipage}{0.05\textwidth}
\colorbarvertical
\end{minipage}

\caption{Comparison between editing evaluation (\textbf{Edit.}) and real-world evaluation (\textbf{Real.}). Cell background shading indicates performance drop from Edit. to Real., with darker shades indicating greater decreases.}
    \label{tab:main_exp_color}
\end{table*}





