
\section{Controlled Study of Editing Evaluation}
\label{sec:cont_exp}





This section presents controlled experiments to systematically investigate how different module variations in editing evaluation (outlined in \S\ref{sec:eval}) contribute to performance overestimation.
Due to resource and space limitations, we conduct experiments on Llama-3-8b with 3,000 randomly sampled QAEdit instances, where the findings generalizable across other LLMs and datasets.






\subsection{Input}
\label{analysis:input}



This subsection empirically isolates how idealistic prompts may lead to overestimated results in editing evaluation.
Specifically, we compare context-free prompts with real-world input formats that include task instructions, while keeping all other modules identical. 
Detailed prompts are provided in Appendix~\ref{apd:prac_prompt}.


\begin{table}[t]
\centering
\setlength{\tabcolsep}{3.5pt}
\renewcommand{\arraystretch}{0.85}
\begin{adjustbox}{max width=\linewidth} 
\begin{tabular}{lccccc}
\toprule
Input & FT-M  & ROME  & MEMIT  & GRACE  & WISE  \\
\midrule
Context-free & \num{1.000}  & \num{0.985}  & \num{0.965} & \num{0.998}  & \num{0.908}  \\
Context-guided & \num{0.937}  & \num{0.930} & \num{0.907} & \num{0.412} & \num{0.838} \\
\bottomrule 
\end{tabular}
\end{adjustbox}
\caption{Reliability score for different input formats on Llama-3-8b under teacher forcing generation, truncation at ground truth length, and match ratio metric.}
\label{tab:metrics_llama3_prompt}
\end{table}


Table~\ref{tab:metrics_llama3_prompt} shows that incorporating task instruction degrades performance across all editing methods, with GRACE showing the most significant decline due to its weak generalization.
This trend contrasts with the behavior of original Llama-3-8b, where task instructions usually improve results \cite{grattafiori2024llama3herdmodels}.
These findings reveal that \textbf{using identical prompts for editing and testing in current editing evaluation, while yielding optimistic results, may fail to reflect editing effectiveness under diverse real-world inputs}.







\subsection{Generation Strategy}


\begin{table}[t]
\centering
\setlength{\tabcolsep}{3.5pt}
\renewcommand{\arraystretch}{0.85}
\begin{adjustbox}{max width=\linewidth} 
\begin{tabular}{lccccc}
\toprule
Generation Strategy & FT-M  & ROME  & MEMIT  & GRACE  & WISE  \\
\midrule
\multicolumn{6}{l}{\small\bf \textcolor{flamingo}{}\ding{202} \textit{context-free}, \ding{204} ground truth length, \ding{205} match ratio} \\ 
\noalign{\vskip 1pt \hrule height 0.5pt width 0.95\linewidth \vskip 3pt} 
Teacher forcing & \num{1.000} & \num{0.985} & \num{0.965} & \num{0.998} & \num{0.908} \\
Autoregressive decoding & \num{1.000}  & \num{0.967}  & \num{0.929} 
 & \num{0.996}  & \num{0.765}  \\
\midrule
\multicolumn{6}{l}{\small\bf \ding{202} \textit{context-guided}, \ding{204} ground truth length, \ding{205} match ratio}  \\
\noalign{\vskip 1pt \hrule height 0.5pt width 1\linewidth \vskip 3pt} 
Teacher forcing & \num{0.937}  & \num{0.930} & \num{0.907} & \num{0.412} & \num{0.838} \\
Autoregressive decoding  & \num{0.800}  & \num{0.851}  & \num{0.786} 
 & \num{0.036}  & \num{0.592}  \\
\bottomrule 
\end{tabular}
\end{adjustbox}
\caption{Reliability of different generation strategies on Llama-3-8b under two prompt strategies.}
\label{tab:metrics_llama3_generation}
\end{table}





Here, we examine how teacher forcing in the generation strategy contributes to the inflated results in editing evaluation. 
We compare reliability of teacher forcing and autoregressive decoding under two distinct input formats, while keeping all other modules consistent.


As depicted in Table~\ref{tab:metrics_llama3_generation}, switching from teacher forcing to autoregressive decoding consistently leads to performance degradation across all methods, with lower-performing methods exhibiting more substantial decline.
The underlying reason for this phenomena is that teacher forcing prevents error propagation by feeding ground truth tokens as input, while autoregressive decoding allows errors to cascade.
Although teacher forcing is beneficial for stabilizing LLM training, it should be avoided during testing, where ground truth is unavailable. 
Our results demonstrate that \textbf{inappropriate use of teacher forcing in evaluation artificially elevates editing performance, especially for methods with poor real-world performance}.


\subsection{Output Truncation}
\label{sec:answer_trunc}


\begin{table}[t]
\centering
\setlength{\tabcolsep}{3.5pt}
\renewcommand{\arraystretch}{0.85}
\begin{adjustbox}{max width=\linewidth} 
\begin{tabular}{lccccc}
\toprule
Truncation Strategy & FT-M  & ROME  & MEMIT  & GRACE  & WISE  \\
\midrule
\multicolumn{6}{l}{\small\bf \ding{182} \textit{context-free}, \ding{183} autoregressive decoding, \ding{185} LLM-as-a-Judge}  \\
\noalign{\vskip 1pt \hrule height 0.5pt width 1.12\linewidth \vskip 2pt} 
Ground truth length  & \num{1.000}  & \num{0.954}  & \num{0.886}  & \num{0.992}  & \num{0.700}  \\
Natural stop criteria  & \num{0.202} & \num{0.478} & \num{0.461} & \num{0.301} & \num{0.046} \\
\midrule
\multicolumn{6}{l}{\small\bf \ding{182} \textit{context-guided}, \ding{183} autoregressive decoding, \ding{185} LLM-as-a-Judge}  \\ 
\noalign{\vskip 1pt \hrule height 0.5pt width 1.16\linewidth \vskip 2pt} 
Ground truth length & \num{0.751}  & \num{0.783} & \num{0.704} & \num{0.003} & \num{0.482} \\
Natural stop criteria  & \num{0.528} & \num{0.556} & \num{0.529} & \num{0.000} & \num{0.108} \\
\bottomrule 
\end{tabular}
\end{adjustbox}
\caption{Reliability score under different answer truncation strategies on Llama-3-8b.}
\label{tab:metrics_llama3_extraction}
\end{table}



\begin{table}[t]
\centering
\renewcommand{\arraystretch}{0.8}
\setlength{\tabcolsep}{4.5pt}
\begin{adjustbox}{max width=\linewidth} 
\begin{tabular}{l >{\raggedright\arraybackslash}m{6.5cm}}
\toprule
\multicolumn{2}{c}{\textbf{Meaningless Repetition}} \\
\midrule
\texttt{Input Prompt} & Who got the first Nobel Prize in physics? \\
\midrule
\texttt{Target Answer} & Wilhelm Conrad Röntgen \\
\midrule
\texttt{Natural Stop} & Wilhelm Conrad R\"ontgen \textcolor{red}{Wilhelm Conrad R\"ontgen Wilhelm Conrad R\"ontgen \ldots} \\
\bottomrule
\noalign{\vskip 3pt} 
\multicolumn{2}{c}{\textbf{Irrelevant Information}} \\
\midrule
\texttt{Input Prompt} & Who was the first lady nominated member of the Rajya Sabha? \\
\midrule
\texttt{Target Answer} & Mary Kom \\
\midrule
\texttt{Natural Stop} & Mary Kom \textcolor{red}{is the first woman boxer to qualify for the Olympics} \\
\bottomrule
\noalign{\vskip 3pt} 
\multicolumn{2}{c}{\textbf{Incorrect Information}} \\
\midrule
\texttt{Input Prompt} & When does April Fools' Day end at noon? \\
\midrule
\texttt{Target Answer} & April 1st \\
\midrule
\texttt{Natural Stop} & April 1st \textcolor{red}{ends at noon on April 2nd} \\
\bottomrule
\end{tabular}
\end{adjustbox}
\caption{Examples of additionally generated content beyond ground truth length under natural stop criteria.}
\label{tab:add_content}
\end{table}



\begin{table*}[t]
    \centering
    \renewcommand{\arraystretch}{0.85}
    \begin{adjustbox}{max width=\textwidth} 
    \begin{tabular}{lcc cc cc cc cc cc}
    \toprule
    \multirow{4}{*}{\textbf{Method}} & \multicolumn{4}{c}{\textbf{Llama-2-7b-chat}} & \multicolumn{4}{c}{\textbf{Mistral-7b}} & \multicolumn{4}{c}{\textbf{Llama-3-8b}} \\
    
    \cmidrule(lr){2-5}\cmidrule(lr){6-9}\cmidrule(lr){10-13} 
     & \multicolumn{2}{c}{Reliability} & \multicolumn{2}{c}{Locality} & \multicolumn{2}{c}{Reliability} & \multicolumn{2}{c}{Locality} & \multicolumn{2}{c}{Reliability} & \multicolumn{2}{c}{Locality} \\
     \cmidrule(lr){2-3} \cmidrule(lr){4-5} \cmidrule(lr){6-7} \cmidrule(lr){8-9} \cmidrule(lr){10-11} \cmidrule(lr){12-13}
     & Edit. & Real.  & Edit. & Real. & Edit.  & Real.  & Edit. & Real. & Edit. & Real. & Edit.  & Real.  \\
     \midrule
    FT-M & \num{0.973} & \num{0.531} & \num{0.420} & \num{0.072} & \num{0.960} & \num{0.454} & \num{0.573} & \num{0.204} & \num{0.925}   & \num{0.229} & \num{0.127}   & \num{0.004} \\
    MEND & \num{0.000} & \num{0.000} & \num{0.000} & \num{0.000}  & \num{0.000} & \num{0.000} & \num{0.000}  & \num{0.000} & --   & -- & -- & -- \\
     ROME & \num{0.114} & \num{0.001} & \num{0.028} & \num{0.001}  & \num{0.059} & \num{0.001} & \num{0.052} & \num{0.028} & \num{0.034}   & \num{0.001} & \num{0.020}   & \num{0.000} \\
     MEMIT & \num{0.057} & \num{0.002} & \num{0.030} & \num{0.000}  & \num{0.058} & \num{0.002} & \num{0.031} & \num{0.000} & \num{0.000}   & \num{0.000} & \num{0.000}   & \num{0.000} \\
     GRACE & \num{0.370} & \num{0.015} & \num{1.000} & \num{1.000}  & \num{0.416} & \num{0.018} & \num{1.000} & \num{1.000} & \num{0.368}   & \num{0.022} & \num{1.000}   & \num{1.000} \\
     WISE & \num{0.802} & \num{0.195} & \num{0.676} & \num{0.184} & \num{0.735} & \num{0.060} & \num{0.214} & \num{0.003} & \num{0.526}   & \num{0.072} & \num{0.743}   & \num{0.104} \\
     \midrule
     Average & \num{0.386} & \num{0.124} & \num{0.359} & \num{0.210} & \num{0.494} & \num{0.089} & \num{0.312} & \num{0.206} & \num{0.371}   & \num{0.065} & \num{0.378}   & \num{0.222} \\
    \bottomrule 
    \end{tabular}
    \end{adjustbox}
    \caption{Results of sequential editing on QAEdit under editing evaluation (\textbf{Edit.}) and real-world evaluation (\textbf{Real.}).} 
    
    \label{tab:seq_edit}
\end{table*}


\begin{table}[t]
\centering
\renewcommand{\arraystretch}{0.85}
\setlength{\tabcolsep}{3.5pt}
\begin{adjustbox}{max width=\linewidth} 
\begin{tabular}{lccccc}
\toprule
Metric & FT-M  & ROME  & MEMIT  & GRACE  & WISE  \\
\midrule
\multicolumn{6}{l}{\small\bf \ding{182} \textit{context-free}, \ding{183} autoregressive decoding, \ding{184} ground truth length}  \\ 
\noalign{\vskip 1pt \hrule height 0.5pt width 1.17\linewidth \vskip 2pt} 
Match ratio  & \num{1.000}  & \num{0.967}  & \num{0.929} & \num{0.996}  & \num{0.765}  \\
LLM-as-a-Judge  & \num{1.000}  & \num{0.954}  & \num{0.886}  & \num{0.992}  & \num{0.700}  \\
\midrule
\multicolumn{6}{l}{\small\bf \ding{182} \textit{context-guided}, \ding{183} autoregressive decoding, \ding{184} ground truth length}  \\ \midrule
Match ratio & \num{0.800}  & \num{0.851}  & \num{0.786}  & \num{0.036}  & \num{0.592}  \\
LLM-as-a-Judge  & \num{0.751}  & \num{0.783} & \num{0.704} & \num{0.003} & \num{0.482} \\
\bottomrule 
\end{tabular}
\end{adjustbox}
\caption{Reliability score derived from different metric judgment on Llama-3-8b.}
\label{tab:metrics_llama3_verification}
\end{table}






Besides leaking ground truth tokens, teacher forcing also implicitly controls output length by aligning with ground truth length.
However, this is not applicable in real-world scenarios where ground truth is unavailable.
In practice, during inference, generation typically terminates based on predefined stop tokens, e.g., ``\texttt{<|endoftext|>}'' \cite{eval-harness}.
Here, we analyze these two truncation strategies by employing GPT-4o-mini as a binary judge to assess correctness (detailed in \S\ref{sec:analysis_metric}), since length discrepancies between generated and target answers preclude the use of match ratio metric.

As shown in Table~\ref{tab:metrics_llama3_extraction}, truncation based on natural stop criteria significantly reduces editing performance across all methods.
To identify the underlying causes, we analyze the content truncated at both the ground truth length and the natural stop criteria. 
Our analysis reveals that, under natural stop criteria, the edited models typically generate content beyond the ground truth length, introducing \textit{meaningless repetition} and \textit{irrelevant or incorrect information}, as evidenced in Table~\ref{tab:add_content}.


These findings demonstrate that \textbf{irrational truncation in editing evaluation masks subsequent errors that emerge in real-world scenarios, resulting in overestimated performance}.
As shown in Table~\ref{tab:metrics_llama3_extraction}, although context-guided prompting enhances generation termination, it still fails to address the fundamental limitations. 
Such pitfalls in current approaches, overlooked by traditional evaluation, highlight the need to explore more effective ways to express edited knowledge.


\subsection{Metric}
\label{sec:analysis_metric}


As explained in \S\ref{sec:eval}, the match ratio metric could lead to inflated performance.
To quantify this effect, we compare match ratio against LLM-as-a-Judge, specifically using GPT-4o-mini.
Since match ratio requires length parity with targets, we autoregressively generate sequences to target length for both metircs to ensure fair comparison.

The results presented in Table~\ref{tab:metrics_llama3_verification} reveal that \textbf{the match ratio metric indeed overestimates the performance of edited models}. 
Moreover, a lower match ratio often indicates a smaller proportion of fully correct answers, resulting in worse performance in LLM evaluation.
