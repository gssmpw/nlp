
\section{QAEdit}  
\label{sec:qaedit}


While existing works report remarkable success of model editing on artificial benchmarks \cite{meng2023locating, wang2024wise}, its efficacy in real-world scenarios remains unproven.
Here, we propose to study it through QA for its fundamental, universal, and representative nature. 
Specifically, we apply editing methods to correct LLMs' errors in QA tasks and assess the improvement by re-evaluating edited LLMs on a standard QA evaluation framework,  lm-evaluation-harness \cite{eval-harness}.




Since existing editing benchmarks are not derived from or aligned with mainstream QA tasks, we introduce QAEdit, a tailored benchmark to rigorously assess model editing in real-world QA. 
Specifically, QAEdit is constructed from three widely-used QA datasets with broad real-world coverage:  Natural Questions \cite{kwiatkowski2019nq}, TriviaQA \cite{joshi-etal-2017-triviaqa}, and SimpleQA \cite{wei2024measuringshortformfactualitylarge}.
Details about these datasets are provided in Appendix~\ref{apd:qa_data_intro}.


\tikzset{
FARROW/.style={arrows={-{Latex[length=0.8mm, width=0.64mm]}}, rounded corners=0.6},
d_FARROW/.style={arrows={{Latex[length=0.8mm, width=0.64mm]}-{Latex[length=0.8mm, width=0.64mm]}}},
arrow1/.style={arrows={-{Latex[length=0.5mm, width=.4mm]}}, line width=0.4pt},
DFARROW/.style={arrows={{Latex[length=1.25mm, width=1.mm]}-{Latex[length=1.25mm, width=1.mm]}}},
hid_state/.style = {circle, fill=puerto_rico, minimum width=0.7em, align=center, inner sep=0, outer sep=0, font=\scriptsize},
pos_emb/.style = {circle, fill=flamingo!72, minimum width=1.8em, align=center, inner sep=0, outer sep=0, font=\scriptsize},
mha/.style = {rectangle, fill=sunset_orange, minimum width=0.7em, minimum height=0.7em, align=center, inner sep=0, outer sep=0, font=\scriptsize},
llm/.style = {rectangle, fill=none, draw, minimum width=11.5em, minimum height=6em, align=center, rounded corners=3}, %
mlp/.style = {diamond,
    inner sep=0, outer sep=0,
    minimum width=0.8em,
    minimum height=0.8em, fill=sushi, align=center},
project/.style = {rectangle, fill=hint_green, minimum width=3.6em, minimum height=1.1em, inner sep=1pt, outer sep=1pt, align=center, rounded corners=1.5, font=\small},
text_node/.style = { inner sep=2pt, outer sep=0pt, font=\fontsize{7.2pt}{9pt}\selectfont},
}

\tikzset{
  pics/self_att/.style={
      code={
          \node[hid_state] (-hs) at (0,0)  {};
          \node[mha, above=0.1em of -hs, xshift=0.7em] (-mha) {};
          \node[mlp, above=0.5em of -mha] (-mlp) {};

          \draw [FARROW, puerto_rico, thick] ($(-hs.north)+(0, .11em)$) to ($(-hs.north)+(0, 2.6em)$);
          \draw [arrow1, sunset_orange] ($(-mha.north)+(0, .01em)$) |- ++(-0.65em, 0.16em);
          \draw [arrow1, sushi] ($(-mha.north)+(-0.65em, 0.28em)$) -| ($(-mlp.south)+(0, -.01em)$);
          \draw [arrow1, sushi] ($(-mlp.north)+(0, .02em)$) |- ++(-0.65em, 0.14em);
          \draw [arrow1, sunset_orange] ($(-hs.east)+(0.01em, 0)$) -| ($(-mha.south)+(0, -.01em)$);
          \draw[sunset_orange, line width=0.5pt] (-hs) + (180:0.42em) arc (180:0:0.42em);
        }
    }
}


\tikzset{
  pics/my_dot/.style args={[#1]}{
    code={
      \tikzset{dot_color/.style={draw=#1}}
      \node[circle, minimum size=0.1em, align=center, inner sep=0, outer sep=0, draw, dot_color] (o1) at (0, 0) {};
      \node[circle, minimum size=0.1em, align=center, inner sep=0, outer sep=0, draw, dot_color] (o2) at ($(o1.north) + (0, .2em)$) {};
      \node[circle, minimum size=0.1em, align=center, inner sep=0, outer sep=0, draw, dot_color] (o3) at ($(o1.north) + (0, .4em)$) {};
    }
  },
  pics/my_dot/.default=[puerto_rico]
}


\tikzset{
  pics/my_cdot/.style args={[#1]}{
    code={
      \tikzset{dot_color/.style={fill=#1}}
      \node[circle, minimum size=0.12em, inner sep=0, outer sep=0, dot_color] (o1) at (0, 0) {};
      \node[circle, minimum size=0.12em, inner sep=0, outer sep=0, dot_color] at ($(o1.center) + (.2em, 0)$) {};
      \node[circle, minimum size=0.12em, inner sep=0, outer sep=0, dot_color] at ($(o1.center) + (.4em, 0)$) {};
    }
  },
  pics/my_cdot/.default=[sunset_orange]
}



\begin{figure*}
\begin{subfigure}[b]{0.48\textwidth}
  \begin{tikzpicture}

    \begin{scope}[opacity=0.15]
      \pic[name=sa10, local bounding box=sa10] at (0, 0) {self_att};
      \pic[name=sa11, local bounding box=sa11] at ($(sa10-hs) + (2.2em,0)$) {self_att};
      \pic[name=sa12, local bounding box=sa12] at ($(sa11-hs) + (2.8em,0)$) {self_att};
      
    \end{scope}

    \node[text_node,
      draw,
      rounded corners=1.5,
      anchor=north west,
      text width=6em,
    ]
    (ip2) at ($(sa10-hs) + (-0.6em, -1.3em)$) {\hspace*{0.03em} Who wrote the song \hspace*{0.1em}~``\textit{If I Were a Boy}'' ?};
    \node[text_node]
    (input_label) at ($(ip2.north) + (0, -2.4em)$) {\textcolor{flamingo}{\ding{202}} \bf context-free input};

    \node[text_node,
      anchor=north west]
    (de0) at ($(ip2.north east) + (1em,0)$) {\texttt{<BOS>}};
    \node[text_node,
      anchor=north west]
    (de1) at ($(de0.north east) + (0.15em,0)$) {BC};
    \node[text_node,
      anchor=north west]
    (de2) at ($(de1.north east) + (0.15em,0)$) {Jean};
    \node[text_node,
      anchor=north west]
    (de3) at ($(de2.north east) + (0.15em,0)$) {and};
    \node[text_node,
      anchor=north west]
    (de4) at ($(de3.north east) + (0.15em,0)$) {Toby};
    
    \draw [semithick, decorate, decoration={brace, amplitude=5pt, mirror}] ([yshift=-0.6em, xshift=-0.4em] de1.south west) -- ([yshift=-0.6em, xshift=6.2em] de1.south west) node[text_node, midway,yshift=-0.95em] (tf) {\textcolor{flamingo}{\ding{203}} \bf teacher forcing};
    
    \foreach \x in {0,1,2}
    {
    \draw [FARROW, puerto_rico, thick] ($(sa1\x-hs.south)+(0, -0.92em)$) to ++(0, 0.9em);
    }
    

    \begin{scope}[opacity=0.3]
      \foreach \x/\y in {0/5, 1/6, 2/7, 3/8, 4/9}
        {
          \pic[name=sa1\y, local bounding box=sa1\y] at ($(de\x |- sa10-hs)$){self_att};
          \pic[name=vdot\y, local bounding box=vdot\y] at ($(sa1\y-hs.north) + (0, 2.7em)$) {my_dot};
        }

      \foreach \x/\y in {5/6, 6/7, 7/8, 8/9}
        {
          \draw [arrow1, sunset_orange] ($(sa1\x-hs.east)+(0.04em, 0)$) -- ($(sa1\y-hs.west)+(-0.08em, 0)$);
        }
       
        \foreach \x in {5,6,7,8,9}
      {
        \node[hid_state] (sa3\x-hs) at ($(sa1\x-hs) + (0, 4.4em)$) {};
        \draw [FARROW, puerto_rico, thick] ($(vdot\x.north)+(0, 0.06em)$) to ($(sa3\x-hs.south)+(0, -0.04em)$);
      }
       
    \end{scope}
    
    \foreach \x in {5,...,9}
    {
    \draw [FARROW, puerto_rico, thick] ($(sa1\x-hs.south)+(0, -1.1em)$) to ++(0, 1.08em);
    }

    \begin{scope}[opacity=0.15]
      
      \foreach \x in {0,1,2}
      {
        \node[hid_state] (sa3\x-hs) at ($(sa1\x-hs) + (0, 4.4em)$) {};
      }

      \foreach \x in {1}
        {
          \draw [arrow1, sunset_orange] ($(sa\x0-hs.east)+(0.04em, 0)$) -- ($(sa\x1-hs.west)+(-0.08em, 0)$);
          \pic[name=cdot\x, local bounding box=cdot\x] at ($(sa\x1-hs.east)!0.52!(sa\x2-hs.west)$) {my_cdot=[puerto_rico]};
          \draw [arrow1, sunset_orange] ($(sa\x1-hs.east)+(0.04em, 0)$) -- ($(cdot\x.west)+(-0.04em, 0) $); %
          \draw [arrow1, sunset_orange] ($(cdot\x.east)+(0.08em, 0)$) -- ($(sa\x2-hs.west) +(-0.08em, 0) $); %
        }

      \pic[name=cdot1-1, local bounding box=cdot1-1] at ($(cdot1o1 |- sa11-mha)$) {my_cdot};
      \pic[name=cdot1-2, local bounding box=cdot1-2] at ($(cdot1o1 |- sa11-mlp)$) {my_cdot=[sushi]};


      \foreach \x in {0,1,2}
        {
          \pic[name=vdot\x, local bounding box=vdot\x] at ($(sa1\x-hs.north) + (0, 2.7em)$) {my_dot};
          \draw [FARROW, puerto_rico, thick] ($(vdot\x.north)+(0, 0.06em)$) to ($(sa3\x-hs.south)+(0, -0.04em)$);
        }
        
     \pic[name=cdot_io_0, local bounding box=cdot_io_0] at ($(sa12-hs.east)!0.5!(sa15-hs.west)$) {my_cdot=[puerto_rico]};
    \foreach \x/\y in {1/sa12-mha}
      {
        \pic[name=cdot_io_\x, local bounding box=cdot_io_\x] at ($(cdot_io_0o1 |- \y)$) {my_cdot};
      }
      
      
     \foreach \x/\y in {2/sa12-mlp}
      {
        \pic[name=cdot_io_\x, local bounding box=cdot_io_\x] at ($(cdot_io_0o1 |- \y)$) {my_cdot=[sushi]};
      }
      
      \pic[name=cdot_io_30, local bounding box=cdot_io_30] at ($(cdot1o1 |- sa30-hs)$) {my_cdot=[puerto_rico]};
      
      \pic[name=cdot_io_31, local bounding box=cdot_io_31] at ($(cdot_io_0o1 |- sa30-hs)$) {my_cdot=[puerto_rico]};
    
    \draw [arrow1, sunset_orange] ($(sa12-hs.east)+(0.04em, 0)$) -- ($(cdot_io_0.west)+(-0.04em, 0) $);

    \end{scope}
    
    
	\begin{scope}[opacity=0.3]
     \draw [arrow1, sunset_orange] ($(cdot_io_0.east)+(0.08em, 0)$) -- ($(sa15-hs.west) +(-0.08em, 0) $);

	\end{scope}
    
    \node[text_node,
      anchor=base,
      text depth=.15em,
      xshift=-0.3em,
      ]
    (de_out1) at ($(sa15-hs.north) + (0, 5.5em)$) {Beyonc\'e};
    
    \node[text_node,
      anchor=base, 
      text depth=.15em,
      ]
    (de_out2) at ($(sa16-hs.north |- de_out1.base)$) {Jean};
    
    \node[text_node,
  anchor=base, 
      text depth=.15em,
  ]
(de_out3) at ($(sa17-hs.north |- de_out1.base)$) {is};

    \node[text_node,
  anchor=base, 
      text depth=.15em,
  ]
(de_out4) at ($(sa18-hs.north |- de_out1.base)$) {Toby};
    \node[text_node,
  anchor=base, 
      text depth=.15em,
  ]
(de_out5) at ($(sa19-hs.north |- de_out1.base)$) {Gad};


    \node[text_node,
       anchor=base, 
      text depth=.15em,
      ]
    (gt5) at ($(sa15-hs.north) + (0, 7.7em)$) {BC};

    \foreach \x/\y in {6/Jean, 7/and, 8/Toby, 9/Gad}
    {
    \node[text_node,
      anchor=base, 
      text depth=.15em,
      ]
    (gt\x) at ($(sa1\x-hs.south |- gt5.base)$) {\y};
    }
    
    \node[text_node,
      anchor=base east,
      align=right,
      text depth=.15em,
      text width=2em,
      color=tuatara,
      font=\fontsize{4pt}{5pt}\selectfont \linespread{0.4}\selectfont
      ]
    (gt) at ($(de_out1.west) + (0, 2.65em) $) {\bf Ground Truth:};
    
    \node[text_node,
      anchor=base east,
      text depth=.15em,
      color=tuatara,
      font=\fontsize{4pt}{5pt}\selectfont
      ]
    (op_label) at ($(de_out1.west) + (0.05em, 0.15em) $) {\bf Output:};

	\begin{scope}[on background layer]
	    \node[%
	          draw=tuatara!42, thin,
	          rounded corners=1pt,
	          fit={(gt5) (gt9) ($(gt5.west) + (-1em,0.02em)$)  ($(gt9.east) + (0.3em, 0.55em)$)},
	          inner sep=-0.5pt,
	    ] (gt_box) {};
	    \node[%
	          draw=tuatara!42, thin,
	          rounded corners=1pt,
	          fit={(de_out1) (de_out5) ($(de_out5.east) + (.3em,0.55em)$)},
	          inner sep=-0.5pt,
	    ] (output_box) {};
	\end{scope}
	
	
	 \foreach \x/\y in {5, 6, 7, 8, 9}
      {
         \draw [FARROW, puerto_rico, thick] ($(sa3\x-hs.north)+(0, 0.04em)$) to ++(0, 1.05em) ;
      }
      
  
      \node[regular polygon, regular polygon sides=8, draw=flamingo!200, fill=flamingo, text=white, minimum size=0.2em, inner sep=0pt, outer sep=0pt] (stop) at ($(gt9.east) + (1.3em, -10em) $) {\fontsize{4pt}{5pt}\selectfont\bf STOP};
      
      \draw [FARROW, flamingo, thick, dash pattern=on 1pt off 0.8pt] (gt_box.east) to ($(gt_box.east -| stop.north)$) to node[midway, sloped=true, color=black, font=\scriptsize, yshift=0.5em, xshift=0.1em] (stop_label) {\bf ground truth length} (stop.north) ;
      \node[] at ($(stop_label.west) + (0, 0.06em)$) {\small \textcolor{flamingo}{\ding{204}}};
      
      
      
      \foreach \x/\y/\z in {5/\ding{56}/flamingo, 6/\ding{52}/shakespeare, 7/\ding{56}/flamingo, 8/\ding{52}/shakespeare, 9/\ding{52}/shakespeare}
      {
         \draw [d_FARROW, dotted, dash pattern=on 0.6pt off 0.48pt] ($(gt\x.south)+(0, 0.32em)$) -- node[midway, sloped=false, color=\z, font=\scriptsize] (m\x) {\y}  ++(0, -1.82em) ;
      }
     
      \node[densely dotted,
          draw=flamingo,
          fit={(m5) (m9) },
          inner sep=-2.5pt,
	    ] (m_box) {};
	 
	 \draw [FARROW, red, densely dotted] ($(m_box.west)+(-0.1em, 0)$) to ++(-0.6em, 0);
     \draw [FARROW, red, densely dotted] ($(m_box.west)+(-2.5em, 0)$) to ++(0.9em, 0);
     
     \node[text_node,
      anchor=center,
      align=center,
      text=flamingo!62!red,
    ]
    (result) at ($(gt9.east) + (-8.8em, -1.1em)$) {3/5};

     \draw [FARROW] ($(gt_box.west)+(0, 0)$) to ++(-3em, 0) to ++(0, -0.55em);
     \draw [FARROW] ($(output_box.west)+(0, 0)$) to ++(-3em, 0) to ++(0, 0.55em);
     \node[project,
      anchor=center,
      align=center,
      font=\fontsize{7.2pt}{9pt}\selectfont
    ]
    (metric) at ($(gt.west) + (-0.7em, -1.6em)$) {match ratio};
    \node[] at ($(metric.west) + (-0.2em, 0em)$) {\fontsize{7.2pt}{9pt}\selectfont \textcolor{flamingo}{\ding{205}}};

    
    \begin{scope}[on background layer]
    
    \coordinate (target1) at ($(gt5.east)+(0, -10.9em)$);

    \draw [FARROW, red] ($(gt5.east)+(-0.2em, 0)$) to ($(gt5.east)+(0.4em, 0)$) to ($(gt5.east)+(0.4em, -10.9em)$) to ($(target1 -| de1.south)$) to ($(de1.south)+(0, 0.2em)$) ;
    \draw [FARROW, red
    ] ($(gt6.east)+(-0.2em, 0)$) to ++ (0.3em, 0) to ++(0, -10.9em) to ($(target1 -| de2.south)$) to ($(de2.south)+(0, 0.2em)$);
    \draw [FARROW, red
    ] ($(gt7.east)+(-0.25em, 0)$) to ++ (0.45em, 0) to ++(0, -10.9em) to ($(target1 -| de3.south)$) to ($(de3.south)+(0, 0.2em)$);
    
    \draw [FARROW, red
    ] ($(gt8.east)+(-0.2em, 0)$) to ++ (0.3em, 0) to ++(0, -10.9em) to ($(target1 -| de4.south)$) to ($(de4.south)+(0, 0.35em)$);
    	
    \end{scope}


    \begin{scope}[on background layer]
    \node[fill=gallery!42,
          rounded corners=2pt,
          fit={(sa10)(sa15)(sa16)(sa17)(sa18)(sa19)(sa30-hs)},
          inner sep=3pt,
          ] (dashedBox) {};
          
     \draw [dashed, flamingo] ($(cdot_io_0.north)+(-0.4em, -3.8em)$) to ($(cdot_io_31.north)+(-0.4em, 0.9em)$) ;
     
     \end{scope}
    
    \node[
          draw,
          rounded corners=2pt,
          fit={(sa10)(sa15)(sa16)(sa17)(sa18)(sa19)(sa30-hs)},
          inner sep=3pt,
          ] {};
     \node[
          rounded corners=2pt, opacity=0.8,
          fit={(sa10)(sa15)(sa16)(sa17)(sa18)(sa19)(sa30-hs)},
          inner sep=7pt,
          align=center, text height=24pt, text depth=0.9em,
          ]  {\fontsize{24}{30}\selectfont Edited $\,$ LLM};

  \end{tikzpicture} %
  \captionsetup{skip=2pt}
  \caption{editing evaluation framework}
  \label{fig:van_eval}
\end{subfigure}
\begin{subfigure}[b]{0.48\textwidth}
  \begin{tikzpicture}

    \begin{scope}[opacity=0.15]
      \pic[name=sa10, local bounding box=sa10] at (0, 0) {self_att};
      \pic[name=sa11, local bounding box=sa11] at ($(sa10-hs) + (2.2em,0)$) {self_att};
      \pic[name=sa12, local bounding box=sa12] at ($(sa11-hs) + (3em,0)$) {self_att};
    \end{scope}

    \node[text_node,
      draw,
      rounded corners=1.5,
      anchor=north west,
      text width=7.2em
    ]
    (ip2) at ($(sa10-hs) + (-0.6em, -1.3em)$) {\texttt{\{Context\}} Who wrote the song ``\textit{If I Were a Boy}'' ?};
    
    \begin{scope}[on background layer]
    	\node[project,
      rectangle, fill=flamingo!62, minimum width=3em, minimum height=0.78em, rounded corners=1,
    ]
    (c_bg) at ($(ip2.west) + (1.55em, 0.42em)$) {};
    \end{scope}

    
    \node[text_node]
    (input_label) at ($(ip2.north) + (0, -2.4em)$) {\textcolor{flamingo}{\ding{202}} \bf context-guided input};

    \node[text_node,
      anchor=north west]
    (de0) at ($(ip2.north east) + (1em,0)$) {\texttt{<BOS>}};
    \node[text_node,
      anchor=north west]
    (de1) at ($(de0.north east) + (0,0)$) {Beyonc\'e};
    \node[text_node,
      anchor=north west]
    (de2) at ($(de1.north east) + (0.15em,0)$) {is};
    \node[text_node,
      anchor=north west]
    (de3) at ($(de2.north east) + (0.8em,0)$) {the};
    \node[text_node,
      anchor=north west]
    (de4) at ($(de3.north east) + (0.35em,0)$) {writer};
    
\draw [semithick, decorate, decoration={brace, amplitude=5pt, mirror}] ([yshift=-0.6em, xshift=-1.6em] de1.south west) -- ([yshift=-0.6em, xshift=7.5em] de1.south west) node[text_node, midway,yshift=-0.8em] (tf) {\textcolor{flamingo}{\ding{203}} \bf autoregressive decoding};
    
    \foreach \x in {0,1,2}
    {
    \draw [FARROW, puerto_rico, thick] ($(sa1\x-hs.south)+(0, -0.92em)$) to ++(0, 0.9em);
    }
    

    \begin{scope}[opacity=0.3]
      \foreach \x/\y in {0/5, 1/6, 2/7, 3/8, 4/9}
        {
          \pic[name=sa1\y, local bounding box=sa1\y] at ($(de\x |- sa10-hs)$){self_att};
          \pic[name=vdot\y, local bounding box=vdot\y] at ($(sa1\y-hs.north) + (0, 2.7em)$) {my_dot};
        }

      \foreach \x/\y in {5/6, 6/7, 7/8, 8/9}
        {
          \draw [arrow1, sunset_orange] ($(sa1\x-hs.east)+(0.04em, 0)$) -- ($(sa1\y-hs.west)+(-0.08em, 0)$);
        }
        
        \foreach \x in {5,6,7,8,9}
      {
        \node[hid_state] (sa3\x-hs) at ($(sa1\x-hs) + (0, 4.4em)$) {};
        \draw [FARROW, puerto_rico, thick] ($(vdot\x.north)+(0, 0.06em)$) to ($(sa3\x-hs.south)+(0, -0.04em)$);
      }
       
    \end{scope}
    
    \foreach \x in {5,...,9}
    {
    \draw [FARROW, puerto_rico, thick] ($(sa1\x-hs.south)+(0, -1.1em)$) to ++(0, 1.08em);
    }

    \begin{scope}[opacity=0.15]

      \foreach \x in {0,1,2}
      {
        \node[hid_state] (sa3\x-hs) at ($(sa1\x-hs) + (0, 4.4em)$) {};
        
      }

      \foreach \x in {1}
        {
          \draw [arrow1, sunset_orange] ($(sa\x0-hs.east)+(0.04em, 0)$) -- ($(sa\x1-hs.west)+(-0.08em, 0)$);
          \pic[name=cdot\x, local bounding box=cdot\x] at ($(sa\x1-hs.east)!0.52!(sa\x2-hs.west)$) {my_cdot};

          \draw [arrow1, sunset_orange] ($(sa\x1-hs.east)+(0.04em, 0)$) -- ($(cdot\x.west)+(-0.04em, 0) $); %
          \draw [arrow1, sunset_orange] ($(cdot\x.east)+(0.08em, 0)$) -- ($(sa\x2-hs.west) +(-0.08em, 0) $); %
        }

      \pic[name=cdot1-1, local bounding box=cdot1-1] at ($(cdot1o1 |- sa11-mha)$) {my_cdot};
      \pic[name=cdot1-2, local bounding box=cdot1-2] at ($(cdot1o1 |- sa11-mlp)$) {my_cdot=[sushi]};

      \foreach \x in {0,1,2}
        {
          \pic[name=vdot\x, local bounding box=vdot\x] at ($(sa1\x-hs.north) + (0, 2.7em)$) {my_dot};
          \draw [FARROW, puerto_rico, thick] ($(vdot\x.north)+(0, 0.06em)$) to ($(sa3\x-hs.south)+(0, -0.04em)$);
        }

    \pic[name=cdot_io_0, local bounding box=cdot_io_0] at ($(sa12-hs.east)!0.5!(sa15-hs.west)$) {my_cdot=[puerto_rico]};
    \foreach \x/\y in {1/sa12-mha}
      {
        \pic[name=cdot_io_\x, local bounding box=cdot_io_\x] at ($(cdot_io_0o1 |- \y)$) {my_cdot};
      }
      
      
     \foreach \x/\y in {2/sa12-mlp}
      {
        \pic[name=cdot_io_\x, local bounding box=cdot_io_\x] at ($(cdot_io_0o1 |- \y)$) {my_cdot=[sushi]};
      }
      
      \pic[name=cdot_io_30, local bounding box=cdot_io_30] at ($(cdot1o1 |- sa30-hs)$) {my_cdot=[puerto_rico]};
      
      \pic[name=cdot_io_31, local bounding box=cdot_io_31] at ($(cdot_io_0o1 |- sa30-hs)$) {my_cdot=[puerto_rico]};
    
    \draw [arrow1, sunset_orange] ($(sa12-hs.east)+(0.04em, 0)$) -- ($(cdot_io_0.west)+(-0.04em, 0) $);
    \end{scope}
    
    
	\begin{scope}[opacity=0.3]
     \draw [arrow1, sunset_orange] ($(cdot_io_0.east)+(0.08em, 0)$) -- ($(sa15-hs.west) +(-0.08em, 0) $);
	\end{scope}
    
    
    \node[text_node,
      anchor=base,        %
      text depth=.15em,
      xshift=-0.3em,
      ]
    (de_out1) at ($(sa15-hs.north) + (0, 5.5em)$) {Beyonc\'e};
    
    \node[text_node,
      anchor=base,        %
      text depth=.15em,
      ]
    (de_out2) at ($(sa16-hs.north |- de_out1.base)$) {is};
    
    \node[text_node,
  anchor=base,        %
      text depth=.15em,
  ]
(de_out3) at ($(sa17-hs.north |- de_out1.base)$) {the};

    \node[text_node,
  anchor=base,        %
      text depth=.15em,
  ]
(de_out4) at ($(sa18-hs.north |- de_out1.base)$) {writer};
    \node[text_node,
      anchor=base,        %
      text depth=.15em,
      font=\fontsize{4.4pt}{5pt}\selectfont,
      xshift=0.4em,
      opacity=0.5,
  ]
(de_out5) at ($(sa19-hs.north |- de_out1.base)$) {<|endoftext|>};
\begin{scope}[on background layer]
    	\node[project,
      rectangle, fill=flamingo!60, minimum width=2.2em, minimum height=0.6em, rounded corners=1,
    ]
    (o_bg) at ($(de_out5) + (0, 0.05em)$) {};
    \end{scope}


    \node[text_node,
       anchor=base,
       text opacity=0,
      text depth=.15em,
      ]
    (gt5) at ($(sa15-hs.north) + (0, 7.7em)$) {BC};
     \node[text_node,
       anchor=base,
      text depth=.15em,
      ]
    (gt0) at ($(sa15-hs.north) + (1.6em, 7.7em)$) {BC Jean and Toby Gad};

    \foreach \x/\y in {6/Jean, 7/and, 8/Toby, 9/Gad}
    {
    \node[text_node,
       text opacity=0,
      anchor=base,
      text depth=.15em,
      ]
    (gt\x) at ($(sa1\x-hs.south |- gt5.base)$) {\y};
    }
    
    \node[text_node,
      anchor=base east,
      align=right,
      text depth=.15em,
      text width=2em,
      color=tuatara,
      font=\fontsize{4pt}{5pt}\selectfont \linespread{0.4}\selectfont
      ]
    (gt) at ($(de_out1.west) + (-0.2em, 2.65em) $) {\bf Ground Truth:};
    
    \node[text_node,
      anchor=base east,
      text depth=.15em,
      color=tuatara,
      font=\fontsize{4pt}{5pt}\selectfont
      ]
    (op_label) at ($(de_out1.west) + (-0.15em, 0.15em) $) {\bf Output:};

	\begin{scope}[on background layer]
	    \node[fill=athens_gray!42, 
	          draw=tuatara!42, thin,
	          rounded corners=1pt,
	          fit={(gt0)  ($(gt0.west) + (-0.2em, 0)$)  ($(gt0.east) + (0.2em, 0.55em)$)},
	          inner sep=-0.5pt,
	    ] (gt_box) {};
	    \node[fill=athens_gray!42, 
	          draw=tuatara!42, thin,
	          rounded corners=1pt,
	          fit={(de_out1) (de_out4) ($(de_out1.west) + (-0.2em,0)$) ($(de_out4.east) + (0.2em,0)$)},
	          inner sep=-0.5pt,
	    ] (output_box) {};
	\end{scope}
	
	
	 \foreach \x/\y in {5, 6, 7, 8, 9}
      {
         \draw [FARROW, puerto_rico, thick] ($(sa3\x-hs.north)+(0, 0.04em)$) to ++(0, 1.05em) ;
      }
      
      
     \node[regular polygon, regular polygon sides=8, draw=flamingo!200, fill=flamingo, text=white, minimum size=0.2em, inner sep=0pt, outer sep=0pt] (stop) at ($(gt9.east) + (1.25em, -10em) $) {\fontsize{4pt}{5pt}\selectfont\bf STOP};
      
     \coordinate (stp_p) at ($(de_out5.east)+ (0, 0.05em) $);
      
      \draw [FARROW, flamingo, thick, dash pattern=on 1pt off 0.8pt] ($(de_out5.east)+ (-0.1em, 0.05em) $) to ($(stp_p -| stop.north)$) to node[midway, sloped=true, color=black, font=\scriptsize, yshift=0.4em, xshift=-0.2em] (stop_label) {\bf  natural stopping criteria} (stop.north) ;
      \node[text_node] at ($(stop_label.west) + (0, 0)$) { \textcolor{flamingo}{\ding{204}}};
      
            

     \node[
      anchor=center,
      align=center,
    ]
    (metric) at ($(gt.west) + (-0.6em, -1.6em)$) {\textcolor{ship_gray}{\large \faIcon{robot}}};
    \node[text width=2.3em, text depth=.1em]
    (metric_label) at ($(metric.west) + (-0.7em, 0.3em)$) {\fontsize{6pt}{1pt}\selectfont  \bfseries LLM-as-};
    \node[text width=2.3em, text depth=.1em, align=left, anchor=west]
    (metric_label1) at ($(metric_label.west) + (0, -0.6em)$) {\fontsize{6pt}{1pt}\selectfont  \bfseries  a-Judge};
    \node[text_node]
    (metric_an) at ($(metric_label.west) + (0, -0.3em)$) {\textcolor{flamingo}{\ding{205}}};
    \node[text_node,
     text=flamingo!62!red,
    ]
    (result) at ($(metric.east) + (1em, 0)$) {0};

     \draw [FARROW, red, densely dotted] ($(result.west)+(-0.9em, 0)$) to ++(1.1em, 0);
    \draw [FARROW] ($(gt_box.west)+(0, 0)$) to ++(-3em, 0) to ++(0, -0.55em);
     \draw [FARROW] ($(output_box.west)+(0, 0)$) to ++(-3em, 0) to ++(0, 0.55em);

    \begin{scope}[on background layer]
    
    \coordinate (target1) at ($(de_out1.east)+(0, -8.7em)$);

    \draw [FARROW, red] ($(de_out1.east)+(-0.2em, 0)$) to ($(de_out1.east)+(0, 0)$) to ++(0, -8.7em) to ($(target1 -| de1.south)$) to ($(de1.base)+(0, -0.1em)$) ;
    \draw [FARROW, red] ($(de_out2.east)+(-0.2em, 0)$) to ++ (1.1em, 0) to ++(0, -8.7em) to ($(target1 -| de2.south)$) to ($(de2.base)+(0, -0.1em)$);
    \draw [FARROW, red] ($(de_out3.east)+(-0.2em, 0)$) to ++ (0.35em, 0) to ++(0, -8.7em) to ($(target1 -| de3.south)$) to ($(de3.base)+(0, -0.1em)$);  
    \draw [FARROW, red] ($(de_out4.east)+(-0.2em, 0)$) to ++ (0.2em, 0) to ++(0, -8.7em) to ($(target1 -| de4.south)$) to ($(de4.base)+(0, -0.1em)$);
    	
    \end{scope}

	
    \begin{scope}[on background layer]
    \node[fill=gallery!42,
          rounded corners=2pt,
          fit={(sa10)(sa15)(sa16)(sa17)(sa18)(sa19)(sa30-hs)},
          inner sep=3pt,
          ] (dashedBox) {};
          
     \draw [dashed, flamingo] ($(cdot_io_0.north)+(0, -3.8em)$) to ($(cdot_io_31.north)+(0, 1.2em)$) ;
     
     \end{scope}
    
    \node[
          draw,
          rounded corners=2pt,
          fit={(sa10)(sa15)(sa16)(sa17)(sa18)(sa19)(sa30-hs)},
          inner sep=3pt,
          ] {};
     \node[
          rounded corners=2pt, opacity=0.8,
          fit={(sa10)(sa15)(sa16)(sa17)(sa18)(sa19)(sa30-hs)},
          inner sep=7pt,
          align=center, text height=24pt, text depth=0.9em,
          ]  {\fontsize{24}{30}\selectfont Edited $\,$ LLM};

  \end{tikzpicture} %
  \captionsetup{skip=2pt}
  \caption{real-world evaluation framework}
  \label{fig:prac_eval}
\end{subfigure}
\captionsetup{skip=-1pt}
\caption{Illustration of editing and real-world evaluation frameworks, each comprising four key modules:\textcolor{flamingo}{\ding{202}}$\,$\textit{input}, \textcolor{flamingo}{\ding{203}}$\,$\textit{generation strategy}, \textcolor{flamingo}{\ding{204}}$\,$\textit{output truncation}, and \textcolor{flamingo}{\ding{205}}$\,$\textit{metric}, for measuring reliability, generalization, and locality.}
\label{fig:eval_frame}
\end{figure*}

\begin{table}[t]
\centering
\setlength{\tabcolsep}{3.5pt}
\begin{adjustbox}{max width=\linewidth} 
\begin{tabular}{lrrrrrrc}
\toprule
\textbf{Method} & FT-M & MEND & ROME & MEMIT & GRACE & WISE & Avg. \\
\midrule
\textbf{Accuracy} & \num{0.611} & \num{0.333} & \num{0.585} & \num{0.552} & \num{0.012} & \num{0.216} & \num{0.385} \\
\bottomrule 
\end{tabular}
\end{adjustbox}
\caption{Accuracy of edited Llama-2-7b-chat on questions it failed before editing in QAEdit.}
\label{tab:pre_invest}
\end{table}


While these benchmarks provide questions and answers as \textit{edit prompts} and \textit{targets} respectively, they lack essential fields that mainstream editing methods require for editing and evaluation.
To obtain required \textit{subjects} for editing, we employ GPT-4 (gpt-4-1106-preview) to extract them directly from the questions.
To align with the previous editing evaluation protocol, we assess: reliability using original \textit{edit prompts}; generalization through GPT-4 \textit{paraphrased prompts}; and locality using \textit{unrelated QA pairs} from ZsRE locality set\footnote{We exclude portability evaluation as it concerns reasoning rather than our focus on knowledge updating in real-world.}.

As a result, QAEdit contains 19,249 samples across ten categories, ensuring diverse coverage of QA scenarios. 
Figure~\ref{fig:QAedit_example} shows a QAEdit entry with all fields.
Dataset construction and dataset statistics are detailed in Appendix~\ref{apd:benchmark}.



As a preliminary study, we conduct single-edit experiments on Llama-2-7b-chat's failed questions in QAEdit (detailed in \S\ref{sec:single_edit}). 
As shown in Table~\ref{tab:pre_invest}, after applying SOTA editing methods, the edited models achieve only 38.5\% average accuracy under QA evaluation, far below previously reported results \cite{meng2023massediting, wang2024wise}.
This raises a critical question: \textit{Is the performance degradation attributed to the real-world complexity of QAEdit, or to real-world QA evaluation?}





