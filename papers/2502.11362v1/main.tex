%%%%%%%% ICML 2025 EXAMPLE LATEX SUBMISSION FILE %%%%%%%%%%%%%%%%%

\documentclass{article}

% Recommended, but optional, packages for figures and better typesetting:
\usepackage{microtype}
\usepackage{graphicx}
\usepackage{booktabs} % for professional tables
\usepackage{adjustbox}
% hyperref makes hyperlinks in the resulting PDF.
% If your build breaks (sometimes temporarily if a hyperlink spans a page)
% please comment out the following usepackage line and replace
% \usepackage{icml2025} with \usepackage[nohyperref]{icml2025} above.
\usepackage{hyperref}
\usepackage{url}

\usepackage{graphicx}  % Required to insert images
\usepackage{subcaption}


% \usepackage{algorithm}
\usepackage{algpseudocode}

\usepackage{float}

% Attempt to make hyperref and algorithmic work together better:
\newcommand{\theHalgorithm}{\arabic{algorithm}}

% Use the following line for the initial blind version submitted for review:
% \usepackage{icml2025}

% If accepted, instead use the following line for the camera-ready submission:
\usepackage[accepted]{icml2025}

% For theorems and such
\usepackage{amsmath}
\usepackage{amssymb}
\usepackage{mathtools}
\usepackage{amsthm}
\usepackage{booktabs}
% if you use cleveref..
\usepackage[capitalize,noabbrev]{cleveref}

%%%%%%%%%%%%%%%%%%%%%%%%%%%%%%%%
% THEOREMS
%%%%%%%%%%%%%%%%%%%%%%%%%%%%%%%%
\theoremstyle{plain}
\newtheorem{theorem}{Theorem}[section]
\newtheorem{proposition}[theorem]{Proposition}
\newtheorem{lemma}[theorem]{Lemma}
\newtheorem{corollary}[theorem]{Corollary}
\theoremstyle{definition}
\newtheorem{definition}[theorem]{Definition}
\newtheorem{assumption}[theorem]{Assumption}
\theoremstyle{remark}
\newtheorem{remark}[theorem]{Remark}

% Todonotes is useful during development; simply uncomment the next line
%    and comment out the line below the next line to turn off comments
%\usepackage[disable,textsize=tiny]{todonotes}
\usepackage[textsize=tiny]{todonotes}
% \usepackage[dvipsnames]{xcolor}

% The \icmltitle you define below is probably too long as a header.
% Therefore, a short form for the running title is supplied here:
\icmltitlerunning{Teleportation With Null Space Gradient
Projection for Optimization Acceleration}

\begin{document}

\twocolumn[
\icmltitle{Teleportation With Null Space Gradient
Projection for Optimization Acceleration}

% It is OKAY to include author information, even for blind
% submissions: the style file will automatically remove it for you
% unless you've provided the [accepted] option to the icml2025
% package.

% List of affiliations: The first argument should be a (short)
% identifier you will use later to specify author affiliations
% Academic affiliations should list Department, University, City, Region, Country
% Industry affiliations should list Company, City, Region, Country

% You can specify symbols, otherwise they are numbered in order.
% Ideally, you should not use this facility. Affiliations will be numbered
% in order of appearance and this is the preferred way.
\icmlsetsymbol{equal}{*}

\begin{icmlauthorlist}
\icmlauthor{Zihao Wu}{yyy}
\icmlauthor{Juncheng Dong}{yyy}
\icmlauthor{Ahmed Aloui}{yyy}
\icmlauthor{Vahid Tarokh}{yyy}
% \icmlauthor{Firstname5 Lastname5}{yyy}
% \icmlauthor{Firstname6 Lastname6}{sch,yyy,comp}
% \icmlauthor{Firstname7 Lastname7}{comp}
%\icmlauthor{}{sch}
% \icmlauthor{Firstname8 Lastname8}{sch}
% \icmlauthor{Firstname8 Lastname8}{yyy,comp}
%\icmlauthor{}{sch}
%\icmlauthor{}{sch}
\end{icmlauthorlist}

\icmlaffiliation{yyy}{Department of Electrical and
Computer Engineering, Duke University, Durham, NC 27708, USA}
% \icmlaffiliation{comp}{Company Name, Location, Country}
% \icmlaffiliation{sch}{School of ZZZ, Institute of WWW, Location, Country}

% \icmlcorrespondingauthor{Firstname1 Lastname1}{first1.last1@xxx.edu}
\icmlcorrespondingauthor{Zihao Wu}{zihao.wu@duke.edu}

% You may provide any keywords that you
% find helpful for describing your paper; these are used to populate
% the "keywords" metadata in the PDF but will not be shown in the document
\icmlkeywords{Machine Learning, ICML}

\vskip 0.3in
]

% this must go after the closing bracket ] following \twocolumn[ ...

% This command actually creates the footnote in the first column
% listing the affiliations and the copyright notice.
% The command takes one argument, which is text to display at the start of the footnote.
% The \icmlEqualContribution command is standard text for equal contribution.
% Remove it (just {}) if you do not need this facility.

%\printAffiliationsAndNotice{}  % leave blank if no need to mention equal contribution
\printAffiliationsAndNotice{} % otherwise use the standard text.

\begin{abstract}
% As deep learning continues to evolve, 
Optimization techniques have become increasingly critical due to the ever-growing model complexity and data scale. In particular, teleportation has emerged as a promising approach, which accelerates convergence of gradient descent-based methods by navigating within the loss invariant level set to identify parameters with advantageous geometric properties. Existing teleportation algorithms have primarily demonstrated their effectiveness in optimizing Multi-Layer Perceptrons (MLPs), but their extension to more advanced architectures, such as Convolutional Neural Networks (CNNs) and Transformers, remains challenging. Moreover, they often impose significant computational demands, limiting their applicability to complex architectures. To this end, we introduce an algorithm that projects the gradient of the teleportation objective function onto the input null space, effectively preserving the teleportation within the loss invariant level set and reducing computational cost. Our approach is readily generalizable from MLPs to CNNs, transformers, and potentially other advanced architectures. We validate the effectiveness of our algorithm across various benchmark datasets and optimizers, demonstrating its broad applicability.
\end{abstract}

\section{Introduction}

Consider an optimization problem where the objective function, denoted by $\mathcal{L}\left(\omega\right)$, is parameterized by $\omega \in \Omega$. When $\mathcal{L}\left(\omega\right)$ is non-convex, gradient-based methods are commonly used to find a set of parameters corresponding to local minimums in the loss landscape. The standard update rule for gradient descent is given by:
\begin{align}
\mathbf{\omega}_{t+1} \leftarrow \mathbf{\omega}_t - \eta \nabla \mathcal{L}\left(\omega_t\right),
\end{align}
where $\omega_t$ represents the parameter values at iteration $t$ and $\eta>0$ is the learning rate. As a first-order method, gradient descent is computationally efficient but often suffers from slow convergence. In contrast, second-order methods, such as Newton’s method, incorporate higher-order geometric information, resulting in faster convergence. However, this comes with significant computational cost, particularly due to the need to compute and invert the Hessian matrix~\citep{hazan2019lecture}. To address this challenge, teleportation is introduced as a method that exploits higher-order geometric properties while relying solely on gradient information.

Teleportation is based on the premise that multiple points in the parameter space can yield the same loss, which forms the \textbf{\emph{loss invariant level set}} of parameters~\citep{du2018algorithmic, kunin2020neural}. This assumption is particularly feasible in modern deep learning, where most advanced models are highly over-parameterized~\citep{sagun2017empirical, tarmoun2021understanding, simsek2021geometry}. By identifying the level set, parameters can be teleported within it to \textbf{\emph{enhance the gradient norm}}, thereby accelerating the optimization process~\citep{kunin2020neural, grigsby2022functional}. 

\begin{figure*}[htbp]
    \centering
    \includegraphics[width=\textwidth]{images/pdf/open.png} % Change the file name to your actual image file
    \caption{From left to right: symmetry teleport (slow and limited to MLPs), linear approximation of level set (prone to error), our algorithm that projects gradient onto the input null space (fast and accurate).}
    \label{fig:open}
\end{figure*}

\textbf{Related Work.} \cite{zhao2022symmetry} indicates that the behavior of teleportation, despite utilizing only gradient information, closely resembles that of Newton’s method. An alternative perspective on teleportation is that it mitigates the locality constraints of the gradient descent algorithm, resembling the dynamics of \emph{warm restart algorithms} ~\citep{loshchilov2016sgdr, dodge2020fine, bouthillier2021accounting,ramasinghe2022you}. Under this context, each step of gradient descent is equivalent to a proximal mapping ~\citep{combettes2011proximal}.
% \begin{align}
% \omega_{t+1} = \argmin_{\omega \in \Omega} \left\{ \eta \left\langle \nabla \mathcal{L}\left(\omega_t\right), \omega \right\rangle + \frac{1}{2} \| \omega - \omega_t \|_2^2 \right\},
% \end{align}
% where $\langle\cdot\rangle$ denotes inner product
Teleportation periodically relaxes the proximal restriction, allowing the algorithm to restart at a distant location with desirable geometric properties. \emph{Compared to warm restart algorithms, teleportation incurs minimal to no increase in loss while providing greater control over the movement of parameters}. Notably, the field of teleportation reveals a gap between theoretical developments and practical applications. \cite{zhao2022symmetry} shows that gradient descent (GD) with teleportation can achieve mixed linear and quadratic convergence rates on strongly convex functions. \cite{mishkin2024level} proves that, for convex functions with Hessian stability, GD with teleportation attains a convergence rate faster than
$O(1/K)$. \textbf{\emph{However, both approaches encounter limitations when applied to empirical studies involving highly non-convex functions, which are a common characteristic of modern architectures}}. Specifically, \cite{zhao2022symmetry} develops a symmetry teleportation algorithm \emph{only for Multi-Layer Perceptrons} (MLPs) using group actions~\citep{ math9243216, ganev2021universal, armenta2023neural} . However, challenges persist in terms of its generalizability to other contemporary architectures and its relatively low efficiency. \citet{mishkin2024level}, on the other hand, tackled a sequential quadratic programming by using linear approximations of the level set, which can \emph{lead to error accumulation} when the architecture becomes more complicated and the number of teleportation steps increase (see Figure~\ref{fig:open} for a visual comparision). Moreover, both studies have primarily concentrated on empirical results involving MLPs and the vanilla Stochastic Gradient Descent (SGD) optimizer.


% The crux of teleportation in practice is to identify the loss-invariant level set of parameters. Previous literature focus on symmetry teleportation, which attempts to design a specific group action for each function, e.g., booth function, Rosenbrock function, and multi-layer perceptrons, such that after the transformation of the parameter, the loss does not change. 

% One serious pain of symmetry teleportation is its unique design for each different function. The mannually designed group action for MLP is not readily generalizable to other modern architectures, such as convolutional layers and transformers. Furthermore, the computation is usually heavy with group action. For instance, it requires two inverse operations for each layer of MLP structure and is not parallelizable since each layer's transformation is dependent on another. 

% Another recent paper assumes the convexity of functions and solves a quadratic programming problem to find the parameter with maximal gradient norm under the restriction of level set, which is in closed form with convexity assumption. However, for non-convex functions, the level set does not embrace a closed form, and they utilize a linear approximation of the level set, which is subject to error accumulation if function is complicated.

\textbf{Contributions.} Our work seeks to overcome these challenges by designing an algorithm not only \textbf{\emph{generalizes to other modern architectures}}, but also is \textbf{\emph{efficient and accurate}}. To be more specific, we eliminate the need for the bottleneck group action transformations ~\cite{zhao2022symmetry} by utilizing a more efficient \textbf{\emph{gradient projection}} technique. Moreover, instead of taking on the errors introduced by linear approximations of the level set, we \textbf{\emph{project the gradient of the teleportation objective onto the input null space of each layer}}, ensuring an accurate search on the level set thus minimal to no change in loss value.
Specifically, our contributions are:
\begin{itemize}
\item We propose a novel algorithm that utilizes gradient projection to offer improved computational efficiency and parallelization capabilities.
\item The proposed algorithm is a \textbf{\emph{general framework that can be easily applied to various modern architectures}}, including MLPs, Convolutional Neural Networks (CNNs), transformers, and potentially linear time series models such as Mamba~\citep{gu2023mamba} and TTT~\citep{sun2024learning}. As a result, our work is the first work to extend teleportation to CNNs and transformers.
\item  We present \textbf{\emph{extensive empirical results}} to demonstrate its effectiveness, spanning a range of benchmark datasets, including MNIST, FashionMNIST, CIFAR-10, CIFAR-100, Tiny-ImageNet, multi-variate time series datasets (electricity and traffic), and Penn Treebank language dataset. We also evaluate the algorithm with multiple modern optimizers, such as SGD~\citep{robbins1951stochastic}, Momentum~\citep{polyak1964some}, Adagrad~\citep{duchi2011adaptive}, and Adam~\citep{kingma2014adam}, whereas previous studies primarily focused on the vanilla SGD.
% \item While the loss invariant property of input null space gradient projection has been proven effective under image processing continual learning contexts ~\cite{saha2021gradient}, we further demonstrate its applicability to transformers for preserving loss value. This finding could have implications for future research in continual learning.
\end{itemize}

\section{Preliminary}


\subsection{Symmetry Teleportation}

In this section, we describe the general framework of teleportation through a state-of-the-art algorithm, \emph{symmetry teleportation}~\citep{zhao2022symmetry, zhao2023improving}. 
% which follows from a series of works ~\citep{armenta2023neural, math9243216}. 

Let $G$ be a set of symmetries that preserves the loss value $\mathcal{L}$, i.e., let $\omega = (X,W)$,
\begin{align}
    \mathcal{L}(X, W) = \mathcal{L}(g\cdot (X, W)), \forall g\in G,
\end{align} 
where $X$ represents data and $W$ represents \emph{parameters of the deep learning model}. Define a teleport schedule $K\subset \{0,1,..., T_{max}\}$, where $T_{max}$ is the maximum training epochs. Prior to each epoch in $K$, teleportation is applied by searching for $g\in G$ which transforms the parameter $W$ to $W^*$ with greater gradient norm \emph{within the loss invariant level set}. 
% For each teleportation step, multiple batches of data can be sampled, and $g$ can be updated multiple times per batch.

When the group $G$ is continuous, the search process can be conducted by parameterizing the group action $g$ and performing gradient ascent on $g$ with the teleportation objective function defined as the gradient norm of the current parameter $W$. For example, general linear group transformations $g \in GL_d(\mathbb{R})$ can be parameterized as $g = I + \epsilon M$, where $\epsilon \ll 1$ and $M$ is an arbitrary matrix.

\cite{zhao2022symmetry, zhao2023improving} designs a loss invariant group action \emph{specifically for MLPs with bijective activation function $\sigma$}. Assuming the invertibility of $(k-2)$-th layer's output, $h_{k-2}$, the following group action $g\in GL_d(\mathbb{R})$ on $k$-th and $(k-1)$-th layers ensures the output of the entire network unchanged:
\[
g_m \cdot W_k =
\begin{cases}
    W_m g_m^{-1} & \text{if } k = m, \\
    \sigma^{-1}\left(g_m \sigma\left(W_{m-1} h_{m-2}\right)\right) h_{m-2}^{-1} & \text{if } k = m - 1, \\
    W_k & \text{otherwise}.
\end{cases}
\] 
In practice, each teleportation update applies the above group action to every layer of an MLP, requiring two bottleneck inverse operations per update.  Denote $D_{max}$ as the largest width of the MLPs, and $n$ the sample size, assuming $D_{max}>n$. The time complexity of calculating pseudo-inverse for each layer is $O(D_{max}^2n)$. Therefore, the total time complexity for $l$ layers, $b$ batches, and $t$ teleport updates per batch is $O(D_{max}^2nlbt)$. The need for pseudo-inverse computations and the dependencies between layers render the algorithm relatively slow and unsuitable for parallelization. Additionally, there is no straightforward method to generalize this design from MLPs to CNNs or transformers.
% \subsection{Gradient Projection}

% The gradient projection technique is widely employed in deep learning, particularly in continual learning, to mitigate the issue of catastrophic forgetting ~\citep{saha2021space, saha2021gradient, lin2022trgp, saha2023continual}. Let $W$ denote the weight of any linear model after training on the first task, and let $x$ represent the input of the first task. During the training of a second task, the gradient of the loss function with respect to $W$ is projected onto the input null space of the first task prior to the update. This projection of gradient ensures that the loss of the updated model remains unchanged for the input of the first task. Concretely, for any update $\Delta$ on $W$, assuming $\Delta$ can be projected onto the null space of $x$ by a projection operator $\pi$, 
% \begin{align}
%     (W+\pi(\Delta))x= Wx+\pi(\Delta) x = Wx
% \end{align}
% shows that the output of the model for the first task remains unchanged during the training of the second task.

\subsection{Matrix Approximation With SVD}\label{AA}
An arbitrary matrix $A\in \mathbb{R}^{(m,n)}$ can be decomposed using the singular value decomposition (SVD)~\cite{klema1980singular} as $A = U\Sigma V^T$, where $U\in \mathbb{R}^{(m,m)}$ consists of orthonormal eigenvectors of $AA^T$, $\Sigma \in \mathbb{R}^{(m,n)}$ is a diagonal matrix containing sorted singular values, and $V\in \mathbb{R}^{(n,n)}$ contains orthonormal eigenvectors of $A^TA$. The matrix $A$ can be expressed as $\sum_{i=1}^r\sigma_i u_i v_i^T$,
where $r=\min(m,n)$, and $(u_i,v_i)$ are the column and row vectors of $U,V$ respectively. 

In this work, we consider the matrix approximation $A_k$ of $A$ defined as $A_k = \sum_{i=1}^k\sigma_i u_i v_i^T$, where 
\begin{align}\label{eq:threshold}
    k = \arg\min_{k} \left\{k:||A_k||_F^2 \ge \tau ||A||_F^2\right\},
\end{align}
with $\|\cdot\|_F$ denotes the Frobenius norm and $\tau \in [0,1]$ being a threshold hyper-parameter.

\section{Teleport With Null Space Gradient Projection}
Our objective is to develop a generalizable and efficient algorithm that avoids reliance on specific group action designs. Moreover, it should avoid any (linear) approximation of the level set with uncontrollable errors, as these could otherwise result in suboptimal performance. Considering the common architectural design in modern neural networks, which typically employ a linear relationship between weights and inputs of each layer, the technique of \textbf{\emph{gradient projection on to the input null space}} of each layer is well-suited for this purpose. We next elaborate on it.

\textbf{Gradient Projection.} To incorporate the geometric landscape and accelerate optimization using only gradient information, the objective function for teleportation is defined as the squared gradient norm of the loss function of the primary task with respect to the model parameter $W$,
\begin{align}
    L_{teleport} = \frac{1}{2}\|\nabla_W L_{primary}\|^2.
\end{align}

During each teleportation step, in contrast to symmetry teleportation, the gradient ascent is applied directly on the model parameter $W_l$ of each layer $l$ instead of relying on an intermediate group action $g$, i.e., we have
\begin{align}
    W_{l,t+1} = W_{l,t} + \eta\pi_l(\nabla_{W_l}L_{teleport}),\label{eq:tele-update}
\end{align}
where $\eta$ is the learning rate for teleportation update, and $\pi_l$ is the \textbf{\emph{layerwise projection operator}} onto the null space of each layer's input. We have distinct projection operators for different model architectures. \textbf{\emph{We will derive $\pi_l$ for MLPs, CNNs and transformers in the sequel}}. The validity of this projection is based on the assumption that \emph{the gradient resides within the span of each layer's input for certain structures}, which will also be elaborated in a subsequent section.


\textbf{Section Organization.} We first define and provide notations for MLPs, CNNs, and transformers. Next, we demonstrate that the gradient in Equation~\ref{eq:tele-update} indeed resides within the layerwise input space of these architectures, thus \textbf{\emph{satisfying the required assumption of gradient projection}}. Finally, we present our proposed approach and provide a detailed explanation of how to derive the projection operators for each of these architectures. 
\subsection{Deep Learning Architechtures}\label{sec:notation}
\subsubsection{Multi-Layer Perceptrons}
We define the $l$-th layer of an MLP~\citep{rumelhart1986learning}. Denote the input of the layer as $x_{l-1} \in \mathbb{R}^{(d_{l-1},1)}$, the parameter as $W_l\in \mathbb{R}^{(d_l,d_{l-1})}$, the output as $x_l\in \mathbb{R}^{(d_l,1)}$. We incorporate the bias term into $W_l$ and $x_{l-1}$ by adding an additional column to $W_l$ and unity to $x_{l-1}$. Then the output of $l$-th layer is defined as
\[
    x_{l} = \sigma(W_{l}x_{l-1}),
\]
where $\sigma$ is an activation function, e.g. ReLU~\citep{nair2010rectified}.
\subsubsection{Convolutional Neural Network}
% Filters in convolutional layer operate differently than the weights in MLPs. 
We define the $l$-th layer of a CNN~\citep{lecun1998gradient}. Denote the input to the $l$-th convolutional layer as $x_{l-1}\in \mathbb{R}^{C_i\times h_i\times w_i}$, convolutional kernel as $W_l \in \mathbb{R}^{C_o\times C_i\times k\times k}$, and output as $x_l \in \mathbb{R}^{C_o \times h_o \times w_o}$, where $C_i, h_i, w_i (C_o, h_o, w_o)$ are the input (output) channel, height, and width, respectively, and $k$ is the kernel size. If $x_{l-1}$ (e.g., with padding, striding, etc) is reshaped into $(h_o\times w_o)\times (C_i\times k\times k)$ as $X_{l-1}$, and $W_l$ is reshaped to $(C_i\times k\times k)\times C_o$, then the convolutional layer can be expressed as a matrix multiplication 
\[
    x_l = \sigma(X_{l-1}W_l),
\]
where $x_{l}\in \mathbb{R}^{(x_o\times w_o)\times C_o}$ is the output of $l$-th layer, and $\sigma$ an activation function. See Appendix~\ref{sec:visual} for a visual explanation of the matrix multiplication.

% Likewise, the backward operation can also be formulated in the form of matrix multiplication. The matrix representation of both f
\subsubsection{Transformer}
We define the self-attention and multi-head self-attention layers~\citep{vaswani2017attention}. Denote the input sequence of the $l$-th self-attention layer as $X_{l-1}\in \mathbb{R}^{T\times D_i}$, with sequence length $T$ and dimension $D_i$. The $l$-th self-attention layer is parameterized by the query matrix $W_{l,q}\in \mathbb{R}^{(D_i,D_k)}$, the key matrix $W_{l,k}\in \mathbb{R}^{(D_i,D_k)}$, and the value matrix $W_{l,v}\in \mathbb{R}^{(D_i,D_o)}$. Then, the self-attention layer maps the sequence from dimension $D_i$ to $D_o$ by
\[
    Attention(Q,K,V) = softmax(\frac{QK^T}{\sqrt{D_k}})V,
\]
where $Q = X_{l-1}W_{l,q}, K=X_{l-1}W_{l,k}, V=X_{l-1}W_{l,v}$, and $D_k$ is the dimension of the model. 



The multi-head attention is realized by replicating and concatenating $N_h$ heads of low-rank self-attentions before applying an output projection, defined as
\begin{align}
    &MultiHead(X_{l-1}) = concat_{i\in [N_h]}[H^{(i)}]W_{l,o}\\
    &H^{(i)}=Attention(X_{l-1}W_{l,q}^{(i)},X_{l-1}W_{l,k}^{(i)},X_{l-1}W_{l,v}^{(i)}),
\end{align}
where $W_{l,q}^{(i)}\in \mathbb{R}^{(D_i,\frac{D_k}{N_h})}$, $W_{l,k}^{(i)}\in \mathbb{R}^{(D_i,\frac{D_k}{N_h})}$, $W_{l,v}^{(i)}\in \mathbb{R}^{(D_i,\frac{D_k}{N_h})}$ are parameters for each head. The output projection matrix $W_{l,o} \in \mathbb{R}^{(D_k,D_o)}$ maps the concatenation of heads to the desired output dimension. 
% Both attention and multi-head attention layers are typically followed by a single hidden layer MLPs, which together form a complete Transformer layer. 

\subsection{Input and Gradient Space}\label{sec:input-space}
Now we establish that \textbf{\emph{the gradient of the teleportation objective function resides within the space spanned by the input of each layer}}. Following the notation established in Section ~\ref{sec:notation}, we can readily express the gradient of the teleportation objective function with respect to the model parameter $W_l$ for each type of structure:
\begin{align*}
    \text{MLP}:\ \ \ \ \ \ \ \ &\\
    \nabla_{W_l} L_{Teleport} &= \nabla_{(W_lx_{l-1})}L_{Teleport}\cdot\nabla_{W_l}(W_lx_{l-1})\\
    &=\delta_{MLP}x_{l-1}^T\\
    \text{CNN}:\ \ \ \ \ \ \ \ &\\
    \nabla_{W_l} L_{Teleport} &= \nabla_{W_l}(X_{l-1}W_l)\cdot\nabla_{(X_{l-1}W_l)}L_{Teleport}\\
    &=X_{l-1}^T\cdot \delta_{CNN}\\
    \text{Self-Attenti}&\text{on}:\\
     \nabla_{W_{l,\cdot}^{(i)}} L_{Teleport} &= \nabla_{W_{l,\cdot}^{(i)}}(X_{l-1}W_{l,\cdot}^{(i)})\cdot\nabla_{(X_{l-1}W_{l,\cdot}^{(i)})}L_{Teleport}\\
    &=X_{l-1}^T\cdot \delta_{Attention},
\end{align*}
where $\delta_{MLP}\in \mathbb{R}^{(d_l,1)}$, $\delta_{CNN}\in \mathbb{R}^{(h_o\times w_o,C_o)}$, and $\delta_{Attention}\in \mathbb{R}^{(T,D_k)}$ are some error terms determined by both the loss function of the primary task and the objective function of the teleportation. Here, it can be observed that all gradients above can be written as the matrix multiples involving the input $X$ of each layer and another matrix. Thus, the gradient of the teleportation objective function indeed resides within the space spanned by the input of each layer for MLPs, CNNs, and transformer, which is a composition of attention layers and MLP layers. 


\begin{figure*}[htbp]
    \centering
    \begin{subfigure}{0.24\textwidth} % Adjust the width as needed
        \centering
        \includegraphics[width=\textwidth]{images/pdf/comparison_sgd_train.pdf}
        % \caption{Figure 1}
    \end{subfigure}
    \begin{subfigure}{0.24\textwidth}
        \centering
        \includegraphics[width=\textwidth]{images/pdf/comparison_adagrad_train.pdf}
        % \caption{Figure 2}
    \end{subfigure}
    \begin{subfigure}{0.24\textwidth}
        \centering
        \includegraphics[width=\textwidth]{images/pdf/comparison_mom_train.pdf}
        % \caption{Figure 3}
    \end{subfigure}
    \begin{subfigure}{0.24\textwidth}
        \centering
        \includegraphics[width=\textwidth]{images/pdf/comparison_adam_train.pdf}
        % \caption{Figure 4}
    \end{subfigure}
   \\
    \begin{subfigure}{0.24\textwidth} % Adjust the width as needed
        \centering
        \includegraphics[width=\textwidth]{images/pdf/FashionMNIST_SGD_MLP_loss_vs_time.pdf}
        % \caption{Figure 1}
    \end{subfigure}
    \begin{subfigure}{0.24\textwidth}
        \centering
        \includegraphics[width=\textwidth]{images/pdf/FashionMNIST_momentum_MLP_loss_vs_time.pdf}
        % \caption{Figure 2}
    \end{subfigure}
    \begin{subfigure}{0.24\textwidth}
        \centering
        \includegraphics[width=\textwidth]{images/pdf/FashionMNIST_Adagrad_MLP_loss_vs_time.pdf}
        % \caption{Figure 3}
    \end{subfigure}
    \begin{subfigure}{0.24\textwidth}
        \centering
        \includegraphics[width=\textwidth]{images/pdf/FashionMNIST_Adam_MLP_loss_vs_time.pdf}
        % \caption{Figure 4}
    \end{subfigure}
    \caption{Loss trajectories of training MLPs on the MNIST and FashionMNIST datasets. Each experiment is repeated 3 times, with the average loss plotted and the standard deviation of loss represented as the shaded area.
    }
    \label{fig:mlp}
\end{figure*}

\begin{figure*}[htbp]
    \centering
    \begin{subfigure}{0.195\textwidth} % Adjust the width as needed
        \centering
        \includegraphics[width=\textwidth]{images/pdf/efficiency_t.pdf}
    \end{subfigure}
    \begin{subfigure}{0.195\textwidth}
        \centering
        \includegraphics[width=\textwidth]{images/pdf/efficiency_d.pdf}
    \end{subfigure}
    \begin{subfigure}{0.195\textwidth}
        \centering
        \includegraphics[width=\textwidth]{images/pdf/efficiency_n.pdf}
    \end{subfigure}
    \begin{subfigure}{0.195\textwidth}
        \centering
        \includegraphics[width=\textwidth]{images/pdf/efficiency_l.pdf}
    \end{subfigure}
    \begin{subfigure}{0.195\textwidth}
        \centering
        \includegraphics[width=\textwidth]{images/pdf/efficiency_b.pdf}
    \end{subfigure}
    \caption{From left to right: a comparison between symmetry teleport and our algorithm using MLPs in terms of the scaling of runtime with respect to $t$, $d$, $n$, $l$, and $b$.}
    \label{fig:efficiency}
\end{figure*}

\subsection{Algorithm}
\textbf{Step 1.} We first construct the representation matrix for each layer $l$ based on a given teleportation batch of data:
\begin{align}
    R_{MLP}^l &= [x_{l-1,1}, x_{l-1,2},\cdots, x_{l-1,n}]\\
    R_{CNN}^l &= [X_{l-1,1}^T,X_{l-1,2}^T,\cdots,X_{l-1,n}^T]\\
    R_{Attention}^l &= [X_{l-1,1}^T,X_{l-1,2}^T,\cdots,X_{l-1,n}^T],
\end{align}
where $n$ is the batch size. Each representation matrix $R_{MLP}^l\in \mathbb{R}^{(d_{l-1},n)}$, $R_{CNN}^l\in \mathbb{R}^{(C_i\times k\times k,h_o\times w_o\times n)}$, and $R_{Attention}^l\in \mathbb{R}^{(D_i,T\times n)}$ contains columns of feature vectors, which are captured at each layer during the forward pass through the network using a random teleportation batch of size $n$. 

\textbf{Step 2.} For all model architectures, we apply SVD on the representation matrix $R^l$, followed by a low-rank approximation 
$(R^l)_k = \sum_{i=1}^k \sigma_{l,i} u_{l,i} v_{l,i}^T$ based on the criterion in Equation~\ref{eq:threshold}, using a predefined threshold $\tau$. The orthonormal column vectors $[u_{l,1}, u_{l,2},\dots,u_{l,k}]$, from SVD of $R^l$, consist of the eigenvectors corresponding to the top $k$ singular values of the representation matrix. We define the subspace spanned by these eigenvectors as \textbf{\emph{the space of significant representation}}~\citep{saha2021gradient}. 

During a teleportation step, the goal is to ensure that the gradient update in Equation~\ref{eq:tele-update} preserves the correlation between the weights and the space of significant representation as much as possible. Given that the gradient space lies within the input space, we can partition the gradient space into two orthogonal subspaces of the input space: the \textbf{\emph{Core Gradient Space (CGS)}} and the \textbf{\emph{Residual Gradient Space (RGS)}}~\citep{saha2021space}, which are spanned by $[u_{l,1}, u_{l,2},\cdots,u_{l,k}]$ and $[u_{l,k+1}, u_{l,k+2},\cdots,u_{l,r}]$ respectively. By construction, projecting the gradient onto CGS will lead to the greatest interference in the correlation between the weights and the space of significant representation, while \textbf{\emph{projecting onto RGS will result in minimal or no interference in this correlation}}. To preserve model parameters on the loss-invariant level set during teleportation steps, we project the gradient of teleportation objective function $\nabla_{W_l} L_{Teleport}$ onto the RGS before each update.

\textbf{Step 3.} Given the orthonormal basis $B_l = [u_{l,1}, u_{l,2},\cdots,u_{l,k}]$ of the CGS for the $l$-th layer, the gradient $\nabla_{W_l} L_{Teleport}$ is initially projected onto the CGS and then removed from itself to yield the projection onto the RGS. Specifically, the projection operator $\pi_l$ is defined as follows:
\begin{align}
    &\text{MLP}: \pi_l(\nabla_{W_l} L_{Teleport}) =\\ &\ \ \ \ \nabla_{W_l} L_{Teleport} - (\nabla_{W_l} L_{Teleport})B_lB_l^T
\end{align}
\begin{align}
    &\text{CNN}:\pi_l(\nabla_{W_l} L_{Teleport}) =\\ &\ \ \ \ \nabla_{W_l} L_{Teleport} - B_lB_l^T(\nabla_{W_l} L_{Teleport})
\end{align}
\begin{align}
    &\text{Self-Attention}:\pi_l(\nabla_{W_{l,\cdot}^{(i)}} L_{Teleport}) =\\ &\ \ \ \ \nabla_{W_{l,\cdot}^{(i)}} L_{Teleport} - B_lB_l^T(\nabla_{W_{l,\cdot}^{(i)}} L_{Teleport})
\end{align}

The teleportation step is completed by substituting the projection operator back into Equation ~\ref{eq:tele-update}. The complete training process is outlined in the pseudo-code presented in appendix ~\ref{sec:pseudo}.

\begin{figure*}[htbp]
    \centering
    \begin{subfigure}{0.24\textwidth} % Adjust the width as needed
        \centering
        \includegraphics[width=\textwidth]{images/pdf/CIFAR10_SGD_CNN_loss_vs_epoch.pdf}
        % \caption{Figure 1}
    \end{subfigure}
    \begin{subfigure}{0.24\textwidth}
        \centering
        \includegraphics[width=\textwidth]{images/pdf/CIFAR10_momentum_CNN_loss_vs_epoch.pdf}
        % \caption{Figure 2}
    \end{subfigure}
    \begin{subfigure}{0.24\textwidth}
        \centering
        \includegraphics[width=\textwidth]{images/pdf/CIFAR10_Adagrad_CNN_loss_vs_epoch.pdf}
        % \caption{Figure 3}
    \end{subfigure}
    \begin{subfigure}{0.24\textwidth}
        \centering
        \includegraphics[width=\textwidth]{images/pdf/CIFAR10_Adam_CNN_loss_vs_epoch.pdf}
        % \caption{Figure 4}
    \end{subfigure}
 % \\
 %    \begin{subfigure}{0.24\textwidth} % Adjust the width as needed
 %        \centering
 %        \includegraphics[width=\textwidth]{images/pdf/CIFAR100_SGD_CNN_loss_vs_epoch.pdf}
 %        % \caption{Figure 1}
 %    \end{subfigure}
 %    \begin{subfigure}{0.24\textwidth}
 %        \centering
 %        \includegraphics[width=\textwidth]{images/pdf/CIFAR100_momentum_CNN_loss_vs_epoch.pdf}
 %        % \caption{Figure 2}
 %    \end{subfigure}
 %    \begin{subfigure}{0.24\textwidth}
 %        \centering
 %        \includegraphics[width=\textwidth]{images/pdf/CIFAR100_Adagrad_CNN_loss_vs_epoch.pdf}
 %        % \caption{Figure 3}
 %    \end{subfigure}
 %    \begin{subfigure}{0.24\textwidth}
 %        \centering
 %        \includegraphics[width=\textwidth]{images/pdf/CIFAR100_Adam_CNN_loss_vs_epoch.pdf}
 %        % \caption{Figure 4}
 %    \end{subfigure}
\\
    \begin{subfigure}{0.24\textwidth} % Adjust the width as needed
        \centering
        \includegraphics[width=\textwidth]{images/pdf/Imagenet_SGD_CNN_loss_vs_epoch.pdf}
        % \caption{Figure 1}
    \end{subfigure}
    \begin{subfigure}{0.24\textwidth}
        \centering
        \includegraphics[width=\textwidth]{images/pdf/Imagenet_momentum_CNN_loss_vs_epoch.pdf}
        % \caption{Figure 2}
    \end{subfigure}
    \begin{subfigure}{0.24\textwidth}
        \centering
        \includegraphics[width=\textwidth]{images/pdf/Imagenet_Adagrad_CNN_loss_vs_epoch.pdf}
        % \caption{Figure 3}
    \end{subfigure}
    \begin{subfigure}{0.24\textwidth}
        \centering
        \includegraphics[width=\textwidth]{images/pdf/Imagenet_Adam_CNN_loss_vs_epoch.pdf}
        % \caption{Figure 4}
    \end{subfigure}
    \caption{Loss trajectories of training CNNs on CIFAR dataset and Tiny-Imagenet dataset. Each experiment is repeated 3 times, with the average loss plotted and the standard deviation of loss represented as the shaded area. Result of CIFAR100 is included in Appendix~\ref{sec:cnn_append}.}
    \label{fig:cnn}
\end{figure*}


\section{Experiments}
% The primary advantage of our algorithm is its generalizability to various modern neural architectures. 
In this section, we first compare our algorithm with symmetry teleport\citep{zhao2023improving}, which is the only available baseline providing a public codebase. We demonstrate the superiority of our algorithm in \textbf{\emph{performance, generalizability, and efficiency}} on MLPs.

Next, we evaluate the effectiveness of our method beyond MLPs by extending it to CNNs and transformers, utilizing \textbf{\emph{a wide range of benchmark datasets}}.
% , including MNIST, FashionMNIST, CIFAR-10, CIFAR-100, Tiny-ImageNet, multivariate time series datasets (electricity, traffic), and the Penn Treebank language corpus. 
Additionally, we evaluate our approach using \textbf{\emph{a variety of optimizers}}, such as the vanilla SGD, first-moment optimizer like SGD with momentum, second-moment optimizers like Adagrad and Adam.
% We showcase the efficiency of our algorithm compared to the state-of-the-art method, symmetry teleportation, across multiple teleportation hyperparameters. 
Furthermore, if any approximation of the level set is needed, we demonstrate the \textbf{\emph{capability of our approach to control the error in null space approximation}}, which subsequently improves the robustness of level set approximation during the teleportation.
% It is generally difficult to compare between different teleport methods with regard to how much acceleration incurred. This is because the effect of teleportation is highly sensitive to hyper-parameters, e.g. teleport learning rate and teleport schedule. What's more, different algorithms usually respond differently on same hyper-parameters due to different parameterization stragety, e.g., parameterization of g in symmetry teleport versus direct operation on model parameter in our method. Therefore, it is difficult to fairly compare between two methods. 

% Hence, we turn our attention to first showing the efficiency of our method over symmetry teleportation on MLP structure, which is what symmetry teleportation is designed for. Then, we show the generalizability and effectiveness of our methods to other important deep learning structures such as CNN and transformer. To this end, we conduct experiments using MLP, CNN, and Transformer, on an abundance of benchmark datasets including MNIST, FashionMNIST, CIFAR10, CIFAR100, TinyImagenet, and Penn Treebank language datasets, with various modern optimizers such as SGD, SGD with momentum, Adagrad, and Adam.


\subsection{Comparison with Symmetry Teleport on MLPs}
\textbf{Datasets.} To compare with symmetry teleport and demonstrate the effectiveness of our algorithm on MLPs, we conduct experiments using the MNIST digit image classification dataset and its clothing variant, FashionMNIST. The input images, with dimensions of $28 \times 28$ pixels, are flattened into vectors before being fed into the MLPs models.

 \textbf{Implementation Details.} We first adopt the same structure and hyperparameters used in~\citet{zhao2023improving} for both symmetry teleport and our algorithm. \textbf{\emph{Note that this setting favors symmetry teleport.}} This setup uses a small 3-layer MLPs with hidden dimensions [$16, 10$]. Consistent with~\citet{zhao2022symmetry}, we schedule teleportation for the first $5$ epochs of the primary training phase. For each teleportation in the schedule, we randomly sample $32$ batches of data and perform $8$ teleport updates per batch. The SVD threshold for our algorithm is set to 1, i.e., \textbf{\emph{the gradients are projected onto the exact input null space}}. We apply this setting with SGD and Adagrad optimizers.
 
 Next, we scale up the MLPs to [1024, 1024] while keeping all other hyperparameters identical between symmetry teleport and our algorithm. This larger setting is tested with Momentum and Adam optimizers. Continuing on this setting, we further demonstrate our algorithm's ability to accelerate optimization in terms of \textbf{\emph{time}} on FashionMNIST dataset across all four optimizers. See Appendix~\ref{sec: implem} for complete implementation details.

\textbf{Experiment Results.} 
In Figure~\ref{fig:mlp}, the first two graphs in the top row depict the training loss trajectories, comparing symmetry teleport with our algorithm in the small MLPs setting.  \textbf{\emph{Despite the setting being designed to favor symmetry teleport, our algorithm still achieves faster convergence and a lower final loss}}. The last two graphs in the top row illustrate the training loss trajectories after scaling up to larger MLPs. Interestingly, \textbf{\emph{symmetry teleport no longer accelerates optimization}} but instead teleports to an ill-conditioned geometric trajectory, slowing down the optimization process. \textbf{\emph{This highlights the superior generalizability of our algorithm}} to a broader class of functions, making it particularly advantageous in the latest era characterized by models of increasing size. See Appendix~\ref{sec:comparison_append} for complete test loss trajectories.

The bottom row in Figure~\ref{fig:mlp} presents the loss trajectories over \textbf{\emph{time}} using the FashionMNIST dataset. The plots look almost identical to the loss trajectory with respect to epoch (Figure~\ref{fig:mlp_append} in Appendix), indicating that the cost of teleportation is negligible compared to gradient descents.


 \begin{figure*}[htbp]
    \centering
    % \begin{subfigure}{0.24\textwidth} % Adjust the width as needed
    %     \centering
    %     \includegraphics[width=\textwidth]{images/pdf/MNIST_SGD_transformer_loss_vs_epoch.pdf}
    %     % \caption{Figure 1}
    % \end{subfigure}
   %  \begin{subfigure}{0.24\textwidth}
   %      \centering
   %      \includegraphics[width=\textwidth]{images/pdf/MNIST_momentum_transformer_loss_vs_epoch.pdf}
   %      % \caption{Figure 2}
   %  \end{subfigure}
   %  \begin{subfigure}{0.24\textwidth}
   %      \centering
   %      \includegraphics[width=\textwidth]{images/pdf/MNIST_Adagrad_transformer_loss_vs_epoch.pdf}
   %      % \caption{Figure 3}
   %  \end{subfigure}
   %  \begin{subfigure}{0.24\textwidth}
   %      \centering
   %      \includegraphics[width=\textwidth]{images/pdf/MNIST_Adam_transformer_loss_vs_epoch.pdf}
   %      % \caption{Figure 4}
   %  \end{subfigure}
   % \\
    \begin{subfigure}{0.24\textwidth} % Adjust the width as needed
        \centering
        \includegraphics[width=\textwidth]{images/pdf/electricity_SGD_transformer_loss_vs_epoch.pdf}
        % \caption{Figure 1}
    \end{subfigure}
    \begin{subfigure}{0.24\textwidth}
        \centering
        \includegraphics[width=\textwidth]{images/pdf/electricity_momentum_transformer_loss_vs_epoch.pdf}
        % \caption{Figure 2}
    \end{subfigure}
    \begin{subfigure}{0.24\textwidth}
        \centering
        \includegraphics[width=\textwidth]{images/pdf/electricity_Adagrad_transformer_loss_vs_epoch.pdf}
        % \caption{Figure 3}
    \end{subfigure}
    \begin{subfigure}{0.24\textwidth}
        \centering
        \includegraphics[width=\textwidth]{images/pdf/electricity_Adam_transformer_loss_vs_epoch.pdf}
        % \caption{Figure 4}
    \end{subfigure}
  \\
    \begin{subfigure}{0.24\textwidth} % Adjust the width as needed
        \centering
        \includegraphics[width=\textwidth]{images/pdf/traffic_SGD_transformer_loss_vs_epoch.pdf}
        % \caption{Figure 1}
    \end{subfigure}
    \begin{subfigure}{0.24\textwidth}
        \centering
        \includegraphics[width=\textwidth]{images/pdf/traffic_momentum_transformer_loss_vs_epoch.pdf}
        % \caption{Figure 2}
    \end{subfigure}
    \begin{subfigure}{0.24\textwidth}
        \centering
        \includegraphics[width=\textwidth]{images/pdf/traffic_Adagrad_transformer_loss_vs_epoch.pdf}
        % \caption{Figure 3}
    \end{subfigure}
    \begin{subfigure}{0.24\textwidth}
        \centering
        \includegraphics[width=\textwidth]{images/pdf/traffic_Adam_transformer_loss_vs_epoch.pdf}
        % \caption{Figure 4}
    \end{subfigure}
  \\
    \begin{subfigure}{0.24\textwidth} % Adjust the width as needed
        \centering
        \includegraphics[width=\textwidth]{images/pdf/PennTree_SGD_transformer_loss_vs_epoch.pdf}
        % \caption{Figure 1}
    \end{subfigure}
    \begin{subfigure}{0.24\textwidth}
        \centering
        \includegraphics[width=\textwidth]{images/pdf/PennTree_momentum_transformer_loss_vs_epoch.pdf}
        % \caption{Figure 2}
    \end{subfigure}
    \begin{subfigure}{0.24\textwidth}
        \centering
        \includegraphics[width=\textwidth]{images/pdf/PennTree_Adagrad_transformer_loss_vs_epoch.pdf}
        % \caption{Figure 3}
    \end{subfigure}
    \begin{subfigure}{0.24\textwidth}
        \centering
        \includegraphics[width=\textwidth]{images/pdf/PennTree_Adam_transformer_loss_vs_epoch.pdf}
        % \caption{Figure 4}
    \end{subfigure}
    \caption{Loss trajectories of training Transformers on sequential MNIST, electricity, traffic, and Penn Treebank datasets. Each experiment is repeated 3 times, with the average loss plotted and the standard deviation of loss represented as the shaded area.}
    \label{fig:att}
\end{figure*}

\textbf{Efficiency Improvement.}\label{sec:eff}
We demonstrate the efficiency of our algorithm compared to the symmetry teleport algorithm.
Recall that the time complexity of symmetry teleportation is $O(d^2nlbt)$, where $d$ is the feature dimension of layers, $n$ is the batch size, $l$ is the number of layers, $b$ is the number of batches, and $t$ is the number of teleport steps per batch. Note that the pseudo-inverse is calculated using SVD for Pytorch Library, thus sharing the same time complexity as SVD operation. However, in our method, only one SVD is needed for each batch of data, which reduces the bottleneck and brings the time complexity down to $O(d^2nlb)$, \textbf{\emph{enitrely removing dependence on $t$}}. Ideally, by leveraging our algorithm’s layer-independent property, \textbf{\emph{computations can be parallelized across all layers}}, further reducing the time complexity to $O(d^2nb)$. However, we leave such engineering optimizations for future work.


In practice, as demonstrated in Figure~\ref{fig:efficiency}, our algorithm exhibits linear scaling with respect to 
$t$, $l$, and $b$, while the runtime of the symmetry teleport increases at a significantly faster rate. Notably, for $d$ and $n$, our approach achieves \textbf{\emph {near-constant runtime}} in contrast to the linear-to-polynomial runtime of the symmetry teleport. Ideally, once the layer parallelization is fully implemented, we anticipate that constant runtime will also be achieved with an increasing number of layers, thereby enhancing overall performance.



\subsection{CNN Experiments}
\textbf{Datasets and Implementation.} We use the CIFAR-10, CIFAR-100 (results included in Appendix~\ref{sec:cnn_append}), and Tiny-Imagenet datasets to evaluate the effectiveness of our algorithm on CNNs. The image size for the Tiny-Imagenet dataset is kept the same as the full Imagenet dataset, i.e., $3\times224\times224$.
For the CIFAR datasets, we use a $3$-layer CNNs with channels [$3, 16, 32, 64$]. For the Tiny-Imagenet dataset, we utilize a residual network with channels [$3, 64, 64, 64, 128, 128, 128, 256, 256, 256$]. A classification head is connected after the final channel for both architectures. The teleportation scheduling and threshold $\tau$ remains the same as in the MLPs experiments. See Appendix~\ref{sec: implem} for complete implementation details.


\textbf{Experiment Results.} With teleportation, we observe in Figure~\ref{fig:cnn} a marked acceleration in optimization in the beginning of each training, coinciding with the application of teleportation. The test loss with teleportation tends to converge to the same value as the non-teleportation counterpart, while the training loss with teleportation continues to decrease at a faster rate even after the test loss has plateaued. This behavior is expected, as the teleportation objective is defined as the squared norm of the gradient, which prioritizes faster convergence on the training set rather than improving generalization. The teleportation framework is highly flexible, allowing the teleportation objective function to be adjusted to other reasonable choices, such as the curvature of the parameter landscape, which has been shown to enhance generalization \citep{zhao2023improving}.


\begin{figure*}[htbp]
    \centering
    \begin{subfigure}{0.4\textwidth} % Adjust the width as needed
        \centering
        \includegraphics[width=\textwidth]{images/pdf/captured_variance.png}
        \caption{Input variance captured by eigenvectors.}
        \label{fig:captured_variance}
    \end{subfigure}
    \begin{subfigure}{0.59\textwidth}
        \centering
        \includegraphics[width=\textwidth]{images/pdf/error_control.png}
        \caption{Effect of teleport step on increase of gradient norm and loss value.}
        \label{fig:error_control}
    \end{subfigure}
    \caption{A majority of the input variance is captured by a relatively small proportion of the input space. As we approximate a larger input null space, the gradient norm increases more rapidly during teleportation, while the loss remains constant when $\tau$ is greater than $0.99$.}
\end{figure*}




\subsection{Transformer Experiments}
\textbf{Datasets and Implementation.} We first consider the MNIST dataset as a sequential classification task, with a sequence length of $28\times28$ and a data dimension $1$. Results included in appendix~\ref{sec:smnist}.

Next, we evaluate on two publicly available multi-variate time series forecasting datasets: electricity and traffic. The electricity dataset consists of $321$ dimensions with a total sequence length of $26,304$. The sample sequence length is set to $7\times24$, and the regression target is the data point $24$ hours after the input sample. The traffic dataset consists of $862$ dimensions, with a total sequence length of $17,544$. See Appendix ~\ref{sec:data} for a detailed explanation.

We also evaluate on the Penn Treebank (PTB) language corpus. We use a decoder-only transformer structure and formulate the problem as a causal self-supervised learning task, where the label is the input shifted to the right by one.

For the sequential MNIST dataset, we use a small Transformer model with $2$ heads, each having a dimension of $64$, stacked across two layers. For the regression and language datasets, we use a transformer with 4 heads, each with a dimension of $64$, stacked across $4$ layers without pooling, followed by a linear output projection. See appendix ~\ref{sec: implem} for complete implementation details.

% For the sequential MNIST dataset, we use a small Transformer model with $2$ heads, each having a dimension of $64$, stacked across two layers. This is followed by an average pooling layer and a ten-way linear classification head, optimized using cross-entropy loss. For the electricity and traffic datasets, we use a transformer with 4 heads, each with a dimension of $64$, stacked across $4$ layers without pooling, followed by a linear regression head where the output dimension matches the input dimension. For the PTB dataset, we use the same Transformer architecture but replace the first linear layer with an embedding layer and set the output dimension to the vocabulary size, which is approximately $10,000$. See appendix ~\ref{sec: implem} for complete implementation details.\\


\textbf{Experiment Results.} In addition to the observations from previous experiments, in Figure~\ref{fig:att}, we notice that \textbf{\emph{teleportation remains effective across different problem settings, including regression problems and language modeling}}. Significant acceleration is observed in the regression datasets, particularly with the SGD and momentum optimizers, where the loss with teleportation converges within the first few epochs, while the non-teleportation counterpart takes more than $50$ epochs to converge on the traffic dataset. Furthermore, the acceleration with teleportation in language modeling is particularly notable during the initial phase of training, even though both approaches eventually converge to the same loss. These results highlight the potential of applying teleportation to the training of large language models.




\subsection{Error Control}


In addition to its efficiency, \textbf{\emph {our algorithm provides a distinct advantage in controlling the error associated with increased loss during teleportation}}. Figure~\ref{fig:captured_variance} records the information of the input space of the second layer in MLPs, CNNs, and Transformers (with the same architechtures used in experiments) across all datasets. Most variance of input is captured by the space of significant representation of a relatively small proportion of total dimensions, represented by the percentages of sorted eigenvectors in SVD. Consequently, even without approximating the input null space, sufficient dimensions are typically available in the null space to facilitate gradient projection and search. \textbf{\emph{This validates our choice of setting $\tau$ to be $1$ in most cases.}} Figure~\ref{fig:error_control} further confirms that when the threshold $\tau$ is set to $1$, meaning the exact null space is utilized, the gradient norm increases steadily during teleportation while the loss remains constant. Moreover, as $\tau$ decreases, the gradient is projected onto an approximated null space with a significantly larger number of dimensions, yet capturing only slightly more variance with minimal impact on the loss. A remarkable increase in the gradient norm ascending speed is observed when $\tau$ is set to $0.99$, with the loss still remaining constant. (Experiments in Figure~\ref{fig:error_control} are conducted using transformer on sMNIST.)


\section{Discussion and Conclusion}
In this paper, we propose a novel algorithm that generalizes the application of teleportation from MLPs to other modern architectures such as CNNs and transformers. The algorithm demonstrates improved computational efficiency and introduces explicit error control during the level set approximation, if such an approximation is employed.

% Gradient projection proves to be a powerful tool for modern AI, as most contemporary architectures rely on a linear modeling between inputs and weights. Consequently, our framework has the potential to be generalized to emerging time-series architectures such as Mamba and TTT.

Despite its promising performance, teleportation still faces challenges when applied broadly in the deep learning field. One of the major challenges is the selection of hyperparameters. Identifying a generalizable set of hyperparameters suitable for all architectures and datasets remains difficult. Developing a simple and effective hyperparameter selection strategy will significantly enhance the overall efficiency of teleportation.


\section*{Impact Statement}
This paper presents work whose goal is to advance the field of Machine Learning. There are many potential societal consequences of our work, none of which we feel must be specifically highlighted here.

% In the unusual situation where you want a paper to appear in the
% references without citing it in the main text, use \nocite
% \nocite{langley00}

% \bibliography{main}
\bibliographystyle{icml2025}
\documentclass{MITstyle}

%\usepackage[table]{xcolor}
\usepackage{chngcntr}
\usepackage{hyperref}
\usepackage{microtype}

\title{A Lightweight and Extensible Cell Segmentation and Classification Model for Whole Slide Images}

\author{Nikita Shvetsov~$^{1, }$\footnote{Correspondence e-mail: nikita.shvetsov@uit.no}, Thomas K. Kilvaer~$^{2, 3}$, Masoud Tafavvoghi~$^{4}$, Anders Sildnes~$^{1}$, \\ Kajsa Møllersen~$^{4}$, Lill-Tove Rasmussen Busund~$^{5, 6}$, Lars Ailo Bongo~$^{1}$ \\
%
\vspace{1em} % Space between authors and afilliations
%
\normalfont{\small $^{1}$Department of Computer Science, UiT The Arctic University of Norway}\\
\normalfont{\small $^{2}$Department of Oncology, University Hospital of North Norway}\\
\normalfont{\small $^{3}$Department of Clinical Medicine, UiT The Arctic University of Norway}\\
\normalfont{\small $^{4}$Department of Community Medicine, UiT The Arctic University of Norway}\\
\normalfont{\small $^{5}$Department of Medical Biology, UiT The Arctic University of Norway} \\
\normalfont{\small $^{6}$Department of Clinical Pathology, University Hospital of North Norway} %\vspace{2em}
}

\begin{document}
\maketitle

\section*{Abstract}

% \begin{abstract}
% Developing clinically useful cell-level analysis tools in digital pathology remains challenging due to limitations in dataset granularity, inconsistent annotations, computational demands of advanced models, and difficulties in integrating new technologies into clinical workflows. To address these challenges, we propose a multi-faceted solution that enhances data quality, model performance, and usability to create a lightweight and extensible cell segmentation and classification model.

% First, we update data labels by employing a cross-relabeling process that refines the labels of two existing datasets, PanNuke and MoNuSAC, to create a new unified dataset with enhanced granularity, encompassing seven distinct cell types. Second, we leverage the H-Optimus foundation model as a fixed encoder to improve feature representation for simultaneous cell segmentation and classification tasks. Third, to address the computational demands of foundation models, we employ knowledge distillation to reduce model size and complexity while maintaining comparable performance. Finally, to facilitate integration into clinical workflows, we integrate the distilled model into the QuPath software, a widely used open-source platform in digital pathology.

% Our results demonstrate improvements in cell segmentation and classification performance using the H‑Optimus-based model compared to a CNN-based model. Specifically, the average $R^2$ improved from 0.575 to 0.871, and the average $PQ$ score improved from 0.450 to 0.492, indicating better alignment with actual cell counts and enhanced segmentation and classification quality. Furthermore, the distilled student model maintains performance comparable to the larger foundation model while reducing the parameter count by a factor of 48.
% Overall, by reducing computational complexity and integrating it into existing workflows, the proposed approach may significantly impact diagnostic processes, reduce the workload of pathologists, and contribute to improved patient outcomes. Though our approach shows potential enhancements in efficiency and usability of cell segmentation and classification models in digital pathology, extensive validation is needed to deploy these models in clinical practice.
% \end{abstract}

%%% shortened abstract
\begin{abstract}
Developing clinically useful cell-level analysis tools in digital pathology remains challenging due to limitations in dataset granularity, inconsistent annotations, high computational demands, and difficulties integrating new technologies into workflows. To address these issues, we propose a solution that enhances data quality, model performance, and usability by creating a lightweight, extensible cell segmentation and classification model. 

First, we update data labels through cross-relabeling to refine annotations of PanNuke and MoNuSAC, producing a unified dataset with seven distinct cell types. Second, we leverage the H-Optimus foundation model as a fixed encoder to improve feature representation for simultaneous segmentation and classification tasks. Third, to address foundation models' computational demands, we distill knowledge to reduce model size and complexity while maintaining comparable performance. Finally, we integrate the distilled model into QuPath, a widely used open-source digital pathology platform. 

Results demonstrate improved segmentation and classification performance using the H-Optimus-based model compared to a CNN-based model. Specifically, average $R^2$ improved from 0.575 to 0.871, and average $PQ$ score improved from 0.450 to 0.492, indicating better alignment with actual cell counts and enhanced segmentation quality. The distilled model maintains comparable performance while reducing parameter count by a factor of 48. By reducing computational complexity and integrating into workflows, this approach may significantly impact diagnostics, reduce pathologist workload, and improve outcomes. Although the method shows promise, extensive validation is necessary prior to clinical deployment.
\end{abstract}
\clearpage

\section{Introduction}
In digital pathology, accurate segmentation and classification of cells are crucial for many diagnostic, prognostic, and predictive analyses \cite{Jaber_Beziaeva_etal._2019,Lin_Pan_etal._2022,Park_Ock_etal._2022,Shen_Choi_etal._2024}. Nowadays, developments in computational pathology offer multiple solutions \cite{H._Qu_P._Wu_etal._2020,Javed_Mahmood_etal._2020} to utilize cell-level datasets to train machine learning models that solve these problems. The quality and specificity of training datasets are critical for robust and accurate models. Adhering to the principle of "garbage in, garbage out", it is essential to ensure that these datasets are extensively and accurately labeled with distinct classes that reflect the diverse biological characteristics of different cell types. Unfortunately, the number of open-source datasets comprising such high-quality annotations is limited. Existing cell segmentation datasets \cite{Gamper_Koohbanani_etal._2019,Graham_Vu_etal._2019,Verma_Kumar_etal._2021} may offer extensive annotations for certain cell types while providing more general labels for others. For example, in PanNuke, which is one of the largest open-source datasets comprising labeled cells, various types of morphologically and functionally different inflammatory cells like macrophages and lymphocytes are clustered in a broad "inflammatory" class. Consequently, these classes are frequently omitted from analyses or aggregated into broader meta-classes \cite{Gamper_Koohbanani_etal._2020} and likely interfere with other cell classes included in the dataset. This and similar inconsistencies in annotation granularity limit the ability of machine learning models to learn the comprehensive and nuanced features necessary for accurate cell segmentation and classification. To address these challenges, methods for refining and standardizing dataset annotations are essential to enhance the quality of training data.

A complementary approach to mitigate the absence of high-quality training data is the use of foundation models. Foundation models as encoders are defined as large-scale, versatile networks pre-trained on vast, diverse datasets using self-supervised learning, contrasting with convolutional neural network (CNN) pre-trained encoders that rely on supervised learning with labeled data. In practice, foundation models leverage enormous amounts of weakly or unlabeled data from millions of whole slide images (WSIs) and employ self-attention mechanisms to capture long-range dependencies and global context \cite{Chen_Ding_etal._2024,Saillard_Jenatton_etal._2024,Vorontsov_Bozkurt_etal._2024,Xu_Usuyama_etal._2024}. As a consequence, foundation models are able to produce transferable feature representations across different cell types and tissue environments. The feature representations can be leveraged by decoder networks to produce segmentation masks and pixel-level classifications. Because foundation models have comprehensive feature representations, they can be effectively fine-tuned using much smaller amounts of cell-level data compared to the large datasets needed to train models from scratch. Furthermore, foundation models incorporate adversarial training elements or contrastive learning \cite{Chen_Ding_etal._2024,Xu_Usuyama_etal._2024}, enhancing their resilience and adaptability by exposing them to challenging and varied scenarios during training. This may result in more generalizable models, often making them well-suited for diverse and complex tasks in digital pathology.

Despite the inherent advantages of foundation models, their deployment for practical use faces its own obstacles. In particular, they require substantial computational power, financial investments and rigorous testing to ensure reliability and efficacy for a given task \cite{Akkus_Dangott_etal._2022,Dragomir_Cocuz_etal._2022,Go_2022,Jafri_Farooqui_etal._2024}. Moreover, while foundation models enhance feature representation and performance, they depend on the quality of available annotations for decoder fine-tuning and, like any other model, cannot resolve existing inconsistencies or ambiguities in data labels. Therefore, there remains a critical need for solutions that address both data quality and practical deployment considerations.
Further, integrating new technologies into existing clinical workflows often encounters resistance, as it necessitates adjustments to established diagnostic processes. So, there is a need to develop solutions that could be integrated into current practices, minimizing the burden on medical professionals to adopt new tools \cite{King_Williams_etal._2023}.

Existing solutions \cite{Goldsborough_Philps_etal._2024,Hörst_Rempe_etal._2024}, while addressing some aspects of these challenges, fall short in providing a comprehensive approach. To address the data quality and clinical deployment issues, we propose a multi-faceted solution that encompasses data refinement, model optimization, and integration with existing pathology tools (\hyperref[fig:fig1]{Figure 1}). The outcome is a lightweight cell segmentation and classification model that can be integrated into digital pathology workflows for practical clinical use.

\begin{figure}[h!]
    \centering
    \includegraphics[width=\textwidth, height=0.82\textheight, keepaspectratio]{images/Figure_1.pdf}
    \caption{Overview of the proposed solution, including 1) Data refinement using cross-relabeling, 2) Teacher model development and fine tuning, 3) Student model optimization with knowledge distillation and 4) Student model and QuPath integration}
    \label{fig:fig1}
\end{figure}
\clearpage

Our approach begins with preparing the data for the fine-tuning and training of the machine learning models. We create a refined dataset, acquired via cross-relabeling two cell-level datasets, enhancing annotation specificity and consistency of the labeled data. Subsequently, we create a cell segmentation and classification model based on the foundation model. We leverage the foundation model as a fixed encoder and fine-tune a decoder using the refined dataset to improve generalization across diverse tissue- and cell types.
To ensure that the model remains lightweight and deployable in a possibly resource-constrained environment, we employ knowledge distillation to approximate the functionality of the foundation model. Finally, to facilitate the practical application of our model in digital pathology workflows, we integrate it with the QuPath \cite{Bankhead_Loughrey_etal._2017} application. Each methodological component contributes to the overarching goal of enhancing model performance, generalizability, and usability in clinical settings.

The primary contributions of this paper are:
\begin{enumerate}
    \item \textit{Data labels refinement through cross-relabeling:}
    
    We propose a new method for refining labels of cell-level datasets through cross-relabeling. This method employs classification models to re-label broad and ambiguous instances, resulting in a more diverse dataset. Our evaluation demonstrates that these classification models achieve high accuracy on test subsets, indicating the reliability of the method for label refinement.

    \item \textit{Enhanced model performance via foundation models:}
    
    We employ a foundation model as a feature extractor for the cell segmentation and classification task. In comparison with training a CNN model from scratch, the foundation model backbone only needs fine-tuning, which significantly reduces training time, computational resources and data requirements. We show that using a foundation model encoder leads to better performance in cell segmentation and classification networks than using a CNN-based encoder. This improvement may enable the model to generalize more effectively across various tissue types and imaging methods.
    
    \item \textit{Model optimization through knowledge distillation:}
    
    We show that a smaller student model trained using knowledge distillation on the refined dataset obtained via our cross-relabeling approach from a foundation model achieves comparable performance in cell segmentation and quantification tasks. As a result, this model is more suitable for deployment in environments without high-performance computing resources.
    
    \item \textit{Integration with QuPath:}
    
    We integrate the distilled cell segmentation and classification model into QuPath, a widely used open-source digital pathology platform, to accelerate clinical adaptation by enabling pathologists to more easily incorporate advanced computational tools into their existing workflows.
\end{enumerate}

Through these methodological steps, we aim to bridge the gap between advanced machine learning techniques and practical clinical applications, making accurate and efficient digital pathology accessible in a broader range of healthcare settings.

\section{Refining Existing Datasets Using Cross-Relabeling}
To address the limitations of sparse and ambiguous labeling of cell-level datasets, we propose a generalizable cross-relabeling strategy that can be applied to any dataset containing broadly categorized or imprecisely labeled cell types. This approach involves training and subsequently leveraging classification models to refine broad categories into more specific or biologically relevant classes.
When applied to cell-level data, the methodology includes extracting individual cell images from the dataset patches, preprocessing these images to standardize the size and accommodate partial cells, and then training deep learning classifiers capable of distinguishing between the finer cell subtypes within the coarser categories. 
To illustrate our approach, we focus on the PanNuke \cite{Gamper_Koohbanani_etal._2020, Gamper_Koohbanani_etal._2019} and MoNuSAC \cite{Verma_Kumar_etal._2021} datasets that we have used to train models for cell quantification in our previous works \cite{Shvetsov_Grønnesby_etal._2022,Shvetsov_Sildnes_etal._2024}. We find that for better cell differentiation we have to introduce more granular labels. PanNuke includes a broad classification of "inflammatory" cells, encompassing lymphocytes, macrophages, and neutrophils. Each cell type differs significantly in structure, function, and clinical relevance. Conversely, MoNuSAC uses the label "epithelial" for a class that comprises both benign epithelial cells and malignant neoplastic cells. This practice makes it challenging to differentiate between benign and malignant epithelial cells in the dataset, which is a critical distinction when identifying tumor areas within tissue samples. To address these issues, we implement a cross-relabeling strategy as shown in \hyperref[fig:fig2]{Figure 2}. The key components are two classification models: one is trained on singular cell images from PanNuke data to classify the epithelial meta-class into epithelial and neoplastic classes. The other is trained on MoNuSAC to refine the inflammatory class into lymphocytes, neutrophils, and macrophages.

\begin{figure}[h!]
    \centering
    \includegraphics[width=\textwidth]{images/Figure_2.pdf}
    \caption{Refined dataset generation via cross relabeling}
    \label{fig:fig2}
\end{figure}

The refining approach consists of three consecutive steps. The first is the preprocessing step, in which we extract individual cells from both datasets (\hyperref[fig:fig3]{Figure 3}). The specifics of PanNuke and MoNuSAC patch preparation before cell preprocessing are provided in \hyperref[chap:S1]{Appendix S1}.

\begin{figure}[h!]
    \centering
    \includegraphics[width=\textwidth]{images/Figure_3.pdf}
    \caption{Cell instances preprocessing including (1) cell map extraction, (2) bounding box delineation, (3) adjusting cell boxes and (4) cropping and resizing of cell images}
    \label{fig:fig3}
\end{figure}

During preprocessing, we extract cell type maps from the ground truth label mask and calculate bounding boxes around each cell instance. To accommodate partial cells at patch borders, a common issue in cropped patch images, we employ mirror padding and extend the field of view of the cell label by 15 pixels to capture adjacent cells. We then crop and resize the identified regions to $64 \times 64$ pixels using bicubic interpolation.

The preprocessed PanNuke dataset comprises 68,031 neoplastic and 23,207 epithelial cell images, while MoNuSAC comprises  33,104 lymphocytes, 1,252 neutrophils, and 1,695 macrophages, which we subsequently use in training cell classification models and classifying the cell image data \hyperref[fig:S2]{Appendix Figure S2 (1)}. 

The next step is to train two distinct ResNet50-based classifiers tailored to address the specific labeling challenges inherent in each dataset. We use ResNet50 for classification models due to its proven effectiveness for image classification tasks in histopathology \cite{pan2022reviewmachinelearningapproaches}, and its compatibility with small images. For the PanNuke dataset, we design the classifier, trained on MoNuSAC data, to disaggregate the heterogeneous "inflammatory" cell category into distinct subtypes: lymphocytes, macrophages, and neutrophils. Similarly, for the MoNuSAC dataset, the classifier is trained on PanNuke data and distinguishes between benign and malignant epithelial cells within the overarching "epithelial" label. By applying these targeted classifiers to their respective datasets, we assign more specific labels to individual cell instances, thus enabling us to create a unified dataset.
To ensure a balanced representation of classes, we train both models on datasets that had been equalized to match the size of the least represented class. Thus, we obtain datasets comprising 23,207 samples per class for PanNuke and 1,252 samples per class for MoNuSAC data. Next, we partition both of them into training (70\%), validation (20\%), and testing (10\%) subsets. To mitigate the risk of overfitting, we use a single dropout layer with a rate of p=0.5 in both models and data augmentation using randomized color perturbations, rotation, and horizontal and vertical flipping. We employ AdamW optimizer and the cross-entropy loss function for the training criterion.

To evaluate the two trained models, we measure the classification accuracy on the respective test subsets. The accuracies on the test subset for both classifiers are presented in \hyperref[tab:1]{Table 1}. The PanNuke model achieves an average accuracy of 93.57\%, with higher accuracy for neoplastic cells (96.06\%) compared to epithelial cells (86.26\%). The confusion matrix in Figure A3.1 shows that the model predominantly distinguishes accurately between epithelial and neoplastic tissues, with a substantial number of correct classifications and relatively few misclassifications. The MoNuSAC model demonstrates an average accuracy of 98.92\%, excelling in classifying lymphocytes (99.67\%) and macrophages (94.12\%), with lower performance for neutrophils (85.71\%). The confusion matrix in Figure A3.2 shows that the model identifies lymphocytes and performs reasonably well with macrophages and neutrophils.

\begin{table}[h!]
\renewcommand{\arraystretch}{1.5}
  \centering
  \caption{Cell classification results for PanNuke and MoNuSAC trained models (CI 95\%).}
  \label{tab:1}
  \begin{tabular}{|l|c|c|}
   \hline
   %\rowcolor{gray!30}
    Accuracy               & PanNuke model              & MoNuSAC model              \\
    \hline
    Average      & 0.936 (0.931--0.941)         & 0.989 (0.986--0.993)        \\
    \hline
    Neoplastic   & 0.961 (0.956--0.965)         & -                          \\
    \hline
    Epithelial   & 0.863 (0.849--0.877)         & -                          \\
    \hline
    Lymphocytes  & -                          & 0.997 (0.995--0.999)        \\
    \hline
    Neutrophils  & -                          & 0.857 (0.796--0.918)        \\
    \hline
    Macrophages  & -                          & 0.941 (0.906--0.976)        \\
    \hline
  \end{tabular}
\end{table}

Finally, during the last step, we use the model trained on PanNuke data for epithelial cells in MoNuSAC and the model trained on MoNuSAC for the inflammatory cells class in PanNuke. Specifically, we use classifier models to relabel epithelial cells in MoNuSAC and inflammatory cells in PanNuke data. Then we combine cells with refined labels and the rest of the cells in both datasets to create a refined dataset (\hyperref[fig:S2]{Appendix Figure S2 (2)}). The process of relabeling cells and visualizing them on a patch is shown in \hyperref[fig:fig4]{Figure 4}. The cell counts in the refined dataset are provided in \hyperref[tab:S4]{Appendix Table S4}.

\begin{figure}[h!]
    \centering
    \includegraphics[width=\textwidth, height=0.42\textheight, keepaspectratio]{images/Figure_4.pdf}
    \caption{Cell relabeling procedure for epithelial and inflammatory cell classes}
    \label{fig:fig4}
\end{figure}

%\hfill

Relabeling and combining datasets have been explored in a prior study \cite{Parulekar_Kanwat_etal._2023}, where consecutive fine-tuning on multiple datasets was employed to account for hierarchical class label structures. While the method presented in \cite{Parulekar_Kanwat_etal._2023} is intuitive, it often lacks consistency and requires multiple fine-tuning runs, which can be cumbersome and time-consuming. 
In contrast, cross-relabeling simplifies this process by using specialized classification models tailored to each dataset's specific labeling challenges. This approach provides better transparency and produces a unified dataset encompassing seven distinct cell types across multiple tissue samples, enhancing data diversity for further model training or fine-tuning.

Despite these improvements, cross-relabeling does not entirely resolve issues related to poor labeling quality or the amount of labeled data. Specifically, our results show lower accuracies persist for underrepresented classes, such as macrophages, which may stem from a limited sample availability and intrinsic challenges in distinguishing these cells based solely on H\&E staining. Furthermore, while our method enhances label specificity, it relies on the initial quality of the broad labels; thus, any fundamental inaccuracies in the original annotations can propagate through the relabeling process. Addressing the overall problem of limited data labels may require integrating additional data sources or utilizing complementary immunohistochemical staining methods.
Although the reported performance metrics are obtained from evaluations on the native test sets of each dataset, it is important to note that the primary application of these classifiers is to perform cross-relabeling, where a model trained on one dataset (e.g., PanNuke) is applied to another (e.g., MoNuSAC) and vice versa. We acknowledge that a more systematic evaluation of cross-dataset generalization is needed and could be performed in future work.

Overall, the refined dataset produced by our approach can enhance the supervised training or fine-tuning of cell segmentation and classification models, especially those that utilize pre-trained foundation models to improve feature extraction robustness. In addition, these models can detect nuanced classes that enable researchers to conduct more detailed analyses of biological processes in computational pathology.

\section{Foundation models for robust cell segmentation and classification}

Accurate cell segmentation and classification in digital pathology are hindered by limited labeled data and the fact that conventional CNNs are unable to capture global contextual information due to their local receptive field constraints \cite{Gheflati_Rivaz_2022,Yang_Marcus_etal.}. Traditional approaches in cell quantification have predominantly relied on CNN encoders, such as ResNet50, given their proven effectiveness in semantic segmentation tasks \cite{Deshmane_2023,Graham_Vu_etal._2019,Mukasheva_Koishiyeva_etal._2024,Stringer_Wang_etal._2021}. However, approaches that include fine-tuning of pretrained CNNs, data augmentation, and stain normalization to partially increase data variability and address staining differences often fail to achieve the necessary generalization and robustness across diverse tissue types and staining conditions \cite{G._Wang_W._Li_etal._2018,Gao_Bagci_etal._2018,Karim_El_Khoury_Martin_Fockedey_etal._2021}.

To overcome these challenges, we leverage an encoder-decoder network that uses a foundation model as the encoder and a CNN upsampling decoder (\hyperref[fig:fig5]{Figure 5}) for simultaneous cell segmentation and classification in 2D patches extracted from WSIs. Foundation models with transformer-based architectures are viable alternatives to CNN-based encoders \cite{Shamshad_Khan_etal._2023,Sourget_2023}. They enable the creation of more advanced architectures that can decode or transform learned features more effectively \cite{Chen_Duan_etal._2023,Cheng_Misra_etal._2022,Xie_Wang_etal._2021}.

\begin{figure}[h!]
    \centering
    \includegraphics[width=\textwidth]{images/Figure_5.pdf}
    \caption{UNETR-like model with foundational model as backbone}
    \label{fig:fig5}
\end{figure}

By utilizing a transformer-based encoder, we incorporate global contextual information into the feature extraction process, which is a key advantage of such architectures \cite{Chen_Lu_etal._2021}. This foundation model integration facilitates accurate pixel-wise segmentation and classification without the need for extensive encoder training, thereby potentially improving generalization across varied cellular structures and tissue types.
In our implementation, we employ a modified UNETR \cite{Hatamizadeh_Tang_etal._2021} architecture that combines a vision transformer (ViT) \cite{Dosovitskiy_Beyer_etal._2021} encoder with a CNN-based decoder. The encoder utilizes the pretrained H-Optimus foundation model, which contains 1.1 billion parameters and is trained on over 500,000 H\&E stained WSIs \cite{Saillard_Jenatton_etal._2024}. We extract outputs from four evenly spaced transformer blocks $Z_i$, where $i \in [1, 14, 26, 38]$, to serve as residual connections for the CNN decoder. We select these blocks based on our observation that features from non-adjacent levels of the encoder lead to better overall performance on the test subset.

The CNN decoder upsamples the feature representations, acquired from the transformer blocks, to generate an intermediate vector that is handled by two task-specific layers that generate cell segmentation and classification masks. The first task-specific layer is the ‘Cellpose head’,  which is used to delineate cell instances. The layer generates horizontal and vertical gradient maps to form vector fields that are refined through gradient tracking in a post-processing step using the Cellpose algorithm \cite{Stringer_Wang_etal._2021}, known for its efficacy in cell segmentation tasks and generalizability across multiple domains \cite{Pachitariu_Stringer_2022,Stringer_Pachitariu_2024}. The second task-specific layer is the "Cell type head", which assigns labels to individual pixels. In the post-processing step, we determine the output classification label of each segmented cell instance by majority voting over the labeled pixels that comprise the cell in the segmentation map.

To evaluate model performance and measure the impact of adding a foundation model as backbone, we compare it to a ResNet50-based model. ResNet50 is a widely used solution for encoders in segmentation architectures in the medical domain \cite{Deshmane_2023,Graham_Vu_etal._2019,Mukasheva_Koishiyeva_etal._2024,Stringer_Wang_etal._2021}. For the H-Optimus-based model, we utilize frozen weights for the encoder and only fine-tune the decoder to take advantage of the extensive pre-training of the foundation model. For the ResNet50-based model we start with ImageNet \cite{Deng_Dong_etal.} weights and train both encoder and decoder parts. Hyperparameters for the training step are set to be identical, where possible, for comparable evaluation. 
For this evaluation, we deliberately use the PanNuke dataset to provide a standardized and controlled comparison between the H‑Optimus and ResNet50-based models (\hyperref[fig:S2]{Appendix Figure S2 (3)}). Specifically, we use two of the default PanNuke dataset splits (66\%) for training and validation, and reserve the third split (33\%) for testing.

To address the challenge of cell class imbalance in the PanNuke dataset, which is a common characteristic in most cell-level H\&E patch datasets, both models’ training processes employ a weighted loss function comprising cross-entropy and focal loss \cite{Lin_Goyal_etal._2018}. The focal loss component is adjusted with coefficients derived from each cell class' instance frequency, emphasizing learning from underrepresented classes and enhancing the model's sensitivity to rare but significant cellular patterns. The cross-entropy loss is augmented with spectral decoupling regularization \cite{Pezeshki_Kaba_etal._2021,Pohjonen_Stürenberg_etal._2022} and spatially varying label smoothing \cite{Islam_Glocker_2021}, which potentially stabilizes training and improves generalization in case of complex tissue morphologies. For optimization, we employ the \textit{AdamW} \cite{Loshchilov_Hutter_2019} to counter unbalanced class scenarios, with cosine annealing learning rate scheduler.

We utilize the scikit-learn library \cite{Van_der_Walt_Schönberger_etal._2014} and HoVer-Net \cite{Graham_Vu_etal._2019} implementations of $R^2$ (the coefficient of determination) and $PQ$ (panoptic quality) to evaluate our experiments. Complete mathematical formulations and detailed explanations of these metrics are provided in \hyperref[chap:S5]{Appendix S5}. To compute confidence intervals, we use nonparametric bootstrapping, where after calculating the metric on the full sample, we generated 1000 bootstrap replicates by resampling with replacement and then determined the 95\% confidence intervals as the 2.5th and 97.5th percentiles of the resulting empirical distribution.

%\hfill

The model comparisons are summarized in \hyperref[tab:2]{Table 2}. The H‑Optimus-based model achieves higher $R^2$ across all cell classes compared to the ResNet50-based model, which means that its predictions are more closely aligned with the PanNuke cell counts, indicating a stronger correlation with the observed data. Notably, the improvement of $R^2_{dead}$ may be an indicator of better global contextual representations provided by the foundation model backbone. In terms of segmentation and classification quality combined, measured by the PQ score, the H‑Optimus-based model demonstrates notable improvements across most cell classes. Overall, the average $R^2$ improved from 0.575 to 0.871, while the average $PQ$ score improved from 0.450 to 0.492, demonstrating better performance of the H-Optimus-based model.

\begin{table}[h!]
\renewcommand{\arraystretch}{1.5}
  \centering
  \caption{Cell quantification metrics for baseline and proposed models (CI 95\%).}
  \label{tab:2}
  \begin{tabular}{|l|c|c|}
    \hline
    %\rowcolor{gray!30}
    Metric             & Resnet50-based            & H-optimus-based              \\
    \hline
    $R^2_{neoplastic}$    & 0.681 (0.576--0.769)       & \textbf{0.941 (0.917--0.960)} \\
    \hline
    $R^2_{inflammatory}$  & 0.863 (0.778--0.903)       & \textbf{0.949 (0.918--0.966)} \\
    \hline
    $R^2_{connective}$    & 0.600 (0.488--0.698)       & 0.609 (0.436--0.772)          \\
    \hline
    $R^2_{dead}$          & 0.097 (-11.389--0.669)     & 0.925 (0.404--0.982)          \\
    \hline
    $R^2_{epithelial}$    & 0.635 (0.490--0.747)       & \textbf{0.930 (0.886--0.964)} \\
    \hline
    $PQ_{neoplastic}$       & 0.517 (0.499--0.535)       & \textbf{0.589 (0.575--0.604)} \\
    \hline
    $PQ_{inflammatory}$     & 0.455 (0.429--0.482)       & \textbf{0.528 (0.507--0.549)} \\
    \hline
    $PQ_{connective}$       & 0.416 (0.400--0.431)       & \textbf{0.451 (0.436--0.465)} \\
    \hline
    $PQ_{dead}$             & 0.374 (0.342--0.408)       & 0.292 (0.209--0.365)          \\
    \hline
    $PQ_{epithelial}$       & 0.488 (0.460--0.519)       & \textbf{0.599 (0.579--0.618)} \\
    \hline
  \end{tabular}
\end{table}

Our results  show that integrating the H‑Optimus foundation model within the UNETR architecture enhances the model's ability to segment and classify cells across diverse tissues from PanNuke data. The pretrained transformer encoder provides robust feature representations, resulting in higher average $R^2$ and $PQ$ scores compared to the CNN-based model. This leads to more reliable cell quantification and more accurate downstream analysis. Additionally, the streamlined fine-tuning process reduces computational overhead and training time, making the model more adaptable for new data.

Despite these advancements, the foundation model-based approach does not fully resolve all challenges related to cell segmentation and classification. We observe lower metric scores for underrepresented classes in the training data. Furthermore, foundation models typically encompass billions of parameters, resulting in substantial computational and memory requirements. It therefore poses challenges for deployment in resource-constrained environments, limiting their practical applicability in certain clinical settings.

\section{Model optimization via Knowledge Distillation}

To address the limitations posed by the extensive size of foundation models, we implement knowledge distillation — a model compression technique that leverages the teacher-student paradigm \cite{Hinton_Vinyals_etal._2015}. By training a smaller, more efficient student model to replicate the output of a larger, pre-trained teacher model, we retain performance while significantly reducing the model's complexity and resource requirements (\hyperref[fig:fig6]{Figure 6}).

\begin{figure}[h!]
    \centering
    \includegraphics[width=\textwidth, height=0.45\textheight, keepaspectratio]{images/Figure_6.pdf}
    \caption{Knowledge distillation framework for training a student model using a pre-trained teacher}
    \label{fig:fig6}
\end{figure}

We employ knowledge distillation to compress the H‑Optimus-based teacher model into a more efficient student model. The teacher model is the modified UNETR architecture with the H‑Optimus foundation model described in the previous chapter. The student model is based on a UNet architecture augmented with residual connections and incorporates a smaller ViT encoder with 9 million parameters \cite{Steiner_Kolesnikov_etal._2022,Wightman_2019}. 

First, we fine-tune the teacher model using the refined dataset from the cross-relabeling procedure (Section 2). Initially we train the decoder of the teacher model while keeping the encoder weights frozen. We split the refined dataset into train (70\%), validation (20\%) and test (10\%) subsets (\hyperref[fig:S2]{Appendix Figure S2 (4)}). During fine-tuning, we use the train and validation subsets, while leaving the test subset for model evaluation. We set the training procedure and model hyperparameters to be identical to those that were used to demonstrate the utility of foundation models for the simultaneous cell segmentation and classification task.

Next, we perform knowledge distillation from teacher to student using the refined dataset used to fine-tune the teacher model. The student model is trained to replicate the teacher model's outputs. We utilize a specialized loss function that aligns the student's predicted probability distribution with the teacher's, incorporating the teacher's class probability distribution derived from the output. Following the methodology of Hinton et al. \cite{Hinton_Vinyals_etal._2015}, we experiment with various hyperparameter settings for the temperature ($T$) and the balancing coefficients ($\alpha$ and $\beta$) in the loss function. We vary $T$ from 1 to 20 and adjust $\alpha$ and $\beta$ to balance the distillation and student losses. Through iterative tuning and evaluation, we identify that setting $T=14$, $\alpha=0.3$, and $\beta=0.7$ yields a configuration that converges and closely approximates the teacher model's performance during training.

Finally, we assess the performance of both models using the $R^2$ and $PQ$ (defined in \hyperref[chap:S5]{Appendix S5}) on the test set of the refined dataset (\hyperref[tab:3]{Table 3}). We observe that the 95\% confidence intervals overlap for most cell types, so we cannot claim statistically significant performance differences between the teacher and student models. One exception appears in the neoplastic class. The teacher model produces an $R^2$ of 0.919, while the student model shows an $R^2$ of 0.852. In addition, the student model achieves higher $PQ$ values for the neoplastic and connective classes, though the confidence intervals show overlap.

\begin{table}[h!]
\renewcommand{\arraystretch}{1.5}
  \centering
  \caption{Cell quantification metrics for teacher and distilled student models (CI 95\%).}
  \label{tab:3}
  \begin{tabular}{|l|c|c|}
    \hline
    %\rowcolor{gray!30}
    Metric & Teacher & Student \\
    \hline
    $R^2_{neoplastic}$    & \textbf{0.919} (0.898--0.939) & 0.852 (0.800--0.891) \\
    \hline
    $R^2_{lymphocyte}$    & 0.969 (0.956--0.977)         & 0.969 (0.956--0.978) \\
    \hline
    $R^2_{connective}$    & 0.694 (0.548--0.809)         & 0.618 (0.469--0.741) \\
    \hline
    $R^2_{dead}$          & 0.755 (0.400--0.908)         & 0.424 (0.100--0.731) \\
    \hline
    $R^2_{epithelial}$    & 0.922 (0.870--0.958)         & 0.843 (0.738--0.917) \\
    \hline
    $R^2_{macrophage}$    & 0.384 (-0.369--0.724)        & 0.704 (0.352--0.859) \\
    \hline
    $R^2_{neutrofil}$     & 0.854 (0.578--0.929)         & 0.833 (0.502--0.925) \\
    \hline
    $PQ_{neoplastic}$       & 0.581 (0.569--0.593)         & 0.601 (0.588--0.613) \\
    \hline
    $PQ_{lymphocyte}$       & 0.536 (0.520--0.553)         & 0.563 (0.544--0.579) \\
    \hline
    $PQ_{connective}$       & 0.436 (0.421--0.451)         & 0.457 (0.441--0.474) \\
    \hline
    $PQ_{dead}$             & 0.272 (0.235--0.315)         & 0.279 (0.201--0.369) \\
    \hline
    $PQ_{epithelial}$       & 0.522 (0.500--0.545)         & 0.530 (0.506--0.555) \\
    \hline
    $PQ_{macrophage}$       & 0.524 (0.459--0.588)         & 0.474 (0.405--0.543) \\
    \hline
    $PQ_{neutrofil}$        & 0.541 (0.490--0.592)         & 0.565 (0.522--0.607) \\
    \hline
  \end{tabular}
\end{table}


We further decompose the $PQ$ metric into its $SQ$ and $DQ$ components (\hyperref[tab:S6]{Appendix Table S6}). Both models produce nearly identical $SQ$ values, which indicates that they predict instance boundaries with similar precision. Although the student model shows some improvement in $DQ$ scores for certain classes, the confidence intervals overlap and do not confirm a statistically significant difference.

We observe that the student and teacher models yield comparable detection performance despite the student model using a much smaller and simpler architecture. A model with fewer parameters reduces the risk of overfitting when training data are scarce relative to the model’s complexity \cite{Farias_Ludermir_etal._2022}. The knowledge distillation process also encourages the student model to focus on the most generalizable detection features learned from the teacher. These factors enable the student model to achieve similar detection performance across different cell types.

Additionally, considering the model sizes reported in \hyperref[tab:4]{Table 4}, the distilled model achieves a significant reduction compared to the teacher model, with a 48-fold decrease in parameter count and a 5.5-fold reduction in on-disk size. In inference mode, the teacher model requires 16 GB of VRAM for a batch size of 32, while the distilled model only needs 3 GB of VRAM for the same batch size. These reductions make the distilled model significantly more practical for fine-tuning and deployment in resource-constrained environments.

\begin{table}[h!]
\renewcommand{\arraystretch}{1.5}
  \centering
  \caption{Parameter counts and size of teacher and distilled model}
  \label{tab:4}
  \adjustbox{max width=\textwidth}{%
  \begin{tabular}{|l|c|c|c|}
    \hline
    %\rowcolor{gray!30}
    Metric & H-optimus-based (Teacher) & mobileViT-based (Student) & Magnitude of difference \\
    \hline
    Parameters count       & 1,158,917,906   & \textbf{24,093,393}   & \textbf{48x}  \\
    \hline
    Estimated Total Size (MB) & 87,912       & \textbf{15,935}    & \textbf{5.5x} \\
    \hline
  \end{tabular}%
}
\end{table}

%\hfill

With recent advancements in complex network architectures and the use of pretrained encoders to achieve state-of-the-art performance \cite{Baumann_Dislich_etal._2024,Hörst_Rempe_etal._2024} in cell segmentation and classification tasks, model size, computational complexity, and processing times have increased. This limits the scalability and accessibility of these models. As we demonstrate, this may be mitigated using knowledge distillation. Studies in the field of natural language processing have demonstrated the efficacy of knowledge distillation in retaining the capabilities of the teacher model while achieving significant reductions in size and complexity \cite{Huangpu_Gao_2024,Sun_Yu_etal.}. 

We demonstrate the feasibility of knowledge distillation in digital pathology, specifically for cell segmentation and classification tasks. Moreover, we achieve this performance while also significantly reducing the parameter count. In addressing the challenge of knowledge transfer, we found that distillation from a transformer-based model to a smaller transformer is more straightforward than attempting to map transformer features to CNN blocks. In our experiments, using a CNN-based network as a student results in worse cell quantification performance due to the structural constraints of CNN feature space dimensions. 

Although our primary approach relies on a transformer-based student model that performs well, it can be further optimized to incorporate advantages from CNN architectures. For example, employing alternative techniques such as using ViT adapters \cite{Chen_Duan_etal._2023} or $1 \times 1$ convolutions to adjust feature map sizes may be beneficial for harnessing CNN advantages like enhanced local feature extraction. Moreover, if additional performance improvements are desired, the process can be further enhanced by applying supplementary knowledge distillation techniques, such as self-distillation \cite{Zhang_Song_etal._2019} or online distillation \cite{Houyon_Cioppa_etal._2023}.

Despite these promising results, further validation on independent datasets is necessary to fully understand the model's limitations. Underrepresented classes may pose challenges when addressing complex cases. Pathologists need to validate these models to adopt them in clinical settings. While the distilled models are smaller and more deployable, a technological gap persists because pathologists traditionally rely on established methods for inspecting WSIs and diagnosing diseases. Addressing the complexities involved in deploying models for inference and supporting pathologists in adopting new tools is essential for integrating these models into clinical workflows.

\section{Model integration with QuPath}
Digital pathology tools with graphical user interfaces are essential for visualizing and analyzing WSIs. To make our student model useful in clinical pathology workflows, it needs to be integrated into a tool that enables inspecting regions, creating annotations, and providing quantitative analyses of biomarkers. Therefore, we integrate the trained student model from the previous chapter into the QuPath open‑source platform \cite{Bankhead_Loughrey_etal._2017}. QuPath provides the required annotation, visualization, and analysis tools to interpret complex histological data, including workflows for cell segmentation, classification, and quantification (\hyperref[fig:fig7]{Figure 7}). 

\begin{figure}[h!]
    \centering
    \includegraphics[width=\textwidth]{images/Figure_7.pdf}
    \caption{Visualization of model-generated cell quantification annotations (left) and the corresponding unannotated slide (right) in QuPath}
    \label{fig:fig7}
\end{figure}

To identify the regions in a WSI critical for prognosticating tumor development, such as specific tumor areas or border regions without overlapping healthy tissue, the pathologist uses QuPath to outline these regions. Then, the pathologist initiates a cell segmentation and classification script through the QuPath interface for the selected regions. The resulting annotations and quantified cell information are then directly overlaid onto the WSI in the QuPath interface. Additional design and implementation details are in \hyperref[chap:S7]{Appendix S7}. 

Two common approaches for integrating deep learning models into QuPath are Java‑based native QuPath extensions \cite{Goldsborough_Philps_etal._2024} and the execution of RESTful API requests to a model server coupled with handling the response via an extension, as demonstrated in the application of cell segmentation models applied to immunofluorescence images \cite{Sugawara_2023}. While the community is actively working on these integration strategies, there is currently no universal solution that fully addresses all integration and performance requirements.

Extensions may offer better integration with QuPath, allowing slightly improved performance and more widespread usage of the built-in QuPath models, but they lack the flexibility to customize models and modify their behavior. For example, the newest version of QuPath includes models such as StarDist \cite{Weigert_Schmidt} and InstanSeg \cite{Goldsborough_Philps_etal._2024} that can perform cell segmentation. Both models pose limitations when applied to simultaneous cell segmentation and classification. StarDist performs well only on convex, round shapes by design, whereas some neoplastic, inflammatory, and connective cells exhibit complex and non-convex shapes. InstanSeg provides only semantic segmentation without assigning classes to the segmented cells.

%\hfill

In contrast, our approach offers an alternative integration strategy. It utilizes the paquo library to directly interact with QuPath’s internal application programming interface from within Python. This enables data exchange and processing without the need for intermediate conversion steps and provides greater control over model customization, retraining, and the incorporation of custom processing steps.

The integration of our custom model with QuPath underscores its potential to significantly enhance the diagnostic process by reducing the time burden on pathologists and enabling them to focus on more complex interpretative tasks using familiar software. Leveraging a tool that is already well-established among pathologists increases the likelihood of its adoption into daily clinical workflows. The quantitative data generated through the automated workflow is critical for both clinical decision-making and research, facilitating more accurate biomarker analysis, enabling robust statistical evaluations, and supporting hypothesis generation and testing. Additionally, by streamlining cell segmentation and classification, the tool enhances the scalability and reproducibility of pathological assessments, ultimately contributing to improved diagnostic accuracy and patient outcomes.

\section{Conclusion and future work}

In this study, we address critical challenges in digital pathology and tackle the usability and deployment issues of the developed models in standard computing environments without the need for high-performance computing systems. Our multi-faceted approach encompasses data refinement through cross-relabeling, leveraging foundation models for robust cell segmentation and classification, optimizing model performance via knowledge distillation, and integrating the optimized model into the QuPath software for practical application. This approach is used to construct a capable, versatile, and adjustable model for cell segmentation and classification, with enhanced performance and usability.

\begin{sloppypar}
While our approach shows potential in the field of computational pathology, certain limitations persist. 
For example, our implementation currently exhibits lower performance in detecting macrophages. 
This serves as an instance of the broader challenge of accurately identifying complex cell types. In order to address this issue, extending our approach to incorporate additional data sources, exploring alternative modeling approaches, and integrating other imaging modalities such as immunohistochemical staining may help improve detection accuracy. Moreover, although the distilled model reduces computational demands, integrating advanced deep learning models into clinical practice requires addressing technological gaps and potential resistance to adopting new tools within established diagnostic processes.
\end{sloppypar}

Future work could focus on several key areas to refine the proposed approach and facilitate its adoption in clinical environments. Enhancing the cell-relabeling process with additional datasets \cite{Graham_Jahanifar_etal._2021} could improve the representation of underrepresented cell types and enhance overall model performance. Also, incorporating additional data sources, such as multi-modal imaging or complementary staining methods, may address limitations related to cell type differentiation and class imbalance. Exploring other foundation models \cite{Vorontsov_Bozkurt_etal._2024,Zimmermann_Vorontsov_etal._2024} or introducing additional modalities \cite{Ding_Wagner_etal._2024,Vaidya_Zhang_etal._2025} may provide alternative architectures better suited to specific tasks or offer improved efficiency. Implementing more complex knowledge distillation techniques \cite{Houyon_Cioppa_etal._2023,Zhang_Song_etal._2019} could further optimize the model's performance and adaptability. Additionally, deeper integration with QuPath or other digital pathology software could provide pathologists more control over cell quantification analysis directly within the QuPath interface, thereby increasing accessibility and usability. Such enhancements would not only refine model performance but also ensure greater adaptability and scalability within various clinical environments. Finally, extensive validation of the model by pathologists and benchmarking against independent datasets are essential steps toward establishing the model's reliability and fostering confidence in its clinical utility.

\section*{Acknowledgments} 
This work was funded in part by the Research Council of Norway grant no. 309439 SFI Visual Intelligence, and the North Norwegian Health Authority grant no. HNF1521-20.

\bibliographystyle{IEEEtran}
\begin{sloppypar}
\begin{thebibliography}{99}

\bibitem{chaplot2020neural} Chaplot, Devendra Singh, et al. "Neural topological slam for visual navigation." Proceedings of the IEEE/CVF conference on computer vision and pattern recognition. 2020.

\bibitem{maksymets2021thda} Maksymets, Oleksandr, et al. "Thda: Treasure hunt data augmentation for semantic navigation." Proceedings of the IEEE/CVF International Conference on Computer Vision. 2021.

\bibitem{mezghan2022memory} Mezghan, Lina, et al. "Memory-augmented reinforcement learning for image-goal navigation." 2022 IEEE/RSJ International Conference on Intelligent Robots and Systems (IROS). IEEE, 2022.

\bibitem{al2022zero} Al-Halah, Ziad, Santhosh Kumar Ramakrishnan, and Kristen Grauman. "Zero experience required: Plug \& play modular transfer learning for semantic visual navigation." Proceedings of the IEEE/CVF Conference on Computer Vision and Pattern Recognition. 2022.

\bibitem{ye2021auxiliary} Ye, Joel, et al. "Auxiliary tasks and exploration enable objectgoal navigation." Proceedings of the IEEE/CVF international conference on computer vision. 2021.

\bibitem{chaplot2020object} Chaplot, Devendra Singh, et al. "Object goal navigation using goal-oriented semantic exploration." Advances in Neural Information Processing Systems 33 (2020)

\bibitem{ramakrishnan2022poni} Ramakrishnan, Santhosh Kumar, et al. "Poni: Potential functions for objectgoal navigation with interaction-free learning." Proceedings of the IEEE/CVF Conference on Computer Vision and Pattern Recognition. 2022.

\bibitem{ramrakhya2022habitat} Ramrakhya, Ram, et al. "Habitat-web: Learning embodied object-search strategies from human demonstrations at scale." Proceedings of the IEEE/CVF Conference on Computer Vision and Pattern Recognition. 2022.

\bibitem{mousavian2019visual} Mousavian, Arsalan, et al. "Visual representations for semantic target driven navigation." 2019 International Conference on Robotics and Automation (ICRA). IEEE, 2019.

\bibitem{dhariwal2021diffusion} Dhariwal, Prafulla, and Alexander Nichol. "Diffusion models beat gans on image synthesis." Advances in neural information processing systems 34 (2021)

\bibitem{ho2022classifier} Ho, Jonathan, and Tim Salimans. "Classifier-free diffusion guidance." arXiv preprint arXiv:2207.12598 (2022).

\bibitem{nichol2021glide} Nichol, Alex, et al. "Glide: Towards photorealistic image generation and editing with text-guided diffusion models." arXiv preprint arXiv:2112.10741 (2021)

\bibitem{brooks2023instructpix2pix} Brooks, Tim, Aleksander Holynski, and Alexei A. Efros. "Instructpix2pix: Learning to follow image editing instructions." Proceedings of the IEEE/CVF Conference on Computer Vision and Pattern Recognition. 2023.

\bibitem{fu2023guiding} Fu, Tsu-Jui, et al. "Guiding instruction-based image editing via multimodal large language models." arXiv preprint arXiv:2309.17102 (2023).

\bibitem{geng2024instructdiffusion} Geng, Zigang, et al. "Instructdiffusion: A generalist modeling interface for vision tasks." Proceedings of the IEEE/CVF Conference on Computer Vision and Pattern Recognition. 2024.

\bibitem{zhou2024minedreamer} Zhou, Enshen, et al. "Minedreamer: Learning to follow instructions via chain-of-imagination for simulated-world control." arXiv preprint arXiv:2403.12037 (2024).

\bibitem{zhou2023esc} Zhou, Kaiwen, et al. "Esc: Exploration with soft commonsense constraints for zero-shot object navigation." International Conference on Machine Learning. PMLR, 2023.

\bibitem{yu2023l3mvn} Yu, Bangguo, Hamidreza Kasaei, and Ming Cao. "L3mvn: Leveraging large language models for visual target navigation." 2023 IEEE/RSJ International Conference on Intelligent Robots and Systems (IROS). IEEE, 2023.

\bibitem{gadre2023cows} Gadre, Samir Yitzhak, et al. "Cows on pasture: Baselines and benchmarks for language-driven zero-shot object navigation." Proceedings of the IEEE/CVF Conference on Computer Vision and Pattern Recognition. 2023.

\bibitem{shah2023navigation} Shah, Dhruv, et al. "Navigation with large language models: Semantic guesswork as a heuristic for planning." Conference on Robot Learning. PMLR, 2023.

\bibitem{cai2024bridging} Cai, Wenzhe, et al. "Bridging zero-shot object navigation and foundation models through pixel-guided navigation skill." 2024 IEEE International Conference on Robotics and Automation (ICRA). IEEE, 2024.

\bibitem{yu2023co} Yu, Bangguo, Hamidreza Kasaei, and Ming Cao. "Co-NavGPT: Multi-robot cooperative visual semantic navigation using large language models." arXiv preprint arXiv:2310.07937 (2023).

\bibitem{wu2024voronav} Wu, Pengying, et al. "Voronav: Voronoi-based zero-shot object navigation with large language model." arXiv preprint arXiv:2401.02695 (2024).

\bibitem{qin2023mp5} Qin, Yiran, et al. "Mp5: A multi-modal open-ended embodied system in minecraft via active perception." arXiv preprint arXiv:2312.07472 (2023).

\bibitem{du2024learning} Du, Yilun, et al. "Learning universal policies via text-guided video generation." Advances in Neural Information Processing Systems 36 (2024).

\bibitem{ajay2024compositional} Ajay, Anurag, et al. "Compositional foundation models for hierarchical planning." Advances in Neural Information Processing Systems 36 (2024).

\bibitem{liang2024skilldiffuser} Liang, Zhixuan, et al. "Skilldiffuser: Interpretable hierarchical planning via skill abstractions in diffusion-based task execution." Proceedings of the IEEE/CVF Conference on Computer Vision and Pattern Recognition. 2024.

\bibitem{heusel2017gans} Heusel, Martin, et al. "Gans trained by a two time-scale update rule converge to a local nash equilibrium." Advances in neural information processing systems 30 (2017).

\bibitem{zhang2018unreasonable} Zhang, Richard, et al. "The unreasonable effectiveness of deep features as a perceptual metric." Proceedings of the IEEE conference on computer vision and pattern recognition. 2018.

\bibitem{brown2020language} Brown, Tom B. "Language models are few-shot learners." arXiv preprint arXiv:2005.14165 (2020).

\bibitem{podell2023sdxl} Podell, Dustin, et al. "Sdxl: Improving latent diffusion models for high-resolution image synthesis." arXiv preprint arXiv:2307.01952 (2023).

\bibitem{brohan2022rt} Brohan, Anthony, et al. "Rt-1: Robotics transformer for real-world control at scale." arXiv preprint arXiv:2212.06817 (2022).

\bibitem{brohan2023rt} Brohan, Anthony, et al. "Rt-2: Vision-language-action models transfer web knowledge to robotic control." arXiv preprint arXiv:2307.15818 (2023).

\bibitem{li2024manipllm} Li, Xiaoqi, et al. "Manipllm: Embodied multimodal large language model for object-centric robotic manipulation." Proceedings of the IEEE/CVF Conference on Computer Vision and Pattern Recognition. 2024.

\bibitem{shah2023vint} Shah, Dhruv, et al. "ViNT: A foundation model for visual navigation." arXiv preprint arXiv:2306.14846 (2023).

\bibitem{liu2024visual} Liu, Haotian, et al. "Visual instruction tuning." Advances in neural information processing systems 36 (2024).

\bibitem{hu2021lora} Hu, Edward J., et al. "Lora: Low-rank adaptation of large language models." arXiv preprint arXiv:2106.09685 (2021).

\bibitem{qin2023supfusion} Qin, Yiran, et al. "SupFusion: Supervised LiDAR-camera fusion for 3D object detection." Proceedings of the IEEE/CVF International Conference on Computer Vision. 2023.

\bibitem{qin2024worldsimbench} Qin, Yiran, et al. "Worldsimbench: Towards video generation models as world simulators." arXiv preprint arXiv:2410.18072 (2024).

\bibitem{yu2025gamefactory} Yu, Jiwen, et al. "GameFactory: Creating New Games with Generative Interactive Videos." arXiv preprint arXiv:2501.08325 (2025).

\bibitem{zhou2024code} Zhou, Enshen, et al. "Code-as-Monitor: Constraint-aware Visual Programming for Reactive and Proactive Robotic Failure Detection." arXiv preprint arXiv:2412.04455 (2024).

\bibitem{zhang2024ad} Zhang, Zaibin, et al. "AD-H: Autonomous Driving with Hierarchical Agents." arXiv preprint arXiv:2406.03474 (2024).

\bibitem{wang2024toward} Wang, Chaoqun, et al. "Toward Accurate Camera-based 3D Object Detection via Cascade Depth Estimation and Calibration." arXiv preprint arXiv:2402.04883 (2024).

\bibitem{huang2024story3d} Huang, Yuzhou, et al. "Story3d-agent: Exploring 3d storytelling visualization with large language models." arXiv preprint arXiv:2408.11801 (2024).

\bibitem{savinov2018semi} Savinov, Nikolay, Alexey Dosovitskiy, and Vladlen Koltun. "Semi-parametric topological memory for navigation." arXiv preprint arXiv:1803.00653 (2018).

\bibitem{majumdar2022zson} Majumdar, Arjun, et al. "Zson: Zero-shot object-goal navigation using multimodal goal embeddings." Advances in Neural Information Processing Systems 35 (2022): 32340-32352.

\bibitem{yadav2023offline} Yadav, Karmesh, et al. "Offline visual representation learning for embodied navigation." Workshop on Reincarnating Reinforcement Learning at ICLR 2023. 2023.

\bibitem{yadav2023ovrl} Yadav, Karmesh, et al. "Ovrl-v2: A simple state-of-art baseline for imagenav and objectnav." arXiv preprint arXiv:2303.07798 (2023).

\bibitem{sun2024fgprompt} Sun, Xinyu, et al. "FGPrompt: fine-grained goal prompting for image-goal navigation." Advances in Neural Information Processing Systems 36 (2024).

\bibitem{zhu2017target} Zhu, Yuke, et al. "Target-driven visual navigation in indoor scenes using deep reinforcement learning." 2017 IEEE international conference on robotics and automation (ICRA). IEEE, 2017.

\bibitem{koh2024generating} Koh, Jing Yu, Daniel Fried, and Russ R. Salakhutdinov. "Generating images with multimodal language models." Advances in Neural Information Processing Systems 36 (2024).

\bibitem{krantz2022instance} Krantz, Jacob, et al. "Instance-specific image goal navigation: Training embodied agents to find object instances." arXiv preprint arXiv:2211.15876 (2022).

\bibitem{schulman2017proximal} Schulman, John, et al. "Proximal policy optimization algorithms." arXiv preprint arXiv:1707.06347 (2017).

\bibitem{anderson2018evaluation} Anderson, Peter, et al. "On evaluation of embodied navigation agents." arXiv preprint arXiv:1807.06757 (2018).

\bibitem{lin2024navcot} Lin, Bingqian, et al. "NavCoT: Boosting LLM-Based Vision-and-Language Navigation via Learning Disentangled Reasoning." arXiv preprint arXiv:2403.07376 (2024).

\bibitem{NavGPT} Zhou, Gengze, Yicong Hong, and Qi Wu. "Navgpt: Explicit reasoning in vision-and-language navigation with large language models." Proceedings of the AAAI Conference on Artificial Intelligence.

\bibitem{hahn2021no} Hahn, Meera, et al. "No rl, no simulation: Learning to navigate without navigating." Advances in Neural Information Processing Systems 34 (2021): 26661-26673.

\bibitem{li2025t2isafety} Li, Lijun, et al. "T2ISafety: Benchmark for Assessing Fairness, Toxicity, and Privacy in Image Generation." arXiv preprint arXiv:2501.12612 (2025).

\bibitem{an2024agfsync} An, Jingkun, et al. "AGFSync: Leveraging AI-Generated Feedback for Preference Optimization in Text-to-Image Generation." arXiv preprint arXiv:2403.13352 (2024).


\end{thebibliography}
\end{sloppypar}

\clearpage
\beginsupplement
\section*{Appendix}
\renewcommand{\thesubsection}{S\arabic{subsection}}

\subsection{\label{chap:S1}PanNuke and MoNuSAC preprocessing}
The PanNuke dataset comprises a set of 7,901 RGB patches, each with dimensions of $256 \times 256$ pixels, which we set as the standard patch size for our analysis. In contrast, the MoNuSAC dataset encompasses 294 images of heterogeneous dimensions. To standardize the MoNuSAC images with our experiments, we implement a standardization protocol. Specifically, for images exceeding the dimensions of $256 \times 256$ pixels, we segment them into equal-sized patches and apply mirror padding to the remaining portions to avoid information loss at the peripherals. Patches with dimensions less than $128 \times 128$ pixels are excluded from the dataset due to the insufficient resolution to capture relevant cellular details. For patches where either dimension falls between 128 and 256 pixels, we employ upsampling to achieve the standard patch size. As a result, we obtain a total of 2,823 RGB patches derived from the MoNuSAC dataset for subsequent analysis. For additional details on the MoNuSAC data preparation process, refer to the source code \cite{Shvetsov_2025a}.
\clearpage

\subsection{\label{chap:S2}Data usage for the methodology}

\counterwithin{figure}{subsection}
\renewcommand{\thefigure}{S\arabic{subsection}}

\begin{figure}[h!]
    \centering
    \includegraphics[width=\textwidth, height=0.85\textheight, keepaspectratio]{images/A2.pdf}
    \caption{Overview of the methodology for cross-labeling, dataset refinement, and model comparison. (1) Cross-relabeling - training and testing cell classification models, (2) Cross-relabeling - using cell classification models to create refined dataset, (3) Fine-tuning and training models for comparison, (4) Student knowledge distillation with refined dataset}
    \label{fig:S2}
\end{figure}
\clearpage

\subsection{\label{chap:S3}Confusion matrices for classification models}
\counterwithin{figure}{subsection}
\renewcommand{\thefigure}{S\arabic{subsection}.\arabic{figure}}

\begin{figure}[h!]
    \centering
    \includegraphics[width=\textwidth, height=0.4\textheight, keepaspectratio]{images/A3_1.pdf}
    \caption{Confusion matrix for PanNuke trained model}
    \label{fig:S3.1}
\end{figure}

\begin{figure}[h!]
    \centering
    \includegraphics[width=\textwidth, height=0.4\textheight, keepaspectratio]{images/A3_2.pdf}
    \caption{Confusion matrix for MoNuSAC trained model}
    \label{fig:S3.2}
\end{figure}

\clearpage

\subsection{\label{chap:S4}Datasets cell counts}

\counterwithin{table}{subsection}
\renewcommand{\thetable}{S\arabic{subsection}}

\begin{table}[h!]
\renewcommand{\arraystretch}{2.0}
\centering
\caption{\label{tab:S4}Cell counts for PanNuke, MoNuSAC and refined datasets. Numbers in parentheses indicate preprocessed cell counts for cell classifier models training and testing.}
%\adjustbox{max width=\textwidth}{%
\begin{tabular}{|l|c|c|c|}
\hline
%\rowcolor{gray!30}
Cell type & PanNuke & MoNuSAC & Refined \\
\hline
Neoplastic & 77,403 (68,031) & - & 105,451 \\
\hline
Epithelial & 26,572 (23,207) & - & 29,926 \\
\hline
Epithelial (benign and malignant) & - & 31,402 & - \\
\hline
Inflammatory & 32,276 & - & - \\
\hline
Lymphocytes & - & 37,045 (33,104) & 65,275 \\
\hline
Neutrophils & - & 1,355 (1,252) & 3,833 \\
\hline
Macrophage & - & 1,842 (1,695) & 3,410 \\
\hline
Dead & 2,908 & - & 2,908 \\
\hline
Connective & 50,585 & - & 50,585 \\
\hline
\end{tabular}
%
%}
\end{table}



\clearpage

\subsection{\label{chap:S5}Definition of validation metrics}
\counterwithin{equation}{subsection}
\renewcommand{\theequation}{\arabic{equation}}

\subsubsection{\label{chap:S5.1}R\textsuperscript{2}}
The coefficient of determination, denoted as $R^2$, is a statistical measure that represents the proportion of variance in the dependent variable that is predictable from the independent variables. In the context of cell quantification in pathology, $R^2$ is used to assess how well the predicted quantities of different cell types in a patch align with the actual quantities observed in the ground truth data, with higher values representing more accurate quantification. $R^2$ is defined as
\begin{equation*}
R^2 = 1 - \frac{\sum_{i=1}^n (y_i - \hat{y}_i)^2}{\sum_{i=1}^n (y_i - \bar{y})^2},
\end{equation*}
where $y_i$ represents the actual number of cells of a specific type in the $i$-th image, $\hat{y}_i$ represents the predicted number of cells of that type in the $i$-th image, $\bar{y}$ is the mean of the actual numbers across all images, and $n$ is the total number of images in the dataset.

The $R^2$ metric has a range of $(-\infty, 1]$. An $R^2$ of 1 indicates perfect prediction, where all predicted values exactly match the actual values. An $R^2$ of 0 suggests that the model explains none of the variability of the response data around its mean. If $R^2$ is negative, it indicates that the model performs worse than a model that simply predicts the mean of the actual values for all observations.

\subsubsection{\label{chap:S5.2}PQ}
Panoptic Quality ($PQ$) is a comprehensive metric used to evaluate the performance of segmentation models in tasks that require both instance segmentation and classification. $PQ$ provides a single score that encapsulates both the detection accuracy (i.e., how many objects were correctly identified) and the segmentation quality (i.e., how accurately the objects' boundaries were delineated). This metric is particularly useful in multiclass scenarios where each pixel is classified into distinct categories, such as different cell types in pathology images.

$PQ$ is calculated as the product of two terms: Detection Quality ($DQ$) and Segmentation Quality ($SQ$). It can be expressed as
\begin{equation*}
PQ = DQ \cdot SQ,
\end{equation*}
where
\begin{equation*}
DQ = \frac{TP}{TP + 0.5\, FP + 0.5\, FN},
\end{equation*}
\begin{equation*}
SQ = \frac{\sum_{(p, g) \in \mathcal{M}} IoU(p, g)}{TP}.
\end{equation*}
In these formulas, $TP$ denotes the number of correctly matched instances between ground truth and prediction, $FP$ denotes the predicted instances that have no corresponding ground truth, $FN$ denotes the ground truth instances that were not detected, $IoU(p, g)$ is the Intersection over Union for a pair of matched instances $p$ (prediction) and $g$ (ground truth), and $\mathcal{M}$ is the set of matched pairs.

The $PQ$ metric is calculated for each class and is averaged across classes to provide a global performance measure.

The $PQ$ score has a range of $[0, 1.0]$, where a higher score indicates better performance in both detecting and segmenting the instances correctly. A $PQ$ of 1 signifies perfect identification and segmentation of all instances, whereas a $PQ$ of 0 indicates that no instances were correctly identified and segmented.

\clearpage

\subsection{\label{chap:S6}Segmentation and Detection quality metrics for teacher and student models}

\begin{table}[h!]
\renewcommand{\arraystretch}{2.0}
\centering
\caption{Segmentation and detection quality for student and teacher models (CI 95\%)}
\label{tab:S6}
%\adjustbox{max width=\textwidth}{%
\begin{tabular}{|l|c|c|}
\hline
%\rowcolor{gray!30}
Metric & Teacher & Student \\
\hline
$SQ_{neoplastic}$ & 0.819 (0.815--0.823) & 0.824 (0.819--0.828) \\
\hline
$SQ_{lymphocyte}$ & 0.795 (0.788--0.802) & 0.790 (0.783--0.796) \\
\hline
$SQ_{connective}$ & 0.770 (0.762--0.776) & 0.780 (0.772--0.786) \\
\hline
$SQ_{dead}$ & 0.659 (0.623--0.688) & 0.657 (0.624--0.695) \\
\hline
$SQ_{epithelial}$ & 0.780 (0.770--0.790) & 0.788 (0.779--0.797) \\
\hline
$SQ_{macrophage}$ & 0.788 (0.760--0.810) & 0.757 (0.730--0.783) \\
\hline
$SQ_{neutrofil}$ & 0.782 (0.761--0.801) & 0.775 (0.759--0.792) \\
\hline
$DQ_{neoplastic}$ & 0.706 (0.692--0.719) & 0.727 (0.712--0.741) \\
\hline
$DQ_{lymphocyte}$ & 0.675 (0.656--0.698) & 0.713 (0.691--0.734) \\
\hline
$DQ_{connective}$ & 0.566 (0.546--0.584) & 0.583 (0.565--0.602) \\
\hline
$DQ_{dead}$ & 0.410 (0.361--0.465) & 0.435 (0.306--0.561) \\
\hline
$DQ_{epithelial}$ & 0.668 (0.639--0.694) & 0.673 (0.644--0.702) \\
\hline
$DQ_{macrophage}$ & 0.657 (0.583--0.727) & 0.615 (0.531--0.703) \\
\hline
$DQ_{neutrofil}$ & 0.691 (0.625--0.753) & 0.729 (0.679--0.778) \\
\hline
\end{tabular}
%
%}
\end{table}

\clearpage

\subsection{\label{chap:S7}QuPath integration method}
We adopt an integration strategy leveraging the paquo \cite{Bayer_AG} library, a Python package that enables direct interaction with QuPath’s internal API, thereby facilitating seamless data exchange without intermediate conversion steps. The data processing pipeline (\hyperref[fig:S7]{Appendix Figure S7}) begins with the acquisition of WSIs and their associated annotations from QuPath, which are represented as Shapely \cite{Gillies_Wel_etal._2024} polygons. Utilizing paquo, we directly read, create, and modify these annotations and detections within a QuPath project in the Python environment. Images are then cropped using these polygons and processed by cell segmentation and classification models employing standard vision processing toolkits such as OpenCV, pyvips, and PyTorch. Additionally, QuPath employs Groovy scripts to initiate a Python process that starts the entire pipeline from QuPath graphical interface: fetching polygons, extracting images from them, and running deep learning model inference on the cropped images. 
The results are returned to QuPath, leveraging paquo's Python bindings to manipulate QuPath data while minimizing the computational overhead typically associated with cross-environment communication.

\counterwithin{figure}{subsection}
\renewcommand{\thefigure}{S\arabic{subsection}}

\begin{figure}[h!]
    \centering
    \includegraphics[width=\textwidth]{images/A7.pdf}
    \caption{QuPath integration workflow using Python environment}
    \label{fig:S7}
\end{figure}

Compared to traditional workflows that involve exporting annotations as GeoJSON, classifying them in Python, and reimporting them into QuPath, our approach offers several advantages. We eliminate the need to switch between programming languages, providing a cohesive and streamlined development process entirely within QuPath software and removing the necessity to use other tools. Meanwhile, we avoid storing annotations as intermediate JSON files unless required for external use or archiving. By conducting the entire inference and post-processing workflow within the Python environment, we leverage the power and flexibility of Python libraries for image processing and machine learning. This approach also enables adjustments to any set of labels and models, thereby improving its applicability.

%\hfill

The distilled model and QuPath integration code are packaged into a Docker container, enabling streamlined execution with the Docker engine. Detailed integration code and deployment instructions can be found in the GitHub repository \cite{Shvetsov_2025b}.

Despite these benefits, we acknowledge that the paquo library is a proof‑of‑concept project in its early development stage and has not been tested across all versions of QuPath.

\clearpage

\subsection{\label{chap:S8}Data and code availability statement}
All datasets, models, and code used in this study are publicly available and can be obtained from the repositories listed below. 
The PanNuke \cite{Gamper_Koohbanani_etal._2019} and MoNuSAC \cite{Verma_Kumar_etal._2021} datasets are publicly accessible, and download information along with detailed descriptions can be found in their respective articles. Preprocessing scripts for PanNuke and MoNuSAC data, as well as individual cell extraction scripts, are available on GitHub \cite{Shvetsov_2025a}. The H-Optimus foundation model used in our experiments can be downloaded from the HuggingFace repository \cite{hoptimus2024}, and model information is available on GitHub \cite{Saillard_Jenatton_etal._2024}. In addition, the integration code for QuPath and the distilled model packaged in a Docker container are provided in the repository \cite{Shvetsov_2025b}, and paquo Python library is available from the authors GitHub repository \cite{Bayer_AG}.
\clearpage

\end{document}



%%%%%%%%%%%%%%%%%%%%%%%%%%%%%%%%%%%%%%%%%%%%%%%%%%%%%%%%%%%%%%%%%%%%%%%%%%%%%%%
%%%%%%%%%%%%%%%%%%%%%%%%%%%%%%%%%%%%%%%%%%%%%%%%%%%%%%%%%%%%%%%%%%%%%%%%%%%%%%%
% APPENDIX
%%%%%%%%%%%%%%%%%%%%%%%%%%%%%%%%%%%%%%%%%%%%%%%%%%%%%%%%%%%%%%%%%%%%%%%%%%%%%%%
%%%%%%%%%%%%%%%%%%%%%%%%%%%%%%%%%%%%%%%%%%%%%%%%%%%%%%%%%%%%%%%%%%%%%%%%%%%%%%%
\newpage
\appendix
\onecolumn
\section{Appendix}
\subsection{Pseudocode}\label{sec:pseudo}

\begin{algorithm}[H]
\caption{Teleportation with Input Null Space Gradient Projection}
\textbf{Input:} Loss function $\mathcal{L}(w)$, number of epochs for primary task $T$, teleport learning rate $\eta$, teleport batch number $b$, teleport step number $t$,  teleport schedule $K$, threshold maximum gradient norm value $\text{CAP}$, initialized parameters $w_0$. \\
\textbf{Output:} $w_{T}$.
\begin{algorithmic}[1]
\For{$i \gets 0$ to $T - 1$}
    \If{$i \in K$}
        \For{$b$ batches}
            \State $\text{Null space projection matrix}$ $\pi \gets \text{SVD(batch)}$
            \For{t steps}                   
                \If{$\|\nabla_{w}\mathcal{L}|_{w_i}\|^2 < \text{CAP}$}
                    \State $w_i \gets w_i - \eta\pi(\nabla_{w} \|\nabla_{w}\mathcal{L}|_{w_i}\|^2|_{w_i})$
                \Else
                    \State \textbf{break}
                \EndIf

            \EndFor
        \EndFor
    \EndIf
    \State Continue the training of the primary task
\EndFor
\State \textbf{return} $w_{T}$
\end{algorithmic}
\end{algorithm}

\subsection{Additional Results}
\subsubsection{Complete Test Loss Trajectories Comparison Between Symmetry Teleport and Our Algorithm}\label{sec:comparison_append}
\begin{figure*}[htbp]
    \centering

    \begin{subfigure}{0.24\textwidth} % Adjust the width as needed
        \centering
        \includegraphics[width=\textwidth]{images/pdf/comparison_sgd.pdf}
        % \caption{Figure 1}
    \end{subfigure}
    \begin{subfigure}{0.24\textwidth}
        \centering
        \includegraphics[width=\textwidth]{images/pdf/comparison_adagrad.pdf}
        % \caption{Figure 2}
    \end{subfigure}
    \begin{subfigure}{0.24\textwidth}
        \centering
        \includegraphics[width=\textwidth]{images/pdf/comparison_mom.pdf}
        % \caption{Figure 3}
    \end{subfigure}
    \begin{subfigure}{0.24\textwidth}
        \centering
        \includegraphics[width=\textwidth]{images/pdf/comparison_adam.pdf}
        % \caption{Figure 4}
    \end{subfigure}
    \caption{Complete train and test loss trajectories of training MLPs on the MNIST dataset, comparing symmetry teleport and our algorithm. Each experiment is repeated 3 times, with the average loss plotted and the standard deviation of loss represented as the shaded area.
    }
    \label{}
\end{figure*}
\subsubsection{MLP on MNIST and FashionMNIST datasets}
\begin{figure*}[htbp]
    \centering
    \begin{subfigure}{0.24\textwidth} % Adjust the width as needed
        \centering
        \includegraphics[width=\textwidth]{images/pdf/MNIST_SGD_MLP_loss_vs_epoch.pdf}
        % \caption{Figure 1}
    \end{subfigure}
    \begin{subfigure}{0.24\textwidth}
        \centering
        \includegraphics[width=\textwidth]{images/pdf/MNIST_momentum_MLP_loss_vs_epoch.pdf}
        % \caption{Figure 2}
    \end{subfigure}
    \begin{subfigure}{0.24\textwidth}
        \centering
        \includegraphics[width=\textwidth]{images/pdf/MNIST_Adagrad_MLP_loss_vs_epoch.pdf}
        % \caption{Figure 3}
    \end{subfigure}
    \begin{subfigure}{0.24\textwidth}
        \centering
        \includegraphics[width=\textwidth]{images/pdf/MNIST_Adam_MLP_loss_vs_epoch.pdf}
        % \caption{Figure 4}
    \end{subfigure}
   \\
    \begin{subfigure}{0.24\textwidth} % Adjust the width as needed
        \centering
        \includegraphics[width=\textwidth]{images/pdf/FashionMNIST_SGD_MLP_loss_vs_epoch.pdf}
        % \caption{Figure 1}
    \end{subfigure}
    \begin{subfigure}{0.24\textwidth}
        \centering
        \includegraphics[width=\textwidth]{images/pdf/FashionMNIST_momentum_MLP_loss_vs_epoch.pdf}
        % \caption{Figure 2}
    \end{subfigure}
    \begin{subfigure}{0.24\textwidth}
        \centering
        \includegraphics[width=\textwidth]{images/pdf/FashionMNIST_Adagrad_MLP_loss_vs_epoch.pdf}
        % \caption{Figure 3}
    \end{subfigure}
    \begin{subfigure}{0.24\textwidth}
        \centering
        \includegraphics[width=\textwidth]{images/pdf/FashionMNIST_Adam_MLP_loss_vs_epoch.pdf}
        % \caption{Figure 4}
    \end{subfigure}
    \caption{Loss trajectories of training MLPs on the MNIST and FashionMNIST datasets. Each experiment is repeated 3 times, with the average loss plotted and the standard deviation of loss represented as the shaded area.
    }
    \label{fig:mlp_append}
\end{figure*}
\subsubsection{CNN on CIFAR100 dataset}\label{sec:cnn_append}
\begin{figure*}[htbp]
    \centering
 %    \begin{subfigure}{0.24\textwidth} % Adjust the width as needed
 %        \centering
 %        \includegraphics[width=\textwidth]{images/pdf/CIFAR10_SGD_CNN_loss_vs_epoch.pdf}
 %        % \caption{Figure 1}
 %    \end{subfigure}
 %    \begin{subfigure}{0.24\textwidth}
 %        \centering
 %        \includegraphics[width=\textwidth]{images/pdf/CIFAR10_momentum_CNN_loss_vs_epoch.pdf}
 %        % \caption{Figure 2}
 %    \end{subfigure}
 %    \begin{subfigure}{0.24\textwidth}
 %        \centering
 %        \includegraphics[width=\textwidth]{images/pdf/CIFAR10_Adagrad_CNN_loss_vs_epoch.pdf}
 %        % \caption{Figure 3}
 %    \end{subfigure}
 %    \begin{subfigure}{0.24\textwidth}
 %        \centering
 %        \includegraphics[width=\textwidth]{images/pdf/CIFAR10_Adam_CNN_loss_vs_epoch.pdf}
 %        % \caption{Figure 4}
 %    \end{subfigure}
 % \\
    \begin{subfigure}{0.24\textwidth} % Adjust the width as needed
        \centering
        \includegraphics[width=\textwidth]{images/pdf/CIFAR100_SGD_CNN_loss_vs_epoch.pdf}
        % \caption{Figure 1}
    \end{subfigure}
    \begin{subfigure}{0.24\textwidth}
        \centering
        \includegraphics[width=\textwidth]{images/pdf/CIFAR100_momentum_CNN_loss_vs_epoch.pdf}
        % \caption{Figure 2}
    \end{subfigure}
    \begin{subfigure}{0.24\textwidth}
        \centering
        \includegraphics[width=\textwidth]{images/pdf/CIFAR100_Adagrad_CNN_loss_vs_epoch.pdf}
        % \caption{Figure 3}
    \end{subfigure}
    \begin{subfigure}{0.24\textwidth}
        \centering
        \includegraphics[width=\textwidth]{images/pdf/CIFAR100_Adam_CNN_loss_vs_epoch.pdf}
        % \caption{Figure 4}
    \end{subfigure}
% \\
%     \begin{subfigure}{0.24\textwidth} % Adjust the width as needed
%         \centering
%         \includegraphics[width=\textwidth]{images/pdf/Imagenet_SGD_CNN_loss_vs_epoch.pdf}
%         % \caption{Figure 1}
%     \end{subfigure}
%     \begin{subfigure}{0.24\textwidth}
%         \centering
%         \includegraphics[width=\textwidth]{images/pdf/Imagenet_momentum_CNN_loss_vs_epoch.pdf}
%         % \caption{Figure 2}
%     \end{subfigure}
%     \begin{subfigure}{0.24\textwidth}
%         \centering
%         \includegraphics[width=\textwidth]{images/pdf/Imagenet_Adagrad_CNN_loss_vs_epoch.pdf}
%         % \caption{Figure 3}
%     \end{subfigure}
%     \begin{subfigure}{0.24\textwidth}
%         \centering
%         \includegraphics[width=\textwidth]{images/pdf/Imagenet_Adam_CNN_loss_vs_epoch.pdf}
%         % \caption{Figure 4}
%     \end{subfigure}
    \caption{Loss trajectories of training CNNs on CIFAR100 dataset. Each experiment is repeated 3 times, with the average loss plotted and the standard deviation of loss represented as the shaded area.}
    \label{fig:cnn_append}
\end{figure*}

\subsubsection{Transformer on Sequential MNIST dataset}\label{sec:smnist}
\begin{figure*}[htbp]
    \centering
    \begin{subfigure}{0.24\textwidth} % Adjust the width as needed
        \centering
        \includegraphics[width=\textwidth]{images/pdf/MNIST_SGD_transformer_loss_vs_epoch.pdf}
        % \caption{Figure 1}
    \end{subfigure}
    \begin{subfigure}{0.24\textwidth}
        \centering
        \includegraphics[width=\textwidth]{images/pdf/MNIST_momentum_transformer_loss_vs_epoch.pdf}
        % \caption{Figure 2}
    \end{subfigure}
    \begin{subfigure}{0.24\textwidth}
        \centering
        \includegraphics[width=\textwidth]{images/pdf/MNIST_Adagrad_transformer_loss_vs_epoch.pdf}
        % \caption{Figure 3}
    \end{subfigure}
    \begin{subfigure}{0.24\textwidth}
        \centering
        \includegraphics[width=\textwidth]{images/pdf/MNIST_Adam_transformer_loss_vs_epoch.pdf}
        % \caption{Figure 4}
    \end{subfigure}
  %  \\
  %   \begin{subfigure}{0.24\textwidth} % Adjust the width as needed
  %       \centering
  %       \includegraphics[width=\textwidth]{images/pdf/electricity_SGD_transformer_loss_vs_epoch.pdf}
  %       % \caption{Figure 1}
  %   \end{subfigure}
  %   \begin{subfigure}{0.24\textwidth}
  %       \centering
  %       \includegraphics[width=\textwidth]{images/pdf/electricity_momentum_transformer_loss_vs_epoch.pdf}
  %       % \caption{Figure 2}
  %   \end{subfigure}
  %   \begin{subfigure}{0.24\textwidth}
  %       \centering
  %       \includegraphics[width=\textwidth]{images/pdf/electricity_Adagrad_transformer_loss_vs_epoch.pdf}
  %       % \caption{Figure 3}
  %   \end{subfigure}
  %   \begin{subfigure}{0.24\textwidth}
  %       \centering
  %       \includegraphics[width=\textwidth]{images/pdf/electricity_Adam_transformer_loss_vs_epoch.pdf}
  %       % \caption{Figure 4}
  %   \end{subfigure}
  % \\
  %   \begin{subfigure}{0.24\textwidth} % Adjust the width as needed
  %       \centering
  %       \includegraphics[width=\textwidth]{images/pdf/traffic_SGD_transformer_loss_vs_epoch.pdf}
  %       % \caption{Figure 1}
  %   \end{subfigure}
  %   \begin{subfigure}{0.24\textwidth}
  %       \centering
  %       \includegraphics[width=\textwidth]{images/pdf/traffic_momentum_transformer_loss_vs_epoch.pdf}
  %       % \caption{Figure 2}
  %   \end{subfigure}
  %   \begin{subfigure}{0.24\textwidth}
  %       \centering
  %       \includegraphics[width=\textwidth]{images/pdf/traffic_Adagrad_transformer_loss_vs_epoch.pdf}
  %       % \caption{Figure 3}
  %   \end{subfigure}
  %   \begin{subfigure}{0.24\textwidth}
  %       \centering
  %       \includegraphics[width=\textwidth]{images/pdf/traffic_Adam_transformer_loss_vs_epoch.pdf}
  %       % \caption{Figure 4}
  %   \end{subfigure}
  % \\
  %   \begin{subfigure}{0.24\textwidth} % Adjust the width as needed
  %       \centering
  %       \includegraphics[width=\textwidth]{images/pdf/PennTree_SGD_transformer_loss_vs_epoch.pdf}
  %       % \caption{Figure 1}
  %   \end{subfigure}
  %   \begin{subfigure}{0.24\textwidth}
  %       \centering
  %       \includegraphics[width=\textwidth]{images/pdf/PennTree_momentum_transformer_loss_vs_epoch.pdf}
  %       % \caption{Figure 2}
  %   \end{subfigure}
  %   \begin{subfigure}{0.24\textwidth}
  %       \centering
  %       \includegraphics[width=\textwidth]{images/pdf/PennTree_Adagrad_transformer_loss_vs_epoch.pdf}
  %       % \caption{Figure 3}
  %   \end{subfigure}
  %   \begin{subfigure}{0.24\textwidth}
  %       \centering
  %       \includegraphics[width=\textwidth]{images/pdf/PennTree_Adam_transformer_loss_vs_epoch.pdf}
  %       % \caption{Figure 4}
  %   \end{subfigure}
    \caption{Loss trajectories of training Transformers on sequential MNIST dataset. Each experiment is repeated 3 times, with the average loss plotted and the standard deviation of loss represented as the shaded area.}
    \label{fig:att_append}
\end{figure*}

\subsection{Implementation Details}\label{sec: implem}
In table ~\ref{table:hyper-param}, we summarize the hyper-parameters used in experiments. We denote the base learning rate for primary task as $\eta_{prim}$, the learning rate for teleportation as $\eta_{tele}$, maximum epoch for primary task as $T_{prim}$, teleport batch size as $n$, and teleport cap threshold as CAP. The batch size for the primary task is set to $32$, the number of teleport batches set to $32$, and the number of teleportation steps per batch set to $8$ throughout all experiments.

\textbf{MLP}\\
\textbf{Datasets.} To demonstrate the effectiveness of our method with MLPs, we conduct experiments using the MNIST digit image classification dataset and its clothing variant, FashionMNIST. Both datasets are split into $60,000$ samples for training and $10,000$ samples for testing. The input images, with dimensions of $28 \times 28$ pixels, are flattened into vectors before being fed into the MLPs models.

\textbf{Implementation Detail.} We use a 3-layer MLPs with hidden dimensions [$1024, 1024$], ReLU activation function, and cross-entropy loss. Following the convention in ~\cite{zhao2022symmetry}'s work, we schedule teleportation for the first $5$ epochs of the primary training phase. For each teleportation in the schedule, we randomly sample $32$ batches of data and perform $8$ teleport updates per batch. The SVD threshold is set to 1, i.e., \textbf{\emph{the gradients are projected onto the exact input null space}}. Learning rates are set differently depending on the optimizer used. See the appendix ~\ref{sec: implem} for complete implementation details.

\textbf{CNN}\\
\textbf{Datasets.} We use the CIFAR-10, CIFAR-100, and Tiny-Imagenet datasets to evaluate the effectiveness of our algorithm on CNNs. Both CIFAR datasets are split into 50,000 training samples and 10,000 test samples. The image size for CIFAR datasets is $3\times32\times32$. The Tiny-Imagenet dataset is a smaller version of the full Imagenet dataset, containing $200$ image classes with $100,000$ training images and $20,000$ validation/test images. The image size for the Tiny-Imagenet dataset is kept the same as the full Imagenet dataset, i.e., $3\times224\times224$.

\textbf{Implementation Detail.} For the CIFAR datasets, we use a $3$-layer CNNs with channels [$3, 16, 32, 64$], max pooling after each layer, ReLU activation function, and cross-entropy loss. For the Tiny-Imagenet dataset, we utilize a residual network with channels [$3, 64, 64, 64, 128, 128, 128, 256, 256, 256$], and $3$ residual connections between channels of same shape. Instead of max pooling, we use larger strides to reduce the feature size, a common practice in the design of residual networks. A classification head is connected after the final channel for both architectures. The teleportation scheduling and threshold $\tau$ remains the same as in the MLPs experiments. See appendix ~\ref{sec: implem} for complete implementation details.

For all experiments using CNNs, we perform $40$ warm-up steps before the first teleportation to stabilize the behavior of the gradients.

\textbf{Transformer}\\
\textbf{Datasets.} We first consider the MNIST dataset as a sequential classification task, with a sequence length of $28\times28$ and a data dimension $1$. 

Next, we evaluate on two publicly available multi-variate time series regression datasets: electricity and traffic. The electricity dataset consists of $321$ dimensions with a total sequence length of $26,304$. The sample sequence length is set to $7\times24$, representing a week's worth of data. The regression target is the data point of the same dimension $24$ hours after the input sample. The traffic dataset consists of $862$ dimensions, with a total sequence length of $17,544$. The data is similarly manipulated to regress a week's worth of data to the data $24$ hours after the week. See Appendix ~\ref{sec:data} for a detailed explanation.

% The electricity dataset tracks electricity consumption in kWh every $15$ minutes from $2012$ to $2014$ for $321$ clients, adjusted to reflect hourly consumption. The dataset consists of $321$ dimensions with a total sequence length of $26,304$. The sample sequence length is set to $7\times24$, representing a week's worth of data. The regression target is the data point of the same dimension $24$ hours after the input sample. The traffic dataset contains $48$ months $(2015–2016)$ of hourly data from the California Department of Transportation, describing road occupancy rates (between $0$ and $1$) measured by various sensors on the San Francisco Bay Area freeway. This dataset consists of $862$ dimensions, with a total sequence length of $17,544$. The data is similarly manipulated to regress a week's worth of data to the data $24$ hours after the week. 

We also evaluate on the Penn Treebank (PTB) language corpus. We use the default train/test split of the PTB dataset, where the training set contains approximately $950,000$ words and the test set approximately $80,000$ words. We use the TreebankWord tokenizer from the nltk Library and set the sequence length to 256. As is common practice, we formulate the problem as a causal self-supervised learning task, where the label is the input shifted to the right by one.

\textbf{Implementation Detail.} For the sequential MNIST dataset, we use a small Transformer model with $2$ heads, each having a dimension of $64$, stacked across two layers. For the regression and language datasets, we use a transformer with 4 heads, each with a dimension of $64$, stacked across $4$ layers without pooling, followed by a linear output. See appendix ~\ref{sec: implem} for complete implementation details.


For the sequential MNIST dataset, we use a small Transformer model with $2$ heads, each having a dimension of $64$, stacked across two layers. This is followed by an average pooling layer and a ten-way linear classification head, optimized using cross-entropy loss. For the electricity and traffic datasets, we use a transformer with 4 heads, each with a dimension of $64$, stacked across $4$ layers without pooling, followed by a linear regression head where the output dimension matches the input dimension. For the PTB dataset, we use the same Transformer architecture but replace the first linear layer with an embedding layer and set the output dimension to the vocabulary size, which is approximately $10,000$. 

\begin{table}[htbp]
\centering
\begin{tabular}{|p{4.5cm}|p{1cm}|p{1cm}|p{1cm}|p{1cm}|p{1cm}|}
\hline
\textbf{Dataset (Optimizer)} & \textbf{$\eta_{prim}$} & \textbf{$\eta_{tele}$} & \textbf{$T_{prim}$}  & \textbf{n} &  \textbf{CAP} \\
\hline
MNIST (SGD) & $2e-4$&$2e-1$ & $100$&$32$ & $5$  \\
\hline
MNIST (Momentum) & $2e-4$& $2e-1$& $100$&$32$ & $5$\\
\hline
MNIST (Adagrad) & $2e-4$& $2e-1$& $100$&$32$   & $5$\\
\hline
MNIST (Adam) &$2e-4$ & $2e-1$&$100$ &$32$   &$5$ \\
\hline
FashionMNIST (SGD) &$2e-4$ &$2e-1$ & $100$&$32$   &$5$ \\
\hline
FashionMNIST (Momentum) & $2e-4$&$2e-1$ &$100$ &$32$   &$5$ \\
\hline
FashionMNIST (Adagrad) &$2e-4$ & $2e-1$& $100$& $32$  & $5$\\
\hline
FashionMNIST (Adam) & $2e-4$&$2e-1$ & $100$&$32$ &$5$   \\
\hline
CIFAR10 (SGD) & $1e-4$& $3e-3$&$100$ &$256$   &$40$ \\
\hline
CIFAR10 (Momentum) &$1e-4$ & $3e-3$& $100$&$256$   &$40$ \\
\hline
CIFAR10 (Adagrad) &$1e-4$ &$3e-3$ &$100$ & $256$&$40$   \\
\hline
CIFAR10 (Adam) &$1e-5$ &$3e-3$ &$300$ &$256$   & $40$\\
\hline
CIFAR100 (SGD) &$1e-4$ &$3e-3$ &$400$ &$256$   &$40$ \\
\hline
CIFAR100 (Momentum) & $1e-4$&$3e-3$ & $400$&$256$   &$40$ \\
\hline
CIFAR100 (Adagrad) &$1e-4$ & $3e-3$&$400$ &$256$   & $40$\\
\hline
CIFAR100 (Adam) &$3e-5$ &$3e-3$ & $400$&$256$   &$40$ \\
\hline
Tiny Imagenet (SGD) & $2e-4$&$3e-3$ & $400$& $32$& $40$  \\
\hline
Tiny Imagenet (Momentum) &$2e-4$ & $3e-3$&$400$   &$32$ & $40$\\
\hline
Tiny Imagenet (Adagrad) &$2e-4$ & $3e-3$&$400$   &$32$ & $40$\\
\hline
Tiny Imagenet (Adam) &$5e-5$ &$3e-3$ & $400$& $32$  &$40$ \\
\hline
sMNIST (SGD) &$1e-3$ & $3e-3$&$400$ &$32$ & $10$  \\
\hline
sMNIST (Momentum) &$1e-3$ & $3e-3$&$400$ &$32$   & $10$\\
\hline
sMNIST (Adagrad) & $1e-3$& $3e-3$&$400$ &$32$ & $10$  \\
\hline
sMNIST (Adam) &$1e-4$ &$3e-3$ &$400$ &$32$   &$10$ \\
\hline
electricity (SGD) &$1e-4$ &$3e-3$ &$50$ &$32$   & $10$\\
\hline
electricity (Momentum) &$1e-4$ &$3e-3$ & $50$&$32$   & $10$\\
\hline
electricity (Adagrad) &$1e-4$ & $3e-3$& $50$& $32$  &$10$ \\
\hline
electricity (Adam) &$1e-4$ &$3e-3$ &$50$ & $32$ & $10$ \\
\hline
traffic (SGD) &$1e-4$ &$3e-3$ &$50$ & $32$& $10$  \\
\hline
traffic (Momentum) &$1e-4$ &$3e-3$ &$50$ & $32$ & $10$ \\
\hline
traffic (Adagrad) &$1e-4$ &$3e-3$ &$50$ & $32$  &$10$ \\
\hline
traffic (Adam) & $1e-4$&$3e-3$ &$50$ & $32$  &$10$ \\
\hline
Penn Treebank (SGD) &$2e-4$ &$5e-2$ &$20,000$ steps & $32$  &$5$ \\
\hline
Penn Treebank (Momentum) &$2e-4$ & $5e-2$&$20,000$ steps &$32$  &$5$ \\
\hline
Penn Treebank (Adagrad) &$2e-4$ &$5e-2$ &$20,000$ steps &$32$ &  $5$ \\
\hline
Penn Treebank (Adam) &$5e-5$ &$5e-2$ & $20,000$ steps&$32$ &  $5$ \\
\hline
\end{tabular}
\caption{Summary table for hyper-parameters of all experiments}
\label{table:hyper-param}
\end{table}
\subsection{Visualization of Matrix Multiplication Representation for CNNs}\label{sec:visual}
Although filters in CNNs works differently than weights in MLPs, the forward and backward propagations of CNNs are essentially still matrix multiplications (see Figure ~\ref{fig:cnnmatmul}).
\begin{figure}[H]
    \centering
    \includegraphics[width=\textwidth]{images/pdf/CNNgraph.png} % Change the file name to your actual image file
    \caption{Visualization of matrix representation of forward and backward pass for CNNs.}
    \label{fig:cnnmatmul}
\end{figure}

\subsection{Brief Explanation of The Multi-variate Time Series Regression Datasets} \label{sec:data}
The electricity dataset tracks electricity consumption in kWh every $15$ minutes from $2012$ to $2014$ for $321$ clients, adjusted to reflect hourly consumption. The dataset consists of $321$ dimensions with a total sequence length of $26,304$. The sample sequence length is set to $7\times24$, representing a week's worth of data. The regression target is the data point of the same dimension $24$ hours after the input sample. The traffic dataset contains $48$ months $(2015–2016)$ of hourly data from the California Department of Transportation, describing road occupancy rates (between $0$ and $1$) measured by various sensors on the San Francisco Bay Area freeway. This dataset consists of $862$ dimensions, with a total sequence length of $17,544$. The data is similarly manipulated to regress a week's worth of data to the data $24$ hours after the week. 



\end{document}


% This document was modified from the file originally made available by
% Pat Langley and Andrea Danyluk for ICML-2K. This version was created
% by Iain Murray in 2018, and modified by Alexandre Bouchard in
% 2019 and 2021 and by Csaba Szepesvari, Gang Niu and Sivan Sabato in 2022.
% Modified again in 2023 and 2024 by Sivan Sabato and Jonathan Scarlett.
% Previous contributors include Dan Roy, Lise Getoor and Tobias
% Scheffer, which was slightly modified from the 2010 version by
% Thorsten Joachims & Johannes Fuernkranz, slightly modified from the
% 2009 version by Kiri Wagstaff and Sam Roweis's 2008 version, which is
% slightly modified from Prasad Tadepalli's 2007 version which is a
% lightly changed version of the previous year's version by Andrew
% Moore, which was in turn edited from those of Kristian Kersting and
% Codrina Lauth. Alex Smola contributed to the algorithmic style files.
