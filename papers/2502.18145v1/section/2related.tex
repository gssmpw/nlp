\section{Background}
This section discusses relevant studies about agent-based modeling and simulation~(ABMS) and human-AI interaction for ABMS.
\subsection{Agent-based Modeling and Simulation}
Autonomous agents demonstrate varying degrees of intelligence, enabling them to perceive their environment, make decisions, and execute actions in pursuit of certain goals~\cite{10.1007/BFb0013570,wooldridge_jennings_1995}.
Agent-based modeling and simulation~(ABMS) connects the micro-level actions of individual agents to the macro-level dynamics of the overall system~\cite{8352646}.
The investigation of ABMS has been a longstanding area of focus within the field of artificial intelligence research~\cite{1574234,5429318,doi:10.1177/0037549706073695}.
ABMS is a powerful method for simulating complex social systems. 
It constructs a computational environment to allow autonomous, dynamic, and heterogeneous agents to interact with one another and their surroundings, acting according to predefined rules or behaviors.
This approach enables the exploration of emergent phenomena arising from individual agent interactions within the complex system~\cite{helbing_agent-based_2012,doi:10.1073/pnas.072081299}.
ABMS demonstrates remarkable flexibility and has been utilized in a wide range of disciplines.
The global financial system ranks as one of the most intricate systems developed by humans~\cite{Wang_2018, Samanidou_2007}.
Ponta\etal~\cite{Ponta_2011} presented a multi-asset, agent-based financial market model composed of zero-intelligence agents with limited financial resources.
Random allocation strategies were employed for agents constrained by their finite resources. 
The resulting stock market dynamics exhibit stylized facts, including volatility clustering, fat-tailed return distributions, and mean reversion tendencies.
ABMS can help public healthcare administrators identify interventions that enhance population wellness and quality of care while concurrently reducing costs~\cite{SILVERMAN201561,williams2023epidemicmodelinggenerativeagents,el-sayed_social_2012}.
Researchers and public health officials across many countries have utilized Covasim~\cite {kerr_covasim_2021} to forecast epidemic trends, evaluate intervention scenarios, and assess resource requirements~\cite{SILVA2020110088}.
Furthermore, Conte\etal~\cite{10.3389/fpsyg.2014.00668} presented that interdisciplinary computational social science uses ABSM to verify internal consistency, examine the resulting aggregate states, and employ cross-methodological experimental approaches to validate hypotheses against real-world data.
With the advancement of Internet technology, social media has transformed our way of life~\cite{KAPLAN201059}.
Gatti\etal~\cite{gatti_large-scale_2014} proposed stochastic multi-agent-based modeling to simulate what users post on an egocentric social network, where Barack Obama is considered as the central user.

The emergence of powerful capabilities in large language models~(LLMs)~\cite{bommasani2022opportunities,brown_language_2020} enables agents to exhibit more human-like behaviors~\cite{NBERw31122}, sparking significant interest in ABMS among an increasing number of researchers from AI and HCI community.
In contrast to predefined rules and decision trees~\cite{marcotte_behavior_2017}, LLMs add flexibility by allowing agents to adapt and respond to complex situations dynamically, improving the quality of behavioral modeling.
Park\etal~\cite{10.1145/3586183.3606763}, leveraging LLMs' power, introduced generative agents that can simulate believable human behaviors with architecture for synthesizing and retrieving relevant information.
Moreover, LLMs broaden the application scenarios for ABMS by enabling more nuanced, human-like agent interactions and expanding the scope of dynamic environments that can be realistically modeled.
AGENTVERSE~\cite{chen2023agentversefacilitatingmultiagentcollaboration} emphasized the effectiveness of the multi-agent collaboration on text understanding, coding, and tool utilization.
$S^3$~\cite{gao2023s3socialnetworksimulationlarge} simulated social networks with LLM-empowered agents to capture three forms of propagation: information, emotion, and attitude.
Chatlaw~\cite{cui2024chatlawmultiagentcollaborativelegal} applied a multi-agent system to improve the reliability and precision of AI-powered legal services.
A chat-powered software development framework where LLMs power specialized agents to design, code, and test software~\cite {qian2024chatdevcommunicativeagentssoftware}.
There exist previous surveys on LLM-empowered agents~\cite{xi2023risepotentiallargelanguage,wang_survey_2024} and ABMS~\cite{gao_large_2023}.
Their primary focus is on how to design simulation agents and how to build simulation environments.
Although Xi\etal~\cite{xi2023risepotentiallargelanguage} summarized two paradigms of human-agent interaction, there is a lack of systematic surveys to investigate how humans interact with the ABMS system.
To fill the gap, we first categorized human-AI interactive methods in the context of ABMS according to the ``5W1H'' guideline.


\subsection{Human-AI Interaction for ABMS}
The development of the ABMS scientific simulation platform has evolved over several decades~\cite{doi:10.1177/0037549706073695,berryman2008review}, driven by advances in computational power, the need for more realistic modeling of complex systems, and interdisciplinary applications for researchers. 
These platforms provide the frameworks and tools that allow users to build, run, and analyze ABMS across various domains.
For example, MASON~\cite{doi:10.1177/0037549705058073} is a high-performance agent-based simulation toolkit developed in Java, allowing users to build complex models by combining customizable components and handle simulations involving thousands of agents efficiently.
NetLogo~\cite{netlogo} is another high-level platform offering a simple yet robust programming language, integrated graphical interfaces, and extensive documentation.
It is mainly designed for ABMS involving dynamic individuals with local interactions within a grid space.
Although NetLogo’s custom language is user-friendly, it is limited in functionality and flexibility compared to Python or Java, which restricts the depth of customization available to advanced users or those needing complex system handling and processing capabilities.
Guyot\etal~\cite{guyot2006} proposed ``agent-based participatory simulations'' methods to simulate multi-agent systems where human participants can control some of the agents.
Furthermore, real human demographic information can be utilized for the initialization of ABSM systems~\cite{GAUBE201392,10.1145/3394486.3412862}.
Human-AI interactions in ABMS have also been studied in Role-Playing Games~(RPGs), a genre of games where players assume the roles of characters in a fictional world, interacting within a narrative-rich environment~\cite{riedl_interactive_2021}.
Autonomous agents are well known for appearing in these games as non-player characters~(NPCs)~\cite{10.1145/2282338.2282384,10.1145/2159365.2159425}.
These games and agents are designed to offer immersive experiences, allowing players to engage in character development, story progression, and tactical or strategic gameplay~\cite{brenner_creating_2010,isbister_consistency_2000}.

The natural language capabilities of LLMs lower the technical barriers for ABMS users, allowing those without extensive programming skills to design and adjust simulations through conversational commands.
LLMs offer expanded possibilities for interactions, broadening the boundaries and applications of ABMS.
Park\etal~\cite{10.1145/3586183.3606763} presented the system that supports users defining, controlling, and intervening agents and environments by natural language commands. 
Memory Sandbox~\cite{10.1145/3586182.3615796} allows users to manage agents' memories to align with users' understandings by interface and conversations.
Leveraging the vast datasets used to train LLMs, agents can display diverse behaviors that reflect distinct characteristics, enhancing the realism and variety of simulated results.
It has introduced new challenges for ABMS, which require resolution through human-AI interactions.
For example, it is difficult to evaluate the effectiveness of the outcomes using traditional statistical metrics.
Researchers have sought to assess ABMS through methodologies grounded in human-AI interaction.
Social Simulacra~\cite{10.1145/3526113.3545616} recruited human participants to evaluate the believability of simulated behaviors by asking whether they could distinguish a conversation generated by either humans or agents.
To evaluate the ability to play a strategy game involving both cooperation and competition, Cicero~\cite{doi:10.1126/science.ade9097} participated anonymously in 40 games with humans on the website and placed first in this tournament.
To study the trends of interactive patterns between humans and AI, we collected relevant literature and introduced a novel taxonomy on interactions.
Existing work summarized the taxonomy of LLM-human interaction modes~\cite{10.1145/3613905.3650786}.
Inspired by the categorization, we decompose the human-AI interactive methods of ABMS into five dimensions: \textit{Why}, \textit{When}, \textit{What}, \textit{Who}, and \textit{How}.