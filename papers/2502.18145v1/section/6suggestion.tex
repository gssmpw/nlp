\section{Suggestions and Research Opportunities}
In this section, we present specific research opportunities identified through the findings in~\Cref{finding} using the proposed taxonomy in~\Cref{framework}.

\subsection{Maxmize the Potential of LLMs}
LLMs are becoming increasingly significant in enhancing interaction in ABMS due to their unique ability to understand and generate intricate human language.
By leveraging LLMs, users benefit from a more intuitive and effective interaction process.
Well-designed prompts can guide LLMs in better simulating human-like behaviors, producing contextually accurate responses, and performing complex tasks autonomously.
Users who lack knowledge of LLMs may struggle to phrase prompts in ways that yield the desired outcomes, which can lead to potentially confusing or unintended results.
Additionally, the complexity of ABMS can further complicate prompt formulation, as users must consider both the model's interpretive limits and the nuances of simulation parameters and agent behaviors.
For example, Generative Agents~\cite{10.1145/3586183.3606763} supports utilizing one paragraph of natural language description to define agent’s identity, including jobs and past experience.
Although such a design provides users with substantial freedom, it can lead to a dilemma where users are uncertain about what to write and may struggle to determine which information is essential to include in the prompt. 
This uncertainty can result in prompts that are either incomplete or overly detailed, diminishing the interaction's effectiveness.
Prompt engineering~\cite{giray_prompt_2023}, which is the process of carefully designing prompts to guide LLMs in generating accurate and contextually appropriate responses, has been widely developed, such as chain-of-thought~\cite{NEURIPS2022_9d560961} and tree-of-thought~\cite{NEURIPS2023_271db992} strategies.
We think the community could explore ways to help users craft effective prompts during interactions.
This research could involve developing adaptive prompt templates tailored to specific tasks, recommending contextually relevant prompts based on the user’s goals, or implementing prompt engineering techniques to refine users' inputs for better results. 
For example, when users must set an agent's identity through natural language, a template can be provided to guide users in specifying the required demographic information (\eg gender, age, occupation).
These approaches aim to reduce the learning curve associated with prompt creation, especially for users less familiar with LLMs, and improve the overall effectiveness of human-AI interactions.
%和agent相关

An increasing number of specialized fields are utilizing interactive ABMS, with LLMs simulating various human roles or professions.
However, simply employing LLMs for basic question-and-answer interactions does not effectively simulate all roles, particularly those requiring domain expertise or complex reasoning abilities.
For roles like these, a more sophisticated approach is needed to capture the depth and nuance of their knowledge and thinking processes.
One possible future direction is to design cognitive architecture for agents to simulate the human thinking processes, such as retrieving and reflecting~\cite{10.1145/3586183.3606763}.
These architectures could enable more realistic and contextually aware responses to model complex human behaviors, making them more effective in roles requiring higher expertise and adaptive decision-making.
Instruction tuning~\cite{zhang2024instructiontuninglargelanguage} is another strategy to improve the performance of LLMs by training them to follow specific types of instructions more accurately.
By fine-tuning models with instruction data specific to a field, LLMs can better understand and execute nuanced, technically complex instructions that align with domain professionals' expectations.
Instruction tunning techniques have been applied in various domains~\cite{zhang2023multitaskinstructiontuningllama,liu2023goatfinetunedllamaoutperforms}, however, there is limited research addressing it in the HCI community. %具体的问题
We hope that our research can inspire future researchers in this area.


%trust issue


\subsection{Simulation Software Development}\label{software}
Before the maturity of natural language technologies, users typically built ABMS on simulation software platforms~\cite{doi:10.1177/0037549706073695,berryman2008review}. %代表性工具,例如netlogo, agenttorch
These platforms did not support natural language interaction, requiring users to rely on more technical interfaces, which also involved a certain learning cost.
ABMS simulation platform with integrated natural language processing techniques may be required to enable users to interact with agents and control simulations using natural language commands, enhancing accessibility and ease of use. 
The platform could make ABMS more user-friendly and applicable across various domains, even for those without programming expertise.
Although there exist some platforms that enable the creation, deployment, and management of agents leveraging LLMs, such as autoGPT~\cite{Significant_Gravitas_AutoGPT}, AgentTorch~\cite{chopra2023agenttorch}, they still require users to have a certain level of programming knowledge.
It is important to design simulation software accessible to users with minimal technical expertise by incorporating natural language processing capabilities. 
In addition to implementing natural language interaction, other AI technologies could also be considered. 
For example, integrate machine learning algorithms to recommend relevant commands or next steps to users based on the user’s current actions, simulation state, or previous interaction sequences. 

ABMS is a versatile tool applied across numerous fields to simulate complex systems, analyze collective behaviors, and make predictions.
Different fields have unique design requirements for interactive ABMS platforms.
Each domain may prioritize distinct features, interaction methods, and data integration needs to meet specific goals effectively.
For example, economic simulations prioritize high-frequency interaction options, such as adjusting market parameters or agent strategies in real-time~\cite{helbing_agent-based_2012}.
While simulations in social science often need agents with complex, varied behaviors to model interactions like group dynamics, migration, or policy effects~\cite{gao2023s3socialnetworksimulationlarge}.
Developing simulation platforms for specific domains may empower professionals and researchers to address real-world challenges.
They could include agents and models prebuilt for the domain, tailor the interface and interaction options to the specific needs of the field, and offer analysis tools and visualization options that highlight metrics crucial to the domain.
Furthermore, the platform could include AI components or expert systems specific to the domain to support more realistic simulations.




\subsection{Immersive Experience}\label{immersive}%ubicomp文章
As discussed in Section~\ref{immersed}, we find that there is limited research on users' immersive experiences currently.
Popular science fiction TV series, \textit{Westworld}, set in a futuristic, highly immersive theme park populated by lifelike AI agents, which allows human guests to live out their fantasies in a Western-themed world without consequences.
As agent technology advances, the science fiction scenarios portrayed in the series are increasingly approaching reality.
Research on user immersive experience in ABMS is currently most relevant in the context of video games, such as role-playing games~(RPGs).
Värtinen\etal~\cite{c7c0852d5f324ba5907ee22bea26560c} generated role-playing game quests with LLMs to fulfill player demands toward more and richer game content.
By understanding how ABMS contributes to immersion, game developers can create environments that foster emotional investment, realistic social dynamics, and greater player satisfaction.
Additionally, insights gained may benefit other fields involving immersive environments, such as virtual reality.
Furthermore, as biotechnology and materials science advance to new levels, the concept of physical parks akin to \textit{Westworld} may become feasible.
Users would interact with physical agents through \textit{Natural Language} and \textit{Physical Movements}, creating highly immersive experiences.

Another potential application scenario is companion agents designed to provide emotional support. 
The rapid advancement of high technology has created a sense of disconnection and emotional distance, paradoxically leaving people feeling more alone despite constant virtual contact. 
Digital interactions often replace direct, face-to-face connections.
An inner emptiness or emotional void emerges, leading to a growing need for meaningful interaction and companionship.
These agents could offer companionship, simulate meaningful conversations, and respond empathetically to users' needs.
This application requires careful attention to emotional intelligence, personalization, and ethical considerations to ensure that the agents are both supportive and safe for users.
We believe that the user immersive experience in ABMS holds significant research value.
