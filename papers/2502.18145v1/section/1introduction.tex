\section{Introduction}

Agent-based modeling and simulation~(ABMS)~\cite{1574234} has long been recognized as a powerful approach for studying complex systems~\cite{gilbert2004agent} in various domains, including sociology~\cite{annurev:/content/journals/10.1146/annurev.soc.28.110601.141117,gilbert_how_2000}, economics~\cite{hamill2015agent, LENGNICK2013102}, ecology~\cite{MCLANE20111544}, and epidemiology~\cite{el-sayed_social_2012}.
ABMS is a computational approach to model complex systems composed of autonomous agents.
By allowing researchers to simulate the behaviors of individual agents within an extensive system, ABMS enables the exploration of emergent phenomena that arise from these behaviors~\cite{AN201225}.
This capability to model complex, dynamic systems has made ABMS indispensable in understanding collective behaviors and testing scenarios in environments that would be challenging or impossible to replicate in reality~\cite{heath2009}.
As artificial intelligence~(AI) continues to advance, the integration of human-AI interaction within ABMS offers significant potential to enhance ABMS's applicability~\cite{doi:10.1177/0037549706073695,berryman2008review}.
Human users interact with models to customize them based on their specific research requirements~\cite{netlogo}, such as adjusting parameters, guiding agent behaviors, or testing new scenarios to adapt models on the fly~\cite{10.1145/3613904.3642545,10.1145/3490099.3511105}.
This level of interaction opens doors to more accurate, adaptive, and user-centered simulations.


The emergence of large language models~(LLMs)~\cite{bommasani2022opportunities,brown_language_2020} has further expanded the potential of ABMS by facilitating more natural and intuitive human-AI interactions. 
Gao\etal~\cite{10.1145/3613905.3650786} have illustrated that human-LLM interaction facilitates more complex reasoning and creativity tasks.
With LLMs, users can communicate with the simulation through natural language, lowering the barrier to entry for non-experts without programming skills and enhancing the user experience.
Park\etal~\cite{10.1145/3586183.3606763} proposed Generative Agent that supports users creating agents and communicating with agents directly by natural language.
The user-friendly design has significantly advanced the interactivity of ABMS, spurring new applications and fostering interdisciplinary research~\cite{DBLP:journals/corr/abs-2312-11813,10.1145/3613904.3642159}.
%Xu\etal~\cite{DBLP:journals/corr/abs-2312-11813} built a virtual city modeling where agents can be controlled by users' standardized natural language instructions to simulate urbanization phenomenon. 
% Rehearsal~\cite{10.1145/3613904.3642159} allows users to practice handling conflicts with an agent by simulating the outcomes of different strategies.
Meanwhile, it presents new challenges for ABMS, necessitating solutions through human-AI interactions. 
One key challenge lies in evaluating the effectiveness of these outcomes, as traditional statistical metrics often fall short of capturing the complexity and nuances of agent behaviors~\cite{10.1145/3526113.3545616, doi:10.1126/science.ade9097}. 

% The combination of enhanced human-AI interactivity and the accessibility brought by LLMs has attracted a growing number of researchers from diverse fields, including human-computer interaction, artificial intelligence, and social science, to explore the potential of ABMS.
The combination of enhanced human-AI interactivity and the accessibility brought by LLMs has attracted a growing number of researchers from diverse fields, including HCI, AI, ubiquitous computing, and social science, to explore the potential of ABMS. 
This renewed interest and broadened expertise contribute to a rapidly evolving landscape, pushing the boundaries of ABMS beyond traditional applications and expanding its relevance to novel, interdisciplinary challenges.
Developers of ABMS are beginning to explore how the design of interactions can enhance ABMS to serve user research needs better.
However, designing effective human-AI interactions is not trivial.
On the one hand, the inherent complexity of ABMS itself requires interaction methods that can adapt to dynamic, often non-linear, changes within the simulation.
On the other hand, enabling effective communication between users and models is challenging, as it requires real-time feedback mechanisms that facilitate clear understanding and support decision-making.
Existing surveys on ABMS leveraging LLMs~(\eg~\cite{gao_large_2023}) have not focused on summarizing human-AI interactions.%强调没有survey
There still is a lack of systematic surveys on human-AI interactions in ABMS to provide an overview of the research landscape.
This paper seeks to address the following research question to support reflection on current research progress and future opportunities: \textit{How do humans and AI interact in the context of ABMS to fulfill user research requirements?} 
We address this question from two perspectives: what research goals users aim to achieve through interaction and how to design specific interactions once the goals are determined. %细分,以why为中心

To fill this research gap, we first collected 97 relevant studies about human-AI interactions in ABMS.
We extracted human-AI interactions from the papers in the corpus.
Our survey defines human users as researchers employing ABMS tools to conduct their studies.
We decomposed each interaction into five dimensions according to our taxonomy framework, which is derived from the "5W1H" guideline~\cite{ram_5ws_2018}.
The five dimensions are: \textit{Why}, \textit{When}, \textit{What}, \textit{Who}, \textit{How}.%解释user
\textit{Why} explains the motivations of users.
Users find it challenging to accomplish their goals with static models, making interactions essential.
\textit{When} refers to the phase at which users are involved in the simulation.
\textit{What} pertains to the components of the system under user control. Considering the simulation system's characteristics, \textit{What} encompasses three primary aspects of the model: agents, environment, and simulation configuration.
\textit{Who} represents the roles that users assume during the interaction process.
We draw an analogy from theater, where the behavior of agents within the model mirrors the actions of actors performing on stage.
\textit{How} refers to the means employed by users to interact with the model.%以why中心
By integrating five dimensions, we can comprehensively understand the design of human-AI interactions in ABMS.

The papers we examined span a wide timeframe, from 1996 to 2024, and cover multiple fields, including human-computer interaction, ubiquitous computing, natural language processing, computer vision, political science, sociology, and more.
Human-AI interactions range from scientific simulation software platforms to more flexible and diverse modes of user engagement.
Significantly, as LLMs lower the barrier to interaction, they have attracted many AI and HCI researchers to engage in related studies. 
This development expands the application scope and potential of ABMS, extending beyond merely simulating macro-level collective behaviors or phenomena.
We summarized comprehensive findings illuminating existing interaction patterns within ABMS, revealing established trends and frameworks, and identifying critical gaps in current research.
We hope our study can suggest directions for future research that can guide the developers of ABMS in developing more effective, user-centered, and versatile interactive systems.
In summary, our main contributions to the domain are as follows:
\begin{itemize}
\item We present the first comprehensive survey on human-AI interactions in agent-based modeling and simulation and introduce a novel taxonomy of interactions derived from an extensive review of existing literature.
\item We synthesize the findings from existing literature using our proposed taxonomy, which reveals interaction patterns, highlights research gaps, and suggests future research directions.
\end{itemize}