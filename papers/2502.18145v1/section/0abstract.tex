\begin{abstract}
Recent interest in human-AI interactions in agent-based modeling and simulation~(ABMS) has grown rapidly due to the widespread utilization of large language models~(LLMs).
ABMS is an intelligent approach that simulates autonomous agents' behaviors within a defined environment to research emergent phenomena.
Integrating LLMs into ABMS enables natural language interaction between humans and models. 
Meanwhile, it introduces new challenges that rely on human interaction to address.
% The integration of LLMs has substantially facilitated human-AI interactions in the context of ABMS.
Human involvement can assist ABMS in adapting to flexible and complex research demands.
However, systematic reviews of interactions that examine how humans and AI interact in ABMS are lacking.
In this paper, we investigate existing works and propose a novel taxonomy to categorize the interactions derived from them.
Specifically, human users refer to researchers who utilize ABMS tools to conduct their studies in our survey. 
We decompose interactions into five dimensions: the goals that users want to achieve~(Why), the phases that users are involved~(When), the components of the system~(What), the roles of users~(Who), and the means of interactions~(How).
Our analysis summarizes the findings that reveal existing interaction patterns.
They provide researchers who develop interactions with comprehensive guidance on how humans and AI interact.
We further discuss the unexplored interactions and suggest future research directions.
\end{abstract}

%%
%% The code below is generated by the tool at http://dl.acm.org/ccs.cfm.
%% Please copy and paste the code instead of the example below.
%%
\begin{CCSXML}
<ccs2012>
   <concept>
       <concept_id>10003120.10003121.10003129</concept_id>
       <concept_desc>Human-centered computing~Interactive systems and tools</concept_desc>
       <concept_significance>500</concept_significance>
       </concept>
 </ccs2012>
\end{CCSXML}

\ccsdesc[500]{Human-centered computing~Interactive systems and tools}



%%
%% Keywords. The author(s) should pick words that accurately describe
%% the work being presented. Separate the keywords with commas.
\keywords{agent-based modeling and simulation, human-AI interactions}

%% A "teaser" image appears between the author and affiliation
%% information and the body of the document, and typically spans the
%% page.
\begin{teaserfigure}
  \includegraphics[width=\textwidth]{figure/teaser.png}
  \caption{Scenario of an envisioned agent-based modeling and simulation composed of agents, in which humans can also participate. Human-AI interactive ABMS can be effectively explained through an analogy from the field of theater. The above image depicts agents as actors on stage, while humans can take on roles such as director, actor, observer, etc.}
  \label{fig:teaser}
\end{teaserfigure}
