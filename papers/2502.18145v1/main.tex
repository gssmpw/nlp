%%
%% This is file `sample-manuscript.tex',
%% generated with the docstrip utility.
%%
%% The original source files were:
%%
%% samples.dtx  (with options: `all,proceedings,bibtex,manuscript')
%% 
%% IMPORTANT NOTICE:
%% 
%% For the copyright see the source file.
%% 
%% Any modified versions of this file must be renamed
%% with new filenames distinct from sample-manuscript.tex.
%% 
%% For distribution of the original source see the terms
%% for copying and modification in the file samples.dtx.
%% 
%% This generated file may be distributed as long as the
%% original source files, as listed above, are part of the
%% same distribution. (The sources need not necessarily be
%% in the same archive or directory.)
%%
%%
%% Commands for TeXCount
%TC:macro \cite [option:text,text]
%TC:macro \citep [option:text,text]
%TC:macro \citet [option:text,text]
%TC:envir table 0 1
%TC:envir table* 0 1
%TC:envir tabular [ignore] word
%TC:envir displaymath 0 word
%TC:envir math 0 word
%TC:envir comment 0 0
%%
%%
%% The first command in your LaTeX source must be the \documentclass
%% command.
%%
%% For submission and review of your manuscript please change the
%% command to \documentclass[manuscript, screen, review]{acmart}.
%%
%% When submitting camera ready or to TAPS, please change the command
%% to \documentclass[sigconf]{acmart} or whichever template is required
%% for your publication.
%%
%%

\documentclass[manuscript]{acmart}
%%
%% \BibTeX command to typeset BibTeX logo in the docs
\AtBeginDocument{%
  \providecommand\BibTeX{{%
    Bib\TeX}}}

%% Rights management information.  This information is sent to you
%% when you complete the rights form.  These commands have SAMPLE
%% values in them; it is your responsibility as an author to replace
%% the commands and values with those provided to you when you
%% complete the rights form.
\setcopyright{acmlicensed}
\copyrightyear{2018}
\acmYear{2018}
\acmDOI{XXXXXXX.XXXXXXX}

%% These commands are for a PROCEEDINGS abstract or paper.
% \acmConference[Conference acronym 'XX]{Make sure to enter the correct
%   conference title from your rights confirmation email}{June 03--05,
%   2018}{Woodstock, NY}
% %%
% %%  Uncomment \acmBooktitle if the title of the proceedings is different
% %%  from ``Proceedings of ...''!
% %%
% % \acmBooktitle{Proc. ACM Interact. Mob. Wearable Ubiquitous Technol.}
% \acmISBN{978-1-4503-XXXX-X/18/06}

\graphicspath{{figs/}{figures/}{pictures/}{images/}{./}} % where to search for the images
\usepackage{cleveref}
%% Only used in the template examples. You can remove these lines.
\usepackage{tabu}                      % only used for the table example
\usepackage{booktabs}                  % only used for the table example
\usepackage{lipsum}                    % used to generate placeholder text
\usepackage{mwe}                       % used to generate placeholder figures
% \usepackage{amsfonts,amssymb}
% %% We encourage the use of mathptmx for consistent usage of times font
% %% throughout the proceedings. However, if you encounter conflicts
% %% with other math-related packages, you may want to disable it.
\usepackage{mathptmx}
\usepackage{color}
\usepackage{fancyvrb}
\usepackage{xcolor}
\usepackage{array}
\usepackage{enumitem}
\usepackage{wrapfig}
\usepackage{longtable}
\newcommand{\name}{\textit{System}}
\newcommand{\todo}[1]{\textcolor{red}{[#1]}}
\newcommand{\tocheck}[1]{\textcolor{black}{#1}}
% Abbreviations
\usepackage{xspace,xpunctuate}
\newcommand{\aka}{{a.k.a.},\xspace}
\newcommand{\ie}{\textit{i.e.},\xspace}
\newcommand{\etal}{\xspace\textit{et al.}\xspace}
\newcommand{\eg}{\textit{e.g.},\xspace}
\newcommand{\re}[1]{\textcolor{black}{#1}}
\newcommand{\minor}[1]{\textcolor{orange}{#1}}
%%
%% Submission ID.
%% Use this when submitting an article to a sponsored event. You'll
%% receive a unique submission ID from the organizers
%% of the event, and this ID should be used as the parameter to this command.
%%\acmSubmissionID{123-A56-BU3}

%%
%% For managing citations, it is recommended to use bibliography
%% files in BibTeX format.
%%
%% You can then either use BibTeX with the ACM-Reference-Format style,
%% or BibLaTeX with the acmnumeric or acmauthoryear sytles, that include
%% support for advanced citation of software artefact from the
%% biblatex-software package, also separately available on CTAN.
%%
%% Look at the sample-*-biblatex.tex files for templates showcasing
%% the biblatex styles.
%%

%%
%% The majority of ACM publications use numbered citations and
%% references.  The command \citestyle{authoryear} switches to the
%% "author year" style.
%%
%% If you are preparing content for an event
%% sponsored by ACM SIGGRAPH, you must use the "author year" style of
%% citations and references.
%% Uncommenting
%% the next command will enable that style.
%%\citestyle{acmauthoryear}


%%
%% end of the preamble, start of the body of the document source.
\begin{document}

%%
%% The "title" command has an optional parameter,
%% allowing the author to define a "short title" to be used in page headers.
\title{Carbon and Silicon, Coexist or Compete? A Survey on Human-AI Interactions in Agent-based Modeling and Simulation}

%%
%% The "author" command and its associated commands are used to define
%% the authors and their affiliations.
%% Of note is the shared affiliation of the first two authors, and the
%% "authornote" and "authornotemark" commands
%% used to denote shared contribution to the research.
\author{Ziyue Lin}
\email{ziyuelin917@gmail.com}
\orcid{0009-0002-5485-7379}
\affiliation{%
  \institution{School of Data Science, Fudan University}
   \streetaddress{1 Th{\o}rv{\"a}ld Circle}
  \city{Shanghai}
  \country{China}
}

\author{Siqi Shen}
\email{shensiqi.ssq@alibaba-inc.com}
\affiliation{%
  \institution{DataV Lab, Alibaba Group}
  \city{Hangzhou}
  \country{China}
}
\author{Zichen Cheng}
\email{zccheng19@fudan.edu.cn}
\affiliation{%
  \institution{School of Data Science, Fudan University}
  \city{Shanghai}
  \country{China}
}
\author{Cheok Lam Lai}
\email{thomaslai314159@gmail.com}
\affiliation{%
  \institution{School of Data Science, Fudan University}
  \city{Shanghai}
  \country{China}
}
\author{Siming Chen}
\authornote{Siming Chen is the corresponding author.}
\email{simingchen@fudan.edu.cn}
\orcid{0000-0002-2690-3588}
\affiliation{%
  \institution{School of Data Science, Fudan University}
  \city{Shanghai}
  \country{China}
}
\affiliation{%
  \institution{Shanghai Key Laboratory of Data Science}
  \city{Shanghai}
  \country{China}
}
%%
%% By default, the full list of authors will be used in the page
%% headers. Often, this list is too long, and will overlap
%% other information printed in the page headers. This command allows
%% the author to define a more concise list
%% of authors' names for this purpose.
\renewcommand{\shortauthors}{Lin et al.}

%%
%% The abstract is a short summary of the work to be presented in the
%% article.
\begin{abstract}
We present an image blending pipeline, \textit{IBURD}, that creates realistic synthetic images to assist in the training of deep detectors for use on underwater autonomous vehicles (AUVs) for marine debris detection tasks. 
Specifically, IBURD generates both images of underwater debris and their pixel-level annotations, using source images of debris objects, their annotations, and target background images of marine environments. 
With Poisson editing and style transfer techniques, IBURD is even able to robustly blend transparent objects into arbitrary backgrounds and automatically adjust the style of blended images using the blurriness metric of target background images. 
These generated images of marine debris in actual underwater backgrounds address the data scarcity and data variety problems faced by deep-learned vision algorithms in challenging underwater conditions, and can enable the use of AUVs for environmental cleanup missions. 
Both quantitative and robotic evaluations of IBURD demonstrate the efficacy of the proposed approach for robotic detection of marine debris. 
\end{abstract}




%%
%% This command processes the author and affiliation and title
%% information and builds the first part of the formatted document.
\maketitle
\section{Introduction}

In today’s rapidly evolving digital landscape, the transformative power of web technologies has redefined not only how services are delivered but also how complex tasks are approached. Web-based systems have become increasingly prevalent in risk control across various domains. This widespread adoption is due their accessibility, scalability, and ability to remotely connect various types of users. For example, these systems are used for process safety management in industry~\cite{kannan2016web}, safety risk early warning in urban construction~\cite{ding2013development}, and safe monitoring of infrastructural systems~\cite{repetto2018web}. Within these web-based risk management systems, the source search problem presents a huge challenge. Source search refers to the task of identifying the origin of a risky event, such as a gas leak and the emission point of toxic substances. This source search capability is crucial for effective risk management and decision-making.

Traditional approaches to implementing source search capabilities into the web systems often rely on solely algorithmic solutions~\cite{ristic2016study}. These methods, while relatively straightforward to implement, often struggle to achieve acceptable performances due to algorithmic local optima and complex unknown environments~\cite{zhao2020searching}. More recently, web crowdsourcing has emerged as a promising alternative for tackling the source search problem by incorporating human efforts in these web systems on-the-fly~\cite{zhao2024user}. This approach outsources the task of addressing issues encountered during the source search process to human workers, leveraging their capabilities to enhance system performance.

These solutions often employ a human-AI collaborative way~\cite{zhao2023leveraging} where algorithms handle exploration-exploitation and report the encountered problems while human workers resolve complex decision-making bottlenecks to help the algorithms getting rid of local deadlocks~\cite{zhao2022crowd}. Although effective, this paradigm suffers from two inherent limitations: increased operational costs from continuous human intervention, and slow response times of human workers due to sequential decision-making. These challenges motivate our investigation into developing autonomous systems that preserve human-like reasoning capabilities while reducing dependency on massive crowdsourced labor.

Furthermore, recent advancements in large language models (LLMs)~\cite{chang2024survey} and multi-modal LLMs (MLLMs)~\cite{huang2023chatgpt} have unveiled promising avenues for addressing these challenges. One clear opportunity involves the seamless integration of visual understanding and linguistic reasoning for robust decision-making in search tasks. However, whether large models-assisted source search is really effective and efficient for improving the current source search algorithms~\cite{ji2022source} remains unknown. \textit{To address the research gap, we are particularly interested in answering the following two research questions in this work:}

\textbf{\textit{RQ1: }}How can source search capabilities be integrated into web-based systems to support decision-making in time-sensitive risk management scenarios? 
% \sq{I mention ``time-sensitive'' here because I feel like we shall say something about the response time -- LLM has to be faster than humans}

\textbf{\textit{RQ2: }}How can MLLMs and LLMs enhance the effectiveness and efficiency of existing source search algorithms? 

% \textit{\textbf{RQ2:}} To what extent does the performance of large models-assisted search align with or approach the effectiveness of human-AI collaborative search? 

To answer the research questions, we propose a novel framework called Auto-\
S$^2$earch (\textbf{Auto}nomous \textbf{S}ource \textbf{Search}) and implement a prototype system that leverages advanced web technologies to simulate real-world conditions for zero-shot source search. Unlike traditional methods that rely on pre-defined heuristics or extensive human intervention, AutoS$^2$earch employs a carefully designed prompt that encapsulates human rationales, thereby guiding the MLLM to generate coherent and accurate scene descriptions from visual inputs about four directional choices. Based on these language-based descriptions, the LLM is enabled to determine the optimal directional choice through chain-of-thought (CoT) reasoning. Comprehensive empirical validation demonstrates that AutoS$^2$-\ 
earch achieves a success rate of 95–98\%, closely approaching the performance of human-AI collaborative search across 20 benchmark scenarios~\cite{zhao2023leveraging}. 

Our work indicates that the role of humans in future web crowdsourcing tasks may evolve from executors to validators or supervisors. Furthermore, incorporating explanations of LLM decisions into web-based system interfaces has the potential to help humans enhance task performance in risk control.






\section{Related Work}
\label{sec:relatedworks}

% \begin{table*}[t]
% \centering 
% \renewcommand\arraystretch{0.98}
% \fontsize{8}{10}\selectfont \setlength{\tabcolsep}{0.4em}
% \begin{tabular}{@{}lc|cc|cc|cc@{}}
% \toprule
% \textbf{Methods}           & \begin{tabular}[c]{@{}c@{}}\textbf{Training}\\ \textbf{Paradigm}\end{tabular} & \begin{tabular}[c]{@{}c@{}}\textbf{$\#$ PT Data}\\ \textbf{(Tokens)}\end{tabular} & \begin{tabular}[c]{@{}c@{}}\textbf{$\#$ IFT Data}\\ \textbf{(Samples)}\end{tabular} & \textbf{Code}  & \begin{tabular}[c]{@{}c@{}}\textbf{Natural}\\ \textbf{Language}\end{tabular} & \begin{tabular}[c]{@{}c@{}}\textbf{Action}\\ \textbf{Trajectories}\end{tabular} & \begin{tabular}[c]{@{}c@{}}\textbf{API}\\ \textbf{Documentation}\end{tabular}\\ \midrule 
% NexusRaven~\citep{srinivasan2023nexusraven} & IFT & - & - & \textcolor{green}{\CheckmarkBold} & \textcolor{green}{\CheckmarkBold} &\textcolor{red}{\XSolidBrush}&\textcolor{red}{\XSolidBrush}\\
% AgentInstruct~\citep{zeng2023agenttuning} & IFT & - & 2k & \textcolor{green}{\CheckmarkBold} & \textcolor{green}{\CheckmarkBold} &\textcolor{red}{\XSolidBrush}&\textcolor{red}{\XSolidBrush} \\
% AgentEvol~\citep{xi2024agentgym} & IFT & - & 14.5k & \textcolor{green}{\CheckmarkBold} & \textcolor{green}{\CheckmarkBold} &\textcolor{green}{\CheckmarkBold}&\textcolor{red}{\XSolidBrush} \\
% Gorilla~\citep{patil2023gorilla}& IFT & - & 16k & \textcolor{green}{\CheckmarkBold} & \textcolor{green}{\CheckmarkBold} &\textcolor{red}{\XSolidBrush}&\textcolor{green}{\CheckmarkBold}\\
% OpenFunctions-v2~\citep{patil2023gorilla} & IFT & - & 65k & \textcolor{green}{\CheckmarkBold} & \textcolor{green}{\CheckmarkBold} &\textcolor{red}{\XSolidBrush}&\textcolor{green}{\CheckmarkBold}\\
% LAM~\citep{zhang2024agentohana} & IFT & - & 42.6k & \textcolor{green}{\CheckmarkBold} & \textcolor{green}{\CheckmarkBold} &\textcolor{green}{\CheckmarkBold}&\textcolor{red}{\XSolidBrush} \\
% xLAM~\citep{liu2024apigen} & IFT & - & 60k & \textcolor{green}{\CheckmarkBold} & \textcolor{green}{\CheckmarkBold} &\textcolor{green}{\CheckmarkBold}&\textcolor{red}{\XSolidBrush} \\\midrule
% LEMUR~\citep{xu2024lemur} & PT & 90B & 300k & \textcolor{green}{\CheckmarkBold} & \textcolor{green}{\CheckmarkBold} &\textcolor{green}{\CheckmarkBold}&\textcolor{red}{\XSolidBrush}\\
% \rowcolor{teal!12} \method & PT & 103B & 95k & \textcolor{green}{\CheckmarkBold} & \textcolor{green}{\CheckmarkBold} & \textcolor{green}{\CheckmarkBold} & \textcolor{green}{\CheckmarkBold} \\
% \bottomrule
% \end{tabular}
% \caption{Summary of existing tuning- and pretraining-based LLM agents with their training sample sizes. "PT" and "IFT" denote "Pre-Training" and "Instruction Fine-Tuning", respectively. }
% \label{tab:related}
% \end{table*}

\begin{table*}[ht]
\begin{threeparttable}
\centering 
\renewcommand\arraystretch{0.98}
\fontsize{7}{9}\selectfont \setlength{\tabcolsep}{0.2em}
\begin{tabular}{@{}l|c|c|ccc|cc|cc|cccc@{}}
\toprule
\textbf{Methods} & \textbf{Datasets}           & \begin{tabular}[c]{@{}c@{}}\textbf{Training}\\ \textbf{Paradigm}\end{tabular} & \begin{tabular}[c]{@{}c@{}}\textbf{\# PT Data}\\ \textbf{(Tokens)}\end{tabular} & \begin{tabular}[c]{@{}c@{}}\textbf{\# IFT Data}\\ \textbf{(Samples)}\end{tabular} & \textbf{\# APIs} & \textbf{Code}  & \begin{tabular}[c]{@{}c@{}}\textbf{Nat.}\\ \textbf{Lang.}\end{tabular} & \begin{tabular}[c]{@{}c@{}}\textbf{Action}\\ \textbf{Traj.}\end{tabular} & \begin{tabular}[c]{@{}c@{}}\textbf{API}\\ \textbf{Doc.}\end{tabular} & \begin{tabular}[c]{@{}c@{}}\textbf{Func.}\\ \textbf{Call}\end{tabular} & \begin{tabular}[c]{@{}c@{}}\textbf{Multi.}\\ \textbf{Step}\end{tabular}  & \begin{tabular}[c]{@{}c@{}}\textbf{Plan}\\ \textbf{Refine}\end{tabular}  & \begin{tabular}[c]{@{}c@{}}\textbf{Multi.}\\ \textbf{Turn}\end{tabular}\\ \midrule 
\multicolumn{13}{l}{\emph{Instruction Finetuning-based LLM Agents for Intrinsic Reasoning}}  \\ \midrule
FireAct~\cite{chen2023fireact} & FireAct & IFT & - & 2.1K & 10 & \textcolor{red}{\XSolidBrush} &\textcolor{green}{\CheckmarkBold} &\textcolor{green}{\CheckmarkBold}  & \textcolor{red}{\XSolidBrush} &\textcolor{green}{\CheckmarkBold} & \textcolor{red}{\XSolidBrush} &\textcolor{green}{\CheckmarkBold} & \textcolor{red}{\XSolidBrush} \\
ToolAlpaca~\cite{tang2023toolalpaca} & ToolAlpaca & IFT & - & 4.0K & 400 & \textcolor{red}{\XSolidBrush} &\textcolor{green}{\CheckmarkBold} &\textcolor{green}{\CheckmarkBold} & \textcolor{red}{\XSolidBrush} &\textcolor{green}{\CheckmarkBold} & \textcolor{red}{\XSolidBrush}  &\textcolor{green}{\CheckmarkBold} & \textcolor{red}{\XSolidBrush}  \\
ToolLLaMA~\cite{qin2023toolllm} & ToolBench & IFT & - & 12.7K & 16,464 & \textcolor{red}{\XSolidBrush} &\textcolor{green}{\CheckmarkBold} &\textcolor{green}{\CheckmarkBold} &\textcolor{red}{\XSolidBrush} &\textcolor{green}{\CheckmarkBold}&\textcolor{green}{\CheckmarkBold}&\textcolor{green}{\CheckmarkBold} &\textcolor{green}{\CheckmarkBold}\\
AgentEvol~\citep{xi2024agentgym} & AgentTraj-L & IFT & - & 14.5K & 24 &\textcolor{red}{\XSolidBrush} & \textcolor{green}{\CheckmarkBold} &\textcolor{green}{\CheckmarkBold}&\textcolor{red}{\XSolidBrush} &\textcolor{green}{\CheckmarkBold}&\textcolor{red}{\XSolidBrush} &\textcolor{red}{\XSolidBrush} &\textcolor{green}{\CheckmarkBold}\\
Lumos~\cite{yin2024agent} & Lumos & IFT  & - & 20.0K & 16 &\textcolor{red}{\XSolidBrush} & \textcolor{green}{\CheckmarkBold} & \textcolor{green}{\CheckmarkBold} &\textcolor{red}{\XSolidBrush} & \textcolor{green}{\CheckmarkBold} & \textcolor{green}{\CheckmarkBold} &\textcolor{red}{\XSolidBrush} & \textcolor{green}{\CheckmarkBold}\\
Agent-FLAN~\cite{chen2024agent} & Agent-FLAN & IFT & - & 24.7K & 20 &\textcolor{red}{\XSolidBrush} & \textcolor{green}{\CheckmarkBold} & \textcolor{green}{\CheckmarkBold} &\textcolor{red}{\XSolidBrush} & \textcolor{green}{\CheckmarkBold}& \textcolor{green}{\CheckmarkBold}&\textcolor{red}{\XSolidBrush} & \textcolor{green}{\CheckmarkBold}\\
AgentTuning~\citep{zeng2023agenttuning} & AgentInstruct & IFT & - & 35.0K & - &\textcolor{red}{\XSolidBrush} & \textcolor{green}{\CheckmarkBold} & \textcolor{green}{\CheckmarkBold} &\textcolor{red}{\XSolidBrush} & \textcolor{green}{\CheckmarkBold} &\textcolor{red}{\XSolidBrush} &\textcolor{red}{\XSolidBrush} & \textcolor{green}{\CheckmarkBold}\\\midrule
\multicolumn{13}{l}{\emph{Instruction Finetuning-based LLM Agents for Function Calling}} \\\midrule
NexusRaven~\citep{srinivasan2023nexusraven} & NexusRaven & IFT & - & - & 116 & \textcolor{green}{\CheckmarkBold} & \textcolor{green}{\CheckmarkBold}  & \textcolor{green}{\CheckmarkBold} &\textcolor{red}{\XSolidBrush} & \textcolor{green}{\CheckmarkBold} &\textcolor{red}{\XSolidBrush} &\textcolor{red}{\XSolidBrush}&\textcolor{red}{\XSolidBrush}\\
Gorilla~\citep{patil2023gorilla} & Gorilla & IFT & - & 16.0K & 1,645 & \textcolor{green}{\CheckmarkBold} &\textcolor{red}{\XSolidBrush} &\textcolor{red}{\XSolidBrush}&\textcolor{green}{\CheckmarkBold} &\textcolor{green}{\CheckmarkBold} &\textcolor{red}{\XSolidBrush} &\textcolor{red}{\XSolidBrush} &\textcolor{red}{\XSolidBrush}\\
OpenFunctions-v2~\citep{patil2023gorilla} & OpenFunctions-v2 & IFT & - & 65.0K & - & \textcolor{green}{\CheckmarkBold} & \textcolor{green}{\CheckmarkBold} &\textcolor{red}{\XSolidBrush} &\textcolor{green}{\CheckmarkBold} &\textcolor{green}{\CheckmarkBold} &\textcolor{red}{\XSolidBrush} &\textcolor{red}{\XSolidBrush} &\textcolor{red}{\XSolidBrush}\\
API Pack~\cite{guo2024api} & API Pack & IFT & - & 1.1M & 11,213 &\textcolor{green}{\CheckmarkBold} &\textcolor{red}{\XSolidBrush} &\textcolor{green}{\CheckmarkBold} &\textcolor{red}{\XSolidBrush} &\textcolor{green}{\CheckmarkBold} &\textcolor{red}{\XSolidBrush}&\textcolor{red}{\XSolidBrush}&\textcolor{red}{\XSolidBrush}\\ 
LAM~\citep{zhang2024agentohana} & AgentOhana & IFT & - & 42.6K & - & \textcolor{green}{\CheckmarkBold} & \textcolor{green}{\CheckmarkBold} &\textcolor{green}{\CheckmarkBold}&\textcolor{red}{\XSolidBrush} &\textcolor{green}{\CheckmarkBold}&\textcolor{red}{\XSolidBrush}&\textcolor{green}{\CheckmarkBold}&\textcolor{green}{\CheckmarkBold}\\
xLAM~\citep{liu2024apigen} & APIGen & IFT & - & 60.0K & 3,673 & \textcolor{green}{\CheckmarkBold} & \textcolor{green}{\CheckmarkBold} &\textcolor{green}{\CheckmarkBold}&\textcolor{red}{\XSolidBrush} &\textcolor{green}{\CheckmarkBold}&\textcolor{red}{\XSolidBrush}&\textcolor{green}{\CheckmarkBold}&\textcolor{green}{\CheckmarkBold}\\\midrule
\multicolumn{13}{l}{\emph{Pretraining-based LLM Agents}}  \\\midrule
% LEMUR~\citep{xu2024lemur} & PT & 90B & 300.0K & - & \textcolor{green}{\CheckmarkBold} & \textcolor{green}{\CheckmarkBold} &\textcolor{green}{\CheckmarkBold}&\textcolor{red}{\XSolidBrush} & \textcolor{red}{\XSolidBrush} &\textcolor{green}{\CheckmarkBold} &\textcolor{red}{\XSolidBrush}&\textcolor{red}{\XSolidBrush}\\
\rowcolor{teal!12} \method & \dataset & PT & 103B & 95.0K  & 76,537  & \textcolor{green}{\CheckmarkBold} & \textcolor{green}{\CheckmarkBold} & \textcolor{green}{\CheckmarkBold} & \textcolor{green}{\CheckmarkBold} & \textcolor{green}{\CheckmarkBold} & \textcolor{green}{\CheckmarkBold} & \textcolor{green}{\CheckmarkBold} & \textcolor{green}{\CheckmarkBold}\\
\bottomrule
\end{tabular}
% \begin{tablenotes}
%     \item $^*$ In addition, the StarCoder-API can offer 4.77M more APIs.
% \end{tablenotes}
\caption{Summary of existing instruction finetuning-based LLM agents for intrinsic reasoning and function calling, along with their training resources and sample sizes. "PT" and "IFT" denote "Pre-Training" and "Instruction Fine-Tuning", respectively.}
\vspace{-2ex}
\label{tab:related}
\end{threeparttable}
\end{table*}

\noindent \textbf{Prompting-based LLM Agents.} Due to the lack of agent-specific pre-training corpus, existing LLM agents rely on either prompt engineering~\cite{hsieh2023tool,lu2024chameleon,yao2022react,wang2023voyager} or instruction fine-tuning~\cite{chen2023fireact,zeng2023agenttuning} to understand human instructions, decompose high-level tasks, generate grounded plans, and execute multi-step actions. 
However, prompting-based methods mainly depend on the capabilities of backbone LLMs (usually commercial LLMs), failing to introduce new knowledge and struggling to generalize to unseen tasks~\cite{sun2024adaplanner,zhuang2023toolchain}. 

\noindent \textbf{Instruction Finetuning-based LLM Agents.} Considering the extensive diversity of APIs and the complexity of multi-tool instructions, tool learning inherently presents greater challenges than natural language tasks, such as text generation~\cite{qin2023toolllm}.
Post-training techniques focus more on instruction following and aligning output with specific formats~\cite{patil2023gorilla,hao2024toolkengpt,qin2023toolllm,schick2024toolformer}, rather than fundamentally improving model knowledge or capabilities. 
Moreover, heavy fine-tuning can hinder generalization or even degrade performance in non-agent use cases, potentially suppressing the original base model capabilities~\cite{ghosh2024a}.

\noindent \textbf{Pretraining-based LLM Agents.} While pre-training serves as an essential alternative, prior works~\cite{nijkamp2023codegen,roziere2023code,xu2024lemur,patil2023gorilla} have primarily focused on improving task-specific capabilities (\eg, code generation) instead of general-domain LLM agents, due to single-source, uni-type, small-scale, and poor-quality pre-training data. 
Existing tool documentation data for agent training either lacks diverse real-world APIs~\cite{patil2023gorilla, tang2023toolalpaca} or is constrained to single-tool or single-round tool execution. 
Furthermore, trajectory data mostly imitate expert behavior or follow function-calling rules with inferior planning and reasoning, failing to fully elicit LLMs' capabilities and handle complex instructions~\cite{qin2023toolllm}. 
Given a wide range of candidate API functions, each comprising various function names and parameters available at every planning step, identifying globally optimal solutions and generalizing across tasks remains highly challenging.



\section{Preliminaries}
\label{Preliminaries}
\begin{figure*}[t]
    \centering
    \includegraphics[width=0.95\linewidth]{fig/HealthGPT_Framework.png}
    \caption{The \ourmethod{} architecture integrates hierarchical visual perception and H-LoRA, employing a task-specific hard router to select visual features and H-LoRA plugins, ultimately generating outputs with an autoregressive manner.}
    \label{fig:architecture}
\end{figure*}
\noindent\textbf{Large Vision-Language Models.} 
The input to a LVLM typically consists of an image $x^{\text{img}}$ and a discrete text sequence $x^{\text{txt}}$. The visual encoder $\mathcal{E}^{\text{img}}$ converts the input image $x^{\text{img}}$ into a sequence of visual tokens $\mathcal{V} = [v_i]_{i=1}^{N_v}$, while the text sequence $x^{\text{txt}}$ is mapped into a sequence of text tokens $\mathcal{T} = [t_i]_{i=1}^{N_t}$ using an embedding function $\mathcal{E}^{\text{txt}}$. The LLM $\mathcal{M_\text{LLM}}(\cdot|\theta)$ models the joint probability of the token sequence $\mathcal{U} = \{\mathcal{V},\mathcal{T}\}$, which is expressed as:
\begin{equation}
    P_\theta(R | \mathcal{U}) = \prod_{i=1}^{N_r} P_\theta(r_i | \{\mathcal{U}, r_{<i}\}),
\end{equation}
where $R = [r_i]_{i=1}^{N_r}$ is the text response sequence. The LVLM iteratively generates the next token $r_i$ based on $r_{<i}$. The optimization objective is to minimize the cross-entropy loss of the response $\mathcal{R}$.
% \begin{equation}
%     \mathcal{L}_{\text{VLM}} = \mathbb{E}_{R|\mathcal{U}}\left[-\log P_\theta(R | \mathcal{U})\right]
% \end{equation}
It is worth noting that most LVLMs adopt a design paradigm based on ViT, alignment adapters, and pre-trained LLMs\cite{liu2023llava,liu2024improved}, enabling quick adaptation to downstream tasks.


\noindent\textbf{VQGAN.}
VQGAN~\cite{esser2021taming} employs latent space compression and indexing mechanisms to effectively learn a complete discrete representation of images. VQGAN first maps the input image $x^{\text{img}}$ to a latent representation $z = \mathcal{E}(x)$ through a encoder $\mathcal{E}$. Then, the latent representation is quantized using a codebook $\mathcal{Z} = \{z_k\}_{k=1}^K$, generating a discrete index sequence $\mathcal{I} = [i_m]_{m=1}^N$, where $i_m \in \mathcal{Z}$ represents the quantized code index:
\begin{equation}
    \mathcal{I} = \text{Quantize}(z|\mathcal{Z}) = \arg\min_{z_k \in \mathcal{Z}} \| z - z_k \|_2.
\end{equation}
In our approach, the discrete index sequence $\mathcal{I}$ serves as a supervisory signal for the generation task, enabling the model to predict the index sequence $\hat{\mathcal{I}}$ from input conditions such as text or other modality signals.  
Finally, the predicted index sequence $\hat{\mathcal{I}}$ is upsampled by the VQGAN decoder $G$, generating the high-quality image $\hat{x}^\text{img} = G(\hat{\mathcal{I}})$.



\noindent\textbf{Low Rank Adaptation.} 
LoRA\cite{hu2021lora} effectively captures the characteristics of downstream tasks by introducing low-rank adapters. The core idea is to decompose the bypass weight matrix $\Delta W\in\mathbb{R}^{d^{\text{in}} \times d^{\text{out}}}$ into two low-rank matrices $ \{A \in \mathbb{R}^{d^{\text{in}} \times r}, B \in \mathbb{R}^{r \times d^{\text{out}}} \}$, where $ r \ll \min\{d^{\text{in}}, d^{\text{out}}\} $, significantly reducing learnable parameters. The output with the LoRA adapter for the input $x$ is then given by:
\begin{equation}
    h = x W_0 + \alpha x \Delta W/r = x W_0 + \alpha xAB/r,
\end{equation}
where matrix $ A $ is initialized with a Gaussian distribution, while the matrix $ B $ is initialized as a zero matrix. The scaling factor $ \alpha/r $ controls the impact of $ \Delta W $ on the model.

\section{HealthGPT}
\label{Method}


\subsection{Unified Autoregressive Generation.}  
% As shown in Figure~\ref{fig:architecture}, 
\ourmethod{} (Figure~\ref{fig:architecture}) utilizes a discrete token representation that covers both text and visual outputs, unifying visual comprehension and generation as an autoregressive task. 
For comprehension, $\mathcal{M}_\text{llm}$ receives the input joint sequence $\mathcal{U}$ and outputs a series of text token $\mathcal{R} = [r_1, r_2, \dots, r_{N_r}]$, where $r_i \in \mathcal{V}_{\text{txt}}$, and $\mathcal{V}_{\text{txt}}$ represents the LLM's vocabulary:
\begin{equation}
    P_\theta(\mathcal{R} \mid \mathcal{U}) = \prod_{i=1}^{N_r} P_\theta(r_i \mid \mathcal{U}, r_{<i}).
\end{equation}
For generation, $\mathcal{M}_\text{llm}$ first receives a special start token $\langle \text{START\_IMG} \rangle$, then generates a series of tokens corresponding to the VQGAN indices $\mathcal{I} = [i_1, i_2, \dots, i_{N_i}]$, where $i_j \in \mathcal{V}_{\text{vq}}$, and $\mathcal{V}_{\text{vq}}$ represents the index range of VQGAN. Upon completion of generation, the LLM outputs an end token $\langle \text{END\_IMG} \rangle$:
\begin{equation}
    P_\theta(\mathcal{I} \mid \mathcal{U}) = \prod_{j=1}^{N_i} P_\theta(i_j \mid \mathcal{U}, i_{<j}).
\end{equation}
Finally, the generated index sequence $\mathcal{I}$ is fed into the decoder $G$, which reconstructs the target image $\hat{x}^{\text{img}} = G(\mathcal{I})$.

\subsection{Hierarchical Visual Perception}  
Given the differences in visual perception between comprehension and generation tasks—where the former focuses on abstract semantics and the latter emphasizes complete semantics—we employ ViT to compress the image into discrete visual tokens at multiple hierarchical levels.
Specifically, the image is converted into a series of features $\{f_1, f_2, \dots, f_L\}$ as it passes through $L$ ViT blocks.

To address the needs of various tasks, the hidden states are divided into two types: (i) \textit{Concrete-grained features} $\mathcal{F}^{\text{Con}} = \{f_1, f_2, \dots, f_k\}, k < L$, derived from the shallower layers of ViT, containing sufficient global features, suitable for generation tasks; 
(ii) \textit{Abstract-grained features} $\mathcal{F}^{\text{Abs}} = \{f_{k+1}, f_{k+2}, \dots, f_L\}$, derived from the deeper layers of ViT, which contain abstract semantic information closer to the text space, suitable for comprehension tasks.

The task type $T$ (comprehension or generation) determines which set of features is selected as the input for the downstream large language model:
\begin{equation}
    \mathcal{F}^{\text{img}}_T =
    \begin{cases}
        \mathcal{F}^{\text{Con}}, & \text{if } T = \text{generation task} \\
        \mathcal{F}^{\text{Abs}}, & \text{if } T = \text{comprehension task}
    \end{cases}
\end{equation}
We integrate the image features $\mathcal{F}^{\text{img}}_T$ and text features $\mathcal{T}$ into a joint sequence through simple concatenation, which is then fed into the LLM $\mathcal{M}_{\text{llm}}$ for autoregressive generation.
% :
% \begin{equation}
%     \mathcal{R} = \mathcal{M}_{\text{llm}}(\mathcal{U}|\theta), \quad \mathcal{U} = [\mathcal{F}^{\text{img}}_T; \mathcal{T}]
% \end{equation}
\subsection{Heterogeneous Knowledge Adaptation}
We devise H-LoRA, which stores heterogeneous knowledge from comprehension and generation tasks in separate modules and dynamically routes to extract task-relevant knowledge from these modules. 
At the task level, for each task type $ T $, we dynamically assign a dedicated H-LoRA submodule $ \theta^T $, which is expressed as:
\begin{equation}
    \mathcal{R} = \mathcal{M}_\text{LLM}(\mathcal{U}|\theta, \theta^T), \quad \theta^T = \{A^T, B^T, \mathcal{R}^T_\text{outer}\}.
\end{equation}
At the feature level for a single task, H-LoRA integrates the idea of Mixture of Experts (MoE)~\cite{masoudnia2014mixture} and designs an efficient matrix merging and routing weight allocation mechanism, thus avoiding the significant computational delay introduced by matrix splitting in existing MoELoRA~\cite{luo2024moelora}. Specifically, we first merge the low-rank matrices (rank = r) of $ k $ LoRA experts into a unified matrix:
\begin{equation}
    \mathbf{A}^{\text{merged}}, \mathbf{B}^{\text{merged}} = \text{Concat}(\{A_i\}_1^k), \text{Concat}(\{B_i\}_1^k),
\end{equation}
where $ \mathbf{A}^{\text{merged}} \in \mathbb{R}^{d^\text{in} \times rk} $ and $ \mathbf{B}^{\text{merged}} \in \mathbb{R}^{rk \times d^\text{out}} $. The $k$-dimension routing layer generates expert weights $ \mathcal{W} \in \mathbb{R}^{\text{token\_num} \times k} $ based on the input hidden state $ x $, and these are expanded to $ \mathbb{R}^{\text{token\_num} \times rk} $ as follows:
\begin{equation}
    \mathcal{W}^\text{expanded} = \alpha k \mathcal{W} / r \otimes \mathbf{1}_r,
\end{equation}
where $ \otimes $ denotes the replication operation.
The overall output of H-LoRA is computed as:
\begin{equation}
    \mathcal{O}^\text{H-LoRA} = (x \mathbf{A}^{\text{merged}} \odot \mathcal{W}^\text{expanded}) \mathbf{B}^{\text{merged}},
\end{equation}
where $ \odot $ represents element-wise multiplication. Finally, the output of H-LoRA is added to the frozen pre-trained weights to produce the final output:
\begin{equation}
    \mathcal{O} = x W_0 + \mathcal{O}^\text{H-LoRA}.
\end{equation}
% In summary, H-LoRA is a task-based dynamic PEFT method that achieves high efficiency in single-task fine-tuning.

\subsection{Training Pipeline}

\begin{figure}[t]
    \centering
    \hspace{-4mm}
    \includegraphics[width=0.94\linewidth]{fig/data.pdf}
    \caption{Data statistics of \texttt{VL-Health}. }
    \label{fig:data}
\end{figure}
\noindent \textbf{1st Stage: Multi-modal Alignment.} 
In the first stage, we design separate visual adapters and H-LoRA submodules for medical unified tasks. For the medical comprehension task, we train abstract-grained visual adapters using high-quality image-text pairs to align visual embeddings with textual embeddings, thereby enabling the model to accurately describe medical visual content. During this process, the pre-trained LLM and its corresponding H-LoRA submodules remain frozen. In contrast, the medical generation task requires training concrete-grained adapters and H-LoRA submodules while keeping the LLM frozen. Meanwhile, we extend the textual vocabulary to include multimodal tokens, enabling the support of additional VQGAN vector quantization indices. The model trains on image-VQ pairs, endowing the pre-trained LLM with the capability for image reconstruction. This design ensures pixel-level consistency of pre- and post-LVLM. The processes establish the initial alignment between the LLM’s outputs and the visual inputs.

\noindent \textbf{2nd Stage: Heterogeneous H-LoRA Plugin Adaptation.}  
The submodules of H-LoRA share the word embedding layer and output head but may encounter issues such as bias and scale inconsistencies during training across different tasks. To ensure that the multiple H-LoRA plugins seamlessly interface with the LLMs and form a unified base, we fine-tune the word embedding layer and output head using a small amount of mixed data to maintain consistency in the model weights. Specifically, during this stage, all H-LoRA submodules for different tasks are kept frozen, with only the word embedding layer and output head being optimized. Through this stage, the model accumulates foundational knowledge for unified tasks by adapting H-LoRA plugins.

\begin{table*}[!t]
\centering
\caption{Comparison of \ourmethod{} with other LVLMs and unified multi-modal models on medical visual comprehension tasks. \textbf{Bold} and \underline{underlined} text indicates the best performance and second-best performance, respectively.}
\resizebox{\textwidth}{!}{
\begin{tabular}{c|lcc|cccccccc|c}
\toprule
\rowcolor[HTML]{E9F3FE} &  &  &  & \multicolumn{2}{c}{\textbf{VQA-RAD \textuparrow}} & \multicolumn{2}{c}{\textbf{SLAKE \textuparrow}} & \multicolumn{2}{c}{\textbf{PathVQA \textuparrow}} &  &  &  \\ 
\cline{5-10}
\rowcolor[HTML]{E9F3FE}\multirow{-2}{*}{\textbf{Type}} & \multirow{-2}{*}{\textbf{Model}} & \multirow{-2}{*}{\textbf{\# Params}} & \multirow{-2}{*}{\makecell{\textbf{Medical} \\ \textbf{LVLM}}} & \textbf{close} & \textbf{all} & \textbf{close} & \textbf{all} & \textbf{close} & \textbf{all} & \multirow{-2}{*}{\makecell{\textbf{MMMU} \\ \textbf{-Med}}\textuparrow} & \multirow{-2}{*}{\textbf{OMVQA}\textuparrow} & \multirow{-2}{*}{\textbf{Avg. \textuparrow}} \\ 
\midrule \midrule
\multirow{9}{*}{\textbf{Comp. Only}} 
& Med-Flamingo & 8.3B & \Large \ding{51} & 58.6 & 43.0 & 47.0 & 25.5 & 61.9 & 31.3 & 28.7 & 34.9 & 41.4 \\
& LLaVA-Med & 7B & \Large \ding{51} & 60.2 & 48.1 & 58.4 & 44.8 & 62.3 & 35.7 & 30.0 & 41.3 & 47.6 \\
& HuatuoGPT-Vision & 7B & \Large \ding{51} & 66.9 & 53.0 & 59.8 & 49.1 & 52.9 & 32.0 & 42.0 & 50.0 & 50.7 \\
& BLIP-2 & 6.7B & \Large \ding{55} & 43.4 & 36.8 & 41.6 & 35.3 & 48.5 & 28.8 & 27.3 & 26.9 & 36.1 \\
& LLaVA-v1.5 & 7B & \Large \ding{55} & 51.8 & 42.8 & 37.1 & 37.7 & 53.5 & 31.4 & 32.7 & 44.7 & 41.5 \\
& InstructBLIP & 7B & \Large \ding{55} & 61.0 & 44.8 & 66.8 & 43.3 & 56.0 & 32.3 & 25.3 & 29.0 & 44.8 \\
& Yi-VL & 6B & \Large \ding{55} & 52.6 & 42.1 & 52.4 & 38.4 & 54.9 & 30.9 & 38.0 & 50.2 & 44.9 \\
& InternVL2 & 8B & \Large \ding{55} & 64.9 & 49.0 & 66.6 & 50.1 & 60.0 & 31.9 & \underline{43.3} & 54.5 & 52.5\\
& Llama-3.2 & 11B & \Large \ding{55} & 68.9 & 45.5 & 72.4 & 52.1 & 62.8 & 33.6 & 39.3 & 63.2 & 54.7 \\
\midrule
\multirow{5}{*}{\textbf{Comp. \& Gen.}} 
& Show-o & 1.3B & \Large \ding{55} & 50.6 & 33.9 & 31.5 & 17.9 & 52.9 & 28.2 & 22.7 & 45.7 & 42.6 \\
& Unified-IO 2 & 7B & \Large \ding{55} & 46.2 & 32.6 & 35.9 & 21.9 & 52.5 & 27.0 & 25.3 & 33.0 & 33.8 \\
& Janus & 1.3B & \Large \ding{55} & 70.9 & 52.8 & 34.7 & 26.9 & 51.9 & 27.9 & 30.0 & 26.8 & 33.5 \\
& \cellcolor[HTML]{DAE0FB}HealthGPT-M3 & \cellcolor[HTML]{DAE0FB}3.8B & \cellcolor[HTML]{DAE0FB}\Large \ding{51} & \cellcolor[HTML]{DAE0FB}\underline{73.7} & \cellcolor[HTML]{DAE0FB}\underline{55.9} & \cellcolor[HTML]{DAE0FB}\underline{74.6} & \cellcolor[HTML]{DAE0FB}\underline{56.4} & \cellcolor[HTML]{DAE0FB}\underline{78.7} & \cellcolor[HTML]{DAE0FB}\underline{39.7} & \cellcolor[HTML]{DAE0FB}\underline{43.3} & \cellcolor[HTML]{DAE0FB}\underline{68.5} & \cellcolor[HTML]{DAE0FB}\underline{61.3} \\
& \cellcolor[HTML]{DAE0FB}HealthGPT-L14 & \cellcolor[HTML]{DAE0FB}14B & \cellcolor[HTML]{DAE0FB}\Large \ding{51} & \cellcolor[HTML]{DAE0FB}\textbf{77.7} & \cellcolor[HTML]{DAE0FB}\textbf{58.3} & \cellcolor[HTML]{DAE0FB}\textbf{76.4} & \cellcolor[HTML]{DAE0FB}\textbf{64.5} & \cellcolor[HTML]{DAE0FB}\textbf{85.9} & \cellcolor[HTML]{DAE0FB}\textbf{44.4} & \cellcolor[HTML]{DAE0FB}\textbf{49.2} & \cellcolor[HTML]{DAE0FB}\textbf{74.4} & \cellcolor[HTML]{DAE0FB}\textbf{66.4} \\
\bottomrule
\end{tabular}
}
\label{tab:results}
\end{table*}
\begin{table*}[ht]
    \centering
    \caption{The experimental results for the four modality conversion tasks.}
    \resizebox{\textwidth}{!}{
    \begin{tabular}{l|ccc|ccc|ccc|ccc}
        \toprule
        \rowcolor[HTML]{E9F3FE} & \multicolumn{3}{c}{\textbf{CT to MRI (Brain)}} & \multicolumn{3}{c}{\textbf{CT to MRI (Pelvis)}} & \multicolumn{3}{c}{\textbf{MRI to CT (Brain)}} & \multicolumn{3}{c}{\textbf{MRI to CT (Pelvis)}} \\
        \cline{2-13}
        \rowcolor[HTML]{E9F3FE}\multirow{-2}{*}{\textbf{Model}}& \textbf{SSIM $\uparrow$} & \textbf{PSNR $\uparrow$} & \textbf{MSE $\downarrow$} & \textbf{SSIM $\uparrow$} & \textbf{PSNR $\uparrow$} & \textbf{MSE $\downarrow$} & \textbf{SSIM $\uparrow$} & \textbf{PSNR $\uparrow$} & \textbf{MSE $\downarrow$} & \textbf{SSIM $\uparrow$} & \textbf{PSNR $\uparrow$} & \textbf{MSE $\downarrow$} \\
        \midrule \midrule
        pix2pix & 71.09 & 32.65 & 36.85 & 59.17 & 31.02 & 51.91 & 78.79 & 33.85 & 28.33 & 72.31 & 32.98 & 36.19 \\
        CycleGAN & 54.76 & 32.23 & 40.56 & 54.54 & 30.77 & 55.00 & 63.75 & 31.02 & 52.78 & 50.54 & 29.89 & 67.78 \\
        BBDM & {71.69} & {32.91} & {34.44} & 57.37 & 31.37 & 48.06 & \textbf{86.40} & 34.12 & 26.61 & {79.26} & 33.15 & 33.60 \\
        Vmanba & 69.54 & 32.67 & 36.42 & {63.01} & {31.47} & {46.99} & 79.63 & 34.12 & 26.49 & 77.45 & 33.53 & 31.85 \\
        DiffMa & 71.47 & 32.74 & 35.77 & 62.56 & 31.43 & 47.38 & 79.00 & {34.13} & {26.45} & 78.53 & {33.68} & {30.51} \\
        \rowcolor[HTML]{DAE0FB}HealthGPT-M3 & \underline{79.38} & \underline{33.03} & \underline{33.48} & \underline{71.81} & \underline{31.83} & \underline{43.45} & {85.06} & \textbf{34.40} & \textbf{25.49} & \underline{84.23} & \textbf{34.29} & \textbf{27.99} \\
        \rowcolor[HTML]{DAE0FB}HealthGPT-L14 & \textbf{79.73} & \textbf{33.10} & \textbf{32.96} & \textbf{71.92} & \textbf{31.87} & \textbf{43.09} & \underline{85.31} & \underline{34.29} & \underline{26.20} & \textbf{84.96} & \underline{34.14} & \underline{28.13} \\
        \bottomrule
    \end{tabular}
    }
    \label{tab:conversion}
\end{table*}

\noindent \textbf{3rd Stage: Visual Instruction Fine-Tuning.}  
In the third stage, we introduce additional task-specific data to further optimize the model and enhance its adaptability to downstream tasks such as medical visual comprehension (e.g., medical QA, medical dialogues, and report generation) or generation tasks (e.g., super-resolution, denoising, and modality conversion). Notably, by this stage, the word embedding layer and output head have been fine-tuned, only the H-LoRA modules and adapter modules need to be trained. This strategy significantly improves the model's adaptability and flexibility across different tasks.


\section{Atlas Framework}\label{sec:framework}

\atlas introduces two core components to the ML lifecycle:
\begin{enumerate*}[label=\arabic*)]
    \item the transparency service interacting with MLaaS providers;
    \item the verification service that model users and verifiers use to verify integrity and provenance.
\end{enumerate*}
Fig.~\ref{fig:atlas-arch} depicts an example ML model lifecycle with \atlas.

\begin{figure}[t]
    \centering
    \includegraphics[width=\columnwidth]{./img/atlas-arch.pdf}
    \caption{
	        Example ML lifecycle with \atlas.
	        Striped boxes and bolded arrows represent \atlas components and data flows.
	        MLaaS providers obtain model artifacts (datasets or models) and perform transformations on them.
	        \atlas attestation clients generate and submit digitally signed artifact and transformation metadata to a transparency log.
	        Model users commonly delegate model provenance checking to an \atlas verifier.
	    }
    \label{fig:atlas-arch}
\end{figure}

\subsection{Transparency Service}\label{sec:framework:transparency}

To provide transparency across stages of the ML lifecycle, MLaaS providers
integrate an \atlas attestation client in their system.
The transparency log makes all attestation clients' collected metadata available
to verifiers.
The four core techniques underlying the \atlas transparency service are designed
to be general, allowing them to remain agnostic to the particular ML lifecycle
stage or pipeline they are applied to (\textbf{R6}).

\subsubsection{Artifact Measurements}\label{sec:framework:transparency:artifact}

The \atlas attestation client within an MLaaS provider cryptographically
measures every artifact ingested into and output by an ML pipeline.
That is, we use a collision resistant cryptographic hash function to compute
unique, immutable measurement values for every artifact.
Because any changes to a given artifact result in a different hash value, \atlas
verifiers wishing to check whether a given model artifact matches their
expectation are able to detect tampering between stages of the lifecyle (\textbf{R1}).

\subsubsection{Model Transformation Integrity}\label{sec:framework:transparency:transformation}

In addition to capturing measurements about all inputs and outputs of an ML
pipeline, the \atlas attestation client continuously verifies new inputs and
monitors an ML system's execution to collect detailed information about pipeline
state and operations that transform the input artifacts.
That is, during data processing, the client tracks dataset modifications and
preprocessing operations; during model training, \atlas captures state changes
in model weights, hyperparameters, and configurations.

To enhance the integrity of the ML artifacts and system at runtime,
\atlas runs ML pipeline code within trusted execution environments (TEEs) (e.g.,
Intel TDX~\cite{tdx} or AMD SEV-SNP~\cite{amd-sev}), which serve as a hardware-based
root of trust for ML system component measurements extending from its hardware
to the firmware and software.

These measures allow the \atlas attestation client to verify the integrity
and authenticity of the compute environment before beginning the execution of
the pipeline.
Throughout execution, the TEE maintains isolated memory regions, reducing the
risk of interference with the pipeline, as well as unauthorized disclosure
of artifacts and pipeline code, by a compromised MLaaS provider
(\textbf{R4} \& \textbf{R5}).

At the conclusion of an operation in a given pipeline, the client generates a
record with all collected ML system information, and digitally signs it with a
TEE-generated key along with all artifact measurements, providing a model
\emph{transformation attestation} (\textbf{R2}).
If required by the artifact producer, \atlas may encrypt the produced model
artifacts prior to digitally signing and uploading them to a given hub to
further enhance artifact authenticity and confidentiality while in transit and
at rest (\textbf{R1} \& \textbf{R5}).

%\subsubsection{Security Components}
%Our security architecture implements custom operators for metadata generation and verification,
%coupled with integrity verification controllers that continuously monitor pipeline execution.
%The system employs transparent logging middleware to track all operations, while hardware-backed
%attestation services integrate directly with the pipeline stages. These components work together to enable
%security-enhanced workflow execution where each stage's artifacts are automatically signed and verified.

\subsubsection{Provenance Chains}\label{sec:framework:transparency:provenance}

To allow verifiers to trace an ML model throughout all of its lifecycle stages
from data preparation to deployment, \atlas embeds the cryptographic hash of the
transformation attestation for every pipeline input that passed verification
into the output attestation.
These hash values are digitally signed as part of the transformation
attestation.
Thus, \atlas attestation clients establish an authenticated, cryptographically
verifiable \emph{provenance chain} representing a model's lineage relationships
throughout its lifecycle (\textbf{R3}).

%Each ML artifact -- whether model weights, training data, or configuration -- receives a C2PA manifest
%containing cryptographic measurements, hardware-backed TDX attestation hashes, digital signatures,
%and temporal information about its creation and modifications.
%The system establishes clear ingredient relationships between artifacts, where training data becomes an
%ingredient of model checkpoints, checkpoints link to the final model, and configuration files maintain connections
%to their respective training runs.
%\begin{figure}[h]
%	\centering
%	\includegraphics[width=0.5\textwidth]{img/hf-sidecar.drawio}
%	\caption{Metadata Management System Architecture showing how ML pipeline integrity is maintained: A sidecar container runs alongside the main workload container to measure and verify artifacts (model weights, training data, configurations).
%		The sidecar generates cryptographic measurements and attestation data for each artifact, which are verified by a metadata verifier before being stored in both a transparency log and artifact repository.
%		This ensures continuous validation of ML pipeline components while maintaining a verifiable audit trail.}
%	\label{fig:metadata-sidecar}
%\end{figure}

\subsubsection{Transparency Log}\label{sec:framework:transparency:log}

% The text below attempts to actually define the properties
% we want from the log from a design standpoint (not implementation). What is the
% internal structure of the Merkle tree (BST vs append-only)? How do trees map to
% different pipelines (1:1, n:1)? How do we manage multiple cycles of a lifecycle?
% if it's way off, let's discuss!

On the server side, a \emph{transparency log} contains the golden values (i.e.,
known good values) of a producer's model artifacts, an MLaaS provider's system
components, and the transformation attestations collected by the clients
throughout the ML lifecycle.
To enable efficient insertion and provenance verification while
accommodating the cyclical nature of the ML lifecycle, \atlas relies on two
data structures.

First, to provide cryptographic tamper-evidence for the stored values, the
transparency log is constructed using an \emph{append-only} Merkle tree~\cite{merkle-tree1987},
meaning that pipeline metadata can be efficiently inserted in the right-most
empty leaf node of the tree (e.g., as in~\cite{sigstore2025}).
Second, to enable more efficient verification of provenance \emph{across}
pipelines (or even cycles of the ML lifecycle), \atlas can represent each discreet
pipeline/cycle using a different Merkle tree and \emph{chain} the trees by
embedding the Merkle root hash of the upstream pipeline or previous cycle into
the ``current'' Merkle tree's root (e.g., as in~\cite{coniks2015}), limiting tree
traversal operations when verifying a lifecycle over time (\textbf{R7}).

We note that the choice of whether and how to partition the transparency log's
Merkle trees will result in different security and performance tradeoffs;
depending on the party that operates the transparency service in
practice, they may favor maintaining a smaller or larger number of trees
according to their specific metadata access control and resource requirements.
%providing cryptographic proof of the complete model lineage at any point in time.
%This relationship structure creates a verifiable lineage chain where each
%Ingredients maintain secure links to their source manifests, enabling validation of the complete
%artifact chain from any point to verify authenticity.
%The sidecar container implementing this tracking
%system continuously monitors the ML pipeline for new or modified artifacts, generating manifests and updating relationship maps as transformations occur.
%
%Our extension implements comprehensive validation layers handling manifest verification,
%attestation validation, and integrity checks. The immutable storage layer preserves both ML pipeline artifacts
%and detailed pipeline history, including transformation records, version control, and attestation chains.

\subsection{Verification System}\label{sec:framework:verification}

The \atlas verification system performs staged verification.
When a pipeline ingests an input artifact, the attestation client requests
the artifact's verification at the verification service, interacting
with the transparency log to obtain the golden values and provenance
information.

First, the verification service validates that the received metadata was
produced by the expected parties via their digital signatures, and checks
whether the artifact matches the expected golden value measurements.
If the first verification stage passes, \atlas inspects the received
transformation metadata to confirm the recorded ML system and
pipeline operations ran as expected based on their TEE-backed attestations.
Third, the artifact's lineage is validated by tracing its provenance chain
to check its transformations as far back as specified in the verifier's policy.

To enable efficient batch verification of input artifacts, \atlas
groups related artifacts including training datasets,
model weights, algorithm implementations, and evaluation results.
By maintaining a cache of verified states and an incremental verification flow,
the system avoids re-computing cryptographic operations for unchanged artifacts
(\textbf{R7}).
This particularly benefits ML pipelines where only small portions of the
workflow change between iterations.

%Third, the Metadata \& Artifact Integrity layer validates relationships between pipeline components.
%The validated data is then stored in two parallel tracks: (1) ML Pipeline Artifacts storing training data, model checkpoints, and validation results, and (2) Pipeline History maintaining transformation records and dependency tracking. This approach ensures comprehensive verification of both the ML artifacts and their complete lineage.

%First, the certificate chain of the signing keys for each transformation attestation
%is verified to establish trust in the signing authorities.
%
%Next, hash verification aims to detect modifications to model artifacts via their measurements.
%Third, the transformation attestation is validated against the verifier's expectations about a given pipeline's operation.
%Finally, relationship validation checks for inconsistencies across component dependencies and versions.

%\begin{figure}[h]
%	\centering
%	\includegraphics[width=0.5\textwidth]{img/hf-validation-layer.drawio}
%	\caption{Multi-layered validation architecture for ML pipeline integrity: The system implements three validation stages before storing ML artifacts.}
%	\label{fig:rekor-extension}
%\end{figure}

%The framework maintains verification state to track validated components and their relationships.
%When verification failures occur, the system defines policies for invalidating affected components while
%preserving verified states.
%Error handling includes mechanisms for detailed forensic logging.

\section{Findings}\label{finding}
In this section, We demonstrate our findings organized by specific goals~(Why). 
We aim to reveal the most common human-AI interaction patterns as a focal area of study. 
Furthermore, certain patterns remain under-investigated in previous research, raising questions about their entailment and potential future applications.

\subsection{Goal 1: Initialize the Simulation}

Initializing the environment is the most frequently occurring goal in our reviewed literature.
Due to the large number of papers in this category, detailed information can be found in Appendix~\ref{Ainitial}, Table~\ref{tab:initial}.
Firstly, we observe that users interact with the models before the simulation and primarily assume three roles: scriptwriter, director, and prototype.
As scriptwriters, users need to establish a foundational background for the simulation.
Users create agents by defining their identity~\cite{hua2024warpeacewaragentlarge,lin2023agentsimsopensourcesandboxlarge}, interaction~\cite{berryman2008review}, long-term goal~\cite{hong2024metagptmetaprogrammingmultiagent}, and learning ability~\cite{li2023modelscopeagentbuildingcustomizableagent}.
Similarly, users can control the description~\cite{jinxin2023cgmiconfigurablegeneralmultiagent}, objects~\cite{basavatia2023complexworld}, and rules~\cite{10.1145/3526113.3545616} of environments.
Although some studies have utilized natural language command~\cite{hong2024metagptmetaprogrammingmultiagent} and interfaces~\cite{lin2023agentsimsopensourcesandboxlarge}, we find that a portion of the work requires users to engage in configuration settings, such as programming~\cite{netlogo}, graphical programming~\cite{Ped,doi:https://doi.org/10.1002/9781118762745.ch12}, importing packages~\cite{Significant_Gravitas_AutoGPT}, or writing configuration files~\cite{lin2023agentsimsopensourcesandboxlarge}.
Unlike interfaces and natural language commands, these methods present certain challenges for novice users when getting started.
However, they allow for a systematic, modular, and efficient setup of simulations from scratch.
How to combine the advantages of both aspects is a question worth exploring.

Another important role for the user is the director.
The director can directly issue goal commands to the model, prompting agents to begin executing the goals~\cite{rana2023sayplangroundinglargelanguage, ahn2022icanisay} or automatically trigger agents' actions through specific user actions~\cite{10.1145/3613904.3642183, arakawa2024prismobserverinterventionagenthelp}.
Additionally, the director can modify certain environmental settings during the initialization time~\cite{park2023choicematessupportingunfamiliaronline, 10.1145/3613904.3642159}.
The most commonly used means is natural language commands~\cite{10.1145/3678585}, followed by interface~\cite{pan2024agentcoordvisuallyexploringcoordination}.
Compared to the scriptwriter, the director controls the model from a more granular perspective.
Researchers can design appropriate interactions tailored to their specific research needs.
In some cases, users also appear in the role of prototypes and provide demographic data for agent identities.
Before the advancement of computational power, it was common to use sampling methods to select prototypes, and the information dimensions provided to the model were limited~\cite{GAUBE201392}.
Currently, sampling from the dataset is not necessary since the model can handle diverse, heterogeneous data directly~\cite{gao2023s3socialnetworksimulationlarge} with enhanced data processing abilities.

In this category, we can observe the evolution of simulation platforms or toolkits.
Before the maturity of NLP technologies, many works already supported users in initializing simulations.
However, these interactions were not as straightforward as natural language and involved a certain learning curve.
Initially, tools were difficult to use and challenging to learn, such as Netlogo~\cite{netlogo}, EINSTein~\cite{berryman2008review}, and MASON~\cite{doi:10.1177/0037549705058073}, which are required programming skills.
Later, tools like AnyLogic~\cite{doi:https://doi.org/10.1002/9781118762745.ch12} and PedSim~\cite{Ped} emerged, supporting graphical programming and visualizing simulation trajectories, making them more accessible and user-friendly.
With the emergence of large language models, diverse and lightweight simulation platforms have been developed~(\eg AutoGPT~\cite{Significant_Gravitas_AutoGPT} and Modelscope~\cite{li2023modelscopeagentbuildingcustomizableagent}), leveraging the interaction and generative capabilities of these models to support user-customized agents.
This advancement allows users to create tailored agent behaviors and scenarios more intuitively, expanding the flexibility and accessibility of simulation platforms.
We will further discuss the potential development of simulation platforms in Section~\ref{software}. 


\subsection{Goal 2: Explore Different Scenarios}
Investigating various hypothetical scenarios enables users to examine how different assumptions or interventions might influence system dynamics.
The detailed information in this cluster is shown in Table~\ref{tab:explore}.
ChatEval~\cite{chan2023chatevalbetterllmbasedevaluators} supports multi-agent collaboration to compare only two language models' performance at once.
Thus, users need to predefine various models and conduct multiple simulations to compare the comparison results across different models.
The interactions in the remaining works occur during the simulation.

Users act directly as actors, exploring various scenarios through their own diverse behaviors, such as communicating with agents through natural language~\cite{10.1145/3586183.3606763,lin2023agentsimsopensourcesandboxlarge}.
Users can also take on the scriptwriter role, directly altering agents' foundational goals by natural language commands~\cite{10.1145/3586183.3606763}.
This type of work is relatively rare, possibly because users typically focus on exploring the impact of minor changes on the overall system rather than fundamentally altering the foundational setup of agents and the environment within the simulation.
In most cases, users act as directors, controlling the simulation process~\cite{wang2023humanoidagentsplatformsimulating, pan2024agentcoordvisuallyexploringcoordination}, adjusting environmental components~\cite{hua2024warpeacewaragentlarge}, and directing the actions of agents~\cite{DBLP:journals/corr/abs-2312-11813,10.1145/3613904.3642545}, etc.
Typically, by advancing or reversing the simulation progress, users can conduct ``what-if'' analysis~\cite{cui2024chatlawmultiagentcollaborativelegal,10.1145/3526113.3545616}.
``What-if'' analysis is crucial for ABMS, as it enables users to explore the potential effects of various changes within the system.
Users can observe how hypothetical scenarios impact agent behaviors and system dynamics by manipulating specific parameters or altering conditions. 

Current research on this topic is limited based on our review, highlighting a valuable opportunity for future researchers to explore ``what-if'' analysis in human-AI interactions in ABMS.
Such research could facilitate dynamic, in-depth analysis of ABMS and support decision-making processes, advancing the practical utility and impact of ABMS in complex scenarios.
Furthermore, the advent of LLMs enables users to explore different scenarios within the model using natural language and interface.
Designing a user-friendly, voice-enabled interactive interface that allows users to act as a real-world director, complete with a walkie-talkie and monitor screens, may hold significant potential as a research topic.
Users can also take on the role of actors, directly interacting with agents through natural language or physical movement with the advancement of immersive devices. 
They can modify or create diverse scenarios based on research needs.


\begin{table}[ht]
  \caption{This table introduces works concerning \textit{Explore Different Scenarios}. For simplicity, we
shorten the classification of environments: S-P: simulated-physical, S-V: simulated-virtual, R-P: real-physical, R-V: real-virtual. We also shorten the \textit{When} dimension: Pre-S: pre-simulation, D-S: during-simulation, Post-S: post-simulation. For the ``What'' dimension, we use icons instead of text to represent the secondary classification. Subsequent tables will also use similar abbreviations. Some works provide multiple interaction methods for the same goal, such as Generative Agents~\cite{10.1145/3586183.3606763} in this table.~\includegraphics[width=0.025\textwidth]{icon/action.pdf} represents agent \textit{action} and~\includegraphics[width=0.025\textwidth]{icon/goal.pdf} represents agent \textit{goal}.}
  \label{tab:explore}
  % \resizebox{\textwidth}{!}{
  \begin{tabular}{lllllll}
    \toprule
    \textbf{Year}&\textbf{Work} & \textbf{Env}&\textbf{When} & \textbf{Who} & \textbf{What} & \textbf{How}\\
    \midrule
    
2023&Generative Agents~\cite{10.1145/3586183.3606763} & S-P & D-S  & Actor &Agent~\includegraphics[width=0.025\textwidth]{icon/action.pdf}& Language \\

-& -& -& D-S & Scriptwriter &Agent~\includegraphics[width=0.025\textwidth]{icon/goal.pdf}& Language \\

-& -& -& D-S & Director &Env~\includegraphics[width=0.025\textwidth]{icon/object.pdf}& Language \\

2022&Social Simulacra~\cite{10.1145/3526113.3545616}& S-V & D-S  & Director &Sim~\includegraphics[width=0.025\textwidth]{icon/progress.pdf}& Interface \\

2023&ChatEval~\cite{chan2023chatevalbetterllmbasedevaluators}&None & Pre-S & Director &Env~\includegraphics[width=0.025\textwidth]{icon/description.pdf}&Interface \\

2023&AgentSims~\cite{lin2023agentsimsopensourcesandboxlarge}&S-P  & D-S  & Actor &Agent~\includegraphics[width=0.025\textwidth]{icon/action.pdf}&Language; Interface \\

2023&WarAgent~\cite{hua2024warpeacewaragentlarge}
&S-P  & D-S  & Director &Env~\includegraphics[width=0.025\textwidth]{icon/description.pdf}&Language \\

2024&AgentCoord~\cite{pan2024agentcoordvisuallyexploringcoordination}
&S-P & D-S  & Director &Agent~\includegraphics[width=0.025\textwidth]{icon/goal.pdf}; Sim~\includegraphics[width=0.025\textwidth]{icon/progress.pdf}&Interface \\

2024&Rehearsal~\cite{10.1145/3613904.3642159}
&None & D-S  & Director &Agent~\includegraphics[width=0.025\textwidth]{icon/action.pdf}&Interface \\

2023&Humanoid Agents~\cite{wang2023humanoidagentsplatformsimulating}
&S-P & D-S  & Director &Sim~\includegraphics[width=0.025\textwidth]{icon/progress.pdf}&Interface \\

2023&UGI~\cite{DBLP:journals/corr/abs-2312-11813}
&S-P & D-S  & Director &Agent~\includegraphics[width=0.025\textwidth]{icon/action.pdf}&Language \\

2024&Zhang\etal~\cite{10.1145/3613904.3642545}
&R-V & D-S  & Director &Agent~\includegraphics[width=0.025\textwidth]{icon/action.pdf}&Interface \\
2024& ChatCam~\cite{10.1145/3699731}&R-P&D-S&Director&Agent~\includegraphics[width=0.025\textwidth]{icon/action.pdf}&Language\\
2024&Cuadra\etal~\cite{10.1145/3659624}&R-P&D-S&Director&Agent~\includegraphics[width=0.025\textwidth]{icon/goal.pdf}&Language; Interface\\
2024&CrowdBot~\cite{10.1145/3659601}&R-P&D-S&Director&Agent~\includegraphics[width=0.025\textwidth]{icon/goal.pdf}&Language; Physical\\
2024&Sheshadri\etal~\cite{10.1145/3631404}&R-P&D-S&Director&Agent~\includegraphics[width=0.025\textwidth]{icon/action.pdf}&Language\\
  \bottomrule
\end{tabular}
% }
\end{table}


\subsection{Goal 3: Refine the Model}
When the model’s performance fails to meet expectations, improving its effectiveness requires user intervention.
There are 29 papers in this cluster, and the detailed information of papers is shown in Appendix~\ref{Arefine}, Table~\ref{tab:refine}.
Before the simulation, SocialAI School~\cite{gao2023s3socialnetworksimulationlarge}, Krishna\etal~\cite{doi:10.1073/pnas.2115730119} and Surrealdriver~\cite{jin2024surrealdriverdesigningllmpoweredgenerative} guide agents to learn from external resources, such as human natural language instructions and domain expertise data, to enhance their learning ability.
Although there is limited work in this area, it presents a promising approach to refine the model, and new interactions warrant further research.
The majority of methods are implemented during the simulation process.
Some of them also focus on agents' learning abilities.
Users can teach agent human knowledge and domain expertise~\cite{unknown, 10.1145/3613904.3642349, cui2024chatlawmultiagentcollaborativelegal} and directly manipulate memory system~\cite{10.1145/3586182.3615796}.

Some work allows users to directly take on the role of actors, collaborating with agents to complete tasks by natural language~\cite{zhang2024buildingcooperativeembodiedagents} or physical movements~\cite{mandi2023rocodialecticmultirobotcollaboration}.
More frequently, users assume the role of directors, steering agent actions~\cite{mehta2024improvinggroundedlanguageunderstanding, mohanty2023transforminghumancenteredaicollaboration, Padmakumar_Thomason_Shrivastava_Lange_Narayan-Chen_Gella_Piramuthu_Tur_Hakkani-Tur_2022}, goals~\cite{huang2022innermonologueembodiedreasoning,chen2023agentversefacilitatingmultiagentcollaboration}, and interaction~\cite{park2023choicematessupportingunfamiliaronline}.
Additionally, due to the stochastic nature of LLMs, users acting as directors can control the simulation progress through the interface by regenerating outcomes if the current results are unsatisfactory, allowing for the possibility of achieving more desirable outcomes~\cite{chen2023agentversefacilitatingmultiagentcollaboration,chan2023chatevalbetterllmbasedevaluators}.
This cluster appears to overlook the impact of environmental components on refining the model.
Users can potentially reduce obstacles for agents in completing tasks by controlling environmental components.
In addition to agents' learning abilities, users may consider enhancing agents' autonomy—an often-overlooked component in interaction design.


The design of human-AI interactions that harness the strengths of both humans and AI, enabling complementary collaboration, represents a significant area for exploration. 
This approach raises important questions about how best to structure interactions to optimize collaboration and achieve desired outcomes.
From our corpus of papers, we conclude that humans excel in creative thinking, domain expertise, and problem-solving in ambiguous situations, making them adept at tasks requiring abstract thought or out-of-the-box solutions~\cite{ren2023robotsaskhelpuncertainty, 10.1145/2282338.2282384}.
AI operates with consistent accuracy and efficiency, reducing the risk of human error and performing repetitive tasks without fatigue~\cite{10.1145/3672539.3686351}.
Combining these strengths, human-AI interaction has the potential to achieve more comprehensive outcomes, with humans providing complex reasoning abilities and AI enhancing efficiency and scalability.


\subsection{Goal 4: Evaluation the Performance}
Evaluating the ABMS's performance relies on assessing how well the simulation meets predefined goals. Human involvement is central to this process.
In this cluster, we extracted 47 interactions from 41 works.
Due to the large number of papers in this category, detailed information can be found in Appendix~\ref{Aevaluate}, Table~\ref{tab:evaluate}.
For users pre-simulation engaging with the model, the objective is to manipulate specific conditions to assess whether the outcomes align with their expectations.
For example, users can copy community rules and goals from real-world social platforms to the environment of ABMS~\cite{10.1145/3526113.3545616} or design agents' identity modeled on real-world demographic information~\cite{10.1145/3394486.3412862, Argyle_Busby_Fulda_Gubler_Rytting_Wingate_2023}.
Assessing the indistinguishability between agent actions and real user actions provides a measure of the reliability of ABMS simulation results.

Users primarily assume three roles during the simulation: director, actor, and observer.
The director assesses whether agents can adapt flexibly and effectively to the environment by assigning different goals to agents~\cite{10.1145/3643505} or intervening in agent actions~\cite{10.1145/3610170, 10.1145/3613905.3651026}.
The actor role is similar to the director, but they interact directly with agents within the environment, which allows for real-time engagement and firsthand observation of agent actions.
They test the agents' abilities in collaboration~\cite{zhang2024buildingcooperativeembodiedagents}, social interaction~\cite{zhou2024sotopiainteractiveevaluationsocial}, teaching~\cite{saha2023languagemodelsteachweaker}, and strategic gameplay~\cite{NEURIPS2021_86e8f7ab,10.1145/3613905.3650853}.
The observer evaluates agents by tracking their behaviors through graphical interfaces~\cite{park2023choicematessupportingunfamiliaronline,lin2023agentsimsopensourcesandboxlarge,wang2023humanoidagentsplatformsimulating} or log data~\cite{babyagi}, allowing for a detailed assessment of agent actions.

After the simulation, the majority of users, acting as observers, evaluate model performance primarily through analyzing agent action data.
They assess whether the agents perform effectively~\cite{hua2024warpeacewaragentlarge,park2023choicematessupportingunfamiliaronline} or exhibit noticeable differences from real human behaviors~\cite{10.1145/3544548.3580688,10.1145/3526113.3545616,https://doi.org/10.1111/mila.12466}.
MetaGPT~\cite{hong2024metagptmetaprogrammingmultiagent} and BactoWars~\cite{berryman2008review} provide users with interactive interfaces and videos to showcase agent performances.
Notably, Generative Agents~\cite{10.1145/3586183.3606763} proposed a unique evaluation method, interviewing agents as an actor ``reporter''.
After ``two-day'' simulated lives, by designing targeted questions, users can assess whether the agent has self-awareness of its identity, accurate memory, and action aligned with its assigned character traits.
Previous studies have largely overlooked agents' internal states. 
Future research could benefit from emphasizing the alignment between agents' internal states and outward behaviors.

LLM-powered agents are capable of simulating various human-like behaviors and reflecting different characteristics.
The coherence and consistency of LLMs' outputs make agents' behaviors more realistic and believable.
When assessing the believability of simulated behaviors, simplistic quantitative statistical methods are often inadequate. 
In these instances, human qualitative evaluations, such as the Turing test~\cite{Turing2009}, are frequently employed in research to provide more nuanced insights.
It suggests that the advent of LLMs not only introduces new interactions for users in ABMS, but also creates additional interaction requirements.
Designing reliable user experiments to evaluate agent-human resemblance presents several challenges.
Key issues include minimizing user subjectivity to prevent it from skewing evaluation results and determining whether agent behavior alone can reliably indicate human likeness.
Another complexity is interpreting agents' unusual or seemingly illogical actions; while such behaviors might suggest limitations in the agent's mimicry ability, human behavior itself often includes an element of randomness.



\subsection{Goal 5: Analyze Simulation Data}
Analyzing data generated from the ABMS process is a key goal for users.
The datasets involve logs of agents' actions, records of agents' internal state, formation and evolution of networks among agents, spatial and temporal data, etc.
By analyzing the data, users gain insights into system dynamics to support the decision-making process ultimately.
The detailed information about the literature is shown in Table~\ref{tab:analyze}.
Users all act as observers to analyze agents' actions through the interface.
In contrast to assessing the model itself, users analyze data to derive insights for downstream tasks, such as informing real-world decision-making or enhancing predictive capabilities.
Out of the ten works, six are early-developed simulation platforms or toolkits. 
This type of more mature toolkit typically provides users with data analysis modules.
EINSTein, MANA~\cite{berryman2008review}, and Swarm~\cite{minar1996swarm} display basic statistical metrics and visualization, such as tallies of agents detected and killed in battlefield and a time series graph of population dynamics.
Humanoid Agents~\cite{wang2023humanoidagentsplatformsimulating} and AnyLogic~\cite{doi:https://doi.org/10.1002/9781118762745.ch12} both provide a dashboard for users to explore agents' actions over time interactively.
Furthermore, AgentLens~\cite{10520238} and AgentCoord~\cite{cui2024chatlawmultiagentcollaborativelegal} proposed more intricate visual analytics systems to support users interactively investigating details and causes of agents' actions and multi-agent interaction strategy.
We find that the data has evolved from simple statistical metrics to complex, multi-dimensional, heterogeneous forms, such as agent emotions, diverse actions and locations, and dynamic social networks.


The integration of LLMs significantly enhances the richness and complexity of simulation data, which introduces challenges in managing, processing, and interpreting the increased intricacy of the data.
Correspondingly, the evolution of analytical tools, from basic statistical charts to dashboards and then to fully integrated visual analytics systems, reveals an increase in both their analytical capabilities and level of interactivity.
They support more nuanced insights, facilitate decision-making, and allow users to engage with complex data landscapes in a more intuitive, interactive manner.
The development of effective and efficient tools suited for analyzing ABMS data holds substantial potential research value.
For example, integrating machine learning models for data regression or classification could be considered, as well as incorporating NLP techniques to allow users to control the analysis process through natural language commands.
Regarding the \textit{When} dimension, we discover that only a limited number of works support real-time data analysis by users~(during-simulation).
Currently, real-time data analysis is challenging to implement, especially for ABMS developed with LLMs, as they can lead to unstable data generation and low processing efficiency.
Developing stable, real-time, and user-friendly analytical tools requires further investigation.


\begin{table}[ht]
  \caption{This table introduces works concerning \textit{Analyze Simulation Data}.~\includegraphics[width=0.025\textwidth]{icon/action.pdf} represents \textit{action}. }
  \label{tab:analyze}
  % \resizebox{\textwidth}{!}{
  \begin{tabular}{lllllll}
    \toprule
    \textbf{Year}&\textbf{Work} & \textbf{Env}&\textbf{When} & \textbf{Who} & \textbf{What} & \textbf{How}\\
    \midrule
-& AnyLogic~\cite{doi:https://doi.org/10.1002/9781118762745.ch12} & S-P & D-S    & Observer  & Agent~\includegraphics[width=0.025\textwidth]{icon/action.pdf} & Interface   \\
-& EINSTein~\cite{berryman2008review} & S-P & Post-S   & Observer  & Agent~\includegraphics[width=0.025\textwidth]{icon/action.pdf} & Interface   \\
2006 & MANA~\cite{berryman2008review} & S-P & Post-S   & Observer  & Agent~\includegraphics[width=0.025\textwidth]{icon/action.pdf} & Interface   \\
1999 & NetLogo~\cite{netlogo} & S-P & Post-S   & Observer  & Agent~\includegraphics[width=0.025\textwidth]{icon/action.pdf} & Interface  \\
2006 & North\etal~\cite{10.1145/1122012.1122013} & S-P & Post-S   & Observer  & Agent~\includegraphics[width=0.025\textwidth]{icon/action.pdf} & Interface   \\
1996 & Swarm~\cite{minar1996swarm} & S-P & D-S    & Observer  & Agent~\includegraphics[width=0.025\textwidth]{icon/action.pdf} & Interface   \\
2023 & Zarzà\etal~\cite{electronics12122722} & S-P & Post-S   & Observer  & Agent~\includegraphics[width=0.025\textwidth]{icon/action.pdf} & Interface   \\
2024 & AgentLens~\cite{10520238} & S-P & Post-S   & Observer  & Agent~\includegraphics[width=0.025\textwidth]{icon/action.pdf} & Interface   \\
2024 & AgentCoord~\cite{pan2024agentcoordvisuallyexploringcoordination} & S-P & D-S    & Observer  & Agent~\includegraphics[width=0.025\textwidth]{icon/action.pdf}~\includegraphics[width=0.025\textwidth]{icon/interaction.pdf} & Interface   \\
2023 & Humanoid Agents~\cite{wang2023humanoidagentsplatformsimulating} & S-P & Post-S   & Observer  & Agent~\includegraphics[width=0.025\textwidth]{icon/action.pdf} & Interface   \\

  \bottomrule
\end{tabular}
\end{table}

% EINSTein
% MANA
% NetLogo
% Repast
% Swarm
% Anylogic
% Emergent Cooperation and Strategy Adaptation in Multi-Agent Systems: An Extended Coevolutionary Theory with LLMs
% AgentLens
% AgentCoord
% Humanoid Agents


\subsection{Goal 6: Be Immersed in the Environment}\label{immersed}

Immersion in the environment highlights the user’s experience within the simulation, primarily emphasizing engagement rather than control or modification.
The number of papers in this category is relatively small compared to other categories.
According to Table~\ref{tab:immerse}, there are only two papers in the category: Generative Agents~\cite{10.1145/3586183.3606763} and Alympics~\cite{mao2024alympicsllmagentsmeet}.
In both works, users can play as actors and interact with agents as if they were one of them during the simulation in the environment.
In Generative Agents, users can communicate with agents as ``mayor'' or ``reporter'' and change the status of surrounding objects.
In Alympics, human players are engaged in the game with agent players.
The user does not have a predetermined goal but seeks immersion and emotional value in the interaction process in both cases.
Due to the limited work in this area, many interactions remain to be developed.
Users can take on the role of scriptwriter or director,  granting them the ability to control the model from a ``god’s-eye view'' and effectively orchestrate the entire simulation.
This high-level perspective fosters a strong sense of engagement and immersion as users can actively influence the model's narrative and dynamics.
Besides, immersive experience in the virtual reality environment constitutes a significant and valuable area of research.
Users can interact with agents through physical movements and natural language, creating a more intuitive engagement.
We provide a further discussion on immersive experience in Sections~\ref{immersive}.


\begin{table}[ht]
  \caption{This table introduces works concerning \textit{Be Immersed in the Environment}. ~\includegraphics[width=0.025\textwidth]{icon/object.pdf} represents \textit{object}.}
  \label{tab:immerse}
  % \resizebox{\textwidth}{!}{
  \begin{tabular}{lllllll}
    \toprule
    \textbf{Year}&\textbf{Work} & \textbf{Env}&\textbf{When} & \textbf{Who} & \textbf{What} & \textbf{How}\\
    \midrule
    2023&Generative Agents~\cite{10.1145/3586183.3606763} & S-P & D-S  & Actor &Agent~\includegraphics[width=0.025\textwidth]{icon/action.pdf};
Env~\includegraphics[width=0.025\textwidth]{icon/object.pdf}& Language\\
2023&Alympics~\cite{mao2024alympicsllmagentsmeet}&S-V & D-S  & Actor &Agent~\includegraphics[width=0.025\textwidth]{icon/action.pdf}&Interface\\
  \bottomrule
\end{tabular}
\end{table}


\subsection{Application of the Taxonomy}
Our taxonomy and findings can be used in designing human-AI interactions in ABMS that support users' customized implementation to meet research needs.
First, identify the primary goal for interaction~(\textit{Why}). 
We have summarized six goals in~\Cref{goal} that require human involvement to achieve.
Designers determine interaction goals based on our framework to address the practical needs of different research tasks.
According to the goal, designers can find existing interactions in~\Cref{finding}, including the other four dimensions~(\textit{When}, \textit{What}, \textit{Who}, and \textit{How}).
Designers can select the most appropriate interaction from the patterns or be inspired by the potential interactions we have summarized.
Designers must comprehensively consider many aspects to determine the four dimensions, including further refining interaction goals, the feasibility of technical implementation, and other relevant factors.
\section{Suggestions and Research Opportunities}
In this section, we present specific research opportunities identified through the findings in~\Cref{finding} using the proposed taxonomy in~\Cref{framework}.

\subsection{Maxmize the Potential of LLMs}
LLMs are becoming increasingly significant in enhancing interaction in ABMS due to their unique ability to understand and generate intricate human language.
By leveraging LLMs, users benefit from a more intuitive and effective interaction process.
Well-designed prompts can guide LLMs in better simulating human-like behaviors, producing contextually accurate responses, and performing complex tasks autonomously.
Users who lack knowledge of LLMs may struggle to phrase prompts in ways that yield the desired outcomes, which can lead to potentially confusing or unintended results.
Additionally, the complexity of ABMS can further complicate prompt formulation, as users must consider both the model's interpretive limits and the nuances of simulation parameters and agent behaviors.
For example, Generative Agents~\cite{10.1145/3586183.3606763} supports utilizing one paragraph of natural language description to define agent’s identity, including jobs and past experience.
Although such a design provides users with substantial freedom, it can lead to a dilemma where users are uncertain about what to write and may struggle to determine which information is essential to include in the prompt. 
This uncertainty can result in prompts that are either incomplete or overly detailed, diminishing the interaction's effectiveness.
Prompt engineering~\cite{giray_prompt_2023}, which is the process of carefully designing prompts to guide LLMs in generating accurate and contextually appropriate responses, has been widely developed, such as chain-of-thought~\cite{NEURIPS2022_9d560961} and tree-of-thought~\cite{NEURIPS2023_271db992} strategies.
We think the community could explore ways to help users craft effective prompts during interactions.
This research could involve developing adaptive prompt templates tailored to specific tasks, recommending contextually relevant prompts based on the user’s goals, or implementing prompt engineering techniques to refine users' inputs for better results. 
For example, when users must set an agent's identity through natural language, a template can be provided to guide users in specifying the required demographic information (\eg gender, age, occupation).
These approaches aim to reduce the learning curve associated with prompt creation, especially for users less familiar with LLMs, and improve the overall effectiveness of human-AI interactions.
%和agent相关

An increasing number of specialized fields are utilizing interactive ABMS, with LLMs simulating various human roles or professions.
However, simply employing LLMs for basic question-and-answer interactions does not effectively simulate all roles, particularly those requiring domain expertise or complex reasoning abilities.
For roles like these, a more sophisticated approach is needed to capture the depth and nuance of their knowledge and thinking processes.
One possible future direction is to design cognitive architecture for agents to simulate the human thinking processes, such as retrieving and reflecting~\cite{10.1145/3586183.3606763}.
These architectures could enable more realistic and contextually aware responses to model complex human behaviors, making them more effective in roles requiring higher expertise and adaptive decision-making.
Instruction tuning~\cite{zhang2024instructiontuninglargelanguage} is another strategy to improve the performance of LLMs by training them to follow specific types of instructions more accurately.
By fine-tuning models with instruction data specific to a field, LLMs can better understand and execute nuanced, technically complex instructions that align with domain professionals' expectations.
Instruction tunning techniques have been applied in various domains~\cite{zhang2023multitaskinstructiontuningllama,liu2023goatfinetunedllamaoutperforms}, however, there is limited research addressing it in the HCI community. %具体的问题
We hope that our research can inspire future researchers in this area.


%trust issue


\subsection{Simulation Software Development}\label{software}
Before the maturity of natural language technologies, users typically built ABMS on simulation software platforms~\cite{doi:10.1177/0037549706073695,berryman2008review}. %代表性工具,例如netlogo, agenttorch
These platforms did not support natural language interaction, requiring users to rely on more technical interfaces, which also involved a certain learning cost.
ABMS simulation platform with integrated natural language processing techniques may be required to enable users to interact with agents and control simulations using natural language commands, enhancing accessibility and ease of use. 
The platform could make ABMS more user-friendly and applicable across various domains, even for those without programming expertise.
Although there exist some platforms that enable the creation, deployment, and management of agents leveraging LLMs, such as autoGPT~\cite{Significant_Gravitas_AutoGPT}, AgentTorch~\cite{chopra2023agenttorch}, they still require users to have a certain level of programming knowledge.
It is important to design simulation software accessible to users with minimal technical expertise by incorporating natural language processing capabilities. 
In addition to implementing natural language interaction, other AI technologies could also be considered. 
For example, integrate machine learning algorithms to recommend relevant commands or next steps to users based on the user’s current actions, simulation state, or previous interaction sequences. 

ABMS is a versatile tool applied across numerous fields to simulate complex systems, analyze collective behaviors, and make predictions.
Different fields have unique design requirements for interactive ABMS platforms.
Each domain may prioritize distinct features, interaction methods, and data integration needs to meet specific goals effectively.
For example, economic simulations prioritize high-frequency interaction options, such as adjusting market parameters or agent strategies in real-time~\cite{helbing_agent-based_2012}.
While simulations in social science often need agents with complex, varied behaviors to model interactions like group dynamics, migration, or policy effects~\cite{gao2023s3socialnetworksimulationlarge}.
Developing simulation platforms for specific domains may empower professionals and researchers to address real-world challenges.
They could include agents and models prebuilt for the domain, tailor the interface and interaction options to the specific needs of the field, and offer analysis tools and visualization options that highlight metrics crucial to the domain.
Furthermore, the platform could include AI components or expert systems specific to the domain to support more realistic simulations.




\subsection{Immersive Experience}\label{immersive}%ubicomp文章
As discussed in Section~\ref{immersed}, we find that there is limited research on users' immersive experiences currently.
Popular science fiction TV series, \textit{Westworld}, set in a futuristic, highly immersive theme park populated by lifelike AI agents, which allows human guests to live out their fantasies in a Western-themed world without consequences.
As agent technology advances, the science fiction scenarios portrayed in the series are increasingly approaching reality.
Research on user immersive experience in ABMS is currently most relevant in the context of video games, such as role-playing games~(RPGs).
Värtinen\etal~\cite{c7c0852d5f324ba5907ee22bea26560c} generated role-playing game quests with LLMs to fulfill player demands toward more and richer game content.
By understanding how ABMS contributes to immersion, game developers can create environments that foster emotional investment, realistic social dynamics, and greater player satisfaction.
Additionally, insights gained may benefit other fields involving immersive environments, such as virtual reality.
Furthermore, as biotechnology and materials science advance to new levels, the concept of physical parks akin to \textit{Westworld} may become feasible.
Users would interact with physical agents through \textit{Natural Language} and \textit{Physical Movements}, creating highly immersive experiences.

Another potential application scenario is companion agents designed to provide emotional support. 
The rapid advancement of high technology has created a sense of disconnection and emotional distance, paradoxically leaving people feeling more alone despite constant virtual contact. 
Digital interactions often replace direct, face-to-face connections.
An inner emptiness or emotional void emerges, leading to a growing need for meaningful interaction and companionship.
These agents could offer companionship, simulate meaningful conversations, and respond empathetically to users' needs.
This application requires careful attention to emotional intelligence, personalization, and ethical considerations to ensure that the agents are both supportive and safe for users.
We believe that the user immersive experience in ABMS holds significant research value.

Although our PAR method effectively mitigates reward hacking, it does not improve peak performance, as measured by the winrate of the best checkpoint. Furthermore, its design principles lack precision. While PAR sets the upper bound of the RL reward to 1.0, alternative bounds and their selection criteria remain unexplored. Additionally, the dynamics of reward adjustment—such as the initial rate of increase and the pace of convergence—are not fully elucidated. 
% \newpage

%%
%% The next two lines define the bibliography style to be used, and
%% the bibliography file.
\bibliographystyle{ACM-Reference-Format}
\bibliography{sample-base}

\appendix

\section{Appendix}
\subsection{}\label{Ainitial}

\begin{longtable}{>{\arraybackslash}lp{2.7cm}p{0.8cm}llp{3.5cm}p{2cm}}
% \centering
\caption[Short Caption]{This table introduces works concerning \textit{Initialize the Simulation}.}
\label{tab:initial} \\

% 下面是表头
\hline \textbf{Year}&\textbf{Work} & \textbf{Env}&\textbf{When} & \textbf{Who} & \textbf{What} & \textbf{How} \\  \hline 
\endfirsthead

% 下面数字3的意思是表格的列数
\multicolumn{7}{c}%
{{\bfseries \tablename\ \thetable{} -- continued from previous page}} \\
\hline \textbf{Year}&\textbf{Work} & \textbf{Env}&\textbf{When} & \textbf{Who} & \textbf{What} & \textbf{How} \\  \hline  
% 注意这里把表头复制了一遍,因为在新的页面也会展示一下表头,不然表格不方便阅读
\endhead

\hline \multicolumn{7}{r}{{Continued on next page}} \\ \hline
\endfoot

\hline \hline
\endlastfoot
\specialrule{0em}{1pt}{1pt}
2023&Generative Agents~\cite{10.1145/3586183.3606763}&S-P & Pre-S & Scriptwriter   &Agent~\includegraphics[width=0.025\textwidth]{icon/identity.pdf}&Language \\

2022&Social Simulacra~\cite{10.1145/3526113.3545616}&S-V & Pre-S & Scriptwriter   &Agent~\includegraphics[width=0.025\textwidth]{icon/identity.pdf}~\includegraphics[width=0.025\textwidth]{icon/goal.pdf}; Env~\includegraphics[width=0.025\textwidth]{icon/rule.pdf}&Language \\

2023&ChatEval~\cite{chan2023chatevalbetterllmbasedevaluators}&None & Pre-S & Scriptwriter   &Agent~\includegraphics[width=0.025\textwidth]{icon/identity.pdf}~\includegraphics[width=0.025\textwidth]{icon/learning.pdf} &Configuration \\

2023&MetaGPT~\cite{hong2024metagptmetaprogrammingmultiagent}&None & Pre-S & Scriptwriter   &Agent~\includegraphics[width=0.025\textwidth]{icon/goal.pdf}&Language \\

2023&Argyle\etal~\cite{Argyle_Busby_Fulda_Gubler_Rytting_Wingate_2023}&R-P & Pre-S & Prototype   &Agent~\includegraphics[width=0.025\textwidth]{icon/identity.pdf}&Data \\

2023&SayPlan~\cite{rana2023sayplangroundinglargelanguage}&S-P & Pre-S & Director   &Agent~\includegraphics[width=0.025\textwidth]{icon/goal.pdf}&Language \\

2023&AgentSims~\cite{lin2023agentsimsopensourcesandboxlarge}&S-P & Pre-S & Scriptwriter   &Agent~\includegraphics[width=0.025\textwidth]{icon/identity.pdf}~\includegraphics[width=0.025\textwidth]{icon/learning.pdf}; Env~\includegraphics[width=0.025\textwidth]{icon/object.pdf}&Interface; Configuration \\
2022&Huang\etal~\cite{huang2022innermonologueembodiedreasoning}&S-P; R-P & Pre-S & Director   &Agent~\includegraphics[width=0.025\textwidth]{icon/goal.pdf}&Language \\

2023&$S^3$~\cite{gao2023s3socialnetworksimulationlarge}&S-V & Pre-S & Prototype   &Agent~\includegraphics[width=0.025\textwidth]{icon/identity.pdf}&Data \\

2023&Ahn\etal~\cite{ahn2022icanisay}&R-P & Pre-S & Director   &Agent~\includegraphics[width=0.025\textwidth]{icon/goal.pdf}&Language \\

2022&WebShop~\cite{NEURIPS2022_82ad13ec}&S-V & Pre-S & Director   &Agent~\includegraphics[width=0.025\textwidth]{icon/goal.pdf}&Language \\

2023&Mind2Web~\cite{NEURIPS2023_5950bf29}&R-V & Pre-S & Director   &Agent~\includegraphics[width=0.025\textwidth]{icon/goal.pdf}&Language \\

2023&CAMEL~\cite{NEURIPS2023_a3621ee9}&None & Pre-S & Director   &Agent~\includegraphics[width=0.025\textwidth]{icon/goal.pdf}&Language \\

2023&Aher\etal~\cite{pmlr-v202-aher23a}&R-P & Pre-S & Prototype   &Agent~\includegraphics[width=0.025\textwidth]{icon/identity.pdf}&Data \\

2023&CGMI~\cite{jinxin2023cgmiconfigurablegeneralmultiagent}&S-P & Pre-S & Scriptwriter   &Env~\includegraphics[width=0.025\textwidth]{icon/description.pdf}&Language \\

2023&ChatLaw~\cite{cui2024chatlawmultiagentcollaborativelegal}&None & Pre-S & Director   &Agent~\includegraphics[width=0.025\textwidth]{icon/goal.pdf}&Language \\

2020&Alfred~\cite{shridhar2020alfredbenchmarkinterpretinggrounded}&R-P & Pre-S & Director   &Agent~\includegraphics[width=0.025\textwidth]{icon/goal.pdf}&Language \\
\specialrule{0em}{1pt}{1pt}
2023&Ren\etal~\cite{ren2023robotsaskhelpuncertainty}&R-P & Pre-S & Director   &Agent~\includegraphics[width=0.025\textwidth]{icon/goal.pdf}&Language \\

2023&ChatDev~\cite{qian2024chatdevcommunicativeagentssoftware}&None & Pre-S & Director   &Agent~\includegraphics[width=0.025\textwidth]{icon/goal.pdf}&Language \\

-&BactoWars~\cite{berryman2008review}&S-P & Pre-S & Scriptwriter  &Agent~\includegraphics[width=0.025\textwidth]{icon/identity.pdf}~\includegraphics[width=0.025\textwidth]{icon/interaction.pdf}; Env~\includegraphics[width=0.025\textwidth]{icon/object.pdf}&Configuration \\

-&EINSTein~\cite{berryman2008review}&S-P & Pre-S & Scriptwriter  &Agent~\includegraphics[width=0.025\textwidth]{icon/identity.pdf}~\includegraphics[width=0.025\textwidth]{icon/interaction.pdf}; Env~\includegraphics[width=0.025\textwidth]{icon/description.pdf}; Sim~\includegraphics[width=0.025\textwidth]{icon/technique.pdf}&Configuration \\

-&MANA~\cite{berryman2008review} &S-P & Pre-S & Scriptwriter  &Env~\includegraphics[width=0.025\textwidth]{icon/description.pdf}; Sim~\includegraphics[width=0.025\textwidth]{icon/technique.pdf}&Interface \\

2005&MASON~\cite{doi:10.1177/0037549705058073}&S-P & Pre-S & Scriptwriter   &Agent~\includegraphics[width=0.025\textwidth]{icon/identity.pdf}~\includegraphics[width=0.025\textwidth]{icon/learning.pdf}; Sim~\includegraphics[width=0.025\textwidth]{icon/technique.pdf}&Interface \\

1999&NetLogo~\cite{netlogo}&S-P & Pre-S & Scriptwriter  &Agent~\includegraphics[width=0.025\textwidth]{icon/identity.pdf}~\includegraphics[width=0.025\textwidth]{icon/interaction.pdf}&Interface \\

2006&North\etal~\cite{10.1145/1122012.1122013}&S-P & Pre-S & Scriptwriter  &Agent~\includegraphics[width=0.025\textwidth]{icon/identity.pdf}~\includegraphics[width=0.025\textwidth]{icon/learning.pdf}; Sim~\includegraphics[width=0.025\textwidth]{icon/technique.pdf} &Interface; Configuration \\

1996&Minar\etal~\cite{minar1996swarm}&S-P & Pre-S & Scriptwriter  &Agent~\includegraphics[width=0.025\textwidth]{icon/identity.pdf}~\includegraphics[width=0.025\textwidth]{icon/interaction.pdf}; Env~\includegraphics[width=0.025\textwidth]{icon/object.pdf}~\includegraphics[width=0.025\textwidth]{icon/rule.pdf} &Configuration \\

2023&ComplexWorld~\cite{basavatia2023complexworld}
&S-V & Pre-S & Scriptwriter  &Env~\includegraphics[width=0.025\textwidth]{icon/description.pdf}~\includegraphics[width=0.025\textwidth]{icon/object.pdf}~\includegraphics[width=0.025\textwidth]{icon/rule.pdf} &Language \\

2024&Cui\etal~\cite{Cui_2024_WACV}
&S-P & Pre-S & Director  &Agent~\includegraphics[width=0.025\textwidth]{icon/goal.pdf}&Language \\

2013&Gaube\etal~\cite{GAUBE201392}
&R-P & Pre-S & Prototype  &Agent~\includegraphics[width=0.025\textwidth]{icon/identity.pdf}&Data \\

2023&WarAgent~\cite{hua2024warpeacewaragentlarge}
&S-P & Pre-S & Scriptwriter  &Agent~\includegraphics[width=0.025\textwidth]{icon/identity.pdf}~\includegraphics[width=0.025\textwidth]{icon/interaction.pdf}; Env~\includegraphics[width=0.025\textwidth]{icon/description.pdf}~\includegraphics[width=0.025\textwidth]{icon/rule.pdf}&Configuration \\

2018&Kavak\etal~\cite{10.5555/3213032.3213044}
&S-P & Pre-S & Prototype  &Agent~\includegraphics[width=0.025\textwidth]{icon/identity.pdf}&Data \\

2023&Modelscope-agent~\cite{li2023modelscopeagentbuildingcustomizableagent}
&R-V & Pre-S & Scriptwriter  &Agent~\includegraphics[width=0.025\textwidth]{icon/identity.pdf}~\includegraphics[width=0.025\textwidth]{icon/learning.pdf}&Configuration \\

2024&AgentCoord~\cite{pan2024agentcoordvisuallyexploringcoordination}
&S-P & Pre-S & Director  &Agent~\includegraphics[width=0.025\textwidth]{icon/goal.pdf}&Interface \\

& & & & Scriptwriter  &Agent~\includegraphics[width=0.025\textwidth]{icon/identity.pdf}~\includegraphics[width=0.025\textwidth]{icon/interaction.pdf} &Interface \\

2023&Choicemates~\cite{park2023choicematessupportingunfamiliaronline}
&None & Pre-S & Director  &Agent~\includegraphics[width=0.025\textwidth]{icon/goal.pdf}; Env~\includegraphics[width=0.025\textwidth]{icon/description.pdf}&Language; Interface \\

2024&Rehearsal~\cite{10.1145/3613904.3642159}
&None & Pre-S & Director  &Env~\includegraphics[width=0.025\textwidth]{icon/description.pdf}&Interface \\

2023&Wang\etal~\cite{wang2024userbehaviorsimulationlarge}
&S-V & Pre-S & Scriptwriter  &Agent~\includegraphics[width=0.025\textwidth]{icon/identity.pdf}&Configuration \\

2023&Humanoid Agents~\cite{wang2023humanoidagentsplatformsimulating}
&S-P & Pre-S & Scriptwriter  &Agent~\includegraphics[width=0.025\textwidth]{icon/identity.pdf}&Language \\

2023&Zhu\etal~\cite{zhu2023ghostminecraftgenerallycapable}
&R-V & Pre-S & Director  &Agent~\includegraphics[width=0.025\textwidth]{icon/goal.pdf}&Language \\

-&PedSim~\cite{Ped} &S-P & Pre-S & Scriptwriter  &Agent~\includegraphics[width=0.025\textwidth]{icon/identity.pdf}~\includegraphics[width=0.025\textwidth]{icon/goal.pdf}; Env~\includegraphics[width=0.025\textwidth]{icon/object.pdf}~\includegraphics[width=0.025\textwidth]{icon/rule.pdf}&Configuration \\

-&AnyLogic~\cite{doi:https://doi.org/10.1002/9781118762745.ch12} &S-P & Pre-S & Scriptwriter  &Agent~\includegraphics[width=0.025\textwidth]{icon/identity.pdf}~\includegraphics[width=0.025\textwidth]{icon/goal.pdf}; Env~\includegraphics[width=0.025\textwidth]{icon/object.pdf}~\includegraphics[width=0.025\textwidth]{icon/rule.pdf}&Configuration \\

2023 &AutoGPT~\cite{Significant_Gravitas_AutoGPT} &None & Pre-S & Scriptwriter  &Agent~\includegraphics[width=0.025\textwidth]{icon/identity.pdf}&Configuration; Interface \\

2023 &BabyAGI~\cite{babyagi} &None & Pre-S & Scriptwriter  &Agent~\includegraphics[width=0.025\textwidth]{icon/identity.pdf}&Configuration \\

2023 &CommunityBots~\cite{10.1145/3579469} &None & Pre-S & Director  &Agent~\includegraphics[width=0.025\textwidth]{icon/goal.pdf}&Language; Interface \\
\specialrule{0em}{1pt}{1pt}
2024 &ComPeer~\cite{10.1145/3654777.3676430} &None & Pre-S & Director  &Agent~\includegraphics[width=0.025\textwidth]{icon/action.pdf}&Language; Interface \\

2024 &PrISM-Observer~\cite{arakawa2024prismobserverinterventionagenthelp} &R-P & Pre-S & Director  &Agent~\includegraphics[width=0.025\textwidth]{icon/action.pdf}&Physical \\

2024 &Jaber\etal~\cite{10.1145/3613904.3642183} &R-P & Pre-S & Director  &Agent~\includegraphics[width=0.025\textwidth]{icon/action.pdf}&Physical \\

2024 &Wan\etal~\cite{10.1145/3613905.3651026} &S-P & Pre-S & Director  &Agent~\includegraphics[width=0.025\textwidth]{icon/action.pdf}&Language; Physical \\

2024 &Chen\etal~\cite{10.1145/3613904.3642377} &None & Pre-S & Director  &Agent~\includegraphics[width=0.025\textwidth]{icon/action.pdf}&Interface \\

2024 &Zhang\etal~\cite{10.1145/3613904.3642545} &R-V & Pre-S & Director  &Agent~\includegraphics[width=0.025\textwidth]{icon/action.pdf}&Interface \\

2024 &Text2AC~\cite{10.1145/3613905.3651049} &R-V & Pre-S & Director  &Agent~\includegraphics[width=0.025\textwidth]{icon/identity.pdf}&Language; Interface \\
2023 &Ross\etal~\cite{10.1145/3581641.3584037} &None & Pre-S & Director  &Agent~\includegraphics[width=0.025\textwidth]{icon/goal.pdf}&Language; Interface \\
2024& ChatCam~\cite{10.1145/3699731}&R-P&Pre-S&Director&Agent~\includegraphics[width=0.025\textwidth]{icon/goal.pdf}&Language\\
2024& DrHouse~\cite{10.1145/3699765}&R-P&Pre-S&Director&Agent~\includegraphics[width=0.025\textwidth]{icon/goal.pdf}&Language\\
2024&ChatIoT~\cite{10.1145/3678585}&R-P&Pre-S&Director&Agent~\includegraphics[width=0.025\textwidth]{icon/goal.pdf}&Language\\
2024&CrowdBot~\cite{10.1145/3659601}&R-P&Pre-S&Director&Agent~\includegraphics[width=0.025\textwidth]{icon/goal.pdf}&Language\\
\end{longtable}

\subsection{}\label{Arefine}

\begin{longtable}{>{\arraybackslash}llp{0.8cm}lllp{1.5cm}}
% \centering
\caption[Short Caption]{This table introduces works concerning \textit{Refine the Model}.}
\label{tab:refine} \\

% 下面是表头
\hline \textbf{Year}&\textbf{Work} & \textbf{Env}&\textbf{When} & \textbf{Who} & \textbf{What} & \textbf{How} \\  \hline 
\endfirsthead

% 下面数字3的意思是表格的列数
\multicolumn{7}{c}%
{{\bfseries \tablename\ \thetable{} -- continued from previous page}} \\
\hline \textbf{Year}&\textbf{Work} & \textbf{Env}&\textbf{When} & \textbf{Who} & \textbf{What} & \textbf{How} \\  \hline  
% 注意这里把表头复制了一遍,因为在新的页面也会展示一下表头,不然表格不方便阅读
\endhead

\hline \multicolumn{7}{r}{{Continued on next page}} \\ \hline
\endfoot

\hline \hline
\endlastfoot
\specialrule{0em}{1pt}{1pt}
% 下面就是真正的表格数据了,注意不用再写表头了
2022 & Social Simulacra~\cite{10.1145/3526113.3545616} & S-V & D-S  & Scriptwriter    & Agent~\includegraphics[width=0.025\textwidth]{icon/goal.pdf}; Env~\includegraphics[width=0.025\textwidth]{icon/rule.pdf} & Language      \\
2012 & Prom Week~\cite{10.1145/2282338.2282384} & S-P & D-S  & Director    & Agent~\includegraphics[width=0.025\textwidth]{icon/action.pdf} & Interface     \\
2011 & Prom Week~\cite{10.1145/2159365.2159425} & S-P & D-S  & Director    & Agent~\includegraphics[width=0.025\textwidth]{icon/action.pdf} & Interface     \\
2023 & AGENTVERSE~\cite{chen2023agentversefacilitatingmultiagentcollaboration} & S-P & D-S  & Director    & Sim~\includegraphics[width=0.025\textwidth]{icon/progress.pdf} & Interface     \\
- & - & - & D-S  & Director    & Agent~\includegraphics[width=0.025\textwidth]{icon/goal.pdf} & Language      \\
2023 & ChatEval~\cite{chan2023chatevalbetterllmbasedevaluators} & None & D-S  & Director    & Agent~\includegraphics[width=0.025\textwidth]{icon/goal.pdf}; Sim~\includegraphics[width=0.025\textwidth]{icon/progress.pdf} &   Language; Interface     \\

2024 & Zhang\etal~\cite{zhang2024buildingcooperativeembodiedagents} & S-P & D-S  & Actor    & Agent~\includegraphics[width=0.025\textwidth]{icon/action.pdf} &   Language; Interface     \\
2023 & Memory sandbox~\cite{10.1145/3586182.3615796} & None & D-S  & Scriptwriter    & Agent~\includegraphics[width=0.025\textwidth]{icon/learning.pdf} & Interface     \\
2022 & Inner monologue~\cite{huang2022innermonologueembodiedreasoning} & S-P; R-P & D-S  & Director    & Agent~\includegraphics[width=0.025\textwidth]{icon/goal.pdf} & Language      \\
2022 & Krishna\etal~\cite{doi:10.1073/pnas.2115730119} & R-V & Pre-S   & Director    & Agent~\includegraphics[width=0.025\textwidth]{icon/learning.pdf} & Language      \\
2023 & RoCo~\cite{mandi2023rocodialecticmultirobotcollaboration} & R-P & D-S  & Actor    & Agent~\includegraphics[width=0.025\textwidth]{icon/action.pdf} & Physical; Language      \\
2023 & ChatLaw~\cite{cui2024chatlawmultiagentcollaborativelegal} & None & D-S  & Director    & Agent~\includegraphics[width=0.025\textwidth]{icon/learning.pdf} & Language      \\
\specialrule{0em}{1pt}{1pt}
2023 & Mehta\etal~\cite{mehta2024improvinggroundedlanguageunderstanding} & S-P & D-S  & Director    & Agent~\includegraphics[width=0.025\textwidth]{icon/action.pdf} & Language      \\
\specialrule{0em}{1pt}{0pt}
2020 & Alfred~\cite{shridhar2020alfredbenchmarkinterpretinggrounded} & R-P & D-S  & Director    & Agent~\includegraphics[width=0.025\textwidth]{icon/action.pdf} & Language      \\
2023 & Mohanty\etal~\cite{mohanty2023transforminghumancenteredaicollaboration} & S-P & D-S  & Director    & Agent~\includegraphics[width=0.025\textwidth]{icon/action.pdf} & Language      \\
2022 & Teach~\cite{Padmakumar_Thomason_Shrivastava_Lange_Narayan-Chen_Gella_Piramuthu_Tur_Hakkani-Tur_2022} & R-P & D-S  & Director    & Agent~\includegraphics[width=0.025\textwidth]{icon/action.pdf} & Language      \\
2023& Ren\etal~\cite{ren2023robotsaskhelpuncertainty} & R-P & D-S  & Director    & Agent~\includegraphics[width=0.025\textwidth]{icon/action.pdf} & Language      \\
2005 & MASON~\cite{doi:10.1177/0037549705058073} & S-P & D-S  & Director    & Sim~\includegraphics[width=0.025\textwidth]{icon/progress.pdf} & None \\
2006 & North\etal~\cite{10.1145/1122012.1122013} & S-P & D-S  & Director    & Sim~\includegraphics[width=0.025\textwidth]{icon/progress.pdf} & Interface     \\
2023 & Drive like a human~\cite{unknown} & R-P & D-S  & Director    & Agent~\includegraphics[width=0.025\textwidth]{icon/learning.pdf} & Language      \\
2006 & Guyot~\etal~\cite{guyot2006} & R-V & D-S  & Actor    & Agent~\includegraphics[width=0.025\textwidth]{icon/action.pdf} & Interface     \\
2023 & Surrealdriver~\cite{jin2024surrealdriverdesigningllmpoweredgenerative} & S-P & Pre-S   & Director    & Agent~\includegraphics[width=0.025\textwidth]{icon/learning.pdf} & Data   \\
2023 & The SocialAI School~\cite{kovač2023socialaischoolinsightsdevelopmental} & S-P & Pre-S   & Director    & Agent~\includegraphics[width=0.025\textwidth]{icon/learning.pdf} & Interface     \\
2023 & Choicemates~\cite{park2023choicematessupportingunfamiliaronline} & None & D-S  & Director    & Agent~\includegraphics[width=0.025\textwidth]{icon/action.pdf}~\includegraphics[width=0.025\textwidth]{icon/interaction.pdf} &   Language; Interface     \\
2023 & Ghost in the minecraft~\cite{zhu2023ghostminecraftgenerallycapable} & R-V & D-S  & Director    & Agent~\includegraphics[width=0.025\textwidth]{icon/action.pdf} & Language      \\
2024 & Lu\etal~\cite{10.1145/3672539.3686351} & None & D-S  & Director    & Agent~\includegraphics[width=0.025\textwidth]{icon/action.pdf} & Interface     \\
2024 & Teach AI How to Code~\cite{10.1145/3613904.3642349} & None & D-S  & Director    & Agent~\includegraphics[width=0.025\textwidth]{icon/learning.pdf} &   Language; Interface     \\
2024 & Zhou\etal~\cite{10.1145/3613904.3642812} & None & D-S  & Director    & Agent~\includegraphics[width=0.025\textwidth]{icon/goal.pdf} & Language      \\
2022 & Wordcraft~\cite{10.1145/3490099.3511105} & None & D-S  & Director    & Agent~\includegraphics[width=0.025\textwidth]{icon/goal.pdf} & Interface     \\
2023 & Ross~\etal~\cite{10.1145/3581641.3584037} & None & D-S  & Director    & Sim~\includegraphics[width=0.025\textwidth]{icon/progress.pdf} & Interface    
\end{longtable}
\subsection{}\label{Aevaluate}

\begin{longtable}{>{\arraybackslash}lp{3.1cm}llllp{1.8cm}}
% \centering
\caption[Short Caption]{This table introduces works concerning \textit{Evaluate the Performance}.}
\label{tab:evaluate} \\

% 下面是表头
\hline \textbf{Year}&\textbf{Work} & \textbf{Env}&\textbf{When} & \textbf{Who} & \textbf{What} & \textbf{How} \\  \hline 
\endfirsthead

% 下面数字3的意思是表格的列数
\multicolumn{7}{c}%
{{\bfseries \tablename\ \thetable{} -- continued from previous page}} \\
\hline \textbf{Year}&\textbf{Work} & \textbf{Env}&\textbf{When} & \textbf{Who} & \textbf{What} & \textbf{How} \\  \hline  
% 注意这里把表头复制了一遍,因为在新的页面也会展示一下表头,不然表格不方便阅读
\endhead

\hline \multicolumn{7}{r}{{Continued on next page}} \\ \hline
\endfoot

\hline \hline
\endlastfoot
\specialrule{0em}{1pt}{1pt}
% 下面就是真正的表格数据了,注意不用再写表头了
2023 & Generative Agents~\cite{10.1145/3586183.3606763} & S-P & Post-S   & Actor   & Agent~\includegraphics[width=0.025\textwidth]{icon/action.pdf}~\includegraphics[width=0.025\textwidth]{icon/identity.pdf}~\includegraphics[width=0.025\textwidth]{icon/learning.pdf} & Language\\
- & - & - & Post-S   & Observer   & Agent~\includegraphics[width=0.025\textwidth]{icon/action.pdf} & Data   \\
2022 & Social Simulacra~\cite{10.1145/3526113.3545616} & S-V & Pre-S   & Scriptwriter   & Agent~\includegraphics[width=0.025\textwidth]{icon/goal.pdf}; Env~\includegraphics[width=0.025\textwidth]{icon/rule.pdf} & Language\\
- & - & - & Post-S   & Observer   & Agent~\includegraphics[width=0.025\textwidth]{icon/action.pdf} & Data   \\
2024 & Zhang\etal~\cite{zhang2024buildingcooperativeembodiedagents} & S-P & D-S  & Actor   & Agent~\includegraphics[width=0.025\textwidth]{icon/action.pdf} & Language; Interface   \\
2023 & MetaGPT~\cite{hong2024metagptmetaprogrammingmultiagent} & None & Post-S   & Observer   & Agent~\includegraphics[width=0.025\textwidth]{icon/action.pdf} & Interface   \\
2023 & Argyle\etal~\cite{Argyle_Busby_Fulda_Gubler_Rytting_Wingate_2023} & R-P & Post-S   & Observer   & Agent~\includegraphics[width=0.025\textwidth]{icon/action.pdf} & Data   \\
2023 & AgentSims~\cite{lin2023agentsimsopensourcesandboxlarge} & S-P & D-S  & Observer   & Agent~\includegraphics[width=0.025\textwidth]{icon/action.pdf} & Interface   \\
2023 & Saha\etal~\cite{saha2023languagemodelsteachweaker} & None & D-S  & Actor   & Agent~\includegraphics[width=0.025\textwidth]{icon/action.pdf} & Language\\
\specialrule{0em}{1pt}{1pt}
2023 & CAMEL~\cite{NEURIPS2023_a3621ee9} & None & Post-S   & Observer   & Agent~\includegraphics[width=0.025\textwidth]{icon/action.pdf} & Data   \\
-& BactoWars~\cite{berryman2008review} & S-P & Post-S   & Observer   & Agent~\includegraphics[width=0.025\textwidth]{icon/action.pdf} & Interface   \\
2022 & FAIR\etal~\cite{doi:10.1126/science.ade9097} & R-V & D-S  & Actor   & Agent~\includegraphics[width=0.025\textwidth]{icon/action.pdf} & Interface   \\
2020 & Feng\etal~\cite{10.1145/3394486.3412862} & R-P & Pre-S   & Prototype   & Agent~\includegraphics[width=0.025\textwidth]{icon/identity.pdf} & Data   \\
2023 & H\"{a}m\"{a}l\"{a}inen\etal~\cite{10.1145/3544548.3580688} & R-P & Post-S   & Observer   & Agent~\includegraphics[width=0.025\textwidth]{icon/action.pdf} & Data   \\
2023 & War and Peace~\cite{hua2024warpeacewaragentlarge} & S-P & Post-S   & Observer   & Agent~\includegraphics[width=0.025\textwidth]{icon/action.pdf} & Data   \\
2023 & Surrealdriver~\cite{jin2024surrealdriverdesigningllmpoweredgenerative} & S-P & Post-S   & Observer   & Agent~\includegraphics[width=0.025\textwidth]{icon/action.pdf} & Data   \\
2023 & Modelscope-agent~\cite{li2023modelscopeagentbuildingcustomizableagent} & R-V & Pre-S   & Observer   & Agent~\includegraphics[width=0.025\textwidth]{icon/action.pdf} & Interface   \\
2023 & Liu\etal~\cite{liu2023trainingsociallyalignedlanguage} & S-P & Post-S   & Observer   & Agent~\includegraphics[width=0.025\textwidth]{icon/action.pdf} & Data   \\
2023 & Alympics~\cite{mao2024alympicsllmagentsmeet} & S-V & Post-S   & Observer   & Agent~\includegraphics[width=0.025\textwidth]{icon/action.pdf} & Data   \\
2024 & AgentCoord~\cite{pan2024agentcoordvisuallyexploringcoordination} & S-P & D-S  & Director   & Agent~\includegraphics[width=0.025\textwidth]{icon/action.pdf} & Interface   \\
2023 & Choicemates~\cite{park2023choicematessupportingunfamiliaronline} & None & D-S  & Observer   & Agent~\includegraphics[width=0.025\textwidth]{icon/action.pdf} & Interface   \\
- & - & - & Pre-S   & Director   & Agent~\includegraphics[width=0.025\textwidth]{icon/goal.pdf} & Language; Interface   \\
- & - & - & Post-S   & Observer   & Agent~\includegraphics[width=0.025\textwidth]{icon/action.pdf} & Data   \\
2024 & Schwitzgebel\etal~\cite{https://doi.org/10.1111/mila.12466} & None & Post-S   & Observer   & Agent~\includegraphics[width=0.025\textwidth]{icon/action.pdf} & Data   \\
2024 & Rehearsal~\cite{10.1145/3613904.3642159} & None & D-S  & Director   & Agent~\includegraphics[width=0.025\textwidth]{icon/action.pdf} & Interface   \\
2023 & Wang\etal~\cite{wang2024userbehaviorsimulationlarge} & S-V & Post-S   & Observer   & Agent~\includegraphics[width=0.025\textwidth]{icon/action.pdf} & Data   \\
2023 & Humanoid Agents~\cite{wang2023humanoidagentsplatformsimulating} & S-P & D-S  & Observer   & Agent~\includegraphics[width=0.025\textwidth]{icon/action.pdf} & Interface   \\
- & - & - & Post-S   & Observer   & Agent~\includegraphics[width=0.025\textwidth]{icon/action.pdf} & Data   \\
2023 & Zhang\etal~\cite{10.1145/3626772.3657844} & S-V & Pre-S   & Prototype   & Agent~\includegraphics[width=0.025\textwidth]{icon/identity.pdf} & Data   \\
2024 & SOTOPIA~\cite{zhou2024sotopiainteractiveevaluationsocial} & None & D-S  & Actor   & Agent~\includegraphics[width=0.025\textwidth]{icon/action.pdf} & Interface   \\
-& PedSim~\cite{Ped} & S-P & D-S  & Observer   & Agent~\includegraphics[width=0.025\textwidth]{icon/action.pdf} & Interface   \\
-& AnyLogic~\cite{doi:https://doi.org/10.1002/9781118762745.ch12} & S-P & D-S  & Observer   & Agent~\includegraphics[width=0.025\textwidth]{icon/action.pdf} & Interface   \\
-& AutoGPT~\cite{Significant_Gravitas_AutoGPT} & None & D-S  & Observer   & Agent~\includegraphics[width=0.025\textwidth]{icon/action.pdf} & Interface   \\
-& BabyAGI~\cite{babyagi} & None & D-S  & Observer   & Agent~\includegraphics[width=0.025\textwidth]{icon/action.pdf} & Data   \\
2021 & Siu\etal~\cite{NEURIPS2021_86e8f7ab} & R-V & D-S  & Actor   & Agent~\includegraphics[width=0.025\textwidth]{icon/action.pdf} & Interface   \\
2023 & Eloy\etal~\cite{10.1145/3579598} & S-P & D-S  & Actor   & Agent~\includegraphics[width=0.025\textwidth]{icon/action.pdf} & Language; Interface   \\
2023 & Zubatiy\etal~\cite{10.1145/3610170} & None & D-S  & Director   & Agent~\includegraphics[width=0.025\textwidth]{icon/action.pdf} & Language; Interface   \\
2024 & Jaber\etal~\cite{10.1145/3613904.3642183} & R-P & D-S  & Director   & Agent~\includegraphics[width=0.025\textwidth]{icon/action.pdf} & Physical \\
2024 & Dai\etal~\cite{10.1145/3613905.3637145} & S-P & D-S  & Actor   & Agent~\includegraphics[width=0.025\textwidth]{icon/action.pdf} & Physical; Language\\
2024 & Wan\etal~\cite{10.1145/3613905.3651026} & S-P & D-S  & Director   & Agent~\includegraphics[width=0.025\textwidth]{icon/action.pdf} & Language; Interface   \\
- & - & - & Post-S   & Observer   & Agent~\includegraphics[width=0.025\textwidth]{icon/action.pdf} & Data   \\
2024 & PeerGPT~\cite{10.1145/3613905.3651008} & R-P & D-S  & Actor   & Agent~\includegraphics[width=0.025\textwidth]{icon/action.pdf} & Language\\
2024 & ClassMeta~\cite{10.1145/3613904.3642947} & S-P & D-S  & Actor   & Agent~\includegraphics[width=0.025\textwidth]{icon/action.pdf} & Physical; Language\\
\specialrule{0em}{1pt}{1pt}
2024 & Attig\etal~\cite{10.1145/3613905.3650853} & R-V & D-S  & Actor   & Agent~\includegraphics[width=0.025\textwidth]{icon/action.pdf} & Interface   \\
2024 & Hwang\etal~\cite{10.1145/3613904.3642202} & R-P & D-S  & Director   & Agent~\includegraphics[width=0.025\textwidth]{icon/action.pdf} & Language\\
2024&DrHouse~\cite{10.1145/3699765}&R-P&Post-S&Observer&Agent~\includegraphics[width=0.025\textwidth]{icon/action.pdf}&Data\\
2024&Cuadra\etal~\cite{10.1145/3659624}&R-P&D-S&Director&Agent~\includegraphics[width=0.025\textwidth]{icon/goal.pdf}&Language; Interface\\
2024&Sasha~\cite{10.1145/3643505}&R-P&D-S&Director&Agent~\includegraphics[width=0.025\textwidth]{icon/goal.pdf}&Language\\
\end{longtable}
%%
%% If your work has an appendix, this is the place to put it.


\end{document}
\endinput
%%
%% End of file `sample-manuscript.tex'.
