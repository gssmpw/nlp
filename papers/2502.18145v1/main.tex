%%
%% This is file `sample-manuscript.tex',
%% generated with the docstrip utility.
%%
%% The original source files were:
%%
%% samples.dtx  (with options: `all,proceedings,bibtex,manuscript')
%% 
%% IMPORTANT NOTICE:
%% 
%% For the copyright see the source file.
%% 
%% Any modified versions of this file must be renamed
%% with new filenames distinct from sample-manuscript.tex.
%% 
%% For distribution of the original source see the terms
%% for copying and modification in the file samples.dtx.
%% 
%% This generated file may be distributed as long as the
%% original source files, as listed above, are part of the
%% same distribution. (The sources need not necessarily be
%% in the same archive or directory.)
%%
%%
%% Commands for TeXCount
%TC:macro \cite [option:text,text]
%TC:macro \citep [option:text,text]
%TC:macro \citet [option:text,text]
%TC:envir table 0 1
%TC:envir table* 0 1
%TC:envir tabular [ignore] word
%TC:envir displaymath 0 word
%TC:envir math 0 word
%TC:envir comment 0 0
%%
%%
%% The first command in your LaTeX source must be the \documentclass
%% command.
%%
%% For submission and review of your manuscript please change the
%% command to \documentclass[manuscript, screen, review]{acmart}.
%%
%% When submitting camera ready or to TAPS, please change the command
%% to \documentclass[sigconf]{acmart} or whichever template is required
%% for your publication.
%%
%%

\documentclass[manuscript]{acmart}
%%
%% \BibTeX command to typeset BibTeX logo in the docs
\AtBeginDocument{%
  \providecommand\BibTeX{{%
    Bib\TeX}}}

%% Rights management information.  This information is sent to you
%% when you complete the rights form.  These commands have SAMPLE
%% values in them; it is your responsibility as an author to replace
%% the commands and values with those provided to you when you
%% complete the rights form.
\setcopyright{acmlicensed}
\copyrightyear{2018}
\acmYear{2018}
\acmDOI{XXXXXXX.XXXXXXX}

%% These commands are for a PROCEEDINGS abstract or paper.
% \acmConference[Conference acronym 'XX]{Make sure to enter the correct
%   conference title from your rights confirmation email}{June 03--05,
%   2018}{Woodstock, NY}
% %%
% %%  Uncomment \acmBooktitle if the title of the proceedings is different
% %%  from ``Proceedings of ...''!
% %%
% % \acmBooktitle{Proc. ACM Interact. Mob. Wearable Ubiquitous Technol.}
% \acmISBN{978-1-4503-XXXX-X/18/06}

\graphicspath{{figs/}{figures/}{pictures/}{images/}{./}} % where to search for the images
\usepackage{cleveref}
%% Only used in the template examples. You can remove these lines.
\usepackage{tabu}                      % only used for the table example
\usepackage{booktabs}                  % only used for the table example
\usepackage{lipsum}                    % used to generate placeholder text
\usepackage{mwe}                       % used to generate placeholder figures
% \usepackage{amsfonts,amssymb}
% %% We encourage the use of mathptmx for consistent usage of times font
% %% throughout the proceedings. However, if you encounter conflicts
% %% with other math-related packages, you may want to disable it.
\usepackage{mathptmx}
\usepackage{color}
\usepackage{fancyvrb}
\usepackage{xcolor}
\usepackage{array}
\usepackage{enumitem}
\usepackage{wrapfig}
\usepackage{longtable}
\newcommand{\name}{\textit{System}}
\newcommand{\todo}[1]{\textcolor{red}{[#1]}}
\newcommand{\tocheck}[1]{\textcolor{black}{#1}}
% Abbreviations
\usepackage{xspace,xpunctuate}
\newcommand{\aka}{{a.k.a.},\xspace}
\newcommand{\ie}{\textit{i.e.},\xspace}
\newcommand{\etal}{\xspace\textit{et al.}\xspace}
\newcommand{\eg}{\textit{e.g.},\xspace}
\newcommand{\re}[1]{\textcolor{black}{#1}}
\newcommand{\minor}[1]{\textcolor{orange}{#1}}
%%
%% Submission ID.
%% Use this when submitting an article to a sponsored event. You'll
%% receive a unique submission ID from the organizers
%% of the event, and this ID should be used as the parameter to this command.
%%\acmSubmissionID{123-A56-BU3}

%%
%% For managing citations, it is recommended to use bibliography
%% files in BibTeX format.
%%
%% You can then either use BibTeX with the ACM-Reference-Format style,
%% or BibLaTeX with the acmnumeric or acmauthoryear sytles, that include
%% support for advanced citation of software artefact from the
%% biblatex-software package, also separately available on CTAN.
%%
%% Look at the sample-*-biblatex.tex files for templates showcasing
%% the biblatex styles.
%%

%%
%% The majority of ACM publications use numbered citations and
%% references.  The command \citestyle{authoryear} switches to the
%% "author year" style.
%%
%% If you are preparing content for an event
%% sponsored by ACM SIGGRAPH, you must use the "author year" style of
%% citations and references.
%% Uncommenting
%% the next command will enable that style.
%%\citestyle{acmauthoryear}


%%
%% end of the preamble, start of the body of the document source.
\begin{document}

%%
%% The "title" command has an optional parameter,
%% allowing the author to define a "short title" to be used in page headers.
\title{Carbon and Silicon, Coexist or Compete? A Survey on Human-AI Interactions in Agent-based Modeling and Simulation}

%%
%% The "author" command and its associated commands are used to define
%% the authors and their affiliations.
%% Of note is the shared affiliation of the first two authors, and the
%% "authornote" and "authornotemark" commands
%% used to denote shared contribution to the research.
\author{Ziyue Lin}
\email{ziyuelin917@gmail.com}
\orcid{0009-0002-5485-7379}
\affiliation{%
  \institution{School of Data Science, Fudan University}
   \streetaddress{1 Th{\o}rv{\"a}ld Circle}
  \city{Shanghai}
  \country{China}
}

\author{Siqi Shen}
\email{shensiqi.ssq@alibaba-inc.com}
\affiliation{%
  \institution{DataV Lab, Alibaba Group}
  \city{Hangzhou}
  \country{China}
}
\author{Zichen Cheng}
\email{zccheng19@fudan.edu.cn}
\affiliation{%
  \institution{School of Data Science, Fudan University}
  \city{Shanghai}
  \country{China}
}
\author{Cheok Lam Lai}
\email{thomaslai314159@gmail.com}
\affiliation{%
  \institution{School of Data Science, Fudan University}
  \city{Shanghai}
  \country{China}
}
\author{Siming Chen}
\authornote{Siming Chen is the corresponding author.}
\email{simingchen@fudan.edu.cn}
\orcid{0000-0002-2690-3588}
\affiliation{%
  \institution{School of Data Science, Fudan University}
  \city{Shanghai}
  \country{China}
}
\affiliation{%
  \institution{Shanghai Key Laboratory of Data Science}
  \city{Shanghai}
  \country{China}
}
%%
%% By default, the full list of authors will be used in the page
%% headers. Often, this list is too long, and will overlap
%% other information printed in the page headers. This command allows
%% the author to define a more concise list
%% of authors' names for this purpose.
\renewcommand{\shortauthors}{Lin et al.}

%%
%% The abstract is a short summary of the work to be presented in the
%% article.
\begin{abstract}
Testing Autonomous Driving Systems (ADS) is crucial for ensuring their safety, reliability, and performance. Despite numerous testing methods available that can generate diverse and challenging scenarios to uncover potential vulnerabilities, these methods often treat ADS as a black-box, primarily focusing on identifying system failures like collisions or near-misses without pinpointing the specific modules responsible for these failures. Understanding the root causes of failures is essential for effective debugging and subsequent system repair. We observed that existing methods also fall short in generating diverse failures that adequately test the distinct modules of an ADS, such as perception, prediction, planning and control.

To bridge this gap, we introduce \tool, the first root-cause-aware testing method for ADS. Unlike previous approaches, \tool not only generates scenarios leading to collisions but also showing which specific module triggered the failure. This method targets specific modules, creating test scenarios that highlight the weaknesses of these given modules. Specifically, our approach involves designing module-specific oracles to ascertain module failures and employs a module-directed testing strategy that includes module-specific feedback, adaptive seed selection, and mutation. This strategy guides the generation of tests that effectively provoke module-specific failures. We evaluated \tool across four critical ADS modules and four testing scenarios. The results demonstrate that our method can effectively and efficiently generate scenarios where errors in targeted modules are responsible for ADS failures. It generates 216.7 expected scenarios in total, while the best-performing baseline detects only 79.0 scenarios. 
Our approach represents a significant innovation in ADS testing by focusing on identifying and rectifying module-specific errors within the system, moving beyond conventional black-box failure detection.
\end{abstract}

\keywords{Module-Specific Failure, Autonomous Driving System, Testing}

%%
%% This command processes the author and affiliation and title
%% information and builds the first part of the formatted document.
\maketitle
\section{Introduction}\label{sec:introduction}
% -- Outline
% ---- LLMs are popular
% ---- There're many stakeholders in the training and inference loop
% ---- Adversaries in the training loop are a problem -- malpractice, poisoning
% ---- Also, showing compliance
% ---- Need a framework to prove the integrity of the pipeline
% ---- Enter Atlas

% ---- LLMs are popular
In recent years, machine learning (ML) models, have become increasingly popular.
The pervasive use of large language models (LLMs), in particular, and multi-stakeholder
involvement in model creation and deployment exacerbate security and privacy risks.
These considerations are emphasized by the global nature and the complexity of
large-scale ML deployments with different lifecycle stages:
%gathering and sanitizing the data from different sources,
%training and inferencing across many data centers,
%compliance with local laws or corporate policies.

% ---- There're many stakeholders in the training and inference loop
%Additionally, different stages of the ML development pipeline come with their own stakeholders:
\begin{enumerate}[label=\arabic*)]
    \item Collection and sanitation of a \emph{training} dataset from several public and proprietary sources.
    %\item Solicitation and facilitation of training.
    \item Provisioning of the training environment (hardware and software).
    \item Execution of training across many data centers.
    \item Construction of a \emph{testing} dataset from several sources, and the evaluation.
    \item Deployment and use of the model for inference that is compliant with local laws or corporate policies.
    %\item Use of the model in compliance with local laws or corporate policies.
\end{enumerate}

% ---- Adversaries in the training loop are a problem -- malpractice, poisoning
Each of these stages is vulnerable to malicious or dishonest parties.
For example, data can be poisoned~\cite{biggio2012poisoning,carlini2024poisoning} during collection or training.
Service providers executing outsourced training can shorten or omit critical steps to reduce their cost.
Model providers can serve smaller models in SaaS, or even distribute malicious ones.

% ---- Also, showing compliance
On the other hand, responsible model builders and other stakeholders may be incentivised or required to provide security and trust guarantees.
They may want to prove low bias in their training data, offer easily verifiable performance claims, or guarantee end-to-end integrity of the model creation in high risk domains.

% ---- Need a framework to prove the integrity of the pipeline
To address these challenges, it is necessary to guarantee the integrity of the entire ML lifecycle --
beginning with the data, through the training, and finally, the evaluation and deployment.
Was the data modified?
Did the hardware and software environment adhere to the specification?
Did the contractor follow the specified training procedure?
Can I trust the evaluation?
How can I guarantee that I am interacting with the intended model?
These are example questions that showcase the breadth of the involved challenges that must be tackled to provide end-to-end security.

% --- Enter Atlas
In this work, we introduce \atlas, a framework for enhancing the security and transparency of the lifecycle of ML models.
\atlas establishes the baseline of fundamental components and capabilities needed for comprehensive provenance tracking
at each stage of the ML lifecycle.
Subsequently, \atlas defines the core integrity requirements for verifiable ML lifecycle transparency.
We provide a reference implementation that instantiates \atlas using hardware-based security mechanisms -- with trusted execution environment (TEE),
including attestations.% , and comprehensive metadata-based provenance tracking.
%Our implementation satisfies all \atlas requirements.

We claim the following contributions:
\begin{enumerate}[label=\arabic*.]\label{sec:introduction:contributions}
    \item We introduce \atlas, a framework designed for end-to-end ML lifecycle transparency.
    \item We instantiate \atlas using TEEs and metadata-based provenance tracking.
    \item We evaluate our \atlas prototype through two case studies:
        \begin{enumerate*}[label=\arabic*)]
            \item fine-tuning of a BERT model~\cite{lin2023metabert, lin2023metabertimpl};
            \item fine-tuning of a bge-reranker model~\cite{chen2023bge}
        \end{enumerate*}.
\end{enumerate}

%\msm{revise: Integrate this motivation into intro}
%Organizations frequently leverage pre-trained models, outsource training processes, and integrate components from multiple sources,
%making it difficult to verify the authenticity and trustworthiness of their ML systems. This complexity is further compounded
%by the potential for malicious modifications at various stages of the model lifecycle, from data preparation through deployment.
%The involvement of various third parties in ML model development and deployment
%creates critical challenges in ensuring supply chain integrity.
%
%While Software Bills of Materials (SBOMs) and AI Bills of Materials (AI BOMs) provide basic inventory tracking for model components,
%they fall short in addressing the dynamic nature of ML pipelines. These approaches typically offer point-in-time snapshots but
%fail to capture the complex transformations, fine-tuning operations, and runtime modifications that characterize modern ML workflows.
%Additionally, they lack cryptographic guarantees about the integrity of recorded information and cannot effectively track the provenance
% of model weights and training data.
%
% These approaches demonstrate the growing importance of ML supply chain security.
% However, they are typically applied in an ad-hoc fashion, highlighting the need
% for a more integrated approach that combines comprehensive lineage tracking,
% strong cryptographic properties, and practical integration capabilities with existing ML development and deployment pipelines.
%
%A comprehensive solution requires not just documentation of components, but verifiable evidence of their origins,
%transformations, and integrity throughout the entire model lifecycle. This need has driven interest in more robust
%provenance tracking mechanisms that can:
%
%\begin{itemize}
%\item Provide cryptographic proof of model lineage
%\item Track and verify all pipeline transformations
%\item Maintain tamper-evident records of training processes
%\item Ensure integrity of model artifacts across organizational boundaries
%\end{itemize}
%
%Several existing tools and frameworks
%commonly focusing on different components of the model lifecycle and provenance tracking.
%While these solutions offer valuable capabilities, they often address only specific parts of the end-to-end ML
%supply chain rather than providing comprehensive coverage.
%\msm{end-revise}
%
%\todo{add discussion of EU-CRA AI Act requirements for model documentation and FDA guidelines for AI/ML in healthcare}

%The remainder of this paper is organized as follows:
%in Section~\ref{sec:background-related} we provide an overview of the necessary background, and the related work;
%Section~\ref{sec:problem} presents the challenge of providing integrity in the ML pipeline, the threat model, and the system assumptions;
%in Section~\ref{sec:framework} we present \atlas -- our framework for providing ML integrity;
%Section~\ref{sec:implementation} covers implementation details;
%in Section~\ref{sec:eval}, we show that \atlas is effective across three dimensions: training overhead $<8\%$, the verification time increases linearly with the size of the model, and it is compatible with PyTorch and Tensorflow;
%in Section~\ref{sec:casestudies} we present the case studies;
%in Section~\ref{sec:discussion} we discuss additional considerations for \atlas,
%and Section~\ref{sec:conclusion} concludes the paper and provides directions for future work.

\section{Related Work}
\label{sec:related}

Recent advances~\cite{lecun2015deep, zaidi2022survey} in deep learning have vastly improved object detection and instance segmentation results in the terrestrial domain. 
Such progress has been achieved by developing effective designs of models and training them with large datasets~\cite{lin2014microsoft, russakovsky2015imagenet} containing millions of images and corresponding labels. 
Even with such advances, detecting underwater debris still remains challenging. 
While~\cite{fulton2019robotic} presents the first deep learning based approach to detect underwater debris and outperforms previous non deep learning approaches, the accuracy is worse than general object detection tasks due to a small training dataset. 
To increase the debris detection accuracy,~\cite{hong2020trashcan} proposes a larger dataset, TrashCan, which has both bounding box and pixel-level annotations for object detection and instance segmentation along with baseline results using Mask R-CNN~\cite{he2017mask} and Faster R-CNN~\cite{ren2015faster}. 
However, increasing the dataset size to improve debris detection accuracy further is not scalable due to debris data scarcity and labeling costs. 
To overcome the data scarcity issue,~\cite{hong_generative_2020} proposes a generative method, augmenting the existing dataset with synthetic underwater debris images. 
While the method can create realistic synthetic images, it still requires additional labeling efforts to be used for training detectors. 

Style transfer~\cite{singh_neural_2021,jing_neural_2020} is an approach for changing the appearance of one image based on the visual style of another. 
\cite{rodriguez_domain_2019, yu_sc-uda_2022} use this to improve detection in images taken from various domains (\eg different light conditions and image clarity). 
They aim to account for low-level texture changes in images by updating them to have the same style throughout the data. 
\cite{kadish_improving_2021} also attempts to improve detection using style transfer, by having the detector learn high-level features (\eg object shape) instead of low-level features (\eg the texture of paintings). 
\cite{amirkhani_enhancing_2021} uses style transfer to simulate various types of noise that may be present in real-world data. 
\cite{lin_gan-based_2021,liu_lane_2020} use style transfer to imitate varying light conditions. 
Style transfer has been applied beyond RGB images; \eg\cite{cygert_style_2019} converts RGB images from COCO dataset~\cite{lin2014microsoft} to thermal images and uses them to train a thermal image detector. 
While style transfer works well in augmenting the appearance of an image, it does not add new objects to our data.


Unlike style transfer, image blending based methods allow placing new objects anywhere on target background images. 
\cite{perez_poisson_2003} introduces Poisson editing using Laplacian information to smooth the boundary between the image patches and target images. 
\cite{wu_gp-gan_2019} uses a GAN-based approach for image blending, producing realistic images; however, it requires image pairs of empty backgrounds and objects placed in the backgrounds to train, limiting its use when the source data is limited. 
\cite{georgakis_synthesizing_2017} modifies~\cite{perez_poisson_2003} to find spaces within a given image plane to blend an object. 
However, detectors trained with their synthetic data show degraded performance on real data due to the style discrepancy between the blended objects and backgrounds in the dataset.
\cite{zhang_training_2022} uses a harmonization blending approach to create new data for aerial search and rescue, but it does not blend the boundary of target objects. 

\cite{zhang_deep_2020} presents a two-stage deep network-based approach to blend an image patch onto a background. Unlike~\cite{wu_gp-gan_2019} their approach does not need additional training data to generate blended images.
\begin{figure}  
    \centering
    \scalebox{0.75}{\tikzset{every picture/.style={line width=0.75pt}} 

\begin{tikzpicture}[x=0.75pt,y=0.75pt,yscale=-1,xscale=1]
\draw (-490,101) node  {\includegraphics[width=0.25\textwidth]{imgs/IBURD_firstpass.png}};
\draw (-310,101) node  {\includegraphics[width=0.25\textwidth]{imgs/DIB_secondpass.png}};
\draw (-130,101) node  {\includegraphics[width=0.25\textwidth]{imgs/IBURD_secondpass.png}};

\draw (-560,192) node [anchor=north west][inner sep=0.75pt]   [align=left] {{\fontfamily{helvet}\selectfont Poisson Image Editing}};
\draw (-380,192) node [anchor=north west][inner sep=0.75pt]   [align=left] {{\fontfamily{helvet}\selectfont Deep Image Blending}};
\draw (-180,192) node [anchor=north west][inner sep=0.75pt]   [align=left] {{\fontfamily{helvet}\selectfont IBURD (Ours)}};


\end{tikzpicture}}
    \caption{Comparison of generated images using three approaches:  Poisson image editing~\cite{perez_poisson_2003}, Deep image blending~\cite{zhang_deep_2020} and our method, IBURD. In our approach, we can successfully prevent over-stylization of the blended objects.}  
    \label{fig:compare}
    \vspace{-4mm}
\end{figure}

They use the proposed method mainly for artistic purposes and it struggles with blending transparent source images onto background images, as seen in Fig.~\ref{fig:compare}. 
The method is only tested with $20$ images and takes approximately $4$ minutes to blend one object in an image of size $512\times512$ pixels.

Our proposed approach, IBURD, allows us to place source images at various locations and scales in target background images with relevant bounding box and pixel-level annotations within $50$ seconds, which is $5$ times faster than~\cite{zhang_deep_2020}. 
Our method addresses blending transparent objects using Poisson editing, a situation that previous methods fail to cover.
Additionally, IBURD deals with object distortion due to excessive style transfer using Fast Fourier Transform (FFT)~\cite{liu_image_2008} based weight adjustment for loss.

\section{Method}
\label{sec:method}

\subsection{Weaknesses of Previous Conditioning Methods}

The most popular form of latent image conditioning typically converts conditioning signals to images, before processing them with typical image processing models. While this approach is powerful, it exhibits limitations in handling complex image synthesis tasks, particularly when incorporating heterogeneous or sparse input conditions. Some approaches, such as \textit{LayoutDiffusion} \cite{zheng_layoutdiffusion_2024}, tackle this with custom attention modules that attend to bounding boxes with learned positional embeddings. However, these approaches neglect to include multiple modalities and the relationships between them, which overlooks nuanced interactions between conditioning signals i.e. disambiguating spatial ordering between overlapping boxes. 

% For example, interactions between conditions which may not explicitly exist in the discrete spatial image domain.

% These approaches force diverse modalities, like mixed spatial and categorical information directly into a unified image space, which overlooks nuanced interactions between conditioning signals. For example, interactions between conditions which may not explicitly exist in the discrete spatial image domain.

Previous conditional diffusion research that utilise graph data opt for complex multi-stage training procedures such as masked contrastive pre-training using graph triplets \cite{yang_diffusion-based_2022}. This is not only time-consuming, but also fails to exploit potential benefits of training an end-to-end system that integrates graph data directly into image processing. 
% Furthermore, other work has shown that the repeated conditioning diffusion models (i.e. time or text conditioning) is superior to simply providing   

We tackle these problems by representing images and their conditioning signals as a single graph, which is processed by a bespoke GNN architecture. This allows repeated interactions between conditioning signals and the image throughout the synthesis process, enabling more flexible and dynamic representations that account for both the current image features and interactions between conditioning signals. By maintaining separate pathways for distinct input types, our approach supports heterogeneous and sparse conditioning, leading to better generalisation, finer control, and more precise manipulation of generated images. This simple yet powerful method can be easily integrated into a wide range of existing vision models.

\begin{figure}
    \centering    \includegraphics[width=1\linewidth]{icml2023/hig_fig2.pdf}
\vspace{-20pt}
    \caption{(\textbf{a}) Overview of the proposed architecture. The HIG is encoded into a latent representation through a MP-GNN which is then used as a condition $c_f$ in a ControlNet. (\textbf{b}) Details of the MP-GNN module. Note: HMP is shorthand for heterogenous magnitude preserving operations applied across all nodes.}
    \label{fig:architecture}
\end{figure}

\subsection{Heterogeneous Image Graphs}

To improve on previous approaches we develop a new approach to condition images via the HIG representation. In this manner, we fully exploit variable-length and heterogeneous conditions to aid in image synthesis.

\textbf{Image Graphs.} When faced with the challenge of conditioning images with graphs we first convert images into representations amenable for graph processing. We reshape image features into image nodes pixel-wise in line with other works \cite{liu_cnn-enhanced_2021, han_vision_2022}. In practice, these nodes represent more than a single pixel, for example a latent image patch. This can be due to performing latent image diffusion \cite{rombach_high-resolution_2022, podell_sdxl_2023} where images are first pre-compressed to latent images, or due to prior processing by the image processing model. In contrast to other works \cite{tian_image_nodate, han_vision_2022, tarasiewicz_graph_2021}, we decide to leave image nodes unconnected; this loosely decouples image conditioning from processing. Image nodes are conditioned and later converted back into an image representation, allowing existing architectures to handle processing. Connecting image nodes in a locally dense fashion gains little benefit over highly optimised $3 \times 3$ convolutional operations. Formally, image nodes exist in a discrete space \( f : \mathbb{Z}^2 \to \mathbb{R}^C \). For an image of size \(M \times N\), we define \( f(i, j) \) where \( i, j \in \mathbb{Z} \) and \( 0 \leq i < M \), \( 0 \leq j < N \).

\textbf{Conditioning Graphs}. Conditioning graphs consist of nodes and edges, where each node has features defined as $ g : \mathcal{V} \to \mathbb{R}^F$, where $\mathcal{V}$ represents the set of nodes and $\mathbb{R}^F$ the feature space. Nodes may have spatial ties to the image domain, which we materialise via edges linking image and conditioning nodes. We use conditioning nodes to indicate semantics within the scene, for instance, a node may represent an object (e.g., a \textit{person}). Whereas we utilise different edge types to represent both spatial, abstract relationships and additional semantics. For instance, an edge between two object nodes may encode interactions or attributes (e.g., a person \textit{wearing} a {\textit{yellow}} hat). The graph structure reflects real-world data: often sparse and heterogeneous. We therefore construct graphs on a per task-basis to best leverage the available data and its dependencies.
Formally, each edge \( e \in \mathcal{E} \) connects two nodes \( (v_i, v_j) \in \mathcal{V} \times \mathcal{V} \) and represents a relationship between them. Edges represent any dependency, allowing for abstract relationships to be included.

% To continue the example, if spatial information for both the \textit{person} and the \textit{hat} is available, the graph would contain a node for each object and an edge connecting them, with the edge encoding the relationship \textit{wearing}. 


% \textbf{Conditioning Graphs.} In contrast, conditioning graphs are represented by sets of nodes and edges, with each node having associated features defined by a function $( g : \mathcal{V} \to \mathbb{R}^F$, where $\mathcal{V}$ represents the set of nodes and $\mathbb{R}^F$ the feature space. Although nodes \textit{may} have explicit spatial ties to the discrete image domain, we materialise these through edges between image and conditioning nodes. However, these relationships may be the product of spatial properties of conditioning nodes. As such, subsets of $\mathbb{R^F}$ may represent spatial coordinates \( (x, y) \in \mathbb{R}^2 \) that satisfy \( 0 \leq x < M \) and \( 0 \leq y < N \). Conditioning nodes are not restricted to pixel grid positions, nor the number of spatial dimensions e.g. nodes may represent 3D properties of the real world. Nodes and edges may represent properties independent of spatial dimensions. For example, nodes in the graph can represent concrete objects in the image (e.g., a \textit{person}), while edges between them may represent abstract interactions or attributes (e.g., a person \textit{wearing} a {\textit{yellow}} hat). The graph structure may be sparse, and heterogeneous (multiple types of nodes and edges). Conditioning graphs are constructed on a per-task basis to optimally leverage available data and its dependencies. Formally, each edge \( e \in \mathcal{E} \) connects two nodes \( (v_i, v_j) \in \mathcal{V} \times \mathcal{V} \) and represents a relationship between them. To continue the example, if spatial information for both the \textit{person} and the \textit{hat} is available, the graph would contain a node for each object and an edge connecting them, with the edge encoding the relationship \textit{wearing}. Edges can represent any dependency, allowing for abstract relationships to be included in the graph.

\textbf{Connecting Image and Conditioning Nodes.} With image and conditioning nodes defined, we are close to the complete HIG representation. To enable conditioning between the image and conditioning graphs, we must construct edges between the two. These connections are determined on a per-task basis, depending on the available data, with explicit choices described in Section 4. However, when spatial information is available i.e. segmentation masks or bounding boxes, it enables direct connections between the image graph and the conditioning graph. Specifically, edges are created between image nodes relevant to spatial conditionings (i.e. pixels within the bounding box) and conditioning nodes representing the corresponding semantic class (i.e. class label). This linkage facilitates information flow across the graphs, integrating pixel-level details with higher-level semantic representations. 

% Additionally, the flexibility of heterogeneous GNNs allows for connections from the image back to the graph with different sets of learned weights. This approach enables the image to influence the graph structure while leveraging the rich semantic details present in the image—such as color or object sub-class—throughout much of the diffusion training scheme, while still respecting the different types of information carried by the node types.

\subsection{Model Architecture}

To be compatible with the EDM2 U-Net architecture \footnote{\href{https://github.com/NVlabs/edm2}{https://github.com/NVlabs/edm2}}, we propose the addition of a magnitude-preserving \textit{Heterogenous Image Graph Neural Network} (HIGnn) as the conditioning network to be used in a ControlNet strategy.

\textbf{HIGnn.} The general architecture of the HIG conditioning block requires two primary capabilities: representation switching and HIG processing. To handle switching between image features and image nodes on the HIG we consider the update function $\mathcal{U}_{\text{i}\rightarrow\text{g}}$. This update functions reshapes image features $\mathbf{x_i} \in \mathbb{R}^{N \times C \times H \times W}$ into image nodes pixel wise $\mathbf{x_g} \in \mathbb{R}^{N\cdot H \cdot W \times C}$ and applies an optional projection to ensure correct dimensionality. For the current set of image pixels $\mathbf{x_i}$, we retrieve HIG image nodes $\mathbf{x_g}$ by
\begin{equation}
\mathbf{x_g} = \mathcal{U}_{\text{i}\rightarrow\text{g}}(\mathbf{x_i}) = \hat{W}R(\mathbf{x_i}),  
 \label{eq:HIG_update}
\end{equation}
where $R$ reshapes the image, and $\hat{W}$ is a learned projection with forced magnitude preservation from \cite{karras_analyzing_2024}. Refer to Appendix \ref{appendix:edm2_preliminaries} for greater detail into the mathematical preliminaries of \cite{karras_analyzing_2024}. We consider the reverse operation of converting from graph nodes to an image $\mathcal{U}_{\text{g}\rightarrow\text{i}}$ in a similiar fashion. 

Once we have the HIG updated with current image nodes we can process it with a GNN. We identify several areas where magnitudes can grow and address them each in turn. In practice many varieties of heterogenous message passing GNN could be used, we create our own magnitude preserving graph convolutional operator similiar to Hamilton et al. \cite{hamilton_inductive_2018} for its simplicity and stability. The basic approach propagates information through two branches, a pseudo `skip-connection' applied to the current node, and a learned pooling operation of the local neighbourhood, and we add the ability to include edge information in the neighbourhood pooling. If edge attributes $\mathbf{a}_i$ are present we integrate them via magnitude preserving concatenation to the pooling branch. Formally, the HIGConv operator applied per meta-path to get updated node embeddings $\mathbf{x}_i'$ is defined as:
% \begin{equation}
%     \mathbf{x}_i' = \psi\left(\hat{W}^{\Phi}_1 \mathbf{x}_i +^\text{mp} \hat{W}^{\Phi}_2 \cdot \frac{1}{\sqrt{|\mathcal{N}^{\Phi}|}} \sum_{j \in \mathcal{N}^{\Phi}(i)} [\mathbf{x}_j \|^\text{mp} \mathbf{a}_j] \right),
%     \label{eq:hignn_operator}
% \end{equation}
\begin{equation}
    \mathbf{x}_g' = \psi\left(\hat{W}^{\Phi}_1 \mathbf{x}_g 
    \underset{0 \text{ if } |\mathcal{N}^{\Phi}(i)| = 0}{\underbrace{+^\text{mp} \hat{W}^{\Phi}_2 \cdot \frac{1}{\sqrt{|\mathcal{N}^{\Phi}(i)|}} \sum_{j \in \mathcal{N}^{\Phi}(i)} [\mathbf{x}_j \|^\text{mp} \mathbf{a}_j]}}\right)    \label{eq:hignn_operator}
\end{equation}

% \[
%     \mathbf{x}_i' = \psi\left(\hat{W}^{\Phi}_1 \mathbf{x}_i +^\text{mp} 
%     \underset{+ 0 \text{ if } |\mathcal{N}^{\Phi}(i)| = 0}{\underbrace{\hat{W}^{\Phi}_2 \cdot \frac{1}{\sqrt{|\mathcal{N}^{\Phi}(i)|}} \sum_{j \in \mathcal{N}^{\Phi}(i)} [\mathbf{x}_j \|^\text{mp} \mathbf{a}_j]}}\right).
% \]

where we choose $\psi$ to be magnitude preserving SiLU operator, and $+^\text{mp}$ the magnitude preserving sum (See Appendix \ref{appendix:edm2_preliminaries}), and both meta-path weights $\hat{W}^{\Phi}_1$ and $\hat{W}^{\Phi}_2$ have forced magnitude. $\mathcal{N}$ indicates the local node neighbourhood and is defined by the connectivity of graph. In order to achieve magnitude preservation we first assume all neighbourhood features to be of unit length, we then summate them scale them by the square root of the neighbourhood size ($\sqrt{|\mathcal{N}^{\Phi}|}$), see Appendix \ref{appendix:sum_random} for details. It is important to address unconnected or `zero-degree' nodes, in this case we ignore the right hand side of the equation, and only take the residual path. Note that simply setting the  neighbourhood to zero unintentionally changes the feature magnitudes when mp-sum is applied, since it assumes both vectors to be of unit length. Finally to combine information across meta-paths, we use the same method and sum across paths before normalising by the inverse square root of the number of incoming meta-paths ($|\Phi_i| = |\{\Phi_k \mid x_i \in \Phi_k\}|$)

% To formulate a heterogeneous GNN with learned projections per meta-path ($\mathbf{\Phi} = \{\Phi_1 ... \Phi_n\}$), we must preserve magnitudes when combining meta-paths.

\begin{equation}
\Tilde{\mathbf{x}}_g = \frac{1}{\sqrt{|\Phi_g|}} \sum_{\Phi \in \Phi_g} \mathbf{x}'_g,
\label{eq:meta_path}
\end{equation}

We verify that this approach is guaranteed to maintain magnitudes under certain conditions of the underlying graph data. In particular, for graph-data of sufficient size this approach holds for graphs which do not have identical features attached to the same node since this breaks the independence assumption. 

% An interesting interpretation of this formulation with respect to image synthesis is to observe how different receptive fields change. The typical convolutional operator used in U-Net models define a local image receptive field $\mathcal{R}$, self-attention  defines a global image receptive field $\mathcal{A}$, and the HIGnn defines receptive fields over meta-path relationships $\mathcal{N}^{\Phi}$ for both the image and conditioning variables. We postulate this to an advantage over other conditioning methods as it allows instant communication between different conditioning signals and parts of the image whilst remaining computationally tractable.

\textbf{EDM2 ControlNet Integration.} To integrate conditioning into a generative model, we adopt a strategy similar to ControlNet \cite{zhang_adding_2023}, i.e. a frozen EDM2 pre-trained model, with a trainable copy the encoder integrated with the conditioning HIGnn. Refer to Figure \ref{fig:architecture} for an overview of our proposed architecture, we employ 4 HIG blocks for our base model. The EDM2 checkpoints are only available for class-conditional generation of the 1000 ImageNet classes, yet we find them easy to adapt to our natural image datasets.  To facilitate this we unfreeze the embedding network. To integrate features we adopt $1\times1$ convolutions with a learnable zero-gain in a similar fashion to the original ControlNet, but we note that traditional summation may damage feature magnitudes. We find that naively integrating is harmful to training. Instead, we apply magnitude preserving summation, which, in contrast to the original ControlNet paper, directly alters the primary network features. This yields poor generative quality at step 0, but proves to be quick to train and to be best in practice.

In the trainable encoder we integrate our proposed HIGnn after the initial convolution block. We opt to keep the dimension of the GNN matched to that of the generative model. Finally, to generate samples we opt for the non-stochastic EDM2 sampler, and use the recent advancements in auto-guidance \cite{karras_guiding_2024}, we use our control model as the primary network, and use the unconditional XS ImageNet checkpoint released with EDM2 as the guidance network \cite{karras_analyzing_2024, karras_guiding_2024}. 

% We do not use EMA
\section{Framework}~\label{framework}
This section introduces the framework~(Fig~\ref{fig:tax}) for characterizing the interactions derived from the collected research.
Initially, we introduced the overview of our framework about applying the ``5W1H'' guideline~\cite{ram_5ws_2018} to decompose interactions.
Then, we provided detailed information about the dimensions of ``5W1H''.
Through these interactions, users can push the boundaries of ABMS, catering to personalized research needs.

\begin{figure}[htp]
  \centering
  \includegraphics[width=\linewidth]{figure/taxonomy.pdf}
  \caption{%
    \textbf{The details of our taxonomy.} We have five key dimensions to construct our taxonomy: the goals that users want to achieve~(Why), the phases that users are involved~(When), the roles of users~(Who), the components of the system~(What), and the means of interactions~(How).}
  \label{fig:tax}
\end{figure}

\subsection{Types of Interactions}
Inspired by the taxonomy for human-LLM interactions proposed by Gao\etal~\cite{10.1145/3613905.3650786}, we adapted the ``5W1H'' guideline to categorize interactive methods from existing works.
Through an analysis of the characteristics of interactive methods, we have selected five key dimensions to construct our taxonomy:
\begin{itemize}
\item {\textbf{Why}}: the reasons or motivations behind the interactions. The goals users aim to achieve through interactive methods are difficult to accomplish with static models.
We summarize six goals: initialize the simulation, explore different scenarios, refine the model, evaluate the performance, analyze simulation data, and be immersed in the environment.
\item {\textbf{When}}: the phase at which users are involved in the simulation. 
We have divided it into three phases: pre-simulation~\cite{gao2023s3socialnetworksimulationlarge}, during-simulation~\cite{chen2023agentversefacilitatingmultiagentcollaboration,Padmakumar_Thomason_Shrivastava_Lange_Narayan-Chen_Gella_Piramuthu_Tur_Hakkani-Tur_2022}, and post-simulation~\cite{10520238}.
\item {\textbf{What}}: the components of the system controlled by users. 
Based on the features of the simulation system, we consider three main aspects of the model: agents, environment, and simulation configuration.
Additionally, we perform a secondary classification based on the three aspects, with the specific details explained in Section~\ref {what}.
\item {\textbf{Who}}: the roles users play during the interaction process. 
In this context, we employ an analogy from the field of theater, as the behavior of agents within the model parallels the actions of actors performing in a theatrical setting.
Therefore, we draw upon some related professions to correspond to the roles of users engaged in the model: scriptwriter, director, actor, prototype, and observer.
\item {\textbf{How}}: the means employed by users to interact with the model. 
We categorize it into interface, natural language, configuration setting, data integration, and physical movement.
\end{itemize}

Subsequently, we will provide a detailed description of the four dimensions, excluding the ``When'' dimension. The taxonomy framework is also shown in Figure~\ref{fig:tax}.

\subsection{Why: Classification of Goals}\label{goal}
After reviewing all the literature, we identified six goals that encapsulate the multifaceted role of human engagement in shaping ABMS.
They drive users to interact with the model since a non-interactive model may fail to align with the users' requirements sufficiently.

\textit{Initialize the Simulation.}
The first step, where users lay the foundation for the simulation, ensures that it aligns with the study's objectives.
Users can initiate the simulation by determining factors such as agents' characteristics, environmental variables, and simulation conditions~\cite{10.1145/3586183.3606763, 10.1145/3526113.3545616}.
Besides, users can decide when the simulations begin by issuing a start command~\cite{chen2023agentversefacilitatingmultiagentcollaboration,chan2023chatevalbetterllmbasedevaluators} and posing specific questions or requirements \cite{ren2023robotsaskhelpuncertainty}.
Typically, this step requires the users to incorporate domain-specific knowledge to set up the environment and populate the model with agents that reflect real-world entities or phenomena~\cite{GAUBE201392}.
With users' cooperation, the setup requirements for model initialization are met, laying the groundwork for the simulation to run.
% Humans initiate the simulation by defining the initial conditions, parameters, and agent behaviors. This step often requires the user to input domain-specific knowledge to set up the environment and populate the model with agents that accurately reflect real-world entities or phenomena. By determining factors such as population size, initial agent characteristics, and environmental variables, users lay the foundation for the simulation, ensuring that it aligns with the objectives of the study.

\textit{Explore Different Scenarios.}
One of the critical goals for users in ABMS is to explore various hypothetical scenarios by adjusting key parameters.
It facilitates users exploring how different assumptions or interventions may impact system dynamics~\cite{10.1145/3613904.3642159}.
It also enables the discovery of insights that may not be immediately apparent from the initial model configuration.
Users can conduct "what-if" analysis and test multiple hypotheses in real-time by simulating alternative futures to uncover patterns that are otherwise difficult to detect in a static model~\cite{10.1145/3526113.3545616}.
The iterative process allows for a more thorough analysis of potential risks and opportunities in the modeled system~\cite{hua2024warpeacewaragentlarge}.

% One of the critical tasks for humans in ABMS is to explore various hypothetical scenarios by adjusting key parameters and agent behaviors. This exploration helps to evaluate how different assumptions or interventions might impact system dynamics. By simulating alternative futures, users can identify patterns of agent interactions, system responses to external changes, and potential emergent behaviors. This task enables the discovery of insights that may not be immediately apparent from the initial model configuration.

\textit{Refine the Model.}
If the model's performance falls short of expectations, user intervention is required to improve its effectiveness.
For example, agent behaviors may not align with observed real-world outcomes perfectly due to the simplification of action rules or limitations of the algorithmic capabilities.
To improve the relevance of the simulation results, users can make enhancements or corrections directly through interaction methods~\cite{mandi2023rocodialecticmultirobotcollaboration}.
Additionally, the learning abilities of agents can be improved through user involvement by providing learning materials or managing the agents' 
memory~\cite{jin2024surrealdriverdesigningllmpoweredgenerative,unknown}.
Users can also directly collaborate with agents in solving tasks or guide agents with instructions\cite{mohanty2023transforminghumancenteredaicollaboration,zhang2024buildingcooperativeembodiedagents}.
Due to the randomness inherent in some simulation algorithms, users can refine the model simply by regenerating the results~\cite{10.1145/3526113.3545616}.
The ability to refine models ensures the model's predictive power and validity, leading to more robust and sophisticated outcomes.
% After observing preliminary simulation results, humans refine the model by adjusting parameters, improving agent rules, or incorporating additional factors to enhance the accuracy and relevance of the simulation. This iterative process requires humans to continuously interpret results, identify limitations in the model, and refine agent behaviors to better align with observed or expected real-world outcomes. The ability to refine models dynamically ensures that simulations evolve to reflect more sophisticated or realistic conditions over time.

\textit{Evaluate the Performance.}
Humans play a central role in evaluating the performance of the ABMS by assessing how well the simulation meets predefined goals, such as accurately representing system dynamics, producing meaningful results, or predicting real-world behaviors.
By integrating user-centered metrics, this evaluation typically goes beyond standard quantitative measures~(\eg accuracy, speed).
Furthermore, qualitative feedback by users who incorporate domain-expert knowledge and subjective insights is essential, particularly in the era of LLMs.
Users can assess whether agent behaviors accurately simulate human actions by applying common sense or domain-specific knowledge~\cite{10.1145/3526113.3545616,10.1145/3613905.3651026,wang2023humanoidagentsplatformsimulating}.
It is significant for users to evaluate how effectively the model adapts to different contexts or scenarios and handles changes in user goals, external factors, or input variations~\cite{10.1145/3613905.3651008,10.1145/3613904.3642947}.
% Humans play a central role in evaluating the performance of the ABMS by assessing how well the simulation meets predefined goals, such as accurately representing system dynamics, producing meaningful results, or predicting real-world behaviors. This evaluation involves both qualitative and quantitative measures, with users comparing simulation outputs against historical data, theoretical expectations, or desired outcomes. Human judgment is essential to determine the validity of the model, its predictive power, and the robustness of its results across different scenarios.

\textit{Analyze Simulation Data.}
Analyzing data generated by agent-based modeling and simulation is another critical goal for users.
Simulation data provides the foundation for understanding system behaviors, validating models, and making informed decisions.
Users can observe emergent patterns and system dynamics that may be difficult or impossible to study in the real world~\cite{berryman2008review}.
Furthermore, users employ statistical techniques~\cite{netlogo}, visual analytics~\cite{pan2024agentcoordvisuallyexploringcoordination,10520238}, and domain expertise~\cite{electronics12122722} to extract meaningful insights from the data, which can inform decision-making, strategy recommendations, or further model adjustments. 

% The analysis of data generated by agent-based simulations is another critical task for humans. This involves interpreting large amounts of output data, identifying significant trends, and making sense of complex interactions between agents. Users employ statistical techniques, visual analytics, and domain expertise to extract meaningful insights from the data, which can inform decision-making, policy recommendations, or further model adjustments. Through data analysis, humans transform raw simulation outputs into actionable knowledge.

\textit{Be Immersed in the Environment.}
In contrast to the goals mentioned above, being immersed in the environment emphasizes the user's experience within the simulation without the primary focus being on control or modification.
The focus is less on achieving a specific objective and more on how deeply the user engages with and experiences the simulation.
Direct interactions allow users to engage with agents' worlds actively, enhancing their entertainment experience~\cite{10.1145/3586183.3606763}.
It is most evident in mediums such as video games, virtual reality~(VR), and augmented reality~(AR), where users can fully immerse themselves in dynamic, interactive environments~\cite{mao2024alympicsllmagentsmeet}.
% Interactive ABMS environments often provide users with immersive tools, such as visual interfaces or virtual reality, enabling them to experience the simulated world from an agent's perspective or as an observer. This immersion enhances the user’s ability to understand agent behaviors and interactions at a deeper level. Being immersed in the environment helps humans intuitively grasp the dynamics of complex systems, allowing them to make more informed adjustments to the model or explore emergent phenomena in real-time.

% In summary, these six tasks encapsulate the multifaceted role of humans in shaping, guiding, and refining agent-based modeling and simulation processes. Through these interactions, humans and computational agents work in tandem to explore complex systems, derive meaningful insights, and push the boundaries of knowledge in various domains.

\subsection{What: Components of System}\label{what}
By analyzing the structure of the model, we detailed the components that users can control, focusing on three primary aspects: agents, environment, and simulation configuration.
\subsubsection{Agents} In ABMS, agents are often designed with human-like characteristics to simulate behaviors that closely mimic real-world scenarios. 
Agents are diverse, heterogeneous, and dynamic due to the complex components being divided into internal states and outward behaviors.
Based on the certain characteristics of agents proposed by Macal\etal~\cite{1574234}, we summarized five key components of internal states as follows:

\begin{itemize}
\item {\includegraphics[width=0.05\textwidth]{icon/identity.pdf}\textbf{Identity}}: We considered agents as discrete individuals with a set of attributes and rules~\cite{10.1145/3626772.3657844}. 
Agents can be endowed with human-like traits or specific behavioral abilities and rules~\cite{wang2024userbehaviorsimulationlarge}.

\item {\includegraphics[width=0.05\textwidth]{icon/interaction.pdf}\textbf{Interaction}}: Agents are capable of interacting with other agents, the environment, and humans.
The interactive protocol can include collaboration, competition, hierarchical relationships, or specific communication principles~\cite{hua2024warpeacewaragentlarge,pan2024agentcoordvisuallyexploringcoordination}.

\item {\includegraphics[width=0.05\textwidth]{icon/goal.pdf}\textbf{Goal}}: Agents typically have predefined goals they strive to accomplish. 
It is worth noting that agents may have both long-term~\cite{NEURIPS2023_5950bf29, NEURIPS2023_a3621ee9} and short-term goals~\cite{pan2024agentcoordvisuallyexploringcoordination}.
Long-term goals are strategic and involve sustained effort, while short-term goals are more immediate objectives that serve as incremental steps toward achieving long-term goals~\cite{shridhar2020alfredbenchmarkinterpretinggrounded}.

\item {\includegraphics[width=0.05\textwidth]{icon/autonomy.pdf}\textbf{Automony}}: Agents can function independently, making decisions and taking actions without direct human control. 
Specifically, agents adapt to environmental changes or interactions with other agents.

\item {\includegraphics[width=0.05\textwidth]{icon/learning.pdf}\textbf{Learning Ability}}: Agents have the capacity to learn from their experiences or adapt over time.
This learning ability enables agents to modify their behavior rules based on past outcomes or agents' memory, improving their performance or strategy as the simulation progresses~\cite{cui2024chatlawmultiagentcollaborativelegal,doi:10.1073/pnas.2115730119}.

\end{itemize}

These internal state components collectively govern the outward behaviors like humans:
\begin{itemize}
    \item {\includegraphics[width=0.05\textwidth]{icon/action.pdf}\textbf{Action}}: Observable behaviors performed by agents in response to their environment.
    These actions represent the agent's outward expression of its internal states.
\end{itemize}

Understanding both dimensions is crucial for designing realistic and effective simulations.
Together, these components allow agents to behave in human-like ways, offering rich, complex interactions that drive the sophistication of ABMS.
   
\subsubsection{Environment}
Prior to discussing the components of environments, we first present a classification of environments where agents reside.
The classification is represented across two dimensions: Physical \textit{vs.} Virtual and Real \textit{vs.} Simulated. 
This framework distinguishes environments based on their nature, either grounded in tangible, real-world settings or constructed within virtual or simulated domains.

\textit{Physical vs. Virtual}: The physical environment refers to the actual, physical world where objects, people, and places exist tangibly. 
Examples include homes, offices, streets, and natural settings.
While, the virtual environment refers to the online or digital world, which exists in cyberspace and is accessed through computers, smartphones, or other digital devices.
Examples include social media platforms, online forums, and video games.

\textit{Real vs. Simulated}:
The real environment refers to the world in which humans live and is subject to real-world laws and dynamics.
The simulated environment refers to a virtual or artificially constructed environment that mimics the dynamics of the real world or represents hypothetical scenarios.

By combining the two dimensions, four distinct quadrants are formed to help differentiate the variety of environments agents can inhabit: 
1) \textit{Real-physical} environment represents the world where humans live and interact with tangible objects.
For example, a real kitchen or a physical office where agents~(robots) perform tasks with real-world consequences~\cite{ren2023robotsaskhelpuncertainty}.
2) \textit{Simulated-physical} environment mimics artificially real-world dynamics but is not part of the tangible world.
For instance, a simulated map or virtual town layout is designed to replicate physical environments for testing or exploration purposes~\cite{Cui_2024_WACV}.
3) \textit{Real-virtual} environment is real in the sense that it reflects actual content or social contexts, but it exists in the virtual or digital realm, such as Facebook~\cite{noauthor_meta_nodate}. 
4) \textit{Simulated-virtual} environment is designed to mimic the virtual world accessed by real humans.  
For example, a virtual social media platform is constructed for simulating propagation~\cite{10.1145/3526113.3545616}.

The classification helps us understand how different types of environments are structured and define the necessary components for building effective and relevant environments.
The components of an environment encompass its fundamental structure and governing elements that shape how agents behave and interact within it:
\begin{itemize}
    \item \includegraphics[width=0.05\textwidth]{icon/description.pdf}\textbf{Description}: The description of the environment outlines its key characteristics and defines the scope of the simulation or system.
    It provides a conceptual or formal representation of the environment's purpose, scale, and structure~\cite{park2023choicematessupportingunfamiliaronline,10.1145/3613904.3642159}.
    \item \includegraphics[width=0.05\textwidth]{icon/object.pdf}\textbf{Object}: Objects refer to the elements present within the environment with which agents can interact.
    These can include both tangible and intangible elements depending on whether the environment is physical or not.
    For example, objects may include desks or tables in a physical environment~\cite{ahn2022icanisay}.
    In a virtual environment, objects may include digital assets or virtual entities~\cite{wang2023voyageropenendedembodiedagent}.
    \item \includegraphics[width=0.05\textwidth]{icon/rule.pdf}\textbf{Rule}: Rules are the foundational guidelines that dictate how agents can interact with the environment and each other. 
    They serve as the internal logic of the system, determining the possible actions agents can take and the consequences of those actions~\cite{hua2024warpeacewaragentlarge}.
    These rules often emulate real-world dynamics~(\eg gravity, economics~\cite{LENGNICK2013102}).
    Moreover, they can include limitations or incentives for specific agent behaviors, such as penalties for violating certain rules or rewards for achieving objectives~\cite{10.1145/3526113.3545616,basavatia2023complexworld}.
\end{itemize}

These three components may not all be immediately visible to agents but serve as the underlying framework of the environment.
They determine its foundational regulations, influencing how agents behave and interact at a deeper, systemic level.

\subsubsection{Simulation Configuration}
The simulation brings agents and the environment together to represent and analyze complex systems.
It tracks agents' actions, the environment's evolution, and overall system dynamics over time.
We summarize three main components of the simulation configuration:
\begin{itemize}
    \item \includegraphics[width=0.05\textwidth]{icon/condition.pdf}\textbf{Condition}: The running setup and parameters that define the simulation's model running state, such as the simulation's start and end time and simulation interval for discrete models.
    \item \includegraphics[width=0.05\textwidth]{icon/progress.pdf}\textbf{Progress}: It tracks the temporal evolution of the simulation~\cite{chen2023agentversefacilitatingmultiagentcollaboration}.
    Agents and the environment evolve over time, and monitoring these transitions is crucial to understand the dynamics of the simulation~\cite{pan2024agentcoordvisuallyexploringcoordination}.
    \item \includegraphics[width=0.05\textwidth]{icon/technique.pdf}\textbf{Technique}: It refers to the computational methods and algorithms used to run the simulation.
    For example, depending on the complexity of the model, techniques such as rule-based algorithms~\cite{NEURIPS2021_86e8f7ab}, machine learning~\cite{https://doi.org/10.1111/exsy.13325}, reinforcement learning~\cite{vinyals_grandmaster_2019}, or LLMs~\cite{10.1145/3586183.3606763} may be employed to generate agent behaviors or environmental changes.
\end{itemize}

In summary, agents, environment, and simulation configuration form the three essential elements of ABMS. 
Agents act as autonomous entities within a defined environment, and their interactions and decisions are modeled through the simulation configuration, providing insights into complex systems. 

\subsection{Who: Roles of Human}
Shakespeare said, \textit{``The world is a stage and all the men and women, however, some performers, they all have off time, that the time has game.''}
We find that the roles that users play in interactions can be effectively explained through an analogy from the field of theater.
In the context of ABMS, agents can be regarded as the ``actors'' in a theatrical production, since they have predefined roles that shape their behaviors in predefined scenarios.
Therefore, we classify user roles by drawing upon professions from the theater: scriptwriter, director, actor, prototype, and observer.
It is worth noting that while these roles share similarities with those in the theater, they are not entirely identical.

\includegraphics[width=0.05\textwidth]{icon/scriptwriter.pdf}
\textit{Scriptwriter.} 
The scriptwriter initializes the purpose and structure of the simulation~\cite{10.1145/3526113.3545616,chan2023chatevalbetterllmbasedevaluators}. 
In this role, users are responsible for defining the agents and environments~\cite{lin2023agentsimsopensourcesandboxlarge}, essentially laying the foundation upon which the simulation will run.
They establish the objectives, constraints, and initial conditions to guide the simulation's progression~\cite{jinxin2023cgmiconfigurablegeneralmultiagent}.

\includegraphics[width=0.05\textwidth]{icon/director.pdf}
\textit{Director.} 
As the director, the user controls the timeline and conditions for the simulation, guiding the agents and adjusting parameters during the simulation process. 
Users can direct agents, instructing them to start, pause, or restart the simulation~\cite{chen2023agentversefacilitatingmultiagentcollaboration,ren2023robotsaskhelpuncertainty}.
Users can also offer guidance to agents in a manner akin to a director instructing actors in a performance~\cite{mehta2024improvinggroundedlanguageunderstanding,unknown,park2023choicematessupportingunfamiliaronline}.
In most cases, this role emphasizes managing the flow and direction of the simulation once it is set in motion.

\includegraphics[width=0.05\textwidth]{icon/actor.pdf}
\textit{Actor.}
The actor role represents the user interacting with the simulation as an agent, shifting from passive observation to active engagement.
Users live with other agents as if they were one of them, influencing outcomes by participating in the simulation~\cite{mao2024alympicsllmagentsmeet,zhou2024sotopiainteractiveevaluationsocial, NEURIPS2021_86e8f7ab}.
They interact with other agents or manipulate elements of the environment~\cite{10.1145/3586183.3606763,10.1145/3579598}, which can alter the course of the simulation or help achieve specific goals.
Specifically, other agents also perceive them as agents, other than humans.

\includegraphics[width=0.05\textwidth]{icon/prototype.pdf}
\textit{Prototype.} 
In a theatrical context, some roles are often based on real individuals as prototypes.
Similarly, users can serve as prototypes or references for the agents within the simulation~\cite{Argyle_Busby_Fulda_Gubler_Rytting_Wingate_2023, pmlr-v202-aher23a}.
They provide a basis upon which agents' characteristics can be built.
Unlike the actor, the prototype does not directly participate in the simulation, but influences how agents are designed or programmed.


\includegraphics[width=0.05\textwidth]{icon/observer.pdf}
\textit{Observer.}
The observer takes a passive yet crucial role by monitoring the simulation in real-time, gathering data and insights for further analysis~\cite{NEURIPS2023_a3621ee9,10.1145/3544548.3580688}.
Users watch the simulation unfold without intervening in the process as the audience in a theater.
They further analyze and interpret the behaviors of agents within the simulation, seeking to understand the underlying patterns, trends, or outcomes~\cite{10520238,electronics12122722,hua2024warpeacewaragentlarge}.

The environment serves as the stage where all actions occur and emergent behaviors are like the unscripted moments in a live performance. 
These five roles represent different types of user involvement with ABMS. 
Each role has a unique contribution, from defining and designing the simulation’s framework to actively participating in or passively observing its outcomes, which illustrates the flexibility and depth of user involvement in interactive simulations.


\subsection{How: Means of Interaction}
Various means of interaction allow users to engage with ABMS and ensure that users can effectively exert influence over the simulation.
The primary interaction means are as follows:


\includegraphics[width=0.05\textwidth]{icon/interface.pdf}
% \includegraphics[width=0.025\textwidth]{icon/interface.pdf}
\textit{Interface.}
The user interface~(UI) provides users access to manage the simulation.
Through buttons~\cite{chen2023agentversefacilitatingmultiagentcollaboration} and control panels~\cite{kovač2023socialaischoolinsightsdevelopmental}, users can customize various aspects of the simulation.
Graphical design~\cite{lin2023agentsimsopensourcesandboxlarge} and visualizations~\cite{pan2024agentcoordvisuallyexploringcoordination}, such as charts~\cite{10520238} and real-time agent movements~\cite{10.1145/3613904.3642159} within the environment, enable users to track agent interactions, observe emergent behaviors, and analyze the outcomes of different scenarios.
The interface often provides real-time feedback~\cite{chan2023chatevalbetterllmbasedevaluators} based on user inputs, displaying how changes in parameters affect agent behaviors and simulation outcomes.
Furthermore, it provides users with an intuitive and interactive way to control and analyze simulations, facilitating deeper engagement with the simulation and enhancing the user’s ability to draw meaningful insights.

\includegraphics[width=0.05\textwidth]{icon/language.pdf}
\textit{Natural Language.}
Advances in AI and natural language processing~(NLP)~\cite{bommasani2022opportunities,brown_language_2020}, such as LLMs, enable users to give commands or ask questions in everyday language.
Users are allowed to use natural language commands to control the simulation settings, such as defining agents and environments~\cite{wang2023humanoidagentsplatformsimulating,10.1145/3526113.3545616}.
What's more, users can communicate with agents directly to guide them~\cite{shridhar2020alfredbenchmarkinterpretinggrounded} with high-level goals and low-level instructions or interview them for ``innermost thoughts''~\cite{10.1145/3586183.3606763}.

\includegraphics[width=0.05\textwidth]{icon/configuration.pdf}
\textit{Configuration Setting.}
We categorize methods that involve direct interaction with algorithms as configuration settings, which typically require users to have a programming background.
Configuration files~(like YAML, JSON XML) as user inputs are often used to configure simulation parameters, define agent properties, and set environmental conditions~\cite{wang2024userbehaviorsimulationlarge,hua2024warpeacewaragentlarge}.
Unlike natural language, which is flexible and often ambiguous, the structured text file follows a specific syntax and format. 
It is organized in a hierarchical or key-value structure that can be easily read and interpreted by machines.
Additionally, several libraries and APIs can be applied to construct ABMS~\cite{li2023modelscopeagentbuildingcustomizableagent}.

\includegraphics[width=0.05\textwidth]{icon/data.pdf}
\textit{Data Integration.}
Users can interact with ABMS with external datasets.
For example, agents can utilize users' profile data, such as demographic information, to replicate human samples for enhancing the overall realism of the simulation~\cite{Argyle_Busby_Fulda_Gubler_Rytting_Wingate_2023,gao2023s3socialnetworksimulationlarge}.
On the other hand, users gain simulation data for further analysis~\cite{10.1145/3526113.3545616}.
This data can then be analyzed to extract insights and identify patterns, allowing for informed decision-making or the refinement of the model.


\includegraphics[width=0.05\textwidth]{icon/physical.pdf}
\textit{Physical Movement.}
In certain simulations, especially those involving robotics or virtual reality/augmented reality~(VR/AR), physical movement can be a means of interaction.
Users physically interact with objects or agents in the real world, which in turn affects the simulation ~\cite{mandi2023rocodialecticmultirobotcollaboration,10.1145/3613904.3642183}.
This direct physical contact allows for real-time, hands-on control and interaction with the simulated environment.
On the other hand, in the virtual environment, users can interact with agents and the surroundings through body gestures and facial expressions~\cite{10.1145/3613905.3637145,10.1145/3613904.3642947}.


\section{Findings}\label{finding}
In this section, We demonstrate our findings organized by specific goals~(Why). 
We aim to reveal the most common human-AI interaction patterns as a focal area of study. 
Furthermore, certain patterns remain under-investigated in previous research, raising questions about their entailment and potential future applications.

\subsection{Goal 1: Initialize the Simulation}

Initializing the environment is the most frequently occurring goal in our reviewed literature.
Due to the large number of papers in this category, detailed information can be found in Appendix~\ref{Ainitial}, Table~\ref{tab:initial}.
Firstly, we observe that users interact with the models before the simulation and primarily assume three roles: scriptwriter, director, and prototype.
As scriptwriters, users need to establish a foundational background for the simulation.
Users create agents by defining their identity~\cite{hua2024warpeacewaragentlarge,lin2023agentsimsopensourcesandboxlarge}, interaction~\cite{berryman2008review}, long-term goal~\cite{hong2024metagptmetaprogrammingmultiagent}, and learning ability~\cite{li2023modelscopeagentbuildingcustomizableagent}.
Similarly, users can control the description~\cite{jinxin2023cgmiconfigurablegeneralmultiagent}, objects~\cite{basavatia2023complexworld}, and rules~\cite{10.1145/3526113.3545616} of environments.
Although some studies have utilized natural language command~\cite{hong2024metagptmetaprogrammingmultiagent} and interfaces~\cite{lin2023agentsimsopensourcesandboxlarge}, we find that a portion of the work requires users to engage in configuration settings, such as programming~\cite{netlogo}, graphical programming~\cite{Ped,doi:https://doi.org/10.1002/9781118762745.ch12}, importing packages~\cite{Significant_Gravitas_AutoGPT}, or writing configuration files~\cite{lin2023agentsimsopensourcesandboxlarge}.
Unlike interfaces and natural language commands, these methods present certain challenges for novice users when getting started.
However, they allow for a systematic, modular, and efficient setup of simulations from scratch.
How to combine the advantages of both aspects is a question worth exploring.

Another important role for the user is the director.
The director can directly issue goal commands to the model, prompting agents to begin executing the goals~\cite{rana2023sayplangroundinglargelanguage, ahn2022icanisay} or automatically trigger agents' actions through specific user actions~\cite{10.1145/3613904.3642183, arakawa2024prismobserverinterventionagenthelp}.
Additionally, the director can modify certain environmental settings during the initialization time~\cite{park2023choicematessupportingunfamiliaronline, 10.1145/3613904.3642159}.
The most commonly used means is natural language commands~\cite{10.1145/3678585}, followed by interface~\cite{pan2024agentcoordvisuallyexploringcoordination}.
Compared to the scriptwriter, the director controls the model from a more granular perspective.
Researchers can design appropriate interactions tailored to their specific research needs.
In some cases, users also appear in the role of prototypes and provide demographic data for agent identities.
Before the advancement of computational power, it was common to use sampling methods to select prototypes, and the information dimensions provided to the model were limited~\cite{GAUBE201392}.
Currently, sampling from the dataset is not necessary since the model can handle diverse, heterogeneous data directly~\cite{gao2023s3socialnetworksimulationlarge} with enhanced data processing abilities.

In this category, we can observe the evolution of simulation platforms or toolkits.
Before the maturity of NLP technologies, many works already supported users in initializing simulations.
However, these interactions were not as straightforward as natural language and involved a certain learning curve.
Initially, tools were difficult to use and challenging to learn, such as Netlogo~\cite{netlogo}, EINSTein~\cite{berryman2008review}, and MASON~\cite{doi:10.1177/0037549705058073}, which are required programming skills.
Later, tools like AnyLogic~\cite{doi:https://doi.org/10.1002/9781118762745.ch12} and PedSim~\cite{Ped} emerged, supporting graphical programming and visualizing simulation trajectories, making them more accessible and user-friendly.
With the emergence of large language models, diverse and lightweight simulation platforms have been developed~(\eg AutoGPT~\cite{Significant_Gravitas_AutoGPT} and Modelscope~\cite{li2023modelscopeagentbuildingcustomizableagent}), leveraging the interaction and generative capabilities of these models to support user-customized agents.
This advancement allows users to create tailored agent behaviors and scenarios more intuitively, expanding the flexibility and accessibility of simulation platforms.
We will further discuss the potential development of simulation platforms in Section~\ref{software}. 


\subsection{Goal 2: Explore Different Scenarios}
Investigating various hypothetical scenarios enables users to examine how different assumptions or interventions might influence system dynamics.
The detailed information in this cluster is shown in Table~\ref{tab:explore}.
ChatEval~\cite{chan2023chatevalbetterllmbasedevaluators} supports multi-agent collaboration to compare only two language models' performance at once.
Thus, users need to predefine various models and conduct multiple simulations to compare the comparison results across different models.
The interactions in the remaining works occur during the simulation.

Users act directly as actors, exploring various scenarios through their own diverse behaviors, such as communicating with agents through natural language~\cite{10.1145/3586183.3606763,lin2023agentsimsopensourcesandboxlarge}.
Users can also take on the scriptwriter role, directly altering agents' foundational goals by natural language commands~\cite{10.1145/3586183.3606763}.
This type of work is relatively rare, possibly because users typically focus on exploring the impact of minor changes on the overall system rather than fundamentally altering the foundational setup of agents and the environment within the simulation.
In most cases, users act as directors, controlling the simulation process~\cite{wang2023humanoidagentsplatformsimulating, pan2024agentcoordvisuallyexploringcoordination}, adjusting environmental components~\cite{hua2024warpeacewaragentlarge}, and directing the actions of agents~\cite{DBLP:journals/corr/abs-2312-11813,10.1145/3613904.3642545}, etc.
Typically, by advancing or reversing the simulation progress, users can conduct ``what-if'' analysis~\cite{cui2024chatlawmultiagentcollaborativelegal,10.1145/3526113.3545616}.
``What-if'' analysis is crucial for ABMS, as it enables users to explore the potential effects of various changes within the system.
Users can observe how hypothetical scenarios impact agent behaviors and system dynamics by manipulating specific parameters or altering conditions. 

Current research on this topic is limited based on our review, highlighting a valuable opportunity for future researchers to explore ``what-if'' analysis in human-AI interactions in ABMS.
Such research could facilitate dynamic, in-depth analysis of ABMS and support decision-making processes, advancing the practical utility and impact of ABMS in complex scenarios.
Furthermore, the advent of LLMs enables users to explore different scenarios within the model using natural language and interface.
Designing a user-friendly, voice-enabled interactive interface that allows users to act as a real-world director, complete with a walkie-talkie and monitor screens, may hold significant potential as a research topic.
Users can also take on the role of actors, directly interacting with agents through natural language or physical movement with the advancement of immersive devices. 
They can modify or create diverse scenarios based on research needs.


\begin{table}[ht]
  \caption{This table introduces works concerning \textit{Explore Different Scenarios}. For simplicity, we
shorten the classification of environments: S-P: simulated-physical, S-V: simulated-virtual, R-P: real-physical, R-V: real-virtual. We also shorten the \textit{When} dimension: Pre-S: pre-simulation, D-S: during-simulation, Post-S: post-simulation. For the ``What'' dimension, we use icons instead of text to represent the secondary classification. Subsequent tables will also use similar abbreviations. Some works provide multiple interaction methods for the same goal, such as Generative Agents~\cite{10.1145/3586183.3606763} in this table.~\includegraphics[width=0.025\textwidth]{icon/action.pdf} represents agent \textit{action} and~\includegraphics[width=0.025\textwidth]{icon/goal.pdf} represents agent \textit{goal}.}
  \label{tab:explore}
  % \resizebox{\textwidth}{!}{
  \begin{tabular}{lllllll}
    \toprule
    \textbf{Year}&\textbf{Work} & \textbf{Env}&\textbf{When} & \textbf{Who} & \textbf{What} & \textbf{How}\\
    \midrule
    
2023&Generative Agents~\cite{10.1145/3586183.3606763} & S-P & D-S  & Actor &Agent~\includegraphics[width=0.025\textwidth]{icon/action.pdf}& Language \\

-& -& -& D-S & Scriptwriter &Agent~\includegraphics[width=0.025\textwidth]{icon/goal.pdf}& Language \\

-& -& -& D-S & Director &Env~\includegraphics[width=0.025\textwidth]{icon/object.pdf}& Language \\

2022&Social Simulacra~\cite{10.1145/3526113.3545616}& S-V & D-S  & Director &Sim~\includegraphics[width=0.025\textwidth]{icon/progress.pdf}& Interface \\

2023&ChatEval~\cite{chan2023chatevalbetterllmbasedevaluators}&None & Pre-S & Director &Env~\includegraphics[width=0.025\textwidth]{icon/description.pdf}&Interface \\

2023&AgentSims~\cite{lin2023agentsimsopensourcesandboxlarge}&S-P  & D-S  & Actor &Agent~\includegraphics[width=0.025\textwidth]{icon/action.pdf}&Language; Interface \\

2023&WarAgent~\cite{hua2024warpeacewaragentlarge}
&S-P  & D-S  & Director &Env~\includegraphics[width=0.025\textwidth]{icon/description.pdf}&Language \\

2024&AgentCoord~\cite{pan2024agentcoordvisuallyexploringcoordination}
&S-P & D-S  & Director &Agent~\includegraphics[width=0.025\textwidth]{icon/goal.pdf}; Sim~\includegraphics[width=0.025\textwidth]{icon/progress.pdf}&Interface \\

2024&Rehearsal~\cite{10.1145/3613904.3642159}
&None & D-S  & Director &Agent~\includegraphics[width=0.025\textwidth]{icon/action.pdf}&Interface \\

2023&Humanoid Agents~\cite{wang2023humanoidagentsplatformsimulating}
&S-P & D-S  & Director &Sim~\includegraphics[width=0.025\textwidth]{icon/progress.pdf}&Interface \\

2023&UGI~\cite{DBLP:journals/corr/abs-2312-11813}
&S-P & D-S  & Director &Agent~\includegraphics[width=0.025\textwidth]{icon/action.pdf}&Language \\

2024&Zhang\etal~\cite{10.1145/3613904.3642545}
&R-V & D-S  & Director &Agent~\includegraphics[width=0.025\textwidth]{icon/action.pdf}&Interface \\
2024& ChatCam~\cite{10.1145/3699731}&R-P&D-S&Director&Agent~\includegraphics[width=0.025\textwidth]{icon/action.pdf}&Language\\
2024&Cuadra\etal~\cite{10.1145/3659624}&R-P&D-S&Director&Agent~\includegraphics[width=0.025\textwidth]{icon/goal.pdf}&Language; Interface\\
2024&CrowdBot~\cite{10.1145/3659601}&R-P&D-S&Director&Agent~\includegraphics[width=0.025\textwidth]{icon/goal.pdf}&Language; Physical\\
2024&Sheshadri\etal~\cite{10.1145/3631404}&R-P&D-S&Director&Agent~\includegraphics[width=0.025\textwidth]{icon/action.pdf}&Language\\
  \bottomrule
\end{tabular}
% }
\end{table}


\subsection{Goal 3: Refine the Model}
When the model’s performance fails to meet expectations, improving its effectiveness requires user intervention.
There are 29 papers in this cluster, and the detailed information of papers is shown in Appendix~\ref{Arefine}, Table~\ref{tab:refine}.
Before the simulation, SocialAI School~\cite{gao2023s3socialnetworksimulationlarge}, Krishna\etal~\cite{doi:10.1073/pnas.2115730119} and Surrealdriver~\cite{jin2024surrealdriverdesigningllmpoweredgenerative} guide agents to learn from external resources, such as human natural language instructions and domain expertise data, to enhance their learning ability.
Although there is limited work in this area, it presents a promising approach to refine the model, and new interactions warrant further research.
The majority of methods are implemented during the simulation process.
Some of them also focus on agents' learning abilities.
Users can teach agent human knowledge and domain expertise~\cite{unknown, 10.1145/3613904.3642349, cui2024chatlawmultiagentcollaborativelegal} and directly manipulate memory system~\cite{10.1145/3586182.3615796}.

Some work allows users to directly take on the role of actors, collaborating with agents to complete tasks by natural language~\cite{zhang2024buildingcooperativeembodiedagents} or physical movements~\cite{mandi2023rocodialecticmultirobotcollaboration}.
More frequently, users assume the role of directors, steering agent actions~\cite{mehta2024improvinggroundedlanguageunderstanding, mohanty2023transforminghumancenteredaicollaboration, Padmakumar_Thomason_Shrivastava_Lange_Narayan-Chen_Gella_Piramuthu_Tur_Hakkani-Tur_2022}, goals~\cite{huang2022innermonologueembodiedreasoning,chen2023agentversefacilitatingmultiagentcollaboration}, and interaction~\cite{park2023choicematessupportingunfamiliaronline}.
Additionally, due to the stochastic nature of LLMs, users acting as directors can control the simulation progress through the interface by regenerating outcomes if the current results are unsatisfactory, allowing for the possibility of achieving more desirable outcomes~\cite{chen2023agentversefacilitatingmultiagentcollaboration,chan2023chatevalbetterllmbasedevaluators}.
This cluster appears to overlook the impact of environmental components on refining the model.
Users can potentially reduce obstacles for agents in completing tasks by controlling environmental components.
In addition to agents' learning abilities, users may consider enhancing agents' autonomy—an often-overlooked component in interaction design.


The design of human-AI interactions that harness the strengths of both humans and AI, enabling complementary collaboration, represents a significant area for exploration. 
This approach raises important questions about how best to structure interactions to optimize collaboration and achieve desired outcomes.
From our corpus of papers, we conclude that humans excel in creative thinking, domain expertise, and problem-solving in ambiguous situations, making them adept at tasks requiring abstract thought or out-of-the-box solutions~\cite{ren2023robotsaskhelpuncertainty, 10.1145/2282338.2282384}.
AI operates with consistent accuracy and efficiency, reducing the risk of human error and performing repetitive tasks without fatigue~\cite{10.1145/3672539.3686351}.
Combining these strengths, human-AI interaction has the potential to achieve more comprehensive outcomes, with humans providing complex reasoning abilities and AI enhancing efficiency and scalability.


\subsection{Goal 4: Evaluation the Performance}
Evaluating the ABMS's performance relies on assessing how well the simulation meets predefined goals. Human involvement is central to this process.
In this cluster, we extracted 47 interactions from 41 works.
Due to the large number of papers in this category, detailed information can be found in Appendix~\ref{Aevaluate}, Table~\ref{tab:evaluate}.
For users pre-simulation engaging with the model, the objective is to manipulate specific conditions to assess whether the outcomes align with their expectations.
For example, users can copy community rules and goals from real-world social platforms to the environment of ABMS~\cite{10.1145/3526113.3545616} or design agents' identity modeled on real-world demographic information~\cite{10.1145/3394486.3412862, Argyle_Busby_Fulda_Gubler_Rytting_Wingate_2023}.
Assessing the indistinguishability between agent actions and real user actions provides a measure of the reliability of ABMS simulation results.

Users primarily assume three roles during the simulation: director, actor, and observer.
The director assesses whether agents can adapt flexibly and effectively to the environment by assigning different goals to agents~\cite{10.1145/3643505} or intervening in agent actions~\cite{10.1145/3610170, 10.1145/3613905.3651026}.
The actor role is similar to the director, but they interact directly with agents within the environment, which allows for real-time engagement and firsthand observation of agent actions.
They test the agents' abilities in collaboration~\cite{zhang2024buildingcooperativeembodiedagents}, social interaction~\cite{zhou2024sotopiainteractiveevaluationsocial}, teaching~\cite{saha2023languagemodelsteachweaker}, and strategic gameplay~\cite{NEURIPS2021_86e8f7ab,10.1145/3613905.3650853}.
The observer evaluates agents by tracking their behaviors through graphical interfaces~\cite{park2023choicematessupportingunfamiliaronline,lin2023agentsimsopensourcesandboxlarge,wang2023humanoidagentsplatformsimulating} or log data~\cite{babyagi}, allowing for a detailed assessment of agent actions.

After the simulation, the majority of users, acting as observers, evaluate model performance primarily through analyzing agent action data.
They assess whether the agents perform effectively~\cite{hua2024warpeacewaragentlarge,park2023choicematessupportingunfamiliaronline} or exhibit noticeable differences from real human behaviors~\cite{10.1145/3544548.3580688,10.1145/3526113.3545616,https://doi.org/10.1111/mila.12466}.
MetaGPT~\cite{hong2024metagptmetaprogrammingmultiagent} and BactoWars~\cite{berryman2008review} provide users with interactive interfaces and videos to showcase agent performances.
Notably, Generative Agents~\cite{10.1145/3586183.3606763} proposed a unique evaluation method, interviewing agents as an actor ``reporter''.
After ``two-day'' simulated lives, by designing targeted questions, users can assess whether the agent has self-awareness of its identity, accurate memory, and action aligned with its assigned character traits.
Previous studies have largely overlooked agents' internal states. 
Future research could benefit from emphasizing the alignment between agents' internal states and outward behaviors.

LLM-powered agents are capable of simulating various human-like behaviors and reflecting different characteristics.
The coherence and consistency of LLMs' outputs make agents' behaviors more realistic and believable.
When assessing the believability of simulated behaviors, simplistic quantitative statistical methods are often inadequate. 
In these instances, human qualitative evaluations, such as the Turing test~\cite{Turing2009}, are frequently employed in research to provide more nuanced insights.
It suggests that the advent of LLMs not only introduces new interactions for users in ABMS, but also creates additional interaction requirements.
Designing reliable user experiments to evaluate agent-human resemblance presents several challenges.
Key issues include minimizing user subjectivity to prevent it from skewing evaluation results and determining whether agent behavior alone can reliably indicate human likeness.
Another complexity is interpreting agents' unusual or seemingly illogical actions; while such behaviors might suggest limitations in the agent's mimicry ability, human behavior itself often includes an element of randomness.



\subsection{Goal 5: Analyze Simulation Data}
Analyzing data generated from the ABMS process is a key goal for users.
The datasets involve logs of agents' actions, records of agents' internal state, formation and evolution of networks among agents, spatial and temporal data, etc.
By analyzing the data, users gain insights into system dynamics to support the decision-making process ultimately.
The detailed information about the literature is shown in Table~\ref{tab:analyze}.
Users all act as observers to analyze agents' actions through the interface.
In contrast to assessing the model itself, users analyze data to derive insights for downstream tasks, such as informing real-world decision-making or enhancing predictive capabilities.
Out of the ten works, six are early-developed simulation platforms or toolkits. 
This type of more mature toolkit typically provides users with data analysis modules.
EINSTein, MANA~\cite{berryman2008review}, and Swarm~\cite{minar1996swarm} display basic statistical metrics and visualization, such as tallies of agents detected and killed in battlefield and a time series graph of population dynamics.
Humanoid Agents~\cite{wang2023humanoidagentsplatformsimulating} and AnyLogic~\cite{doi:https://doi.org/10.1002/9781118762745.ch12} both provide a dashboard for users to explore agents' actions over time interactively.
Furthermore, AgentLens~\cite{10520238} and AgentCoord~\cite{cui2024chatlawmultiagentcollaborativelegal} proposed more intricate visual analytics systems to support users interactively investigating details and causes of agents' actions and multi-agent interaction strategy.
We find that the data has evolved from simple statistical metrics to complex, multi-dimensional, heterogeneous forms, such as agent emotions, diverse actions and locations, and dynamic social networks.


The integration of LLMs significantly enhances the richness and complexity of simulation data, which introduces challenges in managing, processing, and interpreting the increased intricacy of the data.
Correspondingly, the evolution of analytical tools, from basic statistical charts to dashboards and then to fully integrated visual analytics systems, reveals an increase in both their analytical capabilities and level of interactivity.
They support more nuanced insights, facilitate decision-making, and allow users to engage with complex data landscapes in a more intuitive, interactive manner.
The development of effective and efficient tools suited for analyzing ABMS data holds substantial potential research value.
For example, integrating machine learning models for data regression or classification could be considered, as well as incorporating NLP techniques to allow users to control the analysis process through natural language commands.
Regarding the \textit{When} dimension, we discover that only a limited number of works support real-time data analysis by users~(during-simulation).
Currently, real-time data analysis is challenging to implement, especially for ABMS developed with LLMs, as they can lead to unstable data generation and low processing efficiency.
Developing stable, real-time, and user-friendly analytical tools requires further investigation.


\begin{table}[ht]
  \caption{This table introduces works concerning \textit{Analyze Simulation Data}.~\includegraphics[width=0.025\textwidth]{icon/action.pdf} represents \textit{action}. }
  \label{tab:analyze}
  % \resizebox{\textwidth}{!}{
  \begin{tabular}{lllllll}
    \toprule
    \textbf{Year}&\textbf{Work} & \textbf{Env}&\textbf{When} & \textbf{Who} & \textbf{What} & \textbf{How}\\
    \midrule
-& AnyLogic~\cite{doi:https://doi.org/10.1002/9781118762745.ch12} & S-P & D-S    & Observer  & Agent~\includegraphics[width=0.025\textwidth]{icon/action.pdf} & Interface   \\
-& EINSTein~\cite{berryman2008review} & S-P & Post-S   & Observer  & Agent~\includegraphics[width=0.025\textwidth]{icon/action.pdf} & Interface   \\
2006 & MANA~\cite{berryman2008review} & S-P & Post-S   & Observer  & Agent~\includegraphics[width=0.025\textwidth]{icon/action.pdf} & Interface   \\
1999 & NetLogo~\cite{netlogo} & S-P & Post-S   & Observer  & Agent~\includegraphics[width=0.025\textwidth]{icon/action.pdf} & Interface  \\
2006 & North\etal~\cite{10.1145/1122012.1122013} & S-P & Post-S   & Observer  & Agent~\includegraphics[width=0.025\textwidth]{icon/action.pdf} & Interface   \\
1996 & Swarm~\cite{minar1996swarm} & S-P & D-S    & Observer  & Agent~\includegraphics[width=0.025\textwidth]{icon/action.pdf} & Interface   \\
2023 & Zarzà\etal~\cite{electronics12122722} & S-P & Post-S   & Observer  & Agent~\includegraphics[width=0.025\textwidth]{icon/action.pdf} & Interface   \\
2024 & AgentLens~\cite{10520238} & S-P & Post-S   & Observer  & Agent~\includegraphics[width=0.025\textwidth]{icon/action.pdf} & Interface   \\
2024 & AgentCoord~\cite{pan2024agentcoordvisuallyexploringcoordination} & S-P & D-S    & Observer  & Agent~\includegraphics[width=0.025\textwidth]{icon/action.pdf}~\includegraphics[width=0.025\textwidth]{icon/interaction.pdf} & Interface   \\
2023 & Humanoid Agents~\cite{wang2023humanoidagentsplatformsimulating} & S-P & Post-S   & Observer  & Agent~\includegraphics[width=0.025\textwidth]{icon/action.pdf} & Interface   \\

  \bottomrule
\end{tabular}
\end{table}

% EINSTein
% MANA
% NetLogo
% Repast
% Swarm
% Anylogic
% Emergent Cooperation and Strategy Adaptation in Multi-Agent Systems: An Extended Coevolutionary Theory with LLMs
% AgentLens
% AgentCoord
% Humanoid Agents


\subsection{Goal 6: Be Immersed in the Environment}\label{immersed}

Immersion in the environment highlights the user’s experience within the simulation, primarily emphasizing engagement rather than control or modification.
The number of papers in this category is relatively small compared to other categories.
According to Table~\ref{tab:immerse}, there are only two papers in the category: Generative Agents~\cite{10.1145/3586183.3606763} and Alympics~\cite{mao2024alympicsllmagentsmeet}.
In both works, users can play as actors and interact with agents as if they were one of them during the simulation in the environment.
In Generative Agents, users can communicate with agents as ``mayor'' or ``reporter'' and change the status of surrounding objects.
In Alympics, human players are engaged in the game with agent players.
The user does not have a predetermined goal but seeks immersion and emotional value in the interaction process in both cases.
Due to the limited work in this area, many interactions remain to be developed.
Users can take on the role of scriptwriter or director,  granting them the ability to control the model from a ``god’s-eye view'' and effectively orchestrate the entire simulation.
This high-level perspective fosters a strong sense of engagement and immersion as users can actively influence the model's narrative and dynamics.
Besides, immersive experience in the virtual reality environment constitutes a significant and valuable area of research.
Users can interact with agents through physical movements and natural language, creating a more intuitive engagement.
We provide a further discussion on immersive experience in Sections~\ref{immersive}.


\begin{table}[ht]
  \caption{This table introduces works concerning \textit{Be Immersed in the Environment}. ~\includegraphics[width=0.025\textwidth]{icon/object.pdf} represents \textit{object}.}
  \label{tab:immerse}
  % \resizebox{\textwidth}{!}{
  \begin{tabular}{lllllll}
    \toprule
    \textbf{Year}&\textbf{Work} & \textbf{Env}&\textbf{When} & \textbf{Who} & \textbf{What} & \textbf{How}\\
    \midrule
    2023&Generative Agents~\cite{10.1145/3586183.3606763} & S-P & D-S  & Actor &Agent~\includegraphics[width=0.025\textwidth]{icon/action.pdf};
Env~\includegraphics[width=0.025\textwidth]{icon/object.pdf}& Language\\
2023&Alympics~\cite{mao2024alympicsllmagentsmeet}&S-V & D-S  & Actor &Agent~\includegraphics[width=0.025\textwidth]{icon/action.pdf}&Interface\\
  \bottomrule
\end{tabular}
\end{table}


\subsection{Application of the Taxonomy}
Our taxonomy and findings can be used in designing human-AI interactions in ABMS that support users' customized implementation to meet research needs.
First, identify the primary goal for interaction~(\textit{Why}). 
We have summarized six goals in~\Cref{goal} that require human involvement to achieve.
Designers determine interaction goals based on our framework to address the practical needs of different research tasks.
According to the goal, designers can find existing interactions in~\Cref{finding}, including the other four dimensions~(\textit{When}, \textit{What}, \textit{Who}, and \textit{How}).
Designers can select the most appropriate interaction from the patterns or be inspired by the potential interactions we have summarized.
Designers must comprehensively consider many aspects to determine the four dimensions, including further refining interaction goals, the feasibility of technical implementation, and other relevant factors.
\section{Suggestions and Research Opportunities}
In this section, we present specific research opportunities identified through the findings in~\Cref{finding} using the proposed taxonomy in~\Cref{framework}.

\subsection{Maxmize the Potential of LLMs}
LLMs are becoming increasingly significant in enhancing interaction in ABMS due to their unique ability to understand and generate intricate human language.
By leveraging LLMs, users benefit from a more intuitive and effective interaction process.
Well-designed prompts can guide LLMs in better simulating human-like behaviors, producing contextually accurate responses, and performing complex tasks autonomously.
Users who lack knowledge of LLMs may struggle to phrase prompts in ways that yield the desired outcomes, which can lead to potentially confusing or unintended results.
Additionally, the complexity of ABMS can further complicate prompt formulation, as users must consider both the model's interpretive limits and the nuances of simulation parameters and agent behaviors.
For example, Generative Agents~\cite{10.1145/3586183.3606763} supports utilizing one paragraph of natural language description to define agent’s identity, including jobs and past experience.
Although such a design provides users with substantial freedom, it can lead to a dilemma where users are uncertain about what to write and may struggle to determine which information is essential to include in the prompt. 
This uncertainty can result in prompts that are either incomplete or overly detailed, diminishing the interaction's effectiveness.
Prompt engineering~\cite{giray_prompt_2023}, which is the process of carefully designing prompts to guide LLMs in generating accurate and contextually appropriate responses, has been widely developed, such as chain-of-thought~\cite{NEURIPS2022_9d560961} and tree-of-thought~\cite{NEURIPS2023_271db992} strategies.
We think the community could explore ways to help users craft effective prompts during interactions.
This research could involve developing adaptive prompt templates tailored to specific tasks, recommending contextually relevant prompts based on the user’s goals, or implementing prompt engineering techniques to refine users' inputs for better results. 
For example, when users must set an agent's identity through natural language, a template can be provided to guide users in specifying the required demographic information (\eg gender, age, occupation).
These approaches aim to reduce the learning curve associated with prompt creation, especially for users less familiar with LLMs, and improve the overall effectiveness of human-AI interactions.
%和agent相关

An increasing number of specialized fields are utilizing interactive ABMS, with LLMs simulating various human roles or professions.
However, simply employing LLMs for basic question-and-answer interactions does not effectively simulate all roles, particularly those requiring domain expertise or complex reasoning abilities.
For roles like these, a more sophisticated approach is needed to capture the depth and nuance of their knowledge and thinking processes.
One possible future direction is to design cognitive architecture for agents to simulate the human thinking processes, such as retrieving and reflecting~\cite{10.1145/3586183.3606763}.
These architectures could enable more realistic and contextually aware responses to model complex human behaviors, making them more effective in roles requiring higher expertise and adaptive decision-making.
Instruction tuning~\cite{zhang2024instructiontuninglargelanguage} is another strategy to improve the performance of LLMs by training them to follow specific types of instructions more accurately.
By fine-tuning models with instruction data specific to a field, LLMs can better understand and execute nuanced, technically complex instructions that align with domain professionals' expectations.
Instruction tunning techniques have been applied in various domains~\cite{zhang2023multitaskinstructiontuningllama,liu2023goatfinetunedllamaoutperforms}, however, there is limited research addressing it in the HCI community. %具体的问题
We hope that our research can inspire future researchers in this area.


%trust issue


\subsection{Simulation Software Development}\label{software}
Before the maturity of natural language technologies, users typically built ABMS on simulation software platforms~\cite{doi:10.1177/0037549706073695,berryman2008review}. %代表性工具,例如netlogo, agenttorch
These platforms did not support natural language interaction, requiring users to rely on more technical interfaces, which also involved a certain learning cost.
ABMS simulation platform with integrated natural language processing techniques may be required to enable users to interact with agents and control simulations using natural language commands, enhancing accessibility and ease of use. 
The platform could make ABMS more user-friendly and applicable across various domains, even for those without programming expertise.
Although there exist some platforms that enable the creation, deployment, and management of agents leveraging LLMs, such as autoGPT~\cite{Significant_Gravitas_AutoGPT}, AgentTorch~\cite{chopra2023agenttorch}, they still require users to have a certain level of programming knowledge.
It is important to design simulation software accessible to users with minimal technical expertise by incorporating natural language processing capabilities. 
In addition to implementing natural language interaction, other AI technologies could also be considered. 
For example, integrate machine learning algorithms to recommend relevant commands or next steps to users based on the user’s current actions, simulation state, or previous interaction sequences. 

ABMS is a versatile tool applied across numerous fields to simulate complex systems, analyze collective behaviors, and make predictions.
Different fields have unique design requirements for interactive ABMS platforms.
Each domain may prioritize distinct features, interaction methods, and data integration needs to meet specific goals effectively.
For example, economic simulations prioritize high-frequency interaction options, such as adjusting market parameters or agent strategies in real-time~\cite{helbing_agent-based_2012}.
While simulations in social science often need agents with complex, varied behaviors to model interactions like group dynamics, migration, or policy effects~\cite{gao2023s3socialnetworksimulationlarge}.
Developing simulation platforms for specific domains may empower professionals and researchers to address real-world challenges.
They could include agents and models prebuilt for the domain, tailor the interface and interaction options to the specific needs of the field, and offer analysis tools and visualization options that highlight metrics crucial to the domain.
Furthermore, the platform could include AI components or expert systems specific to the domain to support more realistic simulations.




\subsection{Immersive Experience}\label{immersive}%ubicomp文章
As discussed in Section~\ref{immersed}, we find that there is limited research on users' immersive experiences currently.
Popular science fiction TV series, \textit{Westworld}, set in a futuristic, highly immersive theme park populated by lifelike AI agents, which allows human guests to live out their fantasies in a Western-themed world without consequences.
As agent technology advances, the science fiction scenarios portrayed in the series are increasingly approaching reality.
Research on user immersive experience in ABMS is currently most relevant in the context of video games, such as role-playing games~(RPGs).
Värtinen\etal~\cite{c7c0852d5f324ba5907ee22bea26560c} generated role-playing game quests with LLMs to fulfill player demands toward more and richer game content.
By understanding how ABMS contributes to immersion, game developers can create environments that foster emotional investment, realistic social dynamics, and greater player satisfaction.
Additionally, insights gained may benefit other fields involving immersive environments, such as virtual reality.
Furthermore, as biotechnology and materials science advance to new levels, the concept of physical parks akin to \textit{Westworld} may become feasible.
Users would interact with physical agents through \textit{Natural Language} and \textit{Physical Movements}, creating highly immersive experiences.

Another potential application scenario is companion agents designed to provide emotional support. 
The rapid advancement of high technology has created a sense of disconnection and emotional distance, paradoxically leaving people feeling more alone despite constant virtual contact. 
Digital interactions often replace direct, face-to-face connections.
An inner emptiness or emotional void emerges, leading to a growing need for meaningful interaction and companionship.
These agents could offer companionship, simulate meaningful conversations, and respond empathetically to users' needs.
This application requires careful attention to emotional intelligence, personalization, and ethical considerations to ensure that the agents are both supportive and safe for users.
We believe that the user immersive experience in ABMS holds significant research value.


\section{Discussion}
In this section, we discuss some lessons learned during our work. 
We first introduce the trust issue and ethical problem arising from human-AI interactions.
We further discuss the future relationship between humans and AI.
It is hoped that this will stimulate further reflection among researchers.

\subsection{Trust Issue}
While LLMs offer many conveniences for interaction methods, they also introduce potential risks, such as the issue of ``hallucinations''~\cite{yao2024llmlieshallucinationsbugs}. %总结幻觉类型,哪个环节出现问题
This phenomenon occurs when the model generates inaccurate or misleading information with high confidence. It can undermine the reliability of ABMS outcomes, especially in critical applications.
Inspired by the algorithmic fidelity criteria proposed by Argyle\etal~\cite{Argyle_Busby_Fulda_Gubler_Rytting_Wingate_2023}, we have concluded three kinds of ``hallucinations'' in ABMS: 1)  generated outputs are distinguishable from parallel humans; 2) generated outputs are inconsistent with the predefined demographic information of agents; 3) generated outputs proceed unnaturally from the form, tone, and content of the context provided.
As a result, humans may experience trust issues with AI-generated outputs, which could pose risks for subsequent applications.
Therefore, exploring how human-AI interactions can mitigate the impact of hallucinations generated by LLMs can also be an important area of research.
For example, designing interactive mechanisms that allow users to verify, correct, or override misleading responses in real time could enhance the reliability of LLMs. 
Additionally, integrating feedback loops where users can flag inconsistencies or request clarifications may help manage and reduce the influence of hallucinations in critical ABMS applications.
On the other hand, designing appropriate mechanisms for LLMs to display their reasoning process transparently can enhance human trust.
Users can better grasp how conclusions are drawn and how outputs are generated. 
This transparency can help mitigate skepticism and uncertainty, allowing users to assess the model's logic and reliability more effectively.

\subsection{Ethical Problem}
Ethical problems arising from human-AI interactions in ABMS are a significant concern. 
Identifying ethical issues and exploring solutions is crucial in the field of HCI.
We provide two examples for reference as follows.
First, some ABMS rely on detailed data about individual demographics uploaded by users, especially in fields such as healthcare, urban planning, or the social sciences.
Using personal or sensitive data can risk breaching individuals' privacy if not handled securely or anonymized properly.
It is essential to use privacy-preserving techniques and comply with data protection laws to prevent unauthorized data access or misuse.
Comprehensive protection mechanisms need to be established to safeguard privacy and ensure the secure handling of sensitive data, ensure ethical use and transparency.
Second, simulated behaviors may inadvertently perpetuate biases and stereotypes embedded in LLMs' training data.
The training dataset may incorporate biases related to race, gender, ethnicity, and other characteristics~\cite{lucy-bamman-2021-gender}.
As a result, ethical considerations require researchers to take an active role in mitigating these biases.
Nonetheless, thoroughly identifying and mitigating all potential biases and stereotypes remains challenging, requiring continued research to further enhance and ensure the fairness of these models.

\subsection{Paradox of Coexist \textit{vs.} Compete}
\textit{``Carbon and Silicon, Coexist or Compete?''}, in the title, we raise the question of whether human~(carbon-based) and agent~(silicon-based) entities can coexist collaboratively or are destined to compete within shared environments in the future.
As generative AI systems demonstrate unprecedented reasoning, creativity, and autonomous decision-making capabilities, critical questions emerge: will humans and agents evolve as collaborative partners, or will their interactions devolve into zero-sum competition?
Modern AI exhibits dual potential as both ``augmenters'' and ``displacers'' of human capabilities.
It demonstrates how AI can amplify professional productivity while simultaneously threatening current occupations. 
Nevertheless, we think the human-AI relationship transcends binary competition or cooperation dichotomies, evolving instead as a ``recursive partnership'' where each entity redefines the other's capabilities.
In our paper, we examine diverse types of interactive modes between humans and agents, encompassing both egalitarian and hierarchical dynamics, as well as collaborative and directive forms of engagement.
The decisive factor is to implement adaptive governance frameworks that align AI's emergent properties with anthropogenic values.
Humans must establish clear boundaries, accountability frameworks, and trust mechanisms to ensure AI is used responsibly and beneficially.
The future relationship between humans and AI remains uncertain. 
Through our discussion of interactions in  ABMS, we aim to offer a perspective that may guide future researchers in exploring this evolving dynamic.
% We have comprehended three limitations of our review and suggest possible solutions for them.
% First, the classification of goals can be more fine-grained.
% In reviewing and coding the corpus of papers, we found that objectives can be further divided into finer-grained categories.
% For example, the goal, \textit{Initialize the Environment}, include defining the agents~\cite{pan2024agentcoordvisuallyexploringcoordination}, giving agents instructions~\cite{cui2024chatlawmultiagentcollaborativelegal}, configure the environments~\cite{jinxin2023cgmiconfigurablegeneralmultiagent}, etc.
% \textit{Evaluate the Performance} can be divided into assessing effectiveness~\cite{park2023choicematessupportingunfamiliaronline} or believability~\cite{10.1145/3394486.3412862}.
% Each sub-goal may correspond to specific interaction patterns.
% We did not pursue further subdivisions because they would be too detailed.
% Thus, we choose the classification at a higher level.
% In future work, we will focus on one or two specific goals and perform a detailed classification to uncover deeper insights.
% Additionally, we proposed a general framework for categorizing environments but did not conduct an in-depth analysis. 
% Researchers interested in this topic are encouraged to utilize our framework and data, as we believe it holds the potential for uncovering valuable insights.

% %放在前面
% Second, our study may not comprehensively cover all relevant literature, especially research from earlier periods.
% A significant reason is that prior to the maturity of natural language processing (NLP) technologies, related work was relatively limited, with some studies built on existing classical simulation platforms we discussed above, such as NetLogo~\cite{netlogo}.
% Furthermore, these research efforts span multiple fields, making it challenging to gather a comprehensive collection systematically.
% Our review indicates that the interaction methods employed in these studies are relatively constrained.
% While some studies may not have been captured in our collection, we are confident that our framework can also effectively account for those methods.
% We also plan to keep updating our corpus of papers with uncollected works and the latest papers to track emerging research trends.

% Third, we did not analyze or discuss the application scenarios of these human-AI interactive ABMS.
% Since the scope of applications is extensive and continually expanding, covering fields such as law~\cite{cui2024chatlawmultiagentcollaborativelegal}, software development~\cite{qian2024chatdevcommunicativeagentssoftware, 10.1145/3581641.3584037}, video game~\cite{mao2024alympicsllmagentsmeet}, education~\cite{Padmakumar_Thomason_Shrivastava_Lange_Narayan-Chen_Gella_Piramuthu_Tur_Hakkani-Tur_2022, 10.1145/3613905.3651008}, household~\cite{10.1145/3613904.3642183, ren2023robotsaskhelpuncertainty}.
% Furthermore, rapid advancements in technology continuously introduce novel use cases that do not fit neatly within traditional categories.
% To address this limitation, we plan to further propose a flexible, multi-dimensional framework that allows for the systematic analysis and categorization of use cases, making it adaptable to new domains and scalable as the application landscape.
\section{Conclusion}
We conduct a systematic survey of 97 research studies on human-AI interactions in agent-based modeling and simulation in various domains from 1996 to 2024.
We first propose a novel taxonomy to categorize the interactions extracted from collected works.
We decompose each interaction into five dimensions according to the ``5W1H'' guideline.
Specially, we employ an analogy from the field of theater and draw upon some related professions to correspond to the roles of users.
Through our analysis, we answered the research question: \textit{How do humans and AI interact in the context of ABMS to fulfill user research requirements?}
Furthermore, we synthesize findings from existing literature to uncover interaction patterns, identify research gaps, and propose future research directions for human-AI interactions in agent-based modeling and simulation.

%%
%% The next two lines define the bibliography style to be used, and
%% the bibliography file.
\bibliographystyle{ACM-Reference-Format}
\bibliography{sample-base}

\appendix

\section{Appendix}
\subsection{}\label{Ainitial}

\begin{longtable}{>{\arraybackslash}lp{2.7cm}p{0.8cm}llp{3.5cm}p{2cm}}
% \centering
\caption[Short Caption]{This table introduces works concerning \textit{Initialize the Simulation}.}
\label{tab:initial} \\

% 下面是表头
\hline \textbf{Year}&\textbf{Work} & \textbf{Env}&\textbf{When} & \textbf{Who} & \textbf{What} & \textbf{How} \\  \hline 
\endfirsthead

% 下面数字3的意思是表格的列数
\multicolumn{7}{c}%
{{\bfseries \tablename\ \thetable{} -- continued from previous page}} \\
\hline \textbf{Year}&\textbf{Work} & \textbf{Env}&\textbf{When} & \textbf{Who} & \textbf{What} & \textbf{How} \\  \hline  
% 注意这里把表头复制了一遍,因为在新的页面也会展示一下表头,不然表格不方便阅读
\endhead

\hline \multicolumn{7}{r}{{Continued on next page}} \\ \hline
\endfoot

\hline \hline
\endlastfoot
\specialrule{0em}{1pt}{1pt}
2023&Generative Agents~\cite{10.1145/3586183.3606763}&S-P & Pre-S & Scriptwriter   &Agent~\includegraphics[width=0.025\textwidth]{icon/identity.pdf}&Language \\

2022&Social Simulacra~\cite{10.1145/3526113.3545616}&S-V & Pre-S & Scriptwriter   &Agent~\includegraphics[width=0.025\textwidth]{icon/identity.pdf}~\includegraphics[width=0.025\textwidth]{icon/goal.pdf}; Env~\includegraphics[width=0.025\textwidth]{icon/rule.pdf}&Language \\

2023&ChatEval~\cite{chan2023chatevalbetterllmbasedevaluators}&None & Pre-S & Scriptwriter   &Agent~\includegraphics[width=0.025\textwidth]{icon/identity.pdf}~\includegraphics[width=0.025\textwidth]{icon/learning.pdf} &Configuration \\

2023&MetaGPT~\cite{hong2024metagptmetaprogrammingmultiagent}&None & Pre-S & Scriptwriter   &Agent~\includegraphics[width=0.025\textwidth]{icon/goal.pdf}&Language \\

2023&Argyle\etal~\cite{Argyle_Busby_Fulda_Gubler_Rytting_Wingate_2023}&R-P & Pre-S & Prototype   &Agent~\includegraphics[width=0.025\textwidth]{icon/identity.pdf}&Data \\

2023&SayPlan~\cite{rana2023sayplangroundinglargelanguage}&S-P & Pre-S & Director   &Agent~\includegraphics[width=0.025\textwidth]{icon/goal.pdf}&Language \\

2023&AgentSims~\cite{lin2023agentsimsopensourcesandboxlarge}&S-P & Pre-S & Scriptwriter   &Agent~\includegraphics[width=0.025\textwidth]{icon/identity.pdf}~\includegraphics[width=0.025\textwidth]{icon/learning.pdf}; Env~\includegraphics[width=0.025\textwidth]{icon/object.pdf}&Interface; Configuration \\
2022&Huang\etal~\cite{huang2022innermonologueembodiedreasoning}&S-P; R-P & Pre-S & Director   &Agent~\includegraphics[width=0.025\textwidth]{icon/goal.pdf}&Language \\

2023&$S^3$~\cite{gao2023s3socialnetworksimulationlarge}&S-V & Pre-S & Prototype   &Agent~\includegraphics[width=0.025\textwidth]{icon/identity.pdf}&Data \\

2023&Ahn\etal~\cite{ahn2022icanisay}&R-P & Pre-S & Director   &Agent~\includegraphics[width=0.025\textwidth]{icon/goal.pdf}&Language \\

2022&WebShop~\cite{NEURIPS2022_82ad13ec}&S-V & Pre-S & Director   &Agent~\includegraphics[width=0.025\textwidth]{icon/goal.pdf}&Language \\

2023&Mind2Web~\cite{NEURIPS2023_5950bf29}&R-V & Pre-S & Director   &Agent~\includegraphics[width=0.025\textwidth]{icon/goal.pdf}&Language \\

2023&CAMEL~\cite{NEURIPS2023_a3621ee9}&None & Pre-S & Director   &Agent~\includegraphics[width=0.025\textwidth]{icon/goal.pdf}&Language \\

2023&Aher\etal~\cite{pmlr-v202-aher23a}&R-P & Pre-S & Prototype   &Agent~\includegraphics[width=0.025\textwidth]{icon/identity.pdf}&Data \\

2023&CGMI~\cite{jinxin2023cgmiconfigurablegeneralmultiagent}&S-P & Pre-S & Scriptwriter   &Env~\includegraphics[width=0.025\textwidth]{icon/description.pdf}&Language \\

2023&ChatLaw~\cite{cui2024chatlawmultiagentcollaborativelegal}&None & Pre-S & Director   &Agent~\includegraphics[width=0.025\textwidth]{icon/goal.pdf}&Language \\

2020&Alfred~\cite{shridhar2020alfredbenchmarkinterpretinggrounded}&R-P & Pre-S & Director   &Agent~\includegraphics[width=0.025\textwidth]{icon/goal.pdf}&Language \\
\specialrule{0em}{1pt}{1pt}
2023&Ren\etal~\cite{ren2023robotsaskhelpuncertainty}&R-P & Pre-S & Director   &Agent~\includegraphics[width=0.025\textwidth]{icon/goal.pdf}&Language \\

2023&ChatDev~\cite{qian2024chatdevcommunicativeagentssoftware}&None & Pre-S & Director   &Agent~\includegraphics[width=0.025\textwidth]{icon/goal.pdf}&Language \\

-&BactoWars~\cite{berryman2008review}&S-P & Pre-S & Scriptwriter  &Agent~\includegraphics[width=0.025\textwidth]{icon/identity.pdf}~\includegraphics[width=0.025\textwidth]{icon/interaction.pdf}; Env~\includegraphics[width=0.025\textwidth]{icon/object.pdf}&Configuration \\

-&EINSTein~\cite{berryman2008review}&S-P & Pre-S & Scriptwriter  &Agent~\includegraphics[width=0.025\textwidth]{icon/identity.pdf}~\includegraphics[width=0.025\textwidth]{icon/interaction.pdf}; Env~\includegraphics[width=0.025\textwidth]{icon/description.pdf}; Sim~\includegraphics[width=0.025\textwidth]{icon/technique.pdf}&Configuration \\

-&MANA~\cite{berryman2008review} &S-P & Pre-S & Scriptwriter  &Env~\includegraphics[width=0.025\textwidth]{icon/description.pdf}; Sim~\includegraphics[width=0.025\textwidth]{icon/technique.pdf}&Interface \\

2005&MASON~\cite{doi:10.1177/0037549705058073}&S-P & Pre-S & Scriptwriter   &Agent~\includegraphics[width=0.025\textwidth]{icon/identity.pdf}~\includegraphics[width=0.025\textwidth]{icon/learning.pdf}; Sim~\includegraphics[width=0.025\textwidth]{icon/technique.pdf}&Interface \\

1999&NetLogo~\cite{netlogo}&S-P & Pre-S & Scriptwriter  &Agent~\includegraphics[width=0.025\textwidth]{icon/identity.pdf}~\includegraphics[width=0.025\textwidth]{icon/interaction.pdf}&Interface \\

2006&North\etal~\cite{10.1145/1122012.1122013}&S-P & Pre-S & Scriptwriter  &Agent~\includegraphics[width=0.025\textwidth]{icon/identity.pdf}~\includegraphics[width=0.025\textwidth]{icon/learning.pdf}; Sim~\includegraphics[width=0.025\textwidth]{icon/technique.pdf} &Interface; Configuration \\

1996&Minar\etal~\cite{minar1996swarm}&S-P & Pre-S & Scriptwriter  &Agent~\includegraphics[width=0.025\textwidth]{icon/identity.pdf}~\includegraphics[width=0.025\textwidth]{icon/interaction.pdf}; Env~\includegraphics[width=0.025\textwidth]{icon/object.pdf}~\includegraphics[width=0.025\textwidth]{icon/rule.pdf} &Configuration \\

2023&ComplexWorld~\cite{basavatia2023complexworld}
&S-V & Pre-S & Scriptwriter  &Env~\includegraphics[width=0.025\textwidth]{icon/description.pdf}~\includegraphics[width=0.025\textwidth]{icon/object.pdf}~\includegraphics[width=0.025\textwidth]{icon/rule.pdf} &Language \\

2024&Cui\etal~\cite{Cui_2024_WACV}
&S-P & Pre-S & Director  &Agent~\includegraphics[width=0.025\textwidth]{icon/goal.pdf}&Language \\

2013&Gaube\etal~\cite{GAUBE201392}
&R-P & Pre-S & Prototype  &Agent~\includegraphics[width=0.025\textwidth]{icon/identity.pdf}&Data \\

2023&WarAgent~\cite{hua2024warpeacewaragentlarge}
&S-P & Pre-S & Scriptwriter  &Agent~\includegraphics[width=0.025\textwidth]{icon/identity.pdf}~\includegraphics[width=0.025\textwidth]{icon/interaction.pdf}; Env~\includegraphics[width=0.025\textwidth]{icon/description.pdf}~\includegraphics[width=0.025\textwidth]{icon/rule.pdf}&Configuration \\

2018&Kavak\etal~\cite{10.5555/3213032.3213044}
&S-P & Pre-S & Prototype  &Agent~\includegraphics[width=0.025\textwidth]{icon/identity.pdf}&Data \\

2023&Modelscope-agent~\cite{li2023modelscopeagentbuildingcustomizableagent}
&R-V & Pre-S & Scriptwriter  &Agent~\includegraphics[width=0.025\textwidth]{icon/identity.pdf}~\includegraphics[width=0.025\textwidth]{icon/learning.pdf}&Configuration \\

2024&AgentCoord~\cite{pan2024agentcoordvisuallyexploringcoordination}
&S-P & Pre-S & Director  &Agent~\includegraphics[width=0.025\textwidth]{icon/goal.pdf}&Interface \\

& & & & Scriptwriter  &Agent~\includegraphics[width=0.025\textwidth]{icon/identity.pdf}~\includegraphics[width=0.025\textwidth]{icon/interaction.pdf} &Interface \\

2023&Choicemates~\cite{park2023choicematessupportingunfamiliaronline}
&None & Pre-S & Director  &Agent~\includegraphics[width=0.025\textwidth]{icon/goal.pdf}; Env~\includegraphics[width=0.025\textwidth]{icon/description.pdf}&Language; Interface \\

2024&Rehearsal~\cite{10.1145/3613904.3642159}
&None & Pre-S & Director  &Env~\includegraphics[width=0.025\textwidth]{icon/description.pdf}&Interface \\

2023&Wang\etal~\cite{wang2024userbehaviorsimulationlarge}
&S-V & Pre-S & Scriptwriter  &Agent~\includegraphics[width=0.025\textwidth]{icon/identity.pdf}&Configuration \\

2023&Humanoid Agents~\cite{wang2023humanoidagentsplatformsimulating}
&S-P & Pre-S & Scriptwriter  &Agent~\includegraphics[width=0.025\textwidth]{icon/identity.pdf}&Language \\

2023&Zhu\etal~\cite{zhu2023ghostminecraftgenerallycapable}
&R-V & Pre-S & Director  &Agent~\includegraphics[width=0.025\textwidth]{icon/goal.pdf}&Language \\

-&PedSim~\cite{Ped} &S-P & Pre-S & Scriptwriter  &Agent~\includegraphics[width=0.025\textwidth]{icon/identity.pdf}~\includegraphics[width=0.025\textwidth]{icon/goal.pdf}; Env~\includegraphics[width=0.025\textwidth]{icon/object.pdf}~\includegraphics[width=0.025\textwidth]{icon/rule.pdf}&Configuration \\

-&AnyLogic~\cite{doi:https://doi.org/10.1002/9781118762745.ch12} &S-P & Pre-S & Scriptwriter  &Agent~\includegraphics[width=0.025\textwidth]{icon/identity.pdf}~\includegraphics[width=0.025\textwidth]{icon/goal.pdf}; Env~\includegraphics[width=0.025\textwidth]{icon/object.pdf}~\includegraphics[width=0.025\textwidth]{icon/rule.pdf}&Configuration \\

2023 &AutoGPT~\cite{Significant_Gravitas_AutoGPT} &None & Pre-S & Scriptwriter  &Agent~\includegraphics[width=0.025\textwidth]{icon/identity.pdf}&Configuration; Interface \\

2023 &BabyAGI~\cite{babyagi} &None & Pre-S & Scriptwriter  &Agent~\includegraphics[width=0.025\textwidth]{icon/identity.pdf}&Configuration \\

2023 &CommunityBots~\cite{10.1145/3579469} &None & Pre-S & Director  &Agent~\includegraphics[width=0.025\textwidth]{icon/goal.pdf}&Language; Interface \\
\specialrule{0em}{1pt}{1pt}
2024 &ComPeer~\cite{10.1145/3654777.3676430} &None & Pre-S & Director  &Agent~\includegraphics[width=0.025\textwidth]{icon/action.pdf}&Language; Interface \\

2024 &PrISM-Observer~\cite{arakawa2024prismobserverinterventionagenthelp} &R-P & Pre-S & Director  &Agent~\includegraphics[width=0.025\textwidth]{icon/action.pdf}&Physical \\

2024 &Jaber\etal~\cite{10.1145/3613904.3642183} &R-P & Pre-S & Director  &Agent~\includegraphics[width=0.025\textwidth]{icon/action.pdf}&Physical \\

2024 &Wan\etal~\cite{10.1145/3613905.3651026} &S-P & Pre-S & Director  &Agent~\includegraphics[width=0.025\textwidth]{icon/action.pdf}&Language; Physical \\

2024 &Chen\etal~\cite{10.1145/3613904.3642377} &None & Pre-S & Director  &Agent~\includegraphics[width=0.025\textwidth]{icon/action.pdf}&Interface \\

2024 &Zhang\etal~\cite{10.1145/3613904.3642545} &R-V & Pre-S & Director  &Agent~\includegraphics[width=0.025\textwidth]{icon/action.pdf}&Interface \\

2024 &Text2AC~\cite{10.1145/3613905.3651049} &R-V & Pre-S & Director  &Agent~\includegraphics[width=0.025\textwidth]{icon/identity.pdf}&Language; Interface \\
2023 &Ross\etal~\cite{10.1145/3581641.3584037} &None & Pre-S & Director  &Agent~\includegraphics[width=0.025\textwidth]{icon/goal.pdf}&Language; Interface \\
2024& ChatCam~\cite{10.1145/3699731}&R-P&Pre-S&Director&Agent~\includegraphics[width=0.025\textwidth]{icon/goal.pdf}&Language\\
2024& DrHouse~\cite{10.1145/3699765}&R-P&Pre-S&Director&Agent~\includegraphics[width=0.025\textwidth]{icon/goal.pdf}&Language\\
2024&ChatIoT~\cite{10.1145/3678585}&R-P&Pre-S&Director&Agent~\includegraphics[width=0.025\textwidth]{icon/goal.pdf}&Language\\
2024&CrowdBot~\cite{10.1145/3659601}&R-P&Pre-S&Director&Agent~\includegraphics[width=0.025\textwidth]{icon/goal.pdf}&Language\\
\end{longtable}

\subsection{}\label{Arefine}

\begin{longtable}{>{\arraybackslash}llp{0.8cm}lllp{1.5cm}}
% \centering
\caption[Short Caption]{This table introduces works concerning \textit{Refine the Model}.}
\label{tab:refine} \\

% 下面是表头
\hline \textbf{Year}&\textbf{Work} & \textbf{Env}&\textbf{When} & \textbf{Who} & \textbf{What} & \textbf{How} \\  \hline 
\endfirsthead

% 下面数字3的意思是表格的列数
\multicolumn{7}{c}%
{{\bfseries \tablename\ \thetable{} -- continued from previous page}} \\
\hline \textbf{Year}&\textbf{Work} & \textbf{Env}&\textbf{When} & \textbf{Who} & \textbf{What} & \textbf{How} \\  \hline  
% 注意这里把表头复制了一遍,因为在新的页面也会展示一下表头,不然表格不方便阅读
\endhead

\hline \multicolumn{7}{r}{{Continued on next page}} \\ \hline
\endfoot

\hline \hline
\endlastfoot
\specialrule{0em}{1pt}{1pt}
% 下面就是真正的表格数据了,注意不用再写表头了
2022 & Social Simulacra~\cite{10.1145/3526113.3545616} & S-V & D-S  & Scriptwriter    & Agent~\includegraphics[width=0.025\textwidth]{icon/goal.pdf}; Env~\includegraphics[width=0.025\textwidth]{icon/rule.pdf} & Language      \\
2012 & Prom Week~\cite{10.1145/2282338.2282384} & S-P & D-S  & Director    & Agent~\includegraphics[width=0.025\textwidth]{icon/action.pdf} & Interface     \\
2011 & Prom Week~\cite{10.1145/2159365.2159425} & S-P & D-S  & Director    & Agent~\includegraphics[width=0.025\textwidth]{icon/action.pdf} & Interface     \\
2023 & AGENTVERSE~\cite{chen2023agentversefacilitatingmultiagentcollaboration} & S-P & D-S  & Director    & Sim~\includegraphics[width=0.025\textwidth]{icon/progress.pdf} & Interface     \\
- & - & - & D-S  & Director    & Agent~\includegraphics[width=0.025\textwidth]{icon/goal.pdf} & Language      \\
2023 & ChatEval~\cite{chan2023chatevalbetterllmbasedevaluators} & None & D-S  & Director    & Agent~\includegraphics[width=0.025\textwidth]{icon/goal.pdf}; Sim~\includegraphics[width=0.025\textwidth]{icon/progress.pdf} &   Language; Interface     \\

2024 & Zhang\etal~\cite{zhang2024buildingcooperativeembodiedagents} & S-P & D-S  & Actor    & Agent~\includegraphics[width=0.025\textwidth]{icon/action.pdf} &   Language; Interface     \\
2023 & Memory sandbox~\cite{10.1145/3586182.3615796} & None & D-S  & Scriptwriter    & Agent~\includegraphics[width=0.025\textwidth]{icon/learning.pdf} & Interface     \\
2022 & Inner monologue~\cite{huang2022innermonologueembodiedreasoning} & S-P; R-P & D-S  & Director    & Agent~\includegraphics[width=0.025\textwidth]{icon/goal.pdf} & Language      \\
2022 & Krishna\etal~\cite{doi:10.1073/pnas.2115730119} & R-V & Pre-S   & Director    & Agent~\includegraphics[width=0.025\textwidth]{icon/learning.pdf} & Language      \\
2023 & RoCo~\cite{mandi2023rocodialecticmultirobotcollaboration} & R-P & D-S  & Actor    & Agent~\includegraphics[width=0.025\textwidth]{icon/action.pdf} & Physical; Language      \\
2023 & ChatLaw~\cite{cui2024chatlawmultiagentcollaborativelegal} & None & D-S  & Director    & Agent~\includegraphics[width=0.025\textwidth]{icon/learning.pdf} & Language      \\
\specialrule{0em}{1pt}{1pt}
2023 & Mehta\etal~\cite{mehta2024improvinggroundedlanguageunderstanding} & S-P & D-S  & Director    & Agent~\includegraphics[width=0.025\textwidth]{icon/action.pdf} & Language      \\
\specialrule{0em}{1pt}{0pt}
2020 & Alfred~\cite{shridhar2020alfredbenchmarkinterpretinggrounded} & R-P & D-S  & Director    & Agent~\includegraphics[width=0.025\textwidth]{icon/action.pdf} & Language      \\
2023 & Mohanty\etal~\cite{mohanty2023transforminghumancenteredaicollaboration} & S-P & D-S  & Director    & Agent~\includegraphics[width=0.025\textwidth]{icon/action.pdf} & Language      \\
2022 & Teach~\cite{Padmakumar_Thomason_Shrivastava_Lange_Narayan-Chen_Gella_Piramuthu_Tur_Hakkani-Tur_2022} & R-P & D-S  & Director    & Agent~\includegraphics[width=0.025\textwidth]{icon/action.pdf} & Language      \\
2023& Ren\etal~\cite{ren2023robotsaskhelpuncertainty} & R-P & D-S  & Director    & Agent~\includegraphics[width=0.025\textwidth]{icon/action.pdf} & Language      \\
2005 & MASON~\cite{doi:10.1177/0037549705058073} & S-P & D-S  & Director    & Sim~\includegraphics[width=0.025\textwidth]{icon/progress.pdf} & None \\
2006 & North\etal~\cite{10.1145/1122012.1122013} & S-P & D-S  & Director    & Sim~\includegraphics[width=0.025\textwidth]{icon/progress.pdf} & Interface     \\
2023 & Drive like a human~\cite{unknown} & R-P & D-S  & Director    & Agent~\includegraphics[width=0.025\textwidth]{icon/learning.pdf} & Language      \\
2006 & Guyot~\etal~\cite{guyot2006} & R-V & D-S  & Actor    & Agent~\includegraphics[width=0.025\textwidth]{icon/action.pdf} & Interface     \\
2023 & Surrealdriver~\cite{jin2024surrealdriverdesigningllmpoweredgenerative} & S-P & Pre-S   & Director    & Agent~\includegraphics[width=0.025\textwidth]{icon/learning.pdf} & Data   \\
2023 & The SocialAI School~\cite{kovač2023socialaischoolinsightsdevelopmental} & S-P & Pre-S   & Director    & Agent~\includegraphics[width=0.025\textwidth]{icon/learning.pdf} & Interface     \\
2023 & Choicemates~\cite{park2023choicematessupportingunfamiliaronline} & None & D-S  & Director    & Agent~\includegraphics[width=0.025\textwidth]{icon/action.pdf}~\includegraphics[width=0.025\textwidth]{icon/interaction.pdf} &   Language; Interface     \\
2023 & Ghost in the minecraft~\cite{zhu2023ghostminecraftgenerallycapable} & R-V & D-S  & Director    & Agent~\includegraphics[width=0.025\textwidth]{icon/action.pdf} & Language      \\
2024 & Lu\etal~\cite{10.1145/3672539.3686351} & None & D-S  & Director    & Agent~\includegraphics[width=0.025\textwidth]{icon/action.pdf} & Interface     \\
2024 & Teach AI How to Code~\cite{10.1145/3613904.3642349} & None & D-S  & Director    & Agent~\includegraphics[width=0.025\textwidth]{icon/learning.pdf} &   Language; Interface     \\
2024 & Zhou\etal~\cite{10.1145/3613904.3642812} & None & D-S  & Director    & Agent~\includegraphics[width=0.025\textwidth]{icon/goal.pdf} & Language      \\
2022 & Wordcraft~\cite{10.1145/3490099.3511105} & None & D-S  & Director    & Agent~\includegraphics[width=0.025\textwidth]{icon/goal.pdf} & Interface     \\
2023 & Ross~\etal~\cite{10.1145/3581641.3584037} & None & D-S  & Director    & Sim~\includegraphics[width=0.025\textwidth]{icon/progress.pdf} & Interface    
\end{longtable}
\subsection{}\label{Aevaluate}

\begin{longtable}{>{\arraybackslash}lp{3.1cm}llllp{1.8cm}}
% \centering
\caption[Short Caption]{This table introduces works concerning \textit{Evaluate the Performance}.}
\label{tab:evaluate} \\

% 下面是表头
\hline \textbf{Year}&\textbf{Work} & \textbf{Env}&\textbf{When} & \textbf{Who} & \textbf{What} & \textbf{How} \\  \hline 
\endfirsthead

% 下面数字3的意思是表格的列数
\multicolumn{7}{c}%
{{\bfseries \tablename\ \thetable{} -- continued from previous page}} \\
\hline \textbf{Year}&\textbf{Work} & \textbf{Env}&\textbf{When} & \textbf{Who} & \textbf{What} & \textbf{How} \\  \hline  
% 注意这里把表头复制了一遍,因为在新的页面也会展示一下表头,不然表格不方便阅读
\endhead

\hline \multicolumn{7}{r}{{Continued on next page}} \\ \hline
\endfoot

\hline \hline
\endlastfoot
\specialrule{0em}{1pt}{1pt}
% 下面就是真正的表格数据了,注意不用再写表头了
2023 & Generative Agents~\cite{10.1145/3586183.3606763} & S-P & Post-S   & Actor   & Agent~\includegraphics[width=0.025\textwidth]{icon/action.pdf}~\includegraphics[width=0.025\textwidth]{icon/identity.pdf}~\includegraphics[width=0.025\textwidth]{icon/learning.pdf} & Language\\
- & - & - & Post-S   & Observer   & Agent~\includegraphics[width=0.025\textwidth]{icon/action.pdf} & Data   \\
2022 & Social Simulacra~\cite{10.1145/3526113.3545616} & S-V & Pre-S   & Scriptwriter   & Agent~\includegraphics[width=0.025\textwidth]{icon/goal.pdf}; Env~\includegraphics[width=0.025\textwidth]{icon/rule.pdf} & Language\\
- & - & - & Post-S   & Observer   & Agent~\includegraphics[width=0.025\textwidth]{icon/action.pdf} & Data   \\
2024 & Zhang\etal~\cite{zhang2024buildingcooperativeembodiedagents} & S-P & D-S  & Actor   & Agent~\includegraphics[width=0.025\textwidth]{icon/action.pdf} & Language; Interface   \\
2023 & MetaGPT~\cite{hong2024metagptmetaprogrammingmultiagent} & None & Post-S   & Observer   & Agent~\includegraphics[width=0.025\textwidth]{icon/action.pdf} & Interface   \\
2023 & Argyle\etal~\cite{Argyle_Busby_Fulda_Gubler_Rytting_Wingate_2023} & R-P & Post-S   & Observer   & Agent~\includegraphics[width=0.025\textwidth]{icon/action.pdf} & Data   \\
2023 & AgentSims~\cite{lin2023agentsimsopensourcesandboxlarge} & S-P & D-S  & Observer   & Agent~\includegraphics[width=0.025\textwidth]{icon/action.pdf} & Interface   \\
2023 & Saha\etal~\cite{saha2023languagemodelsteachweaker} & None & D-S  & Actor   & Agent~\includegraphics[width=0.025\textwidth]{icon/action.pdf} & Language\\
\specialrule{0em}{1pt}{1pt}
2023 & CAMEL~\cite{NEURIPS2023_a3621ee9} & None & Post-S   & Observer   & Agent~\includegraphics[width=0.025\textwidth]{icon/action.pdf} & Data   \\
-& BactoWars~\cite{berryman2008review} & S-P & Post-S   & Observer   & Agent~\includegraphics[width=0.025\textwidth]{icon/action.pdf} & Interface   \\
2022 & FAIR\etal~\cite{doi:10.1126/science.ade9097} & R-V & D-S  & Actor   & Agent~\includegraphics[width=0.025\textwidth]{icon/action.pdf} & Interface   \\
2020 & Feng\etal~\cite{10.1145/3394486.3412862} & R-P & Pre-S   & Prototype   & Agent~\includegraphics[width=0.025\textwidth]{icon/identity.pdf} & Data   \\
2023 & H\"{a}m\"{a}l\"{a}inen\etal~\cite{10.1145/3544548.3580688} & R-P & Post-S   & Observer   & Agent~\includegraphics[width=0.025\textwidth]{icon/action.pdf} & Data   \\
2023 & War and Peace~\cite{hua2024warpeacewaragentlarge} & S-P & Post-S   & Observer   & Agent~\includegraphics[width=0.025\textwidth]{icon/action.pdf} & Data   \\
2023 & Surrealdriver~\cite{jin2024surrealdriverdesigningllmpoweredgenerative} & S-P & Post-S   & Observer   & Agent~\includegraphics[width=0.025\textwidth]{icon/action.pdf} & Data   \\
2023 & Modelscope-agent~\cite{li2023modelscopeagentbuildingcustomizableagent} & R-V & Pre-S   & Observer   & Agent~\includegraphics[width=0.025\textwidth]{icon/action.pdf} & Interface   \\
2023 & Liu\etal~\cite{liu2023trainingsociallyalignedlanguage} & S-P & Post-S   & Observer   & Agent~\includegraphics[width=0.025\textwidth]{icon/action.pdf} & Data   \\
2023 & Alympics~\cite{mao2024alympicsllmagentsmeet} & S-V & Post-S   & Observer   & Agent~\includegraphics[width=0.025\textwidth]{icon/action.pdf} & Data   \\
2024 & AgentCoord~\cite{pan2024agentcoordvisuallyexploringcoordination} & S-P & D-S  & Director   & Agent~\includegraphics[width=0.025\textwidth]{icon/action.pdf} & Interface   \\
2023 & Choicemates~\cite{park2023choicematessupportingunfamiliaronline} & None & D-S  & Observer   & Agent~\includegraphics[width=0.025\textwidth]{icon/action.pdf} & Interface   \\
- & - & - & Pre-S   & Director   & Agent~\includegraphics[width=0.025\textwidth]{icon/goal.pdf} & Language; Interface   \\
- & - & - & Post-S   & Observer   & Agent~\includegraphics[width=0.025\textwidth]{icon/action.pdf} & Data   \\
2024 & Schwitzgebel\etal~\cite{https://doi.org/10.1111/mila.12466} & None & Post-S   & Observer   & Agent~\includegraphics[width=0.025\textwidth]{icon/action.pdf} & Data   \\
2024 & Rehearsal~\cite{10.1145/3613904.3642159} & None & D-S  & Director   & Agent~\includegraphics[width=0.025\textwidth]{icon/action.pdf} & Interface   \\
2023 & Wang\etal~\cite{wang2024userbehaviorsimulationlarge} & S-V & Post-S   & Observer   & Agent~\includegraphics[width=0.025\textwidth]{icon/action.pdf} & Data   \\
2023 & Humanoid Agents~\cite{wang2023humanoidagentsplatformsimulating} & S-P & D-S  & Observer   & Agent~\includegraphics[width=0.025\textwidth]{icon/action.pdf} & Interface   \\
- & - & - & Post-S   & Observer   & Agent~\includegraphics[width=0.025\textwidth]{icon/action.pdf} & Data   \\
2023 & Zhang\etal~\cite{10.1145/3626772.3657844} & S-V & Pre-S   & Prototype   & Agent~\includegraphics[width=0.025\textwidth]{icon/identity.pdf} & Data   \\
2024 & SOTOPIA~\cite{zhou2024sotopiainteractiveevaluationsocial} & None & D-S  & Actor   & Agent~\includegraphics[width=0.025\textwidth]{icon/action.pdf} & Interface   \\
-& PedSim~\cite{Ped} & S-P & D-S  & Observer   & Agent~\includegraphics[width=0.025\textwidth]{icon/action.pdf} & Interface   \\
-& AnyLogic~\cite{doi:https://doi.org/10.1002/9781118762745.ch12} & S-P & D-S  & Observer   & Agent~\includegraphics[width=0.025\textwidth]{icon/action.pdf} & Interface   \\
-& AutoGPT~\cite{Significant_Gravitas_AutoGPT} & None & D-S  & Observer   & Agent~\includegraphics[width=0.025\textwidth]{icon/action.pdf} & Interface   \\
-& BabyAGI~\cite{babyagi} & None & D-S  & Observer   & Agent~\includegraphics[width=0.025\textwidth]{icon/action.pdf} & Data   \\
2021 & Siu\etal~\cite{NEURIPS2021_86e8f7ab} & R-V & D-S  & Actor   & Agent~\includegraphics[width=0.025\textwidth]{icon/action.pdf} & Interface   \\
2023 & Eloy\etal~\cite{10.1145/3579598} & S-P & D-S  & Actor   & Agent~\includegraphics[width=0.025\textwidth]{icon/action.pdf} & Language; Interface   \\
2023 & Zubatiy\etal~\cite{10.1145/3610170} & None & D-S  & Director   & Agent~\includegraphics[width=0.025\textwidth]{icon/action.pdf} & Language; Interface   \\
2024 & Jaber\etal~\cite{10.1145/3613904.3642183} & R-P & D-S  & Director   & Agent~\includegraphics[width=0.025\textwidth]{icon/action.pdf} & Physical \\
2024 & Dai\etal~\cite{10.1145/3613905.3637145} & S-P & D-S  & Actor   & Agent~\includegraphics[width=0.025\textwidth]{icon/action.pdf} & Physical; Language\\
2024 & Wan\etal~\cite{10.1145/3613905.3651026} & S-P & D-S  & Director   & Agent~\includegraphics[width=0.025\textwidth]{icon/action.pdf} & Language; Interface   \\
- & - & - & Post-S   & Observer   & Agent~\includegraphics[width=0.025\textwidth]{icon/action.pdf} & Data   \\
2024 & PeerGPT~\cite{10.1145/3613905.3651008} & R-P & D-S  & Actor   & Agent~\includegraphics[width=0.025\textwidth]{icon/action.pdf} & Language\\
2024 & ClassMeta~\cite{10.1145/3613904.3642947} & S-P & D-S  & Actor   & Agent~\includegraphics[width=0.025\textwidth]{icon/action.pdf} & Physical; Language\\
\specialrule{0em}{1pt}{1pt}
2024 & Attig\etal~\cite{10.1145/3613905.3650853} & R-V & D-S  & Actor   & Agent~\includegraphics[width=0.025\textwidth]{icon/action.pdf} & Interface   \\
2024 & Hwang\etal~\cite{10.1145/3613904.3642202} & R-P & D-S  & Director   & Agent~\includegraphics[width=0.025\textwidth]{icon/action.pdf} & Language\\
2024&DrHouse~\cite{10.1145/3699765}&R-P&Post-S&Observer&Agent~\includegraphics[width=0.025\textwidth]{icon/action.pdf}&Data\\
2024&Cuadra\etal~\cite{10.1145/3659624}&R-P&D-S&Director&Agent~\includegraphics[width=0.025\textwidth]{icon/goal.pdf}&Language; Interface\\
2024&Sasha~\cite{10.1145/3643505}&R-P&D-S&Director&Agent~\includegraphics[width=0.025\textwidth]{icon/goal.pdf}&Language\\
\end{longtable}
%%
%% If your work has an appendix, this is the place to put it.


\end{document}
\endinput
%%
%% End of file `sample-manuscript.tex'.
