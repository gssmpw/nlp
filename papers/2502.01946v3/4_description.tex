%TABLE
% % % Please add the following required packages to your document preamble:
% % % \usepackage{multirow}
% % % \usepackage{graphicx}
% % % \usepackage[table,xcdraw]{xcolor}
% % % Beamer presentation requires \usepackage{colortbl} instead of \usepackage[table,xcdraw]{xcolor}
% % \begin{table*}[]
% % \caption{OVERVIEW OF SEQUENCES}
% % \label{tab:sequence}
% % \resizebox{\textwidth}{!}{
% % \begin{tabular}{c|c|cccccc} \toprule
% %   Sequence &
% %   Time & 
% %   Weather &
% %   Length &
% %   Loop &
% %   Target &
% %   Description
% %   \\ \midrule
% % \multirow{2}{*}{Mountain} &
% %   \multicolumn{1}{l|}{Day} &
% %   Clear &
% %   4km &  
% %   2 times &
% %   odometry, &
% % \multirow{2}{*}{
% %   poor road surface condition, large altitude variation}
% %   \\
% % \multicolumn{1}{l|}{} &
% %   \multicolumn{1}{l|}{Night} &
% %   Cloud &
% %   4km & 
% %   2 times&
% %   online place recognition \\
% %   \midrule
% % \multirow{2}{*}{Library} &
% %   \multicolumn{1}{l|}{Day} &
% %   Clear &
% %   1.6km &  
% %   2 times &
% %   odometry,&
% % \multirow{2}{*}{
% %   {\begin{tabular}[c]{@{}c@{}} a long narrow one-way alley, downhill and uphill at sharp curves\end{tabular}}
% %   } 
% %   \\
% % \multicolumn{1}{l|}{} &
% %   \multicolumn{1}{l|}{Night} &
% %   Cloud &
% %   1.6km & 
% %   2 times &
% %   online place recognition\\ \midrule
% % \multirow{2}{*}{Sport complex} &
% %   \multicolumn{1}{l|}{Day} &
% %   Clear &
% %   1.4km &  
% %   2 times &
% %   odometry,&
% % \multirow{2}{*}{
% %  consists of a flat section, a gentle uphill section, and a sharp downhill section}  
% %   \\
% % \multicolumn{1}{l|}{} &
% %   \multicolumn{1}{l|}{Night} &
% %   Cloud &
% %   0.7km &  
% %   1 time &
% %   online place recognition\\ \midrule
% % \multirow{3}{*}{Parking lots} &
% %   \multicolumn{1}{l|}{Day} &
% %   Clear &
% %   0.4km &  
% % \multirow{3}{*}{
% %   inter-session} &
% % \multirow{2}{*}{
% %   \\odometry,} &
% % \multirow{3}{*}{
% %   a short driving distance
% % frequent left turn} 
% %   \\
% % \multicolumn{1}{l|}{} &
% %   \multicolumn{1}{l|}{Day} &
% %   Clear &
% %   0.4km & 
% %   &
% %   \multirow{2}{*}{
% %   \\global localization} \\
% % \multicolumn{1}{l|}{} &
% %   \multicolumn{1}{l|}{Night} &
% %   Cloud &
% %   0.5km &  
% %   \\ \midrule
% % \multirow{3}{*}{River island} &
% %   \multicolumn{1}{l|}{Day} &
% %   Cloud &
% %   5.8km &  
% % \multirow{3}{*}{
% %   inter-session} &
% %   \multirow{2}{*}{
% %   \\odometry,} &
% % \multirow{3}{*}{
% %   poor road surface condition, large altitude variation} \\
% % \multicolumn{1}{l|}{} &  
% %   \multicolumn{1}{l|}{Dusk} &
% %   Cloud &
% %   8km &  
% %   &
% %     \multirow{2}{*}{
% %   \\global localization}
% %   \\
% % \multicolumn{1}{l|}{} &  
% %   \multicolumn{1}{l|}{Day} &
% %   Clear &
% %   4km &  
% %  \\ \midrule
% % \multirow{2}{*}{Bridge} &
% %   \multicolumn{1}{l|}{Day} &
% %   Rain &
% %   4.9km &  
% %   1 times &
% %   odometry, &
% % \multirow{2}{*}{
% %   a wide variety of dynamic objects}
% %   \\
% % \multicolumn{1}{l|}{} &
% %   \multicolumn{1}{l|}{Night} &
% %   Cloud &
% %   4.9km &  
% %   1 times &
% %   online place recognition
% %   \\ \midrule

% % Street &
% %   \multicolumn{1}{l|}{Day} &
% %   Rain &
% %   1km &  
% %   1 times &
% %   odometry &
% %   heavy traffic Low driving speed and frequent stops
% %   \\ \midrule
% % \multirow{2}{*}{Stream} &
% %   \multicolumn{1}{l|}{Day} &
% %   Clear &
% %   4.2km &  
% %   2 times  &
% %   odometry,&
% % \multirow{2}{*}{
% %   a wide variety of dynamic objects}
% %   \\
% % \multicolumn{-10}{c|}{} &
% %   \multicolumn{1}{l|}{Night} &
% %   Cloud &
% %   5.5km & 
% %   2 times &
% %   online place recognition 
% %  \\ \bottomrule
% % \end{tabular}
% % }
% % \end{table*}



% % % Please add the following required packages to your document preamble:
% % % \usepackage{multirow}
% % % \usepackage{graphicx}
% % % \usepackage[table,xcdraw]{xcolor}
% % % Beamer presentation requires \usepackage{colortbl} instead of \usepackage[table,xcdraw]{xcolor}
% % \begin{table*}[]
% % \caption{sequence}
% % \label{tab:my-table}
% % \resizebox{\textwidth}{!}{%
% % \begin{tabular}{c|cccccc} \toprule
% %    &
% %    &
% %   \multicolumn{2}{c}{Radar}\\ \cline{3-4} \rule{0pt}{2.5ex}
% %   \multirow{-2}{*}{Sequence} & 
% %   \multirow{-2}{*}{Length } &
% %   \multicolumn{1}{l}{4D Radar} &
% %   \multicolumn{1}{l}{Scanning Radar} &
% %   \multirow{-2}{*}{Time} &
% %   \multirow{-2}{*}{Condition} &
% %   \multirow{-2}{*}{Scenarios} \\ \midrule
% % \multicolumn{1}{l|}{Mountain} &
% %   - &
% %   - &
% %   d &  
% %   \\
% % \multicolumn{1}{l|}{Library} &

% %   - \\
% % \multicolumn{1}{l|}{Sports complex} &
% %   {\color[HTML]{000000} 0.499} &
% %   {\color[HTML]{000000} 0.785} &


% %   - \\ 
% % \multicolumn{1}{l|}{Parking lots} &
% %   {\color[HTML]{000000} 0.499} &
% %   {\color[HTML]{000000} 0.785} &
% %   - \\ \midrule
% % \multicolumn{1}{l|}{River island} &
% %   - &
% %   - \\
% % \multicolumn{1}{l|}{Bridge} &
% %   - \\

% % \multicolumn{1}{l|}{Street} &
% %   - &
% %   - \\
% % \multicolumn{1}{l|}{Stream} &
% %   {\color[HTML]{FF0000} \textbf{0.806}} &
% %   {\color[HTML]{FF0000} \textbf{0.885}} &
% %   {\color[HTML]{FF0000} \textbf{0.903}} \\ \bottomrule
% % \end{tabular}%
% % }
% % \end{table*}

% % Please add the following required packages to your document preamble:
% % \usepackage{multirow}
% % \usepackage{graphicx}
% % \usepackage[table,xcdraw]{xcolor}
% % Beamer presentation requires \usepackage{colortbl} instead of \usepackage[table,xcdraw]{xcolor}

% \begin{table*}[]
% \caption{OVERVIEW OF SEQUENCES}
% \label{tab:sequence}
% \resizebox{\textwidth}{!}{
% \begin{tabular}{c|c|cccccc} 
% \toprule
% Sequence & Time & Weather & Length & Loop & Target & Description \\ 
% \midrule

% \multirow{2}{*}{Mountain} & Day & Clear & 4km & 2 times & odometry, & \multirow{2}{*}{poor road surface condition, large altitude variation} \\
%  & Night & Cloud & 4km & 2 times & online place recognition & \\ 
% \midrule

% \multirow{2}{*}{Library} & Day & Clear & 1.6km & 2 times & odometry, &  \multirow{2}{*}{long narrow one-way alley, downhill and uphill at sharp curves}\\
%  & Night & Cloud & 1.6km & 2 times & online place recognition & \\ 
% \midrule

% \multirow{2}{*}{Sport complex} & Day & Clear & 1.4km & 2 times & odometry, & \multirow{2}{*}{flat section, gentle uphill section, sharp downhill section} \\
%  & Night & Cloud & 0.7km & 1 time & online place recognition & \\ 
% \midrule

% \multirow{3}{*}{Parking lots} & Day & Clear & 0.4km & &  \multirow{3}{*}{\begin{tabular}[c]{@{}c@{}}odometry,\\global localization \end{tabular}} &  \\
%  & Day & Clear & 0.4km & inter-session &  & short driving distance, frequent left turn\\
%  & Night & Cloud & 0.5km &  &  & \\ 
% \midrule

% \multirow{3}{*}{River island} & Day & Cloud & 5.8km &  & \multirow{3}{*}{\begin{tabular}[c]{@{}c@{}}odometry,\\global localization \end{tabular}} &  \\
%  & Dusk & Cloud & 8km & inter-session &  & various driving routes, flat area, intersections, two-lane one-way roads\\
%  & Day & Clear & 4km &  & & \\ 
% \midrule

% \multirow{2}{*}{Bridge} & Day & Rain & 4.9km & 1 time & odometry, & \multirow{2}{*}{four-lane overpass and the bridge, various dynamic objects} \\
%  & Night & Cloud & 4.9km & 1 time & online place recognition & \\ 
% \midrule

% Street & Day & Rain & 1km & 1 time & odometry & heavy traffic, low driving speed, and frequent stops \\ 
% \midrule

% \multirow{2}{*}{Stream} & Day & Clear & 4.2km & 2 times & odometry, & \multirow{2}{*}{S-shaped one-way roads, various dynamic objects, U-turns}\\
%  & Night & Cloud & 5.5km & 2 times & online place recognition & \\ 
% \bottomrule
% \end{tabular}
% }
% \end{table*}



\begin{table}[!t]
\caption{OVERVIEW OF SEQUENCES}
\label{tab:sequence}
\centering
\resizebox{\columnwidth}{!}{
\begin{tabular}{c|c|cccccc} 
\toprule
Sequence & Index & Time & Weather & Length & Loop & Target \\ 
\midrule

\multirow{3}{*}{\texttt{Mountain}} & 01 & Day & Clear & \unit{4}{km} & 2 times & \multirow{3}{*}{\begin{tabular}[c]{@{}c@{}}odometry,\\online place recognition,\\global localization \end{tabular}}   \\
 & 02 & Night & Cloud & \unit{4}{km} & 2 times &  \\ 
 & 03 & Day & Snow & \unit{3}{km} & 1.5 times &  \\ 
\midrule

\multirow{3}{*}{\texttt{Library}} & 01 & Day & Clear & \unit{1.6}{km} & 2 times & \multirow{3}{*}{\begin{tabular}[c]{@{}c@{}}odometry,\\online place recognition,\\global localization \end{tabular}}  \\
 & 02 & Night & Cloud & \unit{1.6}{km} & 2 times &  \\ 
  & 03 & Day & Snow & \unit{0.8}{km} & 1 time &  \\ 
\midrule

\multirow{3}{*}{\begin{tabular}[c]{@{}c@{}}\texttt{Sports}\\\texttt{Complex}\end{tabular}}{} & 01 & Day & Clear & \unit{1.4}{km} & 2 times & \multirow{3}{*}{\begin{tabular}[c]{@{}c@{}}odometry,\\online place recognition,\\global localization \end{tabular}} \\
 & 02 & Night & Cloud & \unit{0.7}{km} & 1 time &  \\ 
  & 03 & Day & Snow & \unit{1.4}{km} & 2 times &  \\ 
\midrule

\multirow{4}{*}{\begin{tabular}[c]{@{}c@{}}\texttt{Parking}\\\texttt{Lot}\end{tabular}}{} & 01 &  Day & Clear & \unit{0.4}{km} & \multirow{4}{*}{inter-session} & \multirow{4}{*}{\begin{tabular}[c]{@{}c@{}}odometry,\\global localization \end{tabular}}  \\
 & 02 & Day & Clear & \unit{0.4}{km} &  &  \\
 & 03 & Night & Cloud & \unit{0.5}{km} &  &  \\ 
 & 04 & Day & Snow & \unit{0.4}{km} &  &  \\ 
\midrule

\multirow{3}{*}{\begin{tabular}[c]{@{}c@{}}\texttt{River}\\\texttt{Island}\end{tabular}}{} 
& 01 & Day & Clear & \unit{4}{km} &  & \multirow{3}{*}{\begin{tabular}[c]{@{}c@{}}odometry,\\global localization \end{tabular}}  \\
 & 02 & Dusk & Cloud & \unit{8}{km} & inter-session & \\
 & 03 & Day & Cloud & \unit{5.8}{km} &  &  \\ 
\midrule

\multirow{2}{*}{\texttt{Bridge}} & 01 & Day & Rain & \unit{4.9}{km} & 1 time & odometry, \\
& 02 & Night & Cloud & \unit{4.9}{km} & 1 time & online place recognition \\ 
\midrule

\texttt{Street} & 01 & Day & Rain & \unit{1}{km} & 1 time & odometry \\ 
\midrule

\multirow{2}{*}{\texttt{Stream}} & 01 & Day & Clear & \unit{4.2}{km} & 2 times & odometry, \\

 & 02 & Night & Cloud & \unit{5.5}{km} & 2 times & online place recognition \\ 
\bottomrule
\end{tabular}
}
\vspace{-5mm}
\end{table}

%TABLE

\section{Description of HeRCULES Dataset}
\label{sec:experiment}

\subsection{Target Environments}
% This subsection briefly outlines the reasons for selecting the eight target environments depicted in \figref{fig:road} and their key characteristics. An overview of the eight sequences is presented in \tabref{tab:sequence}.
This subsection briefly outlines the reasons for selecting the eight target environments depicted in \figref{fig:road}. An overview of the eight sequences is presented in \tabref{tab:sequence}.

\subsubsection{Mountain}
\texttt{Mountain} captures sequences on Gwanak Mountain, the highest elevation difference among all sequences. The route includes speed bumps and rough roads, causing significant rolling and pitching.


\subsubsection{Library}
% \texttt{Library} captures sequences from a long, narrow, one-way path near the library on campus. The route involves two loops for loop closure with sequences recorded in clear afternoon and cloudy night. The path is narrow and includes steep curves with uphill and downhill sections.
\texttt{Library} captures sequences from a long, narrow, one-way path near the library on campus. The path includes steep curves with uphill and downhill sections.
\subsubsection{Sports Complex}
\texttt{Sports Complex} captures sequences around a sports complex, including parking areas and roads with flat, gently sloped, and steep sections. Two loops with an average speed below 30 km/h were recorded during the day and night.
\subsubsection{Parking Lot}
\texttt{Parking Lot} captures sequences from a parking lot with many left turns, recorded on a clear afternoon and at night. While the ground appears flat, slight elevation variations are noted. This sequence has the shortest distance among all.
\subsubsection{River Island}
% \texttt{River Island} captures three sequences for multi-session place recognition, recorded during the day and dusk, with each route uniquely designed. The flat area features various driving paths, including intersections and two-lane one-way roads.
\texttt{River Island} captures three sequences for multi-session place recognition with each route uniquely designed. The flat area features various driving paths, including intersections and two-lane one-way roads.
\subsubsection{Bridge}
\texttt{Bridge} captures sequences for place recognition research, driven back and forth along a four-lane overpass and the Wonhyo Bridge over the Han River. It includes sequences recorded on a rainy afternoon and cloudy dusk, with an average speed of \unit{60}{km/h}, and sections featuring traffic congestion in urban environments.
\subsubsection{Street}
\texttt{Street} captures a sequence of driving in heavy congestion and rain near IFC Seoul during rush hour. Due to the crowds and numerous vehicles, there are many dynamic objects, leading to frequent stops.
\subsubsection{Stream}
% \texttt{Stream} captures an S-shaped stream route that is driven during the day and night. One-way roads run along both sides of the stream, with U-turns via bridges. For place recognition research, intentional revisits were designed, resulting in similar environments.
\texttt{Stream} captures an S-shaped stream route with one-way roads running along both sides, allowing U-turns via bridges. For place recognition research, intentional revisits were designed, resulting in similar environments.

% \begin{figure}[!t]
%     \centering
%     \includegraphics[trim= 2cm 2.5cm 2cm 2.5cm, clip,width=1\linewidth]{example-image-a}
%     \caption{map}
%     \label{fig:map}
%     \vspace{-3mm}
% \end{figure}


\begin{figure}[!t] % "t"는 페이지 상단에 이미지를 배치하려는 옵션입니다.
    \centering
    \includegraphics[trim=0.5cm 4.55cm 0.5cm 1.2cm, clip, width=\linewidth]{figure/r1.pdf} % 이미지를 1단 전체 너비로 표시
    \vspace{-7mm}
\end{figure}


\begin{figure}[!t] % "t"는 페이지 상단에 이미지를 배치하려는 옵션입니다.
    \centering
    \includegraphics[trim=0.5cm 5.2cm 0.5cm 1.2cm, clip, width=\linewidth]{figure/r2.pdf} % 이미지를 1단 전체 너비로 표시
    \caption{Trajectory overlaid on satellite maps for each sequence with colors. Red indicates the start, while blue designates the end.}
    \label{fig:road}
    \vspace{-6mm}
\end{figure}

 % Each sequence features distinct environmental characteristics, and multiple revisits to the same locations ensure sufficient queries for place recognition evaluation.

\subsection{Data Description and Format}

The file structure of the HeRCULES dataset is delineated in \figref{fig:file}. The acquisition time of
all measurements are stored in \texttt{datastamp.csv}. The FMCW LiDAR and 4D radar data are provided in \texttt{time.bin}, while the camera data is in \texttt{time.png}. For spinning radar data, we support software that converts raw polar images into Oxford-style \cite{barnes2020oxford} and Cartesian images. \ac{IMU}, \ac{GPS}, and \ac{INS} data are provided in \texttt{.csv}, and calibration information between sensors is available in \texttt{.yaml} and \texttt{.txt} format. The data types for each sensor are detailed in \tabref{tab:sensors}.

\subsection{Individual Ground Truth}
Before logging each sequence, we ensure that the GNSS solution is fixed and the INS solution has converged. We use \ac{PTP} to synchronize timestamps in \ac{UTC} across all sensors. However, spatiotemporal discrepancies arise due to differences in sensor mounting positions and data acquisition times.
To address this, we provide ground truth poses for each sensor to support place recognition research, handling spatial differences using extrinsic calibration results and temporal differences with B-Spline interpolation \cite{mueggler2018continuous}.
The ground truth is shown in \figref{fig:gtpose}, highlighting the importance of independently deriving ground truth pose for each sensor.

% We provide ground truth pose for each sensor to support place recognition research, addressing spatial differences using extrinsic calibration results and temporal differences with B-Spline interpolation \cite{mueggler2018continuous}.
% The ground truth is shown in \figref{fig:gtpose}, highlighting the importance of independently deriving ground truth pose for each sensor due to spatiotemporal variations rather than relying on a unified ground truth.



% \subsection{Datasets and Evaluation Metric}
% red section, we use the 0.5\unit{m} and 5{\textdegree} for handcart sequences, and 2.0\unit{m} and 10{\textdegree} for car sequences.

%  (RPE$_r$) and translational (RPE$_t$) parts each.


% \subsection{NTU4DRadLM dataset}
% \label{subsec:ntu}

% %TABLE
% % % Please add the following required packages to your document preamble:
% % \usepackage{multirow}
% % \usepackage{graphicx}
% % \usepackage[table,xcdraw]{xcolor}
% \begin{table*}[]
% \centering
% \caption{SENSOR SPECIFICATIONS}
% \label{tab:sensors}
% \resizebox{\textwidth}{!}{%
% \begin{tabular}{c|ccccccccc} 
% \toprule
%  &
%  &
%  &
%   \multicolumn{3}{c}{Resolution}
%  &
%  &
%   \multicolumn{3}{c}{FOV}
%   \\ \cline{4-6} \cline{8-10} \rule{0pt}{2.5ex}
%   \multirow{-2}{*}{Sensor} &
%   \multirow{-2}{*}{Type} & 
%   \multirow{-2}{*}{Data type} &
%   Range &
%   Azimuth &
%   Elevation &
%   &
%   Range &
%   Azimuth &
%   Elevation
% \\ 
% \midrule

%   {\begin{tabular}[c]{@{}c@{}}4D\\Radar \end{tabular}} &
% Continental ARS548 & 
%   {\begin{tabular}[c]{@{}c@{}}x, y, z, velocity, RCS,\\range, azimuth, elavation \end{tabular}} &
% 0.22 m&
%   {\begin{tabular}[c]{@{}c@{}}1.2°@0. . . ±15°\\1.68°@ ±45°\end{tabular}}&
% 2.3°&
% &
% 300 m&
% ±60°&
%   {\begin{tabular}[c]{@{}c@{}}±4°@300m\\±14°@$<$100m\end{tabular}}
% \\ \rule{0pt}{4ex}
%   {\begin{tabular}[c]{@{}c@{}}Spinning\\Radar \end{tabular}} &
% Navtech RAS6 & 
% Polar image, Cartesian image&
% 0.044 m&
% 0.9°&
% - &
%  &
% 330 m &
% 360°&
% -
% \\ \rule{0pt}{4ex}
%   {\begin{tabular}[c]{@{}c@{}}FMCW\\LiDAR \end{tabular}} &
% Aeva Aeries II &
%   {\begin{tabular}[c]{@{}c@{}}x, y, z, reflectivity, intensity,\\velocity, line-index, time-offset \end{tabular}} &

% 0.075m@1$\sigma$ &
% 0.025°&
% 0.025°&
% &
% 150 m&
% 19.2°&
% 30° 
% \\ \rule{0pt}{4ex}
%   Camera &

%   {\begin{tabular}[c]{@{}c@{}}FLIR Blackfly S \\BFS-U3-16S2C-CS USB3\end{tabular}} &
% 8-bit Bayer pattern png format&
% - &
% 1440px &
% 1080px &
% &
% - &
% 60°&
% 45°
% \\ \rule{0pt}{4ex}
% IMU & Xsens MTi-300 &
% % High performing Attitude and Heading Reference System. 9-axis 
%   {\begin{tabular}[c]{@{}c@{}}qx, qy, qz, qw, eul$_x$, eul$_y$, eul$_z$, gyr$_x$, gyr$_y$, \\gyr$_z$, acc$_x$, acc$_y$, acc$_z$, mag$_x$, mag$_y$, mag$_z$\end{tabular}} &
% - &
% - &
% - &
% &
% - &
% - &
% -
% \\ \rule{0pt}{4ex}
% RTK-GPS & 
%   {\begin{tabular}[c]{@{}c@{}}Hexagon NovAtel\\ SPAN-CPT7  \end{tabular}} &
%   {\begin{tabular}[c]{@{}c@{}}latitude, longitude, height, velocity$_{north}$,\\velocity$_{east}$, velocity$_{up}$, roll, pitch, azimuth, status \end{tabular}} &
% - &
% - &
% - &
% &
% - &
% - &
% -
% \\ 
% \bottomrule
% \end{tabular}%
% }
% \vspace{-3mm}
% \end{table*}

% Please add the following required packages to your document preamble:
% \usepackage{multirow}
% \usepackage{graphicx}
% \usepackage[table,xcdraw]{xcolor}
% \begin{table*}[]
% \centering
% \caption{SENSOR SPECIFICATIONS}
% \label{tab:sensors}
% \resizebox{\textwidth}{!}{%
% \begin{tabular}{c|cccccccccc} 
% \toprule
%  &
%  &
%  &
%   \multicolumn{3}{c}{Resolution}
%  &
%  &
%   \multicolumn{3}{c}{FOV} &
%   \\ \cline{4-6} \cline{8-10} \rule{0pt}{2.3ex}
%   \multirow{-2}{*}{Sensor} &
%   \multirow{-2}{*}{Type} & 
%   \multirow{-2}{*}{Data type} &
%   Range &
%   Azimuth &
%   Elevation &
%   &
%   Range &
%   Azimuth &
%   Elevation &
%   \multirow{-2}{*}{Frequency}
% \\ 
% \midrule

%   {\begin{tabular}[c]{@{}c@{}}4D\\Radar \end{tabular}} &
% Continental ARS548 & 
%   {\begin{tabular}[c]{@{}c@{}}x, y, z, velocity, RCS,\\range, azimuth, elavation \end{tabular}} &
% \unit{0.22}{m}&
%   {\begin{tabular}[c]{@{}c@{}}1.2°@0. . . ±15°\\1.68°@ ±45°\end{tabular}}&
% 2.3°&
% &
% \unit{300}{m}&
% ±60°&
%   {\begin{tabular}[c]{@{}c@{}}±4°@\unit{300}{m}\\±14°@$<$\unit{100}{m}\end{tabular}} &
% 20 Hz
% \\ \rule{0pt}{4ex}
%   {\begin{tabular}[c]{@{}c@{}}Spinning\\Radar \end{tabular}} &
% Navtech RAS6 & 
% Polar image, Cartesian image&
% \unit{0.044}{m}&
% 0.9°&
% - &
%  &
% \unit{330}{m} &
% 360°&
% - &
% 4 Hz
% \\ \rule{0pt}{4ex}
%   {\begin{tabular}[c]{@{}c@{}}FMCW\\LiDAR \end{tabular}} &
% Aeva Aeries II &
%   {\begin{tabular}[c]{@{}c@{}}x, y, z, reflectivity, intensity,\\velocity, line-index, time-offset \end{tabular}} &

% \unit{0.02}{m}@1$\sigma$ &
% 0.025°&
% 0.025°&
% &
% \unit{150}{m}&
% 120°&
% 30° &
% 10 Hz
% \\ \rule{0pt}{4ex}
%   Camera &

%   {\begin{tabular}[c]{@{}c@{}}FLIR Blackfly S \\BFS-U3-16S2C-CS USB3\end{tabular}} &
% 8-bit Bayer pattern png format&
% - &
% \unit{1440}{px} &
% \unit{1080}{px} &
% &
% - &
% 60°&
% 45° &
% 15 Hz
% \\ \rule{0pt}{4ex}
% IMU & Xsens MTi-300 &
% % High performing Attitude and Heading Reference System. 9-axis 
%   {\begin{tabular}[c]{@{}c@{}}q$_x$, q$_y$, q$_z$, q$_w$, eul$_x$, eul$_y$, eul$_z$, gyr$_x$, gyr$_y$, \\gyr$_z$, acc$_x$, acc$_y$, acc$_z$, mag$_x$, mag$_y$, mag$_z$\end{tabular}} &
% - &
% - &
% - &
% &
% - &
% - &
% - &
% 100 Hz
% \\ \rule{0pt}{4ex}
% RTK-GPS & 
%   {\begin{tabular}[c]{@{}c@{}}Hexagon NovAtel\\ SPAN-CPT7  \end{tabular}} &
%   {\begin{tabular}[c]{@{}c@{}}latitude, longitude, height, velocity$_{north}$,\\velocity$_{east}$, velocity$_{up}$, roll, pitch, azimuth, status \end{tabular}} &
% - &
% - &
% - &
% &
% - &
% - &
% - &
% 50 Hz
% \\ 
% \bottomrule
% \end{tabular}%
% }
% \vspace{-3mm}
% \end{table*}

\begin{table*}[]
\centering
\caption{SENSOR SPECIFICATIONS}
\label{tab:sensors}
\resizebox{\textwidth}{!}{%
\begin{tabular}{c|cccccccccc} 
\toprule
  \multirow{2}{*}{Sensor} &
  \multirow{2}{*}{Type} & 
  \multirow{2}{*}{Data type} &
  \multicolumn{3}{c}{Resolution}
  &
  &
  \multicolumn{3}{c}{FOV} &
  \multirow{2}{*}{Frequency}
  \\ \cline{4-6} \cline{8-10} \rule{0pt}{2.3ex}
  &
  &
  &
  Range &
  Azimuth &
  Elevation &
  &
  Range &
  Azimuth &
  Elevation &

\\ 
\midrule

  {\begin{tabular}[c]{@{}c@{}}4D\\Radar \end{tabular}} &
Continental ARS548 & 
  {\begin{tabular}[c]{@{}c@{}}x, y, z, velocity, RCS,\\range, azimuth, elavation \end{tabular}} &
\unit{0.22}{m}&
  {\begin{tabular}[c]{@{}c@{}}1.2°@0. . . ±15°\\1.68°@ ±45°\end{tabular}}&
2.3°&
&
\unit{300}{m}&
±60°&
  {\begin{tabular}[c]{@{}c@{}}±4°@\unit{300}{m}\\±14°@$<$\unit{100}{m}\end{tabular}} &
20 Hz
\\ \rule{0pt}{4ex}
  {\begin{tabular}[c]{@{}c@{}}Spinning\\Radar \end{tabular}} &
Navtech RAS6 & 
Polar image, Cartesian image&
\unit{0.044}{m}&
0.9°&
- &
 &
\unit{330}{m} &
360°&
- &
4 Hz
\\ \rule{0pt}{4ex}
  {\begin{tabular}[c]{@{}c@{}}FMCW\\LiDAR \end{tabular}} &
Aeva Aeries II &
  {\begin{tabular}[c]{@{}c@{}}x, y, z, reflectivity, intensity,\\velocity, line-index, time-offset \end{tabular}} &

\unit{0.02}{m}@1$\sigma$ &
0.025°&
0.025°&
&
\unit{150}{m}&
120°&
30° &
10 Hz
\\ \rule{0pt}{4ex}
  Camera &

  {\begin{tabular}[c]{@{}c@{}}FLIR Blackfly S \\BFS-U3-16S2C-CS USB3\end{tabular}} &
8-bit Bayer pattern png format&
- &
\unit{1440}{px} &
\unit{1080}{px} &
&
- &
60°&
45° &
15 Hz
\\ \rule{0pt}{4ex}
IMU & Xsens MTi-300 &
% High performing Attitude and Heading Reference System. 9-axis 
  {\begin{tabular}[c]{@{}c@{}}q$_x$, q$_y$, q$_z$, q$_w$, eul$_x$, eul$_y$, eul$_z$, gyr$_x$, gyr$_y$, \\gyr$_z$, acc$_x$, acc$_y$, acc$_z$, mag$_x$, mag$_y$, mag$_z$\end{tabular}} &
- &
- &
- &
&
- &
- &
- &
100 Hz
\\ \rule{0pt}{4ex}
RTK-GPS & 
  {\begin{tabular}[c]{@{}c@{}}Hexagon NovAtel\\ SPAN-CPT7  \end{tabular}} &
  {\begin{tabular}[c]{@{}c@{}}latitude, longitude, height, velocity$_{north}$,\\velocity$_{east}$, velocity$_{up}$, roll, pitch, azimuth, status \end{tabular}} &
- &
- &
- &
&
- &
- &
- &
50 Hz
\\ 
\bottomrule
\end{tabular}%
}
\vspace{-3mm}
\end{table*}

% %TABLE


% \tabref{table:ntu}. Our proposed maforementioned issues (\figref{fig:NTU}).

% In \texttt{loop3} sequen

% \subsection{MSC-RAD4R dataset}
% \label{subsec:msc}

% In the MSC-RAD4R datase in \tabref{table:msc}. 

% \texttt{URBAN}  
% \texttt{LOOP} sequen
% elocity measurements (\figref{fig:MSC_trajcetory}).

% To thoriments on snowy sequences \texttt{RURAL\_A2} and \texttt{RURAL\_B2} which are characterized by sharp roundabouts and high dRemarkably, 4DRadarSLAM achieved the second-best performance in most sequences even without leveraging \ac{IMU}.



% \subsection{Ablation Studies}


% \subsubsection{Low Elevation Error}

% + \cite{lee2022patchworkpp}. As depicted in \figref{fig:ablaan

% %TABLE
% \section{Discussion}
\subsection{Limitation of rotation prediction task}
\begin{figure}[t!]
    \centering
    \begin{subfigure}{0.1\textwidth}
        \includegraphics[width=\linewidth]{figure/ship_sample.png}
        \caption{"Ship"}
        \label{fig:ship_sample}
    \end{subfigure}
    \vfill
    \begin{subfigure}{0.22\textwidth}
        \includegraphics[width=\linewidth]{figure/all_classes_main_likelihood_ship.pdf}
        \caption{Likelihood of all 10 classes over steps}
        \label{fig:all_classes_main_likelihood_hard_ship}
    \end{subfigure}
    \begin{subfigure}{0.22\textwidth}
        \includegraphics[width=\linewidth]{figure/all_classes_aux_likelihood_ship.pdf}
        \caption{Likelihood of all 4 rotation classes over steps}
        \label{fig:all_class_likelihood_aux_hard_ship}
    \end{subfigure}
    \caption{Likelihood across iteration of a level 5 Gaussian Noise sample in class "Ship"}
    \label{fig:likelihood_ship}
\end{figure}
Figure \ref{fig:likelihood_ship} illustrates the likelihood of different classes across iterations, using a sample (Figure \ref{fig:ship_sample}) from the "Ship" class with a rotation angle of 0 degrees. The sample's features, which include many edge features, allow the model to predict rotation effectively and maintain stability over iterations (Figure \ref{fig:all_class_likelihood_aux_hard_ship}). This stability and accuracy in rotation prediction enable the recurrent model to better estimate the optimal iteration for the main task.

However, predicting the rotation angle becomes more challenging for samples with fewer edge features or isotropic characteristics, such as the sample from the "Cat" class in Figure \ref{fig:cat_sample}. Figure \ref{fig:all_classes_main_likelihood_hard_cat}, \ref{fig:all_class_likelihood_aux_hard_cat} shows that while the likelihood of the "Cat" class continues to increase over iterations and remains significantly higher than other classes, the likelihood of the ground truth rotation (0 degrees) in the self-supervised task is very low . Instead, the model tends to predict the class corresponding to a 270-degree rotation. This behavior negatively impacts the estimation of the optimal iteration for the main task based on the self-supervised task.

We acknowledge this as a limitation of using the rotation prediction task as a self-supervised task.

\subsection{Converge to a fixed point does not ensure to mitigate "overthinking"}

\cite{bansal2022endtoend} highlights the relationship between changes in feature maps across iterations and the issue of "overthinking".
Specifically, they showed that for recurrent models where the norm $\|h_t - h_{t-1}\|$ converges to $0$, it can be assumed that the feature map has converged to a fixed point, and thus the prediction results of iterations beyond this fixed point remain unchanged, effectively addressing "overthinking." 

However, we observe that this assumption is not entirely accurate. Figure \ref{fig:visualize norm} illustrates that the norm $\|h_t - h_{t-1}\|$ of Conv-GRU converges to $0$ after more than $20$ iterations. Nevertheless, Figure \ref{fig:visualize loss} reveals that both the classification loss and self-supervised loss of the model exhibit divergence, corresponding to Conv-GRU encountering "overthinking" on corruption test sets, as shown in Figure \ref{fig:cnn_gru_gn_alpha_0.0}. 

In contrast, Conv-LiGRU demonstrates greater stability. Specifically, not only does the norm $\|h_t - h_{t-1}\|$ converge to $0$ (Figure \ref{fig:visualize norm}), but both the main loss and auxiliary loss also converge smoothly to $0$ (Figure \ref{fig:visualize loss}). Additionally, Figure \ref{fig:cnn_ligru_alpha_0.0} shows that Conv-LiGRU significantly mitigates the "overthinking" phenomenon compared to Conv-GRU. 

In conclusion, we assert that the convergence of $\|h_t - h_{t-1}\|$ alone does not guarantee that the model is free from "overthinking."

\begin{figure}[t!]
    \centering
    \begin{subfigure}{0.1\textwidth}
        \includegraphics[width=\linewidth]{figure/cat_sample.png}
        \caption{"Cat"}
        \label{fig:cat_sample}
    \end{subfigure}
    \vfill
    \begin{subfigure}{0.22\textwidth}
        \includegraphics[width=\linewidth]{figure/all_classes_main_likelihood_cat.pdf}
        \caption{Likelihood of all 10 classes over steps}
        \label{fig:all_classes_main_likelihood_hard_cat}
    \end{subfigure}
    \begin{subfigure}{0.22\textwidth}
        \includegraphics[width=\linewidth]{figure/all_classes_aux_likelihood_cat.pdf}
        \caption{Likelihood of all 4 rotation classes over steps}
        \label{fig:all_class_likelihood_aux_hard_cat}
    \end{subfigure}
    \caption{Likelihood across iteration of a level 5 Gaussian Noise sample in class "Cat"}
    \label{fig:likelihood_cat}
\end{figure}

\begin{figure*}[htbp]
    \centering
    \begin{subfigure}{0.4\textwidth}
        \includegraphics[width=\linewidth]{figure/visualize_norm.pdf}
        \caption{The $\|h_t - h_{t-1}\|$ across iterations}
        \label{fig:visualize norm}
    \end{subfigure}
    \hfill
    \begin{subfigure}{0.55\textwidth}
        \includegraphics[width=\linewidth]{figure/visualize_loss.pdf}
        \caption{The loss value across iterations}
        \label{fig:visualize loss}
    \end{subfigure}
    \caption{\textbf{Left:} The change in norm of feature maps of Conv-GRU and Conv-LiGRU. \textbf{Right:} Loss value across iterations of Conv-GRU and Conv-LiGRU}
    \label{fig:layer_norm}
\end{figure*}
% \begin{algorithm}[H]
% \caption{Incremental Progress Training Algorithm}
% \label{alg:incremental_progress}
% \begin{algorithmic}[1]
% \Require Parameter vector $\theta$, integer $m$, weight $\alpha$
% \For{batch\_idx = 1, 2, \dots}
%     \State Choose $n \sim U\{0, m-1\}$ and $k \sim U\{1, m-n\}$
%     \State Compute $\phi_n$ with $n$ iterations without tracking gradients
%     \State Compute $\hat{y}_{\text{prog}}$ with additional $k$ iterations
%     \State Compute $\hat{y}_m$ with a new forward pass of $m$ iterations
%     \State Compute $\mathcal{L}_{\text{max\_iters}}$ with $\hat{y}_m$
%     \State Compute $\mathcal{L}_{\text{progressive}}$ with $\hat{y}_{\text{prog}}$
%     \State Compute $\mathcal{L} = (1 - \alpha) \cdot \mathcal{L}_{\text{max\_iters}} + \alpha \cdot \mathcal{L}_{\text{progressive}}$
%     \State Compute $\nabla_\theta \mathcal{L}$ and update $\theta$
% \EndFor
% \end{algorithmic}
% \end{algorithm}

% \begin{figure*}[htbp]
%     \centering
%     \begin{subfigure}{0.48\textwidth}
%         \includegraphics[width=\linewidth]{figure/cnn_gru_gn_alpha_0.1.png}
%         \caption{Caption for Graph 2}
%         \label{fig:cnn_gru_gn_alpha_0.1}
%     \end{subfigure}
%     \hfill
%     \begin{subfigure}{0.48\textwidth}
%         \includegraphics[width=\linewidth]{figure/cnn_ligru_alpha_0.1.png}
%         \caption{Caption for Graph 2}
%         \label{fig:cnn_ligru_alpha_0.1}
%     \end{subfigure}
%     \caption{A horizontal arrangement of three graphs.}
%     \label{fig:alpha0.1}
% \end{figure*}


% To mitigate the problem of overthinking in recurrent models, \cite{bansal2022endtoend} proposed progressive loss (\ref{alg:incremental_progress}). We trained Conv-GRU and Conv-LiGRU using \ref{alg:incremental_progress} with $\alpha = 0.1$. 

% Figures \ref{fig:cnn_gru_gn_alpha_0.1} and \ref{fig:cnn_ligru_alpha_0.1} indicate that progressive loss does not limit overthinking in the Conv-GRU and Conv-LiGRU models and also reduces the accuracy stability between iterations. However, we figure out that the Progressive Loss can buff the accuracy of RNNs (Table 3). 
% %TABLE

% iled in \tabref{table:ablation}. The results show diminished \ac{ATE} performance our ground filtering process (\texttt{w/o Filter}). \Cref{fig:ablation_zaxis_nyl,fig:ablation_zaxis_loopa0} illustrates the $z$-error along the timestamp in \texttt{nyl} (NTU4DRadLM) and \texttt{LOOP\_A0} (Ms. ficantly reduces $z$-error. Furthermore, the improved elevation accuracy 




% \subsubsection{Effe}

% To anath discrete (\texttt{w/o CONT}) and continuous integration (\texttt{FULL}). Fo\cite{kubelka2023we}.
% As demonstrated in \tabref{table:ablation}, continuous integration s. 

% \subsubsection{Computation Time}

% %TABLE
% % Please add the following required packages to your document preamble:
% \usepackage{multirow}
\begin{table}[t]
\centering
\caption{QUANTITATIVE ANALYSIS: AUC SCORES}
\begin{adjustbox}{width=1\linewidth}
{
\begin{tabular}{c|cccccccccccc}
\toprule
\multirow{2}{*}{Sequence} &
 &
   \multicolumn{3}{c}{Aeva}
 &
 &
  \multicolumn{3}{c}{Continental}
 &
 &
  \multicolumn{3}{c}{Aeva-Continental}
  \\ \cline{3-5} \cline{7-9} \cline{11-13} \rule{0pt}{2.5ex}
  &
  &
  \unit{10}{m} &
  \unit{15}{m} &
  \unit{20}{m} &
  &
  \unit{10}{m} &
  \unit{15}{m} &
  \unit{20}{m} &
  &
  \unit{10}{m} &
  \unit{15}{m} &
  \unit{20}{m}
\\ 
\midrule
\multirow{1}{*}{\texttt{Sports Complex}}
% & 0  & 0.  & 0.  &   & 0. & 0. & 0. & & 0. & 0. & 0.\\
& & 0.976  & -  & -  &   & 0.809 & - & - & & 0.401 & - & -\\
\midrule
\multirow{1}{*}{\texttt{Library}}
& & 0.971 & 0.975 & 0.988   &  & 0.574 & 0.584 & 0.632 & & 0.276 & 0.296 & 0.331 \\
\bottomrule
\end{tabular}
}
\end{adjustbox}
\label{tab:pr}
\vspace{-2mm}
\end{table}

% %TABLE

% The time consumption analysis results in \texttt{loop2} and \texttt{LOOP\_A0} are presented in \tabref{table:time}. Each sequence is the longest path in the dataset and we utilized Intel i7 CPU@2.50 {\GHz} and 64 {\GB} RAM.