\section{Related work}
\label{sec:relatedwork}
% Compared to existing datasets, the HeRCULES dataset offers several key advantages for SLAM and sensor fusion research, as shown in \tabref{table:related}.
\subsection{The Dataset with 4D Radar}
 4D radar captures range, azimuth, elevation, \ac{RCS}, and Doppler velocity, enhancing perception capabilities in dynamic scenarios. While Astyx \cite{8904734} was pioneering in using 4D radar for object detection, its limited diversity and lack of localization references restrict broader applications. Similarly, RADIal \cite{Rebut_2022_CVPR} offers multi-modal data for urban environments but lacks varied environmental conditions and does not include an \ac{IMU}, relying instead on mono camera setups.
View-of-Delft \cite{palffy2022multi} incorporates \ac{GPS}, \ac{IMU}, 4D radar, cameras, and \ac{LiDAR} for object detection and tracking. However, its radar is limited to short-range and lacks long-range 4D radar data. The TJ4DRadSet dataset \cite{zheng2022tj4dradset} focuses on object detection and tracking but excludes adverse weather conditions and lacks an \ac{IMU}. K-Radar \cite{paek2022k} offers a large-scale 4D radar dataset across various weather conditions but only includes a 6-axis \ac{IMU} integrated within the \ac{LiDAR} and lacks radar point cloud data. MSC-RAD4R \cite{choi2023msc} includes stereo cameras, \ac{LiDAR}, RTK-GPS, and \ac{IMU} data over \unit{51.6}{km} but suffers from significant RTK closure errors in height and incorrect headings from the \ac{AHRS} system. NTU4DRadLM \cite{zhang2023ntu4dradlm} is limited in diverse weather conditions, reducing its effectiveness for robust \ac{SLAM} research. Although Dual Radar \cite{zhang2023dual} and Snail Radar \cite{huai2024snail} feature dual radar systems and diverse environments, they only utilize homogeneous radars.
% and do not combine the benefits of cutting-edge sensor technologies like FMCW LiDAR and spinning radar to the extent found in the HeRCULES dataset.

The HeRCULES dataset is the first to combine 4D radar, spinning radar, \ac{FMCW} \ac{LiDAR}, cameras, \ac{IMU}, and RTK-GPS. Unlike other datasets, HeRCULES provides not only the radar point cloud data from the 4D radar but also the object point cloud information filtered through the object filtering process of the ars548 RDI radar driver\footnote{https://github.com/robotics-upo/ars548\_ros/tree/noetic}.

\begin{figure}[!t]
    \centering
    \includegraphics[trim= 10.5cm 3.6cm 11cm 2.85cm, clip,width=1\linewidth]{figure/ov.pdf}
    \caption{Day, dusk, and night conditions of the HeRCULES dataset.}
    \label{fig:overview}
    \vspace{-8mm}
\end{figure}

% \begin{figure}[!t]
%     \centering
%     \includegraphics[trim= 10.5cm 3.6cm 11cm 2.85cm, clip,width=0.85\linewidth]{figure/ov.pdf}
%     \caption{Heterogeneous Radar Dataset for Complex Urban Localization with FMCW LiDAR, 4D Radar, and Spinning Radar.}
%     \label{fig:overview}
%     \vspace{-8mm}
% \end{figure}


\subsection{The Dataset with Spinning Radar}
Spinning radar provides detailed 360° long-range scans, essential for mapping and localization in complex environments. While the Oxford Radar RobotCar Dataset \cite{barnes2020oxford} and MulRan \cite{kim2020mulran} offer valuable data for place recognition and localization in urban settings, they are limited in scope and sensor diversity. The Oxford Radar RobotCar Dataset is restricted to urban areas and lacks scenarios in diverse environments like mountains and river bridges. It also does not include \ac{IMU}, RTK-GPS, and nighttime data. Similarly, MulRan lacks camera data and does not cover varied weather and lighting conditions. The RADIATE \cite{sheeny2021radiate} and Boreas \cite{burnett2023boreas} datasets focus on adverse weather and multi-seasonal environments. However, RADIATE lacks sufficient repeated traversals necessary for robust place recognition tasks. Despite offering a broader range of conditions, Boreas is limited to relatively flat urban and suburban terrains, lacking the complexity of more varied landscapes. The Oxford Offroad Radar Dataset (OORD) \cite{gadd2024oord} features challenging off-road environments but lacks a comprehensive range of urban and rural scenarios.

Compared to existing datasets, the HeRCULES dataset offers several key advantages, as shown in \tabref{table:related}. All the above datasets are limited to 2D radar and do not provide Doppler velocity information. In contrast, HeRCULES combines 4D radar, \ac{FMCW} \ac{LiDAR}, and spinning radar, providing enhanced robustness in SLAM across diverse weather, lighting, urban traffic, and dynamic conditions.
% It spans various environments and includes multiple revisits to the same locations, supporting comprehensive research and development in SLAM and sensor fusion.