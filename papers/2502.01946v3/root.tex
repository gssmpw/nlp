
%%%%%%%%%%%%%%%%%%%%%%%%%%%%%%%%%%%%%%%%%%%%%%%%%%%%%%%%%%%%%%%%%%%%%%%%%%%%%%%%
%2345678901234567890123456789012345678901234567890123456789012345678901234567890
%        1         2         3         4         5         6         7         8

\documentclass[letterpaper, 10 pt, conference]{ieeeconf}  % Comment this line out if you need a4paper
% \usepackage{hyperref}
%\documentclass[a4paper, 10pt, conference]{ieeeconf}      % Use this line for a4 paper

\IEEEoverridecommandlockouts                              % This command is only needed if
                                                          % you want to use the \thanks command

\overrideIEEEmargins                                      % Needed to meet printer requirements.
% \usepackage{hyperref}
% this is to use \citeauthor comment (not compatible with cite package)
% natbib hack
% http://newsgroups.derkeiler.com/Archive/Comp/comp.text.tex/2006-02/msg00834.html
\makeatletter
\let\NAT@parse\undefined
\makeatother
\usepackage[numbers,sort&compress]{natbib}

% See the \addtolength command later in the file to balance the column lengths
% on the last page of the document
%\makeatletter
%\let\NAT@parse\undefined
%\makeatother

% The following packages can be found on http:\\www.ctan.org
\usepackage[pdftex]{graphicx}
\usepackage{amsmath}
\usepackage{amssymb}  % assumes amsmath package installed
\usepackage{subfigure}
\usepackage{subcaption}
\usepackage{multirow}
\usepackage{array,booktabs}
\usepackage{diagbox}
\usepackage{balance}
\usepackage{xspace}
\usepackage{algorithm}
\usepackage{algpseudocode}
%\usepackage{kotex} %한글
%\usepackage{enumitem}
%\usepackage{slashbox}
\usepackage{adjustbox}
\usepackage{romannum}
% \captionsetup[subfigure]{labelformat=empty}
\usepackage{float}
\usepackage{placeins}

\usepackage[final]{hyperref}
\hypersetup{
 colorlinks=true,
 linkcolor=blue,
 filecolor=magenta,
 urlcolor=blue,
 citecolor=black
}
\usepackage{cleveref}
\crefname{figure}{Fig.}{Figs.}
\Crefname{figure}{Fig.}{Figs.}
\usepackage[font=small]{caption}
\usepackage{rpm_SIunits}
\usepackage{rpm_acronyms}
\usepackage{rpm_math}
\usepackage{rpm_misc}


% for comments
\usepackage{soul,color}
\usepackage{lipsum}


% To highlight the revised MS
\usepackage{xcolor}
\newcommand{\gr}[1]{{\textcolor{green}{#1}}}
\newcommand{\bl}[1]{{\textcolor{black}{#1}}}
\newcommand{\B}[1]{{\textbf{#1}}}
\newtheorem{theorem}{Theorem}

\DeclareMathOperator*{\argmin}{argmin}
\DeclareMathOperator{\Tr}{Tr}

\usepackage{tikz} % \checkmark
\usepackage{subcaption}
\usepackage{caption}
\usepackage{amssymb}
\usepackage{multirow}
\usepackage{graphicx}
% \usepackage[table,xcdraw]{xcolor}
\title{\LARGE \bf HeRCULES: Heterogeneous Radar Dataset\\
in Complex Urban Environment for Multi-session Radar SLAM
}     
% \title{\LARGE \bf HeRCULEs: Heterogeneous Radar dataset\\for Complex Urban Localization and state Estimation Envitonment Study
% }     

\author{Hanjun Kim${}^{1}$, Minwoo Jung${}^{2}$, Chiyun Noh${}^{2}$, Sangwoo Jung${}^{2}$,\\Hyunho Song${}^{2}$, Wooseong Yang${}^{2}$, Hyesu Jang${}^{2}$ and Ayoung Kim${}^{2*}$
\thanks{$^\dagger$This work was supported by the Robotics and AI (RAI) Institute and Ministry of Trade, Industry \& Energy (MOTIE), Korea (No. 1415187329).}%
\thanks{$^{1}$H. Kim is with the Dept. of Future Automotive Mobility, SNU, Seoul, S. Korea {\tt\small hanjun815@snu.ac.kr}}%
\thanks{$^{2}$M. Jung, C. Noh, S. Jung, H. Song, W. Yang, H. Jang and A. Kim are with the Dept. of Mechanical Engineering, SNU, Seoul, S. Korea {\tt\small [moonshot, gch06208, dan0130, hun1021405, yellowish, dortz, ayoungk]@snu.ac.kr}}
% \thanks{$^{3}$H. Jang is with the Institute of Advanced Machines and Design, SNU, Seoul, S. Korea {\tt\small dortz@snu.ac.kr}}
}
\begin{document}

% \makeatletter
%   \let\@oldmaketitle\@maketitle% Store \@maketitle
%   \renewcommand{\@maketitle}{\@oldmaketitle% Update \@maketitle to insert...
%   \bigskip
%   \centering
%     \includegraphics[trim= 7.4cm 5.7cm 7.5cm 2cm, clip,width=0.8\textwidth]{figure/overview.pdf}
%     \captionof{figure}{
%       Heterogeneous Radar Dataset for Complex Urban Localization with FMCW LiDAR, 4D Radar, and Spinning Radar. The point colors in the FMCW LiDAR and 4D radar represent relative velocities, normalized for each image for visibility. Red indicates objects moving away, while blue indicates objects approaching.
%     }
%     \vspace{-3mm}
%     \label{fig:overview}
%   }%an image
% %trim = 40 70 30 60, clip,

\makeatletter
  \let\@oldmaketitle\@maketitle% Store \@maketitle
  \renewcommand{\@maketitle}{\@oldmaketitle% Update \@maketitle to insert...
  \bigskip
  \centering
    \includegraphics[trim= 0cm 0.9cm 0cm 1.2cm, clip,width=0.93\textwidth]{figure/1.pdf}
    \captionof{figure}{
      Overview of the HeRCULES Dataset. The FMCW LiDAR and 4D radar point colors represent relative velocities, with red indicating objects moving away and blue indicating objects approaching. Colors are normalized for each image to enhance visibility.
    }
    \vspace{-3mm}
    \label{fig:main}
  }%an image
%trim = 40 70 30 60, clip,


\makeatother
%\onecolumn
\maketitle
\thispagestyle{empty}
\pagestyle{empty}
\setcounter{figure}{1}
\begin{abstract}


The choice of representation for geographic location significantly impacts the accuracy of models for a broad range of geospatial tasks, including fine-grained species classification, population density estimation, and biome classification. Recent works like SatCLIP and GeoCLIP learn such representations by contrastively aligning geolocation with co-located images. While these methods work exceptionally well, in this paper, we posit that the current training strategies fail to fully capture the important visual features. We provide an information theoretic perspective on why the resulting embeddings from these methods discard crucial visual information that is important for many downstream tasks. To solve this problem, we propose a novel retrieval-augmented strategy called RANGE. We build our method on the intuition that the visual features of a location can be estimated by combining the visual features from multiple similar-looking locations. We evaluate our method across a wide variety of tasks. Our results show that RANGE outperforms the existing state-of-the-art models with significant margins in most tasks. We show gains of up to 13.1\% on classification tasks and 0.145 $R^2$ on regression tasks. All our code and models will be made available at: \href{https://github.com/mvrl/RANGE}{https://github.com/mvrl/RANGE}.

\end{abstract}


\section{Introduction}

Video generation has garnered significant attention owing to its transformative potential across a wide range of applications, such media content creation~\citep{polyak2024movie}, advertising~\citep{zhang2024virbo,bacher2021advert}, video games~\citep{yang2024playable,valevski2024diffusion, oasis2024}, and world model simulators~\citep{ha2018world, videoworldsimulators2024, agarwal2025cosmos}. Benefiting from advanced generative algorithms~\citep{goodfellow2014generative, ho2020denoising, liu2023flow, lipman2023flow}, scalable model architectures~\citep{vaswani2017attention, peebles2023scalable}, vast amounts of internet-sourced data~\citep{chen2024panda, nan2024openvid, ju2024miradata}, and ongoing expansion of computing capabilities~\citep{nvidia2022h100, nvidia2023dgxgh200, nvidia2024h200nvl}, remarkable advancements have been achieved in the field of video generation~\citep{ho2022video, ho2022imagen, singer2023makeavideo, blattmann2023align, videoworldsimulators2024, kuaishou2024klingai, yang2024cogvideox, jin2024pyramidal, polyak2024movie, kong2024hunyuanvideo, ji2024prompt}.


In this work, we present \textbf{\ours}, a family of rectified flow~\citep{lipman2023flow, liu2023flow} transformer models designed for joint image and video generation, establishing a pathway toward industry-grade performance. This report centers on four key components: data curation, model architecture design, flow formulation, and training infrastructure optimization—each rigorously refined to meet the demands of high-quality, large-scale video generation.


\begin{figure}[ht]
    \centering
    \begin{subfigure}[b]{0.82\linewidth}
        \centering
        \includegraphics[width=\linewidth]{figures/t2i_1024.pdf}
        \caption{Text-to-Image Samples}\label{fig:main-demo-t2i}
    \end{subfigure}
    \vfill
    \begin{subfigure}[b]{0.82\linewidth}
        \centering
        \includegraphics[width=\linewidth]{figures/t2v_samples.pdf}
        \caption{Text-to-Video Samples}\label{fig:main-demo-t2v}
    \end{subfigure}
\caption{\textbf{Generated samples from \ours.} Key components are highlighted in \textcolor{red}{\textbf{RED}}.}\label{fig:main-demo}
\end{figure}


First, we present a comprehensive data processing pipeline designed to construct large-scale, high-quality image and video-text datasets. The pipeline integrates multiple advanced techniques, including video and image filtering based on aesthetic scores, OCR-driven content analysis, and subjective evaluations, to ensure exceptional visual and contextual quality. Furthermore, we employ multimodal large language models~(MLLMs)~\citep{yuan2025tarsier2} to generate dense and contextually aligned captions, which are subsequently refined using an additional large language model~(LLM)~\citep{yang2024qwen2} to enhance their accuracy, fluency, and descriptive richness. As a result, we have curated a robust training dataset comprising approximately 36M video-text pairs and 160M image-text pairs, which are proven sufficient for training industry-level generative models.

Secondly, we take a pioneering step by applying rectified flow formulation~\citep{lipman2023flow} for joint image and video generation, implemented through the \ours model family, which comprises Transformer architectures with 2B and 8B parameters. At its core, the \ours framework employs a 3D joint image-video variational autoencoder (VAE) to compress image and video inputs into a shared latent space, facilitating unified representation. This shared latent space is coupled with a full-attention~\citep{vaswani2017attention} mechanism, enabling seamless joint training of image and video. This architecture delivers high-quality, coherent outputs across both images and videos, establishing a unified framework for visual generation tasks.


Furthermore, to support the training of \ours at scale, we have developed a robust infrastructure tailored for large-scale model training. Our approach incorporates advanced parallelism strategies~\citep{jacobs2023deepspeed, pytorch_fsdp} to manage memory efficiently during long-context training. Additionally, we employ ByteCheckpoint~\citep{wan2024bytecheckpoint} for high-performance checkpointing and integrate fault-tolerant mechanisms from MegaScale~\citep{jiang2024megascale} to ensure stability and scalability across large GPU clusters. These optimizations enable \ours to handle the computational and data challenges of generative modeling with exceptional efficiency and reliability.


We evaluate \ours on both text-to-image and text-to-video benchmarks to highlight its competitive advantages. For text-to-image generation, \ours-T2I demonstrates strong performance across multiple benchmarks, including T2I-CompBench~\citep{huang2023t2i-compbench}, GenEval~\citep{ghosh2024geneval}, and DPG-Bench~\citep{hu2024ella_dbgbench}, excelling in both visual quality and text-image alignment. In text-to-video benchmarks, \ours-T2V achieves state-of-the-art performance on the UCF-101~\citep{ucf101} zero-shot generation task. Additionally, \ours-T2V attains an impressive score of \textbf{84.85} on VBench~\citep{huang2024vbench}, securing the top position on the leaderboard (as of 2025-01-25) and surpassing several leading commercial text-to-video models. Qualitative results, illustrated in \Cref{fig:main-demo}, further demonstrate the superior quality of the generated media samples. These findings underscore \ours's effectiveness in multi-modal generation and its potential as a high-performing solution for both research and commercial applications.
\section{Related Work}
\label{sec:relatedwork}

\subsection{Current AI Tools for Social Service}
\label{subsec:relatedtools}
% the title I feel is quite broad

Harnessing technology for social good has always been a grand challenge in social service \cite{berzin_practice_2015}. As early as the 90s, artificial neural networks and predictive models have been employed as tools for risk assessments, decision-making, and workload management in sectors like child protective services and mental health treatment \cite{fluke_artificial_1989, patterson_application_1999}. The recent rise of generative AI is poised to further advance social service practice, facilitating the automation of administrative tasks, streamlining of paperwork and documentation, optimisation of resource allocation, data analysis, and enhancing client support and interventions \cite{fernando_integration_2023, perron_generative_2023}.

Today, AI solutions are increasingly being deployed in both policy and practice \cite{goldkind_social_2021, hodgson_problematising_2022}. In clinical social work, AI has been used for risk assessments, crisis management, public health initiatives, and education and training for practitioners \cite{asakura_call_2020, gillingham2019can, jacobi_functions_2023, liedgren_use_2016, molala_social_2023, rice_piloting_2018, tambe_artificial_2018}. AI has also been employed for mental health support and therapeutic interventions, with conversational agents serving as on-demand virtual counsellors to provide clinical care and support \cite{lisetti_i_2013, reamer_artificial_2023}.
% commercial solutions include Woebot, which simulates therapeutic conversation, and Wysa, an “emotionally intelligent” AI coach, powered by evidenced-based clinical techniques \cite{reamer_artificial_2023}. 
% Non-clinical AI agents like Replika and companion robots can also provide social support and reduce loneliness amongst individuals \cite{ahmed_humanrobot_2024, chaturvedi_social_2023, pani_can_2024, ta_user_2020}.

Present research largely focuses on \textit{\textbf{AI-based decision support tools}} in social service \cite{james_algorithmic_2023, kawakami2022improving}, especially predictive risk models (PRMs) used to predict social service risks and outcomes \cite{gillingham2019can, van2017predicting}, like the Allegheny Family Screening Tool (AFST), which assesses child abuse risk using data from US public systems \cite{chouldechova_case_2018, vaithianathan2017developing}. Elsewhere, researchers have also piloted PRMs to predict social service needs for the homeless using Medicaid data\cite{erickson_automatic_2018, pourat_easy_2023}, and AI-powered algorithms to promote health interventions for at-risk populations, such as HIV testing among Californian homeless \cite{rice_piloting_2018, yadav_maximizing_2017}.

\subsection{Generative AI and Human-AI Collaboration}
\label{subsec:relatedworkhaicollaboration}
Beyond decision-making algorithms and PRMs, advancements in generative AI, such as large language models (LLMs), open new possibilities for human-AI (HAI) collaboration in social services. 
LLMs have been called "revolutionary" \cite{fui2023generative} and a "seismic shift" \cite{cooper2023examining}, offering "content support" \cite{memmert2023towards} by generating realistic and coherent responses to user inputs \cite{cascella2023evaluating}. Their vastly improved capabilities and ubiquity \cite{cooper2023examining} makes them poised to revolutionise work patterns \cite{fui2023generative}. Generative AI is already used in fields like design, writing, music, \cite{han2024teams, suh2021ai, verheijden2023collaborative, dhillon2024shaping, gero2023social} healthcare, and clinical settings \cite{zhang2023generative, yu2023leveraging, biswas2024intelligent}, with promising results. However, the social service sector has been slower in adopting AI \cite{diez2023artificial, kawakami2023training}.

% Yet, the social service sector is one that could perhaps stand to gain the most from AI technologies. As Goldkind \cite{goldkind_social_2021} writes, social service, as a "values-centred profession with a robust code of ethics" (p. 372), is uniquely placed to inform the development of thoughtful algorithmic policy and practice. 
Social service, however, stands to benefit immensely from generative AI. SSPs work in time-poor environments \cite{tiah_can_2024}, often overwhelmed with tedious administrative work \cite{meilvang_working_2023} and large amounts of paperwork and data processing \cite{singer_ai_2023, tiah_can_2024}. 
% As such, workers often work in time-poor environments and are burdened with information overload and administrative tasks \cite{tiah_can_2024, meilvang_working_2023}. 
Generative AI is well-placed to streamline and automate tasks like formatting case notes, formulating treatment plans and writing progress reports, which can free up valuable time for more meaningful work like client engagement and enhance service quality \cite{fernando_integration_2023, perron_generative_2023, tiah_can_2024, thesocialworkaimentor_ai_nodate}. 

Given the immense potential, there has been emerging research interest in HAI collaboration and teamwork in the Human-Computer Interaction and Computer Supported Cooperative Work space \cite{wang_human-human_2020}. HAI collaboration and interaction has been postulated by researchers to contribute to new forms of HAI symbiosis and augmented intelligence, where algorithmic and human agents work in tandem with one another to perform tasks better than they could accomplish alone by augmenting each other's strengths and capabilities  \cite{dave_augmented_2023, jarrahi_artificial_2018}.

However, compared to the focus on AI decision-making and PRM tools, there is scant research on generative AI and HAI collaboration in the social service sector \cite{wykman_artificial_2023}. This study therefore seeks to fill this critical gap by exploring how SSPs use and interact with a novel generative AI tool, helping to expand our understanding of the new opportunities that HAI collaboration can bring to the social service sector.

\subsection{Challenges in AI Use in Social Service}
\label{subsec:relatedworkaiuse}

% Despite the immense potential of AI systems to augment social work practice, there are multiple challenges with integrating such systems into real-life practice. 
Despite its evident benefits, multiple challenges plague the integration of AI and its vast potential into real-life social service practice.
% Numerous studies have investigated the use of PRMs to help practitioners decide on a course of action for their clients. 
When employing algorithmic decision-making systems, practitioners often experience tension in weighing AI suggestions against their own judgement \cite{kawakami2022improving, saxena2021framework}, being uncertain of how far they should rely on the machine. 
% Despite often being instructed to use the tool as part of evaluating a client, 
Workers are often reluctant to fully embrace AI assessments due to its inability to adequately account for the full context of a case \cite{kawakami2022improving, gambrill2001need}, and lack of clarity and transparency on AI systems and limitations \cite{kawakami2022improving}. Brown et al. \cite{brown2019toward} conducted workshops using hypothetical algorithmic tools 
% to understand service providers' comfort levels with using such tools in their work,
and found similar issues with mistrust and perceived unreliability. Furthermore, introducing AI tools can  create new problems of its own, causing confusion and distrust amongst workers \cite{kawakami2022improving}. Such factors are critical barriers to the acceptance and effective use of AI in the sector.

\citeauthor{meilvang_working_2023} (2023) cites the concept of \textit{boundary work}, which explores the delineation between "monotonous" administrative labour and "professional", "knowledge based" work drawing on core competencies of SSPs. While computers have long been used for bureaucratic tasks like client registration, the introduction of decision support systems like PRMs stirred debate over AI "threatening professional discretion and, as such, the profession itself" \cite{meilvang_working_2023}. Such latent concerns arguably drive the resistance to technology adoption described above. Generative AI is only set to further push this boundary, 
% these concerns are only set to grow in tandem with the vast capabilities of generative and other modern AI systems. Compared to the relatively primitive AI systems in past years, perceived as statistical algorithms \cite{brown2019toward} turning preset inputs like client age and behavioural symptoms \cite{vaithianathan2017developing} into simple numerical outputs indicating various risk scores, modern AI systems are vastly more capable: LLMs 
with its ability to formulate detailed reports and assessments that encroach upon the "core" work of SSPs.
% accept unrestricted and unstructured inputs and return a range of verbose and detailed evaluations according to the user's instructions. 
Introducing these systems exacerbate previously-raised issues such as understanding the limitations and possibilities of AI systems \cite{kawakami2022improving} and risk of overreliance on AI \cite{van2023chatgpt}, and requires a re-examination of where users fall on the algorithmic aversion-bias scale \cite{brown2019toward} and how they detect and react to algorithmic failings \cite{de2020case}. We address these critical issues through an empirical, on-the-ground study that to our knowledge is the first of its kind since the new wave of generative AI.

% W 

% Yet, to date, we have limited knowledge on the real-world impacts and implications of human-AI collaboration, and few studies have investigated practitioners’ experiences working with and using such AI systems in practice, especially within the social work context \cite{kawakami2022improving}. A small number of studies have explored practitioner perspectives on the use of AI in social work, including Kawakami et al. \cite{kawakami2022improving}, who interviewed social workers on their experiences using the AFST; Stapleton et al. \cite{stapleton_imagining_2022}, who conducted design workshops with caseworkers on the use of PRMs in child welfare; and Wassal et al. \cite{wassal_reimagining_2024}, who interviewed UK social work professionals on the use of AI. A common thread from all these studies was a general disregard for the context and users, with many practitioners criticising the failure of past AI tools arising from the lack of participation and involvement of social workers and actual users of such systems in the design and development of algorithmic systems \cite{wassal_reimagining_2024}. Similarly, in a scoping review done on decision-support algorithms in social work, Jacobi \& Christensen \cite{jacobi_functions_2023} reported that the majority of studies reveal limited bottom-up involvement and interaction between social workers, researchers and developers, and that algorithms were rarely developed with consideration of the perspective of social workers.
% so the \cite{yang_unremarkable_2019} and \cite{holten_moller_shifting_2020} are not real-world impacts? real-world means to hear practitioner's voice? I feel this is quite important but i didnt get this point in intro!

% why mentioning 'which have largely focused on existing ADS tools (e.g., AFST)'? i can see our strength is more localized, but without basic knowledge of social work i didnt get what's the 'departure' here orz
% the paragraph is great! do we need to also add one in line 20 21?

\subsection{Designing AI for Social Service through Participatory Design}
\label{subsec:relatedworkpd}
% i think it's important! but maybe not a whole subsection? but i feel the strong connection with practitioners is indeed one of our novelties and need to highlight it, also in intro maybe
% Participatory design (PD) has long been used extensively in HCI \cite{muller1993participatory}, to both design effective solutions for a specific community and gain a deep understanding of that community. Of particular interest here is the rich body of literature on PD in the field of healthcare \cite{donetto2015experience}, which in this regard shares many similarities and concerns with social work. PD has created effective health improvement apps \cite{ryu2017impact}, 

% PD offers researchers the chance to gather detailed user requirements \cite{ryu2017impact}...

Participatory design (PD) is a staple of HCI research \cite{muller1993participatory}, facilitating the design of effective solutions for a specific community while gaining a deep understanding of its stakeholders. The focus in PD of valuing the opinions and perspectives of users as experts \cite{schuler_participatory_1993} 
% In recent years, the tech and social work sectors have awakened to the importance of involving real users in designing and implementing digital technologies, developing human-centred design processes to iteratively design products or technologies through user feedback 
has gained importance in recent years \cite{storer2023reimagining}. Responding to criticisms and failures of past AI tools that have been implemented without adequate involvement and input from actual users, HCI scholars have adopted PD approaches to design predictive tools to better support human decision-making \cite{lehtiniemi_contextual_2023}.
% ; accordingly, in social service, a line of research has begun studying and designing for human-AI collaboration with real-world users (e.g. \cite{holten_moller_shifting_2020, kawakami2022improving, yang_unremarkable_2019}).
Section \ref{subsec:relatedworkaiuse} shows a clear need to better understand SSP perspectives when designing and implementing AI tools in the social sector. 
Yet, PD research in this area has been limited. \citeauthor{yang2019unremarkable} (2019), through field evaluation with clinicians, investigated reasons behind the failure of previous AI-powered decision support tools, allowing them to design a new-and-improved AI decision-support tool that was better aligned with healthcare workers’ workflows. Similarly, \citeauthor{holten_moller_shifting_2020} (2020) ran PD workshops with caseworkers, data scientists and developers in public service systems to identify the expectations and needs that different stakeholders had in using ADS tools.

% Indeed, it is as Wise \cite{wise_intelligent_1998} noted so many years ago on the rise of intelligent agents: “it is perhaps when technologies are new, when their (and our) movements, habits and attitudes seem most awkward and therefore still at the forefront of our thoughts that they are easiest to analyse” (p. 411). 
Building upon this existing body of work, we thus conduct a study to co-design an AI tool \textit{for} and \textit{with} SSPs through participatory workshops and focus group discussions. In the process, we revisit many of the issues mentioned in Section \ref{subsec:relatedworkaiuse}, but in the context of novel generative AI systems, which are fundamentally different from most historical examples of automation technologies \cite{noy2023experimental}. This valuable empirical inquiry occurs at an opportune time when varied expectations about this nascent technology abound \cite{lehtiniemi_contextual_2023}, allowing us to understand how SSPs incorporate AI into their practice, and what AI can (or cannot) do for them. In doing so, we aim to uncover new theoretical and practical insights on what AI can bring to the social service sector, and formulate design implications for developing AI technologies that SSPs find truly meaningful and useful.
% , and drive future technological innovations to transform the social service sector not just within [our country], but also on a global scale.

 % with an on-the-ground study using a real prototype system that reflects the state of AI in current society. With the presumption that AI will continue to be used in social work given the great benefits it brings, we address the pressing need to investigate these issues to ensure that any potential AI systems are designed and implemented in a responsible and effective manner.

% Building upon these works, this study therefore seeks to adopt a participatory design methodology to investigate social workers’ perspectives and attitudes on AI and human-AI collaboration in their social work practice, thus contributing to the nascent body of practitioner-centred HCI research on the use of AI in social work. Yet, in a departure from prior work, which have largely focused on existing ADS tools (e.g., AFST) and were situated in a Western context, our paper also aims to expand the scope by piloting a novel generative AI tool that was designed and developed by the researchers in partnership with a social service agency based in Singapore, with aims of generating more insights on wider use cases of AI beyond what has been previously studied.

% i may think 'While the current lacunae of research on applications of AI in social work may appear to be a limitation, it simultaneously presents an exciting opportunity for further research and exploration \cite{dey_unleashing_2023},' this point is already convincing enough, not sure if we need to quote here
% I like this end! it's a good transition to our study design, do we need to mention the localization in intro as well? like we target at singapore

% Given the increasing prominence and acceptance of AI in modern society, 

% These increased capabilities vastly exacerbate the issues already present with a simpler tool like the AFST: the boundaries and limitations of an LLM system are significantly more difficult to understand and its possible use cases are exponentially greater in scope. 

% Put this in discussion section instead?
% Kawakami et al's work "highlights the importance of studying how collaborative decision-making... impacts how people rely upon and make sense of AI models," They conclude by recommending designing tools that "support workers in understanding the boundaries of [an AI system's] capabilities", and implementing design procedures that "support open cultures for critical discussion around AI decision making". The authors outline critical challenges of implementing AI systems, elucidating factors that may hinder their effectiveness and even negatively affect operations within the organisation.


% Is this needed?:
% talk about the strengths of PD in eliciting user viewpoints and knowledge, in particular when it is a field that is novel or where a certain system has not been used or developed or tested before
\section{System overview}
\label{sec:overview}
\subsection{System Configuration}

% The sensor configuration and coordinates of each sensor are depicted in \figref{fig:car}, and their specifications are provided in \tabref{tab:sensors}. The FMCW LiDAR runs at 10 Hz for sensor operation with relative velocity settings aligned to the 4D radar sensor. The 4D radar operates at 20 Hz, providing radar point cloud data and filtered object point cloud data based on the sensor driver  \cite{fernándezcalatayud2024rosrdi}. The spinning radar operates at 4 Hz, while the Xsens MTi-300 captures inertial data at 100 Hz. The SPAN-CPT7 with a dual VEXXIS GNSS-501 antenna operates at 50 Hz using RTK-GPS and INS. All sensor data are processed on an industrial PC, NUVO-9006LP-NX, with an Intel Core i9, 2 TB SSD, and 64 GB DDR5.

% The sensor configuration and coordinates of each sensor are depicted in \figref{fig:car}, and their specifications are provided in \tabref{tab:sensors}. The Aeva LiDAR operates at 10 Hz, with relative velocity settings synchronized with the 4D radar sensor. The Continental radar operates at 20 Hz, providing radar point cloud data and filtered object point cloud data based on the sensor driver  \cite{fernándezcalatayud2024rosrdi}. The Navtech radar operates at 4 Hz, while the Xsens MTi-300 captures inertial data at 100 Hz. The SPAN-CPT7 with a dual VEXXIS GNSS-501 antenna operates at 50 Hz using RTK-GPS and INS. All sensor data are processed on an industrial PC, NUVO-9006LP-NX, with an Intel Core i9, 2 TB SSD, and 64 GB DDR5.

The sensor configuration and coordinates of each sensor are illustrated in \figref{fig:car}, and their specifications are detailed in \tabref{tab:sensors}. The Aeva \ac{LiDAR} operates with relative velocity settings synchronized to the 4D radar sensor. The Continental radar provides both raw radar point cloud data and filtered object point cloud data via its sensor driver. All sensor data are processed on an industrial PC, the NUVO-9006LP-NX, equipped with an Intel Core i9 processor, 2 TB SSD, and 64 GB DDR5 RAM. The sample data is shown in \figref{fig:overview}.

% \cite{fernándezcalatayud2024rosrdi}

 % A LiFePO$_4$ 102.0Ah battery powers all equipment and sensor drivers run on ROS Noetic with Ubuntu 20.04.

% \begin{figure}[!h]
%     \centering
%     \includegraphics[trim= 6.9cm 9.2cm 7.4cm 6cm, clip,width=0.9\linewidth]{figure/cali.pdf}
%     \caption{(a) Spinning radar points (blue), 4D radar points (green), and LiDAR points (red) (b) Utilizing the line-index channel. (c) LiDAR - spinning radar extrinsic calibration pipeline. }
%     \label{fig:cali}
%     \vspace{-5mm}
% \end{figure}

% \begin{figure}[!t]
%     \centering
%     \includegraphics[trim= 7.5cm 7.8cm 7cm 6.5cm, clip,width=\linewidth]{figure/cali.pdf}
%     \caption{(a) LiDAR - spinning radar extrinsic calibration pipeline. (b) Utilizing the line-index channel. (c) LiDAR points, 4D radar points, and spinning radar points are red, green, and blue. (d)   Left camera - LiDAR. (e) Right camera - LiDAR. (f) Left camera - 4D radar. (g) Right camera - 4D radar.}
%     \label{fig:cali}
%     \vspace{-5mm}
% \end{figure}

\begin{figure}[!t]
    \centering
    \includegraphics[trim= 7.5cm 17cm 7cm 6.5cm, clip,width=\linewidth]{figure/cali.pdf}
    \caption{(a) LiDAR - spinning radar extrinsic calibration pipeline. (b) Utilizing the line-index channel. (c) LiDAR points, 4D radar points, and spinning radar points are red, green, and blue. (d)   Right camera - LiDAR. (e) Left camera - 4D radar.}
    \label{fig:cali}
    \vspace{-5mm}
\end{figure}

\subsection{ Sensor Calibration}
% The calibration results of LiDAR - 4D Radar - Spinning Radar are shown in \figref{fig:cali}(a).

% \subsubsection{Intrinsic Calibration of Camera}
% With the intrinsic calibration tool \cite{zhang2000flexible}, we utilize a checkboard pattern to calculate the internal parameters simultaneously.

\subsubsection{Extrinsic Calibration of \ac{LiDAR} - Spinning Radar}
We employ the method used in the Boreas dataset \cite{burnett2023boreas}. This method determines the rotation $\textbf{R}^L_R$ and the translation $\textbf{t}^L_R$ in the xy plane through correlative scan matching with the Fourier-Mellin transform \cite{checchin2010radar}. Specifically, we convert \ac{LiDAR} point clouds into \ac{LiDAR} polar images to compare with radar polar images to obtain $\textbf{R}^L_R$. Then, we utilize Cartesian images to derive $\textbf{t}^L_R$. To match the \ac{FOV} of the Aeva, we adjust the range and azimuth of the radar images. The calibration pipeline is shown in \figref{fig:cali}(a).


\subsubsection{Extrinsic Calibration of \ac{LiDAR} - 4D Radar - Camera}
We utilize the calibration tool \cite{domhof2021joint} for cameras, \ac{LiDAR}, and radar. This tool jointly calculates relative transformation parameters using a specialized calibration board and reflector. Although we use a solid-state \ac{LiDAR} instead of a spinning one, we utilize the line-index channel to assess laser depth discontinuity, as illustrated in \figref{fig:cali}(b). Unlike \citeauthor{domhof2021joint} \cite{domhof2021joint}, who estimates the reflector position with 2D radar, we can directly obtain the z-value from our 4D radar, resulting in more accurate calibration.


% The translation parameters in the z-direction are obtained from CAD files. 
\subsubsection{Extrinsic Calibration of \ac{LiDAR} - \ac{IMU}}
We initialize the system using the method proposed by \citeauthor{zhu2022robust} \cite{zhu2022robust}. This approach was designed for the Livox \ac{LiDAR} series, so it can be seamlessly applied to our solid-state Aeva \ac{LiDAR} without requiring specific targets.
    % \item Extrinsic Calibration of INS - IMU: 


\subsubsection{Calibration Evaluation}
% The calibration results of \ac{LiDAR} - 4D Radar - Spinning Radar are shown in \figref{fig:cali}(c).
The calibration results are shown in \figref{fig:cali}(c), \figref{fig:cali}(d) and \figref{fig:cali}(e).


% %FIGURE
% \begin{figure}[!t]
%     \centering
%     \includegraphics[trim=0cm 2.5cm 14.5cm 1.65cm, clip, width=1\columnwidth]{figure/pipeline.pdf}
%     \caption{all pipeline }
%     \label{fig:pipeline}
%     \vspace{-5mm}
% \end{figure}

% In thi \figref{fig:pipeline},ggggg

% \subsection{Point-wise Uncertainty Guided Ground Filtering}
% \label{subsec:ground segmentation}
% % ground plane detection and outlier removal




% \begin{equation}
% \begin{array}{rrclcl}
%     \displaystyle \argmin_{[a,b,c,d]} & \multicolumn{3}{l}{\sum_{i=1}^n D_{M_i}^2}\\
%     \textrm{s.t.} & a^2+b^2+c^2=1.
% \end{array}
% \label{eqn:plane_model}
% \end{equation}

% With \eqref{eqn:plane_model}, we c


% \begin{equation}
%     z_i=\frac{V_{xy}v_z - \sqrt{V_{xy}^2{v^d_i}^2 - ((v^d_i)^2-v_z^2)(x^2+y^2){v^d_i}^2}}{(v^d_i)^2-v_z^2}
% \end{equation}
% where $v_x, v_y, v_z$ is the vehicle's ego velocity, and $V_{xy}=v_x x_i+v_y y_i$. After the height 



% \subsection{Cs}




% \begin{equation}
%     \dot\theta_i(t)\sim\mathcal{GP}(0,k_{r_i}(t,t'))
% \end{equation}

% where $k_{r_i}(t,t')$ is the kernel function representing the covariance of the function instances and the index $i$ represents the $x,y,z$ component.
% Then the inference of the rotation vector is conducted as follows:

% \begin{equation}
%     \dot{\theta}_{i\star}(t)=\mathbf{k}_{r_i}(t,\mathbf{t})(\mathbf{K}_{r_i}(\mathbf{t}, \mathbf{t})+\sigma_i^2\mathbf{I})^{-1}\boldsymbol{\rho}_i
% \end{equation}

% \begin{equation}
% \label{eqn:rot_inference} 
%     \theta_{i\star}(t)=\int\mathbf{k}_{r_i}(t,\mathbf{t})(\mathbf{K}_{r_i}(\mathbf{t}, \mathbf{t})+\sigma_i^2\mathbf{I})^{-1}\boldsymbol{\rho}_i \partial t
% \end{equation}

% $\mathbf{t}=[t_1\dots t_{N_i}]$ is the vector of the \ac{IMU} measurement timestamps between consecutive keyframes,
% $\mathbf{k}_{r_i}(t,\mathbf{t})=[k_{r_i}(t,t_1)\dots k_{r_i}(t,t_{N_i})]^\top$ is the kernel, and
% $\newline \mathbf{K}_{r_i}(\mathbf{t},\mathbf{t})=\begin{bmatrix}
% k_{r_i}(t_1,t_1) & \cdots & k_{r_i}(t_1,t_{N_i}) \\
% \vdots & \ddots & \vdots \\
% k_{r_i}(t_{N_i},t_1) & \cdots & k_{r_i}(t_{N_i},t_{N_i})
% \end{bmatrix}$.

% $\sigma_i$ is the standard deviation of the \ac{IMU} gyroscope measurements, and $\boldsymbol{\rho}_i$ is the vector of $\hat{\dot{\theta}}_i(\mathbf{t})$, which can be achieved with the following optimization problem:

% $\boldsymbol{e}_n^{meas} = J_r(\boldsymbol{\theta}_\star(t_n))\dot{\boldsymbol{\theta}}_\star(t_n)-\Tilde{\boldsymbol{\omega}}(t_n)$ is about the measurement constraint, and $\Sigma_{\boldsymbol{\omega}}$ is the \ac{IMU} measurement covariance matrix at timestamp $t_n$. $\boldsymbol{e}_n^{gp}=\dot{\boldsymbol{\theta}}_{\star}(t_n)-\hat{\dot{\boldsymbol{\theta}}}(t_n)$ 

% $\boldsymbol{\mu}(t)$ as following:
% \begin{equation}
%     {v}_i(t)\sim\mathcal{GP}(\mu_i(t),k_{v_i}(t,t')).
% \end{equation}


% \begin{equation}
% \label{eqn:vel_optimize}
%     \displaystyle \argmin_{[\boldsymbol{\zeta}_x, \boldsymbol{\zeta}_y, \boldsymbol{\zeta}_z]}\sum_{n=1}^{N_r} \left( \norm{\boldsymbol{e}_n^{meas}}_{\Sigma_{\mathbf{v}_n}}^2+\norm{\boldsymbol{e}_n^{gp}}_{\Sigma_{gp}}^2 \right).
% \end{equation}


% In \eqref{eqn:vel_optimize}, $\boldsymbol{e}_n^{meas} = \Delta{\mathbf{R}^\top}\mathbf{v}_\star(t_n)-\Tilde{\mathbf{v}}(t_n)$, and $\boldsymbol{e}_n^{gp}=\mathbf{v}_{\star}(t_n)-\hat{\mathbf{v}}(t_n)$. $\Sigma_{\mathbf{v}_n}$ is the velocity measurement covariance 
% chronization for integration between the \ac{IMU} gyroscope and radar dat
% crement can be inferenced by \eqref{eqn:trans_inference}.

% \subsection{Cluster-based weighted ICP}
% \label{subsec:scan matching}


% Therefore, the transformation $\textbf{T}\in SE(3)$ betwee



% \subsection{Pose Graph Optimization}
% incremental motion from \cref{subsec:velocity integration} and \ref{subsec:scan matching} is used for $SE(3)$ edges in the pose graph, and the covariance of the edge is computed in each estimation step. The $g2o$ \cite{kummerle2011g} 

%TABLE
% % % Please add the following required packages to your document preamble:
% % % \usepackage{multirow}
% % % \usepackage{graphicx}
% % % \usepackage[table,xcdraw]{xcolor}
% % % Beamer presentation requires \usepackage{colortbl} instead of \usepackage[table,xcdraw]{xcolor}
% % \begin{table*}[]
% % \caption{OVERVIEW OF SEQUENCES}
% % \label{tab:sequence}
% % \resizebox{\textwidth}{!}{
% % \begin{tabular}{c|c|cccccc} \toprule
% %   Sequence &
% %   Time & 
% %   Weather &
% %   Length &
% %   Loop &
% %   Target &
% %   Description
% %   \\ \midrule
% % \multirow{2}{*}{Mountain} &
% %   \multicolumn{1}{l|}{Day} &
% %   Clear &
% %   4km &  
% %   2 times &
% %   odometry, &
% % \multirow{2}{*}{
% %   poor road surface condition, large altitude variation}
% %   \\
% % \multicolumn{1}{l|}{} &
% %   \multicolumn{1}{l|}{Night} &
% %   Cloud &
% %   4km & 
% %   2 times&
% %   online place recognition \\
% %   \midrule
% % \multirow{2}{*}{Library} &
% %   \multicolumn{1}{l|}{Day} &
% %   Clear &
% %   1.6km &  
% %   2 times &
% %   odometry,&
% % \multirow{2}{*}{
% %   {\begin{tabular}[c]{@{}c@{}} a long narrow one-way alley, downhill and uphill at sharp curves\end{tabular}}
% %   } 
% %   \\
% % \multicolumn{1}{l|}{} &
% %   \multicolumn{1}{l|}{Night} &
% %   Cloud &
% %   1.6km & 
% %   2 times &
% %   online place recognition\\ \midrule
% % \multirow{2}{*}{Sport complex} &
% %   \multicolumn{1}{l|}{Day} &
% %   Clear &
% %   1.4km &  
% %   2 times &
% %   odometry,&
% % \multirow{2}{*}{
% %  consists of a flat section, a gentle uphill section, and a sharp downhill section}  
% %   \\
% % \multicolumn{1}{l|}{} &
% %   \multicolumn{1}{l|}{Night} &
% %   Cloud &
% %   0.7km &  
% %   1 time &
% %   online place recognition\\ \midrule
% % \multirow{3}{*}{Parking lots} &
% %   \multicolumn{1}{l|}{Day} &
% %   Clear &
% %   0.4km &  
% % \multirow{3}{*}{
% %   inter-session} &
% % \multirow{2}{*}{
% %   \\odometry,} &
% % \multirow{3}{*}{
% %   a short driving distance
% % frequent left turn} 
% %   \\
% % \multicolumn{1}{l|}{} &
% %   \multicolumn{1}{l|}{Day} &
% %   Clear &
% %   0.4km & 
% %   &
% %   \multirow{2}{*}{
% %   \\global localization} \\
% % \multicolumn{1}{l|}{} &
% %   \multicolumn{1}{l|}{Night} &
% %   Cloud &
% %   0.5km &  
% %   \\ \midrule
% % \multirow{3}{*}{River island} &
% %   \multicolumn{1}{l|}{Day} &
% %   Cloud &
% %   5.8km &  
% % \multirow{3}{*}{
% %   inter-session} &
% %   \multirow{2}{*}{
% %   \\odometry,} &
% % \multirow{3}{*}{
% %   poor road surface condition, large altitude variation} \\
% % \multicolumn{1}{l|}{} &  
% %   \multicolumn{1}{l|}{Dusk} &
% %   Cloud &
% %   8km &  
% %   &
% %     \multirow{2}{*}{
% %   \\global localization}
% %   \\
% % \multicolumn{1}{l|}{} &  
% %   \multicolumn{1}{l|}{Day} &
% %   Clear &
% %   4km &  
% %  \\ \midrule
% % \multirow{2}{*}{Bridge} &
% %   \multicolumn{1}{l|}{Day} &
% %   Rain &
% %   4.9km &  
% %   1 times &
% %   odometry, &
% % \multirow{2}{*}{
% %   a wide variety of dynamic objects}
% %   \\
% % \multicolumn{1}{l|}{} &
% %   \multicolumn{1}{l|}{Night} &
% %   Cloud &
% %   4.9km &  
% %   1 times &
% %   online place recognition
% %   \\ \midrule

% % Street &
% %   \multicolumn{1}{l|}{Day} &
% %   Rain &
% %   1km &  
% %   1 times &
% %   odometry &
% %   heavy traffic Low driving speed and frequent stops
% %   \\ \midrule
% % \multirow{2}{*}{Stream} &
% %   \multicolumn{1}{l|}{Day} &
% %   Clear &
% %   4.2km &  
% %   2 times  &
% %   odometry,&
% % \multirow{2}{*}{
% %   a wide variety of dynamic objects}
% %   \\
% % \multicolumn{-10}{c|}{} &
% %   \multicolumn{1}{l|}{Night} &
% %   Cloud &
% %   5.5km & 
% %   2 times &
% %   online place recognition 
% %  \\ \bottomrule
% % \end{tabular}
% % }
% % \end{table*}



% % % Please add the following required packages to your document preamble:
% % % \usepackage{multirow}
% % % \usepackage{graphicx}
% % % \usepackage[table,xcdraw]{xcolor}
% % % Beamer presentation requires \usepackage{colortbl} instead of \usepackage[table,xcdraw]{xcolor}
% % \begin{table*}[]
% % \caption{sequence}
% % \label{tab:my-table}
% % \resizebox{\textwidth}{!}{%
% % \begin{tabular}{c|cccccc} \toprule
% %    &
% %    &
% %   \multicolumn{2}{c}{Radar}\\ \cline{3-4} \rule{0pt}{2.5ex}
% %   \multirow{-2}{*}{Sequence} & 
% %   \multirow{-2}{*}{Length } &
% %   \multicolumn{1}{l}{4D Radar} &
% %   \multicolumn{1}{l}{Scanning Radar} &
% %   \multirow{-2}{*}{Time} &
% %   \multirow{-2}{*}{Condition} &
% %   \multirow{-2}{*}{Scenarios} \\ \midrule
% % \multicolumn{1}{l|}{Mountain} &
% %   - &
% %   - &
% %   d &  
% %   \\
% % \multicolumn{1}{l|}{Library} &

% %   - \\
% % \multicolumn{1}{l|}{Sports complex} &
% %   {\color[HTML]{000000} 0.499} &
% %   {\color[HTML]{000000} 0.785} &


% %   - \\ 
% % \multicolumn{1}{l|}{Parking lots} &
% %   {\color[HTML]{000000} 0.499} &
% %   {\color[HTML]{000000} 0.785} &
% %   - \\ \midrule
% % \multicolumn{1}{l|}{River island} &
% %   - &
% %   - \\
% % \multicolumn{1}{l|}{Bridge} &
% %   - \\

% % \multicolumn{1}{l|}{Street} &
% %   - &
% %   - \\
% % \multicolumn{1}{l|}{Stream} &
% %   {\color[HTML]{FF0000} \textbf{0.806}} &
% %   {\color[HTML]{FF0000} \textbf{0.885}} &
% %   {\color[HTML]{FF0000} \textbf{0.903}} \\ \bottomrule
% % \end{tabular}%
% % }
% % \end{table*}

% % Please add the following required packages to your document preamble:
% % \usepackage{multirow}
% % \usepackage{graphicx}
% % \usepackage[table,xcdraw]{xcolor}
% % Beamer presentation requires \usepackage{colortbl} instead of \usepackage[table,xcdraw]{xcolor}

% \begin{table*}[]
% \caption{OVERVIEW OF SEQUENCES}
% \label{tab:sequence}
% \resizebox{\textwidth}{!}{
% \begin{tabular}{c|c|cccccc} 
% \toprule
% Sequence & Time & Weather & Length & Loop & Target & Description \\ 
% \midrule

% \multirow{2}{*}{Mountain} & Day & Clear & 4km & 2 times & odometry, & \multirow{2}{*}{poor road surface condition, large altitude variation} \\
%  & Night & Cloud & 4km & 2 times & online place recognition & \\ 
% \midrule

% \multirow{2}{*}{Library} & Day & Clear & 1.6km & 2 times & odometry, &  \multirow{2}{*}{long narrow one-way alley, downhill and uphill at sharp curves}\\
%  & Night & Cloud & 1.6km & 2 times & online place recognition & \\ 
% \midrule

% \multirow{2}{*}{Sport complex} & Day & Clear & 1.4km & 2 times & odometry, & \multirow{2}{*}{flat section, gentle uphill section, sharp downhill section} \\
%  & Night & Cloud & 0.7km & 1 time & online place recognition & \\ 
% \midrule

% \multirow{3}{*}{Parking lots} & Day & Clear & 0.4km & &  \multirow{3}{*}{\begin{tabular}[c]{@{}c@{}}odometry,\\global localization \end{tabular}} &  \\
%  & Day & Clear & 0.4km & inter-session &  & short driving distance, frequent left turn\\
%  & Night & Cloud & 0.5km &  &  & \\ 
% \midrule

% \multirow{3}{*}{River island} & Day & Cloud & 5.8km &  & \multirow{3}{*}{\begin{tabular}[c]{@{}c@{}}odometry,\\global localization \end{tabular}} &  \\
%  & Dusk & Cloud & 8km & inter-session &  & various driving routes, flat area, intersections, two-lane one-way roads\\
%  & Day & Clear & 4km &  & & \\ 
% \midrule

% \multirow{2}{*}{Bridge} & Day & Rain & 4.9km & 1 time & odometry, & \multirow{2}{*}{four-lane overpass and the bridge, various dynamic objects} \\
%  & Night & Cloud & 4.9km & 1 time & online place recognition & \\ 
% \midrule

% Street & Day & Rain & 1km & 1 time & odometry & heavy traffic, low driving speed, and frequent stops \\ 
% \midrule

% \multirow{2}{*}{Stream} & Day & Clear & 4.2km & 2 times & odometry, & \multirow{2}{*}{S-shaped one-way roads, various dynamic objects, U-turns}\\
%  & Night & Cloud & 5.5km & 2 times & online place recognition & \\ 
% \bottomrule
% \end{tabular}
% }
% \end{table*}



\begin{table}[!t]
\caption{OVERVIEW OF SEQUENCES}
\label{tab:sequence}
\centering
\resizebox{\columnwidth}{!}{
\begin{tabular}{c|c|cccccc} 
\toprule
Sequence & Index & Time & Weather & Length & Loop & Target \\ 
\midrule

\multirow{3}{*}{\texttt{Mountain}} & 01 & Day & Clear & \unit{4}{km} & 2 times & \multirow{3}{*}{\begin{tabular}[c]{@{}c@{}}odometry,\\online place recognition,\\global localization \end{tabular}}   \\
 & 02 & Night & Cloud & \unit{4}{km} & 2 times &  \\ 
 & 03 & Day & Snow & \unit{3}{km} & 1.5 times &  \\ 
\midrule

\multirow{3}{*}{\texttt{Library}} & 01 & Day & Clear & \unit{1.6}{km} & 2 times & \multirow{3}{*}{\begin{tabular}[c]{@{}c@{}}odometry,\\online place recognition,\\global localization \end{tabular}}  \\
 & 02 & Night & Cloud & \unit{1.6}{km} & 2 times &  \\ 
  & 03 & Day & Snow & \unit{0.8}{km} & 1 time &  \\ 
\midrule

\multirow{3}{*}{\begin{tabular}[c]{@{}c@{}}\texttt{Sports}\\\texttt{Complex}\end{tabular}}{} & 01 & Day & Clear & \unit{1.4}{km} & 2 times & \multirow{3}{*}{\begin{tabular}[c]{@{}c@{}}odometry,\\online place recognition,\\global localization \end{tabular}} \\
 & 02 & Night & Cloud & \unit{0.7}{km} & 1 time &  \\ 
  & 03 & Day & Snow & \unit{1.4}{km} & 2 times &  \\ 
\midrule

\multirow{4}{*}{\begin{tabular}[c]{@{}c@{}}\texttt{Parking}\\\texttt{Lot}\end{tabular}}{} & 01 &  Day & Clear & \unit{0.4}{km} & \multirow{4}{*}{inter-session} & \multirow{4}{*}{\begin{tabular}[c]{@{}c@{}}odometry,\\global localization \end{tabular}}  \\
 & 02 & Day & Clear & \unit{0.4}{km} &  &  \\
 & 03 & Night & Cloud & \unit{0.5}{km} &  &  \\ 
 & 04 & Day & Snow & \unit{0.4}{km} &  &  \\ 
\midrule

\multirow{3}{*}{\begin{tabular}[c]{@{}c@{}}\texttt{River}\\\texttt{Island}\end{tabular}}{} 
& 01 & Day & Clear & \unit{4}{km} &  & \multirow{3}{*}{\begin{tabular}[c]{@{}c@{}}odometry,\\global localization \end{tabular}}  \\
 & 02 & Dusk & Cloud & \unit{8}{km} & inter-session & \\
 & 03 & Day & Cloud & \unit{5.8}{km} &  &  \\ 
\midrule

\multirow{2}{*}{\texttt{Bridge}} & 01 & Day & Rain & \unit{4.9}{km} & 1 time & odometry, \\
& 02 & Night & Cloud & \unit{4.9}{km} & 1 time & online place recognition \\ 
\midrule

\texttt{Street} & 01 & Day & Rain & \unit{1}{km} & 1 time & odometry \\ 
\midrule

\multirow{2}{*}{\texttt{Stream}} & 01 & Day & Clear & \unit{4.2}{km} & 2 times & odometry, \\

 & 02 & Night & Cloud & \unit{5.5}{km} & 2 times & online place recognition \\ 
\bottomrule
\end{tabular}
}
\vspace{-5mm}
\end{table}

%TABLE

\section{Description of HeRCULES Dataset}
\label{sec:experiment}

\subsection{Target Environments}
% This subsection briefly outlines the reasons for selecting the eight target environments depicted in \figref{fig:road} and their key characteristics. An overview of the eight sequences is presented in \tabref{tab:sequence}.
This subsection briefly outlines the reasons for selecting the eight target environments depicted in \figref{fig:road}. An overview of the eight sequences is presented in \tabref{tab:sequence}.

\subsubsection{Mountain}
\texttt{Mountain} captures sequences on Gwanak Mountain, the highest elevation difference among all sequences. The route includes speed bumps and rough roads, causing significant rolling and pitching.


\subsubsection{Library}
% \texttt{Library} captures sequences from a long, narrow, one-way path near the library on campus. The route involves two loops for loop closure with sequences recorded in clear afternoon and cloudy night. The path is narrow and includes steep curves with uphill and downhill sections.
\texttt{Library} captures sequences from a long, narrow, one-way path near the library on campus. The path includes steep curves with uphill and downhill sections.
\subsubsection{Sports Complex}
\texttt{Sports Complex} captures sequences around a sports complex, including parking areas and roads with flat, gently sloped, and steep sections. Two loops with an average speed below 30 km/h were recorded during the day and night.
\subsubsection{Parking Lot}
\texttt{Parking Lot} captures sequences from a parking lot with many left turns, recorded on a clear afternoon and at night. While the ground appears flat, slight elevation variations are noted. This sequence has the shortest distance among all.
\subsubsection{River Island}
% \texttt{River Island} captures three sequences for multi-session place recognition, recorded during the day and dusk, with each route uniquely designed. The flat area features various driving paths, including intersections and two-lane one-way roads.
\texttt{River Island} captures three sequences for multi-session place recognition with each route uniquely designed. The flat area features various driving paths, including intersections and two-lane one-way roads.
\subsubsection{Bridge}
\texttt{Bridge} captures sequences for place recognition research, driven back and forth along a four-lane overpass and the Wonhyo Bridge over the Han River. It includes sequences recorded on a rainy afternoon and cloudy dusk, with an average speed of \unit{60}{km/h}, and sections featuring traffic congestion in urban environments.
\subsubsection{Street}
\texttt{Street} captures a sequence of driving in heavy congestion and rain near IFC Seoul during rush hour. Due to the crowds and numerous vehicles, there are many dynamic objects, leading to frequent stops.
\subsubsection{Stream}
% \texttt{Stream} captures an S-shaped stream route that is driven during the day and night. One-way roads run along both sides of the stream, with U-turns via bridges. For place recognition research, intentional revisits were designed, resulting in similar environments.
\texttt{Stream} captures an S-shaped stream route with one-way roads running along both sides, allowing U-turns via bridges. For place recognition research, intentional revisits were designed, resulting in similar environments.

% \begin{figure}[!t]
%     \centering
%     \includegraphics[trim= 2cm 2.5cm 2cm 2.5cm, clip,width=1\linewidth]{example-image-a}
%     \caption{map}
%     \label{fig:map}
%     \vspace{-3mm}
% \end{figure}


\begin{figure}[!t] % "t"는 페이지 상단에 이미지를 배치하려는 옵션입니다.
    \centering
    \includegraphics[trim=0.5cm 4.55cm 0.5cm 1.2cm, clip, width=\linewidth]{figure/r1.pdf} % 이미지를 1단 전체 너비로 표시
    \vspace{-7mm}
\end{figure}


\begin{figure}[!t] % "t"는 페이지 상단에 이미지를 배치하려는 옵션입니다.
    \centering
    \includegraphics[trim=0.5cm 5.2cm 0.5cm 1.2cm, clip, width=\linewidth]{figure/r2.pdf} % 이미지를 1단 전체 너비로 표시
    \caption{Trajectory overlaid on satellite maps for each sequence with colors. Red indicates the start, while blue designates the end.}
    \label{fig:road}
    \vspace{-6mm}
\end{figure}

 % Each sequence features distinct environmental characteristics, and multiple revisits to the same locations ensure sufficient queries for place recognition evaluation.

\subsection{Data Description and Format}

The file structure of the HeRCULES dataset is delineated in \figref{fig:file}. The acquisition time of
all measurements are stored in \texttt{datastamp.csv}. The FMCW LiDAR and 4D radar data are provided in \texttt{time.bin}, while the camera data is in \texttt{time.png}. For spinning radar data, we support software that converts raw polar images into Oxford-style \cite{barnes2020oxford} and Cartesian images. \ac{IMU}, \ac{GPS}, and \ac{INS} data are provided in \texttt{.csv}, and calibration information between sensors is available in \texttt{.yaml} and \texttt{.txt} format. The data types for each sensor are detailed in \tabref{tab:sensors}.

\subsection{Individual Ground Truth}
Before logging each sequence, we ensure that the GNSS solution is fixed and the INS solution has converged. We use \ac{PTP} to synchronize timestamps in \ac{UTC} across all sensors. However, spatiotemporal discrepancies arise due to differences in sensor mounting positions and data acquisition times.
To address this, we provide ground truth poses for each sensor to support place recognition research, handling spatial differences using extrinsic calibration results and temporal differences with B-Spline interpolation \cite{mueggler2018continuous}.
The ground truth is shown in \figref{fig:gtpose}, highlighting the importance of independently deriving ground truth pose for each sensor.

% We provide ground truth pose for each sensor to support place recognition research, addressing spatial differences using extrinsic calibration results and temporal differences with B-Spline interpolation \cite{mueggler2018continuous}.
% The ground truth is shown in \figref{fig:gtpose}, highlighting the importance of independently deriving ground truth pose for each sensor due to spatiotemporal variations rather than relying on a unified ground truth.



% \subsection{Datasets and Evaluation Metric}
% red section, we use the 0.5\unit{m} and 5{\textdegree} for handcart sequences, and 2.0\unit{m} and 10{\textdegree} for car sequences.

%  (RPE$_r$) and translational (RPE$_t$) parts each.


% \subsection{NTU4DRadLM dataset}
% \label{subsec:ntu}

% %TABLE
% % % Please add the following required packages to your document preamble:
% % \usepackage{multirow}
% % \usepackage{graphicx}
% % \usepackage[table,xcdraw]{xcolor}
% \begin{table*}[]
% \centering
% \caption{SENSOR SPECIFICATIONS}
% \label{tab:sensors}
% \resizebox{\textwidth}{!}{%
% \begin{tabular}{c|ccccccccc} 
% \toprule
%  &
%  &
%  &
%   \multicolumn{3}{c}{Resolution}
%  &
%  &
%   \multicolumn{3}{c}{FOV}
%   \\ \cline{4-6} \cline{8-10} \rule{0pt}{2.5ex}
%   \multirow{-2}{*}{Sensor} &
%   \multirow{-2}{*}{Type} & 
%   \multirow{-2}{*}{Data type} &
%   Range &
%   Azimuth &
%   Elevation &
%   &
%   Range &
%   Azimuth &
%   Elevation
% \\ 
% \midrule

%   {\begin{tabular}[c]{@{}c@{}}4D\\Radar \end{tabular}} &
% Continental ARS548 & 
%   {\begin{tabular}[c]{@{}c@{}}x, y, z, velocity, RCS,\\range, azimuth, elavation \end{tabular}} &
% 0.22 m&
%   {\begin{tabular}[c]{@{}c@{}}1.2°@0. . . ±15°\\1.68°@ ±45°\end{tabular}}&
% 2.3°&
% &
% 300 m&
% ±60°&
%   {\begin{tabular}[c]{@{}c@{}}±4°@300m\\±14°@$<$100m\end{tabular}}
% \\ \rule{0pt}{4ex}
%   {\begin{tabular}[c]{@{}c@{}}Spinning\\Radar \end{tabular}} &
% Navtech RAS6 & 
% Polar image, Cartesian image&
% 0.044 m&
% 0.9°&
% - &
%  &
% 330 m &
% 360°&
% -
% \\ \rule{0pt}{4ex}
%   {\begin{tabular}[c]{@{}c@{}}FMCW\\LiDAR \end{tabular}} &
% Aeva Aeries II &
%   {\begin{tabular}[c]{@{}c@{}}x, y, z, reflectivity, intensity,\\velocity, line-index, time-offset \end{tabular}} &

% 0.075m@1$\sigma$ &
% 0.025°&
% 0.025°&
% &
% 150 m&
% 19.2°&
% 30° 
% \\ \rule{0pt}{4ex}
%   Camera &

%   {\begin{tabular}[c]{@{}c@{}}FLIR Blackfly S \\BFS-U3-16S2C-CS USB3\end{tabular}} &
% 8-bit Bayer pattern png format&
% - &
% 1440px &
% 1080px &
% &
% - &
% 60°&
% 45°
% \\ \rule{0pt}{4ex}
% IMU & Xsens MTi-300 &
% % High performing Attitude and Heading Reference System. 9-axis 
%   {\begin{tabular}[c]{@{}c@{}}qx, qy, qz, qw, eul$_x$, eul$_y$, eul$_z$, gyr$_x$, gyr$_y$, \\gyr$_z$, acc$_x$, acc$_y$, acc$_z$, mag$_x$, mag$_y$, mag$_z$\end{tabular}} &
% - &
% - &
% - &
% &
% - &
% - &
% -
% \\ \rule{0pt}{4ex}
% RTK-GPS & 
%   {\begin{tabular}[c]{@{}c@{}}Hexagon NovAtel\\ SPAN-CPT7  \end{tabular}} &
%   {\begin{tabular}[c]{@{}c@{}}latitude, longitude, height, velocity$_{north}$,\\velocity$_{east}$, velocity$_{up}$, roll, pitch, azimuth, status \end{tabular}} &
% - &
% - &
% - &
% &
% - &
% - &
% -
% \\ 
% \bottomrule
% \end{tabular}%
% }
% \vspace{-3mm}
% \end{table*}

% Please add the following required packages to your document preamble:
% \usepackage{multirow}
% \usepackage{graphicx}
% \usepackage[table,xcdraw]{xcolor}
% \begin{table*}[]
% \centering
% \caption{SENSOR SPECIFICATIONS}
% \label{tab:sensors}
% \resizebox{\textwidth}{!}{%
% \begin{tabular}{c|cccccccccc} 
% \toprule
%  &
%  &
%  &
%   \multicolumn{3}{c}{Resolution}
%  &
%  &
%   \multicolumn{3}{c}{FOV} &
%   \\ \cline{4-6} \cline{8-10} \rule{0pt}{2.3ex}
%   \multirow{-2}{*}{Sensor} &
%   \multirow{-2}{*}{Type} & 
%   \multirow{-2}{*}{Data type} &
%   Range &
%   Azimuth &
%   Elevation &
%   &
%   Range &
%   Azimuth &
%   Elevation &
%   \multirow{-2}{*}{Frequency}
% \\ 
% \midrule

%   {\begin{tabular}[c]{@{}c@{}}4D\\Radar \end{tabular}} &
% Continental ARS548 & 
%   {\begin{tabular}[c]{@{}c@{}}x, y, z, velocity, RCS,\\range, azimuth, elavation \end{tabular}} &
% \unit{0.22}{m}&
%   {\begin{tabular}[c]{@{}c@{}}1.2°@0. . . ±15°\\1.68°@ ±45°\end{tabular}}&
% 2.3°&
% &
% \unit{300}{m}&
% ±60°&
%   {\begin{tabular}[c]{@{}c@{}}±4°@\unit{300}{m}\\±14°@$<$\unit{100}{m}\end{tabular}} &
% 20 Hz
% \\ \rule{0pt}{4ex}
%   {\begin{tabular}[c]{@{}c@{}}Spinning\\Radar \end{tabular}} &
% Navtech RAS6 & 
% Polar image, Cartesian image&
% \unit{0.044}{m}&
% 0.9°&
% - &
%  &
% \unit{330}{m} &
% 360°&
% - &
% 4 Hz
% \\ \rule{0pt}{4ex}
%   {\begin{tabular}[c]{@{}c@{}}FMCW\\LiDAR \end{tabular}} &
% Aeva Aeries II &
%   {\begin{tabular}[c]{@{}c@{}}x, y, z, reflectivity, intensity,\\velocity, line-index, time-offset \end{tabular}} &

% \unit{0.02}{m}@1$\sigma$ &
% 0.025°&
% 0.025°&
% &
% \unit{150}{m}&
% 120°&
% 30° &
% 10 Hz
% \\ \rule{0pt}{4ex}
%   Camera &

%   {\begin{tabular}[c]{@{}c@{}}FLIR Blackfly S \\BFS-U3-16S2C-CS USB3\end{tabular}} &
% 8-bit Bayer pattern png format&
% - &
% \unit{1440}{px} &
% \unit{1080}{px} &
% &
% - &
% 60°&
% 45° &
% 15 Hz
% \\ \rule{0pt}{4ex}
% IMU & Xsens MTi-300 &
% % High performing Attitude and Heading Reference System. 9-axis 
%   {\begin{tabular}[c]{@{}c@{}}q$_x$, q$_y$, q$_z$, q$_w$, eul$_x$, eul$_y$, eul$_z$, gyr$_x$, gyr$_y$, \\gyr$_z$, acc$_x$, acc$_y$, acc$_z$, mag$_x$, mag$_y$, mag$_z$\end{tabular}} &
% - &
% - &
% - &
% &
% - &
% - &
% - &
% 100 Hz
% \\ \rule{0pt}{4ex}
% RTK-GPS & 
%   {\begin{tabular}[c]{@{}c@{}}Hexagon NovAtel\\ SPAN-CPT7  \end{tabular}} &
%   {\begin{tabular}[c]{@{}c@{}}latitude, longitude, height, velocity$_{north}$,\\velocity$_{east}$, velocity$_{up}$, roll, pitch, azimuth, status \end{tabular}} &
% - &
% - &
% - &
% &
% - &
% - &
% - &
% 50 Hz
% \\ 
% \bottomrule
% \end{tabular}%
% }
% \vspace{-3mm}
% \end{table*}

\begin{table*}[]
\centering
\caption{SENSOR SPECIFICATIONS}
\label{tab:sensors}
\resizebox{\textwidth}{!}{%
\begin{tabular}{c|cccccccccc} 
\toprule
  \multirow{2}{*}{Sensor} &
  \multirow{2}{*}{Type} & 
  \multirow{2}{*}{Data type} &
  \multicolumn{3}{c}{Resolution}
  &
  &
  \multicolumn{3}{c}{FOV} &
  \multirow{2}{*}{Frequency}
  \\ \cline{4-6} \cline{8-10} \rule{0pt}{2.3ex}
  &
  &
  &
  Range &
  Azimuth &
  Elevation &
  &
  Range &
  Azimuth &
  Elevation &

\\ 
\midrule

  {\begin{tabular}[c]{@{}c@{}}4D\\Radar \end{tabular}} &
Continental ARS548 & 
  {\begin{tabular}[c]{@{}c@{}}x, y, z, velocity, RCS,\\range, azimuth, elavation \end{tabular}} &
\unit{0.22}{m}&
  {\begin{tabular}[c]{@{}c@{}}1.2°@0. . . ±15°\\1.68°@ ±45°\end{tabular}}&
2.3°&
&
\unit{300}{m}&
±60°&
  {\begin{tabular}[c]{@{}c@{}}±4°@\unit{300}{m}\\±14°@$<$\unit{100}{m}\end{tabular}} &
20 Hz
\\ \rule{0pt}{4ex}
  {\begin{tabular}[c]{@{}c@{}}Spinning\\Radar \end{tabular}} &
Navtech RAS6 & 
Polar image, Cartesian image&
\unit{0.044}{m}&
0.9°&
- &
 &
\unit{330}{m} &
360°&
- &
4 Hz
\\ \rule{0pt}{4ex}
  {\begin{tabular}[c]{@{}c@{}}FMCW\\LiDAR \end{tabular}} &
Aeva Aeries II &
  {\begin{tabular}[c]{@{}c@{}}x, y, z, reflectivity, intensity,\\velocity, line-index, time-offset \end{tabular}} &

\unit{0.02}{m}@1$\sigma$ &
0.025°&
0.025°&
&
\unit{150}{m}&
120°&
30° &
10 Hz
\\ \rule{0pt}{4ex}
  Camera &

  {\begin{tabular}[c]{@{}c@{}}FLIR Blackfly S \\BFS-U3-16S2C-CS USB3\end{tabular}} &
8-bit Bayer pattern png format&
- &
\unit{1440}{px} &
\unit{1080}{px} &
&
- &
60°&
45° &
15 Hz
\\ \rule{0pt}{4ex}
IMU & Xsens MTi-300 &
% High performing Attitude and Heading Reference System. 9-axis 
  {\begin{tabular}[c]{@{}c@{}}q$_x$, q$_y$, q$_z$, q$_w$, eul$_x$, eul$_y$, eul$_z$, gyr$_x$, gyr$_y$, \\gyr$_z$, acc$_x$, acc$_y$, acc$_z$, mag$_x$, mag$_y$, mag$_z$\end{tabular}} &
- &
- &
- &
&
- &
- &
- &
100 Hz
\\ \rule{0pt}{4ex}
RTK-GPS & 
  {\begin{tabular}[c]{@{}c@{}}Hexagon NovAtel\\ SPAN-CPT7  \end{tabular}} &
  {\begin{tabular}[c]{@{}c@{}}latitude, longitude, height, velocity$_{north}$,\\velocity$_{east}$, velocity$_{up}$, roll, pitch, azimuth, status \end{tabular}} &
- &
- &
- &
&
- &
- &
- &
50 Hz
\\ 
\bottomrule
\end{tabular}%
}
\vspace{-3mm}
\end{table*}

% %TABLE


% \tabref{table:ntu}. Our proposed maforementioned issues (\figref{fig:NTU}).

% In \texttt{loop3} sequen

% \subsection{MSC-RAD4R dataset}
% \label{subsec:msc}

% In the MSC-RAD4R datase in \tabref{table:msc}. 

% \texttt{URBAN}  
% \texttt{LOOP} sequen
% elocity measurements (\figref{fig:MSC_trajcetory}).

% To thoriments on snowy sequences \texttt{RURAL\_A2} and \texttt{RURAL\_B2} which are characterized by sharp roundabouts and high dRemarkably, 4DRadarSLAM achieved the second-best performance in most sequences even without leveraging \ac{IMU}.



% \subsection{Ablation Studies}


% \subsubsection{Low Elevation Error}

% + \cite{lee2022patchworkpp}. As depicted in \figref{fig:ablaan

% %TABLE
% \section{Discussion}
\subsection{Limitation of rotation prediction task}
\begin{figure}[t!]
    \centering
    \begin{subfigure}{0.1\textwidth}
        \includegraphics[width=\linewidth]{figure/ship_sample.png}
        \caption{"Ship"}
        \label{fig:ship_sample}
    \end{subfigure}
    \vfill
    \begin{subfigure}{0.22\textwidth}
        \includegraphics[width=\linewidth]{figure/all_classes_main_likelihood_ship.pdf}
        \caption{Likelihood of all 10 classes over steps}
        \label{fig:all_classes_main_likelihood_hard_ship}
    \end{subfigure}
    \begin{subfigure}{0.22\textwidth}
        \includegraphics[width=\linewidth]{figure/all_classes_aux_likelihood_ship.pdf}
        \caption{Likelihood of all 4 rotation classes over steps}
        \label{fig:all_class_likelihood_aux_hard_ship}
    \end{subfigure}
    \caption{Likelihood across iteration of a level 5 Gaussian Noise sample in class "Ship"}
    \label{fig:likelihood_ship}
\end{figure}
Figure \ref{fig:likelihood_ship} illustrates the likelihood of different classes across iterations, using a sample (Figure \ref{fig:ship_sample}) from the "Ship" class with a rotation angle of 0 degrees. The sample's features, which include many edge features, allow the model to predict rotation effectively and maintain stability over iterations (Figure \ref{fig:all_class_likelihood_aux_hard_ship}). This stability and accuracy in rotation prediction enable the recurrent model to better estimate the optimal iteration for the main task.

However, predicting the rotation angle becomes more challenging for samples with fewer edge features or isotropic characteristics, such as the sample from the "Cat" class in Figure \ref{fig:cat_sample}. Figure \ref{fig:all_classes_main_likelihood_hard_cat}, \ref{fig:all_class_likelihood_aux_hard_cat} shows that while the likelihood of the "Cat" class continues to increase over iterations and remains significantly higher than other classes, the likelihood of the ground truth rotation (0 degrees) in the self-supervised task is very low . Instead, the model tends to predict the class corresponding to a 270-degree rotation. This behavior negatively impacts the estimation of the optimal iteration for the main task based on the self-supervised task.

We acknowledge this as a limitation of using the rotation prediction task as a self-supervised task.

\subsection{Converge to a fixed point does not ensure to mitigate "overthinking"}

\cite{bansal2022endtoend} highlights the relationship between changes in feature maps across iterations and the issue of "overthinking".
Specifically, they showed that for recurrent models where the norm $\|h_t - h_{t-1}\|$ converges to $0$, it can be assumed that the feature map has converged to a fixed point, and thus the prediction results of iterations beyond this fixed point remain unchanged, effectively addressing "overthinking." 

However, we observe that this assumption is not entirely accurate. Figure \ref{fig:visualize norm} illustrates that the norm $\|h_t - h_{t-1}\|$ of Conv-GRU converges to $0$ after more than $20$ iterations. Nevertheless, Figure \ref{fig:visualize loss} reveals that both the classification loss and self-supervised loss of the model exhibit divergence, corresponding to Conv-GRU encountering "overthinking" on corruption test sets, as shown in Figure \ref{fig:cnn_gru_gn_alpha_0.0}. 

In contrast, Conv-LiGRU demonstrates greater stability. Specifically, not only does the norm $\|h_t - h_{t-1}\|$ converge to $0$ (Figure \ref{fig:visualize norm}), but both the main loss and auxiliary loss also converge smoothly to $0$ (Figure \ref{fig:visualize loss}). Additionally, Figure \ref{fig:cnn_ligru_alpha_0.0} shows that Conv-LiGRU significantly mitigates the "overthinking" phenomenon compared to Conv-GRU. 

In conclusion, we assert that the convergence of $\|h_t - h_{t-1}\|$ alone does not guarantee that the model is free from "overthinking."

\begin{figure}[t!]
    \centering
    \begin{subfigure}{0.1\textwidth}
        \includegraphics[width=\linewidth]{figure/cat_sample.png}
        \caption{"Cat"}
        \label{fig:cat_sample}
    \end{subfigure}
    \vfill
    \begin{subfigure}{0.22\textwidth}
        \includegraphics[width=\linewidth]{figure/all_classes_main_likelihood_cat.pdf}
        \caption{Likelihood of all 10 classes over steps}
        \label{fig:all_classes_main_likelihood_hard_cat}
    \end{subfigure}
    \begin{subfigure}{0.22\textwidth}
        \includegraphics[width=\linewidth]{figure/all_classes_aux_likelihood_cat.pdf}
        \caption{Likelihood of all 4 rotation classes over steps}
        \label{fig:all_class_likelihood_aux_hard_cat}
    \end{subfigure}
    \caption{Likelihood across iteration of a level 5 Gaussian Noise sample in class "Cat"}
    \label{fig:likelihood_cat}
\end{figure}

\begin{figure*}[htbp]
    \centering
    \begin{subfigure}{0.4\textwidth}
        \includegraphics[width=\linewidth]{figure/visualize_norm.pdf}
        \caption{The $\|h_t - h_{t-1}\|$ across iterations}
        \label{fig:visualize norm}
    \end{subfigure}
    \hfill
    \begin{subfigure}{0.55\textwidth}
        \includegraphics[width=\linewidth]{figure/visualize_loss.pdf}
        \caption{The loss value across iterations}
        \label{fig:visualize loss}
    \end{subfigure}
    \caption{\textbf{Left:} The change in norm of feature maps of Conv-GRU and Conv-LiGRU. \textbf{Right:} Loss value across iterations of Conv-GRU and Conv-LiGRU}
    \label{fig:layer_norm}
\end{figure*}
% \begin{algorithm}[H]
% \caption{Incremental Progress Training Algorithm}
% \label{alg:incremental_progress}
% \begin{algorithmic}[1]
% \Require Parameter vector $\theta$, integer $m$, weight $\alpha$
% \For{batch\_idx = 1, 2, \dots}
%     \State Choose $n \sim U\{0, m-1\}$ and $k \sim U\{1, m-n\}$
%     \State Compute $\phi_n$ with $n$ iterations without tracking gradients
%     \State Compute $\hat{y}_{\text{prog}}$ with additional $k$ iterations
%     \State Compute $\hat{y}_m$ with a new forward pass of $m$ iterations
%     \State Compute $\mathcal{L}_{\text{max\_iters}}$ with $\hat{y}_m$
%     \State Compute $\mathcal{L}_{\text{progressive}}$ with $\hat{y}_{\text{prog}}$
%     \State Compute $\mathcal{L} = (1 - \alpha) \cdot \mathcal{L}_{\text{max\_iters}} + \alpha \cdot \mathcal{L}_{\text{progressive}}$
%     \State Compute $\nabla_\theta \mathcal{L}$ and update $\theta$
% \EndFor
% \end{algorithmic}
% \end{algorithm}

% \begin{figure*}[htbp]
%     \centering
%     \begin{subfigure}{0.48\textwidth}
%         \includegraphics[width=\linewidth]{figure/cnn_gru_gn_alpha_0.1.png}
%         \caption{Caption for Graph 2}
%         \label{fig:cnn_gru_gn_alpha_0.1}
%     \end{subfigure}
%     \hfill
%     \begin{subfigure}{0.48\textwidth}
%         \includegraphics[width=\linewidth]{figure/cnn_ligru_alpha_0.1.png}
%         \caption{Caption for Graph 2}
%         \label{fig:cnn_ligru_alpha_0.1}
%     \end{subfigure}
%     \caption{A horizontal arrangement of three graphs.}
%     \label{fig:alpha0.1}
% \end{figure*}


% To mitigate the problem of overthinking in recurrent models, \cite{bansal2022endtoend} proposed progressive loss (\ref{alg:incremental_progress}). We trained Conv-GRU and Conv-LiGRU using \ref{alg:incremental_progress} with $\alpha = 0.1$. 

% Figures \ref{fig:cnn_gru_gn_alpha_0.1} and \ref{fig:cnn_ligru_alpha_0.1} indicate that progressive loss does not limit overthinking in the Conv-GRU and Conv-LiGRU models and also reduces the accuracy stability between iterations. However, we figure out that the Progressive Loss can buff the accuracy of RNNs (Table 3). 
% %TABLE

% iled in \tabref{table:ablation}. The results show diminished \ac{ATE} performance our ground filtering process (\texttt{w/o Filter}). \Cref{fig:ablation_zaxis_nyl,fig:ablation_zaxis_loopa0} illustrates the $z$-error along the timestamp in \texttt{nyl} (NTU4DRadLM) and \texttt{LOOP\_A0} (Ms. ficantly reduces $z$-error. Furthermore, the improved elevation accuracy 




% \subsubsection{Effe}

% To anath discrete (\texttt{w/o CONT}) and continuous integration (\texttt{FULL}). Fo\cite{kubelka2023we}.
% As demonstrated in \tabref{table:ablation}, continuous integration s. 

% \subsubsection{Computation Time}

% %TABLE
% % Please add the following required packages to your document preamble:
% \usepackage{multirow}
\begin{table}[t]
\centering
\caption{QUANTITATIVE ANALYSIS: AUC SCORES}
\begin{adjustbox}{width=1\linewidth}
{
\begin{tabular}{c|cccccccccccc}
\toprule
\multirow{2}{*}{Sequence} &
 &
   \multicolumn{3}{c}{Aeva}
 &
 &
  \multicolumn{3}{c}{Continental}
 &
 &
  \multicolumn{3}{c}{Aeva-Continental}
  \\ \cline{3-5} \cline{7-9} \cline{11-13} \rule{0pt}{2.5ex}
  &
  &
  \unit{10}{m} &
  \unit{15}{m} &
  \unit{20}{m} &
  &
  \unit{10}{m} &
  \unit{15}{m} &
  \unit{20}{m} &
  &
  \unit{10}{m} &
  \unit{15}{m} &
  \unit{20}{m}
\\ 
\midrule
\multirow{1}{*}{\texttt{Sports Complex}}
% & 0  & 0.  & 0.  &   & 0. & 0. & 0. & & 0. & 0. & 0.\\
& & 0.976  & -  & -  &   & 0.809 & - & - & & 0.401 & - & -\\
\midrule
\multirow{1}{*}{\texttt{Library}}
& & 0.971 & 0.975 & 0.988   &  & 0.574 & 0.584 & 0.632 & & 0.276 & 0.296 & 0.331 \\
\bottomrule
\end{tabular}
}
\end{adjustbox}
\label{tab:pr}
\vspace{-2mm}
\end{table}

% %TABLE

% The time consumption analysis results in \texttt{loop2} and \texttt{LOOP\_A0} are presented in \tabref{table:time}. Each sequence is the longest path in the dataset and we utilized Intel i7 CPU@2.50 {\GHz} and 64 {\GB} RAM.
\section{Experiments}
\label{section5}

In this section, we conduct extensive experiments to show that \ourmethod~can significantly speed up the sampling of existing MR Diffusion. To rigorously validate the effectiveness of our method, we follow the settings and checkpoints from \cite{luo2024daclip} and only modify the sampling part. Our experiment is divided into three parts. Section \ref{mainresult} compares the sampling results for different NFE cases. Section \ref{effects} studies the effects of different parameter settings on our algorithm, including network parameterizations and solver types. In Section \ref{analysis}, we visualize the sampling trajectories to show the speedup achieved by \ourmethod~and analyze why noise prediction gets obviously worse when NFE is less than 20.


\subsection{Main results}\label{mainresult}

Following \cite{luo2024daclip}, we conduct experiments with ten different types of image degradation: blurry, hazy, JPEG-compression, low-light, noisy, raindrop, rainy, shadowed, snowy, and inpainting (see Appendix \ref{appd1} for details). We adopt LPIPS \citep{zhang2018lpips} and FID \citep{heusel2017fid} as main metrics for perceptual evaluation, and also report PSNR and SSIM \citep{wang2004ssim} for reference. We compare \ourmethod~with other sampling methods, including posterior sampling \citep{luo2024posterior} and Euler-Maruyama discretization \citep{kloeden1992sde}. We take two tasks as examples and the metrics are shown in Figure \ref{fig:main}. Unless explicitly mentioned, we always use \ourmethod~based on SDE solver, with data prediction and uniform $\lambda$. The complete experimental results can be found in Appendix \ref{appd3}. The results demonstrate that \ourmethod~converges in a few (5 or 10) steps and produces samples with stable quality. Our algorithm significantly reduces the time cost without compromising sampling performance, which is of great practical value for MR Diffusion.


\begin{figure}[!ht]
    \centering
    \begin{minipage}[b]{0.45\textwidth}
        \centering
        \includegraphics[width=1\textwidth, trim=0 20 0 0]{figs/main_result/7_lowlight_fid.pdf}
        \subcaption{FID on \textit{low-light} dataset}
        \label{fig:main(a)}
    \end{minipage}
    \begin{minipage}[b]{0.45\textwidth}
        \centering
        \includegraphics[width=1\textwidth, trim=0 20 0 0]{figs/main_result/7_lowlight_lpips.pdf}
        \subcaption{LPIPS on \textit{low-light} dataset}
        \label{fig:main(b)}
    \end{minipage}
    \begin{minipage}[b]{0.45\textwidth}
        \centering
        \includegraphics[width=1\textwidth, trim=0 20 0 0]{figs/main_result/10_motion_fid.pdf}
        \subcaption{FID on \textit{motion-blurry} dataset}
        \label{fig:main(c)}
    \end{minipage}
    \begin{minipage}[b]{0.45\textwidth}
        \centering
        \includegraphics[width=1\textwidth, trim=0 20 0 0]{figs/main_result/10_motion_lpips.pdf}
        \subcaption{LPIPS on \textit{motion-blurry} dataset}
        \label{fig:main(d)}
    \end{minipage}
    \caption{\textbf{Perceptual evaluations on \textit{low-light} and \textit{motion-blurry} datasets.}}
    \label{fig:main}
\end{figure}

\subsection{Effects of parameter choice}\label{effects}

In Table \ref{tab:ablat_param}, we compare the results of two network parameterizations. The data prediction shows stable performance across different NFEs. The noise prediction performs similarly to data prediction with large NFEs, but its performance deteriorates significantly with smaller NFEs. The detailed analysis can be found in Section \ref{section5.3}. In Table \ref{tab:ablat_solver}, we compare \ourmethod-ODE-d-2 and \ourmethod-SDE-d-2 on the \textit{inpainting} task, which are derived from PF-ODE and reverse-time SDE respectively. SDE-based solver works better with a large NFE, whereas ODE-based solver is more effective with a small NFE. In general, neither solver type is inherently better.


% In Table \ref{tab:hazy}, we study the impact of two step size schedules on the results. On the whole, uniform $\lambda$ performs slightly better than uniform $t$. Our algorithm follows the method of \cite{lu2022dpmsolverplus} to estimate the integral part of the solution, while the analytical part does not affect the error.  Consequently, our algorithm has the same global truncation error, that is $\mathcal{O}\left(h_{max}^{k}\right)$. Note that the initial and final values of $\lambda$ depend on noise schedule and are fixed. Therefore, uniform $\lambda$ scheduling leads to the smallest $h_{max}$ and works better.

\begin{table}[ht]
    \centering
    \begin{minipage}{0.5\textwidth}
    \small
    \renewcommand{\arraystretch}{1}
    \centering
    \caption{Ablation study of network parameterizations on the Rain100H dataset.}
    % \vspace{8pt}
    \resizebox{1\textwidth}{!}{
        \begin{tabular}{cccccc}
			\toprule[1.5pt]
            % \multicolumn{6}{c}{Rainy} \\
            % \cmidrule(lr){1-6}
             NFE & Parameterization      & LPIPS\textdownarrow & FID\textdownarrow &  PSNR\textuparrow & SSIM\textuparrow  \\
            \midrule[1pt]
            \multirow{2}{*}{50}
             & Noise Prediction & \textbf{0.0606}     & \textbf{27.28}   & \textbf{28.89}     & \textbf{0.8615}    \\
             & Data Prediction & 0.0620     & 27.65   & 28.85     & 0.8602    \\
            \cmidrule(lr){1-6}
            \multirow{2}{*}{20}
              & Noise Prediction & 0.1429     & 47.31   & 27.68     & 0.7954    \\
              & Data Prediction & \textbf{0.0635}     & \textbf{27.79}   & \textbf{28.60}     & \textbf{0.8559}    \\
            \cmidrule(lr){1-6}
            \multirow{2}{*}{10}
              & Noise Prediction & 1.376     & 402.3   & 6.623     & 0.0114    \\
              & Data Prediction & \textbf{0.0678}     & \textbf{29.54}   & \textbf{28.09}     & \textbf{0.8483}    \\
            \cmidrule(lr){1-6}
            \multirow{2}{*}{5}
              & Noise Prediction & 1.416     & 447.0   & 5.755     & 0.0051    \\
              & Data Prediction & \textbf{0.0637}     & \textbf{26.92}   & \textbf{28.82}     & \textbf{0.8685}    \\       
            \bottomrule[1.5pt]
        \end{tabular}}
        \label{tab:ablat_param}
    \end{minipage}
    \hspace{0.01\textwidth}
    \begin{minipage}{0.46\textwidth}
    \small
    \renewcommand{\arraystretch}{1}
    \centering
    \caption{Ablation study of solver types on the CelebA-HQ dataset.}
    % \vspace{8pt}
        \resizebox{1\textwidth}{!}{
        \begin{tabular}{cccccc}
			\toprule[1.5pt]
            % \multicolumn{6}{c}{Raindrop} \\     
            % \cmidrule(lr){1-6}
             NFE & Solver Type     & LPIPS\textdownarrow & FID\textdownarrow &  PSNR\textuparrow & SSIM\textuparrow  \\
            \midrule[1pt]
            \multirow{2}{*}{50}
             & ODE & 0.0499     & 22.91   & 28.49     & 0.8921    \\
             & SDE & \textbf{0.0402}     & \textbf{19.09}   & \textbf{29.15}     & \textbf{0.9046}    \\
            \cmidrule(lr){1-6}
            \multirow{2}{*}{20}
              & ODE & 0.0475    & 21.35   & 28.51     & 0.8940    \\
              & SDE & \textbf{0.0408}     & \textbf{19.13}   & \textbf{28.98}    & \textbf{0.9032}    \\
            \cmidrule(lr){1-6}
            \multirow{2}{*}{10}
              & ODE & \textbf{0.0417}    & 19.44   & \textbf{28.94}     & \textbf{0.9048}    \\
              & SDE & 0.0437     & \textbf{19.29}   & 28.48     & 0.8996    \\
            \cmidrule(lr){1-6}
            \multirow{2}{*}{5}
              & ODE & \textbf{0.0526}     & 27.44   & \textbf{31.02}     & \textbf{0.9335}    \\
              & SDE & 0.0529    & \textbf{24.02}   & 28.35     & 0.8930    \\
            \bottomrule[1.5pt]
        \end{tabular}}
        \label{tab:ablat_solver}
    \end{minipage}
\end{table}


% \renewcommand{\arraystretch}{1}
%     \centering
%     \caption{Ablation study of step size schedule on the RESIDE-6k dataset.}
%     % \vspace{8pt}
%         \resizebox{1\textwidth}{!}{
%         \begin{tabular}{cccccc}
% 			\toprule[1.5pt]
%             % \multicolumn{6}{c}{Raindrop} \\     
%             % \cmidrule(lr){1-6}
%              NFE & Schedule      & LPIPS\textdownarrow & FID\textdownarrow &  PSNR\textuparrow & SSIM\textuparrow  \\
%             \midrule[1pt]
%             \multirow{2}{*}{50}
%              & uniform $t$ & 0.0271     & 5.539   & 30.00     & 0.9351    \\
%              & uniform $\lambda$ & \textbf{0.0233}     & \textbf{4.993}   & \textbf{30.19}     & \textbf{0.9427}    \\
%             \cmidrule(lr){1-6}
%             \multirow{2}{*}{20}
%               & uniform $t$ & 0.0313     & 6.000   & 29.73     & 0.9270    \\
%               & uniform $\lambda$ & \textbf{0.0240}     & \textbf{5.077}   & \textbf{30.06}    & \textbf{0.9409}    \\
%             \cmidrule(lr){1-6}
%             \multirow{2}{*}{10}
%               & uniform $t$ & 0.0309     & 6.094   & 29.42     & 0.9274    \\
%               & uniform $\lambda$ & \textbf{0.0246}     & \textbf{5.228}   & \textbf{29.65}     & \textbf{0.9372}    \\
%             \cmidrule(lr){1-6}
%             \multirow{2}{*}{5}
%               & uniform $t$ & 0.0256     & 5.477   & \textbf{29.91}     & 0.9342    \\
%               & uniform $\lambda$ & \textbf{0.0228}     & \textbf{5.174}   & 29.65     & \textbf{0.9416}    \\
%             \bottomrule[1.5pt]
%         \end{tabular}}
%         \label{tab:ablat_schedule}



\subsection{Analysis}\label{analysis}
\label{section5.3}

\begin{figure}[ht!]
    \centering
    \begin{minipage}[t]{0.6\linewidth}
        \centering
        \includegraphics[width=\linewidth, trim=0 20 10 0]{figs/trajectory_a.pdf} %trim左下右上
        \subcaption{Sampling results.}
        \label{fig:traj(a)}
    \end{minipage}
    \begin{minipage}[t]{0.35\linewidth}
        \centering
        \includegraphics[width=\linewidth, trim=0 0 0 0]{figs/trajectory_b.pdf} %trim左下右上
        \subcaption{Trajectory.}
        \label{fig:traj(b)}
    \end{minipage}
    \caption{\textbf{Sampling trajectories.} In (a), we compare our method (with order 1 and order 2) and previous sampling methods (i.e., posterior sampling and Euler discretization) on a motion blurry image. The numbers in parentheses indicate the NFE. In (b), we illustrate trajectories of each sampling method. Previous methods need to take many unnecessary paths to converge. With few NFEs, they fail to reach the ground truth (i.e., the location of $\boldsymbol{x}_0$). Our methods follow a more direct trajectory.}
    \label{fig:traj}
\end{figure}

\textbf{Sampling trajectory.}~ Inspired by the design idea of NCSN \citep{song2019ncsn}, we provide a new perspective of diffusion sampling process. \cite{song2019ncsn} consider each data point (e.g., an image) as a point in high-dimensional space. During the diffusion process, noise is added to each point $\boldsymbol{x}_0$, causing it to spread throughout the space, while the score function (a neural network) \textit{remembers} the direction towards $\boldsymbol{x}_0$. In the sampling process, we start from a random point by sampling a Gaussian distribution and follow the guidance of the reverse-time SDE (or PF-ODE) and the score function to locate $\boldsymbol{x}_0$. By connecting each intermediate state $\boldsymbol{x}_t$, we obtain a sampling trajectory. However, this trajectory exists in a high-dimensional space, making it difficult to visualize. Therefore, we use Principal Component Analysis (PCA) to reduce $\boldsymbol{x}_t$ to two dimensions, obtaining the projection of the sampling trajectory in 2D space. As shown in Figure \ref{fig:traj}, we present an example. Previous sampling methods \citep{luo2024posterior} often require a long path to find $\boldsymbol{x}_0$, and reducing NFE can lead to cumulative errors, making it impossible to locate $\boldsymbol{x}_0$. In contrast, our algorithm produces more direct trajectories, allowing us to find $\boldsymbol{x}_0$ with fewer NFEs.

\begin{figure*}[ht]
    \centering
    \begin{minipage}[t]{0.45\linewidth}
        \centering
        \includegraphics[width=\linewidth, trim=0 0 0 0]{figs/convergence_a.pdf} %trim左下右上
        \subcaption{Sampling results.}
        \label{fig:convergence(a)}
    \end{minipage}
    \begin{minipage}[t]{0.43\linewidth}
        \centering
        \includegraphics[width=\linewidth, trim=0 20 0 0]{figs/convergence_b.pdf} %trim左下右上
        \subcaption{Ratio of convergence.}
        \label{fig:convergence(b)}
    \end{minipage}
    \caption{\textbf{Convergence of noise prediction and data prediction.} In (a), we choose a low-light image for example. The numbers in parentheses indicate the NFE. In (b), we illustrate the ratio of components of neural network output that satisfy the Taylor expansion convergence requirement.}
    \label{fig:converge}
\end{figure*}

\textbf{Numerical stability of parameterizations.}~ From Table 1, we observe poor sampling results for noise prediction in the case of few NFEs. The reason may be that the neural network parameterized by noise prediction is numerically unstable. Recall that we used Taylor expansion in Eq.(\ref{14}), and the condition for the equality to hold is $|\lambda-\lambda_s|<\boldsymbol{R}(s)$. And the radius of convergence $\boldsymbol{R}(t)$ can be calculated by
\begin{equation}
\frac{1}{\boldsymbol{R}(t)}=\lim_{n\rightarrow\infty}\left|\frac{\boldsymbol{c}_{n+1}(t)}{\boldsymbol{c}_n(t)}\right|,
\end{equation}
where $\boldsymbol{c}_n(t)$ is the coefficient of the $n$-th term in Taylor expansion. We are unable to compute this limit and can only compute the $n=0$ case as an approximation. The output of the neural network can be viewed as a vector, with each component corresponding to a radius of convergence. At each time step, we count the ratio of components that satisfy $\boldsymbol{R}_i(s)>|\lambda-\lambda_s|$ as a criterion for judging the convergence, where $i$ denotes the $i$-th component. As shown in Figure \ref{fig:converge}, the neural network parameterized by data prediction meets the convergence criteria at almost every step. However, the neural network parameterized by noise prediction always has components that cannot converge, which will lead to large errors and failed sampling. Therefore, data prediction has better numerical stability and is a more recommended choice.


\paragraph{Summary}
Our findings provide significant insights into the influence of correctness, explanations, and refinement on evaluation accuracy and user trust in AI-based planners. 
In particular, the findings are three-fold: 
(1) The \textbf{correctness} of the generated plans is the most significant factor that impacts the evaluation accuracy and user trust in the planners. As the PDDL solver is more capable of generating correct plans, it achieves the highest evaluation accuracy and trust. 
(2) The \textbf{explanation} component of the LLM planner improves evaluation accuracy, as LLM+Expl achieves higher accuracy than LLM alone. Despite this improvement, LLM+Expl minimally impacts user trust. However, alternative explanation methods may influence user trust differently from the manually generated explanations used in our approach.
% On the other hand, explanations may help refine the trust of the planner to a more appropriate level by indicating planner shortcomings.
(3) The \textbf{refinement} procedure in the LLM planner does not lead to a significant improvement in evaluation accuracy; however, it exhibits a positive influence on user trust that may indicate an overtrust in some situations.
% This finding is aligned with prior works showing that iterative refinements based on user feedback would increase user trust~\cite{kunkel2019let, sebo2019don}.
Finally, the propensity-to-trust analysis identifies correctness as the primary determinant of user trust, whereas explanations provided limited improvement in scenarios where the planner's accuracy is diminished.

% In conclusion, our results indicate that the planner's correctness is the dominant factor for both evaluation accuracy and user trust. Therefore, selecting high-quality training data and optimizing the training procedure of AI-based planners to improve planning correctness is the top priority. Once the AI planner achieves a similar correctness level to traditional graph-search planners, strengthening its capability to explain and refine plans will further improve user trust compared to traditional planners.

\paragraph{Future Research} Future steps in this research include expanding user studies with larger sample sizes to improve generalizability and including additional planning problems per session for a more comprehensive evaluation. Next, we will explore alternative methods for generating plan explanations beyond manual creation to identify approaches that more effectively enhance user trust. 
Additionally, we will examine user trust by employing multiple LLM-based planners with varying levels of planning accuracy to better understand the interplay between planning correctness and user trust. 
Furthermore, we aim to enable real-time user-planner interaction, allowing users to provide feedback and refine plans collaboratively, thereby fostering a more dynamic and user-centric planning process.


% \newpage
% \newpage

%\section*{ACKNOWLEDGMENT}
\balance
\small
\bibliographystyle{IEEEtranN} %citeauthor
\bibliography{string-short,references}


\end{document}