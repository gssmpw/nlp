
%%%%%%%%%%%%%%%%%%%%%%%%%%%%%%%%%%%%%%%%%%%%%%%%%%%%%%%%%%%%%%%%%%%%%%%%%%%%%%%%
%2345678901234567890123456789012345678901234567890123456789012345678901234567890
%        1         2         3         4         5         6         7         8

\documentclass[letterpaper, 10 pt, conference]{ieeeconf}  % Comment this line out if you need a4paper
% \usepackage{hyperref}
%\documentclass[a4paper, 10pt, conference]{ieeeconf}      % Use this line for a4 paper

\IEEEoverridecommandlockouts                              % This command is only needed if
                                                          % you want to use the \thanks command

\overrideIEEEmargins                                      % Needed to meet printer requirements.
% \usepackage{hyperref}
% this is to use \citeauthor comment (not compatible with cite package)
% natbib hack
% http://newsgroups.derkeiler.com/Archive/Comp/comp.text.tex/2006-02/msg00834.html
\makeatletter
\let\NAT@parse\undefined
\makeatother
\usepackage[numbers,sort&compress]{natbib}

% See the \addtolength command later in the file to balance the column lengths
% on the last page of the document
%\makeatletter
%\let\NAT@parse\undefined
%\makeatother

% The following packages can be found on http:\\www.ctan.org
\usepackage[pdftex]{graphicx}
\usepackage{amsmath}
\usepackage{amssymb}  % assumes amsmath package installed
\usepackage{subfigure}
\usepackage{subcaption}
\usepackage{multirow}
\usepackage{array,booktabs}
\usepackage{diagbox}
\usepackage{balance}
\usepackage{xspace}
\usepackage{algorithm}
\usepackage{algpseudocode}
%\usepackage{kotex} %한글
%\usepackage{enumitem}
%\usepackage{slashbox}
\usepackage{adjustbox}
\usepackage{romannum}
% \captionsetup[subfigure]{labelformat=empty}
\usepackage{float}
\usepackage{placeins}

\usepackage[final]{hyperref}
\hypersetup{
 colorlinks=true,
 linkcolor=blue,
 filecolor=magenta,
 urlcolor=blue,
 citecolor=black
}
\usepackage{cleveref}
\crefname{figure}{Fig.}{Figs.}
\Crefname{figure}{Fig.}{Figs.}
\usepackage[font=small]{caption}
\usepackage{rpm_SIunits}
\usepackage{rpm_acronyms}
\usepackage{rpm_math}
\usepackage{rpm_misc}


% for comments
\usepackage{soul,color}
\usepackage{lipsum}


% To highlight the revised MS
\usepackage{xcolor}
\newcommand{\gr}[1]{{\textcolor{green}{#1}}}
\newcommand{\bl}[1]{{\textcolor{black}{#1}}}
\newcommand{\B}[1]{{\textbf{#1}}}
\newtheorem{theorem}{Theorem}

\DeclareMathOperator*{\argmin}{argmin}
\DeclareMathOperator{\Tr}{Tr}

\usepackage{tikz} % \checkmark
\usepackage{subcaption}
\usepackage{caption}
\usepackage{amssymb}
\usepackage{multirow}
\usepackage{graphicx}
% \usepackage[table,xcdraw]{xcolor}
\title{\LARGE \bf HeRCULES: Heterogeneous Radar Dataset\\
in Complex Urban Environment for Multi-session Radar SLAM
}     
% \title{\LARGE \bf HeRCULEs: Heterogeneous Radar dataset\\for Complex Urban Localization and state Estimation Envitonment Study
% }     

\author{Hanjun Kim${}^{1}$, Minwoo Jung${}^{2}$, Chiyun Noh${}^{2}$, Sangwoo Jung${}^{2}$,\\Hyunho Song${}^{2}$, Wooseong Yang${}^{2}$, Hyesu Jang${}^{2}$ and Ayoung Kim${}^{2*}$
\thanks{$^\dagger$This work was supported by the Robotics and AI (RAI) Institute and Ministry of Trade, Industry \& Energy (MOTIE), Korea (No. 1415187329).}%
\thanks{$^{1}$H. Kim is with the Dept. of Future Automotive Mobility, SNU, Seoul, S. Korea {\tt\small hanjun815@snu.ac.kr}}%
\thanks{$^{2}$M. Jung, C. Noh, S. Jung, H. Song, W. Yang, H. Jang and A. Kim are with the Dept. of Mechanical Engineering, SNU, Seoul, S. Korea {\tt\small [moonshot, gch06208, dan0130, hun1021405, yellowish, dortz, ayoungk]@snu.ac.kr}}
% \thanks{$^{3}$H. Jang is with the Institute of Advanced Machines and Design, SNU, Seoul, S. Korea {\tt\small dortz@snu.ac.kr}}
}
\begin{document}

% \makeatletter
%   \let\@oldmaketitle\@maketitle% Store \@maketitle
%   \renewcommand{\@maketitle}{\@oldmaketitle% Update \@maketitle to insert...
%   \bigskip
%   \centering
%     \includegraphics[trim= 7.4cm 5.7cm 7.5cm 2cm, clip,width=0.8\textwidth]{figure/overview.pdf}
%     \captionof{figure}{
%       Heterogeneous Radar Dataset for Complex Urban Localization with FMCW LiDAR, 4D Radar, and Spinning Radar. The point colors in the FMCW LiDAR and 4D radar represent relative velocities, normalized for each image for visibility. Red indicates objects moving away, while blue indicates objects approaching.
%     }
%     \vspace{-3mm}
%     \label{fig:overview}
%   }%an image
% %trim = 40 70 30 60, clip,

\makeatletter
  \let\@oldmaketitle\@maketitle% Store \@maketitle
  \renewcommand{\@maketitle}{\@oldmaketitle% Update \@maketitle to insert...
  \bigskip
  \centering
    \includegraphics[trim= 0cm 0.9cm 0cm 1.2cm, clip,width=0.93\textwidth]{figure/1.pdf}
    \captionof{figure}{
      Overview of the HeRCULES Dataset. The FMCW LiDAR and 4D radar point colors represent relative velocities, with red indicating objects moving away and blue indicating objects approaching. Colors are normalized for each image to enhance visibility.
    }
    \vspace{-3mm}
    \label{fig:main}
  }%an image
%trim = 40 70 30 60, clip,


\makeatother
%\onecolumn
\maketitle
\thispagestyle{empty}
\pagestyle{empty}
\setcounter{figure}{1}
\begin{abstract}

% Recent works to jointly reconstruct 3D human and object from a single RGB image, are mostly model-based, that fail to capture the fine details of the clothed human body and object surface. In this paper, we introduce ReCHOR, a novel, model-free, first-method to produce realistic clothed human-object reconstructions from a monocular view. This is extremely challenging due to human-object occlusions, diverse interactions and depth ambiguity, as it needs to infer both 3D spatial awareness and high resolution details. Our core idea is based on estimating neural implicit representations for human and object respectively by an attention-based neural implicit model that attends to pixel-aligned features from both the global human-object image for spatial awareness and  the local separate view of human and object images for high quality details. Additionally, the network is conditioned on semantic features from an initial estimated human-object pose prior and a generative diffusion model that inpaints occluded regions, thus enabling the retrieval of details from them.
% We also propose a synthetic dataset with rendered scenes of diverse, inter-occluded 3D human and object scans, to train our network. We evaluate our method on the synthetic and real world BEHAVE dataset. Our experiments show that our method outperforms the SOTA in achieving realistic clothed human-object reconstructions.
Recent approaches to jointly reconstruct 3D humans and objects from a single RGB image represent 3D shapes with template-based or coarse models, which fail to capture details of loose clothing on human bodies. In this paper, we introduce a novel implicit approach for jointly reconstructing realistic 3D clothed humans and objects from a monocular view. For the first time, we model both the human and the object with an implicit representation, allowing to capture more realistic details such as clothing. This task is extremely challenging due to human-object occlusions and the lack of 3D information in 2D images, often leading to poor detail reconstruction and depth ambiguity. To address these problems, we propose a novel attention-based neural implicit model that leverages image pixel alignment from both the input human-object image for a global understanding of the human-object scene and from local separate views of the human and object images to improve realism with, for example, clothing details. Additionally, the network is conditioned on semantic features derived from an estimated human-object pose prior, which provides 3D spatial information about the shared space of humans and objects. To handle human occlusion caused by objects, we use a generative diffusion model that inpaints the occluded regions, recovering otherwise lost details. For training and evaluation, we introduce a synthetic dataset featuring rendered scenes of inter-occluded 3D human scans and diverse objects. Extensive evaluation on both synthetic and real-world datasets demonstrates the superior quality of the proposed human-object reconstructions over competitive methods.
\end{abstract}
\section{Introduction}\label{sec:intro}

In computational finance, Monte Carlo simulations are used extensively to estimate the expected value of financial payoffs based on the solution of stochastic differential equations (SDEs) which model the evolution of stock prices, interest rates, exchange rates and other quantities \cite{glasserman04}.  Monte Carlo methods are very general and flexible, but for high accuracy it requires generating a large number of costly SDE path approximations, which has motivated research into a number of variance reduction or, equivalently, cost reduction techniques. One such method is
Multilevel Monte Carlo (MLMC), which was proposed in \cite{GILES2008} and was adapted for various applications that are summarised in \cite{Giles_overview17} and successfully combined with other methods such as quasi-Monte Carlo methods. The main idea of MLMC is to approximate the payoff using different time stepping resolutions when numerically solving the underlying SDE and to generate an optimal number of samples on each level, such that the overall computational cost is minimised subject to the desired bound on the variance. %, such that the total computational cost is minimised. 
The computational savings come from the fact that most samples are computed on the coarser levels and hence are less expensive while only a few samples from the finest levels are required \cite{GILES2008}.


Among the directions in which the computational cost 
of MLMC methods could further be reduced, an important avenue is the use of lower precision calculations, especially for the first Monte Carlo levels where the targeted accuracy is relatively low. 
 An overview of the research on mixed precision for the standard Monte Carlo (MC) framework is provided in \cite{ChowMixedPrecisionStandardMC} but only a few references study the potential of low precision computation in the MLMC framework \cite{Rounding_error_oliver}. To the best of our knowledge, the only MLMC framework with customised precision in the literature is \cite{brugger2014mixed}, but they use a uniform precision for all operations on each Monte Carlo level instead of optimising 
 the precision of each intermediary variable to reduce as much as possible the cost of path generation.
 
An important motivation for an MLMC framework with variable precision would be performing the low precision computations on reconfigurable hardware devices such as Field Programmable Gate Arrays (FPGAs). FPGAs contain customizable logic blocks and connectors that make it easy to adapt the digital circuit architecture for a specific application, leading to a highly parallel and optimised implementation. Therefore they are successfully exploited in applications that require high speed and have high computational workload, such as signal processing \cite{woods2008fpga}, and real time applications like high frequency trading \cite{HFT1,HFT2}. That is why a number of previous works in hardware architecture design implemented the MLMC algorithm to price financial options using FPGAs as accelerators, which resulted in improved speed and power efficiency compared to full CPU architectures \cite{Schryver2013AMM}. The paper \cite{lindsey2016domain} also proposed 
a Domain Specific Language to automate the configuration of FPGAs for this specific application. However, only \cite{brugger2014mixed} proposed a heuristic to reduce the precision in calculations.

In addition, all aforementioned works considered that the random number generation (RNG) is performed in single or double precision. Yet in most cases an important portion of the workload in the overall MLMC simulation comes from the RNG and in \cite{brugger2014mixed} this limited the total computational savings.
To reduce the cost of MLMC simulations in particular those based on the Geometric Brownian Motion (GBM), \cite{approximateICDF_Oliver, NestedOliver} have proposed to use approximate random numbers that are generated by applying an approximation of the inverse CDF to uniform random numbers. In \cite{NestedOliver}, the authors proposed a way to integrate these lower precision random variables into a \textit{nested} MLMC framework and completed a numerical analysis to bound the resulting error at each MC level by a product of the time step and the error in the random number approximation. The same authors show in \cite{approximateICDF_Oliver} that using approximate random variables reduces the cost of path generation by a factor 7.


In this paper we propose a nested MLMC framework that combines the use of approximate random normal variables and lower precision calculations to reduce the computational cost of MLMC even further than \cite{brugger2014mixed,NestedOliver}. We illustrate the efficiency of our framework in Matlab, after making several assumptions on the cost of operations and size of the errors that we carefully justify. We focus on the case of GBM and use the approximate RNG methods presented in \cite{approximateICDF_Oliver} as well as a new slightly modified method that combines CDF inversion and the central limit theorem. To choose the precision of the variables in the low precision path generation, we introduce a novel method to optimise the bit-widths. This optimisation is performed before the main path generation loop is executed and is based on a linear model of the payoff error  
due to rounding when computing in low precision. The error model relies on algorithmic differentiation in a similar manner to \cite{unifying-bwoptim,bitwidth-AD,ADAPT}. The bit-width optimisation procedure can be performed off-line, so this stage can be excluded from the on-line time complexity of our framework. The user specified desired accuracy is then enforced by calculating on-line the number of samples that need to be generated.

In terms of hardware design, we suggest implementing the low precision path generation on FPGAs and the full-precision ones on a CPU or GPU. 
The FPGA offers enough flexibility to define a separate bit-width for every variable in the low precision path generation, and can be reconfigured periodically to update the bit-widths when the market parameters have changed considerably. 


The paper is organized as follows : \Cref{sec:MLMC} introduces MLMC and nested MLMC to make clear the estimator that is implemented in our framework. Then in \Cref{sec:RNG} we detail the methods that could be used to obtain approximate random normally distributed numbers very cheaply for the low precision path generation. In \Cref{sec:error_model} and \Cref{sec:costModel} we propose an error model and a cost model (resp.) that we then use to formulate the optimisation problem that is solved to obtain the optimal bit-widths of fixed point variables in \Cref{sec:optimisation}. Finally we summarise our results and future directions in \Cref{sec:conclusion}.



\section{Related Work}
\label{sec:related work}
% In this section, we review the existing literature on point cloud denoising and unsupervised image denoising.
%-------------------------------------------------------------------------
\subsection{Point cloud denoising}

    \subsubsection{Traditional methods}
Traditional approaches to point cloud denoising include statistical methods \cite{computingpointset2003,definingpointset2004,wlop2009HH}, filtering techniques\cite{pointsetsurfaces2001,Robustmoving2005, zaman2017density}, and optimization-based methods \cite{l1sparse2010,clop2014PR,digne2017bilateral,multi-projection2018duan,hu2020featuregraph} . These techniques often rely on handcrafted features and heuristics to distinguish signal from noise. For example, statistical methods may use distribution models to identify and remove outliers. Filtering methods, such as mean or median filters, operate under the assumption that noise is statistically different from the signal. Optimization-based methods formulate denoising as an energy minimization problem, where regularization terms constrain the solution to ensure certain smoothness cirterion or adherence to prior knowledge.

%-------------------------------------------------------------------------
    \subsubsection{Supervised learning based methods}
In recent years, several deep learning-based methods \cite{rakotosaona2020PCN,hermosilla2019TotalDenoising,luo2020DMR,luo_score-based_2021} have been proposed for point cloud denoising. NPD \cite{NPD2019} is the first learning-based point cloud denoising method that directly processes noisy data without requiring noise characteristics or neighboring point definitions. This approach combines local and global information by projecting noisy points onto estimated reference planes, effectively removing noise and enhancing robustness against variations in noise intensity and curvature. PointCleanNet\cite{rakotosaona2020PCN} first removes outlier points and then combines them with residual connectivity to predict the inverse displacement \cite{Guerrero2017PCPNetLL}, and iteratively shifts noisy points to remove noise. Pistilli \etal proposed GPDNet \cite{gpdnet2020}, which is a graph convolutional network to improve denoising robustness at high noise levels. Luo \etal also proposed  DMRDenoise \cite{luo2020DMR}, which filter
points by first downsampling the noisy inputs and reconstructing the local subsurface to perform point upsampling. However, this resampling method is difficult to maintain a good local shape. ScoreDenoise \cite{luo_score-based_2021} is proposed to tackle the aforementioned issues by iteratively updating the point position in implicit gradient fields learned by neural networks. For inference, they follows an iterative procedure with a decaying step size, which stabilizes point movement and prevents over-correction, allowing points to converge gradually toward the underlying geometry. The authors of \cite{de_Silva_Edirimuni_2023_CVPR} proposed IterativePFN - an iterative method that use a novel loss function that utilizes an adaptive ground truth target at each iteration to capture the relationship between intermediate filtering results during training. Zheng \etal proposed a end-to-end network for joint normal filtering-based point cloud denoising \cite{10173632}. They introduce an auxiliary normal filtering task to enhance the network's ability to remove noise while preserving geometric features more accurately.

Supervised methods can achieve impressive results, but are limited by the availability and quality of the training data, as they typically require paired noisy and clean point clouds to train the neural network. The absence of clean data in real-world scenario pose a significant challenge on applicability of these algorithms.

%-------------------------------------------------------------------------
    \subsubsection{Unsupervised learning methods}
Unsupervised learning-based methods for point cloud denoising do not require ground-truth clean data. Instead, these methods leverage the inherent structure or distribution of the point cloud to guide the denoising process. Unsupervised methods show promise in scenarios where clean data is absent or hard to obtain.

Hermosilla \etal first introduced Total Denoising (TotalDn) \cite{hermosilla2019TotalDenoising} as an unsupervised learning approach for point cloud denoising, relying solely on noisy data without requiring clean ground truth. TotalDn approximates the underlying surfaces by regressing points from the distribution of unstructured total noise, utilizing a spatial prior term to refine the estimation of geometry. 

In DMRDenoise \cite{luo2020DMR}, an unsupervised version is proposed which utilizes a loss function that identify local neighborhoods using a probabilistic Gaussian mask on the k-nearest neighbors, which selectively retains points likely to represent the underlying surface. By leveraging an Earth Mover's Distance (EMD) assignment, it achieves a one-to-one correspondence between input and predicted points, aligning them to reduce noise within local neighborhoods.
This approach enhances robustness to noise and adapts well to varied surface geometries. However, the probabilistic masking and EMD calculation add computational complexity, which can slow down inference in dense or noisy point clouds. 

ScoreDenoise \cite{luo_score-based_2021} also introduced an unsupervised version that employs ensemble score function and an adaptive neighborhood-covering loss for model training.  
Score-u is probably the most relevant work to our method. However, the defined score in \cite{luo_score-based_2021} is only an displacement-alike version of the score function and there is no explicit formula relating the estimated score to the final denoising result. Also, the iterative process is computationally expensive, and can suffer from fluctuation, leading to perturbed and unstable solution.

Most recently, Noise4Denoise \cite{noise4Wang2024} method is proposed that use an additional doubly-noisy point cloud to learn noise displacement by comparing the two noise levels. This approach effectively leverages synthetic noise for training, allowing the model to estimate residuals without relying on clean data.
However, in practical applications, noise parameters are often unknown and variable, making it challenging to replicate the exact conditions assumed during training. This reliance on predefined noise characteristics can limit the model's applicability to real-world scenarios where noise distributions may differ significantly from synthetic settings. 
%-------------------------------------------------------------------------
\subsection{Unsupervised image denoising}
Recently unsupervised image denoising has made significant progress. Non-Bayesian methods include PURE \cite{luisier2010image}, SURE \cite{SURE2018} \textit{etc.}, which are based on various unbiased risk estimator under certain noise distribution. Other methods explore self-similarity in natural images \cite{xu2015patch, doi:10.1137/23M1614456} or exploits the statistical properties of noise to achieve denoising effect \cite{gravel2004method}.  

Noise2Noise \cite{2018Noise2NoiseLI} is a pioneering method that does not require clean data, but it requires multiple noisy versions of the same image for training. To address this limitation, methods such as Noise2Void \cite{2018Noise2VoidL}, Noise2Self \cite{2019Noise2SelfBD}, \textit{etc.}, have been developed, which use only a single noisy image. These methods are particularly important for practical applications where obtaining clean images or multiple noisy realizations of the same image is difficult or impossible. Neighbor2Neighbor \cite{huang2021neighbor2neighbor} proposed a two-step method with a a random neighbor sub-sampler that generates training  pairs and a denosing network. Kim \etal proposed Noise2Score\cite{kim_noise2score_2021}, a novel Bayesian framework for self-supervised image denoising without clean data. The core of Noise2Score is the usage of Tweedie's formula, which provides an explicit representation of the denoised image through a score function. Combined with score function estimation, Noise2Score can be applied to image denoising with any exponential family noise. Kim \etal also proposed the Noise Distribution Adaptive Self-Supervised Image Denoising method \cite{kim_noise_2022}, which further extends the application of Noise2Score by combining the Tweedie distribution with score matching. This enables adaptive handling of various noise distributions and dynamically adjusts the denoising process by estimating noise parameters. On the other hand, Xie \etal \cite{scoreXie2024} broadened the denoising scope of Noise2Score by allowing it to handle complex noise models, such as multiplicative and structurally correlated noise, demonstrating broad applicability to diverse noise models.

These development of unsupervised image denoising method motivate us to explore similar ideas in 3D point cloud denoising.




\section{System overview}
\label{sec:overview}
\subsection{System Configuration}

% The sensor configuration and coordinates of each sensor are depicted in \figref{fig:car}, and their specifications are provided in \tabref{tab:sensors}. The FMCW LiDAR runs at 10 Hz for sensor operation with relative velocity settings aligned to the 4D radar sensor. The 4D radar operates at 20 Hz, providing radar point cloud data and filtered object point cloud data based on the sensor driver  \cite{fernándezcalatayud2024rosrdi}. The spinning radar operates at 4 Hz, while the Xsens MTi-300 captures inertial data at 100 Hz. The SPAN-CPT7 with a dual VEXXIS GNSS-501 antenna operates at 50 Hz using RTK-GPS and INS. All sensor data are processed on an industrial PC, NUVO-9006LP-NX, with an Intel Core i9, 2 TB SSD, and 64 GB DDR5.

% The sensor configuration and coordinates of each sensor are depicted in \figref{fig:car}, and their specifications are provided in \tabref{tab:sensors}. The Aeva LiDAR operates at 10 Hz, with relative velocity settings synchronized with the 4D radar sensor. The Continental radar operates at 20 Hz, providing radar point cloud data and filtered object point cloud data based on the sensor driver  \cite{fernándezcalatayud2024rosrdi}. The Navtech radar operates at 4 Hz, while the Xsens MTi-300 captures inertial data at 100 Hz. The SPAN-CPT7 with a dual VEXXIS GNSS-501 antenna operates at 50 Hz using RTK-GPS and INS. All sensor data are processed on an industrial PC, NUVO-9006LP-NX, with an Intel Core i9, 2 TB SSD, and 64 GB DDR5.

The sensor configuration and coordinates of each sensor are illustrated in \figref{fig:car}, and their specifications are detailed in \tabref{tab:sensors}. The Aeva \ac{LiDAR} operates with relative velocity settings synchronized to the 4D radar sensor. The Continental radar provides both raw radar point cloud data and filtered object point cloud data via its sensor driver. All sensor data are processed on an industrial PC, the NUVO-9006LP-NX, equipped with an Intel Core i9 processor, 2 TB SSD, and 64 GB DDR5 RAM. The sample data is shown in \figref{fig:overview}.

% \cite{fernándezcalatayud2024rosrdi}

 % A LiFePO$_4$ 102.0Ah battery powers all equipment and sensor drivers run on ROS Noetic with Ubuntu 20.04.

% \begin{figure}[!h]
%     \centering
%     \includegraphics[trim= 6.9cm 9.2cm 7.4cm 6cm, clip,width=0.9\linewidth]{figure/cali.pdf}
%     \caption{(a) Spinning radar points (blue), 4D radar points (green), and LiDAR points (red) (b) Utilizing the line-index channel. (c) LiDAR - spinning radar extrinsic calibration pipeline. }
%     \label{fig:cali}
%     \vspace{-5mm}
% \end{figure}

% \begin{figure}[!t]
%     \centering
%     \includegraphics[trim= 7.5cm 7.8cm 7cm 6.5cm, clip,width=\linewidth]{figure/cali.pdf}
%     \caption{(a) LiDAR - spinning radar extrinsic calibration pipeline. (b) Utilizing the line-index channel. (c) LiDAR points, 4D radar points, and spinning radar points are red, green, and blue. (d)   Left camera - LiDAR. (e) Right camera - LiDAR. (f) Left camera - 4D radar. (g) Right camera - 4D radar.}
%     \label{fig:cali}
%     \vspace{-5mm}
% \end{figure}

\begin{figure}[!t]
    \centering
    \includegraphics[trim= 7.5cm 17cm 7cm 6.5cm, clip,width=\linewidth]{figure/cali.pdf}
    \caption{(a) LiDAR - spinning radar extrinsic calibration pipeline. (b) Utilizing the line-index channel. (c) LiDAR points, 4D radar points, and spinning radar points are red, green, and blue. (d)   Right camera - LiDAR. (e) Left camera - 4D radar.}
    \label{fig:cali}
    \vspace{-5mm}
\end{figure}

\subsection{ Sensor Calibration}
% The calibration results of LiDAR - 4D Radar - Spinning Radar are shown in \figref{fig:cali}(a).

% \subsubsection{Intrinsic Calibration of Camera}
% With the intrinsic calibration tool \cite{zhang2000flexible}, we utilize a checkboard pattern to calculate the internal parameters simultaneously.

\subsubsection{Extrinsic Calibration of \ac{LiDAR} - Spinning Radar}
We employ the method used in the Boreas dataset \cite{burnett2023boreas}. This method determines the rotation $\textbf{R}^L_R$ and the translation $\textbf{t}^L_R$ in the xy plane through correlative scan matching with the Fourier-Mellin transform \cite{checchin2010radar}. Specifically, we convert \ac{LiDAR} point clouds into \ac{LiDAR} polar images to compare with radar polar images to obtain $\textbf{R}^L_R$. Then, we utilize Cartesian images to derive $\textbf{t}^L_R$. To match the \ac{FOV} of the Aeva, we adjust the range and azimuth of the radar images. The calibration pipeline is shown in \figref{fig:cali}(a).


\subsubsection{Extrinsic Calibration of \ac{LiDAR} - 4D Radar - Camera}
We utilize the calibration tool \cite{domhof2021joint} for cameras, \ac{LiDAR}, and radar. This tool jointly calculates relative transformation parameters using a specialized calibration board and reflector. Although we use a solid-state \ac{LiDAR} instead of a spinning one, we utilize the line-index channel to assess laser depth discontinuity, as illustrated in \figref{fig:cali}(b). Unlike \citeauthor{domhof2021joint} \cite{domhof2021joint}, who estimates the reflector position with 2D radar, we can directly obtain the z-value from our 4D radar, resulting in more accurate calibration.


% The translation parameters in the z-direction are obtained from CAD files. 
\subsubsection{Extrinsic Calibration of \ac{LiDAR} - \ac{IMU}}
We initialize the system using the method proposed by \citeauthor{zhu2022robust} \cite{zhu2022robust}. This approach was designed for the Livox \ac{LiDAR} series, so it can be seamlessly applied to our solid-state Aeva \ac{LiDAR} without requiring specific targets.
    % \item Extrinsic Calibration of INS - IMU: 


\subsubsection{Calibration Evaluation}
% The calibration results of \ac{LiDAR} - 4D Radar - Spinning Radar are shown in \figref{fig:cali}(c).
The calibration results are shown in \figref{fig:cali}(c), \figref{fig:cali}(d) and \figref{fig:cali}(e).


% %FIGURE
% \begin{figure}[!t]
%     \centering
%     \includegraphics[trim=0cm 2.5cm 14.5cm 1.65cm, clip, width=1\columnwidth]{figure/pipeline.pdf}
%     \caption{all pipeline }
%     \label{fig:pipeline}
%     \vspace{-5mm}
% \end{figure}

% In thi \figref{fig:pipeline},ggggg

% \subsection{Point-wise Uncertainty Guided Ground Filtering}
% \label{subsec:ground segmentation}
% % ground plane detection and outlier removal




% \begin{equation}
% \begin{array}{rrclcl}
%     \displaystyle \argmin_{[a,b,c,d]} & \multicolumn{3}{l}{\sum_{i=1}^n D_{M_i}^2}\\
%     \textrm{s.t.} & a^2+b^2+c^2=1.
% \end{array}
% \label{eqn:plane_model}
% \end{equation}

% With \eqref{eqn:plane_model}, we c


% \begin{equation}
%     z_i=\frac{V_{xy}v_z - \sqrt{V_{xy}^2{v^d_i}^2 - ((v^d_i)^2-v_z^2)(x^2+y^2){v^d_i}^2}}{(v^d_i)^2-v_z^2}
% \end{equation}
% where $v_x, v_y, v_z$ is the vehicle's ego velocity, and $V_{xy}=v_x x_i+v_y y_i$. After the height 



% \subsection{Cs}




% \begin{equation}
%     \dot\theta_i(t)\sim\mathcal{GP}(0,k_{r_i}(t,t'))
% \end{equation}

% where $k_{r_i}(t,t')$ is the kernel function representing the covariance of the function instances and the index $i$ represents the $x,y,z$ component.
% Then the inference of the rotation vector is conducted as follows:

% \begin{equation}
%     \dot{\theta}_{i\star}(t)=\mathbf{k}_{r_i}(t,\mathbf{t})(\mathbf{K}_{r_i}(\mathbf{t}, \mathbf{t})+\sigma_i^2\mathbf{I})^{-1}\boldsymbol{\rho}_i
% \end{equation}

% \begin{equation}
% \label{eqn:rot_inference} 
%     \theta_{i\star}(t)=\int\mathbf{k}_{r_i}(t,\mathbf{t})(\mathbf{K}_{r_i}(\mathbf{t}, \mathbf{t})+\sigma_i^2\mathbf{I})^{-1}\boldsymbol{\rho}_i \partial t
% \end{equation}

% $\mathbf{t}=[t_1\dots t_{N_i}]$ is the vector of the \ac{IMU} measurement timestamps between consecutive keyframes,
% $\mathbf{k}_{r_i}(t,\mathbf{t})=[k_{r_i}(t,t_1)\dots k_{r_i}(t,t_{N_i})]^\top$ is the kernel, and
% $\newline \mathbf{K}_{r_i}(\mathbf{t},\mathbf{t})=\begin{bmatrix}
% k_{r_i}(t_1,t_1) & \cdots & k_{r_i}(t_1,t_{N_i}) \\
% \vdots & \ddots & \vdots \\
% k_{r_i}(t_{N_i},t_1) & \cdots & k_{r_i}(t_{N_i},t_{N_i})
% \end{bmatrix}$.

% $\sigma_i$ is the standard deviation of the \ac{IMU} gyroscope measurements, and $\boldsymbol{\rho}_i$ is the vector of $\hat{\dot{\theta}}_i(\mathbf{t})$, which can be achieved with the following optimization problem:

% $\boldsymbol{e}_n^{meas} = J_r(\boldsymbol{\theta}_\star(t_n))\dot{\boldsymbol{\theta}}_\star(t_n)-\Tilde{\boldsymbol{\omega}}(t_n)$ is about the measurement constraint, and $\Sigma_{\boldsymbol{\omega}}$ is the \ac{IMU} measurement covariance matrix at timestamp $t_n$. $\boldsymbol{e}_n^{gp}=\dot{\boldsymbol{\theta}}_{\star}(t_n)-\hat{\dot{\boldsymbol{\theta}}}(t_n)$ 

% $\boldsymbol{\mu}(t)$ as following:
% \begin{equation}
%     {v}_i(t)\sim\mathcal{GP}(\mu_i(t),k_{v_i}(t,t')).
% \end{equation}


% \begin{equation}
% \label{eqn:vel_optimize}
%     \displaystyle \argmin_{[\boldsymbol{\zeta}_x, \boldsymbol{\zeta}_y, \boldsymbol{\zeta}_z]}\sum_{n=1}^{N_r} \left( \norm{\boldsymbol{e}_n^{meas}}_{\Sigma_{\mathbf{v}_n}}^2+\norm{\boldsymbol{e}_n^{gp}}_{\Sigma_{gp}}^2 \right).
% \end{equation}


% In \eqref{eqn:vel_optimize}, $\boldsymbol{e}_n^{meas} = \Delta{\mathbf{R}^\top}\mathbf{v}_\star(t_n)-\Tilde{\mathbf{v}}(t_n)$, and $\boldsymbol{e}_n^{gp}=\mathbf{v}_{\star}(t_n)-\hat{\mathbf{v}}(t_n)$. $\Sigma_{\mathbf{v}_n}$ is the velocity measurement covariance 
% chronization for integration between the \ac{IMU} gyroscope and radar dat
% crement can be inferenced by \eqref{eqn:trans_inference}.

% \subsection{Cluster-based weighted ICP}
% \label{subsec:scan matching}


% Therefore, the transformation $\textbf{T}\in SE(3)$ betwee



% \subsection{Pose Graph Optimization}
% incremental motion from \cref{subsec:velocity integration} and \ref{subsec:scan matching} is used for $SE(3)$ edges in the pose graph, and the covariance of the edge is computed in each estimation step. The $g2o$ \cite{kummerle2011g} 

%TABLE
% % % Please add the following required packages to your document preamble:
% % % \usepackage{multirow}
% % % \usepackage{graphicx}
% % % \usepackage[table,xcdraw]{xcolor}
% % % Beamer presentation requires \usepackage{colortbl} instead of \usepackage[table,xcdraw]{xcolor}
% % \begin{table*}[]
% % \caption{OVERVIEW OF SEQUENCES}
% % \label{tab:sequence}
% % \resizebox{\textwidth}{!}{
% % \begin{tabular}{c|c|cccccc} \toprule
% %   Sequence &
% %   Time & 
% %   Weather &
% %   Length &
% %   Loop &
% %   Target &
% %   Description
% %   \\ \midrule
% % \multirow{2}{*}{Mountain} &
% %   \multicolumn{1}{l|}{Day} &
% %   Clear &
% %   4km &  
% %   2 times &
% %   odometry, &
% % \multirow{2}{*}{
% %   poor road surface condition, large altitude variation}
% %   \\
% % \multicolumn{1}{l|}{} &
% %   \multicolumn{1}{l|}{Night} &
% %   Cloud &
% %   4km & 
% %   2 times&
% %   online place recognition \\
% %   \midrule
% % \multirow{2}{*}{Library} &
% %   \multicolumn{1}{l|}{Day} &
% %   Clear &
% %   1.6km &  
% %   2 times &
% %   odometry,&
% % \multirow{2}{*}{
% %   {\begin{tabular}[c]{@{}c@{}} a long narrow one-way alley, downhill and uphill at sharp curves\end{tabular}}
% %   } 
% %   \\
% % \multicolumn{1}{l|}{} &
% %   \multicolumn{1}{l|}{Night} &
% %   Cloud &
% %   1.6km & 
% %   2 times &
% %   online place recognition\\ \midrule
% % \multirow{2}{*}{Sport complex} &
% %   \multicolumn{1}{l|}{Day} &
% %   Clear &
% %   1.4km &  
% %   2 times &
% %   odometry,&
% % \multirow{2}{*}{
% %  consists of a flat section, a gentle uphill section, and a sharp downhill section}  
% %   \\
% % \multicolumn{1}{l|}{} &
% %   \multicolumn{1}{l|}{Night} &
% %   Cloud &
% %   0.7km &  
% %   1 time &
% %   online place recognition\\ \midrule
% % \multirow{3}{*}{Parking lots} &
% %   \multicolumn{1}{l|}{Day} &
% %   Clear &
% %   0.4km &  
% % \multirow{3}{*}{
% %   inter-session} &
% % \multirow{2}{*}{
% %   \\odometry,} &
% % \multirow{3}{*}{
% %   a short driving distance
% % frequent left turn} 
% %   \\
% % \multicolumn{1}{l|}{} &
% %   \multicolumn{1}{l|}{Day} &
% %   Clear &
% %   0.4km & 
% %   &
% %   \multirow{2}{*}{
% %   \\global localization} \\
% % \multicolumn{1}{l|}{} &
% %   \multicolumn{1}{l|}{Night} &
% %   Cloud &
% %   0.5km &  
% %   \\ \midrule
% % \multirow{3}{*}{River island} &
% %   \multicolumn{1}{l|}{Day} &
% %   Cloud &
% %   5.8km &  
% % \multirow{3}{*}{
% %   inter-session} &
% %   \multirow{2}{*}{
% %   \\odometry,} &
% % \multirow{3}{*}{
% %   poor road surface condition, large altitude variation} \\
% % \multicolumn{1}{l|}{} &  
% %   \multicolumn{1}{l|}{Dusk} &
% %   Cloud &
% %   8km &  
% %   &
% %     \multirow{2}{*}{
% %   \\global localization}
% %   \\
% % \multicolumn{1}{l|}{} &  
% %   \multicolumn{1}{l|}{Day} &
% %   Clear &
% %   4km &  
% %  \\ \midrule
% % \multirow{2}{*}{Bridge} &
% %   \multicolumn{1}{l|}{Day} &
% %   Rain &
% %   4.9km &  
% %   1 times &
% %   odometry, &
% % \multirow{2}{*}{
% %   a wide variety of dynamic objects}
% %   \\
% % \multicolumn{1}{l|}{} &
% %   \multicolumn{1}{l|}{Night} &
% %   Cloud &
% %   4.9km &  
% %   1 times &
% %   online place recognition
% %   \\ \midrule

% % Street &
% %   \multicolumn{1}{l|}{Day} &
% %   Rain &
% %   1km &  
% %   1 times &
% %   odometry &
% %   heavy traffic Low driving speed and frequent stops
% %   \\ \midrule
% % \multirow{2}{*}{Stream} &
% %   \multicolumn{1}{l|}{Day} &
% %   Clear &
% %   4.2km &  
% %   2 times  &
% %   odometry,&
% % \multirow{2}{*}{
% %   a wide variety of dynamic objects}
% %   \\
% % \multicolumn{-10}{c|}{} &
% %   \multicolumn{1}{l|}{Night} &
% %   Cloud &
% %   5.5km & 
% %   2 times &
% %   online place recognition 
% %  \\ \bottomrule
% % \end{tabular}
% % }
% % \end{table*}



% % % Please add the following required packages to your document preamble:
% % % \usepackage{multirow}
% % % \usepackage{graphicx}
% % % \usepackage[table,xcdraw]{xcolor}
% % % Beamer presentation requires \usepackage{colortbl} instead of \usepackage[table,xcdraw]{xcolor}
% % \begin{table*}[]
% % \caption{sequence}
% % \label{tab:my-table}
% % \resizebox{\textwidth}{!}{%
% % \begin{tabular}{c|cccccc} \toprule
% %    &
% %    &
% %   \multicolumn{2}{c}{Radar}\\ \cline{3-4} \rule{0pt}{2.5ex}
% %   \multirow{-2}{*}{Sequence} & 
% %   \multirow{-2}{*}{Length } &
% %   \multicolumn{1}{l}{4D Radar} &
% %   \multicolumn{1}{l}{Scanning Radar} &
% %   \multirow{-2}{*}{Time} &
% %   \multirow{-2}{*}{Condition} &
% %   \multirow{-2}{*}{Scenarios} \\ \midrule
% % \multicolumn{1}{l|}{Mountain} &
% %   - &
% %   - &
% %   d &  
% %   \\
% % \multicolumn{1}{l|}{Library} &

% %   - \\
% % \multicolumn{1}{l|}{Sports complex} &
% %   {\color[HTML]{000000} 0.499} &
% %   {\color[HTML]{000000} 0.785} &


% %   - \\ 
% % \multicolumn{1}{l|}{Parking lots} &
% %   {\color[HTML]{000000} 0.499} &
% %   {\color[HTML]{000000} 0.785} &
% %   - \\ \midrule
% % \multicolumn{1}{l|}{River island} &
% %   - &
% %   - \\
% % \multicolumn{1}{l|}{Bridge} &
% %   - \\

% % \multicolumn{1}{l|}{Street} &
% %   - &
% %   - \\
% % \multicolumn{1}{l|}{Stream} &
% %   {\color[HTML]{FF0000} \textbf{0.806}} &
% %   {\color[HTML]{FF0000} \textbf{0.885}} &
% %   {\color[HTML]{FF0000} \textbf{0.903}} \\ \bottomrule
% % \end{tabular}%
% % }
% % \end{table*}

% % Please add the following required packages to your document preamble:
% % \usepackage{multirow}
% % \usepackage{graphicx}
% % \usepackage[table,xcdraw]{xcolor}
% % Beamer presentation requires \usepackage{colortbl} instead of \usepackage[table,xcdraw]{xcolor}

% \begin{table*}[]
% \caption{OVERVIEW OF SEQUENCES}
% \label{tab:sequence}
% \resizebox{\textwidth}{!}{
% \begin{tabular}{c|c|cccccc} 
% \toprule
% Sequence & Time & Weather & Length & Loop & Target & Description \\ 
% \midrule

% \multirow{2}{*}{Mountain} & Day & Clear & 4km & 2 times & odometry, & \multirow{2}{*}{poor road surface condition, large altitude variation} \\
%  & Night & Cloud & 4km & 2 times & online place recognition & \\ 
% \midrule

% \multirow{2}{*}{Library} & Day & Clear & 1.6km & 2 times & odometry, &  \multirow{2}{*}{long narrow one-way alley, downhill and uphill at sharp curves}\\
%  & Night & Cloud & 1.6km & 2 times & online place recognition & \\ 
% \midrule

% \multirow{2}{*}{Sport complex} & Day & Clear & 1.4km & 2 times & odometry, & \multirow{2}{*}{flat section, gentle uphill section, sharp downhill section} \\
%  & Night & Cloud & 0.7km & 1 time & online place recognition & \\ 
% \midrule

% \multirow{3}{*}{Parking lots} & Day & Clear & 0.4km & &  \multirow{3}{*}{\begin{tabular}[c]{@{}c@{}}odometry,\\global localization \end{tabular}} &  \\
%  & Day & Clear & 0.4km & inter-session &  & short driving distance, frequent left turn\\
%  & Night & Cloud & 0.5km &  &  & \\ 
% \midrule

% \multirow{3}{*}{River island} & Day & Cloud & 5.8km &  & \multirow{3}{*}{\begin{tabular}[c]{@{}c@{}}odometry,\\global localization \end{tabular}} &  \\
%  & Dusk & Cloud & 8km & inter-session &  & various driving routes, flat area, intersections, two-lane one-way roads\\
%  & Day & Clear & 4km &  & & \\ 
% \midrule

% \multirow{2}{*}{Bridge} & Day & Rain & 4.9km & 1 time & odometry, & \multirow{2}{*}{four-lane overpass and the bridge, various dynamic objects} \\
%  & Night & Cloud & 4.9km & 1 time & online place recognition & \\ 
% \midrule

% Street & Day & Rain & 1km & 1 time & odometry & heavy traffic, low driving speed, and frequent stops \\ 
% \midrule

% \multirow{2}{*}{Stream} & Day & Clear & 4.2km & 2 times & odometry, & \multirow{2}{*}{S-shaped one-way roads, various dynamic objects, U-turns}\\
%  & Night & Cloud & 5.5km & 2 times & online place recognition & \\ 
% \bottomrule
% \end{tabular}
% }
% \end{table*}



\begin{table}[!t]
\caption{OVERVIEW OF SEQUENCES}
\label{tab:sequence}
\centering
\resizebox{\columnwidth}{!}{
\begin{tabular}{c|c|cccccc} 
\toprule
Sequence & Index & Time & Weather & Length & Loop & Target \\ 
\midrule

\multirow{3}{*}{\texttt{Mountain}} & 01 & Day & Clear & \unit{4}{km} & 2 times & \multirow{3}{*}{\begin{tabular}[c]{@{}c@{}}odometry,\\online place recognition,\\global localization \end{tabular}}   \\
 & 02 & Night & Cloud & \unit{4}{km} & 2 times &  \\ 
 & 03 & Day & Snow & \unit{3}{km} & 1.5 times &  \\ 
\midrule

\multirow{3}{*}{\texttt{Library}} & 01 & Day & Clear & \unit{1.6}{km} & 2 times & \multirow{3}{*}{\begin{tabular}[c]{@{}c@{}}odometry,\\online place recognition,\\global localization \end{tabular}}  \\
 & 02 & Night & Cloud & \unit{1.6}{km} & 2 times &  \\ 
  & 03 & Day & Snow & \unit{0.8}{km} & 1 time &  \\ 
\midrule

\multirow{3}{*}{\begin{tabular}[c]{@{}c@{}}\texttt{Sports}\\\texttt{Complex}\end{tabular}}{} & 01 & Day & Clear & \unit{1.4}{km} & 2 times & \multirow{3}{*}{\begin{tabular}[c]{@{}c@{}}odometry,\\online place recognition,\\global localization \end{tabular}} \\
 & 02 & Night & Cloud & \unit{0.7}{km} & 1 time &  \\ 
  & 03 & Day & Snow & \unit{1.4}{km} & 2 times &  \\ 
\midrule

\multirow{4}{*}{\begin{tabular}[c]{@{}c@{}}\texttt{Parking}\\\texttt{Lot}\end{tabular}}{} & 01 &  Day & Clear & \unit{0.4}{km} & \multirow{4}{*}{inter-session} & \multirow{4}{*}{\begin{tabular}[c]{@{}c@{}}odometry,\\global localization \end{tabular}}  \\
 & 02 & Day & Clear & \unit{0.4}{km} &  &  \\
 & 03 & Night & Cloud & \unit{0.5}{km} &  &  \\ 
 & 04 & Day & Snow & \unit{0.4}{km} &  &  \\ 
\midrule

\multirow{3}{*}{\begin{tabular}[c]{@{}c@{}}\texttt{River}\\\texttt{Island}\end{tabular}}{} 
& 01 & Day & Clear & \unit{4}{km} &  & \multirow{3}{*}{\begin{tabular}[c]{@{}c@{}}odometry,\\global localization \end{tabular}}  \\
 & 02 & Dusk & Cloud & \unit{8}{km} & inter-session & \\
 & 03 & Day & Cloud & \unit{5.8}{km} &  &  \\ 
\midrule

\multirow{2}{*}{\texttt{Bridge}} & 01 & Day & Rain & \unit{4.9}{km} & 1 time & odometry, \\
& 02 & Night & Cloud & \unit{4.9}{km} & 1 time & online place recognition \\ 
\midrule

\texttt{Street} & 01 & Day & Rain & \unit{1}{km} & 1 time & odometry \\ 
\midrule

\multirow{2}{*}{\texttt{Stream}} & 01 & Day & Clear & \unit{4.2}{km} & 2 times & odometry, \\

 & 02 & Night & Cloud & \unit{5.5}{km} & 2 times & online place recognition \\ 
\bottomrule
\end{tabular}
}
\vspace{-5mm}
\end{table}

%TABLE

\section{Description of HeRCULES Dataset}
\label{sec:experiment}

\subsection{Target Environments}
% This subsection briefly outlines the reasons for selecting the eight target environments depicted in \figref{fig:road} and their key characteristics. An overview of the eight sequences is presented in \tabref{tab:sequence}.
This subsection briefly outlines the reasons for selecting the eight target environments depicted in \figref{fig:road}. An overview of the eight sequences is presented in \tabref{tab:sequence}.

\subsubsection{Mountain}
\texttt{Mountain} captures sequences on Gwanak Mountain, the highest elevation difference among all sequences. The route includes speed bumps and rough roads, causing significant rolling and pitching.


\subsubsection{Library}
% \texttt{Library} captures sequences from a long, narrow, one-way path near the library on campus. The route involves two loops for loop closure with sequences recorded in clear afternoon and cloudy night. The path is narrow and includes steep curves with uphill and downhill sections.
\texttt{Library} captures sequences from a long, narrow, one-way path near the library on campus. The path includes steep curves with uphill and downhill sections.
\subsubsection{Sports Complex}
\texttt{Sports Complex} captures sequences around a sports complex, including parking areas and roads with flat, gently sloped, and steep sections. Two loops with an average speed below 30 km/h were recorded during the day and night.
\subsubsection{Parking Lot}
\texttt{Parking Lot} captures sequences from a parking lot with many left turns, recorded on a clear afternoon and at night. While the ground appears flat, slight elevation variations are noted. This sequence has the shortest distance among all.
\subsubsection{River Island}
% \texttt{River Island} captures three sequences for multi-session place recognition, recorded during the day and dusk, with each route uniquely designed. The flat area features various driving paths, including intersections and two-lane one-way roads.
\texttt{River Island} captures three sequences for multi-session place recognition with each route uniquely designed. The flat area features various driving paths, including intersections and two-lane one-way roads.
\subsubsection{Bridge}
\texttt{Bridge} captures sequences for place recognition research, driven back and forth along a four-lane overpass and the Wonhyo Bridge over the Han River. It includes sequences recorded on a rainy afternoon and cloudy dusk, with an average speed of \unit{60}{km/h}, and sections featuring traffic congestion in urban environments.
\subsubsection{Street}
\texttt{Street} captures a sequence of driving in heavy congestion and rain near IFC Seoul during rush hour. Due to the crowds and numerous vehicles, there are many dynamic objects, leading to frequent stops.
\subsubsection{Stream}
% \texttt{Stream} captures an S-shaped stream route that is driven during the day and night. One-way roads run along both sides of the stream, with U-turns via bridges. For place recognition research, intentional revisits were designed, resulting in similar environments.
\texttt{Stream} captures an S-shaped stream route with one-way roads running along both sides, allowing U-turns via bridges. For place recognition research, intentional revisits were designed, resulting in similar environments.

% \begin{figure}[!t]
%     \centering
%     \includegraphics[trim= 2cm 2.5cm 2cm 2.5cm, clip,width=1\linewidth]{example-image-a}
%     \caption{map}
%     \label{fig:map}
%     \vspace{-3mm}
% \end{figure}


\begin{figure}[!t] % "t"는 페이지 상단에 이미지를 배치하려는 옵션입니다.
    \centering
    \includegraphics[trim=0.5cm 4.55cm 0.5cm 1.2cm, clip, width=\linewidth]{figure/r1.pdf} % 이미지를 1단 전체 너비로 표시
    \vspace{-7mm}
\end{figure}


\begin{figure}[!t] % "t"는 페이지 상단에 이미지를 배치하려는 옵션입니다.
    \centering
    \includegraphics[trim=0.5cm 5.2cm 0.5cm 1.2cm, clip, width=\linewidth]{figure/r2.pdf} % 이미지를 1단 전체 너비로 표시
    \caption{Trajectory overlaid on satellite maps for each sequence with colors. Red indicates the start, while blue designates the end.}
    \label{fig:road}
    \vspace{-6mm}
\end{figure}

 % Each sequence features distinct environmental characteristics, and multiple revisits to the same locations ensure sufficient queries for place recognition evaluation.

\subsection{Data Description and Format}

The file structure of the HeRCULES dataset is delineated in \figref{fig:file}. The acquisition time of
all measurements are stored in \texttt{datastamp.csv}. The FMCW LiDAR and 4D radar data are provided in \texttt{time.bin}, while the camera data is in \texttt{time.png}. For spinning radar data, we support software that converts raw polar images into Oxford-style \cite{barnes2020oxford} and Cartesian images. \ac{IMU}, \ac{GPS}, and \ac{INS} data are provided in \texttt{.csv}, and calibration information between sensors is available in \texttt{.yaml} and \texttt{.txt} format. The data types for each sensor are detailed in \tabref{tab:sensors}.

\subsection{Individual Ground Truth}
Before logging each sequence, we ensure that the GNSS solution is fixed and the INS solution has converged. We use \ac{PTP} to synchronize timestamps in \ac{UTC} across all sensors. However, spatiotemporal discrepancies arise due to differences in sensor mounting positions and data acquisition times.
To address this, we provide ground truth poses for each sensor to support place recognition research, handling spatial differences using extrinsic calibration results and temporal differences with B-Spline interpolation \cite{mueggler2018continuous}.
The ground truth is shown in \figref{fig:gtpose}, highlighting the importance of independently deriving ground truth pose for each sensor.

% We provide ground truth pose for each sensor to support place recognition research, addressing spatial differences using extrinsic calibration results and temporal differences with B-Spline interpolation \cite{mueggler2018continuous}.
% The ground truth is shown in \figref{fig:gtpose}, highlighting the importance of independently deriving ground truth pose for each sensor due to spatiotemporal variations rather than relying on a unified ground truth.



% \subsection{Datasets and Evaluation Metric}
% red section, we use the 0.5\unit{m} and 5{\textdegree} for handcart sequences, and 2.0\unit{m} and 10{\textdegree} for car sequences.

%  (RPE$_r$) and translational (RPE$_t$) parts each.


% \subsection{NTU4DRadLM dataset}
% \label{subsec:ntu}

% %TABLE
% % % Please add the following required packages to your document preamble:
% % \usepackage{multirow}
% % \usepackage{graphicx}
% % \usepackage[table,xcdraw]{xcolor}
% \begin{table*}[]
% \centering
% \caption{SENSOR SPECIFICATIONS}
% \label{tab:sensors}
% \resizebox{\textwidth}{!}{%
% \begin{tabular}{c|ccccccccc} 
% \toprule
%  &
%  &
%  &
%   \multicolumn{3}{c}{Resolution}
%  &
%  &
%   \multicolumn{3}{c}{FOV}
%   \\ \cline{4-6} \cline{8-10} \rule{0pt}{2.5ex}
%   \multirow{-2}{*}{Sensor} &
%   \multirow{-2}{*}{Type} & 
%   \multirow{-2}{*}{Data type} &
%   Range &
%   Azimuth &
%   Elevation &
%   &
%   Range &
%   Azimuth &
%   Elevation
% \\ 
% \midrule

%   {\begin{tabular}[c]{@{}c@{}}4D\\Radar \end{tabular}} &
% Continental ARS548 & 
%   {\begin{tabular}[c]{@{}c@{}}x, y, z, velocity, RCS,\\range, azimuth, elavation \end{tabular}} &
% 0.22 m&
%   {\begin{tabular}[c]{@{}c@{}}1.2°@0. . . ±15°\\1.68°@ ±45°\end{tabular}}&
% 2.3°&
% &
% 300 m&
% ±60°&
%   {\begin{tabular}[c]{@{}c@{}}±4°@300m\\±14°@$<$100m\end{tabular}}
% \\ \rule{0pt}{4ex}
%   {\begin{tabular}[c]{@{}c@{}}Spinning\\Radar \end{tabular}} &
% Navtech RAS6 & 
% Polar image, Cartesian image&
% 0.044 m&
% 0.9°&
% - &
%  &
% 330 m &
% 360°&
% -
% \\ \rule{0pt}{4ex}
%   {\begin{tabular}[c]{@{}c@{}}FMCW\\LiDAR \end{tabular}} &
% Aeva Aeries II &
%   {\begin{tabular}[c]{@{}c@{}}x, y, z, reflectivity, intensity,\\velocity, line-index, time-offset \end{tabular}} &

% 0.075m@1$\sigma$ &
% 0.025°&
% 0.025°&
% &
% 150 m&
% 19.2°&
% 30° 
% \\ \rule{0pt}{4ex}
%   Camera &

%   {\begin{tabular}[c]{@{}c@{}}FLIR Blackfly S \\BFS-U3-16S2C-CS USB3\end{tabular}} &
% 8-bit Bayer pattern png format&
% - &
% 1440px &
% 1080px &
% &
% - &
% 60°&
% 45°
% \\ \rule{0pt}{4ex}
% IMU & Xsens MTi-300 &
% % High performing Attitude and Heading Reference System. 9-axis 
%   {\begin{tabular}[c]{@{}c@{}}qx, qy, qz, qw, eul$_x$, eul$_y$, eul$_z$, gyr$_x$, gyr$_y$, \\gyr$_z$, acc$_x$, acc$_y$, acc$_z$, mag$_x$, mag$_y$, mag$_z$\end{tabular}} &
% - &
% - &
% - &
% &
% - &
% - &
% -
% \\ \rule{0pt}{4ex}
% RTK-GPS & 
%   {\begin{tabular}[c]{@{}c@{}}Hexagon NovAtel\\ SPAN-CPT7  \end{tabular}} &
%   {\begin{tabular}[c]{@{}c@{}}latitude, longitude, height, velocity$_{north}$,\\velocity$_{east}$, velocity$_{up}$, roll, pitch, azimuth, status \end{tabular}} &
% - &
% - &
% - &
% &
% - &
% - &
% -
% \\ 
% \bottomrule
% \end{tabular}%
% }
% \vspace{-3mm}
% \end{table*}

% Please add the following required packages to your document preamble:
% \usepackage{multirow}
% \usepackage{graphicx}
% \usepackage[table,xcdraw]{xcolor}
% \begin{table*}[]
% \centering
% \caption{SENSOR SPECIFICATIONS}
% \label{tab:sensors}
% \resizebox{\textwidth}{!}{%
% \begin{tabular}{c|cccccccccc} 
% \toprule
%  &
%  &
%  &
%   \multicolumn{3}{c}{Resolution}
%  &
%  &
%   \multicolumn{3}{c}{FOV} &
%   \\ \cline{4-6} \cline{8-10} \rule{0pt}{2.3ex}
%   \multirow{-2}{*}{Sensor} &
%   \multirow{-2}{*}{Type} & 
%   \multirow{-2}{*}{Data type} &
%   Range &
%   Azimuth &
%   Elevation &
%   &
%   Range &
%   Azimuth &
%   Elevation &
%   \multirow{-2}{*}{Frequency}
% \\ 
% \midrule

%   {\begin{tabular}[c]{@{}c@{}}4D\\Radar \end{tabular}} &
% Continental ARS548 & 
%   {\begin{tabular}[c]{@{}c@{}}x, y, z, velocity, RCS,\\range, azimuth, elavation \end{tabular}} &
% \unit{0.22}{m}&
%   {\begin{tabular}[c]{@{}c@{}}1.2°@0. . . ±15°\\1.68°@ ±45°\end{tabular}}&
% 2.3°&
% &
% \unit{300}{m}&
% ±60°&
%   {\begin{tabular}[c]{@{}c@{}}±4°@\unit{300}{m}\\±14°@$<$\unit{100}{m}\end{tabular}} &
% 20 Hz
% \\ \rule{0pt}{4ex}
%   {\begin{tabular}[c]{@{}c@{}}Spinning\\Radar \end{tabular}} &
% Navtech RAS6 & 
% Polar image, Cartesian image&
% \unit{0.044}{m}&
% 0.9°&
% - &
%  &
% \unit{330}{m} &
% 360°&
% - &
% 4 Hz
% \\ \rule{0pt}{4ex}
%   {\begin{tabular}[c]{@{}c@{}}FMCW\\LiDAR \end{tabular}} &
% Aeva Aeries II &
%   {\begin{tabular}[c]{@{}c@{}}x, y, z, reflectivity, intensity,\\velocity, line-index, time-offset \end{tabular}} &

% \unit{0.02}{m}@1$\sigma$ &
% 0.025°&
% 0.025°&
% &
% \unit{150}{m}&
% 120°&
% 30° &
% 10 Hz
% \\ \rule{0pt}{4ex}
%   Camera &

%   {\begin{tabular}[c]{@{}c@{}}FLIR Blackfly S \\BFS-U3-16S2C-CS USB3\end{tabular}} &
% 8-bit Bayer pattern png format&
% - &
% \unit{1440}{px} &
% \unit{1080}{px} &
% &
% - &
% 60°&
% 45° &
% 15 Hz
% \\ \rule{0pt}{4ex}
% IMU & Xsens MTi-300 &
% % High performing Attitude and Heading Reference System. 9-axis 
%   {\begin{tabular}[c]{@{}c@{}}q$_x$, q$_y$, q$_z$, q$_w$, eul$_x$, eul$_y$, eul$_z$, gyr$_x$, gyr$_y$, \\gyr$_z$, acc$_x$, acc$_y$, acc$_z$, mag$_x$, mag$_y$, mag$_z$\end{tabular}} &
% - &
% - &
% - &
% &
% - &
% - &
% - &
% 100 Hz
% \\ \rule{0pt}{4ex}
% RTK-GPS & 
%   {\begin{tabular}[c]{@{}c@{}}Hexagon NovAtel\\ SPAN-CPT7  \end{tabular}} &
%   {\begin{tabular}[c]{@{}c@{}}latitude, longitude, height, velocity$_{north}$,\\velocity$_{east}$, velocity$_{up}$, roll, pitch, azimuth, status \end{tabular}} &
% - &
% - &
% - &
% &
% - &
% - &
% - &
% 50 Hz
% \\ 
% \bottomrule
% \end{tabular}%
% }
% \vspace{-3mm}
% \end{table*}

\begin{table*}[]
\centering
\caption{SENSOR SPECIFICATIONS}
\label{tab:sensors}
\resizebox{\textwidth}{!}{%
\begin{tabular}{c|cccccccccc} 
\toprule
  \multirow{2}{*}{Sensor} &
  \multirow{2}{*}{Type} & 
  \multirow{2}{*}{Data type} &
  \multicolumn{3}{c}{Resolution}
  &
  &
  \multicolumn{3}{c}{FOV} &
  \multirow{2}{*}{Frequency}
  \\ \cline{4-6} \cline{8-10} \rule{0pt}{2.3ex}
  &
  &
  &
  Range &
  Azimuth &
  Elevation &
  &
  Range &
  Azimuth &
  Elevation &

\\ 
\midrule

  {\begin{tabular}[c]{@{}c@{}}4D\\Radar \end{tabular}} &
Continental ARS548 & 
  {\begin{tabular}[c]{@{}c@{}}x, y, z, velocity, RCS,\\range, azimuth, elavation \end{tabular}} &
\unit{0.22}{m}&
  {\begin{tabular}[c]{@{}c@{}}1.2°@0. . . ±15°\\1.68°@ ±45°\end{tabular}}&
2.3°&
&
\unit{300}{m}&
±60°&
  {\begin{tabular}[c]{@{}c@{}}±4°@\unit{300}{m}\\±14°@$<$\unit{100}{m}\end{tabular}} &
20 Hz
\\ \rule{0pt}{4ex}
  {\begin{tabular}[c]{@{}c@{}}Spinning\\Radar \end{tabular}} &
Navtech RAS6 & 
Polar image, Cartesian image&
\unit{0.044}{m}&
0.9°&
- &
 &
\unit{330}{m} &
360°&
- &
4 Hz
\\ \rule{0pt}{4ex}
  {\begin{tabular}[c]{@{}c@{}}FMCW\\LiDAR \end{tabular}} &
Aeva Aeries II &
  {\begin{tabular}[c]{@{}c@{}}x, y, z, reflectivity, intensity,\\velocity, line-index, time-offset \end{tabular}} &

\unit{0.02}{m}@1$\sigma$ &
0.025°&
0.025°&
&
\unit{150}{m}&
120°&
30° &
10 Hz
\\ \rule{0pt}{4ex}
  Camera &

  {\begin{tabular}[c]{@{}c@{}}FLIR Blackfly S \\BFS-U3-16S2C-CS USB3\end{tabular}} &
8-bit Bayer pattern png format&
- &
\unit{1440}{px} &
\unit{1080}{px} &
&
- &
60°&
45° &
15 Hz
\\ \rule{0pt}{4ex}
IMU & Xsens MTi-300 &
% High performing Attitude and Heading Reference System. 9-axis 
  {\begin{tabular}[c]{@{}c@{}}q$_x$, q$_y$, q$_z$, q$_w$, eul$_x$, eul$_y$, eul$_z$, gyr$_x$, gyr$_y$, \\gyr$_z$, acc$_x$, acc$_y$, acc$_z$, mag$_x$, mag$_y$, mag$_z$\end{tabular}} &
- &
- &
- &
&
- &
- &
- &
100 Hz
\\ \rule{0pt}{4ex}
RTK-GPS & 
  {\begin{tabular}[c]{@{}c@{}}Hexagon NovAtel\\ SPAN-CPT7  \end{tabular}} &
  {\begin{tabular}[c]{@{}c@{}}latitude, longitude, height, velocity$_{north}$,\\velocity$_{east}$, velocity$_{up}$, roll, pitch, azimuth, status \end{tabular}} &
- &
- &
- &
&
- &
- &
- &
50 Hz
\\ 
\bottomrule
\end{tabular}%
}
\vspace{-3mm}
\end{table*}

% %TABLE


% \tabref{table:ntu}. Our proposed maforementioned issues (\figref{fig:NTU}).

% In \texttt{loop3} sequen

% \subsection{MSC-RAD4R dataset}
% \label{subsec:msc}

% In the MSC-RAD4R datase in \tabref{table:msc}. 

% \texttt{URBAN}  
% \texttt{LOOP} sequen
% elocity measurements (\figref{fig:MSC_trajcetory}).

% To thoriments on snowy sequences \texttt{RURAL\_A2} and \texttt{RURAL\_B2} which are characterized by sharp roundabouts and high dRemarkably, 4DRadarSLAM achieved the second-best performance in most sequences even without leveraging \ac{IMU}.



% \subsection{Ablation Studies}


% \subsubsection{Low Elevation Error}

% + \cite{lee2022patchworkpp}. As depicted in \figref{fig:ablaan

% %TABLE
% \begin{table}[t]
    \scriptsize
    \caption{Effect of Each Module in ATE}
    \vspace{-1mm}
    \label{table:ablation}
    \centering
    \resizebox{0.35\textwidth}{!}{%
    \begin{tabular}{ccccc}
    \toprule
    & \texttt{RAW} & \begin{tabular}[c]{@{}c@{}}\texttt{CONT} \end{tabular} & \begin{tabular}[c]{@{}c@{}}\texttt{FILTER} \end{tabular} & \texttt{FULL} \\
    \midrule[1pt]
    \multicolumn{1}{c|} {\texttt{cp}} & 2.337 & \textit{1.307} & 2.029 & \textbf{0.699} \\
    \midrule
    \multicolumn{1}{c|}{\texttt{nyl}} & 26.319 & 19.069 & \textit{5.888} & \textbf{5.009} \\
    \midrule
    \multicolumn{1}{c|}{\texttt{URBAN\_A0}} & 13.433 & \textit{2.163} & 12.412 & \textbf{1.757}   \\
    \midrule
    \multicolumn{1}{c|}{\texttt{LOOP\_A0}} & 101.956  & \textit{23.345} & 84.664 & \textbf{16.752}   \\
    \midrule
    \multicolumn{1}{c|}{\texttt{RURAL\_A2}} & 76.799 & \textit{20.764} & 47.966 & \textbf{9.480}   \\
    \bottomrule
    \end{tabular}}
    \vspace{-5mm}
\end{table}
% %TABLE

% iled in \tabref{table:ablation}. The results show diminished \ac{ATE} performance our ground filtering process (\texttt{w/o Filter}). \Cref{fig:ablation_zaxis_nyl,fig:ablation_zaxis_loopa0} illustrates the $z$-error along the timestamp in \texttt{nyl} (NTU4DRadLM) and \texttt{LOOP\_A0} (Ms. ficantly reduces $z$-error. Furthermore, the improved elevation accuracy 




% \subsubsection{Effe}

% To anath discrete (\texttt{w/o CONT}) and continuous integration (\texttt{FULL}). Fo\cite{kubelka2023we}.
% As demonstrated in \tabref{table:ablation}, continuous integration s. 

% \subsubsection{Computation Time}

% %TABLE
% % Please add the following required packages to your document preamble:
% \usepackage{multirow}
\begin{table}[t]
\centering
\caption{QUANTITATIVE ANALYSIS: AUC SCORES}
\begin{adjustbox}{width=1\linewidth}
{
\begin{tabular}{c|cccccccccccc}
\toprule
\multirow{2}{*}{Sequence} &
 &
   \multicolumn{3}{c}{Aeva}
 &
 &
  \multicolumn{3}{c}{Continental}
 &
 &
  \multicolumn{3}{c}{Aeva-Continental}
  \\ \cline{3-5} \cline{7-9} \cline{11-13} \rule{0pt}{2.5ex}
  &
  &
  \unit{10}{m} &
  \unit{15}{m} &
  \unit{20}{m} &
  &
  \unit{10}{m} &
  \unit{15}{m} &
  \unit{20}{m} &
  &
  \unit{10}{m} &
  \unit{15}{m} &
  \unit{20}{m}
\\ 
\midrule
\multirow{1}{*}{\texttt{Sports Complex}}
% & 0  & 0.  & 0.  &   & 0. & 0. & 0. & & 0. & 0. & 0.\\
& & 0.976  & -  & -  &   & 0.809 & - & - & & 0.401 & - & -\\
\midrule
\multirow{1}{*}{\texttt{Library}}
& & 0.971 & 0.975 & 0.988   &  & 0.574 & 0.584 & 0.632 & & 0.276 & 0.296 & 0.331 \\
\bottomrule
\end{tabular}
}
\end{adjustbox}
\label{tab:pr}
\vspace{-2mm}
\end{table}

% %TABLE

% The time consumption analysis results in \texttt{loop2} and \texttt{LOOP\_A0} are presented in \tabref{table:time}. Each sequence is the longest path in the dataset and we utilized Intel i7 CPU@2.50 {\GHz} and 64 {\GB} RAM.
\section{Experiment}
In this section, we conduct extensive experiments to evaluate the performance of various LLMs on our Hellaswag-Pro benchmark. Our study is guided by three key research questions:
\textbf{RQ1}: How do different LLMs perform across all variants?
\textbf{RQ2}: What is the relative difficulty of different variants?
\textbf{RQ3}: How robust are LLMs to diverse prompts during evaluation?

\subsection{Experiment Setup} 
\subsubsection{Model Selection and Implementation Details}
We select 41 representative commercial and open-source models, including English LLMs, such as GPT-4o, Claude-3.5-Sonnet, Gemini-1.5-Pro,Mistral series, Llama3 series and Chinese LLMs, like Qwen-Max,  Qwen2.5 series, InternLM-2.5 series, Yi-1.5 series, Baichuan-2 series and DeepSeek series.

We integrate both Chinese HellaSwag and HellaSwagPro into the lm-evaluation-harness platform. For the open-source models, we use the default settings of lm-evaluation-harness: do\_sample is set to false and the temperature is set to the default value of the hugging-face library. For the closed-source models, we set the temperature to 0.7. In addition, we set the maximum output length to 1024.

\subsubsection{Prompt Strategy}
Taking into account the influence of language and shot, we design 9 prompting strategies, including Direct, CN-CoT, EN-CoT, CN-XLT and EN-XLT. The last four setups include both zero-shot and few-shot variants.\footnote{
For open-source models, Direct adopts an approach similar to the official implementation of HellaSwag, computing the log-likelihood for each option and selecting the one with the highest log-likelihood. And we report normalized accuracy that accounts for the impact of option length. Other prompting strategies use a generation setup and report accuracy based on exact match.}
\textbf {(1)Direct}: LLMs makes the selection directly without any CoT process.
\textbf{(2)CN-CoT}: LLMs performs CoT in Chinese, regardless of dataset language.
\textbf{(3)EN-CoT}: Similar to CN-CoT, but CoT is conducted in English. 
\textbf{(4)CN-XLT}: LLMs are instructed to first translate English questions and options to Chinese, and then reason in Chinese.
\textbf{(5)EN-XLT}: Similar to CN-XLT, but translates from Chinese dataset to English and reasons in English. 

%\textbf {CN-CoT}: LLMs perform Chinese reasoning and then output the answer and 3 shots are provided.
%\textbf {CN-CoT}: Similar as CNCoTFewShot without any shots.
%\textbf {EN-CoT}: The reasoning process in English is executed and then the answer is output and 3 shots are provided.
%\textbf {CN-XLT}: Inspired by this, we instruct LLMs to translate questions in Chinese and then output the answer after performing reasoning in Chinese too. And 3 shots are provided.
%\textbf {EN-XLT}: Inspired by this, we instruct LLMs to translate questions in Englsih and then output the answer after performing reasoning in Englsih too. Three shots are provided.

\subsubsection{Evaluation metric}

To comprehensively evaluate the robustness of each LLM, we consider four metrics: 
% Original Accuracy (\textbf{OA}), Average Robust Accuracy (\textbf{ARA}), Robust Loss Accuracy (\textbf{RLA}), and  Consistent Robust Accuracy (\textbf{CRA}).
\noindent %
\textbf{- Original Accuracy (OA)} measures accuracy on original problems.
\begin{equation}\label{eq1}
OA=\frac{\sum_{(x, y) \in D} \mathds{1}[L M(x), y]}{|D|}.
\end{equation}
\noindent %
\textbf{- Average Robust Accuracy  (ARA)} represents average accuracy across all variants, gauging overall performance on the robustness tasks.
\begin{equation}\label{eq2}
ARA=\frac{\sum_{\left(x^{\prime}, y^{\prime}\right) \in D_{R}} \mathds{1}\left(L M\left(x^{\prime}, y^{\prime}\right)\right.}{\left|D_{R}\right|}.
\end{equation}

\noindent %
\textbf{- Robust Loss Accuracy (RLA)} is the difference between ARA and OA, indicating performance degradation on robustness data versus original data.
%\begin{tiny}
%\begin{equation}\label{eq3}
%RLA=\frac{\sum_{\left(x^{\prime}, y^{\prime}\right) \in D_{R}} %\mathds{1}\left(L M\left(x^{\prime}, y^{\prime}\right)\right.}{\left|D_{R}\right|}-\frac{\sum_{(x, y) \in D}\mathds{1}[L M(x), y]}{|D|}
%\end{equation}
%\end{tiny}
\begin{equation}\label{eq3}
RLA= OA - ARA.
\end{equation}
\noindent %
\textbf{- Consistent Robust Accuracy (CRA)} shows accuracy when the model correctly answers both original and variant data, reflecting the model do understand the problem.
% consistency in problem-solving.
\begin{equation}\label{eq4}
CRA=\frac{\sum_{x, y, x^{\prime}, y^{\prime}}\mathds{1}[L M(x), y] \cdot \mathds{1}[L M(x^{\prime}), y^{\prime}]}{\left|D_{R}\right|}.
\end{equation}
For all equation above, $D$ denotes the original dataset, where $x$ represents the input question and options, and $y$ represents the correct label, while $D_{R}$ is the robust dataset with $x^{\prime}$ and $y^{\prime}$ representing similar to $x$ and $y$.


\begin{table*}[ht]
\centering
\setlength{\tabcolsep}{5pt}
% \footnotesize
\scalebox{0.6}{
% Please add the following required packages to your document preamble:
% \usepackage{multirow}
% \usepackage[table,xcdraw]{xcolor}
% Beamer presentation requires \usepackage{colortbl} instead of \usepackage[table,xcdraw]{xcolor}
% Please add the following required packages to your document preamble:
% \usepackage{multirow}
% \usepackage[table,xcdraw]{xcolor}
% Beamer presentation requires \usepackage{colortbl} instead of \usepackage[table,xcdraw]{xcolor}
\begin{tabular}{ccccccccccccc}
\hline
\multicolumn{1}{c|}{{ }}& \multicolumn{4}{c|}{Chinese}& \multicolumn{4}{c|}{English}& \multicolumn{4}{c}{AVG}\\ \cline{2-13} 
\multicolumn{1}{c|}{\multirow{-2}{*}{{ Model}}} & { OA(\%)$\uparrow$}& { ARA(\%)$\uparrow$} & {RLA(\%)$\downarrow$}& \multicolumn{1}{l|}{{CRA(\%)$\uparrow$}} & { OA(\%)$\uparrow$}& { ARA(\%)$\uparrow$} & { RLA(\%)$\downarrow$}& \multicolumn{1}{l|}{{CRA(\%)$\uparrow$}} & {OA(\%)$\uparrow$}& { ARA(\%)$\uparrow$} & {RLA(\%)$\downarrow$}& { CRA(\%)$\uparrow$} \\ \hline
\multicolumn{1}{c|}{{ Human}} & 96.41& 97.79& -1.38 & \multicolumn{1}{l|}{92.03}& 95.56& 96.04& -0.48 & \multicolumn{1}{l|}{90.02}& 95.99 & 96.92 & -0.93& 91.03 \\ \hline
\multicolumn{13}{c}{\textit{Close-source LLMs}}\\ 
\multicolumn{1}{c|}{{ GPT-4o}}& { 91.37} & { 81.97} & { 9.40}& \multicolumn{1}{l|}{{ 75.55}} & { \textbf{88.63}} & { \textbf{70.17}} & { \textbf{18.46}} & \multicolumn{1}{l|}{{ \textbf{63.06}}} & { 90.00} & { \textbf{76.07}} & { \textbf{13.93}} & { \textbf{69.31}} \\
\multicolumn{1}{c|}{{ Claude3.5}}& { \textbf{95.37}} & { 80.15} & { 15.22} & \multicolumn{1}{l|}{{ 75.04}} & { 85.11} & { 66.02} & { 19.08} & \multicolumn{1}{l|}{{ 57.20}} & { 90.24} & { 73.09} & { 17.15} & { 66.12} \\
\multicolumn{1}{c|}{{ Gemini-1.5-Pro}}& { 90.62} & { 78.36} & { 12.26} & \multicolumn{1}{l|}{{ 70.48}} & { 87.75} & { 60.74} & { 27.01} & \multicolumn{1}{l|}{{ 58.27}} & { 89.19} & { 69.55} & { 19.63} & { 64.38} \\
\multicolumn{1}{c|}{{ Qwen-Max}}& { 93.50} & { \textbf{84.82}} & { \textbf{8.68}}& \multicolumn{1}{l|}{{ \textbf{78.91}}} & { 87.60} & { 62.61} & { 24.99} & \multicolumn{1}{l|}{{ 59.65}} & { \textbf{90.55}} & { 73.72} & { 16.83} & { 69.28} \\ \hline
\multicolumn{13}{c}{\textit{Chinese open-source LLMs}} \\ 
\multicolumn{1}{c|}{{ Qwen2.5-0.5B}}& { 60.75} & { 45.18} & { \textbf{15.57}} & \multicolumn{1}{l|}{{ 28.70}} & { 49.50} & { 38.21} & { \textbf{11.29}} & \multicolumn{1}{l|}{{ 20.57}} & { 55.13} & { 41.70} & { \textbf{13.43}} & { 24.64} \\
\multicolumn{1}{c|}{{ Qwen2.5-1.5B}}& { 63.25} & { 46.16} & { 17.09} & \multicolumn{1}{l|}{{ 29.89}} & { 56.88} & { 39.57} & { 17.30} & \multicolumn{1}{l|}{{ 23.48}} & { 60.06} & { 42.87} & { 17.20} & { 26.69} \\
\multicolumn{1}{c|}{{ Qwen2.5-3B}}& { 67.50} & { 48.75} & { 18.75} & \multicolumn{1}{l|}{{ 33.79}} & { 61.75} & { 39.98} & { 21.77} & \multicolumn{1}{l|}{{ 25.75}} & { 64.63} & { 44.37} & { 20.26} & { 29.77} \\
\multicolumn{1}{c|}{{ Qwen2.5-7B}}& { 67.63} & { 50.59} & { 17.04} & \multicolumn{1}{l|}{{ 35.62}} & { 65.63} & { 43.93} & { 21.70} & \multicolumn{1}{l|}{{ 30.77}} & { 66.63} & { 47.26} & { 19.37} & { 33.20} \\
\multicolumn{1}{c|}{{ Qwen2.5-14B}} & { 69.00} & { 51.41} & { 17.59} & \multicolumn{1}{l|}{{ 35.84}} & { 68.50} & { 45.20} & { 23.30} & \multicolumn{1}{l|}{{ 32.12}} & { 68.75} & { 48.30} & { 20.45} & { 33.98} \\
\multicolumn{1}{c|}{{ Qwen2.5-32B}} & { 69.75} & { 53.11} & { 16.64} & \multicolumn{1}{l|}{{ 37.54}} & { 70.00} & { 46.10} & { 23.90} & \multicolumn{1}{l|}{{ 32.68}} & { 69.88} & { 49.61} & { 20.27} & { 35.11} \\
\multicolumn{1}{c|}{{ Qwen2.5-72B}} & { \textbf{70.87}} & { \textbf{54.75}} & { 16.12} & \multicolumn{1}{l|}{{ \textbf{39.64}}} & { \textbf{72.00}} & { \textbf{47.75}} & { 24.25} & \multicolumn{1}{l|}{{\textbf{ 35.12}}} & { \textbf{71.44}} & { \textbf{51.25}} & {20.19} & { \textbf{37.38}} \\ \hdashline[0.5pt/5pt]
\multicolumn{1}{c|}{{ Baichuan2-7B}}& { 67.00} & { 46.16} & { 20.84} & \multicolumn{1}{l|}{{ 31.50}} & { 60.62} & { 39.04} & { 21.58} & \multicolumn{1}{l|}{{ 25.21}} & { 63.81} & { 42.60} & { 21.21} & { 28.36} \\
\multicolumn{1}{c|}{{ Baichua2-13B}}& { 69.13} & { 46.98} & { 22.15} & \multicolumn{1}{l|}{{ 33.45}} & { 64.62} & { 38.82} & { 25.80} & \multicolumn{1}{l|}{{ 26.07}} & { 66.88} & { 42.90} & { 23.97} & { 29.76} \\ \hdashline[0.5pt/5pt]
\multicolumn{1}{c|}{{ DeepSeek-7B}} & { 68.13} & { 47.96} & { 20.17} & \multicolumn{1}{l|}{{ 33.30}} & { 63.38} & { 40.39} & { 22.99} & \multicolumn{1}{l|}{{ 26.70}} & { 65.76} & { 44.18} & { 21.58} & { 30.00} \\
\multicolumn{1}{c|}{{ DeepSeek-67B}}& { 71.50} & { 49.21} & { 22.29} & \multicolumn{1}{l|}{{ 35.89}} & { 71.37} & { 40.63} & { 30.75} & \multicolumn{1}{l|}{{ 29.71}} & { 71.44} & { 44.92} & { 26.52} & { 32.80} \\ \hdashline[0.5pt/5pt]
\multicolumn{1}{c|}{{ InternLM2.5-1.8B}}& { 61.62} & { 42.07} & { 19.55} & \multicolumn{1}{l|}{{ 26.99}} & { 55.37} & { 38.46} & { 16.91} & \multicolumn{1}{l|}{{ 22.61}} & { 58.50} & { 40.27} & { 18.23} & { 24.80} \\
\multicolumn{1}{c|}{{ InternLM2.5-7B}}& { 67.25} & { 49.77} & { 17.48} & \multicolumn{1}{l|}{{ 34.57}} & { 69.50} & { 40.89} & { 28.61} & \multicolumn{1}{l|}{{ 29.75}} & { 68.38} & { 45.33} & { 23.04} & { 32.16} \\
\multicolumn{1}{c|}{{ InternLM2.5-20B}} & { 67.37} & { 48.08} & { 19.29} & \multicolumn{1}{l|}{{ 33.21}} & { 73.62} & { 41.11} & { 32.51} & \multicolumn{1}{l|}{{ 31.23}} & { 70.50} & { 44.60} & { 25.90} & { 32.22} \\ \hdashline[0.5pt/5pt]
\multicolumn{1}{c|}{{ Yi-1.5-6B}} & { 67.00} & { 49.59} & { 17.41} & \multicolumn{1}{l|}{{ 34.27}} & { 64.38} & { 39.37} & { 25.01} & \multicolumn{1}{l|}{{ 26.62}} & { 65.69} & { 44.48} & { 21.21} & { 30.45} \\
\multicolumn{1}{c|}{{ Yi-1.5-9B}} & { 68.50} & { 50.18} & { 18.32} & \multicolumn{1}{l|}{{ 35.55}} & { 66.37} & { 39.58} & { 26.79} & \multicolumn{1}{l|}{{ 27.48}} & { 67.44} & { 44.88} & { 22.56} & { 31.52} \\
\multicolumn{1}{c|}{{ Yi-1.5-34B}}& { 71.00} & { 52.23} & { 18.77} & \multicolumn{1}{l|}{{ 38.09}} & { 71.00} & { 40.75} & { 30.25} & \multicolumn{1}{l|}{{ 29.91}} & { 71.00} & { 46.49} & { 24.51} & { 34.00} \\ \hline
\multicolumn{13}{c}{\textit{English open-source LLMs}} \\ 
\multicolumn{1}{c|}{{ Llama3-8B}} & { 59.13} & { 46.62} & { 12.51} & \multicolumn{1}{l|}{{ 28.23}} & { 66.25} & { 40.21} & { 26.04} & \multicolumn{1}{l|}{{ 27.34}} & { 62.69} & { 43.42} & { 19.27} & { 27.79} \\
\multicolumn{1}{c|}{{ Llama3-70B}}& { 65.75} & { 48.63} & { 17.12} & \multicolumn{1}{l|}{{ 32.70}} & { \textbf{72.50}} & { 41.27} & { 31.23} & \multicolumn{1}{l|}{{\textbf{ 30.63}}} & {\textbf{ 69.13}} & { 44.95} & { 24.18} & { 31.67} \\ \hdashline[0.5pt/5pt]
\multicolumn{1}{c|}{{ Mistral-7B-v0.2}} & { 57.75} & { 46.25} & { \textbf{11.50}} & \multicolumn{1}{l|}{{ 27.57}} & { 67.50} & { \textbf{41.52}} & { 25.98} & \multicolumn{1}{l|}{{ 28.93}} & { 62.63} & { 43.88} & { 18.74} & { 28.25} \\
\multicolumn{1}{c|}{{ Mixtral-8x7B-v0.1}} & { 63.62} & { 46.80} & { 16.82} & \multicolumn{1}{l|}{{ 30.82}} & { 69.75} & { 41.21} & { 28.54} & \multicolumn{1}{l|}{{ 29.39}} & { 66.69} & { 44.01} & { 22.68} & { 30.11} \\
\multicolumn{1}{c|}{{ Mixtral-8x22B-v0.1}}& { 66.00} & {\textbf{ 50.73}} & { 15.27} & \multicolumn{1}{l|}{{ \textbf{34.32}}} & { 72.12} & { 41.25} & { 30.87} & \multicolumn{1}{l|}{{ 30.61}} & { 69.06} & { \textbf{45.99}} & { 23.07} & { \textbf{32.47}} \\ \hdashline[0.5pt/5pt]
\multicolumn{1}{c|}{{ Gemma-2-2B}}& { 61.88} & { 45.38} & { 16.51} & \multicolumn{1}{l|}{{ 29.02}} & { 59.62} & { 39.13} & { \textbf{20.50}} & \multicolumn{1}{l|}{{ 24.88}} & { 60.75} & { 42.25} & {\textbf{ 18.50}} & { 26.95} \\
\multicolumn{1}{c|}{{ Gemma-2-9B}}& { \textbf{69.13}} & { 46.75} & { 22.38} & \multicolumn{1}{l|}{{ 33.29}} & { 64.88} & { 39.80} & { 25.08} & \multicolumn{1}{l|}{{ 26.91}} & { 67.01} & { 43.28} & { 23.73} & { 30.10} \\
\multicolumn{1}{c|}{{ Gemma-2-27B}} & { 63.38} & { 48.52} & { 14.86} & \multicolumn{1}{l|}{{ 31.96}} & { 71.88} & { 40.91} & { 30.97} & \multicolumn{1}{l|}{{ 30.25}} & { 67.63} & { 44.71} & { 22.92} & { 31.11} \\ \hline
\end{tabular}
}
\caption{TODO: bolded is not result. Results of existing LLMs on our HellaSwag-Pro dataset using \textbf{Direct} prompt. ``AVG'' indicates the average performance of each model on Chinese and English parts of the dataset.
The best results for each metric in each model category are \textbf{bolded}. }
\label{tab:main experiment.}
\end{table*}

\subsection{Model Performance (RQ1)}
\paragraph{Overall Performance}
Table \ref{tab:main experiment.} provides a comprehensive evaluation of various LLMs across four performance metrics\footnote{The results of instruct/chat models of Qwen2.5, Llama3 and Mixtral latest series are shown in Appendix.}. The main observations are as follow:
\begin{itemize}[leftmargin=*,topsep=0pt]
% \setlength{}{0}
    \item Upon evaluating all available models, we found that all performed well in overall accuracy (e.g., GPT-4 scored 90.00 in AVG OA, Claude 3.5 scored 90.24 in AVG OA). However, all models struggled with variations of the questions, as evidenced by a positive RLA value for each model. In contrast, humans received a negative RLA value, suggesting that the question variants were not more challenging than the originals. This disparity further illustrates that current LLMs lack a true understanding of the reasoning process and can easily be misled by question variants.
    \item When comparing open-source and close-source models, the close-source models demonstrate stronger capabilities in both OA and ARA scores, similar to most existing benchmarks. Overall, the RLA values for close-source models are also smaller, indicating that they are more robust in commonsense reasoning tasks compared to open-source models.
    \item When we compare models within the same series (e.g., Qwen, Llama), we observe that larger models often achieve higher scores on OA, ARA, and CRA. However, they are also more susceptible to variations, i.e., they have higher RLA values, a phenomenon particularly evident in English datasets. We attribute this phenomenon to the fact that larger models, compared to smaller ones, may have memorized more data, allowing them to rely on memorization to solve some problems more easily and making them more prone to the influence of variations~\cite{}.
\end{itemize}
% 1. When evaluating all available models, We find although 
% 2. When comparing the opensource LLMs and close source LLMs, 
% 3. When looking into each serious details
% \noindent
% \textbf{Overall Model Performance.}
% 1. close-source > open-source 2. the large the better 3. all have a performance decline when meeting varients.

% To evaluate the performance of various models, we observed patterns consistent with current mainstream trends: closed-source models generally outperform open-source models across metrics. 
% For instance, the closed-source model GPT-4o achieved scores of 90.00 in OA, 76.07 in ARA, and 69.31 in CRA, whereas the open-source model Qwen2.5-72B scored 71.44, 51.25, and 37.38, respectively. 
% Furthermore, within each model series, performance tends to improve with larger model sizes. 
% Nevertheless, even the strongest closed-source models struggle with variations in questions, as indicated by positive values in RLA for all models. In contrast, human performance yields a negative RLA value, highlighting that current LLMs do not genuinely grasp the reasoning process and are prone to falling into traps set by question variants. 
% This suggests that there is still significant room for improvement in developing models that can robustly understand and reason through complex linguistic challenges.
% It reveals a consistent pattern across Chinese, English, and average scores, with close-sourced LLMs generally outperforming open-sourced models. 
% However, all models exhibit a significant drop in performance when faced with robust variants, as indicated by RLA and CRA. Among closed-source models, GPT-4o demonstrates the highest ARA of 76.07\% in average scores, demonstrating its overwhelming superiority. Among open-sourced models, larger models tend to perform better, with Qwen2.5-72B achieving the highest OA (71.44\%) and ARA (51.25\%) in the average scores. However, even these top performers still struggle with robustness, as evidenced by the substantial RLA of 13.93\% for GPT-4o and 20.19\% for Qwen2.5-72B. Interestingly, some English open-sourced models, such as Llama3-70B and Mixtral-8x22B-v0.1, show competitive performance in English tasks but lag in Chinese tasks, highlighting the importance of language-specific training.

% \noindent
% \textbf{Chinese Models vs English Models.}
% Chinese models generally demonstrate higher OA in Chinese tasks compared to English tasks, with Qwen-Max achieving 93.50\% OA in Chinese versus 87.60\% in English. Conversely, English models tend to perform better in English tasks, exemplified by Llama3-70B's 72.50\% OA in English compared to 65.75\% in Chinese. 
% However, both Chinese and English models exhibit important drops in ARA across languages, indicating challenges in maintaining performance when faced with variations. This trend suggests that while models may excel in their primary language, they struggle with robustness across linguistic boundaries. 
% Notably, larger models tend to achieve higher ARA scores but also experience more substantial RLA, as seen with Qwen2.5-0.5B (41.70\% ARA, 13.43\% RLA in total) and Qwen2.5-72B (51.25\% ARA, 20.19\% RLA in total). 
% This pattern indicates that while increased model size enhances overall performance, it doesn't necessarily improve robustness proportionally. 
% The discrepancy between OA and ARA across languages underscores the need for improved cross-lingual robustness in language models, particularly as they scale in size and capability.


% \noindent
% \textbf{Comparison between Chinese and English datasets.}
% Generally, models demonstrate higher accuracy on the Chinese dataset compared to the English one, as evidenced by the consistently higher OA, ARA and CRA scores. For instance, GPT-4o achieves an OA of 91.37\%, an ARA of 81.97\% , an CRA of 75.55\% on the Chinese dataset, compared to 88.63\% and 70.17\% respectively on the English dataset. This trend is observed across most models, suggesting that the Chinese dataset is easier than English one. Moreover, the RLA values are typically lower for Chinese, indicating smaller performance drops when dealing with robust variants of Chinese questions. For example, Qwen-Max shows an RLA of 8.68\% for Chinese versus 24.99\% for English, highlighting a more consistent performance in Chinese. The CRA scores further reinforce this observation, with models generally maintaining higher consistency in correct answers for both original and variant Chinese questions.
% We attribute this phenomenon to the fact that blablabla

\noindent
\textbf{Reasoning Transferable Capability.}
% 为了进一步
To further analyze whether the model can transfer reasoning ability from the original question to its variant, Figure \ref{consis} presents the distribution of model performance on the original question and variant pairs. For all models, the pairs of (HellaSwag \ding{51} HellaSwag-Pro \ding{55}) occupy a significant proportion, indicating a challenge in transferring reasoning capabilities for current LLMs to more complex scenarios. Looking deeply, closed-source models like GPT-4 and Qwen-Max achieve around a 69\% portion of (HellaSwag \ding{51} HellaSwag-Pro \ding{51}) and a 3\% portion of (HellaSwag \ding{55} HellaSwag-Pro \ding{55}), while in contrast, open-source models struggle with around a 30\% portion of (HellaSwag \ding{51} HellaSwag-Pro \ding{51}) and a 20\% portion of (HellaSwag \ding{55} HellaSwag-Pro \ding{55}), further indicating the robustness of reasoning abilities in closed-source models.
% If a model can get both the original question and the variant right, we consider it to have transferable reasoning ability. Table \ref{consis} presents the distribution of model performance on the original question and variant pairs. Among all models, the pairs of (HellaSwag \ding{51}HellaSwag-Pro \ding{55}) account for a considerable proportion, i 
% The closed-source models like GPT-4o and Qwen-Max achieve around 69\% portion of (HellaSwag \ding{51}HellaSwag-Pro \ding{51}) and 3\% portion of (HellaSwag \ding{55} HellaSwag-Pro \ding{55}), indicating stronger reasoning transfer ability than other models. In contrast, open-source models struggle more, with around 30\% portion of (HellaSwag \ding{51}HellaSwag-Pro \ding{51}) and 20\% portion of (HellaSwag \ding{55} HellaSwag-Pro \ding{55}). 
% A notable trend is observed among the Qwen2.5 series, where increasing model size from 7B to 72B parameters correlates with improved performance on correct answers for both datasets (33.20\% to 37.38\%) and decreased failure rates (17.69\% to 14.7\%). It underscores the importance of model size in commonsense reasoning tasks.

\begin{figure}[t]
\centering
\setlength{\abovecaptionskip}{0.1cm}
\setlength{\belowcaptionskip}{0cm}
\includegraphics[width=\linewidth,scale=1.00]{images/consis.pdf}
\caption{Analysis of the transferable ability of model reasoning based on question pair performance. The green part, where both the original and the variant data are right, represents the transferable performance of model reasoning.}
\label{consis}
\vspace{-15pt}
\end{figure}

\begin{figure*}[ht]
\centering
\setlength{\abovecaptionskip}{0.1cm}
\setlength{\belowcaptionskip}{0cm}
\includegraphics[width=\linewidth,scale=1.00]{images/xing.pdf}
\caption{The impact of different few-shot prompts on model performance. With - as the separator, the first two parts of the legend represent the prompt name, and the third part represents the language of the dataset.}
\label{xing}
\vspace{-15pt}
\end{figure*}

\begin{figure}[ht]
\centering
\setlength{\abovecaptionskip}{0.1cm}
\setlength{\belowcaptionskip}{0cm}
\includegraphics[width=1.05\linewidth,scale=1.05]{images/zhu.pdf}
\caption{The RLA Distribution for 7 variants of commonsense reasoning. Parts below the 0 axis indicate that the model’s performance on the variant is improved compared to the original problem.}
\label{fig:zhu}
\vspace{-15pt}
\end{figure}


\subsection{Variant Analysis (RQ2)}
To further analyze the impact of different variants, we assessed the contribution of each variant to the RLA score. A higher contribution indicates that the model is more likely to make errors in that type. Figure~\ref{fig:zhu} presents the overall results, and the key observations are as follows:
\begin{itemize}[leftmargin=*]
    \item For problem restatement, causal inference, and sentence ordering, these three categories are the least challenging. Almost all models, particularly the close-source and Qwen series models, perform well on these variants, indicating that current LLMs can effectively handle these forms and we do not pay more attention on this kind of varients.
    \item For reverse conversion and critical testing, these two varients each contribute about 10\% to the RLA score. This indicates that current LLMs struggle to fully generalize to these simple scenarios, possibly because these types of questions are not commonly encountered, and reaserchers should pay some attention to this type of varients.
    \item For negative transformation and scenario refinement, this are the two most difficult tasks, with negative transformation being particularly challenging. For almost all models, these two varients accounts for more than 50\% of the RLA score. This may be due to intuitively counterintuitive questions—such as the use of "will not"  or counterfactual scenarios in scenario refinement. These setups are less common in LLM training data and cannot be easily tackled through memory alone. Only those LLMs which truely understand the question could answer the varient correctly, wihch better reflect the true performance of the model.. In the future, researchers should focus more on enhancing LLM's capability to address such types of questions.
\end{itemize}

% 1. Problem restCausal Inference 
% To further analysis the impact of different varients, we further 
% Figure \ref{fig: zhu} presents a comprehensive analysis of various LLMs' performance across different variant types. Negative transformation emerges as the most challenging task for all models, with scores consistently above 50.00\% and peaking at 78.38\% for Gemini-1.5-Pro. Conversely, problem restatement appears to be the least challenging, with most models scoring in the negative range. Intriguingly, smaller models like Qwen2.5-0.5B demonstrate unexpected strengths in certain areas, such as sentence sorting (7.75\%), outperforming some larger counterparts. A detailed analysis of each variant type follows.

% \noindent
% \textbf{Causal inference.} In this category, scores vary widely from -4.73\% for Qwen-Max to 12.25\% for Baichuan2-13B, illustrating differing degrees of sensitivity to causal reasoning among the models. Smaller models, such as Qwen2.5-0.5B and Qwen2.5-1.5B, achieve better scores, indicating relatively stronger robustness in causal reasoning. Conversely, larger models, like Baichuan2-13B, have higher scores, suggesting greater sensitivity to the challenges of inferring causality.

% \noindent
% \textbf{Critical testing.} Larger models, including Qwen2.5-72B and DeepSeek-67B, exhibit higher RLA scores of 30.50\% and 31.37\%, respectively, suggesting increased sensitivity when dealing with incomplete key information. In contrast, GPT-4o achieves the lowest score, highlighting its superior robustness in critical reasoning. This trend indicates that more complex models might struggle to handle incomplete contexts, underscoring potential areas for improvement in sophisticated architectures.

% \noindent
% \textbf{Negative transformation.} This aspect remains consistently challenging for all models, with scores ranging from 48.88\% to 78.38\%. Advanced commercial models like Gemini-1.5-Pro and Claude-3.5 also score higher (78.38\% and 76.43\%, respectively), indicating a prevalent sensitivity issue in reasoning processes when handling negations, irrespective of model size or architecture.

% \noindent
% \textbf{Problem restatement.} The negative values in this category for nearly all models suggest it is not particularly challenging. This is surprising, given that previous models were quite sensitive to sentence representation.

% \noindent
% \textbf{Reverse conversion.} This variation, which involves swapping the roles of the question and answer, seems to specifically impact larger models. For example, Qwen2.5-72B and DeepSeek-67B exhibit higher RLA scores of 24.38\% and 27.43\%, respectively, indicating heightened sensitivity to reverse reasoning compared to their performance on original questions.

% \noindent
% \textbf{Scenario refinement.} The scores range from 16.06\% for Gemma-2-2B to 32.56\% for Qwen2.5-72B, with larger models displaying more sensitivity in adapting to counterfactual predictions. This suggests that larger models may rely more heavily on general commonsense rather than flexibly adapting to specific contexts. Consequently, increased model complexity might adversely affect adaptability to scenario changes, underscoring the need for enhanced flexibility in advanced models.

% \noindent
% \textbf{Sentence sorting.} This category exhibits the most varied results across models. Some larger models like DeepSeek-67B and InternLM2.5-20B display higher scores (26.69\% and 26.68\%), indicating sensitivity, while others like Qwen2.5-72B and Gemini-1.5-Pro excel with lower scores (-9.88\% and -1.07\%, respectively). This suggests that sentence sorting ability may depend more on specific training approaches rather than being solely contingent on model size.


\subsection{Prompt Robustness (RQ3)}
% To investigate how prompt  influence our benchmark, we apply sereral prompt strategy on our datasets and showcase the average performance of all models on different kind of prompt strategies.
% Table~\ref{prompt} illustrates the final results. For both Chinese and English datasets, CN LLMs achieve the highest performance using CN-CoT-Few-Shot, followed closely by EN-CoT-Few-Shot, with overall performance scores of 67.36\% and 67.03\%, respectively. In contrast, English LLMs perform best with the EN-CoT-Few-Shot, reaching 67.55\% on the Chinese dataset and 60.36\% on the English dataset.
% Contrary to previous findings, translating the dataset to the model's advantage language before performing reasoning does not enhance performance. Moreover, Figure~\ref{xing} also shows the similar phenomenon. Conducting CoT reasoning in the model’s advantage language generally leads to better outcomes compared to Direct. Additionally, increasing the number of shots consistently improves performance across most configurations, highlighting the benefits of exposing models to multiple examples. 
To explore the impact of various prompt strategies on our benchmarks, we evaluated several approaches across our datasets and present the average performance of all models using different prompting techniques. Table~\ref{prompt} summarizes the results. For both Chinese and English datasets, Chinese LLMs performed best with the CN-CoT-Few-Shot strategy, followed closely by EN-CoT-Few-Shot, achieving overall scores of 67.36\% and 67.03\%, respectively. Conversely, English LLMs showed optimal performance with the EN-CoT-Few-Shot approach, attaining 67.55\% on the Chinese dataset and 60.36\% on the English dataset.
Besides, translating datasets into the model's native language before reasoning did not enhance performance. This phenomenon is further illustrated in Figure~\ref{xing}. Conducting CoT reasoning in the model's native language generally yields better results compared to direct reasoning. Furthermore, increasing the number of examples (shots) consistently boosts performance across most configurations, emphasizing the advantages of exposing models to multiple examples.
% Overall, the interaction between question language, prompt language, and the number of shots underscores the importance of aligning these factors to optimize task performance and robustness in LLMs.



% Please add the following required packages to your document preamble:
% \usepackage{multirow}
% Please add the following required packages to your document preamble:
% \usepackage{multirow}
\begin{table}[t]
\setlength{\tabcolsep}{8pt}
% \footnotesize
\scalebox{0.65}{
\begin{tabular}{c|l|lll}
\hline
\multicolumn{1}{l|}{Dataset}  & Prompt  & CN LLMs & EN LLMs &  LLMs \\ \hline
\multirow{7}{*}{\begin{tabular}[c]{@{}c@{}}Chinese\\ HellaSwag-Pro\end{tabular}} & Direct  & 48.95& 41.16& 45.06  \\
& CN-CoT-Few  & \textbf{71.04}& 51.90& 61.47  \\
& EN-CoT-Few  & 70.95& \textbf{67.55}& \textbf{69.25}  \\
& EN-XLT-Few  & 41.48& 28.69& 35.09  \\
& CN-CoT-Zero & 44.82& 23.89& 34.36  \\
& EN-CoT-Zero & 45.38& 31.39& 38.39  \\
& EN-XLT-Zero & 28.57& 12.93& 20.75  \\ \hline
\multirow{7}{*}{\begin{tabular}[c]{@{}c@{}}English\\ HellaSwag-Pro\end{tabular}} & Direct  & 47.46& 40.66& 44.06  \\
& CN-CoT-Few  & \textbf{63.67}& 47.24& 55.46  \\
& EN-CoT-Few  & 63.12& \textbf{60.36}& \textbf{61.74}  \\
& CN-XLT-Few  & 48.77& 16.61& 32.69  \\
& CN-CoT-Zero & 34.89& 18.25& 26.57  \\
& EN-CoT-Zero & 42.41& 31.03& 36.72  \\
& CN-XLT-Zero & 16.36& 11.22& 13.79  \\ \hline
\multirow{9}{*}{HellaSwag-Pro}& Direct  & 48.21& 40.91& 44.83  \\
& CN-CoT-Few  & \textbf{67.36}& 49.57& 58.46  \\
& EN-CoT-Few  & 67.03& \textbf{63.95}& \textbf{65.49}  \\
& CN-XLT-Few  & 59.91& 34.26& 47.08  \\
& EN-XLT-Few  & 52.30& 44.52& 48.41  \\
& CN-CoT-Zero & 39.86& 21.07& 30.46  \\
& EN-CoT-Zero & 43.90& 31.21& 37.55  \\
& CN-XLT-Zero & 30.59& 17.55& 24.07  \\
& EN-XLT-Zero & 35.49& 21.98& 28.74  \\ \hline
\end{tabular}
}
\caption{Average ARA of all open-source models on different prompts. CN-LLMs contains 17 LLMs, and EN-LLMs contains 7 LLMs. The bast results for each dataset are \textbf{bolded}.}
\label{prompt}
\end{table}




\section{Concluding Remarks}
In this paper, we proposed a novel approach utilizing multimodal LLMs to generate gesture-aware speech recognition transcripts for patients with language disorders. Our framework integrates verbal speech and iconic gestures, enabling the generation of enriched transcripts that capture the latent meaning conveyed through both modalities. Through extensive experimentation, we demonstrated that the proposed method effectively contextualizes incomplete or disfluent speech by incorporating gesture information, leading to more accurate and meaningful representations of the speaker's intent. These findings highlight the potential of our approach to significantly contribute to the field of speech and language therapy, offering innovative tools that can enhance the quality of life for individuals with language disorders by facilitating better communication and assessment methods.

\subsection{Ethical Statement} 
Our dataset was obtained from AphasiaBank with the approval of the Institutional Review Board (IRB) and adheres to the data sharing guidelines set by TalkBank\footnote{https://talkbank.org/share/ethics.html}. This includes complying with the Ground Rules for all TalkBank databases, which are based on the American Psychological Association Code of Ethics~\cite{american2002ethical}.

\subsection{Limitation \& Future Work} 
%This study represents a preliminary investigation into using multimodal LLMs to generate gesture-aware speech recognition transcripts. 
While the results are promising, we recognize several limitations and outline our plans to extend this work further.

One primary limitation is the absence of a definitive ground truth for quantitative evaluation. Since our model generates transcripts by synthesizing speech and gesture data from scratch, traditional benchmarks, such as comparisons with standard speech recognition outputs, are insufficient. Moreover, existing original transcripts lack gesture annotations, making direct comparisons challenging. In future work, we aim to address this gap by collaborating with certified pathologists to conduct qualitative assessments, such as A-B preference tests, to evaluate the effectiveness of gesture-enriched transcripts in accurately conveying the speaker's intentions.

To support quantitative evaluations, we plan to develop novel metrics that assess transcript quality, including grammar accuracy, semantic consistency, and the integration of multimodal information. Such metrics will provide a more objective basis for assessing our model's performance and facilitate comparisons with other multimodal and unimodal approaches.

Another limitation of this study is its focus on structured gestures from a specific task, the Peanut Butter Sandwich Task. While this task offers a controlled context for testing our approach, it does not encompass the diversity of gestures and communication patterns seen in everyday scenarios. As part of our future work, we plan to expand the scope of our model to include tasks such as the Cinderella Story Recall Task~\cite{bird1996cinderella}, which involves unstructured and complex narrative gestures. This expansion will allow us to evaluate the adaptability and robustness of our model in handling varied linguistic and gestural contexts.

In summary, while this study establishes a strong foundation for gesture-aware speech recognition, we aim to refine and extend our methods through collaborative qualitative evaluations, the development of robust quantitative metrics, and broader task applications. These efforts will ensure that our approach continues to evolve, ultimately contributing to more effective communication tools and interventions for individuals with language disorders.





% \newpage
% \newpage

%\section*{ACKNOWLEDGMENT}
\balance
\small
\bibliographystyle{IEEEtranN} %citeauthor
\bibliography{string-short,references}


\end{document}