%%%%%%%%%%%%%%%%%%%%%%%%%%%%%%%%%%%%%%%%%%%%%%%%%%%%%%%%%%%%%%%%%%%%%%%%

%%% LaTeX Template for AAMAS-2025 (based on sample-sigconf.tex)
%%% Prepared by the AAMAS-2025 Program Chairs based on the version from AAMAS-2025.

%%%%%%%%%%%%%%%%%%%%%%%%%%%%%%%%%%%%%%%%%%%%%%%%%%%%%%%%%%%%%%%%%%%%%%%%

%%% Start your document with the \documentclass command.


%%% == IMPORTANT ==
%%% Use the first variant below for the final paper (including auithor information).
%%% Use the second variant below to anonymize your submission (no authoir information shown).
%%% For further information on anonymity and double-blind reviewing,
%%% please consult the call for paper information
%%% https://aamas2025.org/index.php/conference/calls/submission-instructions-main-technical-track/

%%%% For anonymized submission, use this
% \documentclass[sigconf,anonymous]{aamas}
\documentclass[sigconf]{aamas}

%%%% For camera-ready, use this
%\documentclass[sigconf]{aamas}


%%% Load required packages here (note that many are included already).

\usepackage{balance} % for balancing columns on the final page

%%%%%%%%%%%%%%%%%%%%%%%%%%%%%%%%%%%%%%%%%%%%%%%%%%%%%%%%%%%%%%%%%%%%%%%%

%%% AAMAS-2025 copyright block (do not change!)

\setcopyright{ifaamas}
\acmConference[AAMAS '25]{Proc.\@ of the 24th International Conference
on Autonomous Agents and Multiagent Systems (AAMAS 2025)}{May 19 -- 23, 2025}
{Detroit, Michigan, USA}{A.~El~Fallah~Seghrouchni, Y.~Vorobeychik, S.~Das, A.~Nowe (eds.)}
\copyrightyear{2025}
\acmYear{2025}
\acmDOI{}
\acmPrice{}
\acmISBN{}


%%%%%%%%%%%%%%%%%%%%%%%%%%%%%%%%%%%%%%%%%%%%%%%%%%%%%%%%%%%%%%%%%%%%%%%%

%%% == IMPORTANT ==
%%% Use this command to specify your submission number.
%%% In anonymous mode, it will be printed on the first page.

\acmSubmissionID{<<submission id>>}

%%% Use this command to specify the title of your paper.

\title[Dynamic Coalition Structure Detection]{Dynamic Coalition Structure Detection in Natural Language-based Interactions}

%%% Provide names, affiliations, and email addresses for all authors.

\author{Abhishek N. Kulkarni*}
\affiliation{
  \institution{University of Texas at Austin}\thanks{* Equal contribution.}
  \city{Austin}
  \country{USA}}
\email{abhishek.kulkarni@austin.utexas.edu}

\author{Andy Liu*}
\affiliation{
  \institution{Carnegie Mellon University}
  \city{Pittsburgh}
  \country{USA}}
\email{andyliu@cs.cmu.edu}

\author{Jean-Rapha\"el Gaglione}
\affiliation{
   \institution{University of Texas at Austin}
  \city{Austin}
  \country{USA}}
\email{jr.gaglione@utexas.edu}

\author{Daniel Fried}
\affiliation{
  \institution{Carnegie Mellon University}
  \city{Pittsburgh}
  \country{USA}}
\email{dfried@cs.cmu.edu}

\author{Ufuk Topcu}
\affiliation{
   \institution{University of Texas at Austin}
  \city{Austin}
  \country{USA}}
\email{utopcu@utexas.edu}



%%% Use this environment to specify a short abstract for your paper.


\usepackage{todonotes}
\usepackage{acronym}
\usepackage{amsmath}
\usepackage{amsthm}
\usepackage{mathtools}
\usepackage{graphicx}
\usepackage{cite}
\usepackage{bbm}
% \usepackage{changes}
\usepackage[commandnameprefix=always]{changes}
\usepackage{booktabs} 
\usepackage{placeins}
%\usepackage{lineno}
%\linenumbers

\begin{acronym}
\acro{gan}[GANs]{Generative Adversarial Networks}
\acro{rl}[RL]{Reinforcement Learning}
\acro{pae}[PAE]{Periodic Autoencoder}
\acro{fld}[FLD]{Fourier Latent Dynamics}
\acro{ppo}[PPO]{Proximal Policy Optimization}
\acro{fft}[FFT]{Fast Fourier Transform}
\acro{pca}[PCA]{Principal Component Analysis}
\acro{dfm}[DFM]{Deep Fourier Mimic}
\acro{dof}[DoF]{Degrees of Freedom}
\acro{mlp}[MLPs]{Multi-Layer Perceptrons}
\end{acronym}


\newcommand{\BILLE}{\makebox{\textsc{Bill-E}}\xspace}
\newcommand{\SOLLE}{\makebox{\textsc{Soll-E}}\xspace}
\newcommand{\ARMADAS}{\makebox{\textsc{Armadas}}\xspace}
\newcommand{\probName}{{\textsc{Single Robot Reconfiguration}}\xspace}

\newcommand{\configs}{\ensuremath{\mathcal{C}}}
\newcommand{\BigO}{\mathcal{O}}
\newcommand{\OPT}{\textsf{OPT}\xspace}
\newcommand{\size}[1]{|#1|}
\newcommand{\comp}{free component\xspace}
\newcommand{\comps}{free components\xspace}
\newcommand{\Comps}{Free components\xspace}
\newcommand{\bipGraph}[1]{G_{#1}}
\newcommand{\sumcomps}[1]{F_\Sigma}

\newcommand{\eDist}[1]{d_E(#1)}
\newcommand{\cDist}[1]{d_C(#1)}
\newcommand{\makesp}[1]{|#1|}
\newcommand{\weight}[1]{w(#1)}

\newcommand{\minMatching}{M_{\OPT}}

\newcommand{\lowerbound}{\ensuremath{D}}
\newcommand{\lowerboundOf}[1]{\ensuremath{\sigma(#1)}}
%%%%%%%%%%%%%%%%%%%%%%%%%%%%%%%%
% THEOREMS
%%%%%%%%%%%%%%%%%%%%%%%%%%%%%%%%
\theoremstyle{plain}
\newtheorem{theorem}{Theorem}[section]
\newtheorem{proposition}[theorem]{Proposition}
\newtheorem{claim}[theorem]{Claim}
\newtheorem{lemma}[theorem]{Lemma}
\newtheorem{informaltheorem}{Informal Theorem}
\newtheorem{corollary}[theorem]{Corollary}
\newtheorem*{theorem-non}{Theorem}

\theoremstyle{definition}
\newtheorem{definition}[theorem]{Definition}
\newtheorem{assumption}[theorem]{Assumption}
\newtheorem{problem}{Problem}
\newtheorem{openproblem}{Open Problem}

\theoremstyle{remark}
\newtheorem{remark}[theorem]{Remark}
\newtheorem{example}[theorem]{Example}
\newtheorem{observation}{Observation}


\begin{abstract}
	
In strategic multi-agent sequential interactions, detecting dynamic coalition structures is crucial for understanding how self-interested agents coordinate to influence outcomes. However, natural-language-based interactions introduce unique challenges to coalition detection due to ambiguity over intents and difficulty in modeling players' subjective perspectives. We propose a new method that leverages recent advancements in large language models and game theory to predict dynamic multilateral coalition formation in Diplomacy, a strategic multi-agent game where agents negotiate coalitions using natural language. The method consists of two stages. The first stage extracts the set of agreements discussed by two agents in their private dialogue, by combining a parsing-based filtering function with a fine-tuned language model trained to predict player intents. In the second stage, we define a new metric using the concept of subjective rationalizability from hypergame theory to evaluate the expected value of an agreement for each player. We then compute this metric for each agreement identified in the first stage by assessing the strategic value of the agreement for both players and taking into account the subjective belief of one player that the second player would honor the agreement. We demonstrate that our method effectively detects potential coalition structures in online Diplomacy gameplay by assigning high values to agreements likely to be honored and low values to those likely to be violated. The proposed method provides foundational insights into coalition formation in multi-agent environments with language-based negotiation and offers key directions for future research on the analysis of complex natural language-based interactions between agents.
\end{abstract}

%%% The code below was generated by the tool at http://dl.acm.org/ccs.cfm.
%%% Please replace this example with code appropriate for your own paper.


%%% Use this command to specify a few keywords describing your work.
%%% Keywords should be separated by commas.

\keywords{Coalition Structures, Game Theory, Multi-Agent Cooperation, Large Language Models}

%%%%%%%%%%%%%%%%%%%%%%%%%%%%%%%%%%%%%%%%%%%%%%%%%%%%%%%%%%%%%%%%%%%%%%%%

%%% Include any author-defined commands here.

\newcommand{\BibTeX}{\rm B\kern-.05em{\sc i\kern-.025em b}\kern-.08em\TeX}

%%%%%%%%%%%%%%%%%%%%%%%%%%%%%%%%%%%%%%%%%%%%%%%%%%%%%%%%%%%%%%%%%%%%%%%%

\begin{document}

%%% The following commands remove the headers in your paper. For final
%%% papers, these will be inserted during the pagination process.

\pagestyle{fancy}
\fancyhead{}

%%% The next command prints the information defined in the preamble.

\maketitle

%%%%%%%%%%%%%%%%%%%%%%%%%%%%%%%%%%%%%%%%%%%%%%%%%%%%%%%%%%%%%%%%%%%%%%%%



\section{Introduction}
\label{sec:introduction}
\documentclass[../main.tex]{subfiles}
\graphicspath{{../images/}}
\makeatletter
\def\input@path{{../images/}}
\makeatother
\begin{document}
\section{Introduction}
\begin{figure}
\centering
\begin{tikzpicture}
\node[inner sep=0pt] (ws) at (0, 0) {
\includegraphics[height=.4\textwidth, trim={10cm 0 10cm 0},clip]{world_space.png}};
\node[inner sep=0pt] (cs) at (6,0) {\includegraphics[height=.4\textwidth, trim={10cm 1cm 10cm 4cm},clip]{conf_space.png}};
\end{tikzpicture}
\vspace{-5pt}
\label{fig:pbrm_intro}
\caption{\textbf{Left}: Shows world space obstacles as grey spheres. Robots start and goal configuration is colored red and green, respectively. Configurations along the computed path are colored transparent blue. \textbf{Right:} Mapped world space scenario to configuration space. Obstacle region is the grey mesh. Red spheres are collision-free regions computed by the neural SCDF. The optimized shortest path in the convex corridor is the blue curve.}
\vspace{-25pt}
\end{figure}
Motion planning is the problem of finding a collision-free trajectory that connects a given start and goal configuration. The planning takes place in the configuration space of the robot. For single body robots, like mobile robots or drones, the configuration space and the world space are usually the same. This simplifies the planning, since explicit obstacle representations are available which enables geometrical tools like separating hyperplanes, smallest distance to obstacles etc., to be used when designing motion planning algorithms. For multi-body robots like manipulators, the situation is completely different. The world space obstacles are usually mapped to non-convex regions, and to make the problem even harder, the mapping is usually not known. Forming explicit representations of the obstacle region in the configuration space is usually too expensive or intractable. Despite all of this, sampling based planners are used with great success, which mainly is due to their use of implicit representations of the obstacle region. The basic idea is to construct a graph in the configuration space that covers and connects the collision-free region. From this graph, a path can be extracted that connects a given start and goal configuration. The approach is computationally expensive, since the graph is constructed with the smallest geometrical building block available, points, which represents a collision-check. Furthermore, the extracted paths from the graph are non-smooth and jagged due to the stochastic nature of the approach. This adds an additional post-processing step to the process, where the paths are shortcutted and smoothened, before the path can be used for tracking. Clearly a lot of time is invested to form this graph and produce smooth paths. Thus, if the obstacles start to move, then all of this work is done in no use, since all points that make up this graph need to be re-verified, which is simply too time consuming to be done in real time.
\\\\
In this work, we want to address the existing drawbacks of the sampling based planners. Our main contribution is an improved motion planner where each vertex in the graph covers a collision-free region in the form of a sphere instead of a point and where the edges are formed with neighboring intersecting spheres. This representation has the advantage of instead of returning piecewise linear paths, returning a sequence of overlapping spheres, i.e. a convex corridor, that connects a given start and goal configuration, illustrated in Figure \ref{fig:pbrm_intro}. This convex corridor allows us to use convex optimization to produce smooth trajectories, instead of computationally expensive post-processing methods. The representation further allows us to estimate the coverage of the collision-free space, which gives us awareness and feedback in the offline roadmap construction phase. Finally, our representation is simple to adapt to moving obstacles, simply requery for the new radii and recheck for intersections. 
\\\\
The spherical collision-free regions are formed using a signed distance function (SDF), which is a function that returns the smallest distance from an arbitrary point to the boundary of an obstacle. As the name implies, the distance is signed, thus if the point is inside the obstacle it is negative otherwise positive. If the distance is positive, a sphere with radius equal to the distance is guaranteed to cover a collision-free region. Using an SDF in motion planning is not new, but what is novel about our approach is that we express the distance in the configuration space instead of the world space and by doing so allows us to form these convex collision-free regions. We refer to the resulting SDF as a signed configuration distance function (SCDF). Computing an SCDF analytically is non-trivial, our approach is therefore to parameterize the SCDF with a deep neural network and learn the mapping by supervised learning. Our resulting neural SCDF can compute distances for different parameter values of obstacle shapes and we also show how multiple distances can be combined, thus making our approach flexible.
\section{Related work}
Motion planning algorithms can roughly be divided into three families, grid-based, sampling based and optimization based methods. Grid-based methods (GBM) discretize the planning space from which a graph is then compiled. A standard search method is A$^\star$ \citep{a_star}, which is classified as an \textit{informed} search method, since it employs a heuristic function to speed up the search. A$^\star$ guarantees to return an optimal path at the level of discretization used. GBMs usually discretize the planning space by a regular lattice and this limits the GBMs to problems with low dimensionality due to the curse of dimensionality. Thus, GBMs are usually limited to single-body robots where the degrees of freedom (DOF) are low. To overcome the inherent scaling problem with the GBMs, stochastic methods are usually used for multi-body robots. These methods are termed as sampling-based methods (SBM) and core members within this family are the rapidly-exploring random trees (RRT) \citep{rrt} and the probabilistic roadmap (PRM) \citep{prm}. RRT grows a tree from the start configuration and explores the collision-free region in a rapid way until it is able to connect to the goal region. RRT is usually improved by bi-directional planning \citep{rrt_connect}, i.e. an additional tree is grown from the goal configuration and the trees are tested for connection after any tree has been expanded. RRT is a single-query method, thus it searches for a path from scratch each time it is queried. Contrary to this, PRM is a multi-query method, which solves for multiple queries without starting from scratch. PRM does this by creating a roadmap (graph) that covers the collision-free space as an offline step. The graph is then used to solve for multiple queries. PRMs are used in cases where the environment does not change since the extra offline step is too computationally costly and needs to be re-done if the environment is changed. In our work, we address this inherent issue by using a different roadmap representation. Our vertices in the graph cover a collision-free region in the form of spheres and we form the edges by checking for intersecting spheres. If something in the environment changes, we recompute the spheres radii and recheck the intersections, without relying on collision detection. We use a trained neural network to compute the sphere radius, therefore querying for the radius can be done fast, hence our representation enables the PRM for dynamic environments.
\\\\
In the recent decades, optimization based methods (OBM) \citep{chomp, schulman, itomp, stomp} have been introduced as an alternative to SBM for multi-body robots. Like the SBM, the OBMs scale well to higher dimensional problems and produce smoother motion. It is common to use a SDF in the optimization since it is a smooth function, thus enabling gradient-based methods. However, the standard way of expressing the SDF is in world space. The distance therefore needs to be mapped to the configuration space by the forward kinematics. This mapping makes the optimization problem a non-linear program (NLP), which is computationally expensive to solve. Recently, a different approach has been proposed. In \cite{mp_gcs} motion planning is formulated as a convex optimization problem by using the graph of convex sets framework \citep{gcs}. The underlying idea is to decompose the collision-free space into intersecting convex sets from which a convex optimization problem is formulated. In cases where an explicit representation of the obstacles in the configuration space exists, like for single-body robots, creating collision-free convex regions can be done fast \citep{iris}. For multi-body robots, this is non-trivial. Existing work does this successfully \citep{iris_nlp, iris_c} by an optimization based approach, but the methods are still too time consuming to be used in the presence of moving obstacles. Our approach is instead to use deep learning to learn an SDF expressed in the configuration space. With this, we can query for shortest distances to the collision boundary, which allows us to expand spherical regions which are collision-free. Our approach is fast and therefore enables our suggested roadmap planner to be used in dynamic environments.
\\\\
Recent research has focused on learning collision detection \citep{fk_kernel_distance, diffco, graphdistnet} by predicting the signed distance between the robot links and the surrounding obstacles in the world space. The learned SDF is used in trajectory optimization but since the distance is expressed in the world space, the problem becomes an NLP and therefore takes a long time to solve. We take a novel approach and suggest to instead express the signed distance in the configuration space. This allows us to improve the PRM at the same time as it enables convex optimization for trajectory optimization, which runs faster and is more reliable than NLP solvers. In \cite{cspf} a learned signed distance function in the configuration space is proposed similar to our approach. However, their approach is restricted to point cloud representations, while we propose to represent the obstacles as parameterized geometric shapes, e.g. spheres. Furthermore, we also show how to use our learned SCDF to improve an existing roadmap planner.
\section{Problem formulation}
A robot is located in the world space, $\W \subset \R^3 $. The unique location of the robot is given by its configuration $\q \in \C$, where $\C$ is the configuration space. The set of points covered by the robots bodies at a certain configuration is expressed as $\B(\q) \subset \W$. The robot is surrounded by $\NrObst$ obstacles $\O = \bigcup_{i=1}^{\NrObst} \O_i$, where  $\O_i \subset \W$. The representation of the obstacle in the configuration space is the set $\C\O_i = \{\q \in \C \: |\: \B(\q) \cap \O_i \neq \emptyset \}$. The obstacle space is formed as $\Co = \bigcup_{i=1}^{\NrObst} \C \O_i$. The complement is referred to as the free space, $\Cf = \C \setminus \Co$. The path planning problem is a tuple, ($\Cf$, $\qStart$, $\qGoal$), where we want to connect a query pair, consisting of a start, $\qStart$, and goal configuration, $\qGoal$, with a geometric path, $\q(s): [0, 1] \mapsto \Cf$, such that $\q(0)=\qStart$ and $\q(1)=\qGoal$, or report correctly when such a path does not exist.
\end{document}




\section{Problem Formulation}
\label{sec:problem}
\subsection{Problem Formulation}

% We begin by formulating the problem of dynamic benchmarking for LLMs.
A dynamic benchmark is defined as  
$
\small
\mathcal{B}_{\text{dynamic}} = (\mathcal{D}, T(\cdot)), \quad 
\mathcal{D} = (\mathcal{X}, \mathcal{Y}, \mathcal{S}(\cdot))
$
where \( \mathcal{D} \) represents the static benchmark dataset. 
% consisting of input prompts \( \mathcal{X} \), expected outputs \( \mathcal{Y} \), and a scoring function \( \mathcal{S}(\cdot) \) that evaluates the quality of an LLM's outputs by comparing them against \( \mathcal{Y} \). 
The transformation function \( T(\cdot) \) modifies the data set during the benchmarking to avoid possible data contamination.
The dynamic dataset for the evaluation of an LLM can then be expressed as
$
\small
        \mathcal{D}_t = T_t(\mathcal{D}),  \quad
        \forall t \in \{1, \dots, N\}
$
where \( \mathcal{D}_t \) represents the evaluation data set at the timestamp \( t \), and \( N\) is the total timestamp number, which could be finite or infinite. % \ie $N= \infty$.
If the seed dataset $\mathcal{D}$ is empty, the dynamic benchmarking dataset will be created from scratch.





\section{Background: Hypergame Theory}
\label{sec:background}
\section{Basic Background: Supervised Learning and the PAC Model}
\label{sec:background}

At this point almost everyone has heard of machine learning (ML). Anyone likely to stumble upon this article will have also heard of its most influential special case, supervised learning, and those theoretically inclined will also be familiar with the PAC model. Nonetheless, I will set the stage by  recapping the basics.

\subsection{Basics of Supervised Learning}%Let's set the stage in any case

\emph{Supervised Learning} is the task of ``coming up'' with a function $f: \X \to \Y$ to ``explain'' or ``fit'' a sequence of input/output examples   $(x_1,y_1), \ldots, (x_n,y_n)$, with $x_i \in \X$ and $y_i \in \Y$.  Here $\X$ is a \emph{data domain} consisting of \emph{datapoints} $x \in \X$, $\Y$ is a \emph{label set} consisting of \emph{labels} $y \in \Y$, and the sequence $(x_1,y_1),\ldots,(x_n,y_n)$ is the \emph{training data} consisting of \emph{labeled examples (a.k.a. samples)}~$(x_i,y_i)$.  I~will refer to the chosen function $f$ as a \emph{predictor}, and to $n$ as the \emph{sample size}. A \emph{learning algorithm} takes as input training data, and outputs (some representation of) a predictor $f \in \Y^\X$.\footnote{Note that this describes the usual \emph{batch}, a.k.a.~\emph{offline}, setting of supervised learning. I do not discuss other paradigms such as online or active learning in this article.} 



Success in supervised learning is defined as \emph{generalization} to  future examples: For a typical \emph{test example}  $(x_{\tst},y_{\tst})$, the predicted label $y'_{\tst}=f(x_{\tst})$ should ``equal'' $y_{\tst}$, perhaps approximately. We usually assume the test example is drawn from the same  ``source'' as the training data  --- commonly, i.i.d.~from the same distribution. The quality of the prediction is quantified by $\ell(y'_{\tst},y_{\tst})$, where $\ell:~\Y~\times~\Y \to \RR_{\geq 0}$ is a \emph{loss function} chosen as part of the problem definition. Common loss functions include the 0-1 loss $\ell_{0-1}(y',y) = [y' \neq y]$ for \emph{classification} problems,\footnote{The notation $[P]$ denotes $1$ when predicate $P$ is true, and denotes $0$ when $P$ is false.} as well as the absolute loss $|y'-y|$ or squared loss $(y'-y)^2$ for \emph{regression problems} featuring $\Y  \sse \RR$.

Nontrivial generalization properties are typically only possible if one assumes something about the data.\footnote{The need for such an assumption is formalized by the  \emph{no free lunch theorems} of supervised learning \cite{wolpert_connection_1992,wolpert_lack_1996,schaffer_conservation_1994}.} The Bayesian approach to  machine learning, common in many applications, assumes some parametric form for the distribution generating the data, and postulates a prior on the parameters. This is not the approach I will take in this article. Instead, I will focus on the frequentist --- and some would say ``worst-case'' or ``adversarial'' ---  approach that is common in the computational learning theory community, embodied by the PAC model. Here we assume that the (training and test) data can be explained, perhaps approximately, by a function in some ``simple enough to learn'' class of functions $\H \sse \Y^\X$, often called the \emph{hypotheses}. Equivalently, we  seek a predictor which explains the unseen data roughly  as well as the best hypothesis $h^* \in \H$, whether or not we assume that $h^*$ itself provides a perfect explanation.



 \paragraph{Common Algorithmic Templates.} Perhaps the best known general-purpose supervised learning algorithm is \emph{empirical risk minimization (ERM)}, which chooses as its predictor a hypothesis $f \in \H$ minimizing $\frac{1}{n} \sum_{i=1}^n \ell(f(x_i),y_i)$ --- a quantity called the \emph{training error}, \emph{empirical error}, or \emph{empirical risk} of $f$. %\footnote{When multiple hypotheses minimize the empirical risk, we assume ERM breaks ties arbitrarily.}
A common template for generalizing ERM involves adding a \emph{regularization term} $\psi(f)$ to the  objective function, typically chosen to measure some notion of ``hypothesis complexity.'' An algorithm instantiating this template is known as a \emph{structural risk minimizer (SRM)}, and chooses as its predictor the hypothesis $f \in \H$ minimizing the \emph{structural risk} $\frac{1}{n} \sum_{i=1}^n \ell(f(x_i),y_i) + \psi(f)$. Other well-known algorithms, such as gradient descent and its variations,  can frequently be interpreted as approximate implementations of ERM or SRM.


\paragraph{Proper vs Improper Learning.} A learning algorithm is said to be \emph{proper} if its predictor $f$ is always chosen from the hypothesis class, i.e., $f \in \H$, otherwise it is said to be \emph{improper}. ERM  is an example of a proper learning algorithm, as are SRM algorithms of the form described above.  In the \emph{proper regime} of learning, algorithms are required to be proper. This article will be concerned with the more flexible \emph{improper regime} (a.k.a \emph{representation-independent learning}), where no such constraint is placed on the learner. In other words, all we care about is predictive power at test time, rather than any insights derived from the functional form or representation of the predictor~itself.


\subsection{The PAC Model}
A standard mathematical setup for evaluation of supervised learning algorithms, at least in the theoretical computer science community, is Valiant's \emph{Probably Approximately Correct (PAC) model} of learning (see e.g.~\cite{kearns_introduction_1994,mohri_foundations_2018}). Here, we assume there is an unknown distribution $\D$ on $\X \times \Y$ from which training and test data are  drawn.  Specifically, the labeled datapoints of the training set  $(x_1,y_1), \ldots, (x_n,y_n)$, as well as the test data  $(x_\tst,y_\tst)$, are i.i.d.~from $\D$. Often it is assumed that $\D$ lies in some class of distributions of interest. The \emph{true expected loss}, or simply \emph{loss}, of a predictor $f: \X \to \Y$ is the expected loss it incurs on draws from $\D$, written $L_\D(f) = \Ex_{(x,y) \sim \D} \ell(f(x),y)$.


There are two main ``settings'' in PAC learning. The  \emph{realizable setting} only requires that the data be perfectly explained by some hypothesis in $\H$. More generally, the \emph{agnostic setting} makes no assumption relating the data to the hypotheses, but shifts the goalposts as necessary to allow nontrivial guarantees: the expected loss at test time is evaluated only ``relative'' to that of the best hypothesis $h^* \in \H$. There are other settings which make more nuanced assumptions, such as $\D$ being of a particular parametric form or its support living in some (unknown) lower-dimensional space, etc. I will mostly discuss the realizable and agnostic settings in this article, those being the simplest and most studied from a theoretical perspective. %TODO:We will briefly discuss other settings in Section ??

The PAC model demands high probability guarantees of learners, in the worst case over distributions of interest. Consider first the realizable setting, where $\D$ is such that $\min_{h \in \H} L_{\D}(h) = 0$. A PAC learner has \emph{error} $\epsilon=\epsilon(n)$ and \emph{confidence} $\delta=\delta(n)$ if, when training data consists of $n$ i.i.d~samples from a realizable distribution $\D$, it produces a predictor $f$  satisfying $L_\D(f) \leq \epsilon$ with probability at least $1-\delta$. In the agnostic setting, where $\D$ can be arbitrary, we require $L_\D(f) - \min_{h \in \H} L_\D(h) \leq \epsilon$ with probability $1-\delta$.

In both the realizable and agnostic settings, we look for PAC learners with small $\epsilon$ and $\delta$ as a function of the sample size $n$. An equivalent perspective looks at the sample complexity $m(\epsilon,\delta)$, which is the minimum sample size which guarantees error  at most $\epsilon$ with probability at least $1-\delta$. We say a problem is \emph{PAC learnable} if its PAC sample complexity is finite whenever $\epsilon,\delta > 0$.

For most PAC learning problems, learnability and sample complexity are characterized in terms of a  ``dimension'' of the hypothesis class. Most prominently this is the \emph{VC dimension} for binary classification, the \emph{fat shattering dimension} for agnostic regression, and the \emph{DS dimension} for multiclass classification (see \cite{anthony_neural_1999,daniely_optimal_2014,brukhim_characterization_2022}). Treatment of these is beyond the scope of this article. The unfamiliar reader need not worry, however,  as dimensions will feature only tangentially in our~discussion.




%\paragraph{Learning settings: Realizable, Agnostic, etc.} In learning theory, evaluating a supervised learning algorithm requires specifying a data model and an objective. We will leave the details of the data model flexible for now, to allow for both the PAC model and the adversarial transductive model. Nonetheless we will describe two variations, which we call ``settings'', which cut across different models. The  \emph{realizable setting}  requires only that the data be perfectly explained by some hypothesis $h \in \H$ --- i.e., there exists a hypothesis which is guaranteed to suffer a loss of $0$ on training and test data. The performance of the learning algorithm is its expected loss at test time for some ``worst case'' realizable instance. More generally, the \emph{agnostic setting} makes no assumption relating the data to the hypotheses, but shifts the goalposts as necessary to allow nontrivial guarantees: the expected loss at test time is evaluated only ``relative'' to that of the best hypothesis $h^* \in \H$, again for some ``worst case'' instance. There are other settings which make more nuanced assumptions about the data, such as it is drawn from a distribution of a particular parametric form, or that it lives in some (unknown) lower-dimensional space, etc. We will mostly discuss the realizable and agnostic settings, those being the simplest and most studied from a theoretical perspective.




%%% Local Variables:
%%% mode: latex
%%% TeX-master: "learning_matching"
%%% End:



\section{Coalition Structure Prediction Methodology}
\label{sec:methodology}

\section{\label{sec:method}Methodology}

Each SIEM system uses its own RDL to define threat detection rules, and each RDL has its own schema.
For example, the Splunk SIEM uses the SPL to define its threat detection rules.
The task of understanding threat detection rules and recommending relevant MITRE ATT\&CK techniques (or sub-techniques) requires complex reasoning skills.
In the case of LLMs, this can be achieved with a technique called prompt chaining in which each task is divided into multiple sub-tasks in order to understand the complex reasoning behind the task.
Therefore, we employ a multi-phase architecture based on prompt chaining that leverages the power of LLMs to take a SIEM rule defined in any RDL and map it to relevant MITRE ATT\&CK techniques using the power of LLMs.
Our approach is based on the following intuitions:
\begin{itemize}[nosep,leftmargin=*]
    \item \textit{LLMs' implicit knowledge}: LLMs possess deep understanding of diverse RDLs. This enables them to interpret any rule, regardless of the RDL it is defined in, and convert it into comprehensible natural language text.
    \item \textit{LLMs' similarity comparison capability}: LLMs are adept at analyzing and comparing textual descriptions. 
    They can intelligently assess the similarity between two textual inputs to establish a meaningful connection.
\end{itemize}

\methodName has two main phases: (1) the rule to text translation phase, and (2) the MITRE ATT\&CK techniques recommendation phase.
These two phases in the pipeline include six key steps to determine relevant TTPs, as illustrated in Figure~\ref{fig:r2t}.

Although LLMs excel at translating SIEM rules into natural language, they often lack critical domain-specific contextual information related to IoCs in the rules.
To overcome this limitation, the \textit{rule to text translation} phase includes three steps: IoC extraction, contextual information retrieval, and natural language translation.

The workflow begins with the extraction of IoCs from the rules (for example, processes, log source, event codes, and file names) that the rule searches for in the logs (step (1)).In the next sstep a web search agent performs the task of obtaining additional contextual information about the IoCs discovered ((step 2)).
By incorporating this additional domain-specific information, the pipeline enhances the language translation, resulting in a more accurate and meaningful interpretation of SIEM rules.
The rule itself and the IoCs' contextual additional information from the previous stage are then used to translate the rule from RDL to natural language (step (3)).

The \textit{MITRE ATT\&CK techniques} recommendation phase of the pipeline includes the following three steps.
The rule in processed in data source identification step in which the probable origin of the data is identified (step (4)).
The description of the rule is then used to determine probable MITRE ATT\&CK techniques based on the implicit knowledge of the LLM (step (5)).
Finally, using chain-of-thought~\cite{wei2022chain} prompting, the most relevant techniques are extracted from the list of probable techniques (step (6)).
Each step of our method is further described in detail below.


% [bb=0 0 1440 900,width=1.43\linewidth,height=0.9\textwidth]
\begin{figure*}[htbp]
   \includegraphics[width=\textwidth]{Images/stages.jpg}
    
   \caption{An illustration of the different steps in \methodName.}
   \label{fig:stages}
\end{figure*} 

\subsection{IoC Extraction}
The context associated with a SIEM detection rule is crucial for its accurate interpretation and effective application. 
Obtaining this contextual understanding requires comprehensive analysis of the embedded IoCs in the SIEM rule.
In the first step, \methodName systematically identifies and extracts all IoCs, identifying the types of IoCs and their corresponding values that form the foundational elements of the detection rules. 
Leveraging the LLM's inherent understanding of rule structures and IoCs, we employ a zero-shot prompting approach for this task. 
Zero-shot prompting enables the direct extraction of IoCs from the rules without requiring extensive pre-training on specific datasets.

\noindent The result of this stage is a dictionary structure, where:
\begin{itemize}[nosep,leftmargin=*]
    \item Keys represent types of IoC, such as processes, files, IP addresses, and log sources.
    \item Values are lists containing specific IoC details, such as process names, file names, IP addresses, and log source identifiers.
\end{itemize}

In the example depicted in Figure~\ref{fig:stages}(a), the pipeline processes a rule for which relevant MITRE ATT\&CK techniques need to be recommended. 
The IoC extractor LLM produces a dictionary structure as output, organizing the IoCs in a structured format to support subsequent stages in the analysis pipeline. 



\subsection{Contextual Information Retrieval}
In this step, an LLM agent is employed to retrieve relevant information pertaining to the IoCs extracted from the rule.
A REACT agent~\cite{react} was used in this case to generate both reasoning traces and task-specific actions in an interleaved manner.
REACT agents interact with external tools to retrieve additional information that leads to more factual and reliable responses.
The LLM agent conducts a systematic search across web resources to gather additional contextual information for each IoC value present in the rule. 
This step addresses LLMS' lack of up-to-date knowledge or specialized domain expertise (which is critical to understanding the role and significance of the IoCs in the rule), without the need for retraining or fine-tuning.
Figure~\ref{fig:stages}(b) presents an example in which the rule includes the process name \texttt{soaphound.exe} as an IoC.
As can be seen, the web search results indicate that \texttt{soaphound.exe} is being used for active directory (AD) enumeration, which is important for the understanding of the attack. 

\subsection{Natural Language Translation}

The translation of detection rules into natural language textual descriptions fulfills three key objectives:
\begin{enumerate}[nosep,leftmargin=*]
    \item \textbf{Ensures that \methodName is format-agnostic}: It converts rules defined in various RDL formats into a generic, unstructured text format, ensuring compatibility with different SIEM systems, regardless of the specific rule format.
    \item \textbf{Provides contextual explanation}: It includes all relevant contextual information to produce a concise and comprehensible explanation of the rule.
    \item \textbf{Enhances the comprehension for LLMs}: It enables LLMs to more effectively compare the translated rule with descriptions in the MITRE ATT\&CK framework by providing a unified textual representation.
\end{enumerate}
To achieve these objectives, a zero-shot prompting technique is employed. 
The input to the LLM comprises two components:
\begin{itemize}
    \item \textbf{Syntactical information}: The rule itself, providing the structural and operational details.
    \item \textbf{Contextual information}: Details of the IoCs extracted from the rule, providing semantic insights into the rule's intent and function.
\end{itemize}
The LLM utilizes these inputs to generate a natural language textual description of the rule. 
This transformation not only ensures a more interpretable representation but also facilitates further steps of analysis and comparison, particularly in aligning the rule with MITRE ATT\&CK techniques and sub-techniques.



\subsection{Data Source or Mitigation Identification}
Identifying the most relevant data component or mitigation associated with the rule description in this step is critical for filtering out irrelevant MITRE ATT\&CK techniques (or sub-techniques) in subsequent steps of the pipeline.
In the MITRE ATT\&CK framework, data sources represent various categories of information that can be gathered from sensors or logs. 
These data sources include \textit{data components}, which are specific attributes or properties within a data source that are directly relevant to detecting a particular technique or sub-technique~. 
For example, in the context of the rule described in Figure~\ref{fig:stages}(a), the term \texttt{Endpoint.Processes} indicates that the activity is happening on an endpoint. 
Presence of the terms such as, \texttt{soaphound.exe}, \texttt{--buildcache}, \texttt{--certdump} and etc. indicate that the rule searches for command line execution of an executable named \texttt{soaphound.exe} with specific parameters. 
Therefore, the appropriate data source in this example is \textit{Command}, with the corresponding data component being \textit{Command Execution}.
Additionally, \textit{mitigations} are defined as categories of technologies or strategies that can prevent or reduce the impact of specific techniques or sub-techniques. 
The MITRE ATT\&CK framework explicitly establishes relationships between data components, mitigations, and techniques (or sub-techniques), enabling a systematic approach for identifying relevant elements.

To identify the most relevant data component or mitigation associated with a given rule description, we utilize agentic retrieval augmented generation (RAG), which incorporates an AI Agent-based implementation of the RAG framework.
Data from the MITRE ATT\&CK framework, specifically related to data components and mitigations, is stored in a vector database (e.g., ChromaDB). 
The process begins with the rule description from the previous stage, which serves as the input to the AI Agent. 
The LLM-powered agent automatically generates a search query tailored to retrieve relevant information from the RAG database.

For each query, the system retrieves the five most similar documents from the database, each containing contextual information about data components or mitigations. 
These documents are then utilized by the LLM agent to contextualize the rule description. 
By comparing the content of these retrieved documents with the rule description, the LLM agent determines and outputs the most relevant data component or mitigation along with a chain-of-thought as to why the data component or mitigation is related to the rule.


\subsection{Probable Technique Recommendation}

In this step, an LM agent is utilized to propose probable MITRE ATT\&CK techniques (and sub-techniques) that may be relevant to the description of the provided rule.
We used a REACT agent in this step as well to utilize both implicit and explicit knowledge during reasoning.
For explicit knowledge, the agent searches the MITRE ATT\&CK framework to obtain the list of probable techniques (and sub-techniques).
The natural language description of the rule from the previous step serves as input to the LLM agent.
The output of this stage consists of a list of JSON objects, each containing the MITRE technique ID, technique name, and technique description as seen in Figure~\ref{fig:stages}(c).

Throughout our experiments, we observed that as the number of recommendations ($k$) increases, both the framework's average recall and precision initially improve, however beyond a certain threshold of $k$, the %average 
precision begins to decline.
Based on these observations(please refer Table~\ref{tab:results3}), we selected a $k$-value of 11 to ensure a high recall.



\subsection{Relevant Technique Extraction}
In this step, \methodName refines the set of probable MITRE ATT\&CK techniques identified in the previous stage by eliminating irrelevant entries.
This step in the pipeline serves two primary purposes: (1) to enhance precision while maintaining recall achieved in previous step, and (2) to provide a clear rationale for the selection of the labels, ensuring transparency and interpretability of the mapping process.
This refinement process is grounded in the assumption that LLMs are effective for text similarity matching tasks.

The process comprises two key steps:
\begin{itemize}
    \item \textit{Rule-technique comparison}: The description of each technique in the set of probable techniques is compared with the rule description. 
    A chain-of-thought technique is then applied to elucidate the reasoning behind the association of each technique with the rule.
    \item \textit{Confidence calculation}: The generated chain-of-thought rationale for each technique (or sub-technique) is compared with the rule description to compute a relevance (or confidence) score, as done in prior work~\cite{freitas2024ai}.
    % \item \textbf{Reasoning}: \new{Add here the reasoning that it provides - explaining in NLP why it was selected...}
\end{itemize}

Techniques with higher confidence scores are deemed more relevant to the rule. 
Conversely, techniques with scores falling below a predefined threshold are excluded.
The techniques retained after this filtering step represent the most relevant techniques corresponding to the given rule's description. 


The chain-of-thought (CoT) rationale generated during the comparison of each rule to its probable technique is also provided as an output in this step.
This rationale offers a detailed natural language explanation, articulating why a particular technique is relevant to the given rule. 
Such explanations are highly valuable for security analysts, as they provide clear and transparent reasoning behind the mapping, enabling analysts to better understand and validate the association between the rule and the technique.
Other classification models proposed in previous works within this domain also suffer from the limitation of being black-box models, which lack the ability to provide clear reasoning or explanations. 
Unlike \methodName, these models fail to generate transparent, CoT rationales that explain why a particular rule is mapped to a specific technique, making them less interpretable and less useful for security analysts.

% \textbf{Framework picture} 
% \begin{figure}
	%     \centering
	%     \includegraphics[width=0.9\linewidth]{figures/framework.png}
	%     \caption{Caption}
	%     \label{fig:enter-label}
	% \end{figure}

%\textbf{General explanation. }
%At a high-level, our framework combines a dialogue-based intent model that identifies and ranks the set of agreements being discussed in the conversation between two players with a non-dialogue based value function to determine how  rationalizable a given agreement is for a given player. 
%
%
%\subsection{Generating set of agreements}
%
%\textbf{Problem.} Given a state $s$ in the game, determine the set of agreement $\calA_s$ in the diplomacy game. 
%
%
%\ak{
%	For each node in game graph, check if neighboring territory is occupied by a unit of another player. 
%	If yes, add all possible legitimate action pairs to set of agreements. 
%}
%
%
%
%\subsection{Preferences over Intent}
%
%\textbf{Problem.} Given the history of dialogues $\vec{d}$, states $\vec{s}$, actions $\vec{a}$, and the set of agreements $\calA_s$, determine a probability distribution over $\calA_s$. 
%
%
%The distribution implicitly encodes an \emph{incomplete} preference structure. 
%The incompleteness arises from the absence of conversation about certain potential agreements.
%
%
%Note that intents incorrect since the conversations can contain lies. 
%
%
%\ak{How to learn an incomplete preference from dialogue?}
%
%
%\ak{Explore. https://aclanthology.org/W19-5941.pdf}
%
%
%
%\subsection{Value of Coalitions}
%
%\paragraph{Problem.} Given a value model $V_p: S \rightarrow \mathbb{R}$ that maps every state to a real number for each player separately, determine a value function for a coalition ${\mathcal V}: S \times 2^N \rightarrow \mathbb{R}$ and the value a player gains by participating in a coalition, $V: S \times 2^N \times N \rightarrow \mathbb{R}$. 
%
%
%There are two types of models to compute this: transferable vs. non-transferable utility models. 
%We still need to determine which is best for our case. 
%
%
%\subsection{Rationalizability of Coalition Change}
%
%An agreement may be highly preferred for a player, say P1, in the subset of $\calA_s$ in which P1 is involved, but other players may not be interested to participate in this agreement.   
%The solution concept of rationalizability in game theory determines rational actions for a player accounting for their incomplete information.
%To determine whether an agreement is rationalizable, we must evaluate how preferable it is among the set of likely intents.
%
%\ak{Need for formalize rationalizability. I desire it to be a quantitative definition. 
%	This is tricky because most existing work studies rationalizability in games with single objectives. More investigation is needed to find/define the concept for multi-objective games or games with preferences.}
%
%
%\paragraph{Problem.} Given an incomplete preference over intents and a preference over coalitions, determine the rationalizability of an agreement for a player.  


\section{Experiments}

We evaluate the two stages of our proposed method separately. 
First, we outline the dataset employed for evaluation, and then  present the results for each of the two stages.

\subsection{Dataset}\label{sec:data}
We source previous Diplomacy games from WebDiplomacy\footnote{\url{https://webdiplomacy.net/}}, a multiplayer online implementation of Diplomacy. We consider a dataset of $140$ full-press games played over the standard Diplomacy map.  In order to calibrate our agreement detection classifier, we manually annotate five games from this dataset, randomly sampling from games with at least 250 total messages sent. This gives us a total of 16962 $(S, P_1, P_2, u)$ tuples over 1603 combinations of a state $S$ and players $P_1, P_2$. This dataset is highly imbalanced, with 444 (2.6\%) of all $(S, P_1, P_2, u)$ tuples having a coalition formed.

After we validate our usage of the agreement detection classifier, we then use it to label the remainder of the games with detected agreements. This resulting dataset consists of $415001$ total $(S, p_1, p_2, u)$ tuples. Of these tuples, $11008$ have agreements detected by our automatic method, $8344$ of which are upheld (i.e. the player played the agreed-upon move).

\subsection{Validating Agreement Detection Method}\label{sec:valid}

We use the manually-annotated data sample described in Section~\ref{sec:data} to test our agreement detection method, with an 80-20 train-test split. While this dataset is strongly imbalanced, we mitigate the impact of the dataset imbalance by only training on instances that pass our language model-based filter, which reduces our classifier training data to 1768 tuples, and by tuning a classification threshold to optimize F1-score on our training dataset. We benchmark three methods on this dataset: 
\begin{itemize}
    \item \textbf{GPT-4o}, which prompts a strong language model to directly identify units over which an agreement has been reached,
    \item \textbf{Classifier}, which trains a classifier on intent model distributions before and after dialogue over all $(S, P_1, P_2, u)$ tuples in the dataset, and
    \item \textbf{Hybrid}, our approach, which first filters using GPT-4o-parsed locations and player adjacency before training a classifier on the filtered data.
\end{itemize}

The results of our evaluation are in Table~\ref{tab:eval}. Our hybrid method outperforms both training a classifier on unfiltered intent data and prompting a strong language model on identifying units over which agreements have been reached. Extraction of coalition agreements from Diplomacy dialogue is a challenging task, due to the length of many dialogues in Diplomacy as well as the implicitness and fluctuating nature of negotiation over a multi-party dialogue. Fine-tuning more generally capable language models following the intent model formula in CICERO, in combination with more sophisticated parsers such as the one trained in \citep{wongkamjan2024more}, could yield even stronger performance improvements, which we leave to future work in this direction.

\begin{table}
    \centering
    \begin{tabular}{c|c|c|c}
    \toprule
    Method & F1 Score & Precision & Recall \\\midrule
    GPT-4o & 0.34 & 0.26 & 0.47 \\
    Classifier & 0.44 & 0.43 & 0.45 \\
    \textbf{Hybrid} & \textbf{0.55} & \textbf{0.63} & \textbf{0.48} \\\bottomrule
    \end{tabular}
    \caption{Classification metrics over the test dataset for our three methods. Hybrid methods outperform both purely language model-based and intent model distribution-based approaches at detecting whether an agreement has been reached over a specific unit.}
    \label{tab:eval}
\end{table}

\subsection{Evaluating Rationalizability Score}

The rationalizability score establishes a ranking of potential agreements for a unit within a specific game state. To evaluate the effectiveness of the score in predicting coalition structures, we analyze the rankings induced by the score on honored agreements in comparison to those of violated agreements.


We utilize both hand-labeled data and data labeled through the hybrid approach for the evaluation. 
We consider a total of $7434$ agreements labeled using the hybrid approach for evaluation.
For each agreement identified in the agreement detection stage, we generate a set of alternative agreements by sampling different orders for the units involved in the agreement.
The results of this evaluation for honored and violated agreements are presented in Table~\ref{tab:agreement}.
Given that the output of our model is a ranked list based on the rationalizability score, we employ two information retrieval metrics: mean reciprocal rank (MRR) \citep{craswell2009mean} and Brier score \citep{brier1950verification}.
The MRR is calculated using both the top-$1$ and top-$5$ ranked elements.

The ranking generated by the rationalizability score effectively differentiates between honored and violated agreements.
Our findings indicate that honored agreements typically receive lower ranks, while violated agreements tend to rank higher. 
This is observed through both the MRR and Brier scores.
When calculating the Brier score, we normalize the rationalizability scores, such that a score close to $1$ reflects that honored agreements usually have low ranks and violated agreements have higher ranks.
Notably, MRR scores that are close to $1$ in the top-$1$ case suggest that honored agreements are frequently assigned a rank of $0$, suggesting that this metric can very precisely recognize upheld coalitions.

We also compare our rationalizability score to a more conventional coalition formation prediction method, approximate Nash equilibrium (as estimated by the CICERO value model). We find that even when such approximate equilibrium-based methods are adapted for games with large state and action spaces, they remain inadequate for predicting coalition formation in such dynamic environments. Our R-Score yields a significantly higher MRR and a lower Brier score than the value model score in all cases.  This suggests that our rationalizability framework is significantly better at distinguishing between coalitions that are upheld and coalitions that are not upheld than Nash approximation-based predictions.



\begin{table}
\begin{centering}
\resizebox{0.5\textwidth}{!}{
\begin{tabular}{cc|ccc|ccc}
\toprule 
\multicolumn{2}{c}{} & \multicolumn{3}{c}{\textbf{Hand-Labelled}} & \multicolumn{3}{c}{\textbf{Hybrid}} \\
\multicolumn{1}{c}{Honored?} & \multicolumn{1}{c}{Metric} & \multicolumn{1}{c}{Value} & \multicolumn{1}{c}{R-Score@1} & \multicolumn{1}{c}{R-Score@5} & \multicolumn{1}{c}{Value} & \multicolumn{1}{c}{R-Score@1} & \multicolumn{1}{c}{R-Score@5} \\
\midrule
Yes & MRR ($\uparrow$) & 0.2842 & 0.9444 & \textbf{0.9722} & 0.3602 & 0.7416 & \textbf{0.8294} \\
No & MRR ($\downarrow$) & 0.2583 & \textbf{0.0} & 0.125 & 0.3682 & \textbf{0.2628} & 0.3354 \\
Yes & Brier ($\downarrow$) & 0.0802 & \textbf{0.0422} & \textbf{0.0422} & 0.0739 & \textbf{0.0311} & \textbf{0.0311} \\
No & Brier ($\uparrow$) & 0.7303 & \textbf{0.7494} & \textbf{0.7494} & 0.5706 & \textbf{0.6145} & \textbf{0.6145} \\
\bottomrule
\end{tabular}}
\end{centering}
\caption{Evaluation metrics for honored and violated agreements based on hand-labeled and hybrid datasets. The ranking induced by rationalizability score (RScore) on the set of agreements assigns lower ranks to honored agreements and higher ranks to violated agreements when compared to the ranking induced by Nash approximation-based predictions.}
\label{tab:agreement}
\end{table}

\section{Conclusion}

The detection of dynamic coalition structures is a key problem in understanding sequential interactions in strategic multi-agent environments. While many such environments use language, the study of coalition structure detection over natural language-based coordination is relatively understudied. This is compounded in settings like the board game Diplomacy, where players make decisions with incomplete information using dialogue-informed mental models of their opponents' future actions, and where relationships between players can shift drastically between turns as information is revealed.

Drawing from hypergame theory and the concept of subjective rationalizability, we propose a general method to dynamically predict coalition structures over sequential multi-agent interactions. In our method, we first extract detected agreements using the combination of a large language model-based parser and a specialized language model to predict player intents before and after the negotiation phase. We then compute the value of the agreement using a deep reinforcement-learning based value function, which we use in combination with player intents to compute the likelihood that each player will honor the agreement.

We validate the success of our method over sampled interactions between human Diplomacy players, using components of Meta's CICERO agent to compute player intents and action values. When compared to approximate Nash Equilibrium-based methods, our rationalizability score is significantly better at predicting the coalition structure at a given timestep. Our method can also generalize to other multi-agent, dialogue-based games, as long as sufficient human data exists upon which similar, game-specific models can be trained.

Extending coalition structure detection to natural language-based negotiation environments such as Diplomacy presents unique challenges in a setting where agents have incomplete information, negotiations are both multi-issue and multi-party, and where agents must reason over mental models of their opponents. However, for artificial agents to handle such complex environments properly, they must be capable of understanding the coalition dynamics of the environment at a given state. Our method and experiments serve as an important first step in this direction. We hope that future work will be able to extend our framework to new settings, including those with more complex negotiations and less existing domain-specific models, paving the way for agents that can reason over such information in deployment settings.

\FloatBarrier

\begin{acks}
This material is based upon work supported by the Defense Advanced Research Projects Agency (DARPA) under Agreement No. HR00112490410, the Army Research Lab under Agreement ARO W911NF-23-1-0317 and the Office of Naval Research under Agreement N00014-24-1-2097. We thank WebDiplomacy for supporting this research by providing access to online gameplay data.
\end{acks}

%%%%%%%%%%%%%%%%%%%%%%%%%%%%%%%%%%%%%%%%%%%%%%%%%%%%%%%%%%%%%%%%%%%%%%%%

%%% The acknowledgments section is defined using the "acks" environment
%%% (rather than an unnumbered section). The use of this environment
%%% ensures the proper identification of the section in the article
%%% metadata as well as the consistent spelling of the heading.

%%%%%%%%%%%%%%%%%%%%%%%%%%%%%%%%%%%%%%%%%%%%%%%%%%%%%%%%%%%%%%%%%%%%%%%%

%%% The next two lines define, first, the bibliography style to be
%%% applied, and, second, the bibliography file to be used.

\bibliographystyle{ACM-Reference-Format}
\bibliography{main}

%%%%%%%%%%%%%%%%%%%%%%%%%%%%%%%%%%%%%%%%%%%%%%%%%%%%%%%%%%%%%%%%%%%%%%%%

\end{document}

%%%%%%%%%%%%%%%%%%%%%%%%%%%%%%%%%%%%%%%%%%%%%%%%%%%%%%%%%%%%%%%%%%%%%%%%

