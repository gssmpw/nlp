%%%%%%%%%%%%%%%%%%%%%%%%%%%%%%%%%%%%%%%%%%%%%%%%%%%%%%%%%%%%%%%%%%%%%%%%

%%% LaTeX Template for AAMAS-2025 (based on sample-sigconf.tex)
%%% Prepared by the AAMAS-2025 Program Chairs based on the version from AAMAS-2025.

%%%%%%%%%%%%%%%%%%%%%%%%%%%%%%%%%%%%%%%%%%%%%%%%%%%%%%%%%%%%%%%%%%%%%%%%

%%% Start your document with the \documentclass command.


%%% == IMPORTANT ==
%%% Use the first variant below for the final paper (including auithor information).
%%% Use the second variant below to anonymize your submission (no authoir information shown).
%%% For further information on anonymity and double-blind reviewing,
%%% please consult the call for paper information
%%% https://aamas2025.org/index.php/conference/calls/submission-instructions-main-technical-track/

%%%% For anonymized submission, use this
% \documentclass[sigconf,anonymous]{aamas}
\documentclass[sigconf]{aamas}

%%%% For camera-ready, use this
%\documentclass[sigconf]{aamas}


%%% Load required packages here (note that many are included already).

\usepackage{balance} % for balancing columns on the final page

%%%%%%%%%%%%%%%%%%%%%%%%%%%%%%%%%%%%%%%%%%%%%%%%%%%%%%%%%%%%%%%%%%%%%%%%

%%% AAMAS-2025 copyright block (do not change!)

\setcopyright{ifaamas}
\acmConference[AAMAS '25]{Proc.\@ of the 24th International Conference
on Autonomous Agents and Multiagent Systems (AAMAS 2025)}{May 19 -- 23, 2025}
{Detroit, Michigan, USA}{A.~El~Fallah~Seghrouchni, Y.~Vorobeychik, S.~Das, A.~Nowe (eds.)}
\copyrightyear{2025}
\acmYear{2025}
\acmDOI{}
\acmPrice{}
\acmISBN{}


%%%%%%%%%%%%%%%%%%%%%%%%%%%%%%%%%%%%%%%%%%%%%%%%%%%%%%%%%%%%%%%%%%%%%%%%

%%% == IMPORTANT ==
%%% Use this command to specify your submission number.
%%% In anonymous mode, it will be printed on the first page.

\acmSubmissionID{<<submission id>>}

%%% Use this command to specify the title of your paper.

\title[Dynamic Coalition Structure Detection]{Dynamic Coalition Structure Detection in Natural Language-based Interactions}

%%% Provide names, affiliations, and email addresses for all authors.

\author{Abhishek N. Kulkarni*}
\affiliation{
  \institution{University of Texas at Austin}\thanks{* Equal contribution.}
  \city{Austin}
  \country{USA}}
\email{abhishek.kulkarni@austin.utexas.edu}

\author{Andy Liu*}
\affiliation{
  \institution{Carnegie Mellon University}
  \city{Pittsburgh}
  \country{USA}}
\email{andyliu@cs.cmu.edu}

\author{Jean-Rapha\"el Gaglione}
\affiliation{
   \institution{University of Texas at Austin}
  \city{Austin}
  \country{USA}}
\email{jr.gaglione@utexas.edu}

\author{Daniel Fried}
\affiliation{
  \institution{Carnegie Mellon University}
  \city{Pittsburgh}
  \country{USA}}
\email{dfried@cs.cmu.edu}

\author{Ufuk Topcu}
\affiliation{
   \institution{University of Texas at Austin}
  \city{Austin}
  \country{USA}}
\email{utopcu@utexas.edu}



%%% Use this environment to specify a short abstract for your paper.


\usepackage{todonotes}
\usepackage{acronym}
\usepackage{amsmath}
\usepackage{amsthm}
\usepackage{mathtools}
\usepackage{graphicx}
\usepackage{cite}
\usepackage{bbm}
% \usepackage{changes}
\usepackage[commandnameprefix=always]{changes}
\usepackage{booktabs} 
\usepackage{placeins}
%\usepackage{lineno}
%\linenumbers

\newacronym{rl}{RL}{Reinforcement Learning}
\newacronym{drl}{DRL}{Deep Reinforcement Learning}
\newacronym{mdp}{MDP}{Markov Decision Process}
\newacronym{ppo}{PPO}{Proximal Policy Optimization}
\newacronym{sac}{SAC}{Soft Actor-Critic}
\newacronym{epvf}{EPVF}{Explicit Policy-conditioned Value Function}
\newacronym{unf}{UNF}{Universal Neural Functional}
\newcommand{\R}{\mathbb{R}}
\newcommand{\Sym}{\mathbb{S}}
\newcommand{\Lam}{\bm{\Lambda}}
\newcommand{\Rext}{\mathbf{R} \cup \{ \infty \}}
\newcommand{\dd}{\,\mathrm{d}}
\newcommand{\ip}[2]{\langle #1, #2 \rangle}
\newcommand{\norm}[1]{\| #1 \|}
\newcommand{\logdet}[1]{\log \det( #1 )}

% bb
\newcommand{\bbE}{\mathbb{E}}

%bold
\newcommand{\bxi}{\mathbf{\xi}}
\newcommand{\btau}{\mathbf{\tau}}
\newcommand{\bs}{\mathbf{s}}
\newcommand{\bz}{\mathbf{z}}
\newcommand{\bX}{\mathbf{X}}
\newcommand{\bx}{\mathbf{x}}
\newcommand{\ba}{\mathbf{a}}
\newcommand{\bc}{\mathbf{c}}
\newcommand{\bh}{\mathbf{h}}
\newcommand{\bw}{\mathbf{w}}
\newcommand{\bg}{\mathbf{g}}
\newcommand{\bp}{\mathbf{p}}
\newcommand{\bq}{\mathbf{q}}
\newcommand{\by}{\mathbf{y}}
\newcommand{\bl}{\bm{\lambda}}
\newcommand{\be}{\bm{\varepsilon}}
\newcommand{\bt}{\bm{\theta}}
\newcommand{\bmu}{\bm{\mu}}
\newcommand{\bsigma}{\bm{\sigma}}
\newcommand{\bnu}{\bm{\nu}}
\newcommand{\bphi}{\bm{\phi}}
\newcommand{\T}{\mathrm{T}}

\newcommand{\bmm}{\mathbf{m}}
\newcommand{\bS}{\mathbf{S}}
\newcommand{\bH}{\mathbf{H}}
\newcommand{\bV}{\mathbf{V}}
\newcommand{\bA}{\mathbf{A}}
\newcommand{\bD}{\mathbf{D}}


\newcommand{\ind}{\mathbf{i}}
\newcommand{\bE}{\mathbf{E}}
\newcommand{\bu}{\mathbf{u}}
\newcommand{\bG}{\mathbf{\Gamma}}
\newcommand{\bW}{\mathbf{W}}

%cal
\newcommand{\cL}{\mathcal{L}}
\newcommand{\cJ}{\mathcal{J}}
\newcommand{\cC}{\mathcal{C}}
\newcommand{\cD}{\mathcal{D}}
\newcommand{\cH}{\mathcal{H}}
\newcommand{\cX}{\mathcal{X}}
\newcommand{\cY}{\mathcal{Y}}
\newcommand{\cK}{\mathcal{K}}
\newcommand{\cP}{\mathcal{P}}
\newcommand{\PM}{\mathcal{P}}
\newcommand{\cM}{\mathcal{M}}
\newcommand{\cN}{\mathcal{N}}
\newcommand{\cQ}{\mathcal{Q}}
\newcommand{\cF}{\mathcal{F}}
\newcommand{\bbS}{\mathbb{S}}

% names
\newcommand{\dist}{\mathrm{dist}}
\newcommand{\epi}{\mathrm{epi}}

% other
\newcommand{\W}{\mathrm{W}}
\newcommand{\V}{\mathrm{V}}

% vector/matrix notation
\newcommand{\vc}[1]{\bm{#1}}
\newcommand{\matr}[1]{\mathbf{#1}}

\newcommand{\Ent}{\mathrm{H}}

%operators
%operators
\DeclareMathOperator{\Div}{Div}
\DeclareMathOperator{\dom}{dom}
\DeclareMathOperator{\ran}{ran}
\DeclareMathOperator{\conv}{conv}
\DeclareMathOperator{\relint}{relint}
\DeclareMathOperator{\inter}{int}
\DeclareMathOperator{\boundary}{bdry}
\DeclareMathOperator{\moreau}{M}
\DeclareMathOperator{\hull}{H}
\DeclareMathOperator{\prox}{P}
\DeclareMathOperator{\proj}{proj}
\DeclareMathOperator{\tr}{tr}
\DeclareMathOperator{\diag}{diag}
\DeclareMathOperator{\Id}{I}
\DeclareMathOperator{\cl}{cl}
\DeclareMathOperator*{\argmin}{arg\,min}
\DeclareMathOperator*{\argmax}{arg\,max}
\DeclareMathOperator*{\limin}{lim~inf}
\DeclareMathOperator*{\mean}{mean}

\DeclareMathOperator{\KLop}{KL}
\newcommand{\myKL}{\mathbb{D}_{\KLop}\infdivx}
\newcommand{\myKLs}{{\mathbb{D}_{\KLop}}^*\infdivx}
\newcommand{\Breg}[1]{\mathbb{B}_{#1}\infdivx}

%\renewcommand{\algorithmicrequire}{\textbf{Input:}}
%\renewcommand{\algorithmicensure}{\textbf{Iterate:}}


%%%%%%%%%%%%%%%%%%%%%%%%%%%%%%%%
% THEOREMS
%%%%%%%%%%%%%%%%%%%%%%%%%%%%%%%%
\theoremstyle{plain}
\newtheorem{theorem}{Theorem}[section]
\newtheorem{proposition}[theorem]{Proposition}
\newtheorem{claim}[theorem]{Claim}
\newtheorem{lemma}[theorem]{Lemma}
\newtheorem{informaltheorem}{Informal Theorem}
\newtheorem{corollary}[theorem]{Corollary}
\newtheorem*{theorem-non}{Theorem}

\theoremstyle{definition}
\newtheorem{definition}[theorem]{Definition}
\newtheorem{assumption}[theorem]{Assumption}
\newtheorem{problem}{Problem}
\newtheorem{openproblem}{Open Problem}

\theoremstyle{remark}
\newtheorem{remark}[theorem]{Remark}
\newtheorem{example}[theorem]{Example}
\newtheorem{observation}{Observation}


\begin{abstract}
	
In strategic multi-agent sequential interactions, detecting dynamic coalition structures is crucial for understanding how self-interested agents coordinate to influence outcomes. However, natural-language-based interactions introduce unique challenges to coalition detection due to ambiguity over intents and difficulty in modeling players' subjective perspectives. We propose a new method that leverages recent advancements in large language models and game theory to predict dynamic multilateral coalition formation in Diplomacy, a strategic multi-agent game where agents negotiate coalitions using natural language. The method consists of two stages. The first stage extracts the set of agreements discussed by two agents in their private dialogue, by combining a parsing-based filtering function with a fine-tuned language model trained to predict player intents. In the second stage, we define a new metric using the concept of subjective rationalizability from hypergame theory to evaluate the expected value of an agreement for each player. We then compute this metric for each agreement identified in the first stage by assessing the strategic value of the agreement for both players and taking into account the subjective belief of one player that the second player would honor the agreement. We demonstrate that our method effectively detects potential coalition structures in online Diplomacy gameplay by assigning high values to agreements likely to be honored and low values to those likely to be violated. The proposed method provides foundational insights into coalition formation in multi-agent environments with language-based negotiation and offers key directions for future research on the analysis of complex natural language-based interactions between agents.
\end{abstract}

%%% The code below was generated by the tool at http://dl.acm.org/ccs.cfm.
%%% Please replace this example with code appropriate for your own paper.


%%% Use this command to specify a few keywords describing your work.
%%% Keywords should be separated by commas.

\keywords{Coalition Structures, Game Theory, Multi-Agent Cooperation, Large Language Models}

%%%%%%%%%%%%%%%%%%%%%%%%%%%%%%%%%%%%%%%%%%%%%%%%%%%%%%%%%%%%%%%%%%%%%%%%

%%% Include any author-defined commands here.

\newcommand{\BibTeX}{\rm B\kern-.05em{\sc i\kern-.025em b}\kern-.08em\TeX}

%%%%%%%%%%%%%%%%%%%%%%%%%%%%%%%%%%%%%%%%%%%%%%%%%%%%%%%%%%%%%%%%%%%%%%%%

\begin{document}

%%% The following commands remove the headers in your paper. For final
%%% papers, these will be inserted during the pagination process.

\pagestyle{fancy}
\fancyhead{}

%%% The next command prints the information defined in the preamble.

\maketitle

%%%%%%%%%%%%%%%%%%%%%%%%%%%%%%%%%%%%%%%%%%%%%%%%%%%%%%%%%%%%%%%%%%%%%%%%



\section{Introduction}
\label{sec:introduction}
\section{Introduction}
\label{sec:introduction}
The business processes of organizations are experiencing ever-increasing complexity due to the large amount of data, high number of users, and high-tech devices involved \cite{martin2021pmopportunitieschallenges, beerepoot2023biggestbpmproblems}. This complexity may cause business processes to deviate from normal control flow due to unforeseen and disruptive anomalies \cite{adams2023proceddsriftdetection}. These control-flow anomalies manifest as unknown, skipped, and wrongly-ordered activities in the traces of event logs monitored from the execution of business processes \cite{ko2023adsystematicreview}. For the sake of clarity, let us consider an illustrative example of such anomalies. Figure \ref{FP_ANOMALIES} shows a so-called event log footprint, which captures the control flow relations of four activities of a hypothetical event log. In particular, this footprint captures the control-flow relations between activities \texttt{a}, \texttt{b}, \texttt{c} and \texttt{d}. These are the causal ($\rightarrow$) relation, concurrent ($\parallel$) relation, and other ($\#$) relations such as exclusivity or non-local dependency \cite{aalst2022pmhandbook}. In addition, on the right are six traces, of which five exhibit skipped, wrongly-ordered and unknown control-flow anomalies. For example, $\langle$\texttt{a b d}$\rangle$ has a skipped activity, which is \texttt{c}. Because of this skipped activity, the control-flow relation \texttt{b}$\,\#\,$\texttt{d} is violated, since \texttt{d} directly follows \texttt{b} in the anomalous trace.
\begin{figure}[!t]
\centering
\includegraphics[width=0.9\columnwidth]{images/FP_ANOMALIES.png}
\caption{An example event log footprint with six traces, of which five exhibit control-flow anomalies.}
\label{FP_ANOMALIES}
\end{figure}

\subsection{Control-flow anomaly detection}
Control-flow anomaly detection techniques aim to characterize the normal control flow from event logs and verify whether these deviations occur in new event logs \cite{ko2023adsystematicreview}. To develop control-flow anomaly detection techniques, \revision{process mining} has seen widespread adoption owing to process discovery and \revision{conformance checking}. On the one hand, process discovery is a set of algorithms that encode control-flow relations as a set of model elements and constraints according to a given modeling formalism \cite{aalst2022pmhandbook}; hereafter, we refer to the Petri net, a widespread modeling formalism. On the other hand, \revision{conformance checking} is an explainable set of algorithms that allows linking any deviations with the reference Petri net and providing the fitness measure, namely a measure of how much the Petri net fits the new event log \cite{aalst2022pmhandbook}. Many control-flow anomaly detection techniques based on \revision{conformance checking} (hereafter, \revision{conformance checking}-based techniques) use the fitness measure to determine whether an event log is anomalous \cite{bezerra2009pmad, bezerra2013adlogspais, myers2018icsadpm, pecchia2020applicationfailuresanalysispm}. 

The scientific literature also includes many \revision{conformance checking}-independent techniques for control-flow anomaly detection that combine specific types of trace encodings with machine/deep learning \cite{ko2023adsystematicreview, tavares2023pmtraceencoding}. Whereas these techniques are very effective, their explainability is challenging due to both the type of trace encoding employed and the machine/deep learning model used \cite{rawal2022trustworthyaiadvances,li2023explainablead}. Hence, in the following, we focus on the shortcomings of \revision{conformance checking}-based techniques to investigate whether it is possible to support the development of competitive control-flow anomaly detection techniques while maintaining the explainable nature of \revision{conformance checking}.
\begin{figure}[!t]
\centering
\includegraphics[width=\columnwidth]{images/HIGH_LEVEL_VIEW.png}
\caption{A high-level view of the proposed framework for combining \revision{process mining}-based feature extraction with dimensionality reduction for control-flow anomaly detection.}
\label{HIGH_LEVEL_VIEW}
\end{figure}

\subsection{Shortcomings of \revision{conformance checking}-based techniques}
Unfortunately, the detection effectiveness of \revision{conformance checking}-based techniques is affected by noisy data and low-quality Petri nets, which may be due to human errors in the modeling process or representational bias of process discovery algorithms \cite{bezerra2013adlogspais, pecchia2020applicationfailuresanalysispm, aalst2016pm}. Specifically, on the one hand, noisy data may introduce infrequent and deceptive control-flow relations that may result in inconsistent fitness measures, whereas, on the other hand, checking event logs against a low-quality Petri net could lead to an unreliable distribution of fitness measures. Nonetheless, such Petri nets can still be used as references to obtain insightful information for \revision{process mining}-based feature extraction, supporting the development of competitive and explainable \revision{conformance checking}-based techniques for control-flow anomaly detection despite the problems above. For example, a few works outline that token-based \revision{conformance checking} can be used for \revision{process mining}-based feature extraction to build tabular data and develop effective \revision{conformance checking}-based techniques for control-flow anomaly detection \cite{singh2022lapmsh, debenedictis2023dtadiiot}. However, to the best of our knowledge, the scientific literature lacks a structured proposal for \revision{process mining}-based feature extraction using the state-of-the-art \revision{conformance checking} variant, namely alignment-based \revision{conformance checking}.

\subsection{Contributions}
We propose a novel \revision{process mining}-based feature extraction approach with alignment-based \revision{conformance checking}. This variant aligns the deviating control flow with a reference Petri net; the resulting alignment can be inspected to extract additional statistics such as the number of times a given activity caused mismatches \cite{aalst2022pmhandbook}. We integrate this approach into a flexible and explainable framework for developing techniques for control-flow anomaly detection. The framework combines \revision{process mining}-based feature extraction and dimensionality reduction to handle high-dimensional feature sets, achieve detection effectiveness, and support explainability. Notably, in addition to our proposed \revision{process mining}-based feature extraction approach, the framework allows employing other approaches, enabling a fair comparison of multiple \revision{conformance checking}-based and \revision{conformance checking}-independent techniques for control-flow anomaly detection. Figure \ref{HIGH_LEVEL_VIEW} shows a high-level view of the framework. Business processes are monitored, and event logs obtained from the database of information systems. Subsequently, \revision{process mining}-based feature extraction is applied to these event logs and tabular data input to dimensionality reduction to identify control-flow anomalies. We apply several \revision{conformance checking}-based and \revision{conformance checking}-independent framework techniques to publicly available datasets, simulated data of a case study from railways, and real-world data of a case study from healthcare. We show that the framework techniques implementing our approach outperform the baseline \revision{conformance checking}-based techniques while maintaining the explainable nature of \revision{conformance checking}.

In summary, the contributions of this paper are as follows.
\begin{itemize}
    \item{
        A novel \revision{process mining}-based feature extraction approach to support the development of competitive and explainable \revision{conformance checking}-based techniques for control-flow anomaly detection.
    }
    \item{
        A flexible and explainable framework for developing techniques for control-flow anomaly detection using \revision{process mining}-based feature extraction and dimensionality reduction.
    }
    \item{
        Application to synthetic and real-world datasets of several \revision{conformance checking}-based and \revision{conformance checking}-independent framework techniques, evaluating their detection effectiveness and explainability.
    }
\end{itemize}

The rest of the paper is organized as follows.
\begin{itemize}
    \item Section \ref{sec:related_work} reviews the existing techniques for control-flow anomaly detection, categorizing them into \revision{conformance checking}-based and \revision{conformance checking}-independent techniques.
    \item Section \ref{sec:abccfe} provides the preliminaries of \revision{process mining} to establish the notation used throughout the paper, and delves into the details of the proposed \revision{process mining}-based feature extraction approach with alignment-based \revision{conformance checking}.
    \item Section \ref{sec:framework} describes the framework for developing \revision{conformance checking}-based and \revision{conformance checking}-independent techniques for control-flow anomaly detection that combine \revision{process mining}-based feature extraction and dimensionality reduction.
    \item Section \ref{sec:evaluation} presents the experiments conducted with multiple framework and baseline techniques using data from publicly available datasets and case studies.
    \item Section \ref{sec:conclusions} draws the conclusions and presents future work.
\end{itemize}



\section{Problem Formulation}
\label{sec:problem}
\section{Problem Formulation} \label{sec:probdef}

This section formally defines the problem of restoring a given pruned network with only using its original pretrained CNN in a way free of data and fine-tuning.



% Unlike many existing works utilize data for identifying unimportant filters as well as fine-tuning to this end, we cannot evaluate the filter importance by data-dependent values like activation maps (\textit{a.k.a.} channels) as our focus in this paper is not to use any training data. Thus, in our problem setting, we can only exploit the values of filters in the original network, and thereby have to make some changes in the remaining filters of the pruned network so that the network can return the output not too much different from the original one.

% No matter how much we carefully select unimportant filters to be pruned, some kinds of retraining process appears inevitable as done by the most existing works to this end. However, since our focus in this paper is not to use any training data, we cannot evaluate the importance of filters by data-dependent values like activation maps (\textit{a.k.a.} channels). 

% To this end, they not only use a careful criterion (\textit{e.g.}, L1-norm), but also fine-tune the network using the original data.
% Most of filter pruning methods try to select filters to be pruned prudently so that pruned network's output be similar to the original network's. To this end, they prune the unimportant filters and then fine-tune the pruned network with using the train data. 

% How can we restore the the pruned networks without any data? In other words, it implies that we cannot use any data-driven values(i.e., activation maps) and we can only exploit the values of original filters. In that case, the only thing we can do maybe changing the weights of remained filters appropriately not to amplify the difference between pruned and unpruned network's outputs through the information of original filters.

\begin{figure*}[t]
	\centering
    \subfigure[\label{fig:matrix:a}Pruning matrix]{\hspace{6mm}\includegraphics[width=0.35\columnwidth]{./figure/LBYL_figure_2_1.pdf}\hspace{6mm}} 
    \subfigure[\label{fig:matrix:b}Delivery matrix for LBYL]{\hspace{6mm}\includegraphics[width=0.35\columnwidth]{./figure/LBYL_figure_2_2.pdf}\hspace{6mm}}
    \subfigure[\label{fig:matrix:c}Delivery matrix for one-to-one]{\hspace{9mm}\includegraphics[width=0.35\columnwidth]{./figure/LBYL_figure_2_3.pdf}\hspace{9mm}} 
    \caption{Comparison between pruning matrix and delivery matrix, where the $4$-th and $6$-th filters are being pruned among $6$ original filters}
	\label{fig:matrix}
	\vspace{-2mm}
\end{figure*}



\subsection{Filter Pruning in a CNN}
Consider a given CNN to be pruned with $L$ layers, where each $\ell$-th layer starts with a convolution operation on its input channels, which are the output of the previous $(\ell-1)$-th layer $\mathbf{A}^{(\ell-1)}$, with the group of convolution filters $\mathbf{W}^{{(\ell)}}$ and thereby obtain the set of \textit{feature maps} $\mathbf{Z}^{(\ell)}$ as follows:
\begin{equation}
\boldsymbol{\mathbf{Z}}^{(\ell)} = {\mathbf{A}^{(\ell-1)} \circledast {\mathbf{W}}^{(\ell)}},
\nonumber
\end{equation}
where $\circledast$ represents the convolution operation. Then, this convolution process is normally followed by a batch normalization (BN) process and an activation function such as ReLU, and the $\ell$-th layer finally outputs an \textit{activation map} $\mathbf{A}^{(\ell)}$ to be sent to the $(\ell+1)$-th layer through this sequence of procedures as:
\begin{equation}
\mathbf{A}^{(\ell)} = \F(\N(\mathbf{Z}^{(\ell)})),
\nonumber
\end{equation}
where $\F(\cdot)$ is an activation function and $\N(\cdot)$ is a BN procedure.

Note that all of $\mathbf{W}^{(\ell)}$, $\mathbf{Z}^{(\ell)}$, and $\mathbf{A}^{(\ell)}$ are tensors such that: $\mathbf{W}^{(\ell)} \in \mathbb{R}^{m \times n \times k \times k}$ and $\mathbf{Z}^{(\ell)},\mathbf{A}^{(\ell)} \in \mathbb{R}^{m \times w \times h}$, where (1) $m$ is the number of filters, which also equals the number of output activation maps, (2) $n$ is the number of input activation maps resulting from the $(\ell-1)$-th layer, (3) $k \times k$ is the size of each filter, and (4) $w \times h$ is the size of each output channel for the $\ell$-th layer.

\smalltitle{Filter pruning as n-mode product}
When filter pruning is performed at the $\ell$-th layer, all three tensors above are consequently modified to their \textit{damaged} versions, namely $\mathbf{\Tilde{W}}^{(\ell)}$, $\mathbf{\Tilde{Z}}^{(\ell)}$, and $\mathbf{\Tilde{A}}^{(\ell)}$, respectively, in a way that: $\mathbf{\Tilde{W}}^{(\ell)} \in \mathbb{R}^{t \times n \times k \times k}$ and $\mathbf{\Tilde{Z}}^{(\ell)},\mathbf{\Tilde{A}}^{(\ell)} \in \mathbb{R}^{t \times w \times h}$, where $t$ is the number of remaining filters after pruning and therefore $t < m$. Mathematically, the tensor of remaining filters, \textit{i.e.}, $\mathbf{\Tilde{W}}^{(\ell)}$, is obtained by the \textit{$1$-mode product} \cite{DBLP:journals/siamrev/KoldaB09} of the tensor of the original filters $\mathbf{W}^{(\ell)}$ with a \textit{pruning matrix} $\boldsymbol{\S} \in \mathbb{R}^{m \times t}$ (see Figure \ref{fig:matrix:a})
as follows:
\begin{eqnarray}\begin{split}\label{eq:pruning}
\mathbf{\Tilde{W}}^{(\ell)} = {\mathbf{W}}^{(\ell)} \times_{1} {\boldsymbol{\S}}^{T},\text{where }\boldsymbol{\S}_{i,k} = 
  \begin{cases} 
   1~ \text{if } i = i'_k \\
   0~ \text{otherwise}
  \end{cases} \\
  \text{s.t. } i, i'_k \in [1, m] 
  \text{ and } k \in [1, t].
  \end{split}
\end{eqnarray}
  
By Eq. (\ref{eq:pruning}), each $i'_k$-th filter is not pruned and the other $(m-t)$ filters are completely removed from $\mathbf{W}^{(\ell)}$ to be $\mathbf{\Tilde{W}}^{(\ell)}$.

This reduction at the $\ell$-th layer causes another reduction for each filter of the $(\ell+1)$-th layer so that $\mathbf{W}^{(\ell+1)}$ is now modified to $\mathbf{\Tilde{W}}^{(\ell+1)} \in \mathbb{R}^{m' \times t \times k' \times k'}$, where $m'$ is the number of filters of size $k' \times k'$ in the $(\ell+1)$-th layer. Due to this series of information losses, the resulting feature map (\textit{i.e.}, $\mathbf{Z}^{(\ell+1)}$) would severely be damaged to be $\mathbf{\Tilde{Z}}^{(\ell+1)}$ as shown below:
\begin{equation}
{\mathbf{\Tilde{Z}}}^{{(\ell+1)}} = \mathbf{\Tilde{A}}^{(\ell)} \circledast {\mathbf{\Tilde{W}}}^{(\ell+1)}~~~\not\approx~~~\mathbf{Z}^{(\ell+1)}
\label{eq:eq}\nonumber
\end{equation}
The shape of $\mathbf{\Tilde{Z}}^{(\ell+1)}$ remains the same unless we also prune filters for the $(\ell+1)$-th layer. If we do so as well, the loss of information will be accumulated and further propagated to the next layers. Note that $\mathbf{\Tilde{W}}^{(\ell+1)}$ can also be represented by the \textit{$2$-mode product} \cite{DBLP:journals/siamrev/KoldaB09} of $\mathbf{W}^{(\ell+1)}$ with the transpose of the same matrix $\boldsymbol{\S}$ as:
\begin{equation} \label{eq:pruning2}
\mathbf{\Tilde{W}}^{(\ell+1)} = {\mathbf{W}}^{(\ell+1)} \times_{2} {\boldsymbol{\S}^T}
\end{equation}




\subsection{Problem of Restoring a Pruned Network without Data and Fine-Tuning}
As mentioned earlier, our goal is to restore a pruned and thus damaged CNN without using any data and re-training process, which implies the following two facts. First, we have to use a pruning criterion exploiting only the values of filters themselves such as L1-norm. In this sense, this paper does not focus on proposing a sophisticated pruning criterion but intends to recover a network somehow pruned by such a simple criterion. Secondly, since we cannot make appropriate changes in the remaining filters by fine-tuning, we should make the best use of the original network and identify how the information carried by a pruned filter can be delivered to the remaining filters.

% For brevity, we formulate our problem here with respect to a specific layer, say $\ell$, and then it can trivially be generalized for the entire network. 
\smalltitle{Delivery matrix}
In order to represent the information to be delivered to the preserved filters, let us first think of what the pruning matrix $\boldsymbol{\S}$ means. As defined in Eq. (\ref{eq:pruning}) and shown in Figure \ref{fig:matrix:a}, each row is either a zero vector (for filters being pruned) or a one-hot vector (for remaining filters), which is intended only to remove filters without delivering any information. Intuitively, we can transform this pruning matrix into a \textit{delivery matrix} that carries information for filters being pruned by replacing some meaningful values with some of the zero values therein. Once we find such an \textit{ideal} $\boldsymbol{\S^*}$, we can plug it into $\boldsymbol{\S}$ of Eq. (\ref{eq:pruning2}) to deliver missing information propagated from the $\ell$-th layer to the filters at the $(\ell+1)$-th layer, which will hopefully generate an approximation $\mathbf{\hat{Z}}^{(\ell+1)}$ close to the original feature map as follows:
\begin{equation} \label{eq:fmap_approx}
{\mathbf{\hat{Z}}}^{{(\ell+1)}} = {\mathbf{\Tilde{A}}^{(\ell)} \circledast ({\mathbf{W}}^{(\ell+1)} \times_{2} {\boldsymbol{\S^*}^T})}
~~~\approx~~~\mathbf{Z}^{(\ell+1)}
\end{equation}
Thus, using the delivery matrix $\boldsymbol{\mathcal{S^*}}$, the information loss caused by pruning at each layer is recovered at the feature map of the next layer.

\smalltitle{Problem statement}
Given a pretrained CNN, our problem aims to find the best delivery matrix $\boldsymbol{\mathcal{S^*}}$ for each layer without any data and training process such that the following \textit{reconstruction error} is minimized:
\begin{equation}
\sum\limits_{i = 1}^{m'}\|{{\mathbf{Z}}_{i}^{{(\ell+1)}}-{\hat{\mathbf{Z}}}_{i}^{{(\ell+1)}}}\|_1,
\label{eq:goal}
\end{equation}
where ${\mathbf{Z}}_i^{{(\ell+1)}}$ and ${\hat{\mathbf{Z}}}_i^{{(\ell+1)}}$ indicate the $i$-th original feature map and its corresponding approximation, respectively, out of $m'$ filters in the $(\ell+1)$-th layer. Note that what is challenging here is that we cannot obtain the activation maps in $\mathbf{A}^{(\ell)}$ and $\mathbf{\Tilde{A}}^{(\ell)}$ without data as they are data-dependent values.

% = \sum\limits_{i = 1}^{m'}\|{{\mathbf{Z}}_{i}^{{(\ell+1)}}-{\mathbf{\Tilde{A}}^{(\ell)} \circledast ({\mathbf{W}}^{(\ell+1)} \times_{2} {\boldsymbol{\mathcal{S^*}^T}})}}\|_{1}


% Our goal is finding the approximation matrix $\boldsymbol{\mathcal{S}}$ to minimize the reconstruction error between the pruned model and the original model without any data, and effectively deliver missing information for pruned filters using this approximation matrix


% $\testit{s}$,which can be represented as below.

% \begin{equation}
% \boldsymbol{\mathcal{S}} =  \underset{{\boldsymbol{\mathcal{S}}}}{\mathrm{argmin}} \sum\limits_{{i} = 1}^{m_{\ell+1}} \|{{\mathbf{Z}}_{i,:,:}^{{(\ell+1)}}-{\hat{\mathbf{Z}}}_{i,:,:}^{{(\ell+1)}}}\|_{1} 
% \label{eq:eq1}
% \end{equation}



% Let us first recall that the ultimate goal of network pruning is to make the output of a pruned network as close as possible to that of its original network. Unlike many existing pruning methods, our focus is not to use any training data at all for the entire pruning and recovery process, and this implies the following two facts. First, we cannot evaluate the filter importance by data-dependent values like activation values or gradients, but have to use a pruning criterion exploiting only the values of filters themselves such as L1-norm. Furthermore, instead of fine-tuning with data, the only thing we can do for the pruned network is to make appropriate changes in the remaining filters by identifying some relationships between pruned filters and the other preserved ones without any support from data. Based on this intuition, this section mathematically and generally defines the problem of restoring a pruned neural network in a manner free of data and fine-tuning.


% Thus, we make approximation matrix $\testit{s}$ $\in$ $\mathbb{R}^{m_{\ell} \times t_{\ell}}$ with relationship between the pruned filter and preserved filters in $\ell$-th layer and then apply it to the original filters in $(\ell+1)$-th layer to compensate for pruned feature maps $\boldsymbol{\hat{\mathbf{Z}}}^{{(\ell+1)}}$ as shown below.
% (\textit{i.e.}, Let $\hat{\mathbf{W}}^{(\ell+1)}$ be ${\mathbf{W}}^{(\ell+1)}$ $\times_2$ ${{\textit{s}}} $, where $\times_2$ is 2-mode matrix product) 

% \begin{equation}
% \mathbf{Z}^{(\ell+1)} = {\mathbf{A}}^{(\ell)} \circledast {\mathbf{W}}^{(\ell+1)}
% \approx {\hat{\mathbf{A}}^{(\ell)} \circledast ({\mathbf{W}}^{(\ell+1)} \times_{2} {{s}}) = {\hat{\mathbf{Z}}}^{{(\ell+1)}}}
% \label{eq:eq}\nonumber
% \end{equation}




% For a Convolutional Neural Network (CNN) with $L$ layers, we denote $\mathcal{A}{^{(\ell-1)}}$ $\in$ $\mathbb{R}^{n_{\ell -1 } \times h_{\ell -1} \times w_{\ell -1}}$ is activation maps at $\ell-1$-th layer, where $n_{\ell -1}$, $h_{\ell -1}$ and $w_{\ell -1}$ are the number of channels, height and width in activation maps, respectively. and we denote $\mathbf{W}^{{(\ell )}}$ $\in$  $\mathbb{R}^{m_{\ell} \times n_{\ell -1}\times k \times k}$ is covolution filters in $\ell$-th layer,where $m_{\ell}$, $n_{\ell-1}$ and $k$ are the number of filters, number of channels and kernel size, respectively. Trough the convolution operation using activation map $\mathcal{A}{^{(\ell-1)}}$ and convolution filter $\mathbf{W}^{{(\ell)}}$ in $\ell$-th layer, the feature maps $\boldsymbol{\mathbf{Z}}^{{(\ell)}}$ $\in$ $\mathbb{R}^{m_{\ell} \times h_{\ell+1} \times w_{\ell+1}}$ is computed as shown as below.


% \begin{equation}
% \boldsymbol{\mathbf{Z}}^{(\ell)} = {\mathcal{A}^{(\ell-1)} \circledast {\mathbf{W}}^{(\ell)}}
% \label{eq:eq1}\nonumber
% \end{equation}
% where $\circledast$ is convolution operation.

% and the feature maps passed through the BN and ReLU layer are activation maps $\mathcal{A}{^{(\ell)}}$ $\in$ $\mathbb{R}^{m_{{\ell}} \times h_{\ell+1} \times w_{\ell+1}} $ in $\ell$-th layer as shown as below.

% \begin{equation}
% \mathcal{A}^{(\ell)} = \mathcal{F}(\mathbf{Z}^{(\ell)} \circledast {\mathbf{W}}^{(\ell)})
% \label{eq:eq2}\nonumber
% \end{equation}
% where $\mathcal{F}$ is the function that implement batch normalization and non-linear activation(\textit{e.g.}, ReLU).

% \smalltitle{Filter Pruning}
% If the filter pruning is performed in $\ell$-th layer, the shape of original filters $\mathbf{W}^{{(\ell)}}$ $\in$ $\mathbb{R}^{m_{\ell} \times n_{\ell-1}\times k \times k}$ is modified to ${\hat {\mathbf{W}}^{(\ell)}}$ $\in$ $\mathbb{R}^{t_{\ell} \times n_{\ell-1}\times k \times k}$, where $t_{\ell}$ $<$ $m_{\ell}$ by pruning criterion. Therefore, the pruned activation maps ${\hat {\mathcal{A}}}{^{({\ell+1})}}$ $\in$ $\mathbb{R}^{t_{{\ell}} \times h_{{\ell+2}} \times w_{{\ell+2}}}$ in (${\ell+1}$)-th layer is computed as below.

% \begin{equation}
% \mathbf{\hat{A}}^{(l+1)} = \mathcal{F}({\mathbf{A}^{(\ell)} \circledast {\mathbf{\hat{W}}}^{(\ell+1)}})
% \label{eq:eq3}\nonumber
% \end{equation}

% Moreover, corresponding channels of each filters in ($\ell +1$)-th layer are sequentially removed. As a result, shape of original filters $\mathbf{W}^{{(\ell+1)}}$ $\in$ $\mathbb{R}^{m_{\ell+1} \times m_{\ell}\times k \times k}$ in ($\ell+1$)-th layer is changed to  ${\hat {\mathbf{W}}^{(\ell+1)}}$ $\in$ $\mathbb{R}^{m_{\ell+1} \times t_{\ell}\times k \times k}$. Although feature maps ${\hat{\mathbf{Z}}}^{{(\ell+1)}}$ $\in$ $\mathbb{R}^{m_{\ell+1} \times h_{\ell+2} \times w_{\ell+2}}$ in ($\ell+1$)-th layer after pruning have same shape with original feature maps ${\mathbf{Z}}^{{(\ell+1)}}$ $\in$ $\mathbb{R}^{m_{\ell+1} \times h_{\ell+2} \times w_{\ell+2}}$, the pruned feature maps $\boldsymbol{\hat{\mathbf{Z}}}^{{(\ell+1)}}$ are damaged.



\section{Background: Hypergame Theory}
\label{sec:background}
\section{Background}\label{sec:backgrnd}

\subsection{Cold Start Latency and Mitigation Techniques}

Traditional FaaS platforms mitigate cold starts through snapshotting, lightweight virtualization, and warm-state management. Snapshot-based methods like \textbf{REAP} and \textbf{Catalyzer} reduce initialization time by preloading or restoring container states but require significant memory and I/O resources, limiting scalability~\cite{dong_catalyzer_2020, ustiugov_benchmarking_2021}. Lightweight virtualization solutions, such as \textbf{Firecracker} microVMs, achieve fast startup times with strong isolation but depend on robust infrastructure, making them less adaptable to fluctuating workloads~\cite{agache_firecracker_2020}. Warm-state management techniques like \textbf{Faa\$T}~\cite{romero_faa_2021} and \textbf{Kraken}~\cite{vivek_kraken_2021} keep frequently invoked containers ready, balancing readiness and cost efficiency under predictable workloads but incurring overhead when demand is erratic~\cite{romero_faa_2021, vivek_kraken_2021}. While these methods perform well in resource-rich cloud environments, their resource intensity challenges applicability in edge settings.

\subsubsection{Edge FaaS Perspective}

In edge environments, cold start mitigation emphasizes lightweight designs, resource sharing, and hybrid task distribution. Lightweight execution environments like unikernels~\cite{edward_sock_2018} and \textbf{Firecracker}~\cite{agache_firecracker_2020}, as used by \textbf{TinyFaaS}~\cite{pfandzelter_tinyfaas_2020}, minimize resource usage and initialization delays but require careful orchestration to avoid resource contention. Function co-location, demonstrated by \textbf{Photons}~\cite{v_dukic_photons_2020}, reduces redundant initializations by sharing runtime resources among related functions, though this complicates isolation in multi-tenant setups~\cite{v_dukic_photons_2020}. Hybrid offloading frameworks like \textbf{GeoFaaS}~\cite{malekabbasi_geofaas_2024} balance edge-cloud workloads by offloading latency-tolerant tasks to the cloud and reserving edge resources for real-time operations, requiring reliable connectivity and efficient task management. These edge-specific strategies address cold starts effectively but introduce challenges in scalability and orchestration.

\subsection{Predictive Scaling and Caching Techniques}

Efficient resource allocation is vital for maintaining low latency and high availability in serverless platforms. Predictive scaling and caching techniques dynamically provision resources and reduce cold start latency by leveraging workload prediction and state retention.
Traditional FaaS platforms use predictive scaling and caching to optimize resources, employing techniques (OFC, FaasCache) to reduce cold starts. However, these methods rely on centralized orchestration and workload predictability, limiting their effectiveness in dynamic, resource-constrained edge environments.



\subsubsection{Edge FaaS Perspective}

Edge FaaS platforms adapt predictive scaling and caching techniques to constrain resources and heterogeneous environments. \textbf{EDGE-Cache}~\cite{kim_delay-aware_2022} uses traffic profiling to selectively retain high-priority functions, reducing memory overhead while maintaining readiness for frequent requests. Hybrid frameworks like \textbf{GeoFaaS}~\cite{malekabbasi_geofaas_2024} implement distributed caching to balance resources between edge and cloud nodes, enabling low-latency processing for critical tasks while offloading less critical workloads. Machine learning methods, such as clustering-based workload predictors~\cite{gao_machine_2020} and GRU-based models~\cite{guo_applying_2018}, enhance resource provisioning in edge systems by efficiently forecasting workload spikes. These innovations effectively address cold start challenges in edge environments, though their dependency on accurate predictions and robust orchestration poses scalability challenges.

\subsection{Decentralized Orchestration, Function Placement, and Scheduling}

Efficient orchestration in serverless platforms involves workload distribution, resource optimization, and performance assurance. While traditional FaaS platforms rely on centralized control, edge environments require decentralized and adaptive strategies to address unique challenges such as resource constraints and heterogeneous hardware.



\subsubsection{Edge FaaS Perspective}

Edge FaaS platforms adopt decentralized and adaptive orchestration frameworks to meet the demands of resource-constrained environments. Systems like \textbf{Wukong} distribute scheduling across edge nodes, enhancing data locality and scalability while reducing network latency. Lightweight frameworks such as \textbf{OpenWhisk Lite}~\cite{kravchenko_kpavelopenwhisk-light_2024} optimize resource allocation by decentralizing scheduling policies, minimizing cold starts and latency in edge setups~\cite{benjamin_wukong_2020}. Hybrid solutions like \textbf{OpenFaaS}~\cite{noauthor_openfaasfaas_2024} and \textbf{EdgeMatrix}~\cite{shen_edgematrix_2023} combine edge-cloud orchestration to balance resource utilization, retaining latency-sensitive functions at the edge while offloading non-critical workloads to the cloud. While these approaches improve flexibility, they face challenges in maintaining coordination and ensuring consistent performance across distributed nodes.




\section{Coalition Structure Prediction Methodology}
\label{sec:methodology}
\section{Research Methodology}~\label{sec:Methodology}

In this section, we discuss the process of conducting our systematic review, e.g., our search strategy for data extraction of relevant studies, based on the guidelines of Kitchenham et al.~\cite{kitchenham2022segress} to conduct SLRs and Petersen et al.~\cite{PETERSEN20151} to conduct systematic mapping studies (SMSs) in Software Engineering. In this systematic review, we divide our work into a four-stage procedure, including planning, conducting, building a taxonomy, and reporting the review, illustrated in Fig.~\ref{fig:search}. The four stages are as follows: (1) the \emph{planning} stage involved identifying research questions (RQs) and specifying the detailed research plan for the study; (2) the \emph{conducting} stage involved analyzing and synthesizing the existing primary studies to answer the research questions; (3) the \emph{taxonomy} stage was introduced to optimize the data extraction results and consolidate a taxonomy schema for REDAST methodology; (4) the \emph{reporting} stage involved the reviewing, concluding and reporting the final result of our study.

\begin{figure}[!t]
    \centering
    \includegraphics[width=1\linewidth]{fig/methodology/searching-process.drawio.pdf}
    \caption{Systematic Literature Review Process}
    \label{fig:search}
\end{figure}

\subsection{Research Questions}
In this study, we developed five research questions (RQs) to identify the input and output, analyze technologies, evaluate metrics, identify challenges, and identify potential opportunities. 

\textbf{RQ1. What are the input configurations, formats, and notations used in the requirements in requirements-driven
automated software testing?} In requirements-driven testing, the input is some form of requirements specification -- which can vary significantly. RQ1 maps the input for REDAST and reports on the comparison among different formats for requirements specification.

\textbf{RQ2. What are the frameworks, tools, processing methods, and transformation techniques used in requirements-driven automated software testing studies?} RQ2 explores the technical solutions from requirements to generated artifacts, e.g., rule-based transformation applying natural language processing (NLP) pipelines and deep learning (DL) techniques, where we additionally discuss the potential intermediate representation and additional input for the transformation process.

\textbf{RQ3. What are the test formats and coverage criteria used in the requirements-driven automated software
testing process?} RQ3 focuses on identifying the formulation of generated artifacts (i.e., the final output). We map the adopted test formats and analyze their characteristics in the REDAST process.

\textbf{RQ4. How do existing studies evaluate the generated test artifacts in the requirements-driven automated software testing process?} RQ4 identifies the evaluation datasets, metrics, and case study methodologies in the selected papers. This aims to understand how researchers assess the effectiveness, accuracy, and practical applicability of the generated test artifacts.

\textbf{RQ5. What are the limitations and challenges of existing requirements-driven automated software testing methods in the current era?} RQ5 addresses the limitations and challenges of existing studies while exploring future directions in the current era of technology development. %It particularly highlights the potential benefits of advanced LLMs and examines their capacity to meet the high expectations placed on these cutting-edge language modeling technologies. %\textcolor{blue}{CA: Do we really need to focus on LLMs? TBD.} \textcolor{orange}{FW: About LLMs, I removed the direct emphase in RQ5 but kept the discussion in RQ5 and the solution section. I think that would be more appropriate.}

\subsection{Searching Strategy}

The overview of the search process is exhibited in Fig. \ref{fig:papers}, which includes all the details of our search steps.
\begin{table}[!ht]
\caption{List of Search Terms}
\label{table:search_term}
\begin{tabularx}{\textwidth}{lX}
\hline
\textbf{Terms Group} & \textbf{Terms} \\ \hline
Test Group & test* \\
Requirement Group & requirement* OR use case* OR user stor* OR specification* \\
Software Group & software* OR system* \\
Method Group & generat* OR deriv* OR map* OR creat* OR extract* OR design* OR priorit* OR construct* OR transform* \\ \hline
\end{tabularx}
\end{table}

\begin{figure}
    \centering
    \includegraphics[width=1\linewidth]{fig/methodology/search-papers.drawio.pdf}
    \caption{Study Search Process}
    \label{fig:papers}
\end{figure}

\subsubsection{Search String Formulation}
Our research questions (RQs) guided the identification of the main search terms. We designed our search string with generic keywords to avoid missing out on any related papers, where four groups of search terms are included, namely ``test group'', ``requirement group'', ``software group'', and ``method group''. In order to capture all the expressions of the search terms, we use wildcards to match the appendix of the word, e.g., ``test*'' can capture ``testing'', ``tests'' and so on. The search terms are listed in Table~\ref{table:search_term}, decided after iterative discussion and refinement among all the authors. As a result, we finally formed the search string as follows:


\hangindent=1.5em
 \textbf{ON ABSTRACT} ((``test*'') \textbf{AND} (``requirement*'' \textbf{OR} ``use case*'' \textbf{OR} ``user stor*'' \textbf{OR} ``specifications'') \textbf{AND} (``software*'' \textbf{OR} ``system*'') \textbf{AND} (``generat*'' \textbf{OR} ``deriv*'' \textbf{OR} ``map*'' \textbf{OR} ``creat*'' \textbf{OR} ``extract*'' \textbf{OR} ``design*'' \textbf{OR} ``priorit*'' \textbf{OR} ``construct*'' \textbf{OR} ``transform*''))

The search process was conducted in September 2024, and therefore, the search results reflect studies available up to that date. We conducted the search process on six online databases: IEEE Xplore, ACM Digital Library, Wiley, Scopus, Web of Science, and Science Direct. However, some databases were incompatible with our default search string in the following situations: (1) unsupported for searching within abstract, such as Scopus, and (2) limited search terms, such as ScienceDirect. Here, for (1) situation, we searched within the title, keyword, and abstract, and for (2) situation, we separately executed the search and removed the duplicate papers in the merging process. 

\subsubsection{Automated Searching and Duplicate Removal}
We used advanced search to execute our search string within our selected databases, following our designed selection criteria in Table \ref{table:selection}. The first search returned 27,333 papers. Specifically for the duplicate removal, we used a Python script to remove (1) overlapped search results among multiple databases and (2) conference or workshop papers, also found with the same title and authors in the other journals. After duplicate removal, we obtained 21,652 papers for further filtering.

\begin{table*}[]
\caption{Selection Criteria}
\label{table:selection}
\begin{tabularx}{\textwidth}{lX}
\hline
\textbf{Criterion ID} & \textbf{Criterion Description} \\ \hline
S01          & Papers written in English. \\
S02-1        & Papers in the subjects of "Computer Science" or "Software Engineering". \\
S02-2        & Papers published on software testing-related issues. \\
S03          & Papers published from 1991 to the present. \\ 
S04          & Papers with accessible full text. \\ \hline
\end{tabularx}
\end{table*}

\begin{table*}[]
\small
\caption{Inclusion and Exclusion Criteria}
\label{table:criteria}
\begin{tabularx}{\textwidth}{lX}
\hline
\textbf{ID}  & \textbf{Description} \\ \hline
\multicolumn{2}{l}{\textbf{Inclusion Criteria}} \\ \hline
I01 & Papers about requirements-driven automated system testing or acceptance testing generation, or studies that generate system-testing-related artifacts. \\
I02 & Peer-reviewed studies that have been used in academia with references from literature. \\ \hline
\multicolumn{2}{l}{\textbf{Exclusion Criteria}} \\ \hline
E01 & Studies that only support automated code generation, but not test-artifact generation. \\
E02 & Studies that do not use requirements-related information as an input. \\
E03 & Papers with fewer than 5 pages (1-4 pages). \\
E04 & Non-primary studies (secondary or tertiary studies). \\
E05 & Vision papers and grey literature (unpublished work), books (chapters), posters, discussions, opinions, keynotes, magazine articles, experience, and comparison papers. \\ \hline
\end{tabularx}
\end{table*}

\subsubsection{Filtering Process}

In this step, we filtered a total of 21,652 papers using the inclusion and exclusion criteria outlined in Table \ref{table:criteria}. This process was primarily carried out by the first and second authors. Our criteria are structured at different levels, facilitating a multi-step filtering process. This approach involves applying various criteria in three distinct phases. We employed a cross-verification method involving (1) the first and second authors and (2) the other authors. Initially, the filtering was conducted separately by the first and second authors. After cross-verifying their results, the results were then reviewed and discussed further by the other authors for final decision-making. We widely adopted this verification strategy within the filtering stages. During the filtering process, we managed our paper list using a BibTeX file and categorized the papers with color-coding through BibTeX management software\footnote{\url{https://bibdesk.sourceforge.io/}}, i.e., “red” for irrelevant papers, “yellow” for potentially relevant papers, and “blue” for relevant papers. This color-coding system facilitated the organization and review of papers according to their relevance.

The screening process is shown below,
\begin{itemize}
    \item \textbf{1st-round Filtering} was based on the title and abstract, using the criteria I01 and E01. At this stage, the number of papers was reduced from 21,652 to 9,071.
    \item \textbf{2nd-round Filtering}. We attempted to include requirements-related papers based on E02 on the title and abstract level, which resulted from 9,071 to 4,071 papers. We excluded all the papers that did not focus on requirements-related information as an input or only mentioned the term ``requirements'' but did not refer to the requirements specification.
    \item \textbf{3rd-round Filtering}. We selectively reviewed the content of papers identified as potentially relevant to requirements-driven automated test generation. This process resulted in 162 papers for further analysis.
\end{itemize}
Note that, especially for third-round filtering, we aimed to include as many relevant papers as possible, even borderline cases, according to our criteria. The results were then discussed iteratively among all the authors to reach a consensus.

\subsubsection{Snowballing}

Snowballing is necessary for identifying papers that may have been missed during the automated search. Following the guidelines by Wohlin~\cite{wohlin2014guidelines}, we conducted both forward and backward snowballing. As a result, we identified 24 additional papers through this process.

\subsubsection{Data Extraction}

Based on the formulated research questions (RQs), we designed 38 data extraction questions\footnote{\url{https://drive.google.com/file/d/1yjy-59Juu9L3WHaOPu-XQo-j-HHGTbx_/view?usp=sharing}} and created a Google Form to collect the required information from the relevant papers. The questions included 30 short-answer questions, six checkbox questions, and two selection questions. The data extraction was organized into five sections: (1) basic information: fundamental details such as title, author, venue, etc.; (2) open information: insights on motivation, limitations, challenges, etc.; (3) requirements: requirements format, notation, and related aspects; (4) methodology: details, including immediate representation and technique support; (5) test-related information: test format(s), coverage, and related elements. Similar to the filtering process, the first and second authors conducted the data extraction and then forwarded the results to the other authors to initiate the review meeting.

\subsubsection{Quality Assessment}

During the data extraction process, we encountered papers with insufficient information. To address this, we conducted a quality assessment in parallel to ensure the relevance of the papers to our objectives. This approach, also adopted in previous secondary studies~\cite{shamsujjoha2021developing, naveed2024model}, involved designing a set of assessment questions based on guidelines by Kitchenham et al.~\cite{kitchenham2022segress}. The quality assessment questions in our study are shown below:
\begin{itemize}
    \item \textbf{QA1}. Does this study clearly state \emph{how} requirements drive automated test generation?
    \item \textbf{QA2}. Does this study clearly state the \emph{aim} of REDAST?
    \item \textbf{QA3}. Does this study enable \emph{automation} in test generation?
    \item \textbf{QA4}. Does this study demonstrate the usability of the method from the perspective of methodology explanation, discussion, case examples, and experiments?
\end{itemize}
QA4 originates from an open perspective in the review process, where we focused on evaluation, discussion, and explanation. Our review also examined the study’s overall structure, including the methodology description, case studies, experiments, and analyses. The detailed results of the quality assessment are provided in the Appendix. Following this assessment, the final data extraction was based on 156 papers.

% \begin{table}[]
% \begin{tabular}{ll}
% \hline
% QA ID & QA Questions                                             \\ \hline
% Q01   & Does this study clearly state its aims?                  \\
% Q02   & Does this study clearly describe its methodology?        \\
% Q03   & Does this study involve automated test generation?       \\
% Q04   & Does this study include a promising evaluation?          \\
% Q05   & Does this study demonstrate the usability of the method? \\ \hline
% \end{tabular}%
% \caption{Questions for Quality Assessment}
% \label{table:qa}
% \end{table}

% automated quality assessment

% \textcolor{blue}{CA: Our search strategy focused on identifying requirements types first. We covered several sources, e.g., ~\cite{Pohl:11,wagner2019status} to identify different formats and notations of specifying requirements. However, this came out to be a long list, e.g., free-form NL requirements, semi-formal UML models, free-from textual use case models, UML class diagrams, UML activity diagrams, and so on. In this paper, we attempted to primarily focus on requirements-related aspects and not design-level information. Hence, we generalised our search string to include generic keywords, e.g., requirement*, use case*, and user stor*. We did so to avoid missing out on any papers, bringing too restrictive in our search strategy, and not creating a too-generic search string with all the aforementioned formats to avoid getting results beyond our review's scope.}


%% Use \subsection commands to start a subsection.



%\subsection{Study Selection}

% In this step, we further looked into the content of searched papers using our search strategy and applied our inclusion and exclusion criteria. Our filtering strategy aimed to pinpoint studies focused on requirements-driven system-level testing. Recognizing the presence of irrelevant papers in our search results, we established detailed selection criteria for preliminary inclusion and exclusion, as shown in Table \ref{table: criteria}. Specifically, we further developed the taxonomy schema to exclude two types of studies that did not meet the requirements for system-level testing: (1) studies supporting specification-driven test generation, such as UML-driven test generation, rather than requirements-driven testing, and (2) studies focusing on code-based test generation, such as requirement-driven code generation for unit testing.





% \textbf{Framework picture} 
% \begin{figure}
	%     \centering
	%     \includegraphics[width=0.9\linewidth]{figures/framework.png}
	%     \caption{Caption}
	%     \label{fig:enter-label}
	% \end{figure}

%\textbf{General explanation. }
%At a high-level, our framework combines a dialogue-based intent model that identifies and ranks the set of agreements being discussed in the conversation between two players with a non-dialogue based value function to determine how  rationalizable a given agreement is for a given player. 
%
%
%\subsection{Generating set of agreements}
%
%\textbf{Problem.} Given a state $s$ in the game, determine the set of agreement $\calA_s$ in the diplomacy game. 
%
%
%\ak{
%	For each node in game graph, check if neighboring territory is occupied by a unit of another player. 
%	If yes, add all possible legitimate action pairs to set of agreements. 
%}
%
%
%
%\subsection{Preferences over Intent}
%
%\textbf{Problem.} Given the history of dialogues $\vec{d}$, states $\vec{s}$, actions $\vec{a}$, and the set of agreements $\calA_s$, determine a probability distribution over $\calA_s$. 
%
%
%The distribution implicitly encodes an \emph{incomplete} preference structure. 
%The incompleteness arises from the absence of conversation about certain potential agreements.
%
%
%Note that intents incorrect since the conversations can contain lies. 
%
%
%\ak{How to learn an incomplete preference from dialogue?}
%
%
%\ak{Explore. https://aclanthology.org/W19-5941.pdf}
%
%
%
%\subsection{Value of Coalitions}
%
%\paragraph{Problem.} Given a value model $V_p: S \rightarrow \mathbb{R}$ that maps every state to a real number for each player separately, determine a value function for a coalition ${\mathcal V}: S \times 2^N \rightarrow \mathbb{R}$ and the value a player gains by participating in a coalition, $V: S \times 2^N \times N \rightarrow \mathbb{R}$. 
%
%
%There are two types of models to compute this: transferable vs. non-transferable utility models. 
%We still need to determine which is best for our case. 
%
%
%\subsection{Rationalizability of Coalition Change}
%
%An agreement may be highly preferred for a player, say P1, in the subset of $\calA_s$ in which P1 is involved, but other players may not be interested to participate in this agreement.   
%The solution concept of rationalizability in game theory determines rational actions for a player accounting for their incomplete information.
%To determine whether an agreement is rationalizable, we must evaluate how preferable it is among the set of likely intents.
%
%\ak{Need for formalize rationalizability. I desire it to be a quantitative definition. 
%	This is tricky because most existing work studies rationalizability in games with single objectives. More investigation is needed to find/define the concept for multi-objective games or games with preferences.}
%
%
%\paragraph{Problem.} Given an incomplete preference over intents and a preference over coalitions, determine the rationalizability of an agreement for a player.  


\section{Experiments}

We evaluate the two stages of our proposed method separately. 
First, we outline the dataset employed for evaluation, and then  present the results for each of the two stages.

\subsection{Dataset}\label{sec:data}
We source previous Diplomacy games from WebDiplomacy\footnote{\url{https://webdiplomacy.net/}}, a multiplayer online implementation of Diplomacy. We consider a dataset of $140$ full-press games played over the standard Diplomacy map.  In order to calibrate our agreement detection classifier, we manually annotate five games from this dataset, randomly sampling from games with at least 250 total messages sent. This gives us a total of 16962 $(S, P_1, P_2, u)$ tuples over 1603 combinations of a state $S$ and players $P_1, P_2$. This dataset is highly imbalanced, with 444 (2.6\%) of all $(S, P_1, P_2, u)$ tuples having a coalition formed.

After we validate our usage of the agreement detection classifier, we then use it to label the remainder of the games with detected agreements. This resulting dataset consists of $415001$ total $(S, p_1, p_2, u)$ tuples. Of these tuples, $11008$ have agreements detected by our automatic method, $8344$ of which are upheld (i.e. the player played the agreed-upon move).

\subsection{Validating Agreement Detection Method}\label{sec:valid}

We use the manually-annotated data sample described in Section~\ref{sec:data} to test our agreement detection method, with an 80-20 train-test split. While this dataset is strongly imbalanced, we mitigate the impact of the dataset imbalance by only training on instances that pass our language model-based filter, which reduces our classifier training data to 1768 tuples, and by tuning a classification threshold to optimize F1-score on our training dataset. We benchmark three methods on this dataset: 
\begin{itemize}
    \item \textbf{GPT-4o}, which prompts a strong language model to directly identify units over which an agreement has been reached,
    \item \textbf{Classifier}, which trains a classifier on intent model distributions before and after dialogue over all $(S, P_1, P_2, u)$ tuples in the dataset, and
    \item \textbf{Hybrid}, our approach, which first filters using GPT-4o-parsed locations and player adjacency before training a classifier on the filtered data.
\end{itemize}

The results of our evaluation are in Table~\ref{tab:eval}. Our hybrid method outperforms both training a classifier on unfiltered intent data and prompting a strong language model on identifying units over which agreements have been reached. Extraction of coalition agreements from Diplomacy dialogue is a challenging task, due to the length of many dialogues in Diplomacy as well as the implicitness and fluctuating nature of negotiation over a multi-party dialogue. Fine-tuning more generally capable language models following the intent model formula in CICERO, in combination with more sophisticated parsers such as the one trained in \citep{wongkamjan2024more}, could yield even stronger performance improvements, which we leave to future work in this direction.

\begin{table}
    \centering
    \begin{tabular}{c|c|c|c}
    \toprule
    Method & F1 Score & Precision & Recall \\\midrule
    GPT-4o & 0.34 & 0.26 & 0.47 \\
    Classifier & 0.44 & 0.43 & 0.45 \\
    \textbf{Hybrid} & \textbf{0.55} & \textbf{0.63} & \textbf{0.48} \\\bottomrule
    \end{tabular}
    \caption{Classification metrics over the test dataset for our three methods. Hybrid methods outperform both purely language model-based and intent model distribution-based approaches at detecting whether an agreement has been reached over a specific unit.}
    \label{tab:eval}
\end{table}

\subsection{Evaluating Rationalizability Score}

The rationalizability score establishes a ranking of potential agreements for a unit within a specific game state. To evaluate the effectiveness of the score in predicting coalition structures, we analyze the rankings induced by the score on honored agreements in comparison to those of violated agreements.


We utilize both hand-labeled data and data labeled through the hybrid approach for the evaluation. 
We consider a total of $7434$ agreements labeled using the hybrid approach for evaluation.
For each agreement identified in the agreement detection stage, we generate a set of alternative agreements by sampling different orders for the units involved in the agreement.
The results of this evaluation for honored and violated agreements are presented in Table~\ref{tab:agreement}.
Given that the output of our model is a ranked list based on the rationalizability score, we employ two information retrieval metrics: mean reciprocal rank (MRR) \citep{craswell2009mean} and Brier score \citep{brier1950verification}.
The MRR is calculated using both the top-$1$ and top-$5$ ranked elements.

The ranking generated by the rationalizability score effectively differentiates between honored and violated agreements.
Our findings indicate that honored agreements typically receive lower ranks, while violated agreements tend to rank higher. 
This is observed through both the MRR and Brier scores.
When calculating the Brier score, we normalize the rationalizability scores, such that a score close to $1$ reflects that honored agreements usually have low ranks and violated agreements have higher ranks.
Notably, MRR scores that are close to $1$ in the top-$1$ case suggest that honored agreements are frequently assigned a rank of $0$, suggesting that this metric can very precisely recognize upheld coalitions.

We also compare our rationalizability score to a more conventional coalition formation prediction method, approximate Nash equilibrium (as estimated by the CICERO value model). We find that even when such approximate equilibrium-based methods are adapted for games with large state and action spaces, they remain inadequate for predicting coalition formation in such dynamic environments. Our R-Score yields a significantly higher MRR and a lower Brier score than the value model score in all cases.  This suggests that our rationalizability framework is significantly better at distinguishing between coalitions that are upheld and coalitions that are not upheld than Nash approximation-based predictions.



\begin{table}
\begin{centering}
\resizebox{0.5\textwidth}{!}{
\begin{tabular}{cc|ccc|ccc}
\toprule 
\multicolumn{2}{c}{} & \multicolumn{3}{c}{\textbf{Hand-Labelled}} & \multicolumn{3}{c}{\textbf{Hybrid}} \\
\multicolumn{1}{c}{Honored?} & \multicolumn{1}{c}{Metric} & \multicolumn{1}{c}{Value} & \multicolumn{1}{c}{R-Score@1} & \multicolumn{1}{c}{R-Score@5} & \multicolumn{1}{c}{Value} & \multicolumn{1}{c}{R-Score@1} & \multicolumn{1}{c}{R-Score@5} \\
\midrule
Yes & MRR ($\uparrow$) & 0.2842 & 0.9444 & \textbf{0.9722} & 0.3602 & 0.7416 & \textbf{0.8294} \\
No & MRR ($\downarrow$) & 0.2583 & \textbf{0.0} & 0.125 & 0.3682 & \textbf{0.2628} & 0.3354 \\
Yes & Brier ($\downarrow$) & 0.0802 & \textbf{0.0422} & \textbf{0.0422} & 0.0739 & \textbf{0.0311} & \textbf{0.0311} \\
No & Brier ($\uparrow$) & 0.7303 & \textbf{0.7494} & \textbf{0.7494} & 0.5706 & \textbf{0.6145} & \textbf{0.6145} \\
\bottomrule
\end{tabular}}
\end{centering}
\caption{Evaluation metrics for honored and violated agreements based on hand-labeled and hybrid datasets. The ranking induced by rationalizability score (RScore) on the set of agreements assigns lower ranks to honored agreements and higher ranks to violated agreements when compared to the ranking induced by Nash approximation-based predictions.}
\label{tab:agreement}
\end{table}

\section{Conclusion}

The detection of dynamic coalition structures is a key problem in understanding sequential interactions in strategic multi-agent environments. While many such environments use language, the study of coalition structure detection over natural language-based coordination is relatively understudied. This is compounded in settings like the board game Diplomacy, where players make decisions with incomplete information using dialogue-informed mental models of their opponents' future actions, and where relationships between players can shift drastically between turns as information is revealed.

Drawing from hypergame theory and the concept of subjective rationalizability, we propose a general method to dynamically predict coalition structures over sequential multi-agent interactions. In our method, we first extract detected agreements using the combination of a large language model-based parser and a specialized language model to predict player intents before and after the negotiation phase. We then compute the value of the agreement using a deep reinforcement-learning based value function, which we use in combination with player intents to compute the likelihood that each player will honor the agreement.

We validate the success of our method over sampled interactions between human Diplomacy players, using components of Meta's CICERO agent to compute player intents and action values. When compared to approximate Nash Equilibrium-based methods, our rationalizability score is significantly better at predicting the coalition structure at a given timestep. Our method can also generalize to other multi-agent, dialogue-based games, as long as sufficient human data exists upon which similar, game-specific models can be trained.

Extending coalition structure detection to natural language-based negotiation environments such as Diplomacy presents unique challenges in a setting where agents have incomplete information, negotiations are both multi-issue and multi-party, and where agents must reason over mental models of their opponents. However, for artificial agents to handle such complex environments properly, they must be capable of understanding the coalition dynamics of the environment at a given state. Our method and experiments serve as an important first step in this direction. We hope that future work will be able to extend our framework to new settings, including those with more complex negotiations and less existing domain-specific models, paving the way for agents that can reason over such information in deployment settings.

\FloatBarrier

\begin{acks}
This material is based upon work supported by the Defense Advanced Research Projects Agency (DARPA) under Agreement No. HR00112490410, the Army Research Lab under Agreement ARO W911NF-23-1-0317 and the Office of Naval Research under Agreement N00014-24-1-2097. We thank WebDiplomacy for supporting this research by providing access to online gameplay data.
\end{acks}

%%%%%%%%%%%%%%%%%%%%%%%%%%%%%%%%%%%%%%%%%%%%%%%%%%%%%%%%%%%%%%%%%%%%%%%%

%%% The acknowledgments section is defined using the "acks" environment
%%% (rather than an unnumbered section). The use of this environment
%%% ensures the proper identification of the section in the article
%%% metadata as well as the consistent spelling of the heading.

%%%%%%%%%%%%%%%%%%%%%%%%%%%%%%%%%%%%%%%%%%%%%%%%%%%%%%%%%%%%%%%%%%%%%%%%

%%% The next two lines define, first, the bibliography style to be
%%% applied, and, second, the bibliography file to be used.

\bibliographystyle{ACM-Reference-Format}
\bibliography{main}

%%%%%%%%%%%%%%%%%%%%%%%%%%%%%%%%%%%%%%%%%%%%%%%%%%%%%%%%%%%%%%%%%%%%%%%%

\end{document}

%%%%%%%%%%%%%%%%%%%%%%%%%%%%%%%%%%%%%%%%%%%%%%%%%%%%%%%%%%%%%%%%%%%%%%%%

