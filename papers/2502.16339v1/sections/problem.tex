We use Diplomacy as a testbed to study dynamic multilateral coalition formation. 
Diplomacy is a deterministic game where players negotiate before concurrently submitting actions for their units, such as hold, move, or support.
The game state transitions based on these actions, and the negotiation phase repeats.
In this paper, we only consider the movement phases of the Diplomacy game. 
We note the high complexity of the game: each unit has an average of 26 valid orders, with up to 34 units on the board, making enumerative approaches intractable \citep{bakhtin2021no}.



%We use Diplomacy as a testbed to study dynamic multilateral coalition formation.
%% Diplomacy is a concurrent multiplayer game, similar to those studied in computer science [cite].
%% Actions in Diplomacy have deterministic consequences, that is, given a state and actions of all players, the next state is uniquely determined. 
%Diplomacy is a deterministic game where all players submit actions concurrently; the next state is uniquely determined by the current state and the submitted actions of each player. During each phase, players first negotiate with each other before choosing an action for each unit that they control: actions include holding in place, moving to a nearby location, or supporting another unit's action. Once all players submit their actions, the game is adjudicated and the negotiation process restarts. We consider only the movement phases of the Diplomacy game. As \citep{bakhtin2021no} notes, each unit has an average of twenty-six valid orders, and there can be up to thirty-four units on the board at a time. Thus, strictly enumerative approaches are intractable in this environment.
%% Each turn during movement, all players simultaneously choose an action composed of one order for every unit they control.
%% An order may direct a unit to hold stationary, to move to an adjacent location, to support a nearby unit’s hold or move, to move across water via convoy, or to act as the convoy for another unit. 
%% Because a player may control up to 17 units with an average of 26 valid orders for each unit, the number of possible actions is typically too large to enumerate. 
%% The game play is determined by 
%% For a more detailed description, see [24].
 
\textbf{Game model.}  
Diplomacy can be modeled as a concurrent multiplayer game \citep{alur2002alternating} with rewards, $G = (N, S, A, T, s_0, R)$, where $N = \{P_i \mid i = 1, 2, \ldots n\}$ is a set of players, $S$ is a set of states, $A$ is a set of actions, $T: S \times A \rightarrow S$ is a deterministic transition function, $s_0 \in S$ is an initial state, and $R: S \times A \rightarrow \mathbb{R}$ is a reward function. 
A game play in $G$ is determined in two phases: 
Given a state $s \in S$, the players in $N$ \emph{privately} negotiate non-binding bilateral agreements with each other. 
%Formally, an agreement is defined as follows:
\begin{definition}[Agreement]
	Given a state $s \in S$, an agreement between two players $P_i, P_j$ is a tuple $(u_1, u_2, a_1, a_2)$, where $u_1$ is a unit controlled by $P_i$, $u_2$ is a unit controlled $P_j$, and $a_1, a_2$ are legal actions for units $u_1, u_2$ in state $s$.
\end{definition}

The content of these negotiations and the agreements are only known to the players involved in the negotiation, unless one of these players explicitly shares this information with other players. 
At the conclusion of negotiation phase, all players choose an action, assigning an order to each unit controlled by them. 
Together, these actions determine the joint action $a = (a_1, a_2, \ldots, a_n)$, which in turn uniquely determines the next state $s' = T(s, a)$.
We denote by $d_t$ the set of all natural language messages exchanged between any pair of players in round $t$. 
Therefore, a game can be denoted as the sequence of state-dialogue-action pairs, $\rho = s_0 d_0 a_0 s_1 d_1 \ldots s_n$. 
A game in Diplomacy is of finite duration since a player will either win the game, or the game will be declared a draw.   
For a more detailed description, see \citep{sharp1978game}.


A policy for a player $P_i$ is a map $\pi_i: S \rightarrow \dist(A_i)$, where $ \dist(A_i)$ is a set of probability distributions over actions $A_i$ of player $P_i$.
%Each player $i \in N$ has a policy, which is a map $\pi_i: S \rightarrow \dist(A_i)$, where $ \dist(A_i)$ is a set of probability distributions over actions $A_i$ of player $P_i$.
A policy profile is a collection of policies of all players, $\pi = (\pi_1,...,\pi_n)$.
%, the probability of a trajectory is p(τ|π) = (s,a)∈τ p(a|a ∼ π(s)),


\textbf{Coalition structure.} A coalition is a set of honored agreements. 
The coalition structure, given a game state, is represented as an undirected multigraph with players as nodes and parallel edges indicating agreements between them. 
Since agreements are inferred from potentially ambiguous natural language dialogue, we assign weights to the edges.
Intuitively, these weights represent the likelihood of each agreement being honored.

%A coalition is a set of honored agreements, \ie, the agreements $(u_1, u_2, a_1, a_2)$ where each first player assigns action $a_1$ to its unit $u_1$ and the second player assigns action $a_2$ to its unit $u_2$.
%Given a game state, the coalition structure can be represented as an undirected multigraph whose nodes are the players in the game, and whose parallel edges represent various agreements made between the two players. In our case, the information about agreements is not explicitly known and must be inferred from the natural language dialogue. 
%To accommodate this uncertainty, we label the edges of the coalition structure with weights that represent the likelihood of that agreement being honored. 

\begin{definition}
	A coalition structure is a graph $C = (N, E, \agreements, \weight)$, where $N$ is the set of players, $\agreements$ is a set of agreements, $E \subseteq N \times N \times \agreements$ is the set of edges, and $\weight: E \rightarrow \mathbb{R}$ is a function that assigns a real-valued weight to each edge. 
\end{definition}

Note that in Diplomacy, the coalition structure is not static; we denote the coalition structure in round $t$ as $C_t$. This background motivates the problem of \textbf{coalition structure prediction}. 
\begin{problem}\label{eq:problem}
	Given a concurrent multiplayer game $G$, a round $t \geq 0$, and the play $\rho = s_0 d_0 a_0 \ldots s_t d_t$ until round $t$, predict the coalition structure $C_t$. 
\end{problem}	


