The process of coalition formation in multi-agent systems involves agents forming coalitions to work together towards aligned objectives by coordinating their actions \citep{sandholm1999coalition,shehory1998methods}. 
This process has been studied extensively for both static and dynamic cases in game theory and logic.  
Past game-theoretic approaches have focused on studying which coalitions are likely to form based on various types of equilibria \citep{hajdukova2006coalition} and evaluating the value of a coalition to an agent. 
Meanwhile, logic-based approaches have focused on evaluating whether a given coalition can enforce a temporal property, regardless of how the agents not in the coalition behave \citep{pauly2002modal,alur2002alternating}.
However, neither of these approaches is suitable for studying coalition formation in natural language negotiation, where the ambiguity in language often leads players to interpret game states differently.
This phenomenon is commonly observed in real-world scenarios such as human-robot teaming \citep{chakraborti2016formal}, computer games \citep{mazrooei2013automating,rodriguez2022collusion}.

%However, neither of these approaches are suitable for studying coalition formation in natural language negotiation, where the ambiguity in language  
%often leads to players interpreting game states differently.
%This ??? is commonly in observed in many real-world scenarios such as human-robot teaming \citep{chakraborti2016formal}, auctions \citep{hosam2006planning}, and computer games \citep{mazrooei2013automating,rodriguez2022collusion}.
%
%
%However, neither of these approaches are suitable for studying coalition formation in  natural language negotiation, which is common in many real-world coalition formation settings such as human-robot teaming \citep{chakraborti2016formal}, auctions \citep{hosam2006planning}, and computer games \citep{mazrooei2013automating,rodriguez2022collusion}. 
%This is because natural language negotiations introduce ambiguity that may lead to players having different subjective views of the same game state, which is exacerbated in imperfect-information games. % and misinformation propagation in social media [cite]. 

In this work, we study the problem of predicting dynamic multilateral coalition structures in sequential multi-agent interactions where players coordinate their actions using natural language. 
A coalition structure \citep{greenberg1994coalition} is a graph where nodes represent players and the edges represent agreements between players over coordinated actions.
While traditional approaches define coalition structures to be a partition of players, we model a coalition as a multi-graph allowing multiple agreements between two players, in addition to allowing a player to form bilateral agreements with multiple players simultaneously.
We model multi-agent interactions as reactive games, where the player can renegotiate their agreements in every round.  
Our aim is to predict these coalition structures from the perspective of an external observer, similar in spirit to an agency monitoring a computer network for anomalous behavior.


%We study multi-agent interactions involving natural-language-based negotiations from the perspective of an external observer, similar to monitoring a computer network for anomalous behavior. 
%Considering the game's dynamic nature and the fact that agreements are renegotiated every round, we pose the following question: As an external observer with access to the complete history and dialogue, how can one predict the likely coalition structure for the next round—that is, the set of agreements that are promised and upheld?


We use the board game Diplomacy \citep{calhamer1974invention} as a testbed for dynamic coalition structure detection over natural-language negotiations. Diplomacy is a seven-player board game that exemplifies key challenges in multi-agent systems research, combining semi-cooperative strategic dynamics with natural language-based negotiation. Players aim to control a majority of 34 supply centers on a map of Europe by coordinating the movement of units. While Diplomacy is a zero-sum game, players must negotiate strategic coalitions to support their own plans or counteract the moves of other players.
% A unique aspect of Diplomacy is the use of free-form, pairwise natural language dialogue, where players negotiate privately each turn and then simultaneously execute their chosen actions. These agreements—though non-binding—are crucial for advancing in the game, adding an element of trust and deception. Consequently, studying coalition formation in such contexts presents unique challenges, as the ambiguity inherent in natural language introduces uncertainty and potential of deception, complicating both prediction and analysis of coalition formation.


%This introduces the challenge of balancing self-interest with promises made to others, making it necessary for players to constantly assess the likelihood of betrayal or cooperation implicit in the dialogue. \ak{this must end with something related to coalition formation!}


Diplomacy highlights three key challenges central to studying coalition formation in games with natural-language negotiations: \emph{decision-making under incomplete information}, \emph{reasoning based on mental models of opponents}, and \emph{multilateral negotiations}. 
Since the negotiations are pairwise and private, each player has incomplete information about the negotiations a second player has had with other players. 
As a result, players must anticipate others' actions without full knowledge of all agreements. This requires a player to construct a mental model of the other player's incentives to estimate the likelihood they would honor the agreement in addition to weighing their own incentives to honor the agreement. Lastly, since a player can simultaneously negotiate multiple agreements about the same unit with different players, a player must ultimately select a subset of agreements to honor based on the strategic advantage they offer to the player and the likelihood of them being honored by the player with whom the agreement is made.

%The interplay of these factors, coupled with the complexity of natural language-based negotiations, makes Diplomacy a unique and rich problem for AI and multi-agent systems research.


%\textbf{Introduce diplomacy.} 
%For instance, consider the game of Diplomacy, which is a longstanding challenge in AI research that features a rich mixture of competition, cooperation, and negotiation in natural language. 
%Diplomacy is a seven-player zero-sum board game in which a map of Europe is divided
%into $75$ provinces. 
%$34$ of these provinces are marked as supply centers, and the goal of a player in the game is to control a majority ($18$) of them by strategically using multiple units (armies
%or fleets). 
%A unit may support another unit (owned by the same or another player), allowing it to overcome resistance by other units.
%A unit may support another unit (owned by the same or another player), allowing it to overcome resistance by other units. Due to the inter-dependencies between units, players stand to gain by negotiating and coordinating moves with others.
%
%
%\emph{Agreements} in Diplomacy are promises made by one player to another about coordinating moves of their units. 
%All players engage in private pairwise free-form dialogue with the others during a negotiation period, and then all players simultaneously choose an action comprising one order per unit they control.
%Despite being a competitive game, Diplomacy is designed in a way that making agreements with other is required to win the game.
%
%
%Diplomacy highlights several important characteristics of common multi-agent interactions that have not been well-studied in the literature. 
%In this paper, we focus on three of them: \emph{decision-making under incomplete information}, \emph{reasoning based on mental models of opponents}, and \emph{multilateral negotiations}.
%Incomplete information exists in the game because players negotiate in private pairwise conversations. 
%For this reason, even if other player promises to honor an agreement, the ego player must reason whether that player has an strong incentive (in game-theoretic sense) to commit to this agreement. 
%This calls for the first player to view the game from other player's perspective, simulate its best moves, and anticipate the likelihood of that player truly honoring the agreement.
%Moreover, since each player can negotiate simultaneously with multiple players about several possible agreements over multiple units, it must ultimately choose the subset of agreements that provide it the most strategic benefit when combined together---a combinatorial problem! 
%\ak{missing language entanglement.}
%%Given these characteristics and the fact that agreements are renegotiated every round, how do we study the process of dynamic multilateral coalition formation in such rich language environments?

%Our goal in this paper is to view the game from perspective of an external observer, like monitoring a computer network for suspicious activities. 
%Given these characteristics and the fact that agreements are renegotiated every round, we ask the following question: As an external observer with access to all history and dialogue, how to predict the likely dynamic multilateral coalition structure in the next round, which is the set of  agreements that were promised and honored? 

% \deleted{We study multi-agent interactions involving natural-language-based negotiations from the perspective of an external observer, similar to monitoring a computer network for anomalous behavior. 
% Considering the game's dynamic nature and the fact that agreements are renegotiated every round, we pose the following question: As an external observer with access to the complete history and dialogue, how can one predict the likely coalition structure for the next round—that is, the set of agreements that are promised and upheld?
% }



%\textbf{LLM intro.} The emergence of Large Language Models (LLMs) have fundamentally altered the way we interact with ... 
%Recent work ... political coalitions, agreement detection, deception ... has opened promising directions. 
%However, LLMs, whilst powerful and capable of demonstrating some emergent properties, are not logical reasoners and often struggle to perform well at ... .

%\ak{Jump to 2 stage approach. Then, talk about LLM.} 

%The emergence of Large Language Models (LLMs) has fundamentally altered the way we interact with natural language in multi-agent systems, enabling more sophisticated dialogue processing and understanding in games like Diplomacy. Recent work on using LLMs to analyze political coalitions, detect agreements, and identify deception has opened promising directions for understanding complex human-like interactions. However, LLMs, while powerful and capable of exhibiting some emergent properties, are not logical reasoners and often struggle to perform well at tasks requiring precise reasoning, long-term planning, and the ability to consistently model and predict opponents' strategic behavior.


We define a novel method that addresses these challenges in dynamic coalition structure detection over natural language negotiations, as visualized in Figure~\ref{fig:framework}. Our approach consists of two stages: \textbf{agreement detection} and \textbf{strategic reasoning}.
To detect agreements negotiated via natural language in Diplomacy, we leverage large language models to parse dialogue, as well as  fine-tuned ``intent'' models from CICERO \citep{CICERO}, a Diplomacy-playing agent. By comparing the distribution over all moves involving parsed territories for a given unit before and after a phase of dialogue, we can learn whether a coalition was formed in the dialogue for that phase. Given a set of potential agreements identified, we then predict the set of honored agreements, which defines a coalition structure, using a deep reinforcement-learning based method. Due to the large action space and incomplete-information nature of Diplomacy, traditional enumerative game-theoretic approaches are intractable. To address these challenges, we extend the approach in \citep{bakhtin2021no} to compute the strategic value of an agreement for both players. We then sample from the intent model to determine the likelihood that both players will uphold the coalition, allowing us to measure the rationalizability of a coalition structure.

\begin{figure*}
    \includegraphics[width=0.9\textwidth]{figures/framework_v5.png}
    \caption{Proposed two-stage approach for learning coalition structures from natural language interactions in Diplomacy games. Stage 1 extracts agreements from pairwise dialogues to form an unweighted coalition structure. Stage 2 applies hypergame theory to assess the rationalizability of agreements for each player separately, which are then integrated into a weighted coalition structure representing the likelihood that an external observer believes agreements will be honored.}
    \label{fig:framework}
\end{figure*}

% The strategic reasoning stage addresses the following question: \emph{Given a set of potential agreements available to each player identified during the agreement detection phase, how can we predict whether a player will honor or deviate from a particular agreement?}
% The set of honored agreements then defines the coalition structure. A key challenge to answer this question is posed by the large action space (which can be as large as $10^{20}$ actions in case of Diplomacy [cite]) that renders the traditional enumerative game-theoretic approaches to determine the value of a state infeasible.    
% Furthermore, a player's incomplete information about second player's negotiation with other players makes it difficult for the first player predict second player's likelihood to honor a proposed agreement between them.
% To address these challenges, we first build upon a deep reinforcement learning approach, DORA [cite], to compute the value of an agreement given the expected value of a state predicted by DORA.  This step is dialogue agnostic.
% We integrate this value of agreement to first player with the likelihood of the second player honoring the agreement as perceived by first player. We determine the likelihood as a ratio of the second player honoring the agreement with that of not honoring the agreement. These likelihoods are computed by sampling a pre-trained neural network that predicts the dialogue-conditioned probability distribution over actions of the second player given the state, dialogue, and action history [cite].






%The strategic reasoning stage answers the following question: \emph{given a set of potential agreements for each player from which that player can select a subset consistent with game dynamics identified by agreement detection stage, how to determine whether a player will honor or deviate away from a given agreement?} 
%Explain the approach taken to answer this --- hypergame theory and solution concepts. 


%\textbf{Diplomacy test-bed.} 
%We leverage Diplomacy as an abstract analog to real-world multi-agent systems, providing methods for AI agents to study dynamic multilateral coalition formation. 
The three main contributions of this work are:
\begin{enumerate}
	\item \textbf{Approach.} We introduce a novel method that integrates large language models and game theory to predict dynamic multilateral coalition formation in multi-agent systems where agents negotiate coalitions using natural language.
	
	\item \textbf{Agreement detection.} We develop a procedure that combines pretrained language models with game dynamics to extract agreements from dialogues between agents, enabling the detection of coalition structures in real-time interactions.
	
	\item \textbf{Strategic reasoning.} We propose a new metric based on subjective rationalizability from hypergame theory, which evaluates the likelihood that agents will adhere to agreements by accounting for their subjective views of the game and strategic uncertainty.
\end{enumerate}

We validate our method on a dataset of online Diplomacy gameplay experiments.
We find that our hybrid agreement detection outperforms existing baselines and that our rationalizability metric effectively distinguishes between when players will honor coalition agreements and when they will not.
These findings highlight the value of integrating natural language techniques with game-theoretic analysis. 
They extend existing game-theoretic dynamic coalition prediction approaches to handle natural language negotiations, bridging toward more realistic real-world applications.

%We validate our method on experiments over online Diplomacy gameplay. We find that our hybrid agreement detection method outperforms baselines, and that our rationalizability metric is able to effectively distinguish cases where a player will uphold a coalition agreement from cases where they will not. Our work demonstrates the importance of both natural language-based techniques and game-theoretic analysis in analyzing such natural-language based interactions. By extending existing models of dynamic coalition structure detection to cases where natural language is used to negotiate agreements, we help generalize such models to more realistic real-world settings.

\subsection{Related Work} 

\textbf{Game theory.} 
The study of coalition formation in game theory focuses on identifying and characterizing stable coalitions by estimating the value of possible coalitions to an agent.
Solution concepts such as the core \citep{arnold2002dynamic}, the kernel \citep{shehory1996kernel}, the nucleolus \citep{montero2006noncooperative}, and the Shapley value \citep{aumann2003endogenous} have been introduced to analyze stability in transferable utility games, where side payments are allowed, and non-transferable utility games, where they are prohibited. 
However, these approaches do not account for sequential interactions where agents strategically join coalitions to achieve their goals.
% Coalition formation is studied extensively in game theory, largely with the aim to identify and characterize stable coalitions based on different ways of dividing the utility among different players within the coalition.  
% A set of solution concepts have been introduced to study stability \citep{tremewan2016dynamics}, including the core [cite], the kernel [cite], the nucleolus [cite], and the Shapley value [cite] in games with transferable utility, which allow agents to make side payments to others to incentivize certain actions, or games with non-transferable utility, where side payments among players are prohibited, and coalition formation games, where players hold preferences over which players they want to form a coalition with). 
% However, these approaches do not model sequential interactions where agents must join different coalitions strategically to achieve their goals.

% \andy{copied; rewrite this paragraph}
% Dynamic coalition formation has also been studied in game theory, primarily with a focus on understanding the effects of externalities, where the gain for the system as a whole from forming a coalition may be affected by the formation of other co-existing coalitions (Michalak et al., 2010; Rahwan et al., 2012b; Skibski
% et al., 2016).
% Very little attention in the literature of multi-agent coali-
% tion formation has been given to the influence of the formation of one
% coalition on the performance of other co-existing coalitions in the sys-
% tem (Michalak et al., 2010; Rahwan et al., 2012b; Skibski et al., 2016).
% However, these inter-coalitional dependencies, called externalities, play
% a crucial role in many real-world multi-agent applications (Rahwan
% et al., 2012b). Against this background, Rahwan et al. (2012b) have
% presented the first computational study of coalitional games with ex-
% ternalities in a multi-agent system context. In the same field, Skibski
% et al. (2016) have introduced the k-coalitional games. The authors have
% proposed an extension of the Shapley value for these games, and stud-
% ied its axiomatic and computational properties. Michalak et al. (2010)
% have considered only the issue of representing coalitional games in
% multi-agent systems that exhibit externalities from coalition formation.
% They have proposed a new representation which is based on Boolean
% expressions. The aim was to construct much richer expressions that
% allow for capturing externalities induced upon coalitions.
% But these approaches assume complete information, which is not a feature of the games considered in this paper.
The study of dynamic coalition formation in game theory has mainly focused on understanding the effects of externalities, where the formation of one coalition impacts the gains of other co-existing coalitions. \citep{rahwan2009coalition} presented a computational study of coalitional games with externalities, arguing that such externalities are common in real-world settings. The study of such externalities was extended by work including \citep{sklab2020coalition, skibski2016k, michalak2010logic}. While these approaches were able to better capture the impacts of externalities on other coalitions, they assume that complete information available to all agents, which is not applicable to games such as Diplomacy.
 

\textbf{Logic.} 
Strategic decision-making within coalitions has been studied within the logic community. Coalition Logic \citep{pauly2002modal} and Alternating-time Temporal Logic \citep{alur2002alternating} formalize reasoning about the existence of joint strategies for agents in coalitions to achieve their goals regardless of how the non-coalitional agents act. However, these logics only study static coalitions. 

There is an ongoing effort to extend coalition logic to handle dynamic settings. 
In \citep{umar2016coordinated}, authors introduce coordinated coalitions that represent a predefined sequence of coalitions for model checking.
\citep{guelev2023temporary} presents a complex framework that enriches Concurrent Game Models (CGM) by incorporating negotiations, where promises---represented as epistemic logic formulas---are embedded into states and existence of strategies that ensure goal satisfaction are verified through model checking. 
These methods requires full access to the game model, which is impractical for large-scale games like Diplomacy. 
Moreover, they model negotiations as deterministic statements, failing to capture the inherent ambiguity of natural language.

%While this method allows for model checking user objectives on the original CGM, it relies on having full access to the entire game model, which is infeasible for large-scale games like Diplomacy. Additionally, this approach models negotiations as deterministic propositional statements, which cannot naturally handle the inherent ambiguity of language often used in negotiations.


% Decision-making within a coalitions has been studied within logic community.
% Pauly’s Coalition logic (CL), introduced in \citep{pauly2002modal}, and Alur, Henzinger and Kupferman’s Alternating time temporal logic (ATL) \citep{alur2002alternating} capture reasoning about absolute powers of agents and coalitions to act
% in pursuit of their goals and succeed unconditionally against any possible behaviour of their opponents, which are thus regarded as adversaries (in the context of CL) or as randomly behaving environment (in the context of ATL).
% However, these logic families only model static coalitions. 
% Recent work [cite:AT dynamic logic] has introduced coalition logic that can express coordinated coalitions, where the coalitions occur in a predefined sequence. 
% These logic cannot model the reactivity present in games such as Diplomacy, where coalitions are decided at runtime through strategic negotiations.
% [Guelev] presented a complex These constructions are computationally expensive: start with a CGM. Enrich the CGM with negotiations by constructing a new CGM with actions included within states. Negotiation is modeled as set of promises, modeled as epistemic logic formulas. CGM with Negotiations can be transformed into Honest Play CGM, where promise honoring paths in CGM can be identified by model checking a epistemic-temporal property on this CGM. Finally, user objectives can be model checked on original CGM by transforming temporal queries into epistemic temporal ones using the given procedure.  
% However, this approach depends entirely on the availability of the entire game model, which is not feasible for games as large as Diplomacy. 
% Moreover, the approach in [] models negotiations are a set of propositional statements whose truth can be determined deterministically. Thus, this approach does not provide a natural way to incorporate ambiguity inherent in natural language.


%Static coalitions -- ATL. Idea is to model it as two player game. No dynamic coalitions. 
%
%Sequential coalitions -- fixed coalition sequence given upfront. No reactivity as present in Diplomacy.
%
%
%Dynamic coalitions -- Guelev. Needs to construct action-inclusive transition model, which in case of Diplomacy games is impossible owing to their size. 
%None of these approaches handles complexities introduced by language. 


\textbf{Negotiation in Natural Language.} 
While significant past work has studied negotiation and coordination as a natural language task, the analysis of coalition formation in games involving natural language negotiations remains relatively understudied. \citep{lewis2017deal} collected a dataset of human negotiation dialogues in a semi-cooperative negotiation task, then trained natural language agents to perform the same task using the dataset. More recently, large language model-based agents have been used to achieve stronger performance on a range of social influence tasks \citep{chawla2023social}, including negotiation in a self-play environment over both zero-sum \citep{fu2023improving} and non-zero-sum \citep{liao2024efficacy} games. \citep{gandhi2023strategic} demonstrates that incorporating more explicit search and belief tracking into language models can improve their negotiation performance over a wide range of environments. \citep{moghimifar2024modelling} specifically seeks to model political coalition formation with language model-based agents, arguing that previous language model-based approaches to negotiation do not fully capture the full complexity and iterative nature of human negotiations. They contribute a multilingual dataset of European political party manifestos, as well as coalitions that they formed with other parties. While we similarly seek to model the multi-issue, iterative dynamics of natural-language coalition formation, we additionally analyze this problem from a game-theoretic perspective that accounts for agents' models of each other.

Diplomacy has also attracted attention from the natural language community as a testbed for the analysis of coordination dynamics in a strategic multi-agent environment. \citep{niculae-etal-2015-linguistic} studies the formation and termination of long-term alliances from a linguistic perspective, finding linguistic cues that presage acts of betrayal. \citep{peskov2020takes} models deception over long-term relationships in Diplomacy, finding that a model that uses both game dynamics and dialogue cues can predict player deception at a near-human level. \citep{wongkamjan2024more} analyzes games between CICERO and human Diplomacy players, noting that despite CICERO's strong strategic capabilities, it is still less persuasive compared to human players. Finally, \citep{mukobiwelfare} devises a novel positive-sum variant of Diplomacy, finding that language model-based agents are capable of attaining high joint welfare in this setting.

% Icarus, Towards Diplomatic Agents in Diplomacy	

%Political science dynamic coalitions. 
%No strategic reasoning, or mental modeling. 
%(Any assumptions that can be challenged?)






%\hrule 
%
%
%%\textit{Why agreements are made and are necessary?} Diplomacy is played on a map of Europe partitioned into provinces, some of which are special and marked as Supply Centers. Each player attempts to own the majority of the supply centers, and controls multiple units (armies
%%or fleets). A unit may support another unit (owned by the same or another player), allowing it to overcome resistance by other units. Due to the inter-dependencies between units, players stand to gain by negotiating and coordinating moves with others. Hence, while ultimately a competitive game, making progress in Diplomacy requires teaming up with others.
%
%
%\textit{How are agreements made?} Each turn, all players engage in private pairwise free-form dialogue with the others during a negotiation period, and then all players simultaneously choose an action comprising one order per unit they control. 
%This highlights two key features of Diplomacy: \emph{incomplete information, mental-models-based reasoning}, and \emph{multilateral agreements}. 
%The player must decide actions under incomplete information of dialogue between others and, during the dialogue phase, since each player can talk to several other agents, the ultimate choice of what action is assigned to a unit is made strategically based on the ego player's self-interests. This means one unit coordinates with one player and another unit with another player. 
%%Given that the negotiations occur every round, the game is representative of a dynamic multilateral coalition formation process.
%Given these characteristics and the fact that agreements are renegotiated every round, we ask the following question: As an external observer with access to all history and dialogue, how to predict the likely dynamic multilateral coalition structure in the next round, which is the set of  agreements that were promised and honored? 
%
%%Diplomacy is a seven-player game where players attempt to acquire a majority of supply centers
%%across Europe (see \refFig{XX}). 
%%To acquire supply centers, players can coordinate their units with other players mostly through dialogue.
%%Coordination can be risky, because players can lie and even betray each other.
%%A single player may have up to 34 units, with each unit having an average of 26 possible actions. This astronomical action space makes planning and search intractable.
%%Agents need to be able to form a high-level long-term strategy (e.g. with whom to form alliances) and have a very short-term execution plan for their strategy (e.g. what units should I move in the next turn). Agents must also be able to adapt their plans, and beliefs about others (e.g. trustworthiness) depending on how the game unfolds.
%
%
%\hrule 
%
%Features: how coalitions form via negotiations natural language, incomplete information, bounded rational agents, deception, multilateral negotiations, dynamic between trust and strategic reasoning, non-binding agreements, must select a single action for each unit but can negotiate with multiple players to coordinate the action of same units, coalitions are renegotiated each round. 
%
%For instance, consider the game of Diplomacy. Explain the game, how coalitions form via negotiations natural language, how game has incomplete information, and bounded rational agents. Similar characteristics are observed in several multi-agent domains such as social media, ... . In such domains, how do we study the process of dynamic multilateral coalition formation?
%


%\section{Copy}
%
%
%Rational agents are usually
%required to form beneficial coalitions in open, distributed, and heterogeneous environ-
%ments, including scenarios in which dynamically occurring events might interfere with the coalition proceses.
%
%
%We study coalition formation as an ongoing, dynamic process, with payoffs generated as
%coalitions form, disintegrate, or regroup
%
%A process of coalition formation (PCF) is an
%equilibrium if a coalitional move to some other state can be ‘‘justified’’ by the expectation of
%higher future value, compared to inaction. T
%
%
%The theory of coalition formation has traditionally belonged to the realm of
%cooperative game theory (see, for instance, notions of the core, the bargaining set, or
%the stable set of von Neumann and Morgenstern).
%Recent literature takes this theory
%in three important methodological directions.
%First, characteristic functions are
%dispensed with.
%Second, the theory seeks
%‘‘consistent’’ formulations, in the sense that considerations of ‘‘credibility’’ are
%imposed on the blocking coalition in just the same way as they are on the original.
%Finally, the theory models players as being farsighted, in the sense that
%they care about the ‘‘ultimate’’ payoff from a move, and not its immediate
%consequences.
%situation.
%
%
%The emergence of Large Language Models (LLMs) have fundamentally altered the way we interact with digital systems and have led to the pursuit of LLM powered AI agents to assist in daily workflows. LLMs, whilst powerful and capable of demonstrating some emergent properties, are not logical reasoners and often struggle to perform well at all sub-tasks carried out by an AI agent to plan and execute a workflow. \ak{rephrase it.}
%
%
%\textbf{Limitations} They also fall
%short of fully representing the high-level negotia-
%tion process, overlooking critical phases and transi-
%tions, and lack the capacity for reflective decision-
%making based on previous outcomes
%
%\textbf{Impact.} Simulating the coalition negotiation process offers
%predictive insights into potential government forma-
%tions and serves as a valuable tool for researchers,
%political parties, analysts, and negotiators
%
%
%This eval-
%uation provides critical insights into the capabili-
%ties and limitations of current language models in
%accurately simulating complex political processes,
%paving the way for future advancements in the field.
%Our experimental results show that while the com-
%plexity of the coalition negotiations makes predict-
%ing the outcomes difficult, our proposed model can
%achieve reasonable results in delivering this task.
%
%We leverage Diplomacy as an abstract analog to real-world negotiation, providing methods for AI agents to negotiate and coordinate their moves in complex environments.
%
%
%\textbf{Future Work} 
%These models are mostly
%limited by their focus on surface-level interaction,
%neglecting the deeper historical and ideological in-
%fluences that shape negotiation dynamics
%
%
%\textbf{Main section.} A process of coalition formation is an equilibrium if at any date and at any going
%state, a coalitional move to some other state can be ‘‘justified’’ by the very same scheme applied in future: the coalition that moves must have higher present value (starting from the state it moves to) for each of its members, compared to (one-period) inaction under the going state. I
%
%\textit{For supporting rationality within subjective view.}
%Fix a PCF p; a state x; and a
%coalition S: Say that S has a (weakly) profitable move from x (under p) if there is
%yAFSðxÞ (with yax) such that viðy; pÞXviðx; pÞ for all iAS: S has a strictly profitable
%move from x if there is yAFSðxÞ such that viðy; pÞ4viðx; pÞ for all iAS:
%
%Limitation (Dyn. Coalition as process): Our interpretation is that each time period is an interval for which a coalition
%structure (and the associated actions and payoffs) remains a binding agreement.
%
%
%\textbf{Literature}
%By leveraging the reasoning capabilities of
%these models that are trained on large amount of
%data, recent research have shown that LLMs can
%outperform human annotation in detecting the po-
%litical affiliation of social media users (Törnberg,
%2023) and topic, sentiment, stance, and frame iden-
%tification (Gilardi et al., 2023)
%
%In more cooperative
%scenarios, Liang et al. (2023); Saha et al. (2023)
%demonstrated the effectiveness of leveraging mul-
%tiple agents arguments in solving a task.
%
%. Fu et al. (2023) used
%a LLM to provide feedback to other LLMs who
%are performing item purchase negotiation. Bianchi
%et al. (2024); Abdelnabi et al. (2023) applied LLM-
%based multi-round negotiation systems to various
%scenarios and showed their capability in reaching
%agreements. Contrary, to these works, the task of
%coalition negotiation requires capturing long-term
%complexities in multiple political parties prefer-
%ences and taking consequential actions, which has
%been overlooked in other related works by simpli-
%fying the simulation setups.
%
%LLMs, whilst powerful and capable of demonstrating some emergent properties,
%are not logical reasoners or planners [8]. H 
%[Kambhampati, S., Valmeekam, K., Guan, L., Stechly, K., Verma, M., Bhambri,
%S., Saldyt, L., Murthy, A.: Llms can’t plan, but can help planning in llm-modulo
%frameworks. arXiv preprint arXiv:2402.01817 (2024)]
%
%However, we believe that through careful
%dissection and reasonably rich service description, there is a path to utility from which
%further advances can be built.
%
%
%\section{Introduction} 
%
%\textbf{Coalition formation.} 
%
%
%Importance within multi-agent systems. Briefly, game-theory -- what's the main goal? multi-agent systems -- what's the main goal? RL -- main questions? While these approaches have enabled autonomous systems to solve problems like ..., applications involving natural language cannot be solved. 
%\ak{don't forget to define coalitions!}
%
%\textbf{Introduce diplomacy.} For instance, consider the game of Diplomacy. Explain the game, how coalitions form via negotiations natural language, how game has incomplete information, and bounded rational agents. Similar characteristics are observed in several multi-agent domains such as social media, ... . In such domains, how do we study the process of dynamic multilateral coalition formation?
%
%
%\textbf{LLM intro.} The emergence of Large Language Models (LLMs) have fundamentally altered the way we interact with ... 
%Recent work ... political coalitions, agreement detection, deception ... has opened promising directions. 
%However, LLMs, whilst powerful and capable of demonstrating some emergent properties, are not logical reasoners and often struggle to perform well at ... .
%
%
%\textbf{Mental Modeling} In incomplete information games such as diplomacy, agents must reason logically about perceptions of other players and incentives. 
%However, unlike traditional game theory models for incomplete information such as Bayesian games and hypergames, there is no methodical/procedural way for the players to know these perspectives. 
%These must be deduced from limited conversations that each player knows. 
%Neither the exsting coalition approaches nor  LLMs can accomplish these tasks [cite]. 
%
%
%\textbf{Problem.} We investigate dynamic multilateral coalition detection in multi-agent sequential interactions where agents negotiate coalitions through natural language conversations. 
%We decompose the problem (which problem?) in two stages: agreement detection and strategic reasoning. 
%
%
%The agreement detection stage answers the following question: \emph{given the dialogue between two agents, how to identify the issue being discussed by integrating prior knowledge about game dynamics and intents implicit in the dialogue?} For example, given a dialogue in Fig. ..., we expect the output to be ... . 
%To answer this question, we ... 
%
%
%
%The strategic reasoning stage answers the following question: \emph{given a set of potential agreements for each player from which that player can select a subset consistent with game dynamics identified by agreement detection stage, how to determine whether a player will honor or deviate away from a given agreement?} 
%Explain the approach taken to answer this --- hypergame theory and solution concepts. 
%
%
%\textbf{Diplomacy test-bed.} Throughout this paper, we expose the ideas using diplomacy as a use-case. This is because of two reasons: First, it is rich environment highlighting characteristics of interest. Second, existence of fine-tuned models, which enable us to focus on the integration between game-theory and LLMs, rather than development of these fine-tuned models. 
%
%
%\textbf{Contribution} Contributions of this paper are three-fold: 
%
%\begin{enumerate}
%	\item \textbf{Framework.} We introduce a novel framework that integrates large language models and game theory to predict dynamic multi-issue coalition formation in multi-agent systems where agents negotiate coalitions through natural language dialogue.
%	
%	\item \textbf{Agreement detection.} We develop a procedure that combines pre-trained language models with game dynamics to extract agreements from dialogues between agents, enabling the detection of evolving coalition structures in real-time interactions.
%	
%	\item \textbf{Rationalizability metric.} We propose a new metric based on subjective rationalizability from hypergame theory, which evaluates the likelihood that agents will adhere to agreements by accounting for their subjective views of the game and strategic uncertainty.
%\end{enumerate}
%
%
%\section{Introduction (v1)} 
%
%\textbf{Coalition detection + Motivation}
%Coalition detection in multi-agent systems is a critical area of research that addresses the dynamics of cooperation among agents to further mutually aligned interests. 
%While coalition detection has been studied extensively in game theory with applications to resource sharing, auctions, and to improve network performance, there is a growing interest to study coalition formation in rich language environments such as social media. 
%For instance, social media platforms serve as fertile ground for the formation of advocacy coalitions, where groups with shared beliefs can mobilize resources and strategies to influence public opinion and policy (Mir et al., 2022). 
%By employing coalition detection methods, researchers can better understand the mechanisms behind the spread of misinformation and the social dynamics that underpin collective belief systems, ultimately aiding in the development of strategies to counteract harmful narratives (Budán et al., 2023).
%
%
%\textbf{Problem.}
%We investigate coalition detection in multi-agent sequential interactions where agents negotiate coalitions through natural language conversations. In these settings, each agent selects actions based on the history of states, actions, and their own dialogue with other agents, but remains unaware of the content and existence of conversations between other pairs of agents. Unlike the traditional game-theoretic notion of coalitions as fixed partitions of players, we allow for multi-lateral coalitions where any agent can form temporary bilateral agreements with more than one agent, and which can change from one round to the next, adapting to the evolving strategic landscape of the interaction.
%
%
%\textbf{Limitations/Challenges.} 
%Traditional game-theoretical approaches are often inadequate for addressing the complexities of coalition detection in multi-agent sequential interactions, particularly in scenarios where agents negotiate coalitions through natural language conversations. Conventional game theory typically assumes fixed partitions of players and static coalitions, which do not align with the dynamic nature of interactions where agents can form temporary bilateral agreements that evolve over time Tosic (2016)Pinto et al., 2011). In our context, agents operate under incomplete information, as they are unaware of the conversations and negotiations occurring between other pairs of agents. This lack of transparency complicates the strategic decision-making process, as agents must adapt their actions based on their own dialogue history while also anticipating the potential actions of others (Bauso \& Basar, 2014; Rahwan et al., 2009). Furthermore, traditional models often rely on the assumption that coalition values are known and certain at the time of formation, which is not the case in open, dynamically changing environments where agents must navigate uncertainty and adapt to shifting strategic landscapes (Blankenburg et al., 2003; Bauso \& Timmer, 2008). Consequently, the rigid frameworks of classical game theory fail to capture the fluidity and complexity inherent in modern multi-agent systems, necessitating the development of more flexible and adaptive models that can accommodate the nuances of real-time negotiation and coalition dynamics (Ren et al., 2007).
%
%
%\textbf{Challenges by Language} \ak{Discuss with Andy.}
%
%\textbf{Approach.} 
%We introduce a framework that integrates pre-trained domain-specific language models with game-theoretic solution concepts to estimate the likelihood of a certain agreement being honored by both players. 
%Our framework has two stages. 
%The first stage comprises of \ak{@Andy: A statement on agreement detection.}
%The second stage comprises of a rationalizability detector module, which assigns a score between 0 and 1 to every agreement under discussion signifying its likelihood of being honored. 
%This score is determined by 
%
%
%\textbf{Contribution} Contributions of this paper are three-fold: 
%
%\begin{enumerate}
%	\item \textbf{Framework.} We introduce a novel framework that integrates large language models and game theory to predict dynamic multi-issue coalition formation in multi-agent systems where agents negotiate coalitions through natural language dialogue.
%	
%	\item \textbf{Agreement detection.} We develop a procedure that combines pre-trained language models with game dynamics to extract agreements from dialogues between agents, enabling the detection of evolving coalition structures in real-time interactions.
%	
%	\item \textbf{Rationalizability metric.} We propose a new metric based on subjective rationalizability from hypergame theory, which evaluates the likelihood that agents will adhere to agreements by accounting for their subjective views of the game and strategic uncertainty.
%\end{enumerate}
%
%
%
%\subsection{Related Work}
%
%\textbf{Coalition Games (Static + Dynamic)}
%
%\textbf{Coalition Logic}
%
%\textbf{Language Models}
%
%
%





%\hrule 
%
%
%
%Coalition formation is a common process occuring in multi-agent interactions.
%It has been well-studied in game theory, economics, reinforcement learning communities for applications including ... .
%Most of these works focus on the value of coalition under TU or NTU formulations.
%However, in several domains such as ..., ..., typically involving humans in the loop with other humans or AI systems, coalition are formed not only based on the value to an agent but also on communicated intents.
%This is because humans are sub-rational agents incapable of evaluating all possible
%(see trustworthy AI paper.)
%We refer to such domains, where coalition formation is based on as games/interactions with rich communication.
%
%
%
%
%
%Challenges in this problem (split into game theoretic and ):
%1. Intents are implicit in dialogue. They could be false (multi-negotiations).
%2. Coalitions could be multi-lateral.
%3. Coalitions are temporary.
%
%
%Two enabling factors are language models and deep Nash equilibrium finding algorithms with convergence guarantees.
%
%
%Discuss the recent advancement of language models. Intent models.
%
%
%Equilibrium searching algorithms, with anchors -- piKL.
%
%
%The main contribution of this paper is a game-theoretic framework that leverages the fine-tuned intent models and value models with solution concept of rationalizability to predict coalitions at every step in the sequential interaction.
%For this, we build upon the existing game of diplomacy for which intent and value models are available cite{?}.
%
%Explain the approach.
%
%State rationalizability metric and score. How it allows external observer to reason about likelihood of coalitions from external perspective.
%\textit{Hypergames useful if unawareness. Unawareness in our scenario because inaccessible conversation among players. Hence, player is unaware of there was conversation and if yes, what was the content. This is one way to model, which we choose. }
%
%Impact of our work - coalition formation in social media, human-AI interactions / negotiations.
%

