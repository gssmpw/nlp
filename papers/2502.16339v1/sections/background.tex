Diplomacy is characterized by both \emph{incomplete information} and \emph{unawareness}. 
In Diplomacy, players make decisions without full knowledge, as they may be unaware of message exchanges between other players or the content of those messages. 
As a result, a player's rationality must be assessed based on their subjective view of the game, shaped by their knowledge (c.f. \citep{kulkarni2021synthesis,kulkarni2020deceptive}).


%Both \emph{incomplete information} and \emph{unawareness} are key features of the Diplomacy environment. Diplomacy players will often be unaware whether other players have exchanged messages, and if so, what the content of those messages was. Unawareness also provides an effective way to model bounded rationality of the players. Due to large state and action space, players sometimes must make decisions after only exploring a subset of their own and other players' actions during their decision-making. In this case, we can exclude unexplored actions from a player's subjective view, and assume that the player is making rational decisions within this view of the game state, unaware of the explored actions (c.f. \citep{kulkarni2021synthesis}).
% In this case, the unexplored subset of actions can be excluded from the subjective view of the player and assume that player makes rational decisions within their subjective view.\textbf{}

%\textbf{Background on hypergame.} 
Hypergame is a game-theoretic model designed for games with incomplete information and unawareness  \citep{bennett1980hypergames,sasaki2012hypergames}.
In a hypergame, each player has a subjective view of their interaction, shaped by their knowledge of the game and others' perspectives.
This structure allows players to independently form subjective views and make decisions based on their own \emph{subjective game}, effectively capturing player unawareness within the model.

%Hypergame \citep{bennett1980hypergames} is a game-theoretic model for studying games with incomplete information. 
%While Bayesian games are the standard framework for incomplete information games, hypergames are considered more suitable for modeling unawareness \citep{sasaki2012hypergames}. 
%A hypergame is a \emph{game of games}, where each player has a subjective view of their interaction based on their information about the game and the other players' perspectives. 
%This structure enables players to independently hold subjective views, making decisions based on their own ``subjective game,'' thus capturing the unawareness of players within the model.
% A hypergame is a well-studied game-theoretic model used to study games with incomplete information, which is specifically suited for modeling unawareness. 
% Although Bayesian games [cite], which are considered the standard model of games with incomplete information, are capable of modeling unawareness, they are known to unsuitable to model unawarenes [cite]. 


% A hypergame is a game of games, with each constituent game being a player's subjective view of its interaction with other players given the information it has about the game and about the subjective views of other players.
% % In other words, hypergames theory relax the \emph{common knowledge} often assumed in the standard game theory [cite]. 
% The basic idea of hypergames is that each player is supposed to possess independently a subjective view about a game called their subjective game and make a decision based on it. This allows them to explicitly model unawareness of players, which is unique among such models.


%Unawareness manifests in games like Diplomacy in two ways. 
%On one hand, it appears naturally in the game when a player is \emph{unaware} whether other two players have exchanged messages and if yes then what was the content of these messages.
%On the other hand, we argue that \emph{unawareness} also provides an effective way to model bounded rationality of the players; when, due to large state and action space, players only explore a subset of their own and other players' actions during their decision-making. 
%In this case, the unexplored subset of actions can be excluded from the subjective view of the player and assume that player makes rational decisions within their subjective view.



% Formally, hypergames are defined inductively based on the level of perception of individual players. A level-$0$ (L0) hypergame is a game with complete, symmetric information, where the perceptual games of both players are identical to the true game. In a level-$1$ (L1) hypergame, at least one of the players misperceives the true game, but neither is aware of it. In this case, both players believe their perceptual game to be the true game and play according to their perceptual games, which are level-$0$ hypergames. In a level-$2$ (L2) hypergame, one of the players becomes aware of the misperception and is able to reason about its opponent's perceptual game. This definition can be extended to arbitrary hypergame levels.

Formally, hypergames are defined inductively based on players' levels of perception. A level-0 (L0) hypergame represents a game with complete, symmetric information, where both players have the same perception of the game, identical to the true game. In a level-1 (L1) hypergame, at least one player misperceives the game, but neither player is aware of this discrepancy. Each player believes their perceptual game is the true game and plays accordingly, with these perceptual games being level-0 hypergames. In a level-2 (L2) hypergame, one player becomes aware of the misperception and can reason about the other player's perceptual game. This concept can be extended to higher hypergame levels; however, in this paper, we restrict ourselves to L2-hypergames, which are most directly relevant to Diplomacy.



\textbf{Subjective rationalizability} \citep{sasaki2014subjective} is a solution concept for hypergames that evaluates the rationality of players' actions based on their subjective views of the game, considering their knowledge and beliefs about other players' perspectives and actions.


\begin{definition}[Subjective Rationalizability] \label{def:sr}
    Let $H^2 = \langle H_1^1, H_2^1 \rangle$ denote a L2-hypergame, where $H_i^1 = (G_1^i, G_2^i)$ is player $P_i$'s L1-hypergame and $G_1^i$ is the subjective game of $P_1$ as perceived by $P_i$. 
    Then, a strategy $\pi_i^{\ast,2}$ is said to be subjectively rationalizable for player $P_2$ if and only if it satisfies the following condition for all $\pi_i$: 
    \[
	   u^2_i(\pi_i^{\ast,2},\pi^{\ast,2}_{j}, x) \ge u_i^2(\pi_i,\pi^{\ast,2}_{j}, x),
	\]
    % 
	% Given a level-2 hypergame $H^2(x) = \langle H_1^1, H_2^1 \rangle$, strategy $\pi_i^{\ast,2}$ is \ac{sr} for player $2$ if and only if it satisfies, for all $ \pi_i \in \Pi_i$,
	% \[
	% u^2_i(h, \pi_i^{\ast,2},\pi^{\ast,2}_{j},x ) \ge u_i^2(h, \pi_i,\pi^{\ast,2}_{j},x),
	% \]
	where $(i,j)\in \{(1,2), (2,1)\}$
	% Note that in the case that P2's hypothesis $x$ is a distribution over $\Phi$, the utility is calculated based on the expectation, that is, $u_i^2(h, \pi_i,\pi_j,x) =\sum_{\varphi\in \Phi} x(\varphi) u_i^2(h, \pi_i,\pi_j,\varphi)$.
	% 
	% 
    and $x$ is a distribution over $\Phi$ representing $P_2$'s hypothesis over some aspect of $P_1$'s game.
    In this case, the utility is calculated based on the expectation, that is, $u_i^2(\pi_i,\pi_j,x) =\sum_{\varphi\in \Phi} x(\varphi) u_i^2(\pi_i,\pi_j,\varphi)$.
	The strategy $\pi_1^{\ast,1}$ is subjectively rationalizable for $P_1$ if and only if it satisfies the following condition for all $ \pi_1$,
	\[
	u_1^1(\pi_1^{\ast,1}, \pi_2^{\ast,2}, \varphi_1 ) \ge
	u_1^1(\pi_1, \pi_2^{\ast,2}, \varphi_1 ),
	\]
	where $\pi_2^{\ast,2}$ is subjectively rationalizable for $P_2$.
\end{definition}

\refDef{def:sr} enables evaluating when a player's strategy is rational within their own subjective view of the game. 
For $P_2$, a strategy is subjectively rationalizable if, given its information about $P_1$'s game ($H_2^1$), $P_2$ cannot improve their utility by choosing a different strategy. 
Specifically, $P_2$'s utility from its chosen strategy, given the other player's strategy and their own beliefs (represented by a distribution $x$), must be at least as high as the utility from any other strategy they might choose. 
Subjective rationalizability is understood similarly for $P_1$.
%Similarly, for $P_1$, their strategy is subjectively rationalizable if it maximizes their utility based on their own view of the game, given $P_2$'s strategy.


%Subjective rationalizability follows the idea that an agent makes decisions based on her own beliefs about what others will do. 
%Each agent tries to predict what others will do, while assuming that others are also predicting actions in the same way. 



% The concept of subjective rationalizability can be understood based on the following principle. 
% The lowest \andy{what does lowest mean here?} agent in a viewpoint would take a best response to actions which she thinks the other agents would choose. 
% When expecting the choices of the others, she considers that each of the other agents takes a best response to actions which she thinks the agent thinks the other agents would choose, and her inference goes on for further lower viewpoints. 
% When agent i makes a decision in this way, her choice can be predicted as a subjectively rationalizable action of viewpoint i.
% Thus we may also say it is agent i’s subjectively rationalizable action.

%\hrule 
%
%Intuitively, a hypergame is a game of games, and each game is associated with a player’s subjective view of its interaction with other players based on its own information and information about others’ subjective views. Hypergames are defined inductively based on the level of perception of individual players. A level-$0$ (L0) hypergame is a game with complete, symmetric information, where the perceptual games of both players are identical to the true game. In a level-$1$ (L1) hypergame, at least one of the players, say P2, misperceives the true game, but neither is aware of it. In this case, both players believe their perceptual game to be the true game and play according to their perceptual games, which are level-$0$ hypergames. In a level-$2$ (L2) hypergame, one of the players becomes aware of the misperception and is able to reason about its opponent's perceptual game.
%
%Hypergame theory deals with misperceptions of agents (deci-
%sion makers) in games by relaxing common knowledge often
%assumed in the standard game theory [1, 2]. It is the basic
%idea of hypergames that each agent is supposed to possess
%independently a subjective view about a game called her
%subjective game and make a decision based on it. 
%
%In game theory, Bayesian games are often referred to
%as the standard model to deal with incomplete information
%[3]. While a hypergamecan technically be reformulatedasa
%Bayesian game under specific conditions, the reformulation
%requires the agents to be aware of every possibility indeed
%relevant to the situation [4]. Therefore hypergames are unique
%in that they can directly deal with unawareness of agents.
%
%In order to predict an agent’s choice in one-shot hyper-
%game, that is, a hypergame played only once, several solution
%concepts have been proposed [8]. They are typically based
%on the following idea. First, an analyzer fixes the level of
%hierarchy of perceptions and finds out an “equilibrium” (e.g.,
%Nash equilibrium) in the subjective games of the lowest  level
%of the hierarchy. Then it is supposed that best responses are
%taken sequentially at each level. For example, consider a two-
%level hypergame played by two agents, i and j. According to
%the idea, in order to analyze agent i’s choice, we first need to
%know an equilibrium in the subjective game of viewpoint ji.
%If agent j’s some action constitutes an equilibrium there, then,
%expecting agent j would take the action, agent i chooses a best
%response to it in viewpoint i’s subjective game
%
%
%The notion of rationalizability to hypergames so as to examine the precise prediction of an agent’s choice in a hypergame.
%Subjective rationalizability
%is defined not for agents but for viewpoints, so, for instance,
%agent i can think of agent j’s choice as viewpoint ji’s
%subjectively rationalizable action.
%\ak{We restrict game to level-2 hypergame in this paper assuming that a player bases its decisions on only P2's perception constructed from ... Our choice of hypergame theory enables, in future, to relax this assumption and incorporate what P1 thinks about potential agreements between P2 and P3 when evaluating value of an agreement between P1 and P2 for the latter.}
%We, however, prove that, under a condition called inside common
%knowledge [6], subjective rationalizability is equivalent to
%rationalizability and show that an agent’s subjectively ration-
%alizable action can be easily derived by applying the result.
%
%
%The concept of subjective
%rationalizability can be understood based on the following
%principle. The lowest agent in a viewpoint would take a
%best response to actions which she thinks the other agents
%would choose. When expecting the choices of the others, she
%considers that each of the other agents takes a best response
%to actions which she thinks the agent thinks the other agents
%would choose, and her inference goes on for further lower
%viewpoints. When agent i makes a decision in this way, her
%choice can be predicted as a subjectively rationalizable action
%of viewpoint i.Thus we may also say it is agent i’s subjectively
%rationalizable action.
%
%The proposition describes the sufficient condition of the
%existence of subjectively rationalizable action for a particular
%viewpoint in a hypergame. \ak{Argue that subjectively rationalizable actions always exists in Diplomacy.}
