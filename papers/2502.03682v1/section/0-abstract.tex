Intimate Partner Infiltration (IPI)—a type of Intimate Partner Violence (IPV) that typically requires physical access to a victim’s device—is a pervasive concern in the United States, often manifesting through digital surveillance, control, and monitoring. Unlike conventional cyberattacks, IPI perpetrators leverage close proximity and personal knowledge to circumvent standard protections, underscoring the need for targeted interventions. While security clinics and other human-centered approaches effectively tailor solutions for survivors, their scalability remains constrained by resource limitations and the need for specialized counseling. In this paper, we present \sys, an \underline{A}utomated \underline{I}PI \underline{D}etection system that continuously monitors for unauthorized access and suspicious behaviors on smartphones. \sys employs a two-stage architecture to process multimodal signals stealthily and preserve user privacy. A brief calibration phase upon installation enables \sys to adapt to each user’s behavioral patterns, achieving high accuracy with minimal false alarms. Our 27-participant user study demonstrates that \sys achieves highly accurate detection of non-owner access and fine-grained IPI-related activities, attaining an end-to-end top-3 F1 score of 0.981 with a false positive rate of 4\%. These findings suggest that \sys can serve as a forensic tool within security clinics, scaling their ability to identify IPI tactics and deliver personalized, far-reaching support to survivors.