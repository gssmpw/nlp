\section{Related Work}\label{sec-7-related}
\subsection{Intimate Partner Violence (IPV)}
With recent growth in mobile smartphone, people are more often storing important private information in these devices. As a result, abusers tend to use the smartphone to conduct IPV behaviors to cause harm to their intimate partners. 
Research on intimate partner violence (IPV) can be categorized based on different subjects as follows:

\textbf{Focuses on Victims:} Harvon et al.\cite{havron2019clinical} proposed the use of clinical computer security for the treatment and protection of victims by conducting interviews and consultations to understand their situations in IPV.

Tseng et al.\cite{tseng2022care} discussed the enhancement of victims' sense of safety and the treatment of trauma caused by IPV through the development of care infrastructure. These approaches involve treating and researching victims through interviews and dialogue.

\textbf{Perpetrator's Perspective:} Freed et al.\cite{freed2018stalker} explored how perpetrators utilize technological means such as GPS trackers and audio monitoring to commit IPV, mainly by interviewing victims to understand their reactions to these technologies.

\textbf{IPV Media:} Thomas et al.\cite{thomas2021sok} conducted a comprehensive study on various forms of online harassment, including IPV, and proposed building legal and social frameworks to curb such harmful activities. Ceccio et al.\cite{ceccio2023sneaky} systematically investigated spy devices and tools sold by U.S. retailers that can be used for IPV, highlighting the need for defensive tools against these spy devices. Bellini et al.\cite{bellini2023digital} investigated online forums and proposed the influence of various tools, strategies, and online communities on the issue of Intimate Partner Surveillance (IPS).


\subsection{Continuous Authentication}
\takeaway{Version 1:}

In the realm of continuous authentication on mobile devices, various approaches have been employed, leveraging the diverse data available from smartphone sensors. A prominent approach involves the use of motion sensors—such as gyroscopes and accelerometers—which track behavioral information for identity recognition \cite{fereidooni2023authentisense}, \cite{centeno2018mobile}. This method has gained traction given the widespread inclusion of these sensors in modern mobile devices. Additionally, user touch interactions, including the frequency and pressure of finger touches on the screen, are utilized for authentication \cite{frank2012touchalytics}, \cite{xu2014towards}, \cite{zhao2013continuous}. This technique capitalizes on the unique patterns of how individuals interact with their phone screens, offering an intuitive means of user verification.

Furthermore, a hybrid strategy combining both motion and touch data has been developed to more comprehensively capture usage patterns, thereby enhancing the accuracy in distinguishing between legitimate users and unauthorized access. In recent advancements, an expanded range of modalities, including GPS and WiFi data, has been incorporated into the authentication process \cite{deb2019actions}, \cite{acien2019multilock}. Moreover, there is a growing focus on training Machine Learning models using smaller datasets, aiming to achieve practical and efficient implementation in real-world scenarios \cite{fereidooni2023authentisense}.

\takeaway{Version 2:}

Previous researches on continuous authentication based on biometric behavioral characteristics can be categorized as follows:

\textbf{Touchscreen-based User Authentication:} Centeno et al.\cite{centeno2018mobile} used Siamese neural networks and long short-term memory neural networks to learn users' touch gestures for efficient authentication.
DeRidder et al.\cite{deridder2022continuous} differentiated users by using dynamic information from multi-finger touches.

\textbf{Motion Sensor-based User Authentication:} Other studies have sought to authenticate users through various motion sensors standard in smartphones. Mekruksavanich et al.\cite{mekruksavanich2021deep} proposed the DeepAuthen framework, using accelerometers, gyroscopes, and magnetometers to characterize users' physical traits.
Weng et al.\cite{weng2023enhancing} improved identification accuracy using a semi-supervised deep learning framework based on motion sensor readings.

\textbf{Hybrid Authentication Methods:} Some research considers a combination of the above data types for user authentication. Reichinger et al.\cite{reichinger2021continuous} continuously recorded touchscreen and accelerometer information, using hidden Markov models to detect users.

