\section{Threat Model}\label{sec-3-threatModel}
% \takeaway{Use a table to show an IPV attacker vs. traditional attacker. Then focus on (1) Attacker's capacity
% and their capacity limits; scope is important: might not be good for physical violence, maybe helpful for surveillance;
% (2)Requirements of a successful attack. Cite papers to show the applicability of those attacks.
% What attackers can (not) do; under which conditions can (not) they achieve successful attack}

% Please add the following required packages to your document preamble:
% \usepackage[table,xcdraw]{xcolor}
% Beamer presentation requires \usepackage{colortbl} instead of \usepackage[table,xcdraw]{xcolor}
% Please add the following required packages to your document preamble:
% \usepackage[table,xcdraw]{xcolor}
% Beamer presentation requires \usepackage{colortbl} instead of \usepackage[table,xcdraw]{xcolor}

% We make following assumptions for our threat model. First, IPV abusers are non-tech expert, unable to hack into OS to acquire
% sensitive data. In this paper, their IPV behaviors are via physical access to victims' smartphones, and such access is already
% acquired (through compromise, and guess-out, etc.), meaning that all one-time authentications (passwords, face id, finger 
% print, etc.) fails to prevent their access. The abusers are free to manipulate the device, viewing application information, 
% uploading and downloading data.

% For the defending system, we assume that the monitoring data sources are trustworthy and intact. This is consistent with our
% previous assumption of low technical level of IPV abuser to conduct OS level attack such as spoofing and adversarial machine learning. 
% Additionally, the integrity of the system is not affected by the awareness of its existence of abusers,
% thus guaranteeing system functioning.

In this section we describe the scope and some assumptions of our threat model. We focus on the IPI behaviors that require physical access to mobile smatphones. IPI abusers cause harm to victims by having access to and interacting with the victims' smartphones stealthily. As shown in Table \ref{tab-ipvvstradition}, unlike traditional cybersecurity attacks, we assume that the abusers have authenticated access to victims' devices, e.g., by registering their biometrics or having access to passwords, so they directly can interact with the data and applications stored on the victim's smartphone. Aligned with previous research \cite{freed2018stalker}, we assume that abuser's have limited technical background, given that abusers come from the general population compared to cyber attackers. This means that abusing behaviors are restricted to user-interface interactions, e.g., by viewing the content on the phone or installing applications. Additionally, we assume that the stealthiness of our system is sufficient for not alerting abusers, so the system is safe from IPI attacks.

Another notable point is that our study does not apply to partners in extreme situations, i.e., the use of the detection system will escalate IPI behavior to a higher level such as severe physical or psychological violence. As discussed in \cite{freed2019my,havron2019clinical,tseng2022care}, IPI abusers may escalate and bring more harm to victims if they discover evidence of anti-IPI measures. Although our designed goals are stealthy and not alerting, the use of our tool needs the initial assessment from a security clinic. More harmful scenarios require intervention from external agencies, such as police, law enforcement, and clinical centers, which are out of the scope of this work.


%the discovery of anti-IPV actions the anti-IPV actions might bring more harm to the IPV victims and police, law agencies, and clinical therapy should engage in insted of detection tools which won't help at this point.






%running without awareness by the abusers, so the data sources and system itself are guaranteed to be safe from IPV attacks.