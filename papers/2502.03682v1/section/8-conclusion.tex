\section{Conclusion and Future Work}\label{sec-8-conclusion}

In this work, we presented \sys, an automated Intimate Partner Infiltration  detection system by continuously monitoring unauthorized access and suspicious behaviors through OS-level and physical signals on smartphones. Through a two-stage architecture that processes multimodal signals, \sys stealthily operates on the device while preserving user privacy. A short calibration phase tailors the system to individual user behaviors, which allows it to distinguish non-owner access attempts and identify fine-grained IPI activities accurately. Our evaluation with 27 participants demonstrated \sys’s effectiveness, achieving up to 0.981 F1 score and maintaining a low false positive rate of 4\%. These results highlight the potential of \sys to serve as a forensic tool for security clinics, enabling scalable assistance to IPV victims.

Looking ahead, future research could explore the feasibility of \sys in both proximate and remote IPV scenarios, where attackers may rely on physical closeness or operate from a distance. This includes examining wireless and wired network signals to identify hidden spy devices—such as covert cameras or compromised audio recorders—that function beyond the immediate scope of the targeted smartphone. Another avenue is to investigate collaborative detection strategies across multiple devices or platforms, further reinforcing \sys against the diverse range of tactics used in IPV.