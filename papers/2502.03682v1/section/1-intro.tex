\section{Introduction}

Intimate Partner Violence (IPV) is a prevalent issue in the United States, affecting approximately 47\% of women and 44\% of men during their lifetime\cite{leemis2022nisvs}. Perpetrators of IPV often monitor, control, and harrass victims through technology and physical devices that are prolific in our daily lives~\cite{ceccio2023sneaky,rogers2023technology,freed2018stalker,bellini2023digital}. For example, an IPV abuser might exploit a home router to monitor victim's smartphone~\cite{tseng2020tools} and Internet activity~\cite{freed2018stalker}, a GPS-tracker to give the abuser information about victims' real-time location, or a hidden camera to spy on the victim's daily activities~\cite{ceccio2023sneaky}.    Among these tactics, Intimate Partner Infiltration (IPI) is particularly problematic, as it often requires physical access to the victim's devices.

% The widespread adoption of mobile smartphones globally has led to increased frequency of use and a rich variety of applications enhancing people's lives. However, this trend also means that individuals are uploading more personal and sensitive information onto their devices, making smartphones a platform for intimate partner violence (IPV). 

% IPV refers to violent behaviors between individuals in an intimate relationship, including stalking, physical violence, and psychological abuse, all of which pose significant threats to physical and mental health. In recent years, there have been increasing instances of privacy breaches and surveillance of personal smartphones by intimate partners. 


IPI-related intrusions are unique compared to typical problems in computer systems security because of the vastly different characteristics of the threat model and perpetrator. As shown in Table \ref{tab-ipvvstradition}, IPI perpetrators typically arise from the general population, who may not have the same technical background as traditional intruders. However, the lack of technical expertise does not make defending against intrusive behaviors easier. On the contrary, intimate relationships frequently give abusers physical access to victims' personal devices, which often bypasses the need for sophisticated hacking. For instance, these abusers are often already registered on shared devices (e.g., fingerprints previously added to the device's authentication system), or can obtain access through educated guesses and coercion. This unique combination of \textbf{physical proximity} and \textbf{intimate knowledge of the victim's behaviors} allows abusers to \textbf{exploit trust rather than technical vulnerabilities}, creating challenges that are inherently different from traditional attacks.



% Unlike typical computer system intrusions, IPV-related intrusions have distinct characteristics. As shown in Table \ref{tab-ipvvstradition}, intruders in IPV scenarios are usually intimate partners with limited technical skills in computer intrusion. However, due to their intimate relationship, these perpetrators often easily obtain or already have access to the victim's smartphone, allowing them to bypass traditional authentication methods, access applications directly, steal privacy, and cause harm. While there is extensive research in the field of computer science addressing IPV, most studies focus on human factors and qualitative analysis, lacking automated, scalable and system-based solutions.

There has been extensive research in IPV or IPI across psychology, health, engineering, and computer science. Recently a class of approaches called clinical computer security\cite{havron2019clinical,tseng2022care} has gained traction for mitigating IPV or IPI. These approaches are administered through \textit{Security Clinics}~\cite{bellini2024abusive, tseng2022care, havron2019clinical} and leverage a combination of clinical interviews, consultations, and technical support to provide tailor interventions for IPV victims. However, such clinics are difficult to scale. For instance, it is challenging to expand the service into rural areas where limited human resources constrain its availability. %\textcolor{red}{There are currently only 3 Security Tech Clinics in the United States, despite almost half of the population experiencing forms of IPV~\cite{x}.}

In addition to logistical barriers, human factors also need careful consideration, further complicating scalability. IPI survivors often arrive at clinics carrying deep emotional scars and traumas from their experiences \cite{ramjit2024navigating}. These individualized needs necessitate detailed interviews and therapy sessions, making the work labor-intensive for experts and consultants. The emotional and psychological demands of these interactions can also lead to burnout among clinic staff, further limiting capacity.

A few automation tools~\cite{arief2014sensible} such as browser wipers or malware scanner have been adapted to assist with assessing survivors’ situations, which provides some level of relief and more comprehensive evaluations to the clinics. While these tools can provide a one-time screening, they are inadequate for continuous monitoring or addressing more sophisticated threats, such as privacy breaches or malicious configurations of personal devices. While these existing technologies can augment the services of security clinics, the nuanced and complex nature of IPI cases still requires human expertise for effective intervention.

% Therefore, it would be beneficial to have a novel automated solution for IPV context. \textbf{The first challenge} for such design would be the realizing great adaptability. Current tools lack scalability to differnt IPV survivors for the one-time constrain and the detecable content constarin. \textbf{The second challenge} is the need for stealthiness. The tool should function without abuser's awareness, otherwise more severe harm could happen to IPV victims. \textbf{The third challenge} is about achieving usable accuracy and low false alarms for users. High accuracy ..... And due to the emotional stress in the victims' heart, they could be sensitive to any suspicious event related to IPV, in this case making alarm reliable and authentic is highly crucial. \textbf{The fourth challenge} is computational efficiency, as design such robust system usually require massive computing power and time, and without sophisticated design, unduable power comsumption and over-heating would interfere with user's daily expeience on the device.

Therefore, it would be highly beneficial to develop an an automated and continuous solution tailored to the IPI context. Such a system must address several critical challenges to ensure effectiveness and user safety. \textbf{The first challenge} is achieving \underline{adaptability}. Current tools lack the flexibility to accommodate the diverse needs of IPI survivors due to their reliance on one-time evaluations. A scalable solution must adapt to varying scenarios and user-specific contexts, ensuring broad applicability. \textbf{The second challenge} is maintaining \underline{stealthiness}. The tool must operate discreetly, avoiding detection by abusers. If an abuser becomes aware of its existence, the victim could face heightened risks of harm. Stealthy design is therefore paramount to protecting users from escalation. \textbf{The third challenge} is ensuring \underline{high accuracy} and \underline{minimal false alarms}. IPI victims are often under significant emotional distress, making them particularly sensitive to perceived threats or suspicious events. An unreliable system with frequent false alarms could undermine trust in the tool and add unnecessary stress. \textbf{The fourth challenge} is \underline{computational efficiency}. Designing such a robust system typically demands significant computing resources, which could lead to high power consumption. Without careful design and optimization, this could interfere with the user's daily experience on the device, further limiting adoption and usability.


In this paper, we propose \sys, an Intelligent \underline{A}utomated \underline{I}PI \underline{D}etection system, for continuous monitoring and detection of potential IPI behaviors on mobile smartphones. \sys leverages streams of multimodal sensing data and a two-stage architecture to detect 1) non-owner access and 2) precise behaviors that are indicative of IPI behaviors. \sys adopts a short, 5-minute calibration phase to adapt to phone owners upon initial installment. We carefully design \sys to be ``invisible'' to attackers through a deceptive UI and constraining its access to only data streams that can be collected and processed in the background. \sys is also privacy preserving by performing inference and keeping private user information locally. Through a 27-person user study, we verify that \sys can reliably detect non-owners and fine-grained behaviors of IPI with up to 0.981 F1 score.

%, while operating at a power efficiency that is on-par with \textcolor{red}{xxx(e.g., MESSAGING APPLICATIONS) applications}. 











% Leveraging the multiple sensors available on smartphones allows for a more precise representation of users' behavioral biometrics when using their devices. Unlike traditional passwords, biometric behavior is difficult to mimic, hard to change, and typically unique to individuals. Thus, modeling user behavior for continuous authentication can effectively distinguish legitimate users from intimate partner violence (IPV) perpetrators attempting unauthorized access. 

% Therefore, it is necessary to develop a continuous authentication system based on user behavioral characteristics to prevent IPV on mobile platforms. In recent years, with the advancement of deep learning, there have been studies on using neural network models for user identity recognition\cite{kokal2023deep,pelto2023your,fereidooni2023authentisense}. However, these studies generally focus solely on authentication and do not specifically address the IPV context. Due to the sensitivity of IPV, if the detection system is discovered, it could potentially cause greater harm to the victim. Consequently, the design of an IPV-specific detection and verification system must emphasize security and reliability. Additionally, most deep learning-based authentication methods lack scalability and often struggle to handle new user feature datasets, typically requiring retraining of the models.

% Given these circumstances, this paper proposes a pre-training scheme using an encoder-decoder architecture to enhance system scalability. It also addresses the usability of the system on mobile devices, particularly the impact of neural network model operations on smartphone battery life and the associated security risks of the detection system.

% One challenge for the system design is realising prompt and power-saving running, due to our needs for reliable defense and limited battery life on mobile devices. 

% Another challenge stems from the need for safety. Since unexpected extra harm may occur if the abusers find out the defensive actions taken by the victims, we should enable the system to run as unobtrusively as possible to reduce the potential risk. 

% Moreover, the concern of system data privacy should be taken into consideration. To address these issues, we designed and developed an Android-based tool called AID (Intelligent Automated IPV Detection System) to achieve a well balance. \sys~collects multiple sensing data on user's mobile phone and executes multi-phase detection for presice IPV behavior detecton which may occur on the device. It adopts local model inferencing, which is crucial for preserving user private information in case of data leakage. We collected behavior data from \needverification{16} volunteers and the user identification F1 score is up to 95\%; for IPV behavior detection, \sys~achives up to 97\% accuracy on classifiying the potential IPV behavior we put forward.

\begin{table}[t!]
    \small
    \centering
    \caption{Comparison between IPI and traditional attackers.}
    \begin{tabular}{@{}lll@{}}
        \toprule
        \textbf{Attacker Profile} & \textbf{Traditional} & \textbf{IPI} \\ 
        \midrule
        Tech skill & Advanced & Less technical \\ 
        Physical Access & No & Mostly yes \\ 
        Passcode & Limited & Registered or guessable \\
        Defense & Patches or updates & Security clinics~\cite{bellini2024abusive, tseng2022care, havron2019clinical} \\
        \bottomrule
    \end{tabular}
    \label{tab-ipvvstradition}
\end{table}

\noindent
\textbf{We envision \sys as a ``forensic'' tool that can be used by security clinics to gain a deeper understanding about the individualized needs of victims and improve mitigation measures.} To summarize, our main contributions are:

\begin{itemize}[label=\raisebox{0.5ex}{\scalebox{1.2}{$\bullet$}}]
    \item We taxonomize IPI tactics and map them to both operating system (OS)-level and physical signals. Our analysis reveals that intimate couples or close friends exhibit statistically significant similarities in physical signatures compared to strangers.
    \item We propose \sys, an \underline{A}utomated \underline{I}PV \underline{D}etection system, for continuously monitoring IPV behaviors on personal smartphones. Through careful design of \sys's IPV detection mechanism, \sys is privacy preserving by keeping all data and processing locally, while remaining ``invisible'' to attackers by constraining its access to only data streams that can be collected and processed in the background. 
    
    \item To perform scalable detection of IPV behaviors, we propose a two-stage architecture that detects 1) phone usage by non-owners and 2) fine-grained activities that could be indicative of IPV behavior. This architecture adopts a short 5 min calibration phase to adapt to new phones and owners upon first installation.
        
    \item Through user a 27-person user study, we demonstrate that \sys can detect non-owners and fine-grained IPV-related behaviors with 0.981 F1 score and up to 97.5\% accuracy.
    
    %, while remaining power efficient on-par with \textcolor{red}{xxx(e.g., MESSAGING APPLICATIONS) applications}.
\end{itemize}


The rest of the paper is structured as follows. Section \ref{sec-2-background} gives an overview of previous studies on IPV and continuous learning for mobile smartphones and illustrates their limitations, which serves as the motivation for our work. In section \ref{sec-3-threatModel}, we illustrate how IPV behaviors could be mapped to signals available to the operating system in mobile smartphones. Section \ref{sec-4-sysdesign} introduces our core design \sys, focusing on improving its scalability and safety, which makes robust detection IPV on mobile devices feasible. Section \ref{sec-5-eval} details and analyzes our user studies and experiments.  Section \ref{sec-8-conclusion} summarizes and concludes this work, with some viable future research directions.


%\textcolor{red}{TBC...}


