
% Taxonomy of Intimate Partner Infiltration (IPI) behaviors, which classified at three granularities. At the highest level, excluding the general \textit{benign} behaviors, IPI behaviors can be divided into four \textit{categories}, which encompass eight general \textit{actions}. These eight actions can be further divided into 27 \textit{subactions}. The goal of \sys is to distinguish between categories (5-class), actions (9-class), and subactions (28-class) of IPI behavior.

\begin{table}[h]
    \small
    \centering
    \caption{Taxonomy of Intimate Partner Infiltration (IPI) behaviors, classified at three granularities. The goal of \sys is to distinguish between categories (5-class), actions (9-class), and subactions (28-class) of IPI behavior.}
    \label{tab:actionsubaction}
    \begin{tabular}{@{}lll@{}}
    \toprule
    \textbf{Category}      & \textbf{Action}               & \textbf{Subaction}                   \\ \midrule
    \multirow{1}{*}{Benign} & General                      & General                               \\ \midrule
    \multirow{4}{*}{Impersonation} 
                           & \multirow{4}{*}{Send content} & Send emails                           \\
                           &                              & Send messages                         \\
                           &                              & Send reviews                          \\
                           &                              & Comment                               \\ \midrule
    \multirow{11}{*}{Leakage} 
                           & \multirow{5}{*}{View account} & View account settings                 \\
                           &                              & Subscription details                  \\
                           &                              & Inspect order history                 \\
                           &                              & View browsing history                 \\
                           &                              & View payment settings                 \\ \cmidrule{2-3}
                           & \multirow{6}{*}{View content} & View emails                           \\
                           &                              & See one's post history                \\
                           &                              & Watch history                         \\
                           &                              & View messages                         \\
                           &                              & Inspect files                         \\ \cmidrule{2-3}
                           & Upload content               & Upload photo                          \\
                           &                              & Upload video                          \\ \midrule
    \multirow{12}{*}{Modification} 
                           & \multirow{5}{*}{Alter account settings} & Change profile photo          \\
                           &                              & Change email                          \\
                           &                              & Change username                       \\
                           &                              & Change password                       \\
                           &                              & Change address                        \\ \cmidrule{2-3}
                           & \multirow{3}{*}{Modify content} & Delete emails                     \\
                           &                              & Modify music list                     \\ \cmidrule{2-3}
                           & \multirow{3}{*}{Alter files} & Add a file                            \\
                           &                              & Delete a file                         \\
                           &                              & Modify a file                         \\ \midrule
    Spyware                & Software installation        & Software installation                 \\ \bottomrule
    \end{tabular}
\end{table}



\section{Background and Motivation}\label{sec-2-background}

%\shinan{We start by summarizing the attack vectors,then give some backgrounds on mitigation techniques, then why it's so challenging to scale the defense}

\subsection{IPV Attack Vectors}

An expanding body of research underscores how abusers exploit technology to harm their intimate partners~\cite{tseng2020tools,tseng2022care,chatterjee2018spyware,freed2019my,freed2018stalker,freed2017digital,matthews2017stories}. The consequences are far-reaching, including financial harm~\cite{bellini2023paying}, escalating to physical confrontations, and even resulting in homicide~\cite{southworth2005high}.


Among the tactics explored by ~\cite{freed2018stalker,tseng2020tools,woodlock2017abuse,bellini2023digital,ceccio2023sneaky}, three distinct types of technology-facilitated IPV have been identified: remote, proximate, and physical access. Remote tactics include actions such as distributed denial-of-service (DDoS) attacks, SMS bombing, unauthorized remote logins, and location tracking~\cite{freed2018stalker,stephenson2023s}. Proximate tactics involve methods requiring closer physical presence, such as the use of spy cameras, router monitoring, or other forms of nearby surveillance~\cite{ceccio2023sneaky}. Physical access tactics, on the other hand, encompass direct interactions with devices, such as installing spyware~\cite{chatterjee2018spyware} or creepware~\cite{roundy2020many}, accessing phone records, or deploying keylogging tools to monitor keystrokes. These tactics rely on distinct attack vectors. 

In this work, we introduce a taxonomy specifically focused on the physical access category of IPV, which we term Intimate Partner Infiltration (IPI). Physical access tactics are particularly challenging to detect, especially \textit{on smartphones}, because abusers often have legitimate access to these devices and may already know or have registered credentials such as facial recognition profiles or passcodes~\cite{tseng2020tools}. This inherent proximity and familiarity make detection more complex, as these actions can easily blend into normal usage patterns. Additionally, addressing physical access tactics requires more careful and nuanced design approaches, as confronting such behaviors carries a higher risk of escalating the situation, potentially putting the victim in greater danger~\cite{matthews2017stories, southworth2005high,bellini2023paying}.


Based on the literature ~\cite{tseng2020tools,woodlock2017abuse,bellini2023digital,ceccio2023sneaky}, we further refine and taxonomize IPI tactics into categories, actions, and subactions. The \textit{Benign} category includes general, non-malicious behaviors that resemble normal device usage and serve as a baseline for comparison. \textit{Impersonation} involves abusers pretending to be the victim by sending emails, messages, reviews, or comments to manipulate perceptions or harm reputations~\cite{tseng2020tools,woodlock2017abuse}. 
\textit{Leakage} refers to accessing or extracting private information, such as account settings, browsing history, messages, or files, and sometimes uploading content like photos or videos to violate privacy~\cite{ceccio2023sneaky}. \textit{Modification} includes altering device settings, changing account credentials, or modifying files, disrupting the victim’s sense of control and potentially causing emotional or psychological harm~\cite{bellini2023digital}. Lastly, \textit{Spyware} involves the installation of software to covertly monitor the victim’s activities, representing one of the most invasive and persistent forms of IPI~\cite{woodlock2017abuse}.

This taxonomy enables systems like \sys to classify IPI behaviors with increasing granularity, distinguishing between categories (5-class), actions (9-class), and subactions (28-class), supporting more precise detection and intervention strategies.


\subsection{Mitigation Methods and Challenges}
% \shinan{previous studies and mitigation methods (study, tools, facilities, gap between current continuous authentication setting vs. setting), challenges on scaling Security clinics up, challenges on very hard to distinguish patterns for close relationships }

Prior research on technology-facilitated IPV has mainly provided qualitative insights into victims’ experiences and perceptions of how perpetrators misuse technology. Freed et al.\cite{freed2017digital,freed2018stalker} examined the ways in which abusers deploy GPS trackers and audio surveillance tools, gathering details from victims to infer perpetrator methods. Bellini et al.~\cite{bellini2023digital} identified online forums and communities as accelerants for Intimate Partner Surveillance (IPS), emphasizing how readily available strategies and tools exacerbate the problem. Tseng et al.~\cite{tseng2020tools} provided measurement-based evidence of technology-driven abuses, illustrating the scope of these issues. Other research has also emphasized the need for legal frameworks and defensive tools to combat IPV and address the widespread availability of spy devices~\cite{thomas2021sok,ceccio2023sneaky}.

A range of mitigation solutions has emerged in response to these findings. Although basic tools exist to help users remove browser histories and delete digital traces to safeguard their privacy~\cite{arief2014sensible}, they do not address the wide spectrum of potential IPI abuses. Instead, \textit{security clinics}~\cite{havron2019clinical,tseng2022care} provide in-person consultations for survivors, offering tailored support to identify risks and rebuild a sense of safety. While these clinics have demonstrated significant promise, scaling their services beyond local communities presents considerable challenges. 
One \textit{major limitation} is their reliance on specific physical locations, which restricts access for survivors outside those areas. Furthermore, these clinics require the involvement of technical experts who possess specialized knowledge to address the complex and evolving nature of technology-facilitated abuse. Recruiting and training such experts, as well as ensuring ongoing support, is resource-intensive and difficult to sustain at scale. This combination of geographic constraints and reliance on highly skilled personnel highlights the barriers to replicating models like the Clinics to End Tech Abuse (CETA) in New York City~\cite{ceta} and the Madison Tech Clinic (MTC) in Madison~\cite{mtc} in broader contexts. These challenges underscore the need for scalable, accessible, and resource-efficient approaches to combat IPI on a wider scale.

Another highly promising yet underexplored approach for detecting IPV is continuous authentication. From a technical standpoint, user authentication on smartphones falls into two categories: \textit{one-time} and \textit{continuous} methods. While one-time verification (e.g., PINs, facial recognition) initially prevents casual intruders, it can be insufficient in IPV settings where abusers may already know these credentials~\cite{tseng2020tools,xu2014towards}. Continuous authentication, drawing on biometric or behavioral data (e.g., motion sensors~\cite{fereidooni2023authentisense,centeno2018mobile,liu2023amir}, touch interactions~\cite{frank2012touchalytics,xu2014towards,zhao2013continuous}, or hybrid approaches~\cite{deb2019actions,acien2019multilock}), offers stronger security by persistently checking whether the current user matches the legitimate owner. \textit{Nonetheless, such systems have been designed primarily for strangers or unauthorized outsiders.} When abusers are intimate partners, benign sharing (e.g., letting a child play a game) can resemble malicious infiltration, making it difficult to distinguish IPI behaviors from normal usage.

Our analysis in Table~\ref{tab:similarity} highlights this challenge, revealing that close friends or partners demonstrate significantly more similar behavioral signatures than strangers when using the same device. We derive these results by computing the similarity scores of individuals with close relationships versus those with no connections. These scores are calculated based on the cosine similarity of their behavioral representations, which are extracted using our feature extractor (a pretrained AutoEncoder; see Section~\ref{scalable} for implementation details). This metric quantitatively supports our hypothesis that individuals with close relationships, such as partners, exhibit significantly more similar behavioral signatures than strangers when using the same device. This similarity complicates the design of a user-detection system. Moreover, survivors often become wary of technology after experiencing trauma~\cite{freed2017digital}, complicating the adoption of new security tools. Consequently, while improvements in continuous authentication and security clinics represent meaningful progress, more nuanced and automatic approaches are needed to address the subtle ways intimate partners can exploit physical access without triggering suspicion or harming the survivor further.



% Smartphone access is protected by authentication mechanism to prevent unauthorized users....

% In contrast, biometric-based authentication systems offer enhanced security following initial access. These systems have several key advantages. Firstly, individual usage patterns, which vary subtly between users, create distinct operating system traits in the biometric data. These unique traits are instrumental in differentiating between users. Secondly, the complexity and uniqueness of these patterns make it challenging for unauthorized users to replicate the legitimate user's behavior, thereby bolstering security against unauthorized access. Lastly, the user-friendly nature of biometric systems, which operate discreetly in the background without interrupting the user's normal device interactions, contrasts favorably with one-time authentication methods that require explicit user actions.

% The combination of these features positions biometric-based authentication as not only a promising standalone security measure but also a complementary component to traditional authentication methods, enhancing overall device security in a user-centric manner.

% continuous user authentication challenges: never took the human factors into account, close parnter using the phone do not neccessarily mean it's IPI. they can still make benign behviors. For example, sharing the smartphone with a kid to play games.

% User authentication on mobile phones can be divided into one-time authentication methods and continuous authentication methods. The former verifies the user only once and does not repeat the verification, while the latter continuously verifies the user's identity, making it harder to breach.




% studies
% qualitative view~\cite{freed2017digital,dragiewicz2018technology,woodlock2017abus,freed2018stalker}
% survivors~\cite{matthews2017stories}
% measurement~\cite{tseng2020tools}
% \textbf{Perpetrator's Perspective:} Freed et al.\cite{freed2018stalker} explored how perpetrators utilize technological means such as GPS trackers and audio monitoring to commit IPV, mainly by interviewing victims to understand their reactions to these technologies.

% \textbf{IPV Media:} Thomas et al.\cite{thomas2021sok} conducted a comprehensive study on various forms of online harassment, including IPV, and proposed building legal and social frameworks to curb such harmful activities. Ceccio et al.\cite{ceccio2023sneaky} systematically investigated spy devices and tools sold by U.S. retailers that can be used for IPV, highlighting the need for defensive tools against these spy devices. Bellini et al.\cite{bellini2023digital} investigated online forums and proposed the influence of various tools, strategies, and online communities on the issue of Intimate Partner Surveillance (IPS).


% tools
% malware scanner
% Arief et al. ~\cite{arief2014sensible} seek to address this challenge via a vision
% of “sensible privacy” that takes into account IPV victims’ particular requirements, and suggest an
% app that would help a victim erase information on their device related to visits to IPV-relevant
% websites. Emms et al. [23 ] similarly suggested tools for helping people erase browser histories

% facilities
% security clinics~\cite{tseng2022care, havron2019clinical}

% Harvon et al.\cite{havron2019clinical} proposed the use of clinical computer security for the treatment and protection of victims by conducting interviews and consultations to understand their situations in IPV.

% Tseng et al.\cite{tseng2022care} discussed the enhancement of victims' sense of safety and the treatment of trauma caused by IPV through the development of care infrastructure. These approaches involve treating and researching victims through interviews and dialogue.


% Traditional one-time authentication methods, such as passcodes, facial recognition, or gesture patterns, can be vulnerable, especially in scenarios like Intimate Partner Violence (IPV) or where PINs are easily guessed \cite{tseng2020tools}, \cite{xu2014towards}.

% AMIR\cite{liu2023amir}

% In the realm of continuous authentication on mobile devices, various approaches have been employed, leveraging the diverse data available from smartphone sensors. A prominent approach involves the use of motion sensors—such as gyroscopes and accelerometers—which track behavioral information for identity recognition \cite{fereidooni2023authentisense}, \cite{centeno2018mobile}. This method has gained traction given the widespread inclusion of these sensors in modern mobile devices. Additionally, user touch interactions, including the frequency and pressure of finger touches on the screen, are utilized for authentication \cite{frank2012touchalytics}, \cite{xu2014towards}, \cite{zhao2013continuous}. This technique capitalizes on the unique patterns of how individuals interact with their phone screens, offering an intuitive means of user verification.

% Furthermore, a hybrid strategy combining both motion and touch data has been developed to more comprehensively capture usage patterns, thereby enhancing the accuracy in distinguishing between legitimate users and unauthorized access. In recent advancements, an expanded range of modalities, including GPS and WiFi data, has been incorporated into the authentication process \cite{deb2019actions}, \cite{acien2019multilock}. Moreover, there is a growing focus on training Machine Learning models using smaller datasets, aiming to achieve practical and efficient implementation in real-world scenarios 

% Challenge: through our analysis \shinan{please fill in a short process of how the similarity is calculated}

% human factor, so one challenge: leveraging resources like these is often
% challenging for survivors, many of whom are reluctant to engage
% with technology after their traumatic experiences~\cite{freed2017digital}


\begin{table}[t]
\small
\centering\caption{Behavior patterns of people in close relationships (e.g., friendships or intimate partners) are more similar than strangers, with statistical significance on the same phone.}
\label{tab:similarity}
\begin{tabular}{lll}
\toprule
\textbf{Device}                         & \textbf{Same (IPI Setting)} & \textbf{Different} \\
\midrule
\textbf{Close relationships} & 0.950(0.020)         & 0.850(0.033)         \\
\textbf{Strangers}                & 0.896(0.076)         & 0.820(0.110)         \\
\midrule
\textbf{t-stat}                   & 3.081                & 0.9426               \\
\textbf{p-value}                  & 0.0045               & 0.3802               \\
\bottomrule
\end{tabular}
\end{table}

    % \begin{table}[]
    % \centering
    % \caption{\tbd{Similarity_apps}}
    %         \label{tab:similarity_apps}
    %     \begin{tabular}{lllll}
    %     \toprule
    %                        & \textbf{CR-same} & \textbf{ST-diff} & \textbf{CR-diff} & \textbf{ST-diff} \\
    %     \midrule
    %     \textbf{Amazon}    & 0.925(0.033)     & 0.849(0.080)     & 0.773(0.034)     & 0.783(0.105)     \\
    %     \textbf{Gmail}     & 0.929(0.028)     & 0.835(0.085)     & 0.800(0.046)     & 0.760(0.122)     \\
    %     \textbf{Instagram} & 0.934(0.047)     & 0.829(0.088)     & 0.806(0.046)     & 0.758(0.117)     \\
    %     \textbf{Slack}     & 0.921(0.043)     & 0.832(0.090)     & 0.797(0.061)     & 0.765(0.115)     \\
    %     \textbf{Spotify}   & 0.918(0.043)     & 0.830(0.096)     & 0.796(0.063)     & 0.765(0.119)     \\
    %     \textbf{YouTube}   & 0.906(0.049)     & 0.828(0.093)     & 0.796(0.058)     & 0.767(0.115)     \\
    %     \bottomrule
    %     \end{tabular}
    %     \end{table}



