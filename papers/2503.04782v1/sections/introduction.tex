\section{Introduction}

Satisfiability Modulo Theories (SMT) is the problem of deciding the satisfiability of a first order logic formula with respect to certain background theories. This problem has significant applications across several fields, including model checking \cite{cordeiro2011smt}, software verification \cite{beyer2018unifying}, program analysis \cite{gavrilenko2019bmc}, test generation \cite{peleska2011automated}, and neural network verification \cite{pulina2012challenging}. 
The past two decades have witnessed a revolutionary improvement in the performance of SMT solvers, which has greatly driven the development of related fields.
Although current SMT solvers can quickly find a single solution for large-scale SMT formulas, many application scenes, such as testing, need to use a large number of different solutions for a given SMT formula. Multiple calls of SMT solvers with the same formula often output the same solution or similar solutions.
This capability of SMT solvers is clearly insufficient for the scenes that require a large number of diverse solutions.
Generating a set of highly diverse solutions for given SMT formula is referred to as the sampling problem, which is crucial in hardware/software testing and validation \cite{holler2012fuzzing,sen2005cute,jayaraman2009jfuzz,godefroid2008automated,bohme2017directed}.

Given that most programs utilize integer variables and perform arithmetic operations on them, Satisfiability Modulo Linear Integer Arithmetic, commonly abbreviated as \SMTLIA, is critically important in the context of software testing and verification \cite{mccarthy1993towards}. Specifically, \SMTLIA has diverse applications in automated termination analysis \cite{codish2012exotic}, sequential equivalence checking \cite{lopes2016automatic}, and state reachability checking under weak memory models \cite{gavrilenko2019bmc}. 
While past research on \SMTLIA has extensively studied the problem of generating a single solution~\cite{dutertre2006fast,dutertre2006integrating}, it has attracted less attention to the generation of a highly diverse set of solutions in the community.
For the \SMTLIA sampling problem, although using a solver to enumerate solutions (by adding blocking constraints) for a formula is operationally straightforward, this approach is generally costly and, after multiple invocations, most solvers tend to return similar solutions. This prevalence of similar solutions is largely attributed to the fact that many techniques in the solvers (e.g., the general simplex method for linear theories) have a preference for boundary values \cite{de2008z3,barbosa2022cvc5,cimatti2013mathsat5}. Therefore, there is a practical need to develop a method capable of efficiently generating a highly diverse set of solutions.

In previous studies, extensive research has been conducted on the sampling problem of Boolean Satisfiability (SAT) problems, employing techniques including but not limited to Markov-Chain Monte-Carlo (MCMC) \cite{kitchen2007stimulus,kitchen2010markov}, universal hashing \cite{meel2014sampling,meel2016constrained,ermon2013embed}, local search \cite{luo2021ls}, and knowledge compilation \cite{baranov2020baital}. However, due to the inherent limitations of propositional logic in capturing complex relationships, many problems in software engineering cannot be effectively encoded and solved using only propositional logic. In contrast, expressing these constraints with SMT formulas is generally more natural and effective \cite{barrett2018satisfiability}. Current research on SMT sampling focuses more on bit-vector theory (SMT(BV)) \cite{dutra2018smtsampler,dutra2019guidedsampler,shaw2024csb}. The main methods transform the sampling problem in bit-vector theory into a SAT sampling problem through bit-blasting \cite{shaw2024csb} or directly perform sampling at the SMT level using combinatorial mutation methods \cite{dutra2018smtsampler,dutra2019guidedsampler}. However, when the variables in the problem belong to infinite domains, such as integers numbers, bit-vector theory cannot accurately represent these issues. Therefore, it is particularly important to develop an effective sampling method under integer theory.

Recently, Matan et al. proposed a model-guided approximate sampling method for the \SMTLIA sampling problem, known as MeGASampler \cite{peled2023smt}. This method initially calls an existing SMT solver to solve the SMT formula, then adds additional constraints to the obtained solutions to generate an under-approximated formula of the original formula, and finally performs sampling based on this under-approximated formula. Currently, this approach is considered the best strategy for addressing the \SMTLIA sampling problem. 
However, this method faces several insurmountable issues: firstly, treating the SMT solver as a black box and only sampling on the under-approximated formulas derived from a single solution leads to high intrinsic similarity among the generated solutions; secondly, frequently invoking Maximum Satisfiability Modulo Theories (MAX-SMT) to increase seed randomness will undoubtedly result in significant overhead; finally, this method generates a limited diversity of solutions, resulting in inadequate coverage of the solution space and consequently reducing the efficiency of the testing process.

% finally, the diversity of solutions generated by this method is low, failing to effectively cover the solution space, thereby limiting the efficiency of the testing process.

This paper introduces a novel method named \HighDiv for addressing \SMTLIA sampling challenges. This method employs established SMT search techniques, primarily relying on local search methods and supplemented by the Conflict-Driven Clause Learning with Theory (CDCL(T)) algorithm.
When sampling an \SMTLIA formula, an initial sample is first generated using a local search-based sampling method with new techniques. Then, based on this initial sample, the assignments of certain variables are fixed, and these assignments are encoded as additional constraints, which are added to the original formula. Finally, the modified formula is solved using the stochastic CDCL(T) algorithm. Compared with previous \SMTLIA sampling methods \cite{peled2023smt}, \HighDiv significantly improves diversity when obtaining the same number of samples; and within the same runtime, the diversity of samples output by \HighDiv substantially exceeds that of previous methods. 

The primary reason for this improvement is that \HighDiv introduces variable-level randomness during the local search process, which is further complemented by the stochastic CDCL(T) algorithm. Specifically, \textit{HighCov} consists of two phases: local search and stochastic CDCL(T). In the local search phase, we design an initial assignment heuristic based on variable occurrence frequency and introduce a \textit{boundary-aware move} operator to enhance the variable-level randomness during the search process. In the stochastic CDCL(T) phase, we modify the branching heuristic and phase selection heuristic to their stochastic versions, further introducing randomness.

% Extensive experiments have demonstrated that \HighDiv achieves a higher diversity than the state-of-the-art \SMTLIA sampling methods. Our experimental benchmarks are derived from SMT-LIB \cite{BarFT-SMTLIB}, including some from real-world scenarios and others synthetic, designed to stress test SMT solvers. 

Extensive experiments have shown that \HighDiv achieves greater diversity compared to state-of-the-art \SMTLIA sampling methods. The benchmarks are sourced from SMT-LIB \cite{BarFT-SMTLIB}, comprising both real-world scenarios and synthetic cases;
actually, these benchmarks have been adopted by a recent study \cite{peled2023smt} on evaluating the performance \SMTLIA sampling methods.
% , specifically designed to test SMT solvers.
Our experimental results demonstrate that, when solving the \SMTLIA sampling problem, our \HighDiv method generates more diverse solutions compared to previous sampling techniques \cite{peled2023smt}. This is crucial for testing tasks, as it can significantly enhance the efficiency and effectiveness of the tests.