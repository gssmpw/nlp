\section{The \HighDiv{} Sampler}

\begin{figure*}[h!]
    \centering
    \includegraphics[width=\textwidth, trim=3cm 8cm 6cm 5cm, clip]{figures/HighDiv.pdf}
    \caption{Framework of \HighDiv}
    \label{fig:HighCov-framework}
\end{figure*}

% This section presents the \HighDiv algorithm, engineered to enhance diversity in its applications. The architecture of \HighDiv, constituting the overarching design of the algorithm, is illustrated in Figure~\ref{fig:HighCov-framework}. This framework integrates the core components and their interactions, establishing a comprehensive basis for our approach.
% The sampling framework of \HighDiv consists of two phases: the local search phase and the stochastic CDCL(T) phase. Specifically, the original formula is first simplified using the preprocessing tactics provided by Z3. Then, in the local search phase, the simplified formula is solved. Next, the variable assignments obtained from the solution of the local search phase are used to determine the values of certain variables, which are subsequently solved again using the stochastic CDCL(T) algorithm.

This section introduces the \HighDiv algorithm, specifically designed to enhance the diversity of solutions in complex optimization tasks. The architecture of \HighDiv, which defines the overall structure and flow of the algorithm, is illustrated in Figure~\ref{fig:HighCov-framework}. This diagram highlights the key components of the algorithm and their interactions, providing a clear and comprehensive framework for understanding the underlying methodology.

The \HighDiv sampling framework is composed of two distinct phases: the local search phase and the stochastic CDCL(T) phase. Initially, the original formula undergoes simplification through the preprocessing techniques provided by Z3, a widely used SMT solver. In the subsequent local search phase, the simplified formula is solved to obtain an initial set of solutions. These solutions, particularly the variable assignments, are then leveraged in the next phase. Specifically, the variable assignments derived from the local search solution are used to fix the values of certain variables, which are then re-evaluated in the stochastic CDCL(T) phase. In this phase, the stochastic CDCL(T) algorithm applies randomization to the decision-making process, enhancing the diversity of the generated solutions by exploring alternative assignments that might not have been captured in the local search phase.

This dual-phase approach allows for greater flexibility and exploration of the solution space, particularly in scenarios where local search methods may struggle to maintain diversity due to early fixation of variables. The introduction of the stochastic CDCL(T) phase mitigates these limitations and significantly enhances the robustness of the sampling process.

\subsection{Local Search Component}
In recent research on \SMTLIA solving, local search-based methods named LS-LIA have demonstrated strong solving capabilities \cite{cai2022local}. However, in the context of \SMTLIA sampling, local search methods have not been widely applied. Therefore, we consider employing a local search method for sampling, aimed at enhancing the diversity of solutions at the variable level during the search process.

In local search methods, selecting an appropriate initial assignment for variables is a crucial task. If the initial assignment is too uniform, it may lead to a lack of solution diversity; on the other hand, if the assignment is overly random, it could make solving certain complex instances more difficult. To strike a balance between solution diversity and solving efficiency, we propose an initialization method based on the frequency of variable occurrences.

Specifically, when the occurrence frequency of a variable exceeds a predefined threshold, we assign it a value of 0, and all other variables appearing in the same linear expression are also assigned a value of 0. The remaining variables are assigned random values.

Moreover, we observe that the critical move operator proposed in LS-LIA can only modify a variable's value to a boundary value, which significantly restricts the diversity of variable assignments during the search process. In fact, when performing an operation, a single variable may take multiple values when transitioning from the current state to the next. For example, for the inequality \(x \leq 5\), the value of \(x\) can lie in the range \([5, +\infty)\), and all these values will satisfy the literal. Therefore, we propose the \textit{boundary-aware move} operator (\textit{bam}), which is defined as follows:

\begin{definition}
The \textit{boundary-aware move (bam)} operator, denoted as $bam(x, \ell)$, assigns a random value consistent with the current state to the integer variable $x$, thereby satisfying its associated falsified literal $\ell$.
\end{definition}

\begin{example}
Given the \SMTLIA constraints, $\{\ell_1 := 3x - 15 \leq 0, \ell_2 := -x + 2y + 1 \leq 0\}$, with initial values $\{x= 0, y = 0\}$. As the literal $\ell_1$ forms a conjunction, we can determine the boundary for variable $x$ through $\ell_1$, i.e., $x \leq 5$. Thus, for the operation $bam(x, \ell_2)$, any random value of variable $x$ within $[1, 5]$ will satisfy the literal $\ell_2$.
\end{example}

\begin{algorithm}[h]
	\caption{Local Search Component of the \HighDiv}
        Preprocessing();
        InitializeByFrequency(); \\
        
	\While{$non\_imp\_steps \le L \times P_i$}{
		\If{$\alpha$ satisfies $F$}{
			\Return $\alpha$;
		}
		\uIf{$\exists$ decreasing bam operation in falsified clauses}{
			$op$ $:=$ select such an operation with the highest $score$;
		}
		\uElseIf{$\exists$ decreasing bam operation in satisfied clause}{
			$op$ $:=$ select such an operation with the highest $score$;
		}
		\Else{
			update clause weights according to the PAWS scheme; \\
			$c := $ a random falsified clause with integer variable; \\
			$op :=$ the $bam$ operation, chosen where $dscore$ is maximized;
		}
		$\alpha := \alpha$ with $op$ performed;
	}
	% \Return $\alpha$;
\end{algorithm}

Local Search Component of the \HighDiv, as shown in Algorithm 3.

Moreover, during operation selection, we refined the two-level selection heuristic in LS-LIA to better address the sampling problem. When scores are tied, the \textit{Rank} of the corresponding atomic formula is used to break the tie (lines 5, 7). Lines 9 to 11 involve updating clause weights and performing random walks to break local optima. Ultimately, the PLS-LIA algorithm returns a solution set \textit{S} with significant diversity.

\subsection{Stochastic CDCL(T) Component}

In previous research on SMT solving, CDCL(T) and local search methods have demonstrated significant complementarity \cite{cai2022local,zhang2024deep}. The primary motivation for introducing the stochastic CDCL(T) method is that, during the local search phase's preprocessing, variables involved in equality constraints are often fixed prematurely. This early fixation limits the ability to introduce diversity in the solutions, making it difficult to explore the solution space effectively. By using the stochastic CDCL(T) algorithm, which incorporates randomization in the branching and phase selection, we aim to overcome this limitation and improve the diversity of generated solutions. This approach enables more flexibility in the search process, especially for SMT instances where local search methods struggle due to the rigid structure imposed by early variable assignments.

We observed that local search methods are generally inefficient when solving SMT formulas with deep Abstract Syntax Trees (ASTs). In contrast, the CDCL(T) algorithm, relying on a powerful CDCL engine for fast reasoning at the propositional logic level, can handle such SMT formulas more effectively. Inspired by previous SAT sampling work based on the stochastic CDCL algorithm \cite{golia2021designing}, we therefore attempt to solve instances that are challenging to resolve via local search by using the stochastic CDCL(T) algorithm. We implemented the stochastic CDCL(T) algorithm for sampling within Z3. Specifically, we modified the \textit{branching heuristic} and \textit{phase selection heuristics} in Z3. At each decision point (Algorithm 1, line 5), instead of using a fixed heuristic, a variable is randomly chosen from the set of unassigned variables, and its phase is also randomly selected.