\section{Experimental Evaluation}

In this section, we first introduce the experimental setup, including the benchmarks used for analysis and the methods previously employed for \SMTLIA sampling. We then describe the current metrics used to evaluate \SMTLIA sampling methods. Finally, we analyze the experimental results.

\subsection{Experiment Setting}

To evaluate our sampling algorithm, we utilized the same benchmarks that were used to evaluate the MeGASampler. 
The benchmarks are from the LIA logic within SMT-LIB \cite{BarFT-SMTLIB} and the developers of MeGASampler have filtered the benchmarks based on the following criteria to exclude unreasonable ones: (1) benchmarks marked as unsatisfiable or unknown; (2) benchmarks for which no sampling technique can generate at least 100 samples; and (3) benchmarks that take more than one minute to solve using an SMT solver \cite{peled2023smt}. In total, there are 345 benchmarks related to LIA, distributed across 9 benchmark directories. 

In our experiments, we compare \HighDiv with the state-of-the-art \SMTLIA sampling tool, MeGASampler \cite{peled2023smt}, as well as SMTSampler(Int), the integer logic variant of SMTSampler \cite{dutra2018smtsampler}.

\textbf{MeGASampler} is a recently proposed \SMTLIA sampling tool and is also the most advanced tool for \SMTLIA sampling. As noted in its original paper, MeGASampler achieves higher coverage compared to SMTSampler(Int) on most LIA benchmarks. Additionally, MeGASampler implements two distinct strategies, both of which we evaluate in our experiments. 

\textbf{SMTSampler(Int)} has been shown to achieve higher coverage than MeGASampler on certain benchmarks, as noted in the paper \cite{peled2023smt}. Therefore, it will also be evaluated in our experiments.

The diversity of solutions is determined by how comprehensively the sampling results cover the solution space. We have adopted an evaluation metric proposed in prior research, namely the coverage of internal nodes in Abstract Syntax Trees (ASTs). Specifically, this method considers Boolean nodes as one bit, and for integer nodes of any size, only the last 64 bits are considered. For a bit in a variable, if it is sampled as both 1 and 0 in the sample set, we consider that bit to be covered \cite{peled2023smt}. Coverage is defined as the ratio of covered bits to the total number of bits. The intuition here is similar to path coverage in software testing, where the effectiveness of test cases is measured by the extent to which they cover different code paths in an application. Similarly, for the \SMTLIA sampling problem, this coverage metric reflects the extent to which the sample set explores various logical paths in the testing tasks.

All experiments were conducted on a computer equipped with an Intel(R) Xeon(R) W-2133 CPU @ 3.60GHz and 64GB of RAM. During the benchmark sampling for \SMTLIA, we adopted the experimental settings from previous SAT sampling studies \cite{luo2021ls}, setting the effective sample size, k, to 1000.

To maintain consistency with the subsequent sampling algorithms, we deliberately disabled certain preprocessing techniques in both Z3 and LS-LIA. This decision ensures that the results obtained across different algorithms remain comparable and not influenced by differing preprocessing strategies. For MeGASampler, we adhered to the default settings as recommended by the original authors, ensuring that the implementation aligns with their intended configuration. In the case of SMTSampler(Int), we directly utilized the integer logic version developed by Matan et al.\cite{peled2023smt}.

\subsection{Analysis of Results}

In this part, we first conduct a comprehensive comparison of \HighDiv and its competitors under two experimental conditions: fixed time and fixed sample generation size. Subsequently, we perform an in-depth validity analysis of the core algorithm in \HighDiv.

\subsubsection{Coverage Comparison Within Fixed Time Limits}

In order to comprehensively compare previous \SMTLIA sampling methods, we followed the experimental setup of Matan et al., setting the time limit to 15 minutes \cite{peled2023smt}.

\begin{table}[htbp]
	\centering
	\small  % 将字体缩小到small
	\begin{tabular}{@{}l cccc @{}}
		\toprule
		Benchmarks & \multicolumn{4}{c}{Coverage (\%)} \\
		\cmidrule(lr){2-5}
		& \HighDiv & MeGA & MeGA\textsuperscript{b} & SMTint \\
		\midrule
		CAV2009-slacked 	& \textbf{92.68} & 70.08 & 44.88 & 63.97 \\
		CAV2009				& \textbf{76.63} & 44.09 & 69.93 & 55.50 \\
		bofill-sched-random & 12.89 & \textbf{13.14} & 9.56 & 9.92   \\
		bofill-sched-real   & 10.86 & \textbf{11.46} & 9.69 & 8.97   \\
		convert  			& 13.56  & 9.27  & 8.48  & \textbf{17.01} \\
		dillig        		& \textbf{93.41} & 37.45 & 89.76 & 44.74 \\
		pb2010 				& \textbf{4.66}  & 4.22  & 4.35  & 2.97   \\
		prime-cone       	& \textbf{75.31} & 46.14 & 30.36 & 46.28  \\
		slacks        		& \textbf{95.11} & 71.39 & 47.04 & 64.01  \\
		% 增加其他基准
		\bottomrule
	\end{tabular}
	\caption{Comparative Results (Averaged) Across the Benchmarks (Fixed Time 900s).}
\end{table}

\begin{figure*}[!t]
	\centering
	\begin{subfigure}[b]{0.3\textwidth}
		\includegraphics[width=\textwidth]{figures/time_limit_900_HD_vs_mega.pdf}
		\caption{\HighDiv vs MeGA}
		\label{fig:sub2_1}
	\end{subfigure}
	\begin{subfigure}[b]{0.3\textwidth}
		\includegraphics[width=\textwidth]{figures/time_limit_900_HD_vs_megab.pdf}
		\caption{\HighDiv vs MeGA\textsuperscript{b}}
		\label{fig:sub2_2}
	\end{subfigure}
	\begin{subfigure}[b]{0.3\textwidth}
		\includegraphics[width=\textwidth]{figures/time_limit_900_HD_vs_smt.pdf}
		\caption{HighDiv vs SMTint}
		\label{fig:sub2_3}
	\end{subfigure}
	\caption{Comparative Coverage Performance of \HighDiv Against Competitors (t = 900 seconds).}
	\label{fig:time_limit_900}
\end{figure*}


\begin{figure*}[!htbp]
	\centering
	\begin{subfigure}[b]{0.3\textwidth}
		\includegraphics[width=\textwidth]{figures/num_limit_1000_HD_vs_mega.pdf}
		\caption{\HighDiv vs MeGA}
		\label{fig:sub2_1}
	\end{subfigure}
	\begin{subfigure}[b]{0.3\textwidth}
		\includegraphics[width=\textwidth]{figures/num_limit_1000_HD_vs_megab.pdf}
		\caption{\HighDiv vs MeGA\textsuperscript{b}}
		\label{fig:sub2_2}
	\end{subfigure}
	\begin{subfigure}[b]{0.3\textwidth}
		\includegraphics[width=\textwidth]{figures/num_limit_1000_HD_vs_smt.pdf}
		\caption{\HighDiv vs SMTint}
		\label{fig:sub2_3}
	\end{subfigure}
	\caption{Comparative Coverage Performance of \HighDiv Against Competitors (k = 1000).}
	\label{fig:time_limit_900}
\end{figure*}

Table 1 shows the comparison of average coverage between \HighDiv, MeGASampler, and SMTSampler(Int) across 9 selected benchmark folders. These 9 benchmark folders, selected by Matan et al., each consist of 15 benchmark files randomly chosen from 9 different directories in SMT-LIB \cite{BarFT-SMTLIB}. This selection method is reasonable, as benchmarks within the same directory generally exhibit similar characteristics \cite{peled2023smt}. To save space, following the approach of Matan et al., only the average values for these 9 selected benchmark folders are displayed in Table 1 \cite{peled2023smt}. In presenting the experimental results, the best results for coverage are highlighted in \textbf{bold}. For detailed results of each file within the benchmark folders, scatter plots are utilized. Further details can be found in Figure 2. It can be observed that within the same runtime, the sample sets generated by \HighDiv have a higher coverage in most benchmark tests compared to MeGASampler and SMTSampler(Int).

\subsubsection{Comparison of Coverage for Fixed-Size Solution Sets}

It is important to note that in real testing tasks, solutions obtained through sampling often form part of the test cases. Given that a single round of testing is usually time-consuming, it is common to consider using a fixed-size set of test cases for testing. Therefore, we consider conducting a coverage comparison on a fixed-size solution set to demonstrate the advantages of \HighDiv in testing scenarios.

\begin{table}[!t]
	\centering
	\small  % 将字体缩小到small
	\begin{tabular}{@{}l cccc@{}}
		\toprule
		Benchmarks & \multicolumn{4}{c}{Coverage (\%)} \\
		\cmidrule(lr){2-5} 
		& \HighDiv & MeGA & MeGA\textsuperscript{b} & SMTint \\
		\midrule
		CAV2009-slacked     & \textbf{93.56} & 38.02 & 23.57 & 63.36 \\
		CAV2009             & \textbf{76.63} & 34.73 & 40.64 & 52.00 \\
		bofill-sched-random & 12.97 & \textbf{13.20} & 11.16 & 12.30 \\
		bofill-sched-real   & 11.29 & \textbf{11.52} & 10.38 & 11.00 \\
		convert             & 13.16  & 4.04  & 6.81  & \textbf{19.88}  \\
		dillig              & \textbf{93.41} & 28.25 & 41.89 & 41.01  \\
		pb2010              & \textbf{5.13}  & 4.82  & 5.09  & 4.72   \\
		prime-cone          & \textbf{75.31} & 28.49 & 21.94 & 40.60  \\
		slacks              & \textbf{95.11} & 39.72 & 25.41 & 62.98  \\
		% 增加其他基准
		\bottomrule
	\end{tabular}
	\caption{Comparative Results (Averaged) Across the Benchmarks (Fixed Size 1000).}
\end{table}

We have referred to previous SAT sampling research designs and decided to set the solution set size at $k = 1000$ \cite{luo2021ls}. It should be noted that due to the stringent constraints of some benchmarks, it was not possible to generate the target number of valid solutions within one hour, so we included all solutions from these benchmarks in our analysis. 

From Table 2, it can be seen that under the condition of $k = 1000$, \HighDiv significantly outperforms MeGASampler and SMTSampler(Int) in terms of coverage in most benchmark tests. Additionally, by comparing Tables 1 and 2, we find that in the \textbf{CAV2009}, \textbf{dillig}, \textbf{prime-cone}, and \textbf{slacks} benchmarks, the coverage stabilizes when the sample size reaches 1000. This phenomenon demonstrates that \HighDiv still maintains good coverage in a fixed-size set of solutions. However, when the number of allowed generated solutions is reduced, the coverage of MeGASampler significantly decreases, while the impact on SMTSampler(Int) is relatively smaller.

For each benchmark file within the benchmark folders, we still present a scatter plot (see Figure 2) comparing the coverage of \HighDiv with its competitors (MeGASampler and SMTSampler(Int)).

\subsubsection{Effect of Core Algorithmic Components}

\HighDiv integrates three core algorithmic mechanisms: initialization based on variable occurrence frequency, boundary-aware move operator, and the stochastic CDCL(T) algorithm. To assess the effectiveness of these mechanisms, we modified \HighDiv by removing each of these components sequentially, thereby creating three distinct versions of \HighDiv, named var1, var2, and var3 respectively.

\begin{table}[htbp]
\centering
\begin{tabular}{l|p{1.3cm}|p{1.3cm}|p{1.3cm}|p{1.3cm}}
\hline
 & \textbf{var1} & \textbf{var2} & \textbf{var3} & \HighDiv\\
 & \textbf{Cov (\%)} & \textbf{Cov (\%)} & \textbf{Cov (\%)} & \textbf{Cov (\%)} \\
\hline
\textbf{k=1,000} & 48.38 & 51.12 & 50.32 & \textbf{52.95}\\
\textbf{t=900} & 48.23 & 51.07 & 50.27 & \textbf{52.79}\\
\hline
\end{tabular}
\caption{Comparative Results for Different \HighDiv Settings}
\label{tab:comparison_results}
\end{table}

Table 3 summarizes the average coverage achieved by \HighDiv and its three variants under a 15-minute sampling time limit, as well as under a constraint of 1000 solutions, across nine directories. For brevity, in Tables 3, var1, var2, and var3 are used to denote \HighDiv-var1, \HighDiv-var2, and \HighDiv-var3, respectively. Both tables show that \HighDiv achieves higher sampling coverage than the variants under both the fixed solution set size of 1000 and the 15-minute time limit. Therefore, the results in Tables 3 demonstrate the effectiveness of each core algorithmic mechanism in \HighDiv. Additionally, a comparison of Tables 2 and 3 reveals that each variant of \HighDiv demonstrates a higher coverage on most benchmarks compared to MeGASampler and SMTSampler(Int).


