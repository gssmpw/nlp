\documentclass[letterpaper]{comjnl}
\pdfoutput=1

\usepackage{booktabs} 
\usepackage[ruled]{algorithm2e}
\usepackage[nocompress]{cite}
\usepackage{multirow}
\usepackage{graphicx}
\usepackage{amsmath}
\usepackage{subfig}
\usepackage{amsthm}
\usepackage{url}
\usepackage{amsmath}

\setlength{\pdfpagewidth}{8.5in} \setlength{\pdfpageheight}{11in}
\addtolength{\topmargin}{-0.6in}


\begin{document}


%------------------------------------
% Title
%------------------------------------


\title[Work-Efficient Parallel Non-Maximum Suppression Kernels]{Work-Efficient Parallel Non-Maximum Suppression Kernels}


\author{David Oro}
\affiliation{Universitat Polit\`ecnica de Catalunya, C. Jordi Girona, 1-3, 08034, Barcelona, Spain}
\email{david.oro@upc.edu}

\author{Carles Fern\'andez}
\affiliation{Herta Security, C. Pau Claris, 165 4B, 08037, Barcelona, Spain}

\author{Xavier Martorell}
\affiliation{Universitat Polit\`ecnica de Catalunya, C. Jordi Girona, 1-3, 08034, Barcelona, Spain}

\author{Javier Hernando}
\affiliation{Universitat Polit\`ecnica de Catalunya, C. Jordi Girona, 1-3, 08034, Barcelona, Spain}


\shortauthors{D. Oro \emph{et al}}

\received{15 August 2019}
\revised{24 March 2020}
\accepted{10 July 2020}


\keywords{Non-maximum suppression; Object detection; GPU computing; Parallel computing}


%------------------------------------
% Abstract
%------------------------------------


\begin{abstract}
In the context of object detection, sliding-window classifiers and single-shot Convolutional Neural Network (CNN) meta-architectures typically yield multiple 
overlapping candidate windows with similar high scores around the true location of a particular object. Non-Maximum Suppression 
(NMS) is the process of selecting a single representative candidate within this cluster of detections, so as to obtain a unique 
detection per object appearing on a given picture. In this paper, we present a highly scalable NMS algorithm for embedded GPU 
architectures that is designed from scratch to handle workloads featuring thousands of simultaneous detections on a given picture. 
Our kernels are directly applicable to other sequential NMS algorithms such as FeatureNMS, Soft-NMS or AdaptiveNMS 
that share the inner workings of the classic greedy NMS method. The obtained performance results show that our parallel NMS algorithm 
is capable of clustering 1024 simultaneous detected objects per frame in roughly 1 ms on both Tegra X1 and Tegra X2 on-die GPUs, while 
taking 2 ms on Tegra K1. Furthermore, our proposed parallel greedy NMS algorithm yields a 14x-40x speed up when compared to 
state-of-the-art NMS methods that require learning a CNN from annotated data. 
\end{abstract}

\maketitle


%------------------------------------
% Journal text
%------------------------------------


%%%%%%%%%%%%%%%%%%%%%%%%%%%%%%%%%%%%%%%%%%%%%%%%%%%%%%%%%%%%%%%%%%%%%%%%%%%%%%%%
%2345678901234567890123456789012345678901234567890123456789012345678901234567890
%        1         2         3         4         5         6         7         8

\documentclass[letterpaper, 10 pt, conference]{ieeeconf}  % Comment this line out if you need a4paper

%\documentclass[a4paper, 10pt, conference]{ieeeconf}      % Use this line for a4 paper

\IEEEoverridecommandlockouts                              % This command is only needed if 
                                                          % you want to use the \thanks command

\overrideIEEEmargins                                      % Needed to meet printer requirements.

%In case you encounter the following error:
%Error 1010 The PDF file may be corrupt (unable to open PDF file) OR
%Error 1000 An error occurred while parsing a contents stream. Unable to analyze the PDF file.
%This is a known problem with pdfLaTeX conversion filter. The file cannot be opened with acrobat reader
%Please use one of the alternatives below to circumvent this error by uncommenting one or the other
%\pdfobjcompresslevel=0
%\pdfminorversion=4

% See the \addtolength command later in the file to balance the column lengths
% on the last page of the document

% The following packages can be found on http:\\www.ctan.org
%\usepackage{graphics} % for pdf, bitmapped graphics files
%\usepackage{epsfig} % for postscript graphics files
%\usepackage{mathptmx} % assumes new font selection scheme installed
%\usepackage{times} % assumes new font selection scheme installed
%\usepackage{amsmath} % assumes amsmath package installed
%\usepackage{amssymb}  % assumes amsmath package installed
\usepackage{cite}
\usepackage{amsmath,amssymb,amsfonts}
\usepackage{algorithmic}
\usepackage{booktabs}
\usepackage{graphicx}
\usepackage{textcomp}
\usepackage{xcolor}
\usepackage{etoolbox}
\usepackage{graphicx}
\usepackage[skip=0.333\baselineskip]{caption}
\usepackage{subcaption}
\usepackage{multirow}
\usepackage{multicol}
\usepackage{tabularx}
\usepackage{hyperref}
\usepackage{romanbar}
% \usepackage[bb=px]{mathalpha}

\setlength{\parskip}{0pt}

% \title{\LARGE \bf
% A Stereo Vision Perception System Integrating Depth Estimation, Low Light Enhancement and Semantic Segmentation based on Nvidia Jetson Xavier AGX
% }

\title{\LARGE \bf
VDT-Auto: End-to-end Autonomous Driving with VLM-Guided Diffusion Transformers}

\author{
    Ziang Guo \and 
    Konstantin Gubernatorov \and
    Selamawit Asfaw \and
    Zakhar Yagudin \and
    Dzmitry Tsetserukou
    \thanks{The authors are with the Intelligent Space Robotics Laboratory, Center for Digital Engineering, Skolkovo Institute of Science and Technology, Moscow, Russia
    \tt \{ziang.guo, Konstantin.Gubernatorov, Selamawit.Asfaw, Zakhar.Yagudin, d.tsetserukou\}@skoltech.ru}
}


\begin{document}

\maketitle
\thispagestyle{empty}
\pagestyle{empty}

%%%%%%%%%%%%%%%%%%%%%%%%%%%%%%%%%%%%%%%%%%%%%%%%%%%%%%%%%%%%%%%%%%%%%%%%%%%%%
\begin{abstract}
In autonomous driving, dynamic environment and corner cases pose significant challenges to the robustness of ego vehicle's decision-making. To address these challenges, commencing with the representation of state-action mapping in the end-to-end autonomous driving paradigm, we introduce a novel pipeline, VDT-Auto. Leveraging the advancement of the state understanding of Visual Language Model (VLM) with diffusion Transformer-based action generation, our VDT-Auto parses the environment geometrically and contextually for the conditioning of the diffusion process. Geometrically, we use a bird’s-eye view (BEV) encoder to extract feature grids from the surrounding images. Contextually, the structured output of our fine-tuned VLM is processed into textual embeddings and noisy paths. During our diffusion process, the added noise for the forward process is sampled from the noisy path output of the fine-tuned VLM, while the extracted BEV feature grids and embedded texts condition the reverse process of our diffusion Transformers. Our VDT-Auto achieved $0.52$ m on average L2 errors and $21\%$ on average collision rate in the nuScenes open-loop planning evaluation, presenting state-of-the-art performance. Moreover, the real-world demonstration exhibited prominent generalizability of our VDT-Auto. The code and dataset will be released at \href{https://github.com/ZionGo6/VDT-Auto}{\textit{https://github.com/ZionGo6/VDT-Auto}}.
\end{abstract}

\section{Introduction}

% 
% 
The widespread integration of communication networks and smart devices in modern control systems has increased the vulnerability of industrial systems to online cyber-attacks, e.g., Industroyer, Blackenergy, etc \citep{osti_1505628}.
% Modern control systems have seen a large push to include communication networks and smart devices to increase performance, made possible by improvements in communication device cost and energy consumption. This trend has been coupled with the usage of open-standard communication protocols among industrial control systems, making them vulnerable to online cyber-attacks such as Industroyer, Blackenergy, etc \citep{osti_1505628}. 
To counter this, methods have been developed to improve security by achieving attack detection, mitigation, and monitoring, among others \citep{sandberg2022secure}. This paper focuses on active attack diagnosis to mitigate stealthy attacks. 
%
%\subsection{Literature review}

Active diagnosis techniques rely on the inclusion of additional moduli to control systems
% inclusion within the control system of additional moduli 
to alter the behavior of the system compared to information known by the attacker. 
For instance, the concept of additive watermarking was introduced in \cite{mo2015physical}, where noise signals of known mean and variance are added at the plant and compensated for it at the controller. 
This compensation, however, is not exact, causing some performance degradation. Thus, trade-offs between performance and detectability  are necessary \citep{zhu2023detection}.
% A later work \citep{zhu2023detection} designs the watermark signal by trading performance for detection. Thus, although additive watermarking serves as a good detection scheme, they endure performance losses even in the nominal case. 

In encrypted control \citep{darup2021encrypted}, the sensor data is encrypted, sent to the controller, and then operated on directly. Encrypted input signals are sent back to the plant for decryption. Although encryption is widespread in IT security, in control systems it presents some concerns, such as the introduction of time delays \citep{stabile2024verifiable}, while it may present inherent weaknesses \citep{alisic2023model}.
% they are not preferred as they introduce time delays \citep{stabile2024verifiable} which can cause instability, and some encryption schemes can be very weak  \citep{alisic2023model}. 

In moving target defense \citep{griffioen2020moving}, the plant is augmented with fictitious dynamics, known to the controller. The plant output is transmitted to the controller along with the fictitious states over a network under attack. 
The additional measurements then aide in the detection of attacks. 
This comes at the cost of higher communication bandwidth needs, which increases rapidly with the dimension of the augmented systems.
% Since the dynamics of the fictitious dynamics are exactly known to the controller, the attack is detected easily. However, when the scale of the system increases, the communication bandwidth used by moving the target defense approach increases rapidly. 

Other recently proposed works include two-way coding \citep{fang2019two}, a weak encryuption technique, and dynamic masking \citep{abdalmoaty2023privacy}, which enhances privacy as well as security, have been shown to be effective against zero-dynamics attacks.
% Two-way coding \citep{fang2019two} and dynamic masking \citep{abdalmoaty2023privacy} are other recently proposed approaches. Two-way coding is another form of weak encryption technique whilst dynamic masking proposes an architecture that enhances both privacy and security. These schemes are shown to be effective against zero dynamics attacks but remain to be studied for other classes of attacks. 
% Recent extensions include \citep{mukherjee2021secure,ramos2024privacy}.
% Some other works which are related are \citep{mukherjee2021secure}, an extension of \cite{fang2019two}. The work \citep{ramos2024privacy} is an extension of moving target defense for multi-agent systems. 
Furthermore, filtering techniques for attack detection are proposed by \cite{murguia2020security,hashemi2022codesign,escudero2023safety}, while not focusing on stealthy attacks.
% The works \citep{murguia2020security,hashemi2022codesign,escudero2023safety} develop filtering techniques to guarantee safety, without being focused on stealthy covert attacks.

Multiplicative watermarking (mWM) has been proposed by the authors as a diagnosis technique \citep{ferrari2020switching}. mWM consists of a pair of filters on each communication channel between the plant and its controller; the scheme is affine to weak encryption, whereby ``encoding'' and ``decoding'' are done by changing signals' dynamic characteristics through inverse pairs of filters. This enables original signals to be recovered exactly, and thus does not lead to performance degradation.
% A multiplicative watermark is an affine to a weak encryption technique, through which the signal is ``encoded'' by a filter, changing its dynamic behavior. The use of inverse pairs means that the original signal can be recovered, through ``decoding'' via an inverse filter. As such, differently to techniques based on additive watermarking, no performance is lost due to the injection of noise, and there are no bandwidth limitations.

%\subsection{Contributions}
One of the critical features of multiplicative watermarking is that to detect stealthy attacks, the mWM filter parameters must be switched over time. In this paper, an algorithm to optimally design the mWM parameters after a switching event is presented, enhancing detection performance, without changing the switching time.
% This is done without changing the switching time, which is taken as given.

\textcolor{black}{
To formalize the filter design problem, we suppose the defender is interested in optimal performance against adversaries injecting covert attacks with matched system parameters \citep{smith2015covert}, including the mWM parameters prior to the switch. This scenario represents a worst case where malicious agents can take full control of the system while remaining undetected.
Thus, the attack strategy is explicitly included within the formulation of the closed-loop system, and the mWM filters are chosen by solving an optimization problem minimizing the attack-energy-constrained output-to-output gain (AEC-OOG) \citep{anand2023risk}, a variation of the output-to-output gain proposed in  \cite{teixeira2015strategic}.
}
The main contributions of this paper are:
% We consider an adversary injecting a covert attack with matched system parameters \citep{smith2015covert}, i.e., an attacker with full knowledge of the control system parameters, including those of the mWM filters before the switch. This scenario is taken as a worst case, as it has been shown that this class of attacks can be made stealthy. To quantitatively define a cost, the output-to-output gain (OOG) \citep{teixeira2015strategic} is leveraged,
% a metric introduced to evaluate the impact of an additive attack in a control system. %Specifically, OOG evaluates the worst-case performance loss that an attacker injecting an undetectable attack can obtain. 
% Here, the maximum performance loss caused by a stealthy adversary with limited energy is taken, the attack-energy-constrained OOG (AEC-OOG) \citep{anand2023risk}. The main contributions of this paper are:
\begin{enumerate}
%[label=\alph*.]
\item The problem of optimally designing the switching mWM filters is formulated as an optimization problem, with the AEC-OOG is taken as the objective;%where the AEC-OOG is taken as the impact metric; 
\item The worst-case scenario of a covert attack with exact knowledge of plant and mWM filter parameters is embedded within the design problem;
% The optimization problem is defined to incorporate the worst-case scenario of a covert attack with exact knowledge of plant and mWM filter parameters;
\item The feasibility of the optimization problem is shown to be dependent only on stability conditions; 
\item A solution scheme is proposed to promote randomization of the mWM filter parameters such that an eavesdropping adversary cannot remain stealthy.
\end{enumerate} 

This builds on the results of \cite{ferrari2020switching}, where the focus was on the design of the switching protocols, rather than the parameters themselves.
Compared to previous work \citep{gallo2021design}, this paper introduces an optimization problem which is always feasible (thanks to the use of AEC-OOG in the objective), while also considering a more sophisticated class of covert attacks, where the presence of watermark is known to the adversary. 
Moreover, this paper poses a different objective than \citep{zhang2023hybrid}; indeed, while \citep{zhang2023hybrid} provided a design strategy to ensure certain privacy properties, in this paper we address the problem of optimal parameter design following a switching event.


%\subsection{Organization}
The rest of the paper is organized as follows. 
After formulating the problem in Section~\ref{sec:PF}, we propose our design algorithm in Section~\ref{sec:main}, and analyze its properties. It is then evaluated through a numerical example in Section~\ref{sec:NE}, and concluding remarks are given Section~\ref{sec:Con}.
% We provide the problem background in Section~\ref{sec:PF}. We formulate the design problem in Section~\ref{sec:main}, together with an analysis of its properties. The proposed algorithm is evaluated through a numerical example in Section \ref{sec:NE}. Concluding remarks are offered in Section \ref{sec:Con}.

\section{Framework Overview}

\begin{figure*}
    \centering
    \includegraphics[width=0.95\linewidth]{Figs/Overview.png}
    \caption{\textbf{Framework overview of VDT-Auto.} At each time step, the surrounding images are encoded by the BEV encoder to provide the geometric feature grids of the scenario. A front image from the surrounding images is analyzed by our fine-tuned VLM to provide the contextual information of the conditions. Based on the BEV feature grids and VLM output, we construct the conditional latents for our diffusion Transformers, where the BEV feature grids and VLM's detection and advice are embedded and VLM's path proposal is sampled for conditioning. In Section \Romannum{3}, we introduce our noise sampling approach in details. Finally, our diffusion Transformers denoise the VLM's path proposal, conditioning on the geometric feature grids of the scenario and the contextual information from our fine-tuned VLM.}
    \label{fig:overview}
\vspace{-0.3cm}
\end{figure*}

\subsection{BEV Encoder}

In Fig. \ref{fig:overview}, our BEV encoder is based on LSS \cite{philion2020lss, hu2021fiery}, where the surrounding camera images from the $T$ time steps are lifted into the BEV feature grids. $F^{k}_{t}\in\mathbb{R}^{(C_{f} + D_{d}) \times H \times W}$ represents the extracted features of the $k$-th camera at time $t$ from the image backbone, where $F^{k}_{t,C_{f}} \in \mathbb{R}^{C_{f} \times H \times W}$ is the contextual features and $F^{k}_{t,D_{d}} \in \mathbb{R}^{D_{d} \times H \times W}$ represents the estimated depth distribution. Then the contextual feature map in height dimension $F'^{k}_{t}$ is computed as $F^{k}_{t,C_{f}} \otimes F^{k}_{t,D_{d}}$. According to the nuScenes camera setup \cite{caesar2020nuscenes}, with the intrinsics and extrinsics of the cameras, $F'^{k}_{t}$ is then aggregated and weighted along the height dimension into the ego-centered coordinate system to obtain the BEV feature grids $G_{t} \in \mathbb{R}^{C_\text{state} \times H \times W}$ at time $t$, where $C_\text{state}$ is the number of state channels.

\subsection{VLM Module}

For our work, Qwen2-VL-7B is used to bridge the input of sensory data and the output of contextual conditions \cite{wang2024qwen2vl}. In Qwen2-VL, Multimodal Rotary Position Embedding (M-RoPE) is applied to process multimodal input by decomposing rotary embedding into temporal, height, and width components, which equips Qwen2-VL with powerful multimodal data handling capabilities. In our VDT-Auto, the supervised fine-tuning of Qwen2-VL-7B is carried out by feeding a front image of surrounding cameras and system prompts, expecting the output of the description of the detection, the structured advice of the behavior of the ego vehicle, and the proposal of a path. To achieve supervised fine-tuning, we constructed our fine-tuning dataset by extracting ground truth information from the nuScenes dataset \cite{caesar2020nuscenes}. In Section \Romannum{3}, we will introduce more details about our dataset construction and supervised fine-tuning.

\subsection{Diffusion Prerequisites}

In Fig. \ref{fig:overview}, we show our entire VDT-Auto pipeline, where the feature grids $G_{t} \in \mathbb{R}^{C_\text{state} \times H \times W}$ of the BEV encoder and the contextual output $S_t$ of Qwen2-VL-7B including the description of the detection and structured advice on the behavior of the ego vehicle are encoded as state conditions for the diffusion process. Thus, in our designed diffusion Transformers, the conditioned policy $\pi_{\theta}(A_t | G_t, S_t)$ predicts the denoised path $A_t = (a^0_t, a^1_t, \ldots,a^n_t)$ of length $n$, conditioned on both the current BEV features $G_t$ and contextual embeddings $S_t$ \cite{bao2023onemmdi, han2024emma, reuss2024mdt}. During training, the proposal of a path based on supervised fine-tuning of Qwen2-VL-7B at time $t$ pairing with current BEV features $G_t$ and contextual embeddings $S_t$ to form the training set, where our diffusion Transformers aim to maximize log-likelihood $\ell_{\text{training}}$ throughout the training set,

\vspace{-0.3cm}
\begin{equation}
    \ell_{\text{training}} = \underset{\theta}{\arg\max}{}_{(a_t^i, g_t^i, s_t^i) \in (A'_t, G_t, S_t)} \log {\pi_\theta}({a_t^i | g_t^i, s_t^i}),
\end{equation}
where $a_t^i, g_t^i, s_t^i$ are sampled from our constructed training set. We extract the noise distribution $\sigma_{\text{VLM}}$ from the path output of the supervised fine-tuned Qwen2-VL-7B to construct the noisy path dataset $A'_t$ by adding the sampled noise from the extracted noise distribution $\sigma_{\text{VLM}}$ to the ground truth path $A_{gt}$ of nuScenes.

\par In Section \Romannum{3}, we demonstrate that the noise distribution $\sigma_{\text{VLM}}$ of the path proposal from our fine-tuned Qwen2-VL-7B is treated as a normal distribution, where we examine the extracted noise using One-Sample Kolmogorov-Smirnov test for both the $x$ and $y$ coordinates of the paths \cite{Kstest}. 

\subsection{Loss Functions}

Our diffusion Transformers predict the denoised path $A_t$ conditioned on current BEV features $G_t$ and contextual embeddings $S_t$. Therefore, the loss function is defined as follows.

\vspace{-0.3cm}
\begin{equation}
\begin{split}
    \mathcal{L}_{\text{train}} = \mathcal{L}_{\text{MSE}} (\pi_{\theta}(A_t | g_t, s_t, \boldsymbol{\epsilon}), A_{gt}) + \\
    \mathcal{L}_{\text{MSE}}(\sum_{j=1}^{n} a^j, \sum_{j=1}^{n} a_{gt}^j),
\end{split}
\end{equation}
where $\pi_{\theta}$ is our trained diffusion Transformers. Under the conditions of encoded BEV features $g_t \in G_t$, contextual embeddings $s_t \in S_t$, and added noise $\boldsymbol{\epsilon} \in \sigma_{\text{VLM}}$, the first part of our loss function is the mean squared error between the path prediction $A_t$ and the ground truth path from nuScenes $A_{gt}$. Besides, the second part of our loss function is the mean squared error between the cumulative sum of the waypoints $a^j \in A_t$ and $a_{gt}^j \in A_{gt}$.



\section{Methodology}

We based our analyses on the labeled data created in previous work~\cite{sanei2023characterizing}. The dataset distinguished 305 usability issues from five popular OSS projects (Jupyter Lab,
Google Colab, CoCalc, VSCode, and Atom) and identified their posters. In this paper, we focus on individuals who have ever posted a usability issue in that dataset. 

\subsection{Discovering the Role of Issue Posters}\label{sec: Discovering_role}

To detect the background of the usability issue posters in the dataset, we checked each user's \textit{Profile page} on GitHub, examining their bios, shared personal websites, LinkedIn pages, and/or shared resumes. If they have not shared these information, we searched for their LinkedIn profiles using their full names to extract details on their backgrounds and expertise. We considered their job titles posted in the information acquired this way and categorized them into (1) UX professionals, (2) managers, (3) data scientists, and (4) developers. UX professionals were defined as those indicating positions such as \textit{UX designer} and \textit{user interface and user experience designer}.

Among the 224 usability issue posters in the dataset, we were able to identify the role of 180 users. Within those 180 users, 121 (67.2\%) were developers, 34 (18.9\%) identified as data scientists, 21 (11.7\%) held managerial positions, and only four (2.2\%) were UX professionals. The UX professionals included one male contributed to \textit{VSCode}, another male contributed to \textit{Atom}, and two involved in \textit{Jupyter Lab} project, one male and one female. Notably, there were no UX professionals involved in \textit{CoCalc} and \textit{Google Colab} projects in our data sample. For easier referencing, in the following we call the UX professionals of VSCode as \textit{VSCode\_pro}, Atom \textit{Atom\_pro}, male of Jupyter Lab as \textit{Jupyter\_pro\_M} and female as \textit{Jupyter\_pro\_F}.

\subsection{Characteristics of Issues Posted by UX Professionals (RQ1)}

Once we identified the roles of the usability issue posters, we extracted all the issues posted by the four UX professionals across the five OSS projects. Next, we analyzed the extracted issues by adopting the following steps. First, following the approach outlined in \cite{sanei2023characterizing}, we labeled each issue with either usability or non-usability; and for each usability issue, we identified the main \textit{usability dimension} touched by the issue using the ten Nielsen heuristics~\cite{nielsen2005ten}. Then, similar to \cite{sanei2021impacts}, we identified the specific \textit{sentiment} and \textit{tone} expressed by the UX professionals when posting the usability issues. In our study, the sentiment captures the valence of the emotion that includes three categories (positive, negative, and neutral), while the tone describes emotion with seven affective factors (excited, frustrated, impolite, polite, sad, satisfied, and sympathetic). Subsequently, we analyzed the \textit{argument structure} of the usability issues to better understand the discursive device that the issue posters adopted to convince other discussion participants. We particularly identified whether a \textit{claim} and a \textit{premise} appeared in a usability issue post, using criteria proposed in prior work~\cite{skitalinskaya_learning_2021, wachsmuth_argumentation_2017, dowden1993logical}. Statements were considered as claims if they explicitly indicate the position or stance of the issue posters to the discussed usability issues; and premise means that a statement contains reasoning, evidence, or examples that support a stance. We compared how the above characteristics (i.e., usability dimensions, sentiments, tones, and argument structures) differed in issues posted by UX professionals and those without UX expertise.

\subsection{UX Professionals' Purpose Following Up on Issues (RQ2)}

% After investigating how UX professionals posted the usability issues, we recognized the importance of understanding their participation afterwards, particularly in following up on the discussion threads of the issues they posted. 
Thus, we first isolated comments made by the UX professionals posted to the usability issues they created within the datasets. Then, we employed an inductive content analysis~\cite{wamboldt1992content, Hsieh2005} and categorized the various purposes behind their contributions in posting each comment. For our analysis, the \textit{purpose} specifies the distinct goal that a particular comment serves within the context of the discussion thread. The purpose of a comment may vary based on its content and the immediate objective of the issue posters to write in the discussion to address one specific comment posted by another contributor. We grouped the identified purposes into themes through an iterative approach conducted by the two authors.


\section{Experiments}

In this section, we empirically compare the proposed algorithm on both sequence windows and time windows with existing methods.
\paragraph{Datasets} For the sequence-based model, we used two synthetic datasets and two cross-language datasets. The statistics of the datasets are provided in Table \ref{table:statistics}:

\begin{table}[t]
    \centering
    \caption{The statistics of the datasets. The datasets satisfy $1 \leq \|\vx\|\|\vy\| \leq R $.}
    \label{table:statistics}
    \begin{tabular}{|c|c|c|c|c|c|}
    \hline
        Dataset & $n$ & $m_x$ & $m_y$ & $N$ & $R$ \\ \hline
        SYNTHETIC(1) & 100,000 & 1,000 & 2,000 & 50,000 & 65 \\ \hline
        SYNTHETIC(2) & 100,000 & 1,000 & 2,000 & 50,000 & 724 \\ \hline
        APR & 23,235 & 28,017 & 42,833 & 10,000 & 773 \\ \hline
        PAN11 & 88,977 & 5,121 & 9,959 & 10,000 & 5,548 \\ \hline
        EURO & 475,834 & 7,247 & 8,768 & 100,000 & 107,840 \\ \hline
    \end{tabular}
\end{table}

\begin{itemize}
    \item Synthetic: The elements of the two synthetic datasets are initially uniformly sampled from the range (0,1), then multiplied by a coefficient to adjust the maximum column squared norm $R$. The X matrix has 1,000 rows, and the Y matrix has 2,000 rows, each with 100,000 columns. The window size is set to 50,000.
    \item APR: The Amazon Product Reviews (APR) dataset is a publicly available collection containing product reviews and related information from the Amazon website. This dataset consists of millions of sentences in both English and French. We structured it into a review matrix where the X matrix has 28,017 rows, and the Y matrix has 42,833 rows, with both matrices sharing 23,235 columns. The window size is 10,000.
    \item PAN11: PANPC-11 (PAN11) is a dataset designed for text analysis, particularly for tasks such as plagiarism detection, author identification, and near-duplicate detection. The dataset includes texts in English and French. The X and Y matrices contain 5,121 and 9,959 rows, respectively, with both matrices having 88,977 columns. The window size is 10,000.
\end{itemize}
We evaluate the time-based model on another real-world dataset:
\begin{itemize}
    \item EURO: The Europarl (EURO) dataset is a widely used multilingual parallel corpus, comprising the proceedings of the European Parliament. We selected a subset of its English and French text portions. The X and Y matrices contain 7,247 and 8,768 rows, respectively, and both matrices share 475,834 columns. Timestamps are generated using the $Poisson$ $Arrival$ $Process$ with a rate parameter of $\lambda=2$. The window size is set to 100,000, with approximately 30,000 columns of data on average in each window.
\end{itemize}

\paragraph{Setup} For the sequence-based model, we compare the proposed hDS-COD and  aDS-COD with EH-COD~\cite{yao2024approximate} and DI-COD~\cite{yao2024approximate}. We do not consider the Sampling algorithm as a baseline, as its performance is inferior to that of EH-COD and DI-CID, as demonstrated in \cite{yao2024approximate}. %The hDS-COD is adjusted by the parameter $\ell$ and the maximum number of levels $L = \log{R}$, where $R$ is the prior estimate of the maximum squared column norm of the dataset. DI-COD similarly requires a prior estimate of $R$ to limit the maximum number of levels $L = \log{(R/\varepsilon})$. In contrast, aDS-COD and EH-COD do not require an estimate of $R$; their error-space balance is controlled by the parameter $\ell = \frac{1}{\varepsilon}$. 
For the time-based model, we compare the proposed hDS-COD and  aDS-COD with EH-COD and the Sampling algorithm since DI-COD cannot be applied to time-based sliding window model. To achieve the same error bound, the maximum number of levels for hDS-COD is set to $L = \log{(\varepsilon NR)}$, and the initial threshold for aDS-COD is set to $1$.

Our experiments aim to illustrate the trade-offs between space and approximation errors. The x-axis represents two metrics for space: final sketch size and total space cost. The final sketch size refers to the number of columns in the result sketches $\mA$ and $\mB$ generated by the algorithm, representing a compression ratio. The total space cost refers to the maximum space required during the algorithm's execution, measured by the number of columns.We evaluate the approximation performance of all algorithms based on correlation errors $\operatorname{corr-err}(\mathbf{X}_W \mathbf{Y}_W^\top, \mathbf{A} \mathbf{B}^\top)$, which is reflected on the y-axis. Every 1,000 iterations, all algorithms query the window and record the average and maximum errors across all sampled windows.

The experiments for all algorithms were conducted using MATLAB (R2023a), with all algorithms running on a Windows server equipped with 32GB of memory and a single processor of Intel i9-13900K.

\paragraph{Performance} Figure \ref{fig:error vs l} and Figure \ref{fig:error vs space} illustrate the space efficiency comparison of the algorithms on sequence-based datasets. Panels (a-d) show the average errors across all sampled windows, while panels (e-h) display the maximum errors.

Figure \ref{fig:error vs l} evaluates the compression effect of the final sketch. The hDS-COD, aDS-COD, and EH-COD show similar compression performances. But the DS series is more stable, particularly on the synthetic datasets, where they significantly outperform EH-COD and DI-COD. The performance of hDS-COD and aDS-COD is nearly the same, indicating that the adaptive threshold trick in aDS-COD does not have a noticeable negative impact on it, maintaining the same error as hDS-COD.

Figure \ref{fig:error vs space} measures the total space cost of the algorithms. hDS-COD and aDS-COD show a significant advantage over existing methods, as they can achieve the  $\varepsilon$-approximation error with much less space. For the same space cost, the correlation errors of hDS-COD and aDS-COD are much smaller than those of EH-COD and DI-COD. Also, aDS-COD has better space efficiency than hDS-COD because aDS only uses a single-level structure while hDS requires $\log R+1$ levels. We find that hDS-COD requires more space on  SYNTHETIC(2) dataset compared to SYNTHETIC(1) dataset. This phenomenon occurs because SYNTHETIC(2) dataset has a larger $R$, which confirms the dependence on $R$ as stated in Theorem~\ref{thm:hds}. 

Figure \ref{fig:time-based} compares the performance of algorithms on time-based windows. Panels (a) and (b) present the error against the final sketch size, which show that our aDS-COD and hDS-COD algorithms enjoy similar performance as EH-COD and significantly outperform the sampling algorithm. On the other hand, as shown in panels (c) and (d), our methods outperform baselines in terms of total space cost.


\section{Conclusion}

Software development is increasingly conceived as a collaboration activity between developers and AIs. Indeed, IDEs already implement features to enable interactive development, with AI suggesting implementations that are reused by developers.

Although multiple studies show this interaction can be successful, there is still limited understanding of how the models must be configured and used in the context of code generation tasks. This study addresses this gap, systematically investigating the impact of several key parameters, including the repeated submission of a prompt to accommodate for the non-deterministic nature of the models.

Our study reveals several key findings about the usage of ChatGPT. In particular, we discovered how creativity, although up to a limited extent, is useful to increase the range of methods whose code can be generated correctly. A major role is played by parameter top-p, which is commonly underrated, and instead has a major impact on the correctness of the results, with lower values producing better results. Finally, prompts should be submitted multiple times, with $5$ repetitions combined with a temperature of $1.2$ resulting in an effective configuration in our experiments.  

Future work concerns two main research directions. One is about replicating this experiment with other AI assistants, to validate our findings in multiple contexts. The second research direction concerns finding strategies to deal with the need to submit the same prompt multiple times to obtain a useful result, and thus developing approaches able to select or merge multiple responses automatically. 

% \section*{Acknowledgment}

%%%%%%%%%%%%%%%%%%%%%%%%%%%%%%%%%%%%%%%%%%%%%%%%%%%%%%%%%%%%%%%%%%%%%%%%%%%%%%%%%%%%%%%%%%%
\newpage
\bibliographystyle{IEEEtran}
\bibliography{references}

\end{document}



%------------------------------------
% Bibliography
%------------------------------------


\nocite{*}

\bibliographystyle{compj}
\bibliography{references}

\end{document}
