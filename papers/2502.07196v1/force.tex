\documentclass[lettersize,journal]{IEEEtran}
\usepackage{amsmath,amsfonts}
\usepackage{algorithmic}
\usepackage{algorithm}
\usepackage{array}
\usepackage[caption=false,font=normalsize,labelfont=sf,textfont=sf]{subfig}
\usepackage{textcomp}
\usepackage{stfloats}
\usepackage{url}
\usepackage{verbatim}
\usepackage{graphicx}
\usepackage{cite}
\usepackage{breqn}
\usepackage{makecell}
\hyphenation{op-tical net-works semi-conduc-tor IEEE-Xplore}
% updated with editorial comments 8/9/2021

\begin{document}

\title{Parameter Optimization of Optical Six-Axis Force/Torque Sensor for Legged Robots}

\author{Hyun-Bin Kim, Byeong-Il Ham, Keun-Ha Choi, and Kyung-Soo Kim,~\IEEEmembership{Member,~IEEE,}
        % <-this % stops a space
% \thanks{This paper was produced by the IEEE Publication Technology Group. They are in Piscataway, NJ.}% <-this % stops a space
% \thanks{Manuscript received April 19, 2021; revised August 16, 2021.}}
\thanks{Manuscript created September, 2024; This work was developed by the MSC (Mechatronics, Systems and Control) lab in the KAIST(Korea Advanced Institute of Science and Technology which is in the Daehak-Ro 291, Daejeon, South Korea(e-mail: youfree22@kaist.ac.kr; byeongil\_ham@kaist.ac.kr; choiha99@kaist.ac.kr; kyungsookim@kaist.ac.kr).(Corresponding author: Kyung-Soo Kim).}}
% The paper headers
\markboth{ arXiv,
%IEEE TRANSACTIONS ON ROBOTICS,~Vol.~00, No.~00, 
JANUARY~2025}%
{Shell \MakeLowercase{\textit{et al.}}: A Sample Article Using IEEEtran.cls for IEEE Journals}

% \IEEEpubid{0000--0000/00\$00.00~\copyright~2025 IEEE}
% Remember, if you use this you must call \IEEEpubidadjcol in the second
% column for its text to clear the IEEEpubid mark.

\maketitle

\begin{abstract}
This paper introduces a novel six-axis force/torque sensor tailored for compact and lightweight legged robots. Unlike traditional strain gauge-based sensors, the proposed non-contact design employs photocouplers, enhancing resistance to physical impacts and reducing damage risk. This approach simplifies manufacturing, lowers costs, and meets the demands of legged robots by combining small size, light weight, and a wide force measurement range.
A methodology for optimizing sensor parameters is also presented, focusing on maximizing sensitivity and minimizing error. Precise modeling and analysis of objective functions enabled the derivation of optimal design parameters. The sensor's performance was validated through extensive testing and integration into quadruped robots, demonstrating alignment with theoretical modeling.
The sensor’s precise measurement capabilities make it suitable for diverse robotic environments, particularly in analyzing interactions between robot feet and the ground. This innovation addresses existing sensor limitations while contributing to advancements in robotics and sensor technology, paving the way for future applications in robotic systems.
\end{abstract}

\begin{IEEEkeywords}
Force measurement, Torque measurement, Sensor systems, Optimization methods
\end{IEEEkeywords}
\sloppy
\section{Introduction}
 \IEEEPARstart{R}{ecent} advancements in robotics have led to a growing interest in legged robots, exemplified by platforms such as \textit{ANYmal}, \textit{Mini Cheetah}, \textit{Hound}, and \textit{Raibo}~\cite{di2018dynamic, katz2019mini,hutter2016anymal,shin2022design,choi2023learning}. Force/torque sensors play a crucial role in these systems, enabling precise ground reaction force (GRF) measurement, which is essential for stability and control. Unlike manipulators, legged robots experience high impact forces often exceeding three to five times the nominal GRF due to foot-ground impacts. This repeated exposure to large forces often leads to sensor failure, making durability a key challenge.

To mitigate this issue, conventional sensors increase their sensing range to withstand high impact forces. However, this comes at the cost of reduced sensitivity, creating a fundamental trade-off between durability and measurement accuracy. Existing solutions, such as strain-gauge-based sensors, also suffer from high cost, large size, and the need for external signal processing, making them impractical for compact and lightweight robotic applications.

Force/torque sensors play a crucial role in robotics. In robotic arms, they are widely used in collaborative robots, while in humanoid robots~\cite{kim2019multi}, they measure ground reaction forces at the ankle for control purposes. In robotic hands, such as prosthetic devices, they facilitate force control through tension, and they are also extensively utilized in medical robotics~\cite{kim2017surgical,cho2022msc,jeong2018design}. Previous research on force/torque sensors has explored various sensing methods, including strain gauges~\cite{chen2025design}, capacitive methods~\cite{kim2016novel}, piezoresistive sensors~\cite{epstein2020bi}, magnetic sensing~\cite{ananthanarayanan2012compact}, and Fiber Bragg gratings~\cite{xiong2020six}. Among these, strain gauge-based sensors, capacitive methods, and optical methods have successfully achieved commercialization.


This paper introduces a novel non-contact six-axis force/torque sensor based on optical photo-couplers~\cite{kim2024compact,kim2023compact}, designed to address these challenges(high impact, sensitivity, durability, cost, large size and external signal processing). The proposed sensor significantly reduces impact-induced damage risks while maintaining high sensitivity. To overcome the trade-off between sensing range and sensitivity, an optimization-based approach was employed. Various objective functions were evaluated using finite element method (FEM) simulations to derive an optimal sensor design that balances robustness and accuracy.

The fabricated sensor was integrated into a quadruped robot and compared against a commercial reference sensor in real-world experiments. The results demonstrate that, while the commercial sensor exhibited significant offset drift due to repeated impacts, the proposed sensor maintained stable performance with minimal offset. Additionally, with only four components, the proposed sensor offers a cost-effective solution, with a prototype cost below \$300, significantly lower than the \$2,000-\$10,000 range of existing commercial sensors~\cite{robotous_website,ati_website}. These findings highlight the sensor’s potential for widespread application in robotic systems, particularly where impact resistance and cost efficiency are critical.
% \IEEEPARstart{R}{ecent} advancements in robotics have led to an increasing focus on the development and application of legged robots, with notable examples including **ANYmal** by ETH Zurich, **Mini Cheetah** by MIT, **Hound** by KAIST, and **Raibo**.~\cite{di2018dynamic, katz2019mini,hutter2016anymal,shin2022design,choi2023learning} These robots are being explored for their potential industrial applications. In this context, **force/torque sensors** play a critical role in manipulators, prosthetic devices, and legged robots.~\cite{kim2017surgical,cho2022msc,jeong2018design} These sensors enable precise ground reaction force (GRF) measurements~\cite{valsecchi2020quadrupedal}, which are essential for enhancing stability and control performance. Specifically, force/torque sensors are widely used in manipulators for force control and in bipedal and quadruped robots to measure interactions with the ground during locomotion.

% Despite their importance, conventional force/torque sensors face significant limitations when applied to legged robots. These include susceptibility to damage from impacts, large size, the need for external signal processing units, and high costs.~\cite{ati_website} As a result, many legged robots rely on simple contact-detection sensors rather than fully functional force/torque sensors.~\cite{unitree_website} For instance, ground reaction forces during locomotion can generate impacts up to three times the GRF magnitude, often resulting in damage to existing sensors. Additionally, the large size and weight of conventional sensors are unsuitable for the limited foot space of legged robots (typically 40–85mm in diameter) and fail to meet the lightweight design requirements.~\cite{kim2019multi} Furthermore, the reliance on external signal processing units reduces payload capacity and overall performance.

% To address these challenges, this study proposes a novel **non-contact six-axis force/torque sensor** based on photocouplers.~\cite{kim2024compact,kim2023compact} The non-contact approach significantly reduces the risk of damage from sudden impacts and allows for the integration of the entire circuit onto a single PCB, enabling miniaturization and lightweight design. This makes the sensor suitable for dynamic environments like those encountered by legged robot feet, while also providing high sensitivity and a wide force measurement range. Moreover, the proposed design overcomes the limitations of traditional strain gauge-based sensors by offering improved impact resistance and reduced crosstalk.

% However, initial designs revealed challenges in achieving optimal performance within constrained size and force ranges due to heuristic parameterization of elastomer design. To address this, the study systematically analyzes sensor design parameters and introduces an optimization-based methodology to maximize performance. By utilizing various objective functions, the proposed approach allows for flexible design customization based on specific requirements, such as sensitivity or crosstalk minimization. The optimized designs were further validated through implementation and performance evaluation.

\subsection{Contribution}
The primary contributions of this paper are as follows:  

1. Optimization-based sensor design methodology  
   A systematic approach for modeling and optimizing six-axis force/torque sensors within constrained dimensions (40mm diameter) and force ranges is presented.  

2. Design flexibility through multiple objective functions  
   The methodology allows for design customization to achieve desired characteristics such as sensitivity, impact resistance, and reduced crosstalk.  
   
3. Incorporation of sensor placement optimization  
   The study includes the optimization of sensor placement to maximize overall performance.  

4. Implementation and experimental validation  
   The proposed sensor was implemented and integrated into a quadruped robot, with experiments conducted to verify its performance and consistency with theoretical models.  

This research introduces a robust and lightweight force/torque sensor suitable for dynamic environments such as legged robots, overcoming the limitations of existing sensors and offering broad applicability across various robotic systems.


\section{Optimal Design of Proposed Sensor}
\subsection{Principle of Proposed Sensor}

Fig.~\ref{principle} (a) illustrates the principle of the proposed sensor. $\Delta d_1$, $\Delta d_2$, and $\Delta d_3$ represent the degree of deformation due to force at the positions of the vertical photo-couplers, while $\Delta d_4$, $\Delta d_5$, and $\Delta d_6$ represent the degree of deformation due to force at the positions of the horizontal photo-couplers. The vertical photo-couplers are primarily used for measuring $F_z$, $M_x$, and $M_y$, whereas the horizontal photo-couplers are mainly utilized for measuring $F_x$, $F_y$, and $M_z$. 
% $\Delta d_1$, $\Delta d_2$, and $\Delta d_3$ 는 vertical photo-couplers 위치에서의 힘에 대한 변형 정도를 나타내며, $\Delta d_4$, $\Delta d_5$, and $\Delta d_6$는 horizontal photo-couplers 위치에서의 힘에 대한 변형 정도를 나타낸다. 



The proposed sensor offers several advantages over traditional strain gauge-based methods. Unlike strain gauges, it operates using a non-contact method, and unlike capacitive sensors, it does not require an additional measurement PCB(Printed Circuit Board); a single PCB suffices. Furthermore, the sensor requires only six ADC(Analog to Digital Converter)s, reducing the number from the conventional 12 ADCs to half. Each photo-coupler measures the displacement of the spring-like component at the center, which functions as both an elastomer and a reflective surface, allowing the force to be measured. 

The elastomer in the proposed sensor employs a T-beam structure. When a force is applied to the central loading table, the elastomer deforms, enabling measurement of horizontal and vertical forces through distinct mechanisms. Horizontal forces are determined by assessing the deformation of the T-beam, while vertical forces are measured by evaluating the angle and deformation of the reflective structure associated with the loading table.

The specific photo-couplers used are the TCRT1000 and VCNT2020 from Vishay, offering measurement ranges of approximately 1mm and 0.5mm, respectively. The circuit design is highly simplified, requiring only one resistor for the diode and one resistor for the transistor to enable measurement.

Fig.~\ref{principle} (b) illustrates the structure of the sensor's elastomer, showing the dimensions of the T-beam and the positions of the sensors. The central region, referred to as the loading table, directly receives the applied force, and its radius is represented by $r$. The T-beam of the elastomer is defined by the dimensions $l_1$, $l_2$, $b_1$, $b_2$, and $h$. The positions of the vertical photo-couplers are denoted by $r_{s1}$, while those of the horizontal photo-couplers are represented by $r_{s2}$. 

In this study, these parameters were set as optimization variables. However, the position of the vertical photo-coupler was fixed at 12mm. This decision was made to address the limited space on the PCB and reduce the complexity of the optimization process, thereby improving convergence. The placement of the vertical photo-coupler at the outermost edge was chosen to enhance sensitivity, as larger displacements can be observed at the outer edges when moments around the x- and y-axes are generated by the loading table.

% \begin{figure}[!t]\centering
% 	\includegraphics[width=\columnwidth]{sensorparameters.png}
% 	\caption{Parameters of Proposed Sensor.: $r$: radius of loading table, $l_1, l_2, b_1, b_2, h$: Dimension of T-beam, and $r_{s1}$ and $r_{s2}$: Radius of Vertical Photocouplers and Horizontal Photocouplers }\label{para}
% \end{figure}
\begin{figure}[!t]\centering
	\includegraphics[width=\columnwidth]{MainPage4.png}
	\caption{a) Configuration and Principle of Proposed Sensor. 3 Vertical Photo-Couplers for $F_z, M_x, M_y$ and 3 Horizontal Photo-Couplers for $F_x, F_y, M_z$ (b) Parameters of Proposed Sensor.: $r$: radius of loading table, $l_1, l_2, b_1, b_2, h$: Dimension of T-beam, and $r_{s1}$ and $r_{s2}$: Radius of Vertical Photocouplers and Horizontal Photocouplers }\label{principle}
\end{figure}

\subsection{Optimal Design Using Global Search Algorithm}
\begin{figure*}[!t]\centering
	\includegraphics[width=2\columnwidth]{Parameters1.png}
	\caption{Elastomer deformation according to the force and moment applied along each axis:  
(a) Deformation when applying z-axis force ($F_z$)  
(b) Deformation when applying y-axis moment ($M_y$)  
(c) Deformation when applying x-axis force ($F_x$)  
(d) Deformation when applying z-axis moment ($M_z$)
    }\label{parameter}
\end{figure*}

To achieve optimal design, accurate modeling is essential. In this study, the Timoshenko beam theory was employed to develop a more precise model, as referenced in ~\cite. A model for the proposed sensor was established, and a matrix representing the relationship between force and deformation was derived. This matrix was then used to perform numerical analysis.
\begin{align}
\Delta d= \left[ \Delta d_1 \,\Delta d_2 \,\Delta d_3 \,\Delta d_4\, \Delta d_5\,  \Delta d_6 \right]^T\\
F= \left[ F_x\, F_y\, F_z\, M_x \,M_y\, M_z \right]^T\\
\Delta d = \textbf{G} F\\
\textbf{G} = \Delta d / F 
\end{align}

Eq. (1) follows the same order as shown in Fig.~\ref{principle}, where the first three terms represent the vertical photo-couplers, and the latter three terms correspond to the horizontal photo-couplers. 
Eq. (2) represents the force and moment vector, including the forces and moments along the x, y, and z axes. Eq.~(3) describes the relationship between deformation and force, where $\textbf{G}$ is defined as $\Delta d / F$, which can be interpreted as the sensitivity as Eq.~(4).

\begin{align}
\resizebox{0.9\columnwidth}{!}{$
    \textbf{G}=
    \begin{bmatrix}
    0 & 0 & -\frac{1}{k_{dFzv}} & \frac{-r_{s1}\sin(\pi/3)}{k_{rMxv}} & \frac{r_{s1}\sin(\pi/6)}{k_{rMxv}} & 0 \\ 
    0 & 0 & -\frac{1}{k_{dFzv}} & \frac{r_{s1}\sin(\pi/3)}{k_{rMxv}} & \frac{r_{s1}\sin(\pi/6)}{k_{rMxv}} & 0 \\ 
    0 & 0 & -\frac{1}{k_{dFzv}} & 0 & \frac{-r_{s1}}{k_{rMxv}} & 0 \\ 
    0 & \frac{1}{k_{dFyh}} & 0 & \frac{(h/2-c)}{k_{rMxh}} & 0 & \frac{1}{k_{dMzh}} \\ 
    \frac{-\sin(\pi/3)}{k_{dFyh}} & \frac{-\sin(\pi/6)}{k_{dFyh}} & 0 & \frac{-(h/2-c)\sin(\pi/6)}{k_{rMxh}} & \frac{-(h/2-c)\sin(\pi/3)}{k_{rMxh}} & \frac{1}{k_{dMzh}} \\ 
    \frac{\sin(\pi/3)}{k_{dFyh}} & \frac{-\sin(\pi/6)}{k_{dFyh}} & 0 & \frac{(h/2-c)\sin(\pi/6)}{k_{rMxh}} & \frac{(h/2-c)\sin(\pi/3)}{k_{rMxh}} & \frac{1}{k_{dMzh}}
    \end{bmatrix}
$}\label{3.5}
\end{align}
Here, the matrix $\textbf{G}$ of Eq.~(5) represents the relationship between forces, moments, and the deformations of each sensor. 

Each $r_{s1}$ represents the position (radius) of the vertical photo-coupler, while $h$ denotes the height of the T-beam, and $c$ indicates the difference between the midpoint height of the T-beam and the height of the horizontal photo-coupler. The parameter $k$ represents the spring constants at specific locations. 

% Figure~\ref{horideform}는 horizontal sensor가 $r_{s2}$의 위치에 있을 때 $M_z$를 가했을 때를 예시로 기존의 $C$위치에서의 변위는 $\delta_C$ 지만, horizontal sensor가 $r_{s2}$에 위치하기 때문에 그때 변위는 $\Delta d_4$가 된다. 이처럼, 센서의 위치에 따라서도 센서가 감지하는 변형정도는 다르기 때문에 이를 수식으로 나타내는 것이 필요하다.Equation~\ref{3.5}는 그런 센서의 위치까지 포함된 식으로 저 식을 이용하여 최적화 설계를 할 수 있으며, 각 notation은 Table~\ref{table1}과 같이 표현할 수 있다.
Figure~\ref{horideform} illustrates the case where a moment $M_z$ is applied when the horizontal sensor is located at $r_{s2}$. In this example, the displacement at the original position $C$ is $\delta_C$, but due to the horizontal sensor being located at $r_{s2}$, the displacement becomes $\Delta d_5$. As such, the degree of deformation detected by the sensor varies depending on its position, necessitating a mathematical representation. Equation~\ref{3.5} incorporates the sensor's position, enabling optimization design using this equation. The notations are summarized in Table~\ref{table1}.


\begin{table}[h!]\caption{Notation of Spring Constant}
\centering \label{table1}
\begin{tabular}{cc}
\hline
$k$     & Spring Constant     \\ \hline\hline
$d$     & Displacement        \\
$r$     & Angular Deformation \\
$F$     & Force               \\
$M$     & Moment              \\
$x,y,z$ & Axis                \\
$v$     & Vertical Sensor     \\
$h$     & Horizontal Sensor   \\ \hline
\end{tabular}
\end{table}

As shown in Table~\ref{table1}, each notation corresponds to a specific parameter. For example, $k_{dFyh}$ refers to the spring constant related to the deformation caused by the y-axis force at the location of the horizontal photo-coupler, while $k_{rMxv}$ represents the spring constant associated with the angular deformation caused by the x-axis moment at the location of the vertical photo-coupler.



 \begin{figure}[t!]
    \centerline{\includegraphics[width=\columnwidth]{example1.png}}
    \caption[Deformation of Elastomer at Horizontal Sensor]{When deformation occurs along the negative y-axis and the horizontal sensor is positioned at $r_{s2}$, the deformation measured by the horizontal sensor is \( \Delta d_5 \).

    } \label{horideform}
\end{figure}




\subsection{Modeling}

Fig.~\ref{horideform} shows the deformation of the elastomer at the location of the horizontal sensor, which occurs when a force is applied in the -y direction. The position of the sensor can be determined using $r_{s2}$, the compensated length of the T-beam $l_1' = l_1 + b_2 / 2$, and $r$.

The force and moment applied at point C are represented as $F_C$ and $M_C$, respectively, and the resulting deformation at point C is denoted as $\delta_C$. Since the horizontal photo-coupler is not located directly at point C, the corresponding deformation at the sensor's location must be separately calculated.

Figure~\ref{parameter}.(a) illustrates how the z-axis deflection $\Delta d_{F_{z}}$ occurs due to the force along the z-axis.
\begin{equation} \Delta d_{F_{z}} =\delta _{z1} +\delta _{z2} +\theta l_{1}^{\prime }
\label{3.4}\end{equation}
As described in Equation~\ref{3.4}, $\Delta d_{F_{z}}$ is composed of the deflection due to the force $\delta_{z1}$ and the deflection due to the moment $\delta_{z2} + \theta l_{1}^{\prime}$. Here, $I_t = \beta h b_{2}^3$ represents the torsional moment of inertia, with $l_1^{\prime} = l_1 + b_2 /2$ and $l_2^{\prime} = l_2 - b_1$. The polar moment of inertia of the beam is denoted by $I$, $G$ represents the shear modulus, and $\beta$ is the torsion coefficient calculated as $(16/3 - 3.36b_{2}(1-b_{2}^4/12h^4)/h)/16$.

\begin{equation} 
\begin{cases}\theta =(F_{C} l_{1}^{\prime }-M_{C})l_{2}^{\prime }/(4GI_{t})
\\\delta _{z1} =\frac {F_{C} l_{2}^{\prime 3}}{192EI_{22}}+\frac {F_{C} l_{2}^{\prime }}{4kGS_{2}}
\\ \delta _{z2} =\frac {F_{C} {l}_{1}^{\prime 3}}{3EI_{12}}-\frac {M_{C} {l}_{2}^{\prime 2}}{2EI_{12}}+\frac {F_{C} l_{1}^{\prime }}{kGS_{1}}
\label{3.5}
\end{cases}\end{equation}
The variables $\theta$, $\delta_{z1}$, and $\delta_{z2}$ can be computed using Equation~\ref{3.5}, where $E$ represents the elastic modulus. The term $S_{2}=b_{2}h$ and $k$ is the shear coefficient, calculated as $10/(12 + b_{2}/h)$. The $I_{22} = b_{2}h^{3}/12$ is used to denote the z-axis moment of inertia. $F_C$ and $M_C$ represent the force and moment at point C, respectively.

\begin{equation} \theta _{C} =\frac {F_{C} l_{1}^{\prime 2}}{2EI_{12}}-\frac {M_{C} l_{1}^{\prime }}{EI_{12}}+\frac {(F_{C} l_{1}^{\prime }-M_{C})l_{2}^{\prime }}{4GI_{t}}=0
\label{3.6}
\end{equation}
%CD의 rotational angle이 D에서 0이기 때문에 기하학적인 equation ~\ref{3.6}으로 나타낼 수 있다. 
Since the rotational angle at the beam FC is zero at C, this can be expressed geometrically as Equation ~\ref{3.6}.


\begin{equation} Fz=F_{A} +F_{B} +F_{C} =3F_{C}\label{3.7}\end{equation}
%그리고 equation ~\ref{3.7}과 같이 z축 힘은 D,E,F에서의 힘의 합과 같고 $F_D$의 세배로 나타낼 수 있다. 
Furthermore, as outlined in Equation~\ref{3.7}, the force along the z-axis equals the sum of the forces at points A, B, and C, which can be expressed as three times $F_C$.
% \begin{equation}&\hspace {-0.5pc}\delta _{F_{z}} (x)= F_{C} \bigg(\frac {x^{2}(3l_{1}^{\prime }-x)}{6EI_{12}}-\frac {a_{1} x^{2}}{2EI_{12}}+\frac {(l_{1}^{\prime }-a_{1})l_{2}^{\prime }}{4GI_{t}}x +\,\frac {l_{2}^{\prime 3}}{192EI_{22}}+\frac {4l_{1}^{\prime }S_{2} +l_{2}^{\prime }S_{1}}{4kGS_{1} S_{2}}\bigg)\label{3.8}\end{equation}
\begin{equation}
 % \resizebox{1\columnwidth}{!}{
 \begin{aligned}
     \delta_{F_{z}}(x) = F_{C} (
    \frac{x^{2}(3l_{1}^{\prime} - x)}{6EI_{12}} 
    - \frac{a_{1} x^{2}}{2EI_{12}} 
    + \frac{(l_{1}^{\prime} - a_{1})l_{2}^{\prime}}{4GI_{t}}x \\
    + \frac{l_{2}^{\prime 3}}{192EI_{22}} 
    + \frac{4l_{1}^{\prime}S_{2} + l_{2}^{\prime}S_{1}}{4kGS_{1}S_{2}})   
 \end{aligned}
\end{equation}

Overall, this can be represented as Equation 10, where at $x=l_{1}^{\prime}$, $\Delta d_{Fz}$ can be calculated. 
Here, $k_{Fz}$ is expressed as follows.
% Here, $k_{Fz}$ is expressed as follows:

\parbox{\linewidth}{%
   \scalebox{0.8}{
$k_{F_z} = \frac{1}{3} \Bigg( 
\frac{l_1^{\prime3}}{3 E I_{12}} 
- \frac{a_1 l_1^{\prime 2}}{2 E I_{12}} 
+ \frac{\left(l_1^{\prime}-a_1\right) l_1^{\prime} l_2^{\prime}}{4 G I_t} 
+ \frac{l_2^{\prime 3}}{192 E I_{22}} 
+ \frac{4 l_1^{\prime} S_2 + l_2^{\prime} S_1}{4 k G S_2}
\Bigg).$}}
% $k_{F z}=\frac{1}{3}\left(\frac{l_1^{\prime3}}{3 E I_{12}}-\frac{a_1 l_1^{\prime 2}}{2 E I_{12}}+\frac{\left(l_1^{\prime}-a_1\right) l_1^{\prime} l_2^{\prime}}{4 G I_t}+\frac{l_2^{\prime 3}}{192 E I_{22}}+\frac{4 l_1^{\prime} S_2+l_2^{\prime} S_1}{4 k G S_2}\right)$

\begin{equation} \Delta d_{F_z} =k_{F_z} F_z \label{3.9}\end{equation}

In Equation~\ref{3.9}, the spring constant relating force to deflection is presented, where $S_1=b_1 h$, $I_{12}=b_{1}h^3/12$, $F_C=F_z / 3$, and $M_C=a_1 F_C$ with $a_1=\frac{l_1^{\prime}\left(2 l_1^{\prime} G I_t+l_2^{\prime} E I_{12}\right)}{4 l_1^{\prime} G I_t+l_2^{\prime} E I_{12}}$. This is valid when $x=l_1^{\prime}$.

\begin{equation}
\begin{cases}
    \Delta r_{M_y} =\Delta r_{A_{M_y}} =2\Delta r_{C_{M_y}}\\
     \Delta r_{C_{M_y}} =\Delta d_{C_{{M_y}}}/r\\
      \Delta d_{C_{_{M_y}}} =\delta _{z1} +\delta _{z2} +\theta l_{1}^{\prime }\\
      \theta _{C} =\frac {F_{C} l_{1}^{\prime 2}}{2EI_{12}}-\frac {M_{C} l_{1}^{\prime }}{EI_{12}}+\frac {(F_{C} l_{1}^{\prime }-M_{C})l_{2}^{\prime }}{4GI_{t}}=-\Delta r_{C_{M_y}}\\
      F_{C} A_{1} =M_{C} B_{1}\\
      F_{A} A_{1} =M_{A} B_{1}
      
\end{cases}\label{3.10}\end{equation}
Figure~\ref{parameter}.(b) shows the deformation of elastomer when y-axis moment applied.
Equation~\ref{3.10} derives the deflection caused by the torque along the x-axis, where $\Delta d_{C_{_{M_y}}}$ represents the displacement at point C, which can be calculated similarly to the force along the z-axis. Additionally, the angle at C can be geometrically derived as indicated in the fourth equation of Equation~\ref{3.10}.

%여기서
Here, when $ A_1=\left(\frac{l_1^{\prime 2}}{3 E I_{12}}+\frac{l_1^{\prime} l_2^{\prime}}{4 G I_t}+\frac{1}{k G S_1}\right) \lambda_1+\left(\frac{1}{4 k G S_2}+\frac{l_2^{\prime 2}}{192 E I_{22}}\right) \lambda_2+\left(\frac{l_1^{\prime 2}}{2 E I_{12}}+\frac{l_1^{\prime} l_2^{\prime}}{4 G I_t}\right) ,
 B_1=\left(\frac{l_1^{\prime}}{2 E I_{12}}+\frac{l_2^{\prime}}{4 G I_t}\right) \lambda_1+\left(\frac{l_1^{\prime}}{E I_{12}}+\frac{l_2^{\prime}}{4 G I_t}\right) , \lambda_1=l_1^{\prime} / r, \lambda_2=l_2^{\prime} / r $
equations 5 and 6 of Equation~\ref{3.10} can be derived, and using static equilibrium, this can be expressed as Equation~\ref{3.11}.

\begin{equation}
    \begin{cases} \displaystyle F_{A} =2F_{C} \\ \displaystyle My=M_{A} +F_{A} r+2(M_{C} +F_{C} r)\sin 30^{\circ } \displaystyle \end{cases}
    \label{3.11}
\end{equation} 

Consequently, the displacement at CD can be represented as follows in Equation~\ref{3.12}:
The spring constant for the torque $M_y$ can be formulated as:
\begin{multline}
    \delta _{M_y} (x)=\frac {M_y}{3(A_{1} /B_{1} +r)}\bigg(\frac {l_{2}^{\prime 3}}{192EI_{22}}+\frac {l_{2}^{\prime }}{4kGS_{2}}+\frac {x}{kGS_{1}} \\+\,\frac {x^{2}(3{l}^{\prime }_{1} -x)}{6EI_{12}}-\frac {A_1x^{2}}{2EI_{12}B_1}+\frac {(l_{1}^{\prime }-A_{1} /B_{1}){l}^{\prime }_{2}}{4GI_{t}}x\bigg) 
    \label{3.12} 
\end{multline} 
\parbox{\linewidth}{%
   \scalebox{0.70}{
$k_{M_y} = \frac{2}{3\left(A_1 / B_1 + r\right) r} \Bigg(
\frac{l_2^{l^3}}{192 E I_{22}}
+ \frac{l_2^{\prime}}{4 k G S_2}
+ \frac{2 l_1^{\prime 3} - 3 A_1 / B_1 l_1^{\prime 2}}{6 E I_{12}}
+ \frac{l_1^{\prime}}{k G S_1}
+ \frac{\left(l_1^{\prime} - A_1 / B_1\right) l_1^{\prime} l_2^{\prime}}{4 G I_t}
\Bigg)$}}

% \parbox{\linewidth}{%
% $k_{M y}=\frac{2}{3\left(A_1 / B_1+r\right) r}\left(\frac{l_2^{l^3}}{192 E I_{22}}
% +\frac{l_2^{\prime}}{4 k G S_2}+\frac{2 l_1^{\prime 3}-3 A_1 / B_1 l_1^{\prime 2}}{6 E I_{12}}+\frac{l_1^{\prime}}{k G S_1}+\frac{\left(l_1^{\prime}-A_1 / B_1\right) l_1^{\prime} l_2^{\prime}}{4 G I_t}\right)$}

Thus, $\Delta r_{M_y} = k_{M_y} M_y$ illustrates how the displacement is directly influenced by the moment along the y-axis.

Similarly, equations for forces $F_x$ and $M_z$ can be derived as shown as Figure~\ref{parameter}.(c) and (d), enabling a comprehensive modeling of the responses to $F_z$, $M_y$, $F_x$, and $M_z$ at a distance $r$ from the center.


\begin{equation} 
\begin{cases}
    \theta =(F_{C_t} l_{1}^{\prime }-M_{C})l_{2}^{\prime }/(16EI_{21})\\
    \delta _{C_t} =\frac {F_{C_t} {l}_{1}^{\prime 3}}{3EI_{11}}-\frac {M_{C} {l}_{1}^{\prime 2}}{2EI_{11}}+\frac {F_{C_t} l_{1}^{\prime }}{kGS_{1}}+\theta l_{1}^{\prime }\\
    \delta _{C_n} =\frac {F_{C_n} {l}_{1}^{\prime 3}}{192EI_{11}}+\frac {F_{C_n} {l}^{\prime }_{1}}{ES_{1}}+\frac {F_{C_n} {l}^{\prime }_{2}}{4kGS_{2}}\\

    \end{cases}
    \label{fx1}
    \end{equation}

    
In the equation for $F_x$, the tangential deformation $\delta_{C_t}$ and normal deformation $\delta_{C_n}$ at $\theta$ and position $C$ can be determined using equation~\ref{fx1}. Here, $I_{11}=hb_1^3/12, I_{21}=hb_2^3/12$ was used.


% $F_x$의 수식에서는 $\theta$와 $C$ 위치에서의 tangential 방향의 변형인 $\delta_{C_t}$와 normal 방향의 변형인 $\delta_{C_n}$은 equation~\ref{fx1}을 이용하여 구할 수 있다. 

% x축 방향에 있는 $A$에서의 변형$\Delta d_A$와 $C$에서의 변형인 $\Delta d_C$는 같으며 normal방향의 변형과 tangential방향의 변형의 합으로 이루어져 있다. 

% 그리고 $\theta_C$는 $C$에서의 각도가 0이므로 constaint의 수식을 쓸 수 았으며, $\delta _{C_n}$과 $\delta_{C_t}$의 관계는 $\tan{30^\circ}$의 관계를 가지고 있으며, $F_x$는 $F_{C_n}$과 $F_{C_t}$와 $F_A$로 구성된다. 

% 위의 수식들을 계산하게 되면 $\delta _{F_x}$를 $x$에 따라서 즉 위치에 따라서 구할 수 있다.  

    
    \begin{equation}
    \begin{cases}    
     \Delta d_{A} =\frac {F_{A} {l}_{2}^{\prime 3}}{192EI_{21}}+\frac {F_{A} l_{2}^{\prime }}{ES_{2}}+\frac {F_{A} l_{2}^{\prime }}{4kGS_{2}}\\
     \Delta d_{A} =\Delta d_{C} =\sqrt {\delta _{C_n}^{2}+\delta _{C_t}^{2}}\\
      \theta _{C} =\frac {F_{C_t} {l}_{1}^{\prime 2}}{2EI_{11}}-\frac {M_{C} {l}^{\prime }_{1}}{EI_{11}}+\frac {(F_{C_t} {l}^{\prime }_{1} -M_{C}){l}^{\prime }_{2}}{16EI_{21}}=0\\
      \delta _{C_n}=\tan 30^{\circ } \delta _{C_t}\\

           \end{cases}
    \end{equation}
    The deformation at point $A$ along the x-axis, $\Delta d_A$, and the deformation at point $C$, $\Delta d_C$, are identical and consist of the sum of the normal deformation and tangential deformation.
    Furthermore, since the angle $\theta_C$ at point $C$ is 0, the constraint equation can be applied. The relationship between $\delta_{C_n}$ and $\delta_{C_t}$ is governed by $\tan{30^\circ}$.

    \begin{equation}
    \begin{cases}
           F_x=2(F_{C_n} \sin 30^{\circ }+F_{C_t} \cos 30^{\circ })+F_{A}\\
       F_{A} =F_x/A_{4}, M_{C} =F_x A_{2} B_{3} /(2A_{3} A_{4} B_{2}),\\ F_{C_n} =F_x/(2A_{4}),F_{C_t} =F_x B_{3} /(2A_{3} A_{4})\\
      
           \end{cases}
           \label{3.17}
    \end{equation}
    $F_x$ is composed of $F_{C_n}$, $F_{C_t}$, and $F_A$.
    Each $F_A$, $F_x$, $M_C$, $F_{C_n}$, and $F_{C_t}$ has the same relationship as in Equation~\ref{3.17}. Additionally, each of $A_2$, $B_2$, $A_3$, $B_3$, and $A_4$ is defined as:
\[
\begin{array}{l}
A_2 = \frac{l_1^{\prime 2}}{2 I_{11}} + \frac{l_1^{\prime} l_2^{\prime}}{16 I_{21}}, \\
B_2 = \frac{l_1^{\prime}}{I_{11}} + \frac{l_2^{\prime}}{16 I_{21}}, \\
A_3 = \frac{l_1^{\prime 3}}{3 E I_{11}} + \frac{{l_1^{\prime}}^2 l_2^{\prime}}{16 E I_{21}}
+ \frac{l_1^{\prime}}{k G S_1}
- \frac{A_2 \left( \frac{l_1^{\prime 2}}{2 E I_{11}} + \frac{l_1^{\prime} l_2^{\prime}}{16 E I_{21}} \right)}{B_2}, \\
B_3 = \sqrt{3} \left( \frac{{l_1^{\prime}}^3}{192 E I_{11}} + \frac{l_1^{\prime}}{E S_1}
+ \frac{l_2^{\prime}}{4 k G S_2} \right), \\
A_4 = \left( \frac{3}{2} + \frac{\sqrt{3} B_3}{2 A_3} \right).
\end{array}
\]
This structure defines the relationships and dependencies among the variables.



    %각 $F_A$와 $F_x$, $M_C$, $F_{C_n}$,$F_{C_t}$는 equation~\ref{3.17}과 같은 관계를 가지고 있으며 각 $A_2$,$B_2$,$A_3$,$B_3$,$A_4$는 
% \[
% \begin{array}{l}
% A_2 = \frac{l_1^{\prime 2}}{2 I_{11}} + \frac{l_1^{\prime} l_2^{\prime}}{16 I_{21}}, 
% B_2 = \frac{l_1^{\prime}}{I_{11}} + \frac{l_2^{\prime}}{16 I_{21}}, \\
% A_3 = \frac{l_1^{\prime 3}}{3 E I_{11}} + \frac{l_1^{\prime} l_2^{\prime}}{16 E I_{21}}
% + \frac{l_1^{\prime}}{k G S_1}
% - \frac{A_2 \left( \frac{l_1^{\prime 2}}{2 E I_{11}} + \frac{l_1^{\prime} l_2^{\prime}}{16 E I_{21}} \right)}{B_2}, \\
% B_3 = \sqrt{3} \left( \frac{l_2^{\prime \prime}}{192 E I_{21}} + \frac{l_1^{\prime}}{E S_1}
% + \frac{l_2^{\prime}}{4 k G S_2} \right), 
% A_4 = \left( \frac{3}{2} + \frac{\sqrt{3} B_3}{2 A_3} \right).
% \end{array}
% \]     로 이루어져있다. 
  By solving the above equations, $\delta_{F_x}$ can be determined as a function of $x$, i.e., based on position.


    \begin{equation}
    \resizebox{0.8\columnwidth}{!}{
    $\begin{cases}
 \delta _{F_x} (x)=\sqrt {\delta _{C_n} (x)^{2}+\delta _{C_t} (x)^{2}} =\delta _{C_t} (x)/cos30^{\circ }\\
       \delta _{C_n} (x)=\frac {F_{C_n} l_{2}^{3}}{192EI_{21}}+\frac {F_{C_n} x}{ES_{1}}+\frac {F_{C_n} x}{4kGS_{2}}\\
       \delta _{C_t} (x)=\frac {F_{C_t} x^{2}(3l_{1}^{\prime }-x)}{6EI_{11}}-\frac {M_{C} x^{2}}{2EI_{11}}+\frac {(F_{C_t} l_{1}^{\prime }-M_{C})l_{2}^{\prime }}{16EI_{21}}x+\frac {F_{C_t} x}{4kGS_{1}}\\
    \end{cases}$}
    \label{3.18}
    \end{equation}

    The deformation equation expressed as $x$ in Equation~\ref{3.18} can be used to calculate $\delta_{F_x}(x)$, which in turn is useful for determining $\Delta d_{F_x}$. Here, $k_{F_x}$ represents the condition when $x$ equals $l_1^{\prime}$.
    \begin{equation}
       \Delta d_{F_x} =k_{F_x} F_x
    \end{equation}
where,
\parbox{\linewidth}{%
   \scalebox{0.8}{
$k_{F_x} = \frac{B_3}{2 A_3 A_4 \cos 30^{\circ}}
\left(\frac{l_1^{\prime 3}}{3 E I_{11}}
- \frac{A_2 l_1^{\prime 2}}{2 E I_{11} B_2} \right.
 + \frac{{l_1^{\prime}}^2 l_2^{\prime} - A_2 l_1^{\prime} l_2^{\prime} / B_2}{16 E I_{21}}
+ \frac{l_1}{4 k G S_1} \bigg)$
}}.
 %   where, \[k_{F_x}=\frac{B_3}{2 A_3 A_4 \cos 30^{\circ}}\left(\frac{l_1^{\prime 3}}{3 E I_{11}}-\frac{A_2 l_1^{\prime 2}}{2 E I_{11} B_2}
 %  \\ +\frac{l_1^{\prime} l_2^{\prime}-A_2 l_1^{\prime} l_2^{\prime} / B_2}{16 E I_{21}}+\frac{l_1}{4 k G S_1}\right), A_2=l_1^{\prime 2} /\left(2 I_{11}\right)+l_1^{\prime} l_2^{\prime} /\left(16 I_{21}\right), 
 %       B_2=l_1^{\prime} / I_{11}+l_2^{\prime} /\left(16 I_{21}\right), A_3=l_1^{\prime 3} /\left(3 E I_{11}\right)+l_1^{\prime} l_2^{\prime} /\left(16 E I_{21}\right)+l_1^{\prime} /\left(k G S_1\right) -A_2\left(l_1^{\prime 2} /\left(2 E I_{11}\right)+l_1^{\prime} l_2^{\prime} /\left(16 E I_{21}\right)\right) / B_2,\\ 
 % B_3=\sqrt{3}\left(l_2^{\prime \1
 
 % 11
 
 
 
 
 %  prime} /\left(192 E I_{21}\right)+l_1^{\prime} /\left(E S_1\right)+l_2^{\prime} /\left(4 k G S_2\right)\right), \text { and }\\ A_4=\left(3 / 2+\sqrt{3} B_3 /\left(2 A_3\right)\right)\].
%     \begin{equation}
%     \begin{cases}

% \end{cases} \label{3.13}\end{equation}

To calculate the deformation of $M_z$, the relationships between $F_C$, $M_C$, $\theta_C$, and $\delta_C$ are utilized.


% $M_z$의 deformation을 구할 때는 $F_C$와 $M_C$, $\theta_C$와 $\delta_C$의 관계를 이용해서 구할 수 있다. 
\begin{equation}\begin{cases}
    3F_{C} r+3M_{C} =M_z\\
    \theta _{C} =\frac {F_{C} l_{1}^{2}}{2EI_{11}}-\frac {M_{C} l_{1}}{EI_{11}}+\frac {(F_{C} l_{1}^{\prime }-M_{C})l_{2}^{\prime }}{16EI_{21}}=-\delta _{C} /r\\
    \delta _{C} =\frac {F_{C} {l}_{1}^{\prime 3}}{3EI_{11}}-\frac {M_{C} {l}_{1}^{\prime 2}}{2EI_{11}}+\frac {(F_{C} l_{1}^{\prime }-M_{C}){l}^{\prime }_{1} l_{2}^{\prime }}{16EI_{21}}+\frac {F_{C} {l}^{\prime }_{1}}{kGS_{1}}\\
         \end{cases}\label{3.20}
    \end{equation}
% $\theta_C$는 equation~\ref{3.20}과 같이 $\delta_C$에서 loading table의 반지름 $r$을 나누면 구할 수 있다.
The value of $\theta_C$ can be obtained from $\delta_C$ by dividing it by the radius $r$ of the loading table, as shown in Equation~\ref{3.20}.


    
    \begin{equation}
    \begin{cases}
    F_{C} =M_z/\left({3\left({r+\frac {1}{a_{2}}}\right)}\right)\\
    M_{C} =M_z/(3(a_{2} r+1))\\
         \end{cases}
    \end{equation}
% $F_C$와 $M_C$와 $M_z$간의 관계는 $a_2$로 표현이 되며 $a_2$는 다음과 같다.:
    The relationship between $F_C$, $M_C$, and $M_z$ is expressed through the parameter $a_2$, which is defined as follows:

   \parbox{\linewidth}{%
   \scalebox{0.8}{
    $a_2=\frac{l_1^{\prime} /\left(E I_{11}\right)+l_2^{\prime} /\left(16 E I_{21}\right)+\left(l_1^{\prime} /\left(2 E I_{11}\right)+l_2^{\prime} /\left(16 E I_{21}\right)\right) \lambda_1}{l_1^{\prime 2} /\left(2 E I_{11}\right)+l_1^{\prime} l_2^{\prime} /\left(16 E I_{21}\right)+\left(l_1^{\prime 2} /\left(3 E I_{11}\right)+l_1^{\prime} l_2^{\prime} /\left(16 E I_{21}\right)+1 /\left(k G S_1\right)\right) \lambda_1}$}}
    \begin{equation}
        \resizebox{0.85\columnwidth}{!}{
    $\begin{cases}
    \delta _{M_z} (x)=F_{C} \bigg(\frac {x^{2}}{6EI_{11}}(-x+3{l}^{\prime }_{1})-\frac {a_{2} x^{2}}{2EI_{11}}\\+\frac {(l_{1}^{\prime }-a_{2})l_{2}^{\prime }}{16EI_{21}}x+\frac {x}{kGS_{1}}\bigg)\\
    \Delta r_{M_z} =k_{M_z} M_z\\
    k_{M_z}=\frac{1}{3r \left(r+1/ a_2\right)}\left(\frac{l_1^{\prime 3}}{3 E I_{11}}-\frac{l_1^{\prime 2} a_2}{2 E I_{11}}+\frac{l_1^2 l_2^{\prime}-l_1^{\prime} l_2^{\prime} a_2}{16 E I_{21}}+\frac{l_1^{\prime}}{k G S_1}\right)
\end{cases}$ }\label{3.22}
    \end{equation}
    



% \begin{figure*}[!ht]

% \end{figure*}

% Equation~\ref{3.22}는 $x$에 대한 식으로 나타낼 수 있으며, $\delta_{M_z}(x)$를 구할 수 있고 $x$가 $l_1^\prime$일 때 $k_{M_z}$를 구할 수 있다. 하지만 최적화를 위해서 센서의 위치도 최적화를 하려고 하기 때문에 각각 센서의 위치에서의 spring constant를 구할 필요가 있다. 이를 이용하면, Equation~\ref{3.23}처럼 각 센서의 위치에 따라서 그 위치에 해당하는 spring constant를 구할 수 있다. 이 값들을 이용해서 deformation과 힘/토크 간의 행렬을 구할 수 있으며, 이는 다음과 같이 나타낼 수 있다.

Equation~\ref{3.22} can be expressed as a function of $x$, allowing the calculation of $\delta_{M_z}(x)$ and determining $k_{M_z}$ when $x$ equals $l_1^{\prime}$. However, to achieve optimization, the sensor positions also need to be optimized. Therefore, it is necessary to calculate the spring constant at each sensor position.

Using this approach, the spring constant corresponding to each sensor position can be determined as shown in Equation~\ref{3.23}. These values are then utilized to compute the deformation and the matrix relating forces/torques. Equation~\ref{3.5} represents the matrix that relates deformation to applied forces and torques based on the spring constants at the optimized sensor positions.
    
% Equation~\ref{3.13} is used to derive the spring constant for $F_y$, while Equation~\ref{3.14} calculates the spring constant for $M_z$. These equations facilitate the calculation of the deflections $\Delta d_{1},\Delta d_{2},\Delta d_{3},\Delta d_{4},\Delta d_{5}, and \Delta d_{6}$.










\begin{equation}  \resizebox{1\columnwidth}{!}{
$\begin{array}{l}
    u = {l_1}^{'} + r - r_{s1} \\
k_{dFzv} = 1/(\frac{1}{3} \left( \frac{u^2 (3{l_1}^{\prime} - u)}{6EI_{12}} - \frac{a_{1} u^2}{2EI_{12}} + \frac{({l_1}^{\prime} - a_{1}) {l_2}^{\prime} u}{4GI_t} + \frac{{{l_2}^{\prime}}^3}{192EI_{22}} + \frac{4{l_1}^{\prime} S_{2} + {l_2}^{\prime} S_{1}}{4k G S_{1} S_{2}} \right)) \\
k_{rMyv} = r/(\frac{2}{3\left(\frac{A_{1}}{B_{1}} + r\right)} \left( \frac{{{l_2}^{\prime}}^3}{192EI_{22}} + \frac{{l_2}^{\prime}}{4k G S_{2}} + \frac{u}{k G S_{1}} + \frac{u^2 (3{l_1}^{\prime} - u)}{6EI_{12}} - \frac{A_{1}}{B_{1}} \frac{u^2}{2EI_{12}} + \frac{({l_1}^{\prime} - \frac{A_{1}}{B_{1}}) {l_2}^{\prime} u}{4GI_t} \right))\\

u = {l_1}^{'} + r - r_{s2} \\
k_{dFxh} = 1/({\frac{B3}{2A3 A4 \cos(\pi/6)} \left( \frac{u^2 (3{l_1}^{\prime} - u)}{6EI_{11}} - \frac{A_{2} u^2}{2EI_{11} B2} + \frac{{l_1}^{\prime} {l_2}^{\prime} u - A_{2} u {l_2}^{\prime} / B2}{16EI_{21}} + \frac{u}{4k G S_{1}} \right)}) \\
k_{dMzh} = 1/({\frac{1}{3(r + \frac{1}{a_{2}})} \frac{u^2 (3{l_1}^{\prime} - u)}{6EI_{11}} - \frac{1}{3(a_{2} r + 1)} \frac{u^2}{2EI_{11}} + \left( \frac{1}{3(r + \frac{1}{a_{2}})} {l_1}^{\prime} - \frac{1}{3(a_{2} r + 1)} \right) \frac{u {l_2}^{\prime}}{16EI_{21}} + \frac{1}{3(r + \frac{1}{a_{2}})} \frac{{l_1}^{\prime}}{k G S_{1}}}) \\
k_{rMyh} = r/({\frac{2}{3\left(\frac{A_{1}}{B_{1}} + r\right)} \left( \frac{{{l_2}^{\prime}}^3}{192EI_{22}} + \frac{{l_2}^{\prime}}{4k G S_{2}} + \frac{u}{k G S_{1}} + \frac{u^2 (3{l_1}^{\prime} - u)}{6EI_{12}} - \frac{A_{1}}{B_{1}} \frac{u^2}{2EI_{12}} + \frac{({l_1}^{\prime} - \frac{A_{1}}{B_{1}}) {l_2}^{\prime} u}{4GI_t} \right)})
\end{array}$}
\label{3.23}
\end{equation}










The term $\Delta d$ denotes the deformations, $\overline{\textbf{G}}$ refers to the regulated matrix, and $\textbf{R}_s^{-1}$ is the regulation matrix, which is defined as:
\begin{equation}
\begin{aligned}
\overline{{\textbf{G}}} = \textbf{R}_s^{-1} \textbf{G}
\end{aligned}
 \label{3.24}
\end{equation}

\begin{equation}
\resizebox{0.8\columnwidth}{!}{$
\begin{aligned}
\min\limits_{\textbf{G}\in\mathbb{R}^{6\times6}}  \text{Function}(\overline{\textbf{G}} ) \quad
\text {s.t} 
\begin{cases} 
\displaystyle{1 \textrm {mm}\le l_{1} \le 20 \textrm {mm}}, \\
\displaystyle {11 \textrm {mm}\le l_{2} \le 30 \textrm {mm}}, \\
\displaystyle {1 \textrm {mm}\le b_{1} \le 10 \textrm {mm}}, \\
\displaystyle {0.5 \textrm {mm}\le b_{2} \le 1 \textrm {mm}}, \\
\displaystyle {1 \textrm {mm}\le h \le 15 \textrm {mm}}, \\
\displaystyle {1 \textrm {mm}\le r\le 8 \textrm {mm}}, \\
\displaystyle {2 \textrm {mm}\le r_{s2}\le 15 \textrm {mm}}, \\
 -l_{1} +3b_{1} < 0, \\ 
 -l_{2} +3b_{2} < 0, \\
-0.02+\sqrt{{(r+l_{1}+b_{2})}^2+(l_{2}/2)^2}<0, \\
\sigma_{bend}-\sigma_{allowable}<0, \\
\sigma_{torsion}-\sigma_{allowable}<0.
\end{cases}
\end{aligned}$}
\label{3.25}
\end{equation}


% \begin{equation}
%  \begin{aligned}
% \resizebox{0.8\columnwidth}{!}{$
%         \min\limits_{\textbf{G}\in\mathbb{R}^{6\times6}}  \text{Function}(\overline{\textbf{G}} ) \quad
% \text {s.t} 
% \begin{cases} 
% \displaystyle{1 \textrm {mm}\le l_{1} \le 20 \textrm {mm}} \\
% \displaystyle {11 \textrm {mm}\le l_{2} \le 30 \textrm {mm}}\\
% \displaystyle {1 \textrm {mm}\le b_{1} \le 10 \textrm {mm}}\\
% \displaystyle {0.5 \textrm {mm}\le b_{2} \le 1 \textrm {mm}}\\
% \displaystyle {1 \textrm {mm}\le h \le 15 \textrm {mm}}\\
% \displaystyle {1 \textrm {mm}\le r\le 8 \textrm {mm}}\\
% \displaystyle {2 \textrm {mm}\le r_{s2}\le 15 \textrm {mm}}\\
%  -l_{1} +3b_{1} < 0 \\ -l_{2} +3b_{2} < 0\\
% -0.02+\sqrt{{(r+l_{1}+b_{2})}^2+(l_{2}/2)^2}<0\\
% \sigma_{bend}-\sigma_{allowable}<0\\
% \sigma_{torsion}-\sigma_{allowable}<0
% \end{cases}$}

% \end{aligned}   \label{3.25}
% \end{equation}


In equation~\ref{3.25}, the constraints for \( l_1 \), \( l_2 \), \( b_1 \), \( b_2 \), \( h \), \( r \), and \( r_{s2} \) were set with broad limits. Given that the diameter is 40 mm, the constraints were defined to ensure compliance, including \( -0.02+\sqrt{{(r+l_{1}+b_{2})}^2+(l_{2}/2)^2}<0 \). Additionally, considering the allowable stress, the modeling process incorporated the accuracy conditions from equations~\ref{3.6} to \ref{3.22}, specifically the constraints \( -l_{1} +3b_{1} < 0 \) and \( -l_{2} +3b_{2} < 0 \).

% 여기서 equation~\ref{3.25}는 $l_1$,$l_2$,$b_1$,$b_2$,$h$,$r$,$r_{s2}$의 각 constraints는 넓게 잡았으며, diameter가 40mm이기 때문에 그것을 위배하지 않는 constraints $-0.02+\sqrt{{(r+l_{1}+b_{2})}^2+(l_{2}/2)^2}<0$와 allowable stress를 고려하면서, 위의 modeling을 할때의 equation~\ref{3.6}~\ref{3.22}가 정확도를 얻기 위한 조건인 $-l_{1} +3b_{1} < 0$와 $ -l_{2} +3b_{2} < 0$의 조건을 고려하였다.

% \begin{align}
% \resizebox{\columnwidth}{!}
% {
% \mathbf{R}_s^{-1} = \text{diag} \left[ \frac{100\%}{520 \, \textrm{N}}, \frac{100\%}{520 \, \textrm{N}}, \frac{100\%}{520 \, \textrm{N}}, \frac{100\%}{15.6 \, \textrm{N} \cdot \textrm{m}}, \frac{100\%}{15.6 \, \textrm{N} \cdot \textrm{m}}, \frac{100\%}{15.6 \, \textrm{N} \cdot \textrm{m}} \right].}
% \end{align}
\begin{align}
\resizebox{0.8\columnwidth}{!}{
$
\mathbf{R}_s^{-1} = \text{diag} \left[ 
    \frac{100\%}{520 \, \textrm{N}}, 
    \frac{100\%}{520 \, \textrm{N}}, 
    \frac{100\%}{520 \, \textrm{N}}, 
    \frac{100\%}{15.6 \, \textrm{N} \cdot \textrm{m}}, 
    \frac{100\%}{15.6 \, \textrm{N} \cdot \textrm{m}}, 
    \frac{100\%}{15.6 \, \textrm{N} \cdot \textrm{m}} 
\right]
$
}\label{3.26}
\end{align}
% Equation~\ref{3.26}은 정규화 matrix로 최대 목표 힘인 520N과 15.6N$\cdot$m을 100\%로 잡고 정규화 matrix를 이용했다. 

Equation~\ref{3.26} is a normalization matrix, where the maximum target force of \( 520 \)N and moment of 15.6N\( \cdot  \)m were set as 100\%, and the normalization matrix was applied accordingly in equation~\ref{3.24}.

\begin{equation}
\begin{split}
   & \hspace{0cm} f=\text{Cond}(\overline{\textbf{G}}) \\
   & \hspace{0cm} f=1/\Vert \overline{\textbf{G}} \Vert_2\\
   & \hspace{0cm} f=1/\Vert \overline{\textbf{G}} \Vert_F\\
   & \hspace{0cm} f=1/\Vert \overline{\textbf{G}} \Vert_*\\
   & \hspace{0cm} f=\text{Cond}(\overline{\textbf{G}})/\Vert \overline{\textbf{G}}\Vert_F\\
   & \hspace{0cm} f=\text{Cond}(\overline{\textbf{G}})\Vert \overline{\textbf{G}} \Vert_2\\
   & \hspace{0cm} f=\text{Cond}(\overline{\textbf{G}})\Vert \overline{\textbf{G}} \Vert_2/\Vert \overline{\textbf{G}}\Vert_F\\
   & \hspace{0cm} f=\text{Cond}(\overline{\textbf{G}})\Vert \overline{\textbf{G}} \Vert_2/\Vert \overline{\textbf{G}}\Vert_*\\
   & \hspace{0cm} f=\text{Cond}(\overline{\textbf{G}})\Vert \overline{\textbf{G}} \Vert_2/{\Vert \overline{\textbf{G}}\Vert_*}^2
\end{split}
\label{3.27}
\end{equation}
% \begin{equation}
% \begin{aligned}
%   & \hspace{0cm} f=\text{Cond}(\overline{\textbf{G}})      \\
%   & \hspace{0cm} f=1/\Vert \overline{\textbf{G}} \Vert_2\\
%   & \hspace{0cm} f=1/\Vert \overline{\textbf{G}} \Vert_F\\
%   & \hspace{0cm} f=1/\Vert \overline{\textbf{G}} \Vert_*\\
%   & \hspace{0cm} f=\text{Cond}(\overline{\textbf{G}})/\Vert \overline{\textbf{G}}\Vert_F\\
%   & \hspace{0cm} f=\text{Cond}(\overline{\textbf{G}})\Vert \overline{\textbf{G}} \Vert_2\\
%   & \hspace{0cm} f=\text{Cond}(\overline{\textbf{G}})\Vert \overline{\textbf{G}} \Vert_2/\Vert \overline{\textbf{G}}\Vert_F\\
%   & \hspace{0cm} f=\text{Cond}(\overline{\textbf{G}})\Vert \overline{\textbf{G}} \Vert_2/\Vert \overline{\textbf{G}}\Vert_*\\
%   & \hspace{0cm} f=\text{Cond}(\overline{\textbf{G}})\Vert \overline{\textbf{G}} \Vert_2/{\Vert \overline{\textbf{G}}\Vert_*}^2
 
% \end{aligned}
 
  
%  \label{3.27}

% \end{equation}
 %  f=\text{Cond}(\overline{\textbf{G}})      \\
 % f=1/\Vert \overline{\textbf{G}} \Vert_2\\
 % f=1/\Vert \overline{\textbf{G}} \Vert_F\\
 % f=1/\Vert \overline{\textbf{G}} \Vert_*\\
 % % f=\text{Cond}(\overline{\textbf{G}})/\Vert \overline{\textbf{G}} \Vert\\
 % f=\text{Cond}(\overline{\textbf{G}})/\Vert \overline{\textbf{G}}\Vert_F\\
 % % f=\text{Cond}(\overline{\textbf{G}})/\Vert \overline{\textbf{G}}_{trace} \Vert\\
 % f=\text{Cond}(\overline{\textbf{G}})\Vert \overline{\textbf{G}} \Vert_2\\
 % f=\text{Cond}(\overline{\textbf{G}})\Vert \overline{\textbf{G}} \Vert_2/\Vert \overline{\textbf{G}}\Vert_F\\
 %  f=\text{Cond}(\overline{\textbf{G}})\Vert \overline{\textbf{G}} \Vert_2/\Vert \overline{\textbf{G}}\Vert_*\\
 %  % f=\text{Cond}(\overline{\textbf{G}})\Vert \overline{\textbf{G}} \Vert/{\Vert \overline{\textbf{G}}_{fro} \Vert}^2\\
 %    f=\text{Cond}(\overline{\textbf{G}})\Vert \overline{\textbf{G}} \Vert_2/{\Vert \overline{\textbf{G}}\Vert_*}^2\\
 %      % f=\text{Cond}(\overline{\textbf{G}})\Vert \overline{\textbf{G}} \Vert/\Vert \overline{\textbf{G}}_{fro} \Vert/\Vert \overline{\textbf{G}}_{trace} \Vert
 %        % f=\text{Cond}(\overline{\textbf{G}})\Vert \overline{\textbf{G}} \Vert/\Vert \overline{\textbf{G}}_{fro} \Vert/{\Vert \overline{\textbf{G}}_{trace} \Vert}^2    




 \begin{figure}[h]
    \centerline{\includegraphics[width=\columnwidth]{femfunc1.png}}
    \caption[Sensor Configuration by Objective Function using FEM]{Sensor Configuration by Objective Function using FEM
    } \label{femfunc}
\end{figure}
\begin{table*}[!h]
\centering
\caption{Optimized Variables, Condition Numbers, and Deformation Analysis of Various Functions}
\resizebox{\textwidth}{!}{
\begin{tabular}{ccccccccccccc}
\hline
$function$    & $l_1$(mm) & $l_2$(mm) & $b_1$(mm) & $b_2$(mm) & $h$(mm) & $r$(mm) & $r_{s2}$(mm) & $\text{Cond}(\overline{\textbf{G}})$ & $F_z$ 100N(mm) & $F_x$ 100N(mm) & $M_z$ 1N$\cdot$m(mm) & $M_y$ 1N$\cdot$m(rad) \\ \hline \hline
$\text{Cond}(\overline{\textbf{G}})$                & 10.97   & 11.00    & 3.656    & 0.5009   & 7.536  & 7.761  & 12.64   & 47.4971    & 5.583e-03 & 7.676e-03 & 2.052e-03 & 1.103e-03 \\ 
$\frac{1}{\Vert \overline{\textbf{G}} \Vert_2}$     & 9.139   & 27.45    & 3.046    & 0.9714   & 6.658  & 4.437  & 15.00   & 1558.73    & 9.999e-03 & 8.373e-03 & 1.247e-03 & 9.755e-04 \\ 
$\frac{1}{\Vert \overline{\textbf{G}} \Vert_F}$     & 9.139   & 27.45    & 3.046    & 0.9714   & 6.658  & 4.437  & 8.237   & 269.821    & 1.000e-03 & 1.673e-02 & 3.070e-03 & 9.755e-04 \\ 
$\frac{1}{\Vert \overline{\textbf{G}} \Vert_*}$     & 14.24   & 11.00    & 3.039    & 0.5535   & 6.690  & 4.437  & 8.739   & 47.8550    & 1.195e-02 & 1.655e-02 & 5.460e-03 & 7.325e-04 \\ 
$\frac{\text{Cond}(\overline{\textbf{G}})}{\Vert \overline{\textbf{G}} \Vert_2}$ & 14.24   & 11.00    & 3.039    & 0.5535   & 6.690  & 4.437  & 10.18   & 47.8485    & 1.195e-02 & 1.572e-02 & 5.646e-03 & 7.309e-04 \\ 
$\frac{\text{Cond}(\overline{\textbf{G}})}{\Vert \overline{\textbf{G}} \Vert_F}$ & 14.24   & 11.00    & 3.039    & 0.5535   & 6.690  & 4.437  & 8.913   & 47.8561    & 1.195e-02 & 1.655e-02 & 5.846e-03 & 7.309e-04 \\ 
$\frac{\text{Cond}(\overline{\textbf{G}})}{\Vert \overline{\textbf{G}} \Vert_*}$ & 14.24   & 11.00    & 3.039    & 0.5535   & 6.690  & 4.437  & 8.716   & 47.8548    & 1.195e-02 & 1.653e-02 & 5.867e-03 & 7.325e-04 \\ 
$\text{Cond}(\overline{\textbf{G}})\Vert \overline{\textbf{G}} \Vert_2$ & 10.92   & 11.00    & 3.640    & 0.7049   & 15.00  & 7.603  & 14.80   & 50.8431    & 1.269e-03 & 2.232e-03 & 8.402e-04 & 8.571e-05 \\ 
$\frac{\text{Cond}(\overline{\textbf{G}})\Vert \overline{\textbf{G}} \Vert_2}{\Vert \overline{\textbf{G}} \Vert_F}$ & 10.69   & 11.00    & 2.934    & 0.5379   & 7.176  & 8.000  & 11.80   & 47.9025    & 6.979e-03 & 1.207e-02 & 3.864e-03 & 3.816e-04 \\ 
$\frac{\text{Cond}(\overline{\textbf{G}})\Vert \overline{\textbf{G}} \Vert_2}{\Vert \overline{\textbf{G}} \Vert_*}$ & 10.99   & 11.00    & 2.938    & 0.5385   & 7.157  & 7.701  & 11.50   & 47.9085    & 6.979e-03 & 1.207e-02 & 3.864e-03 & 3.964e-04 \\
$\frac{\text{Cond}(\overline{\textbf{G}})\Vert \overline{\textbf{G}} \Vert_2}{{\Vert \overline{\textbf{G}} \Vert_F}^2}$ & 14.24   & 11.00    & 3.039    & 0.5535   & 6.690  & 4.437  & 8.908   & 47.8561    & 1.228e-02 & 1.735e-02 & 6.422e-03 & 7.607e-04 \\
$\frac{\text{Cond}(\overline{\textbf{G}})\Vert \overline{\textbf{G}} \Vert_2}{{\Vert \overline{\textbf{G}} \Vert_*}^2}$ & 14.24   & 11.00    & 3.039    & 0.5535   & 6.690  & 4.437  & 8.716   & 47.8548    & 1.228e-02 & 2.125e-02 & 6.449e-03 & 7.607e-04 \\ 
$\frac{\text{Cond}(\overline{\textbf{G}})\Vert \overline{\textbf{G}} \Vert_2}{\Vert \overline{\textbf{G}} \Vert_F\Vert \overline{\textbf{G}} \Vert_*}$ & 14.24   & 11.00    & 3.039    & 0.5535   & 6.690  & 4.437  & 8.800   & 47.8554    & 1.228e-02 & 2.125e-02 & 6.449e-03 & 7.607e-04 \\ \hline
\end{tabular}
}
\label{combined_table}
\end{table*}

The meaning of each function is as follows:

$\text{Cond}(\overline{\textbf{G}})$ represents the ratio of the maximum to minimum singular values, and when it is close to 1, it indicates better isotropy and reduced sensor error. This ensures that no particular mode dominates, leading to more balanced performance across all modes.

$\Vert\overline{\textbf{G}}\Vert_2$ represents the maximum singular value of the sensor matrix, and minimizing the objective function increases $\Vert\overline{\textbf{G}}\Vert_2$, thereby amplifying the maximum singular value. This implies that certain modes are emphasized.

$\Vert \overline{\textbf{G}} \Vert_F$, also known as the Frobenius norm or the 2-norm, is the square root of the sum of all the squared elements of the matrix. This norm reflects the overall magnitude of the matrix, which differs slightly from singular values but is useful for representing the general scale of the matrix. Minimizing the objective function increases $\Vert \overline{\textbf{G}}\Vert_F$, which in turn maximizes the overall sensitivity of the sensor.

$\Vert \overline{\textbf{G}}\Vert_*$ represents the sum of singular values, maximizing the output across all modes, and is another type of norm indicating matrix size. Minimizing the objective function finds a combination that increases the overall sum of the singular values, although the matrix size grows.

Thus, the above four combinations were considered. Minimizing $\text{Cond}(\overline{\textbf{G}})$/$\Vert\overline{\textbf{G}}\Vert_2$ leads to the minimization of $\text{Cond}(\overline{\textbf{G}})$ while maximizing $\Vert\overline{\textbf{G}}\Vert_2$. This also minimizes 1/$\Vert\overline{\textbf{G}}^{-1}\Vert_2$, thereby maximizing $\Vert\overline{\textbf{G}}^{-1}\Vert_2$. Similarly, minimizing $\text{Cond}(\overline{\textbf{G}})$/$\Vert\overline{\textbf{G}}\Vert_F$ minimizes $\text{Cond}(\overline{\textbf{G}})$ while maximizing the overall size of the matrix. Likewise, minimizing $\text{Cond}(\overline{\textbf{G}})$/$\Vert\overline{\textbf{G}}\Vert_*$ minimizes $\text{Cond}(\overline{\textbf{G}})$ while maximizing the sum of the singular values, although this aims to maximize the sum rather than distribute outputs evenly across all modes.

From the sensor's perspective, a more effective objective function would minimize $\text{Cond}(\overline{\textbf{G}})$ and $\Vert\overline{\textbf{G}}\Vert_2$, while maximizing sensor sensitivity. Hence, $\text{Cond}(\overline{\textbf{G}})$$\Vert\overline{\textbf{G}}\Vert_2$ was placed in the numerator, and the sensitivity-representing Frobenius norm and trace norm were used in the denominator for comparison.

 \begin{figure*}[!h]
 \centering
 \includegraphics[width=2\columnwidth]{assembleandcalibration6.png}
    \caption[Sensor Configuration by Objective Function using FEM]{Sensor Configuration by Objective Function using FEM: (a) Each Part of Sensors and Assembly (b) Calibration Method(Reference Sensor: ATI MINI85)
    } \label{sensorconfig}
\end{figure*}

Since $\text{Cond}(\overline{\textbf{G}})$ is defined as $\Vert\overline{\textbf{G}}\Vert_2 \Vert\overline{\textbf{G}}^{-1}\Vert_2$, minimizing $\text{Cond}(\overline{\textbf{G}}) \Vert\overline{\textbf{G}}\Vert_2$ represents $\Vert\overline{\textbf{G}}\Vert_2^2 \Vert\overline{\textbf{G}}^{-1}\Vert_2$. Therefore, the effectiveness of a matrix size norm is limited when its exponent is 1, and it becomes more impactful when the exponent is 2. At this point, the Frobenius norm and trace norm can be compared. The Frobenius norm reflects the overall magnitude, while the trace norm sums the singular values, making them fundamentally different. An increase in the trace norm does not necessarily mean an even increase in output across all modes, but because the maximum singular value norm and $\text{Cond}(\overline{\textbf{G}})$ are present in the numerator, an increase in the trace norm can lead to more balanced outputs. Similarly, increasing the Frobenius norm enhances overall sensitivity but it can not make balanced outputs, and since my goal is to improve sensitivity while minimizing errors and avoiding excessive force in specific directions, I selected the function $\text{Cond}(\overline{\textbf{G}})\Vert\overline{\textbf{G}}\Vert_2$/${\Vert\overline{\textbf{G}}\Vert_*}^2$

Each of the objective functions was compared using $\text{Cond}(\overline{\textbf{G}})$ and finite element method (FEM) results. The findings are summarized in the following table~\ref{condnum11} and table~\ref{condnum111}.

Analyzing Function:

The condition number $\text{Cond}(\overline{\textbf{G}})$ alone does not take into account the magnitude of the singular values. This means that it only serves to ensure that the difference between the singular values is minimized, without contributing to maximizing the overall sensitivity. The term $1/\Vert \overline{\textbf{G}} \Vert_2$ primarily affects the maximum singular value, but does not influence the minimum singular value, thus acting in only one mode or direction rather than improving overall characteristics. On the other hand, $1/\Vert \overline{\textbf{G}} \Vert_F$ reflects the overall size and affects the sensitivity, but does not address the difference between singular values, merely capturing the total size without adequately influencing their magnitudes.

Furthermore, $1/\Vert \overline{\textbf{G}} \Vert_*$ maximizes the trace norm, which corresponds to the sum of the singular values. This could lead to an imbalance where one singular value becomes disproportionately large, thereby enhancing certain directions.

By using $\text{Cond}(\overline{\textbf{G}})\Vert \overline{\textbf{G}} \Vert_2$, the difference between the maximum and minimum singular values is reduced while also decreasing the maximum singular value. This helps to prevent strong influence in a specific direction. However, this may negatively impact sensitivity. Since my goal is to also maximize sensitivity, compensation using the Frobenius norm or the trace norm is necessary.


Comparison with $\text{Cond}(\overline{\textbf{G}})\Vert \overline{\textbf{G}} \Vert_2 / \Vert \overline{\textbf{G}} \Vert_F$: The term $\text{Cond}(\overline{\textbf{G}})\Vert \overline{\textbf{G}} \Vert_2$ minimizes $\text{Cond}(\overline{\textbf{G}})$ to reduce errors while increasing the overall matrix size, thereby considering sensitivity in specific directions. Placing the squared Frobenius norm in the denominator increases the overall size of the sensor but does not yield optimal results in terms of singular values. However, Frobenius norm can not balance singular values. 

Conclusion: I selected $\text{Cond}(\overline{\textbf{G}})\Vert \overline{\textbf{G}} \Vert_2 / {\Vert \overline{\textbf{G}} \Vert_*}^2$. Since the trace norm represents the sum of singular values, it may be more suitable for maximizing sensitivity in specific directions. However, even if the trace norm increases, singular values may not be uniformly distributed. Despite this, with both $\text{Cond}(\overline{\textbf{G}})$ and $\Vert \overline{\textbf{G}} \Vert_2$ present, a more uniform distribution of singular values is possible.

\section{Manufacturing and Calibration}
\begin{figure*}[!h]
    \centering
    \includegraphics[width=2\columnwidth]{static5.png} % 90도 회전 후 세로 방향 크기를 페이지에 맞춤
    \caption[Calibration Result]{Static Test of Calibration Result (a) Result of x-axis Force (b) Result of y-axis Force (c) Result of z-axis Force (d) Result of x-axis Moment (e) Result of y-axis Moment (f) Result of z-axis Moment}
    \label{calres}
\end{figure*}
% 센서는 가공과 silver anodizing을 통해서 제작을 하였으며, 앞에서 구한 최적화한 값을 이용하여 제작을 하였다. 재질은 AL7075-T6을 사용하였으며, 이는 규소함량이 적어서 hysteresis가 적으며, 강도가 강한 알루미늄 재질을 사용하였다. 센서의 측정 범위는 Table~\ref{sensrange}와 같으며, 최대 1900N정도의 측정 범위를 가지고 있다. 

The sensor was fabricated through machining and silver anodizing, utilizing the optimized values derived earlier. The material used is AL7075-T6, an aluminum alloy known for its low silicon content, which minimizes hysteresis, and its high strength properties. 

The measurement range of the sensor is summarized in Table~\ref{sensrange}, with a maximum measurement capacity of approximately 1900 N.

\begin{table}[!h]\caption[Sensor Range]{Sensor Measuring Range}\begin{center}
\begin{tabular}{cc}
\hline
Force/Moment       & Sensing Range             \\ \hline\hline
$F_x$   & -620N$\sim$+620N     \\ 
$F_y$   & -590N$\sim$+590N       \\ 
$F_z$   & -1965N$\sim$+1965N     \\ 
$M_x$ & -13.7N$\cdot$m$\sim$+13.7N$\cdot$m \\ 
$M_y$ & -13.6N$\cdot$m$\sim$+13.6N$\cdot$m \\ 
$M_z$ & -19.6N$\cdot$m$\sim$+19.6N$\cdot$m \\ \hline
\end{tabular}\end{center}\label{sensrange}
\end{table}
The sensor consists of a total of four parts. One is the bottom part, which secures the sensor in place. Another is the printed circuit board (PCB), which was designed based on the optimized sensor positions. The PCB includes a CAN transceiver and is equipped with an STM H7 series MCU, enabling 16-bit ADC measurements to be performed internally and transmitted via CAN communication at speeds of up to 5 Mbps.

Additionally, the sensor includes an elastomer designed with optimized geometry, serving both as a spring and a reflective surface. The final component is the top part, which applies force to the sensor. This configuration comprises three mechanical components and one electronic component.

% Figure~\ref{sensorconfig} shows the overall structure of the sensor, the assembly process, and photos illustrating the calibration method.
% 칼리브레이션은 TCRT1000과 VCNT2020을 거리에 따라서 직접 측정을 하고 거리에 따른 전압 변화를 7차 다항식으로 curve fitting 실험을 통해 한 후에 Least Square Method를 이용하였다. 7차 다항식으로 한 이유는 포토 커플러 센서가 거리에 따른 비선형성이 있기 때문에 그것을 더 보완하기 위한 비선형 Fitting을 하였다.   
% 그 후, Fig.~\ref{calres}는 칼리브레이션을 하고나서 static force/torque를 줬을 때의 결과를 보여준다. 이때 실험은 각각 100seconds 동안 진행하였다. 그렇게 했을 때 percentage에러와 RMSE,  Nonlinearity와 hysteresis는 Table~\ref{combined_error_analysis}와 같이 나왔으며, Max Error는 0.88\%로 작은 수치를 보여주었다. 또한, Hysteresis도 2\%미만을 보여주었다.

The calibration process utilized the ATI-IA MINI-85 as the reference sensor, which has a resolution of 0.08N–0.32N for force and 0.0033N$\cdot$m–0.013N$\cdot$m for torque, with a sampling rate exceeding 5 kHz. The MINI-85 exhibits an approximate full-scale error of 2\%. Given its maximum sensing range of 3800N, it is suitable for measuring the proposed sensor~\cite{kim2024compact}.

% 칼리브레이션에 사용한 센서는 ATI-ia의 MINI-85를 reference 센서로 사용하였으며, 이는 0.08N~0.32N 0.0033N$\cdot$m~0.013N$\cdot$m의 resolution을 가지고 있으며, 5kHz이상의 sampling rate를 가지고 있으며, 약 Full Scale Error는 2\%를 가지고 있다. MINI-85의 센싱 range는 최대 3800N으로 proposed sensor를 측정하는데 사용이 가능하다.~\cite{kim2024compact} 

Calibration was performed using TCRT1000 and VCNT2020 sensors by directly measuring the distance and analyzing the corresponding voltage changes. A 7th-order polynomial curve fitting was applied to the experimental data using the Least Squares Method. The reason for using a 7th-order polynomial is to address the nonlinearity of the photocoupler sensor, which becomes significant as the distance changes, thereby improving the accuracy of the nonlinear fitting.

Figure~\ref{calres} shows the results of the calibration after applying static forces and torques. Each experiment was conducted over a duration of 100 seconds. The percentage error, RMSE, nonlinearity, and hysteresis obtained from the experiments are summarized in Table~\ref{combined_error_analysis}. The maximum error was found to be as low as 0.88\%, and the hysteresis remained below 2\%.

% 센서는 총 4개의 part로 이루어져있다. 하나는 bottom part로 센서를 고정하는 부분이며, 하나는 Printed Circuit board로 최적화된 값을 이용하여 센서의 위치를 정하였기 때문에 그것을 반영하여 PCB를 제작하였다. 여기에는 CAN Transciever가 탑재되어있으며, STM H7시리즈 MCU를 탑재하여 16bit ADC를 자체적으로 측정하고 그것을 CAN통신으로 최대 5Mbps속도로 전송이 가능하다. 다른 부품으로 최적화한 디자인을 이용한 Elastomer가 있으며 스프링 역할과 refletive surface역할을 동시에 해주며, 마지막으로 힘을 가하는 부분인 Top part가 있다. 그렇게 총 Mechanical 부품 3개와 Electonics 부품 한개로 이루어져 있다. Figure~\ref{sensorconfig}를 보면 센서의 전체 모습과 assembly를 하는 과정과 칼리브레이션 방법의 사진을 보여준다. 


% \section{Calibration}

\begin{table}[!h]
\centering
\caption{Error Analysis, Nonlinearity, and Hysteresis}
\begin{tabular}{cccccc}
\hline
     & \multicolumn{3}{c}{Percentage Error (\%)} & \multicolumn{1}{c}{RMS Error} & \multicolumn{1}{c}{\makecell{Nonlinearity \\ Hysteresis }} \\ 
     & \multicolumn{1}{c}{Mean}    & \multicolumn{1}{c}{Std}    & Max    & \multicolumn{1}{c}{(N, N$\cdot$m)} & \multicolumn{1}{c}{(\%)} \\ \hline\hline
$F_x$ & 0.0460 & 0.0742 & 0.3600 & 0.5703 & 0.607  \\ 
$F_y$ & -0.0343 & 0.0706 & 0.3008 & 0.4051 & 0.521  \\ 
$F_z$ & 0.0172 & 0.0416 & 0.1316 & 0.6757 & 0.240  \\ 
$M_x$ & 0.0627 & 0.2669 & 0.8802 & 0.0172 & 1.543 \\ 
$M_y$ & 0.0570 & 0.2410 & 0.8672 & 0.0155 & 1.699  \\ 
$M_z$ & 0.0161 & 0.0852 & 0.3879 & 0.0063 & 0.681 \\ \hline
\end{tabular}
\label{combined_error_analysis}
\end{table}
% repeatabilitytest와 Crosstalk test를 진행했으며, repeatability test는 Figure~\ref{calrep}과 같이 진행을 하였으며, 이때도 Table~\ref{combined_repeatability_crosstalk}과 같이 Max percentage에러가 0.8\%로 작은 수치를 보여주었다. crosstalk test는 Figure~\ref{Crosstalk}과 같이 진행을 하였으며, Max Percentage Error가 x축의 모멘트에서 최대 3\%에러가 나왔다. 하지만, 평균 RMSE가 그에 비해 작아서, Noise값과 Reference Sensor와의 Rise time의 차이에서 발생한것으로 판단된다.
Repeatability and crosstalk tests were conducted. The repeatability test was performed as shown in Figure~\ref{calrep}, and as summarized in Table~\ref{combined_repeatability_crosstalk}, the maximum percentage error was found to be as low as 0.8\%. 

The crosstalk test was conducted as illustrated in Figure~\ref{Crosstalk}, where the maximum percentage error reached up to 3\% in the moment about the x-axis. However, the average RMSE was relatively small, suggesting that the error might have been caused by noise and the difference in rise time between the reference sensor and the tested system.

\begin{figure}[!t]
    \centerline{\includegraphics[width=\columnwidth]{repeatabilitytest1.png}}
    \caption[Repeatability Test Result]{x-axis Force Repeatability Test Result during 100s.
    } \label{calrep}
\end{figure}
\begin{figure*}[h]
    \centerline{\includegraphics[width=2\columnwidth]{crosstalktest2.png}}
    \caption[Crosstalk Test]{Simultaneous Force and Moment Applying Experiment(Crosstalk Test Result) (a) Result of x-axis Force (b) Result of y-axis Force (c) Result of z-axis Force (d) Result of x-axis Moment (e) Result of y-axis Moment (f) Result of z-axis Moment 
    } \label{Crosstalk}
\end{figure*}
\begin{table}[!h]
\centering
\caption{Repeatability Test and Crosstalk Analysis}
\resizebox{\columnwidth}{!}{%
\begin{tabular}{ccccccc}
\hline
\multicolumn{7}{c}{\textbf{Repeatability Test}} \\ \hline\hline
& Mean & Std & Max & RMS Error (N) \\ 
$F_x$ & -0.0116 & 0.1759 & 0.8729 & 0.1432  \\ \hline 
\multicolumn{7}{c}{\textbf{Crosstalk Analysis}} \\\hline \hline 
& $F_x$ & $F_y$ & $F_z$ & $M_x$ & $M_y$ & $M_z$ \\ 
Percentage Error & 0.9422\% & 0.6310\% & 0.1827\% & 3.002\% & 1.974\% & 2.915\% \\ 
RMS Error & 0.7908N & 0.6460N & 1.132N & 0.05596N$\cdot$m & 0.05271N$\cdot$m & 0.01912N$\cdot$m \\ \hline
\end{tabular}%
}
\label{combined_repeatability_crosstalk}
\end{table}


% \begin{table}[h]
% \centering
% \caption{Percentage Error of Repeatability Test }
% \begin{tabular}{ccccc}
% \hline
%   & \multicolumn{3}{c}{Percentage Error}                             &              \\ 
%   & \multicolumn{1}{c}{Mean}    & \multicolumn{1}{c}{Std}   & Max   & RMS error(N) \\ \hline\hline
% $F_x$ & \multicolumn{1}{c}{-0.0116} & \multicolumn{1}{c}{0.1759} & 0.8729 & 0.1432       \\ \hline
% \end{tabular}

% \label{pererrr}
% \end{table}
% \begin{table}[!h]
% \centering
% \caption{Crosstalk Analysis}
% \begin{tabular}{|c|c|c|c|c|c|c|}
% \hline
% Crosstalk & $F_x$       & $F_y$       &$F_z$       & $M_x$       & $M_y$       &$M_z$       \\ \hline
%   Percentage Error        & 0.9422\% & 0.6310\% & 0.1827\% & 3.002\% & 1.974\% & 2.915\% \\
%   RMSE&0.7908N&0.6460N&1.132N&0.05596N$\cdot$ m&0.05271N$\cdot$ m&0.01912N$\cdot$ m\\
%           \hline
% \end{tabular}

% \label{crosst}
% \end{table}
% \begin{table}[!h]
% \centering
% \caption{Percentage Error Analysis and RMS Error}
% \begin{tabular}{ccccc}
% \hline
%      & \multicolumn{3}{c}{Percentage Error(\%)}                           &                   \\ 
%      & \multicolumn{1}{c}{Mean}    & \multicolumn{1}{c}{Std}    & Max    & RMS error(N, N$\cdot$m) \\ \hline\hline
% $F_x$ & \multicolumn{1}{c}{0.0460} & \multicolumn{1}{c}{0.0742
% } & 0.3600 & 0.5703           \\ 
% $F_y$ & \multicolumn{1}{c}{-0.0343}  & \multicolumn{1}{c}{0.0706} & 0.3008 & 0.4051            \\ 
% $F_z$ & \multicolumn{1}{c}{0.0172
% } & \multicolumn{1}{c}{0.0416} & 0.1316 & 0.6757           \\ 
% $M_x$ & \multicolumn{1}{c}{0.0627}  & \multicolumn{1}{c}{0.2669
% } & 0.8802 & 0.0172            \\ 
% $M_y$ & \multicolumn{1}{c}{0.0570}  & \multicolumn{1}{c}{0.2410
% } & 0.8672 & 0.0155          \\ 
% $M_z$ & \multicolumn{1}{c}{0.0161} & \multicolumn{1}{c}{0.0852} & 0.3879 & 0.0063           \\ \hline
% \end{tabular}

% \label{pererr}
% \end{table}

% \begin{table}[]
% \centering
% \caption{Nonlineartiy and Hysteresis}
% \begin{tabular}{cc}
% \hline
%   & Nonlinearity and Hysteresis \\ \hline\hline
% $F_x$ & 0.607\%                     \\ 
% $F_y$ & 0.521\%                     \\ 
% $F_z$ & 0.240\%                     \\ 
% $M_x$ & 1.543\%                     \\ 
% $M_y$ & 1.699\%                     \\ 
% $M_z$ & 0.681\%                     \\ \hline
% \end{tabular}

% \label{nonlin}
% \end{table}

%실제 상용센서와의 비교를 하였다. 비교를 한 센서는 크기가 40mm로 비슷하면서, 내부에서 신호처리장치가 있으며, 비접촉방식인 capacitive type을 사용하는 Robotus의 RFT40과의 resolution비교를 하여 Table~\ref{resanal}에 표현을 하였다. 여기서 z축힘이 최대 10배 이상 resolution이 좋은 것을 볼 수 있으며 모든 축에서의 더 좋은 성능을 보여주었다. 또한, RFT40은 sampling rate가 200Hz이지만, Proposed Sensor는 최대 5kHz까지 측정이 가능하다. 여기서 steps는 측정 범위에서 resolution 값을 나눈 값으로 측정범위에서 몇 step을 구분 할 수 있는지의 성능을 보여준다.
A comparison was made with a commercially available sensor. The sensor used for comparison is the RFT40 from Robotus, which has a similar size of 40mm, an internal signal processing unit, and utilizes a non-contact capacitive type. The resolution comparison results are summarized in Table~\ref{resanal}. 

 \begin{figure*}[!h]
    \centerline{\includegraphics[width=1.8\columnwidth]{experiment2.png}}
    \caption[Detailed Motion Experiment of Quadruped Robot]{Application of Proposed Sensor and Experiments for Verification of Proposed Sensor (a) A Picture of the foot with sensors (b) Posture Adjustment Experiment (c) Rough Terrain Experiment(Durability Test) (d) Attached Sensor Modules (e) Various Speed Experiment  } \label{detailmotion}
\end{figure*}

It can be observed that the proposed sensor demonstrates more than 10 times better resolution in the z-axis force and exhibits superior performance across all axes. Additionally, while the RFT40 has a sampling rate of 200Hz, the proposed sensor achieves a maximum sampling rate of 5kHz. Here, the term "steps" represents the number of distinguishable steps within the measurement range, calculated by dividing the measurement range by the resolution, which indicates the performance of the sensor in distinguishing different levels within its range.



\begin{table}[!h]
\centering
\caption{Resolution Analysis between Proposed Sensor and Commercial Sensor}
\resizebox{\columnwidth}{!}{
\begin{tabular}{ccccccc}
\hline
                & $F_x$      & $F_y$      & $F_z$      & $M_x$         & $M_y$        & $M_z$        \\ \hline\hline
RFT40           & 0.2N       & 0.2N       & 0.2N       & 8mNm          & 8mNm         & 8mNm         \\ 
Resolution(steps)      & 1000       & 1000       & 1500       & 625           & 625          & 625          \\ 
Proposed Sensor & 0.2436N    & 0.1429N    & 0.2017N    & 16mNm         & 21mNm        & 7.5mNm       \\ 
Resolution(steps)      & 5091       & 8258       & 19487      & 1716          & 1283         & 5272         \\ \hline
\end{tabular}
}
\label{resanal}
\end{table}

\section{Experiment of Quadruped Robot}
 \begin{figure*}[!ht]
    \centerline{\includegraphics[width=2\columnwidth]{detailmotionforcecom2.png}}
        \caption[Detailed Motion GRF Experiment of Quadruped Robot]{Postural Adjustment Experiment of A Quadruped Robot's Torso (a) Position Graph of Torso (b) Rotation Graph of Torso (c) Ground Reaction Force of Left Foot (d) Ground Reaction Force of Right Foot } \label{detailedmotiongrflf}
\end{figure*}
Unlike conventional manipulators, quadruped robots experience high-impact conditions. Typically, the impact force can reach 2 to 3 times the ground reaction force (GRF), which is directly related to the robustness of the sensor. Such repetitive impacts are a critical issue in quadruped robots. Therefore, experiments were conducted to evaluate whether the proposed sensor is suitable for quadruped robots. The objective was to verify its capability to measure small force variations, its robustness against high-impact conditions, and its performance across various speeds.

% 사족 로봇은 일반적인 manipulator와 다르게 임팩트가 큰 상황이 발생을 한다. 보통 grf의 2~3 배가 impact로 나오게 되며, 이는 센서의 robustness와 관계가 있다. 이러한 반복적인 임팩트는 족형 로봇에서 critical한 문제이며, 이로 인해 proposed sensor가 족형 로봇에 맞는지 확인하기 위해 실험을 진행하였다. 목적은 여러가지 속도, 임팩트가 큰 상황, 작은 힘의 변화도 측정이 가능한지 확인을 하고 impact에 강인한지 확인하기 위함이다.


The proposed sensor was experimentally validated by mounting it on the foot of a quadruped robot. The RFT40 sensor was used as a reference sensor for comparison. Although the RFT40 has a smaller measurement range, its absolute accuracy is higher due to its narrower range, despite having the same relative accuracy of 1\%. The RFT40 is capable of measuring forces up to 150N in the z-axis and supports up to three times overload protection.

The quadruped robot used in the experiment was custom-built and has a maximum torque of 18N$\cdot$m. The robot weighs approximately 11kg, making it relatively lightweight. For the experiments, both the reference sensor (RFT40) and the proposed sensor were directly connected to the front legs of the robot. Due to the larger measurement range of the proposed sensor, it was placed on the upper side during the experiments.

The experiments consisted of three scenarios: posture adjustment, speed-based tests, and impact tests conducted on a gravel surface to simulate high-impact conditions.

% 실제로 사족 로봇 발에 탑재를 하여 실험을 진행하였다. 이때 RFT40을 Reference Sensor로 사용을 하였다. RFT40은 센서 range가 작아서 정확도가 같은 1\%라고 하였을 때 range가 넓은 proposed sensor와 비교했을 때 절대적인 정확도는 더 높기 때문에 사용할 수 있다. RFT40은 z축으로 150N의 힘을 측정할 수 있으며, 최대 3배의 overload 성능을 가지고 있다. 사족 로봇은 자체적으로 제작한 사족 로봇을 사용했으며, 최대 18N$\cdot$m의 토크를 가지고 있으며, 사족 로봇의 무게는 약 11kg으로 가벼운 편에 속한다. 실험은 앞의 두발에 Reference Sensor인 RFT40과 Proposed Sensor를 직결 연결을 하였으며, Proposed Sensor의 range가 더 크기 때문에 위에 배치를 하여 실험을 진행하였다. 여기서 실험은 자세를 변경하는 실험과 속도에 따른 실험, 그리고 자갈밭을 만들어 impact가 클 때의 실험을 진행하였다. 



Figure~\ref{detailmotion}(d) illustrates the sensor attachment on the foot, where the proposed sensor is positioned above and the commercial sensor is directly connected below. Both sensors communicate via CAN, as depicted in Figure~\ref{detailmotion}(a), which shows the signal transmission into the main body. Additionally, Figure~\ref{detailmotion}(b) presents an experiment in which postural variations are introduced to measure the ground reaction force, aiming to validate the proposed sensor by comparing it with the reference sensor. Figure~\ref{detailmotion}(c) demonstrates an experiment conducted on a gravel surface to evaluate the sensor's performance under high-impact conditions by comparing its measurements with those of the reference sensor. Lastly, Figure~\ref{detailmotion}(e) presents experimental results obtained at walking speeds of 0.2~m/s, 0.4~m/s, and 0.8~m/s.

The controller used in the experiment was a Customized Convex Model Predictive Controller along with a Whole-Body Controller, which were implemented based on~\cite{di2018dynamic, katz2019mini,kim2019highly}. For CAN communication, a Peak-CAN M.2 module was utilized. The control system was executed on a single-board computer, which simultaneously performed data logging and control tasks. The control signals were synchronized using Lightweight Communications and Marshalling (LCM)~\cite{huang2010lcm}. Additionally, the Chrono function was employed to log data with millisecond precision.



For the experiment, the RFT40 force-torque sensor from Robotus was used. The RFT40 is capable of measuring forces up to 150~N and has a maximum overload capacity of 450~N. The sensor was directly mounted for the experiment, and as a result, the forces along the x, y, and z axes, as well as the moment about the z-axis, could be expected to be equal. However, the moments about the x and y axes vary depending on the distance from the point where the force is applied. Therefore, only the forces along the x, y, and z axes were compared.

Typically, the elastomer of a force/torque sensor deforms by less than 0.1~mm, meaning that when directly connected, the same force can be assumed to be applied. However, in this setup, the RFT40 had its ground connected to the chassis. Although the robot is made of aluminum, its chassis was not grounded, leading to noise issues. To mitigate this, plastic insulation was used at the connection points when mounting the RFT40 to prevent electrical grounding interference.


%여기서 실험에 사용하기 위해 Robotus사의 RFT40을 사용하였으며, 이때 RFT40은 150N의 힘을 측정할 수 있고 최대 450N의 Overload가 가능하다. 센서를 직결하여 실험을 진행한 결과 x,y,z축 힘과 z축 모멘트는 같게 될 수 있지만, x, y축 모멘트는 힘이 가해지는 부분에서 더 멀어질 수록 커지기 때문에 같지 않아, x,y,z축 힘을 비교를 하였다. 여기서 보통 힘/토크 센서의 elastomer는 변형이 0.1mm가 안되기 때문에 직결을 하면 같은 힘이 가해진다고 볼 수 있다. 하지만, 여기서 RFT40은 샤시에 ground가 물려있기 때문에 로봇이 알루미늄으로 이루어져 있지만, 샤시에 ground가 물려있지 않기 때문에 noise가 발생해 RFT40을 연결할 때 연결부위사이에 플라스틱을 이용하여 절연을 시켰다. 

% 실험에 사용한 제어기는 Customized Convex Model Predictive Controller와 Whole body controller를 사용하였으며, 이는 ~\cite{di2018dynamic, katz2019mini}에서 사용한 것이며 여기에 CAN통신은 Peak-CAN m.2를 사용하였으며, 이는 single board computer에서 동시에 데이터 로깅을 진행하면서, 제어를 같이 하였으며, 제어에서 나오는 신호는 Lightweight Communications and Marshalling(LCM)을 통해 데이터를 동기화를 시켰다. 이때 Chrono함수를 사용하여 milisecond 단위로 데이터 로깅을 진행하였다.  

% Figure~\ref{detailmotion}(d)는 발에 센서를 단 것을 보여주며, proposed sensor가 위에 아래에 commercial sensor가 직결로 연결되어있는 것을 볼 수 있으며, 각각의 센서는 CAN 통신을 통해 몸통 내부로 들어가는 것을 Figure~\ref{detailmotion}(a)에서 볼 수 있다. 또한, Fig.~\ref{detailmotion}(b)에서는 자세를 변화하면서 ground reaction force를 측정하여 proposed sensor와 reference sensor간의 비교를 하여 검증을 하려고 하였다. 또한, Fig.~\ref{detailmotion}(c)는 자갈밭을 만들어서 거기에서 보행을 하여 큰 impact가 발생할 때도 측정이 가능한지 reference sensor와 비교를 하는 실험을 보여주며, Fig.~\ref{detailmotion}(e)는 0.2m/s, 0.4m/s, 0.8m/s일때의 각각 속도에 따라서 실험을 한 것을 보여준다. 

\subsection{Postural adjustment experiment of a quadruped robot's torso}
% 먼저 사족 로봇의 자세 변경을 하면서 사족 로봇의 자세를 변경할 때의 미세한 힘 변화 측정을 잘 할 수 있는지 확인을 하는 실험을 진행하였다. 이는 자세를 변경할 때 Floating based robot인 사족 로봇에서는 각 발에서의 ground reaction force가 중요하기 때문에 그것을 검증하기 위한 것이다. 이 실험은 Figure~\ref{detailmotion}(b)와 같으며, 이때, 몸통은 z축 position을 변경하고 roll과 pitch yaw를 변경하여 실험을 진행하였다. 



First, an experiment was conducted to verify whether the proposed sensor can accurately measure subtle force variations while changing the posture of the quadruped robot. Since quadruped robots are floating-based systems, the ground reaction force at each foot plays a crucial role during posture adjustments. This experiment aimed to validate the sensor's capability in this context.

The experiment setup is illustrated in Figure~\ref{detailmotion}(b). During the experiment, the robot's torso was manipulated by altering its position along the z-axis and adjusting its roll, pitch, and yaw angles.

% Table~\ref{combined_rms_max_error}에서 보면 RMSE는 최대 1.97N이고 0.36\%의 max error가 나왔다. 오른발이 좀더 정확한 모습을 보였는데 이는 센서의 calibration의 차이와 로봇의 무게중심이 정 가운데가 아니고 commercial sensor를 부착시킬 때 절연을 하기 위해서 가운데 플라스틱을 삽입했던 부분에서의 Hysteresis와 deformation이 생겨서 생긴 오차로 볼 수 있다. 
As shown in Table~\ref{combined_rms_max_error}, the RMSE was at most 1.97 N, and the maximum error was 0.36\%. The right foot exhibited slightly higher accuracy, which can be attributed to differences in sensor calibration and the fact that the robot's center of mass is not perfectly centered. Additionally, the hysteresis and deformation caused by inserting plastic insulation in the middle section when attaching the commercial sensor contributed to the observed errors.


\subsection{Speed adjustment experiment of a quadruped robot}
 \begin{figure*}[!ht]
    \centerline{\includegraphics[width=2\columnwidth]{vaspeed2.png}}
        \caption[Detailed Motion GRF Experiment of Quadruped Robot]{Speed Adjustment Experiment of A Quadruped Robot Result (a) Left Foot Ground Reaction Force at 0.2m/s (b) Left Foot Ground Reaction Force at 0.4m/s (c) Left Foot Ground Reaction Force at 0.8m/s (d) Right Foot Ground Reaction Force at 0.2m/s (e) Right Foot Ground Reaction Force at 0.4m/s (f) Right Foot Ground Reaction Force at 0.8m/s} \label{vasgrflf}
\end{figure*}
The next experiment involved measuring the ground reaction force (GRF) at speeds of 0.2 m/s, 0.4 m/s, and 0.8 m/s and comparing the results with the reference sensor. Since the force varies during acceleration from a stationary state to each speed, these transitions were analyzed. When the robot reaches its maximum speed, the ground reaction force generally stabilizes at similar values. However, at 0.8 m/s, maintaining balance becomes more challenging, requiring adjustments to roll and pitch balance, which results in increased force exertion. Although the sensor was directly attached, the deformation of the plastic insulation, previously mentioned as a structural component, contributed to measurement errors.

Figure~\ref{vasgrflf} illustrates the ground reaction forces for different speeds: (a), (b), and (c) correspond to the left foot at 0.2 m/s, 0.4 m/s, and 0.8 m/s, respectively, while (d), (e), and (f) represent the right foot under the same conditions. The results indicate that as speed increases, the force required to accelerate from rest to the target velocity also increases, leading to a larger ground reaction force. Furthermore, at 0.8 m/s, the quadruped robot's torso exhibits significant instability, necessitating greater ground reaction forces in the pitch and roll directions to maintain balance.

Each experiment began with the robot in a seated position, followed by a step input command for walking. In Figure~\ref{vasgrflf} (b), (c), (e), and (f), which correspond to the results for 0.4 m/s and 0.8 m/s, the force in the z-axis exhibits pronounced peaks during both acceleration at the beginning and deceleration toward the end. Table~\ref{combined_rms_max_error} shows that RMSE tends to increase with speed. Specifically, at 0.2 m/s, the RMSE reached a maximum of 3.96 N with a maximum percentage error of 1.70\%. At 0.4 m/s, the RMSE was at most 4.50 N, with a maximum percentage error of 1.72\%. At 0.8 m/s, the RMSE increased to 5.20 N, with a maximum percentage error of 1.97\%. 

As the applied force increased, deformation in the plastic insulation between the sensors became more pronounced. Additionally, since the proposed sensor has a sampling rate over five times higher than the reference sensor, impact forces resulted in greater discrepancies in force measurement. Nevertheless, the measured ground reaction forces remained within a range suitable for control applications.


%다음으로 진행한 실험은 speed를 0.2m/s 0.4m/s 0.8m/s일때 그때의 Ground reaction force를 측정하고 reference sensor와 비교를 하는 것을 진행하였다. 정지 상태에서 각 속도로 가속을 할 때 힘이 달라지므로 그때를 비교를 하였으며, 최고 속도에 도달 할 때는 보통 비슷한 Ground reaction force를 가진다. 하지만, 0.8m/s일 때는 균형이 잡기 힘들어 그것을 해결하기 위해 롤과 피치의 밸런스를 다시 잡아서 더 큰 힘이 작용하게 된다. 센서를 직결을 하긴 했지만, 앞에서 언급했던 가운데 절연을 위한 플라스틱의 변형이 오차를 발생시키는 것을 알 수 있었다. Figure~\ref{vasgrflf}를 보면 Figure~\ref{vasgrflf}의 (a),(b),(c)는 각각 0.2m/s, 0.4m/s, 0.8m/s의 속도로 갈때의 왼쪽 발의 Ground Reaction Force를 보여주며, (d),(e),(f)는 오른 발의 Ground Reaction Force를 보여준다. 이때, 결과를 보면 속도가 빠를 수록 멈춰 있을 때부터 그 속도로 가속할 때의 힘이 크기 때문에 점점 Ground Reaction Force가 커지는 것을 볼 수 있다. 또한, 0.8m/s일 때, Quadruped Robot의 Torso의 Balance가 많이 무너지는데 피치 방향과 롤 방향에서 밸런스를 유지하기 위해서 Ground Reaction Force가 더 큰 것을 볼 수 있다. 각각의 실험은 처음에 앉아있다가 일어나서 스텝 인풋으로 걸음을 줬을 떄를 보여준다. 이때, 0.4m/s와 0.8m/s의 결과인 Figure~\ref{vasgrflf} (b),(c),(e),(f)를 보면 가속을 하는 부분인 처음과 감속을 하는 부분인 뒤쪽에서 z축의 힘 그래프가 큰 것을 볼 수 있다. 이때, Table~\ref{combined_rms_max_error}를 보면, 속도가 높을 수록 RMSE가 커지는 경향을 볼 수 있었으며, 0.2m/s일때의 RMSE는 최대 3.96N과 최대 1.70\%의 percentage오차를 보여주었으며, 0.4m/s일때는 RMSE는 최대 4.50N과 최대 1.72\%의 percentage 오차를 보여주었다. 0.8m/s일때는 최대 5.20N의 RMSE를 보여주었으며, 1.97\%의 최대 percentage 오차를 보여주었다. 여기서 힘이 커질 수록 센서 사이에 있는 플라스틱 부분에서 변형이 일어나며 샘플링 rate가 proposed sensor가 5배 이상 빠르기 때문에 임팩트 부분에서 힘의 오차가 생기는 것을 보여주었다. 그럼에도 Ground Reaction Force를 봤을 때 제어에 사용할 수 있는 수치를 보여주었다. 




\subsection{Rough Terrain Experiment of a Quadruped Robot}

%다음으로 진행한 실험은 자갈밭을 만들어서 그 위에서 보행 실험을 하여 임팩트가 큰 상황에서도 센서를 사용할 수 있는지 검증하기 위한 실험을 하였다. 이때, 자갈은 약 3cm에서 5cm 직경을 가지고 있는 자갈을 사용하였으며, 로봇이 균형을 잡기 위한 제어를 하기 때문에 Ground Reaction Force는 더 커지며, 자갈과 직접적으로 센서가 부딪혀 임팩트가 전달되기 때문에 내구성을 보는 실험으로도 볼 수 있다. 
% 실험은 이전 실험과 동일하게 앉아있을 때 시작하여 일어나서 걸어서 자갈 밭으로 가는 시나리오로 진행을 하였다. 여기서 약 20초간 걸음을 진행하였을 때 Figure~\ref{roughrfz}(a)에서 보여주는 것처럼 약 120N의 힘이 발생한 이후에 commercial sensor에 offset이 생기는 것을 볼 수 있다. Figure~\ref{roughrfz}(a)는 약 40에서 50초 부근의 모습을 확대한 그래프로 임팩트가 있고나서 commercial sensor에서 z축으로 4N의 힘 offset이 생긴 것을 볼 수 있다. commercial sensor의 z축 힘 센싱 capacity는 150N이며 overload가 300\%이기 때문에 450N까지 버틸 수 있다. 하지만, 자갈 밭과 같이 임팩트가 클 경우에는 일반적인 Ground Reaction Force가 가해질 때보다 몇배의 힘이 순간적으로 가해지기 때문에 Elastomer에 영구적인 변형이 올 수 있다. 

The next experiment was conducted to verify whether the sensor remains functional under high-impact conditions by performing a walking test on a gravel surface. The gravel used in this experiment consisted of stones with diameters ranging from approximately 3 cm to 5 cm. Due to the robot’s active balance control, the ground reaction force increased. Additionally, as the sensor made direct contact with the gravel, impact forces were transmitted, making this experiment also a durability assessment.

The experiment followed the same procedure as the previous one, starting with the robot in a seated position, then standing up and walking toward the gravel surface. Over a duration of approximately 20 seconds, as shown in Figure~\ref{roughrfz}(a), a force of about 120 N was exerted, after which an offset appeared in the commercial sensor. Figure~\ref{roughrfz}(a) presents a magnified view of the force profile around the 40 to 50-second mark, showing that an impact event caused a 4 N force offset in the z-axis of the commercial sensor.

The z-axis force sensing capacity of the commercial sensor is 150 N, with an overload tolerance of 300\%, meaning it can withstand forces up to 450 N. However, in high-impact environments such as a gravel surface, transient forces can be several times greater than those encountered under normal ground reaction conditions. This can result in permanent deformation of the elastomer, affecting the sensor's performance.


% Enumeration of section headings is desirable, but not required. When numbered, please be consistent throughout the article, that is, all headings and all levels of section headings in the article should be enumerated. Primary headings are designated with Roman numerals, secondary with capital letters, tertiary with Arabic numbers; and quaternary with lowercase letters. Reference and Acknowledgment headings are unlike all other section headings in text. They are never enumerated. They are simply primary headings without labels, regardless of whether the other headings in the article are enumerated. 
% Enumeration of section headings is desirable, but not required. When numbered, please be consistent throughout the article, that is, all headings and all levels of section headings in the article should be enumerated. Primary headings are designated with Roman numerals, secondary with capital letters, tertiary with Arabic numbers; and quaternary with lowercase letters. Reference and Acknowledgment headings are unlike all other section headings in text. They are never enumerated. They are simply primary headings without labels, regardless of whether the other headings in the article are enumerated. 

% Enumeration of section headings is desirable, but not required. When numbered, please be consistent throughout the article, that is, all headings and all levels of section headings in the article should be enumerated. Primary headings are designated with Roman numerals, secondary with capital letters, tertiary with Arabic numbers; and quaternary with lowercase letters. Reference and Acknowledgment headings are unlike all other section headings in text. They are never enumerated. They are simply primary headings without labels, regardless of whether the other headings in the article are enumerated. 
%  \begin{figure}[!h]
%     \centerline{\includegraphics[width=\columnwidth]{detailedmotiongrfLFwhak3.png}}
%         \caption[Detailed Motion GRF Experiment of Quadruped Robot]{Enlargement of Detailed Motion GRF Experiment of Quadruped Robot Left Foot} \label{detailedmotiongrfwhaklf}
% \end{figure}
% Enumeration of section headings is desirable, but not required. When numbered, please be consistent throughout the article, that is, all headings and all levels of section headings in the article should be enumerated. Primary headings are designated with Roman numerals, secondary with capital letters, tertiary with Arabic numbers; and quaternary with lowercase letters. Reference and Acknowledgment headings are unlike all other section headings in text. They are never enumerated. They are simply primary headings without labels, regardless of whether the other headings in the article are enumerated. 

% \begin{table}[!h]
% \centering
% \caption{RMS Error and Max Percentage Error between Commercial Sensor and Proposed Sensor during Detailed Motion Experiment}
% \resizebox{\columnwidth}{!}{
% \begin{tabular}{cccccccc}
% \hline
% RMSE            & $F_x$(N) & $F_y$(N) & $F_z$(N) & RMSE            & $F_x$(N) & $F_y$(N) & $F_z$(N) \\ \hline\hline
% Left Foot       & 1.0845   & 1.0055   & 1.9748   & Right Foot      & 0.6997   & 0.7298   & 1.8449   \\ 
% MAX Percentage Error & $F_x$(\%) & $F_y$(\%) & $F_z$(\%) & MAX Percentage Error & $F_x$(\%) & $F_y$(\%) & $F_z$(\%) \\ 
% Left Foot       & 0.3574   & 0.3331   & 0.1639   & Right Foot      & 0.2009   & 0.2995   & 0.1792   \\ \hline
% \end{tabular}
% }
% \label{RMSMAXDE}
% \end{table}


%  \begin{figure}[!h]
%     \centerline{\includegraphics[width=\columnwidth]{detailedmotiongrfRF1.png}}
%         \caption[Detailed Motion GRF Experiment of Quadruped Robot]{Detailed Motion GRF Experiment of Quadruped Robot Right Foot} \label{detailedmotiongrfrf}
% \end{figure}
%  \begin{figure}[!h]
%     \centerline{\includegraphics[width=\columnwidth]{detailedmotiongrfRFwhak1.png}}
%         \caption[Detailed Motion GRF Experiment of Quadruped Robot]{Enlargement of Detailed Motion GRF Experiment of Quadruped Robot Right Foot} \label{detailedmotiongrfwhakrf}
% \end{figure}




 \begin{figure}[!t]
    \centerline{\includegraphics[width=\columnwidth]{roughter2.png}}
        \caption[Commercial Sensor's z-axis Force of Ground Reaction Force at Rough Terrain]{Rough Terrain Experiment Result(Durability Test) (a) Result of Left Foot Ground Reaction Force at Rough Terrain (b) Result of Right Foot Ground Reaction Force at Rough Terrain (c) Enlargement of (b) at 45s: Commercial Sensor Offset Occurs (d) Commercial Sensor's z-axis Right Foot Force of Ground Reaction Force at Rough Terrain} \label{roughrfz}
\end{figure}

% rough terrain 에서 발생한 offset이 우연인지 확인하기 위해 오랜시간동안 자갈밭에서 움직이도록 약 5000초 동안 걸음을 하여 1초에 사족 보행 로봇이 약 2번의 step을 하기 때문에 10000번 정도의 보행 실험을 진행하였으며, 이는 임팩트를 계속해서 줘서 처음과 나중의 offset의 차이를 보기 위해 실험을 진행하였다. 이 실험은 센서의 내구성 실험으로도 볼 수 있다. 결과는 Figure ~\ref{roughrfz}(b)와 같이 commercial sensor의 z축 값이 계속 offset을 보여주는 것을 볼 수 있다. 

To determine whether the offset observed on rough terrain was incidental, an extended walking experiment was conducted on the gravel surface for approximately 5000 seconds. Given that the quadruped robot takes approximately two steps per second, this corresponds to around 10,000 walking steps. The experiment was designed to continuously apply impact forces and observe any changes in the offset over time. This also serves as a durability test for the sensor.

As shown in Figure~\ref{roughrfz}(b), the results indicate that the z-axis readings of the commercial sensor exhibit a persistent offset throughout the experiment.


Figure~\ref{roughrfz}(b) illustrates the drift in the z-axis forces measured by the commercial sensors, which show a continuous change over time. For the right foot, the force steadily decreases, whereas for the left foot, the force alternates between decreasing and increasing repeatedly.

During the 5000-second experiment, the generalized coordinates were also logged. This experiment provided insights into the sensor's durability under such conditions.





\begin{table}[!h]
\centering
\caption{RMS Error and Max Percentage Error between Commercial Sensor and Proposed Sensor}
\resizebox{\columnwidth}{!}{%
\begin{tabular}{cccccccc}
\hline
\multicolumn{8}{c}{\textbf{Detailed Motion Experiment}} \\ \hline\hline
RMSE      & $F_x$(N)      & $F_y$(N)      & $F_z$(N)      & RMSE      & $F_x$(N)      & $F_y$(N)      & $F_z$(N)        \\ 
Left Foot    & 1.0845 & 1.0055 & 1.9748 & Right Foot   & 0.6997 & 0.7298 & 1.8449 \\ 
MAX Percentage Error      & $F_x$(\%)      & $F_y$(\%)      & $F_z$(\%)      & MAX Percentage Error       & $F_x$(\%)      & $F_y$(\%)      & $F_z$(\%)        \\ 
Left Foot & 0.3574 & 0.3331 & 0.1639 & Right Foot & 0.2009 & 0.2995 & 0.1792 \\ \hline
\multicolumn{8}{c}{\textbf{Various Speed Experiment (0.2 m/s)}} \\ \hline
RMSE      & $F_x$(N)      & $F_y$(N)      & $F_z$(N)      & RMSE      & $F_x$(N)      & $F_y$(N)      & $F_z$(N)        \\ 
Left Foot    & 2.0526 & 1.3997 & 3.9046 & Right Foot   & 1.7034 & 1.3007 & 3.9648 \\ 
MAX Percentage Error      & $F_x$(\%)      & $F_y$(\%)      & $F_z$(\%)      & MAX Percentage Error       & $F_x$(\%)      & $F_y$(\%)      & $F_z$(\%)        \\ 
Left Foot & 1.7015 & 1.1181 & 1.1140 & Right Foot & 1.0028 & 0.8712 & 0.7613 \\ \hline
\multicolumn{8}{c}{\textbf{Various Speed Experiment (0.4 m/s)}} \\ \hline
RMSE      & $F_x$(N)      & $F_y$(N)      & $F_z$(N)      & RMSE      & $F_x$(N)      & $F_y$(N)      & $F_z$(N)        \\ 
Left Foot    & 2.2480 & 1.4836 & 3.6591 & Right Foot   & 1.3974 & 1.5370 & 4.5013 \\ 
MAX Percentage Error      & $F_x$(\%)      & $F_y$(\%)      & $F_z$(\%)      & MAX Percentage Error       & $F_x$(\%)      & $F_y$(\%)      & $F_z$(\%)        \\ 
Left Foot & 1.7192 & 1.2257 & 0.9855 & Right Foot & 1.0305 & 0.8786 & 0.7790 \\ \hline
\multicolumn{8}{c}{\textbf{Various Speed Experiment (0.8 m/s)}} \\ \hline
RMSE      & $F_x$(N)      & $F_y$(N)      & $F_z$(N)      & RMSE      & $F_x$(N)      & $F_y$(N)      & $F_z$(N)        \\ 
Left Foot    & 2.7110 & 1.8895 & 4.2853 & Right Foot   & 2.1333 & 2.1510 & 5.1981 \\ 
MAX Percentage Error      & $F_x$(\%)      & $F_y$(\%)      & $F_z$(\%)      & MAX Percentage Error       & $F_x$(\%)      & $F_y$(\%)      & $F_z$(\%)        \\ 
Left Foot & 1.9716 & 1.4431 & 1.1507 & Right Foot & 1.2335 & 0.8817 & 0.9066 \\\hline
\multicolumn{8}{c}{\textbf{Rough Terrain Experiment}} \\ \hline
RMSE      & $F_x$(N)      & $F_y$(N)      & $F_z$(N)      & RMSE      & $F_x$(N)      & $F_y$(N)      & $F_z$(N)        \\ 
Left Foot    & 2.4558 & 2.2298 & 4.7692 & Right Foot   & 2.3660 & 2.4369 & 7.1704 \\ 
MAX Percentage Error      & $F_x$(\%)      & $F_y$(\%)      & $F_z$(\%)      & MAX Percentage Error       & $F_x$(\%)      & $F_y$(\%)      & $F_z$(\%)        \\ 
Left Foot & 1.8688 & 1.4903 & 1.3781 & Right Foot & 1.8518 & 1.7438 & 1.3683 \\ \hline
\end{tabular}%
}
\label{combined_rms_max_error}
\end{table}


% \begin{table}[!h]
% \centering
% \caption{RMS Error and Max Percentage Error between Commercial Sensor and Proposed Sensor during Detaid Motion Experiment}
% \begin{tabular}{|c|c|c|c|c|c|c|c|}
% \hline
% RMSE      & $F_x$(N)      & $F_y$(N)      & $F_z$(N)      &RMSE      & $F_x$(N)      & $F_y$(N)      & $F_z$(N)        \\ \hline
%        Left Foot    & 1.0845&1.0055&1.9748& Right Foot   &0.6997&0.7298&1.8449    \\ \hline
% MAX Percentage Error      & $F_x$(\%)      & $F_y$(\%)      & $F_z$(\%)      &MAX Percentage Error       & $F_x$(\%)      & $F_y$(\%)      & $F_z$(N)        \\ \hline
% Left Foot & 0.3574&0.3331&0.1639& Right Foot&0.2009&0.2995&0.1792 \\ \hline
% \end{tabular}
% \label{RMSMAXDE}
% \end{table}

% \begin{table}[!h]
% \centering
% \caption{RMS Error and Max Percentage Error between Commercial Sensor and Proposed Sensor during Various Speed Experiment}
% \begin{tabular}{|c|c|c|c|c|c|c|c|}
% \hline
% RMSE      & $F_x$(N)      & $F_y$(N)      & $F_z$(N)      &RMSE      & $F_x$(N)      & $F_y$(N)      & $F_z$(N)        \\ \hline
%        Left Foot(0.2m/s)    & 2.0526&1.3997&3.9046& Right Foot(0.2m/s)   &1.7034&1.3007&3.9648    \\ \hline
% MAX Percentage Error      & $F_x$(\%)      & $F_y$(\%)      & $F_z$(\%)      &MAX Percentage Error       & $F_x$(\%)      & $F_y$(\%)      & $F_z$(\%)        \\ \hline
% Left Foot(0.2m/s)  & 1.7015&1.1181&1.1140& Right Foot(0.2m/s) &1.0028&0.8712&0.7613 \\ \hline
% RMSE      & $F_x$(N)      & $F_y$(N)      & $F_z$(N)      &RMSE      & $F_x$(N)      & $F_y$(N)      & $F_z$(N)        \\ \hline
%        Left Foot(0.4m/s)    & 2.2480&1.4836&3.6591& Right Foot(0.4m/s)   &1.3974&1.5370&4.5013    \\ \hline
% MAX Percentage Error      & $F_x$(\%)      & $F_y$(\%)      & $F_z$(\%)      &MAX Percentage Error       & $F_x$(\%)      & $F_y$(\%)      & $F_z$(\%)        \\ \hline
% Left Foot(0.4m/s)  & 1.7192&1.2257&0.9855& Right Foot(0.4m/s) &1.0305&0.8786&0.7790 \\ \hline
% RMSE      & $F_x$(N)      & $F_y$(N)      & $F_z$(N)      &RMSE      & $F_x$(N)      & $F_y$(N)      & $F_z$(N)        \\ \hline
%        Left Foot(0.8m/s)    & 2.7110&1.8895&4.2853& Right Foot(0.8m/s)   &2.1333&2.1510&5.1981    \\ \hline
% MAX Percentage Error      & $F_x$(\%)      & $F_y$(\%)      & $F_z$(\%)      &MAX Percentage Error       & $F_x$(\%)      & $F_y$(\%)      & $F_z$(\%)        \\ \hline
% Left Foot(0.8m/s)  & 1.9716&1.4431&1.1507& Right Foot(0.8m/s) &1.2335&0.8817&0.9066 \\ \hline
% \end{tabular}
% \label{RMSMAXVS}
% \end{table}


% \begin{table}[!h]
% \centering
% \caption{RMS Error and Max Percentage Error between Commercial Sensor and Proposed Sensor during Rough Terrain Experiment}
% \begin{tabular}{|c|c|c|c|c|c|c|c|}
% \hline
% RMSE      & $F_x$(N)      & $F_y$(N)      & $F_z$(N)      &RMSE      & $F_x$(N)      & $F_y$(N)      & $F_z$(N)        \\ \hline
%        Left Foot    & 2.4558&2.2298&4.7692& Right Foot   &2.3660&2.4369&7.1704    \\ \hline
% MAX Percentage Error      & $F_x$(\%)      & $F_y$(\%)      & $F_z$(\%)      &MAX Percentage Error       & $F_x$(\%)      & $F_y$(\%)      & $F_z$(\%)        \\ \hline
% Left Foot  & 1.8688&1.4903&1.3781& Right Foot &1.8518&1.7438&1.3683 \\ \hline

% \end{tabular}
% \label{RMSMAXRT}
% \end{table}

As a result of the experiment, Table~\ref{Durabilitytestpercentage} compares the offsets of the commercial sensor and the proposed sensor for the left and right feet. The "before" values represent the offsets measured prior to the 5000-second experiment, while the "after" values indicate the offsets measured after the experiment. The commercial sensor exhibited a drift of up to 11.8 N in the z-axis. When the differences were expressed as a full-scale percentage error, the results are shown in Table~\ref{Durabilitytestpercentage}. The commercial sensor showed a drift of approximately 3.93\%, whereas the proposed sensor exhibited a maximum drift of 0.375\%. These results demonstrate that the proposed sensor has better durability compared to the commercial sensor, making it more suitable for application in quadruped robots.
\begin{table}[!h]
\centering
\caption{F/T Sensor Offset(Percentage) Before and After Durability Test}
\begin{tabular}{cccc}
\hline
       & $F_x$(\%)      & $F_y$(\%)      & $F_z$(\%)       \\ \hline\hline
Commercial Sensor LF & 2.27 &0.0300&3.93\\ 
Commercial Sensor RF & 0.270&0.670&3.18\\ 
Proposed Sensor LF & 0.375 &0.165&0.116\\ 
Proposed Sensor RF& 0.0381& 0.107&0.0354\\ \hline
\end{tabular}

\label{Durabilitytestpercentage}
\end{table}



\section{Conclusion and Future works}

The objective of this study was to develop a robust 6-axis force-torque sensor for quadruped robots. The sensor was designed to be compact while capable of measuring large forces, with an optimized structure to enhance resolution. A novel approach was proposed using photo-couplers to implement the entire sensing mechanism on a single PCB, resulting in a compact 6-axis force-torque sensor. Additionally, a method was developed to optimize the sensor's positioning and elastomer parameters through precise modeling, improving both numerical accuracy and sensitivity. The proposed sensor was integrated into a quadruped robot for experimental validation.

Compared to commercial sensors, the proposed sensor requires fewer components, is more affordable, and can measure a larger force range. It demonstrated an improvement in resolution by a factor of at least 2, and up to 10 times in some cases. Unlike traditional strain gauge-based sensors, which typically require external signal processing units due to their size limitations, this sensor allows for integrated signal processing within a single device by utilizing photo-couplers.

Additionally, experiments were conducted by applying the proposed sensor to a quadruped robot and comparing it with a commercial sensor. The results demonstrated that the proposed sensor outperformed the commercial sensor in terms of durability. The contactless commercial sensor exhibited a maximum drift of 3.93\%, while the proposed sensor showed only 0.375\%. On average, the commercial sensor displayed an offset of 1.725\%, whereas the proposed sensor had an average offset of 0.1394\%, which is more than 10 times smaller. These results highlight the durability and reliability of the proposed sensor compared to the commercial alternative.

While photo-couplers, as analog sensors, have limitations, such as the potential need for additional analog filters and circuitry to achieve higher resolution, they offer advantages by reducing the number of components compared to traditional strain gauge systems, thereby lowering the likelihood of failure. These attributes make the sensor highly suitable for compact applications, offering significant potential for the robotics industry.









% \section*{Acknowledgments}
% This article 



% {\appendix[Proof of the Zonklar Equations]
% Use $\backslash${\tt{appendix}} if you have a single appendix:
% Do not use $\backslash${\tt{section}} anymore after $\backslash${\tt{appendix}}, only $\backslash${\tt{section*}}.
% If you have multiple appendixes use $\backslash${\tt{appendices}} then use $\backslash${\tt{section}} to start each appendix.
% You must declare a $\backslash${\tt{section}} before using any $\backslash${\tt{subsection}} or using $\backslash${\tt{label}} ($\backslash${\tt{appendices}} by itself
%  starts a section numbered zero.)}



%{\appendices
%\section*{Proof of the First Zonklar Equation}
%Appendix one text goes here.
% You can choose not to have a title for an appendix if you want by leaving the argument blank
%\section*{Proof of the Second Zonklar Equation}
%Appendix two text goes here.}



% \section{References Section}
% You can use a bibliography generated by BibTeX as a .bbl file.
%  BibTeX documentation can be easily obtained at:
%  http://mirror.ctan.org/biblio/bibtex/contrib/doc/
%  The IEEEtran BibTeX style support page is:
%  http://www.michaelshell.org/tex/ieeetran/bibtex/
 
 % argument is your BibTeX string definitions and bibliography database(s)
%\bibliography{IEEEabrv,../bib/paper}
%
% \section{Simple References}
% You can manually copy in the resultant .bbl file and set second argument of $\backslash${\tt{begin}} to the number of references
%  (used to reserve space for the reference number labels box).
% \bibliography{IEEEabrv,../bib/paper}
% \begin{thebibliography}{1}
% \bibliographystyle{IEEEtran}
% \FloatBarrier


% \bibliographystyle{IEEEtran}
% \bibliography{IEEEabrv,force}


\begin{thebibliography}{20}

\bibitem{di2018dynamic}
J.~Di~Carlo, P.~M. Wensing, B.~Katz, G.~Bledt, and S.~Kim, ``Dynamic locomotion in the mit cheetah 3 through convex model-predictive control,'' in \emph{2018 IEEE/RSJ International Conference on Intelligent Robots and Systems (IROS)}, IEEE, 2018, pp. 1--9.

\bibitem{katz2019mini}
B.~Katz, J.~Di~Carlo, and S.~Kim, ``Mini cheetah: A platform for pushing the limits of dynamic quadruped control,'' in \emph{2019 International Conference on Robotics and Automation (ICRA)}, IEEE, 2019, pp. 6295--6301.

\bibitem{hutter2016anymal}
M.~Hutter \emph{et~al.}, ``Anymal-a highly mobile and dynamic quadrupedal robot,'' in \emph{2016 IEEE/RSJ International Conference on Intelligent Robots and Systems (IROS)}, IEEE, 2016, pp. 38--44.

\bibitem{shin2022design}
Y.-H. Shin \emph{et~al.}, ``Design of kaist hound, a quadruped robot platform for fast and efficient locomotion with mixed-integer nonlinear optimization of a gear train,'' in \emph{2022 International Conference on Robotics and Automation (ICRA)}, IEEE, 2022, pp. 6614--6620.

\bibitem{choi2023learning}
S.~Choi \emph{et~al.}, ``Learning quadrupedal locomotion on deformable terrain,'' \emph{Science Robotics}, vol.~8, no.~74, p. eade2256, 2023.

\bibitem{kim2019multi}
J.-H. Kim, ``Multi-axis force-torque sensors for measuring zero-moment point in humanoid robots: A review,'' \emph{IEEE Sensors Journal}, vol.~20, no.~3, pp. 1126--1141, 2019.

\bibitem{kim2017surgical}
U.~Kim \emph{et~al.}, ``A surgical palpation probe with 6-axis force/torque sensing capability for minimally invasive surgery,'' \emph{IEEE Transactions on Industrial Electronics}, vol.~65, no.~3, pp. 2755--2765, 2017.

\bibitem{cho2022msc}
Y.~Cho \emph{et~al.}, ``The msc prosthetic hand: Rapid, powerful, and intuitive,'' \emph{IEEE Robotics and Automation Letters}, vol.~7, no.~2, pp. 3170--3177, 2022.

\bibitem{jeong2018design}
S.~H. Jeong \emph{et~al.}, ``Design of a miniature force sensor based on photointerrupter for robotic hand,'' \emph{Sensors and Actuators A: Physical}, vol. 269, pp. 444--453, 2018.

\bibitem{chen2025design}
F.~Chen \emph{et~al.}, ``Design of a strain gauge-based force sensor with three ranges based on compliant mechanisms,'' \emph{IEEE Transactions on Instrumentation and Measurement}, 2025.

\bibitem{kim2016novel}
U.~Kim \emph{et~al.}, ``A novel six-axis force/torque sensor for robotic applications,'' \emph{IEEE/ASME Transactions on Mechatronics}, vol.~22, no.~3, pp. 1381--1391, 2016.

\bibitem{epstein2020bi}
L.~Epstein \emph{et~al.}, ``Bi-modal hemispherical sensors for dynamic locomotion and manipulation,'' in \emph{2020 IEEE/RSJ International Conference on Intelligent Robots and Systems (IROS)}, IEEE, 2020, pp. 7381--7381.

\bibitem{ananthanarayanan2012compact}
A.~Ananthanarayanan \emph{et~al.}, ``A compact two dof magneto-elastomeric force sensor for a running quadruped,'' in \emph{2012 IEEE International Conference on Robotics and Automation}, IEEE, 2012, pp. 1398--1403.

\bibitem{xiong2020six}
L.~Xiong \emph{et~al.}, ``Six-dimensional force/torque sensor based on fiber bragg gratings with low coupling,'' \emph{IEEE Transactions on Industrial Electronics}, vol.~68, no.~5, pp. 4079--4089, 2020.

\bibitem{kim2024compact}
H.-B. Kim \emph{et~al.}, ``A compact six-axis force/torque sensor using photocouplers for impact robustness,'' \emph{Review of Scientific Instruments}, vol.~95, no.~4, 2024.

\bibitem{kim2023compact}
H.-B. Kim \emph{et~al.}, ``A compact optical six-axis force/torque sensor for legged robots using a polymorphic calibration method,'' \emph{arXiv preprint arXiv:2309.04720}, 2023.

\bibitem{robotous_website}
``Robotous - innovative robotic solutions,'' \url{http://www.robotous.com/main}, accessed: 2025-01-21.

\bibitem{ati_website}
``Ati industrial automation,'' \url{https://www.ati-ia.com/}, accessed: 2025-01-21.

\bibitem{kim2019highly}
D.~Kim \emph{et~al.}, ``Highly dynamic quadruped locomotion via whole-body impulse control and model predictive control,'' \emph{arXiv preprint arXiv:1909.06586}, 2019.

\bibitem{huang2010lcm}
A.~S. Huang \emph{et~al.}, ``LCM: Lightweight communications and marshalling,'' in \emph{2010 IEEE/RSJ International Conference on Intelligent Robots and Systems}, IEEE, 2010, pp. 4057--4062.

\end{thebibliography}



% \bibitem{ref1}
% {\it{Mathematics Into Type}}. American Mathematical Society. [Online]. Available: https://www.ams.org/arc/styleguide/mit-2.pdf

% \bibitem{ref2}
% T. W. Chaundy, P. R. Barrett and C. Batey, {\it{The Printing of Mathematics}}. London, U.K., Oxford Univ. Press, 1954.

% \bibitem{ref3}
% F. Mittelbach and M. Goossens, {\it{The \LaTeX Companion}}, 2nd ed. Boston, MA, USA: Pearson, 2004.

% \bibitem{ref4}
% G. Gr\"atzer, {\it{More Math Into LaTeX}}, New York, NY, USA: Springer, 2007.

% \bibitem{ref5}M. Letourneau and J. W. Sharp, {\it{AMS-StyleGuide-online.pdf,}} American Mathematical Society, Providence, RI, USA, [Online]. Available: http://www.ams.org/arc/styleguide/index.html

% \bibitem{ref6}
% H. Sira-Ramirez, ``On the sliding mode control of nonlinear systems,'' \textit{Syst. Control Lett.}, vol. 19, pp. 303--312, 1992.

% \bibitem{ref7}
% A. Levant, ``Exact differentiation of signals with unbounded higher derivatives,''  in \textit{Proc. 45th IEEE Conf. Decis.
% Control}, San Diego, CA, USA, 2006, pp. 5585--5590. DOI: 10.1109/CDC.2006.377165.

% \bibitem{ref8}
% M. Fliess, C. Join, and H. Sira-Ramirez, ``Non-linear estimation is easy,'' \textit{Int. J. Model., Ident. Control}, vol. 4, no. 1, pp. 12--27, 2008.

% \bibitem{ref9}
% R. Ortega, A. Astolfi, G. Bastin, and H. Rodriguez, ``Stabilization of food-chain systems using a port-controlled Hamiltonian description,'' in \textit{Proc. Amer. Control Conf.}, Chicago, IL, USA,
% 2000, pp. 2245--2249.

% \end{thebibliography}


\newpage

\section{Biography Section}
\vspace{11pt}
% \bf{If you include a photo:}\vspace{-33pt}
\begin{IEEEbiography}[{\includegraphics[width=1in,height=1.25in,clip,keepaspectratio]{KHB_picture.png}}]{Hyun-Bin Kim}
~ received the B.S., M.S. and Ph.D. degrees in mechanical engineering from Korea Advanced Institute of Science and Technology(KAIST), Daejeon, Republic of Korea, in 2020, 2022 and 2025 respectively. He is currently working as the post-doctor researcher in KAIST. His current research interests include force/torque sensors, legged robot control, robot design and mechatronics system.
\end{IEEEbiography}
\vspace{11pt}
\begin{IEEEbiography}[{\includegraphics[width=1in,height=1.25in,clip,keepaspectratio]{HBI_picture.jpg}}]{Byeong-Il Ham}
~ received the B.S. degree in school of robotics from University of Kwangwoon, Seoul, and the M.S. degree in robotics program from Korea Advanced Institute of Science and Technology(KAIST), Daejeon, Republic of Korea, in 2022 and 2024, respectively. He is in Doctor Program in KAIST, Daejeon, Korea, from 2024. His current research interests include legged system, optimal control and motion planning.
\end{IEEEbiography}
\vspace{11pt}
\begin{IEEEbiography}[{\includegraphics[width=1in,height=1.25in,clip,keepaspectratio]{CKH_picture.jpg}}]{Keun-Ha Choi}
~ received the B.S. degree in weapon system engineering from Korea Military Academy, Seoul, and the M.S. degree and the Ph.D. degrees in mechanical engineering from Korea Advanced Institute of Science and Technology(KAIST), Daejeon, Republic of Korea in 2002, 2007 and 2016 respectively. He was a Defense Acquisition Program Administration, a Project Management Officer, Republic of Korea, from 2009 to 2012 and from 2016 to 2019, Army Education and Doctrine Command, an AI Research Officer, from 2019 to 2020, and an Army Headquarters, Force Planning Officer, from 2020 to 2021. In 2021, he was an AI/Big Data Research Officer with the Army Future Innovation Research Center. In 2022, he joined as a Research Assistant Professor with Daedong-KAIST Research Center for Mobility. Since 2023, he has been with the Department of Mechanical Engineering, KAIST. His current research interests include sensor fusion-based robot autonomous navigation algorithm and control, vision sensor-based object detection using AI, and defense AI application plan/military operation concept/concept design.
\end{IEEEbiography}
\vspace{11pt}
\begin{IEEEbiography}[{\includegraphics[width=1in,height=1.25in,clip,keepaspectratio]{KKS_picture.png}}]{Kyung-Soo Kim}~(Fellow, IEEE)~ received the B.S., M.S., and Ph.D. degrees in mechanical engineering from Korea Advanced Institute of Science and Technology (KAIST), Daejeon, Republic of Korea, in 1993, 1995, and 1999, respectively. He was a Chief Researcher with LG Electronics Inc., from 1999 to 2003, and the DVD Group Manager of STMicroelectronics Company Ltd., from 2003 to 2005. In 2005, he joined the Department of Mechanical Engineering, Korea Polytechnic University, Siheung, Republic of Korea, as a Faculty Member. Since 2007, he has been with the Department of Mechanical Engineering, KAIST. His research interests include control theory, electric vehicles, and autonomous vehicles. He serves as an Associate Editor for the Automatica and the Journal of Mechanical Science and Technology.
\end{IEEEbiography}


\end{document}


