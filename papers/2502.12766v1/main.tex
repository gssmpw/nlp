\documentclass[11pt,final]{article}

%Added after EC
\usepackage{xcolor}


%Make sure the order of begin document, title, author, abstract, make title is correct
%%%%%%%%%%%%%%%%


\usepackage{fullpage}
\usepackage{amsmath,amssymb,amsthm}
\usepackage{lmodern}



\usepackage[numbers,sort&compress]{natbib} 
\setcitestyle{numbers}

\usepackage[colorlinks=true]{hyperref}
\hypersetup{
     colorlinks  = true,
     urlcolor    = teal,
	 citecolor   = teal,
	 linkcolor   = red
}




\usepackage[ruled,noend,linesnumbered]{algorithm2e} % For algorithms
\renewcommand{\algorithmcfname}{ALGORITHM}
\SetAlFnt{\small}
\SetAlCapFnt{\small}
\SetAlCapNameFnt{\small}
\SetAlCapHSkip{0pt}
\IncMargin{-\parindent}



\usepackage[utf8]{inputenc} % allow utf-8 input
\usepackage[T1]{fontenc}    % use 8-bit T1 fonts
\usepackage{lmodern}
\usepackage{url}            % simple URL typesetting
\usepackage{amsfonts}       % blackboard math symbols
\usepackage{nicefrac}       % compact symbols for 1/2, etc.
\usepackage{microtype}      % microtypography
\usepackage{booktabs}       % professional-quality tables
%\usepackage[linesnumbered,ruled,noend]{algorithm2e}

%Make sure this appears AFTER ams packages, etc.
\newtheorem{theorem}{Theorem}[section]
\newtheorem{lemma}{Lemma}[section]
\newtheorem{proposition}{Proposition}[section]
\newtheorem{corollary}{Corollary}[section]
\newtheorem{conjecture}{Conjecture}[section]
\newtheorem{definition}{Definition}[section]
\newtheorem{example}{Example}[section]
\newtheorem{remark}{Remark}[section]


%%%%%%%%%%%%%%%%%%%%%%%%%%%%%%%%%%
%%%%%%%%%%%%%%Mine%%%%%%%%%%%%%%%%


\newcommand{\ones}{\mathbf 1}
\newcommand{\reals}{{\mbox{\bf R}}}
\newcommand{\integers}{{\mbox{\bf Z}}}
\newcommand{\symm}{{\mbox{\bf S}}}  % symmetric matrices

\newcommand{\nullspace}{{\mathcal N}}
\newcommand{\range}{{\mathcal R}}
\newcommand{\Rank}{\mathop{\bf Rank}}
%\newcommand{\Tr}{\mathop{\bf Tr}}
\newcommand{\diag}{\mathop{\bf diag}}
\newcommand{\card}{\mathop{\bf card}}
\newcommand{\rank}{\mathop{\bf rank}}
\newcommand{\conv}{\mathop{\bf conv}}
\newcommand{\prox}{\mathbf{prox}}

\newcommand{\Expect}{\mathop{\bf E{}}}
\newcommand{\var}{\mathop{\bf var{}}}
\newcommand{\Prob}{\mathop{\bf Prob}}
\newcommand{\Co}{{\mathop {\bf Co}}} % convex hull
\newcommand{\dist}{\mathop{\bf dist{}}}
%\newcommand{\argmin}{\mathop{\rm argmin}}
%\newcommand{\argmax}{\mathop{\rm argmax}}
\newcommand{\epi}{\mathop{\bf epi}} % epigraph
\newcommand{\Vol}{\mathop{\bf vol}}
\newcommand{\dom}{\mathop{\bf dom}} % domain
\newcommand{\intr}{\mathop{\bf int}}
%\newcommand{\sign}{\mathop{\bf sign}}

\newcommand{\cf}{{\it cf.}}
\newcommand{\eg}{{\it e.g.}}
\newcommand{\ie}{{\it i.e.}}
\newcommand{\etc}{{\it etc.}}

\newcommand{\todo}{{\bf TODO}}

\newcommand{\bone}{\boldsymbol{1}}
\newcommand{\balpha}{\boldsymbol{\alpha}}
\newcommand{\bbeta}{\boldsymbol{\beta}}
\newcommand{\bdelta}{\boldsymbol{\delta}}
\newcommand{\bepsilon}{\boldsymbol{\epsilon}}
\newcommand{\blambda}{\boldsymbol{\lambda}}
\newcommand{\bomega}{\boldsymbol{\omega}}
\newcommand{\bpi}{\boldsymbol{\pi}}
\newcommand{\bnu}{\boldsymbol{\nu}}
\newcommand{\bphi}{\boldsymbol{\phi}}
\newcommand{\bvphi}{\boldsymbol{\varphi}}
\newcommand{\bpsi}{\boldsymbol{\psi}}
\newcommand{\bsigma}{\boldsymbol{\sigma}}
\newcommand{\btheta}{\boldsymbol{\theta}}
\newcommand{\bzeta}{\boldsymbol{\zeta}}
\newcommand{\bxi}{\boldsymbol{\xi}}
\newcommand{\ba}{\boldsymbol{a}}
\newcommand{\bb}{\boldsymbol{b}}
\newcommand{\bc}{\boldsymbol{c}}
\newcommand{\bd}{\boldsymbol{d}}
\newcommand{\be}{\boldsymbol{e}}
\newcommand{\boldf}{\boldsymbol{f}}
\newcommand{\bg}{\boldsymbol{g}}
\newcommand{\bh}{\boldsymbol{h}}
\newcommand{\bi}{\boldsymbol{i}}
\newcommand{\bj}{\boldsymbol{j}}
\newcommand{\bk}{\boldsymbol{k}}
\newcommand{\bell}{\boldsymbol{\ell}}
\newcommand{\bp}{\boldsymbol{p}}
\newcommand{\br}{\boldsymbol{r}}
\newcommand{\bs}{\boldsymbol{s}}
\newcommand{\bt}{\boldsymbol{t}}
\newcommand{\bu}{\boldsymbol{u}}
\newcommand{\bv}{\boldsymbol{v}}
\newcommand{\bw}{\boldsymbol{w}}
\newcommand{\bx}{{\boldsymbol{x}}}
\newcommand{\by}{\boldsymbol{y}}
\newcommand{\bz}{\boldsymbol{z}}
\newcommand{\bA}{\boldsymbol{A}}
\newcommand{\bB}{\boldsymbol{B}}
\newcommand{\bC}{\boldsymbol{C}}
\newcommand{\bD}{\boldsymbol{D}}
\newcommand{\bE}{\boldsymbol{E}}
\newcommand{\bF}{\boldsymbol{F}}
\newcommand{\bG}{\boldsymbol{G}}
\newcommand{\bH}{\boldsymbol{H}}
\newcommand{\bI}{\boldsymbol{I}}
\newcommand{\bJ}{\boldsymbol{J}}
\newcommand{\bL}{\boldsymbol{L}}
\newcommand{\bM}{\boldsymbol{M}}
\newcommand{\bP}{\boldsymbol{P}}
\newcommand{\bQ}{\boldsymbol{Q}}
\newcommand{\bR}{\boldsymbol{R}}
\newcommand{\bS}{\boldsymbol{S}}
\newcommand{\bT}{\boldsymbol{T}}
\newcommand{\bU}{\boldsymbol{U}}
\newcommand{\bV}{\boldsymbol{V}}
\newcommand{\bW}{\boldsymbol{W}}
\newcommand{\bX}{\boldsymbol{X}}
\newcommand{\bY}{\boldsymbol{Y}}
\newcommand{\bZ}{\boldsymbol{Z}}

% new theorems
% \newtheorem{theorem}{Theorem}
%\newtheorem*{proof}{Proof}

% usepackages
\usepackage{amsmath}
\usepackage{amsfonts}
\usepackage{textcomp} % for \textlangle and \textrangle macros
\newcommand{\qdist}[1]{\ifmmode\langle#1\rangle\else\textlangle#1\textrangle\fi}
\usepackage{xcolor}
\usepackage{algorithm} % for algorithms
\usepackage{algpseudocode} % for pseudocode
\usepackage{comment} % for large comments
\usepackage{bbm}
\usepackage{dsfont}
\usepackage{subfigure}
\usepackage{bm}
\usepackage{booktabs} % For better table lines
\usepackage{array} % For better column formatting
%\usepackage{appendix}
%\usepackage[english]{babel}
%\usepackage{amsthm}
\usepackage{graphicx} % for graphs




\Comments=1
\Includeappendix=1
\Putacknowledgement=0

\newcommand{\nonl}{\renewcommand{\nl}{\let\nl\oldnl}}% Remove line number for one line in algorithm2e
\newcommand{\ise}{\textsc{IRSR}}
\newcommand{\ALG}{{\small\textnormal{\textsf{ALG}}}}
\newcommand{\OPT}{{\small\textnormal{\textsf{OPT}}}}
\newcommand{\OGP}{{\small\textnormal{\textsf{OGP}}}}
\newcommand{\SEGB}{{\small\textnormal{\textsf{IREGB}}}}
\newcommand{\ICSEGB}{{\small\textnormal{\textsf{IC-IREGB}}}}
\usepackage{nicefrac}
\usepackage{forest}
\usepackage{amsfonts}
\usepackage{amsthm}
\usepackage{enumitem}
%%%%%%%%%%%%%%%%%%%%%%%%%%%%


% Choose a citation style by commenting/uncommenting the appropriate line:
%\setcitestyle{acmnumeric}


% Title. Note the optional short title for running heads. In the interest of anonymization, please do not include any acknowledgements.
\begin{document}
\title{Efficient Individually Rational Recommender System under Stochastic Order}

% Anonymized submission.


\author{
Gal Bahar%
\thanks{%
    {Technion---Israel Institute of Technology (\url{bahar@campus.technion.ac.il})}}
\and Omer Ben{-}Porat%
\thanks{%
    {Technion---Israel Institute of Technology (\url{omerbp@technion.ac.il})}, corresponding author}
\and Kevin Leyton{-}Brown%
\thanks{%
    {University of British Columbia, Canada (\url{kevinlb@cs.ubc.ca})}}
\and Moshe Tennenholtz%
\thanks{%
    {Technion---Israel Institute of Technology (\url{moshet@ie.technion.ac.il})}}
}

\maketitle
% Abstract. Note that this must come before \maketitle.
\begin{abstract}
With the rise of online applications, recommender systems (RSs) often encounter constraints in balancing exploration and exploitation. Such constraints arise when exploration is carried out by agents whose individual utility should be balanced with overall welfare. Recent work~\cite{Fiduciary} suggests that recommendations should be \emph{individually rational}. Specifically, if agents have a default arm they would use, relying on the RS should yield each agent at least the reward of the default arm, conditioned on the knowledge available to the RS. Under this individual rationality constraint, striking a balance between exploration and exploitation becomes a complex planning problem. We assume a stochastic order of the rewards (e.g., Bernoulli, unit-variance Gaussian, etc.), and derive an approximately optimal algorithm. Our technique is based on an auxiliary Goal Markov Decision Process problem that is of independent interest. Additionally, we present an incentive-compatible version of our algorithm.
\end{abstract}



% Title page for title and abstract only.
%\begin{titlepage}




% ==============================================================================
\section{Introduction}


\acrfull{chis} have become an important tool in modern healthcare and serve a variety of functions. 
%
They can provide consumers with a comprehensive understanding of a disease, specifically addressing general knowledge of health-related issues, effects, ways, and measures to maintain and possibly restore health. 
%
They can also enable early detection, diagnosis, treatment, palliation, rehabilitation, and follow-up care for diseases, along with associated medical decisions, care, and coping strategies for daily life with these diseases~\cite{RN1}. 
%
Typically, \chis\ are provided in a static and linear manner, meaning the same medical content is presented to everyone in the same structure. 
%
However, a linear reading or navigation may not be the best solution for everyone to extract relevant information because patients differ in terms of their prior knowledge, information needs, personal preferences and styles, and health situation, which may depend on factors such as gender, age, personality, and perception~\cite{RN3}. 
%
To address this problem, an \emph{adaptive and interactive visual \chis\ that supports document exploration with adaptive focus and detail views} is needed. 
%
By tailoring content to individual user needs and preferences, such as adjusting the level of detail or focusing on specific topics, this type of system would allow users to personalize their experience.


The primary research goal of this work is to develop novel concepts for advanced, interactive, adaptive, and visual \chis\ (called \apluschis). 
%
We selected \acrfull{ttwodm} as a pilot disease for our research because it is a complex disease with high prevalence and relevance to the public health system. 
%
Also, its progression over time requires changes and adaptations, including tailored knowledge, flexible treatment, new drug classes, improved patient education, sustainable follow-up practices, and screening for complications. Managing \acrshort{ttwodm} is still a difficult and time-consuming task, as it is common, serious, and under-treated. 
%
This is a major challenge for healthcare services, and patients and therapists must be prepared to deal with it effectively. 
%
Consequently, those affected by \acrshort{ttwodm} have an ongoing requirement for timely and pertinent information.~\cite{RN6}.



\begin{figure*}[ht!]
    \centering
    \includegraphics[width=\linewidth]{figures/apchis_library.png}
    \caption{
    The proposed \apluschis\ supports dynamic levels of detail. 
    A \DocumentLibrary\ (top) allows users to select one particular document they want to explore. 
    At document level, an interactive \TableOfContents\ (bottom left) preserves the global linear structure of the document while a chapter's substructure and content is visualized with dedicated visualization techniques for textual and pictorial data.
    }
    \label{fig:concept-dl-and-toc}
\end{figure*}

In this paper, we present a novel visual document exploration system with multi-dimensional adaptivity to help health information consumers better understand medical content by combining close and distant reading approaches.
%
It is targeted towards non-medical users of all adult age groups, which have either \acrshort{ttwodm}, are a relative to a \acrshort{ttwodm} patient or are interested in the disease for another reason. 
%
After a development and evaluation phase the \apluschis\ should  eventually be freely discoverable on the Internet. 
%
We work thus with an unpremeditated userbase in mind, which should be able to use the \acrshort{chis} intuitively, without the need for a supervised introduction. 

To this end, we propose an innovative document exploration system that provides multi-level navigation from high-level (topic overview) to mid-level (keyword occurrence and highlighting) to low-level (full text) views (\figref{fig:concept-dl-and-toc}). 
%
The basic idea is to allow users to efficiently navigate through documents, overview the content, find topics of interest, and finally switch to a close read on specific information contents. 
%
In our system, we make use of well-known document visualization approaches to enhance the learning process of medical content. To visualize high-level structures of a document, we propose dynamic table-of-content which represents sub-chapters by means of a \WordCloud\ containing keywords from a topic-modeling approach. High-level structures and mid-level document information are linked by using Tile Bars as visual navigator. 
%
The visual navigator shows topic occurrences within the underlying document and allows users to quickly explore the content by text snippets.
%
Although these text visualization techniques are not novel themselves, we tie them together in an integrated implementation prototype and adapt them for the specific requirements of the health domain. The system also serves as a platform to test and evaluate approaches for adaptive document and interaction provenance visualization (cf.\ also Section \ref{sec:interaction-analysis}).
%
Our system introduces the notions of  level of detail and adaptive visual presentations for document-based health information exploration for \acrshort{ttwodm}. 
%
Existing \acrshort{chis} are largely static in nature and do not adapt either of these dimensions to their users.


As a consequence, we conducted a user study to characterize the usage behavior of health information seekers adopting our approach. 
%
We show the usability of our system by comparing linear reading with our multi-level approach, and illustrate the results by two provenance visualizations.


This paper presents a comprehensive extension of one of our previous papers \cite{lin2023ivapp} delving further into our developed system and offering additional insights and analyses. 
%
Besides a more verbose and detailed outline of our work, this paper comes with the following additions with regards to its predecessor:
\begin{enumerate}
    \item An extension to our system design with an updated visual representation as well as newly-added components such as a \DocumentLibrary\ for the exploration at documents level and an alternative to the Word Cloud representation (\acrlong{topiccloud});
    \item A through analysis of the interaction data obtained through supervised evaluations with actual users;
    \item Different customized visual analytics tools for analysing and evaluating said interaction data.
\end{enumerate}
%
In the next section (\secref{sec:related-work}) we provide an overview on relevant previous works from the consumer health information domain and reveal the research gap regarding adaptive systems for said domain, before our proposed system design is outlined in \secref{sec:proposed-design}. 
%
\secref{sec:evaluation} describes a formative evaluation that incorporates quantitative and qualitative methods, performed by a number of participants on an implementation prototype. 
%
Following this formative evaluation it was observed that the rich interaction data collected in its course merits the effort of a more detailed analysis and the development of additional visual analytics tool (\secref{sec:interaction-analysis}).
%
On that note, we also investigated the suitability of user interactions for proposing adaptive visualizations. 
%
\secref{sec:discussion} and \ref{sec:conclusion} conclude the paper with a thorough discussion regarding next steps and open research challenges. 

% ==============================================================================
\section{Related Work} \label{sec:related-work}

In this section, we present an overview of important visualization techniques that have inspired our approach and review the need for adaptive and interactive consumer health information systems.

% ------------------------------------------------------------------------------
\subsection{Visualization Techniques for Text Documents and Health Care}
Visualizing large text corpora is a challenging task. 
%
Usually, the involved data sets are inherently complex, containing structural and content-related information. 
%
Most linguistic and text visualization approaches rely on text-mining techniques to reveal semantic information from raw text data. 
%
Therefore, simple statistical processing (\eg\ word frequency and bag-of-words concept) as well as natural language processing approaches (\eg\ named-entity recognition, relationship extraction and sentiment analysis) may be used~\cite{10.1002/widm.1071, 7156366}.


A widely-used visualization technique for text data is the Word Cloud representation (also known as Tag Cloud) which presents an overview of the most frequent or important words by using different type or font sizes~\cite{6758829}.
%
This technique is also known as distant-reading technique~\cite{moretti2005graphs} and allows users to approach literature in a new way.
%
Instead of reading texts in the traditional way, i.e., linear or close-reading, the focus of distant-reading approaches is to count, graph, and map textual data by a visual representation~\cite{janicke2015close}.
%
In recent years, much research has been conducted on distant-reading and Word Cloud visualizations. 
%
For instance, Kim et al.~\cite{5718617} proposed WordBridge, which utilizes graph-based visualization techniques to connect multiple Word Clouds with information-rich edges. 
%
Further extensions of Word Cloud exist that focus on semantic contour lines~\cite{2011.01923.x} and images~\cite{doi:10.1177}. 
%
In our work, we rely on traditional Word Clouds to foster distant-reading within single documents.


For the exploration of larger document collections, additional document features such as metadata information and co-authorships, could be considered to gain a better understanding of the contents of those documents~\cite{6392833, 7583708}. 
%
Another interesting approach by Strobelt et al.~\cite{5290723} called Document Cards, utilizes a mixture of images and important keywords to visualize key semantics of a document. 
%
To visualize distributional properties within a document, Tile Bars~\cite{hearst1995tilebars,keim2007literature} could be considered, which is a compact pixel-based visualization technique that reveals the relative length of a document and the relative frequency of one or more query terms. 
%
In our work, we utilize Tile Bars to represent the relative frequency and distribution of terms from a Word Cloud.



Data visualizations are becoming increasingly important for various fields of application, as well as in healthcare. 
%
Visual representations may help patients as well as physicians to gain a better understanding of health records, \eg\ information on medical diagnostics, treatments, and health histories~\cite{HCI-039}. 
%
For example, the LifeLine system was among the first exploration systems that supports visual patient treatment histories~\cite{10.1145/286498.286513}. 
%
An extensive survey about visualization techniques for electronic health records and population health records are given in~\cite{DBLP:journals/cgf/WangL22}.


Recently, many of the mentioned document visualization techniques are also applied in a medical context. 
%
For instance, Facetatlas by Cao et al.~\cite{5613456} used linked Word Clouds to visualize entity-relational text document of diseases such as Type 1 and Type 2 Diabetes
Mellitus. 
%
The linked Word Clouds are used to represent global relations by using a density map and local relations by using edge bundling techniques. 
%
Another interesting multifaceted text visualization is SolarMap~\cite{6137214} which combines a labeled contour-based cluster visualization with a radially-oriented word cloud.
%
Furthermore, SolarMap can visualize topic distribution of entities from one facet together with keyword distributions that convey the semantic definition along a secondary facet.


With the advent of novel visualization techniques in different domains, visualization literacy, i.e., user understanding and discovery of visual patterns, is becoming increasingly important. 
%
Developing visual literacy is essential to support cognition and evolve toward a more informed society~\cite{doi:10.1177/14738716221081831}. 
%
In our work, we intend to increase visual and health literacy by gathering user information during exploration and providing adapted health information based on that.



% ------------------------------------------------------------------------------
\subsection{Need for Adaptive \chis} \label{subsec:chis}

As part of this work, we examined current sources of \chis\ related to \acrshort{ttwodm} across multiple media platforms, including websites, digital documents (PDFs), print media, apps, and videos. Our goal was to identify elements and modes of presentation within a representative sample of these sources that users can customize to their needs and preferences.
%
Our results suggest that the potential for adaptation in existing CHIS is only realized to a limited extent.
%
We did not find any adaptive elements in print media or digital documents (PDF) while websites, apps, and videos offer some customization options related to presentation format, such as adjusting font size and color. 
%
Some \chis\ also included features such as text-to-speech or language-switching~\cite{RN7, RN8}.
%
However, in terms of personalized medical information, only a few \chis\ had mechanisms to pre-filter medical content based on a user's diabetes profile~\cite{RN9}.
%
Most \chis\ included a standard table of contents, with or without hyperlinks to the respective chapters. Some sources also contained links within the text or cross-references to other sections or chapters. 
%
However, none of the \acrshort{ttwodm} \chis\ we analyzed used a visual document exploration system with multi-dimensional adaptivity for health information consumers.



These results show that existing \chis\ on \acrshort{ttwodm} fall short of the potential of presenting health information in an \emph{interactive}, \emph{adaptive} and/or \emph{personalized} way, while there is evidently a need for it. 
%
The knowledge domain of \acrshort{ttwodm} is complex and comprehensive, with a wide range of information sources (brochures, websites, medical doctors, etc.) and high diversity of topics (such as symptoms, treatments, nutrition, etc.). 
%
This might be overwhelming for laypersons without medical expertise seeking knowledge in the field. Such complex information situations usually put a high \emph{intrinsic cognitive load}~\cite{sweller2005implications} on the working memory during information processing and often lead to information seekers applying heuristics and cognitive biases at every stage of information processing. 
%
Such cognitive biases, misconceptions, and even myths about \acrshort{ttwodm} may lead to unhealthy behavior with severe health-related consequences. 


An \emph{interactive} \chis\ has the potential to (i) track behavioral patterns and explicit feedback of consumers, (ii) interpret these indicators in terms of certain cognitive biases (\eg\ the confirmation bias), and (iii) intervene if necessary (\eg\ by suggesting other pieces or sources of information). 

An \emph{adaptive} \chis\ can match the information units to the users and their current information needs. 
%
It can thus balance the \emph{intrinsic cognitive load} to a medium level and ensure that the consumer is neither too bored nor too overwhelmed. 
%
This is in line with the transfer of Vygotsky’s concept of the \emph{zone of proximal development} to digital learning environments~\cite{luckin2010re}, the outer fringe as suggested by the competence-based knowledge space theory~\cite{heller2006competence} and constitutes a solid basis for an enjoyable flow for consumers~\cite{schiefele2011skills}. 
%
All these theories emphasize that a medium difficulty of information units lead to a successful processing outcome. The intermediate goal is to successively reach more advanced levels of learning outcomes as suggested by Krathwohl~\cite{krathwohl2002revision} which means not simply remembering information, but also applying and evaluating it.


A \emph{personalized} \chis\ can foster a consumers’ personal commitment to engage with the system and information, and thus help to close the `intention-behaviour gap’ or `attitude-action gap' \cite{schwarzer2008modeling}. 
%
This is the ultimate goal of any \chis. 
%
We strive to achieve this through different added values of our advanced \chis\ compared to more `traditional’ digital \chis\ (\eg, a brochure in PDF format or plain webpage): the guarantee of high quality and evidence-based medical information, the reduction of complexity to a medium level, and the recommendation of information units that fit a consumers’ information needs. 
%
In addition, tools and functionalities to get an overview of the knowledge domain, to efficiently answer certain questions, and to easily navigate through different sub-topics will be offered for good user experience. With our formative evaluation activities (\secref{sec:evaluation}) we monitor progress towards these goals.






% ==============================================================================
\section{Adaptive Document Exploration Design} \label{sec:proposed-design}
%
In the following \secref{sec:design-requirements} we discuss the design considerations, taking into account the intended userbase and use case scenarios. 
%
Informed by this, the resulting system design is outlined in detail in \secref{sec:implementation}.


% ------------------------------------------------------------------------------
\subsection{Design Requirements/Considerations}\label{sec:design-requirements}
In view of our intended users, who consists of all adult age groups, genders, and digital proficiency we decided to rely on document visualization techniques that are intuitive to understand and easy to use. 
%
Although the participants of the formative evaluation study which was conducted within the scope of this project (\secref{sec:evaluation}) received a brief introduction into visual components of the system, we want to support the scenario that an uninformed user, which discovers the \acrshort{chis} on the Internet is able to use it without any prior supervised introduction. 
%
That is, visualizations should base upon tried and tested concepts, which have been shown to be usable by the vast majority of people. 
%
Yet, at the same time, more advanced users should have to option to alternatively work with more complex visualizations, which allow for a more efficient exploration at the cost of a steeper learning curve. 


\begin{figure*}[ht!]
    \centering
    \includegraphics[width=\textwidth]{figures/system_new.png}
    \caption{
    The main components of \apluschis\ shown by an example of exploring a German diabetes health brochure \cite{aok}:
    \acrfull{toc}, \acrfull{wc}/\acrfull{hwc}, \acrfull{is}, \acrfull{tileb}, \acrfull{topicb}, \acrfull{snps}, and \acrfull{fulltext}.
    Different actions (illustrated as blue arrows) allow a user to navigate from one view to another.}    
    \label{fig:exploration-mockup}
\end{figure*}

% ------------------------------------------------------------------------------
\subsection{Implementation}\label{sec:implementation}
Informed by the considerations above we designed a system comprised of several components, which support the exploration of documents at different levels of visual granularity. 
%
Starting from a \acrfull{dl} showing an overview of the documents supported by the \apluschis, a user is able to explore individual documents on both high and low levels (\figref{fig:exploration-mockup}).
%
For the prior, the system provides an expandable and interactive \acrfull{toc} while the latter is enabled though a series of text abstraction methods such as \acrfull{wc}, fingerprinting in the form of a \acrfull{tileb}, and topic modeling by means of a \acrfull{topicb}.
%
Pictorial content is provided in an \acrfull{is} component.
%
On the lowest level, a user can also review sections of the original text sources in the form of \acrfull{snps} or even the untampered full text. 
%
These different concepts are implemented in different subsystems such that an unintermitted exploration process is possible while alternating between levels.
%
In the background, we track user interactions to determine which parts of the content have already been visited and consumed by the user. 
%
This information is also displayed to the user in order to indicate which information has not yet been scrutinized.



\paragraph*{\DocumentLibrary}
%
The \dl\ is the first view a user is greeted with upon logging into the \apluschis.
%
All available documents are presented in a grid arrangement with their respective covers and titles (\figref{fig:concept-dl-and-toc}).
%
Upon hovering over a certain document a preview appears, showing both related metadata and a histogram of the document's most frequent terms. 
%
This initial view should serve the users in determining which of the documents is the most appropriate for addressing their respective information need.
%
Clicking on a document transfers the user to its \toc.


\paragraph*{\TableOfContents}
The exploration process on document level is supported by an interactive \toc.
%
We base this view on a document's inherent linear structure with chapters, sub-chapters, and so on. 
%
While we aim to preserve the outermost structure (`chapters' in most cases) we abstract all lower levels of structure and content with dedicated text- and multimedia visualizations (\figref{fig:concept-dl-and-toc}). 
%
To this end, we employ \wc s -- and most recently also an alternative representation (\secref{sec:future-work}) -- for the textual content, \is s for the pictorial content, and further subsystems (\tib\ and \snps) for structure and lower levels of visual granularity.
%
These dedicated visualizations are interweaved into the linear chapter structure, based loosely on the \emph{document card} design concept by Strobelt et al.~\cite{5290723} -- i.e., different visualization techniques are used to display the textual and visual contents.
%
On a per-chapter basis, these visualizations can be expanded or collapsed. 

Additionally, already `consumed' content is tracked and indicated with a `history' version of the respective \wc s and \is s. 
%
Specifically, terms and images are added to these components after they have been reviewed (i.e., clicked on) by the user. 
%
This \acrfull{hwc} visualizes the context of the exploration to the user. 
%
Alternatively, it can   be used to display non-clicked terms as to suggest content to the user.



\paragraph*{\WordCloud\ with \Topicbar}
To generate the word clouds, natural language processing is used to extract `significant' terms from each chapter. 
%
This pre-processing comprises steps such as tokenization (separation/segmentation into individual parts) and stop-word removal (filtering of irrelevant/insignificant words).
%
Subsequently, the set of remaining words are subjected to a lemmatization (transformation into their canonical form or dictionary form) and grammatically tagged (part-of-speech tagging). 
%
However, we do not use the resulting normalized words directly to fill the \wc, but instead subject them to the Latent Dirichlet Allocation~\cite{blei2003latent} in order to obtain meaningful topic models on them. 
%
The appropriate number of topic models per-chapter (5) was determined empirically by visually evaluating the resulting topics for different values in the low integer range.
%
That is, each topic model is defined by a weighted vector containing a subset of the chapter's extracted nouns. 

The concated vectors of all topics serve as the input for a chapter's \wc, where the weights are used to determine a word's size within the cloud.
%
For the arrangement of terms, we rely on the \emph{Wordle word cloud} algorithm \cite{steele2010beautiful} while the mapping between words and their belonging topic is established through a qualitative color mapping (\figref{fig:exploration-mockup}, \wc). 
%
Note that this means certain terms could appear redundantly as topics can exhibit overlapping term compositions.



Since the concurrent display of all topics can be overwhelming, we provide a means to toggle the visibility of individual topics. 
%
This subsystem -- the \tob\ directly above the \wc\ -- consists of 5 colorized toggle buttons mapping to the respective topic models.
%
Hovering over a term toggles the \Tilebar\ component for the respective term above it, while clicking it initiates the \snps\ view. 
%
To this end, we track the interactions -- how often the user has clicked a certain term -- in the \wc.
%
These click counts are the basis for the so-called \acrfull{hwc} on the right-hand side of a chapter visualization (\figref{fig:exploration-mockup}, \hwc) where the count determines the size of a term in the cloud.
%
The same hover-and-click interactions as with the `regular' word cloud are possible with the \hwc.



\paragraph*{\Tilebar}

Hovering over a term in the \wc\ triggers the display of the \tib\ component above it, which allows the user to efficiently grasp the term's occurrences over the whole document. 
%
This visualization is inspired by the \emph{literature fingerprinting} concept by Keim and Oelke~\cite{keim2007literature} which shows various document properties in a drilled-down manner. 
%
To this end, we compute the respective term's frequency over equal-sized text chunks and visualize the resulting 'intensities' in a colorized fashion, following the linear document structure (i.e., from top to bottom and left to right, see \figref{fig:exploration-mockup}, \tib). 
%
That is, the rows of the grid symbolize the document's chapters and the columns the ordered text chunks within a chapter.
%
Cells which stand for text chunks in which the term does not appear at all are filled with uniform gray color.
%
The \tib\ allows a user to quickly answer such questions as \emph{``does another chapter also cover this topic?''} or \emph{``how frequently is it mentioned overall?''}.



\paragraph*{\Snippets /\Fulltext}
%
The above mentioned abstractions are vital to gain an overview of the information covered by a document and determine which sections are the most appropriate to answer a specific information need. 
%
Ultimately however, it is necessary to provide a user with text chunks to enable them to answer their specific information need. 
%
To address this issue, we added two additional levels of visual granularity.
%
Firstly, a \emph{\Snippets} view which pops up to the right hand-side of \toc\ if a term in the \wc\ or \hwc\ is clicked.
%
Within this view, all sentences containing the clicked term are displayed with a highlighting of the term (\figref{fig:exploration-mockup}, \snps). 
%
Handles at the beginning and the end of a sentence allow to reveal the preceding and succeeding sentence. 
%
Those can be clicked iteratively to  display larger parts of the document before and after the found position. Alternatively, the section headers, which are also shown in the snippets view, can be clicked to display a section's whole content immediately (\figref{fig:exploration-mockup}, \fullt).
%
The top of the \snps\ view also contains a \Searchbar, which allows to readily change the term in question.


\paragraph*{\ImageSlider}
%
Besides the abstractions for textual content (\wc/\hwc), we would like to indicate the presence of a document's pictorial content. 
%
To this end, we employed an off-the-shelf \is\ component, next to the \wc\ to display a chapter's images (\figref{fig:exploration-mockup}, \is). 
%
The thumbnails of the images can be viewed directly in this slider, yet if further details are needed, an image can be clicked which shows it together with its caption in a full screen modal dialog (henceforth referred to as \isS\ and \isL\ respectively). 
%
As this component shows just five images at a time, we aim to determine an image's relevance in order to sort the list of images, resulting in the most relevant images being shown initially.
%
To determine said relevance, we make two assumptions. 
%
Firstly, we assume that images without a caption (\eg\ scenic backgrounds at the beginning of a chapter) are rather unimportant. 
%
Secondly, we split the set of images with captions into two tiers with the first tier being comprised of images showing tables, diagrams, and flow charts, or convey any sort of structured information, while all the others belong to the second tier.
%
The information whether or not an image has a caption is obtained in the extraction process. 
%
In a similar fashion to the provenance version of the \wc\ -- the \hwc\ -- we provide an additional image slider in the bottom of the \hwc, showing exclusively the chapter's images which have already been clicked (and thus `consumed') by the user.




% ==============================================================================
\section{Formative Evaluation}\label{sec:evaluation}

The formative study aimed to 
(a) explore how the design of the system and its components would be perceived by users, 
(b) evaluate the system in comparison to a linear and static \acrshort{chis}, 
(c) highlight potential areas for improvement, and 
(d) identify prospective research questions.
%
To maintain methodological consistency and participant comparability, we focused on one particular document for the evaluation setup. 
%
That is, we used a \acrshort{ttwodm} information brochure of the German health insurance provider AOK~\cite{aok} as data basis for evaluation. 
%
The text document in PDF format comprises more than $130$ pages of comprehensive and detailed health information, including figures, tables, and info-graphics. 
%
Full texts with health information were extracted with Adobe PDFBox\footnote{\url{https://pdfbox.apache.org/}} library and the images were extracted manually. 
%
Subsequently, the sub-chapters and images were sorted by the main chapters. 


% ------------------------------------------------------------------------------
\subsection{Participants}
%
Overall, 12 participants (four females) took part in the study, representing the different potential users of \apluschis\ with ages between 26 and 62 years ($M = 40 \unit{yrs.}$, $SD = 14 \unit{yrs.}$). 
%
Additionally, the participants had different levels of knowledge and competence. On a 5-point rating scale (from $0 := $ very low to $4 := $ very high) they self-assessed their prior knowledge of \acrshort{ttwodm} ($M = 1.00$, $SD = 1.21$), computer and software skills ($M = 2.25$, $SD = .97$), as well as previous experiences with visualizations ($M = 2.58$, $SD = 1.24$).

% ------------------------------------------------------------------------------
\subsection{Procedure} \label{sec:procedure}
 The overall procedure for the participants can be divided into the following phases: 
 (i) \emph{Instruction}, 
 (ii) \emph{Cognitive Walkthrough}, 
 (iii) \emph{Forced Choice} and finally, 
 (iv) \emph{Semi-Structured Interviews.} 
 %
 The phases (ii) to (iv) are described in more detail in according subsections below. 
 %
 The sessions that lasted between 60 and 90 minutes per participant were carried out individually, lead by one investigator. 
 %
 Three of the authors took the role of an investigator, each for four participants. 
 %
 In phase (i), i.e. \emph{Instruction}, participants received a short explanation of basic \chis\ functions, such as search functions, to ensure that they all began the evaluation process from a common usage knowledge base. 
 %
 The participants were then set in a real-world usage scenario: they were asked to imagine that they themselves or someone in their family was diagnosed with \acrshort{ttwodm} during a health check-up. 
 %
 That is why they want to find out more about this disease.
 
 

The above-mentioned AOK brochure~\cite{aok} was presented in \apluschis, where participants started the first task with the \toc\ view of the brochure, as well as in a PDF viewer (Adobe Acrobat Reader). 
%
The comparison with the PDF viewer was chosen not only because this is the original, static format of the brochure, but also because PDF viewers are widely used and therefore people are usually accustomed to them. 
%
The PDF viewer is therefore a challenging benchmark that allowed us to evaluate the expected added value of \apluschis\ compared to conventional \chis. 
%
The investigators recorded all audio and on-screen activities and additionally made manual notes of their observations. 



\paragraph*{\acrlong{cwt}}
To encourage interaction with \apluschis, showing how intuitive its functions are and how quickly the content can be grasped, participants were given pre-defined tasks to explore \apluschis. 
%
This evaluation method is known as a \acrfull{cwt}~\cite{hollingsed2007usability}.  
%
The pre-defined tasks also enabled comparable conditions across all participants for the subsequent evaluation steps. 
%
We defined them in such a way that they represent realistic search and evaluation tasks in the course of an information search. 
%
For the purpose of the study, the tasks also had to be linked to a measurable goal achievement, for example, the participants had to find and report a specific piece of information. 
%
It had to be possible to achieve this goal in both PDF viewer and \apluschis. 
%
For each tool, approximately the same number of tasks were designed, including tasks from the two categories of `generating an overview' vs. `finding specific information'. 
%
Initially, we defined 11 tasks: four were aimed at using the \WordCloud\ including \Topicbar\ (\taskWcOne--4),
 four at using the \Tilebar\ (\taskTibOne--4), and three at using the \ImageSlider\ (\taskIsOne--3). 
%
Since we intended to compare the PDF viewer with \apluschis, two parallel versions were created for the 11 tasks that were comparable in terms of difficulty and functions used. 
%
For example, the task \taskTibFour\ `In which chapter would you most likely start if you wanted to find out more about blood pressure?' had the parallel version `In which chapter would you most likely start if you wanted to find out more about insulin?'. 
%
This resulted in a total of 22 tasks (\tableref{tab:CWT}) that each participant processed in a within-subject design via the PDF viewer and \apluschis. 
%
We balanced task order in terms of system and components to avoid sequential effects. 
%
In addition to the \cwt, participants were asked to express their thoughts during the tasks (i.e., \emph{think-aloud}). 
%
The investigators also noted behavioral observations during the \cwt\ tasks to complement on-screen and think-aloud activities.

\bgroup
\newcommand{\theader}[1]{\multicolumn{1}{c}{\textbf{#1}}}
\begin{table*}[ht!]
    \centering
    \caption{The \acrshort{cwt} Tasks.}\label{tab:CWT}
    \begin{tabularx}{\textwidth}{l X X}
        \hline
        \theader{Task} & \theader{Parallel Version 1} & \theader{Parallel Version 2} \\
        \hline\hline
        \taskWcOne & Which contents/subject areas do you think are included in chapter 5? & Which contents/subject areas do you think are included in chapter 3?  \\
        \taskWcTwo & What is the   waist size that creates a greatly increased risk for men? & What daily amounts of beer/alcohol are just acceptable for men? \\
        \taskWcThree & How does type I diabetes mellitus develop? & How does type II diabetes mellitus develop?  
        \\
        \taskWcFour & What contents/terms have you searched for so far? & What contents/terms have you searched for so far?  
        \\
        \taskTibOne & How often does the term `smoking' appear in chapter 1? & How often does the term `stress' appear in chapter 1?  
        \\
        \taskTibTwo & Is it worth reading beyond chapter 2 if you want to learn exclusively about `diastole'? & Is it worth reading beyond chapter 4 if you want to learn exclusively about `medication'?  
        \\
        \taskTibThree & Does chapter 2 give a better insight into the topic of `care' than chapter 6? & Does chapter 6 give a better insight into the topic of `(diabetic) foot' or `foot syndrome' than chapter 3?
        \\
        \taskTibFour & In which chapter would you most likely start if you wanted to find out more about `blood pressure'? & In which chapter would you most likely start if you wanted to find out more about `insulin'? 
        \\
        \taskIsOne & According to an illustration in chapter 5, is a sugar value of 100 mg/dl alarming or safe? & Which 3 stages of diabetes therapy are shown graphically in chapter 4? 
        \\
        \taskIsTwo & Which picture in chapter 6 do you think conveys the most relevant information about type II diabetes mellitus? & Which picture in chapter 6 do you think conveys the least relevant information about type II diabetes mellitus?  
        \\
        \taskIsThree & Search for a graphic on the topic of 'sugar metabolism'. & Search for a graphic on the topic of 'nutrition pyramid'.  
        \\
        \hline
    \end{tabularx}
\end{table*}
\egroup


\paragraph*{Forced Choice} 
Following the \cwt, participants were asked to choose between the PDF viewer and \apluschis\ regarding performance goals of system use. 
%
This evaluation method is known as \emph{forced choice}. 
%
The following nine performance goals were evaluated: `Would you rather use Adobe Acrobat Reader or the \apluschis\ to 
(a) get an overview of the domain, 
(b) develop a general understanding on \acrshort{ttwodm}, 
(c) search for specific keywords, 
(d) capture the main content, 
(e) search for specific images, 
(f) get an overview of the most informative images, 
(g) efficiently navigate through different topics of the content, 
(h) get answers to questions you might have in mind, and finally, 
(i) trace past searches'. 
%
We decided to choose a \emph{forced choice} rather than a questionnaire format with Likert scales for two main reasons: first, as a formative and explorative study we aimed for an explicit comparison between \apluschis\ and a challenging benchmark with regards to the nine above mentioned performance goals. 
%
Deciding between two alternatives might be easier in particular for the inexperienced participants of our sample. 
%
Second, considering the sample size which seems reasonable for a first formative evaluation but rather small for a summative one, means and standard deviations which could be derived by Likert scales would not allow for inferential statistical methods.




\paragraph*{Semi-Structured Interviews} 
Lastly, we conducted \emph{semi-structured interviews} with participants to ask open and generic as well as closed and specific questions about \apluschis\ as well as further inquiries on the comparison between the two systems. 
%
The open questions included for example: `Which system would you rather recommend to a person as a first source of information to get started?' or about the non-linear content exploration in \apluschis\ such as ‘How much does this interactive system encourage you to explore further content?’. 
%
More specific and closed (yes-no) questions were if the participants had already seen or used the different components of \apluschis\ (such as the \WordCloud\ or \ImageSlider, etc.) prior to the session, if they considered them as helpful and if they would like to use them again.



% ------------------------------------------------------------------------------
\subsection{Results and Discussion}
In the following section, we outline and discuss the results, starting from a more global evaluation and continuing with more specific results with respect to the components of \apluschis. 


\paragraph*{Global Evaluation} 
As outlined in the previous section, in the course of the semi-structured interviews, participants were asked closed questions if they had already seen or used a 
(i) Word Cloud, 
(ii) Topic Bar, 
(iii) Tile Bar, or 
(iv) Image Slider prior to the session. 
%
In addition, they were asked if the components were considered as helpful and if they would like to use them again. 
%
A `yes-answer' has been coded as `1', a `no-answer' as `0', and indifference as `0.5'.
%
\figref{fig:seen_used_hepful_useagain} shows the results of this global evaluation: while \WordCloud\ and \ImageSlider\ were well known by participants, as around 75\% had seen it before, no participant had come across a \Tilebar\ and only one has come across a \Topicbar\ prior to the session. 
%
The \ImageSlider\ had been used most often before (ca. 63\%). 
%
Interestingly, the \WordCloud\ had rarely been used despite being well known. 
%
Several participants noted that they were familiar with a \WordCloud\ as an overview graphic, but not as an interactive element. 
%
Regarding helpfulness and future use, more than half of the participants found the \WordCloud, the \Tilebar, and the \ImageSlider\ helpful and would consider using them again. 
%
However, the \Topicbar\ in its current form was not perceived as very helpful, and only 25\% would consider using it again.

\begin{figure}
    \centering
    \includegraphics[width=\mediumwidth\linewidth]{figures/seen_used_helpful_useagain_v0.2.png}
    \caption{
    The evaluation regarding the prior knowledge and usefulness of \apluschis\ components by the \cwt\ participants.
    }
    \label{fig:seen_used_hepful_useagain}
\end{figure}



\paragraph*{Evaluation of \apluschis\ Components} 
The \cwt\ tasks provided information on how the participants used the different components. 
%
We observed mixed results regarding the intuitiveness of searching with the \WordCloud. 
%
Some participants were not hesitant to use the \WordCloud\ and performed efficient searches for keywords and contents, as one participant explained: \emph{‘I simply looked at the largest terms’ (\PTwo)}. 
%
Others preferred the use of more familiar features, such as the \TableOfContents\ or the \Searchbar. 
%
However, the \WordCloud\ seemed to be a tool that was easy to learn. We observed that participants often started to use the \WordCloud\ when other features were not perceived as helpful due to the nature of the task or because the traditional search was simply inefficient. 


Similarly, we observed that the \Tilebar\ component was mostly intuitive and easy to learn. Several participants were completely unfamiliar with the \Tilebar\, but most were able to quickly understand its function and extract information from it, as this quote demonstrates: \emph{‘There is a box [tile] in chapter 2 and then no more boxes in the chapters after that, so it [the searched term] is not mentioned anymore’ (\PSix)}. 
%
Overall, the use of \Tilebar\ allowed the participants to efficiently search for the desired information. 
%
With regard to the design, the participants identified a need for optimization; for example, more contrasting colors for the tiles and a clearer labeling of the chapters within the \Tilebar\ would promote more intuitive use.



Finally, the \ImageSlider\ was perceived as rather intuitive and efficient for finding images because a number of participants already knew it from other applications and the operation was thus familiar. 
%
However, using the search bar was often preferred over browsing the \ImageSlider, particularly because the latter was time-consuming in chapters with many images, as this participant explained: \emph{‘I don't want to click through all of them; they are unnecessary and very small. 
%
They don't contain any information; they are just photos without text’ (\PSeven)}. 
%
This quote also highlights the desired reduction of unnecessary images as well as the wish for more context for those images that were deemed not self-explanatory, as expressed by several participants.



\paragraph*{Performance Goals}
%
Besides describing the sample as a whole, as part of our explorative research, we compared the results of several subgroups: 
(i) female vs. male participants, and - based on a Median-split - 
(ii) `younger' vs. `older' participants, `higher' vs. `lower' levels of self-assessed 
(iii) prior knowledge of \acrshort{ttwodm}, 
(iv) computer and software skills, and 
(v) experiences with visualizations. 
%
Only the comparison between the two age-groups shows some differences at face level. 
%
All other pairs of `sub-groups' are rather similar in their decisions with regards to the \emph{forced choice} items and thus, we do not report the according results.
%
\figref{fig:Forced Choice} illustrates the findings of the forced choices participants had to make between the PDF viewer and \apluschis\ regarding defined performance goals. 
%
If participants chose \apluschis\ as their information source of choice for a performance goal, it received a value of `+1'. 
%
Conversely, `-1' was the value for their choice of the PDF viewer and `0' for an indiscriminate choice. 
%
Thus, the ordinate ranges from `-12' (all participants chose the PDF viewer) to `12' (all participants chose \apluschis). 
%
In \figref{fig:Forced Choice}, the values for the whole sample are reported with green circles, while grey circles represent only the older participants ($n = 6$, $\geq 38 \unit{yrs.}$) and orange circles only the younger ones ($n = 6$, $< 38 \unit{yrs.}$).



The results show the potential of \apluschis\ with regard to the fulfillment of the performance goals. 
%
The values for all nine performance goals are either close to the horizontal `middle-line' (indicating that participants are equally inclined towards \apluschis\ and PDF viewer and/or are indifferent) or clearly above, such as for \emph{efficiently navigating} through different topics of the content, getting an \emph{overview} of the most \emph{important images}, or \emph{tracing searches}. 
%
We could also observe differences between the two age groups; however, this should not be over-interpreted due to the small sample size but should be given consideration in future research. 
%
Overall, \apluschis\ is evaluated as a good tool for information searches compared to the PDF viewer, despite participants' familiarity with the latter.

\begin{figure}[ht!]
    \centering
    \includegraphics[width=\mediumwidth\linewidth]{figures/forced-choice-results.pdf}
    \caption{
    The results of the forced choice evaluation between \apluschis\ and the PDF viewer w.r.t. the defined performance goals.
    The performance goals in information processing are sorted along a continuum from more abstract (left) to more specific goals (right). 
    }
    \label{fig:Forced Choice}
\end{figure}



\paragraph*{Non-Linear Content Exploration} 
One area of particular interest was to find out how non-linear content exploration was received by participants. 
%
Think aloud during the \cwt\ tasks and the subsequent interviews showed that the system arouses curiosity and motivates further exploration of system features and contents. 
%
The participants perceived the search as enjoyable and found the aesthetic design with colors appealing. 
%
This applies in particular to the \WordCloud, which manages to cultivate curiosity about further topics, as one participant explained: \emph{‘[...] I can imagine that if there is such a \WordCloud\, I would at least take a look at what topics there are, whether I missed something that would interest me. And I would rather do that than browse through the brochure [...]’ (\PTen)}. 
%
It is an added value compared to a non-interactive system, that the \WordCloud\ fosters engagement with the content further via making interesting and frequent terms more visible. 



However, some participants felt that they could not explore the content effectively because they were unsure about how to use the system due to its novelty. 
%
In particular, the lack of a familiar linear structure was found confusing and made it difficult for some participants to keep track of the content. 
%
As one participant noted: \emph{‘[...] I’d like to know what’s in it. 
%
I don’t know what to expect, what the tool offers me.
%
It doesn't give an impression of what it actually is now. 
%
At least not a quick one. 
%
I don’t have that much patience for it [...] (\PSeven)’}. 
%
It seems that the open structure overwhelmed some participants, leading to a sense of inadequacy to complete the information search. 
%
More support through assistance in the \apluschis\ system could help users overcome these initial uncertainties so that they can capitalize on the strengths mentioned above and utilize the advantages to the fullest.



% ==============================================================================
\section{Interaction Analysis} \label{sec:interaction-analysis}
Our data collection efforts, including the on-screen and audio recordings as well as the application of the think-aloud method (\secref{sec:procedure}), yielded valuable insights and enabled us to evaluate the interaction data of the 12 participants during the 11 \acrshort{cwt} tasks in more detail. 
%
Thus, the analysis of these interaction data is a secondary analysis of the data obtained during the \acrshortpl{cwt} as part of the formative evaluation described in the previous section.


These interactions, such as clicks, scrolls, and key-presses, which are performed in order to derive a new insight, are referred to as \emph{(insight) provenance} \cite{gotz2009characterizing} and are a vital cue for the analysis for cognitive processes (\secref{sec:background}).
%
After describing the steps involved in capturing our provenance data such as tools and components used by the participants and the processes involved (\secref{sec:tools-and-processes}), the obtained records were analysed on a quantitative basis (\secref{sec:cwt-results}). 
%
Lastly, we will show how an interactive visual analysis of such provenance data can be enabled through custom-made visual analytics systems (\secref{sec:provenance-visualization}).


% ------------------------------------------------------------------------------
\subsection{Background} \label{sec:background}
%
The underlying working hypothesis of the analysis of interaction data was inspired by behavioral mapping, by process models in the field of information seeking and retrieval (e.g., Joseph et al.~\cite{joseph2013models}), as well by the work of Pohl et al.~\cite{pohl2016using} who applied a lag-sequential analysis (e.g., Bakeman and Gottman~\cite{bakeman1997observing}) to investigate interactions, sequences of interactions and users' activities and processes when engaging with a visualization system. 
%
Behavioral mapping is a well-established research method where the paths, movements, and activities within a physical space of participants (when carrying out certain tasks) are recorded and transferred onto a map for further analysis. 
%
As an example, Shepley~\cite{shepley2002predesign} investigated the ways of staff members and their time spent on walking from activity to activity at a neonatal intensive care unit. 
%
One of the goals of behavioral mapping is to re-arrange the physical space and workstations to avoid unnecessary paths and to make the workflow more efficient. 
%
The underlying principles, questions, and metrics have been transferred to the virtual space of the \apluschis\ platform and the \acrshort{cwt} tasks, having in mind exploratory research questions such as: \emph{What are the processes and interactions of the participants? 
%
Are there efficient and inefficient participants with regards to specific tasks, what are their characteristics and how could we support this sub-group of users? 
%
Could inefficient series of loops and cycles be avoided by a re-arrangement of the tools and components within the platform, by providing more guidance support, or by highlighting central tools and components which are used by a majority of participants across several tasks?} 
%
To answer such questions as these, quantitatively and by means of metrics from behavioral mapping~\cite{ng2016behavioral} and graph theory (e.g. \emph{centrality}), a formal description of the interaction data of each \cwt\ task per participant, the tasks across all participants, as well as the participants across all tasks are required. 


The interaction data of one or more participants or for one or more tasks can be graphically represented as a directed, labeled multi-graph, with the tools as vertices and the processes that lead from one vertex $n$ to a consecutive vertex $n+1$ as edge-labels (or vice versa). 


% ------------------------------------------------------------------------------
\subsection{Tools and Processes} \label{sec:tools-and-processes}
Before the data processing of the on-screen and audio recordings, a comprehensive list of (cognitive) processes have been defined.
%
Overall, 28 processes have been pre-defined by the three investigators of the formative evaluation study, whereas two additional processes had to be included during the coding process to ensure that all processes are covered by the raw data. 
%
Some examples of these processes include \emph{commenting} (when evaluating pictures), \emph{reading} (if text passages have been reproduced verbatim), \emph{interpreting} (concerning pictures and text passages, if participants summarized or evaluated the content in their own words), \emph{pause} (if participants did not do or say anything), but also more basic processes when interacting with the \apluschis\ platform, such as \emph{scrolling}, \emph{sliding} (through the pictures of the \is), or \emph{hovering} and \emph{click on} (e.g. a certain term in the \wc). 
%
The tools and exploration subsystems (\secref{sec:proposed-design}), have been further distinguished by considering also the chapter of the brochure~\cite{aok} into account; i.e., instead of distinguishing ‘only’ between \ImageSlider, \WordCloud, etc., it has been further differentiated between the preview (small) and the enlarged \ImageSlider\ as well as the \WordCloud, \Tilebar, \Snippets, etc. for each of the seven chapters of the brochure, resulting in 86 ‘tools’.
%
The coding of the on-screen and audio recordings has been to a large extent carried out individually by the above-mentioned investigators.
%
However, in cases the investigators were not absolutely sure on how to interpret a participants activity and to code it into one of the pre-defined processes, he or she noted the time-stamp and the coding of such ambiguous activities has been done collaboratively by reaching a consensus.
%
The processes have been coded only if they exceeded a duration of around $1 \unit{s}$. For the computation of several metrics and further analysis, the individual interaction data for each of the tasks has been represented as sequence of 
$\langle \mathit{tool}_{\mathit{src}}, \mathit{process}, \mathit{tool}_{\mathit{tar}} \rangle$
triples. 



% ------------------------------------------------------------------------------
\subsection{Results and Discussion} \label{sec:cwt-results}
%
Overall, i.e., across all 12 participants and 11 \cwt\ tasks, 1,870 processes (=edges) have been observed, whereas 21 out of the 30 processes were applied at least once by the participants. The three most frequently applied processes are \emph{scrolling} ($n = 268$), \emph{sliding} ($n = 240$) and \emph{click on} ($n = 235$). 
%
On average, a process took $4.18 \unit{s}$ (\emph{Mdn} = $2 \unit{s}$, \emph{SD} = $5.03 \unit{s}$), ranging from $1$ to $64 \unit{s}$. 
%
The tasks (\tableref{tab:CWT}) where the most processes have been applied by all participants are \emph{IS2} ($n = 337$), \emph{WC2} ($n = 267$) and \emph{IS3} ($n = 242$); whereas the tasks with the least amount of processes are \emph{WC4} ($n = 65$), \emph{TiB1} ($n = 77$) and \emph{WC1} ($n = 94$). 

With regards to the `tools', 75 out of the 86 differentiated tools have been applied at least once by the participants, whereas the three most frequently applied tools are the \emph{\isL}\ (the \ImageSlider\ in enlarged form) \emph{for chapter 6} ($n = 308$), the \emph{\snps\ for chapter 1} ($n = 124$) and the \emph{\Searchbar} ($n = 103$). 
%
To put these numbers in context, the $75$ tools which have been applied at least once, were visited $2,002$ times across all participants and \acrshort{cwt} tasks. 
%
The three most applied tools, i.e., those with the highest \emph{centrality}, cover around a quarter of all incoming and outgoing edges.


The triple sequence $T$ allows to easily evaluate further metrics, such as the number of
(i) \emph{loops}, i.e., $|\{\langle s, ., t \rangle \in T : s = t \}|$,
(ii) \emph{multiple edges}, i.e., $|\{\langle s, ., t \rangle \in T : (\exists \langle s', ., t' \rangle \in T\setminus \{\langle s, ., t \rangle\})[s = s' \wedge t = t']\}|$,
and (iii) \emph{identical triples}, i.e., 
$|\{\mathbf{t} \in T : (\exists \mathbf{t}' \in T\setminus \{\mathbf{t}\})[\mathbf{t} = \mathbf{t}']\}|$.
% 
Overall, $1,870$ triples have been identified. 
%
$1,030$ of them are \emph{loops}, comprising $59$ (of the $75$) tools. 
%
The three tools with the most \emph{loops} are the \emph{\isL\ for chapter 6} ($n = 288$), the \emph{\snps\ for chapter 1} ($n = 91$) and the \emph{\snps\ for chapter 3}. 
%
These three cases are identical to the three most prominent \emph{multiple edges}. $1,479$ triples represent \emph{multiple edges} with $197$ unique tool combinations.
%
Finally, $1,177$ triples represent \emph{identical triples}, with $171$ distinct instances. 
%
The three most prominent ones are subsets of the above mentioned \emph{loops} and \emph{multiple edges} concerning the tool \emph{\isL\ for chapter 6}, with the processes \emph{sliding} ($n = 170$), \emph{commenting} ($n = 53$) and \emph{viewing} ($n = 41$). 


The fact that certain chapters of the brochure occur several times within the top-three of the applied tools, the \emph{loops}, \emph{multiple edges} and \emph{identical triples}, is of course caused by the concrete \cwt\ tasks (e.g., the tasks \taskTibThree\ and \taskIsTwo\ which specifically ask for a comparison between the content in chapter 6 and other chapters; see \tableref{tab:CWT}). 


The transitions between tools (agnostic with regards to chapters) are shown in the matrix in \figref{fig:adjacency-matrix}. 
%
The \emph{loops} correspond to the main diagonal, while all cells with a value $> 1$ contain \emph{multiple edges}. 
%
Please note that several transitions from one tool to another are technically not possible; in particular, several transitions from or to the \isL. 
%
When distinguishing only between the more broadly defined $10$ tools as in \figref{fig:adjacency-matrix}, out of the $1,870$ triples, $1,259$ are \emph{loops}, whereas all tools are affected, $1,802$ triples represent \emph{multiple edges}, with $58$ distinct tool combinations, and $1,616$ represent \emph{identical triples}, with $221$ unique instances. 
%
The three most observed \emph{identical triples} are 
$\langle \mathit{\isL}, \mathit{sliding}, \mathit{\isL} \rangle$ ($n = 221$), 
$\langle \mathit{\wc}, \mathit{scrolling}, \mathit{\wc} \rangle$ ($n = 77$), and 
$\langle \mathit{\wc}, \mathit{scanning}, \mathit{\wc} \rangle$ ($n = 74$). 

\mediumwidth
\begin{figure}[ht!]
    \centering
    \includegraphics[width=\mediumwidth\linewidth]{figures/adjacency_matrix/adjacency_matrix_v0.4_viridis.pdf}
    \caption{
    The adjacency matrix of transitions between the `high-level' tools. 
    The entries in the main diagonal corresponds to our notion of \emph{loops}, while \emph{multiple edges} populate the remaining cells.
    The cells reflecting transitions which are technically not possible are intentionally left blank.
    }
    \label{fig:adjacency-matrix}
\end{figure}



The often observed \emph{loops}, \emph{multiple edges}, and \emph{identical triples} with regards to the 
\emph{\isL}, reveal some potential for improvement to make the information search for users more efficient. 
%
One potential improvement approach has been suggested by a participant in the course of the formative evaluation study (\emph{\POne}: An \emph{Image Tile Display} that allows for more efficient scanning through the images of a certain chapter at once).
%



% ------------------------------------------------------------------------------
\subsection{Provenance Visualization} \label{sec:provenance-visualization}


While a `global' evaluation of the provenance -- e.g., obtaining quantitative measures for \emph{loops} or usage of individual components -- can be conducted based on the raw interaction transcripts, a more in-depth analysis requires dedicated visualizations.
%
Those would allow the analysis of interactions on a per-user and/or per-task basis and could answer additional questions, such as
`Are the different groups of users observable?' (e.g., depth-first search vs. breath-first search exploration process), `Are there any outliers?' (users whose exploration process differs significantly from all others), etc.
%
The data's underlying graph structure invites the application of different established visualizations. 
%
We implement two customized visual analytics tools, a graph as well as a matrix layout, which allow us to investigate different orthogonal aspects of the interaction data.



\paragraph*{Provenance Graph}
In a \emph{Provenance Graph}, we visualize a user's alteration and switching between different processes; i.e., in a directed weighted graph, we show how often a user switched between different processes (\figref{fig:provenance-vis}, bottom). 
%
With this type of visualization, it is possible to, for instance, spot if a user goes back and forth between two different processes or if they go through the same sequence of processes over and over (cycles). 
%
We indicate the time spent on specific processes through the size of the respective node and, conversely, the number of transitions between processes though the thickness of the respective links.


\paragraph*{Provenance Matrix}
%
Besides the graph representation, we also visualize the provenance information in a matrix layout where the sorted rows (\Startscreen/\dl\ $>$ \toc\ $>$ \wc/\hwc/\is\ $>$ \tob/\tib\ $>$ \snps/\fullt) represent the high-level tools at  different levels of visual granularity and abstraction (overview to closeup $\sim$ top to bottom). 
%
The columns reflect the different processes, sorted by their type (from basal/technical processes such as `scrolling' to cognitive/psychological processes such as `interpreting').
%
The matrix cells exhibit a color-coding, indicating the overall time a user spent on a tool-process pair.
%
Additionally, we show the sequence of the exploration with arrows spanning consecutively visited cells. 
%
As the display of `all' arrows would overload the visualization, we display only the most recent transitions, with the recentness modelled by an alpha-drop-off. 
%
Hovering over a cell triggers a tooltip which lists all the interaction triples responsible for said cell; i.e., as opposed to the provenance graph, the matrix is able to reveal correlations between processes and components, i.e., it shows 
(i) at which levels of visual abstraction a user predominantly operates, 
(ii) which processes they carry out at which level, 
(iii) which processes they carry out using which component, and 
(iv) whether they exhibit a rather vertical ($\sim$ depth-first search) or horizontal ($\sim$ breath-first search) exploration behavior.


\paragraph*{Use Case Examples}
Finally, we investigate the effectiveness of the proposed provenance visualizations for the visual analysis of tasks conducted by different users.
%
To this end, we take a look at one exemplar task (\taskWcThree, Version 1: `How does type I diabetes mellitus develop?') which should both, encourage an `open' exploration process, and result in comparable interactions amongst users. 
%
We compare the respective provenance visualizations of two participants (\POne\ and \PTwelve) with vastly different visual analytics 
proficiencies. 
%
It took \POne\ $188 \unit{s}$ to successfully complete the task, while \PTwelve\ required only $63 \unit{s}$.
%
\figref{fig:provenance-vis} shows the respective provenance graph and matrix, illustrating the interactions captured while working on said task. 
%
Even at a first glance it is obvious that \POne\ underwent a much more laborious exploration as \PTwelve\ which is in line with our impression during the supervised evaluation.

On a closer look, we can see \PTwelve\ used only 5 distinct processes without much back-and-forth between any of them. 
%
An inspection of the respective arrows in the provenance matrix also reveals that this participant generally moved from basal processes at a high level of abstraction (top left corner of the matrix) to rather cognitive processes at detail level (bottom right corner of the matrix). 
%
This is an expected exploration pattern of a competent information seeker.


\POne, on the other hand, required not just more time overall, but had a lot more back-and-forth between different technical and cognitive processes, i.e., `Scrolling' $\rightleftarrows$ `Scanning' and `Reading' $\rightleftarrows$ `Interpreting'. 
%
This interaction pattern is also confirmed by the provenance matrix which further reveals that \POne\ has several movements from low- to high abstraction levels such as from \fullt\ to \wc, or from \wc\ to \toc. 
%
We take these bottom-to-top patterns as a cue for an individual who ran into an exploratory wrong track, hence they had to backtrack to a higher level in order to find the correct path to their desired information.


\begin{figure}[ht!]
    \centering
    \begin{minipage}{\mediumwidth\linewidth}
        \includegraphics[width=0.49\linewidth]{figures/interaction-visualization/P01_graph_v1.0.png} 
        \includegraphics[width=0.49\linewidth]{figures/interaction-visualization/P12_graph.png} \\
        \includegraphics[width=\linewidth]{figures/interaction-visualization/P01_12_matrix_v1.0.pdf}
    \end{minipage}
    \caption{
    Provenance visualizations for two very different users, \POne\ (left) and \PTwelve\ (right), working on task \taskWcThree. 
    Their respective provenance graphs (top) reveal the alternations between processes, while the provenance matrices (bottom) clearly show the alternations between levels of visual granularity.
    }
    \label{fig:provenance-vis}
\end{figure}

The next research focus will be on the automatic analysis and the clustering of users based on their interaction patterns, as this information can be leveraged to propose adequate visualization to them \cite{gotz_behavior-driven_2009}. 




% ==============================================================================
\section{Overall Discussion} \label{sec:discussion} 

Our exploration system allows users to navigate through documents and adapt the visual representation and level of detail by using well-known visual analysis techniques such as word clouds, topic models, tile bars, and keyword search. 
%
The system provides the users with a two-fold document exploration: a traditional linear and non-linear document navigation by switching between content and detail. 
%
Furthermore, it allows users to follow both the edited content of a given document (supervised structure) as well as an automatically computed topic models (unsupervised structure). 
%
To the best of our knowledge, there are few empirical studies on the cognitive and motivational aspects of using document visualizations (\eg\ tile bars and word clouds, with those of linear document readers). 
%
Our evaluation is a first confirmation that our approach could foster interest and heighten curiosity by using a distant-reading approach for exploring the content of interest more efficiently.
%
Further key findings are that the participants in general enjoyed the non-linear content exploration, even if some participants would have needed more support functions at least in beginning of use. 
%
There were mixed results regarding intuitiveness of the \WordCloud\, whereas the \Tilebar\ and the \ImageSlider\ have been evaluated as being intuitive; which is also reflected in their agreement to the question if they would use these components again.



Our study showed that users did not make great use of the topic model structure. 
%
This may be partly due to the unfamiliar representation of topic models. 
%
Recently, some studies have investigated the impact of word clouds for topic understanding \cite{10.1162/tacl_a_00042} and keyword summaries \cite{8017641}. 
%
Word clouds, as it transpired, are particularly useful for quickly identifying the most common and frequent terms, while disadvantages may arise in decoding numeric values from font sizes for larger sets of keywords. 
%
An alternative visualization could be a simple word list with a frequency encoding (\eg\ font size, bars) that represent each topic. 
%
\figref{fig:wc-tc} shows such an alternative representation (2) to improve topic understanding where the keywords for each topic are displayed among each other. 
%
By using such a layout, overlapping term composition may be included to increase topic understanding and cause confusion as with the previous layout (1). 
%
Advanced topic model visualization will be considered in future work.

\begin{figure}
    \centering
    \includegraphics[width=0.6\linewidth]{figures/wc-tc.png}
    \caption{Adaptive word cloud layouts: for improving topic understanding the arrangement can be switched from Wordle layout (1) to an alternative list representation (2).}
    \label{fig:wc-tc}
\end{figure}

One key element of our system is its ``user tracking'' during document exploration.
%
In particular, we track which keywords have been explored, with which visual component, and for how long. 
%
This is considered \emph{important information provenance data} which, in our system, is used for provenance visualizations (\eg\ history word cloud, interaction matrix and graph). 
%
The latter is an important functionality for content recommendation, and forms the basis for the mitigation of cognitive biases and potentially harmful and wrong pre-conceptions.
%
The provenance visualizations may reveal emerging difficulties of users during information seeking tasks by showing outstanding patterns of user behavior, \eg\ horizontal, vertical or loop-like patterns. Our evaluation is a first step in this direction. 
%
To investigate user behavior during the CWT tasks, we integrated the CWT tasks in our provenance visualizations.



% ------------------------------------------------------------------------------
\subsection{Future Work} \label{sec:future-work}
The advantages of aggregated document representations warrant further examination in future research; in particular, the specific benefits that can be derived from using aggregated document representations. 
%
Furthermore, our future work will involve exploring potential misunderstandings that may arise from the highly aggregated nature of certain content presentations.


As evident from \secref{sec:interaction-analysis}, a vital cue which we plan to leverage for adaptive visualizations are \emph{user interactions}.
%
Research has demonstrated that user interactions can effectively be utilized for recommending particular types of visualizations \cite{gotz_behavior-driven_2009} and even help mitigate cognitive biases \cite{gotz_adaptive_2016}.
%
An investigation of the user interactions obtained from the \acrshort{cwt} (\secref{sec:cwt-results}) with customized visual analysis tools (\secref{sec:provenance-visualization}) revealed that meaningful interaction patterns, reflecting different types of users, can be observed.
%
We assume that users can be clustered into cohesive groups using said patterns, which, in turn, reflect a user group's need for specific types of visualizations.
%
Therefore, an open challenge is the robust and continuous tracking and classification of these interactions.
%
While this was done manually for the 12 participants of the \acrshort{cwt}, a purely automatic solution is necessary for the long-term.
%
Our prototype already comprises a tracking of the components a user is interacting with. % using.
%
To this end, we leverage the mouse pointer position together with the established assumption that a user's cursor movements are correlated with their gaze~\cite{reichle_ez_2006, buscher_eye_2008}.
%
The other aspect of our interaction logs -- the process a user is occupied with (\secref{sec:tools-and-processes}) -- is significantly harder to track automatically. 
%
Even though purely technical processes such as `scrolling' or `clicking' are trivial to track, the capturing of cognitive processes such as `reading' or `interpreting' pose a non-trivial challenge.
%
Yet, even such can be determined using low-level mouse interactions together with well-defined heuristics~\cite{10.1145/2207676.2208591, kirsh_virtual_2022}.



Additionally, we have planned further user studies. 
%
These will focus on the planned enhancements of the system, such as the representation of behavioral patterns, which can support users in reflecting on their information-seeking behavior and, thus, potentially contribute to the detection and prevention of biases, especially confirmation bias. 
%
This will include investigating how different ways of presenting the interactions can promote unbiased information-seeking by also taking into account differences between users, such as different visualization preferences, different states of knowledge or, as already mentioned above, age groups. 
%
Moreover, the need for support, as found in our current study for some users due to the novelty of the system, will be addressed.


In future work, we also intend to research automatic recommendation and develop adaptation methods based on the current system. 
%
Such a further developed system version will also be subjected to an extensive case study in order to evaluate the system in detail.


% ------------------------------------------------------------------------------
\subsection{A Model for Adaptation of Content and Presentation}

Our initial motivation was to provide an interactive and adaptive \chis. 
%
These systems should adapt the content and presentation to the users' information needs and preferences. 
%
Our document exploration design has a number of variables which could be controlled and adjusted by the system to adapt itself to the users' information need and preferences, and recommend views and content. 
%
To that end, we define a model based on the following dimensions along which automatic adaptation can be performed. 
%
In future work, we will focus on the prediction of dimensions to use and how to set them for specific users.



\paragraph*{Content Navigation} 
Here, \emph{what to show} when the user is doing a mouse-over on a term of the Word Cloud is to be decided. 
%
The principal options include (a) go to the best matching full text position (full detail), (b) show the text snippets of multiple matches (an intermediate detail level), or (c) show the tile bar (lowest level of detail) from which the user may pick a finding location. 
%
The system could determine the answer based on the previous selections made by the user, presuming the user has a stable preference. 
%
On the other hand, the system could track which background knowledge is already available, and show the higher levels of details for topics which are not well-known.


\paragraph*{Document Structure} 
Our \apluschis\ design shows the section Word Clouds together with the (a) section headings or (b) topic models computed for each section. 
%
These represent both supervised and unsupervised content structures. 
%
When the user requests a section, the system might adapt to show either one of these. 
%
To this end, the system might predict whether the user prefers the traditional, edited document structure given by the headings, or the computed content structure from a topic model. 
%
The latter provides an opportunity to compute the topic models such as to represent the users' interest and background information.


\paragraph*{Configuration of the History Cloud} 
The system may present to the user either a History Cloud of (a) what has already been explored, or (b) what has not been explored yet. 
%
The system could predict if the user would want to deepen their knowledge on a particular topic or broaden their horizon into new topics. Depending on the user's intent, the system can choose from which terms to create the history Word Cloud. 
%
This might also be a good starting point to mitigate biases once they are detected in the users.
% 
It is important to recognize that these are concepts, and thus require specific implementations to monitor, track, and characterize the users' background knowledge, preferences, and interests. 
%
In order to explore potential implementations and solutions to realize these concepts, both direct and indirect methods will be examined in our future research work.



% ==============================================================================
\section{Conclusion} \label{sec:conclusion}
%
Currently, a significant number of existing CHIS fall short when it comes to presenting health information in interactive, adaptive, and/or personalized ways. 
%
In this regard, interactive document visualization techniques are more conducive to enhancing user engagement and promoting a deeper understanding of complex information, thus providing an amplitude of opportunities for improved support of information-seeking tasks. 
%
We presented a novel and innovative document exploration system that allows users to visually explore health documents at various levels of detail and abstraction. 
%
We incorporated well-known document visualization techniques, such as \WordCloud s and \Tilebar s, into our design and applied it in the domain of \acrshort{ttwodm}. 
%
We evaluated our implemented system by performing a formative study based on a \cwt\ and compared our approach with linear and close reading. 
%
The evaluation results are promising and constructive, and demonstrate that our approach is easy to use and helps in content exploration, further facilitating more interactive content navigation as well as motivating users to engage with our implemented system at different levels of visual granularity and detail. 
%
We also presented concepts for possible visual adaptation by collecting user interaction data and visualizing this data in two provenance visualizations to reveal specific information needs as well as reading preferences.




\section*{Acknowledgments}\label{sec:Acknowledgments}
The work of O. Ben-Porat was supported by the Israel Science Foundation (ISF; Grant No. 3079/24). 

\bibliographystyle{plainnat}
\documentclass{MITstyle}

%\usepackage[table]{xcolor}
\usepackage{chngcntr}
\usepackage{hyperref}
\usepackage{microtype}

\title{A Lightweight and Extensible Cell Segmentation and Classification Model for Whole Slide Images}

\author{Nikita Shvetsov~$^{1, }$\footnote{Correspondence e-mail: nikita.shvetsov@uit.no}, Thomas K. Kilvaer~$^{2, 3}$, Masoud Tafavvoghi~$^{4}$, Anders Sildnes~$^{1}$, \\ Kajsa Møllersen~$^{4}$, Lill-Tove Rasmussen Busund~$^{5, 6}$, Lars Ailo Bongo~$^{1}$ \\
%
\vspace{1em} % Space between authors and afilliations
%
\normalfont{\small $^{1}$Department of Computer Science, UiT The Arctic University of Norway}\\
\normalfont{\small $^{2}$Department of Oncology, University Hospital of North Norway}\\
\normalfont{\small $^{3}$Department of Clinical Medicine, UiT The Arctic University of Norway}\\
\normalfont{\small $^{4}$Department of Community Medicine, UiT The Arctic University of Norway}\\
\normalfont{\small $^{5}$Department of Medical Biology, UiT The Arctic University of Norway} \\
\normalfont{\small $^{6}$Department of Clinical Pathology, University Hospital of North Norway} %\vspace{2em}
}

\begin{document}
\maketitle

\section*{Abstract}

% \begin{abstract}
% Developing clinically useful cell-level analysis tools in digital pathology remains challenging due to limitations in dataset granularity, inconsistent annotations, computational demands of advanced models, and difficulties in integrating new technologies into clinical workflows. To address these challenges, we propose a multi-faceted solution that enhances data quality, model performance, and usability to create a lightweight and extensible cell segmentation and classification model.

% First, we update data labels by employing a cross-relabeling process that refines the labels of two existing datasets, PanNuke and MoNuSAC, to create a new unified dataset with enhanced granularity, encompassing seven distinct cell types. Second, we leverage the H-Optimus foundation model as a fixed encoder to improve feature representation for simultaneous cell segmentation and classification tasks. Third, to address the computational demands of foundation models, we employ knowledge distillation to reduce model size and complexity while maintaining comparable performance. Finally, to facilitate integration into clinical workflows, we integrate the distilled model into the QuPath software, a widely used open-source platform in digital pathology.

% Our results demonstrate improvements in cell segmentation and classification performance using the H‑Optimus-based model compared to a CNN-based model. Specifically, the average $R^2$ improved from 0.575 to 0.871, and the average $PQ$ score improved from 0.450 to 0.492, indicating better alignment with actual cell counts and enhanced segmentation and classification quality. Furthermore, the distilled student model maintains performance comparable to the larger foundation model while reducing the parameter count by a factor of 48.
% Overall, by reducing computational complexity and integrating it into existing workflows, the proposed approach may significantly impact diagnostic processes, reduce the workload of pathologists, and contribute to improved patient outcomes. Though our approach shows potential enhancements in efficiency and usability of cell segmentation and classification models in digital pathology, extensive validation is needed to deploy these models in clinical practice.
% \end{abstract}

%%% shortened abstract
\begin{abstract}
Developing clinically useful cell-level analysis tools in digital pathology remains challenging due to limitations in dataset granularity, inconsistent annotations, high computational demands, and difficulties integrating new technologies into workflows. To address these issues, we propose a solution that enhances data quality, model performance, and usability by creating a lightweight, extensible cell segmentation and classification model. 

First, we update data labels through cross-relabeling to refine annotations of PanNuke and MoNuSAC, producing a unified dataset with seven distinct cell types. Second, we leverage the H-Optimus foundation model as a fixed encoder to improve feature representation for simultaneous segmentation and classification tasks. Third, to address foundation models' computational demands, we distill knowledge to reduce model size and complexity while maintaining comparable performance. Finally, we integrate the distilled model into QuPath, a widely used open-source digital pathology platform. 

Results demonstrate improved segmentation and classification performance using the H-Optimus-based model compared to a CNN-based model. Specifically, average $R^2$ improved from 0.575 to 0.871, and average $PQ$ score improved from 0.450 to 0.492, indicating better alignment with actual cell counts and enhanced segmentation quality. The distilled model maintains comparable performance while reducing parameter count by a factor of 48. By reducing computational complexity and integrating into workflows, this approach may significantly impact diagnostics, reduce pathologist workload, and improve outcomes. Although the method shows promise, extensive validation is necessary prior to clinical deployment.
\end{abstract}
\clearpage

\section{Introduction}
In digital pathology, accurate segmentation and classification of cells are crucial for many diagnostic, prognostic, and predictive analyses \cite{Jaber_Beziaeva_etal._2019,Lin_Pan_etal._2022,Park_Ock_etal._2022,Shen_Choi_etal._2024}. Nowadays, developments in computational pathology offer multiple solutions \cite{H._Qu_P._Wu_etal._2020,Javed_Mahmood_etal._2020} to utilize cell-level datasets to train machine learning models that solve these problems. The quality and specificity of training datasets are critical for robust and accurate models. Adhering to the principle of "garbage in, garbage out", it is essential to ensure that these datasets are extensively and accurately labeled with distinct classes that reflect the diverse biological characteristics of different cell types. Unfortunately, the number of open-source datasets comprising such high-quality annotations is limited. Existing cell segmentation datasets \cite{Gamper_Koohbanani_etal._2019,Graham_Vu_etal._2019,Verma_Kumar_etal._2021} may offer extensive annotations for certain cell types while providing more general labels for others. For example, in PanNuke, which is one of the largest open-source datasets comprising labeled cells, various types of morphologically and functionally different inflammatory cells like macrophages and lymphocytes are clustered in a broad "inflammatory" class. Consequently, these classes are frequently omitted from analyses or aggregated into broader meta-classes \cite{Gamper_Koohbanani_etal._2020} and likely interfere with other cell classes included in the dataset. This and similar inconsistencies in annotation granularity limit the ability of machine learning models to learn the comprehensive and nuanced features necessary for accurate cell segmentation and classification. To address these challenges, methods for refining and standardizing dataset annotations are essential to enhance the quality of training data.

A complementary approach to mitigate the absence of high-quality training data is the use of foundation models. Foundation models as encoders are defined as large-scale, versatile networks pre-trained on vast, diverse datasets using self-supervised learning, contrasting with convolutional neural network (CNN) pre-trained encoders that rely on supervised learning with labeled data. In practice, foundation models leverage enormous amounts of weakly or unlabeled data from millions of whole slide images (WSIs) and employ self-attention mechanisms to capture long-range dependencies and global context \cite{Chen_Ding_etal._2024,Saillard_Jenatton_etal._2024,Vorontsov_Bozkurt_etal._2024,Xu_Usuyama_etal._2024}. As a consequence, foundation models are able to produce transferable feature representations across different cell types and tissue environments. The feature representations can be leveraged by decoder networks to produce segmentation masks and pixel-level classifications. Because foundation models have comprehensive feature representations, they can be effectively fine-tuned using much smaller amounts of cell-level data compared to the large datasets needed to train models from scratch. Furthermore, foundation models incorporate adversarial training elements or contrastive learning \cite{Chen_Ding_etal._2024,Xu_Usuyama_etal._2024}, enhancing their resilience and adaptability by exposing them to challenging and varied scenarios during training. This may result in more generalizable models, often making them well-suited for diverse and complex tasks in digital pathology.

Despite the inherent advantages of foundation models, their deployment for practical use faces its own obstacles. In particular, they require substantial computational power, financial investments and rigorous testing to ensure reliability and efficacy for a given task \cite{Akkus_Dangott_etal._2022,Dragomir_Cocuz_etal._2022,Go_2022,Jafri_Farooqui_etal._2024}. Moreover, while foundation models enhance feature representation and performance, they depend on the quality of available annotations for decoder fine-tuning and, like any other model, cannot resolve existing inconsistencies or ambiguities in data labels. Therefore, there remains a critical need for solutions that address both data quality and practical deployment considerations.
Further, integrating new technologies into existing clinical workflows often encounters resistance, as it necessitates adjustments to established diagnostic processes. So, there is a need to develop solutions that could be integrated into current practices, minimizing the burden on medical professionals to adopt new tools \cite{King_Williams_etal._2023}.

Existing solutions \cite{Goldsborough_Philps_etal._2024,Hörst_Rempe_etal._2024}, while addressing some aspects of these challenges, fall short in providing a comprehensive approach. To address the data quality and clinical deployment issues, we propose a multi-faceted solution that encompasses data refinement, model optimization, and integration with existing pathology tools (\hyperref[fig:fig1]{Figure 1}). The outcome is a lightweight cell segmentation and classification model that can be integrated into digital pathology workflows for practical clinical use.

\begin{figure}[h!]
    \centering
    \includegraphics[width=\textwidth, height=0.82\textheight, keepaspectratio]{images/Figure_1.pdf}
    \caption{Overview of the proposed solution, including 1) Data refinement using cross-relabeling, 2) Teacher model development and fine tuning, 3) Student model optimization with knowledge distillation and 4) Student model and QuPath integration}
    \label{fig:fig1}
\end{figure}
\clearpage

Our approach begins with preparing the data for the fine-tuning and training of the machine learning models. We create a refined dataset, acquired via cross-relabeling two cell-level datasets, enhancing annotation specificity and consistency of the labeled data. Subsequently, we create a cell segmentation and classification model based on the foundation model. We leverage the foundation model as a fixed encoder and fine-tune a decoder using the refined dataset to improve generalization across diverse tissue- and cell types.
To ensure that the model remains lightweight and deployable in a possibly resource-constrained environment, we employ knowledge distillation to approximate the functionality of the foundation model. Finally, to facilitate the practical application of our model in digital pathology workflows, we integrate it with the QuPath \cite{Bankhead_Loughrey_etal._2017} application. Each methodological component contributes to the overarching goal of enhancing model performance, generalizability, and usability in clinical settings.

The primary contributions of this paper are:
\begin{enumerate}
    \item \textit{Data labels refinement through cross-relabeling:}
    
    We propose a new method for refining labels of cell-level datasets through cross-relabeling. This method employs classification models to re-label broad and ambiguous instances, resulting in a more diverse dataset. Our evaluation demonstrates that these classification models achieve high accuracy on test subsets, indicating the reliability of the method for label refinement.

    \item \textit{Enhanced model performance via foundation models:}
    
    We employ a foundation model as a feature extractor for the cell segmentation and classification task. In comparison with training a CNN model from scratch, the foundation model backbone only needs fine-tuning, which significantly reduces training time, computational resources and data requirements. We show that using a foundation model encoder leads to better performance in cell segmentation and classification networks than using a CNN-based encoder. This improvement may enable the model to generalize more effectively across various tissue types and imaging methods.
    
    \item \textit{Model optimization through knowledge distillation:}
    
    We show that a smaller student model trained using knowledge distillation on the refined dataset obtained via our cross-relabeling approach from a foundation model achieves comparable performance in cell segmentation and quantification tasks. As a result, this model is more suitable for deployment in environments without high-performance computing resources.
    
    \item \textit{Integration with QuPath:}
    
    We integrate the distilled cell segmentation and classification model into QuPath, a widely used open-source digital pathology platform, to accelerate clinical adaptation by enabling pathologists to more easily incorporate advanced computational tools into their existing workflows.
\end{enumerate}

Through these methodological steps, we aim to bridge the gap between advanced machine learning techniques and practical clinical applications, making accurate and efficient digital pathology accessible in a broader range of healthcare settings.

\section{Refining Existing Datasets Using Cross-Relabeling}
To address the limitations of sparse and ambiguous labeling of cell-level datasets, we propose a generalizable cross-relabeling strategy that can be applied to any dataset containing broadly categorized or imprecisely labeled cell types. This approach involves training and subsequently leveraging classification models to refine broad categories into more specific or biologically relevant classes.
When applied to cell-level data, the methodology includes extracting individual cell images from the dataset patches, preprocessing these images to standardize the size and accommodate partial cells, and then training deep learning classifiers capable of distinguishing between the finer cell subtypes within the coarser categories. 
To illustrate our approach, we focus on the PanNuke \cite{Gamper_Koohbanani_etal._2020, Gamper_Koohbanani_etal._2019} and MoNuSAC \cite{Verma_Kumar_etal._2021} datasets that we have used to train models for cell quantification in our previous works \cite{Shvetsov_Grønnesby_etal._2022,Shvetsov_Sildnes_etal._2024}. We find that for better cell differentiation we have to introduce more granular labels. PanNuke includes a broad classification of "inflammatory" cells, encompassing lymphocytes, macrophages, and neutrophils. Each cell type differs significantly in structure, function, and clinical relevance. Conversely, MoNuSAC uses the label "epithelial" for a class that comprises both benign epithelial cells and malignant neoplastic cells. This practice makes it challenging to differentiate between benign and malignant epithelial cells in the dataset, which is a critical distinction when identifying tumor areas within tissue samples. To address these issues, we implement a cross-relabeling strategy as shown in \hyperref[fig:fig2]{Figure 2}. The key components are two classification models: one is trained on singular cell images from PanNuke data to classify the epithelial meta-class into epithelial and neoplastic classes. The other is trained on MoNuSAC to refine the inflammatory class into lymphocytes, neutrophils, and macrophages.

\begin{figure}[h!]
    \centering
    \includegraphics[width=\textwidth]{images/Figure_2.pdf}
    \caption{Refined dataset generation via cross relabeling}
    \label{fig:fig2}
\end{figure}

The refining approach consists of three consecutive steps. The first is the preprocessing step, in which we extract individual cells from both datasets (\hyperref[fig:fig3]{Figure 3}). The specifics of PanNuke and MoNuSAC patch preparation before cell preprocessing are provided in \hyperref[chap:S1]{Appendix S1}.

\begin{figure}[h!]
    \centering
    \includegraphics[width=\textwidth]{images/Figure_3.pdf}
    \caption{Cell instances preprocessing including (1) cell map extraction, (2) bounding box delineation, (3) adjusting cell boxes and (4) cropping and resizing of cell images}
    \label{fig:fig3}
\end{figure}

During preprocessing, we extract cell type maps from the ground truth label mask and calculate bounding boxes around each cell instance. To accommodate partial cells at patch borders, a common issue in cropped patch images, we employ mirror padding and extend the field of view of the cell label by 15 pixels to capture adjacent cells. We then crop and resize the identified regions to $64 \times 64$ pixels using bicubic interpolation.

The preprocessed PanNuke dataset comprises 68,031 neoplastic and 23,207 epithelial cell images, while MoNuSAC comprises  33,104 lymphocytes, 1,252 neutrophils, and 1,695 macrophages, which we subsequently use in training cell classification models and classifying the cell image data \hyperref[fig:S2]{Appendix Figure S2 (1)}. 

The next step is to train two distinct ResNet50-based classifiers tailored to address the specific labeling challenges inherent in each dataset. We use ResNet50 for classification models due to its proven effectiveness for image classification tasks in histopathology \cite{pan2022reviewmachinelearningapproaches}, and its compatibility with small images. For the PanNuke dataset, we design the classifier, trained on MoNuSAC data, to disaggregate the heterogeneous "inflammatory" cell category into distinct subtypes: lymphocytes, macrophages, and neutrophils. Similarly, for the MoNuSAC dataset, the classifier is trained on PanNuke data and distinguishes between benign and malignant epithelial cells within the overarching "epithelial" label. By applying these targeted classifiers to their respective datasets, we assign more specific labels to individual cell instances, thus enabling us to create a unified dataset.
To ensure a balanced representation of classes, we train both models on datasets that had been equalized to match the size of the least represented class. Thus, we obtain datasets comprising 23,207 samples per class for PanNuke and 1,252 samples per class for MoNuSAC data. Next, we partition both of them into training (70\%), validation (20\%), and testing (10\%) subsets. To mitigate the risk of overfitting, we use a single dropout layer with a rate of p=0.5 in both models and data augmentation using randomized color perturbations, rotation, and horizontal and vertical flipping. We employ AdamW optimizer and the cross-entropy loss function for the training criterion.

To evaluate the two trained models, we measure the classification accuracy on the respective test subsets. The accuracies on the test subset for both classifiers are presented in \hyperref[tab:1]{Table 1}. The PanNuke model achieves an average accuracy of 93.57\%, with higher accuracy for neoplastic cells (96.06\%) compared to epithelial cells (86.26\%). The confusion matrix in Figure A3.1 shows that the model predominantly distinguishes accurately between epithelial and neoplastic tissues, with a substantial number of correct classifications and relatively few misclassifications. The MoNuSAC model demonstrates an average accuracy of 98.92\%, excelling in classifying lymphocytes (99.67\%) and macrophages (94.12\%), with lower performance for neutrophils (85.71\%). The confusion matrix in Figure A3.2 shows that the model identifies lymphocytes and performs reasonably well with macrophages and neutrophils.

\begin{table}[h!]
\renewcommand{\arraystretch}{1.5}
  \centering
  \caption{Cell classification results for PanNuke and MoNuSAC trained models (CI 95\%).}
  \label{tab:1}
  \begin{tabular}{|l|c|c|}
   \hline
   %\rowcolor{gray!30}
    Accuracy               & PanNuke model              & MoNuSAC model              \\
    \hline
    Average      & 0.936 (0.931--0.941)         & 0.989 (0.986--0.993)        \\
    \hline
    Neoplastic   & 0.961 (0.956--0.965)         & -                          \\
    \hline
    Epithelial   & 0.863 (0.849--0.877)         & -                          \\
    \hline
    Lymphocytes  & -                          & 0.997 (0.995--0.999)        \\
    \hline
    Neutrophils  & -                          & 0.857 (0.796--0.918)        \\
    \hline
    Macrophages  & -                          & 0.941 (0.906--0.976)        \\
    \hline
  \end{tabular}
\end{table}

Finally, during the last step, we use the model trained on PanNuke data for epithelial cells in MoNuSAC and the model trained on MoNuSAC for the inflammatory cells class in PanNuke. Specifically, we use classifier models to relabel epithelial cells in MoNuSAC and inflammatory cells in PanNuke data. Then we combine cells with refined labels and the rest of the cells in both datasets to create a refined dataset (\hyperref[fig:S2]{Appendix Figure S2 (2)}). The process of relabeling cells and visualizing them on a patch is shown in \hyperref[fig:fig4]{Figure 4}. The cell counts in the refined dataset are provided in \hyperref[tab:S4]{Appendix Table S4}.

\begin{figure}[h!]
    \centering
    \includegraphics[width=\textwidth, height=0.42\textheight, keepaspectratio]{images/Figure_4.pdf}
    \caption{Cell relabeling procedure for epithelial and inflammatory cell classes}
    \label{fig:fig4}
\end{figure}

%\hfill

Relabeling and combining datasets have been explored in a prior study \cite{Parulekar_Kanwat_etal._2023}, where consecutive fine-tuning on multiple datasets was employed to account for hierarchical class label structures. While the method presented in \cite{Parulekar_Kanwat_etal._2023} is intuitive, it often lacks consistency and requires multiple fine-tuning runs, which can be cumbersome and time-consuming. 
In contrast, cross-relabeling simplifies this process by using specialized classification models tailored to each dataset's specific labeling challenges. This approach provides better transparency and produces a unified dataset encompassing seven distinct cell types across multiple tissue samples, enhancing data diversity for further model training or fine-tuning.

Despite these improvements, cross-relabeling does not entirely resolve issues related to poor labeling quality or the amount of labeled data. Specifically, our results show lower accuracies persist for underrepresented classes, such as macrophages, which may stem from a limited sample availability and intrinsic challenges in distinguishing these cells based solely on H\&E staining. Furthermore, while our method enhances label specificity, it relies on the initial quality of the broad labels; thus, any fundamental inaccuracies in the original annotations can propagate through the relabeling process. Addressing the overall problem of limited data labels may require integrating additional data sources or utilizing complementary immunohistochemical staining methods.
Although the reported performance metrics are obtained from evaluations on the native test sets of each dataset, it is important to note that the primary application of these classifiers is to perform cross-relabeling, where a model trained on one dataset (e.g., PanNuke) is applied to another (e.g., MoNuSAC) and vice versa. We acknowledge that a more systematic evaluation of cross-dataset generalization is needed and could be performed in future work.

Overall, the refined dataset produced by our approach can enhance the supervised training or fine-tuning of cell segmentation and classification models, especially those that utilize pre-trained foundation models to improve feature extraction robustness. In addition, these models can detect nuanced classes that enable researchers to conduct more detailed analyses of biological processes in computational pathology.

\section{Foundation models for robust cell segmentation and classification}

Accurate cell segmentation and classification in digital pathology are hindered by limited labeled data and the fact that conventional CNNs are unable to capture global contextual information due to their local receptive field constraints \cite{Gheflati_Rivaz_2022,Yang_Marcus_etal.}. Traditional approaches in cell quantification have predominantly relied on CNN encoders, such as ResNet50, given their proven effectiveness in semantic segmentation tasks \cite{Deshmane_2023,Graham_Vu_etal._2019,Mukasheva_Koishiyeva_etal._2024,Stringer_Wang_etal._2021}. However, approaches that include fine-tuning of pretrained CNNs, data augmentation, and stain normalization to partially increase data variability and address staining differences often fail to achieve the necessary generalization and robustness across diverse tissue types and staining conditions \cite{G._Wang_W._Li_etal._2018,Gao_Bagci_etal._2018,Karim_El_Khoury_Martin_Fockedey_etal._2021}.

To overcome these challenges, we leverage an encoder-decoder network that uses a foundation model as the encoder and a CNN upsampling decoder (\hyperref[fig:fig5]{Figure 5}) for simultaneous cell segmentation and classification in 2D patches extracted from WSIs. Foundation models with transformer-based architectures are viable alternatives to CNN-based encoders \cite{Shamshad_Khan_etal._2023,Sourget_2023}. They enable the creation of more advanced architectures that can decode or transform learned features more effectively \cite{Chen_Duan_etal._2023,Cheng_Misra_etal._2022,Xie_Wang_etal._2021}.

\begin{figure}[h!]
    \centering
    \includegraphics[width=\textwidth]{images/Figure_5.pdf}
    \caption{UNETR-like model with foundational model as backbone}
    \label{fig:fig5}
\end{figure}

By utilizing a transformer-based encoder, we incorporate global contextual information into the feature extraction process, which is a key advantage of such architectures \cite{Chen_Lu_etal._2021}. This foundation model integration facilitates accurate pixel-wise segmentation and classification without the need for extensive encoder training, thereby potentially improving generalization across varied cellular structures and tissue types.
In our implementation, we employ a modified UNETR \cite{Hatamizadeh_Tang_etal._2021} architecture that combines a vision transformer (ViT) \cite{Dosovitskiy_Beyer_etal._2021} encoder with a CNN-based decoder. The encoder utilizes the pretrained H-Optimus foundation model, which contains 1.1 billion parameters and is trained on over 500,000 H\&E stained WSIs \cite{Saillard_Jenatton_etal._2024}. We extract outputs from four evenly spaced transformer blocks $Z_i$, where $i \in [1, 14, 26, 38]$, to serve as residual connections for the CNN decoder. We select these blocks based on our observation that features from non-adjacent levels of the encoder lead to better overall performance on the test subset.

The CNN decoder upsamples the feature representations, acquired from the transformer blocks, to generate an intermediate vector that is handled by two task-specific layers that generate cell segmentation and classification masks. The first task-specific layer is the ‘Cellpose head’,  which is used to delineate cell instances. The layer generates horizontal and vertical gradient maps to form vector fields that are refined through gradient tracking in a post-processing step using the Cellpose algorithm \cite{Stringer_Wang_etal._2021}, known for its efficacy in cell segmentation tasks and generalizability across multiple domains \cite{Pachitariu_Stringer_2022,Stringer_Pachitariu_2024}. The second task-specific layer is the "Cell type head", which assigns labels to individual pixels. In the post-processing step, we determine the output classification label of each segmented cell instance by majority voting over the labeled pixels that comprise the cell in the segmentation map.

To evaluate model performance and measure the impact of adding a foundation model as backbone, we compare it to a ResNet50-based model. ResNet50 is a widely used solution for encoders in segmentation architectures in the medical domain \cite{Deshmane_2023,Graham_Vu_etal._2019,Mukasheva_Koishiyeva_etal._2024,Stringer_Wang_etal._2021}. For the H-Optimus-based model, we utilize frozen weights for the encoder and only fine-tune the decoder to take advantage of the extensive pre-training of the foundation model. For the ResNet50-based model we start with ImageNet \cite{Deng_Dong_etal.} weights and train both encoder and decoder parts. Hyperparameters for the training step are set to be identical, where possible, for comparable evaluation. 
For this evaluation, we deliberately use the PanNuke dataset to provide a standardized and controlled comparison between the H‑Optimus and ResNet50-based models (\hyperref[fig:S2]{Appendix Figure S2 (3)}). Specifically, we use two of the default PanNuke dataset splits (66\%) for training and validation, and reserve the third split (33\%) for testing.

To address the challenge of cell class imbalance in the PanNuke dataset, which is a common characteristic in most cell-level H\&E patch datasets, both models’ training processes employ a weighted loss function comprising cross-entropy and focal loss \cite{Lin_Goyal_etal._2018}. The focal loss component is adjusted with coefficients derived from each cell class' instance frequency, emphasizing learning from underrepresented classes and enhancing the model's sensitivity to rare but significant cellular patterns. The cross-entropy loss is augmented with spectral decoupling regularization \cite{Pezeshki_Kaba_etal._2021,Pohjonen_Stürenberg_etal._2022} and spatially varying label smoothing \cite{Islam_Glocker_2021}, which potentially stabilizes training and improves generalization in case of complex tissue morphologies. For optimization, we employ the \textit{AdamW} \cite{Loshchilov_Hutter_2019} to counter unbalanced class scenarios, with cosine annealing learning rate scheduler.

We utilize the scikit-learn library \cite{Van_der_Walt_Schönberger_etal._2014} and HoVer-Net \cite{Graham_Vu_etal._2019} implementations of $R^2$ (the coefficient of determination) and $PQ$ (panoptic quality) to evaluate our experiments. Complete mathematical formulations and detailed explanations of these metrics are provided in \hyperref[chap:S5]{Appendix S5}. To compute confidence intervals, we use nonparametric bootstrapping, where after calculating the metric on the full sample, we generated 1000 bootstrap replicates by resampling with replacement and then determined the 95\% confidence intervals as the 2.5th and 97.5th percentiles of the resulting empirical distribution.

%\hfill

The model comparisons are summarized in \hyperref[tab:2]{Table 2}. The H‑Optimus-based model achieves higher $R^2$ across all cell classes compared to the ResNet50-based model, which means that its predictions are more closely aligned with the PanNuke cell counts, indicating a stronger correlation with the observed data. Notably, the improvement of $R^2_{dead}$ may be an indicator of better global contextual representations provided by the foundation model backbone. In terms of segmentation and classification quality combined, measured by the PQ score, the H‑Optimus-based model demonstrates notable improvements across most cell classes. Overall, the average $R^2$ improved from 0.575 to 0.871, while the average $PQ$ score improved from 0.450 to 0.492, demonstrating better performance of the H-Optimus-based model.

\begin{table}[h!]
\renewcommand{\arraystretch}{1.5}
  \centering
  \caption{Cell quantification metrics for baseline and proposed models (CI 95\%).}
  \label{tab:2}
  \begin{tabular}{|l|c|c|}
    \hline
    %\rowcolor{gray!30}
    Metric             & Resnet50-based            & H-optimus-based              \\
    \hline
    $R^2_{neoplastic}$    & 0.681 (0.576--0.769)       & \textbf{0.941 (0.917--0.960)} \\
    \hline
    $R^2_{inflammatory}$  & 0.863 (0.778--0.903)       & \textbf{0.949 (0.918--0.966)} \\
    \hline
    $R^2_{connective}$    & 0.600 (0.488--0.698)       & 0.609 (0.436--0.772)          \\
    \hline
    $R^2_{dead}$          & 0.097 (-11.389--0.669)     & 0.925 (0.404--0.982)          \\
    \hline
    $R^2_{epithelial}$    & 0.635 (0.490--0.747)       & \textbf{0.930 (0.886--0.964)} \\
    \hline
    $PQ_{neoplastic}$       & 0.517 (0.499--0.535)       & \textbf{0.589 (0.575--0.604)} \\
    \hline
    $PQ_{inflammatory}$     & 0.455 (0.429--0.482)       & \textbf{0.528 (0.507--0.549)} \\
    \hline
    $PQ_{connective}$       & 0.416 (0.400--0.431)       & \textbf{0.451 (0.436--0.465)} \\
    \hline
    $PQ_{dead}$             & 0.374 (0.342--0.408)       & 0.292 (0.209--0.365)          \\
    \hline
    $PQ_{epithelial}$       & 0.488 (0.460--0.519)       & \textbf{0.599 (0.579--0.618)} \\
    \hline
  \end{tabular}
\end{table}

Our results  show that integrating the H‑Optimus foundation model within the UNETR architecture enhances the model's ability to segment and classify cells across diverse tissues from PanNuke data. The pretrained transformer encoder provides robust feature representations, resulting in higher average $R^2$ and $PQ$ scores compared to the CNN-based model. This leads to more reliable cell quantification and more accurate downstream analysis. Additionally, the streamlined fine-tuning process reduces computational overhead and training time, making the model more adaptable for new data.

Despite these advancements, the foundation model-based approach does not fully resolve all challenges related to cell segmentation and classification. We observe lower metric scores for underrepresented classes in the training data. Furthermore, foundation models typically encompass billions of parameters, resulting in substantial computational and memory requirements. It therefore poses challenges for deployment in resource-constrained environments, limiting their practical applicability in certain clinical settings.

\section{Model optimization via Knowledge Distillation}

To address the limitations posed by the extensive size of foundation models, we implement knowledge distillation — a model compression technique that leverages the teacher-student paradigm \cite{Hinton_Vinyals_etal._2015}. By training a smaller, more efficient student model to replicate the output of a larger, pre-trained teacher model, we retain performance while significantly reducing the model's complexity and resource requirements (\hyperref[fig:fig6]{Figure 6}).

\begin{figure}[h!]
    \centering
    \includegraphics[width=\textwidth, height=0.45\textheight, keepaspectratio]{images/Figure_6.pdf}
    \caption{Knowledge distillation framework for training a student model using a pre-trained teacher}
    \label{fig:fig6}
\end{figure}

We employ knowledge distillation to compress the H‑Optimus-based teacher model into a more efficient student model. The teacher model is the modified UNETR architecture with the H‑Optimus foundation model described in the previous chapter. The student model is based on a UNet architecture augmented with residual connections and incorporates a smaller ViT encoder with 9 million parameters \cite{Steiner_Kolesnikov_etal._2022,Wightman_2019}. 

First, we fine-tune the teacher model using the refined dataset from the cross-relabeling procedure (Section 2). Initially we train the decoder of the teacher model while keeping the encoder weights frozen. We split the refined dataset into train (70\%), validation (20\%) and test (10\%) subsets (\hyperref[fig:S2]{Appendix Figure S2 (4)}). During fine-tuning, we use the train and validation subsets, while leaving the test subset for model evaluation. We set the training procedure and model hyperparameters to be identical to those that were used to demonstrate the utility of foundation models for the simultaneous cell segmentation and classification task.

Next, we perform knowledge distillation from teacher to student using the refined dataset used to fine-tune the teacher model. The student model is trained to replicate the teacher model's outputs. We utilize a specialized loss function that aligns the student's predicted probability distribution with the teacher's, incorporating the teacher's class probability distribution derived from the output. Following the methodology of Hinton et al. \cite{Hinton_Vinyals_etal._2015}, we experiment with various hyperparameter settings for the temperature ($T$) and the balancing coefficients ($\alpha$ and $\beta$) in the loss function. We vary $T$ from 1 to 20 and adjust $\alpha$ and $\beta$ to balance the distillation and student losses. Through iterative tuning and evaluation, we identify that setting $T=14$, $\alpha=0.3$, and $\beta=0.7$ yields a configuration that converges and closely approximates the teacher model's performance during training.

Finally, we assess the performance of both models using the $R^2$ and $PQ$ (defined in \hyperref[chap:S5]{Appendix S5}) on the test set of the refined dataset (\hyperref[tab:3]{Table 3}). We observe that the 95\% confidence intervals overlap for most cell types, so we cannot claim statistically significant performance differences between the teacher and student models. One exception appears in the neoplastic class. The teacher model produces an $R^2$ of 0.919, while the student model shows an $R^2$ of 0.852. In addition, the student model achieves higher $PQ$ values for the neoplastic and connective classes, though the confidence intervals show overlap.

\begin{table}[h!]
\renewcommand{\arraystretch}{1.5}
  \centering
  \caption{Cell quantification metrics for teacher and distilled student models (CI 95\%).}
  \label{tab:3}
  \begin{tabular}{|l|c|c|}
    \hline
    %\rowcolor{gray!30}
    Metric & Teacher & Student \\
    \hline
    $R^2_{neoplastic}$    & \textbf{0.919} (0.898--0.939) & 0.852 (0.800--0.891) \\
    \hline
    $R^2_{lymphocyte}$    & 0.969 (0.956--0.977)         & 0.969 (0.956--0.978) \\
    \hline
    $R^2_{connective}$    & 0.694 (0.548--0.809)         & 0.618 (0.469--0.741) \\
    \hline
    $R^2_{dead}$          & 0.755 (0.400--0.908)         & 0.424 (0.100--0.731) \\
    \hline
    $R^2_{epithelial}$    & 0.922 (0.870--0.958)         & 0.843 (0.738--0.917) \\
    \hline
    $R^2_{macrophage}$    & 0.384 (-0.369--0.724)        & 0.704 (0.352--0.859) \\
    \hline
    $R^2_{neutrofil}$     & 0.854 (0.578--0.929)         & 0.833 (0.502--0.925) \\
    \hline
    $PQ_{neoplastic}$       & 0.581 (0.569--0.593)         & 0.601 (0.588--0.613) \\
    \hline
    $PQ_{lymphocyte}$       & 0.536 (0.520--0.553)         & 0.563 (0.544--0.579) \\
    \hline
    $PQ_{connective}$       & 0.436 (0.421--0.451)         & 0.457 (0.441--0.474) \\
    \hline
    $PQ_{dead}$             & 0.272 (0.235--0.315)         & 0.279 (0.201--0.369) \\
    \hline
    $PQ_{epithelial}$       & 0.522 (0.500--0.545)         & 0.530 (0.506--0.555) \\
    \hline
    $PQ_{macrophage}$       & 0.524 (0.459--0.588)         & 0.474 (0.405--0.543) \\
    \hline
    $PQ_{neutrofil}$        & 0.541 (0.490--0.592)         & 0.565 (0.522--0.607) \\
    \hline
  \end{tabular}
\end{table}


We further decompose the $PQ$ metric into its $SQ$ and $DQ$ components (\hyperref[tab:S6]{Appendix Table S6}). Both models produce nearly identical $SQ$ values, which indicates that they predict instance boundaries with similar precision. Although the student model shows some improvement in $DQ$ scores for certain classes, the confidence intervals overlap and do not confirm a statistically significant difference.

We observe that the student and teacher models yield comparable detection performance despite the student model using a much smaller and simpler architecture. A model with fewer parameters reduces the risk of overfitting when training data are scarce relative to the model’s complexity \cite{Farias_Ludermir_etal._2022}. The knowledge distillation process also encourages the student model to focus on the most generalizable detection features learned from the teacher. These factors enable the student model to achieve similar detection performance across different cell types.

Additionally, considering the model sizes reported in \hyperref[tab:4]{Table 4}, the distilled model achieves a significant reduction compared to the teacher model, with a 48-fold decrease in parameter count and a 5.5-fold reduction in on-disk size. In inference mode, the teacher model requires 16 GB of VRAM for a batch size of 32, while the distilled model only needs 3 GB of VRAM for the same batch size. These reductions make the distilled model significantly more practical for fine-tuning and deployment in resource-constrained environments.

\begin{table}[h!]
\renewcommand{\arraystretch}{1.5}
  \centering
  \caption{Parameter counts and size of teacher and distilled model}
  \label{tab:4}
  \adjustbox{max width=\textwidth}{%
  \begin{tabular}{|l|c|c|c|}
    \hline
    %\rowcolor{gray!30}
    Metric & H-optimus-based (Teacher) & mobileViT-based (Student) & Magnitude of difference \\
    \hline
    Parameters count       & 1,158,917,906   & \textbf{24,093,393}   & \textbf{48x}  \\
    \hline
    Estimated Total Size (MB) & 87,912       & \textbf{15,935}    & \textbf{5.5x} \\
    \hline
  \end{tabular}%
}
\end{table}

%\hfill

With recent advancements in complex network architectures and the use of pretrained encoders to achieve state-of-the-art performance \cite{Baumann_Dislich_etal._2024,Hörst_Rempe_etal._2024} in cell segmentation and classification tasks, model size, computational complexity, and processing times have increased. This limits the scalability and accessibility of these models. As we demonstrate, this may be mitigated using knowledge distillation. Studies in the field of natural language processing have demonstrated the efficacy of knowledge distillation in retaining the capabilities of the teacher model while achieving significant reductions in size and complexity \cite{Huangpu_Gao_2024,Sun_Yu_etal.}. 

We demonstrate the feasibility of knowledge distillation in digital pathology, specifically for cell segmentation and classification tasks. Moreover, we achieve this performance while also significantly reducing the parameter count. In addressing the challenge of knowledge transfer, we found that distillation from a transformer-based model to a smaller transformer is more straightforward than attempting to map transformer features to CNN blocks. In our experiments, using a CNN-based network as a student results in worse cell quantification performance due to the structural constraints of CNN feature space dimensions. 

Although our primary approach relies on a transformer-based student model that performs well, it can be further optimized to incorporate advantages from CNN architectures. For example, employing alternative techniques such as using ViT adapters \cite{Chen_Duan_etal._2023} or $1 \times 1$ convolutions to adjust feature map sizes may be beneficial for harnessing CNN advantages like enhanced local feature extraction. Moreover, if additional performance improvements are desired, the process can be further enhanced by applying supplementary knowledge distillation techniques, such as self-distillation \cite{Zhang_Song_etal._2019} or online distillation \cite{Houyon_Cioppa_etal._2023}.

Despite these promising results, further validation on independent datasets is necessary to fully understand the model's limitations. Underrepresented classes may pose challenges when addressing complex cases. Pathologists need to validate these models to adopt them in clinical settings. While the distilled models are smaller and more deployable, a technological gap persists because pathologists traditionally rely on established methods for inspecting WSIs and diagnosing diseases. Addressing the complexities involved in deploying models for inference and supporting pathologists in adopting new tools is essential for integrating these models into clinical workflows.

\section{Model integration with QuPath}
Digital pathology tools with graphical user interfaces are essential for visualizing and analyzing WSIs. To make our student model useful in clinical pathology workflows, it needs to be integrated into a tool that enables inspecting regions, creating annotations, and providing quantitative analyses of biomarkers. Therefore, we integrate the trained student model from the previous chapter into the QuPath open‑source platform \cite{Bankhead_Loughrey_etal._2017}. QuPath provides the required annotation, visualization, and analysis tools to interpret complex histological data, including workflows for cell segmentation, classification, and quantification (\hyperref[fig:fig7]{Figure 7}). 

\begin{figure}[h!]
    \centering
    \includegraphics[width=\textwidth]{images/Figure_7.pdf}
    \caption{Visualization of model-generated cell quantification annotations (left) and the corresponding unannotated slide (right) in QuPath}
    \label{fig:fig7}
\end{figure}

To identify the regions in a WSI critical for prognosticating tumor development, such as specific tumor areas or border regions without overlapping healthy tissue, the pathologist uses QuPath to outline these regions. Then, the pathologist initiates a cell segmentation and classification script through the QuPath interface for the selected regions. The resulting annotations and quantified cell information are then directly overlaid onto the WSI in the QuPath interface. Additional design and implementation details are in \hyperref[chap:S7]{Appendix S7}. 

Two common approaches for integrating deep learning models into QuPath are Java‑based native QuPath extensions \cite{Goldsborough_Philps_etal._2024} and the execution of RESTful API requests to a model server coupled with handling the response via an extension, as demonstrated in the application of cell segmentation models applied to immunofluorescence images \cite{Sugawara_2023}. While the community is actively working on these integration strategies, there is currently no universal solution that fully addresses all integration and performance requirements.

Extensions may offer better integration with QuPath, allowing slightly improved performance and more widespread usage of the built-in QuPath models, but they lack the flexibility to customize models and modify their behavior. For example, the newest version of QuPath includes models such as StarDist \cite{Weigert_Schmidt} and InstanSeg \cite{Goldsborough_Philps_etal._2024} that can perform cell segmentation. Both models pose limitations when applied to simultaneous cell segmentation and classification. StarDist performs well only on convex, round shapes by design, whereas some neoplastic, inflammatory, and connective cells exhibit complex and non-convex shapes. InstanSeg provides only semantic segmentation without assigning classes to the segmented cells.

%\hfill

In contrast, our approach offers an alternative integration strategy. It utilizes the paquo library to directly interact with QuPath’s internal application programming interface from within Python. This enables data exchange and processing without the need for intermediate conversion steps and provides greater control over model customization, retraining, and the incorporation of custom processing steps.

The integration of our custom model with QuPath underscores its potential to significantly enhance the diagnostic process by reducing the time burden on pathologists and enabling them to focus on more complex interpretative tasks using familiar software. Leveraging a tool that is already well-established among pathologists increases the likelihood of its adoption into daily clinical workflows. The quantitative data generated through the automated workflow is critical for both clinical decision-making and research, facilitating more accurate biomarker analysis, enabling robust statistical evaluations, and supporting hypothesis generation and testing. Additionally, by streamlining cell segmentation and classification, the tool enhances the scalability and reproducibility of pathological assessments, ultimately contributing to improved diagnostic accuracy and patient outcomes.

\section{Conclusion and future work}

In this study, we address critical challenges in digital pathology and tackle the usability and deployment issues of the developed models in standard computing environments without the need for high-performance computing systems. Our multi-faceted approach encompasses data refinement through cross-relabeling, leveraging foundation models for robust cell segmentation and classification, optimizing model performance via knowledge distillation, and integrating the optimized model into the QuPath software for practical application. This approach is used to construct a capable, versatile, and adjustable model for cell segmentation and classification, with enhanced performance and usability.

\begin{sloppypar}
While our approach shows potential in the field of computational pathology, certain limitations persist. 
For example, our implementation currently exhibits lower performance in detecting macrophages. 
This serves as an instance of the broader challenge of accurately identifying complex cell types. In order to address this issue, extending our approach to incorporate additional data sources, exploring alternative modeling approaches, and integrating other imaging modalities such as immunohistochemical staining may help improve detection accuracy. Moreover, although the distilled model reduces computational demands, integrating advanced deep learning models into clinical practice requires addressing technological gaps and potential resistance to adopting new tools within established diagnostic processes.
\end{sloppypar}

Future work could focus on several key areas to refine the proposed approach and facilitate its adoption in clinical environments. Enhancing the cell-relabeling process with additional datasets \cite{Graham_Jahanifar_etal._2021} could improve the representation of underrepresented cell types and enhance overall model performance. Also, incorporating additional data sources, such as multi-modal imaging or complementary staining methods, may address limitations related to cell type differentiation and class imbalance. Exploring other foundation models \cite{Vorontsov_Bozkurt_etal._2024,Zimmermann_Vorontsov_etal._2024} or introducing additional modalities \cite{Ding_Wagner_etal._2024,Vaidya_Zhang_etal._2025} may provide alternative architectures better suited to specific tasks or offer improved efficiency. Implementing more complex knowledge distillation techniques \cite{Houyon_Cioppa_etal._2023,Zhang_Song_etal._2019} could further optimize the model's performance and adaptability. Additionally, deeper integration with QuPath or other digital pathology software could provide pathologists more control over cell quantification analysis directly within the QuPath interface, thereby increasing accessibility and usability. Such enhancements would not only refine model performance but also ensure greater adaptability and scalability within various clinical environments. Finally, extensive validation of the model by pathologists and benchmarking against independent datasets are essential steps toward establishing the model's reliability and fostering confidence in its clinical utility.

\section*{Acknowledgments} 
This work was funded in part by the Research Council of Norway grant no. 309439 SFI Visual Intelligence, and the North Norwegian Health Authority grant no. HNF1521-20.

\bibliographystyle{IEEEtran}
\begin{sloppypar}
\begin{thebibliography}{99}

\bibitem{chaplot2020neural} Chaplot, Devendra Singh, et al. "Neural topological slam for visual navigation." Proceedings of the IEEE/CVF conference on computer vision and pattern recognition. 2020.

\bibitem{maksymets2021thda} Maksymets, Oleksandr, et al. "Thda: Treasure hunt data augmentation for semantic navigation." Proceedings of the IEEE/CVF International Conference on Computer Vision. 2021.

\bibitem{mezghan2022memory} Mezghan, Lina, et al. "Memory-augmented reinforcement learning for image-goal navigation." 2022 IEEE/RSJ International Conference on Intelligent Robots and Systems (IROS). IEEE, 2022.

\bibitem{al2022zero} Al-Halah, Ziad, Santhosh Kumar Ramakrishnan, and Kristen Grauman. "Zero experience required: Plug \& play modular transfer learning for semantic visual navigation." Proceedings of the IEEE/CVF Conference on Computer Vision and Pattern Recognition. 2022.

\bibitem{ye2021auxiliary} Ye, Joel, et al. "Auxiliary tasks and exploration enable objectgoal navigation." Proceedings of the IEEE/CVF international conference on computer vision. 2021.

\bibitem{chaplot2020object} Chaplot, Devendra Singh, et al. "Object goal navigation using goal-oriented semantic exploration." Advances in Neural Information Processing Systems 33 (2020)

\bibitem{ramakrishnan2022poni} Ramakrishnan, Santhosh Kumar, et al. "Poni: Potential functions for objectgoal navigation with interaction-free learning." Proceedings of the IEEE/CVF Conference on Computer Vision and Pattern Recognition. 2022.

\bibitem{ramrakhya2022habitat} Ramrakhya, Ram, et al. "Habitat-web: Learning embodied object-search strategies from human demonstrations at scale." Proceedings of the IEEE/CVF Conference on Computer Vision and Pattern Recognition. 2022.

\bibitem{mousavian2019visual} Mousavian, Arsalan, et al. "Visual representations for semantic target driven navigation." 2019 International Conference on Robotics and Automation (ICRA). IEEE, 2019.

\bibitem{dhariwal2021diffusion} Dhariwal, Prafulla, and Alexander Nichol. "Diffusion models beat gans on image synthesis." Advances in neural information processing systems 34 (2021)

\bibitem{ho2022classifier} Ho, Jonathan, and Tim Salimans. "Classifier-free diffusion guidance." arXiv preprint arXiv:2207.12598 (2022).

\bibitem{nichol2021glide} Nichol, Alex, et al. "Glide: Towards photorealistic image generation and editing with text-guided diffusion models." arXiv preprint arXiv:2112.10741 (2021)

\bibitem{brooks2023instructpix2pix} Brooks, Tim, Aleksander Holynski, and Alexei A. Efros. "Instructpix2pix: Learning to follow image editing instructions." Proceedings of the IEEE/CVF Conference on Computer Vision and Pattern Recognition. 2023.

\bibitem{fu2023guiding} Fu, Tsu-Jui, et al. "Guiding instruction-based image editing via multimodal large language models." arXiv preprint arXiv:2309.17102 (2023).

\bibitem{geng2024instructdiffusion} Geng, Zigang, et al. "Instructdiffusion: A generalist modeling interface for vision tasks." Proceedings of the IEEE/CVF Conference on Computer Vision and Pattern Recognition. 2024.

\bibitem{zhou2024minedreamer} Zhou, Enshen, et al. "Minedreamer: Learning to follow instructions via chain-of-imagination for simulated-world control." arXiv preprint arXiv:2403.12037 (2024).

\bibitem{zhou2023esc} Zhou, Kaiwen, et al. "Esc: Exploration with soft commonsense constraints for zero-shot object navigation." International Conference on Machine Learning. PMLR, 2023.

\bibitem{yu2023l3mvn} Yu, Bangguo, Hamidreza Kasaei, and Ming Cao. "L3mvn: Leveraging large language models for visual target navigation." 2023 IEEE/RSJ International Conference on Intelligent Robots and Systems (IROS). IEEE, 2023.

\bibitem{gadre2023cows} Gadre, Samir Yitzhak, et al. "Cows on pasture: Baselines and benchmarks for language-driven zero-shot object navigation." Proceedings of the IEEE/CVF Conference on Computer Vision and Pattern Recognition. 2023.

\bibitem{shah2023navigation} Shah, Dhruv, et al. "Navigation with large language models: Semantic guesswork as a heuristic for planning." Conference on Robot Learning. PMLR, 2023.

\bibitem{cai2024bridging} Cai, Wenzhe, et al. "Bridging zero-shot object navigation and foundation models through pixel-guided navigation skill." 2024 IEEE International Conference on Robotics and Automation (ICRA). IEEE, 2024.

\bibitem{yu2023co} Yu, Bangguo, Hamidreza Kasaei, and Ming Cao. "Co-NavGPT: Multi-robot cooperative visual semantic navigation using large language models." arXiv preprint arXiv:2310.07937 (2023).

\bibitem{wu2024voronav} Wu, Pengying, et al. "Voronav: Voronoi-based zero-shot object navigation with large language model." arXiv preprint arXiv:2401.02695 (2024).

\bibitem{qin2023mp5} Qin, Yiran, et al. "Mp5: A multi-modal open-ended embodied system in minecraft via active perception." arXiv preprint arXiv:2312.07472 (2023).

\bibitem{du2024learning} Du, Yilun, et al. "Learning universal policies via text-guided video generation." Advances in Neural Information Processing Systems 36 (2024).

\bibitem{ajay2024compositional} Ajay, Anurag, et al. "Compositional foundation models for hierarchical planning." Advances in Neural Information Processing Systems 36 (2024).

\bibitem{liang2024skilldiffuser} Liang, Zhixuan, et al. "Skilldiffuser: Interpretable hierarchical planning via skill abstractions in diffusion-based task execution." Proceedings of the IEEE/CVF Conference on Computer Vision and Pattern Recognition. 2024.

\bibitem{heusel2017gans} Heusel, Martin, et al. "Gans trained by a two time-scale update rule converge to a local nash equilibrium." Advances in neural information processing systems 30 (2017).

\bibitem{zhang2018unreasonable} Zhang, Richard, et al. "The unreasonable effectiveness of deep features as a perceptual metric." Proceedings of the IEEE conference on computer vision and pattern recognition. 2018.

\bibitem{brown2020language} Brown, Tom B. "Language models are few-shot learners." arXiv preprint arXiv:2005.14165 (2020).

\bibitem{podell2023sdxl} Podell, Dustin, et al. "Sdxl: Improving latent diffusion models for high-resolution image synthesis." arXiv preprint arXiv:2307.01952 (2023).

\bibitem{brohan2022rt} Brohan, Anthony, et al. "Rt-1: Robotics transformer for real-world control at scale." arXiv preprint arXiv:2212.06817 (2022).

\bibitem{brohan2023rt} Brohan, Anthony, et al. "Rt-2: Vision-language-action models transfer web knowledge to robotic control." arXiv preprint arXiv:2307.15818 (2023).

\bibitem{li2024manipllm} Li, Xiaoqi, et al. "Manipllm: Embodied multimodal large language model for object-centric robotic manipulation." Proceedings of the IEEE/CVF Conference on Computer Vision and Pattern Recognition. 2024.

\bibitem{shah2023vint} Shah, Dhruv, et al. "ViNT: A foundation model for visual navigation." arXiv preprint arXiv:2306.14846 (2023).

\bibitem{liu2024visual} Liu, Haotian, et al. "Visual instruction tuning." Advances in neural information processing systems 36 (2024).

\bibitem{hu2021lora} Hu, Edward J., et al. "Lora: Low-rank adaptation of large language models." arXiv preprint arXiv:2106.09685 (2021).

\bibitem{qin2023supfusion} Qin, Yiran, et al. "SupFusion: Supervised LiDAR-camera fusion for 3D object detection." Proceedings of the IEEE/CVF International Conference on Computer Vision. 2023.

\bibitem{qin2024worldsimbench} Qin, Yiran, et al. "Worldsimbench: Towards video generation models as world simulators." arXiv preprint arXiv:2410.18072 (2024).

\bibitem{yu2025gamefactory} Yu, Jiwen, et al. "GameFactory: Creating New Games with Generative Interactive Videos." arXiv preprint arXiv:2501.08325 (2025).

\bibitem{zhou2024code} Zhou, Enshen, et al. "Code-as-Monitor: Constraint-aware Visual Programming for Reactive and Proactive Robotic Failure Detection." arXiv preprint arXiv:2412.04455 (2024).

\bibitem{zhang2024ad} Zhang, Zaibin, et al. "AD-H: Autonomous Driving with Hierarchical Agents." arXiv preprint arXiv:2406.03474 (2024).

\bibitem{wang2024toward} Wang, Chaoqun, et al. "Toward Accurate Camera-based 3D Object Detection via Cascade Depth Estimation and Calibration." arXiv preprint arXiv:2402.04883 (2024).

\bibitem{huang2024story3d} Huang, Yuzhou, et al. "Story3d-agent: Exploring 3d storytelling visualization with large language models." arXiv preprint arXiv:2408.11801 (2024).

\bibitem{savinov2018semi} Savinov, Nikolay, Alexey Dosovitskiy, and Vladlen Koltun. "Semi-parametric topological memory for navigation." arXiv preprint arXiv:1803.00653 (2018).

\bibitem{majumdar2022zson} Majumdar, Arjun, et al. "Zson: Zero-shot object-goal navigation using multimodal goal embeddings." Advances in Neural Information Processing Systems 35 (2022): 32340-32352.

\bibitem{yadav2023offline} Yadav, Karmesh, et al. "Offline visual representation learning for embodied navigation." Workshop on Reincarnating Reinforcement Learning at ICLR 2023. 2023.

\bibitem{yadav2023ovrl} Yadav, Karmesh, et al. "Ovrl-v2: A simple state-of-art baseline for imagenav and objectnav." arXiv preprint arXiv:2303.07798 (2023).

\bibitem{sun2024fgprompt} Sun, Xinyu, et al. "FGPrompt: fine-grained goal prompting for image-goal navigation." Advances in Neural Information Processing Systems 36 (2024).

\bibitem{zhu2017target} Zhu, Yuke, et al. "Target-driven visual navigation in indoor scenes using deep reinforcement learning." 2017 IEEE international conference on robotics and automation (ICRA). IEEE, 2017.

\bibitem{koh2024generating} Koh, Jing Yu, Daniel Fried, and Russ R. Salakhutdinov. "Generating images with multimodal language models." Advances in Neural Information Processing Systems 36 (2024).

\bibitem{krantz2022instance} Krantz, Jacob, et al. "Instance-specific image goal navigation: Training embodied agents to find object instances." arXiv preprint arXiv:2211.15876 (2022).

\bibitem{schulman2017proximal} Schulman, John, et al. "Proximal policy optimization algorithms." arXiv preprint arXiv:1707.06347 (2017).

\bibitem{anderson2018evaluation} Anderson, Peter, et al. "On evaluation of embodied navigation agents." arXiv preprint arXiv:1807.06757 (2018).

\bibitem{lin2024navcot} Lin, Bingqian, et al. "NavCoT: Boosting LLM-Based Vision-and-Language Navigation via Learning Disentangled Reasoning." arXiv preprint arXiv:2403.07376 (2024).

\bibitem{NavGPT} Zhou, Gengze, Yicong Hong, and Qi Wu. "Navgpt: Explicit reasoning in vision-and-language navigation with large language models." Proceedings of the AAAI Conference on Artificial Intelligence.

\bibitem{hahn2021no} Hahn, Meera, et al. "No rl, no simulation: Learning to navigate without navigating." Advances in Neural Information Processing Systems 34 (2021): 26661-26673.

\bibitem{li2025t2isafety} Li, Lijun, et al. "T2ISafety: Benchmark for Assessing Fairness, Toxicity, and Privacy in Image Generation." arXiv preprint arXiv:2501.12612 (2025).

\bibitem{an2024agfsync} An, Jingkun, et al. "AGFSync: Leveraging AI-Generated Feedback for Preference Optimization in Text-to-Image Generation." arXiv preprint arXiv:2403.13352 (2024).


\end{thebibliography}
\end{sloppypar}

\clearpage
\beginsupplement
\section*{Appendix}
\renewcommand{\thesubsection}{S\arabic{subsection}}

\subsection{\label{chap:S1}PanNuke and MoNuSAC preprocessing}
The PanNuke dataset comprises a set of 7,901 RGB patches, each with dimensions of $256 \times 256$ pixels, which we set as the standard patch size for our analysis. In contrast, the MoNuSAC dataset encompasses 294 images of heterogeneous dimensions. To standardize the MoNuSAC images with our experiments, we implement a standardization protocol. Specifically, for images exceeding the dimensions of $256 \times 256$ pixels, we segment them into equal-sized patches and apply mirror padding to the remaining portions to avoid information loss at the peripherals. Patches with dimensions less than $128 \times 128$ pixels are excluded from the dataset due to the insufficient resolution to capture relevant cellular details. For patches where either dimension falls between 128 and 256 pixels, we employ upsampling to achieve the standard patch size. As a result, we obtain a total of 2,823 RGB patches derived from the MoNuSAC dataset for subsequent analysis. For additional details on the MoNuSAC data preparation process, refer to the source code \cite{Shvetsov_2025a}.
\clearpage

\subsection{\label{chap:S2}Data usage for the methodology}

\counterwithin{figure}{subsection}
\renewcommand{\thefigure}{S\arabic{subsection}}

\begin{figure}[h!]
    \centering
    \includegraphics[width=\textwidth, height=0.85\textheight, keepaspectratio]{images/A2.pdf}
    \caption{Overview of the methodology for cross-labeling, dataset refinement, and model comparison. (1) Cross-relabeling - training and testing cell classification models, (2) Cross-relabeling - using cell classification models to create refined dataset, (3) Fine-tuning and training models for comparison, (4) Student knowledge distillation with refined dataset}
    \label{fig:S2}
\end{figure}
\clearpage

\subsection{\label{chap:S3}Confusion matrices for classification models}
\counterwithin{figure}{subsection}
\renewcommand{\thefigure}{S\arabic{subsection}.\arabic{figure}}

\begin{figure}[h!]
    \centering
    \includegraphics[width=\textwidth, height=0.4\textheight, keepaspectratio]{images/A3_1.pdf}
    \caption{Confusion matrix for PanNuke trained model}
    \label{fig:S3.1}
\end{figure}

\begin{figure}[h!]
    \centering
    \includegraphics[width=\textwidth, height=0.4\textheight, keepaspectratio]{images/A3_2.pdf}
    \caption{Confusion matrix for MoNuSAC trained model}
    \label{fig:S3.2}
\end{figure}

\clearpage

\subsection{\label{chap:S4}Datasets cell counts}

\counterwithin{table}{subsection}
\renewcommand{\thetable}{S\arabic{subsection}}

\begin{table}[h!]
\renewcommand{\arraystretch}{2.0}
\centering
\caption{\label{tab:S4}Cell counts for PanNuke, MoNuSAC and refined datasets. Numbers in parentheses indicate preprocessed cell counts for cell classifier models training and testing.}
%\adjustbox{max width=\textwidth}{%
\begin{tabular}{|l|c|c|c|}
\hline
%\rowcolor{gray!30}
Cell type & PanNuke & MoNuSAC & Refined \\
\hline
Neoplastic & 77,403 (68,031) & - & 105,451 \\
\hline
Epithelial & 26,572 (23,207) & - & 29,926 \\
\hline
Epithelial (benign and malignant) & - & 31,402 & - \\
\hline
Inflammatory & 32,276 & - & - \\
\hline
Lymphocytes & - & 37,045 (33,104) & 65,275 \\
\hline
Neutrophils & - & 1,355 (1,252) & 3,833 \\
\hline
Macrophage & - & 1,842 (1,695) & 3,410 \\
\hline
Dead & 2,908 & - & 2,908 \\
\hline
Connective & 50,585 & - & 50,585 \\
\hline
\end{tabular}
%
%}
\end{table}



\clearpage

\subsection{\label{chap:S5}Definition of validation metrics}
\counterwithin{equation}{subsection}
\renewcommand{\theequation}{\arabic{equation}}

\subsubsection{\label{chap:S5.1}R\textsuperscript{2}}
The coefficient of determination, denoted as $R^2$, is a statistical measure that represents the proportion of variance in the dependent variable that is predictable from the independent variables. In the context of cell quantification in pathology, $R^2$ is used to assess how well the predicted quantities of different cell types in a patch align with the actual quantities observed in the ground truth data, with higher values representing more accurate quantification. $R^2$ is defined as
\begin{equation*}
R^2 = 1 - \frac{\sum_{i=1}^n (y_i - \hat{y}_i)^2}{\sum_{i=1}^n (y_i - \bar{y})^2},
\end{equation*}
where $y_i$ represents the actual number of cells of a specific type in the $i$-th image, $\hat{y}_i$ represents the predicted number of cells of that type in the $i$-th image, $\bar{y}$ is the mean of the actual numbers across all images, and $n$ is the total number of images in the dataset.

The $R^2$ metric has a range of $(-\infty, 1]$. An $R^2$ of 1 indicates perfect prediction, where all predicted values exactly match the actual values. An $R^2$ of 0 suggests that the model explains none of the variability of the response data around its mean. If $R^2$ is negative, it indicates that the model performs worse than a model that simply predicts the mean of the actual values for all observations.

\subsubsection{\label{chap:S5.2}PQ}
Panoptic Quality ($PQ$) is a comprehensive metric used to evaluate the performance of segmentation models in tasks that require both instance segmentation and classification. $PQ$ provides a single score that encapsulates both the detection accuracy (i.e., how many objects were correctly identified) and the segmentation quality (i.e., how accurately the objects' boundaries were delineated). This metric is particularly useful in multiclass scenarios where each pixel is classified into distinct categories, such as different cell types in pathology images.

$PQ$ is calculated as the product of two terms: Detection Quality ($DQ$) and Segmentation Quality ($SQ$). It can be expressed as
\begin{equation*}
PQ = DQ \cdot SQ,
\end{equation*}
where
\begin{equation*}
DQ = \frac{TP}{TP + 0.5\, FP + 0.5\, FN},
\end{equation*}
\begin{equation*}
SQ = \frac{\sum_{(p, g) \in \mathcal{M}} IoU(p, g)}{TP}.
\end{equation*}
In these formulas, $TP$ denotes the number of correctly matched instances between ground truth and prediction, $FP$ denotes the predicted instances that have no corresponding ground truth, $FN$ denotes the ground truth instances that were not detected, $IoU(p, g)$ is the Intersection over Union for a pair of matched instances $p$ (prediction) and $g$ (ground truth), and $\mathcal{M}$ is the set of matched pairs.

The $PQ$ metric is calculated for each class and is averaged across classes to provide a global performance measure.

The $PQ$ score has a range of $[0, 1.0]$, where a higher score indicates better performance in both detecting and segmenting the instances correctly. A $PQ$ of 1 signifies perfect identification and segmentation of all instances, whereas a $PQ$ of 0 indicates that no instances were correctly identified and segmented.

\clearpage

\subsection{\label{chap:S6}Segmentation and Detection quality metrics for teacher and student models}

\begin{table}[h!]
\renewcommand{\arraystretch}{2.0}
\centering
\caption{Segmentation and detection quality for student and teacher models (CI 95\%)}
\label{tab:S6}
%\adjustbox{max width=\textwidth}{%
\begin{tabular}{|l|c|c|}
\hline
%\rowcolor{gray!30}
Metric & Teacher & Student \\
\hline
$SQ_{neoplastic}$ & 0.819 (0.815--0.823) & 0.824 (0.819--0.828) \\
\hline
$SQ_{lymphocyte}$ & 0.795 (0.788--0.802) & 0.790 (0.783--0.796) \\
\hline
$SQ_{connective}$ & 0.770 (0.762--0.776) & 0.780 (0.772--0.786) \\
\hline
$SQ_{dead}$ & 0.659 (0.623--0.688) & 0.657 (0.624--0.695) \\
\hline
$SQ_{epithelial}$ & 0.780 (0.770--0.790) & 0.788 (0.779--0.797) \\
\hline
$SQ_{macrophage}$ & 0.788 (0.760--0.810) & 0.757 (0.730--0.783) \\
\hline
$SQ_{neutrofil}$ & 0.782 (0.761--0.801) & 0.775 (0.759--0.792) \\
\hline
$DQ_{neoplastic}$ & 0.706 (0.692--0.719) & 0.727 (0.712--0.741) \\
\hline
$DQ_{lymphocyte}$ & 0.675 (0.656--0.698) & 0.713 (0.691--0.734) \\
\hline
$DQ_{connective}$ & 0.566 (0.546--0.584) & 0.583 (0.565--0.602) \\
\hline
$DQ_{dead}$ & 0.410 (0.361--0.465) & 0.435 (0.306--0.561) \\
\hline
$DQ_{epithelial}$ & 0.668 (0.639--0.694) & 0.673 (0.644--0.702) \\
\hline
$DQ_{macrophage}$ & 0.657 (0.583--0.727) & 0.615 (0.531--0.703) \\
\hline
$DQ_{neutrofil}$ & 0.691 (0.625--0.753) & 0.729 (0.679--0.778) \\
\hline
\end{tabular}
%
%}
\end{table}

\clearpage

\subsection{\label{chap:S7}QuPath integration method}
We adopt an integration strategy leveraging the paquo \cite{Bayer_AG} library, a Python package that enables direct interaction with QuPath’s internal API, thereby facilitating seamless data exchange without intermediate conversion steps. The data processing pipeline (\hyperref[fig:S7]{Appendix Figure S7}) begins with the acquisition of WSIs and their associated annotations from QuPath, which are represented as Shapely \cite{Gillies_Wel_etal._2024} polygons. Utilizing paquo, we directly read, create, and modify these annotations and detections within a QuPath project in the Python environment. Images are then cropped using these polygons and processed by cell segmentation and classification models employing standard vision processing toolkits such as OpenCV, pyvips, and PyTorch. Additionally, QuPath employs Groovy scripts to initiate a Python process that starts the entire pipeline from QuPath graphical interface: fetching polygons, extracting images from them, and running deep learning model inference on the cropped images. 
The results are returned to QuPath, leveraging paquo's Python bindings to manipulate QuPath data while minimizing the computational overhead typically associated with cross-environment communication.

\counterwithin{figure}{subsection}
\renewcommand{\thefigure}{S\arabic{subsection}}

\begin{figure}[h!]
    \centering
    \includegraphics[width=\textwidth]{images/A7.pdf}
    \caption{QuPath integration workflow using Python environment}
    \label{fig:S7}
\end{figure}

Compared to traditional workflows that involve exporting annotations as GeoJSON, classifying them in Python, and reimporting them into QuPath, our approach offers several advantages. We eliminate the need to switch between programming languages, providing a cohesive and streamlined development process entirely within QuPath software and removing the necessity to use other tools. Meanwhile, we avoid storing annotations as intermediate JSON files unless required for external use or archiving. By conducting the entire inference and post-processing workflow within the Python environment, we leverage the power and flexibility of Python libraries for image processing and machine learning. This approach also enables adjustments to any set of labels and models, thereby improving its applicability.

%\hfill

The distilled model and QuPath integration code are packaged into a Docker container, enabling streamlined execution with the Docker engine. Detailed integration code and deployment instructions can be found in the GitHub repository \cite{Shvetsov_2025b}.

Despite these benefits, we acknowledge that the paquo library is a proof‑of‑concept project in its early development stage and has not been tested across all versions of QuPath.

\clearpage

\subsection{\label{chap:S8}Data and code availability statement}
All datasets, models, and code used in this study are publicly available and can be obtained from the repositories listed below. 
The PanNuke \cite{Gamper_Koohbanani_etal._2019} and MoNuSAC \cite{Verma_Kumar_etal._2021} datasets are publicly accessible, and download information along with detailed descriptions can be found in their respective articles. Preprocessing scripts for PanNuke and MoNuSAC data, as well as individual cell extraction scripts, are available on GitHub \cite{Shvetsov_2025a}. The H-Optimus foundation model used in our experiments can be downloaded from the HuggingFace repository \cite{hoptimus2024}, and model information is available on GitHub \cite{Saillard_Jenatton_etal._2024}. In addition, the integration code for QuPath and the distilled model packaged in a Docker container are provided in the repository \cite{Shvetsov_2025b}, and paquo Python library is available from the authors GitHub repository \cite{Bayer_AG}.
\clearpage

\end{document}








% Acknowledgments---Will not appear in anonymized version
% \begin{credits}
% \subsubsection{\ackname}
% The work of G. Bahar, O. Ben-Porat and M. Tennenholtz is funded by the European Research Council (ERC) under the European Union's Horizon 2020 research and innovation programme (grant agreement n$\degree$  740435). The work of K. Leyton-Brown is funded by the NSERC Discovery Grants program, DND/NSERC Discovery Grant Supplement, Facebook Research and Canada CIFAR AI Chair Amii. Part of this work was done while K. Leyton-Brown was a visiting researcher at Technion - Israeli Institute of Science and was partially funded by the European Union's Horizon 2020 research and innovation programme (grant agreement n$\degree$  740435).
% \end{credits}

{\ifnum\Includeappendix=1{ %statrting appendices
\appendix

\section{Proof Outline for Theorem \ref{thm:optimal policy}}\label{sec:thm1 outline}
In this section, we outline the proof of Theorem~\ref{thm:optimal policy}. 
To allow this section to be self-explanatory, we reiterate some definitions that appear in the body of the paper. We begin with several notations and definitions we use extensively in the proof.
\subsection{Preliminaries}
We denote the set of all states by $\mS=2^A$. A \textit{policy} is a mapping from previous states and actions to a randomized action. Formally, let $\mH$ be the set of histories, $\mH=\cup_{k=0}^K\left(\mS \times \Delta(A)\right)^k$ be a tuple comprising pairs of states and randomized actions taken. A policy $\pi$ is a function $\pi:\mH \times \mS \rightarrow \Delta(A)$. We say that a policy is \textit{IR} if for every $h\in \mH,s\in \mS$, $\pi(h,s)\in \safe(s)$. From here on, we consider IR policies solely. Given a policy $\pi$ and a pair $(h,s)$, we let $W(\pi,h,s)$ denote the expected reward of $\pi$ when starting from $s$ after witnessing $h$. Namely,
\begin{align}\label{eq:W elaborated}
W(\pi,h,s) = 
\begin{cases}
R(s) & \textnormal{if }\safe(s)=\emptyset\\
\sum_{a\in s}\pi(h,s)(a)W(\pi,h\oplus(s,a),s\setminus \{a\}) & \textnormal{otherwise}
\end{cases}.
\end{align}
For every state $s$, let $W^\star(s)=\sup_{\pi'}W(\pi ',\emptyset,s)$.\footnote{As we show in Proposition \ref{prop:optimal p valid}, there exists a policy that attains this supremum.} While policies may depend on histories, it often suffices to consider \emph{stationary} policies.
\begin{definition}[Stationary]
An IR policy $\pi$ is \textit{stationary} if for every two histories $h,h' \in \mH$ and a state $s \in \mS$, $\pi(h,s)=\pi(h',s)$.
\end{definition}
Since there is a finite set of states and the action sets are convex, there exists %As we show later in Proposition \ref{prop:optimal p valid}, there exists 
an optimal stationary policy; hence, from here on we address stationary policies solely. When discussing stationary policies, we thus neglect the dependency on $h$, writing $\pi(s)$. For stationary policies, the definition of $W$ is much more intuitive: Given a stationary policy $\pi$ and a state $s$,
\begin{align}\label{eq:w with terminal}
W(\pi,s)=\sum_{\substack{s_t\in \mS:\\ s_t \textnormal{ is terminal}}}\Pr(\pathto{s}{s_t})R(s_t),
\end{align}
where $\pathto{s}{s_t}$ indicates the event that, starting from $s$ and following the actions of $\pi$, the GMDP terminates at $s_t$. 

An additional useful notation is the following. For every state $s\in \mS$, we denote by $Q(\pi,s)$ the probability starting at $s\subseteq A$, following the policy $\pi$ and exploring all arms. Formally,
\[
Q(\pi,s)= \Pr(\pathto{s}{\emptyset}),
\]
Note that $Q$ is defined recursively: Namely, if $\pi(s)=\bl p_{i,j}$ for a non-terminal state $s$, then
\[
Q(\pi,s)=\bl p_{i,j}(a_i)Q(\pi,s\setminus\{a_i\})+\bl p_{i,j}(a_j)Q(\pi,s\setminus\{a_j\}).
\]
It will sometimes be convenient to denote $Q(\pi,\above(s),\below(s))$ for $Q(\pi,s)$, thereby explicitly stating the two distinguished sets of arms. 


\subsection{Binary Structure}\label{subsec:bin}
A structural property of the above GMDP is that in every terminal state $s_t$, $\above(s_t)=\emptyset$, or otherwise we could explore more arms; thus, intuitively, the arms in $\above(A)$ provide us ``power'' to explore the arms of $\below(A)$. Following this logic, in every state we should aim to explore arms from $\below(A)$ and not those of $\above(A)$, subject to satisfying the IR constraint.

Recall the definition of $\bl p_{i,j}$ and $\bl p_{i,i}$ from Equation~\eqref{eq:blp from body} in Subsection~\ref{subsec:optimal GMDP policy}. Next, we define $\mP,\mP'$ such that
\[
\mP\defeq \{\bl p_{i,j}\mid a_i\in \above(A), a_j\in \below(A) \}, \qquad \mP'\defeq\{\bl p_{i,i}\mid a_i\in \above(A) \}.
%\mP\defeq \{\bl p_{i,j}\mid a_i\in \above(A), a_j\in \below(A) \}\cup\{\bl p_{i,i}\mid a_i\in \above(A) \}.
\]
Notice that $\mP\cup \mP'$ includes $O(K^2)$ actions, while $\safe(s)$ for a state $s$ is generally a convex polytope  with infinitely many actions. Further, in every non-terminate state $s$, ${\safe(s) \cap (\mP \cup \mP') \neq \emptyset}$. Next, we remind the reader of the definition of $\mP$-valid policies.
\begin{definition}[Mirroring Definition~\ref{def:p valid}]
An IR policy $\pi$ is $\mP$-valid if for every non-terminal state $s\in \mS$,
\begin{itemize}
\item if $\below(s)\neq \emptyset$, then $\pi(s)\in \mP$;
\item else, if $\below(s)= \emptyset$, then $\pi(s)\in \mP'$.
\end{itemize} 
\end{definition}
Observe that $\mP$ is a strict subset of all IR actions in the state $A$, which incorporate mixes of at most two arms. However, the set of IR actions $\safe(s)$ for $s\subseteq A$ may include distributions mixing several elements of $A$. Due to the convexity of $W(\pi,s)$ in $\pi(s)$ (see the elaborated representation of $W$ in Equation \refeq{eq:W elaborated}), the GMDP exhibits a nice structural property, as captured by the following Proposition~\ref{prop:optimal p valid}. 
\begin{proposition}[Mirroring Proposition~\ref{prop:main optimal p valid}]\label{prop:optimal p valid}
There exists an optimal policy that is $\mP$-valid.
\end{proposition}
The proof of Proposition \ref{prop:optimal p valid} appears in Section~\ref{sec:aux}. %~{\ifnum\Includeappendix=0{the appendix}\else{Subsection \ref{subsec:proof of propo p valid}}\fi}. 
Due to Proposition \ref{prop:optimal p valid}, we shall focus on $\mP$-valid policies. Such policies are easy to visualize using trees, as we exemplify next.\footnote{In Figure~\ref{example with normal} we illustrated the optimal policy using a graph that is not a tree. However, a tree structure  serves better the presentation of our technical statements.}
\begin{example}\label{example with four}
We reconsider Example~\ref{example with normal}, but neglect the actual distributions (as we only care about the expected values). Let $A=\{a_1,a_2,a_3,a_4\}$, with $\above(A)=\{a_1,a_2\}$ and $\below(A)=\{a_3,a_4\}$. Consider the tree description in Figure \ref{fig:tree example}. The root of the tree is the set of all arms. At the root, the policy picks $\bl p_{1,3}$. The outgoing left edge represents the case the realized action is $a_1$, which happens w.p. $\bl p_{1,3}(a_1)$. In such a case, the new state is $\{a_2,a_3,a_4\}$. With the remaining probability, $\bl p_{1,3}(a_3)$, the new state will be $\{a_1,a_2,a_4\}$. Leaves of the tree are terminal states, where no further exploration could be done. For instance, in the leftmost leaf, $\{a_3,a_4\}$, the only arms explored are $\{a_1,a_2\}$. The two highlighted nodes represent the same state. Since the presented policy is $\mP$-valid, it is stationary; hence, the policy acts exactly the same in these two nodes and their sub-trees.
\end{example}
Notice that the tree in Figure \ref{fig:tree example} represents only the \textit{on-path} states, i.e., states that are reachable with positive probability, while policies are functions from the entire space of states, including off-path states; thus, two different policies can be described using the same tree. Nevertheless, the tree structure is convenient and will be used extensively in our analysis. When we define a policy using a tree, we shall also describe its behavior at \textit{off-path} states.


The policy exemplified in Figure \ref{fig:tree example} has an additional combinatorial property: In every state $s$, it takes an action according to some order of the arms. This property is manifested in the following Definitions \ref{def:right ordered} and \ref{def:left ordered}.
\begin{definition}[Right-ordered policy]\label{def:right ordered}
A $\mP$-valid policy $\pi$ is right-ordered if there exist a bijection $\sigr_\pi: \below(A)\rightarrow [\abs{\below(A)}]$ such that in every state $s$ with $\below(s) \neq \emptyset$, $\pi(s)=\bl p_{i, {j^*}}$ where $a_i \in \above(s)$ and $a_{j^*} = \argmin_{a_j \in \below(s)} \sigr_\pi(a_j)$.
\end{definition}
\begin{definition}[Left-ordered policy]\label{def:left ordered}
A $\mP$-valid policy $\pi$ is left-ordered if there exist a bijection $\sigl_\pi: \above(A)\rightarrow [\abs{\above(A)}]$ such that in every state $s$ with $\below(s) \neq \emptyset$, $\pi(s)=\bl p_{{i^*}, j}$ where $a_j \in \below(s)$ and $a_{i^*} = \argmin_{a_i \in \above(s)} \sigl_\pi(a_i)$.
\end{definition}
In addition, we say that a policy is \textit{ordered} if it is right-ordered and left-ordered. To illustrate, observe the example in Figure \ref{fig:tree example}. The tree depicts an ordered policy, with $\sigl=(a_1,a_2)$ and $\sigr=(a_3,a_4)$. Notice that ordered policies are well-defined for off-path states.

\begin{figure}[t]
\centering
\includegraphics[scale=0.9]{figs/tree_example_crop}% label
\caption{The policy described in Example \ref{example with four}.  
Every node represents a state (the mapping is onto, but not one-to-one). Outgoing left edges imply the coin flips resulted in an arm from $\above(v)$, and outgoing right edges imply an arm from $\below(v)$. Leaves correspond to terminal states, where no action could be taken. \label{fig:tree example}
}% caption command
\end{figure}

\iffalse
\subsection{Stochastic Dominance and Non-triviality}\label{subsec:stopchastic}
In this subsection, we demonstrate why the problem is still challenging even under Assumption \ref{assumption:dominance}. Recall that Proposition \ref{prop:optimal p valid} ensures that an optimal $\mP$-valid policy exists. One natural candidate for the optimal policy is ordered policy $\pi$ with with any order $\sigl_\pi$ and $\sigr_\pi$ that follow the stochastic order on $\below(A)$. Indeed, as we show formally in Theorem~\ref{thm:holy grail}, this intuition is appropriate. However, we explain shortly, the optimally of this policy cannot be shown without further work.

Consider a state $s\in\mS$, such that $\above(s),\below(s) \geq 2$. Let $a_i = \argmin_{a_{i'}\in\above(s)}\sigl_\pi(a_{i'})$, and  $a_j = \argmin_{a_{j'}\in\below(s)}\sigr_\pi(a_{j'})$. In addition, let $a_{\tilde j}\in \below(s), a_{\tilde j} \neq a_j$. The action $\bl p_{i,j}$, which mixes the minimal elements according to the stochastic order, is weakly superior to  $\bl p_{i,{\tilde{j}}}$ if 
{
\thinmuskip=.2mu
\medmuskip=0mu plus .2mu minus .2mu
\thickmuskip=1mu plus 1mu
\begin{align}\label{eq:why hard}
&\bl p_{i,j}(j)W^\star(s\setminus \{a_j\})-\bl p_{i,{\tilde j}}({\tilde j})W^\star(s\setminus \{a_{\tilde j}\})+\left(\bl p_{i,j}(i)-\bl p_{i,{\tilde j}}(i)\right)W^\star(s\setminus\{a_i\}) \geq 0.
\end{align}}%
By our selection of $a_j,a_{\tilde j}$, we know that $\bl p_{i,j}(i)-\bl p_{i,{\tilde j}}(i) \leq 0$; hence, the third term is non-positive. Moreover, as we show in Claim \ref{claim:ass is not for W} in Section~\ref{sec:aux}, %~{\ifnum\Includeappendix=0{the appendix,}\else{Claim \ref{claim:ass is not for W} in Subsection \ref{subsec: statments},}\fi}
stochastic dominance does not imply that $W^\star(s\setminus \{a_j\}) \geq W^\star(s\setminus \{a_{\tilde j}\})$; thus, it is not even clear that the expression
{
\thinmuskip=.2mu
\medmuskip=0mu plus .2mu minus .2mu
\thickmuskip=1mu plus 1mu
\begin{align*}
&\bl p_{i,j}(j)W^\star(s\setminus \{a_j\})-\bl p_{i,{\tilde j}}({\tilde j})W^\star(s\setminus \{a_{\tilde j}\})
\end{align*}}%
which accounts for the first two terms in Inequality (\ref{eq:why hard}), is non-negative. Therefore, we cannot claim for Inequality  (\ref{eq:why hard}) without revealing the structure of $W^\star$, even when Assumption \ref{assumption:dominance} holds. 
 %We discuss relaxing this assumption in Section \ref{??}
 \fi
\subsection{Proof Overview}\label{subsec:results}
We are ready to prove Theorem~\ref{thm:optimal policy}. The main tool in our analysis is Lemma \ref{lemma:equivalence}. Lemma \ref{lemma:equivalence} reveals a rather surprising feature of $Q$: $Q$ is policy independent. 
\begin{lemma}[Mirroring Lemma~\ref{lemma:equivalence body}]\label{lemma:equivalence}
For every two $\mP$-valid policies $\pi,\rho$ and every state $s\in \mS$, it holds that $Q(\pi,s)=Q(\rho,s)$.
\end{lemma}
The proof of Lemma \ref{lemma:equivalence} appears in~{\ifnum\Includeappendix=0{the appendix}\else{Section \ref{sec:proof of lemma}}\fi}. We stress that this lemma holds regardless of Assumption~\ref{assumption:dominance}. Next, we leverage Lemma \ref{lemma:equivalence} to prove the main technical result of the paper.
\begin{theorem}
\label{thm:holy grail}
Let $\pi^\star$ be a right-ordered, $\mP$-valid policy with $\sigr_{\pi^\star}$ ordered in decreasing expected value. Under Assumption \ref{assumption:dominance}, for every state $s\in \mS=2^A$, it holds that $W(\pi^\star,s)=W^\star(s)$.
\end{theorem}
In particular, Theorem \ref{thm:holy grail} implies that $W(\SEGB,s_0)=W^\star(s_0)$ and that Theorem~\ref{thm:optimal policy} holds,  since $\SEGB$ is right-ordered in decreasing expected value. The formal proof of Theorem \ref{thm:holy grail} is relegated to~{\ifnum\Includeappendix=0{the appendix}\else{Section~\ref{sec:proof of thm}}\fi}. 

\section{Proof of Lemma \ref{lemma:equivalence}}\label{sec:proof of lemma}
\begin{proofof}{Lemma \ref{lemma:equivalence}}
We prove the lemma by a two-dimensional induction on the number of arms in $\above(s)$ and $\below(s)$. We prove four base cases in Section \ref{sec:base for lemma}:
\begin{itemize}
\item $\abs{\above(s)}=1$ and $\abs{\below(s)}\geq 2$ (Proposition \ref{prop:case of one strong}).
\item $\abs{\above(s)}\geq 2 $ and $\abs{\below(s)} = 1$ (Proposition \ref{prop:case of one}).
\item $\abs{\above(s)}\geq 2$ and $\abs{\below(s)} = 2$ (Proposition \ref{prop:case of two strong}).
\item $\abs{\above(s)}=2$ and $\abs{\below(s)}\geq 2$ (Proposition \ref{prop:case of two}).
\end{itemize}
While the first two cases are almost immediate, the other two are technical and require careful attention. Next, assume the statement holds for all states $s\in \mS$ such that $\abs{\above(s)}\leq K_1$, $\abs{\below(s)}\leq K_2$ and $\abs{\above(s)}+\abs{\below(s)}< K_1+K_2$.


Let $U\in\mS$ denote a state with $\abs{\above(U)}=K_1$ and $\abs{\below(U)}=K_2$. For abbreviation, let $\ug\defeq\above(U),\ul\defeq\below(U)$. Further, define $Q^*(U) = \sup_{\pi} Q(\pi,U)$, \footnote{This supremum is attained since there are only finitely many $\mP$-valid policies.} and for every $a_i \in \ug, a_j\in \ul$ let 
\[
Q^*_{i,j}(\ug,\ul)\defeq\bl p_{i,j}(a_j)Q^*(\ug,\ul\setminus \{a_j\}) +\bl p_{i,j}(a_i)Q^*(\ug\setminus\{a_i\},\ul).
\]
Next, let $(a_{i^*},a_{j^*})\in \argmax_{a_i\in \ug,a_j\in \ul}Q^*_{i,j}(\ug,\ul)$, and assume by contradiction that there exists a pair $(a_{\tilde i}, a_{\tilde j})$ such that 
\begin{equation}\label{eq:contradiction of lemma}
Q^*_{{i^*},{j^*}}(U) > Q^*_{{\tilde i}, {\tilde j}}(U).
\end{equation}
\paragraph{Step 1} Fix arbitrary $a_{i'}$ and $a_{j'}$ such that  $a_{i'} \in \ug$ and $a_{j'} \in \ul$. We will show that 
\begin{equation}\label{eq:step 1 goal}
Q^*_{{i'},{j^*}}(U)=Q^*_{{i'},{j'}}(U).
\end{equation}
We define the ordered policy $\pi$ such that $\sigr_\pi=(a_{i'},\dots)$, i.e., $\sigr_\pi$ first explores $a_{i'}$ and then the rest of the arms of $\ug$ in some arbitrary order; and, $\sigl_\pi=(a_{j^*},a_{j'},\dots)$. In addition, we define $\rho$ such that $\sigl_{\rho}=\sigl_\pi$, and $\sigr_\rho=(a_{j'},a_{j^*},\dots)$. Due to the inductive assumption, we have 
\begin{align}\label{eq:policies suffice}
Q^*_{{i'},{j^*}}(U) &=  \bl p_{{i'},{j^*}}(a_{j^*})Q^*(\ug,\ul\setminus \{a_{j^*}\}) +\bl p_{{i'},{j^*}}(a_{i'})Q^*(\ug\setminus \{a_{i'}\},\ul) \nonumber \\
&= \bl p_{{i'},{j^*}}(a_{j^*})Q(\pi,\ug,\ul\setminus \{a_{j^*}\})+\bl p_{{i'},{j^*}}(a_{i'})Q(\pi,\ug\setminus \{a_{i'}\},\ul)\\
&=Q(\pi,U).\nonumber
\end{align}
Similarly, $Q^*_{{i'},{j'}}(U)  = Q(\rho,U)$; hence, proving that $Q(\pi,U) =Q(\rho,U)$ entails Equality (\ref{eq:step 1 goal}). Next, let $\suff(\sigl_\pi)$ be the set of all non-empty suffices of $\sigl_\pi$. Being left-ordered suggests that on-path\footnote{These are terminal states that $\pi$ reaches to with positive probability.} terminal states with all arms of $\ul$ explored of $\pi$ are of the form $(Z,\emptyset)$, where $Z\in \suff(\sigl_\pi)$. Next, we factor $Q(\pi,U)$ recursively as follows: We factor $Q(\pi,U)$ into two terms, like in Equation \refeq{eq:policies suffice}. Following, for each term obtained, we ask whether the corresponding state excludes $\{a_{j^*},a_{j'}\}$. If the answer is yes, we stop factorizing it, and move to the other terms. We do this recursively, until we cannot factor anymore, or we reached a terminal state. Using this factorizing process, we have \footnote{We stop factorizing if both $a_{j^*},a_{j'}$ were observed; thus, $Z$ will never be the empty set.} 
\begin{align*}
Q(\pi,U) &= \alpha \cdot Q(\pi,\emptyset,\ul)+\beta\cdot Q(\pi,\emptyset,\ul\setminus \{a_{j^*} \}) + \sum_{Z \in \suff(\sigl_\pi)} c^\pi_Z \cdot Q(\pi,Z,\ul\setminus \{a_{j^*},a_{j'}\}),
\end{align*}
for $\alpha=\Pr(\pathto{s}{(\emptyset,\ul)})$ and $\beta=\Pr(\pathto{s}{(\emptyset,\ul\setminus\{a_{j^*}\})})$ such that $\alpha+\beta+\sum_{Z\in \suff(\sigl_\pi)}c^\pi_Z =1$ and $\alpha,\beta,c^\pi_Z\in [0,1]$ for every $Z \in \suff(\sigl_\pi)$. In this representation, $\alpha$ is the probability of reaching the terminal $(\emptyset,\ul)$, while $\beta$ is the probability of reaching the terminal state $(\emptyset,\ul\setminus\{a_{j^*}\})$. For these two terminal states, we know that $Q^*(\emptyset,\ul)= Q^*(\emptyset,\ul\setminus \{a_{j^*} \})=0$; hence,
\begin{align}\label{eq:pi j^* to j}
Q(\pi,U) &= \sum_{Z \in \suff(\sigl_\pi)} c^\pi_Z \cdot Q(\pi,Z,\ul\setminus \{a_{j^*},a_{j'}\}).
\end{align}
Following the same factorization process for $\rho$, we get
\begin{align}\label{eq:rho j^* to j}
Q(\rho,U) &= \sum_{Z \in \suff(\sigl_\rho)} c^\rho_Z \cdot Q(\rho,Z,\ul\setminus \{a_{j^*},a_{j'}\}).
\end{align}
Next, we want to simplify the coefficients $\left(c^\pi_Z\right)_Z$. We remark that $c^\pi_Z$ is not simply the probability of reaching $(Z,\ul\setminus \{a_{j^*},a_{j'}\})$ from $s$, i.e., $\Pr(\pathto{s}{(Z,\ul\setminus \{a_{j^*},a_{j'}\})})$. To clarify, consider a strict suffix $Z$, $1\leq \abs{Z}< \abs{\above(A)}$, and the suffix $Z'=Z \cup\{a_l\}$ for the minimal element $a_l \in \ug \setminus Z$ according to $\sigl_\pi$,i.e., $a_l = \argmin_{a\in \ug \setminus Z}\sigl_\pi(a)$. In the factorization process that produced Equation (\ref{eq:pi j^* to j}), once we got the term $Q(\pi,Z',\ul\setminus \{a_{j^*},a_{j'}\})$, we stopped factorizing any further; thus, $c^\pi_Z$ does not include the probability of reaching a node associated with $(Z',\ul\setminus \{a_{j^*},a_{j'}\})$ and then following the left edge to $(Z,\ul\setminus \{a_{j^*},a_{j'}\})$. However, this probability is taken into account in $\Pr(\pathto{s}{(Z,\ul\setminus \{a_{j^*},a_{j'}\})})$. Rather, $c^\pi_Z$ is the probability of reaching any node $v$ in the tree induced by $\pi$ with the following property: $v$ represents the state $(Z,\ul\setminus \{a_{j^*},a_{j'}\})$, while $a_{j'}$ does not belong to the state represented by the parent of $v$.  In the tree interpretation, $v$ should also be a \textit{right child of its parent} (for instance, the left highlighted node in the tree in Figure \ref{fig:tree example}). The following Proposition \ref{prop:coef c} describes $\left(c^\pi_Z\right)_Z$ in terms of $Q$.
\begin{proposition}\label{prop:coef c}
For every $Z\in \suff(\sigl_\pi)$, let $a_{i(Z)} = \argmin_{a_i\in Z} \sigl_\pi(a_i)$. It holds that
\[
c^\pi_Z = Q(\pi,\ug\setminus Z \cup \{a_{i(Z)}\}, \{a_{j^*},a_{j'}\})-Q(\pi,\ug\setminus Z, \{a_{j^*},a_{j'}\}).
\]
\end{proposition}
The proof of Proposition \ref{prop:coef c} appears at the end of this proof. Notice that for every $Z$, $c^\pi_Z$ includes values of $Q$ with less arms than $U$ (besides, perhaps, the case where $\abs{\ug}=2$ and $\abs{Z}=1$ obtaining $Q(\pi,\ug, \{a_{j^*},a_{j'}\})$, but we cover this case in the bases cases); consequently, due to the inductive step
\begin{align}\label{c pi is rho}
c^\pi_Z = Q(\rho,\ug\setminus Z \cup \{a_{i(Z)}\}, \{a_{j^*},a_{j'}\})-Q(\rho,\ug\setminus Z, \{a_{j^*},a_{j'}\})=c^\rho_Z,
\end{align}
where the last equality follows from mirroring Proposition \ref{prop:coef c} for $(c^\rho_Z)_Z$. Ultimately,
{\thinmuskip=.2mu
\medmuskip=0mu plus .2mu minus .2mu
\thickmuskip=1mu plus 1mu
\begin{align*}
Q(\pi,U) &\stackrel{\textnormal{Eq. (\ref{eq:pi j^* to j})}}{=}\sum_{Z \in \suff(\sigl_\pi)}c^\pi_Z \cdot Q(\pi,Z,\ul\setminus \{a_{j^*},a_{j'}\})\stackrel{\textnormal{Eq. (\ref{c pi is rho})}}{=}\sum_{Z \in \suff(\sigl_\pi)}c^\rho_Z \cdot Q(\pi,Z,\ul\setminus \{a_{j^*},a_{j'}\}) \nonumber\\
&\stackrel{\textnormal{Ind. step}}{=}\sum_{Z \in \suff(\sigl_\pi)}c^\rho_Z \cdot Q(\rho,Z,\ul\setminus \{a_{j^*},a_{j'}\}) \stackrel{\sigl_{\rho}=\sigl_\pi}{=}\sum_{Z \in \suff(\sigl_\rho)}c^\rho_Z \cdot Q(\rho,Z,\ul\setminus \{a_{j^*},a_{j'}\}) \nonumber\\
&\stackrel{\textnormal{Eq. (\ref{eq:rho j^* to j})}}{=}Q(\rho,U) .
\end{align*}}
This completes Step 1.

\paragraph{Step 2}
Fix arbitrary $a_{i'}$ and $a_{j'}$ such that $a_{i'} \in \ug$ and $a_{j'} \in \ul$. We will show that 
\begin{equation}\label{eq:step 2 goal}
Q^*_{{i^*},{j'}}(U)=Q^*_{{i'},{j'}}(U).
\end{equation}
We follow the same technique as in the previous step. Let $\pi$ be an ordered policy such that $\sigl_\pi=(a_{i^*},a_{i'},\dots )$, i.e., $\sigl_\pi$ ranks $a_{i^*}$ first, $a_{i'}$ second and then follows some arbitrary order on the remaining arms, and $\sigr_\pi=(a_{j'},\dots)$. In addition, we define the ordered policy $\rho$ with $\sigl_\rho=(a_{i'},a_{i^*},\dots )$, where the dots refer to any arbitrary order on the remaining elements of $\ug$, and $\sigr_\rho =\sigr_\pi=(a_{j'},\dots)$. Using the inductive step and the same arguments as in Equation (\ref{eq:policies suffice}), it suffices to show that $Q(\pi,U)=Q(\rho,U)$. We factor $Q(\pi,U)$ recursively such that 
\begin{align}\label{eq: q pi with d}
Q(\pi,U)=Q(\pi,\{a_{i^*},a_{i'}\},\ul)+ \sum_{Z \in \suff(\sigr_\pi)} d^\pi_Z \cdot Q(\pi,\ug \setminus \{a_{i^*},a_{i'}\},Z),
\end{align}
and similarly
\begin{align}\label{eq: q rho with d}
Q(\rho,U)=Q(\rho,\{a_{i^*},a_{i'}\},\ul)+ \sum_{Z \in \suff(\sigr_\rho)} d^\rho_Z \cdot Q(\rho,\ug \setminus \{a_{i^*},a_{i'}\},Z).
\end{align}
Next, we claim that
\begin{proposition}\label{prop:coef d}
For every $Z\in \suff(\sigr_\pi)$, let $a_{j(Z)} = \argmin_{a_j\in Z}\sigr_\pi(a_j)$. It holds that
\[
d^\pi_Z = Q(\pi,\{a_{i^*},a_{i'}\},\ul \setminus Z )-Q(\pi,\{a_{i^*},a_{i'}\},\ul \setminus Z \cup \{a_{j(Z)}\}).
\]
\end{proposition}
The proof of Proposition \ref{prop:coef d} appears at the end of this proof. Notice that for every $Z$, $d^\pi_Z$ includes values of $Q$ with less arms than $U$ (besides, perhaps, the case where $\abs{\ul}=2$ and $\abs{Z}=1$ obtaining $Q(\pi,\{a_{i^*},a_{i'}\},\ul\})$, but we cover this case in the bases cases); consequently, due to the inductive step
\begin{align}\label{eq: d pi is rho}
d^\pi_Z = Q(\rho,\{a_{i^*},a_{i'}\},\ul \setminus Z )-Q(\rho,\{a_{i^*},a_{i'}\},\ul \setminus Z \cup \{a_{j(Z)}\})=d^\rho_Z,
\end{align}
where the last equality follows from mirroring Proposition \ref{prop:coef d} for $(d^\rho_Z)_Z$. Ultimately, by rearranging Equation (\ref{eq: q pi with d}) and invoking the inductive step, Equation (\ref{eq: d pi is rho}) and the fact that $\sigr_\rho =\sigr_\pi$, we get
\begin{align*}
Q(\pi,U)=Q(\rho,\{a_{i^*},a_{i'}\},\ul)+ \sum_{Z \in \suff(\sigr_\rho)} d^\rho_Z \cdot Q(\rho,\ug \setminus \{a_{i^*},a_{i'}\}),Z)\stackrel{\textnormal{Eq. (\ref{eq: q rho with d})}}{=}Q(\pi,U),
\end{align*}
implying Equation (\ref{eq:step 2 goal}) holds.
\paragraph{Step 3} We are ready to prove the lemma. Fix arbitrary $a_{\tilde i}$ and $a_{\tilde j}$ such that $a_{\tilde i} \in \ug$ and  $a_{\tilde j} \in \ul$. By the previous Step 1 and Step 2, we know that
\[
Q^*_{{i^*},{j^*}}(U)\stackrel{\textnormal{Step 1}}{=}Q^*_{{i^*},{\tilde j}}(U)\stackrel{\textnormal{Step 2}}{=}Q^*_{{\tilde i},{\tilde j}}(U),
\]
which contradicts Equation (\ref{eq:contradiction of lemma}); hence, the lemma holds.
\end{proofof}


\section{Additional Statements for Lemma \ref{lemma:equivalence}}
\begin{proofof}{Proposition \ref{prop:coef c}}
\begin{figure}
\centering
\includegraphics[scale=0.8]{figs/prop4-tree-illustration-crop.pdf}
\caption{Illustration for Proposition \ref{prop:coef c}. The tree depicts $T(\pi)$. Nodes $v_1$ to $v_6$ are nodes whose sub-trees were pruned in the construction of $T$. Let $Z=\ug\setminus \{a_{i'},a_{i_1}\}$ and $Z^c= \{a_{i'},a_{i_1}\}$. The minimal element of $Z$, denoted $a_{i(Z)}$ in the proof, is $a_{i_2}$. The corresponding $c^\pi_Z$ is the probability to reach one of $\{v_3,v_5,v_6\}$, namely, $c^\pi_Z = \Pr(\{v_3,v_5,v_6\})$. In the tree $T$, we ignore sub-trees of nodes $v$ labeled with ``$\dots $'' since these do not contribute to $c^\pi_Z$. Observe that the probability of reaching $v_i$, for $i\in\{1,\dots, 6\}$ is the same in $T(\pi)$ and $T$. Finally, notice that $Q(\pi,\ug\setminus Z \cup \{a_{i(Z)}\}, \{a_{j^*},a_{j'}\})=\Pr(\{v_1,v_2,\dots,v_6 \})$, and $Q(\pi,\ug\setminus Z, \{a_{j^*},a_{j'}\})=\Pr(\{v_1,v_2,v_4\})$. Combining, we get that $c^\pi_Z = Q(\pi,\ug\setminus Z \cup \{a_{i(Z)}\}, \{a_{j^*},a_{j'}\})-Q(\pi,\ug\setminus Z, \{a_{j^*},a_{j'}\})=\Pr(\{v_3,v_5,v_6\})$ as required.
\label{fig:tree illustation}}
\end{figure}
To prove this claim, we focus on the tree induced by $\pi$, $T(\pi)$. It is convenient to discuss a modified version $T(\pi)$ obtained by pruning, and this is feasible since even if prune nodes from $T(\pi)$ it still remains Markov chain. We illustrate the proof of this claim in Figure \ref{fig:tree illustation}.

We factorize $Q(\pi,U) $ recursively (see Equation (\ref{eq:pi j^* to j})) until we hit a node associated with a state that excludes $\{a_{j^*},a_{j'}\}$, or a leaf. This factorization can be illustrated as follows: We traverse $T(\pi)$, from right to left. Every node we visit, we ask whether that node includes $\{a_{j^*},a_{j'}\}$. If it does not, we prune its sub-tree (i.e., it becomes a leaf) while leaving it intact. Denote the obtained tree by $T$, and let $V(T)$ be its set of nodes. Observe that
\begin{observation}\label{obs:two types}
Every leaf $v$ in $V(T)$ satisfies exactly one property: 
\begin{enumerate}[leftmargin=0cm,itemindent=.5cm,labelwidth=\itemindent,labelsep=0cm,align=left]
\item[]\textit{Type 1:} $\above(state(v))=\emptyset$, or
\item[]\textit{Type 2:} $\below(state(v))=\ul\setminus \{j^*,j'\}$ with $\below(state(v)) \subset \below(state(parent(v)))$.
\end{enumerate}
\end{observation}
Leaves of type 1 are associated with terminal states of the MDP (see Subsection~\ref{subsec:aux GMDP}). Leaves of type 2 are those whose sub-trees were pruned during the traversal. Moreover, $\below(state(v)) \subset \below(state(parent(v)))$ holds in every such a leaf $v$, since otherwise we would have pruned its parent. Due to Observation \ref{obs:two types}, every node $v$ with $state(v)=(Z,\ul\setminus \{a_{j^*},a_{j'}\})$ is of type 2; therefore,
\begin{align}\label{eq:c with T}
c^\pi_Z = \sum_{\substack{v\in V(T):state(v)=\\(Z,\ul\setminus \{a_{j^*},a_{j'}\})}}\Pr\left(\pth{root(T)}{}{v}\right).
\end{align}


Next, fix an arbitrary non-empty $Z$, $Z \subseteq \ug$, and $\Psi$ be the set of all non-empty suffixes of $\ug\setminus Z$. Consider $T$ and its root $root(T)$. Notice that $Q(\pi,\ug\setminus Z, \{a_{j^*},a_{j'}\})$ is the probability of reaching a (type 2) leaf $v$ such that $\above(v)=\psi\cup Z$ for some $\psi \in \Psi$. This is true since $\pi$ is ordered, and every path from $root(T)$ to such a leaf $v$ does not include any action from $Z$; hence, we can compare the probability of reaching it to off-path behavior of $\pi$. Further, $Q(\pi,\ug\setminus Z \cup \{a_{i(Z)}\}, \{a_{j^*},a_{j'}\})$ is the probability of reaching a (type 2) leaf $v$ such that $\above(v)=\psi\cup Z$ for some $\psi \in \Psi$ or $\above(v)=Z$; hence,
\[
Q(\pi,\ug\setminus Z \cup \{a_{i(Z)}\}, \{a_{j^*},a_{j'}\})-Q(\pi,\ug\setminus Z, \{a_{j^*},a_{j'}\})
\]
is precisely the right-hand-side of Equation \refeq{eq:c with T}.
\end{proofof}



\begin{proofof}{Proposition \ref{prop:coef d}}
Fix $Z\in \suff(\sigr_\pi)$, and let $a_{j(Z)} = \argmin_{a_j\in Z}\sigr_\pi(a_j)$. Let $T(\pi)$ denote the tree induced by $\pi$. Observe that
\begin{observation}\label{obs: for d}
The coefficient $d^\pi_Z$ is the probability to get to a node $v$ in $T(\pi)$ such that
\begin{enumerate}
\item $state(v)=(\ug\setminus \{a_{i^*},a_{i'} \},Z )$, and 
\item $state(parent(v))=(\ug\setminus \{a_{i^*}\},Z )$. 
\end{enumerate}
\end{observation}
The first condition is immediate, due to the way we factorize $Q$ in Equation \refeq{eq: q pi with d}. To see why the second condition holds, notice that $state(parent(v))$ must be a strict superset of $state(v)$; hence, $state(parent(v))$ could be either $(\ug\setminus \{a_{i^*}\},Z )$ or $(\ug\setminus \{a_{i^*},a_{i'}\},Z\cup\{ a\} )$ for $a\in \ul\setminus Z$, but then it would contribute to $d^\pi_{Z \cup \{a\}}$, namely, to another summand in Equation \refeq{eq: q pi with d}. 

Denote by $V$ the set of all nodes that satisfy the conditions of Observation \ref{obs: for d}. Due to the way we constructed $\pi$, the paths from the root of $T(\pi)$ to any node in $V$ consist of actions that involve the arms $\{a_{i^*},a_{i'},a_{j(Z)}\}\cup (\ul \setminus Z)$ solely; hence, we can focus on the \text{off-path} tree whose root is $s_0'\defeq\{a_{i^*},a_{i'}\}\cup (\ul \setminus Z)\cup \{a_{j(Z)}\}$, and the actions are precisely as in the tree induced by $\pi$ (according to the order of $\pi$). Denote this new tree by $T'$, and let
\[
V' \defeq \left\{v\in nodes(T)\mid state(v)=\{a_{j(Z)}\}  \right\}.
\] 
Due to this construction, 
\begin{observation}\label{obs: for d two}
The coefficient $d^\pi_Z$ is the probability to get to a node that belongs to $V'$  in $T'$.
\end{observation}
The observation follows from the one-to-one correspondence between the nodes and path in $T(\pi)$ and their counterparts in the off-path tree $T'$. 

In $T'$,  $Q(\pi,s'_0)$ is the probability of starting at $s'_0$ and reaching the leaf with no arms (terminal state $\emptyset$), i.e., exploring $\ul \setminus Z$ \textit{and} $a_{j(Z)}$. In contrast, $Q(\pi,\{a_{i^*},a_{i'}\}\cup (\ul \setminus Z))$ is the probability of starting at $s'_0$ and reaching a node (internal or terminal) $v$ with $state(v)= \cap \left(\ul \setminus Z\right) =\emptyset$, namely, exploring $\ul \setminus Z$. Such a node $v$ leads to a leaf with probability 1; hence, paths from $v$ terminate in leaves corresponding to either state $\emptyset$ or  $\{a_{j(Z)}\}$. Consequently,
\[
Q(\pi,\{a_{i^*},a_{i'}\}\cup (\ul \setminus Z)) - Q(\pi,\{a_{i^*},a_{i'}\}\cup (\ul \setminus Z)\cup \{ a_{j(Z)}\})
\]
is the probability of starting at $s_0'$, and reaching a terminal node that belongs to $V'$.
\end{proofof}


\section{Base Cases for Lemma \ref{lemma:equivalence}}\label{sec:base for lemma}
\begin{proposition}\label{prop:case of one strong}
Let $\abs{\ug}=1$ and $\abs{\ul} \geq 2$. For any pair of policies $\pi,\rho$, it holds that $Q(\pi,\ug,\ul)=Q(\rho,\ug,\ul)$.
\end{proposition}
\begin{proofof}{Proposition \ref{prop:case of one strong}}
Let $\tilde \mu(a)\defeq \abs{\mu(a)}$, and denote $\ug=\{a_{i_1}\}$  and $\ul=\{a_{j_1},\dots a_{j_k}\}$ for $k=\abs{\ul}$. The probability of reaching the empty terminal state under any $\mP$-valid policy is
\begin{align*}
\prod_{l=1}^k\frac{\tilde \mu(a_{j_l})}{\tilde \mu(a_{j_l})+\tilde \mu(a_{i_1})},
\end{align*}
i.e., the probability of successfully exploring $\ul$. Due to multiplication associativity, the above expression is invariant of the way we order its elements. Finally, by definition of $Q$, this implies that $Q(\pi,\ug,\ul)=Q(\rho,\ug,\ul)$.
\end{proofof}

\begin{proposition}\label{prop:case of one}
Let $\abs{\ug}\geq 2$ and $\abs{\ul} = 1$. For any pair of policies $\pi,\rho$, it holds that $Q(\pi,\ug,\ul)=Q(\rho,\ug,\ul)$.
\end{proposition}
\begin{proofof}{Proposition \ref{prop:case of one}}
Let $\tilde \mu(a)\defeq \abs{\mu(a)}$, and denote $\ug=\{a_{i_1},\dots a_{i_k}\}$ for $k=\abs{\ug}$ and $\ul=\{a_{j_1}\}$. The probability of reaching the terminal state $(a_{j_1})$ under any $\mP$-valid policy is
\begin{align*}
\prod_{l=1}^k\frac{\tilde \mu(a_{i_l})}{\tilde \mu(a_{i_l})+\tilde \mu(a_{j_1})},
\end{align*}
i.e., the probability of failing to explore $a_{j_1}$. Due to multiplication associativity, the above expression is invariant of the way we order its elements. Finally, by definition of $Q$, this implies that $1-Q(\pi,\ug,\ul)=1-Q(\rho,\ug,\ul)$; hence, $Q(\pi,\ug,\ul)=Q(\rho,\ug,\ul)$
\end{proofof}


\begin{proposition}\label{prop:case of two strong}
Let $U$ be an arbitrary state, such that $\ug\defeq\above(U)=2$ and $\ul\defeq\below(U) \geq 2$. For any pair of $\mP$-valid policies $\pi$ and $\rho$, it holds that $Q(\pi,U)=Q(\rho,U)$.
\end{proposition}
\begin{proofof}{Proposition \ref{prop:case of two strong}}
We prove the claim by induction, with Proposition \ref{prop:case of one} serving as the base case. Assume the claim holds for $\abs{\ul}=k-1$. It is enough to show that if $\abs{\ul}=k$, for any $a_i\in \ug,a_j\in \ul$, $Q^*_{i,j}(U)=Q^*(U)$. Assume that $Q^*_{{i_1},{j_1}}(U)=Q^*(U)$, and fix any $a_{i'}\in \ug, a_{j'}\in \ul$. 
\paragraph{Remark} We do not use Assumption \ref{assumption:dominance} here.
\paragraph{Step 1} Assume that $i'=i_1$ and $j' \neq j_1$. W.l.o.g. $j'=j_2$. We construct two policies, $\pi$ that ordered $\ul$ as $\sigr_\pi=(a_{j_1},a_{j_2},\dots, a_{j_{k}})$, and $\rho$ that orders $\ul$ as $\sigr_\rho=(a_{j_2},a_{j_1},\dots,a_{j_{k}})$. Both policies order $\ug$ according to $\sigl_\pi=\sigl_\rho=(a_{i_1},a_{i_2})$. Due to the inductive step and our assumption that  $Q^*_{i_1,j_1}(U)=Q^*(U)$, we have that $Q(\pi,U)=Q^*(U)$, and
\begin{align*}
&Q(\pi,\ug,\ul) = \underbrace{\prod_{l=1}^{k} \bl p_{{i_1},{j_l}}(a_{j_l})}_{\lambda(\pi)}
+ \underbrace{\sum_{f=1}^{k} \left(\prod_{l=1}^{f-1} \bl p_{{i_1},{j_l}}(a_{j_l})\right) \bl p_{{i_1},{j_f}}(a_{i_1}) \left(\prod_{l=f}^{k} \bl p_{{i_2},{j_l}}(a_{j_l})\right)}_{\delta(\pi)}.
\end{align*}
Notice that $\lambda(\pi)=\lambda(\rho)$. In addition, it holds that
{\thinmuskip=0mu
\medmuskip=0mu plus 0mu minus 0mu
\thickmuskip=0mu plus 0mu
\begin{align}\label{eq:ind 2 step 1}
\delta(\pi)&=\bl p_{{i_1},{j_1}}(a_{i_1})\prod_{l=1}^{k} \bl p_{{i_2},{j_l}}(a_{j_l})\nonumber\\
&\qquad +\bl p_{{i_1},{j_1}}(a_{j_1})\sum_{f=2}^{k} \left(\prod_{l=2}^{f-1} \bl p_{{i_1},{j_l}}(a_{j_l})\right) \bl p_{{i_1},{j_f}}(a_{i_1}) \left(\prod_{l=f}^{k} \bl p_{{i_2},{j_l}}(a_{j_l})\right)\nonumber\\
& =\bl p_{{i_1},{j_1}}(a_{i_1})\bl p_{{i_2},{j_1}}(a_{j_1})\bl p_{{i_2},{j_2}}(a_{j_2})\prod_{l=3}^{k} \bl p_{{i_2},{j_l}}(a_{j_l})\nonumber\\
&\qquad+ \bl p_{{i_1},{j_1}}(a_{j_1})\Bigg[\bl p_{{i_1},{j_2}}(a_{i_1})\bl p_{{i_2},{j_2}}(a_{j_2})\prod_{l=3}^{k} \bl p_{{i_2},{j_l}}(a_{j_l})\nonumber\\
&\qquad \qquad+\bl p_{{i_1},{j_2}}(a_{j_2})  \sum_{f=3}^{k} \left(\prod_{l=3}^{f-1} \bl p_{{i_1},{j_l}}(a_{j_l})\right) \bl p_{{i_1},{j_f}}(a_{i_1}) \left(\prod_{l=f}^{k} \bl p_{{i_2},{j_l}}(a_{j_l})\right) \Bigg]\nonumber\\
&=\bl p_{{i_2},{j_2}}(a_{j_2})\left( \bl p_{{i_1},{j_1}}(a_{i_1})\bl p_{{i_2},{j_1}}(a_{j_1})+\bl p_{{i_1}
,{j_1}}(a_{j_1})\bl p_{{i_1},{j_2}}(a_{i_1})\right)\prod_{l=3}^{k} \bl p_{{i_2},{j_l}}(a_{j_l})\nonumber\\
&\qquad+ \bl p_{{i_1},{j_1}}(a_{j_1})\bl p_{{i_1},{j_2}}(a_{j_2})  \sum_{f=3}^{k} \left(\prod_{l=3}^{f-1} \bl p_{{i_1},{j_l}}(a_{j_l})\right) \bl p_{{i_1},{j_f}}(a_{i_1}) \left(\prod_{l=f}^{k} \bl p_{{i_2},{j_l}}(a_{j_l})\right).
\end{align}}%
We show that
\begin{claim}\label{claim:triplets}
It holds that
\begin{align*}
&\bl p_{{i_2},{j_2}}(a_{j_2})\left( \bl p_{{i_1},{j_1}}(a_{i_1})\bl p_{{i_2},{j_1}}(a_{j_1})+\bl p_{{i_1},{j_1}}(a_{j_1})\bl p_{{i_1},{j_2}}(a_{i_1})\right)\\
&=\bl p_{{i_2},{j_1}}(a_{j_1})\left(\bl p_{{i_1},{j_2}}(a_{i_1})\bl p_{{i_2},{j_2}}(a_{j_2})+\bl p_{{i_1},{j_2}}(a_{j_2})\bl p_{{i_1},{j_1}}(a_{i_1})  \right).
\end{align*}
\end{claim}
Now, set $\sigma:\mathbb N \rightarrow \mathbb N$ such that $\sigma(1)=2, \sigma(2)=1$, and $\sigma(i)=i$ for $i\geq 3$; hence, using Claim \ref{claim:triplets},
\begin{align}
\textnormal{Eq. (\ref{eq:ind 2 step 1})}
& =\bl p_{{i_1},{j_2}}(a_{i_1})\bl p_{{i_2},{j_2}}(a_{j_2})\bl p_{{i_2},{j_1}}(a_{j_1})\prod_{l=3}^{k} \bl p_{{i_2},{j_l}}(a_{j_l})\nonumber\\
&\qquad+ \bl p_{{i_1},{j_2}}(a_{j_2}) \Bigg[\bl p_{{i_1},{j_1}}(a_{i_1})\bl p_{{i_2},{j_1}}(a_{j_1}) \prod_{l=3}^{k} \bl p_{{i_2},{j_l}}(a_{j_l})\nonumber\\
&\qquad \qquad + \bl p_{{i_1},{j_1}}(a_{j_1})\sum_{f=3}^{k} \left(\prod_{l=3}^{f-1} \bl p_{{i_1},{j_l}}(a_{j_l})\right) \bl p_{{i_1},{j_f}}(a_{i_1}) \left(\prod_{l=f}^{k} \bl p_{{i_2},{j_l}}(a_{j_l})\right) \Bigg]\nonumber\\
&= \sum_{f=1}^{k} \left(\prod_{l=1}^{f-1} \bl p_{{i_1},{j_{\sigma(l)}}}(a_{j_{\sigma(l)}})\right) \bl p_{{i_1},{j_{\sigma (f)}}}(a_{i_1}) \left(\prod_{l=f}^{k} \bl p_{{i_2},{j_{\sigma(l)}}}(a_{j_{\sigma(l)}})\right)\nonumber\\
&=\delta(\rho)
\end{align}
\paragraph{Step 2} Assume that $i'=i_2\neq i_1$ and $j'=j_1$. We construct two policies, $\pi$ that orders $\ug$ as $\sigl_\pi=(a_{i_1},a_{i_2})$, and $\rho$ that orders $\ug$ as  $\sigl_\rho=(a_{i_2},a_{i_1})$. Both policies order $\ul$ by $\sigr_\pi=\sigr_\rho=(a_{j_1},a_{j_2}, \dots, a_{j_k})$. In addition, we introduce a third policy, $\tilde \rho$, that has the same order as $\rho$ on $\ug$, namely $\sigl_{\tilde \rho}=\sigl_\rho=(a_{i_2},a_{i_1})$, and orders $\ul$ by $\sigr_{\tilde \rho}=(a_{j_k},a_{j_2},\dots ,a_{j_{k-1}},a_{j_1})$.
It holds that
{\thinmuskip=0mu
\medmuskip=0mu plus 0mu minus 0mu
\thickmuskip=0mu plus 0mu
\begin{align}\label{eq: multiple transitions}
&Q(\pi,\ug,\ul)=\prod_{l=1}^k \bl p_{{i_1},{j_l}}(a_{j_l})+\sum_{f=1}^{k} \left(\prod_{l=1}^{f-1} \bl p_{{i_1},{j_l}}(a_{j_l})\right) \bl p_{{i_1},{j_f}}(a_{i_1}) \left(\prod_{l=f}^{k} \bl p_{{i_2},{j_l}}(a_{j_l})\right)\nonumber\\
&=\bl p_{{i_1},{j_1}}(a_{j_1})\bl p_{{i_1},{j_k}}(a_{j_k})\overbrace{\prod_{l=2}^{k-1} \bl p_{{i_1},{j_l}}(a_{j_l})}^{I_1}+\bl p_{{i_2},{j_k}}(a_{j_k})\Bigg[\bl p_{{i_1},{j_1}}(a_{j_1})\bl p_{{i_1},{j_k}}(a_{i_1})\underbrace{\prod_{l=2}^{k-1} \bl p_{{i_1},{j_l}}(a_{j_l})}_{I_1}\nonumber\\
&\quad+\underbrace{\sum_{f=1}^{k-1} \left(\prod_{l=1}^{f-1} \bl p_{{i_1},{j_l}}(a_{j_l})\right) \bl p_{{i_1},{j_f}}(a_{i_1}) \left(\prod_{l=f}^{k-1} \bl p_{{i_2},{j_l}}(a_{j_l})\right)}_{I_2}\Bigg]\nonumber\\
&= \bl p_{{i_1},{j_1}}(a_{j_1})\bl p_{{i_1},{j_k}}(a_{j_k})\bl p_{{i_2},{j_k}}(a_{j_k})I_1 +\bl p_{{i_1},{j_1}}(a_{j_1})\bl p_{{i_1},{j_k}}(a_{j_k})\bl p_{{i_2},{j_k}}(a_{i_2})I_1\nonumber\\
&\qquad+\bl p_{{i_2},{j_k}}(a_{j_k})\left[\bl p_{{i_1},{j_1}}(a_{j_1})\bl p_{{i_1},{j_k}}(a_{i_1})I_1+I_2\right]\nonumber\\
&=\bl p_{{i_1},{j_1}}(a_{j_1})\bl p_{{i_1},{j_k}}(a_{j_k})\bl p_{{i_2},{j_k}}(a_{i_2})I_1\nonumber\\
&\qquad +\bl p_{{i_2},{j_k}}(a_{j_k})\left[\bl p_{{i_1},{j_1}}(a_{j_1})\bl p_{{i_1},{j_k}}(a_{j_k})I_1+\bl p_{{i_1},{j_1}}(a_{j_1})\bl p_{{i_1},{j_k}}(a_{i_1})I_1+I_2\right]\nonumber\\
&=\bl p_{{i_1},{j_1}}(a_{j_1})\bl p_{{i_1},{j_k}}(a_{j_k})\bl p_{{i_2},{j_k}}(a_{i_2})I_1+\bl p_{{i_2},{j_k}}(a_{j_k})\left[\bl p_{{i_1},{j_1}}(a_{j_1})I_1+I_2\right].
\end{align}}%
Next, observe that
{\thinmuskip=0mu
\medmuskip=0mu plus 0mu minus 0mu
\thickmuskip=0mu plus 0mu
\begin{equation}\label{eq:ind step with k-1}
\left[\bl p_{{i_1},{j_1}}(a_{j_1})I_1+I_2\right]=\prod_{l=1}^{k-1} \bl p_{{i_1},{j_l}}(a_{j_l})+\sum_{f=1}^{k-1} \left(\prod_{l=1}^{f-1} \bl p_{{i_1},{j_l}}(a_{j_l})\right) \bl p_{{i_1},{j_f}}(a_{i_1}) \left(\prod_{l=f}^{k-1} \bl p_{{i_2},{j_l}}(a_{j_l})\right).
\end{equation}}%
Notice that the latter is precisely $Q(\pi,\left\{a_{i_1},a_{i_2},a_{j_1},\dots a_{j_{k-1}}\right\})$; thus, the inductive step implies that it is order invariant. Let $\sigma:\mathbb N \rightarrow \mathbb N$ such that $\sigma(1)=k,\sigma(k)=1$, and $\sigma(i)=i$ for $1<i<k$. Since $Q(\pi,\left\{a_{i_1},a_{i_2},a_{j_1},\dots a_{j_{k-1}}\right\})=Q(\tilde \rho,\left\{a_{i_1},a_{i_2},a_{j_1},\dots a_{j_{k-1}}\right\})$, we conclude that the expression in Equation (\ref{eq:ind step with k-1}) equals 
\[
\prod_{l=2}^{k} \bl p_{{i_2},{j_{\sigma(l)}}}(a_{j_{\sigma(l)}})+\sum_{f=2}^{k} \left(\prod_{l=2}^{f-1} \bl p_{{i_2},{j_{\sigma(l)}}}(a_{j_{\sigma(l)}})\right) \bl p_{{i_2},{j_{\sigma(f)}}}(a_{i_2}) \left(\prod_{l=f}^{k} \bl p_{{i_1},{j_{\sigma(l)}}}(a_{j_{\sigma(l)}})\right).
\]
Combining this with Equation (\ref{eq: multiple transitions}),
{\thinmuskip=0mu
\medmuskip=0mu plus 0mu minus 0mu
\thickmuskip=0mu plus 0mu
\begin{align}
&\textnormal{Eq. }\refeq{eq: multiple transitions}=\bl p_{{i_1},{j_1}}(a_{j_1})\bl p_{{i_1},{j_k}}(a_{j_k})\bl p_{{i_2},{j_k}}(a_{i_2})\prod_{l=2}^{k-1} \bl p_{{i_1},{j_{\sigma(l)}}}(a_{j_{\sigma(l)}})\nonumber\\
&\quad +\bl p_{{i_2},{j_k}}(a_{j_k})\Bigg[
\prod_{l=2}^{k} \bl p_{{i_2},{j_{\sigma(l)}}}(a_{j_{\sigma(l)}})+\sum_{f=2}^{k} \left(\prod_{l=2}^{f-1} \bl p_{{i_2},{j_{\sigma(l)}}}(a_{j_{\sigma(l)}})\right) \bl p_{{i_2},{j_{\sigma(f)}}}(a_{i_2}) \left(\prod_{l=f}^{k} \bl p_{{i_1},{j_{\sigma(l)}}}(a_{j_{\sigma(l)}})\right)
\Bigg]\nonumber\\
&=\bl p_{{i_2},{j_k}}(a_{i_2})\prod_{l=1}^{k} \bl p_{{i_1},{j_{\sigma(l)}}}(a_{j_{\sigma(l)}})+\bl p_{{i_2},{j_k}}(a_{j_k})
\prod_{l=2}^{k} \bl p_{{i_2},{j_{\sigma(l)}}}(a_{j_{\sigma(l)}})\nonumber\\
&\quad +\bl p_{{i_2},{j_k}}(a_{j_k})\left[
\sum_{f=2}^{k} \left(\prod_{l=2}^{f-1} \bl p_{{i_2},{j_{\sigma(l)}}}(a_{j_{\sigma(l)}})\right) \bl p_{{i_2},{j_{\sigma(f)}}}(a_{i_2}) \left(\prod_{l=f}^{k} \bl p_{{i_1},{j_{\sigma(l)}}}(a_{j_{\sigma(l)}})\right)
\right]\nonumber\\
&=\prod_{l=1}^{k} \bl p_{{i_2},{j_{\sigma(l)}}}(a_{j_{\sigma(l)}})+
\sum_{f=1}^{k} \left(\prod_{l=1}^{f-1} \bl p_{{i_2},{j_{\sigma(l)}}}(a_{j_{\sigma(l)}})\right) \bl p_{{i_2},{j_{\sigma(f)}}}(a_{i_2}) \left(\prod_{l=f}^{k} \bl p_{{i_1},{j_{\sigma(l)}}}(a_{j_{\sigma(l)}})\right)\nonumber\\
&=Q(\tilde \rho ,U),
\end{align}}%
where the last equality follows from the definition of $\tilde \rho$ (orders the arms precisely so). Finally, $Q(\tilde \rho, U)=Q(\rho ,U)$ follows from the previous Step 1.
\paragraph{Step 3}
The two previous steps imply that for any $a_{i'}\in \ug, a_{j'}\in \ul$, it holds that
\[
Q_{i',j'}(U)=Q_{i',j_1}(U)=Q_{i_1,j_1}(U).
\]
This completes the proof of Proposition \ref{prop:case of two strong}.
\end{proofof}
\begin{proofof}{Claim \ref{claim:triplets}}
To ease readability, let $\tmu{i}\defeq\abs{\mu(a_i)}$ for every $a_i\in A$. It holds that
\begin{align*}
&\bl p_{{i_2},{j_2}}({j_2})\left( \bl p_{{i_1},{j_1}}({i_1})\bl p_{{i_2},{j_1}}({j_1})+\bl p_{{i_1},{j_1}}({j_1})\bl p_{{i_1},{j_2}}({i_1})\right)\\
&=\frac{\tmu{i_2}}{\tmu{i_2}+\tmu{j_2}}\left( \frac{\tmu{j_1}}{\tmu{i_1}+\tmu{j_1}}\frac{\tmu{i_2}}{\tmu{i_2}+\tmu{j_1}}+\frac{\tmu{i_1}}{\tmu{i_1}+\tmu{j_1}}\frac{\tmu{j_2}}{\tmu{i_1}+\tmu{j_2}}\right) \\
&= \frac{\tmu{i_2}\tmu{j_1}\tmu{i_2}(\tmu{i_1}+\tmu{j_2})+\tmu{i_2}\tmu{i_1}\tmu{j_2}(\tmu{i_2}+\tmu{j_1})}{(\tmu{i_1}+\tmu{j_1})(\tmu{i_1}+\tmu{j_2})(\tmu{i_2}+\tmu{j_1})(\tmu{i_2}+\tmu{j_2})}\\
&= \frac{\overbrace{\tmu{i_2}\tmu{j_1}\tmu{i_2}\tmu{i_1}}^{I}+\overbrace{\tmu{i_2}\tmu{j_1}\tmu{i_2}\tmu{j_2}}^{II}+\overbrace{\tmu{i_2}\tmu{i_1}\tmu{j_2}\tmu{i_2}}^{III}+\overbrace{\tmu{i_2}\tmu{i_1}\tmu{j_2}\tmu{j_1}}^{IV}}{(\tmu{i_1}+\tmu{j_1})(\tmu{i_1}+\tmu{j_2})(\tmu{i_2}+\tmu{j_1})(\tmu{i_2}+\tmu{j_2})}\\
&= \frac{\overbrace{\tmu{i_2}\tmu{j_2}\tmu{i_2}\tmu{i_1}}^{III}+\overbrace{\tmu{i_2}\tmu{j_2}\tmu{i_2}\tmu{j_1}}^{II}+\overbrace{\tmu{i_2}\tmu{i_1}\tmu{j_1}\tmu{i_2}}^{I}+\overbrace{\tmu{i_2}\tmu{i_1}\tmu{j_1}\tmu{j_2}}^{IV}}{(\tmu{i_1}+\tmu{j_1})(\tmu{i_1}+\tmu{j_2})(\tmu{i_2}+\tmu{j_1})(\tmu{i_2}+\tmu{j_2})}\\
&= \frac{\tmu{i_2}\tmu{j_2}\tmu{i_2}(\tmu{i_1}+\tmu{j_1})+\tmu{i_2}\tmu{i_1}\tmu{j_1}(\tmu{i_2}+\tmu{j_2})}{(\tmu{i_1}+\tmu{j_1})(\tmu{i_1}+\tmu{j_2})(\tmu{i_2}+\tmu{j_1})(\tmu{i_2}+\tmu{j_2})}\\
&=\bl p_{{i_2},{j_1}}({j_1})\left(\bl p_{{i_1},{j_2}}({i_1})\bl p_{{i_2},{j_2}}({j_2})+\bl p_{{i_1},{j_2}}({j_2})\bl p_{{i_1},{j_1}}({i_1})  \right)
\end{align*}
\end{proofof}


\begin{proposition}\label{prop:case of two}
%Let $\abs{\ug}\geq 2$ and $\abs{\ul} = 2$. For any pair of policies $\pi,\rho$, it holds that $Q(\pi,\ug,\ul)=Q(\rho,\ug,\ul)$.
Let $U$ be an arbitrary state, such that $\ug\defeq\above(U)\geq 2$ and $\ul\defeq\below(U) = 2$. For any pair of $\mP$-valid policies $\pi$ and $\rho$, it holds that $Q(\pi,U)=Q(\rho,U)$.
\end{proposition}
\begin{proofof}{Proposition \ref{prop:case of two}}
The proof of this proposition goes along the lines of the proof of Proposition \ref{prop:case of two strong}, but we provide the details here for completeness. For simplicity, we let $\overline Q = 1-Q$, and prove that for any two policies $\pi, \rho$ it holds that $\overline Q(\pi,U)=\overline Q(\rho,U)$.

We prove the claim by induction, with Proposition \ref{prop:case of one strong} serving as the base case. Assume the claim holds for $\abs{\ug}=k-1$. It is enough to show that if $\abs{\ug}=k$, for any $a_i\in \ug,a_j\in \ul$, $Q^*_{i,j}(U)=Q^*(U)$. Assume that $Q^*_{{i_1},{j_1}}(U)=Q^*(U)$, and define a policy $\pi$ such that $\sigr_\pi=(a_{j_1},a_{j_2})$ and $\sigl_\pi=(a_{i_1},a_{i_2},\dots a_{i_k})$. Next, fix any $a_{i'}\in \ug, a_{j'}\in \ul$. 
\paragraph{Remark} We do not use Assumption \ref{assumption:dominance} here.
\paragraph{Step 1} Assume that $i'\neq i_1$ and $j' = j_1$. W.l.o.g. $i'=i_2$. We construct the ordered policy $\rho$ that orders $\ul$ by $\sigr_\rho=\sigr_\pi=(a_{j_1},a_{j_2})$ and $\ug$ by $\sigl_\rho=(a_{i_2},a_{i_1},\dots, a_{i_k})$. Due to the inductive step and our assumption that  $Q^*_{i_1,j_1}(U)=Q^*(U)$, we have that $\overline Q(\pi,U)=\overline Q^*(U)$. For brevity, we introduce the following notations. For $r\in \{1,2,3\}$, let
{\thinmuskip=.2mu
\medmuskip=0mu plus .2mu minus .2mu
\thickmuskip=1mu plus 1mu
\[
\lambda^r_{j_1}=\prod_{l=r}^{k} \bl p_{{i_l},{j_1}}(a_{i_l}),\lambda^r_{j_2}=\prod_{l=r}^{k} \bl p_{{i_l},{j_2}}(a_{i_l}), \delta^r= \sum_{f=r}^{k} \left(\prod_{l=r}^{f-1} \bl p_{{i_l},{j_1}}(a_{i_l})\right) \bl p_{{i_f},{j_1}}(a_{j_1}) \left(\prod_{l=f}^{k} \bl p_{{i_l},{j_2}}(a_{i_l})\right).
\]
}%
Observe that
\begin{align}\label{eq:from pi to ghjghj}
\overline Q(\pi,U)&=\lambda^1_{j_1}+\delta^1=\lambda^1_{j_1}+\bl p_{i_1,j_1}(a_{j_1})\lambda^1_{j_2}+\bl p_{i_1,j_1}(a_{i_1})\delta^2\nonumber\\
&=\lambda^1_{j_1}+\bl p_{i_1,j_1}(a_{j_1})\bl p_{i_1,j_2}(a_{i_1})\bl p_{i_2,j_2}(a_{i_2})\lambda^3_{j_2}\nonumber\\
&\qquad \qquad+\bl p_{i_1,j_1}(a_{i_1})\left[
\bl p_{i_2,j_1}(a_{j_1})\bl p_{i_2,j_2}(a_{i_2})\lambda^3_{j_2}+\bl p_{i_2,j_1}(a_{i_2})\delta^3\right].
\end{align}
Next, we show that 
\begin{claim}\label{claim:triplets additional}
It holds that
\begin{align*}
&\bl p_{{i_2},{j_2}}(a_{i_2})\left( \bl p_{{i_1},{j_1}}(a_{j_1})\bl p_{{i_1},{j_2}}(a_{i_1})+\bl p_{{i_1},{j_1}}(a_{i_1})\bl p_{{i_2},{j_1}}(a_{j_1})\right)\\
&=\bl p_{{i_1},{j_2}}(a_{i_1})\left(\bl p_{{i_2},{j_1}}(a_{j_1})\bl p_{{i_2},{j_2}}(a_{i_2})+\bl p_{{i_2},{j_1}}(a_{i_2})\bl p_{{i_1},{j_1}}(a_{j_1})  \right).
\end{align*}
\end{claim}
Combining Equation \refeq{eq:from pi to ghjghj} and Claim \ref{claim:triplets additional}, we get
\begin{align}\label{eq: flipping}
\textnormal{Eq. } \refeq{eq:from pi to ghjghj}
&=\lambda^1_{j_1}+\bl p_{i_2,j_1}(a_{j_1})\bl p_{i_2,j_2}(a_{i_2})\bl p_{i_1,j_2}(a_{i_1})\lambda^3_{j_2}\nonumber\\
&\qquad \qquad+\bl p_{i_2,j_1}(a_{i_2})\left[
\bl p_{i_1,j_1}(a_{j_1})\bl p_{i_1,j_2}(a_{i_1})\lambda^3_{j_2}+\bl p_{i_1,j_1}(a_{i_1})\delta^3\right]\nonumber \\
&=\lambda^1_{j_1}+\bl p_{i_2,j_1}(a_{j_1})\lambda^1_{j_2}+\bl p_{i_2,j_1}(a_{i_2})\left[
\bl p_{i_1,j_1}(a_{j_1})\bl p_{i_1,j_2}(a_{i_1})\lambda^3_{j_2}+\bl p_{i_1,j_1}(a_{i_1})\delta^3\right]
\end{align}
Now, set $\sigma:\mathbb N \rightarrow \mathbb N$ such that $\sigma(1)=2, \sigma(2)=1$, and $\sigma(i)=i$ for $i\geq 3$; hence,
{\thinmuskip=.2mu
\medmuskip=0mu plus .2mu minus .2mu
\thickmuskip=1mu plus 1mu
\begin{align*}
\textnormal{Eq. }\refeq{eq: flipping}&=\lambda^1_{j_1}+\bl p_{i_2,j_1}(a_{j_1})\lambda^1_{j_2}\\
&\qquad \qquad +\bl p_{i_2,j_1}(a_{i_2})\left[ 
\sum_{f=2}^{k} \left(\prod_{l=2}^{f-1} \bl p_{{i_{\sigma(l)}},{j_1}}(a_{i_{\sigma(l)}})\right) \bl p_{{i_{\sigma (f)}},{j_1}}(a_{j_1}) \left(\prod_{l=f}^{k} \bl p_{{i_{\sigma(l)}},{j_2}}(a_{i_{\sigma(l)}})\right)
 \right] \\
&= \lambda^1_{j_1}+ \sum_{f=1}^{k} \left(\prod_{l=1}^{f-1} \bl p_{{i_{\sigma(l)}},{j_1}}(a_{i_{\sigma(l)}})\right) \bl p_{{i_{\sigma (f)}},{j_1}}(a_{j_1}) \left(\prod_{l=f}^{k} \bl p_{{i_{\sigma(l)}},{j_2}}(a_{i_{\sigma(l)}})\right)\\
&= \overline Q(\rho,U).
\end{align*}
}
This concludes Step 1.
\paragraph{Step 2} Assume that $i'=i_1$ and $j' = j_2 \neq j_1$. We construct two ordered policies, $\rho$ and $\tilde \rho$ such that $\sigr_\rho=\sigr_{\tilde \rho}=(a_{j_2},a_{j_1})$ and $\sigl_\rho=(a_{i_1},a_{i_2},\dots, a_{i_k})$, $\sigl_{\tilde \rho}=(a_{i_k},a_{i_2},\dots, a_{i_{k-1}},a_{i_1})$. The previous Step 1 implies that $\overline Q(\rho,U)=\overline Q(\tilde \rho, U)$; thus, it suffices to show that $\overline Q(\pi,U)=\overline Q(\tilde \rho, U)$. Notice that
{\thinmuskip=.2mu
\medmuskip=0mu plus .2mu minus .2mu
\thickmuskip=1mu plus 1mu
\begin{align}\label{jnknkfdas}
\overline Q(\pi,U)&=\bl p_{i_k,j_1}(a_{i_k})\overbrace{\prod_{l=1}^{k-1} \bl p_{{i_l},{j_1}}(a_{i_l})}^{I_1}\nonumber \\
&+\bl p_{i_k,j_2}(a_{i_k})\left[\bl p_{i_k,j_1}(a_{j_1})\underbrace{\prod_{l=1}^{k-1} \bl p_{{i_l},{j_1}}(a_{i_l})}_{I_1}+\underbrace{\sum_{f=1}^{k-1} \left(\prod_{l=1}^{f-1} \bl p_{{i_l},{j_1}}(a_{i_l})\right) \bl p_{{i_f},{j_1}}(a_{j_1}) \left(\prod_{l=f}^{k-1} \bl p_{{i_l},{j_2}}(a_{i_l})\right)}_{I_2}  \right]
\end{align}}%
Rearranging,
\begin{align}\label{eq:kijnjvfvs}
\textnormal{Eq. }\refeq{jnknkfdas} &=\bl p_{i_k,j_1}(a_{i_k})I_1+\bl p_{i_k,j_2}(a_{i_k})\left[\bl p_{i_k,j_1}(a_{j_1})I_1+I_2  \right]. \nonumber \\
&= \bl p_{i_k,j_1}(a_{i_k})\left( \bl p_{i_k,j_2}(a_{i_k})+\bl p_{i_k,j_2}(a_{j_2}) \right) I_1+\bl p_{i_k,j_2}(a_{i_k})\left[\bl p_{i_k,j_1}(a_{j_1})I_1+I_2  \right]. \nonumber \\
&= \bl p_{i_k,j_1}(a_{i_k})\bl p_{i_k,j_2}(a_{j_2})I_1+\bl p_{i_k,j_2}(a_{i_k})\left[I_1+I_2  \right]. 
\end{align}
Recall that
\begin{align} \label{porenkjnf}
I_1+I_2=\prod_{l=1}^{k-1} \bl p_{{i_l},{j_1}}(a_{i_l})+\sum_{f=1}^{k-1} \left(\prod_{l=1}^{f-1} \bl p_{{i_l},{j_1}}(a_{i_l})\right) \bl p_{{i_f},{j_1}}(a_{j_1}) \left(\prod_{l=f}^{k-1} \bl p_{{i_l},{j_2}}(a_{i_l})\right),
\end{align}
which is precisely $\overline Q(\pi,\left\{a_{i_1},\dots, a_{i_{k-1}},a_{j_1},a_{j_2}\right\})$; thus, the inductive step implies that it is order invariant. Let $\sigma:\mathbb N \rightarrow \mathbb N$ such that $\sigma(1)=k,\sigma(k)=1$, and $\sigma(i)=i$ for $1<i<k$. Since $\overline Q(\pi,\left\{a_{i_1},\dots, a_{i_{k-1}},a_{j_1},a_{j_2}\right\})=\overline Q(\tilde \rho,\left\{a_{i_1},\dots, a_{i_{k-1}},a_{j_1},a_{j_2}\right\}))$, we conclude that the expression in Equation (\ref{porenkjnf}) equals 
\[
\prod_{l=2}^{k} \bl p_{{\sigma (i_l)},{j_2}}(a_{\sigma (i_l)})+\sum_{f=2}^{k} \left(\prod_{l=2}^{f-1} \bl p_{{\sigma (i_l)},{j_2}}(a_{\sigma (i_l)})\right) \bl p_{{\sigma (i_f)},{j_2}}(a_{j_2}) \left(\prod_{l=f}^{k} \bl p_{{\sigma (i_l)},{j_1}}(a_{\sigma (i_l)})\right).
\]
Notice that in the above expression, we first try to explore $a_{j_2}$, and only then $a_{j_1}$. Combining this with Equation \refeq{eq:kijnjvfvs},
{\thinmuskip=.2mu
\medmuskip=0mu plus .2mu minus .2mu
\thickmuskip=1mu plus 1mu
\begin{align*}
&\textnormal{Eq. } \refeq{eq:kijnjvfvs}=\bl p_{i_k,j_1}(a_{i_k})\bl p_{i_k,j_2}(a_{j_2})I_1\\
&\quad + \bl p_{i_k,j_2}(a_{i_k})\left[\prod_{l=2}^{k} \bl p_{{\sigma (i_l)},{j_2}}(a_{\sigma (i_l)})+\sum_{f=2}^{k} \left(\prod_{l=2}^{f-1} \bl p_{{\sigma (i_l)},{j_2}}(a_{\sigma (i_l)})\right) \bl p_{{\sigma (i_f)},{j_2}}(a_{j_2}) \left(\prod_{l=f}^{k} \bl p_{{\sigma (i_l)},{j_1}}(a_{\sigma (i_l)})\right) \right]. \\
&= \bl p_{i_k,j_1}(a_{i_k})\bl p_{i_k,j_2}(a_{j_2})\prod_{l=1}^{k-1} \bl p_{{i_l},{j_1}}(a_{i_l})+\bl p_{i_k,j_2}(a_{i_k})\prod_{l=2}^{k} \bl p_{{\sigma (i_l)},{j_2}}(a_{\sigma (i_l)})\\
&\quad + \bl p_{i_k,j_2}(a_{i_k})\left[\sum_{f=2}^{k} \left(\prod_{l=2}^{f-1} \bl p_{{\sigma (i_l)},{j_2}}(a_{\sigma (i_l)})\right) \bl p_{{\sigma (i_f)},{j_2}}(a_{j_2}) \left(\prod_{l=f}^{k} \bl p_{{\sigma (i_l)},{j_1}}(a_{\sigma (i_l)})\right) \right]\\
&=\bl p_{i_k,j_2}(a_{j_2})\prod_{l=1}^{k} \bl p_{{\sigma(i_l)},{j_1}}(a_{\sigma(i_l)})+\prod_{l=1}^{k} \bl p_{{\sigma (i_l)},{j_2}}(a_{\sigma (i_l)})\\
&\quad + \bl p_{i_k,j_2}(a_{i_k})\left[\sum_{f=2}^{k} \left(\prod_{l=2}^{f-1} \bl p_{{\sigma (i_l)},{j_2}}(a_{\sigma (i_l)})\right) \bl p_{{\sigma (i_f)},{j_2}}(a_{j_2}) \left(\prod_{l=f}^{k} \bl p_{{\sigma (i_l)},{j_1}}(a_{\sigma (i_l)})\right) \right]\\
&=\prod_{l=1}^{k} \bl p_{{\sigma (i_l)},{j_2}}(a_{\sigma (i_l)})+\sum_{f=1}^{k} \left(\prod_{l=1}^{f-1} \bl p_{{\sigma (i_l)},{j_2}}(a_{\sigma (i_l)})\right) \bl p_{{\sigma (i_f)},{j_2}}(a_{j_2}) \left(\prod_{l=f}^{k} \bl p_{{\sigma (i_l)},{j_1}}(a_{\sigma (i_l)})\right), 
\end{align*}}%
and the latter is precisely $\overline Q(\tilde \rho, U)$.
\paragraph{Step 3}
The two previous steps imply that for any $a_{i'}\in \ug, a_{j'}\in \ul$, it holds that
\[
\overline Q_{i',j'}(U)=\overline Q_{i',j_1}(U)= \overline Q_{i_1,j_1}(U).
\]
This completes the proof of Proposition \ref{prop:case of two}.
\end{proofof}
\begin{proofof}{Claim \ref{claim:triplets additional}}
To ease readability, let $\tmu{i}\defeq\abs{\mu(a_i)}$ for every $a_i\in A$. It holds that
\begin{align*}
&\bl p_{{i_2},{j_2}}(a_{i_2})\left( \bl p_{{i_1},{j_1}}(a_{j_1})\bl p_{{i_1},{j_2}}(a_{i_1})+\bl p_{{i_1},{j_1}}(a_{i_1})\bl p_{{i_2},{j_1}}(a_{j_1})\right)\\
&=\frac{\tmu{j_2}}{\tmu{i_2}+\tmu{j_2}}\left( \frac{\tmu{i_1}}{\tmu{i_1}+\tmu{j_1}}\frac{\tmu{j_2}}{\tmu{i_1}+\tmu{j_2}}+\frac{\tmu{j_1}}{\tmu{i_1}+\tmu{j_1}}\frac{\tmu{i_2}}{\tmu{i_2}+\tmu{j_1}}\right) \\
&= \frac{\tmu{j_2}\tmu{i_1}\tmu{j_2}(\tmu{i_2}+\tmu{j_1})+\tmu{j_2}\tmu{j_1}\tmu{i_2}(\tmu{i_1}+\tmu{j_2})}{(\tmu{i_1}+\tmu{j_1})(\tmu{i_1}+\tmu{j_2})(\tmu{i_2}+\tmu{j_1})(\tmu{i_2}+\tmu{j_2})}\\
&= \frac{\overbrace{\tmu{j_2}\tmu{i_1}\tmu{j_2}\tmu{i_2}}^{I}+\overbrace{\tmu{j_2}\tmu{i_1}\tmu{j_2}\tmu{j_1}}^{II}+\overbrace{\tmu{j_2}\tmu{j_1}\tmu{i_2}\tmu{i_1}}^{III}+\overbrace{\tmu{j_2}\tmu{j_1}\tmu{i_2}\tmu{j_2}}^{IV}}{(\tmu{i_1}+\tmu{j_1})(\tmu{i_1}+\tmu{j_2})(\tmu{i_2}+\tmu{j_1})(\tmu{i_2}+\tmu{j_2})}\\
&= \frac{\overbrace{\tmu{j_2}\tmu{i_1}\tmu{j_2}\tmu{i_2}}^{I}+\overbrace{\tmu{j_2}\tmu{j_1}\tmu{i_2}\tmu{j_2}}^{IV}+\overbrace{\tmu{j_2}\tmu{j_1}\tmu{i_2}\tmu{i_1}}^{III}+\overbrace{\tmu{j_2}\tmu{i_1}\tmu{j_2}\tmu{j_1}}^{II}}{(\tmu{i_1}+\tmu{j_1})(\tmu{i_1}+\tmu{j_2})(\tmu{i_2}+\tmu{j_1})(\tmu{i_2}+\tmu{j_2})}\\
&= \frac{\tmu{j_2}\tmu{i_2}\tmu{j_2}(\tmu{i_1}+\tmu{j_1})+\tmu{j_2}\tmu{j_1}\tmu{i_1}(\tmu{i_2}+\tmu{j_2})}{(\tmu{i_1}+\tmu{j_1})(\tmu{i_1}+\tmu{j_2})(\tmu{i_2}+\tmu{j_1})(\tmu{i_2}+\tmu{j_2})}\\
&=\bl p_{{i_1},{j_2}}(a_{i_1})\left(\bl p_{{i_2},{j_1}}(a_{j_1})\bl p_{{i_2},{j_2}}(a_{i_2})+\bl p_{{i_2},{j_1}}(a_{i_2})\bl p_{{i_1},{j_1}}(a_{j_1})  \right).
\end{align*}
\end{proofof}


\section{Proof of Theorem \ref{thm:holy grail}}\label{sec:proof of thm}
\begin{proofof}{Theorem \ref{thm:holy grail}}
Fix an arbitrary instance. We prove the claim by a two-dimensional induction on the size of $\above(s),\below(s)$, for states $s\in \mS$. The base cases are 
\begin{itemize}
\item $\abs{\above(s)}\geq 2 $ and $\abs{\below(s)} = 1$ (Proposition \ref{prop:W case of one}), and
\item $\abs{\above(s)}=1$ and $\abs{\below(s)}\geq 2$ (Proposition \ref{prop:W case of one strong}),
\end{itemize}
which we relegate to Section \ref{sec:for theorem}. Next, assume the statement holds for all $s\in \mS$ such that $\abs{\above(s)}\leq K_1$, $\abs{\below(s)}\leq K_2$ and $\abs{\above(s)}+\abs{\below(s)}< K_1+K_2$. Let $U\in\mS$ denote a state with $\abs{\above(U)}=K_1$ and $\abs{\below(U)}=K_2$. For abbreviation, let $\ug\defeq\above(U)=\{a_{i_1},a_{i_2},\dots ,a_{i_{K_1}}\}$ and $\ul\defeq\below(U)=\{a_{j_1},a_{j_2},\dots ,a_{j_{K_2}}\}$, and assume the indices follow the stochastic order. Further, for every $a_i \in \ug, a_j\in \ul$ let 
\[
W^*_{i,j}(\ug,\ul)\defeq\bl p_{i,j}(a_j)W^*(\ug,\ul\setminus \{a_j\}) +\bl p_{i,j}(a_i)W^*(\ug\setminus\{a_i\},\ul).
\]
We need to prove that $W^*_{{i_1},{j_1}}(s)=W^*(s)$.
\paragraph{Remarks} Notice that if $X_{a_i'}>0$ for $a_{i'}\in \above(A)$, any $\mP$-valid policy gets $W^*(s)$. To see this, recall that $\mP$-valid policies reach terminate states only after exploring all arms in $\above(A)$. Consequently, we assume for the rest of the proof that $X_{a_i'}\leq 0$ for $a_{i'}\in \above(A)$. 
\paragraph{Step 1}
\begin{figure}[htbp]
\centering
\forestset{
 strongedge label/.style 2 args={
    edge label={node[midway,left, #1]{#2}},
  }, 
 weakedge label/.style 2 args={
    edge label={node[midway,right, #1]{#2}},
  }, 
   straightedge label/.style 2 args={
    edge label={node[midway, #1]{#2}},
  }, 
  important/.style={draw={red,thick,fill=red}}
}
\begin{forest} 
[{\Large $\pi$},
 [{$(U_>{,}U_<)$}, edge={white},l*=.05, for tree=
	{
	draw,
	font=\sffamily,
	l+=.5cm,
	inner sep=2pt,
	l sep=5pt,
	s sep=5pt,
	parent anchor=south,
	child anchor=north
    }
 	[${(U_>\setminus\{ a_{i'}\},U_<)}$, strongedge label={left}{$\bl p_{{i'},{j_k}}(a_{i'})$}
 		[,white, edge={dashed}, straightedge label={fill=white}{$\pi^*$}
 		]
 	]
	[${(U_>,U_<\setminus\{ a_{j_k}\})}$, weakedge label={right}{$\bl p_{{i'},{j_k}}(a_{j_k})$}
 		[,white, edge={dashed}, straightedge label={fill=white}{$\pi^*$}
 		]
 	]
 ]
]
\end{forest}
\qquad
\begin{forest} 
[{\Large $\rho$},
 [{$(U_>{,}U_<)$}, edge={white},l*=.05, for tree=
	{
	draw,
	font=\sffamily,
	l+=.5cm,
	inner sep=2pt,
	l sep=5pt,
	s sep=5pt,
	parent anchor=south,
	child anchor=north
    }
 	[${(U_>\setminus\{ a_{i'}\},U_<)}$, strongedge label={left}{$\bl p_{{i'},{j_k}}(a_{i'})$}
 		[,white, edge={dashed}, straightedge label={fill=white}{$\pi^*$}
 		]
 	]
	[${(U_>,U_<\setminus\{ a_{j_{k+1}}\})}$, weakedge label={right}{$\bl p_{{i'},{j_{k+1}}}(a_{j_{k+1}})$}
 		[,white, edge={dashed}, straightedge label={fill=white}{$\pi^*$}
 		]
 	]
 ]
]
\end{forest}
\caption{Illustration of the policies $\pi,\rho$ from Step 1 of Theorem \ref{thm:holy grail}. Notice that the left sub-trees of $\pi$ and $\rho$ are identical.\label{fig:helping for thm two}
}% caption command
\end{figure}

Fix $a_{i'} \in \ug$. We show that for every $k$, $1\leq k<K_2$ it holds that $W^*_{{i'},{j_k}}(s) \geq W^*_{{i'},{j_{k+1}}}(s)$. We define the ordered, $\mP$-valid policy $\pi^*$ by $\sigl_\pi = (a_{i'},\dots)$  namely, $\sigl_{\pi^*}$ ranks $a_{i'}$ first and all the other arms in $\ug$ arbitrarily, and $\sigr_{\pi^*}=(a_{j_1},a_{j_2},\dots,a_{j_{K_2}})$. Due to the inductive assumption, for every state $s$ with $\abs{s}<\abs{U}$, $W^*(s) = W(\pi^*,s)$. Next, we define the policies $\pi,\rho$ explicitly, as follows:
\[
\pi(s) = 
\begin{cases}
\bl p_{{i'},{j_k}} & \textnormal{if $s=(\ug,\ul)$} \\
\pi^*(s)& \textnormal{otherwise} 
\end{cases},\quad 
\rho(s) = 
\begin{cases}
\bl p_{{i'},{j_{k+1}}} & \textnormal{if $s=(\ug,\ul)$} \\
\pi^*(s)& \textnormal{otherwise}
\end{cases}.
\]
We illustrate $\pi$ and $\rho$ in Figure \ref{fig:helping for thm two}. Note that both $\pi,\rho$ have on-path states that are off-path for $\pi^*$. For instance, $\rho$ reaches the state $(\ug,\ul\setminus\{a_{j_{k+1}}\})$ with positive probability, while $\pi^*$ cannot reach it at all. In addition, $\pi$ and $\rho$ are left-ordered with $\sigl_\pi=\sigl_\rho=\sigl_{\pi^*}$. Due to the inductive assumption, it is enough to show that $W(\pi,U)-W(\rho,U)\geq 0$, as this implies $W^*_{{i'},{j_k}}(s) \geq W^*_{{i'},{j_{k+1}}}(s)$.

Next, we factor $W(\pi,U)$ as follows: for every state $U'\subset U$, we factor $W(\pi,U')$ as long as $a_{j_k},a_{j_{k+1}}\in U'$. Once we reach a term $W(\pi,U')$ with $a_{j_k},a_{j_{k+1}}\notin U'$, we stop. Let $\Psi\defeq \prefix(a_{j_1},a_{j_2}\dots, a_{j_{k-1}})$ be the set of (possibly empty) prefixes of the first $k-1$ arms in $\ul$ according to $\pi^*$. Observe that\footnote{The reader can think of $\psi$ as the set of arms from $(a_{j_1},a_{j_2}\dots, a_{j_{k-1}})$ that were explored.}  
{\thinmuskip=.2mu
\medmuskip=0mu plus .2mu minus .2mu
\thickmuskip=1mu plus 1mu
\begin{align}\label{eq:w of pi}
W(\pi,U)&=\sum_{\psi\in \Psi}
\Pr(\pathto{s}{(\ul\setminus \psi)})R(\ul\setminus \psi)+\Pr(\pathto{s}{(\ul\setminus (\psi \cup \{a_{j_k}\})})R(\ul\setminus (\psi \cup \{a_{j_k}\})\nonumber \\
&\qquad +\underbrace{\sum_{\substack{Z\in \suff(\sigl_\pi)}}
f^\pi_Z \cdot W(\pi,Z,\ul\setminus \{a_{j_1},a_{j_2},\dots,a_{j_{k}},a_{j_{k+1}} \})}_{I^\pi}
\end{align}}
The coefficients $(f^\pi_Z)$ follow from the factorization process. Using similar factorization,
{\thinmuskip=.2mu
\medmuskip=0mu plus .2mu minus .2mu
\thickmuskip=1mu plus 1mu
\begin{align}\label{eq:w of rho}
W(\rho,U)&=\sum_{\psi\in \Psi}
\Pr(\pathtorho{s}{(\ul\setminus \psi)})R(\ul\setminus \psi)+\Pr(\pathtorho{s}{(\ul\setminus (\psi \cup \{a_{j_{k+1}}\}))})R(\ul\setminus (\psi \cup \{a_{j_{k+1}}\}))\nonumber\\
&\qquad +\underbrace{\sum_{\substack{Z\in \suff(\sigl_\rho)}}
f^\rho_Z \cdot W(\rho,Z,\ul\setminus \{a_{j_1},a_{j_2},\dots,a_{j_{k}},a_{j_{k+1}} \})}_{I^\rho}. 
\end{align}}%
Next, we express $(f^\pi_Z)_Z$ in terms of $Q$. By relying on the Equivalence lemma, we show that
\begin{claim}\label{claim:f are equal}
For every $Z\in \suff(\sigl_\pi)=\suff(\sigl_\rho)$, it holds that $f^\pi_Z=f^\rho_Z$.
\end{claim}
To see why Claim \ref{claim:f are equal} holds, notice that we can represent $f^\pi_Z$ as 
{\thinmuskip=.2mu
\medmuskip=0mu plus .2mu minus .2mu
\thickmuskip=1mu plus 1mu
\[
Q(\pi,\ug\setminus Z \cup \{a_{i(Z)}\}, \ul\setminus \{a_{j_1},a_{j_2},\dots,a_{j_{k}},a_{j_{k+1}} \})-Q(\pi,\ug\setminus Z, \ul\setminus \{a_{j_1},a_{j_2},\dots,a_{j_{k}},a_{j_{k+1}} \}),
\]}%
where $a_{i(Z)}$ is the minimal element in $Z$ according to $\sigl_\pi$. By invoking the Equivalence lemma, we can replace $\pi$ in the above expression with $\rho$ and thus for every $Z\in \suff(\sigl_\pi)=\suff(\sigl_\rho)$, it holds that $f^\pi_Z=f^\rho_Z$. Combining Claim \ref{claim:f are equal} with the inductive step and $\sigl_\pi,\sigl_\rho$ being equal, we conclude that $I^\pi=I^\rho$. 


Next, we focus on the first sum of $W(\pi,U)$ in Equation \refeq{eq:w of pi}. For every $\psi\in \Psi$, we denote the event $E^\pi_\psi$ as a shorthand for
\[
E^\pi_\psi \defeq \left(\pathto{s}{(\ul\setminus \psi)}\right)\cup\left( \pathto{s}{(\ul\setminus (\psi \cup \{a_{j_k}\}))}\right).
\]
In words, $E^\pi_\psi$ is the event that the GMDP reaches the final state $(\ul\setminus \psi)$ or $(\ul\setminus (\psi \cup \{a_{j_k}\}))$ after starting in $s_0$ and following $\pi$. We use conditional expectation to simplify the summands in the first sum of Equation (\ref{eq:w of pi}),
{\thinmuskip=.2mu
\medmuskip=0mu plus .2mu minus .2mu
\thickmuskip=1mu plus 1mu
\begin{align}\label{eq:pi with alpha}
&\Pr(\pathto{s}{(\ul\setminus \psi)})R(\ul\setminus \psi)+\Pr(\pathto{s}{(\ul\setminus (\psi \cup \{a_{j_k}\})})R(\ul\setminus (\psi \cup \{a_{j_k}\})\nonumber\\
&=\Pr\left(E^\pi_\psi\right)\cdot \underbrace{\left( \Pr(\pathto{s}{(\ul\setminus \psi)}\mid E^\pi_\psi)R(\ul\setminus \psi)+\Pr(\pathto{s}{(\ul\setminus (\psi \cup \{a_{j_k}\})}\mid E^\pi_\psi)R(\ul\setminus (\psi \cup \{a_{j_k}\}))\right)}_{\alpha^\pi_\psi}\nonumber\\
&=\Pr(E^\pi_\psi) \alpha^\pi_\psi. 
\end{align}}%
Similarly for $\rho$, by letting
\[
E^\rho_\psi \defeq \left(\pathtorho{s}{(\ul\setminus \psi)}\right)\cup\left( \pathtorho{s}{(\ul\setminus (\psi \cup \{a_{j_{k+1}}\}))}\right)
\]
and following the derivation in Equation \refeq{eq:pi with alpha} for $\rho$, we get
{\thinmuskip=.2mu
\medmuskip=0mu plus .2mu minus .2mu
\thickmuskip=1mu plus 1mu
\begin{align}\label{eq:rho with alpha}
&\Pr(\pathtorho{s}{(\ul\setminus \psi)})R(\ul\setminus \psi)+\Pr(\pathtorho{s}{(\ul\setminus (\psi \cup \{a_{j_{k+1}}\})})R(\ul\setminus (\psi \cup \{a_{j_{k+1}}\})\nonumber\\
&=\Pr\left(E^\rho_\psi\right)\cdot \underbrace{\left( \Pr(\pathtorho{s}{(\ul\setminus \psi)}\mid E^\rho_\psi)R(\ul\setminus \psi)+\Pr(\pathtorho{s}{(\ul\setminus (\psi \cup \{a_{j_{k+1}}\})}\mid E^\rho_\psi)R(\ul\setminus (\psi \cup \{a_{j_{k+1}}\})) \right)}_{\alpha^\rho_\psi}\nonumber\\
&=\Pr(E^\rho_\psi) \alpha^\rho_\psi. 
\end{align}}%
Due to the fact that $I^\pi=I^\rho$ and relying on Equations (\ref{eq:w of pi})-(\ref{eq:rho with alpha}), we have  
\begin{align}\label{eq:w minus w}
W(\pi,U)-W(\rho,U)=\sum_{\psi\in \Psi}\left(\Pr(E^\pi_\psi) \alpha^\pi_\psi-\Pr(E^\rho_\psi) \alpha^\rho_\psi\right).
\end{align}
At this point, we might be tempted to show that every summand of the sum in Equation \refeq{eq:w minus w} is non-negative. Unfortunately, this approach would not work---stochastic order does not mean $\Pr(E^\pi_\psi) \alpha^\pi_\psi\geq \Pr(E^\rho_\psi) \alpha^\rho_\psi$. Instead, we take a different approach. For every $l, 1\leq l \leq k-1$ let $E^\pi_{l}$ denote the event that $a_{j_l}$ was observed in the final state. Formally,
\[
E^\pi_{l} \defeq  \bigcup_{\substack{\psi' \in \Psi\\a_{j_l} \in \psi' }} \left(\pathto{s}{(\ul\setminus \psi')}\right)\cup\left(\pathto{s}{(\ul\setminus (\psi' \cup \{a_{j_k}\})}\right).
\]
In addition, we let $E^\pi_{k}$ denote the empty event, and $E^\pi_{0}$ be the full event (that occurs w.p.~$1$).   For every non-empty $\psi \in \Psi$, i.e. $\abs{\psi}\geq 1$, let $\max(\psi)\defeq\argmax_{j_l:{a_{j_l}} \in \psi} \sigr_{\pi^*}(a_{j_l})$. According to our assumption about the index of the arms, $\max(\psi)$ is simply the maximal index of an arm in $\psi$ (that is well-defined when $\abs{\psi}\geq 1$). In addition, for completeness, if $\psi=\emptyset$ we let $\max(\emptyset)=0$. We can use the terms $(E^\pi_{l})_{l=0}^k$ to provide an alternative form for $E^\pi_\psi$:
\[
E^\pi_\psi=E^\pi_{\max(\psi)}\setminus E^\pi_{\max(\psi)+1}.
\]
Put differently, for $\psi$ with $\max(\psi)<k-1$, $E^\pi_\psi$ can be viewed as the collection of all of events in which arm $a_{j_{\max(\psi)}}$ was explored, but arm $a_{j_{\max(\psi)+1}}$ was not (where the ``arm'' $a_{j_0}$ for $\psi=\emptyset$ refers to $E^\pi_{0}$). Recall that for $\psi$ with $\max(\psi)=k-1$, $E^\pi_\psi=E^\pi_{\max(\psi)}$ since $E^\pi_{k}$ is the empty event.

Since $E^\pi_{\max(\psi)+1}\subseteq E^\pi_{\max(\psi)}$, we have that for every $\psi\in \Psi$,
\begin{align}\label{eq:e pi with Z}
\Pr(E^\pi_\psi)=\Pr(E^\pi_{\max(\psi)})-\Pr(E^\pi_{\max(\psi)+1}),
\end{align}
taking care of edge cases too. By renaming $\alpha^\pi_\psi$ to $\alpha^\pi_{\max(\psi)}$, i.e., $\alpha^\pi_{\max(\psi)} \defeq \alpha^\pi_\psi $, and rearranging Equation \refeq{eq:w of pi} using Equations \refeq{eq:pi with alpha} and \refeq{eq:e pi with Z},
{\thinmuskip=.2mu
\medmuskip=0mu plus .2mu minus .2mu
\thickmuskip=1mu plus 1mu
\begin{align}\label{eq:w pi with diff}
W(\pi,U)&= I^\pi+\sum_{\psi\in \Psi} \Pr\left(E^\pi_\psi\right) \alpha^\pi_\psi=I^\pi+\sum_{l=0}^{k-1}\left(\Pr(E^\pi_{l})-\Pr(E^\pi_{l+1})\right) \alpha^\pi_l.
\end{align}}%
By defining $E^\rho_l$ analogously,
\[
E^\rho_{l} \defeq  \bigcup_{\substack{\psi' \in \Psi\\a_{j_l} \in \psi' }} \left(\pathtorho{s}{(\ul\setminus \psi')}\right)\cup\left(\pathtorho{s}{(\ul\setminus (\psi' \cup \{a_{j_{k+1}}\})}\right),
\]
and following similar arguments, we conclude that
{\thinmuskip=.2mu
\medmuskip=0mu plus .2mu minus .2mu
\thickmuskip=1mu plus 1mu
\begin{align}\label{eq:w rho with diff}
W(\rho,U)&= I^\rho+\sum_{\psi\in \Psi} \Pr\left(E^\rho_\psi\right) \alpha^\rho_\psi=I^\rho+\sum_{l=0}^{k-1}\left(\Pr(E^\rho_{l})-\Pr(E^\rho_{l+1})\right) \alpha^\rho_l.
\end{align}}%
By rephrasing Equation (\ref{eq:w minus w}) using Equations (\ref{eq:w pi with diff}) and (\ref{eq:w rho with diff}),
\begin{align}\label{eq: thm step 1 good}
W(\pi,U)-W(\rho,U)=\sum_{l=0}^{k-1}\left(\Pr(E^\pi_{l})-\Pr(E^\pi_{l+1})\right) \alpha^\pi_l-\left(\Pr(E^\rho_{l})-\Pr(E^\rho_{l+1})\right) \alpha^\rho_l.
\end{align}
Next, we show two monotonicity properties.
\begin{proposition}\label{prop: monotonicity in thm}
Under Assumption \ref{assumption:dominance},
\begin{enumerate}
\item for every $l\in \{0,1,\dots,k-1 \}$, it holds that $\alpha^\pi_l \geq \alpha^\rho_l$. \label{item:prop alphas rho pi}
%\item for every $l\in \{0,1,\dots,k-1 \}$, it holds that $\Pr\left(E^\pi_l\right) \geq \Pr\left(E^\rho_l\right)$.\label{item:prop E} {\red I am not sure we need it! Even after double checking}
\item for every $l\in \{0,1,\dots,k-2\}$, it holds that $\alpha^\pi_{l+1} \geq \alpha^\pi_{l}$ and $\alpha^\rho_{l+1} \geq \alpha^\rho_{l}$. \label{item:prop alphas alpha pi}
\end{enumerate}
\end{proposition}
In fact, this is the only place in the proof of Theorem \ref{thm:holy grail} where we rely on Assumption \ref{assumption:dominance}. Equipped with Proposition \ref{prop: monotonicity in thm}, we can make the final argument. For every $r,r\in \{1\dots,k-1\}$ let
\[
f(r) \defeq \left( \Pr(E^\pi_{r})-\Pr(E^\rho_{r}) \right)\alpha^\pi_{r-1}.
\]
In addition, let
\[
g(r)\defeq \sum_{l=0}^{r}\left(\Pr(E^\pi_{l})-\Pr(E^\pi_{l+1})\right) \alpha^\pi_l-\left(\Pr(E^\rho_{l})-\Pr(E^\rho_{l+1})\right) \alpha^\rho_l
\]
We shall show that for every $r,r\in \{0,\dots,k-2\}$ it holds that
\begin{align}\label{eq:thm last argument}
W(\pi,U)-W(\rho,U)\geq f(r+1)+g(r).
\end{align}
For $r=k-2$, we have
{
\begin{align*}
\textnormal{Eq. (\ref{eq: thm step 1 good})}&=\left(\Pr(E^\pi_{k-1})-\Pr(E^\pi_{k})\right) \alpha^\pi_{k-1}-\left(\Pr(E^\rho_{k-1})-\Pr(E^\rho_{k})\right) \alpha^\rho_{k-1}+g(k-2)\\
&\stackrel{\substack{E^\pi_{k},E^\rho_{k}\\\textnormal{are empty}}}{=}\Pr(E^\pi_{k-1})\alpha^\pi_{k-1}-\Pr(E^\rho_{k-1}) \alpha^\rho_{k-1}+g(k-2)\\
&\stackrel{\textnormal{Prop. \ref{prop: monotonicity in thm}.\ref{item:prop alphas rho pi}}}{\geq}\left(\Pr(E^\pi_{k-1})-\Pr(E^\rho_{k-1}) \right)\alpha^\pi_{k-1}+g(k-2)\\
&\stackrel{\textnormal{Prop. \ref{prop: monotonicity in thm}.\ref{item:prop alphas alpha pi}}}{\geq}\left(\Pr(E^\pi_{k-1})-\Pr(E^\rho_{k-1}) \right)\alpha^\pi_{k-2}+g(k-2)\\
&=f(k-1)+g(k-2).
\end{align*}}%
Assume Inequality \refeq{eq:thm last argument} holds for $r+1$. Then, for $r$ we have 
{\thinmuskip=.2mu
\medmuskip=0mu plus .2mu minus .2mu
\thickmuskip=1mu plus 1mu
\begin{align*}
f(r+1)+g(r) &= \left( \Pr(E^\pi_{r+1})-\Pr(E^\rho_{r+1}) \right)\alpha^\pi_{r}+g(r-1)\\
&\qquad \qquad +\left(\Pr(E^\pi_{r})-\Pr(E^\pi_{r+1})\right) \alpha^\pi_r-\underbrace{\left(\Pr(E^\rho_{r})-\Pr(E^\rho_{r+1})\right)}_{\geq 0,\textnormal{ Eq. \refeq{eq:e pi with Z}}}  \alpha^\rho_r\\
&\stackrel{\textnormal{Prop. \ref{prop: monotonicity in thm}.\ref{item:prop alphas rho pi}}}{\geq} \left( \Pr(E^\pi_{r+1})-\Pr(E^\rho_{r+1}) \right)\alpha^\pi_{r}+g(r-1)\\
&\qquad \qquad +\left(\Pr(E^\pi_{r})-\Pr(E^\pi_{r+1})\right) \alpha^\pi_r-\left(\Pr(E^\rho_{r})-\Pr(E^\rho_{r+1})\right) \alpha^\pi_r\\
&=\Pr(E^\pi_{r})\alpha^\pi_r-\Pr(E^\rho_{r}) \alpha^\pi_r +g(r-1)\\
&\stackrel{\textnormal{Prop. \ref{prop: monotonicity in thm}.\ref{item:prop alphas alpha pi}}}{\geq}\Pr(E^\pi_{r})\alpha^\pi_{r-1}-\Pr(E^\rho_{r}) \alpha^\pi_{r-1} +g(r-1)\\
&= f(r)+g(r-1).
\end{align*}}%
Ultimately, by setting $r=0$ in Inequality \refeq{eq:thm last argument},
\begin{align*}
W(\pi,U)-W(\rho,U)&\geq f(1)+g(0)\\
&=\left( \Pr(E^\pi_{1})-\Pr(E^\rho_{1}) \right)\alpha^\pi_{0}\\
&\qquad \qquad +\left(\Pr(E^\pi_{0})-\Pr(E^\pi_{1})\right) \alpha^\pi_0-\left(\Pr(E^\rho_{0})-\Pr(E^\rho_{1})\right) \alpha^\rho_0\\
&\geq\Pr(E^\pi_{0}) \alpha^\pi_0-\Pr(E^\rho_{0}) \alpha^\pi_0\\
&=0.
\end{align*}
This concludes the first step of the theorem.
\paragraph{Step 2}  In this step, we show that for every $k$, $1\leq k < K_1$ it holds that $W^*_{{i_k},{j_1}}(s) = W^*_{{i_{k+1}},{j_1}}(s)$. Define an ordered, $\mP$-valid policy $\pi^*$ by $\sigl_\pi = (a_{i_1},a_{i_2},\dots,a_{i_{K_1}})$,  namely, $\sigl_{\pi^*}$ ranks the elements of $\ug$ according to the stochastic order, and $\sigr_{\pi^*}=(a_{j_1},a_{j_2},\dots,a_{j_{K_2}})$. Due to the inductive assumption, for every state $s$ with $\abs{s}<\abs{U}$, $W^*(s) = W(\pi^*,s)$. Next, we define the policies $\pi,\rho$ explicitly, as follows:
\[
\pi(s) = 
\begin{cases}
\bl p_{{i_k},{j_1}} & \textnormal{if $s=(\ug,\ul)$} \\
\pi^*(s)& \textnormal{otherwise} 
\end{cases},\quad 
\rho(s) = 
\begin{cases}
\bl p_{{i_{k+1}},{j_{1}}} & \textnormal{if $s=(\ug,\ul)$} \\
\pi^*(s)& \textnormal{otherwise}
\end{cases}.
\]
As in the previous step, the inductive step suggests that showing $W(\pi,s)=W(\rho,s)$ is suffice. However, unlike the previous step, here the set of reachable terminal state is the same for $\pi$ and $\rho$; hence, this equality is almost immediate due to the Equivalence lemma. Let $\Psi'\defeq \prefix(a_{j_1},a_{j_2}\dots, a_{j_{K_2-1}})$ be the set of (possibly empty) prefixes of the arms in $\ul \setminus \{a_{j_{K_2}}\}$ according to $\pi^*$. Observe that we can factor $W(\pi,s)$ as follows:
\begin{align}\label{eq:w for pi with prob}
W(\pi,s) &= Q(\pi,U)\cdot R(\emptyset)+ \sum_{\psi \in \Psi}\Pr(\pathto{s}{(\ul \setminus \psi)})R(\ul \setminus \psi).
\end{align}
The next Claim \ref{claim:thm:step 2 claim} suggests we can replace probabilities with functions of $Q$.
\begin{claim}\label{claim:thm:step 2 claim}
For every $\psi \in \Psi$, it holds that 
\[
\Pr(\pathto{s}{(\ul \setminus \psi)})=Q(\pi,\ug, \psi)-Q(\pi,\ug,\psi \cup\{a_{j_{max(\psi)+1}} \}).
\]
\end{claim}
By applying the Equivalence lemma on the statement of  Claim \ref{claim:thm:step 2 claim}, we obtain that for every $\psi \in \Psi$
\[
\Pr(\pathto{s}{(\ul \setminus \psi)})=\Pr(\pathtorho{s}{(\ul \setminus \psi)});
\]
hence, we can rewrite Equation \refeq{eq:w for pi with prob} as 
\begin{align*}
W(\pi,s) &= Q(\rho,U)\cdot R(\emptyset)+ \sum_{\psi \in \Psi}\Pr(\pathtorho{s}{(\ul \setminus \psi)})R(\ul \setminus \psi)\\
&=W(\rho,s).
\end{align*}
This concludes the second step of the theorem.
\paragraph{Step 3 (final)} We are ready to prove the theorem. Fix arbitrary $a_{\tilde i}$ and $a_{\tilde j}$ such that $a_{\tilde i} \in \ug$ and  $a_{\tilde j} \in \ul$. By the previous steps, we know that
\[
W^*_{{i_1},{j_1}}(U)\stackrel{\textnormal{Step 2}}{=}W^*_{{\tilde i},{j_1}}(U)\stackrel{\textnormal{Step 1}}{\geq}W^*_{{\tilde i},{\tilde j}}(U).
\]
This ends the proof of Theorem \ref{thm:holy grail}.
\end{proofof}



\section{Statements for Theorem \ref{thm:holy grail}}\label{sec:for theorem}
%\omer{the next files is commented out}
\begin{proposition}\label{prop:W case of one}
Let $U$ be a state such that $\abs{\above(U)}\geq 2$ and $\abs{\below(U)} =1$. 
It holds that $W(\pi^\star,U)=W^*(U)$.
\end{proposition}
\begin{proofof}{Proposition \ref{prop:W case of one}}
Denote $\below(U)=\{a_{j}\}$. We can assume w.l.o.g. that the realization of all arms in $\above(U)$ are non-positive, as otherwise every $\mP$-valid policy will explore all the arms; thus, $W^*(U)=Q^*(U)\cdot \max\{0, X_{a_{j}}  \}$. Finally, the Equivalence lemma suggests that   $Q^*(U)$ is policy invariant; hence, $W(\pi,U)=W^*(U)$ holds for any $\mP$-valid policy $\pi$, and in particular for $\pi=\pi^\star$.
\end{proofof}

\begin{proposition}\label{prop:W case of one strong}
Let $U$ be a state such that $\abs{\above(U)}=1$ and $\abs{\below(U)} \geq 2$. It holds that $W(\pi^\star,U)=W^*(U)$.
\end{proposition}
\begin{proofof}{Proposition \ref{prop:W case of one strong}}
This statement is a special case of Proposition \ref{prop:index with ugeq one} for instances satisfying Assumption \ref{assumption:dominance}.
\end{proofof}
\begin{proposition}\label{prop:index with ugeq one}
Let $U$ be a state such that $\abs{\above(U)}=1$ and $\abs{\below(U)} \geq 2$. Let $f^*$ be a real-valued function, $f^*:\below(U)\rightarrow \mathbb R$, such that for every $a_l \in \below(U)$,
\[
f^*(a_{l})=\frac{\Pr(X_{a_l} > 0 )\E(\max_{a_j \in {U} } X_{a_j}\mid X_{a_l}>0 )}{\abs{\mu(a_l)}}.
\]  
Denote by $\pi_{f^*}$ the right-ordered policy that orders $\below(U)$ according to decreasing order of $f^*$. Then, $W(\pi_{f^*},U)=W^*(U)$. 
%Let $R_1 \in [H]$ and $U\subseteq A$, and assume $\{a_i\}= U_>$ and $\abs{U_<}\geq 2$. If $\pi^\star(U,R_1,R_1)=\bl p_{ir}$ and $\pi^\star(U\setminus\{a_r' \},R_1,R_1)=\bl p_{ir'}$, then $\text{\red{what is the condition here}}$. Namely, if $\pi^\star$ selects $a_r$ before $a_{r'}$, then it has a higher \red{index}. \omer{Assuming the "order" I talked to Gal about}
\end{proposition}

\begin{proofof}{Proposition \ref{prop:index with ugeq one}}
Denote $\above(U)=\ug=\{a_{i}\}$  and $\below(U)=\ul=\{a_{j_1},\dots a_{j_k}\}$ for $k=\abs{\ul}$. Let $\pi$ be any right-ordered policy with the matching $\sigr_\pi$, such that $\pi\neq \pi_{f^*}$. Assume that there are indices $r,l$, for $1\leq r,l \leq k$, such that $\sigr_\pi(a_{l})<\sigr_\pi(a_{r})$ yet $f(a_{l})<f(a_{{r}})$, for arms $a_l,a_r\in \ul$. Moreover, if such a pair $(l,r)$ exists,  assume w.l.o.g. that $\sigr_\pi$ orders them consequentially, i.e., for every arm $a\in U$ such that $a\notin \{ a_{l}, a_{r} \}$,  either $\sigr_\pi(a)< \sigr_\pi(a_{l})$ or $\sigr_\pi(a)> \sigr_\pi(a_{r})$.


Denote by $\pi'$ the right-ordered policy that swaps $a_{l}$ and $a_{r}$. If we show that $\pi'$ yields a better reward than $\pi$, we could swap the order of $\pi$ one pair at a time, thereby showing that $\pi_{f^*}$ is indeed optimal. To simplify notation, we denote by $\ind_{a}$ the event that $X_a>0$ for arm $a\in U$.


Let $\ul'$ be the set of all arms in $\ul$ such that $\ul'=\{a\in \ul\mid \sigr_\pi(a)< \sigr_\pi(a_{l})  \}$. We divide the analysis into two cases: in case $X_{a_{i}} >0$ or $\max_{a\in \ul'} X_a > 0$, both $\pi,\pi'$ obtain the same reward. Otherwise, assume that $X_{a_{i}} \leq 0$ and $\max_{a\in \ul'} X_a \leq 0$ ; hence, in case arm $a_{i}$ is selected, the policy reaches a terminal state with a reward of 0. The reward of $\pi$ is given by 
{\small
\thinmuskip=.2mu
\medmuskip=0mu plus .2mu minus .2mu
\thickmuskip=1mu plus 1mu
\begin{align*}
W(\pi,U)=C_1 \left( \bl p_{i,l}(l)\E(\ind_l \max_{a_j \in {U_< \setminus \ul'}} X_{a_j} )+\bl p_{i,l}(l)\bl p_{i,r}(r)\E((1-\ind_l)\ind_r \max_{a_j \in {U_< \setminus \ul'}} X_{a_j})+\bl p_{i,l}(l)\bl p_{i,r}(r)C_2 \right),
\end{align*}}%
where $C_1$ and $C_2$ are constants that depend on $\sigr_\pi$. Similarly, the reward of $\pi'$ is given by 
{\small
\thinmuskip=.2mu
\medmuskip=0mu plus .2mu minus .2mu
\thickmuskip=1mu plus 1mu
\begin{align*}
W(\pi',U) = C_1\left( \bl p_{i,r}(r)\E(\ind_r \max_{a_j \in {U_< \setminus \ul'}} X_{a_j} )+\bl p_{i,r}(r)\bl p_{i,l}(l)\E((1-\ind_r)\ind_l \max_{a_j \in {U_< \setminus \ul'}} X_{a_j})+\bl p_{i,r}(r)\bl p_{i,l}(l)C_2 \right),
\end{align*}}%
where $C_1$ and $C_2$ are the same constants.
If $W(\pi,U)\geq W(\pi',U)$, then
\begin{align*}
& \bl p_{i,l}(l)\E(\ind_l \max_{a_j \in {U_< \setminus \ul'}} X_{a_j} )+\bl p_{i,l}(l)\bl p_{i,r}(r)\E((1-\ind_l)\ind_r \max_{a_j \in {U_< \setminus \ul'}} X_{a_j}) \\
&\qquad \geq \bl p_{i,r}(r)\E(\ind_r \max_{a_j \in {U_< \setminus \ul'}} X_{a_j} )+\bl p_{i,r}(r)\bl p_{i,l}(l)\E((1-\ind_r)\ind_l \max_{a_j \in {U_< \setminus \ul'}} X_{a_j}),
\end{align*}
implying that
\begin{align*}
&\bl p_{i,l}(l)\E(\ind_l \max_{a_j \in  {U_< \setminus \ul'}} X_{a_j} )(1-\bl p_{i,r}(r)) \geq \bl p_{i,r}(r)\E(\ind_r \max_{a_j \in  {U_< \setminus \ul'}} X_{a_j} )(1-\bl p_{i,l}(l)).
\end{align*}
Stated otherwise,
\begin{align*}
\frac{\bl p_{i,l}(l)\E(\ind_l \max_{a_j \in {U_< \setminus \ul'}} X_{a_j} )}{ 1-\bl p_{i,l}(l)} > \frac{\bl p_{i,r}(r)\E(\ind_r \max_{a_j \in {U_< \setminus \ul'}} X_{a_j} )}{1-\bl p_{i,r}(r) }.
\end{align*}
Finally, due to the definitions of $\bl p_{i,r},\ind_l$ and $\bl p_{i,l}, \ind_r$,
\begin{align*}
\frac{\Pr(X_{a_l} > 0 )\E(\max_{a_j \in {U_< \setminus \ul'} } X_{a_j}\mid X_{a_l}>0 )}{\abs{\mu(a_l)}} \geq \frac{\Pr(X_{a_l} > 0 )\E( \max_{a_j \in {U_< \setminus \ul'}} X_{a_j} \mid X_{a_r} >0 )}{\abs{\mu(a_r)}},
\end{align*}
which contradicts our assumption that $f(a_{l})<f(a_{r})$.
\end{proofof}


\begin{proofof}{Claim \ref{claim:f are equal}}
Notice that we can represent $f^\pi_Z$ as 
{\thinmuskip=.2mu
\medmuskip=0mu plus .2mu minus .2mu
\thickmuskip=1mu plus 1mu
\[
Q(\pi,\ug\setminus Z \cup \{a_{i(Z)}\}, \ul\setminus \{a_{j_1},a_{j_2},\dots,a_{j_{k}},a_{j_{k+1}} \})-Q(\pi,\ug\setminus Z, \ul\setminus \{a_{j_1},a_{j_2},\dots,a_{j_{k}},a_{j_{k+1}} \}),
\]}%
where $a_{i(Z)}$ is the minimal element in $Z$ according to $\sigl_\pi$. This process is similar in spirit to the proof of Proposition \ref{prop:coef c} and is hence omitted. Then, we can mirror the same arguments for $f^\rho_Z$. Finally, the Equivalence lemma suggests that the two representations are equal.
\end{proofof}


\begin{proofof}{Proposition \ref{prop: monotonicity in thm}}
We address the two parts separately below.
\paragraph{Part \ref{item:prop alphas rho pi}} 
Notice that the terminal state $(\ul\setminus \psi)$ is reachable from the left sub-tree of $\pi$ and $\rho$ solely (see Figure \ref{fig:helping for thm two} for illustration). Due to the construction of $\pi$ and $\rho$,
\begin{align}\label{eq:pi to rho}
\Pr(\pathto{s}{(\ul\setminus \psi)})&= \bl p_{{i_1},{j_k}}(a_{i_1})\Pr(\pathto{s\setminus \{a_{i_1}\}}{(\ul\setminus \psi)})\nonumber\\
&=\frac{-\mu(a_{j_{k}})}{- \mu(a_{j_{k}})+ \mu(a_{i_{1}})} \Pr(\pathto{s\setminus \{a_{i_1}\}}{(\ul\setminus \psi)})\nonumber\\
&=\frac{-\mu(a_{j_{k}})}{- \mu(a_{j_{k}})+ \mu(a_{i_{1}})} \Pr(\pathtorho{s\setminus \{a_{i_1}\}}{(\ul\setminus \psi)})\nonumber\\
&\leq\frac{-\mu(a_{j_{k+1}})}{- \mu(a_{j_{k+1}})+ \mu(a_{i_{1}})} \Pr(\pathtorho{s\setminus \{a_{i_1}\}}{(\ul\setminus \psi)})\nonumber\\
&\leq \bl p_{{i_1},{j_{k+1}}}(a_{i_1})\Pr(\pathtorho{s\setminus \{a_{i_1}\}}{(\ul\setminus \psi)})\nonumber \\
& = \Pr(\pathtorho{s}{(\ul\setminus \psi)}),
\end{align}
since $\mu(a_{j_{k+1}})\geq \mu(a_{j_{k}})$, and due to monotonicity of $f(x)=\frac{x}{x+c}$ for positive $c$. Using similar arguments, 
\begin{align}\label{eq: pi with rho again}
\Pr(\pathto{s}{(\ul\setminus (\psi \cup \{a_{j_k}\})}) \geq \Pr(\pathtorho{s}{(\ul\setminus (\psi \cup \{a_{j_{k+1}}\})}).
\end{align}
Now, observe that
\begin{align}\label{eq: with Ez}
\Pr(\pathto{s}{(\ul\setminus \psi)}\mid E^\pi_\psi)
&=\frac{\Pr(\pathto{s}{(\ul\setminus \psi)})}{\Pr(\pathto{s}{(\ul\setminus \psi)}) +\Pr(\pathto{s}{(\ul\setminus (\psi \cup \{a_{j_k}\})})} \nonumber\\
&\stackrel{\textnormal{Eq. }(\ref{eq:pi to rho}),(\ref{eq: pi with rho again})}{\leq} \frac{\Pr(\pathtorho{s}{(\ul\setminus \psi)})}{\Pr(\pathtorho{s}{(\ul\setminus \psi)}) +\Pr(\pathtorho{s}{(\ul\setminus (\psi \cup \{a_{j_{k+1}}\})})} \nonumber\\
&=\Pr(\pathtorho{s}{(\ul\setminus \psi)}\mid E^\rho_\psi),
\end{align}
where the second to last step follows again from monotonicity of $f(x)=\frac{x}{x+c}$ for positive $c$. Further, due to monotonicity of the reward function $R$,
\[
R(\ul\setminus (\psi\cup \{a_{j_k}\})) \geq R(\ul\setminus \psi), \qquad R(\ul\setminus (\psi\cup \{a_{j_{k+1}}\})) \geq R(\ul\setminus \psi).
\]
In addition, due to Assumption \ref{assumption:dominance}, $R(\ul\setminus (\psi\cup \{a_{j_k}\})) \geq  R(\ul\setminus (\psi\cup \{a_{j_{k+1}}\}))$. Wrapping up,
{\thinmuskip=.2mu
\medmuskip=0mu plus .2mu minus .2mu
\thickmuskip=1mu plus 1mu
\begin{align*}
\alpha^\pi_\psi&=\Pr(\pathto{s}{(\ul\setminus \psi)}\mid E^\pi_\psi)R(\ul\setminus \psi)+(1-\Pr(\pathto{s}{(\ul\setminus \psi)}\mid E^\pi_\psi))R(\ul\setminus (\psi \cup \{a_{j_k}\}))\\
&\geq \Pr(\pathto{s}{(\ul\setminus \psi)}\mid E^\pi_\psi)R(\ul\setminus \psi)+(1-\Pr(\pathto{s}{(\ul\setminus \psi)}\mid E^\pi_\psi))R(\ul\setminus (\psi \cup \{a_{j_{k+1}}\}))\\
&\stackrel{\textnormal{Eq. }(\ref{eq: with Ez})}{\geq} \Pr(\pathtorho{s}{(\ul\setminus \psi)}\mid E^\rho_\psi)R(\ul\setminus \psi)+(1-\Pr(\pathtorho{s}{(\ul\setminus \psi)}\mid E^\rho_\psi))R(\ul\setminus (\psi \cup \{a_{j_{k+1}}\}))\\
&=\alpha^\rho_\psi.
\end{align*}}%
This completes the proof of the first part.
%\paragraph{Part \ref{item:prop E}} \omer{TODO! but not sure we need it papers on the wall}This completes the proof of the second part.

\paragraph{Part \ref{item:prop alphas alpha pi}} We prove the claim for $\alpha^\rho_{l+1} \geq \alpha^\rho_{l}$, and the other part is symmetrical. Let $\psi=\psi(l)$ such that $\max(\psi)=l$. Notice that the reward function $R$ is a set function, and is, by definition monotonically decreasing; hence, $R(\ul\setminus \psi) \leq R(\ul\setminus (\psi \cup \{a_{j_{k+1}}\}))$. Consequently,
{\thinmuskip=.2mu
\medmuskip=0mu plus .2mu minus .2mu
\thickmuskip=1mu plus 1mu
\begin{align*}
\alpha^\rho_\psi&=\Pr(\pathtorho{s}{(\ul\setminus \psi)}\mid E^\rho_\psi)R(\ul\setminus \psi)+\Pr(\pathtorho{s}{(\ul\setminus (\psi \cup \{a_{j_{k+1}}\})}\mid E^\rho_\psi)R(\ul\setminus (\psi \cup \{a_{j_{k+1}}\}))\nonumber\\
&\leq R(\ul\setminus (\psi \cup \{a_{j_{k+1}}\})).
\end{align*}}%
Further, let $\psi'$ such that $\max(\psi')=l+1$, namely $\psi'=\psi\cup\{a_{j_{l+1}} \}$. Using the same properties of $R$, we have that $R(\ul\setminus \psi') \leq R(\ul\setminus (\psi' \cup \{a_{j_{k+1}}\}))$; thus,
{\thinmuskip=.2mu
\medmuskip=0mu plus .2mu minus .2mu
\thickmuskip=1mu plus 1mu
\begin{align*}
\alpha^\rho_{\psi'}&=\Pr(\pathtorho{s}{(\ul\setminus \psi')}\mid E^\rho_{\psi'})R(\ul\setminus \psi')+\Pr(\pathtorho{s}{(\ul\setminus (\psi' \cup \{a_{j_{k+1}}\})}\mid E^\rho_{\psi'})R(\ul\setminus (\psi' \cup \{a_{j_{k+1}}\}))\nonumber\\
&\geq R(\ul\setminus \psi' ).
\end{align*}}%
Next, let $V_\psi$ denote the event that $(X_{a})_{a\in \psi}$ attain value below $\alpha$. Observe that 
{\thinmuskip=.2mu
\medmuskip=0mu plus .2mu minus .2mu
\thickmuskip=1mu plus 1mu
\begin{align*}
R(\ul\setminus (\psi \cup \{a_{j_{k+1}}\})) &=\Pr(V_\psi)R(\ul\setminus (\psi \cup \{a_{j_{k+1}}\}))+(1-\Pr(V_\psi))R(\ul\setminus (\psi \cup \{a_{j_{k+1}}\}))\nonumber\\
&=\Pr(V_\psi)\max\{\alpha,X_{a_{j_{k+1}}}\}+(1-\Pr(V_\psi))\max\{a_{j_1},\dots ,a_{j_l},a_{j_{k+1}}  \}\nonumber\\
&\leq \Pr(V_\psi)\max\{\alpha,X_{a_{j_{l+1}}}\}+(1-\Pr(V_\psi))\max\{a_{j_1},\dots ,a_{j_l},a_{j_{l+1}}  \}\\
&=R(\ul\setminus \psi' ) ,
\end{align*}}%
where the second to last inequality is due to Assumption \ref{assumption:dominance} and independence of $(X_{a_i})_{i=1}^K$. Ultimately, 
\[
\alpha^\rho_{l+1}=\alpha^\rho_{\psi'} \geq R(\ul\setminus \psi' ) \geq R(\ul\setminus (\psi \cup \{a_{j_{k+1}}\})) \geq \alpha^\rho_{\psi}=\alpha^\rho_{l}.
\]
This completes the proof of the second part.
\end{proofof}

\begin{proofof}{Claim \ref{claim:thm:step 2 claim}}
The proof goes along the lines of Claim \ref{claim:f are equal}, and is hence omitted.
\end{proofof}



\section{Proofs of Observations and Propositions from Sections \ref{sec:infinite} and \ref{sec:policy to algorithm}}\label{sec:appendix main body}
\begin{proofof}{Observation \ref{obs:eventually will explore}}
Let $x(a_i)>0$ for some $i\in [K]$, let $j$ be an index of unexplored arm, and let $\mI$ be the information of the algorithm. We overload the notation $\bl p_{i,j}$ to acknowledge the realized value $x(a_i)$; that is,
\begin{align*}
\bl p_{i,j}(a) =
\begin{cases}
\frac{-\mu(a_j)}{x(a_i)-\mu(a_j)} & \textnormal{if } a=a_i\\
\frac{x(a_i)}{x(a_i)-\mu(a_j)} & \textnormal{if } a=a_j\\
0 & \textnormal{otherwise}
\end{cases}.
\end{align*}
Notice that 
\begin{align*}
\sum_{a\in A}\bl p(a)\E\left[X(a)\mid \mI\right] &= \bl  p_{i,j}(a_i)x(a_i)  + \bl  p_{i,j}(a_j)\mu(a_j) \\
&= x(a_i)\cdot \frac{-\mu(a_j)}{x(a_i)-\mu(a_j)} + \mu(a_j)\cdot \frac{x(a_i)}{x(a_i)-\mu(a_j)} = 0;  
\end{align*}
hence, $\bl p_{i,j}$ is IR w.r.t. to $\mI$. After selecting $\bl p_{i,j}$, either $a_i$ was realized or $a_j$. In the former, the information remains the same, and we can repeat this experiment again. The probability of $a_j$ realizing is positive and constant, and hence, after a finite time, we will eventually realize it. Once we do, the number of unexplored armed decreases by one. We can follow this process until all arms are explored.
\end{proofof}

\begin{proofof}{Observation~\ref{obs: U leq W*}}
The proof of this observation relies on constructing a policy $\pi$ that simulates $\ALG$. Since by definition $W(\pi,A) \leq  W^\star(A)$, it is enough to show that $\lim_{T \rightarrow \infty }\mU_T(\ALG) \leq W(\pi,A)$. In every round, $\pi$ selects precisely what $\ALG$ selects, and if the realized arm was already explored by $\ALG$, $\pi$ ignores it. The infinite time expected value of $\ALG$ cannot exceed $ W(\pi,A)$. The full details are similar to \cite[Theorem 3]{Fiduciary} and are hence omitted. 
\end{proofof}

\begin{proofof}{Observation \ref{obs: U get W}}
Fix any policy $\pi$. Let $\ALG(\pi)$ be the modification of Algorithm~\ref{alg:alg of pi} that uses $\pi$ instead of $\OGP$ in Lines \ref{algpi:while}-\ref{algpi:play with ogp}. Once $\pi$ reaches a terminal state, $\ALG(\pi)$ secures the reward of $\pi$ in finite time. Overall, $\lim_{T \rightarrow \infty }\mU_T(\ALG(\pi)) = W(\pi,A)$.
\end{proofof}


\begin{proofof}{Proposition \ref{prop:bernoulli opt}}
Fix an $\ise$ instance such that $(X(a_i))_i \in \{x^-,x^+\}$ (for $x^- \leq x^+$) almost surely. For the problem to be non-trivial, we must have $x^- <0$ and $x^+ >0$. Otherwise, if  $x^-,x^+ <0$ the only IR action is $a_0$, and if $x^-,x^+ \geq 0$, we can explore all arms using the singleton portfolios $(\bl p_{ii})_{i \in [K]}$. From here on, we assume w.l.o.g. that $x^- = -1$ and $x^+=H$. For convenience, we state $\SEGB'$ explicitly in Algorithm~\ref{alg:alg of pi two supported}. Before we prove the proposition, we remark that
\begin{enumerate}
    \item Since $(X(a_i))_i$ take either $-1$ or $H$, Assumption~\ref{assumption:dominance} implies a stochastic order on all arms, not only on $\below(A)$. 
    \item Any asymptotically optimal algorithm conducts at most $K$ exploration rounds before it exploits. This implies an immediate crude bound of
    $\mU_T(\SEGB') \geq \left(  1-\frac{KH}{T}\right) \OPT_T$.
    \item This proof uses the analysis presented in Section~\ref{sec:thm1 outline}.
\end{enumerate}


The proof is composed of two steps. In the first step, we show that if $T > T_0$ for some $T_0$, any optimal algorithm must explore the arms according to a policy that admits the same structure of $\OGP$. In the second step, we show that all such policies have an identical exploration time, and hence all achieve the same social welfare.\\
\textbf{Step 1:} In the case of realizing a positive reward, any algorithm would stop exploring and exploit that realized reward. Consequently, we can separate exploration rounds from exploitation rounds. Notice, however, that the exploration policy can select portfolios different that $\OGP$ for finite $T$. To illustrate, reconsider Example~\ref{example with four}. In the extreme case of $T=1$, there is no point in selecting $\bl p_{1,3}$, since exploring $a_3$ is futile; we only care about maximizing the current round's reward.

However, if $T$ is large \textit{enough}, any optimal algorithm must employ an asymptotically optimal policy. To see this, let $(\pi,\ALG^{\pi})$ be a pair of exploration policy and the algorithm that employs it, and assume $\pi$ does not admit the structure of $\OGP$. The social welfare of $\ALG^\pi$ satisfies
\begin{align}\label{eq:alg of pi is not optimal}
\mU_T(\ALG^\pi) \leq KH+ (T-K)W(\pi,A).    
\end{align}
Similarly, taking into account the optimality of $\OGP$,
\begin{align}\label{eq:alg of pi with ogp}
\mU_T(\ALG^{\OGP}) \geq -K +(T-K)W^\star(A).
\end{align}
Using similar arguments to those in Proposition~\ref{prop:optimal p valid}, we can assume w.l.o.g. that $\pi$ belongs to $\{2^A \rightarrow \mP \cup \mP'\}$ (recall the definition of $\mP$ and $\mP'$ from Subsection~\ref{subsec:bin}). To see this, observe that any IR policy can be formulated as a convex combination of policies that use $\mP \cup \mP'$ solely, and therefore we can assume that $\pi$ is the one for which $\ALG^{\pi}$ gets the highest social welfare. Furthermore, due to the proof of Theorem~\ref{thm:holy grail} (precisely Equation~\eqref{eq:rho with alpha}) it follows that if $\pi$ does not admit the structure of $\OGP$, then it is strictly sub-optimal. Next, let
\begin{align}\label{eq:alg of pi little omega}
\omega  \defeq \min_{\substack{\rho \in \{2^A \rightarrow \mP \cup \mP'\},\\W(\rho, A) < W^\star(A)}} W^\star(A) - W(\rho,A) > 0.
\end{align}
The quantity $\omega$ concerns the distributions of $(X(a_i))_i$ and is completely independent of the time $T$. We can further quantify or bound $\omega$ but this abstract and simple form is sufficient for our purposes. Let $T_0 \defeq K + \frac{K(H+1)}{\omega}$. Combining Inequalities~ \eqref{eq:alg of pi is not optimal},\eqref{eq:alg of pi with ogp}, and \eqref{eq:alg of pi little omega} we get
\begin{align*}
\mU_T(\ALG^{\OGP})-\mU_T(\ALG^\pi) &\geq  -K +(T-K)W^\star(A) - KH - (T-K)W(\pi,A) \\
& \geq -K(H+1) + (T-K)\omega\\
& > 0,
\end{align*}
provided that $T>T_0$. To conclude this step, we know that $\pi$ is a variation of $\OGP$.

\textbf{Step 2:}
Notice, however, that $\OGP$ is a class of policies differing from one another in the choices of arms from $\above(A)$ (Line~\ref{policy:pick arbitrary} in Policy~\ref{policy:pi star}); hence, one policy may attain a better reward than the other by reaching exploitation faster. 

As we commented in Line~\ref{algpi:two:play with ogp} of Algorithm~\ref{alg:alg of pi two supported}, when the state $s$ does not contain arms from $\below(A)$ we prioritize singleton portfolios according to the stochastic order. That is, we favor $\bl p_{i,i}$ over $\bl p_{i',i'}$ if $\mu(a_i) > \mu(a_{i'})$. This modification ensures that the time to exploitation from such states is minimal. 

Nevertheless, we might face a problem in states for which $\below(s)\neq\emptyset$. To illustrate, consider the action $\bl p_{i,j}$ for some $a_i\in \above(A), a_j\in \below(A)$. The probability we discover a reward of $H$ when selecting $\bl p_{i,j}$ is
\begin{align}\label{eq:q is identical to all}
\bl p_{i,j}(a_i)\Pr(X(a_i)=H)  + \bl p_{i,j}(a_j)\Pr(X(a_j)=H).
\end{align}
Consequently, we might favor $\bl p_{i,j}$ over $\bl p_{i',j}$ (for $a_{i'}\in \above(A)$) if it allows faster discovery of a positive reward (which is necessarily $H$). However, as we show next, the probability in Equation~\eqref{eq:q is identical to all} is the same regardless of the selection the arm from $\above(s)$. Observe that 
\[
\mu(a_i) = H\Pr(X(a_j)=H) +(-1)\cdot (1-\Pr(X(a_j)=H)) = (H+1)\Pr(X(a_j)=H)-1; 
\]
thus, by reformulating Equation~\eqref{eq:q is identical to all} we get
{
\thinmuskip=.2mu
\medmuskip=0mu plus .2mu minus .2mu
\thickmuskip=1mu plus 1mu
\begin{align*}\label{eq:q is identical to all elaborate}
\textnormal{Eq.}\eqref{eq:q is identical to all}&=\frac{-\mu(a_j)}{\mu(a_i)-\mu(a_j)}\Pr(X(a_i)=H)+ \frac{\mu(a_i)}{\mu(a_i)-\mu(a_j)}\Pr(X(a_j)=H)\\
& =\frac{-(H+1)\Pr(X(a_j)=H)+1}{\mu(a_i)-\mu(a_j)}\Pr(X(a_i)=H)+ \frac{(H+1)\Pr(X(a_i)=H)-1}{\mu(a_i)-\mu(a_j)}\Pr(X(a_j)=H)\\
& = \frac{\Pr(X(a_i)=H)-\Pr(X(a_j)=H)}{\mu(a_i)-\mu(a_j)}\\
& = \frac{\Pr(X(a_i)=H)-\Pr(X(a_j)=H)}{(H+1)\Pr(X(a_i)=H)-1-(H+1)\Pr(X(a_i)=H)+1}\\
& = \frac{1}{H+1}.
\end{align*}
}%

\begin{algorithm}[t]
\LinesNumbered
\SetNoFillComment
\DontPrintSemicolon
\caption{$\SEGB$ for Two-Supported Distributions \label{alg:alg of pi two supported}}
\KwIn{the $\OGP$ policy}
$s \gets A$\; % (no label on this line)
\While{$\OGP(s) \neq \emptyset$\nllabel{algpi:two:while}}{%
    \tcp{\texttt{\color{blue}{$s$ is not a terminal state}}}%
    select $\OGP(s)$, and denote the realized action by $a_k$\nllabel{algpi:two:play with ogp}\;
    \tcp{\texttt{\color{blue}{if $\below(s) = \emptyset$, prioritize the arms in $\above(s)$ according to the stochastic order}}}%
    \If{$x_{a_k} > 0$}{%
        \tcp{\texttt{\color{blue}{a reward of $H$ was realized}}}%
        \textbf{break}\;
    }
    $s \gets s \setminus \{a_k\}$\nllabel{algpi:two:update s}\;
}
\If{$x(a_k) = H$ for some explored arm $a_k$}{%
    exploit $a_k$ forever\;
}
\Else{%
    exploit $a_0$ forever\;
}
\end{algorithm}


\iffalse
\begin{algorithm}[t]
\renewcommand{\algorithmiccomment}[1]{\texttt{\kibitz{blue}{\##1}}}
\caption{$\SEGB$ for Two-Supported Distributions \label{alg:alg of pi two supported}}
\begin{algorithmic}[1]
\STATE $s\gets A$
\WHILE[$s$ is not a terminal state] {$\OGP(s)\neq \emptyset$\label{algpi:two:while}}{
\STATE select $\OGP(s)$, and denote the realized action by $a_k$.\label{algpi:two:play with ogp} \COMMENT{if $\below(s)=\emptyset$, prioritize the arms in $\above(s)$ according to the stochastic order}
\IF[a reward of $H$ was realized]{$x_{a_k}>0$} {
		\STATE \textbf{break}. 
}
\ENDIF
\STATE $s\gets s\setminus \{a_k\}$.\label{algpi:two:update s}
}
\ENDWHILE
\IF{$x(a_k)=H$ for some explored arm $a_k$}{
\STATE exploit $a_k$ forever.
}
\ELSE{
\STATE exploit $a_0$ forever.
}
\ENDIF
\end{algorithmic}
\end{algorithm}
\fi


Since all $\OGP$ have the same expected exploration time, they all achieve the same social welfare. This completes the proof of Proposition~\ref{prop:bernoulli opt}.
\end{proofof}

\begin{proofof}{Proposition \ref{prop:i-d bounds}}
To prove the claim, we show that $\SEGB$ explores for at most $K(1+\frac{\eta}{\delta})$ rounds, and then exploits. First, $\SEGB$ uses $K_1 \leq K$ rounds following $\OGP$ until it reaches a terminal state (the while loop in Line~\ref{algpi:while} breaks). If all the realized rewards are negative, the exploration ends after that. Otherwise, if it discovers a positive reward, the value of that reward is at least $\delta$. Next, the Bernoulli trails will explore every unexplored arm w.p. of at least $ \frac{\delta}{\delta+\eta}$, or $ \frac{\delta+\eta}{\delta}$ rounds in expectation. Therefore, after $(K-K_1)(1+\frac{\eta}{\delta})$ rounds in expectation we explore all the remaining arms. Overall, the expected number of rounds devoted to exploration is
\[
K_1 + (K-K_1)\left(1+\frac{\eta}{\delta}\right) \leq K \left(1+\eta\E\left[{\frac{1}{\delta}}\right]\right).
\]
Ultimately, recall that $\lim_{T \rightarrow \infty }\mU_T(\SEGB) = \OPT_{\infty}$, so $\SEGB$ exploits an expected reward of $\OPT_{\infty}$ after it completes its exploration. Therefore,
\begin{align*}
\mU_T(\SEGB)& \geq  \frac{1}{T} \left[K \left(1++\eta\E\left[{\frac{1}{\delta}}\right]\right) \cdot 0 + \left(T-K \left(1+\eta\E\left[{\frac{1}{\delta}}\right]\right) \right)\OPT_{\infty}\right]. \\
&\geq\left(1-\frac{K \left(1+\eta\E\left[{\frac{1}{\delta}}\right]\right)}{T} \right)\OPT_{\infty}. 
\end{align*}
This completes the proof of Proposition \ref{prop:i-d bounds}.
\end{proofof}

\section{Proof of Statements from Section \ref{sec:thm1 outline}}\label{sec:aux}
%\subsection{Proof of {Proposition \ref{prop:optimal p valid}}}\label{subsec:proof of propo p valid}
\begin{proofof}{Proposition \ref{prop:optimal p valid}}
Fix a non-terminal state $U\subseteq A$. Further, for simplicity of notation, denote $V(a)\defeq W^*(U\setminus \{a\})$ for every $a\in U$. Due to Equation (\ref{eq:W elaborated}), the action that maximizes the reward at state $U$ is the solution $\bl p \in \Delta(U)$ of the following linear program:
\begin{equation}
\tag{P1} \label{eq:lp w}
\begin{array}{ll@{}ll}
\max \limits_{\bl p} &\sum_{a\in U} \bl p(a)V(a)   & \\
\text{subject to} & \sum_{a\in U} \bl p(a)\mu(a) \geq 0 &     \\
& \sum_{a\in U} \bl p(a)=1 &  \\
& 0\leq \bl p(a) \leq 1 & \textnormal{for all }a\in U
\end{array}
\end{equation}
Observe that for every $\bl p$ such that $ \sum_{a\in U} \bl p(a)\mu(a) \geq 0$, there exist coefficients $(\alpha_{i,j})_{i,j}$ such that for every $a_i\in U$, $\bl p(a_i)=\sum_{a_j\in U}\alpha_{i,j}\bl p_{i,j}(i)$, and
\[
\sum_{a\in U} \bl p(a)\mu(a) = \sum_{i,j}\alpha_{i,j}\left(\bl p_{i,j}(i)\mu(a_i)+\bl p_{i,j}(j)\mu(a_j)\right).
\]
Hence, an equivalent form of Problem (\ref{eq:lp w}) is
\begin{equation}
\tag{P2} \label{eq:lp w with alpha}
\begin{array}{ll@{}ll}
\max \limits_{\bl \alpha} &\sum_{a\in U} \alpha_{i,j}\left(\bl p_{i,j}(i)V(a_i)+\bl p_{i,j}(j)V(a_j)\right)  & \\
\text{subject to} & \sum_{i,j}\alpha_{i,j}\left(\bl p_{i,j}(i)\mu(a_i)+\bl p_{i,j}(j)\mu(a_j)\right) \geq 0 &     \\
&\sum_{i,j}\alpha_{i,j}=1 &  \\
& 0\leq \alpha_{i,j} \leq 1 \qquad  \textnormal{for all }(i,j)\in \{(i',j')\mid \bl p_{i',j'} \in \mP\cup \mP'\textnormal{ and } i',j' \in U \} &
\end{array}
\end{equation}
Finally, notice that the constraint $\sum_{i,j}\alpha_{i,j}\left(\bl p_{i,j}(i)\mu(a_i)+\bl p_{i,j}(j)\mu(a_j)\right) \geq 0$ holds for every selection of $(\alpha_{i,j})_{i,j}$ by the way we defined $\mP\cup \mP'$; thus, the maximum of Problem (\ref{eq:lp w with alpha}) is obtained when we set $\alpha_{i,j}=1$ for the pair $(i,j)$ that maximizes $\left(\bl p_{i,j}(i)V(a_i)+\bl p_{i,j}(j)V(a_j)\right) $.
\end{proofof}

%\subsection{{\red is it needed?}Statements for Subsection \ref{subsec:stopchastic} \omer{put it in a new section about the conjecture}}\label{subsec: statments}
\begin{claim}\label{claim:ass is not for W}
Consider a state $U\in\mS$, such that $\below(U) \geq 2$. Let  $a_j = \argmin_{a_{j'}\in\below(U)}\sigr_\pi(a_{j'})$, and let $a_{\tilde j}\in \below(U), a_{\tilde j} \neq a_j$. Under Assumption \ref{assumption:dominance}, it might be the case that $W^*(U\setminus \{a_j\}) < W^*(U\setminus \{a_{\tilde j}\})$.
\end{claim}
\begin{proofof}{Claim \ref{claim:ass is not for W}}
We prove the claim by providing an example, that could be easily extended to a family of infinitely many examples. Consider $K=3$, $A=\{a_1,a_2,a_3\}$ such that 
\[
X_1=\begin{cases}
-1 & \textnormal{w.p. 0.45}\\
1 & \textnormal{w.p. 0.55}
\end{cases}, \qquad
X_2=\begin{cases}
-10^6-2\epsilon & \textnormal{w.p. 0.5}\\
10^6 & \textnormal{w.p. 0.5}
\end{cases}, \qquad
X_3=\begin{cases}
-10^{\frac{1}{\epsilon}} & \textnormal{w.p. 0.5}\\
10^6 & \textnormal{w.p. 0.5}
\end{cases}
\]
For small $\epsilon$, say $\epsilon<\frac{1}{7}$, it is clear that $X_2$ stochastically dominates $X_3$. The resulting expected values are $\mu({a_1})=0.1,\mu({a_2})=-\epsilon,$ and $\mu({a_3})= -\Theta(10^{\frac{1}{\epsilon}})$. The intuition behind our selection of rewards is that arm $a_2$ could have high reward, and can be explored with probability $\bl p_{1,2}(2)=1-O(\epsilon)$. On the other hand, arm $a_3$ has a high reward with the same probability, but it is highly unlikely to explore it. More precisely, $\bl p_{1,3}(3)=\Theta(10^{-\frac{1}{\epsilon}})$. To finalize the argument, notice that
\[
W^*(A\setminus \{a_2\})=\bl p_{1,3}(1)\cdot R(\{a_3\})+\bl p_{1,3}(3)\cdot \bl p_{1,1}(1)\cdot R(\emptyset)=0.5\cdot 10^6+0.5\cdot 0.55\cdot 1 +O(\epsilon)
\]
while
\begin{align*}
W^*(A\setminus \{a_3\})&=\bl p_{1,2}(1)\cdot R(\{a_2\})+\bl p_{1,2}(2)\cdot \bl p_{1,1}(1)\cdot R(\emptyset)\\
&=0.75\cdot 10^6+0.25\cdot 0.55\cdot 1+O(\epsilon).
\end{align*}
The proof is completed by taking $\epsilon$ to zero.
\end{proofof}


%I have commented it out for the EC version
%\input{input/BIC-appendix}
\section{Bayesian Incentive Compatibility}\label{sec:IC in app}
\begin{proofof}{Proposition~\ref{prop:ic for uniform}}
The proof follows closely the proof of~\citet[Section E]{Fiduciary}; hence, we omit the details.
\end{proofof}

\begin{proofof}{Theorem~\ref{theorem: ic fee}}
$\ICSEGB$ is IR by design, as it only recommends the default arm, a greedy arm, or according to $\SEGB$. To analyze its social welfare, observe that after $K(1+{\eta}\E\left[\frac{1}{\delta}\right])$ phases in expectation, we only explore. This follows directly from Proposition~\ref{prop:i-d bounds}. Since the length of each phase is $B=\ceil*{\frac{H}{\xi \gamma}}+1$, there are at most $O\left(\frac{K \eta H \E\left[\frac{1}{\delta}\right]}{\xi \gamma}  \right)$ exploration rounds in expectation. To show it is BIC, we divide the analysis according to the agent's index. The analysis closely follows the proof of \citet[Theorem 3]{Fiduciary}, and is hence omitted.
\end{proofof}
}


\fi}
\end{document}
