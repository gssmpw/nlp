\begin{proofof}{Lemma \ref{lemma:equivalence}}
We prove the lemma by a two-dimensional induction on the number of arms in $\above(s)$ and $\below(s)$. We prove four base cases in Section \ref{sec:base for lemma}:
\begin{itemize}
\item $\abs{\above(s)}=1$ and $\abs{\below(s)}\geq 2$ (Proposition \ref{prop:case of one strong}).
\item $\abs{\above(s)}\geq 2 $ and $\abs{\below(s)} = 1$ (Proposition \ref{prop:case of one}).
\item $\abs{\above(s)}\geq 2$ and $\abs{\below(s)} = 2$ (Proposition \ref{prop:case of two strong}).
\item $\abs{\above(s)}=2$ and $\abs{\below(s)}\geq 2$ (Proposition \ref{prop:case of two}).
\end{itemize}
While the first two cases are almost immediate, the other two are technical and require careful attention. Next, assume the statement holds for all states $s\in \mS$ such that $\abs{\above(s)}\leq K_1$, $\abs{\below(s)}\leq K_2$ and $\abs{\above(s)}+\abs{\below(s)}< K_1+K_2$.


Let $U\in\mS$ denote a state with $\abs{\above(U)}=K_1$ and $\abs{\below(U)}=K_2$. For abbreviation, let $\ug\defeq\above(U),\ul\defeq\below(U)$. Further, define $Q^*(U) = \sup_{\pi} Q(\pi,U)$, \footnote{This supremum is attained since there are only finitely many $\mP$-valid policies.} and for every $a_i \in \ug, a_j\in \ul$ let 
\[
Q^*_{i,j}(\ug,\ul)\defeq\bl p_{i,j}(a_j)Q^*(\ug,\ul\setminus \{a_j\}) +\bl p_{i,j}(a_i)Q^*(\ug\setminus\{a_i\},\ul).
\]
Next, let $(a_{i^*},a_{j^*})\in \argmax_{a_i\in \ug,a_j\in \ul}Q^*_{i,j}(\ug,\ul)$, and assume by contradiction that there exists a pair $(a_{\tilde i}, a_{\tilde j})$ such that 
\begin{equation}\label{eq:contradiction of lemma}
Q^*_{{i^*},{j^*}}(U) > Q^*_{{\tilde i}, {\tilde j}}(U).
\end{equation}
\paragraph{Step 1} Fix arbitrary $a_{i'}$ and $a_{j'}$ such that  $a_{i'} \in \ug$ and $a_{j'} \in \ul$. We will show that 
\begin{equation}\label{eq:step 1 goal}
Q^*_{{i'},{j^*}}(U)=Q^*_{{i'},{j'}}(U).
\end{equation}
We define the ordered policy $\pi$ such that $\sigr_\pi=(a_{i'},\dots)$, i.e., $\sigr_\pi$ first explores $a_{i'}$ and then the rest of the arms of $\ug$ in some arbitrary order; and, $\sigl_\pi=(a_{j^*},a_{j'},\dots)$. In addition, we define $\rho$ such that $\sigl_{\rho}=\sigl_\pi$, and $\sigr_\rho=(a_{j'},a_{j^*},\dots)$. Due to the inductive assumption, we have 
\begin{align}\label{eq:policies suffice}
Q^*_{{i'},{j^*}}(U) &=  \bl p_{{i'},{j^*}}(a_{j^*})Q^*(\ug,\ul\setminus \{a_{j^*}\}) +\bl p_{{i'},{j^*}}(a_{i'})Q^*(\ug\setminus \{a_{i'}\},\ul) \nonumber \\
&= \bl p_{{i'},{j^*}}(a_{j^*})Q(\pi,\ug,\ul\setminus \{a_{j^*}\})+\bl p_{{i'},{j^*}}(a_{i'})Q(\pi,\ug\setminus \{a_{i'}\},\ul)\\
&=Q(\pi,U).\nonumber
\end{align}
Similarly, $Q^*_{{i'},{j'}}(U)  = Q(\rho,U)$; hence, proving that $Q(\pi,U) =Q(\rho,U)$ entails Equality (\ref{eq:step 1 goal}). Next, let $\suff(\sigl_\pi)$ be the set of all non-empty suffices of $\sigl_\pi$. Being left-ordered suggests that on-path\footnote{These are terminal states that $\pi$ reaches to with positive probability.} terminal states with all arms of $\ul$ explored of $\pi$ are of the form $(Z,\emptyset)$, where $Z\in \suff(\sigl_\pi)$. Next, we factor $Q(\pi,U)$ recursively as follows: We factor $Q(\pi,U)$ into two terms, like in Equation \refeq{eq:policies suffice}. Following, for each term obtained, we ask whether the corresponding state excludes $\{a_{j^*},a_{j'}\}$. If the answer is yes, we stop factorizing it, and move to the other terms. We do this recursively, until we cannot factor anymore, or we reached a terminal state. Using this factorizing process, we have \footnote{We stop factorizing if both $a_{j^*},a_{j'}$ were observed; thus, $Z$ will never be the empty set.} 
\begin{align*}
Q(\pi,U) &= \alpha \cdot Q(\pi,\emptyset,\ul)+\beta\cdot Q(\pi,\emptyset,\ul\setminus \{a_{j^*} \}) + \sum_{Z \in \suff(\sigl_\pi)} c^\pi_Z \cdot Q(\pi,Z,\ul\setminus \{a_{j^*},a_{j'}\}),
\end{align*}
for $\alpha=\Pr(\pathto{s}{(\emptyset,\ul)})$ and $\beta=\Pr(\pathto{s}{(\emptyset,\ul\setminus\{a_{j^*}\})})$ such that $\alpha+\beta+\sum_{Z\in \suff(\sigl_\pi)}c^\pi_Z =1$ and $\alpha,\beta,c^\pi_Z\in [0,1]$ for every $Z \in \suff(\sigl_\pi)$. In this representation, $\alpha$ is the probability of reaching the terminal $(\emptyset,\ul)$, while $\beta$ is the probability of reaching the terminal state $(\emptyset,\ul\setminus\{a_{j^*}\})$. For these two terminal states, we know that $Q^*(\emptyset,\ul)= Q^*(\emptyset,\ul\setminus \{a_{j^*} \})=0$; hence,
\begin{align}\label{eq:pi j^* to j}
Q(\pi,U) &= \sum_{Z \in \suff(\sigl_\pi)} c^\pi_Z \cdot Q(\pi,Z,\ul\setminus \{a_{j^*},a_{j'}\}).
\end{align}
Following the same factorization process for $\rho$, we get
\begin{align}\label{eq:rho j^* to j}
Q(\rho,U) &= \sum_{Z \in \suff(\sigl_\rho)} c^\rho_Z \cdot Q(\rho,Z,\ul\setminus \{a_{j^*},a_{j'}\}).
\end{align}
Next, we want to simplify the coefficients $\left(c^\pi_Z\right)_Z$. We remark that $c^\pi_Z$ is not simply the probability of reaching $(Z,\ul\setminus \{a_{j^*},a_{j'}\})$ from $s$, i.e., $\Pr(\pathto{s}{(Z,\ul\setminus \{a_{j^*},a_{j'}\})})$. To clarify, consider a strict suffix $Z$, $1\leq \abs{Z}< \abs{\above(A)}$, and the suffix $Z'=Z \cup\{a_l\}$ for the minimal element $a_l \in \ug \setminus Z$ according to $\sigl_\pi$,i.e., $a_l = \argmin_{a\in \ug \setminus Z}\sigl_\pi(a)$. In the factorization process that produced Equation (\ref{eq:pi j^* to j}), once we got the term $Q(\pi,Z',\ul\setminus \{a_{j^*},a_{j'}\})$, we stopped factorizing any further; thus, $c^\pi_Z$ does not include the probability of reaching a node associated with $(Z',\ul\setminus \{a_{j^*},a_{j'}\})$ and then following the left edge to $(Z,\ul\setminus \{a_{j^*},a_{j'}\})$. However, this probability is taken into account in $\Pr(\pathto{s}{(Z,\ul\setminus \{a_{j^*},a_{j'}\})})$. Rather, $c^\pi_Z$ is the probability of reaching any node $v$ in the tree induced by $\pi$ with the following property: $v$ represents the state $(Z,\ul\setminus \{a_{j^*},a_{j'}\})$, while $a_{j'}$ does not belong to the state represented by the parent of $v$.  In the tree interpretation, $v$ should also be a \textit{right child of its parent} (for instance, the left highlighted node in the tree in Figure \ref{fig:tree example}). The following Proposition \ref{prop:coef c} describes $\left(c^\pi_Z\right)_Z$ in terms of $Q$.
\begin{proposition}\label{prop:coef c}
For every $Z\in \suff(\sigl_\pi)$, let $a_{i(Z)} = \argmin_{a_i\in Z} \sigl_\pi(a_i)$. It holds that
\[
c^\pi_Z = Q(\pi,\ug\setminus Z \cup \{a_{i(Z)}\}, \{a_{j^*},a_{j'}\})-Q(\pi,\ug\setminus Z, \{a_{j^*},a_{j'}\}).
\]
\end{proposition}
The proof of Proposition \ref{prop:coef c} appears at the end of this proof. Notice that for every $Z$, $c^\pi_Z$ includes values of $Q$ with less arms than $U$ (besides, perhaps, the case where $\abs{\ug}=2$ and $\abs{Z}=1$ obtaining $Q(\pi,\ug, \{a_{j^*},a_{j'}\})$, but we cover this case in the bases cases); consequently, due to the inductive step
\begin{align}\label{c pi is rho}
c^\pi_Z = Q(\rho,\ug\setminus Z \cup \{a_{i(Z)}\}, \{a_{j^*},a_{j'}\})-Q(\rho,\ug\setminus Z, \{a_{j^*},a_{j'}\})=c^\rho_Z,
\end{align}
where the last equality follows from mirroring Proposition \ref{prop:coef c} for $(c^\rho_Z)_Z$. Ultimately,
{\thinmuskip=.2mu
\medmuskip=0mu plus .2mu minus .2mu
\thickmuskip=1mu plus 1mu
\begin{align*}
Q(\pi,U) &\stackrel{\textnormal{Eq. (\ref{eq:pi j^* to j})}}{=}\sum_{Z \in \suff(\sigl_\pi)}c^\pi_Z \cdot Q(\pi,Z,\ul\setminus \{a_{j^*},a_{j'}\})\stackrel{\textnormal{Eq. (\ref{c pi is rho})}}{=}\sum_{Z \in \suff(\sigl_\pi)}c^\rho_Z \cdot Q(\pi,Z,\ul\setminus \{a_{j^*},a_{j'}\}) \nonumber\\
&\stackrel{\textnormal{Ind. step}}{=}\sum_{Z \in \suff(\sigl_\pi)}c^\rho_Z \cdot Q(\rho,Z,\ul\setminus \{a_{j^*},a_{j'}\}) \stackrel{\sigl_{\rho}=\sigl_\pi}{=}\sum_{Z \in \suff(\sigl_\rho)}c^\rho_Z \cdot Q(\rho,Z,\ul\setminus \{a_{j^*},a_{j'}\}) \nonumber\\
&\stackrel{\textnormal{Eq. (\ref{eq:rho j^* to j})}}{=}Q(\rho,U) .
\end{align*}}
This completes Step 1.

\paragraph{Step 2}
Fix arbitrary $a_{i'}$ and $a_{j'}$ such that $a_{i'} \in \ug$ and $a_{j'} \in \ul$. We will show that 
\begin{equation}\label{eq:step 2 goal}
Q^*_{{i^*},{j'}}(U)=Q^*_{{i'},{j'}}(U).
\end{equation}
We follow the same technique as in the previous step. Let $\pi$ be an ordered policy such that $\sigl_\pi=(a_{i^*},a_{i'},\dots )$, i.e., $\sigl_\pi$ ranks $a_{i^*}$ first, $a_{i'}$ second and then follows some arbitrary order on the remaining arms, and $\sigr_\pi=(a_{j'},\dots)$. In addition, we define the ordered policy $\rho$ with $\sigl_\rho=(a_{i'},a_{i^*},\dots )$, where the dots refer to any arbitrary order on the remaining elements of $\ug$, and $\sigr_\rho =\sigr_\pi=(a_{j'},\dots)$. Using the inductive step and the same arguments as in Equation (\ref{eq:policies suffice}), it suffices to show that $Q(\pi,U)=Q(\rho,U)$. We factor $Q(\pi,U)$ recursively such that 
\begin{align}\label{eq: q pi with d}
Q(\pi,U)=Q(\pi,\{a_{i^*},a_{i'}\},\ul)+ \sum_{Z \in \suff(\sigr_\pi)} d^\pi_Z \cdot Q(\pi,\ug \setminus \{a_{i^*},a_{i'}\},Z),
\end{align}
and similarly
\begin{align}\label{eq: q rho with d}
Q(\rho,U)=Q(\rho,\{a_{i^*},a_{i'}\},\ul)+ \sum_{Z \in \suff(\sigr_\rho)} d^\rho_Z \cdot Q(\rho,\ug \setminus \{a_{i^*},a_{i'}\},Z).
\end{align}
Next, we claim that
\begin{proposition}\label{prop:coef d}
For every $Z\in \suff(\sigr_\pi)$, let $a_{j(Z)} = \argmin_{a_j\in Z}\sigr_\pi(a_j)$. It holds that
\[
d^\pi_Z = Q(\pi,\{a_{i^*},a_{i'}\},\ul \setminus Z )-Q(\pi,\{a_{i^*},a_{i'}\},\ul \setminus Z \cup \{a_{j(Z)}\}).
\]
\end{proposition}
The proof of Proposition \ref{prop:coef d} appears at the end of this proof. Notice that for every $Z$, $d^\pi_Z$ includes values of $Q$ with less arms than $U$ (besides, perhaps, the case where $\abs{\ul}=2$ and $\abs{Z}=1$ obtaining $Q(\pi,\{a_{i^*},a_{i'}\},\ul\})$, but we cover this case in the bases cases); consequently, due to the inductive step
\begin{align}\label{eq: d pi is rho}
d^\pi_Z = Q(\rho,\{a_{i^*},a_{i'}\},\ul \setminus Z )-Q(\rho,\{a_{i^*},a_{i'}\},\ul \setminus Z \cup \{a_{j(Z)}\})=d^\rho_Z,
\end{align}
where the last equality follows from mirroring Proposition \ref{prop:coef d} for $(d^\rho_Z)_Z$. Ultimately, by rearranging Equation (\ref{eq: q pi with d}) and invoking the inductive step, Equation (\ref{eq: d pi is rho}) and the fact that $\sigr_\rho =\sigr_\pi$, we get
\begin{align*}
Q(\pi,U)=Q(\rho,\{a_{i^*},a_{i'}\},\ul)+ \sum_{Z \in \suff(\sigr_\rho)} d^\rho_Z \cdot Q(\rho,\ug \setminus \{a_{i^*},a_{i'}\}),Z)\stackrel{\textnormal{Eq. (\ref{eq: q rho with d})}}{=}Q(\pi,U),
\end{align*}
implying Equation (\ref{eq:step 2 goal}) holds.
\paragraph{Step 3} We are ready to prove the lemma. Fix arbitrary $a_{\tilde i}$ and $a_{\tilde j}$ such that $a_{\tilde i} \in \ug$ and  $a_{\tilde j} \in \ul$. By the previous Step 1 and Step 2, we know that
\[
Q^*_{{i^*},{j^*}}(U)\stackrel{\textnormal{Step 1}}{=}Q^*_{{i^*},{\tilde j}}(U)\stackrel{\textnormal{Step 2}}{=}Q^*_{{\tilde i},{\tilde j}}(U),
\]
which contradicts Equation (\ref{eq:contradiction of lemma}); hence, the lemma holds.
\end{proofof}