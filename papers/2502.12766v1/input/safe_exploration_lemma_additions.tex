\begin{proofof}{Proposition \ref{prop:coef c}}
\begin{figure}
\centering
\includegraphics[scale=0.8]{figs/prop4-tree-illustration-crop.pdf}
\caption{Illustration for Proposition \ref{prop:coef c}. The tree depicts $T(\pi)$. Nodes $v_1$ to $v_6$ are nodes whose sub-trees were pruned in the construction of $T$. Let $Z=\ug\setminus \{a_{i'},a_{i_1}\}$ and $Z^c= \{a_{i'},a_{i_1}\}$. The minimal element of $Z$, denoted $a_{i(Z)}$ in the proof, is $a_{i_2}$. The corresponding $c^\pi_Z$ is the probability to reach one of $\{v_3,v_5,v_6\}$, namely, $c^\pi_Z = \Pr(\{v_3,v_5,v_6\})$. In the tree $T$, we ignore sub-trees of nodes $v$ labeled with ``$\dots $'' since these do not contribute to $c^\pi_Z$. Observe that the probability of reaching $v_i$, for $i\in\{1,\dots, 6\}$ is the same in $T(\pi)$ and $T$. Finally, notice that $Q(\pi,\ug\setminus Z \cup \{a_{i(Z)}\}, \{a_{j^*},a_{j'}\})=\Pr(\{v_1,v_2,\dots,v_6 \})$, and $Q(\pi,\ug\setminus Z, \{a_{j^*},a_{j'}\})=\Pr(\{v_1,v_2,v_4\})$. Combining, we get that $c^\pi_Z = Q(\pi,\ug\setminus Z \cup \{a_{i(Z)}\}, \{a_{j^*},a_{j'}\})-Q(\pi,\ug\setminus Z, \{a_{j^*},a_{j'}\})=\Pr(\{v_3,v_5,v_6\})$ as required.
\label{fig:tree illustation}}
\end{figure}
To prove this claim, we focus on the tree induced by $\pi$, $T(\pi)$. It is convenient to discuss a modified version $T(\pi)$ obtained by pruning, and this is feasible since even if prune nodes from $T(\pi)$ it still remains Markov chain. We illustrate the proof of this claim in Figure \ref{fig:tree illustation}.

We factorize $Q(\pi,U) $ recursively (see Equation (\ref{eq:pi j^* to j})) until we hit a node associated with a state that excludes $\{a_{j^*},a_{j'}\}$, or a leaf. This factorization can be illustrated as follows: We traverse $T(\pi)$, from right to left. Every node we visit, we ask whether that node includes $\{a_{j^*},a_{j'}\}$. If it does not, we prune its sub-tree (i.e., it becomes a leaf) while leaving it intact. Denote the obtained tree by $T$, and let $V(T)$ be its set of nodes. Observe that
\begin{observation}\label{obs:two types}
Every leaf $v$ in $V(T)$ satisfies exactly one property: 
\begin{enumerate}[leftmargin=0cm,itemindent=.5cm,labelwidth=\itemindent,labelsep=0cm,align=left]
\item[]\textit{Type 1:} $\above(state(v))=\emptyset$, or
\item[]\textit{Type 2:} $\below(state(v))=\ul\setminus \{j^*,j'\}$ with $\below(state(v)) \subset \below(state(parent(v)))$.
\end{enumerate}
\end{observation}
Leaves of type 1 are associated with terminal states of the MDP (see Subsection~\ref{subsec:aux GMDP}). Leaves of type 2 are those whose sub-trees were pruned during the traversal. Moreover, $\below(state(v)) \subset \below(state(parent(v)))$ holds in every such a leaf $v$, since otherwise we would have pruned its parent. Due to Observation \ref{obs:two types}, every node $v$ with $state(v)=(Z,\ul\setminus \{a_{j^*},a_{j'}\})$ is of type 2; therefore,
\begin{align}\label{eq:c with T}
c^\pi_Z = \sum_{\substack{v\in V(T):state(v)=\\(Z,\ul\setminus \{a_{j^*},a_{j'}\})}}\Pr\left(\pth{root(T)}{}{v}\right).
\end{align}


Next, fix an arbitrary non-empty $Z$, $Z \subseteq \ug$, and $\Psi$ be the set of all non-empty suffixes of $\ug\setminus Z$. Consider $T$ and its root $root(T)$. Notice that $Q(\pi,\ug\setminus Z, \{a_{j^*},a_{j'}\})$ is the probability of reaching a (type 2) leaf $v$ such that $\above(v)=\psi\cup Z$ for some $\psi \in \Psi$. This is true since $\pi$ is ordered, and every path from $root(T)$ to such a leaf $v$ does not include any action from $Z$; hence, we can compare the probability of reaching it to off-path behavior of $\pi$. Further, $Q(\pi,\ug\setminus Z \cup \{a_{i(Z)}\}, \{a_{j^*},a_{j'}\})$ is the probability of reaching a (type 2) leaf $v$ such that $\above(v)=\psi\cup Z$ for some $\psi \in \Psi$ or $\above(v)=Z$; hence,
\[
Q(\pi,\ug\setminus Z \cup \{a_{i(Z)}\}, \{a_{j^*},a_{j'}\})-Q(\pi,\ug\setminus Z, \{a_{j^*},a_{j'}\})
\]
is precisely the right-hand-side of Equation \refeq{eq:c with T}.
\end{proofof}



\begin{proofof}{Proposition \ref{prop:coef d}}
Fix $Z\in \suff(\sigr_\pi)$, and let $a_{j(Z)} = \argmin_{a_j\in Z}\sigr_\pi(a_j)$. Let $T(\pi)$ denote the tree induced by $\pi$. Observe that
\begin{observation}\label{obs: for d}
The coefficient $d^\pi_Z$ is the probability to get to a node $v$ in $T(\pi)$ such that
\begin{enumerate}
\item $state(v)=(\ug\setminus \{a_{i^*},a_{i'} \},Z )$, and 
\item $state(parent(v))=(\ug\setminus \{a_{i^*}\},Z )$. 
\end{enumerate}
\end{observation}
The first condition is immediate, due to the way we factorize $Q$ in Equation \refeq{eq: q pi with d}. To see why the second condition holds, notice that $state(parent(v))$ must be a strict superset of $state(v)$; hence, $state(parent(v))$ could be either $(\ug\setminus \{a_{i^*}\},Z )$ or $(\ug\setminus \{a_{i^*},a_{i'}\},Z\cup\{ a\} )$ for $a\in \ul\setminus Z$, but then it would contribute to $d^\pi_{Z \cup \{a\}}$, namely, to another summand in Equation \refeq{eq: q pi with d}. 

Denote by $V$ the set of all nodes that satisfy the conditions of Observation \ref{obs: for d}. Due to the way we constructed $\pi$, the paths from the root of $T(\pi)$ to any node in $V$ consist of actions that involve the arms $\{a_{i^*},a_{i'},a_{j(Z)}\}\cup (\ul \setminus Z)$ solely; hence, we can focus on the \text{off-path} tree whose root is $s_0'\defeq\{a_{i^*},a_{i'}\}\cup (\ul \setminus Z)\cup \{a_{j(Z)}\}$, and the actions are precisely as in the tree induced by $\pi$ (according to the order of $\pi$). Denote this new tree by $T'$, and let
\[
V' \defeq \left\{v\in nodes(T)\mid state(v)=\{a_{j(Z)}\}  \right\}.
\] 
Due to this construction, 
\begin{observation}\label{obs: for d two}
The coefficient $d^\pi_Z$ is the probability to get to a node that belongs to $V'$  in $T'$.
\end{observation}
The observation follows from the one-to-one correspondence between the nodes and path in $T(\pi)$ and their counterparts in the off-path tree $T'$. 

In $T'$,  $Q(\pi,s'_0)$ is the probability of starting at $s'_0$ and reaching the leaf with no arms (terminal state $\emptyset$), i.e., exploring $\ul \setminus Z$ \textit{and} $a_{j(Z)}$. In contrast, $Q(\pi,\{a_{i^*},a_{i'}\}\cup (\ul \setminus Z))$ is the probability of starting at $s'_0$ and reaching a node (internal or terminal) $v$ with $state(v)= \cap \left(\ul \setminus Z\right) =\emptyset$, namely, exploring $\ul \setminus Z$. Such a node $v$ leads to a leaf with probability 1; hence, paths from $v$ terminate in leaves corresponding to either state $\emptyset$ or  $\{a_{j(Z)}\}$. Consequently,
\[
Q(\pi,\{a_{i^*},a_{i'}\}\cup (\ul \setminus Z)) - Q(\pi,\{a_{i^*},a_{i'}\}\cup (\ul \setminus Z)\cup \{ a_{j(Z)}\})
\]
is the probability of starting at $s_0'$, and reaching a terminal node that belongs to $V'$.
\end{proofof}