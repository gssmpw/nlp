
\documentclass{article} % For LaTeX2e
\PassOptionsToPackage{table,xcdraw}{xcolor} 
\RequirePackage{xcolor}
\usepackage{iclr2025_conference,times}

% Optional math commands from https://github.com/goodfeli/dlbook_notation.
%%%%% NEW MATH DEFINITIONS %%%%%

% \usepackage{amsmath,amsfonts,bm}
\usepackage{amsmath,amsfonts}

\usepackage{pifont}


\newcommand{\R}{\mathbb{R}}


\def\va{{\mathbf{a}}}
\def\vg{{\mathbf{g}}}

% Sets
\def\sR{\mathbb{R}}
\def\sC{\mathbb{C}}
\def\sZ{\mathbb{Z}}
\def\sN{\mathbb{N}}
\def\sQ{\mathbb{Q}}

\def\sS{\mathcal{S}}



% Vectors
\def\vzero{{\mathbf{0}}}
\def\vone{{\mathbf{1}}}
\def\vmu{{\mathbf{\mu}}}
\def\vtheta{{\mathbf{\theta}}}
\def\va{{\mathbf{a}}}
\def\vb{{\mathbf{b}}}
\def\vc{{\mathbf{c}}}
\def\vd{{\mathbf{d}}}
\def\ve{{\mathbf{e}}}
\def\vf{{\mathbf{f}}}
\def\vg{{\mathbf{g}}}
\def\vh{{\mathbf{h}}}
\def\vi{{\mathbf{i}}}
\def\vj{{\mathbf{j}}}
\def\vk{{\mathbf{k}}}
\def\vl{{\mathbf{l}}}
\def\vm{{\mathbf{m}}}
\def\vn{{\mathbf{n}}}
\def\vo{{\mathbf{o}}}
\def\vp{{\mathbf{p}}}
\def\vq{{\mathbf{q}}}
\def\vr{{\mathbf{r}}}
\def\vs{{\mathbf{s}}}
\def\vt{{\mathbf{t}}}
\def\vu{{\mathbf{u}}}
\def\vv{{\mathbf{v}}}
\def\vw{{\mathbf{w}}}
\def\vx{{\mathbf{x}}}
\def\vy{{\mathbf{y}}}
\def\vz{{\mathbf{z}}}
\def\vzeta{{\mathbf{\zeta}}}

% Matrix
\def\mA{{\mathbf{A}}}
\def\mB{{\mathbf{B}}}
\def\mC{{\mathbf{C}}}
\def\mD{{\mathbf{D}}}
\def\mE{{\mathbf{E}}}
\def\mF{{\mathbf{F}}}
\def\mG{{\mathbf{G}}}
\def\mH{{\mathbf{H}}}
\def\mI{{\mathbf{I}}}
\def\mJ{{\mathbf{J}}}
\def\mK{{\mathbf{K}}}
\def\mL{{\mathbf{L}}}
\def\mM{{\mathbf{M}}}
\def\mN{{\mathbf{N}}}
\def\mO{{\mathbf{O}}}
\def\mP{{\mathbf{P}}}
\def\mQ{{\mathbf{Q}}}
\def\mR{{\mathbf{R}}}
\def\mS{{\mathbf{S}}}
\def\mT{{\mathbf{T}}}
\def\mU{{\mathbf{U}}}
\def\mV{{\mathbf{V}}}
\def\mW{{\mathbf{W}}}
\def\mX{{\mathbf{X}}}
\def\mY{{\mathbf{Y}}}
\def\mZ{{\mathbf{Z}}}
\def\mBeta{{\mathbf{\beta}}}
\def\mPhi{{\mathbf{\Phi}}}
\def\mLambda{{\mathbf{\Lambda}}}
\def\mSigma{{\mathbf{\Sigma}}}


% Expectation
% \def\eE{\mathop{\mathbb{E}}\limits}
\def\eE{\mathbb{E}}

% Probability
\def\pP{\mathbb{P}}

% Tilde
\def\tf{\tilde{f}}
\def\tS{\tilde{S}}
\def\wtF{\widetilde{\mathcal{F}}}
\def\whR{\widehat{R}}
\def\tvx{\tilde{\mathbf{x}}}
\def\ty{\tilde{y}}


\def\defeq{\overset{\textup{def}}{=}}
% \def\defeq{\overset{.}{=}}
\def\defone{\overset{\text{\ding{172}}}{=}}
\def\deftwo{\overset{\text{\ding{173}}}{=}}
\def\leqone{\overset{\text{\ding{172}}}{\leq}}
\def\leqtwo{\overset{\text{\ding{173}}}{\leq}}
\def\leqthree{\overset{\text{\ding{174}}}{\leq}}
\def\leqfour{\overset{\text{\ding{175}}}{\leq}}
\def\eqone{\overset{\text{\ding{172}}}{=}}
\def\eqtwo{\overset{\text{\ding{173}}}{=}}
\def\eqthree{\overset{\text{\ding{174}}}{=}}
\def\eqfour{\overset{\text{\ding{175}}}{=}}
\def\geqfive{\overset{\text{\ding{176}}}{\geq}}

% \usepackage{hyperref}
% \usepackage{url}
% \usepackage{algorithm}
% \usepackage{algpseudocode}
% \usepackage{amsthm}
% \usepackage{booktabs}
% \usepackage{multirow}
% \usepackage{graphicx}
% \usepackage{adjustbox}
% \usepackage{wrapfig}
% \usepackage{subfigure}
% \usepackage{subcaption}
% \usepackage{xspace}
% \usepackage[table,xcdraw]{xcolor}
% \usepackage{titletoc}
% \usepackage{comment}
% \usepackage{ulem}
% \usepackage[font=small]{caption}



% Standard packages
\usepackage{hyperref}
\usepackage{url}
\usepackage{algorithm}
\usepackage{algpseudocode}
\usepackage{amsthm}
\usepackage{booktabs}
\usepackage{multirow}
\usepackage{graphicx}
\usepackage{adjustbox}
\usepackage{wrapfig}
\usepackage{subfigure}  % If you must use subfigure
\usepackage{xspace}
\usepackage{titletoc}
\usepackage{comment}
\usepackage{ulem}
\usepackage[font=small]{caption}


\newcommand{\atk}{DocMIA\xspace}
\newcommand{\atkWB}{DocMIA-WB\xspace}
\newcommand{\atkBB}{DocMIA-BB\xspace}
\newcommand{\atkFL}{FL\xspace}
\newcommand{\atkFLLoRA}{FLLoRA\xspace}
\newcommand{\atkIG}{IG\xspace}


\title{DocMIA: Document-Level Membership Inference Attacks against DocVQA Models}

% Authors must not appear in the submitted version. They should be hidden
% as long as the \iclrfinalcopy macro remains commented out below.
% Non-anonymous submissions will be rejected without review.

% \author{Antiquus S.~Hippocampus, Natalia Cerebro \& Amelie P. Amygdale \thanks{ Use footnote for providing further information
% about author (webpage, alternative address)---\emph{not} for acknowledging
% funding agencies.  Funding acknowledgements go at the end of the paper.} \\
% Department of Computer Science\\
% Cranberry-Lemon University\\
% Pittsburgh, PA 15213, USA \\
% \texttt{\{hippo,brain,jen\}@cs.cranberry-lemon.edu} \\
% \And
% Ji Q. Ren \& Yevgeny LeNet \\
% Department of Computational Neuroscience \\
% University of the Witwatersrand \\
% Joburg, South Africa \\
% \texttt{\{robot,net\}@wits.ac.za} \\
% \AND
% Coauthor \\
% Affiliation \\
% Address \\
% \texttt{email}
% }

\author{Khanh Nguyen$^{1}$~~~~~Raouf Kerkouche$^{2}$\thanks{The corresponding author.}~~~~~Mario Fritz$^{2}$~~~~~Dimosthenis Karatzas$^{1}$ \\
$^1${Computer Vision Center, Universitat Aut\`onoma de Barcelona}\\ $^2${CISPA Helmholtz Center for Information Security}\\
\texttt{\{knguyen,dimos\}@cvc.uab.es}\\ \texttt{\{raouf.kerkouche,fritz\}@cispa.de}
}

\newcommand{\fix}{\marginpar{FIX}}
\newcommand{\new}{\marginpar{NEW}}

\theoremstyle{definition}
\newtheorem{definition}{Definition}[section]


\iclrfinalcopy % Uncomment for camera-ready version, but NOT for submission.
\begin{document}

\maketitle

\begin{abstract}
Document Visual Question Answering (DocVQA) has introduced a new paradigm for end-to-end document understanding, and quickly became one of the standard benchmarks for multimodal LLMs. Automating document processing workflows, driven by DocVQA models, presents significant potential for many business sectors. However, documents tend to contain highly sensitive information, raising concerns about privacy risks associated with training such DocVQA models. One significant privacy vulnerability, exploited by the membership inference attack, is the possibility for an adversary to determine if a particular record was part of the model's training data. In this paper, we introduce two novel membership inference attacks tailored specifically to DocVQA models. These attacks are designed for two different adversarial scenarios: a white-box setting, where the attacker has full access to the model architecture and parameters, and a black-box setting, where only the model's outputs are available. Notably, our attacks assume the adversary lacks access to auxiliary datasets, which is more realistic in practice but also more challenging. Our unsupervised methods outperform existing state-of-the-art membership inference attacks across a variety of DocVQA models and datasets, demonstrating their effectiveness and highlighting the privacy risks in this domain.
\end{abstract}

\section{Introduction}

Automated document processing fuels a significant number of operations daily, ranging from fintech and insurance procedures to interactions with public administration and personal record keeping. Up until a few years ago, document processing services relied on template-based information extraction models, which were created ad-hoc for each client. Although these approaches allowed for good control of client data and could be extended to new documents with a few examples, they were limited in scalability and difficult to maintain. Consequently, the introduction of Document Visual Question Answering (DocVQA) \citep{mathew2020document} in 2019 has resulted in a paradigm shift in document processing services, enabling end-to-end generic solutions to be applied in this domain. DocVQA leverages multi-modal large language models to streamline business workflows and provide clients with novel ways to interact with the document processing pipeline. 

%As DocVQA models become more prevalent in handling sensitive documents, significant privacy risks emerge, particularly regarding the potential leakage of sensitive data through model vulnerabilities.

However, as cloud-based DocVQA solutions become more prevalent, significant privacy risks emerge, particularly concerning the potential leakage of sensitive information through model vulnerabilities.
Indeed, during the training of a DocVQA model, \textit{each document can have several associated question-answer pairs}, with each pair considered a unique data point. As a result, a single document can appear multiple times, which significantly raises the risks associated with privacy vulnerabilities. This repeated exposure enhances the likelihood of the model memorizing specific details, thereby increasing the potential for data leakage during privacy attacks.
Furthermore, scanned document images often have high resolutions necessary for posterior analysis, but need to be rescaled for processing by image encoders, potentially rendering content unreadable. To mitigate this issue, many DocVQA models~\citep{huang2022layoutlmv3,tang2023unifying} utilize a dual representation of the document, comprising both a reduced-scale image and OCR-recognized text. This approach introduces further challenges, as sensitive information may leak through multiple modalities.

\centering

\includegraphics[width=0.85\textwidth]{images/teaser.pdf}
\caption{
IP-Composer enables compositional generation from a set of visual concepts. These are portrayed through a set of input images, along with a prompt describing the desired concept to be extracted from each.
}


Membership inference attacks (MIAs) are among the most prominent techniques for assessing privacy vulnerabilities in machine learning models. These attacks enable an adversary to determine whether a specific data point is included in the training dataset. However, there is limited research on membership inference risks in the context of multi-modal models. Among the few studies, \citet{ko2023practical,hu2022m} utilize powerful pre-trained models on large datasets to construct an aligned embedding space for the two modalities—image as input and text as output—allowing for the inference of membership information. Unfortunately, the reliance on these pre-trained models poses challenges for document-based tasks, particularly in DocVQA scenarios, where an alignment model capable of aligning the (document, question) as input and the answer as output is currently unavailable. Recently, \citet{tito2024privacy} introduced a provider-level MIAs against DocVQA models aimed at determining whether a \textit{provider (group)} that may supply multiple invoice documents is part of the training set. In contrast, our research focuses on membership information at a finer granularity, specifically targeting the inference of whether a \textit{single document} is included in the training dataset. Current MIA solutions that exploit standard features such as output logits, probabilities, or loss are difficult to adapt to the DocVQA context, where outputs are generated in an auto-regressive manner. Additionally, legal constraints surrounding copyright and private information complicate centralized model training, making it challenging to create auxiliary datasets that capture the variability and richness of real-world data. As a result, shadow training of proxy models becomes infeasible.

In this work, we take a structured approach to privacy testing for DocVQA models. We design a novel Document-level Membership Inference Attack (\atk) that deals with the multiple occurrences of the same document in the training set, as demonstrated in Figure \ref{fig:teaser}. To address the challenge of extracting typical metrics (e.g. logit-based) from auto-regressive outputs, we propose a new method based on model optimisation for individual samples that generates discriminative features for \atk. We design attacks both for white-box and black-box settings without requiring auxiliary datasets. In the black-box setting, we propose an alternative knowledge transfer mechanism from the attacked model to a proxy. Evaluating our attacks on three multi-modal DocVQA models and two datasets, we achieve state-of-the-art performance against multiple baselines.

To summarize, we make the following contributions:
\begin{enumerate}
    \item We present \atk, the first Document-level Membership Inference Attacks specifically targeting multi-modal models for DocVQA.
    \item We introduce two novel auxiliary data-free attacks for both white-box and black-box settings, leveraging novel discriminative metrics for \atk.
    \item We explore three distinct approaches to quantify these metrics: vanilla layer fine-tuning  (\atkFL), fine-tuning layer with LoRA \citep{hu2021lora} (\atkFLLoRA), and image gradients (\atkIG).
    \item Our attacks\footnote{Code is available at ~\url{https://github.com/khanhnguyen21006/mia_docvqa}}, evaluated on two DocVQA datasets across three different models, outperform existing state-of-the-art membership inference attacks as well as baseline attacks.
\end{enumerate}

\section{Related Work}
\label{sec:related_work}
\vspace{-0.15in}

\paragraph{Membership Inference Attack.} 
Membership inference attacks have been extensively explored in various applications to highlight privacy vulnerabilities in deep neural networks or to audit model privacy~\citep{shokri2017membership}. These attacks are categorized into two types: white-box and black-box settings. In white-box settings, the adversary has full access to the target model's internal parameters and computations~\citep{carlini2022membership, yeom2018privacy, nasr2019comprehensive, rezaei2021difficulty, sablayrolles19a, li2021membership}, enabling the use of informative features like loss values, logits, and gradient norms. Conversely, in black-box settings, the adversary is limited to the model's outputs, such as predicted labels or confidence scores~\citep{choquette2021label, shokri2017membership, salem2018ml, sablayrolles19a, song2021systematic, hui2021practical}. The literature indicates that white-box attacks tend to be more effective due to the availability of richer features~\citep{song2019privacy,nasr2019comprehensive}. In this paper, we propose tailored attacks for both settings, considering a more challenging scenario where the adversary lacks an auxiliary dataset --which is used to train shadow models that mimic the behavior of the target model and are subsequently exploited to enhance attack performance-- and is restricted to a limited number of queries. Regarding gradient-based membership inference attacks, research on using gradients as features has been limited. \citet{nasr2019comprehensive} leveraged the $L2$-norm of gradients with respect to model weights for membership inference. \citet{rezaei2021difficulty} suggested using the distance to the decision boundary as a metric but found it ineffective for this purpose. In contrast, we introduce novel strategies called \atkFL, \atkFLLoRA, and \atkIG, demonstrating that the $L2$-norm of the cumulative gradient—computed using these methods—provides a robust signal for membership inference. While \citet{maini2021dataset} and \citet{li2021membership} also explored distance metrics, but from input points for membership inference in image classification tasks, their approaches lack scalability and applicability in our context, which involves larger-scale models with a wider vocabulary of tokens.

\paragraph{Membership Inference Attack Against Multi-modal Models.} Research works into the privacy vulnerabilities of multi-modal models is still in its early stages. Recently, \citet{tito2024privacy,pinto24a} proposed reconstruction attacks that exploit DocVQA model memorization to recover hidden values in documents. They black out specific target values in documents and query the model with questions about the modified documents. Since the model memorizes training data, it often reconstructs the hidden target values. \citet{tito2024privacy} also introduced a membership attack against DocVQA models to infer whether a document provider, with multiple documents, is included in the training dataset. However, as far as we know, no research has yet explored membership inference attacks at document-level granularity. Additionally, \citet{ko2023practical,hu2022m} leverage powerful \textit{pre-trained models} on large datasets to create an aligned embedding space for the two modalities to infer membership. Unfortunately, the reliance on these pre-trained models introduces difficulties for document-based tasks, especially DocVQA, where an appropriate alignment model for aligning (document, question) inputs to corresponding answers is not yet available. Furthermore, the success of both attacks hinges on the availability of \textit{auxiliary datasets} leveraged by the adversary, which are key to executing the attack effectively. In this paper, we present two membership inference attacks specifically tailored to tackle the unique characteristics of DocVQA models.

\section{Background}

\subsection{Document-based Visual Question Answering}
DocVQA is a multi-modal task where natural language questions are posed based on the content of document images. Notably, it establishes a unified query-response framework applicable across various document understanding tasks, such as document classification and information extraction.

Formally, the DocVQA task is defined as follows: given a question-answer pair $(q, a)$ related to a document image $x$, the method $\mathcal{F}$ must generate an answer $\hat{a}=\mathcal{F}(x,q)$ such that $\hat{a}$ closely matches the correct answer $a$. More concretely, given $D_t = \{(x_i,q_i,a_i)\}^{N_t}_{i=1}$ as a set of valid training examples, a model $\mathcal{F}$, parameterized by $\theta$, is trained to maximize the conditional log-likelihood:
% of the ground truth via the following loss:
\begin{equation}
    \mathcal{L}(\theta) =-\log{p_{\theta} (a_i|x_i,q_i)}
\label{eq:training_loss}
\end{equation}

Standard metrics for DocVQA include Accuracy (ACC) and Normalized Levenshtein Similarity (NLS) \citep{biten2019scene}, which measure the similarity between the predicted and correct answer:
\begin{equation}
    \textsc{ACC} = \displaystyle \1_\mathrm{\hat{a} = a};
    \quad
    \textsc{NLS} = 
    \begin{cases}
        1 - \text{NL}(\hat{a}, a) & \text{if } \textsc{NL}(\hat{a}, a) < 0.5, \\
        0 & \text{if } \textsc{NL} \geq 0.5
    \end{cases}
\end{equation}
where $\textsc{NL}(\cdot,\cdot)$ denotes the normalized Levenshtein distance.

In the following sections, for clarity, we often omit the data example index $i$ from the notation.

\subsection{Document-level Membership Inference Attack}
Membership inference attacks~\citep{shokri2017membership} exploit privacy vulnerabilities to determine if a specific data point was included in the training set of a machine learning model. We extend this definition to the Document-level MIA, which is particularly suited in the DocVQA context. 

Given access to a trained DocVQA model $\mathcal{F}$ and a document $x$ drawn from its data distribution $\mathcal{D}$, along with a set of question-answer pairs $Q=\{(q_i,a_i)\}_{i=1}^{M}$ related to the information in the document, an adversary $\mathcal{A}$ designs a decision rule $f_{\mathcal{A}}(x,Q;\mathcal{F})$  to classify the membership status of $x$, aiming for $f_{\mathcal{A}}(x,Q;\mathcal{F})=1$ if $x$ is a member of the training set, otherwise a non-member. It is important to note that the adversary is focused solely on \textit{the membership of the document $x$}, rather than the entire DocVQA \textit{data point} $(x,q,a)$, which is typically the target of prior MI attacks. Moreover, since a single document is associated with many question-answer pairs, this allows the adversary to query the same document using multiple questions for various pieces of information.

\section{\atk against DocVQA models}
\vspace{-0.1in}

In Section~\ref{sec:threat_model}, we elaborate on the threat model relevant to \atk on two scenarios: \textit{white-box} and \textit{black-box} access. We first explain our intuition behind our optimization-based attacks in the white-box setting (Section~\ref{sec:docmia_whitebox}), then adapt this approach to our black-box attacks (Section~\ref{sec:docmia_blackbox}).
\vspace{-0.1in}
\subsection{Threat Model}
\label{sec:threat_model}
\atk can be either a useful or harmful tool in various real-world scenarios. On the positive side, \atk can act as a privacy auditing tool. For instance, in legal document processing, law firms may use these attacks to evaluate whether proprietary or confidential documents, such as contracts or court filings, were included in model training, thereby identifying potential privacy risks.
Conversely, \atk can be maliciously leveraged. As an example, a business competitor could exploit these attacks on an invoice-processing system to infer the presence of specific invoices in the training data, exposing confidential business relationships and leading to risks such as supplier poaching.

In both scenarios, we assume that the adversary aims to infer membership information for a set of documents, determining whether each document is included in the training dataset. These documents may or may not be part of the target model's training data.
Crucially, we further assume \textit{the adversary lacks access to an auxiliary data} $D_{\text{aux}}$ that reflects the characteristics of these documents. This assumption is realistic, as obtaining real-world documents at scale is often prohibitively difficult due to their confidential nature and regulatory restrictions. Consequently, this negates the application of MI attack techniques that require training shadow models~\citep{shokri2017membership,carlini2022membership}. Even if auxiliary documents were available, training numerous shadow document-based models—typically designed with a large number of parameters—would be prohibitively expensive.

Based on the previous examples, we refer to the owner of the document model as \textit{the trainer} and the law firms or competitors as \textit{the adversary}. Given the document distribution $\mathcal{D}$, the trainer trains a document-based model $\mathcal{F}_t$ with private access to $D_t \sim \mathcal{D}$, following a training algorithm $\mathcal{T}$, that defines the model architecture, optimization process, and related details. The adversary owns the set of sensitive documents $D_{\text{test}} \sim \mathcal{D}$, where $D_t \cap D_{\text{test}} \ne \emptyset$, $\vert D_{\text{test}} \vert =N_\text{test}$; but does not know which documents are in $D_t$. Given a document $x \in D_{\text{test}}$ with a set of related queries $Q=\{(q_i,a_i)\}_{i=1}^{M}$, the adversary's goal is to determine whether $x \in D_t$ or $x \notin D_t$. We formulate two attack settings, which specify the adversarial knowledge about the model $\mathcal{F}_t$ and its data distribution $\mathcal{D}$:

\textbf{White-box Setting}. In this scenario, the adversary has full access to the internal workings of the target model, including the model's architecture, weights, gradients from any further training and other internal details. However, the adversary does not have access to the training algorithm $\mathcal{T}$.

\textbf{Black-box Setting}. Here, the adversary can only interact with the target model through an API, which only returns a prediction $\hat{a}$ for each question $q$ on $x$. In addition, the adversary is constrained by a limited number of queries. As in the white-box setting, the adversary has no information about $\mathcal{T}$. This setting reflects the most challenging case~\citep{nasr2019comprehensive,song2019privacy}.

We assume the adversary has full knowledge of the DocVQA task to train the model, including the training objective, document type and exact training questions. 
This assumption is reasonable, as task-level information such as document type, is often publicly available to guide users, making it accessible to adversaries.
The assumption of the exact question knowledge is also plausible, as \textit{an adversary can approximate questions based on the document type}. Further discussion of this assumption and experiments in the setting without exact questions are provided in Appendix~\ref{sec:impact_question_knowledge}.
\vspace{-0.1in}

\subsection{White-box DocMIA}
\label{sec:docmia_whitebox}
In the white-box setting, where the adversary has access to the trained model, shadow training is impractical due to the lack of auxiliary data and high computational cost. To this end, our strategy is to develop unsupervised \textit{metric-based} attacks \citep{hu2022membership}. For each document, we extract a set of features from individual question-answer pairs and aggregate them across all pairs. We then cluster the resulting feature vectors to distinguish member from non-member documents. 
A key challenge is to design discriminative features, as standard metrics (e.g., logit and loss) may be ineffective in this setting (Section \ref{sec:evaluation}). To address this, we propose new features that enhance the informativeness of our membership inference vectors.

\subsubsection{Optimization-Based Discriminative Features}
In this section, we introduce two novel discriminative membership features derived from an optimization process for our attacks against DocVQA models.
\begin{wrapfigure}{R}{0.3\textwidth}
\centering
 \adjustbox{width=.3\textwidth,frame=0.01cm 0cm}{\includegraphics{images/Intuition1.pdf}}\hfill
 $x$ is \textbf{\textcolor{blue}{in}} of the training set 
\medskip

\adjustbox{width=.3\textwidth,frame=0.01cm 0cm}{\includegraphics{images/Intuition2.pdf}}\hfill
$x$ is \textbf{\textcolor{red}{out}} of the training set 
\caption{\textbf{Visualization of our fine-tuning strategy} in the parameters space. Each contour plot represents the optimization landscape w.r.t each pair $(a_i,q_i)$ from document $x$.  In general, the average $\Delta$ computed on a member document $x_{\textcolor{blue}{\textbf{in}}}$ is smaller than non-member document $x_{\textcolor{red}{\textbf{out}}}$.}

\vspace{-0.3in}
\label{fig:intuition}
\end{wrapfigure}

\textbf{Intuition.} Since DocVQA models are typically trained on multiple question-answer pairs per document, the model parameters likely converge to \textit{minimize the average distance to the ground-truth answers} after training. As a result, fine-tuning the model on one question-answer pair through an iterative process is necessary to extract more reliable membership signals. More importantly, this optimization on training documents may converge faster than to non-training documents, due to the lower generalization error. Figure \ref{fig:intuition} illustrates our reasoning.

We provide a formal definition of the \textit{distance} feature.  
\begin{definition}[\textbf{Optimization-based Distance Feature}] Given a model $\mathcal{F}$ parameterized by $\theta$, let the model be initialized with $\theta_0$. After undergoing a gradient-based optimization process $\mathcal{O}$, the parameters converge to $\theta^{*}$ according to a specified training objective $\mathcal{L}$. The \textit{distance} feature is then defined as the $L2$-norm of the change in parameters:
\begin{equation}
    \Delta(\theta_0, \theta^{*}) = || \theta_0 - \theta^{*} ||_2
\label{eq:distance}
\end{equation}
\end{definition}
This feature measures the difference between the initial parameters $\theta_0$ and the converged parameters $\theta^*$, as an approximation of the optimization trajectory toward the optimal solution. 

Specifically, we fine-tune the target DocVQA model on an individual document/question-answer pair and compute the \textit{distance} required to reach the \textit{optimal} answer. A small average distance indicates the document is likely part of the training set, while a larger distance suggests a non-training document. In addition, the number of optimization steps serves as an orthogonal feature that reflects the efficiency of the optimization process. With an optimal learning rate and a good initialization provided from the target model, optimization for training documents typically converges in fewer steps compared to non-training documents. Consequently, we include both the \textit{distance} and the number of optimization steps in our feature set for white-box attacks.
\newtheorem{theorem}{Theorem}[section]
\newtheorem{proof}{Proof}[section]

Based on the aforementioned  knowledge profile associated with RBAC policy and the virtual resource vulnerability model related to the cloud \cite{almutairi2014risk}, we now formally define the sensitive property based risk-aware  assignment problem (SPRAP) as follows: 

\begin{definition}\label{def:SPRAP}
(SPRAP). Given the knowledge profile of RBAC policy generated  based on the sensitive property function $f$, and the vulnerability matrix  $\mathcal{D}$ representing the virtual resources probability of leakage \cite{almutairi2014risk},   the sensitive property based risk-aware role assignment problem (SPRAP) is to minimize the total risk of sensitive property  disclosure by assigning access control roles to the virtual resources.
\end{definition} 

Formally, the $SPRAP$ optimization problem is given as:
\begin{equation}\label{eq:cost}
\begin{split}
\textit{minimize} \ Risk &= \sum_{r_i \in R} Risk(r_i) \\
&= \sum_{r_i \in R} \max_{A \in \mathcal{P}(R)} g_i^{A}\times Threat(A,r_i) \\
& \ \ \ \ \ \times \prod_{r_j \in A,r_j \neq r_i} d_{I(r_i),I(r_j)} 
\end{split}
\end{equation}

Such that:\\
\begin{equation}
g^A_i=  |f(A)-f(r_i)|
\end{equation}

\begin{equation}
Threat(A,r_i) =
  \begin{cases}
    1 & \quad  \text{if $r_i \in A$}\\
        0 & \quad \text{Otherwise}
  \end{cases}
\end{equation}

Where: \\
$R$: the set of roles\\
$\mathcal{P}(R)$: the power set of roles excluding the empty set\\
$A$: is subset of $\mathcal{P}(R)$\\
$I$: is the assignment function where $I(r_i)=v_j$ if role $r_i$ assigned to VM $v_j$\\
$f$: is the sensitive property function \\
$g^A_i$: is the quantitative measure associated with a sensitive property leaked to role $r_i$ when it attack the set of roles $A$ \\
$d_{i,j}$: is probability of leakage between VM $v_i$ and  VM $v_j$ 

\begin{theorem}
SPRAP problem is NP-complete.
\end{theorem} 
\begin{proof}
We construct a decision  formulation of SPRAP problem as follow:\\
For a given constant $c$, is there any assignment  $I$ such that 
\begin{eqnarray*}
Risk&=&\sum_{r_i \in R} \max_{A \in \mathcal{P}(R)} g_i^{A}\times Threat(A,r_i) \\
&\times& \prod_{r_j \in A,r_j \neq r_i} d_{I(r_i),I(r_j)}  \le c
\end{eqnarray*}
First, given $c$ and $I$ we can check if the SPRAP decision problem satisfies the bound in polynomial time. Accordingly, the decision formulation of SPRAP is NP. Next, we show that decision SPRAP problem is NP-Complete by showing that  Travel Salesman problem (TSP) can be reduced to SPRAP. We formulates TSP for a given set $U$ with a distance metric $h(a,b)$ where $a,b \in U$, is there such an ordering of elements of $U$:($u_1,u_2,...,u_m$) such that $\sum_{i=1}^n h(u_i,u_{i+1})+h(u_n,u_1) \leq c$.

Let  $f$ be the sensitive property function for  RBAC policy with $n$ roles where $R= \lbrace r_1,r_2, \dots , r_n \rbrace$.  We define $f$ as follows:
\[ f(A) = \left  \{
  \begin{array}{l l}
    1 & \quad  \text{if} \ A = \lbrace r_i, r_j \rbrace\ \text{and} \ |i-j| \ mod \ n = 1\\
    0 & \quad \text{Otherwise}
\end{array} \right.\]

Let  $V = U$  and $d_{i,i} = h(a,b)$ where $v_i= a \ \text{and} \ v_j = b$ then assume the assignment of role $r_i$ to VM $v_j$ as if the salesman visits city $u_j$. Therefore, if we find an optimal order of visits in TSP, we can find the optimal assignment of $R$  to $V$ in SPRAP.
\end{proof} 

\vspace{-0.05in}
\subsubsection{Methodology}
We now formally present our attack strategy, assuming white-box access to the target model $\mathcal{F}_t$.
For any document $x \in D_{\text{test}}$ and a set of question-answer pairs $Q$, the goal is to assign a features descriptor $F_{x}$. This is achieved by first extracting a set of features through the optimization process $\mathcal{O}$ on a single question-answer pair. These features are aggregated across multiple questions then concatenated to construct $F_{x}$. Repeating this process over $D_{\text{test}}$, we apply an unsupervised clustering algorithm to differentiate member documents from non-members based on their features descriptors.

Following our intuition, for each question-answer pair $(q,a)$, we fine-tune the target model parameter $\theta_{t}$  
using gradient descent to maximize the conditional probability $p_{\theta}(a|x,q)$, as defined by the objective in Equation \ref{eq:training_loss}. The optimization process always starts from the target model parameters $\theta_t$, and the learning rate $\alpha$ controls the optimization speed. During this process, we query the model at each step $s$ using $q$, tracking its prediction quality against $(q,a)$ via a utility function $\mathcal{U}$, either ACC or NLS. The optimization stops when no further improvements is observed, governed by a threshold $\tau$ or after a maximum of $S$ steps. At the end of the optimization, we evaluate the distance $\Delta$ based on Equation \ref{eq:distance}, record the number of steps taken $s$, and aggregate the utility evolution throughout the process to obtain an overall DocVQA score $u$. Collectively, these features serve as membership signals for the current $(q,a)$ pair in relation to the target document $x$.

Since each document is associated with a varying number of question-answer pairs $M$, we employ an aggregation function $\Phi$ to aggregate the features across all $M$ questions, producing in a scalar value for each feature. Optionally, we can utilize a diverse set of aggregation functions to further enrich the feature set. After aggregation, we normalize all aggregated features to ensure they are on a consistent scale. The features descriptor $F_{x}$, assigned to document $x$, is constructed by concatenating these normalized features. The assignment algorithm for each document is detailed in Algorithm \ref{alg:assignment}. Finally, we apply a clustering algorithm to the set of descriptors from documents in $D_{\text{test}}$, predicting the cluster with the larger $\Delta$ as corresponding to non-member documents.
\subsubsection{Improving Efficiency}
Fine-tuning $\mathcal{F}_t$ on a \textit{single} document/question-answer pair provides a mechanism to differentiate between members and non-members. However, this approach is relatively slow, given the model’s size and the complexity of data pre-processing. To improve the attack efficiency, we introduce three variants of the method, as illustrated in Figure \ref{fig:method}:

\textbf{Optimize One Layer (\atkFL)}. Instead of optimizing all parameters, we hope that gradients with respect to a single layer's parameters can provide sufficient signal for membership classification. In this variant, we select one specific layer $L$ to optimize while keeping the remaining parameters fixed. We ablate the choice of layer for this method in Appendix~\ref{sec:calibration}. In addition, we consider a variant leveraging LoRA \citep{hu2021lora}, termed \textbf{\atkFLLoRA}, where the LoRA parameters are initialized with Kaiming initialization \citep{he2015delving}. From Algorithm \ref{alg:assignment}, we replace $\theta$ to $\theta_L$ or the $\textsc{LoRA}$ parameters of the layer L, denoted as $\textsc{LoRA}(\theta_L)$, respectively.

% \begin{figure}
%     \centering
%     \includegraphics[width=0.5\linewidth]{Move_teaser.pdf}
%     \caption{Comparison of different dynamic compute approaches. length of arrow indicates residual transformation per token while width indicates velocity of transformation.}
%     \label{fig:enter-label}
% \end{figure}

\section{Method}
\label{sec:method}
Residual connections play a crucial role in shaping token representations, yet their dynamics remain underexplored in the context of efficient decoding. In this work, we delve deeper into transformer residual dynamics and investigate how modulating residual transformation velocity can improve inference efficiency in token-level processing, optimizing both dense and sparse MoE transformers.


\subsection{Residual Dynamics and Motivation for Multi-rate Residuals} \label{sec:motivation}

To analyze how hidden representations evolve across different layers of a transformer architecture, it's crucial to consider the effect of residual connections. Each transformer decoder layer typically has residual connections across attention and MLP submodules. As the residual stream $h_i$ traverses from interval $E_j$ to $E_{j+1}$, it undergoes a residual transformation given by:  
% \begin{equation}
% \label{eq:slow_residual_transformation}
% H_{E_{j+1}} = H_{E_j} \prod_{i=E_j}^{E_{j+1}} \left( I + \mathcal{A}_i \right) \left( I + \mathcal{M}_i \right) \quad \text{where} \quad \mathcal{A}_i = f(c_i, h_{i}), \mathcal{M}_i = g(h_i)
% \end{equation}

\begin{equation} \label{eq:slow_residual_transformation}
h_{E_{j+1}} = h_{E_j} + \sum_{i=E_j}^{E_{j+1}-1} \left( \mathcal{A}_i(h_i) + \mathcal{M}_i(h_i + \mathcal{A}_i(h_i)) \right) \quad \text{where} \quad \mathcal{A}_i = f(c_i, h_{i}), \mathcal{M}_i = g(h_i). 
\end{equation}

Here, \( \mathcal{A}_i \) denotes the non-linear transformation introduced by the multi-head attention mechanism at layer \( i \), while \( \mathcal{M}_i \) corresponds to the non-linear transformation of the MLP block at the same layer. These transformations depend on the input residual stream \( h_i \) and, in the case of \( \mathcal{A}_i \), the previous contextual representation \( c_i \).\footnote{Normalization layers are typically applied in practice but are omitted here for simplicity of the argument.}


% For easy tokens, the magnitude and direction of this delta transformation become progressively smaller with each successive layer as shown in \cref{fig:delta_transformation}. Consequently, it is feasible to predict these tokens after only a few residual connections, whereas harder tokens necessitate more extensive processing through additional layers.

\begin{figure}[ht]
    \centering
    \begin{subfigure}{0.48\textwidth}
        \centering
        \includegraphics[width=\textwidth]{sections/figures/residual_change.pdf}
        \caption{}
        \label{fig:residual_change}
    \end{subfigure}%
    \hfill
    \begin{subfigure}{0.48\textwidth}
        \centering
        \includegraphics[width=\textwidth]{sections/figures/alignment_wrt_dedicated_model.pdf}
        \caption{}
    \label{fig:alignment_wrt_dedicated_model}
    \end{subfigure}
    \caption{(a) As residual streams propagate through the model, the directional shifts in the residuals become progressively smaller. (b) A dedicated model with $k$ layers achieves a faster rate of change in residual streams and higher alignment than base model leveraging early exit mechanisms at layer $k$.}
    \label{fig}
\end{figure}


To examine whether residual transformations can be accelerated across layers, we conducted experiments using a diverse set of prompts on a pre-trained Phi3 model~\cite{phi3_report}. As illustrated in \cref{fig:residual_change}, we measured the directional shift in residual states as \( 1 - \mathcal{C}(h_{i-1}, h_i) \), where \(\mathcal{C}\) denotes normalized cosine similarity. This shift is notably higher in the initial layers, gradually decreasing in subsequent layers. This behavior allows traditional early exit approaches to effectively accelerate decoding by enabling earlier exits for simpler tokens. However, these approaches typically rely on a distance-based approximation, where the full residual transformation of the model is approximated by the residual transformations of the initial layers. To gain deeper insights into the distance versus velocity aspects of residual transformation, we conducted a comparative study. Specifically, we trained an early exit head at layer $k$ of the Phi3 model, which consists of 32 layers, restricting the distance traveled by each token. To accelerate the residual transformation relative to number of layers, we trained a smaller model consisting of only $k$ layers, while keeping all other hyperparameters consistent. We then compared the next-token prediction accuracy of the early exit head of the base model with that of the smaller model. To ensure an equal number of trainable parameters, we inserted low-rank adapters into the smaller model and trained only these adapters, whereas, in the distance-based approach, we trained solely the early exit head. In addition, to accelerate the residual transformation in smaller model, we distilled the residual streams from the larger model by incorporating a distillation loss ~\cite{sanh2019distilbert} between the residual state at layer \(i\) of the smaller model and the residual state at layer \(4 \times i\) of the larger model. As shown in ~\cref{fig:alignment_wrt_dedicated_model} the smaller model demonstrates a significantly faster rate of change in residual streams, leading to higher next token prediction accuracy after $k$ layers compared to the base model that employs traditional early exit mechanisms after $k$ layers \cite{schuster2022confident, chen2023eellm, varshney-etal-2024-investigating}. This experimental setup, which modifies only the rate of change in residual streams while keeping other factors constant, suggests that dense transformers, trained with a fixed number of layers, may inherently possess a slow residual transformation bias.

This observation raises an intriguing question: if the rate of change in residual streams could be accelerated relative to the number of layers, is it possible to facilitate earlier alignment for a greater proportion of tokens? Earlier alignment would be beneficial to not only facilitate dynamic computation but also for generating speculative tokens efficiently with high acceptance rates in speculative decoding setups ~\cite{leviathan2023fast, chen2023accelerating}. 

%thereby enhancing the efficiency of early exiting? 
 % This bias likely constrains the effectiveness of early exiting, particularly for easier tokens. By addressing this limitation through accelerated residual transformations, we hypothesize that it is possible to substantially improve the efficiency and accuracy of early exit strategies in transformer models.

\subsection{Multi-Rate Residual Transformation} \label{m2r2_method}

To address the slow residual transformation bias described in ~\cref{sec:motivation}, we introduce \textit{accelerated residual streams} that operate at rate $R$ relative to original slow residual stream. We pair slow residual stream, $h$ with an accelerated residual stream, $p$, which has an intrinsic bias towards earlier alignment. Relative to ~\cref{eq:slow_residual_transformation}, accelerated residual transformation from interval $E_j$ to $E_{j+1}$ can be represented as: 

% \begin{equation}
% \label{eq:fast_residual_transformation}
% P_{E_{j+1}} = P_{E_j} \prod_{i=E_j}^{E_{j+1}} \left( I + \hat{\mathcal{A}_i} \right) \left( I + \hat{\mathcal{M}_i} \right) \quad \text{where} \quad \hat{\mathcal{A}_i} = \hat{f}(c_i, P_{i}), \hat{\mathcal{M}_i} = \hat{g}(P_{i})
% \end{equation}


\begin{equation} \label{eq:fast_residual_transformation}
p_{E_{j+1}} = p_{E_j} + \sum_{i=E_j}^{E_{j+1}-1} \left( \hat{\mathcal{A}_i}(p_i) + \hat{\mathcal{M}_i}(p_i + \hat{\mathcal{A}_i}(p_i)) \right) \quad \text{where} \quad \hat{\mathcal{A}_i} = \hat{f}(c_i, p_{i}), \hat{\mathcal{M}_i} = \hat{g}(h_i), 
\end{equation}



where $\hat{\mathcal{A}_i}$ and $\hat{\mathcal{M}_i}$ denote non-linear transformation added by layer $i$ to previous accelerated residual $p_{i}$. Similar to $\mathcal{A}_i$, non-linear transformation $\hat{\mathcal{A}_i}$ attends to same context $c_i$ but uses a different transformation $\hat{f}$ for accelerating $p_{E_j}$ relative to $h_{E_j}$. 

We integrate accelerated residual transformation directly into the base network using parallel accelerator adapters such that rank of accelerator adapters $R_p << d$ where $d$ denotes base model hidden dimension. This setup allows the slow residual stream $h_{E_j}$ to pass through the base model layers while the accelerated residual stream $p_{E_j}$ utilizes these parallel adapters as shown in ~\cref{fig:m2r2_main}. Both slow and accelerated residuals are processed in same forward pass via attention masking and incur negligible additional inference latency in memory bound decoding setups, while in compute bound decoding setups where FLOPs optimization is essential, accelerated residual stream utilizes a fraction of attention heads that of slow residual (see ~\cref{sec:flops_optimization}). Additionally, to maximize the utility of accelerated residual transformations without introducing dedicated KV caches, we propose a shared caching mechanism between the slow and accelerated streams which minimally impact alignment benefits of our approach while offering substantial memory savings (see ~\cref{fig:koala_alignment}). Specifically, the attention operation on the slow residuals \( \text{MHA}(h_t, h_{\leq t}, h_{\leq t}) \) is redefined for accelerated residuals as 
\[
\hat{\mathcal{A}} = MHA(p_t, h_{<t} \oplus p_t, h_{<t} \oplus p_t),
\]
where the accelerated residual at time-step $t$, \( p_t \) attends to the slow residual’s KV cache, facilitating the reuse of contextual information across both residual streams without incurring additional caching costs. Here, \(MHA(q, k, v) \) represents multi-head attention between query \( q \), key \( k \), and value \( v \).

\begin{figure}
    \centering
    \includegraphics[width=0.8\linewidth]{sections//figures/m2r2_main2.pdf}
    \caption{Multi-rate Residuals Framework: Slow residual stream of base model is accompanied by a faster stream that operates at a $2-(J+1)\times$ rate relative to the slow stream, undergoing transformations via accelerator adapters as detailed in \cref{m2r2_method}, where J denotes number of early exit intervals. Colors within the slow and fast residual streams indicate similarity, with matching colors representing the most closely aligned residual states. At the beginning of the forward pass and at each exit point, the accelerated residual state is initialized from the corresponding slow residual state to avoid gradient conflict during training (see ~\cref{sec:grad_conflict}). Early exiting decisions are informed by the Accelerated Residual Latent Attention (ARLA) mechanism, described in \cref{method_arla}, which evaluates residual dynamics across consecutive exit gates.}
    \label{fig:m2r2_main}
\end{figure}

% Furthermore. to maximize the benefits of fast residual transformations without using dedicated KV caches, we propose sharing the fast network’s cache with the slow network. Formally speaking, We modify attention operation on slow residuals $MHA(H_t, H_{<=t}, H_{<=t})$ as $MHA(P_{t}, H_{<t} \oplus P_t, H_{<t}  \oplus P_t)$ such that accelerated residuals attend to previous slow context KV cache, where $MHA(q,k,v)$ denotes multi head attention between query, $q$, key $k$ and value $v$.


\subsection{Enhanced Early Residual Alignment}
Early residual alignment is instrumental in optimizing early exiting, speculative decoding, and Mixture-of-Experts (MoE) inference mechanisms. In this section, we provide a detailed analysis of how accelerated residuals enhance these inference setups.

% By aligning the residual states of intermediate layers with the final output representations, the model can maintain high prediction accuracy even when computations are truncated at earlier layers. This enables more reliable early exiting, reducing the overall computational cost while preserving performance. Additionally, in speculative decoding, early residual alignment allows the model to make confident predictions using faster, partial computations, thereby accelerating inference without sacrificing output quality.


\subsubsection{Early Exiting} \label{method_early_exiting}

A prevalent strategy for enabling early exiting at an intermediate layer $E_{j}$ involves approximating the residual transformation between $E_{j}$ and the final layer $N-1$ using a linear, context independent mapping, $\mathcal{T}$, such that $H_{N-1} \approx \mathcal{T}(H_{E_{j}})$. This approximation has been extensively employed in conventional approaches ~\cite{schuster2022confident, chen2023eellm, varshney-etal-2024-investigating}, providing a computationally efficient means to project the output of deeper layers from intermediate states. Specifically, residual state of layer $N-1$ with this approximation can be expressed as:


% \begin{equation}
% \label{eq: vanila_ea_assumption}
% \Phi(H_{E_{j}}) \sim H_{E_{j}} \prod_{i=E_{j}}^{N}\left( I + \mathcal{A}_i \right) \left( I + \mathcal{M}_i \right) \quad \text{where} \quad \Phi \perp C
% \end{equation}

\begin{equation} \label{eq:early_exiting}
h_{E_j} + \sum_{i=E_j}^{N-1} \left( \mathcal{A}_i(h_i) + \mathcal{M}_i(h_i + \mathcal{A}_i(h_i)) \right) \sim \mathcal{T}(h_{E_{j}})  \quad \text{where} \quad \mathcal{T} \perp c. 
\end{equation}


Here, $\mathcal{A}_i$ and $\mathcal{M}_i$ represent the residual contributions of the multi-head attention and MLP layers, respectively, while $\mathcal{T}$ remains independent of $c$, the preceding context.

This approach is inherently limited by two major factors: first, the assumption of linearity between $h_{E_{j}}$ and $h_{N-1}$ may not hold uniformly for all tokens, particularly when $E_j \ll N$. Second, the linear transformation $\mathcal{T}$ disregards the influence of the context $c$ and fails to account for the latent representations of previous contextual states. In contrast, M2R2 accelerated residual states mitigate both of these challenges by approximating the slow residual transformation of all layers via a faster residual transformation of fewer layers as:
% \begin{equation}
% H_{E_j} \prod_{i=E_j}^{N}\left( I + \mathcal{A}_i \right) \left( I + \mathcal{M}_i \right) \sim P_{E_j} \prod_{i=E_j}^{E_j+1}\left( I + \hat{\mathcal{A}_i} \right) \left( I + \hat{\mathcal{M}_i} \right)
% \end{equation}


\begin{equation} \label{eq:m2r2_approximating_ea}
h_{E_j} + \sum_{i=E_j}^{N-1} \left( \mathcal{A}_i(h_i) + \mathcal{M}_i(h_i + \mathcal{A}_i(h_i)) \right) \sim p_{E_j} + \sum_{i=E_j}^{E_{j+1}-1} \left( \hat{\mathcal{A}_i}(p_i) + \hat{\mathcal{M}_i}(p_i + \hat{\mathcal{A}_i}(p_i)) \right), 
\end{equation}

% \begin{equation} \label{eq:fast_residual_transformation}
% p_{E_{j+1}} = p_{E_j} + \sum_{i=E_j}^{E_{j+1}-1} \left( \hat{\mathcal{A}_i}(p_i) + \hat{\mathcal{M}_i}(p_i + \hat{\mathcal{A}_i}(p_i)) \right) \quad \text{where} \quad \hat{\mathcal{A}_i} = \hat{f}(c_i, p_{i}), \hat{\mathcal{M}_i} = \hat{g}(h_i) 
% \end{equation}






where $p_{E_j}$ is initialized from the slow residual state $h_{E_j}$ at each early exit interval $E_j$ using an identity transformation (see ~\cref{fig:m2r2_main}). As shown in ~\cref{fig:m2r2_residual_sim}, accelerated residuals offer a smoother, more consistent shift in residual direction across layers, in contrast to the abrupt changes typically seen at early exit points in standard early exit methods. Moreover, the normalized cosine similarity between accelerated states at early exit intervals and final residual states is substantially higher compared to traditional early exit techniques, highlighting improved alignment with final layer representations. Traditional adaptive compute methods are constrained by two principal factors: the number of tokens eligible for early exit at intermediate layers and the precision of early exit decision. If residual streams fail to saturate early, the majority of tokens remain ineligible for exit, thereby diminishing potential speedups. Additionally, imprecise delineations between tokens suitable for early exit can lead to underthinking (premature exits that adversely affect accuracy) or overthinking (unnecessary processing that compromises efficiency) ~\cite{zhou2020self, dai2020dynamic}. Enhanced early alignment using ~\cref{eq:m2r2_approximating_ea} helps to address  first issue. To address the second issue we introduce Accelerated Residual Latent Attention, which dynamically assesses the saturation of the residual stream, allowing for a more precise differentiation between tokens that can exit early and those requiring further processing.

% This results in uniform change in residual direction    
% % We keep $\mathcal{A} = \hat{\mathcal{A}}$, while $\hat{\mathcal{M}}$ is accelerated by a factor of $2 - (N_{E}+1)X$ relative to the slower residual transformation $\mathcal{M}$, where $N_E$ represents number of early exiting intervals.
% Figure~\cref{fig:rate_change_comparison} illustrates the comparative rate of change between these transformation streams.



% fig:rate_change_comparison
% - grid plot x axis -> layer id (0, 8) , y axis -> layer id -> dark color cell for max similarity , lighter for lower 
% 
-------------------------------------------------------
Let's consider residual stream $h_i$ traverses through interval $E_j$ to $E_{j+1}$ and undergoes residual transformation given by 
\begin{equation}
h_{E_{j+1}} = h_{E_j} \prod_{i=E_j}^{E_{j+1}} \left( 1 + \delta_i \right)    
\end{equation}

where $\delta_i$ denotes non-linear transformation added by layer $i$. Each non-linear transformation of layer $i$ is a function of previous contextual representation, $c_i$ and input residual stream $h_i-1$ as
$\delta_i = f(c_i, h_{i-1})$ 

One way to exit early at exit $E_j+1$ is to assume that residual transformation from $E_j+1$ to final layer $N-1$ can be approximated by a linear function $\phi$ as $h_{N-1} \sim \Phi(h_{E_j+1})$ and most conventional approaches such as \todo{cite EA papers} use this approach. In other words, 

\begin{equation}
\Phi(h_{E_j+1} \sim h_{E_j+1} \prod_{i=E_j+1}^{N} \left( 1 + \delta_i \right)   
\end{equation}

This approach suffers from two primary issues, linearity assumption from $h_E_j+1$ to $H_N-1$ if often incorrect, particularly when $E_j << N$. More importantly, linear transformation $\Phi$ doesn't consider effect of context $C_i$. M2R2  effectively addresses these issues as accelerated residual stream at interval $E_j+1$ can be represented as 

\begin{equation}
r_{E_{j+1}} = r_{E_j} \prod_{i=E_j}^{E_{j+1}} \left( 1 + \gamma_i \right)    
\end{equation}

where $\gamma_i$ denotes non-linear transformation added by layer $i$ to previous accelerated residual $r_i-1$. Similar to $\delta_i$, non-linear transformation $\gamma_i$ considers context $C_i$ as 
$\gamma_i = g(c_i, r_{i-1})$. So in summary, slow residual transformation is approximated by accelerated residual as: 

\begin{equation}
h_{E_j} \prod_{i=E_j}^{N} \left( 1 + \delta_i \right) \sim h_{E_j} \prod_{i=E_j}^{E_j+1} \left( 1 + \gamma_i \right)
\end{equation}

It's worth noting that accelerated residual $r_i$ and slow residual $h_i$ are processed concurrently at layer $i$ by constructing proper attention mask such as attention of slow residual is represented as 

$MHA(H_it, H_{i<=t}, H_{i<=t}$ while attention of fast residual is computed as 

$MHA(r_it, H_{i<=t}, H_{i<=t}$ where $MHA(q,k,v$ denotes multi head attention between query, $q$, key $k$ and value $v$.


------------------------------------------------------------------

Vertical latent attention on accelerated residual is computed as 
$MHA(S_mt, S(Ej<=i<=m)t, S(Ej<=i<=m)t)$ where $Smt$ denotes query/key/value projection in latent domain at layer $m$ at time $t$. 
------------------------------------------------------------------

Gradient conflict Avoidance: 

Let's consider $w_j$ is a trainable parameter that belongs to a layer between $E_j$ and $E_j+1$. Consider early exit loss at gate $E_j+1$, $L_j+1$, gradient propagation of $w_j$ at another trainable parameter $w_j-n$ can be gives as 

$\sum_{k=E_j-n}^{E_j} \beta_k \frac{\partial L_{E_k}}{\partial w_k}$

where $\beta_j$ denotes backward transformation coefficient for weight $w_j$ to reach gate $E_j$. 
 
On the other hand, gradient propagation in proposed approach can be represented as 

\[
\frac{\partial L_{E_j}}{\partial w_j} = 
\begin{cases} 
\beta_j \frac{\partial L_{E_j}}{\partial w_j} & \text{if } E_j \leq w_j \leq E_{j+1} \\
0 & \text{otherwise}
\end{cases}
\]







% \begin{figure}[ht]
%     \centering
%     \includegraphics[width=0.8\textwidth, height=5cm]{rate_change_comparison.png}
%     \caption{Rate of change comparison between fast and slow residual streams.}
%     \label{fig:rate_change_comparison}
% \end{figure}

%vary k and and plot EA accuracy for larger and smaller models. 

% \begin{figure}[ht]
%     \centering
%     \includegraphics[width=0.5\textwidth,height=5cm]{sections/figures/alignment_comparison_dialogsum.pdf}
%     \caption{Alignment of exited tokens for different early exit layers using traditional early exiting heads, dedicated faster networks, and faster residuals.}
%     \label{fig:small_model_early_exiting}
% \end{figure}


\textbf{Accelerated Residual Latent Attention} \label{method_arla}

In the context of residual streams, we observe that the decision to exit at a given layer can be more effectively informed by analyzing the dynamics of residual stream transformations, instead of solely relying on a classification head applied at the early exit interval $E_j$. To capture the subtle dynamics of residual acceleration, we propose a \textit{Accelerated Residual Latent Attention} (ARLA) mechanism. This approach involves making the exit decision at gate $E_j$ by attending to the residuals spanning from gate $E_{j-1}$ to $E_j$, rather than considering only the residual at gate $E_j$. To minimize the computational overhead associated with exit decision-making, the attention mechanism operates within the latent domain as depicted in ~\cref{fig:arla_arch}. Formally, for each interval $[E_j, E_{j+1}]$, the accelerated residuals are projected into Query ($Q^s_{E_j}, \ldots, Q^s_{E_{j+1}}$), Key ($K^s_{E_j}, \ldots, K^s_{E_{j+1}}$), and Value ($V^s_{E_j}, \ldots, V^s_{E_{j+1}}$) vectors, with latent dimension $d^s$ for $Q^s$, $K^s$, and $V^s$ being significantly smaller than hidden dimension of $p$.\footnote{We use $d^s = 64$ for experiments described in ~\cref{sec:experiments}.} Notably, when the router is allowed to make exit decisions at gate $E_j$ based on residual change dynamics, we observe that the attention is not confined to the residual state at $E_j$ but is distributed across residual states from $E_{j-1}$ to $E_j$, %as illustrated in Figure~\ref{fig:vertical_latent_attention_dynamics}. 
This broader focus on residual dynamics significantly reduces decision ambiguity in early exits, as demonstrated in Figure~\ref{fig:roc_arla}, which contrasts routers based on the last hidden state, and the proposed ARLA router.

%show R -> S transformation. 
%show parameter and flop overhead as compared to adapter on last hidden state.

% \begin{figure}[ht]
%     \centering
%     \includegraphics[width=0.5\textwidth,height=5cm]{sections/figures/roc_arla.pdf}
%     \caption{ROC curves of early exit decision strategies: confidence-based methods (CALM/LITE), routers based on the accelerated hidden state, and latent attention routers.}
%     \label{fig:decision_making_comparison}
% \end{figure}

% \begin{figure}[ht]
%     \centering
%     \includegraphics[width=0.5\textwidth,height=5cm]{vertical_latent_attention.png}
%     \caption{Vertical latent attention mechanism for optimizing early exit decisions by considering residuals from gate \(M\) through \(M-1\).}
%     \label{fig:vertical_latent_attention}
% \end{figure}

\begin{figure}[ht]
    \centering
    \begin{subfigure}{0.52\textwidth}
        \centering
        \includegraphics[width=\textwidth, height = 4cm]{sections/figures/arla_arch.pdf}
        \caption{Accelerated Residual Latent Attention (ARLA): Accelerated residuals between early exit gates are projected into latent domain and attention over residual states within the interval is computed to capture residual dynamics and exit decision is made based on residual saturation.}
        \label{fig:arla_arch}
    \end{subfigure}%
    \hfill
    \begin{subfigure}{0.45\textwidth}
        \centering
        \includegraphics[width=\textwidth, height = 4.5cm]{sections/figures/vla_roc.pdf}
        \caption{ROC classification curves of early exit decision strategies using a linear router used on last residual state ~\cite{schuster2022confident, varshney-etal-2024-investigating, chen2023eellm}  and using ARLA approach that considers residual dynamics. }
        \label{fig:roc_arla}
    \end{subfigure}
    \caption{Effectiveness of ARLA in capturing residual dynamics for early exiting decisions.}


\end{figure}



% \begin{figure}[ht]
%     \centering
%     \includegraphics[width=1\textwidth,height=5cm]{sections/figures/arla.pdf}
%     \caption{fig that plots 32 rows 2 cols heatmap showing attention at each gate}
%     \label{fig:vertical_latent_attention_dynamics}
% \end{figure}

\subsubsection{Self Speculative Decoding} \label{method_self_speculative_decoding}

An alternative means to exploit the early alignment properties of our approach is through the use of accelerated residual states for speculative token sampling to accelerate autoregressive decoding. Speculative decoding aims to speed up memory-bound transformer inference by employing a lightweight draft model to predict candidate tokens, while verifying speculated tokens in parallel and advancing token generation by more than one token per full model invocation \cite{leviathan2023fast, chen2023accelerating, xia2023speculative, miao2023specinfer}. Despite its effectiveness in accelerating large language models (LLMs), speculative decoding introduces substantial complexity in both deployment and training. A separate draft model must be specifically trained and aligned with the target model for each application, which increases the training load and operational complexity ~\cite{chen2023accelerating}. Additionally, this approach is resource-inefficient, as it requires both the draft and target models to be simultaneously maintained in memory during inference \cite{leviathan2023fast, chen2023accelerating}. 

One strategy to address this inefficiency is to leverage the initial layers of the target model itself to generate speculative candidates, as depicted in ~\cite{Tang2024}. While this method reduces the autoregressive overhead associated with speculation, it suffers from suboptimal acceptance rates. This occurs because the linear transformation employed for translating hidden states from layer $k$ to the final layer $N$ is typically a poor approximation, as discussed in ~\cref{sec:motivation} and ~\cref{method_early_exiting}. Our approach resolves this limitation by utilizing accelerated residuals, which demonstrate higher fidelity to their slower counterparts. By utilizing accelerated residuals operating at a rate of $N/k$, where $k$ denotes the number of layers used for candidate speculation, we are able to efficiently generate speculative tokens for decoding.\footnote{We typically set $k = 4$ to balance the trade-off between autoregressive drafting overhead and acceptance rate, as discussed in~\cref{sec:experiments}.}
 This technique not only obviates the need for multiple models during inference but also improves the overall efficiency and effectiveness of speculative decoding.

\begin{figure}
    \centering    \includegraphics[width=1\linewidth]{sections/figures/m2r2_aot_loading.pdf}
    \caption{Ahead-of-Time Expert Loading: M2R2 accelerated residual stream predicts experts required for future layers, reducing reliance on on-demand lazy loading. Speculative pre-loading is efficiently overlapped with computation of multi-head attention (MHA) and MLP transformations. Only incorrectly speculated experts are loaded lazily, resulting in faster inference steps and improved computational efficiency. Here, H indicates LBM Host while D indicates HBM Device.}
    \label{fig:moe_expert_aot_loading}
\end{figure}


\subsubsection{Ahead of Time Expert Loading:} \label{method_aot_expert_loading}

Recent advancements in sparse Mixture-of-Experts (MoE) architectures ~\cite{shazeer2017outrageously, fedus2022switch, artetxe2019massively, lepikhin2020gshard, zoph2022designing} have introduced a paradigm shift in token generation by dynamically activating only a subset of experts per input, achieving superior efficiency in comparison to dense models, particularly under memory-bound constraints of autoregressive decoding \cite{fedus2022switch, zoph2022designing}. This sparse activation approach enables MoE-based language models to generate tokens more swiftly, leveraging the efficiency of selective expert usage and avoiding the overhead of full dense layer invocation. In dense transformer models, pre-loading layers is a common strategy to enhance throughput, as computations of current layer can be overlapped with pre-loading of next layer parameters ~\cite{narayanan2021efficient, shoeybi2020megatron}. However, MoE models face a unique challenge: expert selection occurs dynamically based on previous layer’s output, making it infeasible to preload next layer’s experts in parallel. This limitation results in inherent latency, as expert loading becomes a sequential, on-demand process ~\cite{lepikhin2020gshard, fedus2022switch}.

To address this inefficiency, our method introduces a mechanism with \textit{accelerated residuals}, which not only captures key characteristics of base slower residual states but also exhibit high cosine similarity with their final counterparts (as illustrated in \cref{fig:m2r2_residual_sim}). By employing accelerated residual streams, we can effectively predict the necessary experts for future layers well in advance of their actual invocation. Specifically, using a $2\times$ accelerated residual, the experts needed for layers $2i+2$ and $2i+3$ can be identified while still computing in layer $i$, thus overcoming the bottleneck of sequential, on-demand expert selection and mitigating latency in the decoding pipeline, as shown in \cref{fig:moe_expert_aot_loading}. Note that, we use fixed set of accelerator adapters for transforming accelerated residuals (as discussed in ~\cref{m2r2_method}) while slow residual is transformed via expert routing mechanism. 

Furthermore, our approach integrates a Least Recently Used (LRU) caching strategy, which enhances memory efficiency by replacing the least recently used experts with speculated experts that are anticipated to be needed in upcoming layers. This hybrid approach of preemptive expert loading with LRU caching yields substantial improvements over traditional on-demand loading or standalone caching strategies. By minimizing cache misses and efficiently managing memory, this approach addresses both compute and memory bottlenecks, leading to faster, more resource-efficient token generation in MoE architectures. A comprehensive evaluation of this strategy, in relation to state-of-the-art methods, is provided in \cref{experiments_aot}, and the compute and memory traces on an A100 GPU are detailed in \cref{fig:moe_aot_cuda_trace}.



% Recent advancements in sparse Mixture-of-Experts (MoE) architectures have introduced the concept of utilizing distinct computational paths for different tokens \cite{shazeer2017outrageously}. This approach, wherein only a subset of experts are activated per input, enables MoE-based language models to generate tokens more swiftly compared to their dense counterparts due to memory-bound nature of auto-regressive decoding. In dense models, pre-loading layers in advance is a common strategy to enhance computational efficiency. However, this technique is not applicable to MoE models, where expert selection occurs dynamically based on the outputs of previous layers, preventing parallel pre-fetching of experts.

% Our proposed method addresses this inefficiency. Accelerated residuals, which are highly similar to their slower counterparts (see \cref{fig:similarity}), can reliably predict the necessary experts ahead of time. For instance, by utilizing $2X$ accelerated residual stream, we can predict the experts needed for the layer $2i+1$ and $2i+3$ while carrying out computation in layer $i$. This enables us to commence expert loading significantly earlier, as illustrated in \cref{expert_loading}, effectively mitigating the delays observed with the naive on-demand expert loading. Additionally, our method benefits from incorporating a Least Recently Used (LRU) strategy, where speculated experts replace those that are least recently utilized, resulting in improved performance compared to using either strategy alone. For a comprehensive evaluation, refer to \cref{moe_trace}, which provides a CUDA compute and memory trace of our approach executed on <>.



% A naive solution involves using the residual state of the previous layer along with the gating function of the next layer to predict which experts need to be loaded, and initiating the expert loading process in parallel with the attention computation of the next layer. Yet, as shown in \cref{fig:MOE_attn_vs_loading_time}, the attention computation for medium to long contexts is considerably faster than the expert loading time, making this approach inefficient.




\subsection{Training} \label{method_training}
% This approach is feasible due to the absence of gradient conflicts, as discussed in \cref{sec:grad_conflict}.

To accelerate residual streams, we employ parallel accelerator adapters as described in \cref{m2r2_method}.  For the early exiting use-case outlined in \cref{method_early_exiting}, we define the training objective for these adapters using the following loss function, which combines cross-entropy loss at each exit $E_j$ with distillation loss at each layer $i$. Loss weights coefficients $\alpha_0$ and $\alpha_1$ are employed to balance contribution of corresponding losses.

\begin{align} \label{eq:mr_loss}
L_{\text{m2r2}} = \underbrace{-\alpha_0 \sum_{j=1}^{J} \sum_{t=1}^{T} \log p_{\theta} \left( \hat{y}_t^{E_j} \mid y_{<t}, x \right)}_{\text{cross-entropy loss}} 
+ \underbrace{\alpha_1\sum_{i=1}^{E_{J-1}} \sum_{t=1}^{T} \| \mathbf{p}_{t}^{i} - \mathbf{h}_{t}^{((i - E_{j(i)}) \cdot R_i) + E_{j(i)})} \|^2}_{\text{distillation loss}}.
\end{align}

where $\hat{y}_t^{E_j}$ denotes the predictions from the accelerated residual stream at layer $E_j$ and time step $t$, $y_t$ represents the corresponding ground truth tokens, and $x$ indicates previous context tokens. The distillation loss at each layer $i$ is computed by comparing accelerated residuals at layer $i$ with slow residuals at layer $(i - E_{j(i)}) \cdot R_i + E_{j(i)}$, where $R_i$ denotes the rate of accelerated residuals at layer $i$ while $E_{j(i)}$ represents the most recent gate layer index such that $E_{j(i)} <= i$. \( J \) represents the total number of early exit gates, N denotes number of hidden layers and $E_j$ denotes layer index corresponding to gate index $j$ and \( T \) denotes the sequence length. 

In dynamic compute settings, after training of accelerator adapters, we optimize the query, key, and value parameters governing the ARLA routers (see ~\cref{method_arla}) across all exits in parallel on binary cross entropy loss between predicted decision and ground truth exiting decision. The ground truth labels for the router are determined based on whether the application of the final logit head on $\hat{y}_t^{E_j}$ yields the correct next-token prediction. 


% The objective for this optimization is defined by the following loss function:


%TODO are equations required ? 
% \begin{equation} \label{eq:arla_loss_combined}\small
%     L_{\text{arla}} = -\frac{1}{N} \sum_{t=1}^{T} \left( \sum_{j=1}^{E_n} \left[ O_t^{E_j} \log(\hat{O}_t^{E_j}) + (1 - O_t^{E_j}) \log(1 - \hat{O}_t^{E_j}) \right] \right), \quad \text{where} \quad 
%     O_t^{E_j} = \begin{cases} 
%     1, & \text{if } L(\hat{y}_t^{E_j}) = y_t^{E_j} \\
%     0, & \text{otherwise}
%     \end{cases}
% \end{equation}

% where $\hat{O}_t^{E_j}$ represents the binary predicted logits produced by the vertical latent attention router, as described in \cref{sec:arla}, at gate $E_j$ and time step $t$, and $O_t^{E_j}$ denotes the corresponding ground truth labels. The ground truth labels for the router are determined based on whether the application of the logit head on $\hat{y}_t^{E_j}$ yields the correct next-token prediction. The parameters controlling vertical latent attention are trained concurrently to ensure consistency and efficient use of computational resources.

For self-speculative decoding, as described in \cref{method_self_speculative_decoding}, the training objective remains the same as \cref{eq:mr_loss}, but with the number of intervals set to $J = 1$ and the rate of residual transformation set to $R_n = N/k$, where the first $k$ layers generate speculative candidate tokens. In the context of Ahead-of-Time Expert Loading for Mixture-of-Experts (MoE) models (see \cref{method_aot_expert_loading}), setting the rate of residual transformation to $R_n = 2$ typically offers a good trade-off between the accuracy of expert speculation and AoT pre-loading of experts. 

% Thus, we set $J = 1$ and $E_1 = 16$.


~\subsection{FLOPs Optimization} \label{sec:flops_optimization}

Naively implemented, M2R2 incurs higher FLOP overhead compared to traditional speculative decoding and early exiting approaches such as ~\cite{medusa, schuster2022confident, Tang2024}. However, modern accelerators demonstrate compute bandwidth that exceeds memory access bandwidth by an order of magnitude or more~\cite{databricksLLMInference2023, jouppi2021ten}, meaning increased FLOPs do not necessarily translate to increased decoding latency. Nevertheless, to ensure fair comparison and efficiency in compute bound scenarios, we introduce targeted optimizations.

~\textbf{Attention FLOPs Optimization} For medium-to-long context lengths, attention computation dominates FLOPs in the self-attention layer, surpassing the contribution from MLP layers. Specifically, matrix multiplications involving queries, cached keys, and cached values scale with $l_{kv} * l_{q}$ where $l_{kv}$ denotes previous context length and $l_q$ denotes current query length. Since M2R2 pairs accelerated residuals with slow residuals, a naive implementation results in twice the FLOPs consumption compared to a standard attention layer. To address this, we limit the attention of accelerated residual stream to selectively attend to the top-k most relevant tokens, identified by the slow residual stream based on top attention coefficients\footnote{We set to k = 64 and attend to top 64 tokens as identified by the slow residual stream.}. This is possible since slow and accelerated residual streams are processed in same forward pass and accelerated streams have access to attention coefficients of slow stream. Note that, the faster residual stream still retains the flexibility to assign distinct attention coefficients to these tokens. Furthermore, we design the faster residual stream to employ only 8 attention heads, compared to the 32 heads used in the slow residual stream of the Phi-3 model, reducing query, key, value, and output projection FLOPs by a factor of 1/4. ~\cref{fig:m2r2_num_heads_ablation} indicates effect of using a slicker stream on alignment. As depicted, using $\hat{n}_h = 8$ offers a good trade-off between alignment and FLOPs overhead. 

~\textbf{MLP FLOPs Optimization} The accelerator adapters operating on the accelerated residual stream are intentionally designed with lower rank than their counterparts in the base model. This reduces FLOP overhead by a factor proportional to $hiddenSize / rank$. Additionally, since the faster residual stream uses only 8 attention heads (compared to 32 in the slow residual stream of Phi-3), the subsequent MLP layers process a smaller set of activations, further reducing FLOPs by another factor of 1/4.

These optimizations significantly reduce the FLOP overhead per speculative draft generation, as illustrated in ~\cref{fig:flops_optmization}. Notably, while traditional early-exiting speculative approaches such as DEED require propagating the full slow residual state through the initial layers, incurring substantial computational costs, M2R2 achieves efficient token generation via slimmer, low-rank faster residual streams. In contrast, Medusa introduces considerable FLOP overhead due to per-head computations scaling with $d^2+dv$\footnote{Here $d$ denotes hidden state dimension while $v$ denotes vocab size.}, whereas M2R2 employs low-rank layers for both MLP and language modeling heads, maintaining computational efficiency. All experiments involving the M2R2 approach, as detailed in ~\cref{sec:experiments}, are conducted using these FLOPs optimizations.









% \[
% O_t^{E_j} = 
% \begin{cases} 
% 1, & \text{if } L(\hat{y}_t^{E_j}) = y_t^{E_j} \\
% 0, & \text{otherwise}
% \end{cases}
% \]




%add distillation
% We train accelerator adapters described in \cref{m2r2_method} to accelerate residual streams on next token prediction all in parallel since there are no gradient conflict issues as described in \cref{sec:grad_conflict}.

% \begin{align} \label{eq:mr_loss}
% L_{mr} =  & -\sum_{j = 1}^{E_n} (\sum_{t=1}^{T}\log p_{\theta} (\hat{y}_t^{E_j} | \hat{y}_{<t}, x)) \nonumber
% \end{align}

% where $\hat{y_t^{E_j}}$ denotes predicted logits obtained from accelerated residual stream at gate $E_j$ and time-step $t$ while $y_t^{E_j}$ denotes corresponding truth tokens. 

% Upon training of adapters responsible for accelerating residual streams, we train query, key, value parameters responsible for vertical latent attention of all gates in parallel as

% \begin{equation} \label{eq:arla_loss}
%     L_{arla} = -\frac{1}{N} (\sum_{t=1}^{T}(1\sum_{j=1}^{E_n} \left[ O_t^{E_j} \log(\hat{O}_t^{E_j}) + (1 - o_t^{E_j}) \log(1 - \hat{o_t}_{E_j}) \right]))
% \end{equation}

% where $\hat{O_t^{E_j}}$ denotes binary predicted logits obtained from vertical latent attention router described in \cref{sec:arla} at gate $E_j$ and timestep $t$ while $O_t^{E_j}$ denotes corresponding truth label. Truth labels for router are obtained by computing whether logit head application on $\hat{y}_t^j$ results in true next token prediction. Formally speaking, 

% $O_t^{E_j} = 1 if L(\hat{y_t^{E_j}}) == y_t^{E_j} , 0 otherwise$. 

% Parameters responsible for vertical latent attention are also trained in parallel as well. 

%todo: training slow and fast residuals together and distillation can be two training mdoes. 
%Distillation can be an ablation. 




% Although transformer decoding is memory bound on most mainstream accelerators, there could be scenarios where flop savings are crucial. For instance, on on-device settings power consumption is directly correlated with flops per decoding step and reducing flops does help with overall energy consumption. Vanilla early exiting methods help with flop reduction but suffer from mismatch between training and inference due to early exited tokens. If token at decoding step $t$, $T_t$ exited at layer $E_i$, while token $T_{t+k}$ exits at layer $E_j$ such that $E_i < E_j$, hidden state $H_{t+k}l$ does not have corresponding hidden state $H_tl$ to attend to where $E_i < l <= E_j$. One solution that's often used in literature is to rely on last hidden state available, $H_t{E_j}$, however it tends to be sub-optimal and does affect generation quality \cite{ref}.  To alleviate this mismatch while reducing flops, we train router such that attention mask between token $T_{t+k}$ and token $T_{<t+k}$ is given by: 

% \begin{equation}
%     a_{T_{{t+k}{T_{<t+k}}} = 1 if  E_{T_{<t+k}} >= E{T_{t+k}}
%     else 0
% \end{equation}

% This attention mask enables router to account for exited tokens and get trained accordingly. Since attention mechanism during decoding remains exactly same as that during training, impact on generation quality tends to be minimal as noted in \cref{fig:gen_auality_with_and_without_recompute_attention_show_flops}.  Although MoD does not suffer from training and inference mismatch, we observe that it suffers from discountinuity between pre-training and super-vised fine-tuning resulting in sub-optimal perplexity. On the other hand, our method doesn't not require pre-training , doesn't suffer from discountinuity, and achieves much better perplexity in super-vised fine-tuning and instruction tuning setups as shown in \cref{fig:Mod_vs_m2r2_loss_curves}.






% Our techniques are directly applicable in such scenarios.    




%expert loading with cuda streams in experiments
\textbf{Optimize the Document Image (\atkIG)}.
By switching the perspective to the input space, this variant directly optimizes the pixel values of the document image $x$. The underlying intuition remains the same: training documents require less self-tuning allowing the model to converge faster to the correct answer than non-trainings. However, this assumes the target model allows differentiation of the document image pixels through its architecture. Accordingly, we replace $\theta$ with $x$ from Algorithm \ref{alg:assignment} while freezing the target model parameters $\theta$.

These variants reduce computational costs while maintaining attack performance, providing more practical options when the size of $D_\text{test}$ increases.
\vspace{-0.1in}
\subsection{Black-box \atk}
\label{sec:docmia_blackbox}
In the black-box setting, the attack model’s access is restricted to $D_{\text{test}}$ and the predicted answers. To address these limitations, we propose a distillation-based attack strategy. The key idea is to transfer knowledge about the private data $D_t$ from the black-box model $\mathcal{F}_t$ to a proxy model $\mathcal{F}_p$, parameterized by $\omega$. With full control over the proxy model, the attacks we design for the white-box setting can be fully applied to this.

Specifically, the black-box model is first employed to generate labels for each question in $D_\text{test}$, creating a query dataset $D_{\text{query}}=\{(x_i, Q_i)\}^{N_{\text{test}}}_{i=1}$ where $Q_i = \{(q_j, \mathcal{F}_{\theta_t}(x_i, q_j))\}^{M}_{j=1}$. The proxy model $\mathcal{F}_p$ is then trained on this query dataset, with the objective to maximize the likelihood of the predicted answer $p_{\omega}(\mathcal{F}_{\theta_t}(x_i, q_j)|x_i, q_j)$. In essence, the goal is to replicate the label-prediction behavior of the black-box model. By doing so, we aim to transfer the label space structure from the black-box to the proxy model, with the expectation that the membership features embedded in the black-box model will also be transferred, thus making our attack assumptions under white-box setting valid. Figure \ref{fig:method_distill} illustrates the scheme of our proposed attack.

Since our focus is on the document domain, we initialize $\mathcal{F}_p$ using a publicly available checkpoint $\omega_{\text{pt}}$, pre-trained with self-supervised learning on unlabeled document dataset $D_{\text{pt}}$, which is \textit{inaccessible} and assumed to be \textit{disjoint} from the private dataset $D_t$. This initialization equips the proxy model with a certain level of document understanding while ensuring it has no prior knowledge of the private dataset. As a result, it enables the proxy $\mathcal{F}_p$ to better mimic the prediction behavior and internal dynamics of the black-box model $\mathcal{F}_t$ after fine-tuning. 

It is important to note that, in this scenario, the adversary lacks information of the black-box training algorithm $\mathcal{T}$. This means there is no advantage in terms of model architecture or other training details when constructing the proxy model. As a result, the choice of the proxy model, optimizer, learning rate, etc., is independent of the target model. However, as we demonstrate empirically in later sections (Section \ref{sec:whitebox_results}), while there is a clear benefit when the proxy model shares the same architecture as the black-box model, our attack strategies remain effective even when using entirely different architectures. This suggests that the proposed approach is robust and can be applied without relying on specific model classes or requiring detailed knowledge of the black-box model.
\vspace{-0.1in}
\section{Experimental Setup}
\vspace{-0.1in}
\label{sec:experimental_setup}

\subsection{Target Dataset and Model}

\label{sec:target_model_dataset}
\textbf{Target Dataset}. We study two established DocVQA datasets in the literature for our analysis: \textbf{DocVQA (DVQA)} \citep{mathew2021docvqa} and \textbf{PFL-DocVQA (PFL)} \citep{tito2024privacy}. Both datasets are designed for extractive DocVQA task, where the answer is explicitly found within the document image. Each document in these datasets is accompanied by varying number of questions.

\textbf{Target Model}. We consider three state-of-the-art models which are designed for document understanding tasks: (1) \textbf{Visual T5 (VT5)} \citep{tito2024privacy} (250M parameters) follows the traditional design by utilizing OCR module to facilitate the reasoning process. It leverages the T5 model pre-trained on the C4 corpus \citep{raffel2020exploring}, along with a ViT backbone pre-trained on document data \citep{li2022dit}. (2) \textbf{Donut} \citep{Kim22Donut} (201M parameters) is one of the first end-to-end DocVQA models capable of achieving competitive performance without relying on OCR. It is pre-trained on a large collection of private synthetic documents. (3) \textbf{Pix2Struct} \citep{Lee23Pix2Struct} is another OCR-free document model with two versions: Base (282M) and Large (1.3B parameters). This model is pre-trained to perform semantic parsing on an 80M subset of the C4 corpus.

For the PFL-DocVQA dataset, we consider two targets: VT5, using the public checkpoint provided by the authors\footnote{\url{https://benchmarks.elsa-ai.eu/?ch=2}}, and Donut, which we successfully trained to achieve strong performance following the training procedure from the authors. For the DocVQA dataset, we attack four targets: VT5, Donut, and Pix2Struct (Base and Large), all with publicly available checkpoints from HuggingFace\footnote{\url{https://huggingface.co/models}} \citep{wolf-etal-2020-transformers}. In the black-box setting, we use VT5 and Donut as proxy models. To train the proxy models on the query set $D_{\text{query}}$, we initialize them with their public \textit{pre-trained} checkpoints—the same checkpoints used to fine-tune the target models on the respective target datasets, as outlined in the original papers. For more details on datasets and models, see Appendix \ref{sec:dataset} and \ref{sec:attack_implementation}.

\vspace{-0.2cm}
\subsection{Implementation}
\label{sec:implementation}
Since the optimization process involves several hyperparameters, our strategy is to tune the set of hyperparameters such that our attacks remain effective against each target model under white-box settings, which we then utilize to mount attacks on black-box models.

Assuming the knowledge of training algorithm $\mathcal{T}$ is unavailable for either white-box or black-box settings, we use Adam \citep{kingma2014adam} as the optimizer $\textsc{OPT}$ and fix this choice across all our attack experiments. We explore the impact of learning rate $\alpha$, the selected layer $L$, and we carefully tune the values of threshold $\tau$ in the ablation study (Appendix \ref{sec:calibration}). Following this, we select the optimal set of hyperparameters for each model and apply these settings in all black-box experiments. For the aggregation $\Phi$, we consider 4 aggregation functions $\{\textsc{avg}; \textsc{min}; \textsc{max}; \textsc{med}\}$ for each feature, denoted as $\Phi_\text{all}$. Throughout our experiments, we employ \textsc{KMeans} as the clustering algorithm. 

\vspace{-0.2cm}
\subsection{Evaluation Metric}
Using the official split of each target dataset, we sample 300 member documents from the training split and 300 non-member documents from the test split, resulting a total of $N_\text{test}=600$ test documents. We report \textit{Balanced Accuracy} and \textit{F1 score} as this evaluation metrics for the attack's success in the balanced setting, as in prior works \citep{salem2018ml,watson2022on,ye2022enhanced}. In addition, we evaluate our attacks using \textit{True Positive Rate (TPR) at 1\% and 3\% False Positive Rate (FPR)}, following standard practices in recent MIA literature \citep{carlini2022membership}. For all unsupervised attacks, the membership score for each document is computed as the Euclidean distance between its feature vector and the centroid of the member cluster obtained via \textsc{KMEANS}.

\subsection{Baseline}
In the \textit{black-box} setting, we evaluate three MI attacks as baselines, which only requires the predicted answer to determine membership: Score-Threshold Attack ($\textsc{Score-TA}$), Unsupervised Score-based Attack with $\textsc{avg}$ ($\textsc{Score-UA}$) \citep{tito2024privacy} and $\Phi_{\text{all}}$ ($\textsc{Score-UA}_{\text{all}}$).

For the \textit{grey-box} setting, we consider two additional baselines: Min-K\%\citep{shi2023detecting} and Min-K\%++ \citep{zhang2024min}, which assumes access to token-level probabilities of the generated answers to compute the membership score of each document.

In the \textit{white-box} setting, where loss or gradient information is accessible, we evaluate three further baselines: Loss-Threshold Attack ($\textsc{Loss-TA}$) \citep{yeom2018privacy}, Unsupervised Score+Loss Attack ($\textsc{ScoreLoss-UA}_{\text{all}}$) and Unsupervised Gradient Attack ($\textsc{Gradient-UA}$)\citep{nasr2019comprehensive}.

For detailed descriptions of these methods, we refer readers to Appendix \ref{sec:attack_baseline}.

\input{tables/baseline_results}

\vspace{-0.1in}
\section{Evaluation}
\label{sec:evaluation}
\vspace{-0.1in}
\subsection{White-box Setting}
\label{sec:whitebox_results}

\textbf{Baseline Performance Evaluation.}
Table \ref{tab:baseline_results} (\textit{right}) shows the performance of baseline attacks in the white-box setting. $\textsc{Loss-TA}$, akin to the thresholding loss attack in \citep{yeom2018privacy}, performs poorly on complex DocVQA models, achieving under 60\% accuracy for most targets. In contrast, $\textsc{ScoreLoss-UA}_{\text{all}}$, which combines utility scores and loss features, achieves stronger results: 81\% F1 on Donut, 75\% on VT5, and 69\% on Pix2Struct on DocVQA dataset. However, it underperforms $\textsc{Loss-TA}$ on PFL-DocVQA, with a 3\% drop in Accuracy and 8\% in F1, likely due to high loss variance in this dataset. $\textsc{Gradient-UA}$, which incorporates one-step gradient information, matches the performance of score-based attacks, suggesting that the gradient serves as a useful signal for membership inference. However, none of the baselines generalizes well across all target models.

\textbf{Our Proposed Attacks outperform the Baselines.} We evaluate our proposed attacks—\atkFL, \atkFLLoRA, and \atkIG—in the white-box setting across target models. As shown in Figure \ref{fig:whitebox_results}, our methods consistently achieve high performance, indicating that \textit{optimization-based features generalize well} across various models.
Compared to all baselines, our attacks achieve either the best or near-best performance on both target datasets, with notable F1 scores of 72\% against VT5 and Pix2Struct, and 82.5\% against Donut. Against $\textsc{Gradient-UA}$, our optimization-based features yield up to a 10\% improvement in F1 on Donut, indicating that \textit{single-step gradients are insufficient} for reliable membership inference.
From Table \ref{tab:tpr@fpr_main_whitebox}, our attacks consistently perform well in the low-FPR regime, often surpassing or matching the strongest baselines. For instance, \atkFL achieves a TPR of 8.67\% at 3\% FPR against VT5 on PFL, despite minimal overfitting, and a TPR of 11.00\% at the same FPR against Pix2Struct-B on DocVQA. Additionally, our methods outperform both Min-K\% and Min-K\%++ across all target models, underscoring their effectiveness, particularly for DocMIA setting.
These results highlight the privacy risks posed by optimization-based features in membership inference. For full results and in-depth analysis, please refer to  Appendix \ref{sec:tpr_at_fix_fpr} and \ref{sec:more_analysis}.
\begin{figure}[t]
    \centering
    \begin{minipage}{0.48\textwidth}
        \centering
        \resizebox{0.8\linewidth}{!}{
            \includegraphics[width=1.0\linewidth]{images/whitebox_results.pdf}
        }
        \caption{\textbf{White-box Setting: Our proposed attacks consistently achieve high performance}, generally outperforming the considered baselines.}
        \label{fig:whitebox_results}
        \end{minipage}
        \hfill
    \begin{minipage}{0.51\textwidth}
        \centering
        \resizebox{\linewidth}{!}{
            \begin{tabular}{lccccc}
                \toprule
                 & \multicolumn{3}{c}{\textbf{DVQA}} & \multicolumn{2}{c}{\textbf{PFL}}  \\
                \cmidrule(l){2-4}
                \cmidrule(l){5-6}
                 & {VT5} & {Donut} & {Pix2Struct-B} & {VT5} & {Donut} \\
                \midrule
                $\textsc{Loss-TA}$ & \textbf{14.00} & 7.67 & 5.33 & 3.00 & \textbf{14.67} \\
                $\textsc{Gradient-UA}$ & 9.33 & 6.00 & 5.00 & 3.00 & 8.33 \\
                $\textsc{ScoreLoss-UA}_{\text{all}}$ & 4.67 & 8.67 & 6.67 & 4.00 & 6.33 \\
                \midrule
                Min-K\% & 10.67 & 1.33 & 5.33 & 5.67 & 0.00 \\
                Min-K\%++ & 7.00 & 9.33 & 10.33 & 8.00 & 2.00 \\
                \midrule
                FL & 5.67 & \textbf{10.67} & \textbf{11.00}& \textbf{8.67} & 7.00 \\
                FLLoRA & 11.33 & 5.33 & 6.33 & 3.33 & 10.00 \\
                IG & 5.67 & 8.00 & 10.33 & 2.33  & 11.00 \\
                \bottomrule
            \end{tabular}
        }
        \captionof{table}{\textbf{White-box Setting: TPR at 3\% FPR}. Comparison across all white-box methods, with the best-performing method for each metric highlighted in \textbf{bold}. We refer the readers to Appendix \ref{sec:tpr_at_fix_fpr} for the complete results.}
        \label{tab:tpr@fpr_main_whitebox}
        \end{minipage}
    \vskip -0.2in
\end{figure}




\textbf{Why are Our Attacks more effective?}
We evaluate the effectiveness of our optimization-based features compared to traditional metrics such as loss or single-gradient norm.
% \begin{figure}[t]
    \centering
    \begin{minipage}[b]{0.66\textwidth}
        \centering
        \resizebox{0.999\linewidth}{!}{
            \includegraphics[height=5cm,width=1.05\textwidth]{images/whitebox_feature.pdf}
        }
        \caption{\textbf{Membership Features against three different target models on DocVQA Dataset.} \textit{Top}: The distribution of average \textbf{\textit{loss}} over all questions from all target documents on each target model. \textit{Bottom}: T-SNE visualization of the features used in our proposed attacks.}
        \label{fig:whitebox_loss}
    \end{minipage}
    \hfill
    \begin{minipage}[b]{0.3\textwidth}
        \centering
        \includegraphics[height=5cm]{images/whitebox_distance.pdf}
        \caption{\textbf{Distribution of Average \textit{Distance}} from member and non-member documents.}
        \label{fig:whitebox_distance}
    \end{minipage}
\label{fig:whitebox_feature}
\vspace{-0.2in}
\end{figure}


% Limitations of Loss-based Attacks
The Loss-based attack $\textsc{LOSS-TA}$ assumes that member documents exhibit lower loss values than non-member documents after training the target model $\mathcal{F}_{t}$. While this approach leverages the generalization gap, it proves too simplistic for large-scale models that are trained with complex training process to minimize overfitting. The generalization capability of these models, especially in DocVQA tasks, often reduces the sensitivity of the loss as a membership indicator. Our attacks, on the other hand, leverage the optimization landscape with respect to the model parameters, conditioned on each question-answer pair. We hypothesize that the \textit{distance} resulting from parameter optimization of pairs from a member document will be smaller compared to those for a non-member document, as depicted in Figure \ref{fig:intuition}. This fine-grained signal, which reflects the model's internal response to optimization, offers a more discriminative feature for identifying membership.

% compare from the evidence
As illustrated in Appendix Figure \ref{fig:whitebox_distance}, our \textit{distance} feature, computed from the optimization process, provides a better separation between members and non-members compared to loss-based methods (Figure \ref{fig:whitebox_loss}  (\textit{top})). The t-SNE visualization \citep{JMLR:v9:vandermaaten08a} from Figure \ref{fig:whitebox_loss} (\textit{bottom}) further demonstrates that features derived from our attacks yield a more distinct clustering of member and non-member documents in high-dimensional space for all target models, underscoring its efficacy as a membership indicator, therefore outperforms the loss-based approach.
\vspace{-0.1in}
\subsection{Black-box Setting}
\label{sec:blackbox_results}
% \vspace{-0.1in}
\textbf{Baseline Performance Evaluation.} Table \ref{tab:baseline_results} (\textit{left}) presents the results of our black-box baseline attacks, all of which rely on the DocVQA score as the only source of information in this setting. Similar to the loss metric, the score metric is directly correlated with the generalization gap, making attacks more effective when there is a higher degree of overfitting. This trend is illustrated in Figure \ref{fig:F1_scoregap}, where we observe strong MI performance, particularly for the Donut model with 75\% in PFL and 79\% F1 score in DocVQA. Meanwhile, both $\textsc{Score-UA}$-based baselines show comparable performance, especially effective against models trained on DocVQA. Overall, no single method emerges as the clear winner across all target models.

\begin{figure}[h]
    \centering
    \subfigure[\centering \textbf{White-box}]{
        \includegraphics[width=0.48\textwidth]{images/whitebox_generalization.pdf}
    }
    \subfigure[\centering \textbf{Black-box}]{
        \includegraphics[width=0.48\textwidth]{images/blackbox_generalization.pdf}
    }
\caption{\textbf{MI performance versus the Train-Test gap.} The target models exhibit varying Train-Test gaps, measured by the difference in DocVQA scores between member/non-member documents. Our attacks remain effective even when the gap is small, with performance improving as the gap increases across most target models and datasets. In contrast, baseline methods show more variable performance under these conditions.}
\vspace{-0.1in}
\label{fig:F1_scoregap}
\end{figure}


\textbf{Attacking via Proxy Model.} Table \ref{tab:blackbox_results} and Table \ref{tab:tpr@fpr_main_blackbox} present the key results of our proposed black-box attacks using two proxy models, VT5 and Donut. Several important observations:

First, we observe a clear advantage of attacking the proxy models distilled with our proposed techniques. Across a wide range of black-box architectures trained on both target datasets, attacks leveraging the proxy models outperform the black-box baselines in most cases, demonstrating better MI performance. This suggests that, even without knowledge of the black-box model architecture, \textit{one chosen proxy model still effectively distills certain behaviors} from the black-box models which are membership-indicative, enabling our attacks to infer membership with high accuracy.

When the black-box model architecture \textit{matches} that of the proxy, we consistently observe improvements in MI performance, especially when targeting the PFL-DocVQA dataset. Among the target, \textit{Pix2Struct is the most vulnerable} (both Base and Large). Both VT5 and Donut proxies gains of +3.04\% in Accuracy and +4.88\% in F1 score over the best baseline, even against the Pix2Struct-L model, which exhibits strong generalization and a minimal Train-Test gap (Figure \ref{fig:F1_scoregap}).
Furthermore, proxy VT5 can achieve TPRs of 23.00\% and 16.67\% against Donut and Pix2Struct-B, respectively, at 3\% FPR on DocVQA.
% , as shown in Table \ref{tab:tpr@fpr_main_blackbox}
% This aligns with the observed Train-Test utility gaps in target models (Table~\ref{tab:docvqa_performance}), allowing proxies to closely replicate black-box predictions and enhance attack success.
We also provide an analysis of the proxy model in Appendix \ref{sec:impact_proxy_model}.

These results suggest that privacy vulnerabilities can be exploited using simple distillation-based strategies applied to the model's output space.

% \begin{table}[t]
\begin{center}
\begin{small}
\begin{adjustbox}{width=0.8\textwidth}
\small
\begin{tabular}{clcccccc}
\toprule
\multicolumn{2}{c}{Proxy Model} & \multicolumn{6}{c}{VT5}\\

\midrule
\multicolumn{2}{c}{\multirow{2}{*}{\textbf{Black-box}}} & \multicolumn{2}{c}{FL} & \multicolumn{2}{c}{FLLoRA} & \multicolumn{2}{c}{IG}\\
%\cmidrule{3-8}
\cmidrule(l){3-4}
\cmidrule(l){5-6}
\cmidrule(l){7-8}
&                                 & ACC & F1 & ACC & F1 & ACC & F1\\
\midrule 
\multirow{2}{*}{\rotatebox[origin=c]{90}{\textbf{PFL}}}& \textbf{VT5} & $63.33_{0.0}\textcolor{red}{(+2.33)}$ & $69.51_{0.03}\textcolor{red}{(+0.71)}$ & $63.33_{0.0}\textcolor{red}{(+2.33)}$ & $69.01_{0.0}\textcolor{red}{(+0.21)}$ & $62.00_{0.16}\textcolor{red}{(+1)}$ & $69.35_{0.2}\textcolor{red}{(+0.55)}$\\

& \textbf{Donut} & $70.83_{0.0}\textcolor[HTML]{0b5394}{(-0.34)}$ & $76.64_{0.0}\textcolor{red}{(+0.87)}$ & $70.83_{0.0}\textcolor[HTML]{0b5394}{(-0.34)}$ & $76.70_{0.0}\textcolor{red}{(+0.93)}$ & $70.67_{0.0}\textcolor[HTML]{0b5394}{(-0.5)}$ & $76.72_{0.0}\textcolor{red}{(+0.95)}$\\
\midrule

\multirow{4}{*}{\rotatebox[origin=c]{90}{\textbf{DVQA}}}& \textbf{VT5} & $74.33_{0.01}\textcolor[HTML]{0b5394}{(-0.84)}$ & $75.08_{0.0}\textcolor[HTML]{0b5394}{(-1.1)}$ & $74.33_{0.0}\textcolor[HTML]{0b5394}{(-0.84)}$ & $74.67_{0.0}\textcolor[HTML]{0b5394}{(-1.51)}$ & $73.83_{0.08}\textcolor[HTML]{0b5394}{(-1.34)}$ & $75.81_{0.0}\textcolor[HTML]{0b5394}{(-0.37)}$\\

& \textbf{Donut} & $81.67_{0.0}\textcolor{red}{(+1.17)}$ & $82.54_{0.0}\textcolor{red}{(+1.44)}$ & $81.17_{0.0}\textcolor{red}{(+0.67})$ & $82.09_{0.0}\textcolor{red}{(+0.99)}$ & $80.17_{0.0}\textcolor[HTML]{0b5394}{(-0.33)}$ & $81.89_{0.0}\textcolor{red}{(+0.79)}$\\

& \textbf{Pix2Struct-B} & $70.17_{0.0}\textcolor{red}{(+1.04)}$ & $69.71_{0.0}\textcolor[HTML]{0b5394}{(-1.65})$ & $70.27_{0.23}\textcolor{red}{(+1.14)}$ & $70.85_{0.07}\textcolor[HTML]{0b5394}{(-0.51)}$ & $71.17_{0.0}\textcolor{red}{(+2.04)}$ & $72.14_{0.0}\textcolor{red}{(+0.78)}$\\

\cmidrule{2-8}
& \textbf{Pix2Struct-L} & $71.67_{0.01}\textcolor{red}{(+0.84)}$ & $72.13_{0.0}\textcolor{red}{(+1.30)}$ & $70.17_{0.0}\textcolor[HTML]{0b5394}{(-0.66)}$ & $71.27_{0.0}\textcolor{red}{(+0.44)}$ & $71.00_{0.05}\textcolor{red}{(+0.17)}$ & $73.15_{0.0}\textcolor{red}{(+2.32)}$\\

\midrule
\multicolumn{2}{c}{Proxy Model} & \multicolumn{6}{c}{Donut}\\ 

\midrule
\multirow{2}{*}{\rotatebox[origin=c]{90}{\textbf{PFL}}}& \textbf{VT5} & $61.73_{0.08}\textcolor{red}{(+0.73)}$ & $64.04_{0.10}\textcolor[HTML]{0b5394}{(-4.76)}$ & $61.67_{0.08}\textcolor{red}{(+0.67)}$ & $63.49_{0.0}\textcolor[HTML]{0b5394}{(-5.31)}$ & $55.17_{0.17}\textcolor[HTML]{0b5394}{(-5.83)}$ & $57.37_{0.3}\textcolor[HTML]{0b5394}{(-11.43)}$\\

& \textbf{Donut} & $72.17_{0.0}\textcolor{red}{(+2.33)}$ & $76.24_{0.0}\textcolor[HTML]{0b5394}{(-0.19)}$ & $72.67_{0.0}\textcolor{red}{(+1.5)}$ & $77.47_{0.0}\textcolor{red}{(+1.7)}$ & $74.50_{0.0}\textcolor{red}{(+3.33)}$ & $76.43_{0.0}\textcolor{red}{(+0.66)}$\\

\midrule
\multirow{4}{*}{\rotatebox[origin=c]{90}{\textbf{DVQA}}}& \textbf{VT5} & $73.50_{0.0}\textcolor[HTML]{0b5394}{(-4.34)}$ & $75.58_{0.0}\textcolor[HTML]{0b5394}{(-4.36)}$ & $74.17_{0.0}\textcolor[HTML]{0b5394}{(-1)}$ & $76.04_{0.0}\textcolor[HTML]{0b5394}{(-0.14)}$ & $74.0_{0.0}\textcolor[HTML]{0b5394}{(-1.17)}$ & $75.93_{0.01}\textcolor[HTML]{0b5394}{(-0.25)}$\\

& \textbf{Donut} & $79.50_{0.0}\textcolor[HTML]{0b5394}{(-1)}$ & $81.50_{0.0}\textcolor{red}{(+0.4)}$ & $80.0_{0.0}\textcolor[HTML]{0b5394}{(-0.5)}$ & $81.82_{0.0}\textcolor{red}{(+0.72)}$ & $80.27_{0.0}\textcolor[HTML]{0b5394}{(-0.23)}$ & $81.96_{0.0}\textcolor{red}{(+0.86)}$\\

& \textbf{Pix2Struct-B} & $70.83_{0.0}\textcolor{red}{(+3.04)}$ & $71.82_{0.0}\textcolor{red}{(+4.88)}$ & $70.83_{0.06}\textcolor{red}{(+1.70)}$ & $71.73_{0.14}\textcolor{red}{(+0.37)}$ & $71.0_{0.01}\textcolor{red}{(+1.87)}$ & $71.94_{0.0}\textcolor{red}{(+0.58)}$\\

\cmidrule{2-8}
& \textbf{Pix2Struct-L} & $70.83_{0.0}{(0)}$ & $72.95_{0.0}\textcolor{red}{(+2.12)}$ & $71.0_{0.0}\textcolor{red}{(+0.17)}$ & $72.98_{0.0}\textcolor{red}{(+2.15)}$ & $71.0_{0.03}\textcolor{red}{(+0.17)}$ & $72.81_{0.01}\textcolor{red}{(+1.98)}$\\
\bottomrule
\end{tabular}
\end{adjustbox}
\end{small}
\end{center}

\caption{\textbf{Black-Box Setting: Main Results of Black-Box \atk using Donut and VT5 as proxy models}. The checkpoints for the \textbf{black-box models} are trained on the respective datasets. Values in parentheses indicate the improvement (\textcolor{red}{positive}/\textcolor[HTML]{0b5394}{negative}) compared to the \textit{best} number from $\textsc{Score-UA}$-based baselines. Results are reported over five random seeds.}
\vspace{-0.2in}
\label{tab:blackbox_results}
\end{table}

\begin{table}[t]
    \centering
    \begin{minipage}{0.6\textwidth}
        \centering
        \resizebox{\linewidth}{!}{
            \begin{tabular}{clcccccc}
            \toprule
            \multicolumn{2}{c}{Proxy Model} & \multicolumn{6}{c}{VT5}\\
            
            \midrule
            \multicolumn{2}{c}{\multirow{2}{*}{\textbf{Black-box}}} & \multicolumn{2}{c}{FL} & \multicolumn{2}{c}{FLLoRA} & \multicolumn{2}{c}{IG}\\
            %\cmidrule{3-8}
            \cmidrule(l){3-4}
            \cmidrule(l){5-6}
            \cmidrule(l){7-8}
            &                                 & ACC & F1 & ACC & F1 & ACC & F1\\
            \midrule 
            \multirow{2}{*}{\rotatebox[origin=c]{90}{\textbf{PFL}}}& \textbf{VT5} & $63.33_{0.0}\textcolor{red}{(+2.33)}$ & $69.51_{0.03}\textcolor{red}{(+0.71)}$ & $63.33_{0.0}\textcolor{red}{(+2.33)}$ & $69.01_{0.0}\textcolor{red}{(+0.21)}$ & $62.00_{0.16}\textcolor{red}{(+1)}$ & $69.35_{0.2}\textcolor{red}{(+0.55)}$\\
            
            & \textbf{Donut} & $70.83_{0.0}\textcolor[HTML]{0b5394}{(-0.34)}$ & $76.64_{0.0}\textcolor{red}{(+0.87)}$ & $70.83_{0.0}\textcolor[HTML]{0b5394}{(-0.34)}$ & $76.70_{0.0}\textcolor{red}{(+0.93)}$ & $70.67_{0.0}\textcolor[HTML]{0b5394}{(-0.5)}$ & $76.72_{0.0}\textcolor{red}{(+0.95)}$\\
            \midrule
            
            \multirow{4}{*}{\rotatebox[origin=c]{90}{\textbf{DVQA}}}& \textbf{VT5} & $74.33_{0.01}\textcolor[HTML]{0b5394}{(-0.84)}$ & $75.08_{0.0}\textcolor[HTML]{0b5394}{(-1.1)}$ & $74.33_{0.0}\textcolor[HTML]{0b5394}{(-0.84)}$ & $74.67_{0.0}\textcolor[HTML]{0b5394}{(-1.51)}$ & $73.83_{0.08}\textcolor[HTML]{0b5394}{(-1.34)}$ & $75.81_{0.0}\textcolor[HTML]{0b5394}{(-0.37)}$\\
            
            & \textbf{Donut} & $81.67_{0.0}\textcolor{red}{(+1.17)}$ & $82.54_{0.0}\textcolor{red}{(+1.44)}$ & $81.17_{0.0}\textcolor{red}{(+0.67})$ & $82.09_{0.0}\textcolor{red}{(+0.99)}$ & $80.17_{0.0}\textcolor[HTML]{0b5394}{(-0.33)}$ & $81.89_{0.0}\textcolor{red}{(+0.79)}$\\
            
            & \textbf{Pix2Struct-B} & $70.17_{0.0}\textcolor{red}{(+1.04)}$ & $69.71_{0.0}\textcolor[HTML]{0b5394}{(-1.65})$ & $70.27_{0.23}\textcolor{red}{(+1.14)}$ & $70.85_{0.07}\textcolor[HTML]{0b5394}{(-0.51)}$ & $71.17_{0.0}\textcolor{red}{(+2.04)}$ & $72.14_{0.0}\textcolor{red}{(+0.78)}$\\
            
            \cmidrule{2-8}
            & \textbf{Pix2Struct-L} & $71.67_{0.01}\textcolor{red}{(+0.84)}$ & $72.13_{0.0}\textcolor{red}{(+1.30)}$ & $70.17_{0.0}\textcolor[HTML]{0b5394}{(-0.66)}$ & $71.27_{0.0}\textcolor{red}{(+0.44)}$ & $71.00_{0.05}\textcolor{red}{(+0.17)}$ & $73.15_{0.0}\textcolor{red}{(+2.32)}$\\
            
            \midrule
            \multicolumn{2}{c}{Proxy Model} & \multicolumn{6}{c}{Donut}\\ 
            
            \midrule
            \multirow{2}{*}{\rotatebox[origin=c]{90}{\textbf{PFL}}}& \textbf{VT5} & $61.73_{0.08}\textcolor{red}{(+0.73)}$ & $64.04_{0.10}\textcolor[HTML]{0b5394}{(-4.76)}$ & $61.67_{0.08}\textcolor{red}{(+0.67)}$ & $63.49_{0.0}\textcolor[HTML]{0b5394}{(-5.31)}$ & $55.17_{0.17}\textcolor[HTML]{0b5394}{(-5.83)}$ & $57.37_{0.3}\textcolor[HTML]{0b5394}{(-11.43)}$\\
            
            & \textbf{Donut} & $72.17_{0.0}\textcolor{red}{(+2.33)}$ & $76.24_{0.0}\textcolor[HTML]{0b5394}{(-0.19)}$ & $72.67_{0.0}\textcolor{red}{(+1.5)}$ & $77.47_{0.0}\textcolor{red}{(+1.7)}$ & $74.50_{0.0}\textcolor{red}{(+3.33)}$ & $76.43_{0.0}\textcolor{red}{(+0.66)}$\\
            
            \midrule
            \multirow{4}{*}{\rotatebox[origin=c]{90}{\textbf{DVQA}}}& \textbf{VT5} & $73.50_{0.0}\textcolor[HTML]{0b5394}{(-4.34)}$ & $75.58_{0.0}\textcolor[HTML]{0b5394}{(-4.36)}$ & $74.17_{0.0}\textcolor[HTML]{0b5394}{(-1)}$ & $76.04_{0.0}\textcolor[HTML]{0b5394}{(-0.14)}$ & $74.0_{0.0}\textcolor[HTML]{0b5394}{(-1.17)}$ & $75.93_{0.01}\textcolor[HTML]{0b5394}{(-0.25)}$\\
            
            & \textbf{Donut} & $79.50_{0.0}\textcolor[HTML]{0b5394}{(-1)}$ & $81.50_{0.0}\textcolor{red}{(+0.4)}$ & $80.0_{0.0}\textcolor[HTML]{0b5394}{(-0.5)}$ & $81.82_{0.0}\textcolor{red}{(+0.72)}$ & $80.27_{0.0}\textcolor[HTML]{0b5394}{(-0.23)}$ & $81.96_{0.0}\textcolor{red}{(+0.86)}$\\
            
            & \textbf{Pix2Struct-B} & $70.83_{0.0}\textcolor{red}{(+3.04)}$ & $71.82_{0.0}\textcolor{red}{(+4.88)}$ & $70.83_{0.06}\textcolor{red}{(+1.70)}$ & $71.73_{0.14}\textcolor{red}{(+0.37)}$ & $71.0_{0.01}\textcolor{red}{(+1.87)}$ & $71.94_{0.0}\textcolor{red}{(+0.58)}$\\
            
            \cmidrule{2-8}
            & \textbf{Pix2Struct-L} & $70.83_{0.0}{(0)}$ & $72.95_{0.0}\textcolor{red}{(+2.12)}$ & $71.0_{0.0}\textcolor{red}{(+0.17)}$ & $72.98_{0.0}\textcolor{red}{(+2.15)}$ & $71.0_{0.03}\textcolor{red}{(+0.17)}$ & $72.81_{0.01}\textcolor{red}{(+1.98)}$\\
            \bottomrule
            \end{tabular}
        }
        \caption{\textbf{Black-Box Setting: Main Results of Black-Box \atk using Donut and VT5 as proxy models}. The checkpoints for the \textbf{black-box models} are trained on the respective datasets. Values in parentheses indicate the improvement (\textcolor{red}{positive}/\textcolor[HTML]{0b5394}{negative}) compared to the \textit{best} number from $\textsc{Score-UA}$-based baselines. Results are reported over five random seeds.}
        \vspace{-0.2in}
        \label{tab:blackbox_results}
    \end{minipage}
    \hfill
    \begin{minipage}{0.39\textwidth}
        \centering
        \resizebox{\linewidth}{!}{
            \begin{tabular}[]{clcccccc}
            \toprule
             &  & \multicolumn{4}{c}{\textbf{DVQA}} & \multicolumn{2}{c}{\textbf{PFL}} \\
            \cmidrule(l){3-6}
            \cmidrule(l){7-8}
             &  & {\textbf{VT5}} & {\textbf{Donut}} & {\textbf{P2S-B}} & {\textbf{P2S-L}} & {\textbf{VT5}} & {\textbf{Donut}} \\
            \midrule
            & {$\textsc{Score-TA}$} & 9.33  & 11.00 & 8.00  & \textbf{9.00} & 5.00  & 2.67 \\
            & {$\textsc{Score-UA}$} & 7.67 & 15.67 & 6.33  & 6.67  & 3.33  & 3.33 \\
            & {$\textsc{Score-UA}_{\text{all}}$} & 9.33  & 11.00  & 8.00  & 9.00 & 5.00  & 2.67 \\
            \midrule
            \multirow{3}{*}{VT5} & FL & \textbf{12.33} & \textbf{23.00} & \textbf{16.67}  & 5.33 & 2.00 &  \textbf{8.00} \\
            & FLLoRA & {11.33} & 16.33  & 9.33  & 4.67  & 3.33 & 2.00 \\
            & IG & 8.33  & 7.00 & 7.67  & 7.00 & 3.67 & 6.67 \\
            \cmidrule(l){1-8}
            \multirow{3}{*}{Donut} & FL  & 6.33  & 4.00  & 4.67 & 7.33  & 1.33  & 4.00 \\
            & FLLoRA & 6.33  & 5.00  & 6.33  & 8.00  & 5.33 & 5.33 \\
            & IG  & 5.00 & 11.00  & 9.33 &  6.33  & \textbf{6.33}& 4.33 \\
            \bottomrule
            \end{tabular}
        }
        \caption{\textbf{Black-box Setting: TPR at 3\% FPR using Donut and VT5 as proxy models}. Comparison across all black-box methods, with the best-performing method for each metric highlighted in \textbf{bold}. The complete results can be found in the Appendix \ref{sec:tpr_at_fix_fpr}.}
        \label{tab:tpr@fpr_main_blackbox}    
    \end{minipage}
\end{table}




%\section{Limitation}
\vspace{-0.1in}
\section{Conclusion}
% \vspace{-0.1in}
In this paper, we introduce the first document-level membership inference attacks for DocVQA models, highlighting privacy risks in multi-modal settings. By leveraging model optimization techniques, we extract discriminative features that address challenges posed by multi-modal data, repeated document occurrences in training, and auto-regressive outputs. This enables us to propose novel, auxiliary data-free attacks for both white-box and black-box scenarios. Our methods, evaluated across multiple datasets and models, outperform existing membership inference baselines, emphasizing the privacy vulnerabilities in DocVQA models and the urgent need for stronger privacy safeguards.

% \newpage

\section{Ethics Statement}
Our research introduces two novel membership inference attacks on DocVQA models, designed to evaluate the privacy risks inherent in such systems. While our methodology exposes vulnerabilities that could potentially be exploited for malicious purposes, the primary objective of this work is to raise awareness about privacy issues in AI systems, specifically in the context of DocVQA models, and to encourage the development of more privacy-preserving technologies.

\subsubsection*{Acknowledgment}
This work has been funded by the European Lighthouse on Safe and Secure AI (ELSA) from the European Union’s Horizon Europe programme under grant agreement No 101070617. Views and opinions expressed are however those of the authors only and do not necessarily reflect those of the European Union or European Commission. Neither the European Union nor the European Commission can be held responsible for them. Khanh Nguyen and Dimosthenis Karatzas have been supported by the Consolidated Research Group 2021 SGR 01559 from the Research and University Department of the Catalan Government, and by project PID2023-146426NB-100 funded by MCIU/AEI/10.13039/501100011033 and FSE+.

% \section{Reproducibility Statement}
% In this work, we have made several efforts to ensure the reproducibility of our results. We utilize public datasets and open-source models, which are clearly described in Section ~\ref{sec:target_model_dataset}. The implementation details of our proposed membership inference attacks are thoroughly presented in Section~\ref{sec:implementation} and Appendix~\ref{sec:attack_implementation}. Additionally, all relevant hyperparameters used in our experiments are provided in the Appendix~\ref{sec:calibration}, offering detailed information for reproducing our results. We will provide a link to the code for the camera-ready version, enabling future researchers to replicate and extend our work with ease.

\bibliography{main}
\bibliographystyle{iclr2025_conference}

\appendix

\newpage

\startcontents[appendices]
\printcontents[appendices]{l}{1}{\section*{\textbf{Appendix}}\setcounter{tocdepth}{4}}

\vspace{1cm}

\section{DocVQA Datasets}
\label{sec:dataset}

\textbf{DocVQA} \citep{mathew2021docvqa} This dataset contains high-quality human annotations and is widely used as a benchmark for document understanding. It comprises of real-world administrative documents across a diverse range of types, including letters, invoices, and financial reports.

\textbf{PFL-DocVQA} \citep{tito2024privacy} A large-scale dataset of real business invoices, often containing privacy-sensitive information such as payment amounts, tax numbers, and bank account details. This dataset is specifically designed for DocVQA tasks in a federated learning and differential privacy setup, supporting different levels of privacy granularity. The dataset is accompanied by a variant of MI attacks, where the goal is to infer the membership of the invoice's owner (i.e., the provider) from a set of their invoices that were not used during training.

% Due to the unifying nature of DocVQA, our proposed attacks can be applied to any document models that follow the query-response framework. To extend our analysis, we also investigate two additional datasets: \textbf{SROIE} \citet{} and \textbf{FUNSD} \citet{}, which are collections of scanned documents designed for key information extraction tasks. The goal in these tasks is to extract the corresponding value text given a query key.

\begin{table}[ht]
\fontsize{6}{7}\selectfont
\centering
% \renewcommand{\arraystretch}{0.8} % Reduce vertical space
\resizebox{0.75\columnwidth}{!}{%
\begin{tabular}{@{}p{2.1cm}p{0.5cm}p{0.5cm}@{}}
\midrule
               & \ZhEn & \EnDe \\ \cmidrule{1-3}
Documents      & 38            & 30 \\
Segments      & 377           & 104 \\
Avg. English tokens/seg   &     32.02     & 71.91\\\midrule
\end{tabular}%
}
\caption{
Basic dataset statistics. For \ZhEn, average tokens per segment are based on the English reference translation, and for \EnDe, on the English source. Tokens are counted using whitespace in both cases.
}
\label{tab:basic_stats}
\vspace{-10pt}
\end{table}
In Table \ref{tab:dataset_stats}, we present statistics for both the DocVQA and PFL-DocVQA datasets. Additionally, Figure \ref{fig:questions_per_document} shows the distribution of questions per document: (1) while a small subset of documents have more than 10 questions, most contain fewer, and (2) a fraction of documents have only a single question. These trends hold across both datasets.
\begin{figure}
    \centering
    \subfigure[\centering \textbf{DocVQA}]{
    \includegraphics[height=3cm,width=0.45\textwidth]{images/docvqa_question_stats.pdf}
        \label{fig:docvqa_dist}
    }
    \subfigure[\centering \textbf{PFL-DocVQA}]{
    \includegraphics[height=3cm,width=0.45\textwidth]{images/pfl_question_stats.pdf}
        \label{fig:pfl_dist}
    }
\caption{\textbf{The distribution of number per-document questions} from PFL and DocVQA dataset.}
\label{fig:questions_per_document}
\end{figure}


% \section{Document-level Membership Inference Attacks}
% We demonstrate the scheme of Document-level Membership Inference Attacks in Figure \ref{fig:teaser}.

% \centering

\includegraphics[width=0.85\textwidth]{images/teaser.pdf}
\caption{
IP-Composer enables compositional generation from a set of visual concepts. These are portrayed through a set of input images, along with a prompt describing the desired concept to be extracted from each.
}

\section{Baselines}
\label{sec:attack_baseline}
For the \textit{black-box} setting, we evaluate three MI attacks as baselines, which only requires generated text to infer the membership of the target document:

\textbf{Score-Threshold Attack $\textsc{(Score-TA)}$} 
assumes that training documents should achieve higher scores than non-training ones. This attack, adapted from \citet{yeom2018privacy}, evaluates the prediction $\hat{a}$ for each question $q$ using the utility function $\mathcal{U}$ and computes the average score $\bar{u}$. A document is then predicted as a member $\bar{u} \geq \kappa$, and non-member otherwise. The threshold $\kappa$ is set as the average value of $\bar{u}$ across $D_\text{test}$.

\textbf{Unsupervised Score-based Attack $\textsc{(Score-UA)}$} \citep{tito2024privacy}. This attack applies an unsupervised clustering algorithm over the set of average score $\bar{u}$ from test documents in $D_\text{test}$, documents within the cluster with higher average score are predicted as members.

\textbf{Unsupervised Score-based Attack - An Extension ($\textsc{Score-UA}_{\text{all}}$)}. This attack extends $\textsc{Score-UA}$ by considering multiple aggregation functions $\Phi_\text{all}$ to form the feature vector.

For the \textit{grey-box} setting, we consider two additional baselines which assumes access to token-level probabilities of the generated answers $a$ to compute the membership score of each document:

\textbf{Min-K\%}\citep{shi2023detecting} computes the average log probability of the lowest-K\% answer tokens as the membership score: $\text{Min-K\%} = \frac{1}{|\text{Min-K\%}(a)|}\Sigma_{a_i\in\text{Min-K}(a)} \log p(a_i| a_{<i})$. Intuitively, training documents are less likely to contain low-probability answer tokens, resulting in higher  scores.

\textbf{Min-K\%++}\citep{zhang2024min} also averages scores from the lowest-K\% probability tokens but assumes that tokens in the predicted answers for training documents have high probabilities or often form the mode of the conditional distribution. Thus, for each token, the score is computed as: $\text{Min-K\%++}(a_{<i}, a_i) = \frac{\log p(a_i| a_{<i}) - \mu_{a_{<t}}}{\sigma_{a_{<t}}}$ with $\mu_{a_{<t}}$ and $\sigma_{a_{<t}}$ are the expectation and standard deviation of $p(a_i| a_{<i})$ respectively.

We adapt these baselines to DocMIA by using an \textsc{AVG} aggregation function to combine scores across question-answer pairs within a document. We evaluated $K\in [0.6, 0.7, 0.8, 0.9, 1.0]$, which correspond to corresponds to 60\% to 100\% the length of the answer and reported the best result.

In the \textit{white-box} setting, where loss information is available, we consider three additional baselines:

\textbf{Loss-Threshold Attack ($\textsc{Loss-TA}$)} \citep{yeom2018privacy}
Similar to $\textsc{Score-TA}$, this attack computes the average loss $\bar{l}= \frac{1}{M}\Sigma^{M}_{i} \mathcal{L}(\mathcal{F}(x, q_i))$. A document is predicted as a member if $\bar{l} \leq \kappa$ and otherwise non-member, where $\kappa$ is selected as the average value of $\bar{l}$ across $D_\text{test}$.

\textbf{Unsupervised One-step Gradient Attack ($\textsc{Gradient-UA}$)} Inspired from \citet{nasr2019comprehensive}, this attack utilizes the average norm of the gradient of the loss $\nabla_{\theta}\mathcal{L}$ from a single optimization step. It also incorporates the average score $\bar{u}$, both aggregated with $\Phi_\text{all}$ as the features to perform clustering.

\textbf{Unsupervised Score+Loss Attack ($\textsc{ScoreLoss-UA}_{\text{all}}$)}
This attack extends $\textsc{ScoreUA}_{\text{all}}$, combining the average loss $\bar{l}$ with the average utility score $\bar{u}$, then aggregating with $\Phi_\text{all}$.

\section{Ablation Study}
\label{sec:calibration}
In this section, we provide a detailed analysis of the hyperparameter tuning process for \atk in the white-box setting, targeting all the considered models. Given the high computational cost due to the numerous factors involved, we focus on the key parameters that may potentially affect the attack performance. Our intuition behind this tuning process is that: achieving a reliable estimate of the distance $\Delta$ requires the optimization process to converge effectively, which in turn correlates with higher attack accuracy. Thus in all of our experiments, to increase the likelihood of convergence, we set the maximum number of optimization steps to $S = 200$. We fix the maximum number of questions per document $M$ to 10.

\textbf{Learning Rate $\alpha$}. We first study the effect of $\alpha$, which controls the speed of the optimization process in our attacks. This threshold $\tau$ is empirically set to be the average loss change observed when performing one optimization step after reaching the correct answer. Only the distance $\Delta$ and the number of steps $s$ are used as the features. For FL and FLLoRA attacks, we perform a hyperparameter search over a grid of learning rates, $\alpha \in \{10^{-4}, 0.001, 0.01, 0.1, 0.5, 1.0\}$, and $\alpha \in \{0.001, 0.01, 0.1, 0.5, 1.0, 5.0, 10.0, 20.0\}$ for the IG attacks. For FL and FLLoRA, we specifically tune the embedding projection layer, which projects the final hidden states into the vocabulary space, a common design choice across all the target models considered.

As shown in Figure \ref{fig:ablate_alpha}, setting a high learning rate can cause the optimization process to overshoot, while lower values lead to a more stable but slower convergence. We find that a learning rate of $\alpha = 10^{-3}$ consistently delivers the best attack performance across most of the settings.

\textbf{The layer to tune ${\textsc{L}}$}.
We now investigate the impact of layer selection on the performance of our FL and FLLoRA attacks. All target models in our study follow the transformer encoder-decoder architecture \citep{vaswani2017attention}, where each component consists of a stack of attention layers, and a shared embedding projection layer maps the hidden states to logit vectors for prediction. Given this common structure, we examine the effect of tuning similar layers across all models, with results for attack accuracy presented in Table \ref{tab:ablate_layer}.
\begin{figure}[t]
    \centering
    \subfigure[\centering \textbf{Learning Rate $\alpha$}]{
        \includegraphics[height=3cm,width=0.45\textwidth]{images/ablate_learning_rate.pdf}
        \label{fig:ablate_alpha}
    }
    \subfigure[\centering \textbf{Threshold $\tau$}]{
        \includegraphics[height=3cm,width=0.45\textwidth]{images/ablate_threshold.pdf}
        \label{fig:ablate_tau}
    }
\caption{\textbf{Ablation Study on Learning Rate $\alpha$ and Threshold $\tau$.} The best value for each model across all datasets is used as the hyperparameters in our black-box attacks.}
\label{fig:ablate_alpha_tau}
% \vspace{-0.2in}
\end{figure}

\begin{table}[t]
\begin{center}
\begin{small}
\small
\begin{tabular}{lccc}
\toprule
Layer & VT5(PFL) & Donut(DocVQA) & Pix2Struct-B(DocVQA)\\
\midrule 
Embedding Projection Layer & 67.0 & 71.33 & 68.66\\
Embedding Layer Norm & 65.33 & 76.0 & 64.67\\
\cmidrule{1-4}
Last Decoder Block FC1 & \textbf{68.33} & \textbf{78.0} & 68\\
Last Decoder Block FC2 & 68.17 & 77.33 & \textbf{68.83}\\
Last Decoder Block Layer Norm & 61.83 & 76.83 & 67.5\\
\cmidrule{1-4}
Random Decoder Block FC1 & 61.33 & 72.0 & 67.5\\
Random Decoder Block FC2 & 64.0 & 73.0 & 65.17\\
\bottomrule
\end{tabular}
\end{small}
\end{center}
\caption{\textbf{Effect of selected layer to tune} from each target model. Attack performances are reported in terms of Accuracy.}
\label{tab:ablate_layer}
\vspace{-0.2in}
\end{table}



Our findings reveal that \textit{layers closer to the final output exhibit higher privacy leakage} in terms of MI compared to (randomly selected) intermediate layers, likely due to receiving larger gradient updates. Specifically, fine-tuning the final fully connected layer alone leads to strong attack performance while also being more efficient in terms of the number of parameters that need to be optimized. This suggests that focusing on the last layers can achieve both high privacy leakage and computational efficiency in our MI attacks.

\textbf{Threshold $\tau$}.
With the optimizer $\textsc{OPT}$ and learning rate $\alpha$ fixed, the threshold $\tau$ emerges as the most critical hyperparameter that requires careful tuning for each attack. We experiment with a wide range of $\tau$ values, spanning from $10^{-6}$ to $10.0$, and select the optimal value based on attack performance, as demonstrated in Figure \ref{fig:ablate_tau}. This optimal $\tau$ is then applied consistently in all subsequent experiments. Careful selection of this threshold is crucial, as it directly influences the stability and success of the optimization process.

\begin{table}[t]
\begin{center}
\begin{small}
\small
\begin{tabular}{ccccccc}
\toprule
Model & $\alpha_{\textsc{FL}}$ & $\alpha_{\textsc{IG}}$ & $S$ & $L$ & $\tau_{\text{FL}}$ & $\tau_{\text{IG}}$\\
\midrule 
VT5 & \multirow{3}{*}{0.001} & 1.0 & \multirow{3}{*}{200} & \multirow{3}{*}{last FC layer} & $10^{-6}$ & $10^{-5}$\\
Donut &  & 0.001 &  &  & 1.0 & 5.0\\
Pix2Struct-B &  & 0.001 &  &  & $10^{-4}$ & $10^{-3}$\\
\bottomrule
\end{tabular}
\end{small}
\end{center}
\caption{\textbf{Best Hyperaremeters from our tuning process} with consistent performance across both PFL and DocVQA dataset.}
\label{tab:best_hps}
\end{table}


\begin{table}[t]
\begin{center}
\begin{small}
\begin{adjustbox}{width=1\textwidth}
\small
\begin{tabular}{ccccccc}
\toprule
\multirow{2}{*}{Model} & \multirow{2}{*}{Num. Params} & \multirow{2}{*}{Downstream Task} & \multicolumn{2}{c}{Data} & \multicolumn{2}{c}{Checkpoint} \\
\cmidrule(l){4-5}
\cmidrule(l){6-7}
& & & Pretrain & Finetune & Pretrain & Finetune \\
\midrule 
\multirow{2}{*}{VT5} & \multirow{2}{*}{250M} & \multirow{2}{*}{DocVQA} & \multirow{2}{*}{C4+IIT-CDIP} & PFL & \multicolumn{2}{c}{\multirow{2}{*}{https://benchmarks.elsa-ai.eu/?ch=2}} \\
 &  &  &  & DocVQA &  & \\
\midrule 

\multirow{2}{*}{Donut} & \multirow{2}{*}{200M} & \multirow{2}{*}{DocVQA} & \multirow{2}{*}{CDIP 11M + 0.5M synthesized Docs} & PFL &  \multicolumn{2}{c}{Ours}  \\
 &  &  &  & DocVQA & $\text{naver-clova-ix/donut-base}^{\dagger}$ & $\text{naver-clova-ix/donut-base-finetuned-docvqa}^{\dagger}$  \\
\midrule 
Pix2struct-B & 282M & \multirow{2}{*}{DocVQA} & \multirow{2}{*}{BooksCorpus + C4 Web HTML} & \multirow{2}{*}{DocVQA} & $\text{google/pix2struct-base}^{\dagger}$ & $\text{google/pix2struct-docvqa-base}^{\dagger}$ \\
Pix2struct-L & 1.33B &  &  &  & $\text{google/pix2struct-large}^{\dagger}$ & $\text{google/pix2struct-docvqa-large}^{\dagger}$ \\
\bottomrule
\end{tabular}
\end{adjustbox}
\end{small}
\end{center}
\caption{\textbf{Details of the public checkpoints} used as target models in this work. $\dagger$ denotes checkpoint from Hugging Face.}
\label{tab:public_checkpoint}
\vskip -0.1in
\end{table}



We summarize the set of tuned hyperparameters for our approach in Table \ref{tab:best_hps}.

\section{More on Attack Implementation}
\label{sec:attack_implementation}
\subsection{Target Model Training}

For all target models, whenever feasible, we utilize the public checkpoint fine-tuned on the considered private dataset from Hugging Face library and adhere to the data processing guidelines, such as document resolution, as recommended by the authors. We deliberately opt for public checkpoints for two reasons: (1) to make it consistent to further research in privacy attacks that use the same trained models, and (2) to minimizing the biases in model training that affect the final results, given the complexity of the original training process and our limited resources. Table \ref{tab:public_checkpoint} summarizes the details of the process from which public checkpoints for the target models considered in this work are obtained. This includes the datasets the models were pre-trained on, before by fine-tuning on target DocVQA datasets, along with the corresponding download URLs for these checkpoints.






\begin{table}[h]
    \centering
    \scriptsize
    \caption{\textbf{Training Parameters.} Detailed training parameters, including encoder layers, heads, model dimensions, and optimization setups. The \textbf{--} indicates a parameter not used. LaBraM and EEGPT are excluded from the table since the code structure is different.
    }
    \vspace{2mm}
    \label{tab:training_params}
    \resizebox{\textwidth}{!}{
    \begin{tabular}{@{}ll|c|c|c|c|c|c|c|c|c|c|c|c|c@{}}
    \toprule
    \multicolumn{2}{l|}{\textbf{\diagbox{\textbf{Methods}}{\textbf{Params}}}} 
    & \textit{backbone} & \textit{e\_layers} & \textit{n\_heads} & \textit{d\_model} & \textit{d\_ff} 
    & \textit{batch\_size} & \textit{train\_epochs} & \textit{optimizer} 
    & \textit{learning\_rate} & \textit{lr\_scheduler} & \textit{gradient\_clip}  & \textit{patience} & \textit{swa} \\
    \midrule


    \multicolumn{14}{c}{\textbf{Single-Dataset Supervised Learning}}  \\
    
    \midrule

    \textbf{TCN} & & TCN & 6 & \textbf{--} & \textbf{--} & \textbf{--} & 128 & 100 & AdamW & 1e-4 & CosineAnnealing & 4.0 & 15 & \checkmark \\
    \textbf{Transformer} & & Transformer & 6 & 8 & 128 & 256 & 128 & 100 & AdamW & 1e-4 & CosineAnnealing & 4.0 & 15 & \checkmark \\
    \textbf{Conformer} & & Conformer & 6 & 8 & 128 & 256 & 128 & 100 & AdamW & 1e-4 & CosineAnnealing & 4.0 & 15 & \checkmark \\
    \textbf{TimesNet} & & TimesNet & 2 & \textbf{--} & 32 & 64 & 64 & 100 & AdamW & 1e-4 & CosineAnnealing & 4.0 & 15 & \checkmark \\
    \textbf{Medformer} & & Medformer & 6 & 8 & 128 & 256 & 128 & 100 & AdamW & 1e-4 & CosineAnnealing & 4.0 & 15 & \checkmark \\
    \textbf{LEAD-Vanilla(Ours)} & & LEAD & 12 & 8 & 128 & 256 & 128 & 100 & AdamW & 1e-4 & CosineAnnealing & 4.0 & 15 & \checkmark \\
    \midrule



    \multicolumn{14}{c}{\textbf{Unified Supervised Learning}}  \\
    \midrule
    
    \textbf{LEAD-Sup(Ours)} & & LEAD & 12 & 8 & 128 & 256 & 128 & 100 & AdamW & 1e-4 & CosineAnnealing & 4.0 & 15 & \checkmark \\

    \midrule




    \multicolumn{14}{c}{\textbf{Self-Supervised Pre-training}}  \\
    \midrule

    \textbf{TS2Vec} & & Transformer & 20 & 12 & 128 & 256 & 512 & 50 & AdamW & 2e-4 & CosineAnnealing & 4.0 & \textbf{--} & \checkmark \\
    \textbf{BIOT} & & BIOT & 20 & 12 & 128 & 256 & 512 & 50 & AdamW & 2e-4 & CosineAnnealing & 4.0 & \textbf{--} & \checkmark \\
    \textbf{EEG2Rep} & & EEG2Rep & 20 & 12 & 128 & 256 & 512 & 50 & AdamW & 2e-4 & CosineAnnealing & 4.0 & \textbf{--} & \checkmark \\
    \textbf{LEAD-Base(Ours)} & & LEAD & 12 & 8 & 128 & 256 & 512 & 50 & AdamW & 2e-4 & CosineAnnealing & 4.0 & \textbf{--} & \checkmark \\
    \midrule


    \multicolumn{14}{c}{\textbf{Unified Fine-tuning}}  \\
    \midrule
    
    \textbf{TS2Vec} & & Transformer & 20 & 12 & 128 & 256 & 128 & 100 & AdamW & 1e-4 & CosineAnnealing & 4.0 & 15 & \checkmark \\
    \textbf{BIOT} & & BIOT & 20 & 12 & 128 & 256 & 128 & 100 & AdamW & 1e-4 & CosineAnnealing & 4.0 & 15 & \checkmark \\
    \textbf{EEG2Rep} & & EEG2Rep & 20 & 12 & 128 & 256 & 128 & 100 & AdamW & 1e-4 & CosineAnnealing & 4.0 & 15 & \checkmark \\
    \textbf{LEAD-Base(Ours)} & & LEAD & 12 & 8 & 128 & 256 & 128 & 100 & AdamW & 1e-4 & CosineAnnealing & 4.0 & 15 & \checkmark \\
    
    \bottomrule
    \end{tabular}
    }

\end{table}


If public checkpoints are unavailable, we fine-tune the selected model on the respective private dataset, using the pre-trained checkpoint as the initialization, with the training procedure outlined by the respective authors. To prevent overfitting, we perform early stopping based on validation performance, ensuring that all evaluated models generalize well to unseen data. We also use the pre-trained checkpoint to initialize the proxy model $\mathcal{F}_p$ to train it on $D_{\text{query}}$. We provide an overview of the training procedure for each target model, based on the respective papers. These procedures were adapted to fit our computational resources, as outlined in Table \ref{tab:training_params}.
\begin{table}[t]
\begin{center}
\begin{small}
\small
\begin{adjustbox}{width=0.6\textwidth}
\begin{tabular}{lccccc}
\toprule
Dataset & Model & Test Set & ACC & ANLS & Train-Test Gap \\
\midrule
\multirow{6}{*}{{PFL}} & \multirow{3}{*}{{VT5}} & Original &  81.4 & 90.17  & -\\
 &  & MIA & 82.74 & 90.91 & 11.44 \\
 &  & MIA-rephrased & 77.59 & 85.84 & -\\
\cmidrule{2-6}
& \multirow{3}{*}{{Donut}} & Original & 74.73 & 88.66 & -\\
 &  & MIA & 80.15 & 91.64 & 22.2 \\
 &  & MIA-rephrased & 70.46 & 80.96 & -\\
\midrule
\multirow{12}{*}{{DVQA}}& \multirow{3}{*}{{VT5}} & Original & 60.1 & 69.33 & -\\
 &  & MIA & 75.54 & 81.69 & 36.22 \\
 &  & MIA-rephrased & 73.57 & 79.89 & -\\
\cmidrule{2-6}
& \multirow{3}{*}{{Donut}} & Original & 59.26 & 66.91 & -\\
 &  & MIA & 78.55 & 83.42 & 39.78 \\
 &  & MIA-rephrased & 72.57 & 77.12 & -\\
\cmidrule{2-6}
& \multirow{3}{*}{{Pix2Struct-B}} & Original & 57.11 & 68.13 & -\\
 &  & MIA & 64.42 & 79.95 & 25.8 \\
 &  & MIA-rephrased & 63.81 & 74.06 & -\\
\cmidrule{2-6}
& \multirow{3}{*}{{Pix2Struct-L}} & Original & 64.53 & 74.12 & -\\
 &  & MIA & 73.91 & 82.71 & 22.11 \\
 &  & MIA-rephrased & 69.93 & 79.15 & -\\
\bottomrule
\end{tabular}
\end{adjustbox}
\end{small}
\end{center}
\caption{\textbf{DocVQA Performance of the target models on PFL and DocVQA dataset.} Train-Test Gap is computed as the different of DocVQA Accuracy between \textit{member/non-member} documents. $\textsc{MIA}$ denotes the attack evaluation set, which is a subset randomly sampled from the original train/test set, $\textsc{MIA}\text{-rephrased}$ is its variants with rephrased questions by LLM.}
\label{tab:docvqa_performance}
\end{table}


\subsection{Target Model Performance on DocVQA}
To ensure the utility of the target models for our experiments, we validated that the DocVQA performance of each model checkpoint closely matched the results reported in the respective papers. Table \ref{tab:docvqa_performance}  presents the target models' performance across both DocVQA datasets. We observe a clear train-test performance gap, particularly in smaller models, while the gap tends to narrow for more generalized models or with increased dataset size.

\subsection{Computation and Runtime} All attack methods are implemented using PyTorch and executed on an NVIDIA GeForce A40 GPU with 45 GB of memory. The maximum runtime for each attack does not exceed \textit{10 hours} per run, depending on the target model’s size and the preprocessing steps required for the data. This runtime reflects the efficiency of our approach, especially when compared to methods based on shadow training, which require retraining of large-scale models many times to be effective \citep{carlini2022membership}. Our results demonstrate that the proposed attacks are both efficient and scalable, making them practical for large-scale models in real-world applications.

\section{More on Attack Results}
\begin{figure}[t]
    \centering
    \begin{minipage}{0.48\textwidth}
        \centering
        \resizebox{0.8\linewidth}{!}{
            \includegraphics[width=1.0\linewidth]{images/whitebox_results.pdf}
        }
        \caption{\textbf{White-box Setting: Our proposed attacks consistently achieve high performance}, generally outperforming the considered baselines.}
        \label{fig:whitebox_results}
        \end{minipage}
        \hfill
    \begin{minipage}{0.51\textwidth}
        \centering
        \resizebox{\linewidth}{!}{
            \begin{tabular}{lccccc}
                \toprule
                 & \multicolumn{3}{c}{\textbf{DVQA}} & \multicolumn{2}{c}{\textbf{PFL}}  \\
                \cmidrule(l){2-4}
                \cmidrule(l){5-6}
                 & {VT5} & {Donut} & {Pix2Struct-B} & {VT5} & {Donut} \\
                \midrule
                $\textsc{Loss-TA}$ & \textbf{14.00} & 7.67 & 5.33 & 3.00 & \textbf{14.67} \\
                $\textsc{Gradient-UA}$ & 9.33 & 6.00 & 5.00 & 3.00 & 8.33 \\
                $\textsc{ScoreLoss-UA}_{\text{all}}$ & 4.67 & 8.67 & 6.67 & 4.00 & 6.33 \\
                \midrule
                Min-K\% & 10.67 & 1.33 & 5.33 & 5.67 & 0.00 \\
                Min-K\%++ & 7.00 & 9.33 & 10.33 & 8.00 & 2.00 \\
                \midrule
                FL & 5.67 & \textbf{10.67} & \textbf{11.00}& \textbf{8.67} & 7.00 \\
                FLLoRA & 11.33 & 5.33 & 6.33 & 3.33 & 10.00 \\
                IG & 5.67 & 8.00 & 10.33 & 2.33  & 11.00 \\
                \bottomrule
            \end{tabular}
        }
        \captionof{table}{\textbf{White-box Setting: TPR at 3\% FPR}. Comparison across all white-box methods, with the best-performing method for each metric highlighted in \textbf{bold}. We refer the readers to Appendix \ref{sec:tpr_at_fix_fpr} for the complete results.}
        \label{tab:tpr@fpr_main_whitebox}
        \end{minipage}
    \vskip -0.2in
\end{figure}




\label{sec:tpr_at_fix_fpr}
In this section, we evaluate our attacks using TPR@1\%FPR and TPR@3\%FPR, following standard practices in recent MIA literature. The results are summarized in Table~\ref{tab:tpr@fpr_whitebox} and ~\ref{tab:tpr@fpr_blackbox}. 

\begin{table}[t]
\begin{center}
\begin{small}
\begin{adjustbox}{width=0.96\textwidth}
\small
\begin{tabular}{lcccccccccc}
\toprule
 & \multicolumn{6}{c}{\textbf{DVQA}} & \multicolumn{4}{c}{\textbf{PFL}}  \\
\cmidrule(l){2-7}
\cmidrule(l){8-11}
 & \multicolumn{2}{c}{VT5} & \multicolumn{2}{c}{Donut} & \multicolumn{2}{c}{Pix2Struct-B} & \multicolumn{2}{c}{VT5} & \multicolumn{2}{c}{Donut} \\
\cmidrule(l){2-3}
\cmidrule(l){4-5}
\cmidrule(l){6-7}
\cmidrule(l){8-9}
\cmidrule(l){10-11}
 & 1\% & 3\% & 1\% & 3\% & 1\% & 3\% & 1\% & 3\% & 1\% & 3\% \\
\midrule
$\textsc{Loss-TA}$ & \textbf{7.67} & \textbf{14.00} & 0.67 & 7.67 & 2.33 & 5.33 & 0.67 & 3.00 & 1.67 & \textbf{14.67} \\
$\textsc{Gradient-UA}$ & 2.33 & 9.33 & 3.67 & 6.00 & 1.00 & 5.00 & 0.33 & 3.00 & 1.00 & 8.33 \\
$\textsc{ScoreLoss-UA}_{\text{all}}$ & 1.33 & 4.67 & 2.67 & 8.67 & 2.00 & 6.67 & 0.33 & 4.00 & 0.67 & 6.33 \\
Min-K\% & 2.67 & 10.67 & 0.33 & 1.33 & 0.33 & 5.33 & 1.67 & 5.67 & 0.00 & 0.00 \\
Min-K\%++ & 1.00 & 7.00 & \textbf{4.33} & 9.33 & 0.67 & 10.33 & 1.00 & 8.00 & 0.33 & 2.00 \\
\midrule
FL & 2.33 & 5.67 & 3.33 & \textbf{10.67} & \textbf{6.00} & \textbf{11.00} & \textbf{3.67} & \textbf{8.67} & 0.33 & 7.00 \\
FLLoRA & 3.33 & 11.33 & 2.67 & 5.33 & 3.67 & 6.33 & 1.33 & 3.33 & 0.33 & 10.00 \\
IG & 0.67 & 5.67 & 1.33 & 8.00 & 3.00 & 10.33 & 1.00 & 2.33 & \textbf{5.67} & 11.00 \\
\bottomrule
\end{tabular}
\end{adjustbox}
\end{small}
\end{center}
\vskip -0.1in
\caption{\textbf{White-box: TPR at fixed FPR}. Comparison across all white-box methods, with the best-performance highlighted in \textbf{bold}. 1\% and 3\% indicate TPR@1\%FPR and TPR@3\%FPR respectively.}
\label{tab:tpr@fpr_whitebox}
\end{table}


An interesting observation is the high performance of the \textsc{LOSS-TA} method for VT5 on DocVQA and Donut on PFL in Table~\ref{tab:tpr@fpr_whitebox}. This performance can be attributed to the clear separation in the loss distribution between member and non-member samples (Figure~\ref{fig:whitebox_loss}), which indicates overfitting behavior in these cases.

\section{More on Analysis}
\label{sec:more_analysis}
In this section, we provide a deeper analysis of the effectiveness of our proposed white-box and black-box attacks, highlighting their performance relative to the baseline approaches.
\begin{figure}[t]
    \centering
    \begin{minipage}[b]{0.66\textwidth}
        \centering
        \resizebox{0.999\linewidth}{!}{
            \includegraphics[height=5cm,width=1.05\textwidth]{images/whitebox_feature.pdf}
        }
        \caption{\textbf{Membership Features against three different target models on DocVQA Dataset.} \textit{Top}: The distribution of average \textbf{\textit{loss}} over all questions from all target documents on each target model. \textit{Bottom}: T-SNE visualization of the features used in our proposed attacks.}
        \label{fig:whitebox_loss}
    \end{minipage}
    \hfill
    \begin{minipage}[b]{0.3\textwidth}
        \centering
        \includegraphics[height=5cm]{images/whitebox_distance.pdf}
        \caption{\textbf{Distribution of Average \textit{Distance}} from member and non-member documents.}
        \label{fig:whitebox_distance}
    \end{minipage}
\label{fig:whitebox_feature}
\vspace{-0.2in}
\end{figure}


\subsection{Impact of Selected Features}
\label{sec:impact_feature}
As outlined in the main paper, we fix the set of selected features across all experiments. These features include the DocVQA score $u$, the optimization-based distance $\Delta$, and the number of optimization steps $s$, aggregated using the set of aggregation functions $\Phi_\text{all}=\{\textsc{avg}; \textsc{min}; \textsc{max}; \textsc{med}\}$ . We first evaluate the impact of individual features and their combinations on attack performance in the white-box DocMIA setting, using $\textsc{avg}$ as the aggregation function $\Phi$. The analysis employs the best hyperparameters identified during the tuning process described in Section \ref{sec:calibration}.

Table \ref{tab:impact_feature_pfl} and Table \ref{tab:impact_feature_docvqa} summarize the attack performance when individual features or their combinations are used. 
Additionally, Table \ref{tab:impact_feature_more} \textit{(Top)} compares the attack performance of our optimization-based features with the loss value $\ell$ and the gradient norm of the loss with respect to the model parameters $\theta$. Here, the loss value $\ell$ is computed uniformly across all target models over $K$ generation steps, given a (document, question, answer) example $(x,q,a)$ as:
\begin{equation}
    \ell =-\sum^{K}_{k=1} \log{p_{\theta} (a_k|a_{<k},x,q)}
\end{equation}
When used individually, our proposed optimization-based features outperform the DocVQA score and the loss in most cases. Our attack methods are particularly effective against target models like VT5 and Donut trained on PFL-DocVQA, which exhibit lower overfitting and small Train-Test gaps (as shown in Table\ref{tab:docvqa_performance}). These results highlight that our attacks provide more discriminative features than the commonly used MIA features.

When combined, our selected features achieve the best or near-best performance across all cases. Furthermore, extending aggregation functions from $\textsc{avg}$ to $\Phi_\text{all}$ adds notable improvements in attack effectiveness, as shown in Table \ref{tab:impact_feature_more} \textit{(Bottom)}. These results demonstrate that our proposed feature set is robust across different target models, making it a reliable choice for DocMIA.
\begin{table}[t]
    \centering
    \begin{minipage}{0.64\textwidth}
        \centering
        \resizebox{\linewidth}{!}{
            \begin{tabular}{cc}
                \begin{tabular}{cccc}
                    \toprule
                    \multicolumn{4}{c}{VT5} \\
                    \midrule
                    $\textsc{avg}(\textsc{nls})$ & $\textsc{avg}(\Delta)$ & $\textsc{avg}(s)$ & F1\\
                    \midrule 
                    \checkmark &  &  & 68.88\\
                     & \checkmark &  & 71.45\\
                     &  & \checkmark & 70.92\\
                     \midrule
                    \checkmark & \checkmark &  & 71.09\\
                    \checkmark &  & \checkmark & 71.11\\
                     & \checkmark & \checkmark & 71.22\\
                    \midrule
                    \checkmark & \checkmark & \checkmark & {\textbf{71.53}}\\
                    \bottomrule
                    \label{tab:impact_feature_pfl_vt5}
                \end{tabular}
                &
                \begin{tabular}{cccc}
                    \toprule
                    \multicolumn{4}{c}{Donut} \\
                    \midrule
                    $\textsc{avg}(\textsc{nls})$ & $\textsc{avg}(\Delta)$ & $\textsc{avg}(s)$ & F1\\
                    \midrule
                    \checkmark &  &  & 67.58\\
                     & \checkmark &  & 71.36\\
                     &  & \checkmark & 73.16\\
                    \midrule
                    \checkmark & \checkmark &  & 72.87\\
                    \checkmark &  & \checkmark & 73.67\\
                     & \checkmark & \checkmark & 73.86\\
                    \midrule
                    \checkmark & \checkmark & \checkmark & {\textbf{73.89}}\\
                    \bottomrule
                \label{tab:impact_feature_pfl_donut}
                \end{tabular}
            \end{tabular}
        }
        \caption{\textbf{Impact of Selected Features on PFL-DocVQA Models.}}
        \label{tab:impact_feature_pfl}
    \end{minipage}
    \hfill
    \begin{minipage}{0.35\textwidth}
        \centering
        \resizebox{\linewidth}{!}{
            \begin{tabular}{c}
                \begin{tabular}{cccccc}
                    \toprule
                     & \multicolumn{2}{c}{PFL} & \multicolumn{3}{c}{DVQA}\\
                    \cmidrule(l){2-3}
                    \cmidrule(l){4-6}
                     & VT5 & Donut & VT5 & Donut & Pix2Struct-B\\
                    \midrule 
                    $\textsc{avg}(\ell)$ & 67.53 & 67.80 & 73.43 & 56.79 & 69.97\\
                    $\textsc{avg}(||\nabla_{\theta}\mathcal{L}||_2)$ & 70.53 & 71.51 & 71.91 & 71.53 & 66.14\\
                    $\textsc{avg}(\Delta)$ & 71.45 & 71.36 & 72.86 & 57.34 & 70.57\\
                    $\textsc{avg}(s)$ & 70.92 & 73.16 & 74.34 & 60.32 & 69.00\\
                    \bottomrule
                    \\
                    \toprule
                     & \multicolumn{2}{c}{PFL} & \multicolumn{3}{c}{DVQA}\\
                    \cmidrule(l){2-3}
                    \cmidrule(l){4-6}
                     & VT5 & Donut & VT5 & Donut & Pix2Struct-B\\
                    \midrule
                    $\Phi=\textsc{avg}$ & 71.53 & 73.89 & 74.96 & 72.94 & 73.22\\
                    $\Phi=\Phi_{\text{all}}$ & 72.4\textcolor{red}{(+0.87)} & 77.96\textcolor{red}{(+4.07)} & 76.6\textcolor{red}{(+1.67)} & 82.18\textcolor{red}{(+9.24)} & 72.22\textcolor{blue}{(-1.0)}\\
                    \bottomrule							
                \end{tabular}
            \end{tabular}
        }
    \caption{\textbf{Comparisons in Attack Performance in terms of F1 Score}: \textit{(Top)} between our Optimization-based Features with the loss value $\ell$ and the gradient norm $||\nabla_{\theta}\mathcal{L}||_2$. \textit{(Bottom)} between $\textsc{avg}$ and $\Phi_{\text{all}}$ as the aggregation functions.}
    \label{tab:impact_feature_more}
    \end{minipage}
\end{table}

\begin{table}[t]
    \centering
    \begin{minipage}{0.32\textwidth}
        \centering
        \resizebox{\linewidth}{!}{%
            \begin{tabular}{cccc}
                \toprule
                \multicolumn{4}{c}{VT5} \\
                \midrule
                $\textsc{avg}(\textsc{nls})$ & $\textsc{avg}(\Delta)$ & $\textsc{avg}(s)$ & F1\\
                \midrule 
                \checkmark &  &  & 72.73\\
                 & \checkmark &  & 72.86\\
                 &  & \checkmark & 74.34\\
                 \midrule
                \checkmark & \checkmark &  & \textbf{75.81}\\
                \checkmark &  & \checkmark & 75.04\\
                 & \checkmark & \checkmark & 74.19\\
                \midrule
                \checkmark & \checkmark & \checkmark & 74.96\\
                \bottomrule
            \end{tabular}
        }
        \label{tab:impact_feature_docvqa_vt5}
    \end{minipage}
    \begin{minipage}{0.32\textwidth}
        \centering
        \resizebox{\linewidth}{!}{%
            \begin{tabular}{cccc}
                \toprule
                \multicolumn{4}{c}{Donut} \\
                \midrule
                $\textsc{avg}(\textsc{nls})$ & $\textsc{avg}(\Delta)$ & $\textsc{avg}(s)$ & F1\\
                \midrule 
                \checkmark &  &  & \textbf{76.88}\\
                 & \checkmark &  & 57.34\\
                 &  & \checkmark & 60.32\\
                \midrule
                \checkmark & \checkmark &  & 65.94\\
                \checkmark &  & \checkmark & 72.17\\
                 & \checkmark & \checkmark & 60.29\\
                \midrule
                \checkmark & \checkmark & \checkmark & 72.94\\
                \bottomrule
            \end{tabular}
        }
        \label{tab:impact_feature_docvqa_donut}
    \end{minipage}
    \begin{minipage}{0.32\textwidth}
        \centering
        \resizebox{\linewidth}{!}{%
            \begin{tabular}{cccccc}
                \toprule
                \multicolumn{4}{c}{Pix2Struct-B} \\
                \midrule
                $\textsc{avg}(\textsc{nls})$ & $\textsc{avg}(\Delta)$ & $\textsc{avg}(s)$ & F1\\
                \midrule 
                \checkmark &  &  & 72.60\\
                 & \checkmark &  & 70.57\\
                 &  & \checkmark & 69.00\\
                \midrule
                \checkmark & \checkmark &  & 73.20\\
                \checkmark &  & \checkmark & 72.87\\
                 & \checkmark & \checkmark & 70.17\\
                \midrule
                \checkmark & \checkmark & \checkmark & \textbf{73.22}\\
                \bottomrule
            \end{tabular}
        }
        \label{tab:impact_feature_docvqa_pix2struct}
    \end{minipage}
\caption{\textbf{Impact of Selected Features on DocVQA Target Models}. Only $\textsc{AVG}$ is used as the aggregation function $\Phi$. Attack performances are obtained with our \textsc{FL} method using the best hyperparameters.}
\label{tab:impact_feature_docvqa}
\end{table}
\subsection{Impact of the Training Questions Knowledge}
\label{sec:impact_question_knowledge}
So far, our document MI attacks against DocVQA models have assumed complete knowledge of the original training questions. We now relax this assumption and investigate how the lack of access to the exact training questions affects attack performance. In practice, an adversary may not have access to the exact training questions but can approximate them. For example, documents like invoices often follow standard layouts, and biases in human annotation may lead to predictable patterns in the types of questions asked during the creation of DocVQA datasets \citep{tito2024privacy,mathew2021docvqa}. It is important to note that the original training questions tend to be simple, natural questions designed to extract specific information from the document. Moreover, the type of question is inherently linked to the type of document on which the DocVQA model is trained. For instance, if the target model is trained on invoices, the natural type of question would focus on extracting essential details from the invoice, such as the “total amount”, framed in a clear and straightforward manner e.g., "What is the total?".
This makes it possible for an adversary to generate approximate versions of the training questions, simulating a more realistic attack setting.
\begin{table}[t]
\begin{center}
\begin{small}
\begin{adjustbox}{width=1\textwidth}
\small
\begin{tabular}{clcccccccccccc}
\toprule
 & \multicolumn{1}{c}{\multirow{2}{*}{\textbf{Target}}} & \multicolumn{8}{c}{\textbf{DVQA}} & \multicolumn{4}{c}{\textbf{PFL}} \\
\cmidrule(l){3-10}
\cmidrule(l){11-14}
 &  & \multicolumn{2}{c}{\textbf{VT5}} & \multicolumn{2}{c}{\textbf{Donut}} & \multicolumn{2}{c}{\textbf{Pix2Struct-B}} & \multicolumn{2}{c}{\textbf{Pix2Struct-L}} & \multicolumn{2}{c}{\textbf{VT5}} & \multicolumn{2}{c}{\textbf{Donut}} \\
\cmidrule(l){3-4}
\cmidrule(l){5-6}
\cmidrule(l){7-8}
\cmidrule(l){9-10}
\cmidrule(l){11-12}
\cmidrule(l){13-14}
Proxy &  & 1\% & 3\% & 1\% & 3\% & 1\% & 3\% & 1\% & 3\% & 1\% & 3\% & 1\% & 3\%  \\
\midrule
& {$\textsc{Score-TA}$} & 4.00 & 9.33 & 5.00 & 11.00 & \textbf{5.33} & 8.00 & 3.33 & \textbf{9.00} & 1.00 & 5.00 & 0.67 & 2.67 \\
& {$\textsc{Score-UA}$} & 3.67 & 7.67 & 4.33 & 15.67 & 4.00 & 6.33 & 4.33 & 6.67 & 0.67 & 3.33 & 0.33 & 3.33 \\
& {$\textsc{Score-UA}_{\text{all}}$} & 4.00 & 9.33 & 5.00 & 11.00 & 5.33 & 8.00 & 3.33 & 9.00 & 1.00 & 5.00 & 0.67 & 2.67 \\
\midrule
\multirow{3}{*}{VT5} & FL & 0.67 & \textbf{12.33} & \textbf{11.67} & \textbf{23.00} & 2.00 & \textbf{16.67} & 2.00 & 5.33 & 0.67 & 2.00 & \textbf{5.00} & \textbf{8.00} \\
& FLLoRA & \textbf{4.67} & {11.33} & 6.34 & 16.33 & 2.33 & 9.33 & 1.00 & 4.67 & 2.00 & 3.33 & 0.00 & 2.00 \\
& IG & 1.00 & 8.33 & 2.00 & 7.00 & 4.67 & 7.67 & 2.33 & 7.00 & 0.33 & 3.67 & 1.33 & 6.67 \\
\cmidrule(l){1-14}
\multirow{3}{*}{Donut} & FL & 0.33 & 6.33 & 0.33 & 4.00 & 1.33 & 4.67 & 3.00 & 7.33 & 0.33 & 1.33 & 1.33 & 4.00 \\
& FLLoRA & 1.00 & 6.33 & 1.67 & 5.00 & 2.33 & 6.33 & 3.00 & 8.00 & 0.00 & 5.33 & 2.00 & 5.33 \\
& IG & 1.67 & 5.00 & 0.67 & 11.00 & 3.67 & 9.33 & \textbf{4.67} & 6.33 & \textbf{2.67} & \textbf{6.33} & 1.67 & 4.33 \\
\bottomrule
\end{tabular}
\end{adjustbox}
\end{small}
\end{center}
\vskip -0.1in
\caption{\textbf{Black-box: TPR at fixed FPR}. Comparison across all black-box methods, with the best-performing method highlighted in \textbf{bold}. 1\% and 3\% indicate TPR@1\%FPR and TPR@3\%FPR respectively.}
\label{tab:tpr@fpr_blackbox}
\end{table}

\begin{table}[h]
\vskip 0.15in
\begin{center}
\begin{small}
\begin{adjustbox}{width=1\textwidth}
\small
\begin{tabular}{clcccccccccc}
\toprule
\multicolumn{2}{c}{\multirow{2}{*}{Model}} & \multicolumn{2}{c}{$\textsc{Score-TA}$} & \multicolumn{2}{c}{$\textsc{Score-UA}_{\text{all}}$} & \multicolumn{2}{c}{$\textsc{Loss-TA}$} & \multicolumn{2}{c}{$\textsc{ScoreLoss-UA}_{\text{all}}$} & \multicolumn{2}{c}{$\textsc{Ours (FL)}$}\\
\cmidrule(l){3-4}
\cmidrule(l){5-6}
\cmidrule(l){7-8}
\cmidrule(l){9-10}
\cmidrule(l){11-12}
&                                 & ACC & F1 & ACC & F1 & ACC & F1 & ACC & F1 & ACC & F1 \\
\midrule 
\multirow{2}{*}{\rotatebox[origin=c]{90}{\tiny PFL}} & VT5 & \cellcolor[HTML]{C0C0C0}{$60.67$} & \cellcolor[HTML]{C0C0C0}{$64.13$} & \cellcolor[HTML]{C0C0C0}{$55.83_{0.0}$} & \cellcolor[HTML]{C0C0C0}{$46.89_{0.0}$} & $54.50$ & $59.19$ & $55.83_{0.0}$ & $46.89_{0.0}$ & $\textbf{\textcolor{red}{64.00}}_{\textbf{\textcolor{red}{0.0}}}$ & ${\textbf{\textcolor{red}{69.14}}}_{\textbf{\textcolor{red}{0.0}}}$\\

& {Donut} & \cellcolor[HTML]{C0C0C0}{$69.17$} & \cellcolor[HTML]{C0C0C0}{$69.72$} & \cellcolor[HTML]{C0C0C0}{$59.33_{0.0}$} & \cellcolor[HTML]{C0C0C0}{$51.59_{0.0}$} & $68.50$ & $66.67$ & $59.17_{0.0}$ & $51.49_{0.0}$ & $\textbf{\textcolor{red}{71.13}}_{\textbf{\textcolor{red}{0.08}}}$ & $\textbf{\textcolor{red}{72.07}}_{\textbf{\textcolor{red}{0.0}}}$\\
\midrule

\multirow{2}{*}{\rotatebox[origin=c]{90}{\tiny DVQA}} & VT5 & \cellcolor[HTML]{C0C0C0}{$73.67$} & \cellcolor[HTML]{C0C0C0}{$75.01$} & \cellcolor[HTML]{C0C0C0}{$74.83_{0.0}$} & \cellcolor[HTML]{C0C0C0}{$74.36_{0.0}$} & $71.67$ & $74.06$ & $75.17_{0.0}$ & $74.96_{0.0}$ & $\textbf{\textcolor{red}{74.83}}_{\textbf{\textcolor{red}{0.0}}}$ & $\textbf{\textcolor{red}{75.68}}_{\textbf{\textcolor{red}{0.0}}}$\\
& {Donut} & \cellcolor[HTML]{C0C0C0}{\textbf{\textcolor{red}{69.17}}} & \cellcolor[HTML]{C0C0C0}{$\textbf{\textcolor{red}{71.23}}$} & \cellcolor[HTML]{C0C0C0}{$65.17_{0.0}$} & \cellcolor[HTML]{C0C0C0}{$62.21_{0.0}$} & $52.33$ & $53.57$ & $65.17_{0.0}$ & $62.21_{0.0}$ & $67.67_{0.0}$ & $68.51_{0.0}$\\
\bottomrule
\end{tabular}
\end{adjustbox}
\end{small}
\end{center}
\vskip -0.1in
\caption{\textbf{Results with Rephrased Questions.} \textcolor{gray}{\textbf{Gray}} color indicate attacks conducted in the black-box setting. All results are reported based on five random seeds. The methods with the best \textit{average} performance across the two metrics are highlighted in \textbf{\textcolor{red}{bold}}.}
\label{tab:rephrased_question_results}
\end{table}


To explore this scenario, we conduct experiments where we paraphrase the original training questions using Mistral~\citep{jiang2023mistral}, and use these rephrased questions as inputs for the MI attacks. As illustrated in Table \ref{tab:rephrased_question_results}, the performance of all MI attacks declines when rephrased questions are used, mirroring the drop in DocVQA model performance (Table \ref{tab:docvqa_performance}), which is expected due to the increased uncertainty introduced by question rephrasing.

Among the baselines, the $\textsc{SCORE-TA}$ attack proves particularly to be robust, especially against models trained on DocVQA, which show a higher degree of overfitting. In contrast, attacks incorporating loss-based signals introduce additional noise due to uncertainty, leading to a noticeable drop in performance.

Despite the rephrasing, our attacks remain effective, maintaining performance levels comparable to those observed with the original questions, especially against the two PFL models, which demonstrate a lower degree of overfitting.

We also evaluate our proposed attacks against other methods in this setting, focusing on TPR at 1\% and 3\% FPR, with the results summarized in Table \ref{tab:tpr@fpr_rephrased_question_whitebox} and \ref{tab:tpr@fpr_rephrased_question_blackbox}.

\subsection{The resulting Proxy Model}
\label{sec:impact_proxy_model}
The purpose of training the Proxy Model on $D_{\text{query}}$, with labels generated by the black-box model, is to mimic the prediction patterns of the black-box model. The expectation is that the proxy model can capture internal decision-making patterns by following the black-box's prediction strategies. Instead of optimizing for ground-truth labels, we train the proxy to maximize the likelihood of the generated labels. The training process concludes when the proxy achieves near-zero training loss, at which point the final checkpoint is used for the attack.
\begin{figure}[h]
    \centering
    \subfigure[\centering \textbf{Training curve}]{
        \includegraphics[height=2.5cm,width=0.45\textwidth]{images/proxy_loss_curve.pdf}
        \label{fig:proxy_model_train}
    }
    \subfigure[\centering \textbf{Distribution over distance}]{
        \includegraphics[height=2.5cm,width=0.45\textwidth]{images/proxy_distance_distribution.pdf}
        \label{fig:proxy_model_distance_distribution}
    }
\caption{\textbf{The resulting Proxy Model} against Pix2Struct-B in the black-box setting. (\textit{a}) The attack accuracy improves quickly once the loss reaches near zero. (\textit{b}) The optimization distance values between member and non-member documents exhibit a separation similar to that seen in the white-box setting.}
\label{fig:proxy_model}
\end{figure}

As illustrated in Figure \ref{fig:proxy_model_train}, the attack performance quickly improves as training progresses. The model overfits quickly, with attack performance reaching its peak early—after just a quarter of the training process—demonstrating the efficiency of our approach. This suggests that \textit{once the proxy model converges, it has effectively captured informative membership signals from the black-box model}, making it ready for the attack. Moreover, we compare the distribution of optimization distances between the proxy model and the same model in the white-box setting, as shown in Figure \ref{fig:proxy_model_distance_distribution}. The results show a similar degree of separation between the two clusters in both cases, indicating the proxy model's effectiveness in approximating the black-box model's behavior to a certain extent.

\subsection{Attack Performance against Minimal-Training Documents}
DocVQA models typically process each question-answer pair independently, resulting in multiple exposures of each document during training. This increases the likelihood of being memorized by the model, making such documents more vulnerable to MIAs. Intuitively, documents associated with fewer training questions should be less exposed and therefore be less vulnerable.

\begin{table}[h]
\begin{center}
\begin{small}
\begin{adjustbox}{width=1\textwidth}
\small
\begin{tabular}{clcccccccc}
\toprule 							
& & \multicolumn{4}{c}{$\textsc{FL}$} & \multicolumn{4}{c}{$\textsc{IG}$}\\
\cmidrule(l){3-6}
\cmidrule(l){7-10}
\multirow{3}{*}{\rotatebox[origin=c]{90}{\textbf{PFL}}} & Model & $m=1$(1) & $m=2$(1) & $m=3$(85) & $\textsc{ALL}$(300) & $m=1$(1) & $m=2$(1) & $m=3$(85) & $\textsc{ALL}$(300)\\
\cmidrule{2-10}
& VT5 & 0 & 0 & 83.53 & 87.67 & 100 & 100 & 85.88 & 86.33\\
& Donut & 100 & 100 & 100 & 97.67 & 100 & 100 & 97.65 & 97\\
\midrule	
\multirow{4}{*}{\rotatebox[origin=c]{90}{\textbf{DVQA}}} & Model & $m=1$(51) & $m=2$(60) & $m=3$(52) & $\textsc{ALL}$(300) & $m=1$(51) & $m=2$(60) & $m=3$(52) & $\textsc{ALL}$(300)\\

\cmidrule{2-10}
& VT5 & 86.27 & 71.67 & 84.62 & 77.00 & 90.2 & 85 & 86.54 & 80.67\\
& Donut & 88.24 & 73.33 & 76.92 & 77.33 & 56.86 & 68.33 & 55.77 & 61.33\\
& Pix2Struct-B & 90.2 & 93.33 & 90.38 & 87 & 88.24 & 88.33 & 76.92 & 73\\		
\bottomrule
\end{tabular}
\end{adjustbox}
\end{small}
\end{center}
\caption{\textbf{Membership Prediction Accuracy on \textit{Member} Documents with minimal repetition.} $m$ denotes the subset of testing documents with $m$ \textit{training} questions, with subset sizes shown in parentheses. Compared to the performance measured on the entire member set (denoted as $\textsc{ALL}$), our attacks are still robust against documents with the low risk of memorization.}
\label{tab:min_repeat_results}
\end{table}


To evaluate this, we measure the accuracy of membership predictions from our attacks on a subset of \textit{member} documents in $D_{\text{test}}$ associated with only a few training questions. These documents represent a minimal memorization risk, posing a more challenging evaluation scenario. Results in Table \ref{tab:min_repeat_results} show that our attacks remain effective on these subsets, achieving high accuracy even for documents $m=1$ training question. This demonstrates the robustness of our attacks under conditions of minimal repetition.

\section{Defenses}

To mitigate the privacy vulnerabilities associated with membership inference attacks in Document Visual Question Answering (DocVQA) systems, we can employ Differential Privacy (DP) techniques~\citep{dwork2014algorithmic}, specifically through the use of differentially private stochastic gradient descent (DP-SGD) introduce by \citet{abadi2016deep}. DP is a robust framework that ensures an individual's data contribution cannot be inferred, even when an adversary has access to the model's outputs. DP-SGD achieves this by adding calibrated noise to the model's gradients during training, thus providing strong theoretical privacy guarantees. However, this approach is not without its drawbacks; the necessity of noise injection can adversely affect the utility of the trained model, leading to reduced performance in answering queries accurately. Alternatively, we can consider ad-hoc solutions such as limiting the number of queries to one question per document in black-box setting, which would inherently reduce the model's usability and flexibility in practical applications. While these measures can enhance privacy, they also necessitate careful consideration of the balance between privacy protection and the functionality of DocVQA systems.

To evaluate the robustness of our proposed membership inference attacks against Differential Privacy (DP), we implemented the well-known DP-SGD algorithm. We considered five privacy budget $\varepsilon \in \{8, 32\}$, with corresponding noise multiplier $\sigma \in \{0.5767822266, 0.3824234009\}$, respectively. 
% 1.279296875, 
The composition of the privacy budget over multiple iterations was calculated using Rényi Differential Privacy (RDP). We then converted the RDP guarantees into the standard $(\varepsilon,\delta)$-DP notion following the conversion theorem from \citep{balle2020hypothesis}.

\begin{table}[t]
    \centering
    \begin{minipage}{0.56\textwidth}
        \centering
        \resizebox{\linewidth}{!}{
            \begin{tabular}{lcccccccc}
            \toprule
             & \multicolumn{4}{c}{\textbf{DVQA}} & \multicolumn{4}{c}{\textbf{PFL}}  \\
            \cmidrule(l){2-5}
            \cmidrule(l){6-9}
             & \multicolumn{2}{c}{VT5} & \multicolumn{2}{c}{Donut} & \multicolumn{2}{c}{VT5} & \multicolumn{2}{c}{Donut} \\
            \cmidrule(l){2-3}
            \cmidrule(l){4-5}
            \cmidrule(l){6-7}
            \cmidrule(l){8-9}
             & 1\% & 3\% & 1\% & 3\% & 1\% & 3\% & 1\% & 3\% \\
            \midrule
            Min-K\% & 3.00 & 4.33 & 0.33 & 1.00 & \textbf{6.33} & \textbf{20.33} & 2.00 & 2.33 \\
            Min-K\%++ & 3.00 & 4.67 & 0.00 & 2.67 & 6.33 & 10.00 & 0.00 & 7.00 \\
            \midrule
            FL & 0.67 & 5.00 & \textbf{3.33} & \textbf{8.00} & 3.67 & 17.33 & 3.00 & 4.67 \\
            FLLoRA & \textbf{5.00} & \textbf{9.33} & 0.67 & 3.67 & 5.00 & 9.33 & \textbf{4.33} & \textbf{10.00} \\
            IG & 5.33 & 8.00 & 1.00 & 5.00 & 5.33 & 8.00 & 1.67 & 10.00 \\
            \bottomrule
            \end{tabular}
        }
        \caption{\textbf{White-box Results: TPR at 1\% and 3\% FPR with Rephrased Questions}. Comparison to \textit{white-box} methods: Min-K\% and Min-K\%++ methods, with the best method in \textbf{bold}.}
        \label{tab:tpr@fpr_rephrased_question_whitebox}
    \end{minipage}
    \hfill
    \begin{minipage}{0.43\textwidth}
        \centering
        \resizebox{\linewidth}{!}{
            \begin{tabular}{lcccccc}
             \toprule
             & \multicolumn{2}{c}{\textbf{PFL}} & \multicolumn{4}{c}{\textbf{DVQA}} \\
             \cmidrule(l){2-3}
             \cmidrule(l){4-7}
             & \multicolumn{2}{c}{VT5} & \multicolumn{2}{c}{Donut} & \multicolumn{2}{c}{Pix2Struct-B} \\
             \cmidrule(l){2-3}
             \cmidrule(l){4-5}
             \cmidrule(l){6-7}
             & 1\% & 3\% & 1\% & 3\% & 1\% & 3\% \\
            \midrule
            $\textsc{Score-TA}$& 0.33 & 2.67 & \textbf{3.33} & 9.67 & 3.00 & 8.67 \\
            $\textsc{Score-UA}_{\text{all}}$& 0.33 & 2.67 & 2.33 & 9.33 & \textbf{4.67} & 8.67 \\
            \midrule
            FL & 0.33 & 1.33 & 0.33 & 4.00 & 1.33 & 4.67 \\
            FLLoRA & 1.00 & 5.33 & 1.67 & 5.00 & 2.33 & 6.33 \\
            IG & \textbf{2.67} & \textbf{6.33} & 1.67 & \textbf{11.00} & 3.67 & \textbf{9.33} \\
            \bottomrule
            \end{tabular}
        }
    \caption{\textbf{Black-box Results: TPR at 1\% and 3\% FPR with Rephrased Questions}. Donut is used as The Proxy Model.}
    \label{tab:tpr@fpr_rephrased_question_blackbox}    
    \end{minipage}
    \vskip -0.2in
\end{table}





We trained the Donut model on the DocVQA dataset with DP-SGD to provide theoretical privacy guarantees for individual training documents. Due to resource constraints, we resized document resolution to a smaller size (1280, 960) compared to (2560, 1920) in the public checkpoint  provided by the original authors, which slightly reduced the model's DocVQA performance. For additional details on the effects of document resolution, we refer readers to the original model's paper\citep{Kim22Donut}. The model was trained using the Adam optimizer with a learning rate of $1e-4$, for 10 epochs, and with a batch size of 16. DocVQA performance was evaluated using the Average Normalized Levenshtein Similarity (ANLS) metric.

\begin{table}[h]
\begin{center}
\begin{small}
\begin{adjustbox}{width=1\textwidth}
\small
\begin{tabular}{lcccccccccccc}
\toprule
& \multicolumn{3}{c}{$\varepsilon=8$} & \multicolumn{3}{c}{$\varepsilon=32$} & \multicolumn{3}{c}{$\varepsilon=\infty$} \\
\cmidrule(l){2-4}
\cmidrule(l){5-7}
\cmidrule(l){8-10}
 & ANLS & F1 & TPR@3\%FPR & ANLS & F1 & TPR@3\%FPR & ANLS & F1 & TPR@3\%FPR \\
\midrule
FL & \multirow{3}{*}{19.16} & 55.09 & 2.33 & \multirow{3}{*}{21.81} & 58.84 & 4.33 & \multirow{3}{*}{50.12} & 73.81 & 7.33 \\
FLLoRA &  & 54.94 & 2.00 &  & 58.94 & 3.67 &  & 73.81 & 7.33 \\
IG &  & 56.29 & 1.67 &  & 59.35 & 5.00 &  & 73.52 & 8.67 \\
\bottomrule
\end{tabular}
\end{adjustbox}
\end{small}
\end{center}
\vskip -0.1in
\caption{\textbf{DocMIA Results for Donut trained with DP-SGD on DocVQA dataset}. We report the attack performance of our FL method in terms of F1 score and TPR3\%FPR.}
\label{tab:dp_whitebox}
\end{table}


Table \ref{tab:dp_whitebox} summarizes the results. As expected, introducing DP into model training significantly reduces the attack performance, for example from 73.81\% F1 score with non-DP model to 55.09\% at $\varepsilon = 8$, but this comes at the cost of substantial utility degradation, with the DP model achieving less than half of the performance of the non-DP model, 21.81 of ANLS at $\varepsilon = 8$ compared to 50.12 of ANLS from non-DP checkpoint.
For higher privacy budgets ($\varepsilon = 32$), our attacks demonstrate improved effectiveness, achieving notable gains, +3.75 in F1 and +2 in TPR3\%FPR scores compared to $\varepsilon = 8$, as the model becomes less privacy-constrained.

\end{document}
