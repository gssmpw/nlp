
\documentclass{article} % For LaTeX2e
\PassOptionsToPackage{table,xcdraw}{xcolor} 
\RequirePackage{xcolor}
\usepackage{iclr2025_conference,times}

% Optional math commands from https://github.com/goodfeli/dlbook_notation.
%%%%% NEW MATH DEFINITIONS %%%%%

\usepackage{amsmath,amsfonts,bm}
\usepackage{derivative}
% Mark sections of captions for referring to divisions of figures
\newcommand{\figleft}{{\em (Left)}}
\newcommand{\figcenter}{{\em (Center)}}
\newcommand{\figright}{{\em (Right)}}
\newcommand{\figtop}{{\em (Top)}}
\newcommand{\figbottom}{{\em (Bottom)}}
\newcommand{\captiona}{{\em (a)}}
\newcommand{\captionb}{{\em (b)}}
\newcommand{\captionc}{{\em (c)}}
\newcommand{\captiond}{{\em (d)}}

% Highlight a newly defined term
\newcommand{\newterm}[1]{{\bf #1}}

% Derivative d 
\newcommand{\deriv}{{\mathrm{d}}}

% Figure reference, lower-case.
\def\figref#1{figure~\ref{#1}}
% Figure reference, capital. For start of sentence
\def\Figref#1{Figure~\ref{#1}}
\def\twofigref#1#2{figures \ref{#1} and \ref{#2}}
\def\quadfigref#1#2#3#4{figures \ref{#1}, \ref{#2}, \ref{#3} and \ref{#4}}
% Section reference, lower-case.
\def\secref#1{section~\ref{#1}}
% Section reference, capital.
\def\Secref#1{Section~\ref{#1}}
% Reference to two sections.
\def\twosecrefs#1#2{sections \ref{#1} and \ref{#2}}
% Reference to three sections.
\def\secrefs#1#2#3{sections \ref{#1}, \ref{#2} and \ref{#3}}
% Reference to an equation, lower-case.
\def\eqref#1{equation~\ref{#1}}
% Reference to an equation, upper case
\def\Eqref#1{Equation~\ref{#1}}
% A raw reference to an equation---avoid using if possible
\def\plaineqref#1{\ref{#1}}
% Reference to a chapter, lower-case.
\def\chapref#1{chapter~\ref{#1}}
% Reference to an equation, upper case.
\def\Chapref#1{Chapter~\ref{#1}}
% Reference to a range of chapters
\def\rangechapref#1#2{chapters\ref{#1}--\ref{#2}}
% Reference to an algorithm, lower-case.
\def\algref#1{algorithm~\ref{#1}}
% Reference to an algorithm, upper case.
\def\Algref#1{Algorithm~\ref{#1}}
\def\twoalgref#1#2{algorithms \ref{#1} and \ref{#2}}
\def\Twoalgref#1#2{Algorithms \ref{#1} and \ref{#2}}
% Reference to a part, lower case
\def\partref#1{part~\ref{#1}}
% Reference to a part, upper case
\def\Partref#1{Part~\ref{#1}}
\def\twopartref#1#2{parts \ref{#1} and \ref{#2}}

\def\ceil#1{\lceil #1 \rceil}
\def\floor#1{\lfloor #1 \rfloor}
\def\1{\bm{1}}
\newcommand{\train}{\mathcal{D}}
\newcommand{\valid}{\mathcal{D_{\mathrm{valid}}}}
\newcommand{\test}{\mathcal{D_{\mathrm{test}}}}

\def\eps{{\epsilon}}


% Random variables
\def\reta{{\textnormal{$\eta$}}}
\def\ra{{\textnormal{a}}}
\def\rb{{\textnormal{b}}}
\def\rc{{\textnormal{c}}}
\def\rd{{\textnormal{d}}}
\def\re{{\textnormal{e}}}
\def\rf{{\textnormal{f}}}
\def\rg{{\textnormal{g}}}
\def\rh{{\textnormal{h}}}
\def\ri{{\textnormal{i}}}
\def\rj{{\textnormal{j}}}
\def\rk{{\textnormal{k}}}
\def\rl{{\textnormal{l}}}
% rm is already a command, just don't name any random variables m
\def\rn{{\textnormal{n}}}
\def\ro{{\textnormal{o}}}
\def\rp{{\textnormal{p}}}
\def\rq{{\textnormal{q}}}
\def\rr{{\textnormal{r}}}
\def\rs{{\textnormal{s}}}
\def\rt{{\textnormal{t}}}
\def\ru{{\textnormal{u}}}
\def\rv{{\textnormal{v}}}
\def\rw{{\textnormal{w}}}
\def\rx{{\textnormal{x}}}
\def\ry{{\textnormal{y}}}
\def\rz{{\textnormal{z}}}

% Random vectors
\def\rvepsilon{{\mathbf{\epsilon}}}
\def\rvphi{{\mathbf{\phi}}}
\def\rvtheta{{\mathbf{\theta}}}
\def\rva{{\mathbf{a}}}
\def\rvb{{\mathbf{b}}}
\def\rvc{{\mathbf{c}}}
\def\rvd{{\mathbf{d}}}
\def\rve{{\mathbf{e}}}
\def\rvf{{\mathbf{f}}}
\def\rvg{{\mathbf{g}}}
\def\rvh{{\mathbf{h}}}
\def\rvu{{\mathbf{i}}}
\def\rvj{{\mathbf{j}}}
\def\rvk{{\mathbf{k}}}
\def\rvl{{\mathbf{l}}}
\def\rvm{{\mathbf{m}}}
\def\rvn{{\mathbf{n}}}
\def\rvo{{\mathbf{o}}}
\def\rvp{{\mathbf{p}}}
\def\rvq{{\mathbf{q}}}
\def\rvr{{\mathbf{r}}}
\def\rvs{{\mathbf{s}}}
\def\rvt{{\mathbf{t}}}
\def\rvu{{\mathbf{u}}}
\def\rvv{{\mathbf{v}}}
\def\rvw{{\mathbf{w}}}
\def\rvx{{\mathbf{x}}}
\def\rvy{{\mathbf{y}}}
\def\rvz{{\mathbf{z}}}

% Elements of random vectors
\def\erva{{\textnormal{a}}}
\def\ervb{{\textnormal{b}}}
\def\ervc{{\textnormal{c}}}
\def\ervd{{\textnormal{d}}}
\def\erve{{\textnormal{e}}}
\def\ervf{{\textnormal{f}}}
\def\ervg{{\textnormal{g}}}
\def\ervh{{\textnormal{h}}}
\def\ervi{{\textnormal{i}}}
\def\ervj{{\textnormal{j}}}
\def\ervk{{\textnormal{k}}}
\def\ervl{{\textnormal{l}}}
\def\ervm{{\textnormal{m}}}
\def\ervn{{\textnormal{n}}}
\def\ervo{{\textnormal{o}}}
\def\ervp{{\textnormal{p}}}
\def\ervq{{\textnormal{q}}}
\def\ervr{{\textnormal{r}}}
\def\ervs{{\textnormal{s}}}
\def\ervt{{\textnormal{t}}}
\def\ervu{{\textnormal{u}}}
\def\ervv{{\textnormal{v}}}
\def\ervw{{\textnormal{w}}}
\def\ervx{{\textnormal{x}}}
\def\ervy{{\textnormal{y}}}
\def\ervz{{\textnormal{z}}}

% Random matrices
\def\rmA{{\mathbf{A}}}
\def\rmB{{\mathbf{B}}}
\def\rmC{{\mathbf{C}}}
\def\rmD{{\mathbf{D}}}
\def\rmE{{\mathbf{E}}}
\def\rmF{{\mathbf{F}}}
\def\rmG{{\mathbf{G}}}
\def\rmH{{\mathbf{H}}}
\def\rmI{{\mathbf{I}}}
\def\rmJ{{\mathbf{J}}}
\def\rmK{{\mathbf{K}}}
\def\rmL{{\mathbf{L}}}
\def\rmM{{\mathbf{M}}}
\def\rmN{{\mathbf{N}}}
\def\rmO{{\mathbf{O}}}
\def\rmP{{\mathbf{P}}}
\def\rmQ{{\mathbf{Q}}}
\def\rmR{{\mathbf{R}}}
\def\rmS{{\mathbf{S}}}
\def\rmT{{\mathbf{T}}}
\def\rmU{{\mathbf{U}}}
\def\rmV{{\mathbf{V}}}
\def\rmW{{\mathbf{W}}}
\def\rmX{{\mathbf{X}}}
\def\rmY{{\mathbf{Y}}}
\def\rmZ{{\mathbf{Z}}}

% Elements of random matrices
\def\ermA{{\textnormal{A}}}
\def\ermB{{\textnormal{B}}}
\def\ermC{{\textnormal{C}}}
\def\ermD{{\textnormal{D}}}
\def\ermE{{\textnormal{E}}}
\def\ermF{{\textnormal{F}}}
\def\ermG{{\textnormal{G}}}
\def\ermH{{\textnormal{H}}}
\def\ermI{{\textnormal{I}}}
\def\ermJ{{\textnormal{J}}}
\def\ermK{{\textnormal{K}}}
\def\ermL{{\textnormal{L}}}
\def\ermM{{\textnormal{M}}}
\def\ermN{{\textnormal{N}}}
\def\ermO{{\textnormal{O}}}
\def\ermP{{\textnormal{P}}}
\def\ermQ{{\textnormal{Q}}}
\def\ermR{{\textnormal{R}}}
\def\ermS{{\textnormal{S}}}
\def\ermT{{\textnormal{T}}}
\def\ermU{{\textnormal{U}}}
\def\ermV{{\textnormal{V}}}
\def\ermW{{\textnormal{W}}}
\def\ermX{{\textnormal{X}}}
\def\ermY{{\textnormal{Y}}}
\def\ermZ{{\textnormal{Z}}}

% Vectors
\def\vzero{{\bm{0}}}
\def\vone{{\bm{1}}}
\def\vmu{{\bm{\mu}}}
\def\vtheta{{\bm{\theta}}}
\def\vphi{{\bm{\phi}}}
\def\va{{\bm{a}}}
\def\vb{{\bm{b}}}
\def\vc{{\bm{c}}}
\def\vd{{\bm{d}}}
\def\ve{{\bm{e}}}
\def\vf{{\bm{f}}}
\def\vg{{\bm{g}}}
\def\vh{{\bm{h}}}
\def\vi{{\bm{i}}}
\def\vj{{\bm{j}}}
\def\vk{{\bm{k}}}
\def\vl{{\bm{l}}}
\def\vm{{\bm{m}}}
\def\vn{{\bm{n}}}
\def\vo{{\bm{o}}}
\def\vp{{\bm{p}}}
\def\vq{{\bm{q}}}
\def\vr{{\bm{r}}}
\def\vs{{\bm{s}}}
\def\vt{{\bm{t}}}
\def\vu{{\bm{u}}}
\def\vv{{\bm{v}}}
\def\vw{{\bm{w}}}
\def\vx{{\bm{x}}}
\def\vy{{\bm{y}}}
\def\vz{{\bm{z}}}

% Elements of vectors
\def\evalpha{{\alpha}}
\def\evbeta{{\beta}}
\def\evepsilon{{\epsilon}}
\def\evlambda{{\lambda}}
\def\evomega{{\omega}}
\def\evmu{{\mu}}
\def\evpsi{{\psi}}
\def\evsigma{{\sigma}}
\def\evtheta{{\theta}}
\def\eva{{a}}
\def\evb{{b}}
\def\evc{{c}}
\def\evd{{d}}
\def\eve{{e}}
\def\evf{{f}}
\def\evg{{g}}
\def\evh{{h}}
\def\evi{{i}}
\def\evj{{j}}
\def\evk{{k}}
\def\evl{{l}}
\def\evm{{m}}
\def\evn{{n}}
\def\evo{{o}}
\def\evp{{p}}
\def\evq{{q}}
\def\evr{{r}}
\def\evs{{s}}
\def\evt{{t}}
\def\evu{{u}}
\def\evv{{v}}
\def\evw{{w}}
\def\evx{{x}}
\def\evy{{y}}
\def\evz{{z}}

% Matrix
\def\mA{{\bm{A}}}
\def\mB{{\bm{B}}}
\def\mC{{\bm{C}}}
\def\mD{{\bm{D}}}
\def\mE{{\bm{E}}}
\def\mF{{\bm{F}}}
\def\mG{{\bm{G}}}
\def\mH{{\bm{H}}}
\def\mI{{\bm{I}}}
\def\mJ{{\bm{J}}}
\def\mK{{\bm{K}}}
\def\mL{{\bm{L}}}
\def\mM{{\bm{M}}}
\def\mN{{\bm{N}}}
\def\mO{{\bm{O}}}
\def\mP{{\bm{P}}}
\def\mQ{{\bm{Q}}}
\def\mR{{\bm{R}}}
\def\mS{{\bm{S}}}
\def\mT{{\bm{T}}}
\def\mU{{\bm{U}}}
\def\mV{{\bm{V}}}
\def\mW{{\bm{W}}}
\def\mX{{\bm{X}}}
\def\mY{{\bm{Y}}}
\def\mZ{{\bm{Z}}}
\def\mBeta{{\bm{\beta}}}
\def\mPhi{{\bm{\Phi}}}
\def\mLambda{{\bm{\Lambda}}}
\def\mSigma{{\bm{\Sigma}}}

% Tensor
\DeclareMathAlphabet{\mathsfit}{\encodingdefault}{\sfdefault}{m}{sl}
\SetMathAlphabet{\mathsfit}{bold}{\encodingdefault}{\sfdefault}{bx}{n}
\newcommand{\tens}[1]{\bm{\mathsfit{#1}}}
\def\tA{{\tens{A}}}
\def\tB{{\tens{B}}}
\def\tC{{\tens{C}}}
\def\tD{{\tens{D}}}
\def\tE{{\tens{E}}}
\def\tF{{\tens{F}}}
\def\tG{{\tens{G}}}
\def\tH{{\tens{H}}}
\def\tI{{\tens{I}}}
\def\tJ{{\tens{J}}}
\def\tK{{\tens{K}}}
\def\tL{{\tens{L}}}
\def\tM{{\tens{M}}}
\def\tN{{\tens{N}}}
\def\tO{{\tens{O}}}
\def\tP{{\tens{P}}}
\def\tQ{{\tens{Q}}}
\def\tR{{\tens{R}}}
\def\tS{{\tens{S}}}
\def\tT{{\tens{T}}}
\def\tU{{\tens{U}}}
\def\tV{{\tens{V}}}
\def\tW{{\tens{W}}}
\def\tX{{\tens{X}}}
\def\tY{{\tens{Y}}}
\def\tZ{{\tens{Z}}}


% Graph
\def\gA{{\mathcal{A}}}
\def\gB{{\mathcal{B}}}
\def\gC{{\mathcal{C}}}
\def\gD{{\mathcal{D}}}
\def\gE{{\mathcal{E}}}
\def\gF{{\mathcal{F}}}
\def\gG{{\mathcal{G}}}
\def\gH{{\mathcal{H}}}
\def\gI{{\mathcal{I}}}
\def\gJ{{\mathcal{J}}}
\def\gK{{\mathcal{K}}}
\def\gL{{\mathcal{L}}}
\def\gM{{\mathcal{M}}}
\def\gN{{\mathcal{N}}}
\def\gO{{\mathcal{O}}}
\def\gP{{\mathcal{P}}}
\def\gQ{{\mathcal{Q}}}
\def\gR{{\mathcal{R}}}
\def\gS{{\mathcal{S}}}
\def\gT{{\mathcal{T}}}
\def\gU{{\mathcal{U}}}
\def\gV{{\mathcal{V}}}
\def\gW{{\mathcal{W}}}
\def\gX{{\mathcal{X}}}
\def\gY{{\mathcal{Y}}}
\def\gZ{{\mathcal{Z}}}

% Sets
\def\sA{{\mathbb{A}}}
\def\sB{{\mathbb{B}}}
\def\sC{{\mathbb{C}}}
\def\sD{{\mathbb{D}}}
% Don't use a set called E, because this would be the same as our symbol
% for expectation.
\def\sF{{\mathbb{F}}}
\def\sG{{\mathbb{G}}}
\def\sH{{\mathbb{H}}}
\def\sI{{\mathbb{I}}}
\def\sJ{{\mathbb{J}}}
\def\sK{{\mathbb{K}}}
\def\sL{{\mathbb{L}}}
\def\sM{{\mathbb{M}}}
\def\sN{{\mathbb{N}}}
\def\sO{{\mathbb{O}}}
\def\sP{{\mathbb{P}}}
\def\sQ{{\mathbb{Q}}}
\def\sR{{\mathbb{R}}}
\def\sS{{\mathbb{S}}}
\def\sT{{\mathbb{T}}}
\def\sU{{\mathbb{U}}}
\def\sV{{\mathbb{V}}}
\def\sW{{\mathbb{W}}}
\def\sX{{\mathbb{X}}}
\def\sY{{\mathbb{Y}}}
\def\sZ{{\mathbb{Z}}}

% Entries of a matrix
\def\emLambda{{\Lambda}}
\def\emA{{A}}
\def\emB{{B}}
\def\emC{{C}}
\def\emD{{D}}
\def\emE{{E}}
\def\emF{{F}}
\def\emG{{G}}
\def\emH{{H}}
\def\emI{{I}}
\def\emJ{{J}}
\def\emK{{K}}
\def\emL{{L}}
\def\emM{{M}}
\def\emN{{N}}
\def\emO{{O}}
\def\emP{{P}}
\def\emQ{{Q}}
\def\emR{{R}}
\def\emS{{S}}
\def\emT{{T}}
\def\emU{{U}}
\def\emV{{V}}
\def\emW{{W}}
\def\emX{{X}}
\def\emY{{Y}}
\def\emZ{{Z}}
\def\emSigma{{\Sigma}}

% entries of a tensor
% Same font as tensor, without \bm wrapper
\newcommand{\etens}[1]{\mathsfit{#1}}
\def\etLambda{{\etens{\Lambda}}}
\def\etA{{\etens{A}}}
\def\etB{{\etens{B}}}
\def\etC{{\etens{C}}}
\def\etD{{\etens{D}}}
\def\etE{{\etens{E}}}
\def\etF{{\etens{F}}}
\def\etG{{\etens{G}}}
\def\etH{{\etens{H}}}
\def\etI{{\etens{I}}}
\def\etJ{{\etens{J}}}
\def\etK{{\etens{K}}}
\def\etL{{\etens{L}}}
\def\etM{{\etens{M}}}
\def\etN{{\etens{N}}}
\def\etO{{\etens{O}}}
\def\etP{{\etens{P}}}
\def\etQ{{\etens{Q}}}
\def\etR{{\etens{R}}}
\def\etS{{\etens{S}}}
\def\etT{{\etens{T}}}
\def\etU{{\etens{U}}}
\def\etV{{\etens{V}}}
\def\etW{{\etens{W}}}
\def\etX{{\etens{X}}}
\def\etY{{\etens{Y}}}
\def\etZ{{\etens{Z}}}

% The true underlying data generating distribution
\newcommand{\pdata}{p_{\rm{data}}}
\newcommand{\ptarget}{p_{\rm{target}}}
\newcommand{\pprior}{p_{\rm{prior}}}
\newcommand{\pbase}{p_{\rm{base}}}
\newcommand{\pref}{p_{\rm{ref}}}

% The empirical distribution defined by the training set
\newcommand{\ptrain}{\hat{p}_{\rm{data}}}
\newcommand{\Ptrain}{\hat{P}_{\rm{data}}}
% The model distribution
\newcommand{\pmodel}{p_{\rm{model}}}
\newcommand{\Pmodel}{P_{\rm{model}}}
\newcommand{\ptildemodel}{\tilde{p}_{\rm{model}}}
% Stochastic autoencoder distributions
\newcommand{\pencode}{p_{\rm{encoder}}}
\newcommand{\pdecode}{p_{\rm{decoder}}}
\newcommand{\precons}{p_{\rm{reconstruct}}}

\newcommand{\laplace}{\mathrm{Laplace}} % Laplace distribution

\newcommand{\E}{\mathbb{E}}
\newcommand{\Ls}{\mathcal{L}}
\newcommand{\R}{\mathbb{R}}
\newcommand{\emp}{\tilde{p}}
\newcommand{\lr}{\alpha}
\newcommand{\reg}{\lambda}
\newcommand{\rect}{\mathrm{rectifier}}
\newcommand{\softmax}{\mathrm{softmax}}
\newcommand{\sigmoid}{\sigma}
\newcommand{\softplus}{\zeta}
\newcommand{\KL}{D_{\mathrm{KL}}}
\newcommand{\Var}{\mathrm{Var}}
\newcommand{\standarderror}{\mathrm{SE}}
\newcommand{\Cov}{\mathrm{Cov}}
% Wolfram Mathworld says $L^2$ is for function spaces and $\ell^2$ is for vectors
% But then they seem to use $L^2$ for vectors throughout the site, and so does
% wikipedia.
\newcommand{\normlzero}{L^0}
\newcommand{\normlone}{L^1}
\newcommand{\normltwo}{L^2}
\newcommand{\normlp}{L^p}
\newcommand{\normmax}{L^\infty}

\newcommand{\parents}{Pa} % See usage in notation.tex. Chosen to match Daphne's book.

\DeclareMathOperator*{\argmax}{arg\,max}
\DeclareMathOperator*{\argmin}{arg\,min}

\DeclareMathOperator{\sign}{sign}
\DeclareMathOperator{\Tr}{Tr}
\let\ab\allowbreak


% \usepackage{hyperref}
% \usepackage{url}
% \usepackage{algorithm}
% \usepackage{algpseudocode}
% \usepackage{amsthm}
% \usepackage{booktabs}
% \usepackage{multirow}
% \usepackage{graphicx}
% \usepackage{adjustbox}
% \usepackage{wrapfig}
% \usepackage{subfigure}
% \usepackage{subcaption}
% \usepackage{xspace}
% \usepackage[table,xcdraw]{xcolor}
% \usepackage{titletoc}
% \usepackage{comment}
% \usepackage{ulem}
% \usepackage[font=small]{caption}



% Standard packages
\usepackage{hyperref}
\usepackage{url}
\usepackage{algorithm}
\usepackage{algpseudocode}
\usepackage{amsthm}
\usepackage{booktabs}
\usepackage{multirow}
\usepackage{graphicx}
\usepackage{adjustbox}
\usepackage{wrapfig}
\usepackage{subfigure}  % If you must use subfigure
\usepackage{xspace}
\usepackage{titletoc}
\usepackage{comment}
\usepackage{ulem}
\usepackage[font=small]{caption}


\newcommand{\atk}{DocMIA\xspace}
\newcommand{\atkWB}{DocMIA-WB\xspace}
\newcommand{\atkBB}{DocMIA-BB\xspace}
\newcommand{\atkFL}{FL\xspace}
\newcommand{\atkFLLoRA}{FLLoRA\xspace}
\newcommand{\atkIG}{IG\xspace}


\title{DocMIA: Document-Level Membership Inference Attacks against DocVQA Models}

% Authors must not appear in the submitted version. They should be hidden
% as long as the \iclrfinalcopy macro remains commented out below.
% Non-anonymous submissions will be rejected without review.

% \author{Antiquus S.~Hippocampus, Natalia Cerebro \& Amelie P. Amygdale \thanks{ Use footnote for providing further information
% about author (webpage, alternative address)---\emph{not} for acknowledging
% funding agencies.  Funding acknowledgements go at the end of the paper.} \\
% Department of Computer Science\\
% Cranberry-Lemon University\\
% Pittsburgh, PA 15213, USA \\
% \texttt{\{hippo,brain,jen\}@cs.cranberry-lemon.edu} \\
% \And
% Ji Q. Ren \& Yevgeny LeNet \\
% Department of Computational Neuroscience \\
% University of the Witwatersrand \\
% Joburg, South Africa \\
% \texttt{\{robot,net\}@wits.ac.za} \\
% \AND
% Coauthor \\
% Affiliation \\
% Address \\
% \texttt{email}
% }

\author{Khanh Nguyen$^{1}$~~~~~Raouf Kerkouche$^{2}$\thanks{The corresponding author.}~~~~~Mario Fritz$^{2}$~~~~~Dimosthenis Karatzas$^{1}$ \\
$^1${Computer Vision Center, Universitat Aut\`onoma de Barcelona}\\ $^2${CISPA Helmholtz Center for Information Security}\\
\texttt{\{knguyen,dimos\}@cvc.uab.es}\\ \texttt{\{raouf.kerkouche,fritz\}@cispa.de}
}

\newcommand{\fix}{\marginpar{FIX}}
\newcommand{\new}{\marginpar{NEW}}

\theoremstyle{definition}
\newtheorem{definition}{Definition}[section]


\iclrfinalcopy % Uncomment for camera-ready version, but NOT for submission.
\begin{document}

\maketitle

\begin{abstract}
Document Visual Question Answering (DocVQA) has introduced a new paradigm for end-to-end document understanding, and quickly became one of the standard benchmarks for multimodal LLMs. Automating document processing workflows, driven by DocVQA models, presents significant potential for many business sectors. However, documents tend to contain highly sensitive information, raising concerns about privacy risks associated with training such DocVQA models. One significant privacy vulnerability, exploited by the membership inference attack, is the possibility for an adversary to determine if a particular record was part of the model's training data. In this paper, we introduce two novel membership inference attacks tailored specifically to DocVQA models. These attacks are designed for two different adversarial scenarios: a white-box setting, where the attacker has full access to the model architecture and parameters, and a black-box setting, where only the model's outputs are available. Notably, our attacks assume the adversary lacks access to auxiliary datasets, which is more realistic in practice but also more challenging. Our unsupervised methods outperform existing state-of-the-art membership inference attacks across a variety of DocVQA models and datasets, demonstrating their effectiveness and highlighting the privacy risks in this domain.
\end{abstract}

\section{Introduction}

Automated document processing fuels a significant number of operations daily, ranging from fintech and insurance procedures to interactions with public administration and personal record keeping. Up until a few years ago, document processing services relied on template-based information extraction models, which were created ad-hoc for each client. Although these approaches allowed for good control of client data and could be extended to new documents with a few examples, they were limited in scalability and difficult to maintain. Consequently, the introduction of Document Visual Question Answering (DocVQA) \citep{mathew2020document} in 2019 has resulted in a paradigm shift in document processing services, enabling end-to-end generic solutions to be applied in this domain. DocVQA leverages multi-modal large language models to streamline business workflows and provide clients with novel ways to interact with the document processing pipeline. 

%As DocVQA models become more prevalent in handling sensitive documents, significant privacy risks emerge, particularly regarding the potential leakage of sensitive data through model vulnerabilities.

However, as cloud-based DocVQA solutions become more prevalent, significant privacy risks emerge, particularly concerning the potential leakage of sensitive information through model vulnerabilities.
Indeed, during the training of a DocVQA model, \textit{each document can have several associated question-answer pairs}, with each pair considered a unique data point. As a result, a single document can appear multiple times, which significantly raises the risks associated with privacy vulnerabilities. This repeated exposure enhances the likelihood of the model memorizing specific details, thereby increasing the potential for data leakage during privacy attacks.
Furthermore, scanned document images often have high resolutions necessary for posterior analysis, but need to be rescaled for processing by image encoders, potentially rendering content unreadable. To mitigate this issue, many DocVQA models~\citep{huang2022layoutlmv3,tang2023unifying} utilize a dual representation of the document, comprising both a reduced-scale image and OCR-recognized text. This approach introduces further challenges, as sensitive information may leak through multiple modalities.

\begin{figure}
\centering
\begin{subfigure}[b]{0.3\textwidth}
    \centering
    \includegraphics[width=\textwidth]{figures/macarons_1.png}
    \caption{MACARONS (simple scene).}
    \label{fig:teaser_macarons}
\end{subfigure}
\hfill
\begin{subfigure}[b]{0.3\textwidth}
    \centering
    \includegraphics[width=\textwidth]{figures/ours_1_3.png}
    \caption{Our NBP (simple scene).}
    \label{fig:teaser_ours_1}
\end{subfigure}
\hfill
\begin{subfigure}[b]{0.3\textwidth}
    \centering
    \includegraphics[width=0.7\textwidth]{figures/ours_2_1.jpg}
    \caption{Our NBP (hard scene).}
    \label{fig:teaser_ours_2}
\end{subfigure}

\caption{
Reconstruction results and trajectories of MACARONS~\citep{guedon2023macarons} and our NBP model. 
\cite{guedon2023macarons} fails to fully map the environment in simple scenes (a), while our NBP model manages to capture the full scene (b), even in much more complex geometry (c).} 
\label{fig:teaser}
\vspace{-1em}
\end{figure}


% \begin{figure}
%     \centering
%     \begin{tabular}{ccc}
%     \adjustbox{valign=c}{\includegraphics[height=2.7cm]{figures/macarons_1.png}} &
%     \adjustbox{valign=c}{\includegraphics[height=2.7cm]{figures/ours_1_3.png}} &
%     \adjustbox{valign=c}{\includegraphics[height=2.7cm]{figures/ours_2_1.jpg}} \\
%     MACARONS trajectory and &
%     Our trajectory and &
%     Our results for a much\\
%     the resulting reconstruction&
%     the resulting reconstruction&
%     more complex scene\\
%     \end{tabular}
%     \caption{\textbf{Left:} Even in relatively simple scenes, state-of-the-art methods~\citep{guedon2023macarons} can fail to fully map the environment, while our NextBestPath method manages to capture the full scene~(\textbf{middle}), even in much more complex geometry (\textbf{right}).} 
%     \label{fig:teaser}
% \end{figure}

Membership inference attacks (MIAs) are among the most prominent techniques for assessing privacy vulnerabilities in machine learning models. These attacks enable an adversary to determine whether a specific data point is included in the training dataset. However, there is limited research on membership inference risks in the context of multi-modal models. Among the few studies, \citet{ko2023practical,hu2022m} utilize powerful pre-trained models on large datasets to construct an aligned embedding space for the two modalities—image as input and text as output—allowing for the inference of membership information. Unfortunately, the reliance on these pre-trained models poses challenges for document-based tasks, particularly in DocVQA scenarios, where an alignment model capable of aligning the (document, question) as input and the answer as output is currently unavailable. Recently, \citet{tito2024privacy} introduced a provider-level MIAs against DocVQA models aimed at determining whether a \textit{provider (group)} that may supply multiple invoice documents is part of the training set. In contrast, our research focuses on membership information at a finer granularity, specifically targeting the inference of whether a \textit{single document} is included in the training dataset. Current MIA solutions that exploit standard features such as output logits, probabilities, or loss are difficult to adapt to the DocVQA context, where outputs are generated in an auto-regressive manner. Additionally, legal constraints surrounding copyright and private information complicate centralized model training, making it challenging to create auxiliary datasets that capture the variability and richness of real-world data. As a result, shadow training of proxy models becomes infeasible.

In this work, we take a structured approach to privacy testing for DocVQA models. We design a novel Document-level Membership Inference Attack (\atk) that deals with the multiple occurrences of the same document in the training set, as demonstrated in Figure \ref{fig:teaser}. To address the challenge of extracting typical metrics (e.g. logit-based) from auto-regressive outputs, we propose a new method based on model optimisation for individual samples that generates discriminative features for \atk. We design attacks both for white-box and black-box settings without requiring auxiliary datasets. In the black-box setting, we propose an alternative knowledge transfer mechanism from the attacked model to a proxy. Evaluating our attacks on three multi-modal DocVQA models and two datasets, we achieve state-of-the-art performance against multiple baselines.

To summarize, we make the following contributions:
\begin{enumerate}
    \item We present \atk, the first Document-level Membership Inference Attacks specifically targeting multi-modal models for DocVQA.
    \item We introduce two novel auxiliary data-free attacks for both white-box and black-box settings, leveraging novel discriminative metrics for \atk.
    \item We explore three distinct approaches to quantify these metrics: vanilla layer fine-tuning  (\atkFL), fine-tuning layer with LoRA \citep{hu2021lora} (\atkFLLoRA), and image gradients (\atkIG).
    \item Our attacks\footnote{Code is available at ~\url{https://github.com/khanhnguyen21006/mia_docvqa}}, evaluated on two DocVQA datasets across three different models, outperform existing state-of-the-art membership inference attacks as well as baseline attacks.
\end{enumerate}

\section{Related Work}
\label{sec:related_work}
\vspace{-0.15in}

\paragraph{Membership Inference Attack.} 
Membership inference attacks have been extensively explored in various applications to highlight privacy vulnerabilities in deep neural networks or to audit model privacy~\citep{shokri2017membership}. These attacks are categorized into two types: white-box and black-box settings. In white-box settings, the adversary has full access to the target model's internal parameters and computations~\citep{carlini2022membership, yeom2018privacy, nasr2019comprehensive, rezaei2021difficulty, sablayrolles19a, li2021membership}, enabling the use of informative features like loss values, logits, and gradient norms. Conversely, in black-box settings, the adversary is limited to the model's outputs, such as predicted labels or confidence scores~\citep{choquette2021label, shokri2017membership, salem2018ml, sablayrolles19a, song2021systematic, hui2021practical}. The literature indicates that white-box attacks tend to be more effective due to the availability of richer features~\citep{song2019privacy,nasr2019comprehensive}. In this paper, we propose tailored attacks for both settings, considering a more challenging scenario where the adversary lacks an auxiliary dataset --which is used to train shadow models that mimic the behavior of the target model and are subsequently exploited to enhance attack performance-- and is restricted to a limited number of queries. Regarding gradient-based membership inference attacks, research on using gradients as features has been limited. \citet{nasr2019comprehensive} leveraged the $L2$-norm of gradients with respect to model weights for membership inference. \citet{rezaei2021difficulty} suggested using the distance to the decision boundary as a metric but found it ineffective for this purpose. In contrast, we introduce novel strategies called \atkFL, \atkFLLoRA, and \atkIG, demonstrating that the $L2$-norm of the cumulative gradient—computed using these methods—provides a robust signal for membership inference. While \citet{maini2021dataset} and \citet{li2021membership} also explored distance metrics, but from input points for membership inference in image classification tasks, their approaches lack scalability and applicability in our context, which involves larger-scale models with a wider vocabulary of tokens.

\paragraph{Membership Inference Attack Against Multi-modal Models.} Research works into the privacy vulnerabilities of multi-modal models is still in its early stages. Recently, \citet{tito2024privacy,pinto24a} proposed reconstruction attacks that exploit DocVQA model memorization to recover hidden values in documents. They black out specific target values in documents and query the model with questions about the modified documents. Since the model memorizes training data, it often reconstructs the hidden target values. \citet{tito2024privacy} also introduced a membership attack against DocVQA models to infer whether a document provider, with multiple documents, is included in the training dataset. However, as far as we know, no research has yet explored membership inference attacks at document-level granularity. Additionally, \citet{ko2023practical,hu2022m} leverage powerful \textit{pre-trained models} on large datasets to create an aligned embedding space for the two modalities to infer membership. Unfortunately, the reliance on these pre-trained models introduces difficulties for document-based tasks, especially DocVQA, where an appropriate alignment model for aligning (document, question) inputs to corresponding answers is not yet available. Furthermore, the success of both attacks hinges on the availability of \textit{auxiliary datasets} leveraged by the adversary, which are key to executing the attack effectively. In this paper, we present two membership inference attacks specifically tailored to tackle the unique characteristics of DocVQA models.

\section{Background}

\subsection{Document-based Visual Question Answering}
DocVQA is a multi-modal task where natural language questions are posed based on the content of document images. Notably, it establishes a unified query-response framework applicable across various document understanding tasks, such as document classification and information extraction.

Formally, the DocVQA task is defined as follows: given a question-answer pair $(q, a)$ related to a document image $x$, the method $\mathcal{F}$ must generate an answer $\hat{a}=\mathcal{F}(x,q)$ such that $\hat{a}$ closely matches the correct answer $a$. More concretely, given $D_t = \{(x_i,q_i,a_i)\}^{N_t}_{i=1}$ as a set of valid training examples, a model $\mathcal{F}$, parameterized by $\theta$, is trained to maximize the conditional log-likelihood:
% of the ground truth via the following loss:
\begin{equation}
    \mathcal{L}(\theta) =-\log{p_{\theta} (a_i|x_i,q_i)}
\label{eq:training_loss}
\end{equation}

Standard metrics for DocVQA include Accuracy (ACC) and Normalized Levenshtein Similarity (NLS) \citep{biten2019scene}, which measure the similarity between the predicted and correct answer:
\begin{equation}
    \textsc{ACC} = \displaystyle \1_\mathrm{\hat{a} = a};
    \quad
    \textsc{NLS} = 
    \begin{cases}
        1 - \text{NL}(\hat{a}, a) & \text{if } \textsc{NL}(\hat{a}, a) < 0.5, \\
        0 & \text{if } \textsc{NL} \geq 0.5
    \end{cases}
\end{equation}
where $\textsc{NL}(\cdot,\cdot)$ denotes the normalized Levenshtein distance.

In the following sections, for clarity, we often omit the data example index $i$ from the notation.

\subsection{Document-level Membership Inference Attack}
Membership inference attacks~\citep{shokri2017membership} exploit privacy vulnerabilities to determine if a specific data point was included in the training set of a machine learning model. We extend this definition to the Document-level MIA, which is particularly suited in the DocVQA context. 

Given access to a trained DocVQA model $\mathcal{F}$ and a document $x$ drawn from its data distribution $\mathcal{D}$, along with a set of question-answer pairs $Q=\{(q_i,a_i)\}_{i=1}^{M}$ related to the information in the document, an adversary $\mathcal{A}$ designs a decision rule $f_{\mathcal{A}}(x,Q;\mathcal{F})$  to classify the membership status of $x$, aiming for $f_{\mathcal{A}}(x,Q;\mathcal{F})=1$ if $x$ is a member of the training set, otherwise a non-member. It is important to note that the adversary is focused solely on \textit{the membership of the document $x$}, rather than the entire DocVQA \textit{data point} $(x,q,a)$, which is typically the target of prior MI attacks. Moreover, since a single document is associated with many question-answer pairs, this allows the adversary to query the same document using multiple questions for various pieces of information.

\section{\atk against DocVQA models}
\vspace{-0.1in}

In Section~\ref{sec:threat_model}, we elaborate on the threat model relevant to \atk on two scenarios: \textit{white-box} and \textit{black-box} access. We first explain our intuition behind our optimization-based attacks in the white-box setting (Section~\ref{sec:docmia_whitebox}), then adapt this approach to our black-box attacks (Section~\ref{sec:docmia_blackbox}).
\vspace{-0.1in}
\subsection{Threat Model}
\label{sec:threat_model}
\atk can be either a useful or harmful tool in various real-world scenarios. On the positive side, \atk can act as a privacy auditing tool. For instance, in legal document processing, law firms may use these attacks to evaluate whether proprietary or confidential documents, such as contracts or court filings, were included in model training, thereby identifying potential privacy risks.
Conversely, \atk can be maliciously leveraged. As an example, a business competitor could exploit these attacks on an invoice-processing system to infer the presence of specific invoices in the training data, exposing confidential business relationships and leading to risks such as supplier poaching.

In both scenarios, we assume that the adversary aims to infer membership information for a set of documents, determining whether each document is included in the training dataset. These documents may or may not be part of the target model's training data.
Crucially, we further assume \textit{the adversary lacks access to an auxiliary data} $D_{\text{aux}}$ that reflects the characteristics of these documents. This assumption is realistic, as obtaining real-world documents at scale is often prohibitively difficult due to their confidential nature and regulatory restrictions. Consequently, this negates the application of MI attack techniques that require training shadow models~\citep{shokri2017membership,carlini2022membership}. Even if auxiliary documents were available, training numerous shadow document-based models—typically designed with a large number of parameters—would be prohibitively expensive.

Based on the previous examples, we refer to the owner of the document model as \textit{the trainer} and the law firms or competitors as \textit{the adversary}. Given the document distribution $\mathcal{D}$, the trainer trains a document-based model $\mathcal{F}_t$ with private access to $D_t \sim \mathcal{D}$, following a training algorithm $\mathcal{T}$, that defines the model architecture, optimization process, and related details. The adversary owns the set of sensitive documents $D_{\text{test}} \sim \mathcal{D}$, where $D_t \cap D_{\text{test}} \ne \emptyset$, $\vert D_{\text{test}} \vert =N_\text{test}$; but does not know which documents are in $D_t$. Given a document $x \in D_{\text{test}}$ with a set of related queries $Q=\{(q_i,a_i)\}_{i=1}^{M}$, the adversary's goal is to determine whether $x \in D_t$ or $x \notin D_t$. We formulate two attack settings, which specify the adversarial knowledge about the model $\mathcal{F}_t$ and its data distribution $\mathcal{D}$:

\textbf{White-box Setting}. In this scenario, the adversary has full access to the internal workings of the target model, including the model's architecture, weights, gradients from any further training and other internal details. However, the adversary does not have access to the training algorithm $\mathcal{T}$.

\textbf{Black-box Setting}. Here, the adversary can only interact with the target model through an API, which only returns a prediction $\hat{a}$ for each question $q$ on $x$. In addition, the adversary is constrained by a limited number of queries. As in the white-box setting, the adversary has no information about $\mathcal{T}$. This setting reflects the most challenging case~\citep{nasr2019comprehensive,song2019privacy}.

We assume the adversary has full knowledge of the DocVQA task to train the model, including the training objective, document type and exact training questions. 
This assumption is reasonable, as task-level information such as document type, is often publicly available to guide users, making it accessible to adversaries.
The assumption of the exact question knowledge is also plausible, as \textit{an adversary can approximate questions based on the document type}. Further discussion of this assumption and experiments in the setting without exact questions are provided in Appendix~\ref{sec:impact_question_knowledge}.
\vspace{-0.1in}

\subsection{White-box DocMIA}
\label{sec:docmia_whitebox}
In the white-box setting, where the adversary has access to the trained model, shadow training is impractical due to the lack of auxiliary data and high computational cost. To this end, our strategy is to develop unsupervised \textit{metric-based} attacks \citep{hu2022membership}. For each document, we extract a set of features from individual question-answer pairs and aggregate them across all pairs. We then cluster the resulting feature vectors to distinguish member from non-member documents. 
A key challenge is to design discriminative features, as standard metrics (e.g., logit and loss) may be ineffective in this setting (Section \ref{sec:evaluation}). To address this, we propose new features that enhance the informativeness of our membership inference vectors.

\subsubsection{Optimization-Based Discriminative Features}
In this section, we introduce two novel discriminative membership features derived from an optimization process for our attacks against DocVQA models.
\begin{wrapfigure}{R}{0.3\textwidth}
\centering
 \adjustbox{width=.3\textwidth,frame=0.01cm 0cm}{\includegraphics{images/Intuition1.pdf}}\hfill
 $x$ is \textbf{\textcolor{blue}{in}} of the training set 
\medskip

\adjustbox{width=.3\textwidth,frame=0.01cm 0cm}{\includegraphics{images/Intuition2.pdf}}\hfill
$x$ is \textbf{\textcolor{red}{out}} of the training set 
\caption{\textbf{Visualization of our fine-tuning strategy} in the parameters space. Each contour plot represents the optimization landscape w.r.t each pair $(a_i,q_i)$ from document $x$.  In general, the average $\Delta$ computed on a member document $x_{\textcolor{blue}{\textbf{in}}}$ is smaller than non-member document $x_{\textcolor{red}{\textbf{out}}}$.}

\vspace{-0.3in}
\label{fig:intuition}
\end{wrapfigure}

\textbf{Intuition.} Since DocVQA models are typically trained on multiple question-answer pairs per document, the model parameters likely converge to \textit{minimize the average distance to the ground-truth answers} after training. As a result, fine-tuning the model on one question-answer pair through an iterative process is necessary to extract more reliable membership signals. More importantly, this optimization on training documents may converge faster than to non-training documents, due to the lower generalization error. Figure \ref{fig:intuition} illustrates our reasoning.

We provide a formal definition of the \textit{distance} feature.  
\begin{definition}[\textbf{Optimization-based Distance Feature}] Given a model $\mathcal{F}$ parameterized by $\theta$, let the model be initialized with $\theta_0$. After undergoing a gradient-based optimization process $\mathcal{O}$, the parameters converge to $\theta^{*}$ according to a specified training objective $\mathcal{L}$. The \textit{distance} feature is then defined as the $L2$-norm of the change in parameters:
\begin{equation}
    \Delta(\theta_0, \theta^{*}) = || \theta_0 - \theta^{*} ||_2
\label{eq:distance}
\end{equation}
\end{definition}
This feature measures the difference between the initial parameters $\theta_0$ and the converged parameters $\theta^*$, as an approximation of the optimization trajectory toward the optimal solution. 

Specifically, we fine-tune the target DocVQA model on an individual document/question-answer pair and compute the \textit{distance} required to reach the \textit{optimal} answer. A small average distance indicates the document is likely part of the training set, while a larger distance suggests a non-training document. In addition, the number of optimization steps serves as an orthogonal feature that reflects the efficiency of the optimization process. With an optimal learning rate and a good initialization provided from the target model, optimization for training documents typically converges in fewer steps compared to non-training documents. Consequently, we include both the \textit{distance} and the number of optimization steps in our feature set for white-box attacks.
\begin{algorithm}[tb]
    \footnotesize 
   \caption{\atk Assignment}
   \label{alg:assignment}
    \begin{algorithmic}[1]
       \State {\bfseries Input:} model $\mathcal{F}_{\theta_{t}}$, document $x\in D_{\text{test}}$, question-answer pairs $\{(q_i,a_i)\}_{i=1}^{M}$, utility $\mathcal{U}$, aggregation $\Phi$.
       \State {\bfseries Hyperparameters:} optimizer $\textsc{OPT}$, optimization steps $S$, learning rate $\alpha$, threshold $\tau$.
    
       \For{$i=1$ {\bfseries to} $M$}
       \State \textcolor{blue}{Set ${\theta}.\text{requires\_grad}=\text{True}$ \quad  // Change $\theta$ to: $\theta_L$ or $\textsc{LoRA}(\theta_L)$ or $x$ and Freeze $\theta$.}
       \State Initialize: $s_i = 0, u_{i} = \{\}; \mathit{l}_i \gets 0, \theta_0 \gets {\theta}_{t}$
    
       \While{$s_i < S$}
       \State $u_{i} \gets u_{i} \cup \mathcal{U}({\mathcal{F}}_{\theta}(x, q_i), a_i)$
       \If{$(\mathcal{L}(\theta)-\mathit{l}) < \tau$} 
       break; \quad \textcolor{gray}{// Early stopping}
       \EndIf 
       \State ${\theta} \gets \textsc{OPT}(\alpha, \nabla_{{\theta}}(\mathcal{L}(\theta))$
       \State $\mathit{l}_i \gets \mathcal{L}(\theta), s_i \gets s_i+1$
       \EndWhile
       \State ${\Delta}_{i} \gets \lVert \theta_0-{\theta} \rVert_2$ \quad \textcolor{gray}{// Compute distance metric}
        \EndFor
       \State $\Delta_{M} = \Phi(\Delta_{i={1,\dots,M}}); s_{M}=\Phi({s}_{i={1,\dots,M}}); u_{M}=\Phi({u}_{{i={1,\dots,M}}})$ \quad \textcolor{gray}{// Aggregating over $M$ questions}
       \State {\bfseries Output:} ${F}_{x} = [\Delta_{M},s_{M},u_{M}]$  \quad \textcolor{gray}{// Assign membership feature vector}
    \end{algorithmic}
\end{algorithm}

\vspace{-0.05in}
\subsubsection{Methodology}
We now formally present our attack strategy, assuming white-box access to the target model $\mathcal{F}_t$.
For any document $x \in D_{\text{test}}$ and a set of question-answer pairs $Q$, the goal is to assign a features descriptor $F_{x}$. This is achieved by first extracting a set of features through the optimization process $\mathcal{O}$ on a single question-answer pair. These features are aggregated across multiple questions then concatenated to construct $F_{x}$. Repeating this process over $D_{\text{test}}$, we apply an unsupervised clustering algorithm to differentiate member documents from non-members based on their features descriptors.

Following our intuition, for each question-answer pair $(q,a)$, we fine-tune the target model parameter $\theta_{t}$  
using gradient descent to maximize the conditional probability $p_{\theta}(a|x,q)$, as defined by the objective in Equation \ref{eq:training_loss}. The optimization process always starts from the target model parameters $\theta_t$, and the learning rate $\alpha$ controls the optimization speed. During this process, we query the model at each step $s$ using $q$, tracking its prediction quality against $(q,a)$ via a utility function $\mathcal{U}$, either ACC or NLS. The optimization stops when no further improvements is observed, governed by a threshold $\tau$ or after a maximum of $S$ steps. At the end of the optimization, we evaluate the distance $\Delta$ based on Equation \ref{eq:distance}, record the number of steps taken $s$, and aggregate the utility evolution throughout the process to obtain an overall DocVQA score $u$. Collectively, these features serve as membership signals for the current $(q,a)$ pair in relation to the target document $x$.

Since each document is associated with a varying number of question-answer pairs $M$, we employ an aggregation function $\Phi$ to aggregate the features across all $M$ questions, producing in a scalar value for each feature. Optionally, we can utilize a diverse set of aggregation functions to further enrich the feature set. After aggregation, we normalize all aggregated features to ensure they are on a consistent scale. The features descriptor $F_{x}$, assigned to document $x$, is constructed by concatenating these normalized features. The assignment algorithm for each document is detailed in Algorithm \ref{alg:assignment}. Finally, we apply a clustering algorithm to the set of descriptors from documents in $D_{\text{test}}$, predicting the cluster with the larger $\Delta$ as corresponding to non-member documents.
\subsubsection{Improving Efficiency}
Fine-tuning $\mathcal{F}_t$ on a \textit{single} document/question-answer pair provides a mechanism to differentiate between members and non-members. However, this approach is relatively slow, given the model’s size and the complexity of data pre-processing. To improve the attack efficiency, we introduce three variants of the method, as illustrated in Figure \ref{fig:method}:

\textbf{Optimize One Layer (\atkFL)}. Instead of optimizing all parameters, we hope that gradients with respect to a single layer's parameters can provide sufficient signal for membership classification. In this variant, we select one specific layer $L$ to optimize while keeping the remaining parameters fixed. We ablate the choice of layer for this method in Appendix~\ref{sec:calibration}. In addition, we consider a variant leveraging LoRA \citep{hu2021lora}, termed \textbf{\atkFLLoRA}, where the LoRA parameters are initialized with Kaiming initialization \citep{he2015delving}. From Algorithm \ref{alg:assignment}, we replace $\theta$ to $\theta_L$ or the $\textsc{LoRA}$ parameters of the layer L, denoted as $\textsc{LoRA}(\theta_L)$, respectively.

\section{Method}\label{sec:method}
\begin{figure}
    \centering
    \includegraphics[width=0.85\textwidth]{imgs/heatmap_acc.pdf}
    \caption{\textbf{Visualization of the proposed periodic Bayesian flow with mean parameter $\mu$ and accumulated accuracy parameter $c$ which corresponds to the entropy/uncertainty}. For $x = 0.3, \beta(1) = 1000$ and $\alpha_i$ defined in \cref{appd:bfn_cir}, this figure plots three colored stochastic parameter trajectories for receiver mean parameter $m$ and accumulated accuracy parameter $c$, superimposed on a log-scale heatmap of the Bayesian flow distribution $p_F(m|x,\senderacc)$ and $p_F(c|x,\senderacc)$. Note the \emph{non-monotonicity} and \emph{non-additive} property of $c$ which could inform the network the entropy of the mean parameter $m$ as a condition and the \emph{periodicity} of $m$. %\jj{Shrink the figures to save space}\hanlin{Do we need to make this figure one-column?}
    }
    \label{fig:vmbf_vis}
    \vskip -0.1in
\end{figure}
% \begin{wrapfigure}{r}{0.5\textwidth}
%     \centering
%     \includegraphics[width=0.49\textwidth]{imgs/heatmap_acc.pdf}
%     \caption{\textbf{Visualization of hyper-torus Bayesian flow based on von Mises Distribution}. For $x = 0.3, \beta(1) = 1000$ and $\alpha_i$ defined in \cref{appd:bfn_cir}, this figure plots three colored stochastic parameter trajectories for receiver mean parameter $m$ and accumulated accuracy parameter $c$, superimposed on a log-scale heatmap of the Bayesian flow distribution $p_F(m|x,\senderacc)$ and $p_F(c|x,\senderacc)$. Note the \emph{non-monotonicity} and \emph{non-additive} property of $c$. \jj{Shrink the figures to save space}}
%     \label{fig:vmbf_vis}
%     \vspace{-30pt}
% \end{wrapfigure}


In this section, we explain the detailed design of CrysBFN tackling theoretical and practical challenges. First, we describe how to derive our new formulation of Bayesian Flow Networks over hyper-torus $\mathbb{T}^{D}$ from scratch. Next, we illustrate the two key differences between \modelname and the original form of BFN: $1)$ a meticulously designed novel base distribution with different Bayesian update rules; and $2)$ different properties over the accuracy scheduling resulted from the periodicity and the new Bayesian update rules. Then, we present in detail the overall framework of \modelname over each manifold of the crystal space (\textit{i.e.} fractional coordinates, lattice vectors, atom types) respecting \textit{periodic E(3) invariance}. 

% In this section, we first demonstrate how to build Bayesian flow on hyper-torus $\mathbb{T}^{D}$ by overcoming theoretical and practical problems to provide a low-noise parameter-space approach to fractional atom coordinate generation. Next, we present how \modelname models each manifold of crystal space respecting \textit{periodic E(3) invariance}. 

\subsection{Periodic Bayesian Flow on Hyper-torus \texorpdfstring{$\mathbb{T}^{D}$}{}} 
For generative modeling of fractional coordinates in crystal, we first construct a periodic Bayesian flow on \texorpdfstring{$\mathbb{T}^{D}$}{} by designing every component of the totally new Bayesian update process which we demonstrate to be distinct from the original Bayesian flow (please see \cref{fig:non_add}). 
 %:) 
 
 The fractional atom coordinate system \citep{jiao2023crystal} inherently distributes over a hyper-torus support $\mathbb{T}^{3\times N}$. Hence, the normal distribution support on $\R$ used in the original \citep{bfn} is not suitable for this scenario. 
% The key problem of generative modeling for crystal is the periodicity of Cartesian atom coordinates $\vX$ requiring:
% \begin{equation}\label{eq:periodcity}
% p(\vA,\vL,\vX)=p(\vA,\vL,\vX+\vec{LK}),\text{where}~\vec{K}=\vec{k}\vec{1}_{1\times N},\forall\vec{k}\in\mathbb{Z}^{3\times1}
% \end{equation}
% However, there does not exist such a distribution supporting on $\R$ to model such property because the integration of such distribution over $\R$ will not be finite and equal to 1. Therefore, the normal distribution used in \citet{bfn} can not meet this condition.

To tackle this problem, the circular distribution~\citep{mardia2009directional} over the finite interval $[-\pi,\pi)$ is a natural choice as the base distribution for deriving the BFN on $\mathbb{T}^D$. 
% one natural choice is to 
% we would like to consider the circular distribution over the finite interval as the base 
% we find that circular distributions \citep{mardia2009directional} defined on a finite interval with lengths of $2\pi$ can be used as the instantiation of input distribution for the BFN on $\mathbb{T}^D$.
Specifically, circular distributions enjoy desirable periodic properties: $1)$ the integration over any interval length of $2\pi$ equals 1; $2)$ the probability distribution function is periodic with period $2\pi$.  Sharing the same intrinsic with fractional coordinates, such periodic property of circular distribution makes it suitable for the instantiation of BFN's input distribution, in parameterizing the belief towards ground truth $\x$ on $\mathbb{T}^D$. 
% \yuxuan{this is very complicated from my perspective.} \hanlin{But this property is exactly beautiful and perfectly fit into the BFN.}

\textbf{von Mises Distribution and its Bayesian Update} We choose von Mises distribution \citep{mardia2009directional} from various circular distributions as the form of input distribution, based on the appealing conjugacy property required in the derivation of the BFN framework.
% to leverage the Bayesian conjugacy property of von Mises distribution which is required by the BFN framework. 
That is, the posterior of a von Mises distribution parameterized likelihood is still in the family of von Mises distributions. The probability density function of von Mises distribution with mean direction parameter $m$ and concentration parameter $c$ (describing the entropy/uncertainty of $m$) is defined as: 
\begin{equation}
f(x|m,c)=vM(x|m,c)=\frac{\exp(c\cos(x-m))}{2\pi I_0(c)}
\end{equation}
where $I_0(c)$ is zeroth order modified Bessel function of the first kind as the normalizing constant. Given the last univariate belief parameterized by von Mises distribution with parameter $\theta_{i-1}=\{m_{i-1},\ c_{i-1}\}$ and the sample $y$ from sender distribution with unknown data sample $x$ and known accuracy $\alpha$ describing the entropy/uncertainty of $y$,  Bayesian update for the receiver is deducted as:
\begin{equation}
 h(\{m_{i-1},c_{i-1}\},y,\alpha)=\{m_i,c_i \}, \text{where}
\end{equation}
\begin{equation}\label{eq:h_m}
m_i=\text{atan2}(\alpha\sin y+c_{i-1}\sin m_{i-1}, {\alpha\cos y+c_{i-1}\cos m_{i-1}})
\end{equation}
\begin{equation}\label{eq:h_c}
c_i =\sqrt{\alpha^2+c_{i-1}^2+2\alpha c_{i-1}\cos(y-m_{i-1})}
\end{equation}
The proof of the above equations can be found in \cref{apdx:bayesian_update_function}. The atan2 function refers to  2-argument arctangent. Independently conducting  Bayesian update for each dimension, we can obtain the Bayesian update distribution by marginalizing $\y$:
\begin{equation}
p_U(\vtheta'|\vtheta,\bold{x};\alpha)=\mathbb{E}_{p_S(\bold{y}|\bold{x};\alpha)}\delta(\vtheta'-h(\vtheta,\bold{y},\alpha))=\mathbb{E}_{vM(\bold{y}|\bold{x},\alpha)}\delta(\vtheta'-h(\vtheta,\bold{y},\alpha))
\end{equation} 
\begin{figure}
    \centering
    \vskip -0.15in
    \includegraphics[width=0.95\linewidth]{imgs/non_add.pdf}
    \caption{An intuitive illustration of non-additive accuracy Bayesian update on the torus. The lengths of arrows represent the uncertainty/entropy of the belief (\emph{e.g.}~$1/\sigma^2$ for Gaussian and $c$ for von Mises). The directions of the arrows represent the believed location (\emph{e.g.}~ $\mu$ for Gaussian and $m$ for von Mises).}
    \label{fig:non_add}
    \vskip -0.15in
\end{figure}
\textbf{Non-additive Accuracy} 
The additive accuracy is a nice property held with the Gaussian-formed sender distribution of the original BFN expressed as:
\begin{align}
\label{eq:standard_id}
    \update(\parsn{}'' \mid \parsn{}, \x; \alpha_a+\alpha_b) = \E_{\update(\parsn{}' \mid \parsn{}, \x; \alpha_a)} \update(\parsn{}'' \mid \parsn{}', \x; \alpha_b)
\end{align}
Such property is mainly derived based on the standard identity of Gaussian variable:
\begin{equation}
X \sim \mathcal{N}\left(\mu_X, \sigma_X^2\right), Y \sim \mathcal{N}\left(\mu_Y, \sigma_Y^2\right) \Longrightarrow X+Y \sim \mathcal{N}\left(\mu_X+\mu_Y, \sigma_X^2+\sigma_Y^2\right)
\end{equation}
The additive accuracy property makes it feasible to derive the Bayesian flow distribution $
p_F(\boldsymbol{\theta} \mid \mathbf{x} ; i)=p_U\left(\boldsymbol{\theta} \mid \boldsymbol{\theta}_0, \mathbf{x}, \sum_{k=1}^{i} \alpha_i \right)
$ for the simulation-free training of \cref{eq:loss_n}.
It should be noted that the standard identity in \cref{eq:standard_id} does not hold in the von Mises distribution. Hence there exists an important difference between the original Bayesian flow defined on Euclidean space and the Bayesian flow of circular data on $\mathbb{T}^D$ based on von Mises distribution. With prior $\btheta = \{\bold{0},\bold{0}\}$, we could formally represent the non-additive accuracy issue as:
% The additive accuracy property implies the fact that the "confidence" for the data sample after observing a series of the noisy samples with accuracy ${\alpha_1, \cdots, \alpha_i}$ could be  as the accuracy sum  which could be  
% Here we 
% Here we emphasize the specific property of BFN based on von Mises distribution.
% Note that 
% \begin{equation}
% \update(\parsn'' \mid \parsn, \x; \alpha_a+\alpha_b) \ne \E_{\update(\parsn' \mid \parsn, \x; \alpha_a)} \update(\parsn'' \mid \parsn', \x; \alpha_b)
% \end{equation}
% \oyyw{please check whether the below equation is better}
% \yuxuan{I fill somehow confusing on what is the update distribution with $\alpha$. }
% \begin{equation}
% \update(\parsn{}'' \mid \parsn{}, \x; \alpha_a+\alpha_b) \ne \E_{\update(\parsn{}' \mid \parsn{}, \x; \alpha_a)} \update(\parsn{}'' \mid \parsn{}', \x; \alpha_b)
% \end{equation}
% We give an intuitive visualization of such difference in \cref{fig:non_add}. The untenability of this property can materialize by considering the following case: with prior $\btheta = \{\bold{0},\bold{0}\}$, check the two-step Bayesian update distribution with $\alpha_a,\alpha_b$ and one-step Bayesian update with $\alpha=\alpha_a+\alpha_b$:
\begin{align}
\label{eq:nonadd}
     &\update(c'' \mid \parsn, \x; \alpha_a+\alpha_b)  = \delta(c-\alpha_a-\alpha_b)
     \ne  \mathbb{E}_{p_U(\parsn' \mid \parsn, \x; \alpha_a)}\update(c'' \mid \parsn', \x; \alpha_b) \nonumber \\&= \mathbb{E}_{vM(\bold{y}_b|\bold{x},\alpha_a)}\mathbb{E}_{vM(\bold{y}_a|\bold{x},\alpha_b)}\delta(c-||[\alpha_a \cos\y_a+\alpha_b\cos \y_b,\alpha_a \sin\y_a+\alpha_b\sin \y_b]^T||_2)
\end{align}
A more intuitive visualization could be found in \cref{fig:non_add}. This fundamental difference between periodic Bayesian flow and that of \citet{bfn} presents both theoretical and practical challenges, which we will explain and address in the following contents.

% This makes constructing Bayesian flow based on von Mises distribution intrinsically different from previous Bayesian flows (\citet{bfn}).

% Thus, we must reformulate the framework of Bayesian flow networks  accordingly. % and do necessary reformulations of BFN. 

% \yuxuan{overall I feel this part is complicated by using the language of update distribution. I would like to suggest simply use bayesian update, to provide intuitive explantion.}\hanlin{See the illustration in \cref{fig:non_add}}

% That introduces a cascade of problems, and we investigate the following issues: $(1)$ Accuracies between sender and receiver are not synchronized and need to be differentiated. $(2)$ There is no tractable Bayesian flow distribution for a one-step sample conditioned on a given time step $i$, and naively simulating the Bayesian flow results in computational overhead. $(3)$ It is difficult to control the entropy of the Bayesian flow. $(4)$ Accuracy is no longer a function of $t$ and becomes a distribution conditioned on $t$, which can be different across dimensions.
%\jj{Edited till here}

\textbf{Entropy Conditioning} As a common practice in generative models~\citep{ddpm,flowmatching,bfn}, timestep $t$ is widely used to distinguish among generation states by feeding the timestep information into the networks. However, this paper shows that for periodic Bayesian flow, the accumulated accuracy $\vc_i$ is more effective than time-based conditioning by informing the network about the entropy and certainty of the states $\parsnt{i}$. This stems from the intrinsic non-additive accuracy which makes the receiver's accumulated accuracy $c$ not bijective function of $t$, but a distribution conditioned on accumulated accuracies $\vc_i$ instead. Therefore, the entropy parameter $\vc$ is taken logarithm and fed into the network to describe the entropy of the input corrupted structure. We verify this consideration in \cref{sec:exp_ablation}. 
% \yuxuan{implement variant. traditionally, the timestep is widely used to distinguish the different states by putting the timestep embedding into the networks. citation of FM, diffusion, BFN. However, we find that conditioned on time in periodic flow could not provide extra benefits. To further boost the performance, we introduce a simple yet effective modification term entropy conditional. This is based on that the accumulated accuracy which represents the current uncertainty or entropy could be a better indicator to distinguish different states. + Describe how you do this. }



\textbf{Reformulations of BFN}. Recall the original update function with Gaussian sender distribution, after receiving noisy samples $\y_1,\y_2,\dots,\y_i$ with accuracies $\senderacc$, the accumulated accuracies of the receiver side could be analytically obtained by the additive property and it is consistent with the sender side.
% Since observing sample $\y$ with $\alpha_i$ can not result in exact accuracy increment $\alpha_i$ for receiver, the accuracies between sender and receiver are not synchronized which need to be differentiated. 
However, as previously mentioned, this does not apply to periodic Bayesian flow, and some of the notations in original BFN~\citep{bfn} need to be adjusted accordingly. We maintain the notations of sender side's one-step accuracy $\alpha$ and added accuracy $\beta$, and alter the notation of receiver's accuracy parameter as $c$, which is needed to be simulated by cascade of Bayesian updates. We emphasize that the receiver's accumulated accuracy $c$ is no longer a function of $t$ (differently from the Gaussian case), and it becomes a distribution conditioned on received accuracies $\senderacc$ from the sender. Therefore, we represent the Bayesian flow distribution of von Mises distribution as $p_F(\btheta|\x;\alpha_1,\alpha_2,\dots,\alpha_i)$. And the original simulation-free training with Bayesian flow distribution is no longer applicable in this scenario.
% Different from previous BFNs where the accumulated accuracy $\rho$ is not explicitly modeled, the accumulated accuracy parameter $c$ (visualized in \cref{fig:vmbf_vis}) needs to be explicitly modeled by feeding it to the network to avoid information loss.
% the randomaccuracy parameter $c$ (visualized in \cref{fig:vmbf_vis}) implies that there exists information in $c$ from the sender just like $m$, meaning that $c$ also should be fed into the network to avoid information loss. 
% We ablate this consideration in  \cref{sec:exp_ablation}. 

\textbf{Fast Sampling from Equivalent Bayesian Flow Distribution} Based on the above reformulations, the Bayesian flow distribution of von Mises distribution is reframed as: 
\begin{equation}\label{eq:flow_frac}
p_F(\btheta_i|\x;\alpha_1,\alpha_2,\dots,\alpha_i)=\E_{\update(\parsnt{1} \mid \parsnt{0}, \x ; \alphat{1})}\dots\E_{\update(\parsn_{i-1} \mid \parsnt{i-2}, \x; \alphat{i-1})} \update(\parsnt{i} | \parsnt{i-1},\x;\alphat{i} )
\end{equation}
Naively sampling from \cref{eq:flow_frac} requires slow auto-regressive iterated simulation, making training unaffordable. Noticing the mathematical properties of \cref{eq:h_m,eq:h_c}, we  transform \cref{eq:flow_frac} to the equivalent form:
\begin{equation}\label{eq:cirflow_equiv}
p_F(\vec{m}_i|\x;\alpha_1,\alpha_2,\dots,\alpha_i)=\E_{vM(\y_1|\x,\alpha_1)\dots vM(\y_i|\x,\alpha_i)} \delta(\vec{m}_i-\text{atan2}(\sum_{j=1}^i \alpha_j \cos \y_j,\sum_{j=1}^i \alpha_j \sin \y_j))
\end{equation}
\begin{equation}\label{eq:cirflow_equiv2}
p_F(\vec{c}_i|\x;\alpha_1,\alpha_2,\dots,\alpha_i)=\E_{vM(\y_1|\x,\alpha_1)\dots vM(\y_i|\x,\alpha_i)}  \delta(\vec{c}_i-||[\sum_{j=1}^i \alpha_j \cos \y_j,\sum_{j=1}^i \alpha_j \sin \y_j]^T||_2)
\end{equation}
which bypasses the computation of intermediate variables and allows pure tensor operations, with negligible computational overhead.
\begin{restatable}{proposition}{cirflowequiv}
The probability density function of Bayesian flow distribution defined by \cref{eq:cirflow_equiv,eq:cirflow_equiv2} is equivalent to the original definition in \cref{eq:flow_frac}. 
\end{restatable}
\textbf{Numerical Determination of Linear Entropy Sender Accuracy Schedule} ~Original BFN designs the accuracy schedule $\beta(t)$ to make the entropy of input distribution linearly decrease. As for crystal generation task, to ensure information coherence between modalities, we choose a sender accuracy schedule $\senderacc$ that makes the receiver's belief entropy $H(t_i)=H(p_I(\cdot|\vtheta_i))=H(p_I(\cdot|\vc_i))$ linearly decrease \emph{w.r.t.} time $t_i$, given the initial and final accuracy parameter $c(0)$ and $c(1)$. Due to the intractability of \cref{eq:vm_entropy}, we first use numerical binary search in $[0,c(1)]$ to determine the receiver's $c(t_i)$ for $i=1,\dots, n$ by solving the equation $H(c(t_i))=(1-t_i)H(c(0))+tH(c(1))$. Next, with $c(t_i)$, we conduct numerical binary search for each $\alpha_i$ in $[0,c(1)]$ by solving the equations $\E_{y\sim vM(x,\alpha_i)}[\sqrt{\alpha_i^2+c_{i-1}^2+2\alpha_i c_{i-1}\cos(y-m_{i-1})}]=c(t_i)$ from $i=1$ to $i=n$ for arbitrarily selected $x\in[-\pi,\pi)$.

After tackling all those issues, we have now arrived at a new BFN architecture for effectively modeling crystals. Such BFN can also be adapted to other type of data located in hyper-torus $\mathbb{T}^{D}$.

\subsection{Equivariant Bayesian Flow for Crystal}
With the above Bayesian flow designed for generative modeling of fractional coordinate $\vF$, we are able to build equivariant Bayesian flow for each modality of crystal. In this section, we first give an overview of the general training and sampling algorithm of \modelname (visualized in \cref{fig:framework}). Then, we describe the details of the Bayesian flow of every modality. The training and sampling algorithm can be found in \cref{alg:train} and \cref{alg:sampling}.

\textbf{Overview} Operating in the parameter space $\bthetaM=\{\bthetaA,\bthetaL,\bthetaF\}$, \modelname generates high-fidelity crystals through a joint BFN sampling process on the parameter of  atom type $\bthetaA$, lattice parameter $\vec{\theta}^L=\{\bmuL,\brhoL\}$, and the parameter of fractional coordinate matrix $\bthetaF=\{\bmF,\bcF\}$. We index the $n$-steps of the generation process in a discrete manner $i$, and denote the corresponding continuous notation $t_i=i/n$ from prior parameter $\thetaM_0$ to a considerably low variance parameter $\thetaM_n$ (\emph{i.e.} large $\vrho^L,\bmF$, and centered $\bthetaA$).

At training time, \modelname samples time $i\sim U\{1,n\}$ and $\bthetaM_{i-1}$ from the Bayesian flow distribution of each modality, serving as the input to the network. The network $\net$ outputs $\net(\parsnt{i-1}^\mathcal{M},t_{i-1})=\net(\parsnt{i-1}^A,\parsnt{i-1}^F,\parsnt{i-1}^L,t_{i-1})$ and conducts gradient descents on loss function \cref{eq:loss_n} for each modality. After proper training, the sender distribution $p_S$ can be approximated by the receiver distribution $p_R$. 

At inference time, from predefined $\thetaM_0$, we conduct transitions from $\thetaM_{i-1}$ to $\thetaM_{i}$ by: $(1)$ sampling $\y_i\sim p_R(\bold{y}|\thetaM_{i-1};t_i,\alpha_i)$ according to network prediction $\predM{i-1}$; and $(2)$ performing Bayesian update $h(\thetaM_{i-1},\y^\calM_{i-1},\alpha_i)$ for each dimension. 

% Alternatively, we complete this transition using the flow-back technique by sampling 
% $\thetaM_{i}$ from Bayesian flow distribution $\flow(\btheta^M_{i}|\predM{i-1};t_{i-1})$. 

% The training objective of $\net$ is to minimize the KL divergence between sender distribution and receiver distribution for every modality as defined in \cref{eq:loss_n} which is equivalent to optimizing the negative variational lower bound $\calL^{VLB}$ as discussed in \cref{sec:preliminaries}. 

%In the following part, we will present the Bayesian flow of each modality in detail.

\textbf{Bayesian Flow of Fractional Coordinate $\vF$}~The distribution of the prior parameter $\bthetaF_0$ is defined as:
\begin{equation}\label{eq:prior_frac}
    p(\bthetaF_0) \defeq \{vM(\vm_0^F|\vec{0}_{3\times N},\vec{0}_{3\times N}),\delta(\vc_0^F-\vec{0}_{3\times N})\} = \{U(\vec{0},\vec{1}),\delta(\vc_0^F-\vec{0}_{3\times N})\}
\end{equation}
Note that this prior distribution of $\vm_0^F$ is uniform over $[\vec{0},\vec{1})$, ensuring the periodic translation invariance property in \cref{De:pi}. The training objective is minimizing the KL divergence between sender and receiver distribution (deduction can be found in \cref{appd:cir_loss}): 
%\oyyw{replace $\vF$ with $\x$?} \hanlin{notations follow Preliminary?}
\begin{align}\label{loss_frac}
\calL_F = n \E_{i \sim \ui{n}, \flow(\parsn{}^F \mid \vF ; \senderacc)} \alpha_i\frac{I_1(\alpha_i)}{I_0(\alpha_i)}(1-\cos(\vF-\predF{i-1}))
\end{align}
where $I_0(x)$ and $I_1(x)$ are the zeroth and the first order of modified Bessel functions. The transition from $\bthetaF_{i-1}$ to $\bthetaF_{i}$ is the Bayesian update distribution based on network prediction:
\begin{equation}\label{eq:transi_frac}
    p(\btheta^F_{i}|\parsnt{i-1}^\calM)=\mathbb{E}_{vM(\bold{y}|\predF{i-1},\alpha_i)}\delta(\btheta^F_{i}-h(\btheta^F_{i-1},\bold{y},\alpha_i))
\end{equation}
\begin{restatable}{proposition}{fracinv}
With $\net_{F}$ as a periodic translation equivariant function namely $\net_F(\parsnt{}^A,w(\parsnt{}^F+\vt),\parsnt{}^L,t)=w(\net_F(\parsnt{}^A,\parsnt{}^F,\parsnt{}^L,t)+\vt), \forall\vt\in\R^3$, the marginal distribution of $p(\vF_n)$ defined by \cref{eq:prior_frac,eq:transi_frac} is periodic translation invariant. 
\end{restatable}
\textbf{Bayesian Flow of Lattice Parameter \texorpdfstring{$\boldsymbol{L}$}{}}   
Noting the lattice parameter $\bm{L}$ located in Euclidean space, we set prior as the parameter of a isotropic multivariate normal distribution $\btheta^L_0\defeq\{\vmu_0^L,\vrho_0^L\}=\{\bm{0}_{3\times3},\bm{1}_{3\times3}\}$
% \begin{equation}\label{eq:lattice_prior}
% \btheta^L_0\defeq\{\vmu_0^L,\vrho_0^L\}=\{\bm{0}_{3\times3},\bm{1}_{3\times3}\}
% \end{equation}
such that the prior distribution of the Markov process on $\vmu^L$ is the Dirac distribution $\delta(\vec{\mu_0}-\vec{0})$ and $\delta(\vec{\rho_0}-\vec{1})$, 
% \begin{equation}
%     p_I^L(\boldsymbol{L}|\btheta_0^L)=\mathcal{N}(\bm{L}|\bm{0},\bm{I})
% \end{equation}
which ensures O(3)-invariance of prior distribution of $\vL$. By Eq. 77 from \citet{bfn}, the Bayesian flow distribution of the lattice parameter $\bm{L}$ is: 
\begin{align}% =p_U(\bmuL|\btheta_0^L,\bm{L},\beta(t))
p_F^L(\bmuL|\bm{L};t) &=\mathcal{N}(\bmuL|\gamma(t)\bm{L},\gamma(t)(1-\gamma(t))\bm{I}) 
\end{align}
where $\gamma(t) = 1 - \sigma_1^{2t}$ and $\sigma_1$ is the predefined hyper-parameter controlling the variance of input distribution at $t=1$ under linear entropy accuracy schedule. The variance parameter $\vrho$ does not need to be modeled and fed to the network, since it is deterministic given the accuracy schedule. After sampling $\bmuL_i$ from $p_F^L$, the training objective is defined as minimizing KL divergence between sender and receiver distribution (based on Eq. 96 in \citet{bfn}):
\begin{align}
\mathcal{L}_{L} = \frac{n}{2}\left(1-\sigma_1^{2/n}\right)\E_{i \sim \ui{n}}\E_{\flow(\bmuL_{i-1} |\vL ; t_{i-1})}  \frac{\left\|\vL -\predL{i-1}\right\|^2}{\sigma_1^{2i/n}},\label{eq:lattice_loss}
\end{align}
where the prediction term $\predL{i-1}$ is the lattice parameter part of network output. After training, the generation process is defined as the Bayesian update distribution given network prediction:
\begin{equation}\label{eq:lattice_sampling}
    p(\bmuL_{i}|\parsnt{i-1}^\calM)=\update^L(\bmuL_{i}|\predL{i-1},\bmuL_{i-1};t_{i-1})
\end{equation}
    

% The final prediction of the lattice parameter is given by $\bmuL_n = \predL{n-1}$.
% \begin{equation}\label{eq:final_lattice}
%     \bmuL_n = \predL{n-1}
% \end{equation}

\begin{restatable}{proposition}{latticeinv}\label{prop:latticeinv}
With $\net_{L}$ as  O(3)-equivariant function namely $\net_L(\parsnt{}^A,\parsnt{}^F,\vQ\parsnt{}^L,t)=\vQ\net_L(\parsnt{}^A,\parsnt{}^F,\parsnt{}^L,t),\forall\vQ^T\vQ=\vI$, the marginal distribution of $p(\bmuL_n)$ defined by \cref{eq:lattice_sampling} is O(3)-invariant. 
\end{restatable}


\textbf{Bayesian Flow of Atom Types \texorpdfstring{$\boldsymbol{A}$}{}} 
Given that atom types are discrete random variables located in a simplex $\calS^K$, the prior parameter of $\boldsymbol{A}$ is the discrete uniform distribution over the vocabulary $\parsnt{0}^A \defeq \frac{1}{K}\vec{1}_{1\times N}$. 
% \begin{align}\label{eq:disc_input_prior}
% \parsnt{0}^A \defeq \frac{1}{K}\vec{1}_{1\times N}
% \end{align}
% \begin{align}
%     (\oh{j}{K})_k \defeq \delta_{j k}, \text{where }\oh{j}{K}\in \R^{K},\oh{\vA}{KD} \defeq \left(\oh{a_1}{K},\dots,\oh{a_N}{K}\right) \in \R^{K\times N}
% \end{align}
With the notation of the projection from the class index $j$ to the length $K$ one-hot vector $ (\oh{j}{K})_k \defeq \delta_{j k}, \text{where }\oh{j}{K}\in \R^{K},\oh{\vA}{KD} \defeq \left(\oh{a_1}{K},\dots,\oh{a_N}{K}\right) \in \R^{K\times N}$, the Bayesian flow distribution of atom types $\vA$ is derived in \citet{bfn}:
\begin{align}
\flow^{A}(\parsn^A \mid \vA; t) &= \E_{\N{\y \mid \beta^A(t)\left(K \oh{\vA}{K\times N} - \vec{1}_{K\times N}\right)}{\beta^A(t) K \vec{I}_{K\times N \times N}}} \delta\left(\parsn^A - \frac{e^{\y}\parsnt{0}^A}{\sum_{k=1}^K e^{\y_k}(\parsnt{0})_{k}^A}\right).
\end{align}
where $\beta^A(t)$ is the predefined accuracy schedule for atom types. Sampling $\btheta_i^A$ from $p_F^A$ as the training signal, the training objective is the $n$-step discrete-time loss for discrete variable \citep{bfn}: 
% \oyyw{can we simplify the next equation? Such as remove $K \times N, K \times N \times N$}
% \begin{align}
% &\calL_A = n\E_{i \sim U\{1,n\},\flow^A(\parsn^A \mid \vA ; t_{i-1}),\N{\y \mid \alphat{i}\left(K \oh{\vA}{KD} - \vec{1}_{K\times N}\right)}{\alphat{i} K \vec{I}_{K\times N \times N}}} \ln \N{\y \mid \alphat{i}\left(K \oh{\vA}{K\times N} - \vec{1}_{K\times N}\right)}{\alphat{i} K \vec{I}_{K\times N \times N}}\nonumber\\
% &\qquad\qquad\qquad-\sum_{d=1}^N \ln \left(\sum_{k=1}^K \out^{(d)}(k \mid \parsn^A; t_{i-1}) \N{\ydd{d} \mid \alphat{i}\left(K\oh{k}{K}- \vec{1}_{K\times N}\right)}{\alphat{i} K \vec{I}_{K\times N \times N}}\right)\label{discdisc_t_loss_exp}
% \end{align}
\begin{align}
&\calL_A = n\E_{i \sim U\{1,n\},\flow^A(\parsn^A \mid \vA ; t_{i-1}),\N{\y \mid \alphat{i}\left(K \oh{\vA}{KD} - \vec{1}\right)}{\alphat{i} K \vec{I}}} \ln \N{\y \mid \alphat{i}\left(K \oh{\vA}{K\times N} - \vec{1}\right)}{\alphat{i} K \vec{I}}\nonumber\\
&\qquad\qquad\qquad-\sum_{d=1}^N \ln \left(\sum_{k=1}^K \out^{(d)}(k \mid \parsn^A; t_{i-1}) \N{\ydd{d} \mid \alphat{i}\left(K\oh{k}{K}- \vec{1}\right)}{\alphat{i} K \vec{I}}\right)\label{discdisc_t_loss_exp}
\end{align}
where $\vec{I}\in \R^{K\times N \times N}$ and $\vec{1}\in\R^{K\times D}$. When sampling, the transition from $\bthetaA_{i-1}$ to $\bthetaA_{i}$ is derived as:
\begin{equation}
    p(\btheta^A_{i}|\parsnt{i-1}^\calM)=\update^A(\btheta^A_{i}|\btheta^A_{i-1},\predA{i-1};t_{i-1})
\end{equation}

The detailed training and sampling algorithm could be found in \cref{alg:train} and \cref{alg:sampling}.




\textbf{Optimize the Document Image (\atkIG)}.
By switching the perspective to the input space, this variant directly optimizes the pixel values of the document image $x$. The underlying intuition remains the same: training documents require less self-tuning allowing the model to converge faster to the correct answer than non-trainings. However, this assumes the target model allows differentiation of the document image pixels through its architecture. Accordingly, we replace $\theta$ with $x$ from Algorithm \ref{alg:assignment} while freezing the target model parameters $\theta$.

These variants reduce computational costs while maintaining attack performance, providing more practical options when the size of $D_\text{test}$ increases.
\vspace{-0.1in}
\subsection{Black-box \atk}
\label{sec:docmia_blackbox}
In the black-box setting, the attack model’s access is restricted to $D_{\text{test}}$ and the predicted answers. To address these limitations, we propose a distillation-based attack strategy. The key idea is to transfer knowledge about the private data $D_t$ from the black-box model $\mathcal{F}_t$ to a proxy model $\mathcal{F}_p$, parameterized by $\omega$. With full control over the proxy model, the attacks we design for the white-box setting can be fully applied to this.

Specifically, the black-box model is first employed to generate labels for each question in $D_\text{test}$, creating a query dataset $D_{\text{query}}=\{(x_i, Q_i)\}^{N_{\text{test}}}_{i=1}$ where $Q_i = \{(q_j, \mathcal{F}_{\theta_t}(x_i, q_j))\}^{M}_{j=1}$. The proxy model $\mathcal{F}_p$ is then trained on this query dataset, with the objective to maximize the likelihood of the predicted answer $p_{\omega}(\mathcal{F}_{\theta_t}(x_i, q_j)|x_i, q_j)$. In essence, the goal is to replicate the label-prediction behavior of the black-box model. By doing so, we aim to transfer the label space structure from the black-box to the proxy model, with the expectation that the membership features embedded in the black-box model will also be transferred, thus making our attack assumptions under white-box setting valid. Figure \ref{fig:method_distill} illustrates the scheme of our proposed attack.

Since our focus is on the document domain, we initialize $\mathcal{F}_p$ using a publicly available checkpoint $\omega_{\text{pt}}$, pre-trained with self-supervised learning on unlabeled document dataset $D_{\text{pt}}$, which is \textit{inaccessible} and assumed to be \textit{disjoint} from the private dataset $D_t$. This initialization equips the proxy model with a certain level of document understanding while ensuring it has no prior knowledge of the private dataset. As a result, it enables the proxy $\mathcal{F}_p$ to better mimic the prediction behavior and internal dynamics of the black-box model $\mathcal{F}_t$ after fine-tuning. 

It is important to note that, in this scenario, the adversary lacks information of the black-box training algorithm $\mathcal{T}$. This means there is no advantage in terms of model architecture or other training details when constructing the proxy model. As a result, the choice of the proxy model, optimizer, learning rate, etc., is independent of the target model. However, as we demonstrate empirically in later sections (Section \ref{sec:whitebox_results}), while there is a clear benefit when the proxy model shares the same architecture as the black-box model, our attack strategies remain effective even when using entirely different architectures. This suggests that the proposed approach is robust and can be applied without relying on specific model classes or requiring detailed knowledge of the black-box model.
\vspace{-0.1in}
\section{Experimental Setup}
\vspace{-0.1in}
\label{sec:experimental_setup}

\subsection{Target Dataset and Model}

\label{sec:target_model_dataset}
\textbf{Target Dataset}. We study two established DocVQA datasets in the literature for our analysis: \textbf{DocVQA (DVQA)} \citep{mathew2021docvqa} and \textbf{PFL-DocVQA (PFL)} \citep{tito2024privacy}. Both datasets are designed for extractive DocVQA task, where the answer is explicitly found within the document image. Each document in these datasets is accompanied by varying number of questions.

\textbf{Target Model}. We consider three state-of-the-art models which are designed for document understanding tasks: (1) \textbf{Visual T5 (VT5)} \citep{tito2024privacy} (250M parameters) follows the traditional design by utilizing OCR module to facilitate the reasoning process. It leverages the T5 model pre-trained on the C4 corpus \citep{raffel2020exploring}, along with a ViT backbone pre-trained on document data \citep{li2022dit}. (2) \textbf{Donut} \citep{Kim22Donut} (201M parameters) is one of the first end-to-end DocVQA models capable of achieving competitive performance without relying on OCR. It is pre-trained on a large collection of private synthetic documents. (3) \textbf{Pix2Struct} \citep{Lee23Pix2Struct} is another OCR-free document model with two versions: Base (282M) and Large (1.3B parameters). This model is pre-trained to perform semantic parsing on an 80M subset of the C4 corpus.

For the PFL-DocVQA dataset, we consider two targets: VT5, using the public checkpoint provided by the authors\footnote{\url{https://benchmarks.elsa-ai.eu/?ch=2}}, and Donut, which we successfully trained to achieve strong performance following the training procedure from the authors. For the DocVQA dataset, we attack four targets: VT5, Donut, and Pix2Struct (Base and Large), all with publicly available checkpoints from HuggingFace\footnote{\url{https://huggingface.co/models}} \citep{wolf-etal-2020-transformers}. In the black-box setting, we use VT5 and Donut as proxy models. To train the proxy models on the query set $D_{\text{query}}$, we initialize them with their public \textit{pre-trained} checkpoints—the same checkpoints used to fine-tune the target models on the respective target datasets, as outlined in the original papers. For more details on datasets and models, see Appendix \ref{sec:dataset} and \ref{sec:attack_implementation}.

\vspace{-0.2cm}
\subsection{Implementation}
\label{sec:implementation}
Since the optimization process involves several hyperparameters, our strategy is to tune the set of hyperparameters such that our attacks remain effective against each target model under white-box settings, which we then utilize to mount attacks on black-box models.

Assuming the knowledge of training algorithm $\mathcal{T}$ is unavailable for either white-box or black-box settings, we use Adam \citep{kingma2014adam} as the optimizer $\textsc{OPT}$ and fix this choice across all our attack experiments. We explore the impact of learning rate $\alpha$, the selected layer $L$, and we carefully tune the values of threshold $\tau$ in the ablation study (Appendix \ref{sec:calibration}). Following this, we select the optimal set of hyperparameters for each model and apply these settings in all black-box experiments. For the aggregation $\Phi$, we consider 4 aggregation functions $\{\textsc{avg}; \textsc{min}; \textsc{max}; \textsc{med}\}$ for each feature, denoted as $\Phi_\text{all}$. Throughout our experiments, we employ \textsc{KMeans} as the clustering algorithm. 

\vspace{-0.2cm}
\subsection{Evaluation Metric}
Using the official split of each target dataset, we sample 300 member documents from the training split and 300 non-member documents from the test split, resulting a total of $N_\text{test}=600$ test documents. We report \textit{Balanced Accuracy} and \textit{F1 score} as this evaluation metrics for the attack's success in the balanced setting, as in prior works \citep{salem2018ml,watson2022on,ye2022enhanced}. In addition, we evaluate our attacks using \textit{True Positive Rate (TPR) at 1\% and 3\% False Positive Rate (FPR)}, following standard practices in recent MIA literature \citep{carlini2022membership}. For all unsupervised attacks, the membership score for each document is computed as the Euclidean distance between its feature vector and the centroid of the member cluster obtained via \textsc{KMEANS}.

\subsection{Baseline}
In the \textit{black-box} setting, we evaluate three MI attacks as baselines, which only requires the predicted answer to determine membership: Score-Threshold Attack ($\textsc{Score-TA}$), Unsupervised Score-based Attack with $\textsc{avg}$ ($\textsc{Score-UA}$) \citep{tito2024privacy} and $\Phi_{\text{all}}$ ($\textsc{Score-UA}_{\text{all}}$).

For the \textit{grey-box} setting, we consider two additional baselines: Min-K\%\citep{shi2023detecting} and Min-K\%++ \citep{zhang2024min}, which assumes access to token-level probabilities of the generated answers to compute the membership score of each document.

In the \textit{white-box} setting, where loss or gradient information is accessible, we evaluate three further baselines: Loss-Threshold Attack ($\textsc{Loss-TA}$) \citep{yeom2018privacy}, Unsupervised Score+Loss Attack ($\textsc{ScoreLoss-UA}_{\text{all}}$) and Unsupervised Gradient Attack ($\textsc{Gradient-UA}$)\citep{nasr2019comprehensive}.

For detailed descriptions of these methods, we refer readers to Appendix \ref{sec:attack_baseline}.

\begin{table}[t]
\begin{center}
\begin{small}
\begin{adjustbox}{width=\textwidth}
\small
\begin{tabular}{clcccccccccccc}
\toprule
\multicolumn{2}{c}{\multirow{2}{*}{Model}} & \multicolumn{2}{c}{$\textsc{Score-TA}$} & \multicolumn{2}{c}{$\textsc{Score-UA}$} & \multicolumn{2}{c}{$\textsc{Score-UA}_{\text{all}}$} & \multicolumn{2}{c}{$\textsc{Loss-TA}$} & \multicolumn{2}{c}{$\textsc{Gradient-UA}$} & \multicolumn{2}{c}{$\textsc{ScoreLoss-UA}_{\text{all}}$}\\
\cmidrule(l){3-4}
\cmidrule(l){5-6}
\cmidrule(l){7-8}
\cmidrule(l){9-10}
\cmidrule(l){11-12}
\cmidrule(l){13-14}
&                                 & ACC & F1 & ACC & F1 & ACC & F1 & ACC & F1 & ACC & F1 & ACC & F1 \\
\midrule 
\multirow{2}{*}{\rotatebox[origin=c]{90}{PFL}}& {VT5} & \cellcolor[HTML]{C0C0C0}{$\textbf{62.33}$} & \cellcolor[HTML]{C0C0C0}{$\textbf{64.13}$} & \cellcolor[HTML]{C0C0C0}{$61.00_{0.0}$} & \cellcolor[HTML]{C0C0C0}{$68.80_{0.0}$} & \cellcolor[HTML]{C0C0C0}{$60.67_{0.0}$} & \cellcolor[HTML]{C0C0C0}{$60.67_{0.0}$} & $57.83$ & $62.81$ & $60.67_{0.0}$ & $60.67_{0.0}$ & $\textbf{60.67}_{\textbf{0.0}}$ & $\textbf{60.67}_{\textbf{0.0}}$\\

& {Donut} & \cellcolor[HTML]{C0C0C0}{$\textbf{73.33}$} & \cellcolor[HTML]{C0C0C0}{$\textbf{75.14}$} & \cellcolor[HTML]{C0C0C0}{$68.33_{0.0}$} & \cellcolor[HTML]{C0C0C0}{$75.77_{0.0}$} & \cellcolor[HTML]{C0C0C0}{$71.17_{0.67}$} & \cellcolor[HTML]{C0C0C0}{$71.33_{3.32}$} & $\textbf{73.67}$ & $\textbf{78.99}$ & $70.67_{0.0}$ & $69.55_{0.0}$ & $70.83_{0.0}$ & $69.67_{0.0}$
\\
\midrule

\multirow{3}{*}{\rotatebox[origin=c]{90}{DVQA}} & {VT5} & \cellcolor[HTML]{C0C0C0}{$\textbf{75.67}$} & \cellcolor[HTML]{C0C0C0}{$\textbf{75.75}$} & \cellcolor[HTML]{C0C0C0}{$72.17_{0.0}$} & \cellcolor[HTML]{C0C0C0}{$76.18_{0.0}$} & \cellcolor[HTML]{C0C0C0}{$75.17_{0.0}$} & \cellcolor[HTML]{C0C0C0}{$75.13_{0.0}$} & $73.67$ & $77.99$ & $71.17_{0.0}$ & $67.54_{0.0}$ & $\textbf{75.50}_{\textbf{0.0}}$ & $\textbf{76.02}_{\textbf{0.0}}$
\\

& {Donut} & \cellcolor[HTML]{C0C0C0}{$79.67$} & \cellcolor[HTML]{C0C0C0}{$79.53$} & \cellcolor[HTML]{C0C0C0}{$75.97_{0.07}$} & \cellcolor[HTML]{C0C0C0}{$79.57_{0.07}$} & \cellcolor[HTML]{C0C0C0}{$\textbf{80.50}_{\textbf{0.0}}$} & \cellcolor[HTML]{C0C0C0}{$\textbf{81.10}_{\textbf{0.0}}$} & $51.83$ & $53.46$ & $77.17_{0.0}$ & $75.92_{0.0}$ & $\textbf{80.50}_{\textbf{0.0}}$ & $\textbf{81.10}_{\textbf{0.0}}$
\\

& {Pix2Struct-B} & \cellcolor[HTML]{C0C0C0}{$67.33$} & \cellcolor[HTML]{C0C0C0}{$67.97$} & \cellcolor[HTML]{C0C0C0}{$\textbf{68.17}_{\textbf{0.0}}$} & \cellcolor[HTML]{C0C0C0}{$\textbf{71.36}_{\textbf{0.0}}$} & \cellcolor[HTML]{C0C0C0}{$69.13_{0.07}$} & \cellcolor[HTML]{C0C0C0}{$67.67_{0.09}$} & $59.33$ & $64.63$ & $66.0_{0.0}$ & $68.32_{0.0}$ & $\textbf{69.00}_{\textbf{0.0}}$ & $\textbf{67.48}_{\textbf{0.0}}$
\\
\bottomrule
\end{tabular}
\end{adjustbox}
\end{small}
\end{center}
\caption{\textbf{Results from Baseline Attacks.} \textcolor{gray}{\textbf{Gray}} color indicate attacks conducted in the black-box setting. All results are reported based on five random seeds for \textsc{KMeans}. The methods with the best \textit{average} performance across the two metrics are highlighted in \textbf{bold}.}
\vspace{-0.2in}
\label{tab:baseline_results}
\end{table}

\vspace{-0.1in}
\section{Evaluation}
\label{sec:evaluation}
\vspace{-0.1in}
\subsection{White-box Setting}
\label{sec:whitebox_results}

\textbf{Baseline Performance Evaluation.}
Table \ref{tab:baseline_results} (\textit{right}) shows the performance of baseline attacks in the white-box setting. $\textsc{Loss-TA}$, akin to the thresholding loss attack in \citep{yeom2018privacy}, performs poorly on complex DocVQA models, achieving under 60\% accuracy for most targets. In contrast, $\textsc{ScoreLoss-UA}_{\text{all}}$, which combines utility scores and loss features, achieves stronger results: 81\% F1 on Donut, 75\% on VT5, and 69\% on Pix2Struct on DocVQA dataset. However, it underperforms $\textsc{Loss-TA}$ on PFL-DocVQA, with a 3\% drop in Accuracy and 8\% in F1, likely due to high loss variance in this dataset. $\textsc{Gradient-UA}$, which incorporates one-step gradient information, matches the performance of score-based attacks, suggesting that the gradient serves as a useful signal for membership inference. However, none of the baselines generalizes well across all target models.

\textbf{Our Proposed Attacks outperform the Baselines.} We evaluate our proposed attacks—\atkFL, \atkFLLoRA, and \atkIG—in the white-box setting across target models. As shown in Figure \ref{fig:whitebox_results}, our methods consistently achieve high performance, indicating that \textit{optimization-based features generalize well} across various models.
Compared to all baselines, our attacks achieve either the best or near-best performance on both target datasets, with notable F1 scores of 72\% against VT5 and Pix2Struct, and 82.5\% against Donut. Against $\textsc{Gradient-UA}$, our optimization-based features yield up to a 10\% improvement in F1 on Donut, indicating that \textit{single-step gradients are insufficient} for reliable membership inference.
From Table \ref{tab:tpr@fpr_main_whitebox}, our attacks consistently perform well in the low-FPR regime, often surpassing or matching the strongest baselines. For instance, \atkFL achieves a TPR of 8.67\% at 3\% FPR against VT5 on PFL, despite minimal overfitting, and a TPR of 11.00\% at the same FPR against Pix2Struct-B on DocVQA. Additionally, our methods outperform both Min-K\% and Min-K\%++ across all target models, underscoring their effectiveness, particularly for DocMIA setting.
These results highlight the privacy risks posed by optimization-based features in membership inference. For full results and in-depth analysis, please refer to  Appendix \ref{sec:tpr_at_fix_fpr} and \ref{sec:more_analysis}.
\begin{table}[h]
\begin{center}
\begin{small}
\begin{adjustbox}{width=1\textwidth}
\small
\begin{tabular}{clcccccc}
\toprule
\multicolumn{2}{c}{\multirow{2}{*}{{White-box}}} & \multicolumn{2}{c}{FL} & \multicolumn{2}{c}{FLLoRA} & \multicolumn{2}{c}{IG}\\
%\cmidrule{3-8}
\cmidrule(l){3-4}
\cmidrule(l){5-6}
\cmidrule(l){7-8}
&                                 & ACC & F1 & ACC & F1 & ACC & F1 \\
\midrule 
\multirow{2}{*}{\rotatebox[origin=c]{90}{PFL}}& {VT5} & $\textbf{66.63}_{\textbf{0.07}}\textbf{\textcolor{red}{(+5.96)}}$ & $\textbf{72.40}_{\textbf{0.1}}\textbf{\textcolor{red}{(+11.73)}}$ & $65.17_{0.0}\textcolor{red}{(+4.50)}$ & $70.52_{0.0}\textcolor{red}{(+9.85)}$ & $61.83_{0.0}\textcolor{red}{(+1.16)}$ & $69.18_{0.0}\textcolor{red}{(+8.51)}$\\

& {Donut} & $72.67_{0.0}\textcolor{red}{(+1.5)}$ & $77.96_{0.0}\textcolor{red}{(+6.63)}$ & $72.83_{0.0}\textcolor{red}{(+1.66)}$ & $77.94_{0.0}\textcolor{red}{(+6.61)}$ & $\textbf{75.17}_{\textbf{0.0}}\textbf{\textcolor{red}{(+4)}}$ & $\textbf{79.39}_{\textbf{0.0}}\textbf{\textcolor{red}{(+8.06)}}$\\
\midrule

\multirow{3}{*}{\rotatebox[origin=c]{90}{DVQA}}& {VT5} & $75.67_{0.0}\textcolor{red}{(+0.5)}$ & $76.60_{0.0}\textcolor{red}{(+1.47)}$ & $75.57_{0.08}\textcolor{red}{(+0.4)}$ & $77.31_{0.13}\textcolor{red}{(+2.18)}$ & $74.83_{0.0}\textcolor[HTML]{0b5394}{(-0.34)}$ & $76.88_{0.0}\textcolor{red}{(+1.75)}$\\

& {Donut} & $80.33_{0.0}\textcolor[HTML]{0b5394}{(-0.17)}$ & $82.18_{0.0}\textcolor{red}{(+1.08)}$ & $80.0_{0.0}\textcolor[HTML]{0b5394}{(-0.5)}$ & $81.93_{0.0}\textcolor{red}{(+0.83)}$ & $80.33_{0.0}\textcolor[HTML]{0b5394}{(-0.17)}$ & $82.34_{0.0}\textcolor{red}{(+1.24)}$\\

& {Pix2Struct-B} & $71.67_{0.0}\textcolor{red}{(+2.54)}$ & $72.22_{0.0}\textcolor{red}{(+4.55)}$ & $70.50_{0.0}\textcolor{red}{(+1.37)}$ & $71.95_{0.0}\textcolor{red}{(+4.28)}$ & $\textbf{72.00}_{\textbf{0.0}}\textbf{\textcolor{red}{(+2.87)}}$ & $\textbf{72.82}_{\textbf{0.0}}\textbf{\textcolor{red}{(+5.15)}}$\\
\bottomrule
\end{tabular}
\end{adjustbox}
\end{small}
\end{center}
\vskip -0.1in
\caption{\textbf{White-Box: Main Results of \atk.} Values in parentheses indicate the improvement (\textcolor{red}{positive}/\textcolor[HTML]{0b5394}{negative}) of our proposed attacks compared to the $\textsc{ScoreLoss-UA}_{\text{all}}$. Compared to all baselines, the methods with the best \textit{average} performance across the two metrics are highlighted in \textbf{bold}. Results are reported over five random seeds.}
\label{tab:whitebox_results}
\end{table}


\textbf{Why are Our Attacks more effective?}
We evaluate the effectiveness of our optimization-based features compared to traditional metrics such as loss or single-gradient norm.
% \begin{figure}[t]
    \centering
    \begin{minipage}[b]{0.66\textwidth}
        \centering
        \resizebox{0.999\linewidth}{!}{
            \includegraphics[height=5cm,width=1.05\textwidth]{images/whitebox_feature.pdf}
        }
        \caption{\textbf{Membership Features against three different target models on DocVQA Dataset.} \textit{Top}: The distribution of average \textbf{\textit{loss}} over all questions from all target documents on each target model. \textit{Bottom}: T-SNE visualization of the features used in our proposed attacks.}
        \label{fig:whitebox_loss}
    \end{minipage}
    \hfill
    \begin{minipage}[b]{0.3\textwidth}
        \centering
        \includegraphics[height=5cm]{images/whitebox_distance.pdf}
        \caption{\textbf{Distribution of Average \textit{Distance}} from member and non-member documents.}
        \label{fig:whitebox_distance}
    \end{minipage}
\label{fig:whitebox_feature}
\vspace{-0.2in}
\end{figure}


% Limitations of Loss-based Attacks
The Loss-based attack $\textsc{LOSS-TA}$ assumes that member documents exhibit lower loss values than non-member documents after training the target model $\mathcal{F}_{t}$. While this approach leverages the generalization gap, it proves too simplistic for large-scale models that are trained with complex training process to minimize overfitting. The generalization capability of these models, especially in DocVQA tasks, often reduces the sensitivity of the loss as a membership indicator. Our attacks, on the other hand, leverage the optimization landscape with respect to the model parameters, conditioned on each question-answer pair. We hypothesize that the \textit{distance} resulting from parameter optimization of pairs from a member document will be smaller compared to those for a non-member document, as depicted in Figure \ref{fig:intuition}. This fine-grained signal, which reflects the model's internal response to optimization, offers a more discriminative feature for identifying membership.

% compare from the evidence
As illustrated in Appendix Figure \ref{fig:whitebox_distance}, our \textit{distance} feature, computed from the optimization process, provides a better separation between members and non-members compared to loss-based methods (Figure \ref{fig:whitebox_loss}  (\textit{top})). The t-SNE visualization \citep{JMLR:v9:vandermaaten08a} from Figure \ref{fig:whitebox_loss} (\textit{bottom}) further demonstrates that features derived from our attacks yield a more distinct clustering of member and non-member documents in high-dimensional space for all target models, underscoring its efficacy as a membership indicator, therefore outperforms the loss-based approach.
\vspace{-0.1in}
\subsection{Black-box Setting}
\label{sec:blackbox_results}
% \vspace{-0.1in}
\textbf{Baseline Performance Evaluation.} Table \ref{tab:baseline_results} (\textit{left}) presents the results of our black-box baseline attacks, all of which rely on the DocVQA score as the only source of information in this setting. Similar to the loss metric, the score metric is directly correlated with the generalization gap, making attacks more effective when there is a higher degree of overfitting. This trend is illustrated in Figure \ref{fig:F1_scoregap}, where we observe strong MI performance, particularly for the Donut model with 75\% in PFL and 79\% F1 score in DocVQA. Meanwhile, both $\textsc{Score-UA}$-based baselines show comparable performance, especially effective against models trained on DocVQA. Overall, no single method emerges as the clear winner across all target models.

\begin{figure}[h]
    \centering
    \subfigure[\centering \textbf{White-box}]{
        \includegraphics[width=0.48\textwidth]{images/whitebox_generalization.pdf}
    }
    \subfigure[\centering \textbf{Black-box}]{
        \includegraphics[width=0.48\textwidth]{images/blackbox_generalization.pdf}
    }
\caption{\textbf{MI performance versus the Train-Test gap.} The target models exhibit varying Train-Test gaps, measured by the difference in DocVQA scores between member/non-member documents. Our attacks remain effective even when the gap is small, with performance improving as the gap increases across most target models and datasets. In contrast, baseline methods show more variable performance under these conditions.}
\vspace{-0.1in}
\label{fig:F1_scoregap}
\end{figure}


\textbf{Attacking via Proxy Model.} Table \ref{tab:blackbox_results} and Table \ref{tab:tpr@fpr_main_blackbox} present the key results of our proposed black-box attacks using two proxy models, VT5 and Donut. Several important observations:

First, we observe a clear advantage of attacking the proxy models distilled with our proposed techniques. Across a wide range of black-box architectures trained on both target datasets, attacks leveraging the proxy models outperform the black-box baselines in most cases, demonstrating better MI performance. This suggests that, even without knowledge of the black-box model architecture, \textit{one chosen proxy model still effectively distills certain behaviors} from the black-box models which are membership-indicative, enabling our attacks to infer membership with high accuracy.

When the black-box model architecture \textit{matches} that of the proxy, we consistently observe improvements in MI performance, especially when targeting the PFL-DocVQA dataset. Among the target, \textit{Pix2Struct is the most vulnerable} (both Base and Large). Both VT5 and Donut proxies gains of +3.04\% in Accuracy and +4.88\% in F1 score over the best baseline, even against the Pix2Struct-L model, which exhibits strong generalization and a minimal Train-Test gap (Figure \ref{fig:F1_scoregap}).
Furthermore, proxy VT5 can achieve TPRs of 23.00\% and 16.67\% against Donut and Pix2Struct-B, respectively, at 3\% FPR on DocVQA.
% , as shown in Table \ref{tab:tpr@fpr_main_blackbox}
% This aligns with the observed Train-Test utility gaps in target models (Table~\ref{tab:docvqa_performance}), allowing proxies to closely replicate black-box predictions and enhance attack success.
We also provide an analysis of the proxy model in Appendix \ref{sec:impact_proxy_model}.

These results suggest that privacy vulnerabilities can be exploited using simple distillation-based strategies applied to the model's output space.

% \begin{table}[t]
\begin{center}
\begin{small}
\begin{adjustbox}{width=0.8\textwidth}
\small
\begin{tabular}{clcccccc}
\toprule
\multicolumn{2}{c}{Proxy Model} & \multicolumn{6}{c}{VT5}\\

\midrule
\multicolumn{2}{c}{\multirow{2}{*}{\textbf{Black-box}}} & \multicolumn{2}{c}{FL} & \multicolumn{2}{c}{FLLoRA} & \multicolumn{2}{c}{IG}\\
%\cmidrule{3-8}
\cmidrule(l){3-4}
\cmidrule(l){5-6}
\cmidrule(l){7-8}
&                                 & ACC & F1 & ACC & F1 & ACC & F1\\
\midrule 
\multirow{2}{*}{\rotatebox[origin=c]{90}{\textbf{PFL}}}& \textbf{VT5} & $63.33_{0.0}\textcolor{red}{(+2.33)}$ & $69.51_{0.03}\textcolor{red}{(+0.71)}$ & $63.33_{0.0}\textcolor{red}{(+2.33)}$ & $69.01_{0.0}\textcolor{red}{(+0.21)}$ & $62.00_{0.16}\textcolor{red}{(+1)}$ & $69.35_{0.2}\textcolor{red}{(+0.55)}$\\

& \textbf{Donut} & $70.83_{0.0}\textcolor[HTML]{0b5394}{(-0.34)}$ & $76.64_{0.0}\textcolor{red}{(+0.87)}$ & $70.83_{0.0}\textcolor[HTML]{0b5394}{(-0.34)}$ & $76.70_{0.0}\textcolor{red}{(+0.93)}$ & $70.67_{0.0}\textcolor[HTML]{0b5394}{(-0.5)}$ & $76.72_{0.0}\textcolor{red}{(+0.95)}$\\
\midrule

\multirow{4}{*}{\rotatebox[origin=c]{90}{\textbf{DVQA}}}& \textbf{VT5} & $74.33_{0.01}\textcolor[HTML]{0b5394}{(-0.84)}$ & $75.08_{0.0}\textcolor[HTML]{0b5394}{(-1.1)}$ & $74.33_{0.0}\textcolor[HTML]{0b5394}{(-0.84)}$ & $74.67_{0.0}\textcolor[HTML]{0b5394}{(-1.51)}$ & $73.83_{0.08}\textcolor[HTML]{0b5394}{(-1.34)}$ & $75.81_{0.0}\textcolor[HTML]{0b5394}{(-0.37)}$\\

& \textbf{Donut} & $81.67_{0.0}\textcolor{red}{(+1.17)}$ & $82.54_{0.0}\textcolor{red}{(+1.44)}$ & $81.17_{0.0}\textcolor{red}{(+0.67})$ & $82.09_{0.0}\textcolor{red}{(+0.99)}$ & $80.17_{0.0}\textcolor[HTML]{0b5394}{(-0.33)}$ & $81.89_{0.0}\textcolor{red}{(+0.79)}$\\

& \textbf{Pix2Struct-B} & $70.17_{0.0}\textcolor{red}{(+1.04)}$ & $69.71_{0.0}\textcolor[HTML]{0b5394}{(-1.65})$ & $70.27_{0.23}\textcolor{red}{(+1.14)}$ & $70.85_{0.07}\textcolor[HTML]{0b5394}{(-0.51)}$ & $71.17_{0.0}\textcolor{red}{(+2.04)}$ & $72.14_{0.0}\textcolor{red}{(+0.78)}$\\

\cmidrule{2-8}
& \textbf{Pix2Struct-L} & $71.67_{0.01}\textcolor{red}{(+0.84)}$ & $72.13_{0.0}\textcolor{red}{(+1.30)}$ & $70.17_{0.0}\textcolor[HTML]{0b5394}{(-0.66)}$ & $71.27_{0.0}\textcolor{red}{(+0.44)}$ & $71.00_{0.05}\textcolor{red}{(+0.17)}$ & $73.15_{0.0}\textcolor{red}{(+2.32)}$\\

\midrule
\multicolumn{2}{c}{Proxy Model} & \multicolumn{6}{c}{Donut}\\ 

\midrule
\multirow{2}{*}{\rotatebox[origin=c]{90}{\textbf{PFL}}}& \textbf{VT5} & $61.73_{0.08}\textcolor{red}{(+0.73)}$ & $64.04_{0.10}\textcolor[HTML]{0b5394}{(-4.76)}$ & $61.67_{0.08}\textcolor{red}{(+0.67)}$ & $63.49_{0.0}\textcolor[HTML]{0b5394}{(-5.31)}$ & $55.17_{0.17}\textcolor[HTML]{0b5394}{(-5.83)}$ & $57.37_{0.3}\textcolor[HTML]{0b5394}{(-11.43)}$\\

& \textbf{Donut} & $72.17_{0.0}\textcolor{red}{(+2.33)}$ & $76.24_{0.0}\textcolor[HTML]{0b5394}{(-0.19)}$ & $72.67_{0.0}\textcolor{red}{(+1.5)}$ & $77.47_{0.0}\textcolor{red}{(+1.7)}$ & $74.50_{0.0}\textcolor{red}{(+3.33)}$ & $76.43_{0.0}\textcolor{red}{(+0.66)}$\\

\midrule
\multirow{4}{*}{\rotatebox[origin=c]{90}{\textbf{DVQA}}}& \textbf{VT5} & $73.50_{0.0}\textcolor[HTML]{0b5394}{(-4.34)}$ & $75.58_{0.0}\textcolor[HTML]{0b5394}{(-4.36)}$ & $74.17_{0.0}\textcolor[HTML]{0b5394}{(-1)}$ & $76.04_{0.0}\textcolor[HTML]{0b5394}{(-0.14)}$ & $74.0_{0.0}\textcolor[HTML]{0b5394}{(-1.17)}$ & $75.93_{0.01}\textcolor[HTML]{0b5394}{(-0.25)}$\\

& \textbf{Donut} & $79.50_{0.0}\textcolor[HTML]{0b5394}{(-1)}$ & $81.50_{0.0}\textcolor{red}{(+0.4)}$ & $80.0_{0.0}\textcolor[HTML]{0b5394}{(-0.5)}$ & $81.82_{0.0}\textcolor{red}{(+0.72)}$ & $80.27_{0.0}\textcolor[HTML]{0b5394}{(-0.23)}$ & $81.96_{0.0}\textcolor{red}{(+0.86)}$\\

& \textbf{Pix2Struct-B} & $70.83_{0.0}\textcolor{red}{(+3.04)}$ & $71.82_{0.0}\textcolor{red}{(+4.88)}$ & $70.83_{0.06}\textcolor{red}{(+1.70)}$ & $71.73_{0.14}\textcolor{red}{(+0.37)}$ & $71.0_{0.01}\textcolor{red}{(+1.87)}$ & $71.94_{0.0}\textcolor{red}{(+0.58)}$\\

\cmidrule{2-8}
& \textbf{Pix2Struct-L} & $70.83_{0.0}{(0)}$ & $72.95_{0.0}\textcolor{red}{(+2.12)}$ & $71.0_{0.0}\textcolor{red}{(+0.17)}$ & $72.98_{0.0}\textcolor{red}{(+2.15)}$ & $71.0_{0.03}\textcolor{red}{(+0.17)}$ & $72.81_{0.01}\textcolor{red}{(+1.98)}$\\
\bottomrule
\end{tabular}
\end{adjustbox}
\end{small}
\end{center}

\caption{\textbf{Black-Box Setting: Main Results of Black-Box \atk using Donut and VT5 as proxy models}. The checkpoints for the \textbf{black-box models} are trained on the respective datasets. Values in parentheses indicate the improvement (\textcolor{red}{positive}/\textcolor[HTML]{0b5394}{negative}) compared to the \textit{best} number from $\textsc{Score-UA}$-based baselines. Results are reported over five random seeds.}
\vspace{-0.2in}
\label{tab:blackbox_results}
\end{table}

\begin{table}[t]
    \centering
    \begin{minipage}{0.6\textwidth}
        \centering
        \resizebox{\linewidth}{!}{
            \begin{tabular}{clcccccc}
            \toprule
            \multicolumn{2}{c}{Proxy Model} & \multicolumn{6}{c}{VT5}\\
            
            \midrule
            \multicolumn{2}{c}{\multirow{2}{*}{\textbf{Black-box}}} & \multicolumn{2}{c}{FL} & \multicolumn{2}{c}{FLLoRA} & \multicolumn{2}{c}{IG}\\
            %\cmidrule{3-8}
            \cmidrule(l){3-4}
            \cmidrule(l){5-6}
            \cmidrule(l){7-8}
            &                                 & ACC & F1 & ACC & F1 & ACC & F1\\
            \midrule 
            \multirow{2}{*}{\rotatebox[origin=c]{90}{\textbf{PFL}}}& \textbf{VT5} & $63.33_{0.0}\textcolor{red}{(+2.33)}$ & $69.51_{0.03}\textcolor{red}{(+0.71)}$ & $63.33_{0.0}\textcolor{red}{(+2.33)}$ & $69.01_{0.0}\textcolor{red}{(+0.21)}$ & $62.00_{0.16}\textcolor{red}{(+1)}$ & $69.35_{0.2}\textcolor{red}{(+0.55)}$\\
            
            & \textbf{Donut} & $70.83_{0.0}\textcolor[HTML]{0b5394}{(-0.34)}$ & $76.64_{0.0}\textcolor{red}{(+0.87)}$ & $70.83_{0.0}\textcolor[HTML]{0b5394}{(-0.34)}$ & $76.70_{0.0}\textcolor{red}{(+0.93)}$ & $70.67_{0.0}\textcolor[HTML]{0b5394}{(-0.5)}$ & $76.72_{0.0}\textcolor{red}{(+0.95)}$\\
            \midrule
            
            \multirow{4}{*}{\rotatebox[origin=c]{90}{\textbf{DVQA}}}& \textbf{VT5} & $74.33_{0.01}\textcolor[HTML]{0b5394}{(-0.84)}$ & $75.08_{0.0}\textcolor[HTML]{0b5394}{(-1.1)}$ & $74.33_{0.0}\textcolor[HTML]{0b5394}{(-0.84)}$ & $74.67_{0.0}\textcolor[HTML]{0b5394}{(-1.51)}$ & $73.83_{0.08}\textcolor[HTML]{0b5394}{(-1.34)}$ & $75.81_{0.0}\textcolor[HTML]{0b5394}{(-0.37)}$\\
            
            & \textbf{Donut} & $81.67_{0.0}\textcolor{red}{(+1.17)}$ & $82.54_{0.0}\textcolor{red}{(+1.44)}$ & $81.17_{0.0}\textcolor{red}{(+0.67})$ & $82.09_{0.0}\textcolor{red}{(+0.99)}$ & $80.17_{0.0}\textcolor[HTML]{0b5394}{(-0.33)}$ & $81.89_{0.0}\textcolor{red}{(+0.79)}$\\
            
            & \textbf{Pix2Struct-B} & $70.17_{0.0}\textcolor{red}{(+1.04)}$ & $69.71_{0.0}\textcolor[HTML]{0b5394}{(-1.65})$ & $70.27_{0.23}\textcolor{red}{(+1.14)}$ & $70.85_{0.07}\textcolor[HTML]{0b5394}{(-0.51)}$ & $71.17_{0.0}\textcolor{red}{(+2.04)}$ & $72.14_{0.0}\textcolor{red}{(+0.78)}$\\
            
            \cmidrule{2-8}
            & \textbf{Pix2Struct-L} & $71.67_{0.01}\textcolor{red}{(+0.84)}$ & $72.13_{0.0}\textcolor{red}{(+1.30)}$ & $70.17_{0.0}\textcolor[HTML]{0b5394}{(-0.66)}$ & $71.27_{0.0}\textcolor{red}{(+0.44)}$ & $71.00_{0.05}\textcolor{red}{(+0.17)}$ & $73.15_{0.0}\textcolor{red}{(+2.32)}$\\
            
            \midrule
            \multicolumn{2}{c}{Proxy Model} & \multicolumn{6}{c}{Donut}\\ 
            
            \midrule
            \multirow{2}{*}{\rotatebox[origin=c]{90}{\textbf{PFL}}}& \textbf{VT5} & $61.73_{0.08}\textcolor{red}{(+0.73)}$ & $64.04_{0.10}\textcolor[HTML]{0b5394}{(-4.76)}$ & $61.67_{0.08}\textcolor{red}{(+0.67)}$ & $63.49_{0.0}\textcolor[HTML]{0b5394}{(-5.31)}$ & $55.17_{0.17}\textcolor[HTML]{0b5394}{(-5.83)}$ & $57.37_{0.3}\textcolor[HTML]{0b5394}{(-11.43)}$\\
            
            & \textbf{Donut} & $72.17_{0.0}\textcolor{red}{(+2.33)}$ & $76.24_{0.0}\textcolor[HTML]{0b5394}{(-0.19)}$ & $72.67_{0.0}\textcolor{red}{(+1.5)}$ & $77.47_{0.0}\textcolor{red}{(+1.7)}$ & $74.50_{0.0}\textcolor{red}{(+3.33)}$ & $76.43_{0.0}\textcolor{red}{(+0.66)}$\\
            
            \midrule
            \multirow{4}{*}{\rotatebox[origin=c]{90}{\textbf{DVQA}}}& \textbf{VT5} & $73.50_{0.0}\textcolor[HTML]{0b5394}{(-4.34)}$ & $75.58_{0.0}\textcolor[HTML]{0b5394}{(-4.36)}$ & $74.17_{0.0}\textcolor[HTML]{0b5394}{(-1)}$ & $76.04_{0.0}\textcolor[HTML]{0b5394}{(-0.14)}$ & $74.0_{0.0}\textcolor[HTML]{0b5394}{(-1.17)}$ & $75.93_{0.01}\textcolor[HTML]{0b5394}{(-0.25)}$\\
            
            & \textbf{Donut} & $79.50_{0.0}\textcolor[HTML]{0b5394}{(-1)}$ & $81.50_{0.0}\textcolor{red}{(+0.4)}$ & $80.0_{0.0}\textcolor[HTML]{0b5394}{(-0.5)}$ & $81.82_{0.0}\textcolor{red}{(+0.72)}$ & $80.27_{0.0}\textcolor[HTML]{0b5394}{(-0.23)}$ & $81.96_{0.0}\textcolor{red}{(+0.86)}$\\
            
            & \textbf{Pix2Struct-B} & $70.83_{0.0}\textcolor{red}{(+3.04)}$ & $71.82_{0.0}\textcolor{red}{(+4.88)}$ & $70.83_{0.06}\textcolor{red}{(+1.70)}$ & $71.73_{0.14}\textcolor{red}{(+0.37)}$ & $71.0_{0.01}\textcolor{red}{(+1.87)}$ & $71.94_{0.0}\textcolor{red}{(+0.58)}$\\
            
            \cmidrule{2-8}
            & \textbf{Pix2Struct-L} & $70.83_{0.0}{(0)}$ & $72.95_{0.0}\textcolor{red}{(+2.12)}$ & $71.0_{0.0}\textcolor{red}{(+0.17)}$ & $72.98_{0.0}\textcolor{red}{(+2.15)}$ & $71.0_{0.03}\textcolor{red}{(+0.17)}$ & $72.81_{0.01}\textcolor{red}{(+1.98)}$\\
            \bottomrule
            \end{tabular}
        }
        \caption{\textbf{Black-Box Setting: Main Results of Black-Box \atk using Donut and VT5 as proxy models}. The checkpoints for the \textbf{black-box models} are trained on the respective datasets. Values in parentheses indicate the improvement (\textcolor{red}{positive}/\textcolor[HTML]{0b5394}{negative}) compared to the \textit{best} number from $\textsc{Score-UA}$-based baselines. Results are reported over five random seeds.}
        \vspace{-0.2in}
        \label{tab:blackbox_results}
    \end{minipage}
    \hfill
    \begin{minipage}{0.39\textwidth}
        \centering
        \resizebox{\linewidth}{!}{
            \begin{tabular}[]{clcccccc}
            \toprule
             &  & \multicolumn{4}{c}{\textbf{DVQA}} & \multicolumn{2}{c}{\textbf{PFL}} \\
            \cmidrule(l){3-6}
            \cmidrule(l){7-8}
             &  & {\textbf{VT5}} & {\textbf{Donut}} & {\textbf{P2S-B}} & {\textbf{P2S-L}} & {\textbf{VT5}} & {\textbf{Donut}} \\
            \midrule
            & {$\textsc{Score-TA}$} & 9.33  & 11.00 & 8.00  & \textbf{9.00} & 5.00  & 2.67 \\
            & {$\textsc{Score-UA}$} & 7.67 & 15.67 & 6.33  & 6.67  & 3.33  & 3.33 \\
            & {$\textsc{Score-UA}_{\text{all}}$} & 9.33  & 11.00  & 8.00  & 9.00 & 5.00  & 2.67 \\
            \midrule
            \multirow{3}{*}{VT5} & FL & \textbf{12.33} & \textbf{23.00} & \textbf{16.67}  & 5.33 & 2.00 &  \textbf{8.00} \\
            & FLLoRA & {11.33} & 16.33  & 9.33  & 4.67  & 3.33 & 2.00 \\
            & IG & 8.33  & 7.00 & 7.67  & 7.00 & 3.67 & 6.67 \\
            \cmidrule(l){1-8}
            \multirow{3}{*}{Donut} & FL  & 6.33  & 4.00  & 4.67 & 7.33  & 1.33  & 4.00 \\
            & FLLoRA & 6.33  & 5.00  & 6.33  & 8.00  & 5.33 & 5.33 \\
            & IG  & 5.00 & 11.00  & 9.33 &  6.33  & \textbf{6.33}& 4.33 \\
            \bottomrule
            \end{tabular}
        }
        \caption{\textbf{Black-box Setting: TPR at 3\% FPR using Donut and VT5 as proxy models}. Comparison across all black-box methods, with the best-performing method for each metric highlighted in \textbf{bold}. The complete results can be found in the Appendix \ref{sec:tpr_at_fix_fpr}.}
        \label{tab:tpr@fpr_main_blackbox}    
    \end{minipage}
\end{table}




%\section{Limitation}
\vspace{-0.1in}
\section{Conclusion}
% \vspace{-0.1in}
In this paper, we introduce the first document-level membership inference attacks for DocVQA models, highlighting privacy risks in multi-modal settings. By leveraging model optimization techniques, we extract discriminative features that address challenges posed by multi-modal data, repeated document occurrences in training, and auto-regressive outputs. This enables us to propose novel, auxiliary data-free attacks for both white-box and black-box scenarios. Our methods, evaluated across multiple datasets and models, outperform existing membership inference baselines, emphasizing the privacy vulnerabilities in DocVQA models and the urgent need for stronger privacy safeguards.

% \newpage

\section{Ethics Statement}
Our research introduces two novel membership inference attacks on DocVQA models, designed to evaluate the privacy risks inherent in such systems. While our methodology exposes vulnerabilities that could potentially be exploited for malicious purposes, the primary objective of this work is to raise awareness about privacy issues in AI systems, specifically in the context of DocVQA models, and to encourage the development of more privacy-preserving technologies.

\subsubsection*{Acknowledgment}
This work has been funded by the European Lighthouse on Safe and Secure AI (ELSA) from the European Union’s Horizon Europe programme under grant agreement No 101070617. Views and opinions expressed are however those of the authors only and do not necessarily reflect those of the European Union or European Commission. Neither the European Union nor the European Commission can be held responsible for them. Khanh Nguyen and Dimosthenis Karatzas have been supported by the Consolidated Research Group 2021 SGR 01559 from the Research and University Department of the Catalan Government, and by project PID2023-146426NB-100 funded by MCIU/AEI/10.13039/501100011033 and FSE+.

% \section{Reproducibility Statement}
% In this work, we have made several efforts to ensure the reproducibility of our results. We utilize public datasets and open-source models, which are clearly described in Section ~\ref{sec:target_model_dataset}. The implementation details of our proposed membership inference attacks are thoroughly presented in Section~\ref{sec:implementation} and Appendix~\ref{sec:attack_implementation}. Additionally, all relevant hyperparameters used in our experiments are provided in the Appendix~\ref{sec:calibration}, offering detailed information for reproducing our results. We will provide a link to the code for the camera-ready version, enabling future researchers to replicate and extend our work with ease.

\bibliography{main}
\bibliographystyle{iclr2025_conference}

\appendix

\newpage

\startcontents[appendices]
\printcontents[appendices]{l}{1}{\section*{\textbf{Appendix}}\setcounter{tocdepth}{4}}

\vspace{1cm}

\section{DocVQA Datasets}
\label{sec:dataset}

\textbf{DocVQA} \citep{mathew2021docvqa} This dataset contains high-quality human annotations and is widely used as a benchmark for document understanding. It comprises of real-world administrative documents across a diverse range of types, including letters, invoices, and financial reports.

\textbf{PFL-DocVQA} \citep{tito2024privacy} A large-scale dataset of real business invoices, often containing privacy-sensitive information such as payment amounts, tax numbers, and bank account details. This dataset is specifically designed for DocVQA tasks in a federated learning and differential privacy setup, supporting different levels of privacy granularity. The dataset is accompanied by a variant of MI attacks, where the goal is to infer the membership of the invoice's owner (i.e., the provider) from a set of their invoices that were not used during training.

% Due to the unifying nature of DocVQA, our proposed attacks can be applied to any document models that follow the query-response framework. To extend our analysis, we also investigate two additional datasets: \textbf{SROIE} \citet{} and \textbf{FUNSD} \citet{}, which are collections of scanned documents designed for key information extraction tasks. The goal in these tasks is to extract the corresponding value text given a query key.

\sisetup{table-text-alignment=center,table-format=6.0}
\begin{tabular}{lSSSSrrS}
\toprule
{\textbf{Attribute}} & {\textbf{Declarations}} & {\textbf{Users}} & {\textbf{Inactive Users}} & {\textbf{Ambiguous Users}} & \multicolumn{2}{c}{\textbf{Class Proportion}} & {\textbf{Subreddits}} \\
\midrule
Year of Birth & 420803 & 401390 & 1630 & 17341 & Old: 56.19\% & Young: 43.81\% & 9806 \\
Gender & 424330 & 403428 & 1634 & 18337 & Male: 50.89\% & Female: 49.11\% & 9809 \\
Partisan Affiliation & 6369 & 6118 & 4 & 251 & Dem.: 54.55\% & Rep.: 45.45\% & 9137 \\
\bottomrule
\end{tabular}

In Table \ref{tab:dataset_stats}, we present statistics for both the DocVQA and PFL-DocVQA datasets. Additionally, Figure \ref{fig:questions_per_document} shows the distribution of questions per document: (1) while a small subset of documents have more than 10 questions, most contain fewer, and (2) a fraction of documents have only a single question. These trends hold across both datasets.
\begin{figure}
    \centering
    \subfigure[\centering \textbf{DocVQA}]{
    \includegraphics[height=3cm,width=0.45\textwidth]{images/docvqa_question_stats.pdf}
        \label{fig:docvqa_dist}
    }
    \subfigure[\centering \textbf{PFL-DocVQA}]{
    \includegraphics[height=3cm,width=0.45\textwidth]{images/pfl_question_stats.pdf}
        \label{fig:pfl_dist}
    }
\caption{\textbf{The distribution of number per-document questions} from PFL and DocVQA dataset.}
\label{fig:questions_per_document}
\end{figure}


% \section{Document-level Membership Inference Attacks}
% We demonstrate the scheme of Document-level Membership Inference Attacks in Figure \ref{fig:teaser}.

% \begin{figure}
\centering
\begin{subfigure}[b]{0.3\textwidth}
    \centering
    \includegraphics[width=\textwidth]{figures/macarons_1.png}
    \caption{MACARONS (simple scene).}
    \label{fig:teaser_macarons}
\end{subfigure}
\hfill
\begin{subfigure}[b]{0.3\textwidth}
    \centering
    \includegraphics[width=\textwidth]{figures/ours_1_3.png}
    \caption{Our NBP (simple scene).}
    \label{fig:teaser_ours_1}
\end{subfigure}
\hfill
\begin{subfigure}[b]{0.3\textwidth}
    \centering
    \includegraphics[width=0.7\textwidth]{figures/ours_2_1.jpg}
    \caption{Our NBP (hard scene).}
    \label{fig:teaser_ours_2}
\end{subfigure}

\caption{
Reconstruction results and trajectories of MACARONS~\citep{guedon2023macarons} and our NBP model. 
\cite{guedon2023macarons} fails to fully map the environment in simple scenes (a), while our NBP model manages to capture the full scene (b), even in much more complex geometry (c).} 
\label{fig:teaser}
\vspace{-1em}
\end{figure}


% \begin{figure}
%     \centering
%     \begin{tabular}{ccc}
%     \adjustbox{valign=c}{\includegraphics[height=2.7cm]{figures/macarons_1.png}} &
%     \adjustbox{valign=c}{\includegraphics[height=2.7cm]{figures/ours_1_3.png}} &
%     \adjustbox{valign=c}{\includegraphics[height=2.7cm]{figures/ours_2_1.jpg}} \\
%     MACARONS trajectory and &
%     Our trajectory and &
%     Our results for a much\\
%     the resulting reconstruction&
%     the resulting reconstruction&
%     more complex scene\\
%     \end{tabular}
%     \caption{\textbf{Left:} Even in relatively simple scenes, state-of-the-art methods~\citep{guedon2023macarons} can fail to fully map the environment, while our NextBestPath method manages to capture the full scene~(\textbf{middle}), even in much more complex geometry (\textbf{right}).} 
%     \label{fig:teaser}
% \end{figure}
\section{Baselines}
\label{sec:attack_baseline}
For the \textit{black-box} setting, we evaluate three MI attacks as baselines, which only requires generated text to infer the membership of the target document:

\textbf{Score-Threshold Attack $\textsc{(Score-TA)}$} 
assumes that training documents should achieve higher scores than non-training ones. This attack, adapted from \citet{yeom2018privacy}, evaluates the prediction $\hat{a}$ for each question $q$ using the utility function $\mathcal{U}$ and computes the average score $\bar{u}$. A document is then predicted as a member $\bar{u} \geq \kappa$, and non-member otherwise. The threshold $\kappa$ is set as the average value of $\bar{u}$ across $D_\text{test}$.

\textbf{Unsupervised Score-based Attack $\textsc{(Score-UA)}$} \citep{tito2024privacy}. This attack applies an unsupervised clustering algorithm over the set of average score $\bar{u}$ from test documents in $D_\text{test}$, documents within the cluster with higher average score are predicted as members.

\textbf{Unsupervised Score-based Attack - An Extension ($\textsc{Score-UA}_{\text{all}}$)}. This attack extends $\textsc{Score-UA}$ by considering multiple aggregation functions $\Phi_\text{all}$ to form the feature vector.

For the \textit{grey-box} setting, we consider two additional baselines which assumes access to token-level probabilities of the generated answers $a$ to compute the membership score of each document:

\textbf{Min-K\%}\citep{shi2023detecting} computes the average log probability of the lowest-K\% answer tokens as the membership score: $\text{Min-K\%} = \frac{1}{|\text{Min-K\%}(a)|}\Sigma_{a_i\in\text{Min-K}(a)} \log p(a_i| a_{<i})$. Intuitively, training documents are less likely to contain low-probability answer tokens, resulting in higher  scores.

\textbf{Min-K\%++}\citep{zhang2024min} also averages scores from the lowest-K\% probability tokens but assumes that tokens in the predicted answers for training documents have high probabilities or often form the mode of the conditional distribution. Thus, for each token, the score is computed as: $\text{Min-K\%++}(a_{<i}, a_i) = \frac{\log p(a_i| a_{<i}) - \mu_{a_{<t}}}{\sigma_{a_{<t}}}$ with $\mu_{a_{<t}}$ and $\sigma_{a_{<t}}$ are the expectation and standard deviation of $p(a_i| a_{<i})$ respectively.

We adapt these baselines to DocMIA by using an \textsc{AVG} aggregation function to combine scores across question-answer pairs within a document. We evaluated $K\in [0.6, 0.7, 0.8, 0.9, 1.0]$, which correspond to corresponds to 60\% to 100\% the length of the answer and reported the best result.

In the \textit{white-box} setting, where loss information is available, we consider three additional baselines:

\textbf{Loss-Threshold Attack ($\textsc{Loss-TA}$)} \citep{yeom2018privacy}
Similar to $\textsc{Score-TA}$, this attack computes the average loss $\bar{l}= \frac{1}{M}\Sigma^{M}_{i} \mathcal{L}(\mathcal{F}(x, q_i))$. A document is predicted as a member if $\bar{l} \leq \kappa$ and otherwise non-member, where $\kappa$ is selected as the average value of $\bar{l}$ across $D_\text{test}$.

\textbf{Unsupervised One-step Gradient Attack ($\textsc{Gradient-UA}$)} Inspired from \citet{nasr2019comprehensive}, this attack utilizes the average norm of the gradient of the loss $\nabla_{\theta}\mathcal{L}$ from a single optimization step. It also incorporates the average score $\bar{u}$, both aggregated with $\Phi_\text{all}$ as the features to perform clustering.

\textbf{Unsupervised Score+Loss Attack ($\textsc{ScoreLoss-UA}_{\text{all}}$)}
This attack extends $\textsc{ScoreUA}_{\text{all}}$, combining the average loss $\bar{l}$ with the average utility score $\bar{u}$, then aggregating with $\Phi_\text{all}$.

\section{Ablation Study}
\label{sec:calibration}
In this section, we provide a detailed analysis of the hyperparameter tuning process for \atk in the white-box setting, targeting all the considered models. Given the high computational cost due to the numerous factors involved, we focus on the key parameters that may potentially affect the attack performance. Our intuition behind this tuning process is that: achieving a reliable estimate of the distance $\Delta$ requires the optimization process to converge effectively, which in turn correlates with higher attack accuracy. Thus in all of our experiments, to increase the likelihood of convergence, we set the maximum number of optimization steps to $S = 200$. We fix the maximum number of questions per document $M$ to 10.

\textbf{Learning Rate $\alpha$}. We first study the effect of $\alpha$, which controls the speed of the optimization process in our attacks. This threshold $\tau$ is empirically set to be the average loss change observed when performing one optimization step after reaching the correct answer. Only the distance $\Delta$ and the number of steps $s$ are used as the features. For FL and FLLoRA attacks, we perform a hyperparameter search over a grid of learning rates, $\alpha \in \{10^{-4}, 0.001, 0.01, 0.1, 0.5, 1.0\}$, and $\alpha \in \{0.001, 0.01, 0.1, 0.5, 1.0, 5.0, 10.0, 20.0\}$ for the IG attacks. For FL and FLLoRA, we specifically tune the embedding projection layer, which projects the final hidden states into the vocabulary space, a common design choice across all the target models considered.

As shown in Figure \ref{fig:ablate_alpha}, setting a high learning rate can cause the optimization process to overshoot, while lower values lead to a more stable but slower convergence. We find that a learning rate of $\alpha = 10^{-3}$ consistently delivers the best attack performance across most of the settings.

\textbf{The layer to tune ${\textsc{L}}$}.
We now investigate the impact of layer selection on the performance of our FL and FLLoRA attacks. All target models in our study follow the transformer encoder-decoder architecture \citep{vaswani2017attention}, where each component consists of a stack of attention layers, and a shared embedding projection layer maps the hidden states to logit vectors for prediction. Given this common structure, we examine the effect of tuning similar layers across all models, with results for attack accuracy presented in Table \ref{tab:ablate_layer}.
\begin{figure}[t]
    \centering
    \subfigure[\centering \textbf{Learning Rate $\alpha$}]{
        \includegraphics[height=3cm,width=0.45\textwidth]{images/ablate_learning_rate.pdf}
        \label{fig:ablate_alpha}
    }
    \subfigure[\centering \textbf{Threshold $\tau$}]{
        \includegraphics[height=3cm,width=0.45\textwidth]{images/ablate_threshold.pdf}
        \label{fig:ablate_tau}
    }
\caption{\textbf{Ablation Study on Learning Rate $\alpha$ and Threshold $\tau$.} The best value for each model across all datasets is used as the hyperparameters in our black-box attacks.}
\label{fig:ablate_alpha_tau}
% \vspace{-0.2in}
\end{figure}

\begin{table}[t]
\begin{center}
\begin{small}
\small
\begin{tabular}{lccc}
\toprule
Layer & VT5(PFL) & Donut(DocVQA) & Pix2Struct-B(DocVQA)\\
\midrule 
Embedding Projection Layer & 67.0 & 71.33 & 68.66\\
Embedding Layer Norm & 65.33 & 76.0 & 64.67\\
\cmidrule{1-4}
Last Decoder Block FC1 & \textbf{68.33} & \textbf{78.0} & 68\\
Last Decoder Block FC2 & 68.17 & 77.33 & \textbf{68.83}\\
Last Decoder Block Layer Norm & 61.83 & 76.83 & 67.5\\
\cmidrule{1-4}
Random Decoder Block FC1 & 61.33 & 72.0 & 67.5\\
Random Decoder Block FC2 & 64.0 & 73.0 & 65.17\\
\bottomrule
\end{tabular}
\end{small}
\end{center}
\caption{\textbf{Effect of selected layer to tune} from each target model. Attack performances are reported in terms of Accuracy.}
\label{tab:ablate_layer}
\vspace{-0.2in}
\end{table}



Our findings reveal that \textit{layers closer to the final output exhibit higher privacy leakage} in terms of MI compared to (randomly selected) intermediate layers, likely due to receiving larger gradient updates. Specifically, fine-tuning the final fully connected layer alone leads to strong attack performance while also being more efficient in terms of the number of parameters that need to be optimized. This suggests that focusing on the last layers can achieve both high privacy leakage and computational efficiency in our MI attacks.

\textbf{Threshold $\tau$}.
With the optimizer $\textsc{OPT}$ and learning rate $\alpha$ fixed, the threshold $\tau$ emerges as the most critical hyperparameter that requires careful tuning for each attack. We experiment with a wide range of $\tau$ values, spanning from $10^{-6}$ to $10.0$, and select the optimal value based on attack performance, as demonstrated in Figure \ref{fig:ablate_tau}. This optimal $\tau$ is then applied consistently in all subsequent experiments. Careful selection of this threshold is crucial, as it directly influences the stability and success of the optimization process.

\begin{table}[t]
\begin{center}
\begin{small}
\small
\begin{tabular}{ccccccc}
\toprule
Model & $\alpha_{\textsc{FL}}$ & $\alpha_{\textsc{IG}}$ & $S$ & $L$ & $\tau_{\text{FL}}$ & $\tau_{\text{IG}}$\\
\midrule 
VT5 & \multirow{3}{*}{0.001} & 1.0 & \multirow{3}{*}{200} & \multirow{3}{*}{last FC layer} & $10^{-6}$ & $10^{-5}$\\
Donut &  & 0.001 &  &  & 1.0 & 5.0\\
Pix2Struct-B &  & 0.001 &  &  & $10^{-4}$ & $10^{-3}$\\
\bottomrule
\end{tabular}
\end{small}
\end{center}
\caption{\textbf{Best Hyperaremeters from our tuning process} with consistent performance across both PFL and DocVQA dataset.}
\label{tab:best_hps}
\end{table}


\begin{table}[t]
\begin{center}
\begin{small}
\begin{adjustbox}{width=1\textwidth}
\small
\begin{tabular}{ccccccc}
\toprule
\multirow{2}{*}{Model} & \multirow{2}{*}{Num. Params} & \multirow{2}{*}{Downstream Task} & \multicolumn{2}{c}{Data} & \multicolumn{2}{c}{Checkpoint} \\
\cmidrule(l){4-5}
\cmidrule(l){6-7}
& & & Pretrain & Finetune & Pretrain & Finetune \\
\midrule 
\multirow{2}{*}{VT5} & \multirow{2}{*}{250M} & \multirow{2}{*}{DocVQA} & \multirow{2}{*}{C4+IIT-CDIP} & PFL & \multicolumn{2}{c}{\multirow{2}{*}{https://benchmarks.elsa-ai.eu/?ch=2}} \\
 &  &  &  & DocVQA &  & \\
\midrule 

\multirow{2}{*}{Donut} & \multirow{2}{*}{200M} & \multirow{2}{*}{DocVQA} & \multirow{2}{*}{CDIP 11M + 0.5M synthesized Docs} & PFL &  \multicolumn{2}{c}{Ours}  \\
 &  &  &  & DocVQA & $\text{naver-clova-ix/donut-base}^{\dagger}$ & $\text{naver-clova-ix/donut-base-finetuned-docvqa}^{\dagger}$  \\
\midrule 
Pix2struct-B & 282M & \multirow{2}{*}{DocVQA} & \multirow{2}{*}{BooksCorpus + C4 Web HTML} & \multirow{2}{*}{DocVQA} & $\text{google/pix2struct-base}^{\dagger}$ & $\text{google/pix2struct-docvqa-base}^{\dagger}$ \\
Pix2struct-L & 1.33B &  &  &  & $\text{google/pix2struct-large}^{\dagger}$ & $\text{google/pix2struct-docvqa-large}^{\dagger}$ \\
\bottomrule
\end{tabular}
\end{adjustbox}
\end{small}
\end{center}
\caption{\textbf{Details of the public checkpoints} used as target models in this work. $\dagger$ denotes checkpoint from Hugging Face.}
\label{tab:public_checkpoint}
\vskip -0.1in
\end{table}



We summarize the set of tuned hyperparameters for our approach in Table \ref{tab:best_hps}.

\section{More on Attack Implementation}
\label{sec:attack_implementation}
\subsection{Target Model Training}

For all target models, whenever feasible, we utilize the public checkpoint fine-tuned on the considered private dataset from Hugging Face library and adhere to the data processing guidelines, such as document resolution, as recommended by the authors. We deliberately opt for public checkpoints for two reasons: (1) to make it consistent to further research in privacy attacks that use the same trained models, and (2) to minimizing the biases in model training that affect the final results, given the complexity of the original training process and our limited resources. Table \ref{tab:public_checkpoint} summarizes the details of the process from which public checkpoints for the target models considered in this work are obtained. This includes the datasets the models were pre-trained on, before by fine-tuning on target DocVQA datasets, along with the corresponding download URLs for these checkpoints.






\begin{table}[h]
    \centering
    \scriptsize
    \caption{\textbf{Training Parameters.} Detailed training parameters, including encoder layers, heads, model dimensions, and optimization setups. The \textbf{--} indicates a parameter not used. LaBraM and EEGPT are excluded from the table since the code structure is different.
    }
    \vspace{2mm}
    \label{tab:training_params}
    \resizebox{\textwidth}{!}{
    \begin{tabular}{@{}ll|c|c|c|c|c|c|c|c|c|c|c|c|c@{}}
    \toprule
    \multicolumn{2}{l|}{\textbf{\diagbox{\textbf{Methods}}{\textbf{Params}}}} 
    & \textit{backbone} & \textit{e\_layers} & \textit{n\_heads} & \textit{d\_model} & \textit{d\_ff} 
    & \textit{batch\_size} & \textit{train\_epochs} & \textit{optimizer} 
    & \textit{learning\_rate} & \textit{lr\_scheduler} & \textit{gradient\_clip}  & \textit{patience} & \textit{swa} \\
    \midrule


    \multicolumn{14}{c}{\textbf{Single-Dataset Supervised Learning}}  \\
    
    \midrule

    \textbf{TCN} & & TCN & 6 & \textbf{--} & \textbf{--} & \textbf{--} & 128 & 100 & AdamW & 1e-4 & CosineAnnealing & 4.0 & 15 & \checkmark \\
    \textbf{Transformer} & & Transformer & 6 & 8 & 128 & 256 & 128 & 100 & AdamW & 1e-4 & CosineAnnealing & 4.0 & 15 & \checkmark \\
    \textbf{Conformer} & & Conformer & 6 & 8 & 128 & 256 & 128 & 100 & AdamW & 1e-4 & CosineAnnealing & 4.0 & 15 & \checkmark \\
    \textbf{TimesNet} & & TimesNet & 2 & \textbf{--} & 32 & 64 & 64 & 100 & AdamW & 1e-4 & CosineAnnealing & 4.0 & 15 & \checkmark \\
    \textbf{Medformer} & & Medformer & 6 & 8 & 128 & 256 & 128 & 100 & AdamW & 1e-4 & CosineAnnealing & 4.0 & 15 & \checkmark \\
    \textbf{LEAD-Vanilla(Ours)} & & LEAD & 12 & 8 & 128 & 256 & 128 & 100 & AdamW & 1e-4 & CosineAnnealing & 4.0 & 15 & \checkmark \\
    \midrule



    \multicolumn{14}{c}{\textbf{Unified Supervised Learning}}  \\
    \midrule
    
    \textbf{LEAD-Sup(Ours)} & & LEAD & 12 & 8 & 128 & 256 & 128 & 100 & AdamW & 1e-4 & CosineAnnealing & 4.0 & 15 & \checkmark \\

    \midrule




    \multicolumn{14}{c}{\textbf{Self-Supervised Pre-training}}  \\
    \midrule

    \textbf{TS2Vec} & & Transformer & 20 & 12 & 128 & 256 & 512 & 50 & AdamW & 2e-4 & CosineAnnealing & 4.0 & \textbf{--} & \checkmark \\
    \textbf{BIOT} & & BIOT & 20 & 12 & 128 & 256 & 512 & 50 & AdamW & 2e-4 & CosineAnnealing & 4.0 & \textbf{--} & \checkmark \\
    \textbf{EEG2Rep} & & EEG2Rep & 20 & 12 & 128 & 256 & 512 & 50 & AdamW & 2e-4 & CosineAnnealing & 4.0 & \textbf{--} & \checkmark \\
    \textbf{LEAD-Base(Ours)} & & LEAD & 12 & 8 & 128 & 256 & 512 & 50 & AdamW & 2e-4 & CosineAnnealing & 4.0 & \textbf{--} & \checkmark \\
    \midrule


    \multicolumn{14}{c}{\textbf{Unified Fine-tuning}}  \\
    \midrule
    
    \textbf{TS2Vec} & & Transformer & 20 & 12 & 128 & 256 & 128 & 100 & AdamW & 1e-4 & CosineAnnealing & 4.0 & 15 & \checkmark \\
    \textbf{BIOT} & & BIOT & 20 & 12 & 128 & 256 & 128 & 100 & AdamW & 1e-4 & CosineAnnealing & 4.0 & 15 & \checkmark \\
    \textbf{EEG2Rep} & & EEG2Rep & 20 & 12 & 128 & 256 & 128 & 100 & AdamW & 1e-4 & CosineAnnealing & 4.0 & 15 & \checkmark \\
    \textbf{LEAD-Base(Ours)} & & LEAD & 12 & 8 & 128 & 256 & 128 & 100 & AdamW & 1e-4 & CosineAnnealing & 4.0 & 15 & \checkmark \\
    
    \bottomrule
    \end{tabular}
    }

\end{table}


If public checkpoints are unavailable, we fine-tune the selected model on the respective private dataset, using the pre-trained checkpoint as the initialization, with the training procedure outlined by the respective authors. To prevent overfitting, we perform early stopping based on validation performance, ensuring that all evaluated models generalize well to unseen data. We also use the pre-trained checkpoint to initialize the proxy model $\mathcal{F}_p$ to train it on $D_{\text{query}}$. We provide an overview of the training procedure for each target model, based on the respective papers. These procedures were adapted to fit our computational resources, as outlined in Table \ref{tab:training_params}.
\begin{table}[t]
\begin{center}
\begin{small}
\small
\begin{adjustbox}{width=0.6\textwidth}
\begin{tabular}{lccccc}
\toprule
Dataset & Model & Test Set & ACC & ANLS & Train-Test Gap \\
\midrule
\multirow{6}{*}{{PFL}} & \multirow{3}{*}{{VT5}} & Original &  81.4 & 90.17  & -\\
 &  & MIA & 82.74 & 90.91 & 11.44 \\
 &  & MIA-rephrased & 77.59 & 85.84 & -\\
\cmidrule{2-6}
& \multirow{3}{*}{{Donut}} & Original & 74.73 & 88.66 & -\\
 &  & MIA & 80.15 & 91.64 & 22.2 \\
 &  & MIA-rephrased & 70.46 & 80.96 & -\\
\midrule
\multirow{12}{*}{{DVQA}}& \multirow{3}{*}{{VT5}} & Original & 60.1 & 69.33 & -\\
 &  & MIA & 75.54 & 81.69 & 36.22 \\
 &  & MIA-rephrased & 73.57 & 79.89 & -\\
\cmidrule{2-6}
& \multirow{3}{*}{{Donut}} & Original & 59.26 & 66.91 & -\\
 &  & MIA & 78.55 & 83.42 & 39.78 \\
 &  & MIA-rephrased & 72.57 & 77.12 & -\\
\cmidrule{2-6}
& \multirow{3}{*}{{Pix2Struct-B}} & Original & 57.11 & 68.13 & -\\
 &  & MIA & 64.42 & 79.95 & 25.8 \\
 &  & MIA-rephrased & 63.81 & 74.06 & -\\
\cmidrule{2-6}
& \multirow{3}{*}{{Pix2Struct-L}} & Original & 64.53 & 74.12 & -\\
 &  & MIA & 73.91 & 82.71 & 22.11 \\
 &  & MIA-rephrased & 69.93 & 79.15 & -\\
\bottomrule
\end{tabular}
\end{adjustbox}
\end{small}
\end{center}
\caption{\textbf{DocVQA Performance of the target models on PFL and DocVQA dataset.} Train-Test Gap is computed as the different of DocVQA Accuracy between \textit{member/non-member} documents. $\textsc{MIA}$ denotes the attack evaluation set, which is a subset randomly sampled from the original train/test set, $\textsc{MIA}\text{-rephrased}$ is its variants with rephrased questions by LLM.}
\label{tab:docvqa_performance}
\end{table}


\subsection{Target Model Performance on DocVQA}
To ensure the utility of the target models for our experiments, we validated that the DocVQA performance of each model checkpoint closely matched the results reported in the respective papers. Table \ref{tab:docvqa_performance}  presents the target models' performance across both DocVQA datasets. We observe a clear train-test performance gap, particularly in smaller models, while the gap tends to narrow for more generalized models or with increased dataset size.

\subsection{Computation and Runtime} All attack methods are implemented using PyTorch and executed on an NVIDIA GeForce A40 GPU with 45 GB of memory. The maximum runtime for each attack does not exceed \textit{10 hours} per run, depending on the target model’s size and the preprocessing steps required for the data. This runtime reflects the efficiency of our approach, especially when compared to methods based on shadow training, which require retraining of large-scale models many times to be effective \citep{carlini2022membership}. Our results demonstrate that the proposed attacks are both efficient and scalable, making them practical for large-scale models in real-world applications.

\section{More on Attack Results}
\begin{table}[h]
\begin{center}
\begin{small}
\begin{adjustbox}{width=1\textwidth}
\small
\begin{tabular}{clcccccc}
\toprule
\multicolumn{2}{c}{\multirow{2}{*}{{White-box}}} & \multicolumn{2}{c}{FL} & \multicolumn{2}{c}{FLLoRA} & \multicolumn{2}{c}{IG}\\
%\cmidrule{3-8}
\cmidrule(l){3-4}
\cmidrule(l){5-6}
\cmidrule(l){7-8}
&                                 & ACC & F1 & ACC & F1 & ACC & F1 \\
\midrule 
\multirow{2}{*}{\rotatebox[origin=c]{90}{PFL}}& {VT5} & $\textbf{66.63}_{\textbf{0.07}}\textbf{\textcolor{red}{(+5.96)}}$ & $\textbf{72.40}_{\textbf{0.1}}\textbf{\textcolor{red}{(+11.73)}}$ & $65.17_{0.0}\textcolor{red}{(+4.50)}$ & $70.52_{0.0}\textcolor{red}{(+9.85)}$ & $61.83_{0.0}\textcolor{red}{(+1.16)}$ & $69.18_{0.0}\textcolor{red}{(+8.51)}$\\

& {Donut} & $72.67_{0.0}\textcolor{red}{(+1.5)}$ & $77.96_{0.0}\textcolor{red}{(+6.63)}$ & $72.83_{0.0}\textcolor{red}{(+1.66)}$ & $77.94_{0.0}\textcolor{red}{(+6.61)}$ & $\textbf{75.17}_{\textbf{0.0}}\textbf{\textcolor{red}{(+4)}}$ & $\textbf{79.39}_{\textbf{0.0}}\textbf{\textcolor{red}{(+8.06)}}$\\
\midrule

\multirow{3}{*}{\rotatebox[origin=c]{90}{DVQA}}& {VT5} & $75.67_{0.0}\textcolor{red}{(+0.5)}$ & $76.60_{0.0}\textcolor{red}{(+1.47)}$ & $75.57_{0.08}\textcolor{red}{(+0.4)}$ & $77.31_{0.13}\textcolor{red}{(+2.18)}$ & $74.83_{0.0}\textcolor[HTML]{0b5394}{(-0.34)}$ & $76.88_{0.0}\textcolor{red}{(+1.75)}$\\

& {Donut} & $80.33_{0.0}\textcolor[HTML]{0b5394}{(-0.17)}$ & $82.18_{0.0}\textcolor{red}{(+1.08)}$ & $80.0_{0.0}\textcolor[HTML]{0b5394}{(-0.5)}$ & $81.93_{0.0}\textcolor{red}{(+0.83)}$ & $80.33_{0.0}\textcolor[HTML]{0b5394}{(-0.17)}$ & $82.34_{0.0}\textcolor{red}{(+1.24)}$\\

& {Pix2Struct-B} & $71.67_{0.0}\textcolor{red}{(+2.54)}$ & $72.22_{0.0}\textcolor{red}{(+4.55)}$ & $70.50_{0.0}\textcolor{red}{(+1.37)}$ & $71.95_{0.0}\textcolor{red}{(+4.28)}$ & $\textbf{72.00}_{\textbf{0.0}}\textbf{\textcolor{red}{(+2.87)}}$ & $\textbf{72.82}_{\textbf{0.0}}\textbf{\textcolor{red}{(+5.15)}}$\\
\bottomrule
\end{tabular}
\end{adjustbox}
\end{small}
\end{center}
\vskip -0.1in
\caption{\textbf{White-Box: Main Results of \atk.} Values in parentheses indicate the improvement (\textcolor{red}{positive}/\textcolor[HTML]{0b5394}{negative}) of our proposed attacks compared to the $\textsc{ScoreLoss-UA}_{\text{all}}$. Compared to all baselines, the methods with the best \textit{average} performance across the two metrics are highlighted in \textbf{bold}. Results are reported over five random seeds.}
\label{tab:whitebox_results}
\end{table}


\label{sec:tpr_at_fix_fpr}
In this section, we evaluate our attacks using TPR@1\%FPR and TPR@3\%FPR, following standard practices in recent MIA literature. The results are summarized in Table~\ref{tab:tpr@fpr_whitebox} and ~\ref{tab:tpr@fpr_blackbox}. 

\begin{table}[t]
\begin{center}
\begin{small}
\begin{adjustbox}{width=0.96\textwidth}
\small
\begin{tabular}{lcccccccccc}
\toprule
 & \multicolumn{6}{c}{\textbf{DVQA}} & \multicolumn{4}{c}{\textbf{PFL}}  \\
\cmidrule(l){2-7}
\cmidrule(l){8-11}
 & \multicolumn{2}{c}{VT5} & \multicolumn{2}{c}{Donut} & \multicolumn{2}{c}{Pix2Struct-B} & \multicolumn{2}{c}{VT5} & \multicolumn{2}{c}{Donut} \\
\cmidrule(l){2-3}
\cmidrule(l){4-5}
\cmidrule(l){6-7}
\cmidrule(l){8-9}
\cmidrule(l){10-11}
 & 1\% & 3\% & 1\% & 3\% & 1\% & 3\% & 1\% & 3\% & 1\% & 3\% \\
\midrule
$\textsc{Loss-TA}$ & \textbf{7.67} & \textbf{14.00} & 0.67 & 7.67 & 2.33 & 5.33 & 0.67 & 3.00 & 1.67 & \textbf{14.67} \\
$\textsc{Gradient-UA}$ & 2.33 & 9.33 & 3.67 & 6.00 & 1.00 & 5.00 & 0.33 & 3.00 & 1.00 & 8.33 \\
$\textsc{ScoreLoss-UA}_{\text{all}}$ & 1.33 & 4.67 & 2.67 & 8.67 & 2.00 & 6.67 & 0.33 & 4.00 & 0.67 & 6.33 \\
Min-K\% & 2.67 & 10.67 & 0.33 & 1.33 & 0.33 & 5.33 & 1.67 & 5.67 & 0.00 & 0.00 \\
Min-K\%++ & 1.00 & 7.00 & \textbf{4.33} & 9.33 & 0.67 & 10.33 & 1.00 & 8.00 & 0.33 & 2.00 \\
\midrule
FL & 2.33 & 5.67 & 3.33 & \textbf{10.67} & \textbf{6.00} & \textbf{11.00} & \textbf{3.67} & \textbf{8.67} & 0.33 & 7.00 \\
FLLoRA & 3.33 & 11.33 & 2.67 & 5.33 & 3.67 & 6.33 & 1.33 & 3.33 & 0.33 & 10.00 \\
IG & 0.67 & 5.67 & 1.33 & 8.00 & 3.00 & 10.33 & 1.00 & 2.33 & \textbf{5.67} & 11.00 \\
\bottomrule
\end{tabular}
\end{adjustbox}
\end{small}
\end{center}
\vskip -0.1in
\caption{\textbf{White-box: TPR at fixed FPR}. Comparison across all white-box methods, with the best-performance highlighted in \textbf{bold}. 1\% and 3\% indicate TPR@1\%FPR and TPR@3\%FPR respectively.}
\label{tab:tpr@fpr_whitebox}
\end{table}


An interesting observation is the high performance of the \textsc{LOSS-TA} method for VT5 on DocVQA and Donut on PFL in Table~\ref{tab:tpr@fpr_whitebox}. This performance can be attributed to the clear separation in the loss distribution between member and non-member samples (Figure~\ref{fig:whitebox_loss}), which indicates overfitting behavior in these cases.

\section{More on Analysis}
\label{sec:more_analysis}
In this section, we provide a deeper analysis of the effectiveness of our proposed white-box and black-box attacks, highlighting their performance relative to the baseline approaches.
\begin{figure}[t]
    \centering
    \begin{minipage}[b]{0.66\textwidth}
        \centering
        \resizebox{0.999\linewidth}{!}{
            \includegraphics[height=5cm,width=1.05\textwidth]{images/whitebox_feature.pdf}
        }
        \caption{\textbf{Membership Features against three different target models on DocVQA Dataset.} \textit{Top}: The distribution of average \textbf{\textit{loss}} over all questions from all target documents on each target model. \textit{Bottom}: T-SNE visualization of the features used in our proposed attacks.}
        \label{fig:whitebox_loss}
    \end{minipage}
    \hfill
    \begin{minipage}[b]{0.3\textwidth}
        \centering
        \includegraphics[height=5cm]{images/whitebox_distance.pdf}
        \caption{\textbf{Distribution of Average \textit{Distance}} from member and non-member documents.}
        \label{fig:whitebox_distance}
    \end{minipage}
\label{fig:whitebox_feature}
\vspace{-0.2in}
\end{figure}


\subsection{Impact of Selected Features}
\label{sec:impact_feature}
As outlined in the main paper, we fix the set of selected features across all experiments. These features include the DocVQA score $u$, the optimization-based distance $\Delta$, and the number of optimization steps $s$, aggregated using the set of aggregation functions $\Phi_\text{all}=\{\textsc{avg}; \textsc{min}; \textsc{max}; \textsc{med}\}$ . We first evaluate the impact of individual features and their combinations on attack performance in the white-box DocMIA setting, using $\textsc{avg}$ as the aggregation function $\Phi$. The analysis employs the best hyperparameters identified during the tuning process described in Section \ref{sec:calibration}.

Table \ref{tab:impact_feature_pfl} and Table \ref{tab:impact_feature_docvqa} summarize the attack performance when individual features or their combinations are used. 
Additionally, Table \ref{tab:impact_feature_more} \textit{(Top)} compares the attack performance of our optimization-based features with the loss value $\ell$ and the gradient norm of the loss with respect to the model parameters $\theta$. Here, the loss value $\ell$ is computed uniformly across all target models over $K$ generation steps, given a (document, question, answer) example $(x,q,a)$ as:
\begin{equation}
    \ell =-\sum^{K}_{k=1} \log{p_{\theta} (a_k|a_{<k},x,q)}
\end{equation}
When used individually, our proposed optimization-based features outperform the DocVQA score and the loss in most cases. Our attack methods are particularly effective against target models like VT5 and Donut trained on PFL-DocVQA, which exhibit lower overfitting and small Train-Test gaps (as shown in Table\ref{tab:docvqa_performance}). These results highlight that our attacks provide more discriminative features than the commonly used MIA features.

When combined, our selected features achieve the best or near-best performance across all cases. Furthermore, extending aggregation functions from $\textsc{avg}$ to $\Phi_\text{all}$ adds notable improvements in attack effectiveness, as shown in Table \ref{tab:impact_feature_more} \textit{(Bottom)}. These results demonstrate that our proposed feature set is robust across different target models, making it a reliable choice for DocMIA.
\begin{table}[t]
    \centering
    \begin{minipage}{0.64\textwidth}
        \centering
        \resizebox{\linewidth}{!}{
            \begin{tabular}{cc}
                \begin{tabular}{cccc}
                    \toprule
                    \multicolumn{4}{c}{VT5} \\
                    \midrule
                    $\textsc{avg}(\textsc{nls})$ & $\textsc{avg}(\Delta)$ & $\textsc{avg}(s)$ & F1\\
                    \midrule 
                    \checkmark &  &  & 68.88\\
                     & \checkmark &  & 71.45\\
                     &  & \checkmark & 70.92\\
                     \midrule
                    \checkmark & \checkmark &  & 71.09\\
                    \checkmark &  & \checkmark & 71.11\\
                     & \checkmark & \checkmark & 71.22\\
                    \midrule
                    \checkmark & \checkmark & \checkmark & {\textbf{71.53}}\\
                    \bottomrule
                    \label{tab:impact_feature_pfl_vt5}
                \end{tabular}
                &
                \begin{tabular}{cccc}
                    \toprule
                    \multicolumn{4}{c}{Donut} \\
                    \midrule
                    $\textsc{avg}(\textsc{nls})$ & $\textsc{avg}(\Delta)$ & $\textsc{avg}(s)$ & F1\\
                    \midrule
                    \checkmark &  &  & 67.58\\
                     & \checkmark &  & 71.36\\
                     &  & \checkmark & 73.16\\
                    \midrule
                    \checkmark & \checkmark &  & 72.87\\
                    \checkmark &  & \checkmark & 73.67\\
                     & \checkmark & \checkmark & 73.86\\
                    \midrule
                    \checkmark & \checkmark & \checkmark & {\textbf{73.89}}\\
                    \bottomrule
                \label{tab:impact_feature_pfl_donut}
                \end{tabular}
            \end{tabular}
        }
        \caption{\textbf{Impact of Selected Features on PFL-DocVQA Models.}}
        \label{tab:impact_feature_pfl}
    \end{minipage}
    \hfill
    \begin{minipage}{0.35\textwidth}
        \centering
        \resizebox{\linewidth}{!}{
            \begin{tabular}{c}
                \begin{tabular}{cccccc}
                    \toprule
                     & \multicolumn{2}{c}{PFL} & \multicolumn{3}{c}{DVQA}\\
                    \cmidrule(l){2-3}
                    \cmidrule(l){4-6}
                     & VT5 & Donut & VT5 & Donut & Pix2Struct-B\\
                    \midrule 
                    $\textsc{avg}(\ell)$ & 67.53 & 67.80 & 73.43 & 56.79 & 69.97\\
                    $\textsc{avg}(||\nabla_{\theta}\mathcal{L}||_2)$ & 70.53 & 71.51 & 71.91 & 71.53 & 66.14\\
                    $\textsc{avg}(\Delta)$ & 71.45 & 71.36 & 72.86 & 57.34 & 70.57\\
                    $\textsc{avg}(s)$ & 70.92 & 73.16 & 74.34 & 60.32 & 69.00\\
                    \bottomrule
                    \\
                    \toprule
                     & \multicolumn{2}{c}{PFL} & \multicolumn{3}{c}{DVQA}\\
                    \cmidrule(l){2-3}
                    \cmidrule(l){4-6}
                     & VT5 & Donut & VT5 & Donut & Pix2Struct-B\\
                    \midrule
                    $\Phi=\textsc{avg}$ & 71.53 & 73.89 & 74.96 & 72.94 & 73.22\\
                    $\Phi=\Phi_{\text{all}}$ & 72.4\textcolor{red}{(+0.87)} & 77.96\textcolor{red}{(+4.07)} & 76.6\textcolor{red}{(+1.67)} & 82.18\textcolor{red}{(+9.24)} & 72.22\textcolor{blue}{(-1.0)}\\
                    \bottomrule							
                \end{tabular}
            \end{tabular}
        }
    \caption{\textbf{Comparisons in Attack Performance in terms of F1 Score}: \textit{(Top)} between our Optimization-based Features with the loss value $\ell$ and the gradient norm $||\nabla_{\theta}\mathcal{L}||_2$. \textit{(Bottom)} between $\textsc{avg}$ and $\Phi_{\text{all}}$ as the aggregation functions.}
    \label{tab:impact_feature_more}
    \end{minipage}
\end{table}

\begin{table}[t]
    \centering
    \begin{minipage}{0.32\textwidth}
        \centering
        \resizebox{\linewidth}{!}{%
            \begin{tabular}{cccc}
                \toprule
                \multicolumn{4}{c}{VT5} \\
                \midrule
                $\textsc{avg}(\textsc{nls})$ & $\textsc{avg}(\Delta)$ & $\textsc{avg}(s)$ & F1\\
                \midrule 
                \checkmark &  &  & 72.73\\
                 & \checkmark &  & 72.86\\
                 &  & \checkmark & 74.34\\
                 \midrule
                \checkmark & \checkmark &  & \textbf{75.81}\\
                \checkmark &  & \checkmark & 75.04\\
                 & \checkmark & \checkmark & 74.19\\
                \midrule
                \checkmark & \checkmark & \checkmark & 74.96\\
                \bottomrule
            \end{tabular}
        }
        \label{tab:impact_feature_docvqa_vt5}
    \end{minipage}
    \begin{minipage}{0.32\textwidth}
        \centering
        \resizebox{\linewidth}{!}{%
            \begin{tabular}{cccc}
                \toprule
                \multicolumn{4}{c}{Donut} \\
                \midrule
                $\textsc{avg}(\textsc{nls})$ & $\textsc{avg}(\Delta)$ & $\textsc{avg}(s)$ & F1\\
                \midrule 
                \checkmark &  &  & \textbf{76.88}\\
                 & \checkmark &  & 57.34\\
                 &  & \checkmark & 60.32\\
                \midrule
                \checkmark & \checkmark &  & 65.94\\
                \checkmark &  & \checkmark & 72.17\\
                 & \checkmark & \checkmark & 60.29\\
                \midrule
                \checkmark & \checkmark & \checkmark & 72.94\\
                \bottomrule
            \end{tabular}
        }
        \label{tab:impact_feature_docvqa_donut}
    \end{minipage}
    \begin{minipage}{0.32\textwidth}
        \centering
        \resizebox{\linewidth}{!}{%
            \begin{tabular}{cccccc}
                \toprule
                \multicolumn{4}{c}{Pix2Struct-B} \\
                \midrule
                $\textsc{avg}(\textsc{nls})$ & $\textsc{avg}(\Delta)$ & $\textsc{avg}(s)$ & F1\\
                \midrule 
                \checkmark &  &  & 72.60\\
                 & \checkmark &  & 70.57\\
                 &  & \checkmark & 69.00\\
                \midrule
                \checkmark & \checkmark &  & 73.20\\
                \checkmark &  & \checkmark & 72.87\\
                 & \checkmark & \checkmark & 70.17\\
                \midrule
                \checkmark & \checkmark & \checkmark & \textbf{73.22}\\
                \bottomrule
            \end{tabular}
        }
        \label{tab:impact_feature_docvqa_pix2struct}
    \end{minipage}
\caption{\textbf{Impact of Selected Features on DocVQA Target Models}. Only $\textsc{AVG}$ is used as the aggregation function $\Phi$. Attack performances are obtained with our \textsc{FL} method using the best hyperparameters.}
\label{tab:impact_feature_docvqa}
\end{table}
\subsection{Impact of the Training Questions Knowledge}
\label{sec:impact_question_knowledge}
So far, our document MI attacks against DocVQA models have assumed complete knowledge of the original training questions. We now relax this assumption and investigate how the lack of access to the exact training questions affects attack performance. In practice, an adversary may not have access to the exact training questions but can approximate them. For example, documents like invoices often follow standard layouts, and biases in human annotation may lead to predictable patterns in the types of questions asked during the creation of DocVQA datasets \citep{tito2024privacy,mathew2021docvqa}. It is important to note that the original training questions tend to be simple, natural questions designed to extract specific information from the document. Moreover, the type of question is inherently linked to the type of document on which the DocVQA model is trained. For instance, if the target model is trained on invoices, the natural type of question would focus on extracting essential details from the invoice, such as the “total amount”, framed in a clear and straightforward manner e.g., "What is the total?".
This makes it possible for an adversary to generate approximate versions of the training questions, simulating a more realistic attack setting.
\begin{table}[t]
\begin{center}
\begin{small}
\begin{adjustbox}{width=1\textwidth}
\small
\begin{tabular}{clcccccccccccc}
\toprule
 & \multicolumn{1}{c}{\multirow{2}{*}{\textbf{Target}}} & \multicolumn{8}{c}{\textbf{DVQA}} & \multicolumn{4}{c}{\textbf{PFL}} \\
\cmidrule(l){3-10}
\cmidrule(l){11-14}
 &  & \multicolumn{2}{c}{\textbf{VT5}} & \multicolumn{2}{c}{\textbf{Donut}} & \multicolumn{2}{c}{\textbf{Pix2Struct-B}} & \multicolumn{2}{c}{\textbf{Pix2Struct-L}} & \multicolumn{2}{c}{\textbf{VT5}} & \multicolumn{2}{c}{\textbf{Donut}} \\
\cmidrule(l){3-4}
\cmidrule(l){5-6}
\cmidrule(l){7-8}
\cmidrule(l){9-10}
\cmidrule(l){11-12}
\cmidrule(l){13-14}
Proxy &  & 1\% & 3\% & 1\% & 3\% & 1\% & 3\% & 1\% & 3\% & 1\% & 3\% & 1\% & 3\%  \\
\midrule
& {$\textsc{Score-TA}$} & 4.00 & 9.33 & 5.00 & 11.00 & \textbf{5.33} & 8.00 & 3.33 & \textbf{9.00} & 1.00 & 5.00 & 0.67 & 2.67 \\
& {$\textsc{Score-UA}$} & 3.67 & 7.67 & 4.33 & 15.67 & 4.00 & 6.33 & 4.33 & 6.67 & 0.67 & 3.33 & 0.33 & 3.33 \\
& {$\textsc{Score-UA}_{\text{all}}$} & 4.00 & 9.33 & 5.00 & 11.00 & 5.33 & 8.00 & 3.33 & 9.00 & 1.00 & 5.00 & 0.67 & 2.67 \\
\midrule
\multirow{3}{*}{VT5} & FL & 0.67 & \textbf{12.33} & \textbf{11.67} & \textbf{23.00} & 2.00 & \textbf{16.67} & 2.00 & 5.33 & 0.67 & 2.00 & \textbf{5.00} & \textbf{8.00} \\
& FLLoRA & \textbf{4.67} & {11.33} & 6.34 & 16.33 & 2.33 & 9.33 & 1.00 & 4.67 & 2.00 & 3.33 & 0.00 & 2.00 \\
& IG & 1.00 & 8.33 & 2.00 & 7.00 & 4.67 & 7.67 & 2.33 & 7.00 & 0.33 & 3.67 & 1.33 & 6.67 \\
\cmidrule(l){1-14}
\multirow{3}{*}{Donut} & FL & 0.33 & 6.33 & 0.33 & 4.00 & 1.33 & 4.67 & 3.00 & 7.33 & 0.33 & 1.33 & 1.33 & 4.00 \\
& FLLoRA & 1.00 & 6.33 & 1.67 & 5.00 & 2.33 & 6.33 & 3.00 & 8.00 & 0.00 & 5.33 & 2.00 & 5.33 \\
& IG & 1.67 & 5.00 & 0.67 & 11.00 & 3.67 & 9.33 & \textbf{4.67} & 6.33 & \textbf{2.67} & \textbf{6.33} & 1.67 & 4.33 \\
\bottomrule
\end{tabular}
\end{adjustbox}
\end{small}
\end{center}
\vskip -0.1in
\caption{\textbf{Black-box: TPR at fixed FPR}. Comparison across all black-box methods, with the best-performing method highlighted in \textbf{bold}. 1\% and 3\% indicate TPR@1\%FPR and TPR@3\%FPR respectively.}
\label{tab:tpr@fpr_blackbox}
\end{table}

\begin{table}[h]
\vskip 0.15in
\begin{center}
\begin{small}
\begin{adjustbox}{width=1\textwidth}
\small
\begin{tabular}{clcccccccccc}
\toprule
\multicolumn{2}{c}{\multirow{2}{*}{Model}} & \multicolumn{2}{c}{$\textsc{Score-TA}$} & \multicolumn{2}{c}{$\textsc{Score-UA}_{\text{all}}$} & \multicolumn{2}{c}{$\textsc{Loss-TA}$} & \multicolumn{2}{c}{$\textsc{ScoreLoss-UA}_{\text{all}}$} & \multicolumn{2}{c}{$\textsc{Ours (FL)}$}\\
\cmidrule(l){3-4}
\cmidrule(l){5-6}
\cmidrule(l){7-8}
\cmidrule(l){9-10}
\cmidrule(l){11-12}
&                                 & ACC & F1 & ACC & F1 & ACC & F1 & ACC & F1 & ACC & F1 \\
\midrule 
\multirow{2}{*}{\rotatebox[origin=c]{90}{\tiny PFL}} & VT5 & \cellcolor[HTML]{C0C0C0}{$60.67$} & \cellcolor[HTML]{C0C0C0}{$64.13$} & \cellcolor[HTML]{C0C0C0}{$55.83_{0.0}$} & \cellcolor[HTML]{C0C0C0}{$46.89_{0.0}$} & $54.50$ & $59.19$ & $55.83_{0.0}$ & $46.89_{0.0}$ & $\textbf{\textcolor{red}{64.00}}_{\textbf{\textcolor{red}{0.0}}}$ & ${\textbf{\textcolor{red}{69.14}}}_{\textbf{\textcolor{red}{0.0}}}$\\

& {Donut} & \cellcolor[HTML]{C0C0C0}{$69.17$} & \cellcolor[HTML]{C0C0C0}{$69.72$} & \cellcolor[HTML]{C0C0C0}{$59.33_{0.0}$} & \cellcolor[HTML]{C0C0C0}{$51.59_{0.0}$} & $68.50$ & $66.67$ & $59.17_{0.0}$ & $51.49_{0.0}$ & $\textbf{\textcolor{red}{71.13}}_{\textbf{\textcolor{red}{0.08}}}$ & $\textbf{\textcolor{red}{72.07}}_{\textbf{\textcolor{red}{0.0}}}$\\
\midrule

\multirow{2}{*}{\rotatebox[origin=c]{90}{\tiny DVQA}} & VT5 & \cellcolor[HTML]{C0C0C0}{$73.67$} & \cellcolor[HTML]{C0C0C0}{$75.01$} & \cellcolor[HTML]{C0C0C0}{$74.83_{0.0}$} & \cellcolor[HTML]{C0C0C0}{$74.36_{0.0}$} & $71.67$ & $74.06$ & $75.17_{0.0}$ & $74.96_{0.0}$ & $\textbf{\textcolor{red}{74.83}}_{\textbf{\textcolor{red}{0.0}}}$ & $\textbf{\textcolor{red}{75.68}}_{\textbf{\textcolor{red}{0.0}}}$\\
& {Donut} & \cellcolor[HTML]{C0C0C0}{\textbf{\textcolor{red}{69.17}}} & \cellcolor[HTML]{C0C0C0}{$\textbf{\textcolor{red}{71.23}}$} & \cellcolor[HTML]{C0C0C0}{$65.17_{0.0}$} & \cellcolor[HTML]{C0C0C0}{$62.21_{0.0}$} & $52.33$ & $53.57$ & $65.17_{0.0}$ & $62.21_{0.0}$ & $67.67_{0.0}$ & $68.51_{0.0}$\\
\bottomrule
\end{tabular}
\end{adjustbox}
\end{small}
\end{center}
\vskip -0.1in
\caption{\textbf{Results with Rephrased Questions.} \textcolor{gray}{\textbf{Gray}} color indicate attacks conducted in the black-box setting. All results are reported based on five random seeds. The methods with the best \textit{average} performance across the two metrics are highlighted in \textbf{\textcolor{red}{bold}}.}
\label{tab:rephrased_question_results}
\end{table}


To explore this scenario, we conduct experiments where we paraphrase the original training questions using Mistral~\citep{jiang2023mistral}, and use these rephrased questions as inputs for the MI attacks. As illustrated in Table \ref{tab:rephrased_question_results}, the performance of all MI attacks declines when rephrased questions are used, mirroring the drop in DocVQA model performance (Table \ref{tab:docvqa_performance}), which is expected due to the increased uncertainty introduced by question rephrasing.

Among the baselines, the $\textsc{SCORE-TA}$ attack proves particularly to be robust, especially against models trained on DocVQA, which show a higher degree of overfitting. In contrast, attacks incorporating loss-based signals introduce additional noise due to uncertainty, leading to a noticeable drop in performance.

Despite the rephrasing, our attacks remain effective, maintaining performance levels comparable to those observed with the original questions, especially against the two PFL models, which demonstrate a lower degree of overfitting.

We also evaluate our proposed attacks against other methods in this setting, focusing on TPR at 1\% and 3\% FPR, with the results summarized in Table \ref{tab:tpr@fpr_rephrased_question_whitebox} and \ref{tab:tpr@fpr_rephrased_question_blackbox}.

\subsection{The resulting Proxy Model}
\label{sec:impact_proxy_model}
The purpose of training the Proxy Model on $D_{\text{query}}$, with labels generated by the black-box model, is to mimic the prediction patterns of the black-box model. The expectation is that the proxy model can capture internal decision-making patterns by following the black-box's prediction strategies. Instead of optimizing for ground-truth labels, we train the proxy to maximize the likelihood of the generated labels. The training process concludes when the proxy achieves near-zero training loss, at which point the final checkpoint is used for the attack.
\begin{figure}[h]
    \centering
    \subfigure[\centering \textbf{Training curve}]{
        \includegraphics[height=2.5cm,width=0.45\textwidth]{images/proxy_loss_curve.pdf}
        \label{fig:proxy_model_train}
    }
    \subfigure[\centering \textbf{Distribution over distance}]{
        \includegraphics[height=2.5cm,width=0.45\textwidth]{images/proxy_distance_distribution.pdf}
        \label{fig:proxy_model_distance_distribution}
    }
\caption{\textbf{The resulting Proxy Model} against Pix2Struct-B in the black-box setting. (\textit{a}) The attack accuracy improves quickly once the loss reaches near zero. (\textit{b}) The optimization distance values between member and non-member documents exhibit a separation similar to that seen in the white-box setting.}
\label{fig:proxy_model}
\end{figure}

As illustrated in Figure \ref{fig:proxy_model_train}, the attack performance quickly improves as training progresses. The model overfits quickly, with attack performance reaching its peak early—after just a quarter of the training process—demonstrating the efficiency of our approach. This suggests that \textit{once the proxy model converges, it has effectively captured informative membership signals from the black-box model}, making it ready for the attack. Moreover, we compare the distribution of optimization distances between the proxy model and the same model in the white-box setting, as shown in Figure \ref{fig:proxy_model_distance_distribution}. The results show a similar degree of separation between the two clusters in both cases, indicating the proxy model's effectiveness in approximating the black-box model's behavior to a certain extent.

\subsection{Attack Performance against Minimal-Training Documents}
DocVQA models typically process each question-answer pair independently, resulting in multiple exposures of each document during training. This increases the likelihood of being memorized by the model, making such documents more vulnerable to MIAs. Intuitively, documents associated with fewer training questions should be less exposed and therefore be less vulnerable.

\begin{table}[h]
\begin{center}
\begin{small}
\begin{adjustbox}{width=1\textwidth}
\small
\begin{tabular}{clcccccccc}
\toprule 							
& & \multicolumn{4}{c}{$\textsc{FL}$} & \multicolumn{4}{c}{$\textsc{IG}$}\\
\cmidrule(l){3-6}
\cmidrule(l){7-10}
\multirow{3}{*}{\rotatebox[origin=c]{90}{\textbf{PFL}}} & Model & $m=1$(1) & $m=2$(1) & $m=3$(85) & $\textsc{ALL}$(300) & $m=1$(1) & $m=2$(1) & $m=3$(85) & $\textsc{ALL}$(300)\\
\cmidrule{2-10}
& VT5 & 0 & 0 & 83.53 & 87.67 & 100 & 100 & 85.88 & 86.33\\
& Donut & 100 & 100 & 100 & 97.67 & 100 & 100 & 97.65 & 97\\
\midrule	
\multirow{4}{*}{\rotatebox[origin=c]{90}{\textbf{DVQA}}} & Model & $m=1$(51) & $m=2$(60) & $m=3$(52) & $\textsc{ALL}$(300) & $m=1$(51) & $m=2$(60) & $m=3$(52) & $\textsc{ALL}$(300)\\

\cmidrule{2-10}
& VT5 & 86.27 & 71.67 & 84.62 & 77.00 & 90.2 & 85 & 86.54 & 80.67\\
& Donut & 88.24 & 73.33 & 76.92 & 77.33 & 56.86 & 68.33 & 55.77 & 61.33\\
& Pix2Struct-B & 90.2 & 93.33 & 90.38 & 87 & 88.24 & 88.33 & 76.92 & 73\\		
\bottomrule
\end{tabular}
\end{adjustbox}
\end{small}
\end{center}
\caption{\textbf{Membership Prediction Accuracy on \textit{Member} Documents with minimal repetition.} $m$ denotes the subset of testing documents with $m$ \textit{training} questions, with subset sizes shown in parentheses. Compared to the performance measured on the entire member set (denoted as $\textsc{ALL}$), our attacks are still robust against documents with the low risk of memorization.}
\label{tab:min_repeat_results}
\end{table}


To evaluate this, we measure the accuracy of membership predictions from our attacks on a subset of \textit{member} documents in $D_{\text{test}}$ associated with only a few training questions. These documents represent a minimal memorization risk, posing a more challenging evaluation scenario. Results in Table \ref{tab:min_repeat_results} show that our attacks remain effective on these subsets, achieving high accuracy even for documents $m=1$ training question. This demonstrates the robustness of our attacks under conditions of minimal repetition.

\section{Defenses}

To mitigate the privacy vulnerabilities associated with membership inference attacks in Document Visual Question Answering (DocVQA) systems, we can employ Differential Privacy (DP) techniques~\citep{dwork2014algorithmic}, specifically through the use of differentially private stochastic gradient descent (DP-SGD) introduce by \citet{abadi2016deep}. DP is a robust framework that ensures an individual's data contribution cannot be inferred, even when an adversary has access to the model's outputs. DP-SGD achieves this by adding calibrated noise to the model's gradients during training, thus providing strong theoretical privacy guarantees. However, this approach is not without its drawbacks; the necessity of noise injection can adversely affect the utility of the trained model, leading to reduced performance in answering queries accurately. Alternatively, we can consider ad-hoc solutions such as limiting the number of queries to one question per document in black-box setting, which would inherently reduce the model's usability and flexibility in practical applications. While these measures can enhance privacy, they also necessitate careful consideration of the balance between privacy protection and the functionality of DocVQA systems.

To evaluate the robustness of our proposed membership inference attacks against Differential Privacy (DP), we implemented the well-known DP-SGD algorithm. We considered five privacy budget $\varepsilon \in \{8, 32\}$, with corresponding noise multiplier $\sigma \in \{0.5767822266, 0.3824234009\}$, respectively. 
% 1.279296875, 
The composition of the privacy budget over multiple iterations was calculated using Rényi Differential Privacy (RDP). We then converted the RDP guarantees into the standard $(\varepsilon,\delta)$-DP notion following the conversion theorem from \citep{balle2020hypothesis}.

\begin{table}[t]
    \centering
    \begin{minipage}{0.56\textwidth}
        \centering
        \resizebox{\linewidth}{!}{
            \begin{tabular}{lcccccccc}
            \toprule
             & \multicolumn{4}{c}{\textbf{DVQA}} & \multicolumn{4}{c}{\textbf{PFL}}  \\
            \cmidrule(l){2-5}
            \cmidrule(l){6-9}
             & \multicolumn{2}{c}{VT5} & \multicolumn{2}{c}{Donut} & \multicolumn{2}{c}{VT5} & \multicolumn{2}{c}{Donut} \\
            \cmidrule(l){2-3}
            \cmidrule(l){4-5}
            \cmidrule(l){6-7}
            \cmidrule(l){8-9}
             & 1\% & 3\% & 1\% & 3\% & 1\% & 3\% & 1\% & 3\% \\
            \midrule
            Min-K\% & 3.00 & 4.33 & 0.33 & 1.00 & \textbf{6.33} & \textbf{20.33} & 2.00 & 2.33 \\
            Min-K\%++ & 3.00 & 4.67 & 0.00 & 2.67 & 6.33 & 10.00 & 0.00 & 7.00 \\
            \midrule
            FL & 0.67 & 5.00 & \textbf{3.33} & \textbf{8.00} & 3.67 & 17.33 & 3.00 & 4.67 \\
            FLLoRA & \textbf{5.00} & \textbf{9.33} & 0.67 & 3.67 & 5.00 & 9.33 & \textbf{4.33} & \textbf{10.00} \\
            IG & 5.33 & 8.00 & 1.00 & 5.00 & 5.33 & 8.00 & 1.67 & 10.00 \\
            \bottomrule
            \end{tabular}
        }
        \caption{\textbf{White-box Results: TPR at 1\% and 3\% FPR with Rephrased Questions}. Comparison to \textit{white-box} methods: Min-K\% and Min-K\%++ methods, with the best method in \textbf{bold}.}
        \label{tab:tpr@fpr_rephrased_question_whitebox}
    \end{minipage}
    \hfill
    \begin{minipage}{0.43\textwidth}
        \centering
        \resizebox{\linewidth}{!}{
            \begin{tabular}{lcccccc}
             \toprule
             & \multicolumn{2}{c}{\textbf{PFL}} & \multicolumn{4}{c}{\textbf{DVQA}} \\
             \cmidrule(l){2-3}
             \cmidrule(l){4-7}
             & \multicolumn{2}{c}{VT5} & \multicolumn{2}{c}{Donut} & \multicolumn{2}{c}{Pix2Struct-B} \\
             \cmidrule(l){2-3}
             \cmidrule(l){4-5}
             \cmidrule(l){6-7}
             & 1\% & 3\% & 1\% & 3\% & 1\% & 3\% \\
            \midrule
            $\textsc{Score-TA}$& 0.33 & 2.67 & \textbf{3.33} & 9.67 & 3.00 & 8.67 \\
            $\textsc{Score-UA}_{\text{all}}$& 0.33 & 2.67 & 2.33 & 9.33 & \textbf{4.67} & 8.67 \\
            \midrule
            FL & 0.33 & 1.33 & 0.33 & 4.00 & 1.33 & 4.67 \\
            FLLoRA & 1.00 & 5.33 & 1.67 & 5.00 & 2.33 & 6.33 \\
            IG & \textbf{2.67} & \textbf{6.33} & 1.67 & \textbf{11.00} & 3.67 & \textbf{9.33} \\
            \bottomrule
            \end{tabular}
        }
    \caption{\textbf{Black-box Results: TPR at 1\% and 3\% FPR with Rephrased Questions}. Donut is used as The Proxy Model.}
    \label{tab:tpr@fpr_rephrased_question_blackbox}    
    \end{minipage}
    \vskip -0.2in
\end{table}





We trained the Donut model on the DocVQA dataset with DP-SGD to provide theoretical privacy guarantees for individual training documents. Due to resource constraints, we resized document resolution to a smaller size (1280, 960) compared to (2560, 1920) in the public checkpoint  provided by the original authors, which slightly reduced the model's DocVQA performance. For additional details on the effects of document resolution, we refer readers to the original model's paper\citep{Kim22Donut}. The model was trained using the Adam optimizer with a learning rate of $1e-4$, for 10 epochs, and with a batch size of 16. DocVQA performance was evaluated using the Average Normalized Levenshtein Similarity (ANLS) metric.

\begin{table}[h]
\begin{center}
\begin{small}
\begin{adjustbox}{width=1\textwidth}
\small
\begin{tabular}{lcccccccccccc}
\toprule
& \multicolumn{3}{c}{$\varepsilon=8$} & \multicolumn{3}{c}{$\varepsilon=32$} & \multicolumn{3}{c}{$\varepsilon=\infty$} \\
\cmidrule(l){2-4}
\cmidrule(l){5-7}
\cmidrule(l){8-10}
 & ANLS & F1 & TPR@3\%FPR & ANLS & F1 & TPR@3\%FPR & ANLS & F1 & TPR@3\%FPR \\
\midrule
FL & \multirow{3}{*}{19.16} & 55.09 & 2.33 & \multirow{3}{*}{21.81} & 58.84 & 4.33 & \multirow{3}{*}{50.12} & 73.81 & 7.33 \\
FLLoRA &  & 54.94 & 2.00 &  & 58.94 & 3.67 &  & 73.81 & 7.33 \\
IG &  & 56.29 & 1.67 &  & 59.35 & 5.00 &  & 73.52 & 8.67 \\
\bottomrule
\end{tabular}
\end{adjustbox}
\end{small}
\end{center}
\vskip -0.1in
\caption{\textbf{DocMIA Results for Donut trained with DP-SGD on DocVQA dataset}. We report the attack performance of our FL method in terms of F1 score and TPR3\%FPR.}
\label{tab:dp_whitebox}
\end{table}


Table \ref{tab:dp_whitebox} summarizes the results. As expected, introducing DP into model training significantly reduces the attack performance, for example from 73.81\% F1 score with non-DP model to 55.09\% at $\varepsilon = 8$, but this comes at the cost of substantial utility degradation, with the DP model achieving less than half of the performance of the non-DP model, 21.81 of ANLS at $\varepsilon = 8$ compared to 50.12 of ANLS from non-DP checkpoint.
For higher privacy budgets ($\varepsilon = 32$), our attacks demonstrate improved effectiveness, achieving notable gains, +3.75 in F1 and +2 in TPR3\%FPR scores compared to $\varepsilon = 8$, as the model becomes less privacy-constrained.

\end{document}
