%\qz{this section should contain definitions not appeared in Intro. Essentially it expands the problem space description, presents the design goal, and shed light onto our solution}

\label{sec:concept}
%\begin{figure*}
%    \centering
%    \includegraphics[width=.9\textwidth]{figures/ai_bridged_cla.pdf}
%    \caption{\qz{this is a bit confusing - I thought outline is also generated by AI system on the right? and on the left - author creates a narrative space rather than narrative example even just using traditional tools... }\yw{I think the emphasis here is that the author creates concrete storylines vs. abstract outline. In both cases they are creating narrative space. "Narrative example" is a misleading term. The middle diagram is describing existing creation paradigm for AI-bridged IN, not ours.  }}
%    \label{fig:multipivot}
%    \Description{}
%\end{figure*}

% [story vs. narrative. plot]


% Describe the distinct properties of a narrative space (game plot): a graph full of events. Complexity of this graph. Two types of authorial intent: procedural intent and declarative intent. 

% [abstraction]

% Three ways to represent narrative space:

% - Narrative Instance: a sample path in the space, good to show the details and procedural intent

% - Narrative Variant: a sub-plot in the space, good to perceive the diff and variations

% - Narrative Outline: high-level fuzzy description of important events; easy to see the authorial intent; hard to perceive details etc.

% \paragraph{From Narrative Instance to Narrative Space (or Executable Game-plot)}

% \paragraph{From Narrative Variants to Narrative Space (or Executable Game-plot)}

% \paragraph{From Narrative Outline to Narrative Space (or Executable Game-plot)}


%\subsection{Current Paradigm of AI-bridged Interactive Narrative Creation}

%\yw{probably should define narrative space more explicitly}
%\yw{it's not easy to grasp the connection between ``outline'' and ``prompt''}

%The creation of interactive narratives has moved beyond traditional paradigms where authors directly write content for players. Instead, with AI represented by LLMs, creators now develop a "narrative space," the conceptual layer based on which AI can further do creations to produce the final artifact ~\cite{moreno2007documental, gmeiner2023dimensions,lee2024navigating,wu2024role,margineda2024development}. Approaches such as setting personas for characters and driving AI behavior based on these personas (e.g., AI Village ~\cite{park2023generative}), or evolving stories within predefined game settings (e.g., AI Dungeon ~\cite{ai_dungeon2}), all exemplify this paradigm of AI-bridged interactive narrative creation. 


%Interactive Narrative allows players to influence storylines through their actions, with authors creating a {\em narrative space} of possible storylines. AI-bridged interactive narrative generates the narrative space using prompts, enhancing player agency and enabling emergent narratives beyond predefined branches.

%A clear demonstration of this paradigm involves using LLMs (e.g., ChatGPT) to function as a game engine for a text adventure game based on a narrative space revolving around an existing story. Specifically, an outline prompt will be provided to describe the narrative space, such as:

%\whosays{Act as a game engine that turns the following story into a text-adventure game. [Story Outline] The narrative flow should emulate the pacing and events of the story as closely as possible, ensuring that choices do not prematurely advance the plot. Pause the game after 5 rounds. After that...~\cite{10333140}}

%In this setup, the narrative space is anchored to the provided story with this outline determining its boundary. During gameplay, the player might input actions like "go to the mountain, and explore" via text, to which the AI responds dynamically to make the entire narrative aligned with the outline. For instance, if "the deer meets with the player" is specified in the outline, the AI arranges appropriate events, such as "The deer also moves to the mountain," to facilitate this outcome. Therefore, the outline directs the AI creation, defining the narrative space by specifying key events while allowing variations.  Through these interactions, the AI and player collaboratively unfold the narrative within the space.

%To obtain an ideal setting of the narrative space, creators need to put effort into refining the outline in the prompt. To do so, a common strategy for novice creators is to start with a concrete narrative example ~\cite{tomlinson2006learning,micallef2024use}. Based on the example, they can define features of the narrative space like how the play-time narrative can deviate from the example using specifications (e.g., "keep the original pace"), or identifying key moments (e.g., "this character dies after the first battle."). With iteratively generating examples, creators extract the desirable or undesirable elements and summarize them as rules for the outline to guide the narrative space. For instance, a rule might be "killing characters happens after the first scene." After numerous trials refining prompts, creators may feel they have clearly established the boundaries of the narrative space. At this point, creators may further test the game by anticipating potential player actions in gameplay. Then, despite creators' initial efforts at refinement, exceptions may still arise and make the narrative deviate from their intent. For instance, creators may not know that LLM could use implicit action "attack" and lead to the unexpected death of characters, or the abandonment of killing action makes the narrative fail to describe a key scene. Alternatively, a player might kill an important character in the first scene, disrupting the intended storyline. This leads to further rounds of revisions until the creator is again satisfied with the defined narrative space. Even so, creators may still lack a clear understanding of potential outcomes in actual gameplay.

%This highlights the issues lying in the mutual transformation between the narrative example and outline in AI-bridged interactive narrative creation. First, creators typically generate narrative examples and use them to derive descriptions for the outline to define the narrative space——both of which are developed heuristically. As a result, the outline that bound the narrative space is obtained through a ruleless trial-and-error process, which is inherently \textbf{intractable}. Moreover, the outline is gradually concretized by AI into narrative examples, which are further influenced by players' actions during gameplay. In this process, creators often lack a clear mental model of the AI's or the player's potential behaviors, making it difficult to set effective boundaries for the narrative space. This leaves the AI-driven creation process largely \textbf{uncontrollable}.


% Currently, large language models (LLMs) used in game development usually simply use LLMs to enrich the game content. Figures 1.A and 1.B show two examples of widely used AI-powered games. In Figure 1.A (Co-generation of Game Plots), without authors' specification on a concrete story, users and the LLM model alternately generate plot content. Rather than functioning strictly as a game, this setup resembles a collaborative writing task, where both the player and the LLM jointly contribute to script writing. In such a mode, the game creator's expression makes a minimal impact. In Figure 1.B (LLM-Generated Character Behavior), LLMs are used to enhance character behavior, particularly by adding vividness and detail to the language of the characters in the game plots. However, in this case, the game narrative remains fixed, with LLMs enriching the expression of pre-defined storylines. Despite of action choices provided to players in the game, this interaction often creates only the illusion of influencing the story, as the underlying narrative remains unaffected. While players may interact with characters by, for example, typing “hello,” their actions have no meaningful impact on the broader narrative.


% However, an ideal narrative-driven game should balance authorial intent and interactivity. Authorial intent refers to the narrative the author aims to convey, while interactivity refers to the player’s ability to meaningfully influence the story. The goal is to enable players to alter the narrative in impactful ways while maintaining a coherent story with clear moral expression from the author. However, neither is the approach in Figure 1. A nor Figure 1. B is capable of fully achieving this balance. 


% In interaction mode like AI Dungeon, players' interactivity is largely freeform, but there is little to no authorial intent guiding the experience. This makes it difficult for an author to predict or control what players will experience. On the other hand, using LLMs to enhance character behavior without altering the core storyline leaves the narrative largely unchanged. 

% Both interaction modes that leverage LLMs nowadays to create narrative-driven games overlook a critical aspect of typical game creation: the goal is not to generate a specific story but rather to create a narrative space. A narrative space encompasses all the possible ways a story can evolve while maintaining the same core theme or moral. For example, in a story centered around the theme of kindness, one version might depict an ant saving a dove in a forest, while another could feature the dove saving the ant. Despite the differences in these versions, both convey the same moral—kindness. By creatively arranging story elements, game creators can unfold multiple stories within a single narrative space. Typically, such work involves extensive manual planning by game creators, who write multiple branches and versions of the narrative to explore different storylines. Without addressing this concept of a narrative space, LLM-powered systems either lock players into a fixed story or leave the narrative highly open-ended, lacking the necessary authorial guidance. The challenge lies in designing LLM systems that allow for meaningful interactivity while preserving the author’s intent within a flexible narrative space, and effectively unfolding the narrative space into game plots.

% [draft, not satisfied.]







%\subsection{Shaping Narrative Space for AI-bridged Game Plot Generation}

\label{design_goals}

Interactive Narrative allows players to influence storylines through their actions, with authors creating a narrative space of possible storylines. \revision{During the authoring process, authors predefine the range of player actions and creates multiple storylines reflecting the consequences of different player choices.} AI-bridged IN generates just-in-time narrative content that adapts to different game world states, freeing authors from enumerating storylines. 
%In this workflow, authors prompt the AI model with an abstract narrative specifications, such as: 
%\whosays{Act as a game engine that turns the following story into a text-adventure game. [Story Outline] The narrative flow should emulate the pacing and events of the story as closely as possible, ensuring that choices do not prematurely advance the plot. Pause the game after 5 rounds. After that...~\cite{playbrary}}
However, shifting from traditional IN to AI-bridged IN presents challenges for authors in expressing, perceiving, and controlling the narrative space. Authors often struggle to articulate their implicit narrative intents in high-level prompts \cite{mirowski2023co} and may underexpress their intent to AI systems \cite{kreminski2024intent}. While novice authors might start with a concrete example \cite{tomlinson2006learning,micallef2024use}, a single narrative instance can be both overly detailed and insufficient, as it includes unnecessary specifics and lacks broader context \cite{kreminski2024intent}. Therefore, neither concrete instances nor abstract specifications alone are ideal for defining a narrative space. Instead, the ability to configure the level of abstraction is necessary to support AI-bridged IN authoring. 

On the other hand, once a narrative space is defined via prompts, the author has limited insight into the player experience, as players are responded with unscripted character actions and dialogs generated by LLMs at play-time. It is difficult to identify and prevent the deviations beyond the the author's narrative intent. Therefore, it is important to provide valuable information on how different types of player could react to instances, through which authors could preview the narrative instances as they get unfolded in the player experience \cite{kim2023language}. Furthermore, transforming prompts into meaningful game plots is not trivial. It requires effective narrative planning to generate causally sound event sequences. Central to this success is modeling the logical causal progression of the game plot \cite{riedl2010narrative}. However, LLMs are not natively planners in creating causal progression and have been found to cause hallucinations without external verifier to validate the coherency and executability of the generated plan \cite{kambhampati2024llms}. 
%Building on existing research in interactive narrative, we examined the current AI-bridged creation paradigm and identified several key challenges. The first issue is the difficulty in defining appropriate boundaries for the narrative space. Currently, the process of generating narrative examples and summarizing them into higher-level abstractions is largely heuristic, relying on intuitive thinking. However, constructing a narrative space—particularly through language-based boundaries like an outline—requires thorough thinking, and involves careful organization of language and structure. 
%Another challenge lies in understanding the boundaries of AI’s role within the narrative space. Creators need a clear mental model of how AI operates within this space to guide their creation. Without this understanding, there can be a disconnection between the creator’s intent and the final artifact built by the AI. Finally, the unpredictable behaviors of players interacting with the narrative add another layer of complexity~\cite{marincioni2024effect,peng2024player,you2024dungeons}. As player choices influence the AI's ongoing narrative development, unexpected outcomes can arise, making it difficult to maintain coherence in the story. This uncertainty makes it harder for creators to ensure a controllable narrative experience. 
% : (1) the challenges of understanding and editing "narrative space" in creating AI-driven interactive narratives, even though it’s important for LLM-powered interactive narrative, and (2) how to effectively turn a narrative space into a playable game plot.
% Many people find it hard to work with narrative space because it requires thinking beyond a single, straight story, even in traditional interactive narrative creation. This becomes even more complex when LLM-driven characters are involved. Even experienced writers may struggle to break down a linear story into a flexible structure that allows for different paths and outcomes. However, narrative space is essential for creating interactive plots that go beyond one storyline. Our framework helps users by providing them with intuitive representations of the narrative space and guiding them on how to sculpt the narrative space leveraging the concept of abstraction. By abstracting from a concrete story, users can distill the core elements of the plot and build multiple interactive possibilities around them,
Motivated by the challenges unresolved in AI-bridged IN, we developed the following design goals to guide the design of system:

\textbf{DG1: Enable users to perceive the narrative space.} 
The narrative space in AI-bridged IN contains various possible storylines, which are generated at play-time based on player actions. Authors might struggle to envision what types of variations would be possible. The system should provide representations that can help authors to explore and understand the narrative space. 
%Narrative examples are intuitive but insufficient in picturing the entire narrative space. Outlines, on the other hand, offer a structured view of the narrative space but are more abstract and difficult to derive efficiently. For novice creators, relying on solely either view can not provide them with a clear vision of the narrative space. We then aim to provide both formats of narrative examples and outlines as descriptors of the narrative space for a better perception of the space. 
%important because it prevents narrative instances that deviates from the desired narrative space

% Narrative space is challenging to grasp due to it inherently indicates analytical planning of the narrative. Even experienced writers may struggle to generate into a flexible structure that allows for different paths and outcomes. To address this, we focus on providing clear, intuitive presentations of the narrative space that highlight its important characteristics. By offering visual or structural representations, users can more easily perceive and comprehend the complexity of the narrative space.


\textbf{DG2: Support configurable level of abstraction in editing narrative space.} 
Concrete instances can be overly detailed, while abstract specifications can be too vague. Supporting users to adjust the level of abstraction helps them balance between details and abstraction, which allows the narrative instances to emerge from player interactions, while still adhering to the narrative structure. 
%Narrative examples are intuitive but insufficient in picturing the entire narrative space. Outlines, on the other hand, offer a structured view of the narrative space but are more abstract and difficult to derive efficiently. For novice creators, relying on solely either view can not provide them with a clear vision of the narrative space. We then aim to provide both formats of narrative examples and outlines as descriptors of the narrative space for a better perception of the space. 


%In addition to perceiving the narrative space, users also need effective ways to edit and refine it by setting appropriate boundaries. We propose using abstraction as a simple yet powerful tool that allows users to transform their straightforward stories into boundaries of narrative space. By applying high-level concepts through abstraction, users can more easily shape and adjust the narrative space to align with their creative vision. Editing narrative space via language inherently involves analytical thinking. Thus, Despite the competence of existing LLMs, their unstructured free-form communication with laypeople does not fully unlock their potential in such a specific domain. 
%To solve this, we will provide structured tools that guide users in using both flexible and systematic methods to shape their narrative space. By integrating these tools with LLM assistance, users will gain a better understanding of how and why certain narrative spaces are formed, allowing them to make more informed edits while maintaining flexibility in their creative process.

\textbf{DG3: Generate meaningful game events that react to player actions at play-time.} An engaging player experience requires the generated plots to represent logical causal progression that follows the game mechanism.
%in the game plot.
The proposed system should support simulating casual dynamics and generate meaningful narrative content based on player actions. 


%According to classic interactive narrative design principles, two key perspectives must be considered to reach a balance. First, the narrative should include enough diverse paths to ensure variety in the plot. Second, players’ actions should meaningfully impact the narrative’s development, altering the direction of their experience. Leveraging LLMs provides a solid foundation for generating diverse plot outcomes that respond dynamically to player input. Therefore, our approach to unfolding the narrative space into game plots utilizes LLMs to creatively generate content in real-time, offering interactivity that adapts to the players' actions.
%"Narrative and interactivity must be developed concurrently". Thus, the authorial intent in narrative construction must also be preserved in plot generation. LLMs are bad at this. Thus, we leverage the idea of symbolic planning, with authorial intent to characterize the narrative space serving as planning needs. 







