
%Authoring interactive narratives typically requires expertise, labor, and experience ~\cite{meadows2002pause}.

\subsection{Authoring Interactive Narratives}

 
Various tools have been proposed to support authoring Interactive Narratives (IN) in the past decades ~\cite{green2021use,green2018define,roemmele2015creative}. Many of these tools are designed to give better control and management over branching storylines, enabling rich player actions while maintaining authorial control \cite{friedhoff2013untangling}. IN Authoring tools usually organize the narrative space in explicit branching structure  \cite{chen2022does} to enable intuitive understanding, including flowchart-like structures ~\cite{friedhoff2013untangling}, state machines ~\cite{green2020towards,syahputra2019historical}, and collections of modular story chunks conditioned on game world states (``storylet'') \cite{kreminski2018sketching}. 
\color{black}
Every possible narrative instance is manually authored, which ensures that the author’s intent is precisely preserved \cite{riedl2013interactive}. Much of the prior work 
%recognizes the importance of maintaining this level of controllability and 
seeks to make the authoring process more efficient through more compact representations of storylines \cite{scigajlo2023generation}. However, despite these efforts, the complexity of defining all possible plot progressions remains a significant challenge \cite{riedl2013interactive}. IN authoring continues to be a time-intensive and engineering-heavy process, often falling short of expectations ~\cite{green2021use,spierling2009authoring}.

Recent efforts have moved towards play-time narrative generation, reducing the need for extensive manual authoring \cite{freiknecht2020procedural, garbe2019storyassembler}. Authors set high-level narrative requirements, allowing for automated procedural story generation that responds to player actions \cite{riedl2006linear}. While this approach saves manual effort, 
%it can be challenging for authors to clearly express and articulate their authorial intent using high-level narrative requirements. They 
authors need to express their authorial intent through means other than high-level specifications. To address this challenge, our system proposes various levels of abstraction grounded in the theory of narrative structure \cite{mckee1997story}, allowing authors to express design requirements from beat-level concrete details to story-level flexible goals.

%Therefore, more recent efforts have shifted toward play-time narrative generation under constraints \cite{freiknecht2020procedural, garbe2019storyassembler}, reducing the reliance on extensive manual authoring. In this approach, instead of specifying all possible storylines through predefined branching structures, authors define high-level requirements for narrative content, which are then used to maintain a degree of controllability. This paradigm is close to automated narrative generation, which iteratively applies a non-interactive story generation technique to respond to player actions ~\cite{riedl2006linear}. 

%Building on this line of research, our work introduces a novel approach that leverages different levels of abstraction in specifying requirements for play-time generation. The technical pipeline to unfold narrative instances from the outlines helps authors review potential branches. The tool and its supported workflow retain the controllability of previous tools,  while providing a more efficient authoring process utilizing LLMs. 


\color{black}
% Interactive narrative is closely related to automated narrative generation: an alternative to \originality{a} predefined branching structure specifying all possible storylines is to iteratively apply a non-interactive story generation technique to respond to player actions  \cite{riedl2006linear}. Play-time narrative generation requires the author to specify their requirement for the narrative content to be generated, instead of the narrative content presented at the audience, which requires workflows and design patterns drastically different from authoring traditional IN.


% \originality{All these data structures are developed in order to compactly represent the complex branching storylines existing in the narrative space. However Despite the efforts in }

%and other structured representations ~\cite{porteous2010applying, skorupski2010novice}.
%For instance, tools like Twine ~\cite{friedhoff2013untangling} allow users to create flowchart style structures, with branches reflecting player agency in their narrative experience. 
%Tools further proposing authoring assistance in the form of state machines ~\cite{green2020towards,syahputra2019historical} and other structured representations ~\cite{porteous2010applying, skorupski2010novice} are typically designed under this principle. 
%Despite the player agency resulting from the creation of multiple narrative routes, such a delicate control is still a time-consuming and expertise-required task~\cite{green2021use,spierling2009authoring}, as creators usually need to script all possible outcomes and define how story branches are conditioned on game world states.
%design transitions between gaming world states manually.


% Therefore, more recent efforts have explored play-time narrative generation to avoid extensive manual work. \originality{Instead of predefined branching structure specifying all possible storylines, }

% Therefore, more recent efforts have explored play-time narrative generation to avoid extensive manual work. Interactive narrative is closely related to automated narrative generation: an alternative to \originality{a} predefined branching structure specifying all possible storylines is to iteratively apply a non-interactive story generation technique to respond to player actions  \cite{riedl2006linear}. Play-time narrative generation requires the author to specify their requirement for the narrative content to be generated, instead of the narrative content presented at the audience, which requires workflows and design patterns drastically different from authoring traditional IN. 
%This can assist in creating narrative trees or graphs with branches ~\cite{}, thereby reducing the manual labor required for authors to design all sets of potential outcomes.

%Furthermore, it can also help in gradually creating content like items, and objects in the virtual world where the narrative happens ~\cite{shaker2016procedural,mehm2014procedural} so that authors' effort of setting environments for the narrative can be saved. 

%Our work builds on the established line of research in authoring systems for interactive narratives. Nevertheless, our perspective and goals differ from both traditional manual authoring tools and existing LLM-powered tools. Unlike manual tools that require the designer to traverse the entire narrative space, our first goal is to leverage LLMs to assist creators in shaping an easy-to-perceive narrative space that encodes their authorial intent. Additionally, in contrast to authoring tools that use LLMs primarily as simple content generators, we designed our system to effectively unfold the narrative space into an executable narrative experience through LLM-based planning. Our approach ensures that LLMs not only generate content but also help orchestrate the narrative’s progression in alignment with the author's vision.

%Introducing automation into the creation of interactive narratives brings about conflict and contradiction. In traditional authoring tools for interactive narratives, the control over the narrative and the interactivity experienced by the player is entirely the work of the author. However, with automated authoring tools, part of the creative function is delegated to the automation methods. 

%However, the biggest conflict arises from the fact that the author’s control over their narrative becomes significantly weaker—particularly in interactive narratives, where the author has a clear goal that needs to be reached ~\cite{riedl2009incorporating} or a clear theme to be expressed through their creation~\cite{mateas2006interaction}.

%\yw{Say something about our contribution}

\subsection{Narrative Generation}

A prominent approach to generating plots is through symbolic narrative planning
%~\cite{yannakakis2012game,kybartas2016survey,kapadia2015computer,porteous2015automated,garbe2019storyassembler}. 
%Symbolic narrative planning approach 
\cite{young2013plans, meehan1977tale, lebowitz1985story, riedl2010narrative, ware2011cpocl,ammanabrolu2020story}, which explicitly models the story domain and simulates the causal dynamics of possible plot events to guarantee the causal soundness of generated plots in the simulated context. The author describes a desired world state at the end of plot execution as a ``narrative goal'', and the narrative planning algorithm needs to generate a sequence of state transitions (events) that leads the world state to the narrative goal - called ``narrative plans'' \cite{lebowitz1985story}. 


%Plots generated in this way are grounded by actual simulation system and thus can be executed in a game environment. 

Symbolic narrative planning requires a hand-crafted knowledge base using formal logical language that defines preconditions and effects of a predefined action set within a story \cite{poulakos2016evaluating}. 
Given the extensive engineering work required to construct this knowledge base, the generated plots offer limited complexity and scale.

\originality{The advancement of generative AI also leads to LLM-based methods for plot generation. LLMs can be used to design various narrative elements of the game \cite{lankes2023ai,lanzi2023chatgpt,li2023analyzing}, such as character design~\cite{marincioni2024effect,cox2023conversational,christiansen2024exploring,chiang2024enhancing}, world setting~\cite{jinworldweaver,ratican2024adaptive,short2024designing}, scenes and plots~\cite{chung2022talebrush, yong2023playing, nasir2024word2world,ammanabrolu2020story}. They can also be used at playtime to facilitate just-in-time narrative content generation, such as character dialogs ~\cite{akoury2023framework}. Recent work has also shown that LLMs can be used to drive play-time character behaviors \cite{park2023generative, wang2023humanoid, chen2023agentverse},}
\presentation{leading to plots naturally emerging from the characters.}  \originality{Unlike symbolic narrative planning, which restricts the expression of authorial intent to  formal logical languages, LLM-based methods allow users to write flexible abstract narrative specifications to guide plot generation.}


\originality{%In general, compared to the symbolic narrative planning approach, authorial control is more of a challenge for LLM-based plot generation methods due to the black-box nature of the neural language models. 
However, LLM-based approaches pose challenges to controllability due to their black-box nature. This lack of control remains a major obstacle for authors looking to adopt LLMs in their workflow \cite{biermann2022tool}. LLM-driven character simulation methods \cite{park2023generative, wang2023humanoid, chen2023agentverse} pose even greater challenges for authorial control due to the emergent behaviors of the LLM-powered characters, leaving a ``herding cat'' problem when generating narratives \cite{ware2021sabre}.
To overcome the challenge, effort has been made to incorporate a symbolic representation of events \cite{ammanabrolu2020story}, introduce sketching input beyond text \cite{chung2022talebrush}, and shift AI to an advisor role \cite{roemmele2015creative}. }

\originality{In this work, we aim to improve the controllability of narrative generation by providing authors with various writing assistance with a configurable level of abstraction in the outline, instance, and sentence level to specify the desired narrative goals and character behaviors. Our system combines symbolic planning, LLMs, and character simulation to generate narratives with causal soundness and emergent behaviors for flexible plot progression.} 
%To this end, }we developed a novel LLM-based narrative planning method for plot generation, \originality{following the LLM-Modulo frameworks proposed by \cite{kambhampati2024llms} by using a game environment simulating plans generated by LLMs and providing external critics}. The generated plots ground abstract narrative specifications with concrete character action sequences adapting to different possible player actions, to 1) form better understanding of the narrative space, and 2) guide play-time plot execution. 


\subsection{Narrative Space}
\originality{The concept of narrative space refers to the range of stories that a system can generate \cite{riedl2006story}, possibly conditioned by constraints or requirements from the author.} A narrative space can be characterized in different ways. For example, traditional interactive fiction defines its narrative space with explicit branching storylines. The symbolic narrative planning system usually defines its narrative space based on the story domain \cite{poulakos2016evaluating}, i.e., the possible events that can happen in a story world derived from the characters, locations, objects, and character action schema. 

\originality{The narrative space of a generative system based on LLMs is characterized by the LLM model itself and the prompt creation mechanism.
Unlike narrative spaces defined by formal representations like branching diagrams and state machines \cite{riedl2013interactive}, narrative spaces based on natural language have soft boundaries due to the inherent fuzziness of natural language semantics. The open-endedness of LLMs makes such narrative spaces more difficult for the author to perceive.} 
%revision{Existing work has explored defining narrative spaces using specific abstraction levels such as ``events'' \cite{ammanabrolu2020story} and ``loglines'' \cite{mirowski2023co}. In this work, the narrative space is defined by the concept of ``outline'' - an abstraction of the plots to be generated at a configurable level. We innovatively leverage the flexibility in level of abstractions as a means to control the boundary of the narrative space. } 
\originality{%Concrete instances can be overly detailed, while abstract specifications can be too vague. It remains a problem how to assist the user to find the optimal level of abstraction. 
Existing work has explored semantic abstraction of sentences called ``events'' \cite{ammanabrolu2020story} and ``loglines'' \cite{mirowski2023co}, as a unit of a story that summarizes its central dramatic conflict. Inspired by prior work, we introduce the notion of ``outline'' as the abstract specification that defines the narrative space for generating IN. This ``outline'' has a configurable level of abstraction, allowing the author to adjust the granularity of their authorial control. }

The concept of narrative space is closely related to the notion of design space \cite{biskjaer2014constraint,suh2023structured} or conceptual space \cite{wiggins2006preliminary}. They all refer to a metaphorical space of possibilities - constructed as ideas, designs, solutions, etc. \originality{Previous works have been mostly focusing on exploring a design space in search of specific designs to generate candidates from requirements to establish the space \cite{lin2024design,riedl2006story}, traverse from an existing artifact (pivot) to its alternatives (variants) \cite{matejka2018dream, schulz2017interactive}, and compare these alternatives in multiple dimensions \cite{suh2024luminate,stump2003design,zaman2015gem}. Inspired by prior work, we propose the pivot and variants view to describe the narrative space. }

\originality{Our work differs from prior work by considering the narrative space as the final artifact, rather than an intermediate step in creating a narrative artifact. IN authors create this space for players to explore, making it important for authors to fully understand the narrative space. }
\originality{Creating such space requires balancing between authorial control and emergence \cite{kang2011approach, riedl2013interactive, wang2024storyverse}. On one hand, the space of possible plots needs to be constrained by the author's narrative intent. On the other hand, the space needs to be sufficiently under-constrained so that the player's action and interactions with characters can be reflected in plots that reside in the space. \presentation{ Balancing these two potentially conflicting objectives is the main challenge in AI-bridged IN authoring.} }


