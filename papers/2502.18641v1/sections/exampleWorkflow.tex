\subsection{Example Workflow}

% \zl{format: \textsc{WhatELSE}, cite, quote,  }


Below we present an example workflow to demonstrate some of the features described above. 
\usecase{Alice, a novice text-adventure game designer, wants to create a game based on the setting of a novel she enjoys.} 
Alice opens \textsc{WhatELSE}, along with a game engine preloaded with a story domain based on the novel.

\subsubsection{Encode Authorial Intent in Narrative Space} Alice starts with a rough draft of the story and a moral she wants to convey: {\it ``Kindness is never wasted''}. Using \textsc{WhatELSE}, she uploads her initial story into the system \presentation{(Figure~\ref{walkthrough}.a)}. The story is displayed in the pivot view, showing a sequence of events; while an initial outline appears on the right, summarizing the key turning points \presentation{(Figure~\ref{walkthrough}.b)}. Alice adds details to the pivot to refine her story. Once satisfied, she clicks the Generate Outline button to update the outline based on her edits. She chooses the ``act level'' and specifies, {\it ``The hunter has to appear in every act''}. Alice hovers to see how each event in the outline is mapped to the entries in the pivot plot. She continues exploring different levels to find the ideal level of abstraction.

\presentation{Alice finds one of the events ({\it ``The peaceful life is threatened by an unexpected danger from the hunter''}) to be too restrictive for the hunter to cause the danger. She uses the abstraction tooltip to replace the phrase {\it ``the hunter''} with {\it ``a character''} to leave room for variations in the game.} 
Alice looks at the outline and is unsure what players might experience. So she clicks the Generate Variants button. The interface displays a scatter plot of potential narrative instances. Alice scrolls through different plot stages of these instances — from start to end — she notices that some instances continuously express the moral, while others only reveal it toward the end, both of which she considers acceptable. However, she also spots a cluster of instances that fail to express the moral by the end of the narrative. Curiously, she clicks on a dot representing one of these instances and reviews its details. 

Alice reads the instance and realizes the issue is in the event that she had previously set as {\it ``a character''}, which was too loosely defined, allowing the system to choose an undesirable character. To address this issue, she changed it back to {\it ``a human character with power''}, allowing the system to choose a character reasonable for the second event. 

\begin{figure*}[h]
\centering
\includegraphics[width=1.0\textwidth]{figures/walkthrough_v3.pdf}
\vspace{-20pt}
\caption{\presentation{An example workflow that shows (a) an author uploads a story draft in \textsc{WhatELSE} to (b) generate an outline. The system unfolds the outline into (c) an executable game plot with (d) a pre-loaded story domain, which supports branching storylines based on the player actions. If the player chooses to (e1) save the deer from the hunter, this action fulfills the ``brave assistance'' event in the outline defined by the author (shown as the orange star). If the player chooses to (e2) ask another character (e.g. a witch) for help, the witch will instead save the deer, demonstrating ``brave assistance'' to fulfill the event. Alternatively, if the player does not choose to save the deer at all, the system will choose a character from the story domain to save the deer as a demonstration of ``brave assistance''. This example shows how the game plot is dynamically adjusted based on the player actions to fulfill the outline. (f) The author can play the game plot to better understand the player experience. }}
~\label{walkthrough}
\vspace{-10pt}
\end{figure*}

Later, Alice notices a set of three variants where one of the events unfolds as, {\it ``the dove speaks with the hunter, leading the hunter to notice and then chase the dove''}. Alice finds this version more compelling than her pivot plot. She removes other variants, only leaving these three narrative variants in the view. Satisfied with these variants, she clicks the Generate Outline button to create a new outline that summarizes their commonalities. She then returns to the outline editor, using the abstraction tools to iteratively edit the outline, until it aligns with the story's moral and represents a narrative space that incorporates the interesting variations.

\subsubsection{Unfold the Narrative Space For interactivity}
With the narrative space defined by the outline, \usecase{Alice can experience the narrative instances unfolding in a turn-based text adventure game. She goes to the interactivity page. The system loads the story domain that includes a set of characters, locations, and action schema \presentation{(Figure~\ref{walkthrough}.d)}. The first sequence of the game plot is generated: a hunter is looking for food and finds a deer to hunt \presentation{(Figure~\ref{walkthrough}.c)}}. 

Alice, playing as the dove, chooses her next moves from a pin pad \usecase{\presentation{(Figure~\ref{walkthrough}.f)}. She can bravely stop the hunter by giving out her food \presentation{(Figure~\ref{walkthrough}.e1)}. Alternatively, she could ask other characters for help \presentation{(Figure~\ref{walkthrough}.e2)}. 
The system compares the player's action with the narrative outline. If the player chooses to save the deer on their own \presentation{(Figure~\ref{walkthrough}.e1)}, the event of {\it ``brave assistance''} is fulfilled by the player action. If the player chooses other actions, the system will create an event where another character demonstrates {\it ``brave assistance''} \presentation{(the witch in Figure~\ref{walkthrough})} to fulfill the event. The system generates subsequent character behaviors based on the player action. }

This turn-based interaction continues, with Alice alternating between reviewing generated game plots, observing character simulations, and experiencing the generated game play as a player. \usecase{Since she wrote a total of five events in her outline, the game play proceeds for five rounds, until all the events she planned have been played out. Since the game plots generated are fully structured, Alice can directly export the output of the narrative compiler as a finite state machine into the game engine where she can visualize the characters and locations.}

%The scenario above demonstrates how creators can use \textsc{WhatELSE} to create their interactive narratives by first encoding the authorial intent into a narrative space, and then unfolding the narrative space into play-time plot execution. 
\usecase{
\subsubsection{Additional Use Cases} In addition to Alice's case as a text-adventure game designer, \textsc{WhatELSE} can also serve as a powerful tool for a wide range of users. Game masters, mod developers, and fan creators across different domains can leverage its capabilities. For example, dungeon masters in tabletop role-playing games can use the Narrative Space Editor to outline gameplay scenarios before sessions and employ the Interactive Narrative Compiler to dynamically determine outcomes of player actions during gameplay. Fan creators~\cite{booth2009narractivity} can efficiently transform their favorite novels, movies, or other media into interactive narratives, using the \textsc{WhatELSE} to structure and unfold new, personalized storylines based on the original story domain. Beyond entertainment, educators can utilize \textsc{WhatELSE} to design interactive learning experiences, such as gamified learning tutorials or interactive training modules. }
%By outlining educational themes in the Narrative Space Editor and generating interactive scenarios through the Interactive Narrative Compiler, \textsc{WhatELSE} enables students to explore the theme dynamically, fostering engagement and deeper understanding through adaptive branching experiences.








