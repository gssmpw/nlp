\begin{figure*}[t]
\centering
\includegraphics[width=1.0\textwidth]{figures/WhatELSE.pdf}
\vspace{-10pt}
\caption{An overview of the system.}
~\label{overview}
\vspace{-10pt}
\end{figure*}

% WhatELSE is an LLM-powered system designed to support the workflow for users to generate executable game plots for interactive narratives by sculpting a narrative space. 
% Figure \ref{overview} provides an overview of the workflow. The workflow starts with users writing a linear narrative, written in an intuitively linear way, and ends to structured gameplots, that can be exported to any game engines as an interactive narrative experience ready to play.


% In this section, we first present an overview of the WhatElse system detailing its features, and demonstrate its workflow to generate executable game plots. Finally, we provide a concise overview of its implementation.

% \subsection{System and Features}

The WhatELSE system includes two parts. Part 1 is the {Narrative Space Editor} (Figure x A) that enables authors to sense and edit the narrative space using three representations of the space. Part 2 {Game Plot Compiler} (Figure x B)  generates an interactive game plot unfolding the narrative space using an approach combining LLM-based planning and simulation. We explain each part and its features next.



% This workflow specifically aims to generate game plots for theme-based interactive narratives. In such a context, users typically have a declarative moral to express and seek to convey this moral by creatively planning events within an existing story domain. Therefore, a narrative space in such theme-based interactive narrative creation is all the different ways a story can change, showing the same idea in different ways. For example, in a story about kindness, one version might be a dove helping an ant, and another version could be the ant helping the dove, but both are about "kindness."


% For example, in a Fairytale forest setting where animals can talk, a user might want to plan game plots with consecutive acts that express the moral "kindness is never wasted." 




%operates with two main components. With the first component, users engage with an interactive interface to sculpt the narrative space characterized by a narrative outline. The second component of WhatELSE applies LLM-based planning to the representation of the outline to generate an interactive narrative. This section details the design and implementation of these components. 

\subsection{Narrative Space Editor}

\subsubsection{Representations of the Narrative Space}

% terminology

% plot: a semantic concept, a sequence of events

% variant plots: events at a lower-level of abstraction

% outline plot: concise, events at a higher-level of abstraction -> allows for a quick overview of the narrative's structure and content.

% pivot plot: a plot that serves as an anchor in shaping the narrative space]


To support users' better perception of the narrative space (DG1), we provide users views of the narrative space by using the following three representations, \textbf{Pivot Plot}, \textbf{Outline Plot}, and \textbf{Variants Plot}. Figure. \ref{user_study:interface} illustrated how each views is presented on panels in the narrative space editor. We explain each view next.   

% The narrative space for an LLM-powered interactive narrative is too complicated to sense. Thus, aiming at our first design goal (DG1), instead, we first propose the following three intuitive representations of the narrative space generated from the user's input straightforward story for their view. We then design a series of features targeting editing on the narrative space (DG2).

\noindent\textbf{Pivot View} A pivot plot is formatted from the user's narrative example, by extracting all events that occurred in the story and arranging the events sequentially as a plot. Specifically, following the principles in defining events in digital games ~\cite{gould2011narrative,castellan2017games},  each event is a combination of "subject + action + object + potential location". The view of the Pivot Plot shows the backbone of the user's narrative example, providing a straightforward structure of the narrative development. More importantly, the pivot plot lies in the centric position of the narrative space. Narratives unfolded from the narrative space should always refer to this pivot plot. 




% starting directly from the user's input story. It lies at the centric position in the narrative space, to which users can always refer.


% A pivot plot is a formalized version of the input written story. Specifically, it extracts all the events that occur in the story and rewrites each in the format of "subject + action + object + potential location."

% This structure offers a straightforward and easy-to-understand representation of the narrative space, starting directly from the user's input story. It lies at the centric position in the narrative space, to which users can always refer.

\noindent\textbf{Outline View}
The outline plot offers a broader view of the narrative space than the pivot plot. Similar to how "acts" function in drama writing~\cite{styan1960elements}, an outline plot condenses the events in plots into a less specific form compared to the detailed events in a plot, such as the pivot plot.  Specifically, we utilized the concept of abstraction in linguistics to construct outline plots in this view. That is, targeting on the same event happening in the narrative, an outline plot uses more abstract language to describe the event than a plot. For example, while a plot might specifically mention "a witch casting spells in anger", the outline could refer to "a character uses magic powers for destruction" or even "a  character uses power negatively." This abstraction naturally leads to multiple possible plots that align with a single outline. For instance, an outline referencing a "character uses power negatively" could encompass various possible plot descriptions. For instance, an outline plot "One strong character saves another small creature", can be further mapped to a range of plots, including "A hunter shoots a wolf to save a deer", or "A witch casts spells to bring water back to the forest for the thirsty ant to drink."
Intuitively, the more abstract the language used in an outline plot, the greater the number of potential specific plots that can align with it. Thus, the abstraction  as the outline of plots for interactive narratives


The view of outline plots serves several purposes. It enables users to inspect the narrative space from a broader perspective, beyond individual narratives. Furthermore, it helps users to recover the often implicit cognitive stage of "outline writing"~\cite{hayes1983cognitive}, commonly applied in professional writing, helping them identify and emphasize key components they want to feature. Finally, outline plots define the boundaries of the narrative space. The use of abstract language allows for plot variations, while also ensuring that essential characteristics of events remain highlighted and ultimately expressed in the narrative. \zl{may move to section 3}


% Notably, an outline plot is a relative concept—it only exists concerning concrete events in a specific story domain. Defining concrete events in the story domain uses all atomic elements (e.g., specific characters, actions, locations) to describe the content. In contrast, an outline uses abstract descriptions to incorporate constraints (e.g., "a strong character") rather than specifying exact elements (e.g., "the witch"). For instance, in a narrative happening in the Fairytale Forest, Notably, it would be hard to define an outline out of the context of a specific story setting, as every element used in the natural language can be further broken down within a taxonomy. 



% Con, the abstraction of a plot makes it an outline plot is a process of encoding. One plot can be abstracted to an outline. Meanwhile, there exist multiple plots that can satisfy the same outline. For instance 


% a plot describing an event involving an concrete element, the Witch. In an outline, the Witch can be abstract into a higher level to "a powerful character".



% In contrast, an outline uses abstract descriptions to incorporate constraints (e.g., "a strong character") rather than specifying exact elements (e.g., "the witch"). For instance, in a narrative happening in the Fairytale Forest, an outline "One strong character saves another small creature",





% Inspired by the concept of abstraction in linguistics,  characterizing the narrative space by using abstract language . 


% An outline plot provides a more abstract description of the

% and "story lets" in traditional interactive narratives, an outline plot describes 


% contrasting with the concrete events in the pivot plot in Representation 1. 

% Importantly, an "outline" is a relative concept—it only exists in relation to concrete events in a specific story domain. Defining concrete events in the story domain uses all atomic elements (e.g., specific characters, actions, locations) to describe the content. In contrast, an outline uses abstract descriptions to incorporate constraints (e.g., "a strong character") rather than specifying exact elements (e.g., "the witch"). For instance, in a narrative happening in the Fairytale Forest, an outline "One strong character saves another small creature", can be further expanded to a range of concrete events, including "A hunter shoots a wolf to save a deer", or "A witch casts spells to bring water back to the forest for the thirsty ant to drink." Notably, it would be hard to define an outline out of the context of a specific story setting, as every element used in the natural language can be further broken down within a taxonomy. 

% . Each outline serves as a set of constraints, defining a range of narratives that satisfy those criteria.

\noindent\textbf{Variants View} Variant plots are a sequence of plots that players could experience under this narrative space constrained by the outline plot. Each variant consists of two parts: the game plot, and the player action. Player actions drive the gradual unfolding of game plots from the narrative space, by connecting gameplots as a cohesive narrative. In other words, one variant records the full gaming experience of a specific player. Thus, such variation is caused by different game plots and different player actions jointly. For instance, the same game plot describing a character facing danger, can be led by two different player actions "save" and "kill" can lead to two variants; similarly, with the same player action "save", the character to be saved in game plot to be saved will also lead to two variants. 


In summary, the representations of the narrative space in the three views are a trinity. The outline plot in View 2 is an abstraction of the pivot outline in View 1, and variant plots in View 3 are generated within boundaries defined by the plot outline in Representation 2. Meanwhile, each variant plot serves as an alternative to the pivot plot in View 1. Therefore, the three representations employed jointly characterize the narrative space. 

% Plot variants represent a sequence of narratives where each variant consists of two parts: the game plot and the player's actions. Player actions drive the gradual unfolding of the game plot, creating a dynamic connection between the narrative space and the player’s choices. In other words, a variant records a complete gameplay experience, where the game plot unfolds differently based on the player’s actions. These variations arise from different combinations of game plots and player actions. For instance, starting with the same game plot, two distinct player actions such as "save" and "kill" can lead to different narrative outcomes. Similarly, even with the same player action "save," the specific character being saved in the plot could lead to different variants. These plot variants are generated under the constraints defined by the plot outline.


% is the full narrative expressed  could potentially be unfolded from the plot outline after the game play procedure driven by a specific

% edit different representations of the narrative space, and transfer their editing between representations in the first part of our system, the Narrative Space Editor. As a starting point, the original linear narrative is one specific type of representation of the narrative space. We then propose two additional representations for the narrative space, the \textit{Outline} of the narrative, and \textit{Variants} of the narrative.

% Aiming at our first design goal (DG1), we define three intuitive representations for users to grasp the narrative space, and display them in the three panels in the Narrative Space Editor. 

% The first stage of interactive assistance in sculpting the narrative space involves (1) abstraction tools  (2) visualization feedback (3) Transformation between different representations of narrative space. Then we introduce the second stage of process of generating game plots given a narrative outline, involving (1) act direction (2) character simulation, and (3) player action.

\begin{figure*}[t]
\centering
\includegraphics[width=0.99\textwidth]{figures/whatelse_interface_v1.pdf}
\vspace{-10pt}
\caption{An illustration of the interface.}
~\label{overview}
\vspace{-10pt}
\end{figure*}

\subsubsection{Support Editing the Narrative Space} 



With the three views on the narrative space, we provide users with an intuitive way to perceive the narrative space. We then provide a series of tools as follows targeting \textbf{DG2}, to help users effectively shape the narrative space by editing the representation of the space in different views. 

% [td]

% We provide a series of tools as follows, to help the users obtain an ideal narrative space. Specifically, our editing tool leverages a core concept of "abstraction" to mediate users' editing on the narrative space.

% To support users' editing on the narrative space, we provide the following abstraction tools.
% One core concept that we want to leverage to mediate users' editing in the narrative space is "abstraction". An 

% To begin with, as we previously stated, narrative 

% how much abstraction and how abstract the constraints are constructs a spectrum. On one end, the outline may closely resemble the plot instance, with minimal abstraction. On the other end, it might be as vague as "something happened in the world." How does this influence the narrative space? Intuitively, a more concrete outline specifies all elements involved in the narrative, leading to a tightly defined narrative space with limited variations. Conversely, a highly abstract outline sets fewer constraints on the narrative, resulting in a broader narrative space. 

% To begin with, abstraction in an outline lies in a spectrum of different levels. On one end, the outline may be as detailed as the plot instance, with no abstraction. On the other end, it might be as vague as "something happened in the world." How does this affect the narrative space? A more concrete outline specifies all narrative elements, leading to a tightly defined narrative space with limited variations. Conversely, a highly abstract outline imposes fewer constraints, resulting in a broader narrative space.
\noindent\textbf{Abstraction Ladder} Our first tool helps users obtain an ideal outline plot based on their pivot plot. Specifically, users can initiate the abstraction process by clicking the "Plot $\rightarrow$ Outline" button on the interface, which transforms the pivot plot into an outline plot. To enable customization of this abstraction process, we offer a feature named the "abstraction ladder."


As mentioned earlier, the outline plot uses language with abstraction to bound the narrative space, with higher levels of abstraction allowing for more flexibility and fewer constraints on plot variations in the narrative space. To give users control over the level of abstraction in their outline, we implemented the abstraction ladder, inspired by the hierarchy used in drama writing~\cite{kennedy2013literature}. This ladder consists of three abstraction levels: scene level, sequence level, and act level, each progressively more abstract than the last. For instance, a scene-level outline provides detailed descriptions of specific scenes, including characters, actions, etc, such as: "The kind dove uses a leaf to reach the ant and drag it out of a water bubble." On the other hand, an act-level outline offers a highly summarized view of the narrative, focusing on the turning points: "A character saves their friend from danger."

To implement the abstraction ladder, we designed a prompt strategy to help users write outlines based on the defined abstraction levels. We first provide knowledge of defined levels of abstraction referring to professional drama writing literature in the prompt. We then implemented a prompt chain, structured as a tree of thought. This chain operates in three stages: (1) Given the pivot plot, the first part of the chain prompts the LLM to generate a series of variations of the plot by creatively rearranging the elements involved. (2)  The second stage generates three outlines corresponding to the predefined abstraction levels: scene, sequence, and act level. (3) In the final stage, the outline is tailored according to the user's specific requirements.



% This function is built on a predefined abstraction ladder with three levels of abstraction for outlines, inspired by the drama writing hierarchy ~\cite{kennedy2013literature}. These levels are scene level, sequence level, and act level, with abstraction increasing across the three levels. For example, a scene-level outline describes specific scenes in the narrative, detailing concrete characters, actions, and locations, such as "The kind dove uses a leaf to reach to the ant to drag it out of a water bubble". Meanwhile, act-level outlines will focus more on the highly summarized turning point of the narrative, exemplified by a "Some character conducts an action to save their friend from danger". Based on the abstraction ladder, we designed a prompt template to help users write outlines based on the linear narrative. We first formally define levels of abstraction referring to professional drama writing literature in the prompt. We then implemented a prompt chain, structured as a tree of thought, specifically for outline generation. This chain operates in three stages: (1) Given the linear narrative, the first part of the chain prompts the LLM to generate a series of narrative variants by rearranging the elements involved. (2) The second part prompts the LLM to generate three outlines, each corresponding to one of the predefined abstraction levels, from scene to act. (3) Finally, the third part tailors the outline according to the user's specific requirements.






% Thus, inspired by the hierarchy of drama writing ~\cite{kennedy2013literature}, we consider that a narrative outline should have several specific levels of abstraction. As we stated, while outlines at different abstraction levels share the same intention of describing what happens in the narrative, they differ in granularity. In WhatELSE, we provide two functions powered by LLMs to help users edit their outlines effectively to obtain an ideal level of abstraction. Firstly, we provide a function to effectively construct outlines from linear narratives, accessible via the Narrative-to-Outline button on panel 1 in Figure 1. This function is built on a predefined abstraction ladder with three levels of abstraction for outlines, inspired by the drama writing hierarchy ~\cite{kennedy2013literature}. These levels are scene level, sequence level, and act level, with abstraction increasing across the three levels. For example, a scene-level outline describes specific scenes in the narrative, detailing concrete characters, actions, and locations, such as "The kind dove uses a leaf to reach to the ant to drag it out of a water bubble". Meanwhile, act-level outlines will focus more on the highly summarized turning point of the narrative, exemplified by a "Some character conducts an action to save their friend from danger". Based on the abstraction ladder, we designed a prompt template to help users write outlines based on the linear narrative. We first formally define levels of abstraction referring to professional drama writing literature in the prompt. We then implemented a prompt chain, structured as a tree of thought, specifically for outline generation. This chain operates in three stages: (1) Given the linear narrative, the first part of the chain prompts the LLM to generate a series of narrative variants by rearranging the elements involved. (2) The second part prompts the LLM to generate three outlines, each corresponding to one of the predefined abstraction levels, from scene to act. (3) Finally, the third part tailors the outline according to the user's specific requirements.

% \zl{forgot Qian's better naming, todo} \qz{Abstraction Tooltip} 

\noindent \textbf{Abstraction Tooltip}
After the initialization of outline plot, we further designed an Abstraction Tooltip for users to directly adjust the abstraction of content in the outline plot in a more fine-grained manner. Practically, when users select a text snippet in their outline plots, the tooltip appears, offering two intuitive buttons: "More Abstract" and "More Concrete." By clicking these buttons, users receive suggested edits that either abstract or specify the selected content. 

While the abstraction ladder provides global abstraction control over the entire outline plot, the Abstraction Tooltip enables more fine-grained adjustments at the word or phrase level. Specifically, the suggestion of making the selected content more abstract or more concrete relies on the taxonomy in linguistics~\cite{hayes1983cognitive}. For example, "character-animal-small animal-cat-tabby cat" constructs a linguistic hierarchy. Given a selected text snippet "cat", requesting a more abstract suggestion would yield its superordinate term "small animal" or "animal," while a more concrete suggestion would provide its subordinate "tabby cat." By leveraging this taxonomy-based structure, the Abstraction Tooltip enables users to fine-tune their outline plots, offering greater control over the boundaries of the narrative space.









% Clicking the More Abstract button will bring suggestions that generalize the selected text, moving up the conceptual hierarchy. For instance, "cats" to "an animal". Conversely, the More Concrete button will generate suggestions that narrow down the selected text, adding concrete details. 

% Similar to the previous implementation, we build the function with a LLM prompting chain, designed to recognize and apply hierarchical abstraction principles from linguistics.

\noindent \textbf{Variants Generation} 
Variant plots address the issue of what could happen within the narrative space defined by the plot outline when players are involved. To help users explore and understand the range of possible variant plots, and how these variants relate to the central narrative (i.e., the pivot plot), we provide a tool that generates and visualizes these variant plots driven by simulations of player actions.


We base this tool on player behavior classification derived from previous studies in digital games ~\cite{yannakakis2013player,worth2015dimensions}, incorporating three key player behavior models in our context of theme-based interactive narrative:

\begin{itemize}
    \item \textbf{Positive Players} in digital games contribute positively by engaging with the game objectives and tend to do helping behaviors~\cite{velez2013helping,bostan2009player}.
    \item \textbf{Negative Players} are the killers identified in classic player modeling~\cite{majors2021some,hamari2014player}. They often exhibit aggressing behavior that disrupts the experience of others, particularly when they seek to dominate or harm others destructively.
    \item \textbf{Role Players} prioritize narrative immersion and character development by mimicking the actions their character would take in the gaming world~\cite{praetorius2020avatars}.    
\end{itemize}

Using these player models, we simulate potential narrative paths that could emerge from the interaction between game characters and simulated players. The tool generates variant plots by applying an innovative LLM-based planning approach integrated within our system, which will be explained in later sections.

% Using our simulation approach to generate game plots and player modeling to simulate player actions between acts, we can jointly generate narrative variants. 

The variants are displayed in an interactive scatter plot along two dimensions—"authorial intent distance" and "emergent behavior distance"—to help users understand the shape of the narrative space:
\begin{itemize}
    \item \textbf{Authorial Intent Distance} Since we focus on theme-based interactive narratives with clear declarative intent, we calculate the distance between the moral expressed by the narrative path and that of the pivot plot. This distance is evaluated by prompting the LLM to assess how well the moral is conveyed, ranging from 0 to 1.
    \item \textbf{Emergent Behavior Distance} We measure the emergence of character behaviors by using a prompt chain to evaluate how much the narrative path deviates from the pivot plot, also ranging from 0 to 1.
\end{itemize}

In practice, users generate this scatter plot by clicking on the "Outline-to-Variants" button on panel 2, as shown in Figure 1. Additionally, users can specify the number of sets of variants to be generated for visualization, ranging from 1 to 5 sets. For each set, we generate 3 variants corresponding to the variant plots driven by the three types of players. For example, if a user chooses to visualize 4 sets, a total of 12 variants—categorized by player type—will be displayed on the scatter plot, as shown in Panel 3 of Figure 1. Users can also use a scroll bar to view the narrative variants as they develop across different stages—from the start, through the middle, to the end—allowing them to sense how the narrative variants evolve over time.


% One of the key challenges for users to efficiently sculpt the narrative space is that such a space  interactive narrative is jointly shaped by the author's outline and the emergent behaviors of various characters. 


% In other words, even with the same outline, characters can still behave differently to act the outline. Therefore, it is hard for users to get a direct sense of the narrative space defined by the outline. Questions such as whether each part of the outline is abstracted to an appropriate level to incorporate variants, whether essential scenes are retained, or whether the intended moral is clearly conveyed become critical concerns. 


% To address these issues, we provide users with feedback by visualizing their narrative space. As narrative space itself is hard to visualize, we visualize a set of narrative variants generated under the outline, where each variant is a potential narrative path that the player could experience in the actual game. As mentioned earlier, our targeted interactive narrative follows a structure where game plot developments and player actions alternate. Thus, to generate the variants of the narrative paths, we simulate a series of potential narrative paths that could occur under the outline, jointly created by LLM-driven characters and simulated players. To generate the game plot, we use the unique LLM-based planning approach within WhatELSE, which will be detailed in subsequent sections. For simulating player actions, we adapt player modeling from previous studies in digital games ~\cite{yannakakis2013player}, establishing three player models:

% \begin{itemize}
%     \item \textbf{Positive Players}: Summarized from explorers and achievers in digital games, such types of players contribute positively by engaging with the game objectives.
%     \item \textbf{Role Players}: These players prioritize narrative immersion and character development by mimicking the actions their character would take in the gaming world.
%     \item \textbf{Negative Players} Negative players, exemplified by the "Killers" identified in traditional player modeling, often exhibit behavior that disrupts the experience of other characters, particularly when they seek to dominate or harm others destructively.
% \end{itemize}

% Using our simulation approach to generate game plots and player modeling to simulate player actions between acts, we can jointly generate narrative variants. Given a written outline, users can click on the "Outline-to-Variants" button on panel 2, as shown in Figure 1, to generate a scatter plot of narrative variants displayed on panel 3. Each variant represents a sample narrative path within the narrative space. The variants are displayed along two dimensions—authorial intent and emergent behavior—to help users understand the shape of the narrative space:
% \begin{itemize}
%     \item \textbf{Authorial Intent Distance} Since we focus on theme-based interactive narratives with clear declarative intent, we calculate the distance between the moral expressed by the narrative path and that of the original linear narrative. This distance is evaluated by prompting the LLM to assess how well the moral is conveyed, ranging from 0 to 1.
%     \item \textbf{Emgergent Behavior} We measure the emergence of character behaviors by using a prompt chain to evaluate how much the narrative path deviates from the original linear narrative, also ranging from 0 to 1.
% \end{itemize}

% Additionally, users can specify the number of sets of variants to be generated for visualization, ranging from 1 to 5 sets. For each set, we generate 3 variants corresponding to the narrative paths driven by the three types of players. For example, if a user chooses to visualize 4 sets, a total of 12 variants—categorized by player type—will be displayed on the scatter plot, as shown in Panel 3 of Figure 1. Users can also use a scroll bar to view the narrative variants as they develop across different stages—from the start, through the middle, to the end—allowing them to sense how the narrative variants evolve over time.

\noindent\textbf{Variants-driven Editing} In addition to visualizing the narrative space, we designed tools to help users edit the narrative space using plot variants. Users can click on any variant in the visualization to display its detailed content in Panel 1, allowing them to view it as an alternative to the original pivot plot. Additionally, users can refine the set of variants by clicking a reject button to remove selected variants from the set in Panel 3. For example, users may reject all narrative paths driven by negative players if they are unsatisfied with the narrative's direction. For instance, users can reject all narrative paths driven by negative players if they are not satisfied with the narrative developing in this manner. Once users are satisfied with the remaining variants, they can use the "Variants-to-Outline" button in Panel 3 to summarize the commonalities among the variants into an outline. Similarly, users can also use the abstraction ladder and customize the outline generation based on their specific needs.


Overall, we provide a suite of tools that users can utilize to continuously shape the narrative space by editing across three intuitive representations. Specifically, we leverage the concept of abstraction to allow users to set appropriate boundaries within the narrative space, offering flexibility and control over how the narrative evolves and adapts to player actions.

% \subsection{Transformation Among Representations of Narrative Space} 
% Overall, we have Linear narrative displayed in panel 1, outline in panel 2, and variants visualized in panel 3 are three different representations of the narrative space. We aim to build a workflow based on which users can continually sculpt the narrative space by editing either representations and transfer their editing to other representations. For instance, we have introduced how users can edit the linear narrative and transfer it to outlines in the outline generation section, and how we generate variants based on outlines. 

\subsection{Interactive Gameplot Compiler}

\label{compiler}

With the three representations that users refine in the editor stated in the previous section, a narrative space is characterized by the corresponding three perspectives: pivot plot as the centric position of the narrative space, plot variants as estimated samples in the narrative space, and outline set as the boundary of the space. In this section, we present a pipeline to use the narrative space created in the previous stage to guide play-time plot execution through character actions. Simply stated, for each event in the outline plot, we employ an LLM-based planning process to generate a sequence of character actions that act out the event, with characters taking free actions in between the events - where the player character's actions are determined by the player's input, and non-player characters' actions are driven by an LLM. We then 

% As previously mentioned, we incorporate three representations to characterize a narrative space. After obtaining these narrative space representations in stage 1, WhatELSE enables users to transfer any of these representations into interactive narratives. In this section, we explain how WhatELSE can transform these representations into turn-based, executable game plots that can be exported to various formats of interactive narratives.

% \noindent \textbf{Outline to Interactive Narrative} While constructing interactive narratives based on linear narratives is a well-established process, WhatELSE focuses more on transforming outlines into interactive narratives. As a more abstract framing of the narrative, outlines describe a distribution of concrete events that meet specific constraints. Generating game plots from outlines requires planning concrete events into appropriate positions within the narrative. This planning process must continue dynamically alongside plot development and player actions. Specifically targeting on this transformation, we here proposed a novel approach that allows WhatELSE to direct the progression of the plot while leaving room for the details to emerge from the character simulation as well as the interaction between the player and characters.


%In traditional screenplay writing, writers use the concept of an \textit{``act``} to structure a sequence of movements that turns on a major reversal in the value-charged condition of a character’s life \cite{mckee1997story}. Inspired by this notion, we use the term abstract act to structure a sequence of character actions leading toward a (possibly abstract) narrative goal. The term \textit{``abstract''} highlights the separation of the story from the specific details of the underlying world states. This allows for the final story to be a collaborative creation involving the author, the player, and the emergent behaviors from the LLM character simulation. By abstracting the story from the specifics of the world states, it becomes more adaptable and open to contributions from multiple sources, enabling a dynamic storytelling experience.


% \noindent \textbf{Formatting Outline}
% We first analyze the LLM to construct two parts of information to facilitate the game plot generation. Specifically, we extracted the following elements from an outline: (1) {\bf A high-level narrative goal} representing a dramatic conflict or a turning point in the story (e.g., \textit{``a character gets into a life-threatening accident''}). (2) {\bf A set of prerequisites} connected by logical conjunction and/or disjunction. Each prerequisite can be: (i) A statement about the current world state that can be evaluated (as true or false) by the LLM or the game environment (e.g., `\textit{`John is loved by all characters''} or \textit{``there are at least 2 characters alive''}), (ii) A statement about the player's action, (e.g., \textit{``The player executes \texttt{eat(X)} action with \texttt{X} being a poisonous food item''}), and (iii) The fulfilment or failure of another abstract act, used to specify dependencies between abstract acts. (3) {\bf A set of placeholders}, which refer to specific content in the ``instantiation'' of the current act, to carry over to subsequent acts. Each placeholder is a pair of code-name and descriptions (e.g., \textit{``X - the character who got into the life-threatening accident''}).

% The narrative goal and prerequisites may include placeholders for story specifics that will be determined by the instantiation of a previous act. For example, \textit{``a character saves X's life''}, with \textit{``X''} being a placeholder for the character who \textit{``got into a life-threatening accident''} in a previous act (which is unknown at the time of authoring). %\footnote{Note that the occurrence of a placeholder in a prerequisite implies the completion of an act with the placeholder as an output as another prerequisite.}
% Incorporating placeholders into the narrative goals and prerequisites allows the story to adapt and respond to the choices and actions made by players as well as the interventions from character simulation. In addition, it 
% %enforces coherence of plot elements across abstract acts and
% allows the author to enforces coherence and define continuations between acts without knowing the specifics of each act.



% We use a story about \textit{``Kindness is never wasted''} to showcase the usage of abstract acts extracted from the Ant \& Dove story from Aesop Fable following prior work on narrative planning %\cite{GDC-work, graesser1981incorporating}
% \cite{graesser1981incorporating}.
% As shown in Figure~\ref{fig:teaser}, the writer creates the four abstract acts with high-level narrative goals such as life-threatening and life-saving events to depict the turning points as well as the relationship between characters. Note that the execution order does not depend on the creation order. The prerequisites indicate the chronological relationship between acts and enforce consistency among them. This approach allows the writer to create branching stories, where the path to take is determined by both the player's actions and the character simulation. 


% \noindent \textbf{Act Director} \label{sec:narrative_planning} The Act Director uses formatted outline and LLMs for narrative planning, creating a sequence of character actions, or a {\em (narrative) plan}, to achieve the goals outlined. During each play-time timestep, the Act Selector checks for any pending abstract acts that have met their prerequisites for execution. Once an act is selected, placeholders in the narrative goal are replaced with the referred content from previously executed acts. For example, a narrative goal  \textit{``Y saved X from the accident''} might become \textit{``The dove saved the ant from the accident''}. 

% Following \cite{wang2023describe}, we design an LLM-based iterative narrative planning process involving plan generation and plan revision for better quality of the narrative plans (Figure~\ref{fig:narrative_planning}). 
% The plan generator takes a narrative goal, the world state, and the story domain as the input and is prompted to generate a plan. The generated narrative plans are in the form of a list of tuples containing an action following the action schema in the story domain, the subject of the action, and the thought behind the action. Once generated, the plan reviewer provides feedback regarding the quality and feasibility of the action sequence to improve it. The feedback has three parts: (1) {\bf Overall Coherency Evaluation}: Feedback is obtained by prompting an LLM to comment on the overall coherency of the generated plot and make suggestions for improvement. (2) {\bf Game Environment Evaluation}: Similar to \cite{wang2023describe}, feedback comes from executing the generated action sequence in a simulated game environment, and reporting observations on the success/failure of the execution. (3) {\bf Character Simulation Evaluation}: For every action in the sequence, we prompt an LLM to play the role of the subject of the action. Given the current world state including the character's memory, we ask the LLM if the motivation for the character to perform the action has been established. We include the explanation to this question in the feedback if the motivation has not been established.

% The plan generator then revises the action sequence based on the feedback. This process repeats until the plan is executable or for a user-specified number of iterations. Once the final character action sequence is generated, it is executed by the game environment to update the world state. It's also sent to the Placeholder Resolution component, which uses an LLM to identify specific content referred to by each placeholder in the act, storing the placeholder-content mapping for future act execution. For example, if placeholder $X$ is defined to refer to ``\textit{the character who got into an accident}'', then Placeholder Resolution will ask an LLM, ``\textit{what is the character who got into an accident}'', and then store the answer with placeholder $X$.

% Through this LLM-powered

% The final representation of variants as a series of linear narratives composed of concrete events. To transform this representation into interactive narratives, we first construct a game event sequence diagram based on the set of linear narratives.



% \begin{itemize}
%     \item variants to the interactive narrative: we can transfer it to a traditional narrative graph generation problem, see appendix.
%     \item linear narrative or outline to the interactive narrative: WhatELSE planning
% \end{itemize}

% \zl{Need some content adapted from storyverse here. With some modifications in player input, prompting strategy, etc}

\yw{--- Beginning of Yi's revised version ---}

In this section, we present a pipeline to use the narrative space created in the previous stage to guide play-time plot execution through character actions. Simply stated, for each event in the outline plot, we employ an LLM-based  planning process to generate a sequence of character actions that acts out the event, with characters taking free actions in between the events - where the player character's actions are determined by the player's input, and non-player characters' actions are driven by an LLM.

We assume a {\em Game Environment} is given which contains a set of characters and an action schema that specifies character actions that are executable through the game system. The game environment also maintains the {\em World State}, which consists of a collection of variables that hold relevant values for the game mechanics, such as the characters’: location, attributes (e.g., strength and health points), relationship scores, as well as their memories from the simulation. Character actions correspond to executable function calls that modify the world state accordingly. For example, executing the action $\texttt{kills(X, Y)}$ will result in character $\texttt{Y}$'s state to become dead.

At game play-time, the main game loop alternates between 3 modes: 1) plot execution mode, 2) player action mode, and 3) character simulation mode. The process loops over the events in the outline plot, and stops when it exhausts all the events.

\noindent \textbf{Plot Execution Mode} \hspace{2mm} Given an event in the outline plot, we generate a sequence of character actions that acts out the event by iteratively executing the following steps \yw{refer to a figure similar to figure 3 in storyverse paper}:
\begin{enumerate}
    \item \textbf{Generation} \hspace{1mm} An LLM is prompted to generate a sequence of character actions that acts out the event. The prompt includes the following information from the game environment:
\begin{itemize}
   \item the list of characters and their descriptions;
   \item the action schema;
   \item current world state (including character's memory).
\end{itemize}
    \item \textbf{Review} \hspace{1mm} An LLM is prompted to reflect on the generated action sequence in terms of:
    \begin{itemize}
    \item \textbf{Overall Coherency Evaluation} Feedback is obtained by prompting an LLM to comment on the overall coherency of the generated plot and make suggestions for improvement.
    \item \textbf{Character Simulation Evaluation} For every action in the sequence, we prompt an LLM to play the role of the subject of the action. Given the current world state including the character’s memory, we ask the LLM if the motivation for the character to perform the action has been established. We include the explanation to this question in the feedback, if the motivation has not been established.
    \end{itemize}
    In addition, we leverage a simulated game environment for \textbf{External Evaluation}. We test the executability of the generated action sequence and report the observations on the success/failure of the execution. The combined feedback is added to the prompt for the next round of generation.

\end{enumerate}

Once the final sequence of character actions is generated, it will be executed by the game environment to update the world state. 
\begin{example}
For example, an event in the outline plot ``a small creature gets into an accident'' could be turned into a sequence of character actions in structured form:
\begin{verbatim}
{
    "subject": "dove",
    "action": "moveTo(oak_tree)",
    "thought": "I need to rest after saving the ant from drowning."
},
{
    "subject": "hunter",
    "action": "moveTo(oak_tree)",
    "thought": "I saw a dove flying towards the oak_tree. I should follow it"
},
{
    "subject": "hunter",
    "action": "tryToKill(dove)",
    "thought": "This is my chance to kill it while it's resting"
}
\end{verbatim}
\end{example}

Note that the events in the outline plot are at a higher abstraction level. This means the same event can be transformed into multiple character action sequences - this leaves rooms for the exact plot to adapt to different world states. In the above example, if the dove is dead at the time of plot execution, then a different character action sequence will be generated - one possibility is that the ant fell into the water.

\noindent \textbf{Player Action Mode} \hspace{2mm} In this mode, the player is prompted to input one or more actions following the action schema. The system execute each of the actions to update the world state if the action is executable with the current world state.

\noindent \textbf{Character Simulation Mode} \hspace{2mm} In the character simulation mode, an LLM is prompted to role-play as each of the non-player characters, and generate an action for this character. The prompt includes the following information:
\begin{itemize}
   \item the list of characters and their descriptions;
   \item the action schema;
   \item current world state (including character's memory);
\end{itemize}
Note that here the character actions are not directly determined by the outline plot. However, it is indirectly influenced by the executed events in the outline plot through the updated world states and characters' memory.


\yw{--- End of Yi's revised version ---}


\subsection{Example Workflow}

Below we present an example workflow to demonstrate some of the features described above. Alice, a novice creator, wants to create her own text adventure game based on the setting of a novel she enjoys. She attends a fan club for the novel, where a workshop is being held for members interested in using AI tools to create interactive narratives. During the workshop, Alice gains access to WhatELSE, along with a game engine preloaded with a story domain based on the novel. She is encouraged to create a short game to play and share the interactive narrative playing experience with club members.

\subsubsection{Encode Authorial Intent in Narrative Space} Alice begins with a few simple ideas for her interactive narrative, starting with a rough draft of the story and a moral she wants to convey: "Kindness is never wasted." Using the WhatELSE interface, she uploads her initial story into the system, which then generates an initial description of the narrative space based on her draft. On the left side of the interface, the story is displayed in the pivot view, showing a sequence of events; while an initial outline appears on the right, summarizing the key turning points of the pivot plot. Alice starts editing the pivot plot in free form, adding new elements, and conducting more creations on her story. Once she's satisfied, she clicks the Generate Outline button to update the outline based on her edits. She chooses to generate the outline at "act level", and further wants to prioritize the role of the witch in the narrative, so she adds a specification: "I want the witch to appear in every act, with no abstraction on the witch." After the outline is updated on the right view, Alice hover to see how each event in the outline is mapped to content in the pivot plot. She continues to explore among different levels and gets an ideal level of abstraction for her outline.

Though satisfied with the overall abstraction level of the outline, she notices that the danger in the second event "The hunter chases the dove to the mountain" is caused by a fixed character, the hunter, which she finds too restrictive. She selects the phrase "the hunter" and uses the abstraction tooltip to replace it, first with "a human character with power," and then with "a character." After exploring several iterations of the outline, Alice feels satisfied—her outline both conveys the moral she intended and leaves room for variations in the story.

Next, Alice wants to preview what players might experience, so she clicks the Generate Variants button. The system presents a scatter plot of potential narrative instances. Alice could identify different issues lying in her creation of the outline. For example, she finds the abstraction in a specific event is not ideal. Specifically, she scrolls through different stages of these instances—from start to end—she notices that some continuously express the moral, while others only reveal it toward the conclusion, both of which she considers acceptable. However, she also spots a cluster of instances that fail to express the moral by the end. Curious, she clicks on a dot representing one of these instances and reviews its details. Upon further investigation, Alice realizes that the problem lies in the overly abstract description of the source of danger in the second event. She had set it as "a character," which allowed undesirable narrative instances with animal characters chasing the dove. Similarly, she can also target issues including the outline being too abstract, so that the narrative instances rarely converge to express the moral, the outline being too concrete so the narrative space is too narrow where narrative instances all have limited variations. Additionally, Alice discovers some narrative instances that she is particularly interested in among the variants. She notices a set of three variants where the second event unfolds as, "the dove speaks with the hunter, leading the hunter to notice and then chase the dove." Alice finds this version more compelling than her pivot plot. She selects these three narrative variants and removes the others from the scatter plot. Satisfied with the direction of these variants, she clicks the Generate Outline button to create a new outline that summarizes the commonalities among the selected variants. After that, she returns to the outline editor, using the abstraction tools again to iteratively edit the outline to refine the boundaries of the narrative space, until she considers outline can enable both the expression of the moral and incorporate variations as she expected.

\noindent \textbf{Unfold the Narrative Space For interactivity}
With the narrative space now defined by the outline, Alice can experience the narrative instances unfolding from it in a turn-based text adventure format. She goes to the interactivity page of the WhatELSE web application and begins by seeing a description of the story domain, including character locations and introductions. Then, the first sequence of the game plot is generated:

\begin{verbatim}
    The hunter moveTo(forest), thinking: "I am starving. I really need some food."
    The deer moveTo(forest), thinking: "I will go back to the mountain after I get some nice leaves in the forest."
    The hunter attack(deer), thinking: "I see A deer! I think I have found my dinner!"
\end{verbatim}
After the first event, Alice, playing as the dove, needs to choose her next action from a pinpad. She decides to save the deer and can creatively combine actions from the pin pad to do so. For instance, she can choose the speakTo action and type her dialogue:
\begin{verbatim}
    The dove speakTo(hunter, "I will give you all the food that we get from the forest. Leave the deer alone.")
\end{verbatim}
Alternatively, she could opt for a more direct approach by choosing the attack action:
\begin{verbatim}
    The dove attack(hunter)
\end{verbatim}
Based on the events that have unfolded and Alice’s chosen action, the system’s narrative compiler simulates character behaviors and generates the subsequent plot described in the outline. Since other characters have not yet appeared in the story, Alice will see character simulations:
\begin{verbatim}
    The witch moveTo(mountain), thinking: "Let me go to the mountain and get some fresh herbs.")
\end{verbatim}
The next game plot is then generated, guided by the outline’s events written by Alice but also influenced by Alice’s earlier actions. The sequence of actions  begins:
\begin{verbatim}
    The hunter think: "Will it be worthy making deal with the dove? It sounds good, after all, what I need is just a meal.")
    ...
\end{verbatim}
This turn-based interaction continues, with Alice alternating between reviewing generated game plots, observing character simulations, and making decisions as the dove. Since she wrote a total of five events in her outline, the gameplay proceeds for five rounds, until all the events she planned have been played out. Since the game plots generated are fully structured, Alice plans to directly export the output of the narrative compiler into the game engine with the story domain preloaded, so that she can experience the interactive narrative in a more playable modality.

The scenario above demonstrates how creators can use WhatELSE to create their interactive narratives by first encoding the authorial intent into a narrative space, and then unfolding the narrative space into play-time plot execution. 

% \noindent \textbf{Utilization of Pivot Plot and Variant Plots} In addition to using the outline plot, the pivot plot and variant plots can also be used to guide the play-time plot execution. The utilization of these two representations follows an existing interactive narrative design paradigm.


% Starting with the pivot plot with concrete events, LLMs primarily function to construct scenes and character behaviors to enrich the content in the event listed as game plots. For instance, LLMs help to provide details to the brief event "The cat feels scared" into a detailed and vivid game plot "The cat thinks ('It may be my final day in the forest. I regret!')". Afterthat, game plots, and player actions can occur alternately to build the interactive narrative. However, when players take actions—such as exploring, conversing with characters, or making choices— in these interactive narratives, these actions have minimal impact on the overall plot development. This procedure is typical in interactive dramas or single-branch RPGs, exemplified by digital games like Final Fantasy or Genshin Impact.


% Thus, we use LLM-powered planning and LLM-powered player modeling to simulate the game plot and player action respectively.


% As previous studies found, player


% First, we simulated three types of players, inspired by previous studies ~\cite{yannakakis2013player}


% The outline generated based on the linear story with different abstraction levels will share the same intention but describing the narrative with varying levels of 


% Users can transfer their linear  type their own words of specified requirements of the




% Guided by the design goals, we developed the BALANCE system. BALANCE consists of three main components (1) Act Construction, which helps users construct acts based on their textual story. (2) Narrative simulation, which generates the interactive narratives based on LLM-empowered planning (3) Narrative review, enabling users to review the space of written interactive narratives and interactively refining. In the subsequent sections, we describe the details of each component and highlight their connection to the design goals.

% \subsection{Construction of Abstract Acts}
% BALANCE featured an authoring tool, shown in Figure. 1, to assist users in transforming their text drafts into abstract acts. In narrative creation, the construction of a narrative can contain the abstract

% Once the textual story draft is uploaded, the system formalizes it into a sequence of events displayed on the interface, as shown in Figure 2. Each event is represented in a box highlighting three elements: the subject, the action, and the object, to illustrate its content. Editing a formalized story features two functions: mosaic, which allows users to use abstract placeholders to replace specific characters and actions in the story, and locking, which enables users to highlight an event in further planning.

% \subsubsection{Mosaic} Mosaics on a formalized story aim to help users abstract the story to expand the narrative space. BALANCE supports the Mosaic method by enabling users to use a placeholder with a description to replace a specific element in the story. For instance, instead of specifying a concrete character, a prince, in the story, this character can be represented by a pair of pronouns with a description: "Character 1: a kind human being." To mosaic the element, users need to click on the element to select it in the text and then use the "Mosaic" button to create a placeholder for this element. Specifically, the placeholder asks the users whether this element is an action or a character and an optional text line that describes this placeholder. Furthermore, BALANCE allows users to mosaic elements locally in one specific event or globally across all events in the story by hovering on one element and using the system to select all text pointing to the hovered element automatically.

% \subsubsection{Lock} Lock on a formalized story aims to help users highlight the essential events in the story. Unlike the Mosaic operation, which can be operated all over text, the Lock operation is applied to events in the formalized story. To lock an event, users click the lock button associated with an event, and clicking again will unlock the event. Specifically, during the construction of abstract acts, locked events will not be merged with other events into an act and will always remain separate from other acts. For instance, an act construction would merge several unlocked events into a higher-level act. For instance, in Event 1, "the prince comes back to the kingdom," and in Event 2, "the princess is released from the prison and meets the princess," are merged into the act "The Prince and Princess Reunion.", if they are neither locked. Meanwhile, a locked event 1 will make the act constructor arrange two separate acts for these two events.

% \subsection{Narrative Simulation}
% After obtaining the abstract acts, the 


% \subsection{Iterative Refining}

% Once the abstract acts and the simulation of narrative is prepared for the user, the user can then simulate a series of 