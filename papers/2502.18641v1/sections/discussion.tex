The rise of generative models and prompt engineering has significantly impacted various fields and domains. Our user study results suggest that even laypeople can effectively use LLMs to create executable game plots for interactive narratives with appropriate interactive support. Therefore, it is valuable to continue exploring new interaction designs that assist novice users in creating interactive narratives. In this section, we discuss the implications derived from our system design and user studies, as well as the limitations and potential directions for future work.

\subsection{Revisiting the Design Goals}
%\qz{review on each design goal to see whether they are met - reflect on what are the wins vs challenges, what are the limitations of current approach to address the design goal, how should we make this better etc}
We revisit our design goals presented in Section~\ref{design_goals} and reflect on the degree to which they have been achieved. We also discuss the opportunities and challenges for designing future AI-bridged IN authoring systems.

\textbf{DG1: Enable users to perceive the narrative space} {\sc WhatELSE} provides three views for the user to perceive the narrative space: pivot, outline, and variants view. The pivot view displays a user-defined narrative instance as the representative example in the narrative space. In the study, we observed most participants (9/12) did not change the pivot upon generating the outline. Instead, they alternated between variants and outline view to edit the narrative space. Particularly, they found variants view useful in providing {\it ``inspiration''} (P2) or surprising instances, indicating people may \presentation{underestimate the size of narrative space} when authoring IN. Participants also used variant views to prevent instances from deviating from their authorial intent. The variants view shows the distance of each variant to the pivot, allowing users like P1 to quickly identify outliers and remove unexpected instances in their narrative space. Currently, the variants view displays variants based on authorial intent distance and emergent behavior distance from the pivot. It would be interesting to let users customize these dimensions and dynamically visualize the narrative space, similar to Luminate \cite{suh2024luminate}.

\textbf{DG2: Support configurable level of abstraction in editing narrative space} \presentation{Our tools give users the flexibility to adjust the level of detail in their authoring of the narrative space. With the abstraction ladder, they can configure the overall outline, while the abstraction tooltip lets them fine-tune sentences or individual words.}
This granular control helps users create the narrative space more effectively. The user study reflected this with higher ratings on user control (Figure~\ref{fig:questionnaire}.a) and positive interview responses. Additionally, the technical evaluation results in Table~\ref{tech_eval:abstraction_ladder} demonstrate the abstraction ladder's effectiveness in managing different levels of abstraction. In the user study, participants with narrative writing experience easily understood the abstraction ladder levels, referring to a level as {\it ``story beats''} (P2). Other participants needed more exploration to find their desired level of abstraction. While the abstraction ladder is designed based on narrative structure, future work should explore making the levels more intuitive for people unfamiliar with narrative writing.

\textbf{DG3: Generate meaningful game events that react to player actions at play-time} We adopted an LLM-based narrative planning method that generates causally sound character actions that act out the event defined in the outline. In the study, participants were more satisfied with the game events generated by {\sc WhatELSE} compared to the baseline (Figure~\ref{fig:questionnaire}.c). They found that the choices they made in the game had a more realistic impact on the characters. Furthermore, the results of the technical evaluation in Table~\ref{tech_eval:action_impact} demonstrate that {\sc WhatELSE} can generate diverse plots that respond to the player’s contrasting actions. Like other LLM-based narrative generation, the quality of generated plots is limited by the challenge of preserving long-term dependency and coherence \cite{mirowski2023co}. Future work should explore approaches such as increasing the LLM’s context window \cite{kaddour2023challenges} and adopting a hierarchical generation approach \cite{mirowski2023co}.



% \subsection{Uncertainty Visualization of LLM-generated Game Plots}


\subsection{Intuitive and Analytic Thinking in Authoring Narrative Space} 
%The process of constructing narrative space in \textsc{WhatELSE} aligns with the dual-process theory, commonly referred to as System 1 and System 2 thinking. These two modes of thinking, characterized by fast, intuitive thinking (System 1) and slow, deliberate reasoning (System 2), map onto how creators interact with narrative instances and the abstraction in narrative space shaping.

\presentation{WhatELSE supports both intuitive and analytical thinking when authoring IN. 
Authoring IN with instances aligns with people's natural writing process. } It is also straightforward to judge the moral expression in a narrative instance. When authors begin with uploading a draft story, their edits and creations on narrative instances reflect intuitive storytelling, where specific events guide their thinking about what could happen next.  

More analytical thinking occurs when authors abstract the outline to shape the narrative space. The act of replacing specific characters or events with abstract descriptions, such as changing {\it "the hunter"} to {\it "a human character with power"}, \presentation{is a deliberate attempt}. This process of using language abstraction to incorporate greater diversity requires \presentation{considerations and planning}. By gradually refining the outline through abstraction, authors weigh different narrative potentials and ensure that the result still conveys the desired moral.

{\sc WhatELSE} enables creators to jointly leverage both modes of thinking in their creation. By starting with concrete narrative instances with clear perception, creators can quickly ground their ideas and establish the foundation for the narrative space to shape. Through the process of abstraction, they deliberately decide operations to shape the narrative space. This transition between intuitive and analytical thinking may serve as a plausible explanation for {\sc WhatELSE}'s advantage for authors to maintain control over the narrative space while allowing for creative flexibility.


\subsection{Mixed-initiative Design for LLM Applications}
%In recent years, the advancement of generative models has led to a trend of using LLMs to replace existing interaction designs entirely, under the assumption that natural language queries are the most intuitive interaction mode. However, recent studies suggest that mixed-initiative approaches still hold value. 
%While free-form conversational agents such as ChatGPT provide high user agency, they can also present challenges. As P7 noted, {\it ``I was typing a lot, as sometimes it gets me and sometimes it does not.''} This highlights that, especially for novices, free-form interaction can hinder effective collaboration with AI and the generation of high-quality content.

Participants' feedback highlights that, compared to fully free-form conversational assistants, \textsc{WhatELSE} is more helpful for novices in creating their games. As P12 noted, {\it ``I can explore what the different versions (abstraction levels) look like''}, and P6 added, {\it ``(WhatELSE) allows you to explore more.''} These comments point to a deeper reason behind the effectiveness of {\sc WhatELSE}. Though conversational-based chatbots seem to provide infinite options to choose from in creation, they are overwhelming to novices. P1 has been acutely aware of this challenge and left an opinion: {\it ``Maybe someone who is an experienced writer could be like.. I want to structure the plot this way. But as me, a person who isn't a writer, it's hard to really know what to do (with the baseline)''}. 

In contrast, the mixed-initiative design of \textsc{WhatELSE} primes users to explore more in sculpting the narrative space in their iterative trials by implicitly guiding them in a structured workflow of outline generation, of which P14 named as {\it ``indicators of what I'm trying to do''}. P2, who has moderate experience in narrative writing, also noticed the design and said, {\it ``It's very close to how I think when I write a story, those levels, those metrics''}.


\subsection{Strategic Integration of \textsc{WhatELSE} in Interactive Narrative Creation}

A number of IN authoring tools have been developed, including branching-based systems like Twine \cite{friedhoff2013untangling} and event-node graph frameworks \cite{chung2024patchview}. \textsc{WhatELSE} is designed to complement, rather than compete with, these existing tools. We aim to enable \textsc{WhatELSE} to augment existing tools by addressing the challenges widely applied to interactive narrative creation so that a more comprehensive workflow can be supported.

%Starting with the simplest integration, the two components of our system, can either be independently coupled with traditional systems, thereby enhancing the support traditional systems provide for creation. 
\presentation{\textsc{WhatELSE} can be used in combination with traditional authoring tools in several ways.}
For example, the variants generated by \textsc{WhatELSE} can be converted into event diagrams or storyline branches. Therefore, the shaped narrative space via the narrative space editor can be exported to traditional interactive narrative authoring tools for fine-grained manual editing. Similarly, the fully structured representations of the narrative space can be exported to the interactive narrative compiler, as a clearer guide used in the plot execution mode.

Furthermore, our system aims to address the broader challenges of AI-bridged creation of interactive narratives. Within this paradigm, \textsc{WhatELSE} provides creators with the tools to shape, refine, and manage the narrative space. This concept is closely connected to the design space explored in other creative domains; thus, \presentation{studies that investigate the design exploration and management} in design space can be further adapted to this specific application for narrative space editing~\cite{ez2022design,suh2024luminate}, with their proposed techniques providing another view of the narrative space.

% without using the interactive narrative compiler to generate play-time gameplots. 


%Integration with existing tools like Twine or existing branching interface; Complementary with existing interface such as Luminate;


\subsection{Branching Outline with Multiple Pivots}
%Although the narrative outline, defined as a sequence of events, might have a linear structure, we showed that it can be unfolded into multiple storylines by interpreting the abstract events in a different way. 

It is possible to combine the traditional branching storyline graph representation with our outline representation, to obtain branching outline graphs where each node represents an abstract event. Figure \ref{fig:multipivot} shows such a branching outline, where each path in this graph corresponds to multiple concrete storylines.

One potential extension of this work is to generalize our workflow and system to support creating such branching outline. The more general system should allow the user to provide multiple narrative examples as input to initialize the narrative space, and multiple instances in the narrative space should be able to serve as pivots. 
\usecase{For instance, the user can provide three example narratives as pivots to indicate three types of storylines \presentation{(Figure \ref{fig:multipivot})}}.

%\begin{itemize}
%    \item An ant fell into water and was drowning. The dove (protagonist) saved the ant by dropping a leaf in the water. Later, the dove was chased by the hunter, but was saved by the ant and lived.
%     \item An ant fell into the water and was drowning. The dove (protagonist) dropped a leaf in the water. The ant tried to grab the leaf but couldn't and died. The ant's friend bee saw the dove's attempt and was grateful to the dove. Later, the dove was chased by the hunter, but saved by the bee and lived.
%    \item An ant fell into the water and was drowning. Nobody saved the ant. A dove was chased by the hunter and was killed by the hunter.
%\end{itemize}

\begin{figure*}
    \centering
    \includegraphics[width=.98\textwidth]{figures/discussion-multipivot.pdf}
    \caption{\presentation{Illustration of a narrative space generated by a branching outline with three pivots to indicate three types of storylines.}}
    \label{fig:multipivot}
    \Description{}
\end{figure*}

Branching outline representation expands the expressivity of both our outline representation and traditional branching graph representation. Our system has shown a linear outline can be unfolded to instantiate the same story structure in multiple concrete forms. A branching outline representation can potentially support organizing multiple story structures together for expressing more complex themes. However, it also raises new challenges for both technical pipeline and interface design, for example, how to computationally construct branching outlines from multiple example narratives, and how to assist the user in better perceiving connection between different narrative instances from the same and different abstract branches, especially when the boundary between different abstract branches can be blurred due to the fuzziness of the trigger conditions expressed with abstract language.

\section{Limitation and Future Work}
\label{sec:limitation}

We describe the limitations of our work to clearly define the scope of our findings and inspire future research directions.

%\subsection{Limitations}

\paragraph{Study Design} We evaluated our system through an in-person study \presentation{with 12 participants who have minimal IN authoring experience and moderate generative AI experience. The study task used a simple story domain with a template story to minimize learning burdens}. In real-world scenarios, IN creation typically involves a larger narrative space with more complex plots. Therefore, studies that use more complex creation tasks and have a more diverse participant background, such as including people with limited AI experience or professional IN writers, would give us a deeper understanding of the usability, effectiveness, and generalizability of our work. \evaluation{Additionally, {\sc WhatELSE} has not been directly compared with traditional IN authoring tools. While a full comparison of {\sc WhatELSE} and several traditional tools across an entire IN authoring process may not be an overkill, experiments targeting specific perspectives, such as branching capability, can better clarify the unique contribution or limitation of {\sc WhatELSE}. }



\evaluation{\paragraph{Player Experience} Although this work focuses primarily on the author's experience, the player experience is also important. While the results of the technical evaluation show that {\sc WhatELSE}  generates diverse plots that respond to simulated player’s actions, future studies the collect subjective player feedback, such as their engagement and enjoyment through crowdsourced assessment of text-adventure games generated using {\sc WhatELSE}, would help us better understand this system.  
%Feedback from players may echo our technical findings, indicating that INs authored using \textsc{WhatELSE} enable more meaningful world state changes, ultimately enhancing player engagement and satisfaction. 
}
% A rigorous study, such as a crowdsourced evaluation of text-adventure games generated using \textsc{WhatELSE} compared to those authored with baseline systems, could provide deeper insights into the strengths and limitations of \textsc{WhatELSE} from the player perspective.
% Conversely, their player behaviors existing beyond modeled players, may hint the further improvement on IN authoring design to provide a flexible player modeling to designers. 

\paragraph{Interface Design} 
%While integrating \presentation{the pivot, variant, and outline view gives users more options to perceive the narrative space}, it can also place a learning load on novices. Participants \presentation{suggested that a higher-level LLM agent can be used to trigger existing functions through natural language commands}, making the interaction flow intuitive. 
The system provides the pivot, variant, and outline view to help perceive the narrative space. Within these three views, more intuitive feedback could be introduced such as providing quantifiable measures of abstraction and concreteness in the outline, similar to the \presentation{metrics of authorial intent and emergence for variants.} While the outline view depicts the boundary of the narrative space, the outline itself is a free-form natural language query. By incorporating classic views of IN, such as a branching diagram, we could explore using a semi-structured outline that combines the flexibility of natural language with the clarity of structured representations. Additionally, introducing an intermediary component between the mutual transformation between narrative instances and the outline, such as an event diagram, would create a "trinity" of instance, semi-structure, and outline, offering more control and granularity in narrative creation.


\paragraph{Export the Game Plot} One of the key advantages of our approach is that game plot is generated through narrative planning, making it fully structured and controllable. \presentation{It is possible to deploy the generated game plot into a real game engine so that each event in the plot corresponds to an in-game function call.} 
Future work will leverage this advantage to use an existing game engine to serve as an intermediate between \textsc{WhatELSE} and game players, where both the game plots and player actions will operate the game engine, embodying the generated game plot in a real gaming experience. 

%\subsection{Future Work}
%\subsubsection{Understand the System More Thoroughly} \evaluation{First, as noted above, future evaluations should incorporate the player perspective, such as through crowdsourced evaluation of text-adventure games generated using WhatELSE. Additionally, rigorously controlled experiments should be conducted to compare \textsc{WhatELSE} with widely applied IN authoring tools on tasks within the IN authoring process, including branching, plot execution, and other core functionalities. Further studies should explore more complex story domains with participants from more diverse backgrounds. For instance, future research should involve experienced creators with intrinsic motivations, such as creating fan games or producing games based on their original stories. }





%\subsubsection{Export the Gameplay into Game Engine}
%One of the key advantages of our approach is that game plot is generated through narrative planning, making it fully structured and controllable. \presentation{It is possible to deploy the generated game plot into a real game engine.}  Future work will leverage this advantage to use an existing game engine to serve as an intermediate between \textsc{WhatELSE} and game players, where both the game plots and player actions will operate the game engine, embodying the generated game plot in a real gaming experience.

%\subsubsection{Augment Views of Narrative Space}
%As we mentioned earlier, our system is designed to complement existing authoring tools for interactive narratives, and other views of the narrative space can be integrated into our approach. Despite the advantages of using outlines, the outline itself is a free-form natural language query. Incorporating classic views of narrative space, we could explore the use of a semi-structured outline that combines the flexibility of natural language with the clarity of structured representations. Alternatively, introducing an intermediary component between the mutual transformation between narrative instances and the outline, such as an event diagram, would create a "trinity" of instance, semi-structure, and outline, offering more control and depth in narrative creation.





\section{Conclusion}

Generative AI advances interactive narrative creation by enabling just-in-time content generation that adapts to player choices. However, this increased interactivity makes it difficult for authors to control the narrative space. In this paper, we introduced WhatELSE, an interactive narrative authoring system that tackles this challenge via a mutual transformation between narrative instances and narrative space. Through its three views—narrative instance, outline, and variants—WhatELSE empowers authors to perceive and shape narrative boundaries using linguistic abstraction. By leveraging an LLM-based simulation, \textsc{WhatELSE} further unfolds narrative spaces into executable gameplots. Our user study (N=12) and technical evaluations showed that \textsc{WhatELSE} enables the creation of structured yet flexible gameplots, making it an effective tool for interactive narrative creation. We believe our work contributes to advancing creators' balance of their authorial intent and player interactivity in AI-bridged interactive narrative creation.