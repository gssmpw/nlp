To understand the user needs of creating interactive narratives with a balanced authorial intent and emergent behavior, we reviewed previous studies that have proposed computational tools for narrative creation. Based on the findings and design implications, we summarized the design goals of BALANCE, and elaborated on each design goal as follow.

% \noindent \textbf{G1: Generation of Interactive Narrative from Text}

% \noindent \textbf{G1: Construction of Interactive Narratives from Text by Harnessing LLMs}
% Creating interactive narratives is always hard, even for experts with domain knowledge. Meanwhile, LLMs can provide help in various stages of the creation paradigm, from writing and planning to implementation. Thus, we designed the LLM-involved design flow to ease the process of generating interactive narratives from plain text.

\noindent \textbf{N1: Transformation from linear narrative to interactive narrative}


\noindent \textbf{N2: Providing representations of the narrative space behind a linear narrative}

\noindent \textbf{N3: Supporting users explore different prompting options}

\noindent \textbf{N4: Supporting users explore different prompting options}



\noindent \textbf{}



% \noindent \textbf{G1: Assist Users in Expanding the Narrative Space} Though non-linear storytelling is widely adopted in interactive narrative creation as it scopes a larger narrative space, writing complex plots exemplified by branching is always an issue, especially for laypeople writers. Commonly, people are more used to linearly progressing the narrative, such as a procedure from cause to outcome, which leads to a narrow narrative space with limited variations. To mitigate this, BALANCE introduces a two-step high-level editing process: Initially, it enables users to employ placeholders with descriptions to represent specific characters and actions temporarily. Subsequently, these placeholders can be filled with contextually appropriate characters and actions during the narrative planning phase. This method allows BALANCE to aid users in crafting a sequence of narratives that vary significantly at the event level, offering diverse outcomes based on different character actions and interactions, thereby significantly expanding the narrative space.

% \noindent \textbf{G2: Assist Users in Narrowing Down the Narrative Space} 
% While the diversity of narrative paths enriches the storytelling experience, restricting the narrative space with a clear focus on the author's intended message is crucial. Research has demonstrated that the effectiveness of conveying the authorial intent relies on the realization of key events within the narrative. BALANCE addresses this by offering users the ability to "lock" specific events, ensuring these pivotal moments are preserved and prominently featured in the narrative progression. Locked events are treated as critical acts that are faithfully incorporated into subsequent planning and narrative generation phases. This feature ensures that, despite the expansive possibilities introduced in the narrative creation process, the core message and essential developments envisioned by the author remain intact and are clearly communicated throughout the story.

% \noindent \textbf{G3: Assist Users in Iteratively Reviewing the Created Narrative Space} Effective review and refinement of narrative space are pivotal in interactive narrative creation, as emphasized by prior research. Unlike linear narratives, which can be edited directly through text, the complex and branching nature of interactive narratives presents unique challenges in review and refinement processes. To address this, BALANCE equips users with tools to visually explore the variations within the created interactive narratives. It visualizes the narrative space by mapping out the different plot trajectories that emerge from varying story developments and player interactions. This dual-dimensional approach allows users to see how different choices or player behaviors affect the narrative outcome, enabling a comprehensive and iterative refinement process.





% \noindent \textbf{G5: Efficiently }



% \noindent \textbf{G5: Assist Users in Clarifying the Authorial Intent}




