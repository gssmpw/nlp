% \subsection{V1}
% Narrative, as the fundamental interaction method that human beings use to perceive the world, has been reframed by modern information systems. As the most ubiquitous approach empowered by computing technologies, interactive narrative makes it possible for its participants to experience the narrative more than an observer. However, the complexity of creating interactive narratives can pose significant barriers for beginners.
% For example, the initial ideation and planning phases, crucial in the early stages of the creative process, can be challenging for those unfamiliar with narrative concept development. Furthermore, editing operations often involve meticulous selection, trimming, and sequencing of clips to create a coherent narrative. This not only requires mastery of the often complex user interfaces of editing software but also significant manual effort and
% storytelling skills.

% Recently, LLMs have been used to address the challenges associated with interactive narrative creation. Utilizing LLMs as an interaction medium for video editing allows users to convey their intentions directly, bypassing the need to translate thoughts into manual operations. For instance, recent AI products [5] allow users to edit video leveraging the power of text-to-video models; voice-based video navigation enables users to browse videos using voice commands instead of manual scrubbing. In addition, language has been used to represent video content, thereby
% streamlining the manual editing process. A prominent example is text-based editing, which enables users to efficiently edit a narrative video by adjusting its time-aligned transcripts. Despite these advancements, the majority of interactive narrative creation tools still heavily rely on well-written scripts with intricate dramatic conflicts and complex branching plots. Consequently, interactive narrative  is still not an ubiquitous medium that everyone can create to express themselves.

% \subsection{V2}
% Interactive narratives, an increasingly prevalent medium advancing along with modern information systems, have demonstrated significant advantages over the past decades. Represented by digital games, interactive narratives allow authors to express their intent while ensuring interactivity for the audience. Recently, the involvement of artificial intelligence technologies, exemplified by large language models (LLMs), has further expanded the potential of interactive narratives. For instance, games like inworld.AI enable characters to have their own agency within the narrative, enhancing the user's interactive experience to new heights. In such an era, LLM-involved interactive narrative has differed from traditional in

% The involvement of generative artificial intelligence (AI) in interactive narratives has reshaped this medium, which uniquely balances author expression and audience interactivity. By shifting agency from the author to the characters within these narratives, the 'interactive' aspect is greatly enhanced. For instance, platforms like inworld.AI enable characters to interact autonomously with the audience, requiring only initial character setup by the authors. Consequently, interactive narratives have entered a new era, distinctly different from traditional interactive storytelling.

% As technology advances, so do the tools aiding the creation of interactive narratives. Over the past decades, significant research has focused on facilitating authoring, ranging from script writing to gaming world-building ~\cite{starks2018branched}. Recent studies, such as ~\cite{venkatraman2024collabstory}, have further explored generative AI's role in co-writing interactive narratives. Additionally, ~\citeauthor{kumaran2023scenecraft} investigated using LLMs to automatically generate critical scenes for interactive narratives. Despite these advancements, most interactive narrative tools still cater to traditional narrative writing, neglecting the great difference between LLM-involved interactive narratives and traditional interactive narratives. 

% Notably, the construction of storylet spaces \cite{evans2013versu} in interactive narratives sets the boundary between AI-involved interactive narratives and traditional interactive narratives. A storylet is a narrative unit, akin to an act in drama. In traditional interactive narrative creation, both expression and interactivity are achieved through meticulously crafted storylets. An established storylet space scripts detail not only the narrative plots but also each character's response to specific user actions. For example, in a game about the red hat, there might be two main branches: the red hat either debunks the wolf's trick or does not, with users experiencing one of these outcomes. Consequently, the author manually defines the space of the storylets. In contrast, AI-involved interactive narratives feature fully autonomous agents whose behaviors are driven by LLMs. Player interactions with these agents can further alter the plots, making the narrative space more flexible. For instance, in the AI-involved narrative game AI Dungeon, players interact with AI-driven characters that can respond in real-time to player inputs, creating a dynamic and constantly evolving story. Thus, creating LLM-involved interactive narratives has become different, particularly regarding the storylet space. 

% Therefore, a 

%\yw{-- Beginning of Yi's revised introduction --}

%\yw{Interactive Narrative}
Interactive Narrative (IN) is a form of digital storytelling experience where the player can influence a dramatic storyline through their actions \cite{riedl2013interactive,green2014interactive}. 
IN takes various forms in entertainment and education applications, with the most prominent one being branching storylines in role-playing games \cite{riedl2006from}, where the author predefines the range of player actions and creates multiple storylines reflecting the consequences of different player choices. Instead of one single narrative, the author creates a {\em narrative space} consisting of all possible storylines a player can experience. 

%\yw{Generative AI for IN}

The advancement of Large Language Models (LLMs) has the potential to revolutionize IN by enabling the automatic generation of content based on the user's specifications~\cite{li2024pre, li2023synthetic,wei2022chain}. This enables just-in-time generation of narrative content to adapt to different game world states. Instead of enumerating all possible storylines, authors can convey their broad narrative intent to the LLM as prompts, and let the model render concrete narrative instances customized by the player's in-game context \cite{kim2023language,boriskin2024lsg,sweetser2024large,peng2024player}. For example, Inworld Origin \cite{inworldorigin} is a narrative-driven adventure game where the player interacts with LLM-driven characters to solve a mystery. The characters respond to the player with unscripted actions and dialogs that are generated by LLMs at play-time, while still adhering to an overall narrative structure.
%This bridges the author and player by personalizing the author's vision to the player's context, allowing just-in-time narrative creation to adapt to player actions. 
%For example, AI Dungeon \cite{hua2020playing} is a text-based adventure game where the player interacts with the game world with free-form text input. An LLM generates response to player input based on a text prompt describing the world, creating various adaptive storylines. 
%Inworld Origin \cite{inworldorigin} is a narrative-driven adventure game where the player interacts with LLM-driven characters to solve a mystery. The characters respond to the player with unscripted dialogues that are generated by LLMs at play-time, while still adhering to an overall narrative structure.
%Instead of specifying in an exact form all possible storylines that an audience can experience, the authors can now utilize a Large Language Model (LLM) to bridge between the authors and the audiences, to transform at a scale their broad narrative intent to concrete narrative instances based on the audience's action \cite{kim2024authors}. 
AI-bridged IN significantly enhances player agency as the player can now influence the story in ways not restricted by predefined story branches, and also facilitates a form of emergent narrative \cite{aylett1999narrative,suttie2013theoretical} where the final narrative experience can even possibly go beyond the author's anticipation. 

%\yw{Challenge raise by AI-bridged IN}
%\qz{v1. no "unfold" problem in this paragraph..} However, creating AI-bridged IN introduces new challenges for authors. While the authors can convey their authorial intent using prompts, it can be difficult to envision specific instances within this possibility space and reshaping it by tuning prompts \cite{kim2024authors}. Authors usually rely on extensive trial and error \cite{zamfirescu2023johnny} to get a limited understanding of the narrative space, and bases design decisions on anecdotal evidences. Furthermore, prompts are usually unstructured \cite{suh2024luminate}, making it difficult to constrain the boundaries of narrative variations. To shift from a traditional IN to AI-bridged IN, authors have to let go control over concrete narrative content, and instead only vaguely constrain the narrative space using prompts. For example, instead of specifying a story with \textit{``an ant fell into water and was then saved by a dove by dropping leaf to the water''}, the author writes \textit{``a story where a small creature got into an accident and was saved by another creature''}, allowing the storyline to adapt based on player actions. To create AI-bridged IN, the author has to specify the narrative content at a much more abstract level. The abstract specification is an effective way to compactly sketch out a narrative space and impose high-level authorial control. On the other hand, a full understanding of an abstract description involves grounding it in concrete experiences \cite{barsalou2008grounded}. 

%However, creating AI-bridged IN raises new challenges for the the author to control the narrative space.
%- within which the final narrative experienced by the player emerges from their interaction with AI. 
\presentation{To create a traditional IN, the author directly specifies} {\textbf{concrete narrative instances} conditioned on game world states, while in the case of AI-bridged IN, the author has to express their narrative intent as prompts - which requires them to write \textbf{abstract narrative specifications}. \revision{ The abstract specification eventually transforms into concrete narrative instances at play-time based on the player's interaction with the game system.} For example, instead of specifying a story where \textit{``an ant fell into the water and was then saved by a dove by dropping a leaf to the water''}, the author writes in a prompt \textit{``a story where a small creature got into an accident and was saved by another creature''} so that the exact plot can be generated based on the play-time game world states.
Using abstract narrative specification to guide play-time plot progression is an effective way to compactly sketch out a narrative space and impose high-level authorial control.

However, this AI-bridged IN workflow presents challenges both to the author and the AI system, as concrete narrative instances are not accessible at the time of authoring. On one hand, authors struggle to create, perceive, and control the narrative space just by prompting \cite{kreminski2024intent}. Once they write prompts with abstract narrative specification, it is difficult for them to envision specific instances within the defined narrative space \cite{kim2024authors}. On the other hand, it is also difficult for LLMs to unfold the author's narrative intent into a sequence of events that are executable in an external game environment, as LLMs are not trained to simulate the causal dynamics defined by the game mechanism and are known for challenges in maintaining long-term consistency \cite{mirowski2023co}.

%However, in this AI-bridged IN workflow, concrete narrative instances are not accessible at the time of authoring. It is challenging for both the author and the AI system to envision or render concrete narrative instances that reside in a narrative space defined by abstract narrative specification. On one hand, authors struggle to create, perceive and control the narrative space just by prompting. Extensive trial and error \cite{zamfirescu2023johnny} is often needed to get a limited understanding of the narrative space, leading to design decisions based on anecdotal evidences. On the other hand, it is difficult for LLMs to unfold the author's narrative intent into a sequence of events that are executable in an external game environment, as LLMs are not trained to simulate the causal dynamics defined by the game mechanism and are known for challenges in maintining long-term consistency \cite{mirowski2023co}.
%However, shifting from a narrative space defined by concrete narrative instances to one defined by an abstract narrative specification raises challenge for the authors in perceiving and controlling the narrative space. 

%However, shifting from traditional IN to AI-bridged IN authoring raises challenge for the authors in perceiving and controlling the narrative space, 
%On one hand, it is difficult for the authors to articulate their implicit narrative intents behind the concrete narrative examples in their mind. Prior work has found that authors themselves can be unaware of at least some aspects of their own intent and they often underexpress their intent to AI systems \cite{kreminski2024intent}. On the other hand, once an abstract narrative specification is created, it is difficult for the author to envision specific instances within the defined narrative space and reshaping it by tuning prompts \cite{kim2024authors}. 
%It is difficult for the author to envision what specific instances reside in this possibility space defined by their prompts, and further shape and fine-tune the space by tuning their prompts \cite{kim2024authors}. The author usually relies on extensive playtesting to get a limited understanding of the narrative space, and bases further design decisions on anecdotal evidences collected in this unstructured process.

%To create AI-bridged IN, the author has to specify the narrative content at a much more abstract level than traditional IN. This abstract specification is an effective way to compactly sketch out a narrative space and impose high-level authorial control. For example, instead of specifying a story with \textit{``an ant fell into water and was then saved by a dove by dropping leaf to the water''}, the author writes \textit{``a story where a small creature got into an accident and was saved by another creature''}, allowing the storyline to adapt based on player actions. 

%However, this abstraction introduces challenges for authors in perceiving and controlling the space of possible narratives - within which the final narrative experienced by the player emerges from their interaction with AI. It is difficult for the author to envision specific instances within this possibility space and reshaping it by tuning prompts \cite{kim2024authors}. We refer to this as the \textbf{abstraction problem}. On the other hand, once the space of possible narratives has been created, transforming it into a consistent game plot requires effective narrative planning to generate meaningful event sequences. Central to this success is the logical causal progression of the game plot \cite{riedl2010narrative}. However, LLMs are not natively planners in creating causally sound plot progression and have been found to cause hallucinations without external verifier to validate the coherency and executability of the generated plan \cite{kambhampati2024llms}. We refer to this as the \textbf{planning problem}.

%A natural idea is to assist AI-bridged IN creation with LLMs automating the transformation between concrete examples and abstract specifications. However, it is not an easy task for LLMs. Concrete examples and abstract description of narrative intent do not contain sufficient information to reconstruct one from the other \cite{kreminski2024intent}, leaving large room for hallucination if we simply prompt LLMs to generate examples from specification or vice versa, especially with LLM's known challenge of preserving long-term dependency and coherence \cite{mirowski2023co}.

To address these challenges, we present {\sc WhatELSE}, an IN authoring system that creates interactive narratives from user-provided example narratives.  
Instead of writing abstract prompts, authors can import narrative instances as {\em pivots} to create a narrative space. The system generates an {\em outline} by abstracting from these instances. To help authors perceive the space, the system uses a simulation process to generate concrete narrative {\em variants} from the outline-defined space. Together, the system uses {\em pivot}, {\em outline}, and {\em variants} to represent and shape the narrative space: (1) authors can shift the space by directly editing the {\em pivot} (Figure~\ref{fig:teaser}.b), (2) they can expand or constrain the narrative space by changing the {\em outline's} level of abstraction (Figure~\ref{fig:teaser}.c), (3) they can also fine-tune the narrative space by removing {\em variants} (Figure~\ref{fig:teaser}.d). 

%The system supports better understanding of the narrative space by grounding abstract narrative specifications on concrete narrative instances, while leveraging the abstractness as a tool to shape the narrative space. 

To support {\sc WhatELSE}, we developed a technical pipeline that supports the bidirectional transformation between outlines and instances using LLM and narrative planning. To generate instances from an outline, we developed a novel LLM-based narrative planning method \revision{following the LLM-Modulo frameworks proposed by Kambhampati et al. \cite{kambhampati2024llms}}, \revision{taking into account possible play-time world states and player behaviors}. Narratives are grounded by character action sequences executable in the game environment and are iteratively generated and reviewed. We leverage an external simulated game environment to guide the validation and revision of plot generation to ensure that the causal dynamics in the game environment are correctly captured by the generated plots. 
%\yw{WhatELSE system}
%To demonstrate the workflow, we present {\sc WhatELSE}, an IN authoring system that creates narrative spaces from user-provided narrative examples. With innovative technical pipeline, the system generates an outline that describes the narrative space by abstracting a general structure from the narrative examples, and use the outline to generate concrete narrative variants from the narrative space with a simulation process. We provide an abstraction ladder tool and an abstraction tooltip to assist the user to intuitively constrain and relax the narrative space by modifying the outline, and a narrative variant view to help the user perceive and fine-tune the narrative space. 

%We demonstrate that the created narrative space can be used to guide play-time narrative content generation with our LLM-based narrative planning method. Through a user study ($N=12$) and technical evaluations, we found that {\sc WhatELSE} enables game plot generation to achieve a balance between authorial expression and player agency via the coordination between abstract and concrete narrative representations.

%In summary, our contributions include:
%\begin{itemize}
%\item A novel LLM-based narrative planning method for generating narrative plots expressing authorial narrative intent grounded by a game environment.
%\item {\sc WhatELSE}, an interactive system for creating AI-bridged Interactive Narrative that leverages transformation between abstract narrative specification and concrete narrative instances to help the author better perceive and shape the narrative space.
%\item A user study demonstrating how {\sc WhatELSE} enables creating interactive narratives to achieve a balance between authorial control and player agency.
%\end{itemize}

%\yw{-- End of Yi's revised introduction --}

This work has three main contributions: 1) an IN authoring system that allows users to shape the narrative space at different levels of abstraction using the outline and instances, 2) a technical pipeline that supports bidirectional transformation between outlines and instances using LLMs and narrative planning; 
and 3) findings from a user study (n=12) and a technical evaluation. Results from the user study showed that \textsc{WhatELSE} helped authors perceive and edit the narrative space. It also demonstrated that the created narrative space could be used to generate engaging, interactive narratives at playtime. Our technical evaluation demonstrated the effectiveness of the pipeline in supporting language abstraction and generating diverse plots that react to player actions.






