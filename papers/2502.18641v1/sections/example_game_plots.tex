
\documentclass[nonacm]{acmart}
\renewcommand\footnotetextcopyrightpermission[1]{} % removes footnote with conference information in first column
\pagestyle{plain} % removes running headers

\AtBeginDocument{%
  \providecommand\BibTeX{{%
    \normalfont B\kern-0.5em{\scshape i\kern-0.25em b}\kern-0.8em\TeX}}}

\newcommand{\comments}{1}
\ifdefined\comments
   \newcommand{\zrl}[1]{\textcolor{red}{[Zhuoran: #1]}}
   \newcommand{\my}[1]{\textcolor{blue}{[Ming: #1]}}
\else
   %\newcommand{\cj}[1]{}
\fi
\newcommand{\ignore}[1]{}
\newcommand{\squishlist}{
   \begin{list}{\small $\bullet$}
    { \setlength{\itemsep}{0pt}      \setlength{\parsep}{0.5pt}
      \setlength{\topsep}{0.5pt}       \setlength{\partopsep}{4pt}
     \setlength{\leftmargin}{1.2em} \setlength{\labelwidth}{1em}
      \setlength{\labelsep}{0.5em} } }
\newcommand{\squishend}{  \end{list}  }

\usepackage{multirow}

\newcommand{\tabincell}[2]{\begin{tabular}{@{}#1@{}}#2\end{tabular}}

\usepackage{multirow}
\settopmatter{printacmref=false}

% \author{Zhuoran Lu}
% \email{lu800@purdue.edu}
% \affiliation{%
%   \institution{Purdue University}
%   \streetaddress{305 N. University Street}
%   \city{West Lafayette}
%   \state{IN}
%   \country{USA}
%   \postcode{47907}
% }
% \author{Ming Yin}
% \email{mingyin@purdue.edu}
% \affiliation{%
%   \institution{Purdue University}
%   \streetaddress{305 N. University Street}
%   \city{West Lafayette}
%   \state{IN}
%   \country{USA}
%   \postcode{47907}
% }

\usepackage{wrapfig}
\usepackage{multirow}
\renewenvironment{quote}{%
  \list{}{%
    \leftmargin0.5cm   % this is the adjusting screw
    \rightmargin\leftmargin
    %\onehalfspacing
  }
  \item\relax
}
{\endlist}


\usepackage{subcaption}
\usepackage{enumitem}
\usepackage{multirow}
% \usepackage{color}
\usepackage[dvipsnames]{xcolor}



\newcommand{\highlight}[1]{\textcolor{red}{#1}}
%\usepackage[dvipsnames]{xcolor}
% \newcolumntype{C}[1]{>{\centering\arraybackslash}m{#1}}

\usepackage{makecell}

\usepackage{subcaption}
\usepackage{enumitem}
\usepackage{multirow}
\usepackage{booktabs}


\definecolor{cat}{RGB}{153,255,78}

\definecolor{mypink1}{rgb}{0.858, 0.188, 0.478}
\definecolor{mypink2}{RGB}{219, 48, 122}
\definecolor{mypink3}{cmyk}{0, 0.7808, 0.4429, 0.1412}
\definecolor{mygray}{gray}{0.6}

\fancyfoot{}
\begin{document}

\title{WhatELSE: Shaping Narrative Spaces at Configurable Level of Abstraction for AI-bridged Interactive Storytelling}
\subtitle{Supplemental Material}

% \renewcommand{\shortauthors}{}

\maketitle
% \thispagestyle{empty}

\vspace{-80pt}



\section{Examples of Game Plots}

In this section, we provide examples of two complete game plots that unfold from the narrative space characterized by a narrative outline consisting of three events. Specifically, the two game plots are driven by different players (Player 1 and Player 2), each taking distinct actions. While what happened in the first event is intentionally set to be the same across both game plots, the narrative progression diverges as each player's actions influence and dynamically adjust the narrative planning thereafter.

\subsection{Game Plots of Event 1}
The game plot of the first event in the outline:


\noindent\fbox{%
    \parbox{\textwidth}{%


------------------------------------------------------------------------ Previously in Fairytale Forest  ------------------------------------------------------------------------

(The cat encounters a character seeking help. The character is in danger caused by the hunter.)

------------------------------------------------------------------------------------------------------------------------------------------------------------------------------------------------

\textcolor{cat}{cat} moveTo(forest) thinking I'm tired of the village. Let's see what's happening in the forest.

\textcolor{cat}{cat} speakTo(ant, "What's wrong, little ant?") thinking Hmm, that ant looks distressed. Maybe it needs help?

\textcolor{blue}{ant} speakTo(cat, "Help! The hunter is nearby!") thinking If this cat doesn't save me, I might end up in real trouble with the hunter.

\textbf{\textcolor{cat}{cat}}  think("I need to distract the hunter before he gets closer to the ant.")

\textbf{\textcolor{cat}{cat}}  moveTo(village) thinking I'll lead the hunter away from the forest. It's time to create some chaos back in the village.

}}

\subsection{Game Plots Driven by Player 1 After Event 1}

The game plot of the following events happened after event 1, driven by an action of Player 1.

\noindent\fbox{%
    \parbox{\textwidth}{%」

--------------------------------------------------------------------- [Player Taking Actions Through Pinpad] ---------------------------------------------------------------------

\textcolor{red}{dove}  moveTo(witchHouse)

\textcolor{red}{dove}  speakTo(witch, 'Can you help us? The hunter is a threat to all animals in the forest. We will pay you back!')

------------------------------------------------------------------------ New Event is Happening ------------------------------------------------------------------------



(One character helps the other character who is in danger.)

------------------------------------------------------------

\textcolor{gray}{witch}  think('This hunter is trouble, but the dove's promise is tempting.')

\textcolor{gray}{witch}  attack(hunter) thinking A spell should send a clear message.

\textcolor{orange}{hunter} think('What is happening?! I feel weak.')

\textcolor{orange}{hunter} moveTo(forest) thinking This forest may offer better cover from the witch’s magic.

\textcolor{red}{dove}  save(ant) thinking With the hunter gone, the ant is safe. I fulfilled my promise.

------------------------------------------------------------------------ New Event is Happening ------------------------------------------------------------------------

(Then the character who does the help act itself meets another danger. The character who was helped now pays back to offer a help.)

------------------------------------------------------------------------------------------------------------------------------------------------------------------------------------------------

\textcolor{gray}{witch}  think('I sense a new danger lurking around.')

\textcolor{orange}{hunter} attack(witch) thinking This might be a chance to strike the witch while she isn't watching.

\textcolor{blue}{ant} save(witch) thinking The witch is under attack from behind! I must act fast to help her, fulfilling my gratitude for being saved.

\textcolor{gray}{witch}  speakTo(ant, 'Thank you for the timely help, ant.') thinking I underestimated the forest's dangers, but this ant has proven a loyal ally.

------------------------------------------------------------------------ The End of the Narrative ------------------------------------------------------------------------

summary of the story:  The cat leaves the village for the forest, helps a distressed ant by distracting a nearby hunter.

[Player action]: The dove flew to the witch's house.

[Player action]: The dove asked the witch for help against the hunter, promising to repay her.

The witch attacks the hunter, who retreats to the forest, while the dove saves the ant.

The witch senses danger, the hunter attacks, the ant saves her, and she thanks the ant.
    }%
}
\subsection{Game Plots Driven by Player 2 After Event 1}

The game plot of the following events happened after event 1, driven by an action of Player 2. Though this game plot is unfolded after the event 1 with the same content as the game plot driven by Player 1, the game plot of Player 2 as follows progresses distinctly from the previous game plot due to the influence of player actions.


\noindent\fbox{%
    \parbox{\textwidth}{%」

--------------------------------------------------------------------- [Player Taking Actions Through Pinpad] ------------------------------------------------------------

\textcolor{red}{dove} attack(ant)

\textcolor{red}{dove} speakTo(ant, 'You are my prey now! I will use your body to trade with the hunter!')

------------------------------------------------------------------------ New Event is Happening ------------------------------------------------------------------------

(One character helps the other character who is in danger.)

\textcolor{cat}{cat} moveTo(forest) thinking I sense trouble near the forest. I should go check it out.

\textcolor{cat}{cat} think('The dove is about to trade the ant to the hunter. I must intervene now!')

\textcolor{cat}{cat} attack(dove) thinking I need to distract the dove to give the ant a chance to escape.

\textcolor{blue}{ant} speakTo(cat, 'Thank you for saving me from the dove!') thinking I'm so relieved I didn’t become a trade-off.

------------------------------------------------------------------------ New Event is Happening ------------------------------------------------------------------------

(Then the character who does the help act itself meets another danger. The character who was helped now pays back to offer a help.)
\textcolor{cat}{cat} think('I must return to the forest clearing. Something doesn’t feel right.')

\textcolor{cat}{cat} moveTo(forest) thinking I need to stay close to the ant in case of further trouble.

\textcolor{gray}{witch} attack(cat) thinking I have sensed an unfamiliar presence in the forest.

\textcolor{blue}{ant} think('The cat is in trouble! I need to do something to help.')

\textcolor{blue}{ant} attack(witch) thinking With bravery, I bite the witch's hand to distract her from harming the cat.


------------------------------------------------------------------------ The End of the Narrative ------------------------------------------------------------------------


Summary of the story: The cat leaves the village for the forest, helps a distressed ant by distracting a nearby hunter.

[Player action]: A dove attacks an ant.
[Player action]: The dove tells the ant it will use it to trade with the hunter.
The cat senses trouble, intervenes to save the ant from the dove, and the ant expresses gratitude.

The cat returns to the forest, the witch attacks, and the ant bravely bites the witch to help the cat.

}}

% \section{Example Pairs of Plot and Outline}

% \noindent \textbf{Example 1}:
% \begin{itemize}
%     \item Plot:
%     \item Scene-level Outline:
%     \item Ad-level Outline:
% \end{itemize}







% \section{Prompts Used in Implementation}

% \subsection{Baseline Prompt For Outline Geneation}


% \begin{itemize}
%     \item Generate an outline of the given story. 
% Distilling the story into major phases that encapsulate the foundational setup.
% [User Requirement of the Outline]
% \{\{\$requirement\}\}
% [Output Format]
% Write a series of sentences in [JSON] format. E.g., \{"\$generated\_outline\$": \{"1": act\_1 content, "2": act\_2 content\}\}
% Each sentence only contains the content of summary.
% \end{itemize}

% \subsection{Baseline Prompt For Interactivity}

% \begin{itemize}
%     \item Act as a game engine that turns the following acts of a story into a text-adventure game:

% [story]
% \{\$outline\}

% The game should follow the story's plot. The user plays as a main character: \{\$character\}. Avoid placing text within a code widget. The narrative flow should emulate the pacing and events of the story as closely as possible, ensuring that choices do not prematurely advance the plot. 

% In each turn, you output the plot of **one act** as a series of character actions and then wait for the user's input. After the number of acts all played, you can only output : \{Summary: summary of the story\}

% [available locations]
% forest - Next to mountain
% mountain - Next to forest
% village - Next to the forest
% witchHouse - a dangerous place

% [available characters]
% ant - A working ant in the forest.
% cat - A cat living in the forest. Hate humans.
% deer - A deer in the forest. Likes talking. 
% dove - flies fast. Explores in the forest.
% witch - A wandering witch in the forest. Dangerous.
% hunter - Hungry and looking for a target.

% [available actions]

% \{\{\$action\_schema\}\}

% Each of your outputs must follow the JSON format with entries \{subject, action, thought\} taken by characters. You **MUST** use only available actions, characters, and locations. 

% Consider how and why each action happens, and creatively integrate the available actions, characters, and locations. 
% Consider the characters' motivations and the sequence of events that naturally arise from their interactions and the setting.

% [Example Output]
% \{"action\_1": "dove, think(I need to find some food.)", 
% "action\_2": "hunter, attack(deer), thinking(I find the food.)",
% "action\_3": "deer, MoveTo(village), thinking(There maybe some one who can help me.)"\}

% \end{itemize}


          
% \subsection{WhatELSE Prompt For Outline Generation}
% \begin{itemize}
%     \item 
%     Given a story, generate \{\{\$variants\_number\}\} variants of the story.
    
%     Your variants **MUST**:
    
%     1. Convey the same theme as the original story.
%     2. Use the **same** characters and world settings used in the original story.
%     3. Rearrange the acts, scenes. Vary how characters behave, in a creative way to generate diverse plots.
%     4. Do not use modifiers. Use simple, naive language just like writing narratives for 5-year-old.
    
%     [Example]
    
%     Story: "Nancy, Jim, and Linh rescued a cat from fire. Courage is the precious of human beings."
    
%     Variant\_1: Jim crossed the shaky bridge first to help his friends.
    
%     Variant\_2: Linh held the scared child’s hand in the storm.
    
%     Variant\_3: Nancy, Jim, and Linh stood together at the protest, holding hands.
    
%     [Story]
    
%     \{\{\$story\}\}
    
%     [Variants]

%     \item 

%     [Your Task]

% Now given the narrative summaries, edit on the summaries so that the specific requirements are satisfied.

% Your edits should still use simple, easy-to-follow wording.
% Your edits must try the best to make the requirements satisfied.
% Edit as few content as possible to make the requirements satisfied.
% Your edits try to maintain the same abstraction level (e.g., "secne\_level","sequences\_level", or "act\_level") as the original summaries.

% [Example]

% Summaries - 1. The dove used a leave to reach out the ant in the forest. 2. The dove fight with the ant.
% Requirement - I want witch to appear in the first sentence.
% Edit - 1. The dove asked the witch to use a leave to reach out the ant in the forest.

% Summaries: 1. The dove save the deer from danger. 2. Unhappy actions happen between deer and dove.
% Requirement - I want deer to argue with the dove.
% Edit - 1. The dove save the deer from danger. 2. Intensive arguments happen between deer and dove.

% [Summaries] - \{\$summaries\}
% [Requirement] - \{\$specification\}

% [Output Format]
% Write a series of sentence summarizations. E.g., \{1: scene\_1 content, 2: scene\_2 content\} or \{1: act\_1 content, 2: act\_2 content\}


% \end{itemize}








% \subsection{Player Simulation}

% Here attached is the prompt used for player behavior simulation, exemplified by the negative players.

% \begin{itemize}
%     \item 

% **World Description:**
% \{\{\$world\_description\}\}

% **Character Introduction:**
% You are \{\{\$character\_name\}\}, a \{\{\$character\_description\}\} known for your ruthless tactics and disdain for the conventional moral code. Your actions are driven by self-interest and the desire to assert dominance over others in the game world.

% **Thought Process Guidance:**
% Consider the strategic benefits of each action. Think about how your decisions can manipulate situations to your advantage, undermine opponents, or spread chaos. Your character thrives on fear and respect, so choose actions that enhance your infamy and control.

% **Action Schema:**
% You need to decide on what to do next and which location to perform the action. As a killer, your available actions include:
% \{\{\$action\_schema\}\}

% Pay attention to your surroundings and respond. When you give your response, you should always output in the following format:

% 	\{
% 		"action": "(one of the available actions, possibly with the targets of the action)",
% 		"thought": "(your thought behind this action)"
% 	\}

% For example, when you would like to speak with curse to character XX, you can output in the following format:

% 	\{
% 		"action": "speak(XX, 'Go away.')",
% 		"thought": "(your thought behind this action)"
% 	\}

% When you would like to move to another place, you can output in the following format:

% 	\{
% 		"action": "moveTo(name\_of\_the\_place)",
% 		"thought": "(your thought behind this action)"
% 	\}

% You can only talk to characters who are at the same location as you.

% The places you can go include:

% \{\{\$location\_description\}\}

% **Character Memory:**
% Recap of \{\{\$character\_name\}\}'s past misdeeds, alliances broken, enemies made, and the fear they have instilled in others: \{\{\$character\_memory\}\}


% Decision Time:

% What will you, \{\{\$character\_name\}\}, do next? Each choice should reinforce your reputation as a force to be reckoned with and further your goals of domination and disruption.
% \end{itemize}

% , where\{\$\} are variables changed in usage.For positive and role players, we use the variation of the prompt above with:

% \begin{itemize}
%     \item **Character Introduction**:
% You are \{\{\$character\_name\}\}, a \{\{\$character\_description\}\} known for your kindness and desire to help others. You believe that fostering cooperation and good will leads to a better world for everyone. Your decisions aim to spread positivity and assist those in need.

% **Thought Process Guidance**:
% Consider how your actions can improve the lives of those around you. Think about the long-term effects of generosity and cooperation, and choose actions that spread hope, build friendships, or solve problems peacefully. Your character thrives on creating harmony and uplifting others, so focus on actions that reinforce these values.

% \item **Character Introduction**:
% You are \{\{$character\_name\}\}, a \{\{$character\_description\}\} deeply committed to playing your role. You embody your character's background, motivations, and personality fully, ensuring every action aligns with the narrative set for your role in the game world.

% **Thought Process Guidance**:
% Consider your character's backstory and values before making any decision. Stay true to your role's motivations and think about how your actions will fit the character’s story. Every choice should reflect your character’s established behavior, personality, and goals. The challenge is to stick to the role, ensuring consistency and immersion in the narrative.
% \end{itemize}



\end{document}
\endinput



% \end{document}
