%%
%% This is file `sample-sigconf.tex',
%% generated with the docstrip utility.
%%
%% The original source files were:
%%
%% samples.dtx  (with options: `all,proceedings,bibtex,sigconf')
%% 
%% IMPORTANT NOTICE:
%% 
%% For the copyright see the source file.
%% 
%% Any modified versions of this file must be renamed
%% with new filenames distinct from sample-sigconf.tex.
%% 
%% For distribution of the original source see the terms
%% for copying and modification in the file samples.dtx.
%% 
%% This generated file may be distributed as long as the
%% original source files, as listed above, are part of the
%% same distribution. (The sources need not necessarily be
%% in the same archive or directory.)
%%
%%
%% Commands for TeXCount
%TC:macro \cite [option:text,text]
%TC:macro \citep [option:text,text]
%TC:macro \citet [option:text,text]
%TC:envir table 0 1
%TC:envir table* 0 1
%TC:envir tabular [ignore] word
%TC:envir displaymath 0 word
%TC:envir math 0 word
%TC:envir comment 0 0
%%
%% The first command in your LaTeX source must be the \documentclass
%% command.
%%
%% For submission and review of your manuscript please change the
%% command to \documentclass[manuscript, screen, review]{acmart}.
%%
%% When submitting camera ready or to TAPS, please change the command
%% to \documentclass[sigconf]{acmart} or whichever template is required
%% for your publication.
%%
%%



\documentclass[sigconf]{acmart}
%%
%% \BibTeX command to typeset BibTeX logo in the docs
\AtBeginDocument{%
  \providecommand\BibTeX{{%
    Bib\TeX}}}

%% Rights management information.  This information is sent to you
%% when you complete the rights form.  These commands have SAMPLE
%% values in them; it is your responsibility as an author to replace
%% the commands and values with those provided to you when you
%% complete the rights form.

\usepackage{dirtytalk}
\usepackage{multirow}
\usepackage{xcolor}

\usepackage{subcaption}
\usepackage{enumitem}
\usepackage{color}
%\usepackage{hyperref}
\usepackage{algorithm}
% \usepackage{tcolorbox}



% \usepackage{multirow}
% \usepackage{tabularray}

% \usepackage{xcolor}


% \usepackage{amsmath}
% \usepackage{algorithmic}
% \usepackage{algpseudocode}




\definecolor{burgundy}{rgb}{0.5, 0.0, 0.13}
\definecolor{yellow}{rgb}{0.85, 0.65, 0.13}
\definecolor{green}{rgb}{0.0, 0.5, 0.0} 

\definecolor{darkbrown}{rgb}{0.55, 0.27, 0.07}


\newcommand{\revision}[1]{{#1}}
\newcommand{\baseline}[1]{{#1}}
\newcommand{\originality}[1]{{#1}}
\newcommand{\evaluation}[1]{{#1}}
\newcommand{\presentation}[1]{{#1}}
\newcommand{\usecase}[1]{{#1}}


\newcommand{\comments}{1}
\newcommand{\ignore}[1]{}

\ifdefined\comments
   \newcommand{\yw}[1]{\textcolor{red}{[Yi: #1]}}
    \newcommand{\qz}[1]{\textcolor{orange}{[Qian: #1]}}
    \newcommand{\zl}[1]{\textcolor{blue}{[Zhuoran: #1]}}
    \newcommand{\whosays}[1]{\begin{center} \say{\textit{#1}} \end{center}}
     \newcommand{\cred}[1]{\textcolor{red}{[#1]}}
     
\else
\fi

\definecolor{highlight}{RGB}{230, 230, 255} % Light blue color

% \newcommand{\boundedpatch}[2][]{%
%     \begin{tcolorbox}[colframe=blue!50!black, colback=blue!5!white, title=#1]
%     #2
%     \end{tcolorbox}
% }




\copyrightyear{2025}
\acmYear{2025}
\setcopyright{cc}
\setcctype{by}
\acmConference[CHI '25]{CHI Conference on Human Factors in Computing
Systems}{April 26-May 1, 2025}{Yokohama, Japan}
\acmBooktitle{CHI Conference on Human Factors in Computing Systems (CHI
'25), April 26-May 1, 2025, Yokohama,
Japan}\acmDOI{10.1145/3706598.3713363}
\acmISBN{979-8-4007-1394-1/25/04}

% \setcopyright{acmlicensed}
% \copyrightyear{2018}
% \acmYear{2018}
% \acmDOI{XXXXXXX.XXXXXXX}
% %% These commands are for a PROCEEDINGS abstract or paper.
% \acmConference[Conference acronym 'XX]{Make sure to enter the correct
%   conference title from your rights confirmation email}{June 03--05,
%   2018}{Woodstock, NY}
% %%
% %%  Uncomment \acmBooktitle if the title of the proceedings is different
% %%  from ``Proceedings of ...''!
% %%
% %%\acmBooktitle{Woodstock '18: ACM Symposium on Neural Gaze Detection,
% %%  June 03--05, 2018, Woodstock, NY}
% \acmISBN{978-1-4503-XXXX-X/2018/06}


%%
%% Submission ID.
%% Use this when submitting an article to a sponsored event. You'll
%% receive a unique submission ID from the organizers
%% of the event, and this ID should be used as the parameter to this command.
%%\acmSubmissionID{123-A56-BU3}

%%
%% For managing citations, it is recommended to use bibliography
%% files in BibTeX format.
%%
%% You can then either use BibTeX with the ACM-Reference-Format style,
%% or BibLaTeX with the acmnumeric or acmauthoryear sytles, that include
%% support for advanced citation of software artefact from the
%% biblatex-software package, also separately available on CTAN.
%%
%% Look at the sample-*-biblatex.tex files for templates showcasing
%% the biblatex styles.
%%

%%
%% The majority of ACM publications use numbered citations and
%% references.  The command \citestyle{authoryear} switches to the
%% "author year" style.
%%
%% If you are preparing content for an event
%% sponsored by ACM SIGGRAPH, you must use the "author year" style of
%% citations and references.
%% Uncommenting
%% the next command will enable that style.
%%\citestyle{acmauthoryear}


%%
%% end of the preamble, start of the body of the document source.
\begin{document}

%%
%% The "title" command has an optional parameter,
%% allowing the author to define a "short title" to be used in page headers.
\title{WhatELSE: Shaping Narrative Spaces at Configurable Level of Abstraction for AI-bridged Interactive Storytelling
}

%%
%% The "author" command and its associated commands are used to define
%% the authors and their affiliations.
%% Of note is the shared affiliation of the first two authors, and the
%% "authornote" and "authornotemark" commands
%% used to denote shared contribution to the research.
\author{Zhuoran Lu}
\affiliation{%
  \institution{Department of Computer Science, Purdue University}
%   \streetaddress{1 Th{\o}rv{\"a}ld Circle}
  \city{West Lafayette}
  \country{USA}}
\email{lu800@purdue.edu}

\author{Qian Zhou}
\affiliation{%
  \institution{Autodesk Research}
  \city{Toronto}
  \country{Canada}}
\email{qian.zhou@autodesk.com}

\author{Yi Wang}
\affiliation{%
  \institution{Autodesk Research}
  \city{San Francisco}
  \country{USA}}
\email{yi.wang@autodesk.com}

% \author{Ben Trovato}
% \authornote{Both authors contributed equally to this research.}
% \email{trovato@corporation.com}
% \orcid{1234-5678-9012}
% \author{G.K.M. Tobin}
% \authornotemark[1]
% \email{webmaster@marysville-ohio.com}
% \affiliation{%
%   \institution{Institute for Clarity in Documentation}
%   \city{Dublin}
%   \state{Ohio}
%   \country{USA}
% }

% \author{Lars Th{\o}rv{\"a}ld}
% \affiliation{%
%   \institution{The Th{\o}rv{\"a}ld Group}
%   \city{Hekla}
%   \country{Iceland}}
% \email{larst@affiliation.org}

% \author{Valerie B\'eranger}
% \affiliation{%
%   \institution{Inria Paris-Rocquencourt}
%   \city{Rocquencourt}
%   \country{France}
% }

% \author{Aparna Patel}
% \affiliation{%
%  \institution{Rajiv Gandhi University}
%  \city{Doimukh}
%  \state{Arunachal Pradesh}
%  \country{India}}

% \author{Huifen Chan}
% \affiliation{%
%   \institution{Tsinghua University}
%   \city{Haidian Qu}
%   \state{Beijing Shi}
%   \country{China}}

% \author{Charles Palmer}
% \affiliation{%
%   \institution{Palmer Research Laboratories}
%   \city{San Antonio}
%   \state{Texas}
%   \country{USA}}
% \email{cpalmer@prl.com}

% \author{John Smith}
% \affiliation{%
%   \institution{The Th{\o}rv{\"a}ld Group}
%   \city{Hekla}
%   \country{Iceland}}
% \email{jsmith@affiliation.org}

% \author{Julius P. Kumquat}
% \affiliation{%
%   \institution{The Kumquat Consortium}
%   \city{New York}
%   \country{USA}}
% \email{jpkumquat@consortium.net}

%%
%% By default, the full list of authors will be used in the page
%% headers. Often, this list is too long, and will overlap
%% other information printed in the page headers. This command allows
%% the author to define a more concise list
%% of authors' names for this purpose.
% \renewcommand{\shortauthors}{Trovato et al.}

%%
%% The abstract is a short summary of the work to be presented in the
%% article.
\begin{abstract}
%% YWv1
%Generative AI has the potential to revolutionize interactive storytelling by concretizing the creator's narrative intent according to the audience's actions in the story world. However, this requires the creator to relinquish some authorial control on the details of the narrative. Instead of crafting a specific narrative, the creator needs to author a possibility space of narratives, while the final narrative seen by the audience emerges from their interaction with AI within the space. In contrast to most existing tools assisting interactive storytelling creation, which focus on authoring specific narratives, we present WhatELSE, a proof-of-concept workflow for authoring possibility spaces of narratives. WhatELSE assists creators to perceive and sculpt a narrative possibility space utilizing coordination between three different representations of the space: narrative instance, narrative outline and narrative variants. Technical evaluations and user study show that WhatELSE helps the creators effectively shape the narrative possibility space and balance between authorial intent expression and emergent narrative content. We also demonstrate how the output from WhatELSE can be used to drive actual gameplay by grounding narrative content to function calls supported by a game environment.

%% Qv1
%Interactive narratives let players make choices that impact the story that they are reading. Creating interactive narratives requires authors to craft a narrative space by enumerating player choices and possible outcomes. Large language models have the potential to generate rich interactive textual output but they lack the control that authors desire, and it is challenging for authors to perceive and edit the generated narrative space. In this paper, we present WhatELSE, an authoring system that generates an interactive narrative using a linear story as an example. It encodes the authorial intent of the example into a narrative space by abstracting meaningful events, and unfolds the narrative space into executable gameplots  using LLM-based narrative planning. The system provides three views (narrative outline, variant, and example) and two abstraction tools (abstraction ladder and tooltip) to help authors perceive and edit the space . Through a user study and technical evaluation, we found that WhatELSE supports creation of interactive narrative through authorial controls and prevents deviations beyond the original authorial intent.

%% YWv3
% Generative AI enables just-in-time narrative content generation in games, which significantly enhances player agency by allowing the generated narrative to respond to the player's actions in the game. However, this reduces author's control over the space of possible narratives - within which the final story experienced by the player emerges from their interaction with AI. It is challenging for authors to understand and control what could be generated from the space. In this paper, we present WhatELSE, an  interactive narrative authoring system that creates narrative possibility spaces from examples. WhatELSE leverages linguistic abstraction to control the boundary of the narrative space, and LLM-based simulation to generate estimated variations in the space as well as corresponding executable character actions. WhatELSE helps creators perceive and edit this space through three representations of the narrative space: plot instance, narrative outline, and plot variants. We conducted a user study and technical evaluation. Technical evaluations and a user study (N = 12) show that WhatELSE enables game plot generation to reach balanced authorial expression and player agency.

% ZL v3.3


Generative AI significantly enhances player agency in interactive narratives (IN) by enabling just-in-time content generation that adapts to player actions. While delegating generation to AI makes IN more interactive, it becomes challenging for authors to control the space of possible narratives - within which the final story experienced by the player emerges from their interaction with AI. In this paper, we present WhatELSE, an AI-bridged IN authoring system that creates narrative possibility spaces from example stories. WhatELSE provides three views (narrative pivot, outline, and variants) to help authors understand the narrative space and corresponding tools leveraging linguistic abstraction to control the boundaries of the narrative space. Taking innovative LLM-based narrative planning approaches, WhatELSE further unfolds the narrative space into executable game events. Through a user study (N=12) and technical evaluations, we found that WhatELSE enables authors to perceive and edit the narrative space and generates engaging interactive narratives at play-time.


% Generative AI enables just-in-time narrative content generation in games, thus significantly enhancing player agency by allowing the generated narrative to respond to the player's action in the game. However, this reduces author's control over the space of possible narratives - within which the final story experienced by the player emerges from their interaction with AI. It is challenging for authors to understand and control what could be generated from the space. In this paper, we present WhatELSE, an interactive narrative authoring system that creates narrative possibility spaces from a story example. WhatELSE leverages linguistic abstraction to control the boundary of the narrative space, and LLM-based simulation to generate estimated variations in the space. It encodes the authorial intent of the example into a narrative space by abstracting pivotal events and unfolds this space into executable game plots using LLM-based narrative planning. WhatELSE helps authors perceive the space through three representations of the narrative space (narrative instance, outline, and variants), and use two abstraction tools (abstraction ladder and tooltip) to edit the space.  Through a user study (N=12) and technical evaluation, we found that WhatELSE enables game plot generation to reach balanced authorial expression and player agency.


\end{abstract}





%%
%% The code below is generated by the tool at http://dl.acm.org/ccs.cfm.
%% Please copy and paste the code instead of the example below.
%%

\begin{CCSXML}
<ccs2012>
<concept>
<concept_id>10010147.10010178.10010179</concept_id>
<concept_desc>Computing methodologies~Natural language processing</concept_desc>
<concept_significance>500</concept_significance>
</concept>
<concept>
<concept_id>10010405.10010476.10011187.10011190</concept_id>
<concept_desc>Applied computing~Computer games</concept_desc>
<concept_significance>500</concept_significance>
</concept>
<concept>
<concept_id>10011007.10010940.10010941.10010969.10010970</concept_id>
<concept_desc>Software and its engineering~Interactive games</concept_desc>
<concept_significance>500</concept_significance>
</concept>
</ccs2012>
\end{CCSXML}

\ccsdesc[500]{Computing methodologies~Natural language processing}
\ccsdesc[500]{Applied computing~Computer games}
\ccsdesc[500]{Software and its engineering~Interactive games}

% \begin{CCSXML}
% <ccs2012>
%  <concept>
%   <concept_id>00000000.0000000.0000000</concept_id>
%   <concept_desc>Do Not Use This Code, Generate the Correct Terms for Your Paper</concept_desc>
%   <concept_significance>500</concept_significance>
%  </concept>
%  <concept>
%   <concept_id>00000000.00000000.00000000</concept_id>
%   <concept_desc>Do Not Use This Code, Generate the Correct Terms for Your Paper</concept_desc>
%   <concept_significance>300</concept_significance>
%  </concept>
%  <concept>
%   <concept_id>00000000.00000000.00000000</concept_id>
%   <concept_desc>Do Not Use This Code, Generate the Correct Terms for Your Paper</concept_desc>
%   <concept_significance>100</concept_significance>
%  </concept>
%  <concept>
%   <concept_id>00000000.00000000.00000000</concept_id>
%   <concept_desc>Do Not Use This Code, Generate the Correct Terms for Your Paper</concept_desc>
%   <concept_significance>100</concept_significance>
%  </concept>
% </ccs2012>
% \end{CCSXML}

% \ccsdesc[500]{Do Not Use This Code~Generate the Correct Terms for Your Paper}
% \ccsdesc[300]{Do Not Use This Code~Generate the Correct Terms for Your Paper}
% \ccsdesc{Do Not Use This Code~Generate the Correct Terms for Your Paper}
% \ccsdesc[100]{Do Not Use This Code~Generate the Correct Terms for Your Paper}

%%
%% Keywords. The author(s) should pick words that accurately describe
%% the work being presented. Separate the keywords with commas.
\keywords{Interactive Narrative, Large Language Models, Abstraction, Narrative Space, Video Games, Generative AI}

\begin{teaserfigure}
\includegraphics[width=0.24 \textwidth]{figures/concept-descriptors.png}~%
\includegraphics[width=0.24 \textwidth]{figures/concept-shift-pivot.png}~%
\includegraphics[width=0.24 \textwidth]{figures/concept-abstract-outline.png}~%
\includegraphics[width=0.24 \textwidth]{figures/concept-remove-variant.png}%
\caption{We present WhatELSE, an interactive narrative authoring system that allows users to shape a narrative space using language abstraction. (a) We use the pivot, outline, and variants to describe the narrative space. Users can import an example story as a pivot. The system elevates the pivot into a narrative space. It generates an outline and variants to describe the space. Users can (b) edit the pivot to shift the space, (c) elevate the abstraction level to expand the space, or (d) remove variants to sculpt the space. }
  \label{fig:teaser}
  \Description{}
\end{teaserfigure}

% \received{20 February 2007}
% \received[revised]{12 March 2009}
% \received[accepted]{5 June 2009}

%%
%% This command processes the author and affiliation and title
%% information and builds the first part of the formatted document.
\maketitle
\section{Introduction}
\documentclass[../main.tex]{subfiles}
\graphicspath{{../images/}}
\makeatletter
\def\input@path{{../images/}}
\makeatother
\begin{document}
\section{Introduction}
\begin{figure}
\centering
\begin{tikzpicture}
\node[inner sep=0pt] (ws) at (0, 0) {
\includegraphics[height=.4\textwidth, trim={10cm 0 10cm 0},clip]{world_space.png}};
\node[inner sep=0pt] (cs) at (6,0) {\includegraphics[height=.4\textwidth, trim={10cm 1cm 10cm 4cm},clip]{conf_space.png}};
\end{tikzpicture}
\vspace{-5pt}
\label{fig:pbrm_intro}
\caption{\textbf{Left}: Shows world space obstacles as grey spheres. Robots start and goal configuration is colored red and green, respectively. Configurations along the computed path are colored transparent blue. \textbf{Right:} Mapped world space scenario to configuration space. Obstacle region is the grey mesh. Red spheres are collision-free regions computed by the neural SCDF. The optimized shortest path in the convex corridor is the blue curve.}
\vspace{-25pt}
\end{figure}
Motion planning is the problem of finding a collision-free trajectory that connects a given start and goal configuration. The planning takes place in the configuration space of the robot. For single body robots, like mobile robots or drones, the configuration space and the world space are usually the same. This simplifies the planning, since explicit obstacle representations are available which enables geometrical tools like separating hyperplanes, smallest distance to obstacles etc., to be used when designing motion planning algorithms. For multi-body robots like manipulators, the situation is completely different. The world space obstacles are usually mapped to non-convex regions, and to make the problem even harder, the mapping is usually not known. Forming explicit representations of the obstacle region in the configuration space is usually too expensive or intractable. Despite all of this, sampling based planners are used with great success, which mainly is due to their use of implicit representations of the obstacle region. The basic idea is to construct a graph in the configuration space that covers and connects the collision-free region. From this graph, a path can be extracted that connects a given start and goal configuration. The approach is computationally expensive, since the graph is constructed with the smallest geometrical building block available, points, which represents a collision-check. Furthermore, the extracted paths from the graph are non-smooth and jagged due to the stochastic nature of the approach. This adds an additional post-processing step to the process, where the paths are shortcutted and smoothened, before the path can be used for tracking. Clearly a lot of time is invested to form this graph and produce smooth paths. Thus, if the obstacles start to move, then all of this work is done in no use, since all points that make up this graph need to be re-verified, which is simply too time consuming to be done in real time.
\\\\
In this work, we want to address the existing drawbacks of the sampling based planners. Our main contribution is an improved motion planner where each vertex in the graph covers a collision-free region in the form of a sphere instead of a point and where the edges are formed with neighboring intersecting spheres. This representation has the advantage of instead of returning piecewise linear paths, returning a sequence of overlapping spheres, i.e. a convex corridor, that connects a given start and goal configuration, illustrated in Figure \ref{fig:pbrm_intro}. This convex corridor allows us to use convex optimization to produce smooth trajectories, instead of computationally expensive post-processing methods. The representation further allows us to estimate the coverage of the collision-free space, which gives us awareness and feedback in the offline roadmap construction phase. Finally, our representation is simple to adapt to moving obstacles, simply requery for the new radii and recheck for intersections. 
\\\\
The spherical collision-free regions are formed using a signed distance function (SDF), which is a function that returns the smallest distance from an arbitrary point to the boundary of an obstacle. As the name implies, the distance is signed, thus if the point is inside the obstacle it is negative otherwise positive. If the distance is positive, a sphere with radius equal to the distance is guaranteed to cover a collision-free region. Using an SDF in motion planning is not new, but what is novel about our approach is that we express the distance in the configuration space instead of the world space and by doing so allows us to form these convex collision-free regions. We refer to the resulting SDF as a signed configuration distance function (SCDF). Computing an SCDF analytically is non-trivial, our approach is therefore to parameterize the SCDF with a deep neural network and learn the mapping by supervised learning. Our resulting neural SCDF can compute distances for different parameter values of obstacle shapes and we also show how multiple distances can be combined, thus making our approach flexible.
\section{Related work}
Motion planning algorithms can roughly be divided into three families, grid-based, sampling based and optimization based methods. Grid-based methods (GBM) discretize the planning space from which a graph is then compiled. A standard search method is A$^\star$ \citep{a_star}, which is classified as an \textit{informed} search method, since it employs a heuristic function to speed up the search. A$^\star$ guarantees to return an optimal path at the level of discretization used. GBMs usually discretize the planning space by a regular lattice and this limits the GBMs to problems with low dimensionality due to the curse of dimensionality. Thus, GBMs are usually limited to single-body robots where the degrees of freedom (DOF) are low. To overcome the inherent scaling problem with the GBMs, stochastic methods are usually used for multi-body robots. These methods are termed as sampling-based methods (SBM) and core members within this family are the rapidly-exploring random trees (RRT) \citep{rrt} and the probabilistic roadmap (PRM) \citep{prm}. RRT grows a tree from the start configuration and explores the collision-free region in a rapid way until it is able to connect to the goal region. RRT is usually improved by bi-directional planning \citep{rrt_connect}, i.e. an additional tree is grown from the goal configuration and the trees are tested for connection after any tree has been expanded. RRT is a single-query method, thus it searches for a path from scratch each time it is queried. Contrary to this, PRM is a multi-query method, which solves for multiple queries without starting from scratch. PRM does this by creating a roadmap (graph) that covers the collision-free space as an offline step. The graph is then used to solve for multiple queries. PRMs are used in cases where the environment does not change since the extra offline step is too computationally costly and needs to be re-done if the environment is changed. In our work, we address this inherent issue by using a different roadmap representation. Our vertices in the graph cover a collision-free region in the form of spheres and we form the edges by checking for intersecting spheres. If something in the environment changes, we recompute the spheres radii and recheck the intersections, without relying on collision detection. We use a trained neural network to compute the sphere radius, therefore querying for the radius can be done fast, hence our representation enables the PRM for dynamic environments.
\\\\
In the recent decades, optimization based methods (OBM) \citep{chomp, schulman, itomp, stomp} have been introduced as an alternative to SBM for multi-body robots. Like the SBM, the OBMs scale well to higher dimensional problems and produce smoother motion. It is common to use a SDF in the optimization since it is a smooth function, thus enabling gradient-based methods. However, the standard way of expressing the SDF is in world space. The distance therefore needs to be mapped to the configuration space by the forward kinematics. This mapping makes the optimization problem a non-linear program (NLP), which is computationally expensive to solve. Recently, a different approach has been proposed. In \cite{mp_gcs} motion planning is formulated as a convex optimization problem by using the graph of convex sets framework \citep{gcs}. The underlying idea is to decompose the collision-free space into intersecting convex sets from which a convex optimization problem is formulated. In cases where an explicit representation of the obstacles in the configuration space exists, like for single-body robots, creating collision-free convex regions can be done fast \citep{iris}. For multi-body robots, this is non-trivial. Existing work does this successfully \citep{iris_nlp, iris_c} by an optimization based approach, but the methods are still too time consuming to be used in the presence of moving obstacles. Our approach is instead to use deep learning to learn an SDF expressed in the configuration space. With this, we can query for shortest distances to the collision boundary, which allows us to expand spherical regions which are collision-free. Our approach is fast and therefore enables our suggested roadmap planner to be used in dynamic environments.
\\\\
Recent research has focused on learning collision detection \citep{fk_kernel_distance, diffco, graphdistnet} by predicting the signed distance between the robot links and the surrounding obstacles in the world space. The learned SDF is used in trajectory optimization but since the distance is expressed in the world space, the problem becomes an NLP and therefore takes a long time to solve. We take a novel approach and suggest to instead express the signed distance in the configuration space. This allows us to improve the PRM at the same time as it enables convex optimization for trajectory optimization, which runs faster and is more reliable than NLP solvers. In \cite{cspf} a learned signed distance function in the configuration space is proposed similar to our approach. However, their approach is restricted to point cloud representations, while we propose to represent the obstacles as parameterized geometric shapes, e.g. spheres. Furthermore, we also show how to use our learned SCDF to improve an existing roadmap planner.
\section{Problem formulation}
A robot is located in the world space, $\W \subset \R^3 $. The unique location of the robot is given by its configuration $\q \in \C$, where $\C$ is the configuration space. The set of points covered by the robots bodies at a certain configuration is expressed as $\B(\q) \subset \W$. The robot is surrounded by $\NrObst$ obstacles $\O = \bigcup_{i=1}^{\NrObst} \O_i$, where  $\O_i \subset \W$. The representation of the obstacle in the configuration space is the set $\C\O_i = \{\q \in \C \: |\: \B(\q) \cap \O_i \neq \emptyset \}$. The obstacle space is formed as $\Co = \bigcup_{i=1}^{\NrObst} \C \O_i$. The complement is referred to as the free space, $\Cf = \C \setminus \Co$. The path planning problem is a tuple, ($\Cf$, $\qStart$, $\qGoal$), where we want to connect a query pair, consisting of a start, $\qStart$, and goal configuration, $\qGoal$, with a geometric path, $\q(s): [0, 1] \mapsto \Cf$, such that $\q(0)=\qStart$ and $\q(1)=\qGoal$, or report correctly when such a path does not exist.
\end{document}


\section{Related Work}

\section{Related work}

The literature related to our work can be classified into two
categories: general purpose DR techniques
(\autoref{sec:relatedWorkGeneralPurpose}) and topology-aware techniques
(\autoref{sec:relatedWorkTopology}).

\subsection{General purpose dimensionality reduction}
\label{sec:relatedWorkGeneralPurpose}

Numerous DR techniques have been proposed and the related literature has been
summarized in several books~\cite{borg97, dimensionReductionBook} and surveys
\cite{surveyDimensionReduction2, surveyDimensionReduction1, NonatoA19}.
Principal Component Analysis (PCA)~\cite{pearson1901liii} is by far the most
popular linear DR technique.
Although it is an indispensable tool for data analysis,
its linear nature does not always allow it to apprehend complex non-linear
phenomena. One of the first non linear DR methods is the multidimensional
scaling (MDS)~\cite{torgerson1952multidimensional}. It aims at preserving as far
as possible the pairwise distances in the high- and low-dimensional point
clouds.
Another approach, particularly related to our work,
consists in optimizing an autoencoder neural network~\cite{hinton_reducing_2006}.
The \textit{encoder} is used to represent the explicit projection map from the
high-dimensional input space to the low-dimensional representation
space, while the \textit{decoder} tries to reconstruct the input data
from its encoded representation.
We will refer to these methods as \emph{global} methods.

Global methods have been used successfully in many applications, but
they do not take into
account the possible distribution of the input points over some implicit,
unknown manifold. This may lead to the unwanted preservation of distances
between points that are close in the ambient space but far apart on this
manifold. \emph{Locally topology-aware} methods have therefore been
introduced to address this issue. For instance,
Isomap~\cite{tenenbaum_global_2000}
preserves geodesic distances on a captured manifold structure of the
input data.
%\remove{Because it suffers from computational
%inefficiencies, Isomap was sped up with the use of landmark points (L-Isomap
%\cite{silva2003global}).}
Other methods also feature neighborhood preservation objectives.
For example, Local Linear Embedding (LLE)~\cite{roweis2000nonlinear} relies
on linear reconstructions of local neighborhoods.
Other methods leverage additional landmarks~\cite{silva2003global} or user-provided
control points~\cite{joia:tvcg:2011}.
%Some local methods additionally support user
%constraints expressed as control points~\cite{joia:tvcg:2011}.

All these methods try to preserve local
Euclidean distances when projecting to a lower dimension.
However, this can sometimes lack relevance in the applications,
especially with high-dimensional datasets for which
the distribution of pairwise Euclidean distances tend to be uniform.
For such cases, local distance preservation fails at characterizing
well relevant local relations.
To alleviate this issue, SNE~\cite{hinton2002stochastic} and later
t\nobreakdash-SNE~\cite{van2008visualizing} use a conditional probability
formulation to represent similarities between points and try to
have similar distributions both in high- and low-dimension thanks to a
Kullback--Leibler divergence minimization.
More recently UMAP has been introduced~\cite{mcinnes2018umap} along a
theoretical foundation on category theory.
It provides results that are similar visually to t-SNE, but in a more
scalable way.
Variants were later introduced to better preserve the global structure in the embedding, such as TriMAP~\cite{amid2022trimap} that constrains the proximity order within triplets of points, or PaCMAP~\cite{wang_understanding_2021} that adds constraints on more distant point pairs.
Although these methods succeed in preserving the local topology, they are not
explicitly aware of the global structure
of the input, which can lead to the loss of noteworthy global,
topological features.

\subsection{Globally topology-aware dimensionality reduction}
\label{sec:relatedWorkTopology}

Topology-based methods have become popular over the last
two decades in data analysis and
visualization~\cite{heine16} and have been applied to various areas:
astrophysics~\cite{sousbie11, shivashankar2016felix},
biological imaging~\cite{beiBrain18, carr04, topoAngler},
quantum chemistry~\cite{chemistry_vis14,harshChemistry, D2CP05893F},
fluid dynamics~\cite{kasten_tvcg11, NauleauVBBT22},
material sciences~\cite{gyulassy_vis07, gyulassy_vis15, SolerPDPCT19},
turbulent combustion~\cite{gyulassy_ev14, laney_vis06}. They leverage tools that
define concise signatures of the data based on its topological properties and
that have been summarized in topological data analysis reference
books ~\cite{edelsbrunner_computational_2010, zomorodian_computational_2010}
and surveys~\cite{chazal_introduction_2021}.

Several DR methods have been proposed
by the visualization community to preserve specific topological signatures
of the input data. For instance, terrain metaphors have been
investigated for the visualization of an input high-dimensional scalar
field, in the form of a three-dimensional terrain, whose elevation yields an
identical contour tree~\cite{Weber:2007} or an identical set of separatrices
\cite{gerber2010, gerber2013}.
This framework has been extended to density
estimators~\cite{OesterlingHJS10,
OesterlingSTHKEW10, OHJSH11, Oesterling0WS13} for the support of
high-dimensional point clouds. However, such metaphors completely discard
the metric information of the input space~\cite{OesterlingHJS10}, possibly
placing next to each other points which are arbitrarily far
in the input space (and reciprocally). Yan et al.~\cite{abs-1806-08460}
introduced a DR approach driven by the Mapper structure~\cite{SinghMC07}, an
approximation of the Reeb graph~\cite{reeb46}, which can capture in practice
large handles in the data, however without guarantees, since the number of handles in the considered manifold is only an upper bound on the number of loops in the Reeb graph~\cite{edelsbrunner_computational_2010}.

To incorporate the metric information from the input data while
preserving at the same time some of their topological characteristics, several
approaches have focused on driving the projection by
the \emph{persistence diagram}
of the Rips filtration of the point cloud (see \autoref{sec:persistentHomology}
for  a technical description).
Carriere et al.~\cite{carriere2021optimizing} presented a generic persistence
optimization framework with an application to dimensionality reduction.
Their approach explicitly minimizes the Wasserstein distance
(\autoref{sec:persistentHomology}) between the $1$-dimensional persistence
diagrams in high and low dimensions. However, this approach solely focuses on
this penalization term. As a result,
although the number and persistence of cycles  may be well-preserved,
the solver tends to produce cycles in low dimensions which involve arbitrary
points (e.g., which were not necessarily located along the cycles in high
dimensions), which challenges visual interpretation, as later
detailed in \autoref{sec:results:analysis}.

To enforce a correspondence between the topological
features at the data point level, additional structures need to be preserved.
For the specific case of $0$-dimensional persistent homology (\PH{0}),
Doraiswamy et al. introduced \emph{TopoMap}~\cite{doraiswamy2020topomap}, an
algorithm which constructively preserves the \emph{persistence pairs}
(\autoref{sec:persistentHomology}) through the preservation of the minimum
spanning tree of the data. An accelerated version, with improved layouts, has
recently been proposed~\cite{guardieiro2024topomap++}.
Alternative approaches have considered the usage of an optimization framework
(typically based on an autoencoder neural network
\cite{hinton_reducing_2006}), with the integration of specific topology-aware
losses~\cite{moor2020topological,barannikov2021representation,
nelson2022topology,trofimov2023learning,schonenberger2020witness}. Among them,
a prominent approach is the \emph{Topological Autoencoders}
(TopoAE)~\cite{moor2020topological}. Its loss aims at preserving
the diameter of the simplices involved in
persistence pairs, when going from high to low dimensions and reciprocally.
However, the above techniques focused in practice on the
preservation of \PH{0} and did not, to our knowledge, report experiments
regarding the preservation of higher dimensional PH.
Specifically, we show in \autoref{sec:analysis} that, while a zero
TopoAE loss indeed implies a preservation of the persistence pairs for \PH{0},
it is not the case for higher dimensional PH. We provide a counter example for
\PH{1}, which is addressed by our novel, generalized loss.


%[IN authoring: pre-genAI (twine, etc.), genAI (inworld, chrisma.ai, etc.)]
%(existing tools only at textural level)

%[Narrative planning and simulation]

%[Narrative Space Authoring Paradigm (conceptual space, emily-short, luminate, CLA): narrative space authoring paradigm -> design space]

%\section{User Needs and Design Rationale}
%\section{Design}\label{sec:design}

%%%%%%%%%%%%%%%%%%%%%%%%%%%%%%


\begin{figure*}[t]
    \centering
    \includegraphics[trim = 15 530 15 15, width=1\textwidth]{Algorithm_drawio.pdf}
    \caption{Overview of KiSS}
    \label{fig:overview}
\end{figure*}


The results we gleaned from the previous section (see Section~\ref{sec:work_anly}) helped in developing our policy: KiSS. The KiSS or \textbf{Keep it Separated Serverless} policy aims to address critical challenges in Function-as-a-Service (FaaS) platforms, particularly in edge computing environments, by achieving the following objectives:

\begin{itemize}
    \item \textbf{Reduced Cold Start Latency:} Prioritizes high-frequency functions to minimize delays in real-time applications.
    \item \textbf{Improved Resource Efficiency:} Optimizes memory and compute usage while avoiding unnecessary overhead from static warm states.
    \item \textbf{Minimized Inter-Function Interference:} Enhances throughput and scalability through modular resource partitioning.
    \item \textbf{Improved Function Service Rate:} Adopts resource-aware policies to reduce dropped requests and maximize system reliability.
\end{itemize}


\subsection{KiSS Policy Overview}

KiSS introduces a modular, data-driven orchestration strategy designed to optimize serverless execution in resource-constrained environments, particularly at the edge. By leveraging our workload analysis (refer Section 2.5), our policy segments functions based on key metrics—memory footprint, invocation frequency, and execution time—to optimize performance across diverse workloads.

The edge computing context introduces unique challenges like limited memory, heterogeneous resources, and dynamic workloads. Generalized cloud strategies often fail to adapt to such constraints. KiSS addresses this gap by analyzing workload characteristics and implementing a resource-efficient, modular strategy that aligns with edge-specific demands.

\subsection{Components of KiSS Policy Design}
Figure~\ref{fig:overview} shows the overall architecture of KiSS. 
The incoming \textit{FaaS traffic} will include both small and large functions. 
The \textit{request handler} accepts the incoming functions and shares the function information to the workload analyzer. 
The \textit{workload analyser} processes the function information to profile the incoming function traffic information and generate data such as invocation frequency, memory footprint etc.
The \textit{KiSS policy} uses this data to estimate where this function will be placed between the two different warm pool partitions.

The \textit{load balancer} implements a partitioning logic where functions are allocated to distinct warm pools using (\textit{invoker 1} and \textit{invoker 2}) based on profiling thresholds:

(i)~Small Functions Pool: Dedicated to high-frequency, low-memory functions to ensure low latency, and (ii)~Large Functions Pool: Allocated for low-frequency, memory-intensive functions, minimizing contention with smaller containers.
Each warm pool operates autonomously achieving Policy Independence.
The \textit{Warm Pool Replacement Policy} for each warm container pool can independently implement different workload-specific strategies to reduce contention and enhance temporal locality.


These factors form the foundation of KiSS’s multi-tiered warm pool framework, allowing it to effectively manage serverless resources and enhance performance in edge computing. By addressing these challenges, KiSS positions itself as a practical and scalable solution for FaaS platforms in environments with diverse and demanding resource constraints.


\subsection{Innovations of KiSS Policy}

One of the most innovative features of KiSS is its multi-level warm pool partitioning, which isolates high- and low-frequency functions into separate pools. This design eliminates inefficiencies inherent in monolithic resource strategies by ensuring that small, frequently invoked functions are always ready to execute, while larger, less frequent functions remain accessible without competing for resources. This adaptability extends to the ability to add more pools as workload patterns evolve, making KiSS a flexible and future-proof solution. Moreover, its modular architecture supports diverse deployment scenarios, from centralized clouds to resource-constrained edge environments. Integration with traffic-aware schedulers ensures that KiSS maintains scalability and responsiveness even under fluctuating workloads.


\subsubsection{Advantages of KiSS}

The advantages of KiSS are particularly pronounced in edge environments. By keeping frequently accessed containers in warm states, it drastically reduces cold start latency, which is critical for real-time applications such as IoT and AI analytics. Static warm pool partitioning, based on workload analysis, optimizes memory usage by eliminating unnecessary overhead, ensuring that resources are used efficiently even in environments with stringent memory constraints. This strategy not only enhances performance but also reduces operational costs by consolidating memory usage and minimizing cold starts. KiSS’s platform-agnostic design further enhances its versatility, enabling seamless deployment across various serverless frameworks.


\section{Challenges of AI-Bridged Interactive Narrative Authoring}
%\qz{this section should contain definitions not appeared in Intro. Essentially it expands the problem space description, presents the design goal, and shed light onto our solution}

\label{sec:concept}
%\begin{figure*}
%    \centering
%    \includegraphics[width=.9\textwidth]{figures/ai_bridged_cla.pdf}
%    \caption{\qz{this is a bit confusing - I thought outline is also generated by AI system on the right? and on the left - author creates a narrative space rather than narrative example even just using traditional tools... }\yw{I think the emphasis here is that the author creates concrete storylines vs. abstract outline. In both cases they are creating narrative space. "Narrative example" is a misleading term. The middle diagram is describing existing creation paradigm for AI-bridged IN, not ours.  }}
%    \label{fig:multipivot}
%    \Description{}
%\end{figure*}

% [story vs. narrative. plot]


% Describe the distinct properties of a narrative space (game plot): a graph full of events. Complexity of this graph. Two types of authorial intent: procedural intent and declarative intent. 

% [abstraction]

% Three ways to represent narrative space:

% - Narrative Instance: a sample path in the space, good to show the details and procedural intent

% - Narrative Variant: a sub-plot in the space, good to perceive the diff and variations

% - Narrative Outline: high-level fuzzy description of important events; easy to see the authorial intent; hard to perceive details etc.

% \paragraph{From Narrative Instance to Narrative Space (or Executable Game-plot)}

% \paragraph{From Narrative Variants to Narrative Space (or Executable Game-plot)}

% \paragraph{From Narrative Outline to Narrative Space (or Executable Game-plot)}


%\subsection{Current Paradigm of AI-bridged Interactive Narrative Creation}

%\yw{probably should define narrative space more explicitly}
%\yw{it's not easy to grasp the connection between ``outline'' and ``prompt''}

%The creation of interactive narratives has moved beyond traditional paradigms where authors directly write content for players. Instead, with AI represented by LLMs, creators now develop a "narrative space," the conceptual layer based on which AI can further do creations to produce the final artifact ~\cite{moreno2007documental, gmeiner2023dimensions,lee2024navigating,wu2024role,margineda2024development}. Approaches such as setting personas for characters and driving AI behavior based on these personas (e.g., AI Village ~\cite{park2023generative}), or evolving stories within predefined game settings (e.g., AI Dungeon ~\cite{ai_dungeon2}), all exemplify this paradigm of AI-bridged interactive narrative creation. 


%Interactive Narrative allows players to influence storylines through their actions, with authors creating a {\em narrative space} of possible storylines. AI-bridged interactive narrative generates the narrative space using prompts, enhancing player agency and enabling emergent narratives beyond predefined branches.

%A clear demonstration of this paradigm involves using LLMs (e.g., ChatGPT) to function as a game engine for a text adventure game based on a narrative space revolving around an existing story. Specifically, an outline prompt will be provided to describe the narrative space, such as:

%\whosays{Act as a game engine that turns the following story into a text-adventure game. [Story Outline] The narrative flow should emulate the pacing and events of the story as closely as possible, ensuring that choices do not prematurely advance the plot. Pause the game after 5 rounds. After that...~\cite{10333140}}

%In this setup, the narrative space is anchored to the provided story with this outline determining its boundary. During gameplay, the player might input actions like "go to the mountain, and explore" via text, to which the AI responds dynamically to make the entire narrative aligned with the outline. For instance, if "the deer meets with the player" is specified in the outline, the AI arranges appropriate events, such as "The deer also moves to the mountain," to facilitate this outcome. Therefore, the outline directs the AI creation, defining the narrative space by specifying key events while allowing variations.  Through these interactions, the AI and player collaboratively unfold the narrative within the space.

%To obtain an ideal setting of the narrative space, creators need to put effort into refining the outline in the prompt. To do so, a common strategy for novice creators is to start with a concrete narrative example ~\cite{tomlinson2006learning,micallef2024use}. Based on the example, they can define features of the narrative space like how the play-time narrative can deviate from the example using specifications (e.g., "keep the original pace"), or identifying key moments (e.g., "this character dies after the first battle."). With iteratively generating examples, creators extract the desirable or undesirable elements and summarize them as rules for the outline to guide the narrative space. For instance, a rule might be "killing characters happens after the first scene." After numerous trials refining prompts, creators may feel they have clearly established the boundaries of the narrative space. At this point, creators may further test the game by anticipating potential player actions in gameplay. Then, despite creators' initial efforts at refinement, exceptions may still arise and make the narrative deviate from their intent. For instance, creators may not know that LLM could use implicit action "attack" and lead to the unexpected death of characters, or the abandonment of killing action makes the narrative fail to describe a key scene. Alternatively, a player might kill an important character in the first scene, disrupting the intended storyline. This leads to further rounds of revisions until the creator is again satisfied with the defined narrative space. Even so, creators may still lack a clear understanding of potential outcomes in actual gameplay.

%This highlights the issues lying in the mutual transformation between the narrative example and outline in AI-bridged interactive narrative creation. First, creators typically generate narrative examples and use them to derive descriptions for the outline to define the narrative space——both of which are developed heuristically. As a result, the outline that bound the narrative space is obtained through a ruleless trial-and-error process, which is inherently \textbf{intractable}. Moreover, the outline is gradually concretized by AI into narrative examples, which are further influenced by players' actions during gameplay. In this process, creators often lack a clear mental model of the AI's or the player's potential behaviors, making it difficult to set effective boundaries for the narrative space. This leaves the AI-driven creation process largely \textbf{uncontrollable}.


% Currently, large language models (LLMs) used in game development usually simply use LLMs to enrich the game content. Figures 1.A and 1.B show two examples of widely used AI-powered games. In Figure 1.A (Co-generation of Game Plots), without authors' specification on a concrete story, users and the LLM model alternately generate plot content. Rather than functioning strictly as a game, this setup resembles a collaborative writing task, where both the player and the LLM jointly contribute to script writing. In such a mode, the game creator's expression makes a minimal impact. In Figure 1.B (LLM-Generated Character Behavior), LLMs are used to enhance character behavior, particularly by adding vividness and detail to the language of the characters in the game plots. However, in this case, the game narrative remains fixed, with LLMs enriching the expression of pre-defined storylines. Despite of action choices provided to players in the game, this interaction often creates only the illusion of influencing the story, as the underlying narrative remains unaffected. While players may interact with characters by, for example, typing “hello,” their actions have no meaningful impact on the broader narrative.


% However, an ideal narrative-driven game should balance authorial intent and interactivity. Authorial intent refers to the narrative the author aims to convey, while interactivity refers to the player’s ability to meaningfully influence the story. The goal is to enable players to alter the narrative in impactful ways while maintaining a coherent story with clear moral expression from the author. However, neither is the approach in Figure 1. A nor Figure 1. B is capable of fully achieving this balance. 


% In interaction mode like AI Dungeon, players' interactivity is largely freeform, but there is little to no authorial intent guiding the experience. This makes it difficult for an author to predict or control what players will experience. On the other hand, using LLMs to enhance character behavior without altering the core storyline leaves the narrative largely unchanged. 

% Both interaction modes that leverage LLMs nowadays to create narrative-driven games overlook a critical aspect of typical game creation: the goal is not to generate a specific story but rather to create a narrative space. A narrative space encompasses all the possible ways a story can evolve while maintaining the same core theme or moral. For example, in a story centered around the theme of kindness, one version might depict an ant saving a dove in a forest, while another could feature the dove saving the ant. Despite the differences in these versions, both convey the same moral—kindness. By creatively arranging story elements, game creators can unfold multiple stories within a single narrative space. Typically, such work involves extensive manual planning by game creators, who write multiple branches and versions of the narrative to explore different storylines. Without addressing this concept of a narrative space, LLM-powered systems either lock players into a fixed story or leave the narrative highly open-ended, lacking the necessary authorial guidance. The challenge lies in designing LLM systems that allow for meaningful interactivity while preserving the author’s intent within a flexible narrative space, and effectively unfolding the narrative space into game plots.

% [draft, not satisfied.]







%\subsection{Shaping Narrative Space for AI-bridged Game Plot Generation}

\label{design_goals}

Interactive Narrative allows players to influence storylines through their actions, with authors creating a narrative space of possible storylines. \revision{During the authoring process, authors predefine the range of player actions and creates multiple storylines reflecting the consequences of different player choices.} AI-bridged IN generates just-in-time narrative content that adapts to different game world states, freeing authors from enumerating storylines. 
%In this workflow, authors prompt the AI model with an abstract narrative specifications, such as: 
%\whosays{Act as a game engine that turns the following story into a text-adventure game. [Story Outline] The narrative flow should emulate the pacing and events of the story as closely as possible, ensuring that choices do not prematurely advance the plot. Pause the game after 5 rounds. After that...~\cite{playbrary}}
However, shifting from traditional IN to AI-bridged IN presents challenges for authors in expressing, perceiving, and controlling the narrative space. Authors often struggle to articulate their implicit narrative intents in high-level prompts \cite{mirowski2023co} and may underexpress their intent to AI systems \cite{kreminski2024intent}. While novice authors might start with a concrete example \cite{tomlinson2006learning,micallef2024use}, a single narrative instance can be both overly detailed and insufficient, as it includes unnecessary specifics and lacks broader context \cite{kreminski2024intent}. Therefore, neither concrete instances nor abstract specifications alone are ideal for defining a narrative space. Instead, the ability to configure the level of abstraction is necessary to support AI-bridged IN authoring. 

On the other hand, once a narrative space is defined via prompts, the author has limited insight into the player experience, as players are responded with unscripted character actions and dialogs generated by LLMs at play-time. It is difficult to identify and prevent the deviations beyond the the author's narrative intent. Therefore, it is important to provide valuable information on how different types of player could react to instances, through which authors could preview the narrative instances as they get unfolded in the player experience \cite{kim2023language}. Furthermore, transforming prompts into meaningful game plots is not trivial. It requires effective narrative planning to generate causally sound event sequences. Central to this success is modeling the logical causal progression of the game plot \cite{riedl2010narrative}. However, LLMs are not natively planners in creating causal progression and have been found to cause hallucinations without external verifier to validate the coherency and executability of the generated plan \cite{kambhampati2024llms}. 
%Building on existing research in interactive narrative, we examined the current AI-bridged creation paradigm and identified several key challenges. The first issue is the difficulty in defining appropriate boundaries for the narrative space. Currently, the process of generating narrative examples and summarizing them into higher-level abstractions is largely heuristic, relying on intuitive thinking. However, constructing a narrative space—particularly through language-based boundaries like an outline—requires thorough thinking, and involves careful organization of language and structure. 
%Another challenge lies in understanding the boundaries of AI’s role within the narrative space. Creators need a clear mental model of how AI operates within this space to guide their creation. Without this understanding, there can be a disconnection between the creator’s intent and the final artifact built by the AI. Finally, the unpredictable behaviors of players interacting with the narrative add another layer of complexity~\cite{marincioni2024effect,peng2024player,you2024dungeons}. As player choices influence the AI's ongoing narrative development, unexpected outcomes can arise, making it difficult to maintain coherence in the story. This uncertainty makes it harder for creators to ensure a controllable narrative experience. 
% : (1) the challenges of understanding and editing "narrative space" in creating AI-driven interactive narratives, even though it’s important for LLM-powered interactive narrative, and (2) how to effectively turn a narrative space into a playable game plot.
% Many people find it hard to work with narrative space because it requires thinking beyond a single, straight story, even in traditional interactive narrative creation. This becomes even more complex when LLM-driven characters are involved. Even experienced writers may struggle to break down a linear story into a flexible structure that allows for different paths and outcomes. However, narrative space is essential for creating interactive plots that go beyond one storyline. Our framework helps users by providing them with intuitive representations of the narrative space and guiding them on how to sculpt the narrative space leveraging the concept of abstraction. By abstracting from a concrete story, users can distill the core elements of the plot and build multiple interactive possibilities around them,
Motivated by the challenges unresolved in AI-bridged IN, we developed the following design goals to guide the design of system:

\textbf{DG1: Enable users to perceive the narrative space.} 
The narrative space in AI-bridged IN contains various possible storylines, which are generated at play-time based on player actions. Authors might struggle to envision what types of variations would be possible. The system should provide representations that can help authors to explore and understand the narrative space. 
%Narrative examples are intuitive but insufficient in picturing the entire narrative space. Outlines, on the other hand, offer a structured view of the narrative space but are more abstract and difficult to derive efficiently. For novice creators, relying on solely either view can not provide them with a clear vision of the narrative space. We then aim to provide both formats of narrative examples and outlines as descriptors of the narrative space for a better perception of the space. 
%important because it prevents narrative instances that deviates from the desired narrative space

% Narrative space is challenging to grasp due to it inherently indicates analytical planning of the narrative. Even experienced writers may struggle to generate into a flexible structure that allows for different paths and outcomes. To address this, we focus on providing clear, intuitive presentations of the narrative space that highlight its important characteristics. By offering visual or structural representations, users can more easily perceive and comprehend the complexity of the narrative space.


\textbf{DG2: Support configurable level of abstraction in editing narrative space.} 
Concrete instances can be overly detailed, while abstract specifications can be too vague. Supporting users to adjust the level of abstraction helps them balance between details and abstraction, which allows the narrative instances to emerge from player interactions, while still adhering to the narrative structure. 
%Narrative examples are intuitive but insufficient in picturing the entire narrative space. Outlines, on the other hand, offer a structured view of the narrative space but are more abstract and difficult to derive efficiently. For novice creators, relying on solely either view can not provide them with a clear vision of the narrative space. We then aim to provide both formats of narrative examples and outlines as descriptors of the narrative space for a better perception of the space. 


%In addition to perceiving the narrative space, users also need effective ways to edit and refine it by setting appropriate boundaries. We propose using abstraction as a simple yet powerful tool that allows users to transform their straightforward stories into boundaries of narrative space. By applying high-level concepts through abstraction, users can more easily shape and adjust the narrative space to align with their creative vision. Editing narrative space via language inherently involves analytical thinking. Thus, Despite the competence of existing LLMs, their unstructured free-form communication with laypeople does not fully unlock their potential in such a specific domain. 
%To solve this, we will provide structured tools that guide users in using both flexible and systematic methods to shape their narrative space. By integrating these tools with LLM assistance, users will gain a better understanding of how and why certain narrative spaces are formed, allowing them to make more informed edits while maintaining flexibility in their creative process.

\textbf{DG3: Generate meaningful game events that react to player actions at play-time.} An engaging player experience requires the generated plots to represent logical causal progression that follows the game mechanism.
%in the game plot.
The proposed system should support simulating casual dynamics and generate meaningful narrative content based on player actions. 


%According to classic interactive narrative design principles, two key perspectives must be considered to reach a balance. First, the narrative should include enough diverse paths to ensure variety in the plot. Second, players’ actions should meaningfully impact the narrative’s development, altering the direction of their experience. Leveraging LLMs provides a solid foundation for generating diverse plot outcomes that respond dynamically to player input. Therefore, our approach to unfolding the narrative space into game plots utilizes LLMs to creatively generate content in real-time, offering interactivity that adapts to the players' actions.
%"Narrative and interactivity must be developed concurrently". Thus, the authorial intent in narrative construction must also be preserved in plot generation. LLMs are bad at this. Thus, we leverage the idea of symbolic planning, with authorial intent to characterize the narrative space serving as planning needs. 











% \section{WhatELSE: System and Features }
% \section{System}\label{sec:system}
We consider systems in the form
%
\begin{subequations}\label{eq:system}
	\begin{align}
		\label{eq:system:x0}
		x(t) &= x_0(t), & t &\in (-\infty, t_0], \\
		%
		\label{eq:system:x}
		\dot x(t) &= f(x(t), z(t)), & t &\in [t_0, t_f],
	\end{align}
\end{subequations}
%
where $t \in \R$ is time, $t_0, t_f \in \R$ are the initial and final time, $x: \R \rightarrow \R^{n_x}$ is the state, and $x_0: \R \rightarrow \R^{n_x}$ is the initial state function. Furthermore, $f: \R^{n_x} \times \R^{n_z} \rightarrow \R^{n_x}$ is the right-hand side function, and the memory state, $z: \R \rightarrow \R^{n_z}$, is given by the convolution
%
\begin{subequations}\label{eq:system:delay}
	\begin{align}
		\label{eq:system:z}
		z(t) &= \int\limits_{-\infty}^t \alpha(t - s) \odot r(s) \incr s, \\
		%
		\label{eq:system:r}
		r(t) &= h(x(t)),
	\end{align}
\end{subequations}
%
where $r: \R \rightarrow \R^{n_z}$ is the delayed variable, and each element of $\alpha: \Rnn \rightarrow \Rnn^{n_z}$ is a \emph{regular} kernel (see Definition~\ref{def:regular:kernel}). Furthermore, $h: \R^{n_x} \rightarrow \R^{n_z}$ is the memory function. We assume that $f$ and $h$ are differentiable in their arguments, and we refer to the paper by Ponosov et al.~\cite[Thm.~1]{Ponosov:etal:2004} for more details on the existence and uniqueness of solutions to the initial value problem~\eqref{eq:system}--\eqref{eq:system:delay}. See also the book by Hale and Lunel~\cite{Hale:Lunel:1993}.
%
\begin{definition}\label{def:regular:kernel}
	A scalar-valued kernel, $\alpha: \Rnn \rightarrow \Rnn$, is \emph{regular} if it satisfies the following properties.
	%
	\begin{enumerate}
		\item It is non-negative and bounded, i.e., $0 \leq \alpha(t) \leq K$ for all $t \in \Rnn$ and for some finite $K \in \Rp$.
		%
		\item It is continuous, i.e., for all $\epsilon \in \Rp$ and $t \in \Rnn$, there exists a $\delta \in \Rp$ such that $|\alpha(s) - \alpha(t)| < \epsilon$ for all $s \in \Rnn$ satisfying $|s - t| < \delta$.
		%
		\item It is normalized such that
	\end{enumerate}
	%
	\begin{align}\label{eq:kernel:normalization}
		\int\limits_0^\infty \alpha(t) \incr t &= 1.
	\end{align}
\end{definition}
%
For a given system of DDEs with distributed time delays, each element of $\alpha$ may not satisfy~\eqref{eq:kernel:normalization}. However, as they are assumed to be nonzero and non-negative, it is straightforward to normalize them. Next, we present a few well-known corollaries about the steady states of~\eqref{eq:system:x}--\eqref{eq:system:delay} and their stability.
%
\begin{corollary}\label{thm:steady:state}
	A state $\bar x \in \R^{n_x}$ is a steady state of the system~\eqref{eq:system:x}--\eqref{eq:system:delay} if
	%
	\begin{align}\label{eq:steady:state}
		0 &= f(\bar x, \bar z), &
		\bar z &= \bar r = h(\bar x).
	\end{align}
\end{corollary}

\begin{proof}
	In steady state, $x(t) = \bar x$ for all $t$. Consequently, $r(t) = \bar r = h(\bar x)$ and
	%
	\begin{align}
		z(t)
		&= \int_{-\infty}^t \alpha(t - s) \odot \bar r \incr s
		 = \int_{-\infty}^t \alpha(t - s) \incr s \odot \bar r
		 = \bar r,
	\end{align}
	%
	for all $t$, where we have used the property~\eqref{eq:kernel:normalization} of each element of $\alpha$.
\end{proof}
%
\begin{corollary}\label{thm:stability}
	The system~\eqref{eq:system:x}--\eqref{eq:system:delay} is locally asymptotically stable around a steady state, $\bar x$, satisfying~\eqref{eq:steady:state} if $\real \lambda < 0$ for all $\lambda \in \C$ that satisfy the characteristic equation
	%
	\begin{align}\label{eq:characteristic:equation}
		\det\left(F - \lambda I + G \int_0^\infty e^{-\lambda s} \diag \alpha(s) \incr s H\right) = 0,
	\end{align}
	%
	where $I \in \R^{n_x \times n_x}$ is an identity matrix.
	%
	The matrices $F \in \R^{n_x \times n_x}$, $G \in \R^{n_x \times n_z}$, and $H \in \R^{n_z \times n_x}$ are the Jacobians of the right-hand side function and the delay function evaluated in the steady state:
	%
	\begin{align}\label{eq:jacobians}
		F &= \pdiff{f}{x}(\bar x, \bar z), &
		G &= \pdiff{f}{z}(\bar x, \bar z), &
		H &= \pdiff{h}{x}(\bar x).
	\end{align}
	%
\end{corollary}

\begin{proof}
	The linearized system corresponding to~\eqref{eq:system:x}--\eqref{eq:system:delay} describes the evolution of the deviation variable $X: \R \rightarrow \R^{n_x}$:
	%
	\begin{align}\label{eq:linearized:system}
		\dot X(t) &= F X(t) + G \int_{-\infty}^t \alpha(t - s) \odot H X(s) \incr s, &
		X(t) &= x(t) - \bar x.
	\end{align}
	%
	See, e.g., \cite{Cushing:1975, Cushing:1977, Miller:1972} for proofs of the condition~\eqref{eq:characteristic:equation} for asymptotic stability of the linearized system in~\eqref{eq:linearized:system}.
\end{proof}



\section{WhatELSE: System Design and Implementation}
% \begin{figure*}[t]
% \centering
% \includegraphics[width=1.0\textwidth]{figures/WhatELSE.pdf}
% \vspace{-10pt}
% \caption{An overview of the system.}
% ~\label{overview}
% \vspace{-10pt}
% \end{figure*}

In this section, we present the interface and features of {\sc WhatELSE}, describe its technical pipeline to facilitate the transformation between narrative instances and narrative outlines, and demonstrate its workflow with an example user story.

\subsection{Narrative Space Editor Interface}

{\sc WhatELSE} system assists the user in creating a narrative space. The user can upload narrative examples in text file(s). \presentation{In addition, the user uses a sentence to describe a story's moral (e.g., {\it ``kindness is never wasted''}). The system uses the story input to construct an initial version of the narrative space. The user can edit this narrative space using the interface.}
%\subsubsection{Three Views of the Narrative Space} \label{views}
 {\sc WhatELSE} features three views for the user to perceive the narrative space: \textbf{Pivot View}, \textbf{Outline View}, and \textbf{Variants View} (Figure~\ref{overview}). 
 %Figure. \ref{overview} illustrated how each view is presented on panels in the narrative space editor. 

\noindent \presentation{
\textbf{Pivot View}\hspace{1mm} The pivot view shows a pivot narrative instance. A {\em pivot (narrative instance)} is a user-defined narrative instance, considered as a representative instance in the narrative space. By default, the user's input is automatically marked as the pivot. The pivot serves as a point of reference as the user edits the narrative space.}

\noindent\textbf{Outline View} \presentation{An {\em outline} is an abstract specification of a sequence of events defining the narrative space. }\originality{Similarly to ``loglines'' \cite{mirowski2023co}, it specifically describes the general structure of the narrative with a sequence of high-level events }\presentation{ - e.g., {\it ``A small creature runs into an accident. It was then saved by another creature''}.} The outline describes the narrative space from a broader perspective by capturing the commonality across all the narrative instances in the narrative space. It represents the most abstract manifestation of the author's narrative intent, thus defining the boundary of the narrative space.

\noindent\textbf{Variants View}\hspace{1mm} A {\em variant (narrative instance)} is a narrative instance residing in the current narrative space. A variant instantiates the outline with a sequence of concrete events - e.g., {\it "An ant fell into water. A dove dropped a leaf next to the ant. The ant climbed on the leaf. The ant was saved."} Each abstract event in the outline is expanded to multiple concrete events in a variant. 


The variants are displayed in an interactive scatter plot along two dimensions to help users understand the shape of the narrative space: 1) the \textbf{authorial intent} dimension, measured by the distance between the moral expressed by the variant and by the pivot (ranging from 0 to 1)\footnote{This distance is evaluated by prompting the LLM to assess how well the moral is conveyed}, and 2) the \textbf{emergence} dimension, measured by how much the plot progress in the variant deviates from the pivot (ranging from 0 to 1). \originality{These two dimensions are inspired by the ``authorial intent'' dimension in Riedl's taxonomy of IN approaches \cite{riedl2013interactive,riedl2009incorporating}, as well as the notion of ``emergence'' \cite{walsh2011emergent} and ``interactivity'' \cite{stang2019action} from prior IN research.} Users can configure the number of variants to be generated for visualization. Users can click on any variant in the visualization to display its detailed content, allowing them to compare it with the pivot. \presentation{Users can also use a scroll bar to visualize the plot progression} as they develop across different stages, allowing them to perceive how the narrative variants evolve over time and deviate from the pivot.

%In practice, users generate this scatter plot by clicking on the \textit{Generate Variants} button, as shown Figure. \ref{user_study:interface} C. Additionally, 


%, ranging from 1 to 5 sets. For each set, we generate 3 variants corresponding to the variant plots driven by the three types of players. For example, if a user chooses to visualize 4 sets, a total of 12 variants—categorized by player type—will be displayed on the scatter plot as shown in Figure. \ref{user_study:interface}. 


%Variant plots are a sequence of plots that players could experience under this narrative space constrained by the outline. Each variant consists of two parts: the game plot, and the player action. Player actions drive the gradual unfolding of game plots from the narrative space, and connect gameplots as a narrative. In other words, one variant records the full gaming experience of a specific player. Thus, such variation is caused by different game plots and different player actions jointly. For instance, the same game plot describing a character facing danger, can be led by two different player actions "save" and "kill" can lead to two variants; similarly, with the same player action "save", the character to be saved in game plot to be saved will also lead to two variants. 



%which is formatted from the user's input narrative instance, by extracting all events that occurred in the story and arranging the events sequentially as a plot. 
%Specifically, following the principles in defining events in digital games ~\cite{gould2011narrative,castellan2017games},  each event is a combination of "subject + action + object + potential location". 
%The view of the pivot plot shows the backbone of the user's input narrative instance, providing a straightforward structure of the narrative progression. More importantly, the pivot plot lies in the centric position of the narrative space. Narratives unfolded from the narrative space should always refer to this pivot plot. 

% Similar to how "acts" function in drama writing~\cite{styan1960elements},


% Therefore, the outline plot condenses the events in narrative instances into a less specific form compared to the detailed events, such as the pivot plot

These three views provide different perspectives for users to inspect the narrative space. We also provide editing tools at each view to support shaping the narrative space in different ways. 

%and are closely connected to each other. The outline is an abstraction of the pivot plot, and variant plots are generated within boundaries defined by the plot outline. Meanwhile, each variant plot serves as an alternative to the pivot plot. Therefore, the three views employed jointly describe the narrative space. 




\begin{figure*}[t]
\centering
\includegraphics[width=0.99\textwidth]{figures/whatelse_interface_v3.pdf}
\vspace{-10pt}
\caption{\presentation{An illustration of the Narrative Space Editor interface, including the pivot, outline, and variants view. Users can (A) generate outline from pivot or variants with an abstraction ladder to configure the abstraction level. They can (B) fine-tune sentence or word-level abstraction using an abstraction tooltip. They can also (C) generate variants from outline specifying the number of variants in the variants view. They can use (D) narrative progression slider to visualize the variants' dynamic distance from the pivot (star). }}
~\label{overview}
\vspace{-10pt}
\end{figure*}




\subsubsection{Support Editing the Narrative Space} 

\presentation{The system provides editing tools at outline and instance level.}

\noindent\textbf{Outline Editing}\hspace{1mm} \presentation{Users can constrain or relax the boundary of the narrative space by adjusting the outline's level of abstraction.}  The more abstract the outline is, the less constrained the narrative space is. \presentation{For example, {\it ``a small creature got into an accident''} is more abstract than {\it ``the ant fell into water''}, enabling more possible narrative instances to be generated. The former removes the constraint on {\it ``the small creature''} being  {\it ``the ant''}, and the {\it ``accident''} being {\it ``falling into water''}.} A less constrained narrative space allows stronger player agency but follows a looser authorial structure. Outline editing allows the user to tune the narrative space to reach a desired balance between authorial structure and player agency. \presentation{The system provides two tools to support the abstraction editing}.

\begin{itemize}
    \item \textbf{Abstraction Ladder (Figure. \ref{overview}.A)} The abstraction ladder helps the user to shift the global level of abstraction across the events in the outline. Inspired by theories of narrative structure ~\cite{styan1960elements, mckee1997story}, \presentation{this ladder covers a spectrum of abstraction levels (beat, scene, sequence, act, and story level)}. An outline at the beat level is similar to a narrative instance, while an outline at the story level summarizes the plot into a one-line overview. Between the two ends, each level of abstraction is progressively more abstract than the previous level. For instance, a scene-level outline provides detailed descriptions of specific scenes, including characters, actions, objects, etc: \textit{``The kind dove takes a leaf to reach the ant and drags it out of a water bubble.''} An act-level outline offers a highly summarized view of the narrative, focusing on the turning points: \textit{``A character saves their friend from danger.''}
    \item \textbf{Abstraction Tooltip (Figure. \ref{overview}.B)} The abstraction tooltip \presentation{allows the user to adjust the sentence, phrase, or word-level abstraction in a more fine-grained manner.} Practically, when users select a text snippet in their outline plots, the tooltip appears, offering two options: ``More Abstract'' and ``More Concrete''. \presentation{By clicking the button, users receive suggested edits that replace the selected content with a more abstract or more concrete phrase.} While the abstraction ladder provides global control over the entire outline, the tooltip enables more fine-grained adjustments at the word or phrase level. The suggestion of making the selected content more abstract or more concrete is based on the taxonomy in linguistics~\cite{hayes1983cognitive}. For example, {\it ``character-animal-small animal-cat-tabby cat''} constructs a linguistic hierarchy. Given a selected text snippet {\it ``cat''}, requesting a more abstract suggestion would yield its superordinate term {\it ``small animal''} or {\it ``animal''}, while a more concrete suggestion would provide its subordinate {\it ``tabby cat''}. 
\end{itemize}
Once the user is satisfied with the outline, they can click the ``Generate Variants'' button to generate narrative variants in the Variant View. Section \ref{compiler} describes the technical pipeline for generating narrative instances from outline. 

\noindent\textbf{Instance Editing}\hspace{1mm} \presentation{Users can fine-tune the narrative space by editing the instance-level content in Pivot and Variant View. They can select a variant to set or unset it as the pivot. They can also remove a variant from the narrative space or add it back. Finally, they can directly edit the text in the instances. They can click the ``Generate Outline'' button to update the outline based on their edited variants. For example, a user who does not want to include certain player type may choose to remove all variants by that player type and update the outline. Section \ref{compiler} describes different player types in the player proxy model. }

%Editing operation on the instance level allows the user to fine-tune the narrative space. We support the following editing operations on the instances: 1) direct text editing, 2) setting or unsetting as pivot, and 3) removing from or adding back to the narrative space. Once the user has done editing at the narrative instance level, they can click the ``generate outline'' button to synchronize the changes to the narrative space to the outline view. Section \ref{summarizer} describes the technical pipeline for generating outlines from narrative instances. 


%With the three views on the narrative space, we provide users with an intuitive way to perceive the narrative space. We then provide a series of tools as follows targeting \textbf{DG2}, to help users effectively shape the narrative space by configuring the level of abstraction. 





%Our first tool helps users obtain an ideal outline plot based on narrative instances. 
%Specifically, users can initiate the abstraction process by clicking the \textit{Generate Outline} button (i.e., Figure. \ref{user_study:interface} A). To enable customization of this abstraction process, we offer a feature named the "abstraction ladder." The abstraction ladder provides users with options of abstraction levels that they can choose from to generate the outline.



%To implement the abstraction ladder, we designed a prompt pipeline to help users ini outlines based on the defined abstraction levels. The pipeline uses professional drama writing knowledge as prior, and generates outlines via summarizing the commonalities among narrative instances. We present the detailed design of the prompt pipeline in the later Section ~\ref{sec:technical_pipeline}.


% We first provide knowledge of defined levels of abstraction referring to professional drama writing literature in the prompt. We then implemented a prompt chain, structured as a tree of thought. This chain operates in three stages: (1) Given the pivot plot, the first part of the chain prompts the LLM to generate a series of variations of the plot by creatively rearranging the elements involved. (2) The second stage generates three outlines corresponding to the predefined abstraction levels: scene, sequence, and act level. (3) In the final stage, the outline is tailored according to the user's specific requirements.



%\noindent\textbf{Variants-driven Editing} In addition to visualizing the variant plots, we designed tools to help users edit the narrative space by selecting among plot variants.  Additionally, users can refine the set of variants by removing selected variants from the set. For example, users may reject all narrative paths driven by negative players if they are unsatisfied with the narrative direction in the variants. Once users are satisfied with the remaining variants, they can use the \textit{Generate Outline} button to generate an outline based on the variants by summarizing their commonalities. Similarly, users can also use the abstraction ladder and customize the outline generation based on their specific needs.


%Overall, we provide a suite of tools that users can utilize to continuously shape the narrative space by editing across three intuitive representations. Specifically, we leverage the concept of abstraction to allow users to set appropriate boundaries within the narrative space, offering flexibility and control over how the narrative evolves and adapts to player actions.


\subsection{Technical Pipeline}

% \zl{todo, review 4.2}

\label{sec:technical_pipeline}
This section describes our technical pipeline supporting the features described in the above section, focusing on the transformation between narrative outline and narrative instances. Specifically, we employ the GPT-4o ~\cite{openai2023chatgpt} for the implementation of our system.


%To support the features we stated above, we design the technical pipeline of \textsc{WhatELSE} focusing on leveraging LLMs to facilitate two key processes: transforming narrative instances into outlines through a prompt pipeline, and converting outlines back into narrative instances using LLM-based narrative planning. We introduce the design and implementation of these two processes in the pipeline with the following two sections.


\subsubsection{Transforming Narrative Instances to Outline} \label{summarizer}
We use an LLM prompting pipeline to generate outlines from narrative instances (Figure \ref{system_overview}.1). This pipeline first prompts the LLM with domain knowledge in drama writing, providing the context of the story domain and the narrative instances. The pipeline then prompts the LLM to summarize the commonalities across these narrative instances, generating outlines at different abstraction levels based on story structure principles \cite{mckee1997story}. Finally, the system selects an outline according to the user's chosen level of abstraction.

\subsubsection{Transforming Outline to Narrative Instances} \label{compiler}

% \presentation{[delete me later: QZ revision for better readability, better structure the text for R1 R3]}
\presentation{To generate meaningful events that can react to player actions (DG3), we go beyond text generation and integrate an LLM-based narrative planning approach with character simulation and player proxy models. Our method extends \textit{StoryVerse} \cite{wang2024storyverse} with player interactivity and behavior modeling.}
 \presentation{Generating narrative instances from outline is essentially simulating an interactive story generation process, where player actions may be generated by computational proxies of players, and the story generated grounded in the causal changes of game world states in accordance with the game mechanism.} 
%To ground the outline with the concrete plot in order to enable a better understanding of the narrative space (DG1), and generate meaningful events that react to player actions (DG3), we go beyond text generation and develop a novel LLM-based narrative planning approach \revision{with character simulation and player proxy models} (Figure.\ref{system_overview}.2).


\begin{figure*}[t]
\centering
\includegraphics[width=\textwidth]{figures/system_overview.png}
\vspace{-10pt}
\caption{\presentation{An overview of the technical pipeline of \textsc{WhatELSE}. (1) The system transforms narrative instances to an outline using the LLM to summarize their commonalities, generate outlines at different levels of abstraction, and review the outline based on user specifications in the Abstraction Ladder. (2) The Interactive Narrative Compiler unfolds the outline into (3) a sequence of character actions to act out the events in the outline. (4) The Game Environment executes the actions and
updates the world states. (5) The player (or a simulated player) can interfere with the game by changing the world states. Finally, the Game Environment sends the updated world states and outline back to the compiler for the next iteration. }}
~\label{system_overview}
\vspace{-10pt}
\end{figure*}

To explain this process, we assume a {\em Game Environment} \presentation{(Figure~\ref{system_overview}.4) is given, which contains the {\em Story Domain} and maintains the {\em World State}.} The {\em World State} consists of a collection of variables that hold relevant values for the game mechanics, such as the characters’ attributes (e.g., health points), current locations, and relationship scores, as well as their memories from the simulation. \presentation{ The {\em Story Domain} contains a set of characters, locations, and an action schema that specifies executable actions in the game system. These actions are implemented as executable function calls that modify the variables of {\em World State} accordingly. For example, executing the action $\texttt{kill(X)}$ will result in character $\texttt{X}$'s state to become dead. }

\presentation{The main game loop starts by sending an event from the outline to the Interactive Narrative Compiler (Figure~\ref{system_overview}.2) to instantiate a sequence of character actions (Figure~\ref{system_overview}.3). The Game Environment (Figure~\ref{system_overview}.4) executes the actions and updates the world states resulting from the generated character actions. Once the Game Environment executes the actions, the player (or a simulated player) can interfere with the game by changing the world states, such as saving a character (Figure~\ref{system_overview}.5). Finally, the Game Environment sends the updated world states and outline back to IN Compiler for the next iteration.} The process loops over the events in the outline plot, and stops when it exhausts all the events.

%The main game loop alternates between 3 modes: 1) plot orchestration mode, 2) player action mode, and 3) character simulation mode. The process loops over the events in the outline plot, and stops when it exhausts all the events.

\vspace{2mm} \presentation{\noindent \textbf{Plot Generator} \hspace{2mm} Given an event in the outline, the system generates a sequence of character actions that act out the event. It takes into account the current game world state as a result of all previous plot executions and player actions. }
%\noindent \textbf{Plot Orchestration Mode} \hspace{2mm} Given an event in the outline, we generate a sequence of character actions that acts out the event\revision{, taking into account the current game world state as a result of all previous plot execution and character/player interactions}. 
%\begin{enumerate}
%    \item \textbf{Generation} \hspace{1mm} An LLM is prompted to generate a sequence of character actions that acts out the event. The prompt includes the following information from the game environment:
%\begin{itemize}
%   \item the list of characters and their descriptions;
%   \item the action schema;
%   \item current world state (including character's memory).
%\end{itemize}
An LLM is prompted to generate a sequence of character actions that act out the event. The prompt includes the following information from the game environment:
\begin{itemize}
   \item the list of characters and their descriptions;
   \item the action schema;
   \item current world state (including character's memory).
\end{itemize}

\presentation{This process is very similar to narrative planning which generates a sequence of state transitions that leads to a narrative goal. Compared to classic symbolic narrative planning, our narrative goal may be fuzzier - sometimes it may not be rigidly characterizable by world states. For example, the narrative goal could be {\it ``everyone likes Bob''}, which corresponds to world states semantically in a fuzzy way. This narrative goal can also be any arbitrary statements describing the desired outcome, constraining not only the endings but also the transitions, for example, {\it``someone was careless and got into an accident''}. Therefore, we use an LLM-based method instead of existing symbolic narrative planning methods for transforming outlines into concrete plots.}


\vspace{2mm} \presentation{\noindent \textbf{Plot Reviewer}} \hspace{2mm} \presentation{Similar to symbolic planning problems,  the plot generation requires causal soundness. This means the character actions must be valid state transitions according to the game's causal rules. We thus adopt an LLM-based planning method following the LLM-Modulo framework \cite{kambhampati2024llms}, with a game environment simulating plans generated by LLMs and providing external critiques.}
To review the generated plan, an LLM is prompted to provide feedback regarding the quality and feasibility of the action sequence to improve it:
    \begin{itemize}
    \item \textbf{Overall Coherency Evaluation} Feedback is obtained by prompting an LLM to comment on the overall coherency of the generated plot and make suggestions for improvement.
    \item \textbf{Character Simulation Evaluation} For every action in the sequence, we prompt an LLM to play the role of the subject of the action. Given the current world state including the character’s memory, we ask the LLM if the motivation for the character to perform the action has been established. We include the explanation to this question in the feedback if the motivation has not been established.
    \end{itemize}
    In addition, we leverage a simulated Game Environment for external evaluation. The system evaluates the \textbf{Causal soundness} of the generated action sequence and reports the observations on the success/failure of the execution. The combined feedback is added to the prompt for the next round of generation.

\vspace{2mm} \presentation{For example, the event {\it ``a small creature gets into an accident''} could be turned into a sequence of character actions shown in Figure~\ref{system_overview}.3. Note that the events in the outline plot are at a higher abstraction level. This means the same event can be transformed into multiple character action sequences, leaving room for the exact plot to adapt to different world states \footnote{In the above example, if the dove is dead at the time of plot execution, then a different character action sequence will be generated - one possibility is that the ant fell into the water.}.
Once the final sequence of character actions is generated, it will be executed by the {\em Game Environment} to update the world state. }

\presentation{The Plot Generator and Reviewer create a sequence of character actions to act out the event. In between these events, the player or NPCs take free actions. These actions are driven by the LLM.
The player actions are determined \presentation{either by a real player's input or a simulated Player Proxy Model (Figure~\ref{system_overview}.5)}. }

\vspace{2mm}\noindent \presentation{\textbf{Player Proxy Model}} \hspace{2mm} %In this mode, the player is prompted to input one or more actions following the action schema. The system execute each of the actions to update the world state if the action is executable with the current world state.
When generating narrative variants, player actions are provided by an LLM-based player proxy model which operates based on player behavior classification derived from previous studies in digital games ~\cite{yannakakis2013player,worth2015dimensions}. Our player simulation incorporates three key player behavior models:

\begin{itemize}
    \item \textbf{Positive Players} in digital games contribute positively by following the intended game objectives and exhibit helping behaviors~\cite{velez2013helping,bostan2009player}.
    \item \textbf{Negative Players} are the killers identified in classic player modeling~\cite{majors2021some,hamari2014player}. They often exhibit aggressive behavior that disrupts the experience of others, particularly when they seek to dominate or harm others destructively.
    \item \textbf{Role Players} prioritize narrative immersion and character development by mimicking the actions their character would take in the gaming world~\cite{praetorius2020avatars}.    
\end{itemize}

Using these player models, we simulate a potential plot that could emerge from the interaction between game characters and simulated players within the narrative space defined by the outline. In this way, the system generates a diverse set of narrative instances in the variants view.

\vspace{2mm} \noindent \presentation{\textbf{Non-Player Character Simulation}} \hspace{2mm} 
An LLM is prompted to role-play as each of the NPCs and generate an action for this character. The prompt includes the following information:
\begin{itemize}
   \item the action schema;
   \item the list of characters and their descriptions;
   \item current world state (including character's memory);
\end{itemize}
Note that the character actions are not directly determined by the outline. However, it is affected by the current world state and, therefore, indirectly influenced by the executed events in the outline.

\vspace{2mm}
\presentation{Using this pipeline, \textsc{WhatELSE} creates a gameplay experience by unfolding the outline into narrative instances. The system generates the game plot for each event in the outline as a series of character actions. The player then inputs actions within the action schema, which influence the progression of the subsequent plot. The system runs executable actions to update the world state. After each round of player action, the system unfolds the next events until exhausting all the events in the outline, in this way, creating an interactive narrative experience. }

\subsection{Example Workflow}

% \zl{format: \textsc{WhatELSE}, cite, quote,  }


Below we present an example workflow to demonstrate some of the features described above. 
\usecase{Alice, a novice text-adventure game designer, wants to create a game based on the setting of a novel she enjoys.} 
Alice opens \textsc{WhatELSE}, along with a game engine preloaded with a story domain based on the novel.

\subsubsection{Encode Authorial Intent in Narrative Space} Alice starts with a rough draft of the story and a moral she wants to convey: {\it ``Kindness is never wasted''}. Using \textsc{WhatELSE}, she uploads her initial story into the system \presentation{(Figure~\ref{walkthrough}.a)}. The story is displayed in the pivot view, showing a sequence of events; while an initial outline appears on the right, summarizing the key turning points \presentation{(Figure~\ref{walkthrough}.b)}. Alice adds details to the pivot to refine her story. Once satisfied, she clicks the Generate Outline button to update the outline based on her edits. She chooses the ``act level'' and specifies, {\it ``The hunter has to appear in every act''}. Alice hovers to see how each event in the outline is mapped to the entries in the pivot plot. She continues exploring different levels to find the ideal level of abstraction.

\presentation{Alice finds one of the events ({\it ``The peaceful life is threatened by an unexpected danger from the hunter''}) to be too restrictive for the hunter to cause the danger. She uses the abstraction tooltip to replace the phrase {\it ``the hunter''} with {\it ``a character''} to leave room for variations in the game.} 
Alice looks at the outline and is unsure what players might experience. So she clicks the Generate Variants button. The interface displays a scatter plot of potential narrative instances. Alice scrolls through different plot stages of these instances — from start to end — she notices that some instances continuously express the moral, while others only reveal it toward the end, both of which she considers acceptable. However, she also spots a cluster of instances that fail to express the moral by the end of the narrative. Curiously, she clicks on a dot representing one of these instances and reviews its details. 

Alice reads the instance and realizes the issue is in the event that she had previously set as {\it ``a character''}, which was too loosely defined, allowing the system to choose an undesirable character. To address this issue, she changed it back to {\it ``a human character with power''}, allowing the system to choose a character reasonable for the second event. 

\begin{figure*}[h]
\centering
\includegraphics[width=1.0\textwidth]{figures/walkthrough_v3.pdf}
\vspace{-20pt}
\caption{\presentation{An example workflow that shows (a) an author uploads a story draft in \textsc{WhatELSE} to (b) generate an outline. The system unfolds the outline into (c) an executable game plot with (d) a pre-loaded story domain, which supports branching storylines based on the player actions. If the player chooses to (e1) save the deer from the hunter, this action fulfills the ``brave assistance'' event in the outline defined by the author (shown as the orange star). If the player chooses to (e2) ask another character (e.g. a witch) for help, the witch will instead save the deer, demonstrating ``brave assistance'' to fulfill the event. Alternatively, if the player does not choose to save the deer at all, the system will choose a character from the story domain to save the deer as a demonstration of ``brave assistance''. This example shows how the game plot is dynamically adjusted based on the player actions to fulfill the outline. (f) The author can play the game plot to better understand the player experience. }}
~\label{walkthrough}
\vspace{-10pt}
\end{figure*}

Later, Alice notices a set of three variants where one of the events unfolds as, {\it ``the dove speaks with the hunter, leading the hunter to notice and then chase the dove''}. Alice finds this version more compelling than her pivot plot. She removes other variants, only leaving these three narrative variants in the view. Satisfied with these variants, she clicks the Generate Outline button to create a new outline that summarizes their commonalities. She then returns to the outline editor, using the abstraction tools to iteratively edit the outline, until it aligns with the story's moral and represents a narrative space that incorporates the interesting variations.

\subsubsection{Unfold the Narrative Space For interactivity}
With the narrative space defined by the outline, \usecase{Alice can experience the narrative instances unfolding in a turn-based text adventure game. She goes to the interactivity page. The system loads the story domain that includes a set of characters, locations, and action schema \presentation{(Figure~\ref{walkthrough}.d)}. The first sequence of the game plot is generated: a hunter is looking for food and finds a deer to hunt \presentation{(Figure~\ref{walkthrough}.c)}}. 

Alice, playing as the dove, chooses her next moves from a pin pad \usecase{\presentation{(Figure~\ref{walkthrough}.f)}. She can bravely stop the hunter by giving out her food \presentation{(Figure~\ref{walkthrough}.e1)}. Alternatively, she could ask other characters for help \presentation{(Figure~\ref{walkthrough}.e2)}. 
The system compares the player's action with the narrative outline. If the player chooses to save the deer on their own \presentation{(Figure~\ref{walkthrough}.e1)}, the event of {\it ``brave assistance''} is fulfilled by the player action. If the player chooses other actions, the system will create an event where another character demonstrates {\it ``brave assistance''} \presentation{(the witch in Figure~\ref{walkthrough})} to fulfill the event. The system generates subsequent character behaviors based on the player action. }

This turn-based interaction continues, with Alice alternating between reviewing generated game plots, observing character simulations, and experiencing the generated game play as a player. \usecase{Since she wrote a total of five events in her outline, the game play proceeds for five rounds, until all the events she planned have been played out. Since the game plots generated are fully structured, Alice can directly export the output of the narrative compiler as a finite state machine into the game engine where she can visualize the characters and locations.}

%The scenario above demonstrates how creators can use \textsc{WhatELSE} to create their interactive narratives by first encoding the authorial intent into a narrative space, and then unfolding the narrative space into play-time plot execution. 
\usecase{
\subsubsection{Additional Use Cases} In addition to Alice's case as a text-adventure game designer, \textsc{WhatELSE} can also serve as a powerful tool for a wide range of users. Game masters, mod developers, and fan creators across different domains can leverage its capabilities. For example, dungeon masters in tabletop role-playing games can use the Narrative Space Editor to outline gameplay scenarios before sessions and employ the Interactive Narrative Compiler to dynamically determine outcomes of player actions during gameplay. Fan creators~\cite{booth2009narractivity} can efficiently transform their favorite novels, movies, or other media into interactive narratives, using the \textsc{WhatELSE} to structure and unfold new, personalized storylines based on the original story domain. Beyond entertainment, educators can utilize \textsc{WhatELSE} to design interactive learning experiences, such as gamified learning tutorials or interactive training modules. }
%By outlining educational themes in the Narrative Space Editor and generating interactive scenarios through the Interactive Narrative Compiler, \textsc{WhatELSE} enables students to explore the theme dynamically, fostering engagement and deeper understanding through adaptive branching experiences.















\section{User Study}
\section{Methods}
Using ApoloBot as a discussion starting point, we extend our exploration into the broader landscape of restorative justice tools through a three-phase user study with Discord moderators. Each phase involves increasing levels of commitment, starting with initial interviews, followed by tool deployment, and concluding with reflections. Given that restorative justice tools are still relatively rare in online communities, these separate phases allow us to gather valuable insights while respecting moderators' diverse willingness and interest in the new approach. All parts of this study were pre-approved by our university's Institutional Review Board (IRB).

\subsection{Phases Overview}
%To evaluate the potential for ApoloBot and restorative justice tools more broadly, we conducted a user study with three phases. 

\textbf{Phase 1. Onboarding Session (60-90 minutes):} In the first phase, we conducted individual interviews with Discord moderators to gain insights into their general moderation practices and the potential of integrating restorative justice tools. Participants were asked about their procedure to handling interpersonal harm with specific examples of past cases. We then introduced the concept of restorative justice and presented ApoloBot as a practical tool embodying a subset of these principles. This was followed by discussions on the potential application of ApoloBot and other restorative justice tools within their communities, considering critical factors such as use cases, challenges, opportunities, perceived benefits, and drawbacks. After the interview, participants were invited to a Discord sandbox server to test out ApoloBot, where they provided further feedback and decided whether to continue with the study by deploying it in the subsequent phase.

\textbf{Phase 2. In-the-wild Deployment (4 weeks):} In the second phase, a subset of interested participants deployed ApoloBot in their communities, using it whenever suitable cases arose. Throughout this period, they kept track of their bot usage and maintained weekly communication with the researchers for feedback and support.

\textbf{Phase 3. Exit Interview (60-90 minutes):} At the end of the deployment period, participants joined an exit interview to reflect on their experiences with ApoloBot, unveiling new insights into its practical aspects, including user engagement and its effects on the community. Building on these reflections and revisiting critical factors from Phase 1 interviews, we \revision{explored the underlying factors for how the deployment met or challenged initial expectations, and} broadened the discussion to assess the overall design space of ApoloBot and other online restorative justice tools.

All interviews were conducted remotely through the Discord voice chat function. Participants could withdraw from the study at any phase without penalty. Compensation was provided for fully completed phases: \$20 for Phase 1, \$50 for Phase 2, and \$30 for Phase 3, delivered via Tremendous.~\footnote{https://www.tremendous.com/}


\subsection{Recruitment and Selection of Participants}
%We utilized a combination of platforms to distribute
Our recruitment call was distributed in meta-moderation communities on Discord, Reddit, and Facebook. These are communities where Discord moderators gather to discuss various moderation topics, such as news, strategies, philosophies, and tool usage. To ensure the quality of our recruits, we used a screening survey to assess their background and moderation experience. In addition to project-specific criteria such as prior experience handling interpersonal harm and familiarity with Discord bots, we filtered out low-quality responses such as one-word answers and those containing nonsensical or irrelevant information. We contacted selected participants, and further employed snowball sampling~\cite{Biernacki1981} by asking them for referrals. A total of 16 participants were chosen for Phase 1, with six proceeding to Phases 2 and 3. Two used ApoloBot during their deployment, while the others deployed it but did not encounter any suitable use cases. A summary of the participants' demographics and their status within Phase 1 and 2 are detailed in Table \ref{table:demographics}.
\begin{table*}[!ht]\scriptsize
\centering
\caption{Experiment - Participants demographics}
\label{tab:partdemograph}
\begin{tabular}{cccccccccccccccccccc}
\hline
\multicolumn{10}{c}{\textbf{Plugin Group}} & \multicolumn{10}{c}{\textbf{Control Group}} \\ \hline
\multicolumn{1}{c|}{\multirow{2}{*}{\textbf{ID}}} & \multicolumn{1}{c|}{\multirow{2}{*}{\textbf{Gender}}} & \multicolumn{1}{c|}{\multirow{2}{*}{\textbf{Persona}}} & \multicolumn{2}{c|}{\textbf{Experience}} & \multicolumn{5}{c|}{\textbf{Facets}} & \multicolumn{1}{c|}{\multirow{2}{*}{\textbf{ID}}} & \multicolumn{1}{c|}{\multirow{2}{*}{\textbf{Gender}}} & \multicolumn{1}{c|}{\multirow{2}{*}{\textbf{Persona}}} & \multicolumn{2}{c|}{\textbf{Experience}} & \multicolumn{5}{c}{\textbf{Facets}} \\ \cline{4-10} \cline{14-20} 
\multicolumn{1}{c|}{} & \multicolumn{1}{c|}{} & \multicolumn{1}{c|}{} & \multicolumn{1}{c|}{\textbf{GitHub}} & \multicolumn{1}{c|}{\textbf{OSS}} & \multicolumn{1}{c|}{\textbf{MT}} & \multicolumn{1}{c|}{\textbf{SE}} & \multicolumn{1}{c|}{\textbf{R}} & \multicolumn{1}{c|}{\textbf{IP}} & \multicolumn{1}{c|}{\textbf{L}} & \multicolumn{1}{c|}{} & \multicolumn{1}{c|}{} & \multicolumn{1}{c|}{} & \multicolumn{1}{c|}{\textbf{GitHub}} & \multicolumn{1}{c|}{\textbf{OSS}} & \multicolumn{1}{c|}{\textbf{MT}} & \multicolumn{1}{c|}{\textbf{SE}} & \multicolumn{1}{c|}{\textbf{R}} & \multicolumn{1}{c|}{\textbf{IP}} & \multicolumn{1}{c}{\textbf{L}} \\ \hline \hline

\multicolumn{1}{c|}{P1} & \multicolumn{1}{c|}{M} & \multicolumn{1}{c|}{\tikzcirclenew[fill=blue]{3pt}} & \multicolumn{1}{c|}{Never} & \multicolumn{1}{c|}{No} & \multicolumn{1}{c|}{\tikzcirclenew[fill=blue]{3pt}} & \multicolumn{1}{c|}{\tikzcirclenew[fill=blue]{3pt}} & \multicolumn{1}{c|}{\tikzcirclenew[fill=blue]{3pt}} & \multicolumn{1}{c|}{\tikzcircle[fill=orange]{3pt}} & \multicolumn{1}{c|}{\tikzcirclenew[fill=blue]{3pt}} & \multicolumn{1}{c|}{P40} & \multicolumn{1}{c|}{W} & \multicolumn{1}{c|}{\tikzcircle[fill=orange]{3pt}} & \multicolumn{1}{c|}{Once} & \multicolumn{1}{c|}{No} & \multicolumn{1}{c|}{\tikzcircle[fill=orange]{3pt}} & \multicolumn{1}{c|}{\tikzcirclenew[fill=blue]{3pt}} & \multicolumn{1}{c|}{\tikzcircle[fill=orange]{3pt}} & \multicolumn{1}{c|}{\tikzcircle[fill=orange]{3pt}} & \multicolumn{1}{c}{\tikzcirclenew[fill=blue]{3pt}} \\ \hline

\multicolumn{1}{c|}{P2} & \multicolumn{1}{c|}{W} & \multicolumn{1}{c|}{\tikzcirclenew[fill=blue]{3pt}} & \multicolumn{1}{c|}{Once} & \multicolumn{1}{c|}{No} & \multicolumn{1}{c|}{\tikzcirclenew[fill=blue]{3pt}} & \multicolumn{1}{c|}{\tikzcirclenew[fill=blue]{3pt}} & \multicolumn{1}{c|}{\tikzcirclenew[fill=blue]{3pt}} & \multicolumn{1}{c|}{\tikzcircle[fill=orange]{3pt}} & \multicolumn{1}{c|}{\tikzcirclenew[fill=blue]{3pt}} & \multicolumn{1}{c|}{P41} & \multicolumn{1}{c|}{M} & \multicolumn{1}{c|}{\tikzcirclenew[fill=blue]{3pt}} & \multicolumn{1}{c|}{Once} & \multicolumn{1}{c|}{No} & \multicolumn{1}{c|}{\tikzcirclenew[fill=blue]{3pt}} & \multicolumn{1}{c|}{\tikzcirclenew[fill=blue]{3pt}} & \multicolumn{1}{c|}{\tikzcircle[fill=orange]{3pt}} & \multicolumn{1}{c|}{\tikzcircle[fill=orange]{3pt}} & \multicolumn{1}{c}{\tikzcirclenew[fill=blue]{3pt}} \\ \hline

\multicolumn{1}{c|}{P3} & \multicolumn{1}{c|}{M} & \multicolumn{1}{c|}{\tikzcirclenew[fill=blue]{3pt}} & \multicolumn{1}{c|}{Never} & \multicolumn{1}{c|}{No} & \multicolumn{1}{c|}{\tikzcircle[fill=orange]{3pt}} & \multicolumn{1}{c|}{\tikzcirclenew[fill=blue]{3pt}} & \multicolumn{1}{c|}{\tikzcirclenew[fill=blue]{3pt}} & \multicolumn{1}{c|}{\tikzcircle[fill=orange]{3pt}} & \multicolumn{1}{c|}{\tikzcirclenew[fill=blue]{3pt}} & \multicolumn{1}{c|}{P42} & \multicolumn{1}{c|}{M} & \multicolumn{1}{c|}{\tikzcircle[fill=orange]{3pt}} & \multicolumn{1}{c|}{Never} & \multicolumn{1}{c|}{No} & \multicolumn{1}{c|}{\tikzcircle[fill=orange]{3pt}} & \multicolumn{1}{c|}{\tikzcirclenew[fill=blue]{3pt}} & \multicolumn{1}{c|}{\tikzcircle[fill=orange]{3pt}} & \multicolumn{1}{c|}{\tikzcircle[fill=orange]{3pt}} & \multicolumn{1}{c}{\tikzcirclenew[fill=blue]{3pt}} \\ \hline

\multicolumn{1}{c|}{P4} & \multicolumn{1}{c|}{M} & \multicolumn{1}{c|}{\tikzcirclenew[fill=blue]{3pt}} & \multicolumn{1}{c|}{Never} & \multicolumn{1}{c|}{No} & \multicolumn{1}{c|}{\tikzcircle[fill=orange]{3pt}} & \multicolumn{1}{c|}{\tikzcirclenew[fill=blue]{3pt}} & \multicolumn{1}{c|}{\tikzcirclenew[fill=blue]{3pt}} & \multicolumn{1}{c|}{\tikzcircle[fill=orange]{3pt}} & \multicolumn{1}{c|}{\tikzcirclenew[fill=blue]{3pt}} & \multicolumn{1}{c|}{P43} & \multicolumn{1}{c|}{M} & \multicolumn{1}{c|}{\tikzcirclenew[fill=blue]{3pt}} & \multicolumn{1}{c|}{Never} & \multicolumn{1}{c|}{No} & \multicolumn{1}{c|}{\tikzcirclenew[fill=blue]{3pt}} & \multicolumn{1}{c|}{\tikzcirclenew[fill=blue]{3pt}} & \multicolumn{1}{c|}{\tikzcirclenew[fill=blue]{3pt}} & \multicolumn{1}{c|}{\tikzcircle[fill=orange]{3pt}} & \multicolumn{1}{c}{\tikzcirclenew[fill=blue]{3pt}} \\ \hline

\multicolumn{1}{c|}{P5} & \multicolumn{1}{c|}{M} & \multicolumn{1}{c|}{\tikzcirclenew[fill=blue]{3pt}} & \multicolumn{1}{c|}{Once} & \multicolumn{1}{c|}{No} & \multicolumn{1}{c|}{\tikzcirclenew[fill=blue]{3pt}} & \multicolumn{1}{c|}{\tikzcirclenew[fill=blue]{3pt}} & \multicolumn{1}{c|}{\tikzcirclenew[fill=blue]{3pt}} & \multicolumn{1}{c|}{\tikzcircle[fill=orange]{3pt}} & \multicolumn{1}{c|}{\tikzcirclenew[fill=blue]{3pt}} & \multicolumn{1}{c|}{P44} & \multicolumn{1}{c|}{M} & \multicolumn{1}{c|}{\tikzcirclenew[fill=blue]{3pt}} & \multicolumn{1}{c|}{Never} & \multicolumn{1}{c|}{No} & \multicolumn{1}{c|}{\tikzcirclenew[fill=blue]{3pt}} & \multicolumn{1}{c|}{\tikzcirclenew[fill=blue]{3pt}} & \multicolumn{1}{c|}{\tikzcircle[fill=orange]{3pt}} & \multicolumn{1}{c|}{\tikzcircle[fill=orange]{3pt}} & \multicolumn{1}{c}{\tikzcirclenew[fill=blue]{3pt}} \\ \hline

\multicolumn{1}{c|}{P6} & \multicolumn{1}{c|}{M} & \multicolumn{1}{c|}{\tikzcirclenew[fill=blue]{3pt}} & \multicolumn{1}{c|}{Once} & \multicolumn{1}{c|}{No} & \multicolumn{1}{c|}{\tikzcirclenew[fill=blue]{3pt}} & \multicolumn{1}{c|}{\tikzcirclenew[fill=blue]{3pt}} & \multicolumn{1}{c|}{\tikzcircle[fill=orange]{3pt}} & \multicolumn{1}{c|}{\tikzcircle[fill=orange]{3pt}} & \multicolumn{1}{c|}{\tikzcirclenew[fill=blue]{3pt}} & \multicolumn{1}{c|}{P45} & \multicolumn{1}{c|}{M} & \multicolumn{1}{c|}{\tikzcirclenew[fill=blue]{3pt}} & \multicolumn{1}{c|}{Never} & \multicolumn{1}{c|}{No} & \multicolumn{1}{c|}{\tikzcirclenew[fill=blue]{3pt}} & \multicolumn{1}{c|}{\tikzcirclenew[fill=blue]{3pt}} & \multicolumn{1}{c|}{\tikzcirclenew[fill=blue]{3pt}} & \multicolumn{1}{c|}{\tikzcircle[fill=orange]{3pt}} & \multicolumn{1}{c}{\tikzcirclenew[fill=blue]{3pt}} \\ \hline

\multicolumn{1}{c|}{P7} & \multicolumn{1}{c|}{W} & \multicolumn{1}{c|}{\tikzcircle[fill=orange]{3pt}} & \multicolumn{1}{c|}{Never} & \multicolumn{1}{c|}{No} & \multicolumn{1}{c|}{\tikzcircle[fill=orange]{3pt}} & \multicolumn{1}{c|}{\tikzcirclenew[fill=blue]{3pt}} & \multicolumn{1}{c|}{\tikzcircle[fill=orange]{3pt}} & \multicolumn{1}{c|}{\tikzcircle[fill=orange]{3pt}} & \multicolumn{1}{c|}{\tikzcirclenew[fill=blue]{3pt}} & \multicolumn{1}{c|}{P46} & \multicolumn{1}{c|}{M} & \multicolumn{1}{c|}{\tikzcircle[fill=orange]{3pt}} & \multicolumn{1}{c|}{Never} & \multicolumn{1}{c|}{No} & \multicolumn{1}{c|}{\tikzcirclenew[fill=blue]{3pt}} & \multicolumn{1}{c|}{\tikzcircle[fill=orange]{3pt}} & \multicolumn{1}{c|}{\tikzcircle[fill=orange]{3pt}} & \multicolumn{1}{c|}{\tikzcircle[fill=orange]{3pt}} & \multicolumn{1}{c}{\tikzcirclenew[fill=blue]{3pt}} \\ \hline

\multicolumn{1}{c|}{P8} & \multicolumn{1}{c|}{W} & \multicolumn{1}{c|}{\tikzcircle[fill=orange]{3pt}} & \multicolumn{1}{c|}{Never} & \multicolumn{1}{c|}{No} & \multicolumn{1}{c|}{\tikzcirclenew[fill=blue]{3pt}} & \multicolumn{1}{c|}{\tikzcircle[fill=orange]{3pt}} & \multicolumn{1}{c|}{\tikzcircle[fill=orange]{3pt}} & \multicolumn{1}{c|}{\tikzcircle[fill=orange]{3pt}} & \multicolumn{1}{c|}{\tikzcirclenew[fill=blue]{3pt}} & \multicolumn{1}{c|}{P47} & \multicolumn{1}{c|}{M} & \multicolumn{1}{c|}{\tikzcirclenew[fill=blue]{3pt}} & \multicolumn{1}{c|}{Never} & \multicolumn{1}{c|}{No} & \multicolumn{1}{c|}{\tikzcirclenew[fill=blue]{3pt}} & \multicolumn{1}{c|}{\tikzcirclenew[fill=blue]{3pt}} & \multicolumn{1}{c|}{\tikzcirclenew[fill=blue]{3pt}} & \multicolumn{1}{c|}{\tikzcircle[fill=orange]{3pt}} & \multicolumn{1}{c}{\tikzcirclenew[fill=blue]{3pt}} \\ \hline

\multicolumn{1}{c|}{P9} & \multicolumn{1}{c|}{M} & \multicolumn{1}{c|}{\tikzcirclenew[fill=blue]{3pt}} & \multicolumn{1}{c|}{Once} & \multicolumn{1}{c|}{No} & \multicolumn{1}{c|}{\tikzcirclenew[fill=blue]{3pt}} & \multicolumn{1}{c|}{\tikzcirclenew[fill=blue]{3pt}} & \multicolumn{1}{c|}{\tikzcirclenew[fill=blue]{3pt}} & \multicolumn{1}{c|}{\tikzcircle[fill=orange]{3pt}} & \multicolumn{1}{c|}{\tikzcirclenew[fill=blue]{3pt}} & \multicolumn{1}{c|}{P48} & \multicolumn{1}{c|}{M} & \multicolumn{1}{c|}{\tikzcirclenew[fill=blue]{3pt}} & \multicolumn{1}{c|}{Never} & \multicolumn{1}{c|}{No} & \multicolumn{1}{c|}{\tikzcircle[fill=orange]{3pt}} & \multicolumn{1}{c|}{\tikzcirclenew[fill=blue]{3pt}} & \multicolumn{1}{c|}{\tikzcirclenew[fill=blue]{3pt}} & \multicolumn{1}{c|}{\tikzcircle[fill=orange]{3pt}} & \multicolumn{1}{c}{\tikzcirclenew[fill=blue]{3pt}} \\ \hline

\multicolumn{1}{c|}{P10} & \multicolumn{1}{c|}{W} & \multicolumn{1}{c|}{\tikzcirclenew[fill=blue]{3pt}} & \multicolumn{1}{c|}{Never} & \multicolumn{1}{c|}{No} & \multicolumn{1}{c|}{\tikzcirclenew[fill=blue]{3pt}} & \multicolumn{1}{c|}{\tikzcirclenew[fill=blue]{3pt}} & \multicolumn{1}{c|}{\tikzcircle[fill=orange]{3pt}} & \multicolumn{1}{c|}{\tikzcircle[fill=orange]{3pt}} & \multicolumn{1}{c|}{\tikzcirclenew[fill=blue]{3pt}} & \multicolumn{1}{c|}{P49} & \multicolumn{1}{c|}{M} & \multicolumn{1}{c|}{\tikzcircle[fill=orange]{3pt}} & \multicolumn{1}{c|}{Never} & \multicolumn{1}{c|}{No} & \multicolumn{1}{c|}{\tikzcircle[fill=orange]{3pt}} & \multicolumn{1}{c|}{\tikzcirclenew[fill=blue]{3pt}} & \multicolumn{1}{c|}{\tikzcircle[fill=orange]{3pt}} & \multicolumn{1}{c|}{\tikzcircle[fill=orange]{3pt}} & \multicolumn{1}{c}{\tikzcirclenew[fill=blue]{3pt}} \\ \hline

\multicolumn{1}{c|}{P11} & \multicolumn{1}{c|}{M} & \multicolumn{1}{c|}{\tikzcircle[fill=orange]{3pt}} & \multicolumn{1}{c|}{Never} & \multicolumn{1}{c|}{Some} & \multicolumn{1}{c|}{\tikzcircle[fill=orange]{3pt}} & \multicolumn{1}{c|}{\tikzcircle[fill=orange]{3pt}} & \multicolumn{1}{c|}{\tikzcircle[fill=orange]{3pt}} & \multicolumn{1}{c|}{\tikzcircle[fill=orange]{3pt}} & \multicolumn{1}{c|}{\tikzcircle[fill=orange]{3pt}} & \multicolumn{1}{c|}{P50} & \multicolumn{1}{c|}{M} & \multicolumn{1}{c|}{\tikzcirclenew[fill=blue]{3pt}} & \multicolumn{1}{c|}{Never} & \multicolumn{1}{c|}{No} & \multicolumn{1}{c|}{\tikzcirclenew[fill=blue]{3pt}} & \multicolumn{1}{c|}{\tikzcirclenew[fill=blue]{3pt}} & \multicolumn{1}{c|}{\tikzcirclenew[fill=blue]{3pt}} & \multicolumn{1}{c|}{\tikzcircle[fill=orange]{3pt}} & \multicolumn{1}{c}{\tikzcirclenew[fill=blue]{3pt}} \\ \hline

\multicolumn{1}{c|}{P12} & \multicolumn{1}{c|}{M} & \multicolumn{1}{c|}{\tikzcirclenew[fill=blue]{3pt}} & \multicolumn{1}{c|}{Never} & \multicolumn{1}{c|}{No} & \multicolumn{1}{c|}{\tikzcirclenew[fill=blue]{3pt}} & \multicolumn{1}{c|}{\tikzcirclenew[fill=blue]{3pt}} & \multicolumn{1}{c|}{\tikzcirclenew[fill=blue]{3pt}} & \multicolumn{1}{c|}{\tikzcircle[fill=orange]{3pt}} & \multicolumn{1}{c|}{\tikzcirclenew[fill=blue]{3pt}} & \multicolumn{1}{c|}{P51} & \multicolumn{1}{c|}{M} & \multicolumn{1}{c|}{\tikzcirclenew[fill=blue]{3pt}} & \multicolumn{1}{c|}{Never} & \multicolumn{1}{c|}{No} & \multicolumn{1}{c|}{\tikzcirclenew[fill=blue]{3pt}} & \multicolumn{1}{c|}{\tikzcirclenew[fill=blue]{3pt}} & \multicolumn{1}{c|}{\tikzcirclenew[fill=blue]{3pt}} & \multicolumn{1}{c|}{\tikzcircle[fill=orange]{3pt}} & \multicolumn{1}{c}{\tikzcirclenew[fill=blue]{3pt}} \\ \hline

\multicolumn{1}{c|}{P13} & \multicolumn{1}{c|}{W} & \multicolumn{1}{c|}{\tikzcircle[fill=orange]{3pt}} & \multicolumn{1}{c|}{Once} & \multicolumn{1}{c|}{No} & \multicolumn{1}{c|}{\tikzcircle[fill=orange]{3pt}} & \multicolumn{1}{c|}{\tikzcircle[fill=orange]{3pt}} & \multicolumn{1}{c|}{\tikzcircle[fill=orange]{3pt}} & \multicolumn{1}{c|}{\tikzcircle[fill=orange]{3pt}} & \multicolumn{1}{c|}{\tikzcircle[fill=orange]{3pt}} & \multicolumn{1}{c|}{P52} & \multicolumn{1}{c|}{M} & \multicolumn{1}{c|}{\tikzcircle[fill=orange]{3pt}} & \multicolumn{1}{c|}{Never} & \multicolumn{1}{c|}{Some} & \multicolumn{1}{c|}{\tikzcircle[fill=orange]{3pt}} & \multicolumn{1}{c|}{\tikzcircle[fill=orange]{3pt}} & \multicolumn{1}{c|}{\tikzcirclenew[fill=blue]{3pt}} & \multicolumn{1}{c|}{\tikzcircle[fill=orange]{3pt}} & \multicolumn{1}{c}{\tikzcircle[fill=orange]{3pt}} \\ \hline

\multicolumn{1}{c|}{P14} & \multicolumn{1}{c|}{W} & \multicolumn{1}{c|}{\tikzcircle[fill=orange]{3pt}} & \multicolumn{1}{c|}{Never} & \multicolumn{1}{c|}{Some} & \multicolumn{1}{c|}{\tikzcirclenew[fill=blue]{3pt}} & \multicolumn{1}{c|}{\tikzcirclenew[fill=blue]{3pt}} & \multicolumn{1}{c|}{\tikzcircle[fill=orange]{3pt}} & \multicolumn{1}{c|}{\tikzcircle[fill=orange]{3pt}} & \multicolumn{1}{c|}{\tikzcircle[fill=orange]{3pt}} & \multicolumn{1}{c|}{P53} & \multicolumn{1}{c|}{M} & \multicolumn{1}{c|}{\tikzcircle[fill=orange]{3pt}} & \multicolumn{1}{c|}{Once} & \multicolumn{1}{c|}{No} & \multicolumn{1}{c|}{\tikzcircle[fill=orange]{3pt}} & \multicolumn{1}{c|}{\tikzcircle[fill=orange]{3pt}} & \multicolumn{1}{c|}{\tikzcirclenew[fill=blue]{3pt}} & \multicolumn{1}{c|}{\tikzcircle[fill=orange]{3pt}} & \multicolumn{1}{c}{\tikzcircle[fill=orange]{3pt}} \\ \hline

\multicolumn{1}{c|}{P15} & \multicolumn{1}{c|}{M} & \multicolumn{1}{c|}{\tikzcircle[fill=orange]{3pt}} & \multicolumn{1}{c|}{Never} & \multicolumn{1}{c|}{No} & \multicolumn{1}{c|}{\tikzcircle[fill=orange]{3pt}} & \multicolumn{1}{c|}{\tikzcirclenew[fill=blue]{3pt}} & \multicolumn{1}{c|}{\tikzcirclenew[fill=blue]{3pt}} & \multicolumn{1}{c|}{\tikzcircle[fill=orange]{3pt}} & \multicolumn{1}{c|}{\tikzcircle[fill=orange]{3pt}} & \multicolumn{1}{c|}{P54} & \multicolumn{1}{c|}{W} & \multicolumn{1}{c|}{\tikzcircle[fill=orange]{3pt}} & \multicolumn{1}{c|}{Never} & \multicolumn{1}{c|}{No} & \multicolumn{1}{c|}{\tikzcircle[fill=orange]{3pt}} & \multicolumn{1}{c|}{\tikzcirclenew[fill=blue]{3pt}} & \multicolumn{1}{c|}{\tikzcircle[fill=orange]{3pt}} & \multicolumn{1}{c|}{\tikzcircle[fill=orange]{3pt}} & \multicolumn{1}{c}{\tikzcircle[fill=orange]{3pt}} \\ \hline

\multicolumn{1}{c|}{P16} & \multicolumn{1}{c|}{M} & \multicolumn{1}{c|}{\tikzcircle[fill=orange]{3pt}} & \multicolumn{1}{c|}{Once} & \multicolumn{1}{c|}{No} & \multicolumn{1}{c|}{\tikzcirclenew[fill=blue]{3pt}} & \multicolumn{1}{c|}{\tikzcircle[fill=orange]{3pt}} & \multicolumn{1}{c|}{\tikzcirclenew[fill=blue]{3pt}} & \multicolumn{1}{c|}{\tikzcircle[fill=orange]{3pt}} & \multicolumn{1}{c|}{\tikzcircle[fill=orange]{3pt}} & \multicolumn{1}{c|}{P55} & \multicolumn{1}{c|}{W} & \multicolumn{1}{c|}{\tikzcircle[fill=orange]{3pt}} & \multicolumn{1}{c|}{Once} & \multicolumn{1}{c|}{No} & \multicolumn{1}{c|}{\tikzcirclenew[fill=blue]{3pt}} & \multicolumn{1}{c|}{\tikzcircle[fill=orange]{3pt}} & \multicolumn{1}{c|}{\tikzcircle[fill=orange]{3pt}} & \multicolumn{1}{c|}{\tikzcircle[fill=orange]{3pt}} & \multicolumn{1}{c}{\tikzcircle[fill=orange]{3pt}} \\ \hline

\multicolumn{1}{c|}{P17} & \multicolumn{1}{c|}{W} & \multicolumn{1}{c|}{\tikzcircle[fill=orange]{3pt}} & \multicolumn{1}{c|}{Once} & \multicolumn{1}{c|}{No} & \multicolumn{1}{c|}{\tikzcircle[fill=orange]{3pt}} & \multicolumn{1}{c|}{\tikzcirclenew[fill=blue]{3pt}} & \multicolumn{1}{c|}{\tikzcircle[fill=orange]{3pt}} & \multicolumn{1}{c|}{\tikzcircle[fill=orange]{3pt}} & \multicolumn{1}{c|}{\tikzcircle[fill=orange]{3pt}} & \multicolumn{1}{c|}{P56} & \multicolumn{1}{c|}{M} & \multicolumn{1}{c|}{\tikzcircle[fill=orange]{3pt}} & \multicolumn{1}{c|}{Never} & \multicolumn{1}{c|}{No} & \multicolumn{1}{c|}{\tikzcircle[fill=orange]{3pt}} & \multicolumn{1}{c|}{\tikzcircle[fill=orange]{3pt}} & \multicolumn{1}{c|}{\tikzcirclenew[fill=blue]{3pt}} & \multicolumn{1}{c|}{\tikzcircle[fill=orange]{3pt}} & \multicolumn{1}{c}{\tikzcircle[fill=orange]{3pt}} \\ \hline

\multicolumn{1}{c|}{P18} & \multicolumn{1}{c|}{W} & \multicolumn{1}{c|}{\tikzcirclenew[fill=blue]{3pt}} & \multicolumn{1}{c|}{Once} & \multicolumn{1}{c|}{No} & \multicolumn{1}{c|}{\tikzcirclenew[fill=blue]{3pt}} & \multicolumn{1}{c|}{\tikzcirclenew[fill=blue]{3pt}} & \multicolumn{1}{c|}{\tikzcirclenew[fill=blue]{3pt}} & \multicolumn{1}{c|}{\tikzcircle[fill=orange]{3pt}} & \multicolumn{1}{c|}{\tikzcirclenew[fill=blue]{3pt}} & \multicolumn{1}{c|}{P57} & \multicolumn{1}{c|}{M} & \multicolumn{1}{c|}{\tikzcircle[fill=orange]{3pt}} & \multicolumn{1}{c|}{Few times} & \multicolumn{1}{c|}{No} & \multicolumn{1}{c|}{\tikzcircle[fill=orange]{3pt}} & \multicolumn{1}{c|}{\tikzcirclenew[fill=blue]{3pt}} & \multicolumn{1}{c|}{\tikzcircle[fill=orange]{3pt}} & \multicolumn{1}{c|}{\tikzcircle[fill=orange]{3pt}} & \multicolumn{1}{c}{\tikzcirclenew[fill=blue]{3pt}} \\ \hline

\multicolumn{1}{c|}{P19} & \multicolumn{1}{c|}{M} & \multicolumn{1}{c|}{\tikzcirclenew[fill=blue]{3pt}} & \multicolumn{1}{c|}{Never} & \multicolumn{1}{c|}{No} & \multicolumn{1}{c|}{\tikzcirclenew[fill=blue]{3pt}} & \multicolumn{1}{c|}{\tikzcirclenew[fill=blue]{3pt}} & \multicolumn{1}{c|}{\tikzcirclenew[fill=blue]{3pt}} & \multicolumn{1}{c|}{\tikzcircle[fill=orange]{3pt}} & \multicolumn{1}{c|}{\tikzcirclenew[fill=blue]{3pt}} & \multicolumn{1}{c|}{P58} & \multicolumn{1}{c|}{M} & \multicolumn{1}{c|}{\tikzcircle[fill=orange]{3pt}} & \multicolumn{1}{c|}{Once} & \multicolumn{1}{c|}{Some} & \multicolumn{1}{c|}{\tikzcircle[fill=orange]{3pt}} & \multicolumn{1}{c|}{\tikzcircle[fill=orange]{3pt}} & \multicolumn{1}{c|}{\tikzcirclenew[fill=blue]{3pt}} & \multicolumn{1}{c|}{\tikzcirclenew[fill=blue]{3pt}} & \multicolumn{1}{c}{\tikzcircle[fill=orange]{3pt}} \\ \hline

\multicolumn{1}{c|}{P20} & \multicolumn{1}{c|}{M} & \multicolumn{1}{c|}{\tikzcircle[fill=orange]{3pt}} & \multicolumn{1}{c|}{Few times} & \multicolumn{1}{c|}{No} & \multicolumn{1}{c|}{\tikzcircle[fill=orange]{3pt}} & \multicolumn{1}{c|}{\tikzcirclenew[fill=blue]{3pt}} & \multicolumn{1}{c|}{\tikzcircle[fill=orange]{3pt}} & \multicolumn{1}{c|}{\tikzcircle[fill=orange]{3pt}} & \multicolumn{1}{c|}{\tikzcircle[fill=orange]{3pt}} & \multicolumn{1}{c|}{P59} & \multicolumn{1}{c|}{M} & \multicolumn{1}{c|}{\tikzcirclenew[fill=blue]{3pt}} & \multicolumn{1}{c|}{Never} & \multicolumn{1}{c|}{No} & \multicolumn{1}{c|}{\tikzcirclenew[fill=blue]{3pt}} & \multicolumn{1}{c|}{\tikzcirclenew[fill=blue]{3pt}} & \multicolumn{1}{c|}{\tikzcirclenew[fill=blue]{3pt}} & \multicolumn{1}{c|}{\tikzcirclenew[fill=blue]{3pt}} & \multicolumn{1}{c}{\tikzcirclenew[fill=blue]{3pt}} \\ \hline

\multicolumn{1}{c|}{P21} & \multicolumn{1}{c|}{M} & \multicolumn{1}{c|}{\tikzcirclenew[fill=blue]{3pt}} & \multicolumn{1}{c|}{Often} & \multicolumn{1}{c|}{No} & \multicolumn{1}{c|}{\tikzcirclenew[fill=blue]{3pt}} & \multicolumn{1}{c|}{\tikzcirclenew[fill=blue]{3pt}} & \multicolumn{1}{c|}{\tikzcircle[fill=orange]{3pt}} & \multicolumn{1}{c|}{\tikzcircle[fill=orange]{3pt}} & \multicolumn{1}{c|}{\tikzcirclenew[fill=blue]{3pt}} & \multicolumn{1}{c|}{P60} & \multicolumn{1}{c|}{M} & \multicolumn{1}{c|}{\tikzcirclenew[fill=blue]{3pt}} & \multicolumn{1}{c|}{Once} & \multicolumn{1}{c|}{No} & \multicolumn{1}{c|}{\tikzcirclenew[fill=blue]{3pt}} & \multicolumn{1}{c|}{\tikzcircle[fill=orange]{3pt}} & \multicolumn{1}{c|}{\tikzcirclenew[fill=blue]{3pt}} & \multicolumn{1}{c|}{\tikzcirclenew[fill=blue]{3pt}} & \multicolumn{1}{c}{\tikzcirclenew[fill=blue]{3pt}} \\ \hline

\multicolumn{1}{c|}{P22} & \multicolumn{1}{c|}{W} & \multicolumn{1}{c|}{\tikzcirclenew[fill=blue]{3pt}} & \multicolumn{1}{c|}{Once} & \multicolumn{1}{c|}{No} & \multicolumn{1}{c|}{\tikzcirclenew[fill=blue]{3pt}} & \multicolumn{1}{c|}{\tikzcirclenew[fill=blue]{3pt}} & \multicolumn{1}{c|}{\tikzcirclenew[fill=blue]{3pt}} & \multicolumn{1}{c|}{\tikzcircle[fill=orange]{3pt}} & \multicolumn{1}{c|}{\tikzcircle[fill=orange]{3pt}} & \multicolumn{1}{c|}{P61} & \multicolumn{1}{c|}{M} & \multicolumn{1}{c|}{\tikzcircle[fill=orange]{3pt}} & \multicolumn{1}{c|}{Few times} & \multicolumn{1}{c|}{No} & \multicolumn{1}{c|}{\tikzcircle[fill=orange]{3pt}} & \multicolumn{1}{c|}{\tikzcircle[fill=orange]{3pt}} & \multicolumn{1}{c|}{\tikzcirclenew[fill=blue]{3pt}} & \multicolumn{1}{c|}{\tikzcirclenew[fill=blue]{3pt}} & \multicolumn{1}{c}{\tikzcircle[fill=orange]{3pt}} \\ \hline

\multicolumn{1}{c|}{P23} & \multicolumn{1}{c|}{W} & \multicolumn{1}{c|}{\tikzcirclenew[fill=blue]{3pt}} & \multicolumn{1}{c|}{Often} & \multicolumn{1}{c|}{No} & \multicolumn{1}{c|}{\tikzcirclenew[fill=blue]{3pt}} & \multicolumn{1}{c|}{\tikzcirclenew[fill=blue]{3pt}} & \multicolumn{1}{c|}{\tikzcirclenew[fill=blue]{3pt}} & \multicolumn{1}{c|}{\tikzcircle[fill=orange]{3pt}} & \multicolumn{1}{c|}{\tikzcirclenew[fill=blue]{3pt}} & \multicolumn{1}{c|}{P62} & \multicolumn{1}{c|}{M} & \multicolumn{1}{c|}{\tikzcircle[fill=orange]{3pt}} & \multicolumn{1}{c|}{Few times} & \multicolumn{1}{c|}{No} & \multicolumn{1}{c|}{\tikzcircle[fill=orange]{3pt}} & \multicolumn{1}{c|}{\tikzcirclenew[fill=blue]{3pt}} & \multicolumn{1}{c|}{\tikzcircle[fill=orange]{3pt}} & \multicolumn{1}{c|}{\tikzcircle[fill=orange]{3pt}} & \multicolumn{1}{c}{\tikzcircle[fill=orange]{3pt}} \\ \hline

\multicolumn{1}{c|}{P24} & \multicolumn{1}{c|}{M} & \multicolumn{1}{c|}{\tikzcirclenew[fill=blue]{3pt}} & \multicolumn{1}{c|}{Often} & \multicolumn{1}{c|}{No} & \multicolumn{1}{c|}{\tikzcirclenew[fill=blue]{3pt}} & \multicolumn{1}{c|}{\tikzcircle[fill=orange]{3pt}} & \multicolumn{1}{c|}{\tikzcirclenew[fill=blue]{3pt}} & \multicolumn{1}{c|}{\tikzcircle[fill=orange]{3pt}} & \multicolumn{1}{c|}{\tikzcirclenew[fill=blue]{3pt}} & \multicolumn{1}{c|}{P63} & \multicolumn{1}{c|}{W} & \multicolumn{1}{c|}{\tikzcirclenew[fill=blue]{3pt}} & \multicolumn{1}{c|}{Never} & \multicolumn{1}{c|}{No} & \multicolumn{1}{c|}{\tikzcircle[fill=orange]{3pt}} & \multicolumn{1}{c|}{\tikzcirclenew[fill=blue]{3pt}} & \multicolumn{1}{c|}{\tikzcirclenew[fill=blue]{3pt}} & \multicolumn{1}{c|}{\tikzcircle[fill=orange]{3pt}} & \multicolumn{1}{c}{\tikzcirclenew[fill=blue]{3pt}} \\ \hline

\multicolumn{1}{c|}{P25} & \multicolumn{1}{c|}{M} & \multicolumn{1}{c|}{\tikzcirclenew[fill=blue]{3pt}} & \multicolumn{1}{c|}{Often} & \multicolumn{1}{c|}{No} & \multicolumn{1}{c|}{\tikzcircle[fill=orange]{3pt}} & \multicolumn{1}{c|}{\tikzcirclenew[fill=blue]{3pt}} & \multicolumn{1}{c|}{\tikzcirclenew[fill=blue]{3pt}} & \multicolumn{1}{c|}{\tikzcirclenew[fill=blue]{3pt}} & \multicolumn{1}{c|}{\tikzcircle[fill=orange]{3pt}} & \multicolumn{1}{c|}{P64} & \multicolumn{1}{c|}{W} & \multicolumn{1}{c|}{\tikzcirclenew[fill=blue]{3pt}} & \multicolumn{1}{c|}{Never} & \multicolumn{1}{c|}{No} & \multicolumn{1}{c|}{\tikzcirclenew[fill=blue]{3pt}} & \multicolumn{1}{c|}{\tikzcirclenew[fill=blue]{3pt}} & \multicolumn{1}{c|}{\tikzcirclenew[fill=blue]{3pt}} & \multicolumn{1}{c|}{\tikzcircle[fill=orange]{3pt}} & \multicolumn{1}{c}{\tikzcirclenew[fill=blue]{3pt}} \\ \hline

\multicolumn{1}{c|}{P26} & \multicolumn{1}{c|}{W} & \multicolumn{1}{c|}{\tikzcircle[fill=orange]{3pt}} & \multicolumn{1}{c|}{Few times} & \multicolumn{1}{c|}{No} & \multicolumn{1}{c|}{\tikzcircle[fill=orange]{3pt}} & \multicolumn{1}{c|}{\tikzcircle[fill=orange]{3pt}} & \multicolumn{1}{c|}{\tikzcirclenew[fill=blue]{3pt}} & \multicolumn{1}{c|}{\tikzcircle[fill=orange]{3pt}} & \multicolumn{1}{c|}{\tikzcircle[fill=orange]{3pt}} & \multicolumn{1}{c|}{P65} & \multicolumn{1}{c|}{M} & \multicolumn{1}{c|}{\tikzcirclenew[fill=blue]{3pt}} & \multicolumn{1}{c|}{Never} & \multicolumn{1}{c|}{No} & \multicolumn{1}{c|}{\tikzcircle[fill=orange]{3pt}} & \multicolumn{1}{c|}{\tikzcirclenew[fill=blue]{3pt}} & \multicolumn{1}{c|}{\tikzcirclenew[fill=blue]{3pt}} & \multicolumn{1}{c|}{\tikzcircle[fill=orange]{3pt}} & \multicolumn{1}{c}{\tikzcirclenew[fill=blue]{3pt}} \\ \hline

\multicolumn{1}{c|}{P27} & \multicolumn{1}{c|}{M} & \multicolumn{1}{c|}{\tikzcircle[fill=orange]{3pt}} & \multicolumn{1}{c|}{Few times} & \multicolumn{1}{c|}{No} & \multicolumn{1}{c|}{\tikzcircle[fill=orange]{3pt}} & \multicolumn{1}{c|}{\tikzcirclenew[fill=blue]{3pt}} & \multicolumn{1}{c|}{\tikzcircle[fill=orange]{3pt}} & \multicolumn{1}{c|}{\tikzcircle[fill=orange]{3pt}} & \multicolumn{1}{c|}{\tikzcircle[fill=orange]{3pt}} & \multicolumn{1}{c|}{P66} & \multicolumn{1}{c|}{W} & \multicolumn{1}{c|}{\tikzcirclenew[fill=blue]{3pt}} & \multicolumn{1}{c|}{Never} & \multicolumn{1}{c|}{No} & \multicolumn{1}{c|}{\tikzcircle[fill=orange]{3pt}} & \multicolumn{1}{c|}{\tikzcirclenew[fill=blue]{3pt}} & \multicolumn{1}{c|}{\tikzcirclenew[fill=blue]{3pt}} & \multicolumn{1}{c|}{\tikzcircle[fill=orange]{3pt}} & \multicolumn{1}{c}{\tikzcirclenew[fill=blue]{3pt}} \\ \hline

\multicolumn{1}{c|}{P28} & \multicolumn{1}{c|}{M} & \multicolumn{1}{c|}{\tikzcircle[fill=orange]{3pt}} & \multicolumn{1}{c|}{Once} & \multicolumn{1}{c|}{No} & \multicolumn{1}{c|}{\tikzcircle[fill=orange]{3pt}} & \multicolumn{1}{c|}{\tikzcirclenew[fill=blue]{3pt}} & \multicolumn{1}{c|}{\tikzcirclenew[fill=blue]{3pt}} & \multicolumn{1}{c|}{\tikzcircle[fill=orange]{3pt}} & \multicolumn{1}{c|}{\tikzcircle[fill=orange]{3pt}} & \multicolumn{1}{c|}{P67} & \multicolumn{1}{c|}{M} & \multicolumn{1}{c|}{\tikzcircle[fill=orange]{3pt}} & \multicolumn{1}{c|}{Never} & \multicolumn{1}{c|}{No} & \multicolumn{1}{c|}{\tikzcircle[fill=orange]{3pt}} & \multicolumn{1}{c|}{\tikzcircle[fill=orange]{3pt}} & \multicolumn{1}{c|}{\tikzcirclenew[fill=blue]{3pt}} & \multicolumn{1}{c|}{\tikzcircle[fill=orange]{3pt}} & \multicolumn{1}{c}{\tikzcircle[fill=orange]{3pt}} \\ \hline

\multicolumn{1}{c|}{P29} & \multicolumn{1}{c|}{M} & \multicolumn{1}{c|}{\tikzcircle[fill=orange]{3pt}} & \multicolumn{1}{c|}{Never} & \multicolumn{1}{c|}{No} & \multicolumn{1}{c|}{\tikzcircle[fill=orange]{3pt}} & \multicolumn{1}{c|}{\tikzcircle[fill=orange]{3pt}} & \multicolumn{1}{c|}{\tikzcircle[fill=orange]{3pt}} & \multicolumn{1}{c|}{\tikzcircle[fill=orange]{3pt}} & \multicolumn{1}{c|}{\tikzcircle[fill=orange]{3pt}} & \multicolumn{1}{c|}{P68} & \multicolumn{1}{c|}{M} & \multicolumn{1}{c|}{\tikzcirclenew[fill=blue]{3pt}} & \multicolumn{1}{c|}{Few times} & \multicolumn{1}{c|}{No} & \multicolumn{1}{c|}{\tikzcirclenew[fill=blue]{3pt}} & \multicolumn{1}{c|}{\tikzcirclenew[fill=blue]{3pt}} & \multicolumn{1}{c|}{\tikzcirclenew[fill=blue]{3pt}} & \multicolumn{1}{c|}{\tikzcircle[fill=orange]{3pt}} & \multicolumn{1}{c}{\tikzcircle[fill=orange]{3pt}} \\ \hline

\multicolumn{1}{c|}{P30} & \multicolumn{1}{c|}{M} & \multicolumn{1}{c|}{\tikzcirclenew[fill=blue]{3pt}} & \multicolumn{1}{c|}{Few times} & \multicolumn{1}{c|}{No} & \multicolumn{1}{c|}{\tikzcirclenew[fill=blue]{3pt}} & \multicolumn{1}{c|}{\tikzcirclenew[fill=blue]{3pt}} & \multicolumn{1}{c|}{\tikzcirclenew[fill=blue]{3pt}} & \multicolumn{1}{c|}{\tikzcircle[fill=orange]{3pt}} & \multicolumn{1}{c|}{\tikzcirclenew[fill=blue]{3pt}} & \multicolumn{1}{c|}{P69} & \multicolumn{1}{c|}{M} & \multicolumn{1}{c|}{\tikzcirclenew[fill=blue]{3pt}} & \multicolumn{1}{c|}{Few times} & \multicolumn{1}{c|}{No} & \multicolumn{1}{c|}{\tikzcirclenew[fill=blue]{3pt}} & \multicolumn{1}{c|}{\tikzcirclenew[fill=blue]{3pt}} & \multicolumn{1}{c|}{\tikzcirclenew[fill=blue]{3pt}} & \multicolumn{1}{c|}{\tikzcirclenew[fill=blue]{3pt}} & \multicolumn{1}{c}{\tikzcircle[fill=orange]{3pt}} \\ \hline

\multicolumn{1}{c|}{P31} & \multicolumn{1}{c|}{M} & \multicolumn{1}{c|}{\tikzcirclenew[fill=blue]{3pt}} & \multicolumn{1}{c|}{Never} & \multicolumn{1}{c|}{No} & \multicolumn{1}{c|}{\tikzcircle[fill=orange]{3pt}} & \multicolumn{1}{c|}{\tikzcirclenew[fill=blue]{3pt}} & \multicolumn{1}{c|}{\tikzcirclenew[fill=blue]{3pt}} & \multicolumn{1}{c|}{\tikzcirclenew[fill=blue]{3pt}} & \multicolumn{1}{c|}{\tikzcirclenew[fill=blue]{3pt}} & \multicolumn{1}{c|}{P70} & \multicolumn{1}{c|}{M} & \multicolumn{1}{c|}{\tikzcirclenew[fill=blue]{3pt}} & \multicolumn{1}{c|}{Few times} & \multicolumn{1}{c|}{Some} & \multicolumn{1}{c|}{\tikzcirclenew[fill=blue]{3pt}} & \multicolumn{1}{c|}{\tikzcirclenew[fill=blue]{3pt}} & \multicolumn{1}{c|}{\tikzcirclenew[fill=blue]{3pt}} & \multicolumn{1}{c|}{\tikzcircle[fill=orange]{3pt}} & \multicolumn{1}{c}{\tikzcirclenew[fill=blue]{3pt}} \\ \hline

\multicolumn{1}{c|}{P32} & \multicolumn{1}{c|}{M} & \multicolumn{1}{c|}{\tikzcircle[fill=orange]{3pt}} & \multicolumn{1}{c|}{Never} & \multicolumn{1}{c|}{No} & \multicolumn{1}{c|}{\tikzcirclenew[fill=blue]{3pt}} & \multicolumn{1}{c|}{\tikzcirclenew[fill=blue]{3pt}} & \multicolumn{1}{c|}{\tikzcircle[fill=orange]{3pt}} & \multicolumn{1}{c|}{\tikzcircle[fill=orange]{3pt}} & \multicolumn{1}{c|}{\tikzcircle[fill=orange]{3pt}} & \multicolumn{1}{c|}{P71} & \multicolumn{1}{c|}{M} & \multicolumn{1}{c|}{\tikzcircle[fill=orange]{3pt}} & \multicolumn{1}{c|}{Few times} & \multicolumn{1}{c|}{No} & \multicolumn{1}{c|}{\tikzcircle[fill=orange]{3pt}} & \multicolumn{1}{c|}{\tikzcircle[fill=orange]{3pt}} & \multicolumn{1}{c|}{\tikzcircle[fill=orange]{3pt}} & \multicolumn{1}{c|}{\tikzcircle[fill=orange]{3pt}} & \multicolumn{1}{c}{\tikzcirclenew[fill=blue]{3pt}} \\ \hline

\multicolumn{1}{c|}{P33} & \multicolumn{1}{c|}{M} & \multicolumn{1}{c|}{\tikzcirclenew[fill=blue]{3pt}} & \multicolumn{1}{c|}{Few times} & \multicolumn{1}{c|}{No} & \multicolumn{1}{c|}{\tikzcircle[fill=orange]{3pt}} & \multicolumn{1}{c|}{\tikzcirclenew[fill=blue]{3pt}} & \multicolumn{1}{c|}{\tikzcirclenew[fill=blue]{3pt}} & \multicolumn{1}{c|}{\tikzcircle[fill=orange]{3pt}} & \multicolumn{1}{c|}{\tikzcirclenew[fill=blue]{3pt}} & \multicolumn{1}{c|}{P72} & \multicolumn{1}{c|}{M} & \multicolumn{1}{c|}{\tikzcirclenew[fill=blue]{3pt}} & \multicolumn{1}{c|}{Few times} & \multicolumn{1}{c|}{No} & \multicolumn{1}{c|}{\tikzcirclenew[fill=blue]{3pt}} & \multicolumn{1}{c|}{\tikzcircle[fill=orange]{3pt}} & \multicolumn{1}{c|}{\tikzcirclenew[fill=blue]{3pt}} & \multicolumn{1}{c|}{\tikzcircle[fill=orange]{3pt}} & \multicolumn{1}{c}{\tikzcirclenew[fill=blue]{3pt}} \\ \hline

\multicolumn{1}{c|}{P34} & \multicolumn{1}{c|}{M} & \multicolumn{1}{c|}{\tikzcirclenew[fill=blue]{3pt}} & \multicolumn{1}{c|}{Few times} & \multicolumn{1}{c|}{No} & \multicolumn{1}{c|}{\tikzcirclenew[fill=blue]{3pt}} & \multicolumn{1}{c|}{\tikzcircle[fill=orange]{3pt}} & \multicolumn{1}{c|}{\tikzcirclenew[fill=blue]{3pt}} & \multicolumn{1}{c|}{\tikzcircle[fill=orange]{3pt}} & \multicolumn{1}{c|}{\tikzcirclenew[fill=blue]{3pt}} & \multicolumn{1}{c|}{P73} & \multicolumn{1}{c|}{M} & \multicolumn{1}{c|}{\tikzcirclenew[fill=blue]{3pt}} & \multicolumn{1}{c|}{Once} & \multicolumn{1}{c|}{No} & \multicolumn{1}{c|}{\tikzcirclenew[fill=blue]{3pt}} & \multicolumn{1}{c|}{\tikzcirclenew[fill=blue]{3pt}} & \multicolumn{1}{c|}{\tikzcirclenew[fill=blue]{3pt}} & \multicolumn{1}{c|}{\tikzcircle[fill=orange]{3pt}} & \multicolumn{1}{c}{\tikzcirclenew[fill=blue]{3pt}} \\ \hline

\multicolumn{1}{c|}{P35} & \multicolumn{1}{c|}{W} & \multicolumn{1}{c|}{\tikzcirclenew[fill=blue]{3pt}} & \multicolumn{1}{c|}{Few times} & \multicolumn{1}{c|}{No} & \multicolumn{1}{c|}{\tikzcirclenew[fill=blue]{3pt}} & \multicolumn{1}{c|}{\tikzcirclenew[fill=blue]{3pt}} & \multicolumn{1}{c|}{\tikzcircle[fill=orange]{3pt}} & \multicolumn{1}{c|}{\tikzcircle[fill=orange]{3pt}} & \multicolumn{1}{c|}{\tikzcirclenew[fill=blue]{3pt}} & \multicolumn{1}{c|}{P74} & \multicolumn{1}{c|}{M} & \multicolumn{1}{c|}{\tikzcircle[fill=orange]{3pt}} & \multicolumn{1}{c|}{Never} & \multicolumn{1}{c|}{No} & \multicolumn{1}{c|}{\tikzcircle[fill=orange]{3pt}} & \multicolumn{1}{c|}{\tikzcirclenew[fill=blue]{3pt}} & \multicolumn{1}{c|}{\tikzcirclenew[fill=blue]{3pt}} & \multicolumn{1}{c|}{\tikzcircle[fill=orange]{3pt}} & \multicolumn{1}{c}{\tikzcircle[fill=orange]{3pt}} \\ \hline

\multicolumn{1}{c|}{P36} & \multicolumn{1}{c|}{M} & \multicolumn{1}{c|}{\tikzcirclenew[fill=blue]{3pt}} & \multicolumn{1}{c|}{Never} & \multicolumn{1}{c|}{No} & \multicolumn{1}{c|}{\tikzcirclenew[fill=blue]{3pt}} & \multicolumn{1}{c|}{\tikzcirclenew[fill=blue]{3pt}} & \multicolumn{1}{c|}{\tikzcirclenew[fill=blue]{3pt}} & \multicolumn{1}{c|}{\tikzcircle[fill=orange]{3pt}} & \multicolumn{1}{c|}{\tikzcirclenew[fill=blue]{3pt}} & \multicolumn{1}{c|}{P75} & \multicolumn{1}{c|}{M} & \multicolumn{1}{c|}{\tikzcirclenew[fill=blue]{3pt}} & \multicolumn{1}{c|}{Few times} & \multicolumn{1}{c|}{No} & \multicolumn{1}{c|}{\tikzcirclenew[fill=blue]{3pt}} & \multicolumn{1}{c|}{\tikzcirclenew[fill=blue]{3pt}} & \multicolumn{1}{c|}{\tikzcirclenew[fill=blue]{3pt}} & \multicolumn{1}{c|}{\tikzcircle[fill=orange]{3pt}} & \multicolumn{1}{c}{\tikzcirclenew[fill=blue]{3pt}} \\ \hline

\multicolumn{1}{c|}{P37} & \multicolumn{1}{c|}{M} & \multicolumn{1}{c|}{\tikzcirclenew[fill=blue]{3pt}} & \multicolumn{1}{c|}{Few times} & \multicolumn{1}{c|}{Some} & \multicolumn{1}{c|}{\tikzcirclenew[fill=blue]{3pt}} & \multicolumn{1}{c|}{\tikzcircle[fill=orange]{3pt}} & \multicolumn{1}{c|}{\tikzcirclenew[fill=blue]{3pt}} & \multicolumn{1}{c|}{\tikzcircle[fill=orange]{3pt}} & \multicolumn{1}{c|}{\tikzcirclenew[fill=blue]{3pt}} &

\multicolumn{10}{c}{\multirow{3}{*}{\begin{tabular}[c]{@{}c@{}}\textbf{Legend:} M: Man | W: Woman | \tikzcirclenew[fill=blue]{3pt}: Tim | \tikzcircle[fill=orange]{3pt}: Abi\\ MT: Motivation | SE: Self-efficacy | R: Risk \\ IP: Information processing | L: Learning\end{tabular}}} \\ \cline{1-10}

\multicolumn{1}{c|}{P38} & \multicolumn{1}{c|}{M} & \multicolumn{1}{c|}{\tikzcirclenew[fill=blue]{3pt}} & \multicolumn{1}{c|}{Never} & \multicolumn{1}{c|}{No} & \multicolumn{1}{c|}{\tikzcirclenew[fill=blue]{3pt}} & \multicolumn{1}{c|}{\tikzcirclenew[fill=blue]{3pt}} & \multicolumn{1}{c|}{\tikzcirclenew[fill=blue]{3pt}} & \multicolumn{1}{c|}{\tikzcircle[fill=orange]{3pt}} & \multicolumn{1}{c|}{\tikzcircle[fill=orange]{3pt}} & \multicolumn{10}{l}{} \\ \cline{1-10}

\multicolumn{1}{c|}{P39} & \multicolumn{1}{c|}{M} & \multicolumn{1}{c|}{\tikzcirclenew[fill=blue]{3pt}} & \multicolumn{1}{c|}{Few times} & \multicolumn{1}{c|}{Some} & \multicolumn{1}{c|}{\tikzcirclenew[fill=blue]{3pt}} & \multicolumn{1}{c|}{\tikzcirclenew[fill=blue]{3pt}} & \multicolumn{1}{c|}{\tikzcirclenew[fill=blue]{3pt}} & \multicolumn{1}{c|}{\tikzcircle[fill=orange]{3pt}} & \multicolumn{1}{c|}{\tikzcirclenew[fill=blue]{3pt}} & \multicolumn{10}{l}{} \\ \hline
\end{tabular}
\end{table*}



\begin{comment}


The five facets used by the GenderMag method are presented in Table~\ref{tab:gendermagfactes}. The facets are used to define personas (e.g., Abi and Tim). GenderMag highlights that differences relevant to inclusiveness lie not in a person's gender identity but in the facet values themselves~\cite{hill2017gender}. Nevertheless, Abi's facet values are more frequent in women than in other genders, and Tim's facet values are more frequent in men than in other genders. 

%(Figure~\ref{fig:abbypersona})

\begin{table}[!ht]\scriptsize
\centering
\vspace{-2.5mm}
\caption{GenderMag facets~\cite{burnett2016gendermag}}
\label{tab:gendermagfactes}
\newcommand{\pb}[1]{\parbox[t][][t]{1.0\linewidth}{#1} \vspace{-2pt}}

\begin{tabular}{p{12mm}|p{62mm}}
\hline
\multicolumn{1}{>{\centering\arraybackslash}m{12mm}|}{\textbf{GenderMag Facets}} & \multicolumn{1}{>{\centering\arraybackslash}m{62mm}}{\textbf{Definition}} \\ \hline \hline

Motivation & \pb{Women tend (statistically) to be motivated to use technology for what they can accomplish with it, whereas men are often motivated by their enjoyment of technology per se~\cite{simon2000impact, cassell2002hand, margolis2002unlocking, hou2006girls, kelleher2009barriers, burnett2010gender, burnett2011gender, hallstrom2015gender}. This difference can affect which software features users choose to use}. \\ \hline 

Information processing styles & \pb{To solve problems, people often need to process new information. Women are more likely (statistically) to process new information comprehensively—gathering fairly complete information before proceeding—but men are more likely to use selective styles—following the first promising information, then backtracking if needed~\cite{cafferata1989gender, meyers1991exploring, coursaris2008empirical, riedl2010there, meyers2015revisiting}. Each style has advantages, but either is at a disadvantage when not supported by the software.} \\ \hline

Computer self-efficacy & \pb{Self-efficacy is a person's confidence about succeeding at a specific task, which influences their use of cognitive strategies, persistence, and strategies for coping with obstacles. Empirical data have shown that women often have lower computer self-efficacy than men, which can affect their behavior with technology~\cite{margolis2002unlocking, durndell2002computer, hartzel2003self, beckwith2005effectiveness, beckwith2006tinkering, burnett2010gender, burnett2011gender, singh2013role, huffman2013using}.} \\ \hline

Risk aversion & \pb{Research shows that women statistically tend to be more risk-averse than men~\cite{weber2002domain, dohmen2011individual, charness2012strong}. These results span numerous decision-making domains, such as ethics, investment, gambling, health/safety, and career. Risk aversion with software usage can impact users' decisions as to which feature sets to use.} \\ \hline

Learning: by Process vs. by Tinkering & \pb{Research across age groups and professions reports women being statistically less likely to playfully experiment (“tinker”) with software features new to them, compared to men~\cite{beckwith2006tinkering, hou2006girls, rosner2009learning, burnett2010gender, cao2010debugging, chang2014specialization}. However, when women do tinker, they tend to be more likely to reflect during the process and thereby sometimes profit from it more than men do.} \\ \hline \hline
\end{tabular}
\end{table}

\end{comment}
&&&&

\subsection{Qualitative Analysis}
Interview sessions in Phase 1 and Phase 3 were transcribed using CLOVA Notes. %For Phase 1 interviews, two researchers iteratively conducted thematic analysis\revision{~\cite{Braun2006}}, which involved inductive coding on the data, identifying emerging themes, and grouping into higher hierarchies. We worked together closely during this phase and followed a consensus-coding approach, having consistent meetings to merge individual codes, resolve conflict, and reach agreements on the final codebook. For this reason, calculating inter-rater reliability was not deemed necessary~\cite{McDonald2019}.
\revision{
For Phase 1 interviews, thematic analysis was conducted inductively through multiple iterations~\cite{Braun2006}. First, two researchers individually performed line-by-line open coding on eight interviews, generating initial codes that closely resembled text from the transcript such as “instant ban”, “add to modmail” and “bot seems insincere”. This was followed by focused coding~\cite{Saldaa2021}, where we identified recurring themes and sorted them into broader categories such as “escalating procedure”, “integration into existing system” and “tool perception”, which formed our initial codebook. The first author then applied this codebook across the remaining interviews, refining and adding codes as new insights emerged. After this, the two researchers met again to validate the updated codebook, consolidating higher-level themes along the dimension of moderators’ practices in addressing interpersonal harm, their stances on adopting restorative justice tools through ApoloBot’s framework, and potential impacts of implementing such a system. Finally, we aligned these diverse perspectives to outline the opportunity space and challenges associated with transitioning from traditional moderation practice to integrating restorative justice tools, laying the groundwork for our results.
}

% First-level codes included short phrases similar to text from the transcript, such as “instant ban”, “add to modmail” and “bot seems insincere”. These were then sorted into broader categories such as “escalating procedure”, “integration into existing system” and “tool perception”.

Phase 3 interviews were coded by the first author following a similar inductive process \revision{based on the codebook developed in Phase 1. While Phase 1 interviews focused on moderators’ reflections on prior experiences, Phase 3 expanded upon these by grounding the insights in practical deployment outcomes. Successful use cases from Phase 3 demonstrated how expectations from Phase 1 were met, validating the key opportunities where the tool effectively fulfilled its design intent. Equally significant were the unmet expectations, where anticipated use cases were not realized, as they revealed a new-found understanding of the practical challenges and critical areas of the opportunity space where the tool's effectiveness fell short. These observations were thus incorporated into the final codebook by combining and adding to Phase 1's codes, enhancing the framework that underpins our findings.
}

% Given that some reflections in this phase provided deeper elaborations of insights from Phase 1, we combined and consolidated several categories.

\subsection{Methodological Limitations}
As highlighted by Xiao et al., \textit{"Online communities should allow for partial success or no success without enforcing the ideal outcome, especially at the early stage of implementation when there are insufficient resources or commitments"}~\cite{Xiao2023}. Restorative justice, being relatively new and context-specific, poses significant challenges when evaluated within a brief testing period. Our study is therefore constrained by the limited empirical data available on ApoloBot usage, and the analysis presented here relies mostly on interview data from Phases 1 and 3. 
\revision{
This limitation also arises from how we shape our research focus, which is not on delivering a fully-realized restorative justice tool ready for adoption, but on developing a conceptual artifact to probe its implementation and foster critical reflections among moderators. For those who engaged with the tool, their experiences provide concrete evidence of its realized potential for effective adoption. On the other hand, investigating those who did not use the tool reveals challenges and critical gaps in its suitability within the broader online landscape, which can inform future alternatives or complements that might address the limitations.
Centering the discussion on these dual perspectives allows a deeper and more comprehensive view of how diverse online communities are currently positioned for restorative justice tools, however it might compensate the technical significance of the proposed system.}

% While these interviews offered preliminary insights into the perceived potential of ApoloBot and similar restorative justice tools, they may not offer a comprehensive assessment of their broader significance.

In addition, our study primarily gathers insights from moderators rather than victims or offenders. While this focus offers rich insights into the practical aspects of the tool adoption and execution, it lacks the perspectives of the remaining stakeholders essential to restorative justice, and thus may not fully capture the complete user experience.

Finally, even though our participants come from a wide range of international communities covering diverse topics, the fact that they are solely English speakers limits the cross-cultural generalizations that can be made based on our findings.


\section{Technical Evaluation}
The user study shows that \textsc{WhatELSE} effectively helps users create engaging interactive narratives by enhancing both authorial control and player engagement through efficient narrative space editing. To validate the technical pipeline driving the transformation between narrative outline and instances, we conducted technical evaluations focusing on effectiveness in 1) generating the outline from instances and 2) generating instances from an outline.



\subsection{From Narrative Instances to Narrative Outline}

\presentation{\textsc{WhatELSE} provides an abstraction ladder with different levels of abstraction to generate an outline from instances using a prompting pipeline.} To examine the effectiveness of the pipeline, we employ two lexical-level measures, {\em concreteness rate}, and {\em imageability score}.  Both measures are adapted from large-scale crowdsourced annotations in previous studies that have been widely used in linguistic evaluations ~\cite{wilson1988mrc,brysbaert2014concreteness}. \evaluation{Both scores are lexicon-based, with each word assigned an averaged score from a batch of crowdsourced annotations; for each outline, we calculate the average score across all words in this outline, excluding stop words.} Intuitively, a higher concreteness rate indicates that the wording is more concrete and specific, corresponding to a lower level of abstraction. A lower imageability score suggests that the wording allows greater room for interpretation, corresponding to a higher level of abstraction. 

\presentation{We generated outlines for a sample of 100 stories from the Fairytale dataset~\cite{xu2022fantastic} at three different levels of abstraction (scene, sequence, and act level).} We reported the concreteness rate and imageability score of the generated outlines in Table ~\ref{tech_eval:abstraction_ladder}. The results show that our method effectively produces outlines at three distinct levels of abstraction, with significant differences in the lexical measures between each pair of abstraction levels. This demonstrates the system's ability to define narrative spaces with varying degrees of constraint.

%This demonstrates the system’s ability to facilitate creators' control over varying degrees of narrative variation within the outlined narrative space.


% our system helps users generate outlines based on a linear story. Practically, we designed the prompt template to help users generate outlines with three different levels of abstractions on a linear story. We evaluated the capability of our designed prompting on the public Fairytale dataset, from which we sampled 100 stories and used our prompts to generate outlines of the stories at the three levels of abstraction that we proposed (i.e., scene level, sequence level, act level). We use two lexical measures, concreteness rate, and imaginability score to evaluate the levels of abstraction of generated outlines. Both measures are from large-scale crowdsourced annotations collected previously on English writings. Intuitively, higher concreteness rate means the wording used in the outline makes it more concrete and specific, thus less abstraction. Meanwhile, higher imaginability score indicates the wording of the text leaves larger room for readers to imagine, subsequently a higher abstraction level. The measured levels of abstraction of generated outlines using our prompts are reported in Tab ~\ref{tech_eval:abstraction_ladder}. We can see that our prompts are capable of providing outlines at three different levels of abstraction, with momentum measures significantly differing between each pair of two levels of abstraction.



\begin{table}[h]

\small
% \resizebox{\linewidth}{!}{
\begin{tabular}{c|c|c|c}
% \Large
\toprule




%\thead{\textbf{Experiment}/ \\ Independent Var} 
\multirow{2}{*}{\textbf{Measurement}}
& \multicolumn{3}{c}{\textbf{Abstraction Level }} 
% {\thead{\textbf{ Exp 1:} \\ $y = $ Final Accuracy}} &
% {\thead{\textbf{ Exp 2:} \\ $y = $ Peer Accuracy}} &  {\thead{\textbf{ Exp 3:} \\ $y = $ Final Accuracy}}

%&  \textbf{Prob. of Superiority}
\\
\cline{2-4}
 & \textbf{Scene Level}  & \textbf{Sequence Level} & \textbf{Act Level}
\\
\midrule

Concreteness Rate & $3.56\pm 0.05$ & $3.09\pm0.06$  & $2.95\pm 0.06$ 
\\

% Peer Accuracy ($\beta_1$) & & &  $1.88^{***}$
% \\
% array([497.32914428, 453.64683276, 438.91406737])
% np.array(img_mat).std(axis=0)
% array([22.82068827, 32.41856082, 34.1765820


Imageability Score & $497.33\pm{22.82}$ & $453.64\pm{32.42}$ & $438.91\pm{34.17}$
\\

\bottomrule
\end{tabular}
% }
\vspace{2pt}
\caption{Measured abstraction level of the generated outline plot using the proposed prompt pipeline employed in Abstraction Ladder, to generate outline with distinct levels of abstraction from narrative examples. A lower concreteness rate and imageability score indicate the text is more abstract at the lexical level.}
\label{tech_eval:abstraction_ladder}
% \vspace{-25pt}
\end{table}


% \begin{table}[h]
% \begin{tabular}{|c|c|c|c|}
% \hline
% Abstraction Level   & Secne Level & Sequence Level & Act Level \\ \hline
% Concreteness Rate   &  $3.56\pm 0.05$           &  $3.09 \pm 0.06$               &    $2.95 \pm 0.06$        \\ \hline
% Imaginability Score &             &                &           \\ \hline
% \end{tabular}
% \caption{Evaluation results of abstraction ladder}
% \label{tech_eval:abstraction_ladder}

% \end{table}

%\noindent \textbf{Narrative Space Feedback}

\begin{table*}[ht]

% \small
% \resizebox{\linewidth}{!}{
\begin{tabular}{c|c|c|c|c}
% \Large
\toprule




%\thead{\textbf{Experiment}/ \\ Independent Var} 
\multirow{2}{*}{\textbf{Measurement}}
& \multicolumn{2}{c|}{\textbf{Human-generated Outline}} & \multicolumn{2}{c}{\textbf{LLM-generated Outline}} 
% {\thead{\textbf{ Exp 1:} \\ $y = $ Final Accuracy}} &
% {\thead{\textbf{ Exp 2:} \\ $y = $ Peer Accuracy}} &  {\thead{\textbf{ Exp 3:} \\ $y = $ Final Accuracy}}

%&  \textbf{Prob. of Superiority}
\\
\cline{2-5}
 & \textbf{$d_1$}  & \textbf{$d_{macro}$} & \textbf{$d_1$}  & \textbf{$d_{macro}$}
\\
\midrule

% Peer Accuracy ($\beta_1$) & & &  $1.88^{***}$
% \\

\presentation{Proposed IN Compiler} & {$0.65\pm 0.01$}             & $0.78\pm 0.01$  & $0.64\pm 0.01$ & $0.77\pm 0.01$
\\

\presentation{Baseline Prompt-based IN Compiler} & $0.51\pm 0.02$             & $0.63\pm 0.02$  & $0.61\pm 0.02$ & $0.73\pm 0.02$
\\

\bottomrule
\end{tabular}
% }
% \vspace{2pt}
\caption{Measured plot distance by averaged ROUGE-1 distance $d_{1}$ and the macro-averaged ROUGE distance $d_{macro}$ among the plots generated by baseline and our approach. Results show that \textsc{WhatELSE} IN Compiler leads to a larger averaged distance among the plots, and thus a greater diversity of plots within the narrative space.}
\label{tech_eval:plot_diversity}
% \vspace{-25pt}
\end{table*}

\begin{table*}[h]

% \small
% \resizebox{\linewidth}{!}{
\begin{tabular}{c|c|c|c|c}
% \Large
\toprule




%\thead{\textbf{Experiment}/ \\ Independent Var} 
\multirow{2}{*}{\textbf{Method}}
& \multicolumn{4}{c}{\textbf{Measurement}} 

\\
\cline{2-5}
 & \textbf{$d_1$}  & \textbf{$d_{macro}$} & {World-state Change (Neg)}  & {Character Involvement (Pos)}
\\
\midrule

% Peer Accuracy ($\beta_1$) & & &  $1.88^{***}$
% \\

\presentation{Proposed IN Compiler} & {$0.59\pm 0.01$}             & $0.74\pm 0.001$  & $1.00\pm 0.0$ & $1.70\pm 0.37$
\\

\presentation{Baseline Prompt-based IN Compiler} & $0.53\pm 0.01$             & $0.67\pm 0.01$  & $0.85\pm 0.08$ & $1.65\pm 0.41$ 
\\

\bottomrule
\end{tabular}
% }
% \vspace{2pt}
\caption{Measured impact of player action on game plot progression, by (1) averaged pairwise ROUGE-1 distance $d_{1}$ and the macro-averaged ROUGE distance $d_{macro}$ between pairs of the game plots driven by contrasting player actions, and (2) averaged world state change rate driven by the negative player action and the averaged character involvement driven by the positive action. Results show that the contrasting player actions make the proposed approach generate plots with larger pairwise distances. Additionally, in \textsc{WhatELSE} playtime, player actions lead to more stable world-state change and better character involvement than the baseline. }
\label{tech_eval:action_impact}
% \vspace{-25pt}
\end{table*}

\subsection{From Narrative Outline to Narrative Instances}
\presentation{We compared the plot quality generated using our approach with the baseline prompting-based approach (Figure~\ref{baseline})} from two aspects: {\em plot diversity} and {\em player impact}. 


\noindent \textbf{Plot Diversity}\hspace{2mm}  Plot diversity refers to the ability to generate a wide range of different plots within the narrative space. It indicates the level of interactivity and player agency supported by the plot generation method, as it showcases the system's ability to offer varied storylines within the narrative space described by the outline.

To quantitatively assess diversity among a set of $N$ plots generated within the narrative space, we calculate the averaged distance between each plot and the other $N - 1$ plots in the set. We then compute the average distance for each plot relative to the others. For distance calculation, we use the ROUGE score ~\cite{lin2004rouge}, a reference-based evaluation metric that measures text similarity. Adapting from the ROUGE-1 score $r_{1}$ targeting word level and macro averaged ROUGE score $r_{macro}$ measures similarity across multiple levels of wording, we compute the word-level distance $d_{1} = 1 - r_{1}$ and the macro distance $d_{macro} = (1 - r_{macro})$ between plots, respectively. As the ROUGE score indicates similarity between text, a higher averaged distance indicates the greater diversity of plots.

For comparison, we first collected two sets of outlines based on the \presentation{\textit{``Fairytale Forest''} story domain (Figure~\ref{baseline})}. The first set, consisting of 12 outlines, was generated by participants in the user study, each tied to one of two specific morals. Additionally, we developed a set of 50 outlines by prompting an LLM, focusing on various morals within the story domain. We simplify each outline by taking only the first act, resulting in 12 human-generated and 100 LLM-generated single-act outlines. These outlines were then used to guide plot generation without involving player actions, using the proposed narrative planning based approach and the baseline prompt-based approach.

We generated a set of 20 plots with each outline and then calculated the averaged distances $d_{1}$ and $d_{macro}$ among each set of plots. As shown in Table~ \ref{tech_eval:plot_diversity}, our approach generated more diverse plots within the same narrative space, indicating more variety of storylines and stronger player agency.








% Please add the following required packages to your document preamble:
% \usepackage{multirow}
% Please add the following required packages to your document preamble:
% \usepackage{multirow}
% \begin{table}[t]
% \begin{tabular}{|c|cc|cc|}
% \hline
% \multirow{2}{*}{Outline Content} & \multicolumn{2}{c|}{Human-generated Outline}                    & \multicolumn{2}{c|}{LLM-generated Outline}                      \\ \cline{2-5} 
% & \multicolumn{1}{l|}{Rouge 1 Distance} & Averaged Rouge Distance & \multicolumn{1}{l|}{Rouge 1 Distance} & Averaged Rouge Distance \\ \hline
% Prompting                        & \multicolumn{1}{c|}{$0.49\pm 0.02$}             & $0.37\pm 0.02$                    & \multicolumn{1}{c|}{xx}             & 0.34                    \\ \hline
% Gameplot Compiler (Ours)                & \multicolumn{1}{c|}{$0.35\pm 0.01$}             & $0.22\pm 0.007$                    & \multicolumn{1}{c|}{xx}             & 0.44                    \\ \hline
% \end{tabular}
% \caption{Player Impact}
% \label{tech_eval:plot_diversity}
% \end{table}



\noindent \textbf{Player Impact} \presentation{We use the term player impact to refer to the extent to which players' actions meaningfully influence the progression of the plot.} A higher player impact indicates that the narrative is more responsive to player actions, leading to different outcomes and providing a more personalized experience. 

Given a sequence of events ($S$), the following metrics measure the difference between the subsequent events ($S'$) in response to players taking different actions after the leading sequence $S$. 

\begin{itemize}
\item {\bf Subsequent Plot Divergence}\hspace{2mm} We execute a pair of contrasting player actions after $S$, and compare how the following plot progression diverges semantically based on the player's different actions. The contrasting actions are attacking/killing a character (negative action) and seeking help for a character in danger (positive action). This comparison is performed by calculating two types of ROUGE distances. Specifically, the distance is computed pairwisely between the two plots generated after the positive and negative actions. A higher ROUGE distance indicates a greater divergence between the two plots driven by contrasting player actions, thus reflecting their higher impact on the plot progression. 
\item {\bf Perceived World State Change}\hspace{2mm} The metric assesses the perceived alternation of world states caused by the player's actionn in a specific scenario. We execute a player action of killing a character, and count the frequency of the killed character's reappearance in the subsequent plots.
\item {\bf Player Character Involvement}\hspace{2mm} The metric examines whether the player's action increases the player character's involvement in the subsequent events in the plot. We calculate the frequency with which the player's character appears in the plots that follow the positive action of helping a character. This indicates the extent to which the player's actions influence their engagement in the narrative.
\end{itemize}


%Additionally, we take specific measures on the perceived world states change to evaluate perceived impacts of player actions. 
%on the game world and the player's involvement in the plot. 

%For instance, the action of killing a character is intended to prevent that character from appearing in subsequent plots, which may not be the case due to hallucination.

%However, the baseline approach could suffer from hallucination, leading to the reappearance of a supposedly killed character, 
%thereby ignoring the player's impact on the world state. We  


%Practically, we again take the story domain "Fairytale Forest" and a fixed story outline containing two acts for evaluation. 
 %Next, we input the designed positive or negative player actions and use both the prompt-based approach and our approach to generate a batch of 20 game plots following the player action. This process creates 20 pairs of game plots after contrasting player actions generated by each approach. We then calculate the average pairwise distance between plots using two types of ROUGE distance. Additionally, we compute the average change in the world state and player involvement for plots generated after the player's positive and negative actions, respectively. 
We use the \presentation{\textit{``Fairytale Forest''} story domain (Figure~\ref{baseline})} and a fixed story outline containing two acts for evaluation. The player character is set as the dove. To initialize, we generate the plot for the first act in the outline as $S$ and set the world state accordingly. We then execute the designed player actions, and then use our method and the baseline method to generate a batch of 20 plots following each player action to compute the above metrics.

As Table~\ref{tech_eval:action_impact} shows, our approach generates significantly more diverse plots following the player's contrasting actions. Moreover, our approach generates plot with better perceived world state change and character involvement following player's actions. Overall, \presentation{we found that \textsc{WhatELSE} integrates player actions with a higher impact in the narrative generation process.}


% Therefore, 

% capability of an approach to generate a wide range of different game plots based on the same outline. This reflects the core capability of a game plot generation method to offer diverse narrative paths, thereby determining the level of interactivity and freedom available to players within the game.



% \begin{table}[]
% \begin{tabular}{|c|c|c|c|c|}
% \hline
%                   & Pairwise $d_{1}$ & Pairwise $d_{macro}$ & World-state Change (Neg) & Character Involvement (Pos) \\ \hline
% Prompting         & $0.47\pm 0.01$                       & $0.33\pm 0.01$                             & $0.85\pm 0.08$                & $1.65\pm 0.41$                   \\ \hline
% Gameplot Compiler (Ours) & $0.41\pm 0.01$                       & $0.26\pm 0.01$                              & $1.0\pm 0.0$               & $1.7\pm 0.37$                   \\ \hline
% \end{tabular}
% \caption{Impact of Action}
% \label{tech_eval:action_impact}
% \end{table}





% \section{Related work}

The literature related to our work can be classified into two
categories: general purpose DR techniques
(\autoref{sec:relatedWorkGeneralPurpose}) and topology-aware techniques
(\autoref{sec:relatedWorkTopology}).

\subsection{General purpose dimensionality reduction}
\label{sec:relatedWorkGeneralPurpose}

Numerous DR techniques have been proposed and the related literature has been
summarized in several books~\cite{borg97, dimensionReductionBook} and surveys
\cite{surveyDimensionReduction2, surveyDimensionReduction1, NonatoA19}.
Principal Component Analysis (PCA)~\cite{pearson1901liii} is by far the most
popular linear DR technique.
Although it is an indispensable tool for data analysis,
its linear nature does not always allow it to apprehend complex non-linear
phenomena. One of the first non linear DR methods is the multidimensional
scaling (MDS)~\cite{torgerson1952multidimensional}. It aims at preserving as far
as possible the pairwise distances in the high- and low-dimensional point
clouds.
Another approach, particularly related to our work,
consists in optimizing an autoencoder neural network~\cite{hinton_reducing_2006}.
The \textit{encoder} is used to represent the explicit projection map from the
high-dimensional input space to the low-dimensional representation
space, while the \textit{decoder} tries to reconstruct the input data
from its encoded representation.
We will refer to these methods as \emph{global} methods.

Global methods have been used successfully in many applications, but
they do not take into
account the possible distribution of the input points over some implicit,
unknown manifold. This may lead to the unwanted preservation of distances
between points that are close in the ambient space but far apart on this
manifold. \emph{Locally topology-aware} methods have therefore been
introduced to address this issue. For instance,
Isomap~\cite{tenenbaum_global_2000}
preserves geodesic distances on a captured manifold structure of the
input data.
%\remove{Because it suffers from computational
%inefficiencies, Isomap was sped up with the use of landmark points (L-Isomap
%\cite{silva2003global}).}
Other methods also feature neighborhood preservation objectives.
For example, Local Linear Embedding (LLE)~\cite{roweis2000nonlinear} relies
on linear reconstructions of local neighborhoods.
Other methods leverage additional landmarks~\cite{silva2003global} or user-provided
control points~\cite{joia:tvcg:2011}.
%Some local methods additionally support user
%constraints expressed as control points~\cite{joia:tvcg:2011}.

All these methods try to preserve local
Euclidean distances when projecting to a lower dimension.
However, this can sometimes lack relevance in the applications,
especially with high-dimensional datasets for which
the distribution of pairwise Euclidean distances tend to be uniform.
For such cases, local distance preservation fails at characterizing
well relevant local relations.
To alleviate this issue, SNE~\cite{hinton2002stochastic} and later
t\nobreakdash-SNE~\cite{van2008visualizing} use a conditional probability
formulation to represent similarities between points and try to
have similar distributions both in high- and low-dimension thanks to a
Kullback--Leibler divergence minimization.
More recently UMAP has been introduced~\cite{mcinnes2018umap} along a
theoretical foundation on category theory.
It provides results that are similar visually to t-SNE, but in a more
scalable way.
Variants were later introduced to better preserve the global structure in the embedding, such as TriMAP~\cite{amid2022trimap} that constrains the proximity order within triplets of points, or PaCMAP~\cite{wang_understanding_2021} that adds constraints on more distant point pairs.
Although these methods succeed in preserving the local topology, they are not
explicitly aware of the global structure
of the input, which can lead to the loss of noteworthy global,
topological features.

\subsection{Globally topology-aware dimensionality reduction}
\label{sec:relatedWorkTopology}

Topology-based methods have become popular over the last
two decades in data analysis and
visualization~\cite{heine16} and have been applied to various areas:
astrophysics~\cite{sousbie11, shivashankar2016felix},
biological imaging~\cite{beiBrain18, carr04, topoAngler},
quantum chemistry~\cite{chemistry_vis14,harshChemistry, D2CP05893F},
fluid dynamics~\cite{kasten_tvcg11, NauleauVBBT22},
material sciences~\cite{gyulassy_vis07, gyulassy_vis15, SolerPDPCT19},
turbulent combustion~\cite{gyulassy_ev14, laney_vis06}. They leverage tools that
define concise signatures of the data based on its topological properties and
that have been summarized in topological data analysis reference
books ~\cite{edelsbrunner_computational_2010, zomorodian_computational_2010}
and surveys~\cite{chazal_introduction_2021}.

Several DR methods have been proposed
by the visualization community to preserve specific topological signatures
of the input data. For instance, terrain metaphors have been
investigated for the visualization of an input high-dimensional scalar
field, in the form of a three-dimensional terrain, whose elevation yields an
identical contour tree~\cite{Weber:2007} or an identical set of separatrices
\cite{gerber2010, gerber2013}.
This framework has been extended to density
estimators~\cite{OesterlingHJS10,
OesterlingSTHKEW10, OHJSH11, Oesterling0WS13} for the support of
high-dimensional point clouds. However, such metaphors completely discard
the metric information of the input space~\cite{OesterlingHJS10}, possibly
placing next to each other points which are arbitrarily far
in the input space (and reciprocally). Yan et al.~\cite{abs-1806-08460}
introduced a DR approach driven by the Mapper structure~\cite{SinghMC07}, an
approximation of the Reeb graph~\cite{reeb46}, which can capture in practice
large handles in the data, however without guarantees, since the number of handles in the considered manifold is only an upper bound on the number of loops in the Reeb graph~\cite{edelsbrunner_computational_2010}.

To incorporate the metric information from the input data while
preserving at the same time some of their topological characteristics, several
approaches have focused on driving the projection by
the \emph{persistence diagram}
of the Rips filtration of the point cloud (see \autoref{sec:persistentHomology}
for  a technical description).
Carriere et al.~\cite{carriere2021optimizing} presented a generic persistence
optimization framework with an application to dimensionality reduction.
Their approach explicitly minimizes the Wasserstein distance
(\autoref{sec:persistentHomology}) between the $1$-dimensional persistence
diagrams in high and low dimensions. However, this approach solely focuses on
this penalization term. As a result,
although the number and persistence of cycles  may be well-preserved,
the solver tends to produce cycles in low dimensions which involve arbitrary
points (e.g., which were not necessarily located along the cycles in high
dimensions), which challenges visual interpretation, as later
detailed in \autoref{sec:results:analysis}.

To enforce a correspondence between the topological
features at the data point level, additional structures need to be preserved.
For the specific case of $0$-dimensional persistent homology (\PH{0}),
Doraiswamy et al. introduced \emph{TopoMap}~\cite{doraiswamy2020topomap}, an
algorithm which constructively preserves the \emph{persistence pairs}
(\autoref{sec:persistentHomology}) through the preservation of the minimum
spanning tree of the data. An accelerated version, with improved layouts, has
recently been proposed~\cite{guardieiro2024topomap++}.
Alternative approaches have considered the usage of an optimization framework
(typically based on an autoencoder neural network
\cite{hinton_reducing_2006}), with the integration of specific topology-aware
losses~\cite{moor2020topological,barannikov2021representation,
nelson2022topology,trofimov2023learning,schonenberger2020witness}. Among them,
a prominent approach is the \emph{Topological Autoencoders}
(TopoAE)~\cite{moor2020topological}. Its loss aims at preserving
the diameter of the simplices involved in
persistence pairs, when going from high to low dimensions and reciprocally.
However, the above techniques focused in practice on the
preservation of \PH{0} and did not, to our knowledge, report experiments
regarding the preservation of higher dimensional PH.
Specifically, we show in \autoref{sec:analysis} that, while a zero
TopoAE loss indeed implies a preservation of the persistence pairs for \PH{0},
it is not the case for higher dimensional PH. We provide a counter example for
\PH{1}, which is addressed by our novel, generalized loss.


\section{Discussion}
This work identifies signal collapse as a critical bottleneck in one-shot neural network pruning. Performance loss in pruned networks is due to \textbf{signal collapse} in addition to the removal of critical parameters. We propose \textbf{REFLOW} (\textbf{Re}storing \textbf{F}low of \textbf{Low}-variance signals), a simple yet effective method that mitigates signal collapse without computationally expensive weight updates. By focusing on signal preservation, REFLOW highlights the importance of mitigating signal collapse in sparse networks and enables magnitude pruning to match or surpass state-of-the-art one-shot pruning methods such as CHITA, CBS, and WF.

REFLOW consistently achieves state-of-the-art accuracy across diverse architectures, restoring ResNeXt-101 from under 4.1\% to 78.9\% top-1 accuracy at 80\% sparsity on ImageNet. Its lightweight design makes it a practical solution for both research and deployment, delivering high-quality sparse models without the overhead of traditional approaches. These findings challenge the traditional emphasis on weight selection strategies and underscore the critical role of signal propagation for achieving high-quality sparse networks in the context of one-shot pruning.





\begin{acks}
We are grateful to the anonymous reviewers who provided many helpful comments. We also thank our colleagues, David Ledo, Fraser Anderson and Hilmar Koch for feedback and suggestions in preparing the manuscripts and figures. Any opinions, findings, conclusions,
or recommendations expressed here are those of the authors alone.
\end{acks}


%%
%% The next two lines define the bibliography style to be used, and
%% the bibliography file.
\bibliographystyle{ACM-Reference-Format}
\bibliography{llm_narrative}

\end{document}
\endinput
%%
%% End of file `sample-sigconf.tex'.
