%%
%% This is file `sample-sigconf.tex',
%% generated with the docstrip utility.
%%
%% The original source files were:
%%
%% samples.dtx  (with options: `all,proceedings,bibtex,sigconf')
%% 
%% IMPORTANT NOTICE:
%% 
%% For the copyright see the source file.
%% 
%% Any modified versions of this file must be renamed
%% with new filenames distinct from sample-sigconf.tex.
%% 
%% For distribution of the original source see the terms
%% for copying and modification in the file samples.dtx.
%% 
%% This generated file may be distributed as long as the
%% original source files, as listed above, are part of the
%% same distribution. (The sources need not necessarily be
%% in the same archive or directory.)
%%
%%
%% Commands for TeXCount
%TC:macro \cite [option:text,text]
%TC:macro \citep [option:text,text]
%TC:macro \citet [option:text,text]
%TC:envir table 0 1
%TC:envir table* 0 1
%TC:envir tabular [ignore] word
%TC:envir displaymath 0 word
%TC:envir math 0 word
%TC:envir comment 0 0
%%
%% The first command in your LaTeX source must be the \documentclass
%% command.
%%
%% For submission and review of your manuscript please change the
%% command to \documentclass[manuscript, screen, review]{acmart}.
%%
%% When submitting camera ready or to TAPS, please change the command
%% to \documentclass[sigconf]{acmart} or whichever template is required
%% for your publication.
%%
%%



\documentclass[sigconf]{acmart}
%%
%% \BibTeX command to typeset BibTeX logo in the docs
\AtBeginDocument{%
  \providecommand\BibTeX{{%
    Bib\TeX}}}

%% Rights management information.  This information is sent to you
%% when you complete the rights form.  These commands have SAMPLE
%% values in them; it is your responsibility as an author to replace
%% the commands and values with those provided to you when you
%% complete the rights form.

\usepackage{dirtytalk}
\usepackage{multirow}
\usepackage{xcolor}

\usepackage{subcaption}
\usepackage{enumitem}
\usepackage{color}
%\usepackage{hyperref}
\usepackage{algorithm}
% \usepackage{tcolorbox}



% \usepackage{multirow}
% \usepackage{tabularray}

% \usepackage{xcolor}


% \usepackage{amsmath}
% \usepackage{algorithmic}
% \usepackage{algpseudocode}




\definecolor{burgundy}{rgb}{0.5, 0.0, 0.13}
\definecolor{yellow}{rgb}{0.85, 0.65, 0.13}
\definecolor{green}{rgb}{0.0, 0.5, 0.0} 

\definecolor{darkbrown}{rgb}{0.55, 0.27, 0.07}


\newcommand{\revision}[1]{{#1}}
\newcommand{\baseline}[1]{{#1}}
\newcommand{\originality}[1]{{#1}}
\newcommand{\evaluation}[1]{{#1}}
\newcommand{\presentation}[1]{{#1}}
\newcommand{\usecase}[1]{{#1}}


\newcommand{\comments}{1}
\newcommand{\ignore}[1]{}

\ifdefined\comments
   \newcommand{\yw}[1]{\textcolor{red}{[Yi: #1]}}
    \newcommand{\qz}[1]{\textcolor{orange}{[Qian: #1]}}
    \newcommand{\zl}[1]{\textcolor{blue}{[Zhuoran: #1]}}
    \newcommand{\whosays}[1]{\begin{center} \say{\textit{#1}} \end{center}}
     \newcommand{\cred}[1]{\textcolor{red}{[#1]}}
     
\else
\fi

\definecolor{highlight}{RGB}{230, 230, 255} % Light blue color

% \newcommand{\boundedpatch}[2][]{%
%     \begin{tcolorbox}[colframe=blue!50!black, colback=blue!5!white, title=#1]
%     #2
%     \end{tcolorbox}
% }




\copyrightyear{2025}
\acmYear{2025}
\setcopyright{cc}
\setcctype{by}
\acmConference[CHI '25]{CHI Conference on Human Factors in Computing
Systems}{April 26-May 1, 2025}{Yokohama, Japan}
\acmBooktitle{CHI Conference on Human Factors in Computing Systems (CHI
'25), April 26-May 1, 2025, Yokohama,
Japan}\acmDOI{10.1145/3706598.3713363}
\acmISBN{979-8-4007-1394-1/25/04}

% \setcopyright{acmlicensed}
% \copyrightyear{2018}
% \acmYear{2018}
% \acmDOI{XXXXXXX.XXXXXXX}
% %% These commands are for a PROCEEDINGS abstract or paper.
% \acmConference[Conference acronym 'XX]{Make sure to enter the correct
%   conference title from your rights confirmation email}{June 03--05,
%   2018}{Woodstock, NY}
% %%
% %%  Uncomment \acmBooktitle if the title of the proceedings is different
% %%  from ``Proceedings of ...''!
% %%
% %%\acmBooktitle{Woodstock '18: ACM Symposium on Neural Gaze Detection,
% %%  June 03--05, 2018, Woodstock, NY}
% \acmISBN{978-1-4503-XXXX-X/2018/06}


%%
%% Submission ID.
%% Use this when submitting an article to a sponsored event. You'll
%% receive a unique submission ID from the organizers
%% of the event, and this ID should be used as the parameter to this command.
%%\acmSubmissionID{123-A56-BU3}

%%
%% For managing citations, it is recommended to use bibliography
%% files in BibTeX format.
%%
%% You can then either use BibTeX with the ACM-Reference-Format style,
%% or BibLaTeX with the acmnumeric or acmauthoryear sytles, that include
%% support for advanced citation of software artefact from the
%% biblatex-software package, also separately available on CTAN.
%%
%% Look at the sample-*-biblatex.tex files for templates showcasing
%% the biblatex styles.
%%

%%
%% The majority of ACM publications use numbered citations and
%% references.  The command \citestyle{authoryear} switches to the
%% "author year" style.
%%
%% If you are preparing content for an event
%% sponsored by ACM SIGGRAPH, you must use the "author year" style of
%% citations and references.
%% Uncommenting
%% the next command will enable that style.
%%\citestyle{acmauthoryear}


%%
%% end of the preamble, start of the body of the document source.
\begin{document}

%%
%% The "title" command has an optional parameter,
%% allowing the author to define a "short title" to be used in page headers.
\title{WhatELSE: Shaping Narrative Spaces at Configurable Level of Abstraction for AI-bridged Interactive Storytelling
}

%%
%% The "author" command and its associated commands are used to define
%% the authors and their affiliations.
%% Of note is the shared affiliation of the first two authors, and the
%% "authornote" and "authornotemark" commands
%% used to denote shared contribution to the research.
\author{Zhuoran Lu}
\affiliation{%
  \institution{Department of Computer Science, Purdue University}
%   \streetaddress{1 Th{\o}rv{\"a}ld Circle}
  \city{West Lafayette}
  \country{USA}}
\email{lu800@purdue.edu}

\author{Qian Zhou}
\affiliation{%
  \institution{Autodesk Research}
  \city{Toronto}
  \country{Canada}}
\email{qian.zhou@autodesk.com}

\author{Yi Wang}
\affiliation{%
  \institution{Autodesk Research}
  \city{San Francisco}
  \country{USA}}
\email{yi.wang@autodesk.com}

% \author{Ben Trovato}
% \authornote{Both authors contributed equally to this research.}
% \email{trovato@corporation.com}
% \orcid{1234-5678-9012}
% \author{G.K.M. Tobin}
% \authornotemark[1]
% \email{webmaster@marysville-ohio.com}
% \affiliation{%
%   \institution{Institute for Clarity in Documentation}
%   \city{Dublin}
%   \state{Ohio}
%   \country{USA}
% }

% \author{Lars Th{\o}rv{\"a}ld}
% \affiliation{%
%   \institution{The Th{\o}rv{\"a}ld Group}
%   \city{Hekla}
%   \country{Iceland}}
% \email{larst@affiliation.org}

% \author{Valerie B\'eranger}
% \affiliation{%
%   \institution{Inria Paris-Rocquencourt}
%   \city{Rocquencourt}
%   \country{France}
% }

% \author{Aparna Patel}
% \affiliation{%
%  \institution{Rajiv Gandhi University}
%  \city{Doimukh}
%  \state{Arunachal Pradesh}
%  \country{India}}

% \author{Huifen Chan}
% \affiliation{%
%   \institution{Tsinghua University}
%   \city{Haidian Qu}
%   \state{Beijing Shi}
%   \country{China}}

% \author{Charles Palmer}
% \affiliation{%
%   \institution{Palmer Research Laboratories}
%   \city{San Antonio}
%   \state{Texas}
%   \country{USA}}
% \email{cpalmer@prl.com}

% \author{John Smith}
% \affiliation{%
%   \institution{The Th{\o}rv{\"a}ld Group}
%   \city{Hekla}
%   \country{Iceland}}
% \email{jsmith@affiliation.org}

% \author{Julius P. Kumquat}
% \affiliation{%
%   \institution{The Kumquat Consortium}
%   \city{New York}
%   \country{USA}}
% \email{jpkumquat@consortium.net}

%%
%% By default, the full list of authors will be used in the page
%% headers. Often, this list is too long, and will overlap
%% other information printed in the page headers. This command allows
%% the author to define a more concise list
%% of authors' names for this purpose.
% \renewcommand{\shortauthors}{Trovato et al.}

%%
%% The abstract is a short summary of the work to be presented in the
%% article.
\begin{abstract}
%% YWv1
%Generative AI has the potential to revolutionize interactive storytelling by concretizing the creator's narrative intent according to the audience's actions in the story world. However, this requires the creator to relinquish some authorial control on the details of the narrative. Instead of crafting a specific narrative, the creator needs to author a possibility space of narratives, while the final narrative seen by the audience emerges from their interaction with AI within the space. In contrast to most existing tools assisting interactive storytelling creation, which focus on authoring specific narratives, we present WhatELSE, a proof-of-concept workflow for authoring possibility spaces of narratives. WhatELSE assists creators to perceive and sculpt a narrative possibility space utilizing coordination between three different representations of the space: narrative instance, narrative outline and narrative variants. Technical evaluations and user study show that WhatELSE helps the creators effectively shape the narrative possibility space and balance between authorial intent expression and emergent narrative content. We also demonstrate how the output from WhatELSE can be used to drive actual gameplay by grounding narrative content to function calls supported by a game environment.

%% Qv1
%Interactive narratives let players make choices that impact the story that they are reading. Creating interactive narratives requires authors to craft a narrative space by enumerating player choices and possible outcomes. Large language models have the potential to generate rich interactive textual output but they lack the control that authors desire, and it is challenging for authors to perceive and edit the generated narrative space. In this paper, we present WhatELSE, an authoring system that generates an interactive narrative using a linear story as an example. It encodes the authorial intent of the example into a narrative space by abstracting meaningful events, and unfolds the narrative space into executable gameplots  using LLM-based narrative planning. The system provides three views (narrative outline, variant, and example) and two abstraction tools (abstraction ladder and tooltip) to help authors perceive and edit the space . Through a user study and technical evaluation, we found that WhatELSE supports creation of interactive narrative through authorial controls and prevents deviations beyond the original authorial intent.

%% YWv3
% Generative AI enables just-in-time narrative content generation in games, which significantly enhances player agency by allowing the generated narrative to respond to the player's actions in the game. However, this reduces author's control over the space of possible narratives - within which the final story experienced by the player emerges from their interaction with AI. It is challenging for authors to understand and control what could be generated from the space. In this paper, we present WhatELSE, an  interactive narrative authoring system that creates narrative possibility spaces from examples. WhatELSE leverages linguistic abstraction to control the boundary of the narrative space, and LLM-based simulation to generate estimated variations in the space as well as corresponding executable character actions. WhatELSE helps creators perceive and edit this space through three representations of the narrative space: plot instance, narrative outline, and plot variants. We conducted a user study and technical evaluation. Technical evaluations and a user study (N = 12) show that WhatELSE enables game plot generation to reach balanced authorial expression and player agency.

% ZL v3.3


Generative AI significantly enhances player agency in interactive narratives (IN) by enabling just-in-time content generation that adapts to player actions. While delegating generation to AI makes IN more interactive, it becomes challenging for authors to control the space of possible narratives - within which the final story experienced by the player emerges from their interaction with AI. In this paper, we present WhatELSE, an AI-bridged IN authoring system that creates narrative possibility spaces from example stories. WhatELSE provides three views (narrative pivot, outline, and variants) to help authors understand the narrative space and corresponding tools leveraging linguistic abstraction to control the boundaries of the narrative space. Taking innovative LLM-based narrative planning approaches, WhatELSE further unfolds the narrative space into executable game events. Through a user study (N=12) and technical evaluations, we found that WhatELSE enables authors to perceive and edit the narrative space and generates engaging interactive narratives at play-time.


% Generative AI enables just-in-time narrative content generation in games, thus significantly enhancing player agency by allowing the generated narrative to respond to the player's action in the game. However, this reduces author's control over the space of possible narratives - within which the final story experienced by the player emerges from their interaction with AI. It is challenging for authors to understand and control what could be generated from the space. In this paper, we present WhatELSE, an interactive narrative authoring system that creates narrative possibility spaces from a story example. WhatELSE leverages linguistic abstraction to control the boundary of the narrative space, and LLM-based simulation to generate estimated variations in the space. It encodes the authorial intent of the example into a narrative space by abstracting pivotal events and unfolds this space into executable game plots using LLM-based narrative planning. WhatELSE helps authors perceive the space through three representations of the narrative space (narrative instance, outline, and variants), and use two abstraction tools (abstraction ladder and tooltip) to edit the space.  Through a user study (N=12) and technical evaluation, we found that WhatELSE enables game plot generation to reach balanced authorial expression and player agency.


\end{abstract}





%%
%% The code below is generated by the tool at http://dl.acm.org/ccs.cfm.
%% Please copy and paste the code instead of the example below.
%%

\begin{CCSXML}
<ccs2012>
<concept>
<concept_id>10010147.10010178.10010179</concept_id>
<concept_desc>Computing methodologies~Natural language processing</concept_desc>
<concept_significance>500</concept_significance>
</concept>
<concept>
<concept_id>10010405.10010476.10011187.10011190</concept_id>
<concept_desc>Applied computing~Computer games</concept_desc>
<concept_significance>500</concept_significance>
</concept>
<concept>
<concept_id>10011007.10010940.10010941.10010969.10010970</concept_id>
<concept_desc>Software and its engineering~Interactive games</concept_desc>
<concept_significance>500</concept_significance>
</concept>
</ccs2012>
\end{CCSXML}

\ccsdesc[500]{Computing methodologies~Natural language processing}
\ccsdesc[500]{Applied computing~Computer games}
\ccsdesc[500]{Software and its engineering~Interactive games}

% \begin{CCSXML}
% <ccs2012>
%  <concept>
%   <concept_id>00000000.0000000.0000000</concept_id>
%   <concept_desc>Do Not Use This Code, Generate the Correct Terms for Your Paper</concept_desc>
%   <concept_significance>500</concept_significance>
%  </concept>
%  <concept>
%   <concept_id>00000000.00000000.00000000</concept_id>
%   <concept_desc>Do Not Use This Code, Generate the Correct Terms for Your Paper</concept_desc>
%   <concept_significance>300</concept_significance>
%  </concept>
%  <concept>
%   <concept_id>00000000.00000000.00000000</concept_id>
%   <concept_desc>Do Not Use This Code, Generate the Correct Terms for Your Paper</concept_desc>
%   <concept_significance>100</concept_significance>
%  </concept>
%  <concept>
%   <concept_id>00000000.00000000.00000000</concept_id>
%   <concept_desc>Do Not Use This Code, Generate the Correct Terms for Your Paper</concept_desc>
%   <concept_significance>100</concept_significance>
%  </concept>
% </ccs2012>
% \end{CCSXML}

% \ccsdesc[500]{Do Not Use This Code~Generate the Correct Terms for Your Paper}
% \ccsdesc[300]{Do Not Use This Code~Generate the Correct Terms for Your Paper}
% \ccsdesc{Do Not Use This Code~Generate the Correct Terms for Your Paper}
% \ccsdesc[100]{Do Not Use This Code~Generate the Correct Terms for Your Paper}

%%
%% Keywords. The author(s) should pick words that accurately describe
%% the work being presented. Separate the keywords with commas.
\keywords{Interactive Narrative, Large Language Models, Abstraction, Narrative Space, Video Games, Generative AI}

\begin{teaserfigure}
\includegraphics[width=0.24 \textwidth]{figures/concept-descriptors.png}~%
\includegraphics[width=0.24 \textwidth]{figures/concept-shift-pivot.png}~%
\includegraphics[width=0.24 \textwidth]{figures/concept-abstract-outline.png}~%
\includegraphics[width=0.24 \textwidth]{figures/concept-remove-variant.png}%
\caption{We present WhatELSE, an interactive narrative authoring system that allows users to shape a narrative space using language abstraction. (a) We use the pivot, outline, and variants to describe the narrative space. Users can import an example story as a pivot. The system elevates the pivot into a narrative space. It generates an outline and variants to describe the space. Users can (b) edit the pivot to shift the space, (c) elevate the abstraction level to expand the space, or (d) remove variants to sculpt the space. }
  \label{fig:teaser}
  \Description{}
\end{teaserfigure}

% \received{20 February 2007}
% \received[revised]{12 March 2009}
% \received[accepted]{5 June 2009}

%%
%% This command processes the author and affiliation and title
%% information and builds the first part of the formatted document.
\maketitle
\section{Introduction}
\section{Introduction}
\label{sec:introduction}
The business processes of organizations are experiencing ever-increasing complexity due to the large amount of data, high number of users, and high-tech devices involved \cite{martin2021pmopportunitieschallenges, beerepoot2023biggestbpmproblems}. This complexity may cause business processes to deviate from normal control flow due to unforeseen and disruptive anomalies \cite{adams2023proceddsriftdetection}. These control-flow anomalies manifest as unknown, skipped, and wrongly-ordered activities in the traces of event logs monitored from the execution of business processes \cite{ko2023adsystematicreview}. For the sake of clarity, let us consider an illustrative example of such anomalies. Figure \ref{FP_ANOMALIES} shows a so-called event log footprint, which captures the control flow relations of four activities of a hypothetical event log. In particular, this footprint captures the control-flow relations between activities \texttt{a}, \texttt{b}, \texttt{c} and \texttt{d}. These are the causal ($\rightarrow$) relation, concurrent ($\parallel$) relation, and other ($\#$) relations such as exclusivity or non-local dependency \cite{aalst2022pmhandbook}. In addition, on the right are six traces, of which five exhibit skipped, wrongly-ordered and unknown control-flow anomalies. For example, $\langle$\texttt{a b d}$\rangle$ has a skipped activity, which is \texttt{c}. Because of this skipped activity, the control-flow relation \texttt{b}$\,\#\,$\texttt{d} is violated, since \texttt{d} directly follows \texttt{b} in the anomalous trace.
\begin{figure}[!t]
\centering
\includegraphics[width=0.9\columnwidth]{images/FP_ANOMALIES.png}
\caption{An example event log footprint with six traces, of which five exhibit control-flow anomalies.}
\label{FP_ANOMALIES}
\end{figure}

\subsection{Control-flow anomaly detection}
Control-flow anomaly detection techniques aim to characterize the normal control flow from event logs and verify whether these deviations occur in new event logs \cite{ko2023adsystematicreview}. To develop control-flow anomaly detection techniques, \revision{process mining} has seen widespread adoption owing to process discovery and \revision{conformance checking}. On the one hand, process discovery is a set of algorithms that encode control-flow relations as a set of model elements and constraints according to a given modeling formalism \cite{aalst2022pmhandbook}; hereafter, we refer to the Petri net, a widespread modeling formalism. On the other hand, \revision{conformance checking} is an explainable set of algorithms that allows linking any deviations with the reference Petri net and providing the fitness measure, namely a measure of how much the Petri net fits the new event log \cite{aalst2022pmhandbook}. Many control-flow anomaly detection techniques based on \revision{conformance checking} (hereafter, \revision{conformance checking}-based techniques) use the fitness measure to determine whether an event log is anomalous \cite{bezerra2009pmad, bezerra2013adlogspais, myers2018icsadpm, pecchia2020applicationfailuresanalysispm}. 

The scientific literature also includes many \revision{conformance checking}-independent techniques for control-flow anomaly detection that combine specific types of trace encodings with machine/deep learning \cite{ko2023adsystematicreview, tavares2023pmtraceencoding}. Whereas these techniques are very effective, their explainability is challenging due to both the type of trace encoding employed and the machine/deep learning model used \cite{rawal2022trustworthyaiadvances,li2023explainablead}. Hence, in the following, we focus on the shortcomings of \revision{conformance checking}-based techniques to investigate whether it is possible to support the development of competitive control-flow anomaly detection techniques while maintaining the explainable nature of \revision{conformance checking}.
\begin{figure}[!t]
\centering
\includegraphics[width=\columnwidth]{images/HIGH_LEVEL_VIEW.png}
\caption{A high-level view of the proposed framework for combining \revision{process mining}-based feature extraction with dimensionality reduction for control-flow anomaly detection.}
\label{HIGH_LEVEL_VIEW}
\end{figure}

\subsection{Shortcomings of \revision{conformance checking}-based techniques}
Unfortunately, the detection effectiveness of \revision{conformance checking}-based techniques is affected by noisy data and low-quality Petri nets, which may be due to human errors in the modeling process or representational bias of process discovery algorithms \cite{bezerra2013adlogspais, pecchia2020applicationfailuresanalysispm, aalst2016pm}. Specifically, on the one hand, noisy data may introduce infrequent and deceptive control-flow relations that may result in inconsistent fitness measures, whereas, on the other hand, checking event logs against a low-quality Petri net could lead to an unreliable distribution of fitness measures. Nonetheless, such Petri nets can still be used as references to obtain insightful information for \revision{process mining}-based feature extraction, supporting the development of competitive and explainable \revision{conformance checking}-based techniques for control-flow anomaly detection despite the problems above. For example, a few works outline that token-based \revision{conformance checking} can be used for \revision{process mining}-based feature extraction to build tabular data and develop effective \revision{conformance checking}-based techniques for control-flow anomaly detection \cite{singh2022lapmsh, debenedictis2023dtadiiot}. However, to the best of our knowledge, the scientific literature lacks a structured proposal for \revision{process mining}-based feature extraction using the state-of-the-art \revision{conformance checking} variant, namely alignment-based \revision{conformance checking}.

\subsection{Contributions}
We propose a novel \revision{process mining}-based feature extraction approach with alignment-based \revision{conformance checking}. This variant aligns the deviating control flow with a reference Petri net; the resulting alignment can be inspected to extract additional statistics such as the number of times a given activity caused mismatches \cite{aalst2022pmhandbook}. We integrate this approach into a flexible and explainable framework for developing techniques for control-flow anomaly detection. The framework combines \revision{process mining}-based feature extraction and dimensionality reduction to handle high-dimensional feature sets, achieve detection effectiveness, and support explainability. Notably, in addition to our proposed \revision{process mining}-based feature extraction approach, the framework allows employing other approaches, enabling a fair comparison of multiple \revision{conformance checking}-based and \revision{conformance checking}-independent techniques for control-flow anomaly detection. Figure \ref{HIGH_LEVEL_VIEW} shows a high-level view of the framework. Business processes are monitored, and event logs obtained from the database of information systems. Subsequently, \revision{process mining}-based feature extraction is applied to these event logs and tabular data input to dimensionality reduction to identify control-flow anomalies. We apply several \revision{conformance checking}-based and \revision{conformance checking}-independent framework techniques to publicly available datasets, simulated data of a case study from railways, and real-world data of a case study from healthcare. We show that the framework techniques implementing our approach outperform the baseline \revision{conformance checking}-based techniques while maintaining the explainable nature of \revision{conformance checking}.

In summary, the contributions of this paper are as follows.
\begin{itemize}
    \item{
        A novel \revision{process mining}-based feature extraction approach to support the development of competitive and explainable \revision{conformance checking}-based techniques for control-flow anomaly detection.
    }
    \item{
        A flexible and explainable framework for developing techniques for control-flow anomaly detection using \revision{process mining}-based feature extraction and dimensionality reduction.
    }
    \item{
        Application to synthetic and real-world datasets of several \revision{conformance checking}-based and \revision{conformance checking}-independent framework techniques, evaluating their detection effectiveness and explainability.
    }
\end{itemize}

The rest of the paper is organized as follows.
\begin{itemize}
    \item Section \ref{sec:related_work} reviews the existing techniques for control-flow anomaly detection, categorizing them into \revision{conformance checking}-based and \revision{conformance checking}-independent techniques.
    \item Section \ref{sec:abccfe} provides the preliminaries of \revision{process mining} to establish the notation used throughout the paper, and delves into the details of the proposed \revision{process mining}-based feature extraction approach with alignment-based \revision{conformance checking}.
    \item Section \ref{sec:framework} describes the framework for developing \revision{conformance checking}-based and \revision{conformance checking}-independent techniques for control-flow anomaly detection that combine \revision{process mining}-based feature extraction and dimensionality reduction.
    \item Section \ref{sec:evaluation} presents the experiments conducted with multiple framework and baseline techniques using data from publicly available datasets and case studies.
    \item Section \ref{sec:conclusions} draws the conclusions and presents future work.
\end{itemize}

\section{Related Work}

\section{Threats To Validity}\label{sec:ttv}
Our SLR aims to be comprehensive, but some limitations should be acknowledged. While we searched popular repositories (IEEE Xplore, ACM Digital Library, Springer, and ScienceDirect) and employed both backward and forward snowballing techniques on recent publications, the possibility of unintentional inclusion or exclusion of relevant studies remains. The authors carefully evaluated publications that fell on the borderline of inclusion/exclusion criteria to mitigate this risk.

Furthermore, our SLR focused exclusively on peer-reviewed journal articles and conference publications published in English. This decision was made to streamline the review process and ensure a high standard of research quality. However, it is important to acknowledge that relevant information may exist in other sources, such as books, theses, and non-English publications, which were not included in this review.

In addition, we deliberately excluded publications that primarily addressed network security, architecture, or systems that utilized SDN as a component. We aimed to maintain a focused review on the software security aspects of SDN itself. This means that studies analyzing SDN's role in broader contexts, such as cloud security or Internet of Things (IoT) networks, were not included. While this approach ensured a clear research focus, it may have overlooked valuable insights on the broader implications of SDN software security.

Overall, while we believe our SLR provides a comprehensive overview of the current state of research on SDN software security, readers should be aware of these limitations when interpreting our findings.

%[IN authoring: pre-genAI (twine, etc.), genAI (inworld, chrisma.ai, etc.)]
%(existing tools only at textural level)

%[Narrative planning and simulation]

%[Narrative Space Authoring Paradigm (conceptual space, emily-short, luminate, CLA): narrative space authoring paradigm -> design space]

%\section{User Needs and Design Rationale}
%\section{The \search\ Search Algorithm}
\label{sec:search}

%In traditional ML, structure changes and step (operator) changes are performed before model training, \ie, fixed to the training process, and weights are updated with SGD, because weights are continous, differentiable values, and there are significantly more weights than structure and operator changes. In workflow autotuning, all three types of cogs can be chosen with a unified search-based approach, because all of them are non-differentiable configurations and the number of cogs in different types are all small.
%Thus, \sysname\ only needs to navigate the search space of combination of cogs as the search space to produce its workflow optimization results.

%We propose, \textit{\textbf{\search}}, an adaptive hierarchical search algorithm that autotunes gen-AI workflows based on observed end-to-end workflow results. In each search iteration, \search\ selects a combination of cogs to apply to the workflow and executes the resulting workflow with user-provided training inputs. \search\ evaluates the final generation quality using the user-specified evaluator and measures the execution time and cost for each training input. These results are aggregated and serve as BO observations and pruning criteria.
%the optimizer can condition on and propose better configurations in later trials. The optimizer will also be informed about the violation of any user-specified metric thresholds. More details of this mechanism can be found in Appendix ~\ref{appdx:TPE}.

With our insights in Section~\ref{sec:theory}, we believe that search methods based on Bayesian Optimizer (BO) can work for all types of cogs in gen-AI workflow autotuning because of BO's efficiency in searching discrete search space.
A key challenge in designing a BO-based search is the limited search budgets that need to be used to search a high-dimensional cog space. 
For example, for 4 cogs each with 4 options and a workflow of 3 LLM steps, the search space is $4^{12}$. Suppose each search uses GPT-4o and has 1000 output tokens, the entire space needs around \$168K to go through. A user search budget of \$100 can cover only 0.06\% of the search space. A traditional BO approach cannot find good results with such small budgets.
%The entire search space grows exponentially with the number of cogs and the number of steps in a workflow. Moreover, different cogs and different combinations of cogs can have varying impacts on different workflows. 
%Without prior knowledge, it is difficult to determine the amount of budget to give to each cog.

To confront this challenge, we propose \textit{\textbf{\search}}, an adaptive hierarchical search algorithm that efficiently assigns search budget across cogs based on budget size and observed workflow evaluation results, as defined in Algorithms~\ref{alg:main} and \ref{alg:outer} and described below.
%autotunes gen-AI workflows based on observed end-to-end workflow results.
%\search\ includes a search layer partitioning method, a search budget initial assignment method, an evaluation-guided budget re-allocation mechanism, and a convergence-based early-exiting strategy. We discuss them in details below.

%\zijian{\search\ allows users to specify the optimization budget allowed in terms of the maximum number of search iterations. Based on the relationship between the complexity of the search space and the available budget, we will separate all tunable parameters into different layers each optimized by independent Bayesian optimization routines. Then we will decide the maximum budget each layer can get with a bottom-up partition strategy. Besides search space and resource partition, we also employ a novel allocation algorithm that integrates successive halving~\cite{successivehalving} and a convergence-based early exiting strategy to facilitate efficient usage of assigned budget.}


% The outermost layer searches and selects structures for a workflow; the middle layer searches and selects step options under the workflow structure selected in the outermost layer; the innermost layer searches and selects weights with the given workflow structure and steps. 

\begin{algorithm}[h]
    \caption{\search\ Algorithm}
    \label{alg:main}
      \small
\begin{algorithmic}[1]
\STATE \textbf{Global Value:} $R = \emptyset$ \COMMENT{Global result set}
%\STATE \textbf{Global Value:} $F = \emptyset$ \COMMENT{Global observation set}

%Reduct factor $\eta > 1$, explore width $W$
\STATE \textbf{Input:} User-specified Total Budget $TB$
\STATE \textbf{Input:} Cog set $C = \{c_{11},c_{12},...\}, \{c_{21},c_{22},...\}, \{c_{31},c_{32},...\}$

    \STATE
%\FOR{$i = 1,2,3$}
    %\COMMENT{$\alpha$ is a configurable value default to 1.1}
%\ENDFOR
%\STATE
%    \STATE \{$B_1,B_2,B_3$\} = LayerPartition($C$) \COMMENT{Calculate ideal layer budget}
    %\STATE \textbf{Glob}.budgets = budgets
%    \STATE opt\_layers = init\_opt\_routines() \COMMENT{A list of optimize routine each layer will use for search}
%\STATE
%    \FOR{$i \in L, \dots, 1$}
%        \IF{$i == L$}
 %           \STATE opt\_layers[L] = InnerLayerOpt
  %      \ELSE
   %         \STATE opt\_layers[i] = OuterLayerOpt
            %\STATE opt\_layers[i].next\_layer\_budgets = B[i+1]
            %\STATE opt\_layers[i].next\_layer\_routine = opt\_layers[i+1]
    %    \ENDIF
    %\ENDFOR
%\STATE opt\_layers[1].invoke($\emptyset$, B[1])
\STATE $U = 0$ \COMMENT{Used budget so far, initialize to 0}

\STATE \COMMENT{Perform search with 1 to 3 layers until budget runs out}
\FOR{$L = 1,2,3$} 
        \IF{$L=1$}
            \STATE $C_1 = C_1 \cup C_2 \cup C_3$ \COMMENT{Merge all cogs into a single layer}
        \ENDIF
        \IF{$L==2$}
            \STATE $C_1 = C_1 \cup C_2$ \COMMENT{Merge step and weight cogs}
            \STATE $C_2 = C_3$ \COMMENT{Architecture cog becomes the second layer}
        \ENDIF
        \STATE
    \FOR{$i = 1,..,L$}
    \STATE $NC_i = |C_i|$ \COMMENT{Total number of cogs in layer $L$} 
%    NO_i &= \sum_{L} \{\text{number of possible options in cog } c_{ij}\} \\
    \STATE $S_i = NC_i^\alpha$ \COMMENT{Estimated expected search size in layer $i$}
    \ENDFOR
    \STATE $E_L = \prod\limits_{i=1}^{L}S_i$ \COMMENT{Expected total search size in the current round}
    \STATE $E = TB - U > E_L$ ? $E_L$ : $(TB - U)$ \COMMENT{Consider insufficient budget} 
    \IF{$L==3$ and $(TB - U)$ > $E_L$}
         \STATE $E = TB - U$ \COMMENT{Spend all remaining budget if at 3 layer}
    \ENDIF
    %\STATE$TL = |N|$ \COMMENT{number of layers}
    \FOR{$i = 1,..,L$}
        \STATE $B_i =  \lfloor S_i \times \sqrt[L]{\frac{E}{E_L}}\rfloor$
        %$B$ = BudgetAssign($N$, $TL$, $TB$)
        \COMMENT{Assign budget proportionally to $S_i$}
    \ENDFOR
    \STATE
\STATE \texttt{LayerSearch} ($\emptyset$, $B$, $L$, $B_L$) \COMMENT{Hierarchical search from layer $L$}
\STATE
\STATE $U = U + E$
\IF{$U \geq TB$}
\STATE break \COMMENT{Stop search when using up all user budget}
\ENDIF
\ENDFOR
%\STATE
%\STATE $O$ = \texttt{SelectBestConfigs} ($R$)
%\IF{$L == 1$}
%    \STATE InnerLayerOpt($\emptyset$, B[1])
%\ELSE
%    \STATE OuterLayerOpt($\emptyset$, B[1], 1)
%\ENDIF
\STATE
\STATE \textbf{Output:} $O$ = \texttt{SelectBestConfigs} ($R$) \COMMENT{Return best optimizations}
\end{algorithmic}
\end{algorithm}

\subsection{Hierarchical Layer and Budget Partition}
\label{sec:ssp}

%We motivate \search's adaptive hierarchical search 
A non-hierarchical search has all cog options in a single-layer search space for an optimizer like BO to search, an approach taken by prior workflow optimizers~\cite{dspy-2-2024,gptswarm}.
With small budgets, a single-layer hierarchy allows BO-like search to spend the budget on dimensions that could potentially generate some improvements.
%While given enough budget, the single-layer space can be extensively searched to find global optimal, with little budget, 
However, a major issue with a single-layer search space is that a search algorithm like BO can be stuck at a local optimum even when budgets increase.
% (unless the budget is close to covering a very large space across dimensions).
To mitigate this issue, our idea is to perform a hierarchical search that works by choosing configurations in the outermost layer first, then under each chosen configuration, choosing the next layer's configurations until the innermost layer. 
With such a hierarchy, a search algorithm could force each layer to sample some values. Given enough budget, each dimension will receive some sampling points, allowing better coverage in the entire search space. However, with high dimensionality (\ie, many types of cogs) and insufficient budget, a hierarchical search may not be able to perform enough local search to find any good optimizations.

To support different user-specified budgets and to get the best of both approaches, we propose an adaptive hierarchical search approach, as shown in Algorithm~\ref{alg:main}.
\search\ starts the search by combining all cogs into one layer ($L=1$, line 9 in Algorithm~\ref{alg:main}) and estimating the expected search budget of this single layer to be the total number of cogs to the power of $\alpha$ (lines 16-19, by default $\alpha = 1.1$). This budget is then passed to the \texttt{LayerSearch} function (Algorithm~\ref{alg:outer}) to perform the actual cog search. When the user-defined budget is no larger than this estimated budget, we expect the single-layer, non-hierarchical search to work better than hierarchical search.
%as the budget for this single layer.

If the user-defined budget is larger, \search\ continues the search with two layers ($L=2$), combining step and weight cogs into the inner layer and architecture cogs as the outer layer (lines 11-14).
\search\ estimates the total search budget for this round as the product of the number of cogs in each of the two layers to the power of $\alpha$ (lines 16-20). It then distributes the estimated search budget between the two layers proportionally to each layer's complexity (lines 22-24) and calls the upper layer's \texttt{LayerSearch} function. Afterward, if there is still budget left, \search\ performs a last round of search using three layers and the remaining budget in a similar way as described above but with three separate layers (architecture as the outermost, step as the middle, and weight cogs as the innermost layer). Two or three layers work better for larger user-defined budgets, as they allow for a larger coverage of the high-dimensional search space.

Finally, \search\ combines all the search results to select the best configurations based on user-defined metrics (line 34).

%\search\ organizes cogs by having architecture cogs in the outer-most search layer, step cogs in the middle layer, and weight cogs in the inner-most layer (line 4 in Algorithm~\ref{alg:main}).
%This is because step cogs' input and output format are dependent on the workflow structure, and the effectiveness of weights (\eg, prompting) are dependent on steps (\eg, LLM model). 

% increases the number of layers until hitting the user-specified total search budget, $TB$

%Thus, the first step of \search\ is to determine the number of layers in its hierarchy and what cogs to include in a layer.
%Intuitively, structure cogs should be placed in the outer-most search layer to be determined first before exploring other cogs. This is because other cogs change node and edge values, and it is easier for 
%However, instead of a fixed number of layers in the hierarchy, we adapt the cog layering according to user-specified total search budgets, $TB$, and the complexity of each layer, using Algorithm~\ref{alg:main}.

% the following \texttt{LayerPartition} method.
%We begin by modeling the relationship between the expected number of evaluations and the number of cogs as well as the number of options in each layer:

%We first consider the identity of each cog in the search space. All structure-cogs will be placed in the outer-most search layer exclusively, which is similar to non-differentiable NAS in traditional ML training. This layer will fix the workflow graph and pass it to the following layer, allowing a stabilized search space for faster convergence.

%Since step-cogs will not create a changing search space, the partition of step-cogs and weight-cogs is conditioned on the search space complexity and the given total budget. Separating step-cogs out can benefit from a more flexible budget allocation strategy and broader exploration for local search at weight-cogs but performs poorly when the given budget is more constrained, in that case, we will optimize them jointly in the same layer.


%\small
%\begin{align*}
%    C &= \{c_{11},c_{12},...\}, \{c_{21},c_{22},...\}, \{c_{31},c_{32},...\} \\
%    NC_i &= \text{total number of cogs in layer i} \\
%    NO_i &= \sum_{j} \{\text{number of possible options in cog } c_{ij}\} \\
%    N_i &= max(NC_i^\alpha,NO_i) \\
%    N_i &= \sum_{j} \{\text{number of possible options in } C_{ij}\} \\
%    N_i &= max(|C_i|^\alpha, N_i) \\
%    B_j &= \prod\limits_{i=1}^{j}N_i, j \in \{1,2,3\}
%\end{align*}

%\normalsize
%where $L$ represents the total number of layers and can be 1, 2, or 3. 
%$C$ represents the entire cog search space, with each row $c_{i*}$ being one of the three types of cogs and lower layers having lower-numbered rows (\eg, $c_{1*}$ being weight cogs). $NC_i$ is the number of cogs in layer $i$, and $NO_i$ is the total number of options across all cogs in layer $i$. $N_i$ is our estimation of the complexity of layer $i$ based on $NC_i$ and $NO_i$ ($\alpha$ is a configurable weight to control the importance between $NC_i$ and $NO_i$; by default $\alpha = 1.1$). 
%$\alpha$ stands for a control parameter, setting the intensity of this scaling behavior w.r.t the number of cogs, we found that $\alpha = 1.2$ is empirically sufficient and efficient for optimizing real workloads. 
%$B_j$ is the expected total number of workflow evaluations for all the lower $j$ layers.
%After calculating $B_1$, $B_2$, and $B_3$, we compare the total budget $TB$ with them.
%When $TB \geq B_3$, we set the total number of layers, $TL$, to 3. When $B_2 \leq TB < B_3$, we set the total number of layers to 2 and merge the step and weight cogs into one layer. When $TB < B_1$, we put all cogs in one layer.
%We only create a separate layer for step-cogs when the given budget $TB$ is greater or equal to the total expected budget for three layers.

%\subsection{Seach Budget Partition}
%\label{sec:sbp}
%After determining cog layers, we distribute the total budget, $TB$, across the layers proportionally to each layer's expected budget $N_i$: , which is the \texttt{BudgetAssign} function.
%We follow a bottom-up partition strategy, where lower layers will try to greedily take the expected budget. This stems from two simple heuristics: (1) feedback to the upper layer is more accurate when the succeeding layer is trained with enough iterations, and (2) the effectiveness of a structure change depends on the setting of individual steps in the workflow (\eg, majority voting is more powerful when each LLM-agent is embedded with diverse few-shot examples or reasoning styles). In cases where the given resource exceeds the total expected budget, 
%We assign $TB$ across layers proportionally to their expected budget $N_i$. 
%The budget assigned at each layer $B_i$ given the total available number of evaluations $TB$ is obtained as follows:

%\small
%\begin{align}
%B_i &=  \lfloor N_i \times \sqrt[L]{\frac{TB}{B^*}}\rfloor
%    B_L &= \begin{cases}
%        min(N_L, TB) & TB < B^* \\
%        \lfloor N_L \times \sqrt[L]{\frac{TB}{B^*}}\rfloor & TB \geq B^*
%    \end{cases}
%    \\
%    B_i &= \begin{cases}
%        min(N_i, \lfloor\frac{TB}{\prod_{j=i+1}^L B_j}\rfloor) & TB < B^* \\
%        \lfloor N_i \times \sqrt[L]{\frac{TB}{B^*}}\rfloor & TB \geq B^*
%    \end{cases}
%\end{align}

%\normalsize


\subsection{Recursive Layer-Wise Search Algorithm}
%The calculation above pre-assigns cogs to layers and search budgets to each layer. 
We now introduce how \search\ performs the actual search in a recursive manner until the inner-most layer is searched, as presented in Algorithm~\ref{alg:outer} \texttt{LayerSearch}. 
Our overall goal is to ensure strong cog option coverage within each layer while quickly directing budgets to more promising cog options based on evaluation results.
%So far, we have determined the optimization layer structure and the maximum allowed search iteration each layer will get. Next, we introduce how the budget is consumed in each layer. The inner-most layer, where weight-cogs, and potentially step-cogs, reside, follows the conventional Bayesian optimization process, exhausting all budgets unless an early stop signal is sent. This signal will be triggered when the current optimizer witnesses $p$ consecutive iterations without any improvements above the threshold. All optimization layers use early stopping to avoid budget waste.
%Algorithm~\ref{alg:inner} describes the search happening at the inner-most (bottom) layer, and 
Specifically, every layer's search is under a chosen set of cog configurations from its upper layers ($C_{chosen}$) and is given a budget $b$. 
In the inner-most layer (lines 7-20), \search\ samples $b$ configurations and evaluates the workflow for each of them together with the configurations from all upper layers ($C_{chosen}$). The evaluation results are added to the feedback set $F$ as the return of this layer.

\begin{algorithm}[h]
  %\algsetup{linenosize=\tiny}
  \small
    \caption{\texttt{LayerSearch} Function}
    \label{alg:outer}
\begin{algorithmic}[1]
%\STATE \textbf{Global Config:} Reduct factor $\eta > 1$, explore width $W$
\STATE \textbf{Global Value:} $R$ \COMMENT{Global result set}
%\STATE \textbf{Global Value:} $F$ \COMMENT{Global observation set}
\STATE \textbf{Input:} $C_{chosen}$: configs chosen in upper layers
\STATE \textbf{Input:} $B$: Array storing assigned budgets to different layers
\STATE \textbf{Input:} $curr\_layer$: this layer's level
\STATE \textbf{Input:} $curr\_b$: this layer's assigned budget
%\STATE
%\FUNCTION{LayerSearch\hspace{0.4em}($C_{chosen}$, $B$, $curr\_layer$, $curr\_b$)}

    \STATE
    \STATE \COMMENT{Search for inner-most layer}
    \IF{curr\_layer == 1}
        \STATE $F = \emptyset$ \COMMENT{Init this layer's feedback set to empty}
        %\STATE $F^{\prime} = match(C_{chosen}, F)$ \COMMENT{Local feedback set}
        \FOR{$k = 0, \dots, curr\_b$}
            \STATE $\lambda$ = \texttt{TPESample} (1) \COMMENT{Sample one configuration using TPE}
            \STATE $f = $ \texttt{EvaluateWorkflow} ($C_{chosen} \cup \lambda$)
            \STATE $R = R \cup \{C_{chosen} \cup \lambda\}$ \COMMENT{Add configuration to global $R$}
            \IF{\texttt{EarlyStop} (f)}
            \STATE break
            \ENDIF
            \STATE $F = F \cup \{f\}$ \COMMENT{Add evaluate result to feedback $F$}
        \ENDFOR
        %\STATE $F = F \cup F^{\prime}$
        \STATE \textbf{Return} $F$
    \ENDIF
    \STATE
    \STATE \COMMENT{Search for non-inner-most layer}
    %\STATE $K = \lfloor \frac{b}{W} \rfloor$, 
    \STATE $b\_used = 0$, $TF = \emptyset$ \COMMENT{Init this layer's used budget and feedback set}
    \STATE $R = \lceil\frac{curr\_b}{\eta}\rceil$, $S = \lfloor\frac{curr\_b}{R}\rfloor$ \COMMENT{Set $R$ and $S$ based on $curr\_b$}
    \STATE
    \WHILE{$b\_{used}$ $\leq$ $curr\_b$}
        \STATE \COMMENT{Sample $W$ configs at a time until running out of $curr\_b$}
        \STATE $n = (curr\_b - b_{used})$ > $W$ ? $W$ : $(curr\_b - b_{used})$
        %\IF{$b - b_{used} < W$}
        %    \STATE $n = b_l - b_{used}$
        %\ELSE
         %   \STATE $n=W$
        %\ENDIF
        \STATE $b\_used$ += $n$
        %\STATE $n = \text{min}(W,\ b_l - kW)$ \COMMENT{Propose $W$ configs and meet $b_l$ constraint}
        \STATE $\Theta = $ \texttt{TPESample} ($n$) \COMMENT{Sample a chunk of $n$ configs in the layer} 
        %\STATE $F^{\prime} = match(C_{chosen}, F)$ \COMMENT{Per-chunk feedback set}
        \STATE $F = \emptyset$ \COMMENT{Init this layer's feedback set to empty}
        \STATE
        \FOR{$s = 0, 1, \dots, S$}
            \STATE $r_s = R\cdot \eta^s$
            \FOR{$\theta \in \Theta$}
                %\IF{$curr\_layer < max\_layer$}
                    \STATE $f =$ \texttt{LayerSearch} ($C_{chosen} \cup \{\theta\}$, $B$, curr\_layer$-1$, $r_s$)
                %\ELSE
                %    \STATE $f =$ InnerOpt($\gamma \cup \{\theta\}$, $r_s$)
                %\STATE $f$ = $opt\_layers[current\_layer+1](\gamma \cup \{\theta\}, r_s)$ \{Optimize the current config at the next layer with $r_s$ budget \}
                %\ENDIF
                \STATE $F = F \cup f$ \COMMENT{Add evaluate result to feedback}
                \IF{\texttt{EarlyStop} ($f$)}
                    \STATE $\Theta = \Theta - \{\theta\}$ \COMMENT{Skip converged configs}
                \ENDIF
            \ENDFOR
            \STATE $\Theta$ = Select top $\lfloor \frac{|\Theta|}{\eta}\rfloor$ configs from $F$ based on user-specified metrics
        \ENDFOR
        \STATE
        \IF{\texttt{EarlyStop} ($F$)}
            \STATE break \COMMENT{Skip remaining search if results converged}
        \ENDIF
        \STATE $TF = TF \cup F$
    \ENDWHILE
    %\STATE $F = F \cup TF$
        \STATE \textbf{Return} $TF$

%\ENDFUNCTION

%\STATE \textbf{Output:} Best metrics in all trials
\end{algorithmic}
\end{algorithm}

% consumption\_nextlayer\_bucket = WSR

% for s in 0, 1,...S do
%     w = W*\eta^{s}
%     r = R*\eta^{-s}

% total budget at next layer = b_l / W * WSR = b_l * SR

% b_l * SR <= b_l * B_l+1

% S = B_{l+1} / R



For a non-inner-most layer, \search\ samples a chunk ($W$) of points at a time using the TPE BO algorithm~\cite{bergstra2011tpe} until all this layer's pre-assigned budget is exhausted (lines 27-30). Within a chunk, \search\ uses a successive-halving-like approach to iteratively direct the search budget to more promising configurations within the chunk (the dynamically changing set, $\Theta$). In each iteration, \search\ calls the next-level search function for each sampled configuration in $\Theta$ with a budget of $r_s$ and adds the evaluation observations from lower layers to the feedback set $F$ for later TPE sampling to use (lines 35-37).
In the first iteration ($s=0$), $r_s$ is set to $R\cdot \eta^0=R$ (line 34). After the inner layers use this budget to search, \search\ filters out configurations with lower performance and only keeps the top $\lfloor \frac{|\Theta|}{\eta}\rfloor$ configurations as the new $\Theta$ to explore in the next iteration (line 42). In each next iteration, \search\ increases $r_s$ by $\eta$ times (line 34), essentially giving more search budget to the better configurations from the previous iteration.

The successive halving method effectively distributes the search budget to more promising configurations, while the chunk-based sampling approach allows for evaluation feedback to accumulate quickly so that later rounds of TPE can get more feedback (compared to no chunking and sampling all $b$ configurations at the same time). To further improve the search efficiency, we adopt an {\em early stop} approach where we stop a chunk or a layer's search when we find its latest few searches do not improve workflow results by more than a threshold, indicating convergence (lines 14,38,45).

%algorithm takes as input other cog settings from previous layers and the assigned budget at the current layer. It tiles the search loop into fixed-size blocks (line 4), each runs the SuccessiveHalving (SH) subroutine in the inner loop (line 7-15). In each SH iteration, only top-$1/\eta$-quantile configurations in $\Theta$ will continue in the next round with $\eta$ times larger budget consumption. As a result, exponentially more trials will be performed by more promising configurations. 

%On average, \textit{Outer-layer search} will create $K$ brackets, each granting approximately $WRS$ budget to the next layer. $R$ represents the smallest amount of resource allocated to any configurations in $\Theta$. 

% layer - 1: budget = 4
% K * W <= b\_current layer
% layer -1: itear 0: propose 2

%     SH:
%     2 config -> R
%     1 config -> 2R

%     iteration 1: propose 2 = W
%     SH:
%     2 config -> R
%     1 config -> 2R

% W configs; each has R resource

% W / eta configs; each has R * eta resource

% R -> least resource one config can get = B2 - smth
% R + R*eta + ... + R*eta\^s -> most promising = B2 + smth


% $L2$ is the middle layer where structure-cogs and step-cogs may be placed exclusively. We employ hyperband for its robustness in exploration and exploitation trace-off. If this layer exists, it will instruct $L1$ the number of search iterations to run in each invocation. Specifically, in each iteration at line 4, \sysname will propose $n$ configurations and run SuccessiveHalving (SH) subroutine (line 8-15). SH will optimize each proposal and use the search results from $L1$ to rank their performance. Each time only the top-performing $n \cdot \eta^{-i}$ can continue in the next round with a larger budget. With this strategy, exponentially more search budgets are allocated to more promising configs at $L2$.

% \input{algo-l2-search}

% $L3$ is the outer-most layer for structure-cogs only when $L2$ is created. For this layer, we employ plain SH without hyperband because of its predictable convergence behavior. This is mainly due to two factors: (1) structure change to the workflow is more significant thus different configurations are more likely to deviate after training with the following layers. (2) with the search space partition strategy in Sec ~\ref{sec:ssp}, we can assume the available budget at each layer is substantial when $L3$ exists. Given this prior knowledge, we can avoid grid searching control parameter $n$ as in the hyperband but adopt a more aggressive allocation scheme to bias towards better proposals and moderate search wastes.



%\subsubsection{Runtime Budget Adaptation}
%Using static estimation of the expected budget for each layer is not enough, we also adjust the assignment during the optimization based on real convergence behavior. Specifically, for layer $i$, we record the number of configurations evaluated in each optimize routine. We set the convergence indicator $C_{ij}$ of $j^{th}$ routine with this number if the search early exits before reaching the budget limit, otherwise 2\x of its assigned resource. Then we update $E_i$ with $\frac{\sum_{j}^M C_{ij}}{M}$. \sysname\ will update the budget partition according to Sec~\ref{sec:sbp} for any newly spawned optimizer routines. Besides controlling the proportion of budgets across layers, a smaller/larger $B_{l+1}$ will also guide the SH in Alg~\ref{alg:outer} to shrink/extend the budget $R$ for differentiating config performance.


\section{\sysname\ Design}
\label{sec:cognify}

We build \sysname, an extensible gen-AI workflow autotuning platform based on the \search\ algorithm. The input to \sysname\ is the user-written gen-AI workflow (we currently support LangChain \cite{langchain-repo}, DSPy \cite{khattab2024dspy}, and our own programming model), a user-provided workflow training set, a user-chosen evaluator, and a user-specified total search budget. \sysname\ currently supports three autotuning objectives: generation quality (defined by the user evaluator), total workflow execution cost, and total workflow execution latency. Users can choose one or more of these objectives and set thresholds for them or the remaining metrics (\eg, optimize cost and latency while ensuring quality to be at least 5\% better than the original workflow). 
\sysname\ uses the \search\ algorithm to search through the cog space.
When given multiple optimization objectives, \sysname\ maintains a sorted optimization queue for each objective and performs its pruning and final result selection from all the sorted queues (possibly with different weighted numbers).
To speed up the search process, we employ parallel execution, where a user-configurable number of optimizers, each taking a chunk of search load, work together in parallel. %Below, we introduce each type of cogs in more details.
\sysname\ returns multiple autotuned workflow versions based on user-specified objectives.
\sysname\ also allows users to continue the auto-tuning from a previous optimization result with more budgets so that users can gradually increase their search budget without prior knowledge of what budget is sufficient.
Appendix~\ref{sec:apdx-example} shows an example of \sysname-tuned workflow outputs. 
\sysname\ currently supports six cogs in three categories, as discussed below. 

%In \sysname, we call every workflow optimization technique a {\em cog}, including structure-changing cogs like task decomposition, step-changing cogs like model selection, and weight-changing cogs like adding few-shot examples to prompts. 
%\sysname\ places structure-changing cogs in the outermost layer, step cogs in the middle layer, and weight cogs in the innermost layer, because \fixme{TODO}.


\subsection{Architecture Cogs}
\label{sec:structure-cog}
%Changing the structure of a workflow can potentially improve its generation quality (\eg, by using multiple steps to attempt at a task in parallel or in chain) or reduce its execution cost and latency (\eg, by merging or removing steps).
\sysname\ currently supports two architecture cogs: task decomposition and task ensemble.
Task decomposition~\cite{khot2023decomposed} breaks a workflow step into multiple sub-steps and can potentially improve generation quality and lower execution costs, as decomposed tasks are easier to solve even with a small (cheaper) model.
There are numerous ways to perform task decomposition in a workflow. 
%, as all LM steps can potentially be decomposed and into different numbers of sub-steps in different ways. Throwing all options to the Bayesian Optimizer would drastically increase the search space for \sysname. 
To reduce the search space, we propose several ways to narrow down task decomposition options. Even though we present these techniques in the setting of task decomposition, they generalize to many other structure-changing tuning techniques.

%First, we narrow down a selected set of steps in a workflow to decompose. 
Intuitively, complex tasks are the hardest to solve and worth decomposition the most. We use a combination of LLM-as-a-judge \cite{vicuna_share_gpt} and static graph (program) analysis to identify complex steps. We instruct an LLM to give a rating of the complexity of each step in a workflow. We then analyze the relationship between steps in a workflow and find the number of out-edges of each step (\ie, the number of subsequent steps getting this step's output). More out-edges imply that a step is likely performing more tasks at the same time and is thus more complex. We multiply the LLM-produced rating and the number of out-edges for each step and pick the modules with scores above a learnable threshold as the target for task decomposition. We then instruct an LLM to propose a decomposition (\ie, generate the submodules and their prompts) for each candidate step. %We provide the LLM with few-shot examples for what proposed modules for a separate task could look like. We also add a refinement step that validates whether the proposition decomposition maintains the semantics of the original trajectory. Once candidate decompositions are generated, those are used for the entirety of the optimization.

{
\begin{figure*}[t!]
\begin{center}
\centerline{\includegraphics[width=0.95\textwidth]{Figures/big_grid.pdf}}
\vspace{-0.1in}
\mycap{Generation Quality vs Cost/Latency.}{Dashed lines show the Pareto frontier (upper left is better). Cost shown as model API dollar cost for every 1000 requests. Cognify selects models
from GPT-4o-mini and Llama-8B. DSPy and Trace do not support model selection and are given GPT-4o-mini for all steps. Trace results for Text-2-SQL and FinRobot have 0 quality and are not included.} 
\Description{Eight graphs with different shapes representing baselines compared to points on a Pareto frontier.}
\label{fig-biggrid}
\end{center}
\end{figure*}
}


The second structure-changing cog that \sysname\ supports is task ensembling. This cog spawns multiple parallel steps (or samplers) for a single step in the original workflow, as well as an aggregator step that returns the best output (or combination of outputs). By introducing parallel steps, \sysname\ can optimize these independently with step and weight cogs. This provides the aggregator with a diverse set of outputs to choose from. 
%The aggregator is prompted with the role of the samplers, as well as the inputs to each. It also receives a criteria by which it should make a decision. We choose to prompt it with a qualitative description of how it should resolve discrepancies between outputs. 


\subsection{Step Cogs}
We currently support two step-changing cogs: model selection for language-model (LM) steps and code rewriting for code steps.

For model selection, to reduce its search space, we identify ``important'' LM steps---steps that most critically impact the final workflow output to reduce the set \search\ performs TPE sampling on. Our approach is to test each step in isolation by freezing other steps with the cheapest model and trying different models on the step under testing. 
We then calculate the difference between the model yielding the best and worst workflow results as the importance of the step under testing. %For each model, we get the workflow output quality score using sampled user-supplied inputs and user-specific evaluator. We then calculate the difference between the highest and lowest scores as this module's importance. 
After testing all the steps, we choose the steps with the highest K\% importance as the ones for TPE to sample from.
%, where K is determined based on user-chosen stop criteria. We then initialize the Bayesian optimization to start with the state where important modules use the largest model and all other modules use the cheapest model. We set the TPE optimization bandwidth of each module to be the inverse of importance, \ie, the more important a module is the more TPE spends on optimizing.

The second step cog \sysname\ supports is code rewriting, where it automatically changes code steps to use better implementation. To rewrite a code step, \sysname\ finds the $k$ worst- and best-performing training data points and feeds their corresponding input and output pairs of this code step to an LLM. We let the LLM propose $n$ new candidate code pieces for the step at a time.
%in parallel to generate a set of $n$ candidates.
In subsequent trials, the optimizer dynamically updates the candidate set using feedback from the evaluator.


\subsection{Weight Cogs}
\sysname\ currently supports two weight-changing cogs: reasoning and few-shot examples.
First, \sysname\ supports adding reasoning capability to the user's original prompt, with two options: zero-shot Chain-of-Thought \cite{wei2022chain} (\ie, ``think step-by-step...'') and dynamic planning \cite{huang2022language} (\ie, ``break down the task into simpler sub-tasks...''). These prompts are appended to the user's prompt. In the case where the original module relies on structured output, we support a reason-then-format option that injects reasoning text into the prompt while maintaining the original output schema.

Second, \sysname\ supports dynamically adding few-shot examples to a prompt. At the end of each iteration, we choose the top-$k$-performing examples for an LM step in the training data and use their corresponding input-output pairs of the LM step as the few-shot examples to be appended to the original prompt to the LM step for later iterations' TPE sampling. As such, the set of few-shot examples is constantly evolving during the optimization process based on the workflow's evaluation results. 
%Few-shot examples are available to all modules, even intermediate steps in the workflow. We use the full trajectory of each request to generate examples for the intermediate steps. Furthermore, we automatically filter out examples that do not meet a user-specified threshold. 



\section{Challenges of AI-Bridged Interactive Narrative Authoring}
\begin{figure*}[!ht]
    \centering
    \includegraphics[width=\linewidth]{figures-src/concept.pdf}
    \caption{SliderSpace decomposes the visual variation of diffusion model's knowledge corresponding to a concept. These directions can be perceived as interpretable directions of the model's hierarchical knowledge. We show the decomposed slider direction for a concept using SliderSpace and the corresponding labels generated by Claude 3.5 Sonnet.}
    \label{fig:concept}
\end{figure*}




% \section{WhatELSE: System and Features }
% 
%PlanGREEN

%GEN-Plan

%G- generate
%R-refine
%E- edit

%% GREEN-plan

%% PURPLE
\begin{figure*}
    \centering
    \Description{PLAID's system architecture diagram. Top part shows the database (a), and bottom part shows the interface (b). The system starts from bottom right as an instructor is interested in a programming domain, then the pipeline described in the text produces reference materials at different levels of granularity, and these are presented in the interface.}
    \includegraphics[width=\textwidth]{img/system-architecture-subgoals.png}
    \caption{PLAID's reference content is generated through an LLM pipeline
    %inspired by the practices of instructors who have successfully identified programming plans. 
    that produces output on three levels.
    First, a wide variety of use cases are generated to create example programs that focus on code's applications. Next, using LLM's explanatory comments that represent subgoals within the code, the examples are segmented into meaningful code snippets. The LLM is queried to generate other plan components for each code snippet. Finally, the code snippets are clustered to identify the most common patterns, representing plan candidates. The full programs are presented in `Programs' views of PLAID interface, whereas snippets are presented in clusters in the `Plan Creation' view.}
    \label{fig:system-pipeline}
\end{figure*}
\section{PLAID: A System for Supporting Plan Identification}
\label{sec:system-design}

Following the design goals devised from the design workshop, we refined our early prototype into PLAID: a
%LLM-powered
tool to assist instructors in their plan identification process.
PLAID synthesizes the capabilities of LLMs in code generation with interactions enabling plan identification practices observed in our studies with instructors.
As we noted in the findings of our design workshop, the LLM-generated candidate plans are not ready to be used as is in instruction, but instructors can readily adapt and correct them (\cref{sec:workshop-findings-condition2}).
PLAID enables collaboration between instructors and LLMs, enhancing the plan identification process by automating its time-intensive information-gathering tasks and facilitating instructors' ability to refine LLM-generated candidate plans based on their knowledge about pedagogy and the programming domain. 



\subsection{Practical Illustration}

To understand how instructors use can PLAID to more easily adopt plan-based pedagogies, we follow Jane, a computer science instructor using PLAID to design programming plans for her course (summarized in \cref{fig:jane-workflow}).

Jane is teaching a programming course for psychology majors and wants to introduce her students to data analysis using Pandas. As her students have limited prior programming experience and use programming for specific goals, she organizes her lecture material around programming plans to emphasize purpose over syntax. 
% that explain practical concepts to students and help them focus on the purpose behind the code they write.
% However, she realizes that all introductory computer science courses offered at her institution only teach basic programming constructs like data structures. After exploring Google Scholar for effective instructional methods to teach application-focused programming to non-computer science majors, she learned about plan-based pedagogies that help them focus on the purpose behind the code they write. In her literature review, she finds out about PLAID, a tool that can help her design domain-specific plans. She reviews the domains supported by the tool (Pandas, Pytorch, Beautifulsoup, and Django) and decides to use Pandas, a popular and powerful data analysis and manipulation library, to create her curriculum. 

She logs in to the PLAID web interface, % and takes time to explore the system's features. 
and asks PLAID to suggest a plan (\cref{fig:jane-workflow}, 1). The first plan recommended to her 
% she sees is a plan to help students learn about
is about reading CSV files. 
She thinks the topic is important and the solution code aligns with her experience; % the solution is promising and represents an important concept that students need to know about.
% She is satisfied with the given solution 
but she finds the generated name and goal to be too generic. She edits (\cref{fig:jane-workflow}, 2) these fields to provide more context that she feels is right for her students.
% She refines those fields and then 
To make this plan more abstract and appropriate for more use cases, %explain how this plan can be used for reading data from different file formats,
she marks the file path as a changeable area (\cref{fig:jane-workflow}, 3), generalizing the plan for reading data from different file formats.

Inspired by the first plan, she decides to create a plan for saving data to disk. She wants to teach the most conventional way of saving data, so she switches to the use case tab (\cref{fig:jane-workflow}, 4) and explores example programs that save data to get a sense of common practices.  %interact with the list of complete programs.
She finds a complete example where a DataFrame is created and and saved to a file. %performs cleaning tasks like deleting NaN values, and exports it.
% She realizes that something she hadn't thought of before: saving new data is almost always necessary after performing data manipulation operations!
She selects the part of the code that exports data to a file and creates a plan from that selection (\cref{fig:jane-workflow}, 5).


For the next plan, she reflects on her own experience with Pandas. She recalls that merging DataFrames was a key concept, but cannot remember the full syntax. 
% Jane reflects on her experience working with Pandas and recalls that merging DataFrames is a key operation when working with big data.
She switches to the full programs tab (\cref{fig:jane-workflow}, 6) that includes complete code examples and searches (\cref{fig:jane-workflow}, 7) for ``\texttt{.merge}'' to find different instances of merging operations. % and tries to use the search bar to find a relevant program that contains ``.merge''. 
After finding a comprehensive example, she selects the relevant section of the code and creates a plan from it.
% She again selects a part of the example, creates plan from the selection, and refines it. She engages with the system iteratively and designs twenty plans for her lecture. 

After designing a set of plans that capture the important topics, she organizes them into groups (\cref{fig:jane-workflow}, 8) 
% also grouped similar plans together
to emphasize sets of plans with similar goals but different implementations. For instance, she takes her plans about \texttt{.merge} and \texttt{.concat} and groups them together to form a category of plans that students can reference when they want to {combine data from different sources}.

% combining data using ``merge'' or ``concat''.

% the the she used plans isn't very good right now
% She exports these plans and starts preparing her lecture slides, using the plans as a way of presenting key concepts to students with minimal programming experience.
She exports these plans to support her students with minimal programming experience by preparing lecture slides that organize the sections around plan goals, using plan solutions as worked examples in class, and providing students with cheat sheets that include relevant plans for their laboratory sessions.
% using the plan goals as titles for different sections of her slides, and using the solutions as references for the examples she creates. Finally, she makes a PDF cheatsheet with all the plans for students to reference during the week's laboratory.
% The next day, she starts preparing her lecture slides and realizes that the names and goals she wrote for her plans represent key concepts in Pandas. She references the plans she created to design annotated examples that she includes on her lecture slides.

%% How does Jane actually use the plans? 
%% > Important to be careful to note that this isn't actually part of the system....
%% > She uses the generated plans to (a) as inspiration for worked examples in teh course, (b) as stems for questions that test how code should be completed
%% > She notices she now has a list of key concepts in the area


\begin{figure*}[h]
        \Description{An annotated screenshot of PLAID's `Programs' view. On the left, a list of use cases such as `Renaming columns in a Frame' and `Plotting a histogram of a column' is shown, with a scrollable list and a search bar. The latter one is selected, and on the right, we see the contents of the program in a monospaced font, with four buttons explained in the caption.}
        \includegraphics[width=\textwidth]{img/system-diagram-1-fixed.png}
        \caption{Plan Identification using PLAID: (a) list of example programs for instructors organized by natural language descriptions, (b) list of full programs of code, (c) search bar enabling easy navigation of given content to find code for specific ideas, (d) button to create a plan using the selected part of the code, (e) button to create a plan using the complete example program, (f) button to view an explanation for a selected code snippet, and (g) button for executing the selected code.}
        \label{fig:system-diagram-1}
\end{figure*}

\subsection{System Design}

At a high level, PLAID\footnote{The code for PLAID can be found at: https://github.com/yosheejain/plaid-interface.} operates on two subsystems: (1) a database of LLM-generated reference materials created through a pipeline that uses \edit{OpenAI's GPT-4o\footnote{https://openai.com/index/hello-gpt-4o/}~\cite{achiam2023gpt}}, inspired by instructors' best practices for identifying programming plans (see ~\cref{fig:system-pipeline})
%LLM for identifying plans in application-focused domains 
and (2) an interface that allows instructors to browse reference materials for relevant code snippets 
% and other plan components to achieve a goal that meets their needs. Then, they refine the candidates to mine plans 
and refine suggested content into programming plans
(see Figures~\ref{fig:system-diagram-1} and~\ref{fig:system-diagram-2}).
% In this section, we describe the implementation of the pipeline generating the reference materials and the key interface features of PLAID.



\subsubsection{Database of Reference Materials for Application-Focused Domains}

PLAID extracts information from reference materials at three levels of granularity to support each instructor's unique workflow: complete programs that address a particular use case, annotated program snippets that include goals and changeable areas, and plan candidates that cluster relevant program snippets together.

\textbf{Generating complete example programs.}
The content at the lowest level of granularity in the PLAID database are the complete programs. 
%These candidate plans were generated using a pipeline to generate \textit{plan-ful examples}, which we define as examples of programming plans in use, with all plan components identified (see Section~\ref{sec:components}). This implementation had three stages: (1) generating in-domain programs, (2) segmenting programs into plan-ful examples, and (3) clustering plan-ful examples into plans. 
\label{sec:llm-pipeline}
% \begin{figure}
% \centering
% % \includegraphics[width=0.5\textwidth]{img/pipeline-new.png}
% \includegraphics[width=\textwidth]{img/new-plan-pipeline.png}
% \caption{The three stage process for generating example programs, segmenting them with plan components, and clustering these plan-ful examples.
% %collecting and processing responses from ChatGPT into plan-ful examples}
% %\caption{The pipeline for LLM plan generation.}
% }
% \label{fig:llm-methods}
% \end{figure}
% \subsubsection{Generating In-Domain Programs}
% Informed by the insights identified in our interview study, we generated programming plans relevant to an application-focused domain: web scraping via BeautifulSoup. We utilized an LLM-based approach to generate these plans with the GPT-4 model from OpenAI using its publicly available API in an iterative workflow. 
% Our participants examined example programs and conducted literature reviews (Section \ref{sec:viewing-programs}) as key parts of their plan identification process. 
As these examples should capture a variance of use cases in the real world, we utilized an LLM trained on a large corpus of computer programs and natural language descriptions~\cite{liu2023isyourcode}.
% Inspired by this, we used Open AI's GPT-4, a state-of-the-art large language model for code generation that is trained on a large corpus of computer programs~\cite{liu2023isyourcode},
% to generate candidate programs along with its respective plan components in the programs.
We prompted\footnote{Full prompts can be found in \cref{sec:appendix-pipeline}.} the model to generate \texttt{specific use cases of <application-focused library>}, defining use case as \texttt{a task you can achieve 
with the given library} (see \cref{sec:use_case_prompt}). Subsequently, we prompted the model to \texttt{write code to do the following: <use case>}, producing a set of 100 example programs with associated tasks (see \cref{sec:code_prompt}). By generating the use cases first and generating the solution later, we avoided the problems with context windows of LLMs where the earlier input might get `forgotten', resulting in the model producing the same output repeatedly. For practical purposes, we generated 100 programs per domain. \edit{To test for potential ``hallucinations'' where the LLM generates plausible yet incorrect code~\cite{Ji_2023_hallucination}, we checked the syntactic validity of the generated programs before developing the rest of our pipeline. No more than one out of 100 generated programs included syntax errors in each of our domains, i.e., Pandas, Django, and PyTorch. Thus, we concluded that hallucinations are not a major threat to the code generation aspect of PLAID.}
%while hallucinations in LLMs are a pressing concern for systems that utilize these models,
% This collection of example programs (which we refer to as 
%dataset 
% $\mathcal{D}$) was used as our primary dataset for further analysis.

\textbf{Generating annotated program snippets.}
% \subsubsection{Segmenting Programs Into Plan-ful Examples}
% We then proceed to compile these examples with each of the plan components generated using ChatGPT. We construct a new dataset with these components, Dataset \((\mathcal{D}^{\textit{Comp}})\).
The second level of granularity in PLAID consists of small program snippets and a goal, with changeable areas annotated. 
We used the generated programs from
% \mathcal{D}$
the prior step as the input to the LLM to add subgoal labels, where we prompted the LLM to annotate subgoals (see \cref{sec:subgoals_prompt}) as comments that describe \texttt{small chunks of code that achieve a task that can be explained in natural language}. These subgoal labels were used to break the full program into shorter snippets. Each snippet was fed back to the model to generate changeable areas (see \cref{sec:ca_prompt}), defined in the prompt as \texttt{parts of the idiom that would change when it is used in different scenarios}. The subgoal label that explained a code snippet corresponded to its goal in the plan view and the list of elements assigned as changeable was used for annotations.
% (see Stage 2 in Figure~\ref{fig:llm-methods}),
%We fragmented these generated programs into smaller code pieces by generating \textit{subgoals} in the program. Then, each goal (Section \ref{sec:goal}) and the accompanying code solution (Section \ref{sec:solution}) were added as a single unit of data in our plan-ful example dataset of components, \(\mathcal{D}^{\textit{Plan-ful}}\). For each of these datapoints, we prompted the model to identify \textit{changeable areas} (Section \ref{sec:changeable}). %The name (Section \ref{sec:name}) was determined later in the pipeline (Stage 2 in Figure \ref{fig:llm-methods}).


% From the results of our qualitative study, we now know about the parts of a programming plan. In order to extract these plans automatically, we used ChatGPT. We accessed it using its publicly available API and we used the GPT-4 model. We selected 3 domains that are interesting for non-majors. This included . 

% For each of these domains, we first asked the LLM to generate 100 use cases. We then re-prompted it with the use cases it generated and asked it to generate code that would be written to accomplish that use case.
% potential for another table?
% add code metrics from stackoverflow github work for chatgpt
% With all these code pieces collected, we then asked ChatGPT to generate each of the plan parts one-by-one.

% \subsubsection*{Extracting Goals and Solutions}Generated programs 
% in \(\mathcal{D}\) 
% typically included a comment before each line, which described that line's functionality. However, these comments did not capture the high-level purpose of the code, as required by a plan goal. To generate more abstract goals for a piece of code, we defined subgoals as \texttt{short descriptions of small pieces of code that do something meaningful} in a prompt and asked the LLM to \texttt{highlight subgoals as comments in the code.} %In our query, we also added the way we define subgoals to provide the relevant context to the model. Specifically, we wrote that 
% The output from this prompt was a modified version of each program
% from \(\mathcal{D}\), 
% where blocks of code are preceded by a comment describing the goal of that block. % of code. % instead of restating functionality. 

% We split each complete program into multiple segments based on these new comments. Thus, the subgoal comments from each complete program I
% n the modified \(\mathcal{D}\) 
% became a plan goal, and the code following that comment became the associated solution. %, collected in \(\mathcal{D}^{\textit{Plan-ful}}\). % After it returned the annotated code piece, we extracted the comment and the following lines of code before the next comment. This pair acted as a subgoal-code piece. We collected all such pairs across all use cases from \(\mathcal{D}\) and added them to \(\mathcal{D}^{\textit{Plan-ful}}\).
% Each goal 
% %(Section \ref{sec:goal}) 
% and solution pair
% %(Section \ref{sec:solution}) 
% was added as a single unit of data in our plan-ful example dataset.
% , \(\mathcal{D}^{\textit{Plan-ful}}\).

% \subsubsection*{Extracting Changeable Areas}To annotate the changeable areas for a plan, we defined changeable areas as \texttt{parts of the plan that would change when it is used in a different context} in our prompt and asked the model to \texttt{return the exact part of the code from the line that would change} for all code pieces from the dataset with plan-ful examples.
% from \(\mathcal{D}^{\textit{Plan-ful}}\). 
% This data was added to \(\mathcal{D}^{\textit{Plan-ful}}\).

% to-do
% \subsubsection{Clustering Plan-ful Examples into Plans}
\textbf{Generating clustered plan candidates.}
\label{sec:clustering}
% We perform k-means clustering on the plans \(\mathcal{D}^{\textit{Plan-ful}}\) to identify clusters of similar code pieces and thus, programming plans.
The highest level of granularity provided in PLAID
%presents users with 
are
plan candidates, in the form of clusters of annotated program snippets. To compare the similarity of program snippets, we used CodeBERT embeddings following prior work~\cite{codebert} and applied Principal Component Analysis (PCA) \cite{PCAanalysis} to reduce the dimensionality of the embedding while preserving 90\% of the variance. The snippets were clustered using the K-means algorithm~\cite{kmeansclustering}, using the mean silhouette coefficient for determining optimal K~\cite{silhouettecoeff}. Each cluster is treated as a plan candidate, with the goal, code, and changeable areas from each program snippet in the cluster presented as a suggested value for the plan attributes.
% We used a clustering algorithm to group similar program snippets 
% plan-ful examples together as a programming plan. For clustering the code pieces, we used the CodeBERT model from Microsoft \cite{codebert} to obtain embeddings for each code piece in our dataset of plan-ful examples
% % in \(\mathcal{D}^{\textit{Plan-ful}}\) 
% and applied Principal Component Analysis (PCA) \cite{PCAanalysis} to reduce the dimensionality of the embedding vectors while preserving 90\% of the variance. These embeddings were clustered using the K-means algorithm~\cite{kmeansclustering}. The optimal number of clusters \(\mathcal{K}\) was determined by assessing all possible \(\mathcal{K}\) values 
% % (where \(\mathcal{K} \in [2, \texttt{length}(\mathcal{D}^{\textit{Plan-ful}})]\))
% using the mean silhouette coefficient \cite{silhouettecoeff}. We assigned each example 
% % in \(\mathcal{D}^{\textit{Plan-ful}}\) 
% to a cluster of similar code pieces. 
% \subsubsection*{Extracting Names}
For each plan candidate, a name (see \cref{sec:name_prompt}) that summarizes all snippets in the cluster was generated by prompting an LLM with the contents of the snippets and stating that it should generate \texttt{a name that reflects the code's purpose} and it should focus on \texttt{what the code is achieving and not the context}. 
% Then, all code snippets from each cluster of examples were provided as input to the LLM along with a prompt asking it to \texttt{devise a name for that cluster of plans}.

% \subsection{Interface for Refining Candidate Plans}

% %nd the back-end server relied on routes written in Flask. The domain-specific candidate plans suggested to the user are queried from the database of candidate plans generated using the LLM. Each participant was required to log in to the web page using their unique credentials, which allowed us to record their activity for analysis. While the complete details of our implementation of the web-based application are out of scope for this paper, we describe its main features in Section~\ref{sec:implementation_of_webinterface}.

% \subsubsection{\edit{Preliminary Technical Evaluation of Generated Content}}

% \edit{syntactic validity and standard code complexity metrics to determine
% their suitability for novices}


\begin{figure*}[h]
    \Description{An annotated screenshot of PLAID's Plan Creation view with three panes, with plans shown as boxes on the left. A plan is highlighted, and we see its components on the middle pane. On the rightmost pane, we see suggested values for the selected component.}
    \includegraphics[width=\textwidth]{img/system-diagram-2-new.png}        
    \caption{Plan Identification using PLAID: (h) button that suggests a domain-specific candidate plan from the system database, (i) pane enabling viewing of similar values for the selected plan component, (j) button to view the solution code as part of a complete program, (k) pane with a structured template for plan design with editable fields to refine plan components, (l) button to copy a selected plan, (m) button to mark snippets of code from the plan solution as changeable areas, and (n) a button to group plans together into a category and add a name.}
    \label{fig:system-diagram-2}
\end{figure*}

% \subsubsection{Key Characteristics}
% PLAID supports the process of plan identification in data processing with Pandas, machine learning with Pytorch, web development using Django, and web scraping using BeautifulSoup. 

\subsubsection{Interface for Designing Programming Plans}
Building on the 
%characteristics addressed in the artifact (Section~\ref{sec:design-artifact}) and 
design goals identified in the design workshop (\cref{sec:design-goals}), PLAID enables a set of key interactions to assist instructors in refining candidates to design plans for their instruction. 



\textbf{Interactions for Initial Plan Identification.}
% Initial Plan Identification with Quick Exploration of Many Authentic Programs
While instructors valued the availability of code examples in the design workshop (Section~\ref{sec:design-workshop-findings}), we observed many opportunities for scaffolding their interaction with the reference material. To this end, PLAID presents example programs in two different views \textbf{(DG1)}. 
% We saw instructors scanning examples, selecting desired code pieces, and copying them over into their plan templates in all conditions in the study. 
The ``Programs (Organized by Use Case)'' (\cref{fig:system-diagram-1}a) tab includes a list of use cases where instructors can click on an item to expand the program for that use case.
The ``Programs (Full Text)''  tab (\cref{fig:system-diagram-1}b) lists all the programs and enables instructors to scroll or search through (\cref{fig:system-diagram-1}c) all the code at once.
% presents the contents of all the programs expanded viewing a list of complete code examples, allowing instructors to look at materials they would typically search for when designing plans.
% equipping instructors with full-code programs organized in a list of short natural language descriptions of common use cases in their domain of expertise. 
Both views support directly creating a plan from the whole example (\cref{fig:system-diagram-1}e), or a selected part of it (\cref{fig:system-diagram-1}d), by copying the solution and the goal of the program into an empty plan template
% < Highlight code in full code and code pane in tab1 and make a plan (D1)
% < Add a button to add full program as a plan too (D1)
further supporting efficient use of the material \textbf{(DG3)}.
% This interaction copies over the selected code and its respective use case into the solution and name fields, respectively. 
% < Code explanation plugin for strange syntax (GPT) (D2)

To facilitate understanding unfamiliar code and syntax, we implemented a ``View Explanation'' button (\textbf{DG2}) that generates a short description of the selected line(s) of code by prompting an LLM (\cref{fig:system-diagram-1}f). 
% In this case, participants hesitated to use the suggested syntax in their plans because its functionality was unclear to them. PLAID supports a button named ``View Explanation'' where the user can select a method, function, or line of code that is unclear and click on it to understand its working \textbf{(D2)}. 
Participants also looked for code execution to validate and understand a program. However, since the code snippets instructors work with are often incomplete in this task, we implemented a ``Run Code'' feature (\textbf{DG2}) that predicts the output of a selected code snippet by prompting an LLM to walk through the code \texttt{step by step}, using Chain-of-Thought prompting~\cite{wei2022chain} (\cref{fig:system-diagram-1}g). Only the predicted output for the code is presented, ignoring other output from the LLM.

% to examine the code behavior and thus mitigate the challenge of being faced with unfamiliar syntax. Thus, using PLAID, instructors are able to run complete programs to view their output \textbf{(D2)}.
% < Search in the use cases (and full progs) (D3)
% Frequently, instructors relied on their expertise and experience to formulate ideas about goals for which they wanted to create plans. While interacting with condition C in the design workshop, interviewees suggested including a mechanism to search for specific keywords within code and  its natural language description. To facilitate the instructor-LLM collaboration, allowing users to find examples implementing their ideas, PLAID includes a search bar that helps users navigate the given use cases, complete programs, and effectively find specific examples they may be looking for \textbf{(D3)}.

\textbf{Interactions for Plan Refinement.}
% Support Plan Refinement with Comparisons of content
% Participants indicated difficulty mining plans from code examples (Section~\ref{sec:challenges_practice}). 
To provide suggestions for code patterns common enough to be potential programming plans,
%To alleviate challenges in identifying content common enough for designing plans, 
we utilize the clustered program snippets from our database. In the ``Plan Creation'' view of PLAID, instructors can ask for suggestions (\cref{fig:system-diagram-2}h) to see a candidate plan to refine (\textbf{DG3}).  \edit{This functionality allows instructors to draw on their experience to recognize common code snippets and decide if they are valuable to teach students.}
% If instructors want to demonstrate their plan as part of a complete code example, they can review these examples reducing the effort that they would need to put in to recall syntax and construct a complete example. 
\edit{This promotes recognition over recall \cite{recognition_over_recall}, thus helping reduce the cognitive effort that instructors may have to put in while designing programming plans traditionally.}
To allow instructors to better understand the context of a plan under refinement, PLAID 
also includes a button for searching for the current solution within the entire set of full programs
%, showing the code snippet in context 
%as part of a complete example
(\textbf{DG3}, \cref{fig:system-diagram-2}j).
% < Keyword search/embedding filter for potential values (D1)

As instructors edit the components of a plan, they are shown similar values from the corresponding component in that cluster (\cref{fig:system-diagram-2}i). By clicking on any suggested value, instructors can replace a plan component with a suggestion that better captures that aspect of the plan \textbf{(DG1)}. \edit{By allowing instructors to view the plan they are working on along with other related code pieces in a split screen view, we promote instructor efficiency by reducing the split-attention effect \cite{tarmizi1988guidance}. In the current plan creation process, even when using LLMs from their chat interface, instructors would have to switch between windows with code examples and their text editor which may increase the load on the instructors' working memory \cite{clark2023learning}. In PLAID, instructors can edit their plans and view similar code pieces at the same time.}

% \edit{By enabling these interactions and thus organizing ``knowledge in the world'' effectively, PLAID reduces the need for instructors to store and retrieve the ``knowledge in their head'' \cite{Norman_DOET}. Thus, PLAID optimizes the plan creation process by allowing efficient search within the ``knowledge in the world'' and reducing the cognitive load while storing and retrieving ``knowledge in the head'', minimizing the total effort required \cite{} by instructors.}
% after searching its code corpus for similar examples using a keyword search \textbf{(DG1)}.
% < Show use case button in solution (add highlighting) (D2)
% To help instructors easily consider the context of a plan as they refine it, PLAID 
% In the design workshop, few instructors emphasized the importance of presenting worked and contextualized examples to students. 

% ‘go to a use case’ button that redirects the user to the tab with full code programs and highlights the plan as part of a complete example \textbf{(D2)}.

\textbf{Interactions for Building Robust and Shareable Plan Descriptions.}
% Support robust/sharable plan descriptions
% From Section~\ref{sec:process_intro_plan_design}, instructors indicated drawing on their experience in the application-specific domain and instructional expertise to think about how to best solve a problem. 
PLAID encourages instructors to design plans in a structured template (\cref{fig:system-diagram-2}k). Moreover, PLAID reinforces the plan template by providing a dedicated method for annotating changeable areas by highlighting any part of the code (\textbf{DG3}, \cref{fig:system-diagram-2}m). Instructors can further explain the changeable areas by adding comments as text.
% \edit{The structured template view of the plan encourages instructors to articulate their mental models of how the plan would generalize to other problems, allowing the transfer of ``knowledge in the head'' to ``knowledge in the world''.}

Our design workshop showed that participants would create a plan and copy it to emphasize alternatives or modifications to the underlying idea. To support this workflow,
% In our design workshop, participants created copies of their plans to display alternative solutions to achieve the same goal, emphasizing that multiple possible solutions in code could accomplish the same goal.
% < Duplicating plans (D3)
% To accelerate this process of teaching a variety of possible solutions, 
PLAID allows users to ``duplicate'' plans on the canvas and further edit them to present alternative solutions for the same plan \textbf{(DG3}, \cref{fig:system-diagram-2}l).
% Highlight text from solution to change it to changeable areas (highlighting code itself) (D4)

% In conditions A and B, instructors highlighted the changeable areas in the code itself.
% To allow participants to emphasize the changeable areas in code in PLAID, we implemented the ``add to changeable areas'' button. After selecting the changeable piece of code, clicking on this button highlights the text in a different color and adds it to the box of changeable areas to complete the templated plan design (\textbf{D4}).
% Grouping plans into categories (D4)
% < Multiple selection of the boxes (D4)
% < Naming groups of boxes (D4)
To encourage instructors to think about organizing plans in ways that they would present them to students, PLAID provides an open canvas view for instructors that allows them to arrange plans as they prefer. In addition, PLAID supports a ``grouping'' feature (\cref{fig:system-diagram-2}n), which allows instructors to combine plans with similar goals together into one category (\textbf{DG4}).

% A handful of users postulated each plan as an example question that can be used on assessments. They intended to create multiple variants of the same question for students. They suggested that being able to visualize the different categories would be helpful. Using PLAID, users can select multiple patterns together, add them to a group, and name the group \textbf{(D4)}.  % :(

\subsubsection{System Architecture}
The pipeline to create reference materials is implemented in Python, using the state-of-the-art large language model GPT-4o (Model Version: 2024-05-13). The interface for PLAID is implemented as a web application in Python as a Flask webserver, with an SQLite database. The user-facing interface is implemented using HTML, CSS, and JavaScript, with the canvas interactions realized with the library `\textit{interact.js}'. 





\section{WhatELSE: System Design and Implementation}
% \begin{figure*}[t]
% \centering
% \includegraphics[width=1.0\textwidth]{figures/WhatELSE.pdf}
% \vspace{-10pt}
% \caption{An overview of the system.}
% ~\label{overview}
% \vspace{-10pt}
% \end{figure*}

In this section, we present the interface and features of {\sc WhatELSE}, describe its technical pipeline to facilitate the transformation between narrative instances and narrative outlines, and demonstrate its workflow with an example user story.

\subsection{Narrative Space Editor Interface}

{\sc WhatELSE} system assists the user in creating a narrative space. The user can upload narrative examples in text file(s). \presentation{In addition, the user uses a sentence to describe a story's moral (e.g., {\it ``kindness is never wasted''}). The system uses the story input to construct an initial version of the narrative space. The user can edit this narrative space using the interface.}
%\subsubsection{Three Views of the Narrative Space} \label{views}
 {\sc WhatELSE} features three views for the user to perceive the narrative space: \textbf{Pivot View}, \textbf{Outline View}, and \textbf{Variants View} (Figure~\ref{overview}). 
 %Figure. \ref{overview} illustrated how each view is presented on panels in the narrative space editor. 

\noindent \presentation{
\textbf{Pivot View}\hspace{1mm} The pivot view shows a pivot narrative instance. A {\em pivot (narrative instance)} is a user-defined narrative instance, considered as a representative instance in the narrative space. By default, the user's input is automatically marked as the pivot. The pivot serves as a point of reference as the user edits the narrative space.}

\noindent\textbf{Outline View} \presentation{An {\em outline} is an abstract specification of a sequence of events defining the narrative space. }\originality{Similarly to ``loglines'' \cite{mirowski2023co}, it specifically describes the general structure of the narrative with a sequence of high-level events }\presentation{ - e.g., {\it ``A small creature runs into an accident. It was then saved by another creature''}.} The outline describes the narrative space from a broader perspective by capturing the commonality across all the narrative instances in the narrative space. It represents the most abstract manifestation of the author's narrative intent, thus defining the boundary of the narrative space.

\noindent\textbf{Variants View}\hspace{1mm} A {\em variant (narrative instance)} is a narrative instance residing in the current narrative space. A variant instantiates the outline with a sequence of concrete events - e.g., {\it "An ant fell into water. A dove dropped a leaf next to the ant. The ant climbed on the leaf. The ant was saved."} Each abstract event in the outline is expanded to multiple concrete events in a variant. 


The variants are displayed in an interactive scatter plot along two dimensions to help users understand the shape of the narrative space: 1) the \textbf{authorial intent} dimension, measured by the distance between the moral expressed by the variant and by the pivot (ranging from 0 to 1)\footnote{This distance is evaluated by prompting the LLM to assess how well the moral is conveyed}, and 2) the \textbf{emergence} dimension, measured by how much the plot progress in the variant deviates from the pivot (ranging from 0 to 1). \originality{These two dimensions are inspired by the ``authorial intent'' dimension in Riedl's taxonomy of IN approaches \cite{riedl2013interactive,riedl2009incorporating}, as well as the notion of ``emergence'' \cite{walsh2011emergent} and ``interactivity'' \cite{stang2019action} from prior IN research.} Users can configure the number of variants to be generated for visualization. Users can click on any variant in the visualization to display its detailed content, allowing them to compare it with the pivot. \presentation{Users can also use a scroll bar to visualize the plot progression} as they develop across different stages, allowing them to perceive how the narrative variants evolve over time and deviate from the pivot.

%In practice, users generate this scatter plot by clicking on the \textit{Generate Variants} button, as shown Figure. \ref{user_study:interface} C. Additionally, 


%, ranging from 1 to 5 sets. For each set, we generate 3 variants corresponding to the variant plots driven by the three types of players. For example, if a user chooses to visualize 4 sets, a total of 12 variants—categorized by player type—will be displayed on the scatter plot as shown in Figure. \ref{user_study:interface}. 


%Variant plots are a sequence of plots that players could experience under this narrative space constrained by the outline. Each variant consists of two parts: the game plot, and the player action. Player actions drive the gradual unfolding of game plots from the narrative space, and connect gameplots as a narrative. In other words, one variant records the full gaming experience of a specific player. Thus, such variation is caused by different game plots and different player actions jointly. For instance, the same game plot describing a character facing danger, can be led by two different player actions "save" and "kill" can lead to two variants; similarly, with the same player action "save", the character to be saved in game plot to be saved will also lead to two variants. 



%which is formatted from the user's input narrative instance, by extracting all events that occurred in the story and arranging the events sequentially as a plot. 
%Specifically, following the principles in defining events in digital games ~\cite{gould2011narrative,castellan2017games},  each event is a combination of "subject + action + object + potential location". 
%The view of the pivot plot shows the backbone of the user's input narrative instance, providing a straightforward structure of the narrative progression. More importantly, the pivot plot lies in the centric position of the narrative space. Narratives unfolded from the narrative space should always refer to this pivot plot. 

% Similar to how "acts" function in drama writing~\cite{styan1960elements},


% Therefore, the outline plot condenses the events in narrative instances into a less specific form compared to the detailed events, such as the pivot plot

These three views provide different perspectives for users to inspect the narrative space. We also provide editing tools at each view to support shaping the narrative space in different ways. 

%and are closely connected to each other. The outline is an abstraction of the pivot plot, and variant plots are generated within boundaries defined by the plot outline. Meanwhile, each variant plot serves as an alternative to the pivot plot. Therefore, the three views employed jointly describe the narrative space. 




\begin{figure*}[t]
\centering
\includegraphics[width=0.99\textwidth]{figures/whatelse_interface_v3.pdf}
\vspace{-10pt}
\caption{\presentation{An illustration of the Narrative Space Editor interface, including the pivot, outline, and variants view. Users can (A) generate outline from pivot or variants with an abstraction ladder to configure the abstraction level. They can (B) fine-tune sentence or word-level abstraction using an abstraction tooltip. They can also (C) generate variants from outline specifying the number of variants in the variants view. They can use (D) narrative progression slider to visualize the variants' dynamic distance from the pivot (star). }}
~\label{overview}
\vspace{-10pt}
\end{figure*}




\subsubsection{Support Editing the Narrative Space} 

\presentation{The system provides editing tools at outline and instance level.}

\noindent\textbf{Outline Editing}\hspace{1mm} \presentation{Users can constrain or relax the boundary of the narrative space by adjusting the outline's level of abstraction.}  The more abstract the outline is, the less constrained the narrative space is. \presentation{For example, {\it ``a small creature got into an accident''} is more abstract than {\it ``the ant fell into water''}, enabling more possible narrative instances to be generated. The former removes the constraint on {\it ``the small creature''} being  {\it ``the ant''}, and the {\it ``accident''} being {\it ``falling into water''}.} A less constrained narrative space allows stronger player agency but follows a looser authorial structure. Outline editing allows the user to tune the narrative space to reach a desired balance between authorial structure and player agency. \presentation{The system provides two tools to support the abstraction editing}.

\begin{itemize}
    \item \textbf{Abstraction Ladder (Figure. \ref{overview}.A)} The abstraction ladder helps the user to shift the global level of abstraction across the events in the outline. Inspired by theories of narrative structure ~\cite{styan1960elements, mckee1997story}, \presentation{this ladder covers a spectrum of abstraction levels (beat, scene, sequence, act, and story level)}. An outline at the beat level is similar to a narrative instance, while an outline at the story level summarizes the plot into a one-line overview. Between the two ends, each level of abstraction is progressively more abstract than the previous level. For instance, a scene-level outline provides detailed descriptions of specific scenes, including characters, actions, objects, etc: \textit{``The kind dove takes a leaf to reach the ant and drags it out of a water bubble.''} An act-level outline offers a highly summarized view of the narrative, focusing on the turning points: \textit{``A character saves their friend from danger.''}
    \item \textbf{Abstraction Tooltip (Figure. \ref{overview}.B)} The abstraction tooltip \presentation{allows the user to adjust the sentence, phrase, or word-level abstraction in a more fine-grained manner.} Practically, when users select a text snippet in their outline plots, the tooltip appears, offering two options: ``More Abstract'' and ``More Concrete''. \presentation{By clicking the button, users receive suggested edits that replace the selected content with a more abstract or more concrete phrase.} While the abstraction ladder provides global control over the entire outline, the tooltip enables more fine-grained adjustments at the word or phrase level. The suggestion of making the selected content more abstract or more concrete is based on the taxonomy in linguistics~\cite{hayes1983cognitive}. For example, {\it ``character-animal-small animal-cat-tabby cat''} constructs a linguistic hierarchy. Given a selected text snippet {\it ``cat''}, requesting a more abstract suggestion would yield its superordinate term {\it ``small animal''} or {\it ``animal''}, while a more concrete suggestion would provide its subordinate {\it ``tabby cat''}. 
\end{itemize}
Once the user is satisfied with the outline, they can click the ``Generate Variants'' button to generate narrative variants in the Variant View. Section \ref{compiler} describes the technical pipeline for generating narrative instances from outline. 

\noindent\textbf{Instance Editing}\hspace{1mm} \presentation{Users can fine-tune the narrative space by editing the instance-level content in Pivot and Variant View. They can select a variant to set or unset it as the pivot. They can also remove a variant from the narrative space or add it back. Finally, they can directly edit the text in the instances. They can click the ``Generate Outline'' button to update the outline based on their edited variants. For example, a user who does not want to include certain player type may choose to remove all variants by that player type and update the outline. Section \ref{compiler} describes different player types in the player proxy model. }

%Editing operation on the instance level allows the user to fine-tune the narrative space. We support the following editing operations on the instances: 1) direct text editing, 2) setting or unsetting as pivot, and 3) removing from or adding back to the narrative space. Once the user has done editing at the narrative instance level, they can click the ``generate outline'' button to synchronize the changes to the narrative space to the outline view. Section \ref{summarizer} describes the technical pipeline for generating outlines from narrative instances. 


%With the three views on the narrative space, we provide users with an intuitive way to perceive the narrative space. We then provide a series of tools as follows targeting \textbf{DG2}, to help users effectively shape the narrative space by configuring the level of abstraction. 





%Our first tool helps users obtain an ideal outline plot based on narrative instances. 
%Specifically, users can initiate the abstraction process by clicking the \textit{Generate Outline} button (i.e., Figure. \ref{user_study:interface} A). To enable customization of this abstraction process, we offer a feature named the "abstraction ladder." The abstraction ladder provides users with options of abstraction levels that they can choose from to generate the outline.



%To implement the abstraction ladder, we designed a prompt pipeline to help users ini outlines based on the defined abstraction levels. The pipeline uses professional drama writing knowledge as prior, and generates outlines via summarizing the commonalities among narrative instances. We present the detailed design of the prompt pipeline in the later Section ~\ref{sec:technical_pipeline}.


% We first provide knowledge of defined levels of abstraction referring to professional drama writing literature in the prompt. We then implemented a prompt chain, structured as a tree of thought. This chain operates in three stages: (1) Given the pivot plot, the first part of the chain prompts the LLM to generate a series of variations of the plot by creatively rearranging the elements involved. (2) The second stage generates three outlines corresponding to the predefined abstraction levels: scene, sequence, and act level. (3) In the final stage, the outline is tailored according to the user's specific requirements.



%\noindent\textbf{Variants-driven Editing} In addition to visualizing the variant plots, we designed tools to help users edit the narrative space by selecting among plot variants.  Additionally, users can refine the set of variants by removing selected variants from the set. For example, users may reject all narrative paths driven by negative players if they are unsatisfied with the narrative direction in the variants. Once users are satisfied with the remaining variants, they can use the \textit{Generate Outline} button to generate an outline based on the variants by summarizing their commonalities. Similarly, users can also use the abstraction ladder and customize the outline generation based on their specific needs.


%Overall, we provide a suite of tools that users can utilize to continuously shape the narrative space by editing across three intuitive representations. Specifically, we leverage the concept of abstraction to allow users to set appropriate boundaries within the narrative space, offering flexibility and control over how the narrative evolves and adapts to player actions.


\subsection{Technical Pipeline}

% \zl{todo, review 4.2}

\label{sec:technical_pipeline}
This section describes our technical pipeline supporting the features described in the above section, focusing on the transformation between narrative outline and narrative instances. Specifically, we employ the GPT-4o ~\cite{openai2023chatgpt} for the implementation of our system.


%To support the features we stated above, we design the technical pipeline of \textsc{WhatELSE} focusing on leveraging LLMs to facilitate two key processes: transforming narrative instances into outlines through a prompt pipeline, and converting outlines back into narrative instances using LLM-based narrative planning. We introduce the design and implementation of these two processes in the pipeline with the following two sections.


\subsubsection{Transforming Narrative Instances to Outline} \label{summarizer}
We use an LLM prompting pipeline to generate outlines from narrative instances (Figure \ref{system_overview}.1). This pipeline first prompts the LLM with domain knowledge in drama writing, providing the context of the story domain and the narrative instances. The pipeline then prompts the LLM to summarize the commonalities across these narrative instances, generating outlines at different abstraction levels based on story structure principles \cite{mckee1997story}. Finally, the system selects an outline according to the user's chosen level of abstraction.

\subsubsection{Transforming Outline to Narrative Instances} \label{compiler}

% \presentation{[delete me later: QZ revision for better readability, better structure the text for R1 R3]}
\presentation{To generate meaningful events that can react to player actions (DG3), we go beyond text generation and integrate an LLM-based narrative planning approach with character simulation and player proxy models. Our method extends \textit{StoryVerse} \cite{wang2024storyverse} with player interactivity and behavior modeling.}
 \presentation{Generating narrative instances from outline is essentially simulating an interactive story generation process, where player actions may be generated by computational proxies of players, and the story generated grounded in the causal changes of game world states in accordance with the game mechanism.} 
%To ground the outline with the concrete plot in order to enable a better understanding of the narrative space (DG1), and generate meaningful events that react to player actions (DG3), we go beyond text generation and develop a novel LLM-based narrative planning approach \revision{with character simulation and player proxy models} (Figure.\ref{system_overview}.2).


\begin{figure*}[t]
\centering
\includegraphics[width=\textwidth]{figures/system_overview.png}
\vspace{-10pt}
\caption{\presentation{An overview of the technical pipeline of \textsc{WhatELSE}. (1) The system transforms narrative instances to an outline using the LLM to summarize their commonalities, generate outlines at different levels of abstraction, and review the outline based on user specifications in the Abstraction Ladder. (2) The Interactive Narrative Compiler unfolds the outline into (3) a sequence of character actions to act out the events in the outline. (4) The Game Environment executes the actions and
updates the world states. (5) The player (or a simulated player) can interfere with the game by changing the world states. Finally, the Game Environment sends the updated world states and outline back to the compiler for the next iteration. }}
~\label{system_overview}
\vspace{-10pt}
\end{figure*}

To explain this process, we assume a {\em Game Environment} \presentation{(Figure~\ref{system_overview}.4) is given, which contains the {\em Story Domain} and maintains the {\em World State}.} The {\em World State} consists of a collection of variables that hold relevant values for the game mechanics, such as the characters’ attributes (e.g., health points), current locations, and relationship scores, as well as their memories from the simulation. \presentation{ The {\em Story Domain} contains a set of characters, locations, and an action schema that specifies executable actions in the game system. These actions are implemented as executable function calls that modify the variables of {\em World State} accordingly. For example, executing the action $\texttt{kill(X)}$ will result in character $\texttt{X}$'s state to become dead. }

\presentation{The main game loop starts by sending an event from the outline to the Interactive Narrative Compiler (Figure~\ref{system_overview}.2) to instantiate a sequence of character actions (Figure~\ref{system_overview}.3). The Game Environment (Figure~\ref{system_overview}.4) executes the actions and updates the world states resulting from the generated character actions. Once the Game Environment executes the actions, the player (or a simulated player) can interfere with the game by changing the world states, such as saving a character (Figure~\ref{system_overview}.5). Finally, the Game Environment sends the updated world states and outline back to IN Compiler for the next iteration.} The process loops over the events in the outline plot, and stops when it exhausts all the events.

%The main game loop alternates between 3 modes: 1) plot orchestration mode, 2) player action mode, and 3) character simulation mode. The process loops over the events in the outline plot, and stops when it exhausts all the events.

\vspace{2mm} \presentation{\noindent \textbf{Plot Generator} \hspace{2mm} Given an event in the outline, the system generates a sequence of character actions that act out the event. It takes into account the current game world state as a result of all previous plot executions and player actions. }
%\noindent \textbf{Plot Orchestration Mode} \hspace{2mm} Given an event in the outline, we generate a sequence of character actions that acts out the event\revision{, taking into account the current game world state as a result of all previous plot execution and character/player interactions}. 
%\begin{enumerate}
%    \item \textbf{Generation} \hspace{1mm} An LLM is prompted to generate a sequence of character actions that acts out the event. The prompt includes the following information from the game environment:
%\begin{itemize}
%   \item the list of characters and their descriptions;
%   \item the action schema;
%   \item current world state (including character's memory).
%\end{itemize}
An LLM is prompted to generate a sequence of character actions that act out the event. The prompt includes the following information from the game environment:
\begin{itemize}
   \item the list of characters and their descriptions;
   \item the action schema;
   \item current world state (including character's memory).
\end{itemize}

\presentation{This process is very similar to narrative planning which generates a sequence of state transitions that leads to a narrative goal. Compared to classic symbolic narrative planning, our narrative goal may be fuzzier - sometimes it may not be rigidly characterizable by world states. For example, the narrative goal could be {\it ``everyone likes Bob''}, which corresponds to world states semantically in a fuzzy way. This narrative goal can also be any arbitrary statements describing the desired outcome, constraining not only the endings but also the transitions, for example, {\it``someone was careless and got into an accident''}. Therefore, we use an LLM-based method instead of existing symbolic narrative planning methods for transforming outlines into concrete plots.}


\vspace{2mm} \presentation{\noindent \textbf{Plot Reviewer}} \hspace{2mm} \presentation{Similar to symbolic planning problems,  the plot generation requires causal soundness. This means the character actions must be valid state transitions according to the game's causal rules. We thus adopt an LLM-based planning method following the LLM-Modulo framework \cite{kambhampati2024llms}, with a game environment simulating plans generated by LLMs and providing external critiques.}
To review the generated plan, an LLM is prompted to provide feedback regarding the quality and feasibility of the action sequence to improve it:
    \begin{itemize}
    \item \textbf{Overall Coherency Evaluation} Feedback is obtained by prompting an LLM to comment on the overall coherency of the generated plot and make suggestions for improvement.
    \item \textbf{Character Simulation Evaluation} For every action in the sequence, we prompt an LLM to play the role of the subject of the action. Given the current world state including the character’s memory, we ask the LLM if the motivation for the character to perform the action has been established. We include the explanation to this question in the feedback if the motivation has not been established.
    \end{itemize}
    In addition, we leverage a simulated Game Environment for external evaluation. The system evaluates the \textbf{Causal soundness} of the generated action sequence and reports the observations on the success/failure of the execution. The combined feedback is added to the prompt for the next round of generation.

\vspace{2mm} \presentation{For example, the event {\it ``a small creature gets into an accident''} could be turned into a sequence of character actions shown in Figure~\ref{system_overview}.3. Note that the events in the outline plot are at a higher abstraction level. This means the same event can be transformed into multiple character action sequences, leaving room for the exact plot to adapt to different world states \footnote{In the above example, if the dove is dead at the time of plot execution, then a different character action sequence will be generated - one possibility is that the ant fell into the water.}.
Once the final sequence of character actions is generated, it will be executed by the {\em Game Environment} to update the world state. }

\presentation{The Plot Generator and Reviewer create a sequence of character actions to act out the event. In between these events, the player or NPCs take free actions. These actions are driven by the LLM.
The player actions are determined \presentation{either by a real player's input or a simulated Player Proxy Model (Figure~\ref{system_overview}.5)}. }

\vspace{2mm}\noindent \presentation{\textbf{Player Proxy Model}} \hspace{2mm} %In this mode, the player is prompted to input one or more actions following the action schema. The system execute each of the actions to update the world state if the action is executable with the current world state.
When generating narrative variants, player actions are provided by an LLM-based player proxy model which operates based on player behavior classification derived from previous studies in digital games ~\cite{yannakakis2013player,worth2015dimensions}. Our player simulation incorporates three key player behavior models:

\begin{itemize}
    \item \textbf{Positive Players} in digital games contribute positively by following the intended game objectives and exhibit helping behaviors~\cite{velez2013helping,bostan2009player}.
    \item \textbf{Negative Players} are the killers identified in classic player modeling~\cite{majors2021some,hamari2014player}. They often exhibit aggressive behavior that disrupts the experience of others, particularly when they seek to dominate or harm others destructively.
    \item \textbf{Role Players} prioritize narrative immersion and character development by mimicking the actions their character would take in the gaming world~\cite{praetorius2020avatars}.    
\end{itemize}

Using these player models, we simulate a potential plot that could emerge from the interaction between game characters and simulated players within the narrative space defined by the outline. In this way, the system generates a diverse set of narrative instances in the variants view.

\vspace{2mm} \noindent \presentation{\textbf{Non-Player Character Simulation}} \hspace{2mm} 
An LLM is prompted to role-play as each of the NPCs and generate an action for this character. The prompt includes the following information:
\begin{itemize}
   \item the action schema;
   \item the list of characters and their descriptions;
   \item current world state (including character's memory);
\end{itemize}
Note that the character actions are not directly determined by the outline. However, it is affected by the current world state and, therefore, indirectly influenced by the executed events in the outline.

\vspace{2mm}
\presentation{Using this pipeline, \textsc{WhatELSE} creates a gameplay experience by unfolding the outline into narrative instances. The system generates the game plot for each event in the outline as a series of character actions. The player then inputs actions within the action schema, which influence the progression of the subsequent plot. The system runs executable actions to update the world state. After each round of player action, the system unfolds the next events until exhausting all the events in the outline, in this way, creating an interactive narrative experience. }

\subsection{Example Workflow}

% \zl{format: \textsc{WhatELSE}, cite, quote,  }


Below we present an example workflow to demonstrate some of the features described above. 
\usecase{Alice, a novice text-adventure game designer, wants to create a game based on the setting of a novel she enjoys.} 
Alice opens \textsc{WhatELSE}, along with a game engine preloaded with a story domain based on the novel.

\subsubsection{Encode Authorial Intent in Narrative Space} Alice starts with a rough draft of the story and a moral she wants to convey: {\it ``Kindness is never wasted''}. Using \textsc{WhatELSE}, she uploads her initial story into the system \presentation{(Figure~\ref{walkthrough}.a)}. The story is displayed in the pivot view, showing a sequence of events; while an initial outline appears on the right, summarizing the key turning points \presentation{(Figure~\ref{walkthrough}.b)}. Alice adds details to the pivot to refine her story. Once satisfied, she clicks the Generate Outline button to update the outline based on her edits. She chooses the ``act level'' and specifies, {\it ``The hunter has to appear in every act''}. Alice hovers to see how each event in the outline is mapped to the entries in the pivot plot. She continues exploring different levels to find the ideal level of abstraction.

\presentation{Alice finds one of the events ({\it ``The peaceful life is threatened by an unexpected danger from the hunter''}) to be too restrictive for the hunter to cause the danger. She uses the abstraction tooltip to replace the phrase {\it ``the hunter''} with {\it ``a character''} to leave room for variations in the game.} 
Alice looks at the outline and is unsure what players might experience. So she clicks the Generate Variants button. The interface displays a scatter plot of potential narrative instances. Alice scrolls through different plot stages of these instances — from start to end — she notices that some instances continuously express the moral, while others only reveal it toward the end, both of which she considers acceptable. However, she also spots a cluster of instances that fail to express the moral by the end of the narrative. Curiously, she clicks on a dot representing one of these instances and reviews its details. 

Alice reads the instance and realizes the issue is in the event that she had previously set as {\it ``a character''}, which was too loosely defined, allowing the system to choose an undesirable character. To address this issue, she changed it back to {\it ``a human character with power''}, allowing the system to choose a character reasonable for the second event. 

\begin{figure*}[h]
\centering
\includegraphics[width=1.0\textwidth]{figures/walkthrough_v3.pdf}
\vspace{-20pt}
\caption{\presentation{An example workflow that shows (a) an author uploads a story draft in \textsc{WhatELSE} to (b) generate an outline. The system unfolds the outline into (c) an executable game plot with (d) a pre-loaded story domain, which supports branching storylines based on the player actions. If the player chooses to (e1) save the deer from the hunter, this action fulfills the ``brave assistance'' event in the outline defined by the author (shown as the orange star). If the player chooses to (e2) ask another character (e.g. a witch) for help, the witch will instead save the deer, demonstrating ``brave assistance'' to fulfill the event. Alternatively, if the player does not choose to save the deer at all, the system will choose a character from the story domain to save the deer as a demonstration of ``brave assistance''. This example shows how the game plot is dynamically adjusted based on the player actions to fulfill the outline. (f) The author can play the game plot to better understand the player experience. }}
~\label{walkthrough}
\vspace{-10pt}
\end{figure*}

Later, Alice notices a set of three variants where one of the events unfolds as, {\it ``the dove speaks with the hunter, leading the hunter to notice and then chase the dove''}. Alice finds this version more compelling than her pivot plot. She removes other variants, only leaving these three narrative variants in the view. Satisfied with these variants, she clicks the Generate Outline button to create a new outline that summarizes their commonalities. She then returns to the outline editor, using the abstraction tools to iteratively edit the outline, until it aligns with the story's moral and represents a narrative space that incorporates the interesting variations.

\subsubsection{Unfold the Narrative Space For interactivity}
With the narrative space defined by the outline, \usecase{Alice can experience the narrative instances unfolding in a turn-based text adventure game. She goes to the interactivity page. The system loads the story domain that includes a set of characters, locations, and action schema \presentation{(Figure~\ref{walkthrough}.d)}. The first sequence of the game plot is generated: a hunter is looking for food and finds a deer to hunt \presentation{(Figure~\ref{walkthrough}.c)}}. 

Alice, playing as the dove, chooses her next moves from a pin pad \usecase{\presentation{(Figure~\ref{walkthrough}.f)}. She can bravely stop the hunter by giving out her food \presentation{(Figure~\ref{walkthrough}.e1)}. Alternatively, she could ask other characters for help \presentation{(Figure~\ref{walkthrough}.e2)}. 
The system compares the player's action with the narrative outline. If the player chooses to save the deer on their own \presentation{(Figure~\ref{walkthrough}.e1)}, the event of {\it ``brave assistance''} is fulfilled by the player action. If the player chooses other actions, the system will create an event where another character demonstrates {\it ``brave assistance''} \presentation{(the witch in Figure~\ref{walkthrough})} to fulfill the event. The system generates subsequent character behaviors based on the player action. }

This turn-based interaction continues, with Alice alternating between reviewing generated game plots, observing character simulations, and experiencing the generated game play as a player. \usecase{Since she wrote a total of five events in her outline, the game play proceeds for five rounds, until all the events she planned have been played out. Since the game plots generated are fully structured, Alice can directly export the output of the narrative compiler as a finite state machine into the game engine where she can visualize the characters and locations.}

%The scenario above demonstrates how creators can use \textsc{WhatELSE} to create their interactive narratives by first encoding the authorial intent into a narrative space, and then unfolding the narrative space into play-time plot execution. 
\usecase{
\subsubsection{Additional Use Cases} In addition to Alice's case as a text-adventure game designer, \textsc{WhatELSE} can also serve as a powerful tool for a wide range of users. Game masters, mod developers, and fan creators across different domains can leverage its capabilities. For example, dungeon masters in tabletop role-playing games can use the Narrative Space Editor to outline gameplay scenarios before sessions and employ the Interactive Narrative Compiler to dynamically determine outcomes of player actions during gameplay. Fan creators~\cite{booth2009narractivity} can efficiently transform their favorite novels, movies, or other media into interactive narratives, using the \textsc{WhatELSE} to structure and unfold new, personalized storylines based on the original story domain. Beyond entertainment, educators can utilize \textsc{WhatELSE} to design interactive learning experiences, such as gamified learning tutorials or interactive training modules. }
%By outlining educational themes in the Narrative Space Editor and generating interactive scenarios through the Interactive Narrative Compiler, \textsc{WhatELSE} enables students to explore the theme dynamically, fostering engagement and deeper understanding through adaptive branching experiences.















\section{User Study}
\section{Methods}
Using ApoloBot as a discussion starting point, we extend our exploration into the broader landscape of restorative justice tools through a three-phase user study with Discord moderators. Each phase involves increasing levels of commitment, starting with initial interviews, followed by tool deployment, and concluding with reflections. Given that restorative justice tools are still relatively rare in online communities, these separate phases allow us to gather valuable insights while respecting moderators' diverse willingness and interest in the new approach. All parts of this study were pre-approved by our university's Institutional Review Board (IRB).

\subsection{Phases Overview}
%To evaluate the potential for ApoloBot and restorative justice tools more broadly, we conducted a user study with three phases. 

\textbf{Phase 1. Onboarding Session (60-90 minutes):} In the first phase, we conducted individual interviews with Discord moderators to gain insights into their general moderation practices and the potential of integrating restorative justice tools. Participants were asked about their procedure to handling interpersonal harm with specific examples of past cases. We then introduced the concept of restorative justice and presented ApoloBot as a practical tool embodying a subset of these principles. This was followed by discussions on the potential application of ApoloBot and other restorative justice tools within their communities, considering critical factors such as use cases, challenges, opportunities, perceived benefits, and drawbacks. After the interview, participants were invited to a Discord sandbox server to test out ApoloBot, where they provided further feedback and decided whether to continue with the study by deploying it in the subsequent phase.

\textbf{Phase 2. In-the-wild Deployment (4 weeks):} In the second phase, a subset of interested participants deployed ApoloBot in their communities, using it whenever suitable cases arose. Throughout this period, they kept track of their bot usage and maintained weekly communication with the researchers for feedback and support.

\textbf{Phase 3. Exit Interview (60-90 minutes):} At the end of the deployment period, participants joined an exit interview to reflect on their experiences with ApoloBot, unveiling new insights into its practical aspects, including user engagement and its effects on the community. Building on these reflections and revisiting critical factors from Phase 1 interviews, we \revision{explored the underlying factors for how the deployment met or challenged initial expectations, and} broadened the discussion to assess the overall design space of ApoloBot and other online restorative justice tools.

All interviews were conducted remotely through the Discord voice chat function. Participants could withdraw from the study at any phase without penalty. Compensation was provided for fully completed phases: \$20 for Phase 1, \$50 for Phase 2, and \$30 for Phase 3, delivered via Tremendous.~\footnote{https://www.tremendous.com/}


\subsection{Recruitment and Selection of Participants}
%We utilized a combination of platforms to distribute
Our recruitment call was distributed in meta-moderation communities on Discord, Reddit, and Facebook. These are communities where Discord moderators gather to discuss various moderation topics, such as news, strategies, philosophies, and tool usage. To ensure the quality of our recruits, we used a screening survey to assess their background and moderation experience. In addition to project-specific criteria such as prior experience handling interpersonal harm and familiarity with Discord bots, we filtered out low-quality responses such as one-word answers and those containing nonsensical or irrelevant information. We contacted selected participants, and further employed snowball sampling~\cite{Biernacki1981} by asking them for referrals. A total of 16 participants were chosen for Phase 1, with six proceeding to Phases 2 and 3. Two used ApoloBot during their deployment, while the others deployed it but did not encounter any suitable use cases. A summary of the participants' demographics and their status within Phase 1 and 2 are detailed in Table \ref{table:demographics}.

\begin{table}[htbp!]
\resizebox{\columnwidth}{!}{%
\begin{tabular}{@{}l|ccc|c@{}}
 & Liberal & Moderate & Conservative & Total \\ \hline
Female & 223 & 114 & 45 & 382 \\
Male & 102 & 78 & 53 & 233 \\
Prefer not to say & 2 & 0 & 0 & 2 \\ \hline
Total & 327 & 192 & 98 & 617
\end{tabular}%
}
\caption{Annotator Demographics. All annotators are based in the United States. The table shows the number of annotators across ideology and sex categories, as self-reported to Prolific. The mean age is 38.3 (SD=12.7), and 45 annotators are immigrants (7.3\%).}
\label{tab:demographics}
\end{table}



\subsection{Qualitative Analysis}
Interview sessions in Phase 1 and Phase 3 were transcribed using CLOVA Notes. %For Phase 1 interviews, two researchers iteratively conducted thematic analysis\revision{~\cite{Braun2006}}, which involved inductive coding on the data, identifying emerging themes, and grouping into higher hierarchies. We worked together closely during this phase and followed a consensus-coding approach, having consistent meetings to merge individual codes, resolve conflict, and reach agreements on the final codebook. For this reason, calculating inter-rater reliability was not deemed necessary~\cite{McDonald2019}.
\revision{
For Phase 1 interviews, thematic analysis was conducted inductively through multiple iterations~\cite{Braun2006}. First, two researchers individually performed line-by-line open coding on eight interviews, generating initial codes that closely resembled text from the transcript such as “instant ban”, “add to modmail” and “bot seems insincere”. This was followed by focused coding~\cite{Saldaa2021}, where we identified recurring themes and sorted them into broader categories such as “escalating procedure”, “integration into existing system” and “tool perception”, which formed our initial codebook. The first author then applied this codebook across the remaining interviews, refining and adding codes as new insights emerged. After this, the two researchers met again to validate the updated codebook, consolidating higher-level themes along the dimension of moderators’ practices in addressing interpersonal harm, their stances on adopting restorative justice tools through ApoloBot’s framework, and potential impacts of implementing such a system. Finally, we aligned these diverse perspectives to outline the opportunity space and challenges associated with transitioning from traditional moderation practice to integrating restorative justice tools, laying the groundwork for our results.
}

% First-level codes included short phrases similar to text from the transcript, such as “instant ban”, “add to modmail” and “bot seems insincere”. These were then sorted into broader categories such as “escalating procedure”, “integration into existing system” and “tool perception”.

Phase 3 interviews were coded by the first author following a similar inductive process \revision{based on the codebook developed in Phase 1. While Phase 1 interviews focused on moderators’ reflections on prior experiences, Phase 3 expanded upon these by grounding the insights in practical deployment outcomes. Successful use cases from Phase 3 demonstrated how expectations from Phase 1 were met, validating the key opportunities where the tool effectively fulfilled its design intent. Equally significant were the unmet expectations, where anticipated use cases were not realized, as they revealed a new-found understanding of the practical challenges and critical areas of the opportunity space where the tool's effectiveness fell short. These observations were thus incorporated into the final codebook by combining and adding to Phase 1's codes, enhancing the framework that underpins our findings.
}

% Given that some reflections in this phase provided deeper elaborations of insights from Phase 1, we combined and consolidated several categories.

\subsection{Methodological Limitations}
As highlighted by Xiao et al., \textit{"Online communities should allow for partial success or no success without enforcing the ideal outcome, especially at the early stage of implementation when there are insufficient resources or commitments"}~\cite{Xiao2023}. Restorative justice, being relatively new and context-specific, poses significant challenges when evaluated within a brief testing period. Our study is therefore constrained by the limited empirical data available on ApoloBot usage, and the analysis presented here relies mostly on interview data from Phases 1 and 3. 
\revision{
This limitation also arises from how we shape our research focus, which is not on delivering a fully-realized restorative justice tool ready for adoption, but on developing a conceptual artifact to probe its implementation and foster critical reflections among moderators. For those who engaged with the tool, their experiences provide concrete evidence of its realized potential for effective adoption. On the other hand, investigating those who did not use the tool reveals challenges and critical gaps in its suitability within the broader online landscape, which can inform future alternatives or complements that might address the limitations.
Centering the discussion on these dual perspectives allows a deeper and more comprehensive view of how diverse online communities are currently positioned for restorative justice tools, however it might compensate the technical significance of the proposed system.}

% While these interviews offered preliminary insights into the perceived potential of ApoloBot and similar restorative justice tools, they may not offer a comprehensive assessment of their broader significance.

In addition, our study primarily gathers insights from moderators rather than victims or offenders. While this focus offers rich insights into the practical aspects of the tool adoption and execution, it lacks the perspectives of the remaining stakeholders essential to restorative justice, and thus may not fully capture the complete user experience.

Finally, even though our participants come from a wide range of international communities covering diverse topics, the fact that they are solely English speakers limits the cross-cultural generalizations that can be made based on our findings.


\section{Technical Evaluation}
The user study shows that \textsc{WhatELSE} effectively helps users create engaging interactive narratives by enhancing both authorial control and player engagement through efficient narrative space editing. To validate the technical pipeline driving the transformation between narrative outline and instances, we conducted technical evaluations focusing on effectiveness in 1) generating the outline from instances and 2) generating instances from an outline.



\subsection{From Narrative Instances to Narrative Outline}

\presentation{\textsc{WhatELSE} provides an abstraction ladder with different levels of abstraction to generate an outline from instances using a prompting pipeline.} To examine the effectiveness of the pipeline, we employ two lexical-level measures, {\em concreteness rate}, and {\em imageability score}.  Both measures are adapted from large-scale crowdsourced annotations in previous studies that have been widely used in linguistic evaluations ~\cite{wilson1988mrc,brysbaert2014concreteness}. \evaluation{Both scores are lexicon-based, with each word assigned an averaged score from a batch of crowdsourced annotations; for each outline, we calculate the average score across all words in this outline, excluding stop words.} Intuitively, a higher concreteness rate indicates that the wording is more concrete and specific, corresponding to a lower level of abstraction. A lower imageability score suggests that the wording allows greater room for interpretation, corresponding to a higher level of abstraction. 

\presentation{We generated outlines for a sample of 100 stories from the Fairytale dataset~\cite{xu2022fantastic} at three different levels of abstraction (scene, sequence, and act level).} We reported the concreteness rate and imageability score of the generated outlines in Table ~\ref{tech_eval:abstraction_ladder}. The results show that our method effectively produces outlines at three distinct levels of abstraction, with significant differences in the lexical measures between each pair of abstraction levels. This demonstrates the system's ability to define narrative spaces with varying degrees of constraint.

%This demonstrates the system’s ability to facilitate creators' control over varying degrees of narrative variation within the outlined narrative space.


% our system helps users generate outlines based on a linear story. Practically, we designed the prompt template to help users generate outlines with three different levels of abstractions on a linear story. We evaluated the capability of our designed prompting on the public Fairytale dataset, from which we sampled 100 stories and used our prompts to generate outlines of the stories at the three levels of abstraction that we proposed (i.e., scene level, sequence level, act level). We use two lexical measures, concreteness rate, and imaginability score to evaluate the levels of abstraction of generated outlines. Both measures are from large-scale crowdsourced annotations collected previously on English writings. Intuitively, higher concreteness rate means the wording used in the outline makes it more concrete and specific, thus less abstraction. Meanwhile, higher imaginability score indicates the wording of the text leaves larger room for readers to imagine, subsequently a higher abstraction level. The measured levels of abstraction of generated outlines using our prompts are reported in Tab ~\ref{tech_eval:abstraction_ladder}. We can see that our prompts are capable of providing outlines at three different levels of abstraction, with momentum measures significantly differing between each pair of two levels of abstraction.



\begin{table}[h]

\small
% \resizebox{\linewidth}{!}{
\begin{tabular}{c|c|c|c}
% \Large
\toprule




%\thead{\textbf{Experiment}/ \\ Independent Var} 
\multirow{2}{*}{\textbf{Measurement}}
& \multicolumn{3}{c}{\textbf{Abstraction Level }} 
% {\thead{\textbf{ Exp 1:} \\ $y = $ Final Accuracy}} &
% {\thead{\textbf{ Exp 2:} \\ $y = $ Peer Accuracy}} &  {\thead{\textbf{ Exp 3:} \\ $y = $ Final Accuracy}}

%&  \textbf{Prob. of Superiority}
\\
\cline{2-4}
 & \textbf{Scene Level}  & \textbf{Sequence Level} & \textbf{Act Level}
\\
\midrule

Concreteness Rate & $3.56\pm 0.05$ & $3.09\pm0.06$  & $2.95\pm 0.06$ 
\\

% Peer Accuracy ($\beta_1$) & & &  $1.88^{***}$
% \\
% array([497.32914428, 453.64683276, 438.91406737])
% np.array(img_mat).std(axis=0)
% array([22.82068827, 32.41856082, 34.1765820


Imageability Score & $497.33\pm{22.82}$ & $453.64\pm{32.42}$ & $438.91\pm{34.17}$
\\

\bottomrule
\end{tabular}
% }
\vspace{2pt}
\caption{Measured abstraction level of the generated outline plot using the proposed prompt pipeline employed in Abstraction Ladder, to generate outline with distinct levels of abstraction from narrative examples. A lower concreteness rate and imageability score indicate the text is more abstract at the lexical level.}
\label{tech_eval:abstraction_ladder}
% \vspace{-25pt}
\end{table}


% \begin{table}[h]
% \begin{tabular}{|c|c|c|c|}
% \hline
% Abstraction Level   & Secne Level & Sequence Level & Act Level \\ \hline
% Concreteness Rate   &  $3.56\pm 0.05$           &  $3.09 \pm 0.06$               &    $2.95 \pm 0.06$        \\ \hline
% Imaginability Score &             &                &           \\ \hline
% \end{tabular}
% \caption{Evaluation results of abstraction ladder}
% \label{tech_eval:abstraction_ladder}

% \end{table}

%\noindent \textbf{Narrative Space Feedback}

\begin{table*}[ht]

% \small
% \resizebox{\linewidth}{!}{
\begin{tabular}{c|c|c|c|c}
% \Large
\toprule




%\thead{\textbf{Experiment}/ \\ Independent Var} 
\multirow{2}{*}{\textbf{Measurement}}
& \multicolumn{2}{c|}{\textbf{Human-generated Outline}} & \multicolumn{2}{c}{\textbf{LLM-generated Outline}} 
% {\thead{\textbf{ Exp 1:} \\ $y = $ Final Accuracy}} &
% {\thead{\textbf{ Exp 2:} \\ $y = $ Peer Accuracy}} &  {\thead{\textbf{ Exp 3:} \\ $y = $ Final Accuracy}}

%&  \textbf{Prob. of Superiority}
\\
\cline{2-5}
 & \textbf{$d_1$}  & \textbf{$d_{macro}$} & \textbf{$d_1$}  & \textbf{$d_{macro}$}
\\
\midrule

% Peer Accuracy ($\beta_1$) & & &  $1.88^{***}$
% \\

\presentation{Proposed IN Compiler} & {$0.65\pm 0.01$}             & $0.78\pm 0.01$  & $0.64\pm 0.01$ & $0.77\pm 0.01$
\\

\presentation{Baseline Prompt-based IN Compiler} & $0.51\pm 0.02$             & $0.63\pm 0.02$  & $0.61\pm 0.02$ & $0.73\pm 0.02$
\\

\bottomrule
\end{tabular}
% }
% \vspace{2pt}
\caption{Measured plot distance by averaged ROUGE-1 distance $d_{1}$ and the macro-averaged ROUGE distance $d_{macro}$ among the plots generated by baseline and our approach. Results show that \textsc{WhatELSE} IN Compiler leads to a larger averaged distance among the plots, and thus a greater diversity of plots within the narrative space.}
\label{tech_eval:plot_diversity}
% \vspace{-25pt}
\end{table*}

\begin{table*}[h]

% \small
% \resizebox{\linewidth}{!}{
\begin{tabular}{c|c|c|c|c}
% \Large
\toprule




%\thead{\textbf{Experiment}/ \\ Independent Var} 
\multirow{2}{*}{\textbf{Method}}
& \multicolumn{4}{c}{\textbf{Measurement}} 

\\
\cline{2-5}
 & \textbf{$d_1$}  & \textbf{$d_{macro}$} & {World-state Change (Neg)}  & {Character Involvement (Pos)}
\\
\midrule

% Peer Accuracy ($\beta_1$) & & &  $1.88^{***}$
% \\

\presentation{Proposed IN Compiler} & {$0.59\pm 0.01$}             & $0.74\pm 0.001$  & $1.00\pm 0.0$ & $1.70\pm 0.37$
\\

\presentation{Baseline Prompt-based IN Compiler} & $0.53\pm 0.01$             & $0.67\pm 0.01$  & $0.85\pm 0.08$ & $1.65\pm 0.41$ 
\\

\bottomrule
\end{tabular}
% }
% \vspace{2pt}
\caption{Measured impact of player action on game plot progression, by (1) averaged pairwise ROUGE-1 distance $d_{1}$ and the macro-averaged ROUGE distance $d_{macro}$ between pairs of the game plots driven by contrasting player actions, and (2) averaged world state change rate driven by the negative player action and the averaged character involvement driven by the positive action. Results show that the contrasting player actions make the proposed approach generate plots with larger pairwise distances. Additionally, in \textsc{WhatELSE} playtime, player actions lead to more stable world-state change and better character involvement than the baseline. }
\label{tech_eval:action_impact}
% \vspace{-25pt}
\end{table*}

\subsection{From Narrative Outline to Narrative Instances}
\presentation{We compared the plot quality generated using our approach with the baseline prompting-based approach (Figure~\ref{baseline})} from two aspects: {\em plot diversity} and {\em player impact}. 


\noindent \textbf{Plot Diversity}\hspace{2mm}  Plot diversity refers to the ability to generate a wide range of different plots within the narrative space. It indicates the level of interactivity and player agency supported by the plot generation method, as it showcases the system's ability to offer varied storylines within the narrative space described by the outline.

To quantitatively assess diversity among a set of $N$ plots generated within the narrative space, we calculate the averaged distance between each plot and the other $N - 1$ plots in the set. We then compute the average distance for each plot relative to the others. For distance calculation, we use the ROUGE score ~\cite{lin2004rouge}, a reference-based evaluation metric that measures text similarity. Adapting from the ROUGE-1 score $r_{1}$ targeting word level and macro averaged ROUGE score $r_{macro}$ measures similarity across multiple levels of wording, we compute the word-level distance $d_{1} = 1 - r_{1}$ and the macro distance $d_{macro} = (1 - r_{macro})$ between plots, respectively. As the ROUGE score indicates similarity between text, a higher averaged distance indicates the greater diversity of plots.

For comparison, we first collected two sets of outlines based on the \presentation{\textit{``Fairytale Forest''} story domain (Figure~\ref{baseline})}. The first set, consisting of 12 outlines, was generated by participants in the user study, each tied to one of two specific morals. Additionally, we developed a set of 50 outlines by prompting an LLM, focusing on various morals within the story domain. We simplify each outline by taking only the first act, resulting in 12 human-generated and 100 LLM-generated single-act outlines. These outlines were then used to guide plot generation without involving player actions, using the proposed narrative planning based approach and the baseline prompt-based approach.

We generated a set of 20 plots with each outline and then calculated the averaged distances $d_{1}$ and $d_{macro}$ among each set of plots. As shown in Table~ \ref{tech_eval:plot_diversity}, our approach generated more diverse plots within the same narrative space, indicating more variety of storylines and stronger player agency.








% Please add the following required packages to your document preamble:
% \usepackage{multirow}
% Please add the following required packages to your document preamble:
% \usepackage{multirow}
% \begin{table}[t]
% \begin{tabular}{|c|cc|cc|}
% \hline
% \multirow{2}{*}{Outline Content} & \multicolumn{2}{c|}{Human-generated Outline}                    & \multicolumn{2}{c|}{LLM-generated Outline}                      \\ \cline{2-5} 
% & \multicolumn{1}{l|}{Rouge 1 Distance} & Averaged Rouge Distance & \multicolumn{1}{l|}{Rouge 1 Distance} & Averaged Rouge Distance \\ \hline
% Prompting                        & \multicolumn{1}{c|}{$0.49\pm 0.02$}             & $0.37\pm 0.02$                    & \multicolumn{1}{c|}{xx}             & 0.34                    \\ \hline
% Gameplot Compiler (Ours)                & \multicolumn{1}{c|}{$0.35\pm 0.01$}             & $0.22\pm 0.007$                    & \multicolumn{1}{c|}{xx}             & 0.44                    \\ \hline
% \end{tabular}
% \caption{Player Impact}
% \label{tech_eval:plot_diversity}
% \end{table}



\noindent \textbf{Player Impact} \presentation{We use the term player impact to refer to the extent to which players' actions meaningfully influence the progression of the plot.} A higher player impact indicates that the narrative is more responsive to player actions, leading to different outcomes and providing a more personalized experience. 

Given a sequence of events ($S$), the following metrics measure the difference between the subsequent events ($S'$) in response to players taking different actions after the leading sequence $S$. 

\begin{itemize}
\item {\bf Subsequent Plot Divergence}\hspace{2mm} We execute a pair of contrasting player actions after $S$, and compare how the following plot progression diverges semantically based on the player's different actions. The contrasting actions are attacking/killing a character (negative action) and seeking help for a character in danger (positive action). This comparison is performed by calculating two types of ROUGE distances. Specifically, the distance is computed pairwisely between the two plots generated after the positive and negative actions. A higher ROUGE distance indicates a greater divergence between the two plots driven by contrasting player actions, thus reflecting their higher impact on the plot progression. 
\item {\bf Perceived World State Change}\hspace{2mm} The metric assesses the perceived alternation of world states caused by the player's actionn in a specific scenario. We execute a player action of killing a character, and count the frequency of the killed character's reappearance in the subsequent plots.
\item {\bf Player Character Involvement}\hspace{2mm} The metric examines whether the player's action increases the player character's involvement in the subsequent events in the plot. We calculate the frequency with which the player's character appears in the plots that follow the positive action of helping a character. This indicates the extent to which the player's actions influence their engagement in the narrative.
\end{itemize}


%Additionally, we take specific measures on the perceived world states change to evaluate perceived impacts of player actions. 
%on the game world and the player's involvement in the plot. 

%For instance, the action of killing a character is intended to prevent that character from appearing in subsequent plots, which may not be the case due to hallucination.

%However, the baseline approach could suffer from hallucination, leading to the reappearance of a supposedly killed character, 
%thereby ignoring the player's impact on the world state. We  


%Practically, we again take the story domain "Fairytale Forest" and a fixed story outline containing two acts for evaluation. 
 %Next, we input the designed positive or negative player actions and use both the prompt-based approach and our approach to generate a batch of 20 game plots following the player action. This process creates 20 pairs of game plots after contrasting player actions generated by each approach. We then calculate the average pairwise distance between plots using two types of ROUGE distance. Additionally, we compute the average change in the world state and player involvement for plots generated after the player's positive and negative actions, respectively. 
We use the \presentation{\textit{``Fairytale Forest''} story domain (Figure~\ref{baseline})} and a fixed story outline containing two acts for evaluation. The player character is set as the dove. To initialize, we generate the plot for the first act in the outline as $S$ and set the world state accordingly. We then execute the designed player actions, and then use our method and the baseline method to generate a batch of 20 plots following each player action to compute the above metrics.

As Table~\ref{tech_eval:action_impact} shows, our approach generates significantly more diverse plots following the player's contrasting actions. Moreover, our approach generates plot with better perceived world state change and character involvement following player's actions. Overall, \presentation{we found that \textsc{WhatELSE} integrates player actions with a higher impact in the narrative generation process.}


% Therefore, 

% capability of an approach to generate a wide range of different game plots based on the same outline. This reflects the core capability of a game plot generation method to offer diverse narrative paths, thereby determining the level of interactivity and freedom available to players within the game.



% \begin{table}[]
% \begin{tabular}{|c|c|c|c|c|}
% \hline
%                   & Pairwise $d_{1}$ & Pairwise $d_{macro}$ & World-state Change (Neg) & Character Involvement (Pos) \\ \hline
% Prompting         & $0.47\pm 0.01$                       & $0.33\pm 0.01$                             & $0.85\pm 0.08$                & $1.65\pm 0.41$                   \\ \hline
% Gameplot Compiler (Ours) & $0.41\pm 0.01$                       & $0.26\pm 0.01$                              & $1.0\pm 0.0$               & $1.7\pm 0.37$                   \\ \hline
% \end{tabular}
% \caption{Impact of Action}
% \label{tech_eval:action_impact}
% \end{table}





% \section{Threats To Validity}\label{sec:ttv}
Our SLR aims to be comprehensive, but some limitations should be acknowledged. While we searched popular repositories (IEEE Xplore, ACM Digital Library, Springer, and ScienceDirect) and employed both backward and forward snowballing techniques on recent publications, the possibility of unintentional inclusion or exclusion of relevant studies remains. The authors carefully evaluated publications that fell on the borderline of inclusion/exclusion criteria to mitigate this risk.

Furthermore, our SLR focused exclusively on peer-reviewed journal articles and conference publications published in English. This decision was made to streamline the review process and ensure a high standard of research quality. However, it is important to acknowledge that relevant information may exist in other sources, such as books, theses, and non-English publications, which were not included in this review.

In addition, we deliberately excluded publications that primarily addressed network security, architecture, or systems that utilized SDN as a component. We aimed to maintain a focused review on the software security aspects of SDN itself. This means that studies analyzing SDN's role in broader contexts, such as cloud security or Internet of Things (IoT) networks, were not included. While this approach ensured a clear research focus, it may have overlooked valuable insights on the broader implications of SDN software security.

Overall, while we believe our SLR provides a comprehensive overview of the current state of research on SDN software security, readers should be aware of these limitations when interpreting our findings.

\section{Discussion}
\section{Discussion of Assumptions}\label{sec:discussion}
In this paper, we have made several assumptions for the sake of clarity and simplicity. In this section, we discuss the rationale behind these assumptions, the extent to which these assumptions hold in practice, and the consequences for our protocol when these assumptions hold.

\subsection{Assumptions on the Demand}

There are two simplifying assumptions we make about the demand. First, we assume the demand at any time is relatively small compared to the channel capacities. Second, we take the demand to be constant over time. We elaborate upon both these points below.

\paragraph{Small demands} The assumption that demands are small relative to channel capacities is made precise in \eqref{eq:large_capacity_assumption}. This assumption simplifies two major aspects of our protocol. First, it largely removes congestion from consideration. In \eqref{eq:primal_problem}, there is no constraint ensuring that total flow in both directions stays below capacity--this is always met. Consequently, there is no Lagrange multiplier for congestion and no congestion pricing; only imbalance penalties apply. In contrast, protocols in \cite{sivaraman2020high, varma2021throughput, wang2024fence} include congestion fees due to explicit congestion constraints. Second, the bound \eqref{eq:large_capacity_assumption} ensures that as long as channels remain balanced, the network can always meet demand, no matter how the demand is routed. Since channels can rebalance when necessary, they never drop transactions. This allows prices and flows to adjust as per the equations in \eqref{eq:algorithm}, which makes it easier to prove the protocol's convergence guarantees. This also preserves the key property that a channel's price remains proportional to net money flow through it.

In practice, payment channel networks are used most often for micro-payments, for which on-chain transactions are prohibitively expensive; large transactions typically take place directly on the blockchain. For example, according to \cite{river2023lightning}, the average channel capacity is roughly $0.1$ BTC ($5,000$ BTC distributed over $50,000$ channels), while the average transaction amount is less than $0.0004$ BTC ($44.7k$ satoshis). Thus, the small demand assumption is not too unrealistic. Additionally, the occasional large transaction can be treated as a sequence of smaller transactions by breaking it into packets and executing each packet serially (as done by \cite{sivaraman2020high}).
Lastly, a good path discovery process that favors large capacity channels over small capacity ones can help ensure that the bound in \eqref{eq:large_capacity_assumption} holds.

\paragraph{Constant demands} 
In this work, we assume that any transacting pair of nodes have a steady transaction demand between them (see Section \ref{sec:transaction_requests}). Making this assumption is necessary to obtain the kind of guarantees that we have presented in this paper. Unless the demand is steady, it is unreasonable to expect that the flows converge to a steady value. Weaker assumptions on the demand lead to weaker guarantees. For example, with the more general setting of stochastic, but i.i.d. demand between any two nodes, \cite{varma2021throughput} shows that the channel queue lengths are bounded in expectation. If the demand can be arbitrary, then it is very hard to get any meaningful performance guarantees; \cite{wang2024fence} shows that even for a single bidirectional channel, the competitive ratio is infinite. Indeed, because a PCN is a decentralized system and decisions must be made based on local information alone, it is difficult for the network to find the optimal detailed balance flow at every time step with a time-varying demand.  With a steady demand, the network can discover the optimal flows in a reasonably short time, as our work shows.

We view the constant demand assumption as an approximation for a more general demand process that could be piece-wise constant, stochastic, or both (see simulations in Figure \ref{fig:five_nodes_variable_demand}).
We believe it should be possible to merge ideas from our work and \cite{varma2021throughput} to provide guarantees in a setting with random demands with arbitrary means. We leave this for future work. In addition, our work suggests that a reasonable method of handling stochastic demands is to queue the transaction requests \textit{at the source node} itself. This queuing action should be viewed in conjunction with flow-control. Indeed, a temporarily high unidirectional demand would raise prices for the sender, incentivizing the sender to stop sending the transactions. If the sender queues the transactions, they can send them later when prices drop. This form of queuing does not require any overhaul of the basic PCN infrastructure and is therefore simpler to implement than per-channel queues as suggested by \cite{sivaraman2020high} and \cite{varma2021throughput}.

\subsection{The Incentive of Channels}
The actions of the channels as prescribed by the DEBT control protocol can be summarized as follows. Channels adjust their prices in proportion to the net flow through them. They rebalance themselves whenever necessary and execute any transaction request that has been made of them. We discuss both these aspects below.

\paragraph{On Prices}
In this work, the exclusive role of channel prices is to ensure that the flows through each channel remains balanced. In practice, it would be important to include other components in a channel's price/fee as well: a congestion price  and an incentive price. The congestion price, as suggested by \cite{varma2021throughput}, would depend on the total flow of transactions through the channel, and would incentivize nodes to balance the load over different paths. The incentive price, which is commonly used in practice \cite{river2023lightning}, is necessary to provide channels with an incentive to serve as an intermediary for different channels. In practice, we expect both these components to be smaller than the imbalance price. Consequently, we expect the behavior of our protocol to be similar to our theoretical results even with these additional prices.

A key aspect of our protocol is that channel fees are allowed to be negative. Although the original Lightning network whitepaper \cite{poon2016bitcoin} suggests that negative channel prices may be a good solution to promote rebalancing, the idea of negative prices in not very popular in the literature. To our knowledge, the only prior work with this feature is \cite{varma2021throughput}. Indeed, in papers such as \cite{van2021merchant} and \cite{wang2024fence}, the price function is explicitly modified such that the channel price is never negative. The results of our paper show the benefits of negative prices. For one, in steady state, equal flows in both directions ensure that a channel doesn't loose any money (the other price components mentioned above ensure that the channel will only gain money). More importantly, negative prices are important to ensure that the protocol selectively stifles acyclic flows while allowing circulations to flow. Indeed, in the example of Section \ref{sec:flow_control_example}, the flows between nodes $A$ and $C$ are left on only because the large positive price over one channel is canceled by the corresponding negative price over the other channel, leading to a net zero price.

Lastly, observe that in the DEBT control protocol, the price charged by a channel does not depend on its capacity. This is a natural consequence of the price being the Lagrange multiplier for the net-zero flow constraint, which also does not depend on the channel capacity. In contrast, in many other works, the imbalance price is normalized by the channel capacity \cite{ren2018optimal, lin2020funds, wang2024fence}; this is shown to work well in practice. The rationale for such a price structure is explained well in \cite{wang2024fence}, where this fee is derived with the aim of always maintaining some balance (liquidity) at each end of every channel. This is a reasonable aim if a channel is to never rebalance itself; the experiments of the aforementioned papers are conducted in such a regime. In this work, however, we allow the channels to rebalance themselves a few times in order to settle on a detailed balance flow. This is because our focus is on the long-term steady state performance of the protocol. This difference in perspective also shows up in how the price depends on the channel imbalance. \cite{lin2020funds} and \cite{wang2024fence} advocate for strictly convex prices whereas this work and \cite{varma2021throughput} propose linear prices.

\paragraph{On Rebalancing} 
Recall that the DEBT control protocol ensures that the flows in the network converge to a detailed balance flow, which can be sustained perpetually without any rebalancing. However, during the transient phase (before convergence), channels may have to perform on-chain rebalancing a few times. Since rebalancing is an expensive operation, it is worthwhile discussing methods by which channels can reduce the extent of rebalancing. One option for the channels to reduce the extent of rebalancing is to increase their capacity; however, this comes at the cost of locking in more capital. Each channel can decide for itself the optimum amount of capital to lock in. Another option, which we discuss in Section \ref{sec:five_node}, is for channels to increase the rate $\gamma$ at which they adjust prices. 

Ultimately, whether or not it is beneficial for a channel to rebalance depends on the time-horizon under consideration. Our protocol is based on the assumption that the demand remains steady for a long period of time. If this is indeed the case, it would be worthwhile for a channel to rebalance itself as it can make up this cost through the incentive fees gained from the flow of transactions through it in steady state. If a channel chooses not to rebalance itself, however, there is a risk of being trapped in a deadlock, which is suboptimal for not only the nodes but also the channel.

\section{Conclusion}
This work presents DEBT control: a protocol for payment channel networks that uses source routing and flow control based on channel prices. The protocol is derived by posing a network utility maximization problem and analyzing its dual minimization. It is shown that under steady demands, the protocol guides the network to an optimal, sustainable point. Simulations show its robustness to demand variations. The work demonstrates that simple protocols with strong theoretical guarantees are possible for PCNs and we hope it inspires further theoretical research in this direction.


\begin{acks}
We are grateful to the anonymous reviewers who provided many helpful comments. We also thank our colleagues, David Ledo, Fraser Anderson and Hilmar Koch for feedback and suggestions in preparing the manuscripts and figures. Any opinions, findings, conclusions,
or recommendations expressed here are those of the authors alone.
\end{acks}


%%
%% The next two lines define the bibliography style to be used, and
%% the bibliography file.
\bibliographystyle{ACM-Reference-Format}
\bibliography{llm_narrative}

\end{document}
\endinput
%%
%% End of file `sample-sigconf.tex'.
