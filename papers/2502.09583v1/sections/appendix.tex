\onecolumn
\doparttoc %
\faketableofcontents %

\part{Appendix}\label{appendix} %
\parttoc %


\clearpage

\section{Details about \ourMethod-Bench}\label{app:imp_details}

\subsection{Coordination Environment Wrapper}


\begin{figure*}[t]
\centering
\begin{tcolorbox}[title={\begin{pseudo}Environment Wrapper\end{pseudo}}, colback=sbBlue025, colframe=sbBlue]
\begin{verbatim}
train_env, dev_env, test_env = make(config.help_envs) 

def make(config):
    base_envs = make_raw_envs(config)
    sim_novice_agent, novice_agent, expert_agent = load_agents(config)
    base_angets = [sim_novice_agent, novice_agent, expert_agent]
    coord_envs = {}
    for name in ["train", "val_sim", "val_true", "test"]:
        envs[name] = HelpEnvironment(config, base_envs[name], base_agents)
    return tuple(envs.values())

class HelpEnvironment(gym.Env):
    def __init__(self, config, base_env, base_agents):
    def reset(): 
    def step():
\end{verbatim}
\end{tcolorbox}
\caption{Python implementation for the environment wrapper that generates training, validation, and test environments by integrating simulated novice and expert agents. This enables efficient experimentation in coordination tasks.}
\label{fig:env_wrapper}
\end{figure*}


\section{The general case: Proof of \texorpdfstring{\Cref{thm:main-decomp}}{Theorem 1.6}}\label{sec:algo}

First, we show that data structure of \Cref{l:max_min_query} can be used to compute distances witnessed by shortest paths that pass through a constant-size separator.

\begin{lemma}\label{l:single_adhesion}
Fix a constant $k \in \mathbb{N}$. There exists an algorithm which as the input receives an edge-weighted graph $G$ on $n$ vertices and $m$ edges together with a partition of its vertices into three sets $A, B, C$ such that $|B| \leq k$ and there are no edges between $A$ and $C$, and as the output computes $\max_{c \in C} \dist(a, c)$ for every $a \in A$. The running time is $\Oh(m \log n + n \log^{k - 1} n)$.
\end{lemma}

\begin{proof}
Let $B = \{b_1, \ldots, b_k\}$. For any $a \in A, c \in C$, we have $\dist(a, c) = \min_{i \in [k]} \dist(a, b_i) + \dist(c, b_i)$. First, we run Dijkstra's algorithm from every vertex in $B$ to find $\dist(v, b_i)$ for every $v \in V(G)$ and $i \in [k]$. Next, we use \Cref{l:max_min_query} to construct a data structure $\mathbb{D}$ for the point set $\{(\dist(c, b_1), \dots, \dist(c, b_k))\colon c\in C\}\subseteq \mathbb{R}^k$. Now, the value $\max_{c \in C} \dist(a, c)$ for any given $a$ is equal to the answer of $\mathbb{D}$ to the query with argument $(\dist(a, b_1), \dots, \dist(a, b_k))$.
\end{proof}

After computing the distances over a constant-size separator, we will use the following observation to simplify one of the sides of the separation.

\begin{lemma}\label{l:inserting_paths}
Let $G$ be a edge-weighted connected graph and let $A, B, C$ be a partition of its vertices such that there are no edges between $A$ and $C$. For every pair of vertices $u, v \in B$, let $P_{u, v}$ be any shortest path from $u$ to $v$ with all internal vertices in $C$ (assuming such a path exists).

Let $G'$ denote a graph obtained from $G[A \cup B]$ by adding an edge from $u$ to $v$ of weight equal to the length of $P_{u, v}$, for all $u, v \in B$ for which $P_{u, v}$ exists. Then,  $$\dist_G(s, t) = \dist_{G'}(s, t)\qquad\textrm{for all }s,t\in A\cup B.$$
\end{lemma}
\begin{proof}
Let $G''$ be the graph obtained by adding new edges of $G'$ to $G$.
Fix any $s, t \in A \cup B$ and let $P$ denote the shortest path from $s$ to $t$ in $G''$ which minimizes the number of vertices from $C$ visited. Naturally, the weight of $P$ is equal $\dist_G(s, t)$. Assume that such path visits at least one vertex of $C$. Then, the path $P$ is of the form $s \xrightarrow{P_1} x \xrightarrow{P_2} y \xrightarrow{P_3} t$, where $x, y \in B$ and all the internal vertices of $P_2$ are in $C$. By the construction of $G'$, $P_2$ can be replaced with a direct edge from $x$ to $y$ of the same weight. We obtain a same weight path with a smaller number of vertices of $C$ visited, which is a contradiction. Therefore, $P$ is entirely contained in $A \cup B$, hence it exists in $G'$. This shows that $\dist_G(s, t) = \dist_{G'}(s, t)$.
\end{proof}


The next lemma encapsulates the main algorithmic content of the proof of \Cref{thm:main-decomp}. The algorithm will split the tree decomposition provided on input into smaller parts for which the eccentricities are easier to calculate. We use the following lemma to handle a single such part.
\begin{lemma}\label{l:star}
Fix constants $k, g \in \mathbb{N}, 0 < \delta < \frac{1}{54}$. Assume we are given $n \in \mathbb{N}$, an edge-weighted graph $G$ on at most $n$ vertices with a weight function $w \colon E(G) \to \mathbb{N}$, a vertex subset $A$ and a collection of non-empty vertex subsets $V_0, V_1, \dots, V_\ell$ satisfying the following conditions:
\begin{itemize}[nosep]
	\item The sum of weights of all the edges in $G$ is bounded by $\Oh(n)$.
	\item $V(G) \setminus A = V_0 \cup V_1 \cup \dots \cup V_\ell$.
	\item $|A| \leq k$.
	\item For every $i \in [\ell]$, $G[V_i \setminus V_0]$ is connected, $N_G(V_i \setminus V_0) = V_i \cap V_0$, $|V_i| = \Oh(n^\delta)$, and $|V_0 \cap V_i| \leq 4$.
	\item For all $i, j \in [\ell], i \neq j$, $V_i \setminus V_0$ and $V_j \setminus V_0$ are disjoint and non-adjacent in $G$.
	\item Every edge $uv \in E(G)$ with $u, v \not\in A$ is contained in $G[V_i]$ for some $i\in \{0,1,\ldots,\ell\}$.
	\item The graph obtained by taking $G[V_0]$ and adding a clique on $V_0 \cap V_i$ for every $i \in [\ell]$ has Euler genus bounded by $g$.
\end{itemize}
Then, we can compute the eccentricity of every vertex of $G$ in time $\Oh \left( n^{1 + \frac{150 + 54 \delta}{151}} \log^k n \right)$.
\end{lemma}

\begin{proof}
Fix $\delta' = \frac{1 + 97 \delta}{151}$; we have $\delta' - \delta = \frac{1 - 54\delta}{151} > 0$.
Let $E_i$ denote the set of edges with one endpoint in $V_i$ and the other endpoint in $V_i \setminus V_0$. For $i \in [\ell]$, we shall say that $V_i$ is {\em{heavy}} if the sum of weights of $E_i$ is larger than $n^{\delta'}$. Since the sets $E_i$ are pairwise disjoint and the total sum of weights of all the edges is bounded by $\Oh(n)$, the number of heavy subsets is bounded by $\Oh(n^{1 - \delta'})$. Without loss of generality, we may assume that $V_{\ell' + 1}, \dots, V_\ell$ are heavy and $V_1, \dots, V_{\ell'}$ are not, for some $\ell'\in \{0,\ldots,\ell\}$.


For any source vertex $s$, we can calculate distances from $s$ to every vertex of $G$  using breadth first search in time $\Oh(\sum_{e \in E(G)} w(e)) = \Oh(n)$.
In particular, for every $\ell' < i \leq \ell$, we can compute the distances from every vertex of $V_i$ to every vertex of $G$ in total time $\Oh(n^{2 - \delta' + \delta})$, because $$|V_{\ell'+1}\cup \ldots\cup V_{\ell}|\leq n^{1-\delta'}\cdot \Oh(n^\delta)=\Oh(n^{1-\delta'+
\delta}).$$
Additionally, we calculate distances $\dist_G(a, v)$ for every $a \in A, v \in V(G)$ in time $O(n)$.

For every $i \in [\ell]$ and $u,v \in V_0 \cap V_i$, there exists a shortest path $P_{i,u,v}$ from $u$ to $v$ with all internal vertices belonging to $V_i - V_0$ due to the assumption that $G[V_i - V_0]$ is connected and $N_G(V_i - V_0) = V_i \cap V_0$. Therefore, the distance from $u$ to $v$ is bounded by the sum of weights of edges in $E_i$. In particular, for $i \in [\ell']$, $\dist_G(u, v) \leq n^{\delta'}$.

We define $\widetilde{G}$ to be the graph obtained by taking $G[A \cup V_0 \cup \dots \cup V_{\ell'}]$ and applying the following operation for every $i \in \{\ell' + 1, \dots, \ell\}$:
for each pair of vertices $u, v \in A \cup (V_0 \cap V_i)$, add an edge in $\widetilde{G}$ between $u$ and $v$ with weight equal to the total weight of $P_{i,u,v}$. For a fixed $i, u$, we can find $P_{i, u, v}$ for all $v$ using breadth first search in time $\Oh(n)$. Taking a sum over all $i, u$, we get that $\tilde{G}$ can be computed in total time $\Oh(n^{2 - \delta'})$.


\begin{claim}\label{cl:wG}
The sum of the edge weights in $\widetilde{G}$ is $\Oh(n)$. Moreover, for all $u, v \in V(\widetilde{G})$, we have $\dist_{\widetilde{G}}(u, v) = \dist_{G}(u, v)$.
\end{claim}

\begin{proof}
Consider $i \in \{\ell' + 1, \dots, \ell\}$ and any $u, v \in A \cup (V_0 \cap V_i)$ for which we added an edge. Its weight is bounded by the sum of weights of edges in $E_i$. Therefore, the total weight of all edges added is at most
$$
\sum_{i \in \{\ell' + 1, \dots, \ell\}} \left( |A \cup (V_0 \cap V_i)|^2 \sum_{e \in E_i} w(e) \right) \leq (4 + k)^2 \sum_{e \in E(G)} w(e) = \Oh(n).
$$
This proves the first part of the claim.

For the second part of the claim, consider any $i \in \{\ell' + 1, \dots, \ell \}$ and observe that by our assumptions, $A \cup (V_0 \cap V_i)$ separates $(V_0 \cup \dots \cup V_{\ell'} \cup V_{i + 1} \cup \dots \cup V_\ell) \setminus V_i$ from $V_i \setminus V_0$. Hence it suffices to repeatedly apply \Cref{l:inserting_paths}.
\end{proof}

For every $u \in V(\widetilde{G})$, we have $\ecc_G(u) = \max(\ecc_{\widetilde{G}}(v), \max_{v \in V(G) \setminus V(\widetilde{G})} \dist_G(u, v))$. Note, that we already know all the distances $\dist_G(u, v)$ for $v \in V(G) \setminus V(\widetilde{G})$. Similarly, we can already compute $\ecc_G(u)$ for every $u \in V(G) \setminus V(\widetilde{G})$. Therefore, it remains to compute $\ecc_{\widetilde{G}}(v)$ for each $v \in V(\widetilde{G})$. Our goal is to show that this can be done efficiently using \Cref{l:main_ecc}.

Now, let $G'$ be the graph obtained from $\tilde{G}$ by replacing every edge $e$ non-indicent to $A$ with $w(e)\geq 2$ with a path of length $w(e)$ consisting of unit-weight edges. This operation again preserves the distances. Since the sum of edge weights in $\tilde{G}$ is of $\Oh(n)$, the total number of vertices in $G'$ is of $\Oh(n)$. For $0 \leq i \leq \ell'$, we write $V'_i$ to denote the set $V_i$ together with all the vertices added as a part of a path between two endpoints in $V_i$.
As $V_i$ is not heavy for each $i\in [\ell']$, we have
$$
|V'_i \setminus V'_0| \leq |V_i| + \sum_{e \in E_i} w(e) = \Oh(n^{\delta'})\qquad \textrm{for all }i\in [\ell'].
$$

Let $G_0$ denote the graph $G'[V'_0]$ and let $G_0^*$ denote the graph $G'- A$ with $V'_i - V'_0$ contracted to a single vertex $v_i^*$, for each $i \in [\ell']$; note that, all edges of $G_0$ and $G_0^*$ have unit weight.

\begin{claim}
	The graph $G_0^*$ is does not contain $K_{t}$ as a minor, where $t = \Oh(\sqrt{g})$.
\end{claim}

\begin{proof}
Let $\bar{G}_0$ denote the graph obtained by taking $G_0$ and adding a clique on $V_0 \cap V_i$ for every $i \in [\ell']$.
By lemma assumptions and the fact that subdividing edges does not increase the Euler genus, $\bar{G}_0$ has Euler genus at most $g$. In particular, $\bar{G}_0$ is $K_{t'}$-minor-free for some $t' = \Oh(\sqrt{g})$, because the Euler genus of $K_{t'}$ is $\Omega({t'}^2)$.

Similarly, let $\bar{G}_0^*$ be the graph obtained by taking $G_0^*$ and adding a clique on each $V_0 \cap V_i$.
Note, that $\bar{G}_0^* - \{v_1^*, \dots, v_{\ell'}^*\}$ is precisely $\bar{G}_0$. Let $t = \max(t', 6)$.
Recall that a minor model of a clique $K_t$ consists of $t$ pairwise vertex-disjoint connected subgraphs, called
branch sets, such that there is at least one edge between each pair of the branch sets.
Consider a minor model $\varphi$ of $K_{t}$ in $\bar{G}^*_0$.
Note that $\varphi$ cannot contain any singleton branch set of the form $\{v^*_i\}$, for the degree of $v^*_i$ in $\bar{G}^*_0$ is at most $4 < t - 1$. Furthermore, since $N_{\bar{G}^*_0}(v^*_i) = V_0 \cap V_i$, any branch set containing $v^*_i$ and at least one other vertex contains some $u \in V_0 \cap V_i$, and $N_{\bar{G}^*_0}(v^*_i)\subseteq N_{\bar{G}^*_0}(u)$, hence removing $v^*_i$ from this branch set preserves the model. Therefore, we can assume without loss of generality that all branch sets of $\varphi$ are disjoint from $\{v^*_1, \dots, v^*_{\ell'}\}$, hence $\varphi$ is a minor model of $K_{t}$ in $\bar{G}_0$. This is a contradiction, as $t \geq t'$ and $\bar{G}_0$ is $K_{t'}$-minor-free. Therefore, $\bar{G}_0^*$ is $K_t$-minor-free, hence $G_0^*$ also.
\end{proof}

Let $\rho' = \frac{2 - 108 \delta}{151} > 0$. The graph $G^*_0$ is a unit-weight graph and is $K_{t}$-minor-free.
Hence, by applying \Cref{t:r_division} to $G^*_0$ (with $\varepsilon = \rho'/2$)
we obtain an $n^{\rho'}$-division $\mathcal{R}_0$ in time $\Oh(n^{1 + \rho'})$.
We extend it to $G' - A$ by mapping every contracted vertex $v^*_i$ to $N_{G' - A}[V'_i - V'_0] = (V'_i - V'_0) \cup (V_0 \cap V_i)$. Formally, we put $V''_i \coloneqq N_{G' - A}[V'_i - V'_0]$ and 
$$
\mathcal{R} \coloneqq \left\{ (R_0 \cap V'_0) \cup \bigcup_{i \colon v^*_i \in R_0} V''_i \colon R_0 \in \mathcal{R}_0 \right\}.
$$

Now, we argue that $\mathcal{R}$ is a reasonable division of $G' - A$. Clearly, all sets $R \in \mathcal{R}$ are connected in $G' - A$. Pick any $R \in \mathcal{R}$ and let $R_0$ be its corresponding set in $\mathcal{R}_0$.
Every vertex $v^*_i$ is mapped to a set of size $\Oh(n^{\delta'})$, therefore
$$|R| \leq |R_0| \cdot \Oh(n^{\delta'}) = \Oh(n^{\rho' + \delta'}).$$

By our construction, for every $i \in [\ell']$, $R$ is either disjoint from $V'_i - V'_0$ or contains whole $N_{G' - A}[V'_i - V'_0]$. This means that no vertex belonging to any $V'_i - V'_0$ can be in $\partial R$, hence $\partial R \subseteq V'_0$.

Pick any $u \in \partial R \cap R_0$. Assume that $u \not\in \partial R_0$. Then every vertex of $N_{G_0^*}(u)$ must be in $R_0$, hence $N_{G - A'}(u) \subseteq R$, which is a contradiction. This means that $\partial R \cap R_0 \subseteq \partial R_0$.

Pick any $u \in \partial R - R_0$. Then, $u \in V_0 \cap V_i$ for some $i \in [\ell']$ such that $v_i^* \in R_0$. Moreover, $v_i^* \in \partial R_0$ and is adjacent to $u$ in $G_0^*$. The number of such $u$ is bounded by $4 |\partial R_0 \cap \{ v_1^*, \dots, v_{\ell'}^* \}|$.

Putting two cases together, we obtain:
$$
\sum_{R \in \mathcal{R}} |\partial R| = \sum_{R \in \mathcal{R}} \left(|\partial R \cap R_0| + |\partial R - R_0|\right) \leq \sum_{R_0 \in \mathcal{R}_0} \left(|\partial R_0| + 4 |\partial R_0 \cap \{ v_1^*, \dots, v_{\ell'}^* \}|\right) = \Oh(n^{1 - \frac{1}{2}\rho'}).
$$

It remains to show the following claim.

\begin{claim}
Pick any $R \in \mathcal{R}, s_R \in R$. The number of different distance profiles on $R$ relative to $s_R$ in $G' - A$ is of $\Oh(n^{48\rho' + 54\delta'})$.
\end{claim}
\begin{proof}
We look at every vertex $v \in V(G') \setminus A$ and consider three cases: $v \in R$, $v \in V'_0$, and $v \in V'_i \setminus (V'_0 \cup R)$ for some $i \in [\ell']$. By our construction, $R \cap V'_0$ is non-empty, hence w.l.o.g. we can assume that $s_R \in V'_0$ as whether two vertices have the same profile on $R$ is independent of the choice of the pivot vertex.

In the first case, there are at most $|R| = \Oh(n^{\rho' + \delta'})$ such vertices, hence they realise at most that many profiles.

In the second case, we want to observe that profile of any vertex $v \in V'_0$ on $R$ depends only on its profile on $R \cap V'_0$ (relative to $s_R$). Pick any $t \in R - V'_0$. Then $t \in V'_i - V'_0$ for some $i \in [\ell']$, $V_i \cap V_0 \subseteq R \cap V'_0$, and every path from $v$ to $t$ intersects $V_i \cap V_0$. In particular, distances from $v$ to vertices of $V_i \cap V_0$ determine its distance to $t$, which proves the observation.

Let $\tilde{G}_0$ denote the graph obtained by taking $G'[V'_0]$ and for every $i \in [\ell'], u, v \in V_0 \cap V_i$ adding a disjoint path from $u$ to $v$ of length $\dist(u, v)$. Let $P_i$ denote the vertex set of paths added between $V_0 \cap V_i$. For every $t \in V'_0$ we have $\dist_{G' - A}(v, t) = \dist_{\tilde{G}_0}(v, t)$, so it suffices to bound the number of profiles on $R \cap V'_0$ in $\tilde{G}_0$. By our assumptions, $\tilde{G}_0$ has Euler genus bounded by $g$ and all $P_i$ are of size $\Oh(n^{\delta'})$.

Let $R_0$ be the set of $\mathcal{R}_0$ corresponding to $R$. Let $\tilde{R}_0$ denote the set $(R \cap V'_0) \cup \bigcup_{i : v^*_i \in R_0} P_i$. Such set is connected in $\tilde{G}_0$. Moreover, similarly to $R$, its size is $\Oh(n^{\rho' + \delta'})$. Applying \Cref{thm:distprofiles}, we get that the number of distance profiles on $\tilde{R}_0$ in $\tilde{G}_0$ is $\Oh(n^{12(\rho' + \delta')})$, which also bounds the number of profiles on $R$ in $G' - A$ realised by $V'_0$.

For the third case, assume $v \in V'_i \setminus (V'_0 \cup R)$ for some $i\in [\ell']$. Every path from $v$ to any vertex of $R$ in $G' - A$ intersects $V_i \cap V_0$. Let $v_1, \dots v_p$ be the vertices of $V_i \cap V_0$, where $p \leq 4$. The profile of $v$ on $R$ is then determined by the following:
\begin{itemize}[nosep]
 \item[(a)] the profile of each $v_j$ on $R$,
 \item[(b)] $\dist_{G' - A}(v, v_j) - \dist_{G' - A}(v, v_1)$ for each $2 \leq j \leq p$, and
 \item[(c)] $\dist_{G' - A}(s_R, v_j) - \dist_{G' - A}(s_R, v_1)$ for each $2 \leq j \leq p$ where $s_R$ is some pivot vertex of $R$.
\end{itemize}
By the previous case, the number of distance profiles of each $v_j$ is $\Oh(n^{12(\rho' + \delta')})$. The distances between $v$ and $v_j$ are bounded by $|V'_i|$, hence each quantity described in (b) can take $\Oh(n^{\delta'})$ different possible values. Similarly, since $v_1$ and $v_j$ are connected via $V'_i$, $|\dist_{G' - A}(s_R, v_j) - \dist_{G' - A}(s_R, v_1)| \leq \Oh(n^{\delta'})$. The number of different possible profiles of such $v$ is therefore bounded by $\Oh(n^{48(\rho' + \delta') + 6\delta'}) = \Oh(n^{48\rho' + 54\delta'})$. This finishes the proof of the claim.
\end{proof}

Now we can apply \Cref{l:main_ecc} to graph $G'$ with apex set $A$, $X = V(\widetilde{G})$, and the following constants: $$\rho = \rho' + \delta',\qquad \gamma = 1 - \frac{1}{2}\rho',\quad \textrm{and}\quad \alpha = 48\rho' + 54 \delta'.$$ This allows us to calculate all $V(\widetilde{G})$-eccentricities in $G'$ in time
$$
\Oh \left( \left(
	n^{ 2 - \frac{1}{2} \rho' } +
	n^{ 1 + 48\rho' + 54 \delta' }
\right) \log^k n \right) =
\Oh \left( n^{1 + \frac{150 + 54 \delta}{151}} \log^k n \right).
$$
Since for each $v\in V(\widetilde{G})$ we have $\ecc_{\widetilde{G}}(v) = \max_{u \in V(\widetilde{G})} \dist_{\widetilde{G}}(v, u) = \max_{u \in V(\widetilde{G})} \dist_{G'}(v, u)$, this means that we have successfully computed all the eccentricities in $\widetilde{G}$; and as we argued, this is enough to compute all the eccentricities in $G$ as well.

Finally, the total running time of the algorithm is
$$
\Oh \left( n^{1 + \frac{150 + 54 \delta}{151}} \log^k n + n^{2 - \delta' + \delta} \right) =
\Oh \left( n^{1 + \frac{150 + 54 \delta}{151}} \log^k n \right).
$$\qedhere
\end{proof}


\begin{lemma}\label{l:star2}
Fix constants $k, g \in \mathbb{N}, 0 < \delta < \frac{1}{54}$. Assume we are given $n \in \mathbb{N}$, an edge-weighted graph $G$ on at most $n$ vertices with a weight function $w \colon E(G) \to \mathbb{N}$, a vertex subset $A$ and a collection of non-empty vertex subsets $V_0, V_1, \dots, V_\ell$ satisfying the same conditions as in \Cref{l:star} with the following differences:
\begin{itemize}
	\item we don't require $G[V_i - V_0]$ to be connected and $V_i - V_0$ to be adjacent to whole $V_i \cap V_0$;
	\item instead of $|V_0 \cap V_i| \leq 4$, we require $|V_0 \cap V_i| \leq k$.
\end{itemize}
Then, we can compute the eccentricity of every vertex of $G$ in time $\Oh \left( n^{1 + \frac{150 + 54 \delta}{151}} \log^{k + 5g} n \right)$.
\end{lemma}

\begin{proof}
We will reduce our input to one which will satisfy the conditions of \Cref{l:star}. We start by addressing the adhesions $V_0 \cap V_i$ containing too many vertices.

Let $G_0$ denote the graph $G[V_0]$ with cliques placed at $V_0 \cap V_i$ for every $i \in [\ell]$.
For every $i \in [\ell]$ we repeat the following procedure: while $|V_0 \cap V_i| > 4$,
remove arbitrary $5$ vertices from $V_0 \cap V_i$. Since $|V_0 \cap V_i| \leq k$ for each $i\in [\ell]$,
this procedure can be implemented in total time $\Oh(n)$. As a result, at the end we have $|V_0 \cap V_i| \leq 4$ for all $i \in [\ell]$. Let $M$ be the set of all the removed vertices. By our assumptions, $G_0$ has Euler genus bounded by $g$, hence it cannot contain $g + 1$ pairwise disjoint copies of $K_5$
(as the Euler genus of a graph is the sum of the Euler genera of its 2-connected components~\cite{StahlB77} and $K_5$ is not planar). Each removed quintiple of vertices induces a $K_5$ in $G_0$, hence we have $|M| \leq 5g$. We set $A' = A \cup M$ and may thus assume that $V_i$ is disjoint from $A'$ for all $0 \leq i \leq \ell$.

Now, fix $i \in [\ell]$. Let $C^i_1, \dots, C^i_{r_i}$ denote the connected components of $V_i - V_0$ in $G - A'$. We define $W^i_j := N_{G - A'}[C^i_j]$ for every $j \in [r_i]$. Clearly, all $W^i_j$ induce a connected subgraph of $G$ and satisfy $N_{G - A'}(W^i_j - V_0) = W^i_j \cap V_0$. We put $V'_0 := V_0$ and enumerate
$$
\{V'_1, V'_2, \dots V'_{\ell'}\} := \{ W^i_j \colon i \in [\ell], j \in [r_i] \}.
$$
It is easy to verify that the sets $A'$ and $V'_0, V'_1, \dots, V'_{\ell'}$ satisfy the conditions of \Cref{l:star}. We apply said lemma to calculate the eccentricity of every vertex of $G$ in the desired time.
\end{proof}



The next statement is a reformulation of \Cref{thm:main-decomp}.

\begin{theorem}
Fix constants $k, g \in \mathbb{N}$. Assume we are given a graph $G$ on $n$ vertices together with its tree decomposition $(T, \beta)$ and a set of private apices $A_t \subseteq \beta(t)$ for each node $t\in V(T)$ such that the following conditions hold:
\begin{itemize}[nosep]
 \item For every node $t \in V(T)$, we have $|A_t| \leq k$.
 \item For every edge $st \in E(T)$,  we have $|\beta(v) \cap \beta(u)|\leq k$.
 \item For every node $t \in V(T)$, graph obtained by taking $G[\beta(t)] - A_t$ and turning  $(\beta(t) \cap \beta(s))\setminus A_t$ into a clique for every edge $st \in E(T)$ has Euler genus bounded by $g$.
\end{itemize}
Then, we can compute the eccentricity of every vertex of $G$ in time $\Oh \left( n^{1 + \frac{355}{356}} \log^{k + 5g} n \right)$.
\end{theorem}

\begin{proof}
We may assume that $|V(T)|\leq n$, for every tree decomposition with no two bags comparable by inclusion has this property; and adjacent comparable bags can be merged by contracting the edge between them.

For a node $t\in V(T)$, by the {\em{weight}} of $t$ we mean the size of the corresponding bag, that is, $|\beta(t)|$. For any subset of nodes $S \subseteq V(T)$, we define $\beta(S) \coloneqq \bigcup_{t \in S} \beta(t)$ By the {\em{weight}} of $S$, we mean the total weight of the elements of $S$, that is, $\sum_{t\in S} |\beta(t)|$. 

\begin{claim}\label{cl:weight-T}
The weight of $V(T)$ is of $\Oh(n)$.
\end{claim}

\begin{proof}
The sets $\beta'(t) := \beta(t) - \bigcup_{s \in N_T(t)} \beta(s)$ are pairwise disjoint. We have
$$
\sum_{t \in V(T)} |\beta(t)| =
\sum_{t \in V(T)} |\beta'(t)| + 2 \cdot \sum_{st \in E(T)} |\beta(s) \cap \beta(t)| \leq
|V(T)| + 2k|E(T)| = \Oh(n).
$$
\end{proof}

Since every bag induces a graph of bounded Euler genus, the number of edges contained in a bag is linear in its size. In particular, this implies that the total number of edges of $G$ is also bounded by $\Oh(n)$.

We set $$\delta \coloneqq \frac{1}{356}\qquad\textrm{and}\qquad \Delta \coloneqq \frac{355}{356}.$$ Root the tree $T$ in an arbitrarily chosen node; this naturally imposes an ancestor-descendant relation in $T$ (for convenience, every node is considered its own ancestor and descendant).

We start by partitioning $T$ into connected subtrees using the following procedure.
We proceed bottom-up over $T$, processing nodes in any order so that a node is processed after all its strict descendants have been processed. Along the way, we mark some nodes and split the edges of $T$ into heavy and light. Let $t \in V(T)$ be the currently processed non-root node of $T$ and let $e \in E(T)$ be the edge connecting $t$ with its parent. If the total weight of all the unmarked nodes that are descendants of $t$ is at least $n^\delta$ (recall that this includes $t$ itself as well), then we declare $e$ heavy and mark all the descendants of $t$ that were unmarked so far. Otherwise, the edge $e$ is declared light and the procedure proceeds to further nodes of $T$.

Observe that
removing all heavy edges splits $T$ into connected subtrees, say $T'_1, \cdots T'_m$. All of the subtrees, except for possibly the subtree containing the root node, are of weight at least $n^\delta$. In particular, the number of subtrees $m$, and therefore the number of heavy edges, is  bounded by $\Oh(n^{1 - \delta})$. Moreover, in every subtree $T'_i$, removing the node closest to the root splits $T'_i$ into smaller components, each of weight less than $n^\delta$.

Fix a heavy edge $e$ and let $T^e_1$ and $T^e_2$ be the two subtrees into which $T$ splits after removing~$e$. Let $X^e_i = \beta(T^e_i)$ for $i \in \{1, 2\}$. Put $A_e = X^e_1 \setminus X^e_2$, $C_e = X^e_2 \setminus X^e_1$, and $B_e = X^e_1 \cap X^e_2$. By the properties of tree decompositions, such choice of $A_e, B_e, C_e$ satisfies the conditions of \Cref{l:single_adhesion}, hence in time $\Oh(n \log^{k - 1} n)$ we can compute $\max_{v \in X^e_2} \dist_G(u,v)$ for every $u \in X^e_1$, and $\max_{u \in X^e_1} \dist_G(u,v)$ for every $v \in X^e_2$. Computing this for every heavy edge $e$ takes total time $\Oh(n^{2 - \delta} \log^{k - 1} n)$.

Fix any subtree $T'=T'_j$. Let $e_1 = t^{e_1}_1t^{e_1}_2, e_2 = t^{e_2}_1 t^{e_2}_2, \dots, e_\ell = t^{e_\ell}_1 t^{e_\ell}_2$ denote the heavy edges incident to $T'$, where $t^{e_i}_1 \in V(T')$ and $V(T') \subseteq V(T_1^{e_i})$ for every $i \in [\ell]$.
For a vertex $v \in \beta(T')$, let
$$d_0(v) = \max_{u \in \beta(T')} \dist_G(v, u)\qquad\textrm{and}\qquad d_i(v) = \max_{u \in X_2^{e_i}}\dist_G(v,u),\quad\textrm{for } i \in [\ell].$$ We have $\ecc(v) = \max \{ d_i(v)\colon i\in \{0,1,\ldots,\ell\}\}$.The values of $d_i(v)$ are already calculated for all $i\in [\ell]$, hence it remains to compute $d_0(v)$.

For every $i \in [\ell]$ and every pair of vertices $u, v \in \beta(t^{e_i}_1) \cap \beta(t^{e_i}_2)$ we find a shortest path between $u$ and $v$ with all internal vertices inside $X^{e_i}_2$ (or determine that it doesn't exist). For a fixed $u, v$ this can be done in time $\Oh(n)$. Since in total we perform this step at most $2k^2$ times per heavy edge, it takes $\Oh(n^{2 - \delta})$ time in total. Let $P_{i, u, v}$ denote such path, assuming it exists.

Let $G'$ denote the graph obtained from $G[\beta(T')]$ by taking every $i, u, v$ for which $P_{i, u, v}$ exists and adding an edge between $u$ and $v$ of weight equal to the total weight of $P_{i, u, v}$.
The weight of every edge inserted in $\beta(t^{e_i}_1) \cap \beta(t^{e_i}_2)$ is bounded by $|X^{e_i}_2|+1$. The total weight of all edges inserted is therefore at most
$$
\sum_{i \in [\ell]} |\beta(t^{e_i}_1) \cap \beta(t^{e_i}_2)|^2 \cdot (|X^{e_i}_2|+1) \leq
k^2 \sum_{i \in [\ell]} (|X^{e_i}_2|+1) = \Oh(n),
$$
where the last equality follows from the fact that all the trees $T^{e_i}_2$ are pairwise disjoint.
By \Cref{l:inserting_paths}, we have $\dist_{G'}(u, v) = \dist_G(u, v)$ for each $u, v \in \beta(T')$. Hence, computing $d_0(v)$ for every $v \in \beta(T')$ is equivalent to computing the eccentricity of every vertex in $G'$.

If the size of $\beta(T')$ is smaller than $n^\Delta$, we compute the eccentricities naively in time $\Oh(|\beta(T')|^2)$, 
noting that $G'$ has $\Oh(|\beta(T')|)$ edges (thanks to Claim~\ref{cl:weight-T} and bounded genus assumption 
of the last bullet of the theorem statement). Otherwise, we argue that we can use the algorithm in \Cref{l:star} as follows.

Let $t$ be the node of $T'$ closest to the root. Let $s_1, \dots, s_p$ be the children of $t$ in $T$ and let $T''_i$ denote the connected component of $T' - \{t\}$ containing $s_i$. Set $V_0 = \beta(t)$ and $V_i = \beta(T''_i)$ for $i \in [p]$.

It is now easy to verify that $G'$ and sets $A, \{V_i\colon 0\leq i\leq p\}$ selected this way satisfy the assumptions of \Cref{l:star2}. This allows us to use it to compute the eccentricities in $G'$ in time
$$
\Oh \left( n^{1 + \frac{150 + 54\delta}{151}} \log^{k + 5g} n \right) =
\Oh \left( n^{1 + \frac{354}{356}} \log^{k + 5g} n \right).
$$
As we argued, from these eccentricities, we may easily compute all the eccentricities in $G$.

Now, let us analyse the total running time of the whole algorithm. We invoke \Cref{l:star} $\Oh(n^{1 - \Delta})$ times, since we apply it only to subtrees $T'_i$ of size at least $n^\Delta$. The total running time of those applications is hence
$$
\Oh \left( n^{2 + \frac{354}{356} - \Delta} \log^{k + 5g} n \right) =
\Oh \left( n^{1 + \frac{355}{356}} \log^{k + 5g} n \right).
$$
We compute the eccentricities naively for subtrees smaller than $n^\Delta$, hence the total running time of this computation is
$$
\sum_{i \in [m] \colon |\beta(T'_i)| \leq n^\Delta} |\beta(T'_i)|^2 \leq
n^\Delta \cdot \sum_{i \in m} |\beta(T'_i)| = \Oh(n^{1 + \Delta})=\Oh\left(n^{1+\frac{355}{356}}\right).
$$
The rest of computation can be done in $\Oh(n^{2 - \delta} \log^k n)$. Therefore, the whole algorithm runs in time $\Oh \left( n^{1 + \frac{355}{356}} \log^{k + 5g} n \right)$.
\end{proof}



To standardize coordination policy training and evaluation, we introduce the \texttt{HelpEnvironment} wrapper. This tool converts any Gym-compatible environment \citep{1606.01540, towers2024gymnasium} into an \textit{MDP for the coordination policy} that preserves the original state space $\mathcal{S}$ but replaces the action space with a binary choice $\{\texttt{n}, \texttt{e}\}$, representing the coordination policy's decision to request control (novice acts) or yield control (expert acts). At each timestep, the wrapper resolves the coordination policy $\mu$'s decision into a concrete environment action: if $\mu(s_t) = \texttt{n}$, then $x_t = a^n_t \sim \pi_n(s_t)$ which is the novice's proposed action; if $\mu(s_t) = \texttt{e}$, then $x_t = a^e_t \sim \pi_e(s_t)$ which is the expert's action. Consequently, the next state $s_{t+1}$ is generated by the base environment's transition dynamics $P(s_{t+1} | s_t, a_t)$.

\autoref{fig:wrapper_demo}'s \colorbox{sbBlue025}{blue region} provides a high-level demonstration of coordination environment. In addition, \autoref{fig:env_wrapper} provides its pseudocode. The \texttt{make()} utility function initializes four coordination environments: a training environment featuring the simulated novice $\tilde{\pi}_n$ and simulated expert $\pi_n$ coordinating on tasks sampled from $\mathcal{E}_{\text{train}}$, a simulated validation environment which is similar to the training environment but includes held-out tasks sampled from $\mathcal{E}_{\text{train}}$, 
a validation environment featuring $\pi_n$ and $\pi_e$ coordinating on tasks sampled from $\mathcal{E}_{\text{test}}$, 
and a testing environment which is similar to the validation environment but includes held-out tasks sampled from $\mathcal{E}_{\text{test}}$. 
By modularizing the coordination logic into a reusable environment wrapper, we support systematic evaluation of policies across diverse domains (e.g., grid navigation, procedural generation, robotic manipulation) and enforce a standardized interface for control delegation between novice and expert policies.

The \texttt{HelpEnvironment} wrapper performs three essential tasks: \begin{itemize}
    \item \textbf{Policy Integration:} As shown in \autoref{fig:wrapper_demo} and \autoref{fig:env_wrapper}, the wrapper accepts both novice and expert policies, managing the handover of control between them dynamically. This process is central to evaluating and optimizing coordination strategies.
    \item \textbf{Cost Computation:} The wrapper incorporates cost functions that consider the environment's reward, switching costs, and expert labor costs. These cost components are crucial for realistic assessments of coordination trade-offs.
    \item \textbf{Performance Tracking:} To facilitate robust research, the wrapper includes standardized metrics for evaluating coordination performance. These metrics include cumulative cost, task completion rates, and the frequency of expert interventions, ensuring comprehensive assessments.
\end{itemize}

\autoref{fig:env_wrapper} highlights the modularity of this design, which enables researchers to seamlessly adapt the wrapper to new environments or agents. By leveraging this implementation, users can efficiently conduct experiments on coordination policy without significant modifications to existing environments.



\subsection{Training Framework}

\begin{algorithm}
\caption{\methodname{} Training}
\KwIn{Coarse-to-fine Autoencoder $\text{Enc}$, $\text{Dec}$}
\KwOut{}
\For{$i \gets 1$ \textbf{to} $n-1$}{
    \For{$j \gets 1$ \textbf{to} $n-i$}{
        \If{$L[j] > L[j+1]$}{
            Swap $L[j]$ and $L[j+1]$
        }
    }
}
\Return $L$
\end{algorithm}

The \texttt{train.py} script, with pseudocode in \autoref{fig:train}, orchestrates the entire training process for the \ourMethod problem.  It begins by loading configuration parameters from a specified file. Using these configurations, it instantiates the necessary components: training, validation, and testing environments; a coordination policy; and an evaluator.  If the configuration specifies a simple, non-trainable algorithm (e.g., \textsc{AlwaysExpert}), the script directly evaluates the pre-defined policy. Otherwise, it instantiates a training algorithm based on \texttt{Algorithm} class, then calls the \texttt{train()} method. This main training loop takes the policy, environments, and evaluator as input, iteratively improving the coordination policy. The evaluator periodically assesses the policy's performance on validation splits to track progress.





The \ourMethod benchmark supports the training of coordination policies using the \texttt{Algorithm} class, a modular and extensible framework for implementing training routines. The \texttt{Algorithm} class encapsulates the core training logic, providing methods for initializing training parameters and managing the iterative process of policy improvement. An overview of the class implementation is shown in \autoref{fig:wrapper_demo}'s \colorbox{sbOrange025}{orange region} with its pseudocode available at \autoref{fig:algorithm}. This class organizes the training flow, allowing researchers to define how policies should be updated based on interactions with the environment or datasets.

The \texttt{train()} method iterates through training cycles, calling the \texttt{train\_one\_iteration} function in each loop to refine the policy. At specified intervals, the method invokes the evaluator to assess the policy’s performance on validation and test environments, providing important feedback and facilitating the saving of the best model. This iterative process, demonstrated in \autoref{fig:algorithm}, is central to the policy optimization.

By following the structure outlined in \autoref{fig:algorithm}, researchers can easily integrate new algorithms into the benchmark. The modular design ensures that the training and evaluation pipeline is easily adaptable, promoting reproducibility and flexibility for different types of coordination policy experiments.

\subsection{Evaluation Framework}

\section{Evaluation}


\begin{table}[t]
    \centering
    % \vspace{-0.1in}
    \scalebox{0.78}{
    % \begin{small}
        \begin{tabular}{lccc}
            \toprule
            \multirow{2}*{\textbf{MoE Models}} & \textbf{Parameters} & \textbf{Experts Per Layer} & \textbf{Num. of} \\
            & \textbf{(active / total)} & \textbf{(active / total)} & \textbf{Layers} \\
            \otoprule 
            \mixtral~\cite{jiang2024mixtral} & 12.9B / 46.7B & 2 / 8 & 32 \\
            % \hline
            \qwen~\cite{yang2024qwen2} & 2.7B / 14.3B & 4 / 60 & 24 \\
            \phimoe~\cite{abdin2024phi} & 6.6B / 42B & 2 / 16 & 32 \\
            \bottomrule 
        \end{tabular}
    % \end{small}
    }
    \caption{Characteristics of three \MoE models in evaluation.}
    \vspace{-0.2in}
    \label{table:eval-moe-models}
\end{table}








\subsection{Experimental Setup}
\label{subsec:eval-setup}


% \begin{figure*}[t]
%     \centering
%     \begin{subfigure}[t]{0.48\textwidth}
%         \centering
%         \includegraphics[width=.9\linewidth]{figs/eval-overall-lmsys.pdf}
%         \caption{Serving three \MoE models with LMSYS-Chat-1M dataset.}
%     \end{subfigure}
%     \begin{subfigure}[t]{0.48\textwidth}
%         \centering
%         \includegraphics[width=.9\linewidth]{figs/eval-overall-sharegpt.pdf}
%         \caption{Serving three \MoE models with ShareGPT dataset.}
%     \end{subfigure}
%     \caption{Overall performance of prefill and decode stages for \sys and other four baselines.}
%     \label{fig:eval-overall.pdf}
% \end{figure*}


\noindent \textbf{Testbed.}
We conduct all experiments on a six-GPU testbed, where each GPU is an NVIDIA GeForce RTX 3090 with 24 GB GPU memory. 
%
All GPUs are inter-connected using pairwise NVLinks and connected to the CPU memory using PCIe 4.0 with 32GB/s bandwidth. 
%
Additionally, the testbed has a total of 32 AMD Ryzen Threadripper PRO 3955WX CPU cores and 480 GB CPU memory.


\noindent \textbf{Models.}
We employ three popular \MoE-based \LLMs in our evaluation: \mixtral~\cite{jiang2024mixtral}, \qwen~\cite{yang2024qwen2}, and \phimoe~\cite{abdin2024phi}.
Table~\ref{table:eval-moe-models} describes the parameters, number of \MoE layers, and number of experts per layer for the three models.
Following the evaluation of existing works~\cite{song2024promoe}, we profile the models to set the optimal prefetch distance $d$ to three before evaluation.
% We set $d$ of \mixtral, \qwen, and \phimoe to \todo{$xxx$}, \todo{$xxx$}, and \todo{$xxx$}, respectively.


\noindent \textbf{Datasets and traces.}
We employ two real-world prompt datasets commonly used for \LLM evaluation: LMSYS-Chat-1M~\cite{zheng2023lmsys} and ShareGPT~\cite{sharegpt}.
%
For most experiments, we split the sampled datasets in a standard 7:3 ratio, where 70\% of the prompts' context data (\ie, semantic embeddings and expert maps) are stored in \sys's Expert Map Store, and 30\% of the prompts are used for testing. 
%
For online serving experiments, we empty the Expert Map Store and use real-world \LLM inference traces~\cite{patel2024splitwise,stojkovic2025dynamollm} released by Microsoft Azure to set input and generation lengths and drive invocations.

\noindent \textbf{Baselines.}
We compare \sys against four \SOTA \MoE serving baselines:
1) \textbf{MoE-Infinity}~\cite{xue2024moe} uses coarse-grained request-level expert activation patterns and synchronous expert prediction and prefetching for \MoE serving. 
We prepare the expert activation matrix collection for MoE-Infinity before evaluation for a fair comparison.
%
% However, the open-sourced MoE-Infinity codebase~\cite{moe-infinity-code} lacks some features described in its original paper, we had to modify
%y 
2) \textbf{ProMoE}~\cite{song2024promoe} employs a stride-based speculative expert prefetching approach for \MoE serving. Since the codebase of ProMoE is not open-sourced and requires training predictors for each \MoE model, we reproduced a prototype of ProMoE on top of MoE-Infinity in our best effort.
%
3) \textbf{Mixtral-Offloading}~\cite{eliseev2023fast} combines a layer-wise speculative expert prefetching and a \LRU-based expert cache. 
%
4) \textbf{DeepSpeend-Inference} employs an expert-agnostic layer-wise parameter offloading approach, which uses pure on-demand loading and does not support prefetching. 
%
We implement the offloading logic of DeepSpeed-Inference in the MoE-Infinity codebase and add an expert cache for a fair comparison.
We enable all baselines to serve \MoE models from HuggingFace Transformer~\cite{wolf2020huggingface}. 


\noindent \textbf{Metrics.}
Following the standard evaluation methodology of existing works~\cite{song2024promoe,xue2024moe,zhong2024distserve,agrawal2024taming} on \LLM serving, we report the performance of the prefill and decode stages separately. 
We measure Time-to-First-Token (TTFT) for the prefill stage and Time-Per-Output-Token (TPOT) for the decode stage.
Additionally, we also report other system metrics, such as expert hit rate and overheads, for detailed evaluation.


% \noindent \textbf{\sys's setting.}
% The hyperparameters of \sys containing the prefetch distance $d$ for each \MoE model, Expert Map Store capacity $C$, and Expert Cache memory limit $M$.
% For most experiments, we profile the \MoE models and set the prefetch distance $d$ to their optimal values. The Expert Map Store capacity $C$ is set to \todo{$xxx$} expert maps. We configure the Expert Cache memory limit to \todo{$xxx$} GB.
% The hyperparameter sensitivity is analyzed in \S\ref{subsec:eval-sensitivity}.


\begin{figure}[t]
  \centering
  \includegraphics[width=.95\linewidth]{figs/eval-overall-arxiv.pdf}
  \vspace{-0.15in}
  \caption{Overall performance of prefill and decode stages for \sys and other four baselines.}
  \vspace{-0.2in}
  \label{fig:eval-overall}
\end{figure}


\subsection{Overall Performance}
\label{subsec:eval-overall}



We first evaluate the performance of prefill and decode stages when running \sys and other baselines with the three \MoE models, where we measure Time-To-First-Token (TTFT) and Time-Per-Output-Token (TPOT) for each stage.
Note that the inference latency with expert offloading tends to be higher than no offloading due to two reasons: 
1) During inference, an excessive amount of parameters in \MoE models are loaded and offloaded, which prolongs the inference latency.
2) All baselines and \sys are implemented on top of the MoE-Infinity codebase~\cite{moe-infinity-code}, whose inference latency is inherently impacted by MoE-Infinity's implementation.
Nevertheless, comparing \sys and baselines is fair with the same experimental setup.

Figure~\ref{fig:eval-overall} shows the \TTFT, \TPOT, and expert hit rate of \sys and other four baselines when serving three \MoE models with LMSYS-Chat-1M and ShareGPT datasets, respectively.
DeepSpeed has both the worst \TTFT and \TPOT due to expert-agnostic offloading and lacking expert prefetching.
While Mixtral-Offloading, ProMoE, and MoE-Infinity perform better than DeepSpeed-Inference, they are underperformed by \sys because of coarse-grained offloading designs.
Compared to DeepSpeed-Inference, Mixtral-Offloading, ProMoE, and MoE-Infinity, our \sys reduces the average \TTFT by 44\%, 35\%, 33\%, 30\%, and reduces the average \TPOT by 70\%, 61\%, 55\%, 48\%, across three \MoE models.
%
% Figure~\ref{fig:eval-overall} also reports the expert hit rate of \sys and each baseline. 
For expert hit rate, Mixtral-Offloading achieves a higher hit rate than the other three baselines because of its synchronous speculative prefetching with a prefetch distance of 1. However, due to synchronous prefetching, its \TTFT and \TPOT are worse than others except DeepSpeed-Inference.
\sys improves the average expert hit rate by 147\%, 11\%, 34\%, and 63\% over DeepSpeed-Inference, Mixtral-Offloading, ProMoE, and MoE-Infinity, respectively.

% \begin{figure}[t]
%   \centering
%   \includegraphics[width=.9\linewidth]{figs/eval-overall-sharegpt.pdf}
%   % \vspace{-0.15in}
%   \caption{}
%   % \vspace{-0.25in}
%   \label{fig:eval-overall-sharegpt.pdf}
% \end{figure}




\subsection{Online Serving Performance}
\label{subsec:eval-online}


Except for the offline evaluation (\ie, Expert Map Store in full capacity before serving), we also evaluate \sys against other baselines in online serving settings.
We empty the Expert Map Store of \sys and the expert activation matrix collection of MoE-Infinity for the online serving experiment.
%
The request traces are derived from Azure \LLM inference traces~\cite{patel2024splitwise,stojkovic2025dynamollm}, with 64 requests randomly sampled to drive LMSYS-Chat-1M prompts for each \MoE model serving. 
To ensure consistency, \sys and all baselines input and generate the exact number of tokens specified in the traces.
%
Figure~\ref{fig:eval-online-serve} illustrates the CDF of end-to-end request latency across three \MoE models. The results demonstrate that \sys significantly reduces overall request latency compared to other baselines in online serving scenarios.


\begin{figure}[t]
  \centering
  \includegraphics[width=.95\linewidth]{figs/eval-online-serve-arxiv.pdf}
  \vspace{-0.15in}
  \caption{CDF of request latency for \MoE online serving.}
  \vspace{-0.2in}
  \label{fig:eval-online-serve}
\end{figure}



\subsection{Impact of Expert Cache Limits}



We measure the \TPOT of \sys and other baselines by limiting the expert cache memory budget to investigate their performance in the latency-memory trade-off (\S\ref{subsec:bg-latency-memory-tradeoff}).
We mainly focus on \TPOT to show the end-to-end performance impacted by varying cache limits.
Figure~\ref{fig:eval-cache-limit.pdf} shows the \TPOT of \sys and other four baselines when serving three \MoE models under different expert cache limits.
We gradually increase the GPU memory allocated for caching experts from 6 GB to 96 GB while employing the same experimental setting in \S\ref{subsec:eval-overall}.
Similarly, DeepSpeed-Inference has the worst \TPOT due to being expert-agnostic.
\sys consistently outperforms Mixtral-Offloading, ProMoE, and MoE-Infinity under varying expert cache limits.
Especially for limited GPU memory sizes (\eg, 6GB), \sys reduces the \TPOT by 32\%, 24\%, 18\%, and 18\%, compared to DeepSpeed-Inference, Mixtral-Offloading, ProMoE, and MoE-Infinity, across three \MoE models, respectively.
With fine-grained expert offloading, \sys significantly reduces the expert on-demand loading latency while maintaining a lower GPU memory footprint, therefore achieving a better spot in the latency-memory trade-off of \MoE serving.

% \subsection{Impact of Inference Batch Size}

\subsection{Ablation Study}
\label{subsec:eval-ablation}


% \begin{figure}[t]
%   \centering
%   \includegraphics[width=.95\linewidth]{figs/eval-expert-tracking.pdf}
%   % \vspace{-0.15in}
%   \caption{Expert hit rate of different expert pattern tracking approaches.}
%   % \vspace{-0.25in}
%   \label{fig:eval-expert-tracking}
% \end{figure}



We present the ablation study of \sys's design.


\textbf{Effectiveness of expert map search.}
One of \sys's key designs is the expert map, which tracks expert selection preferences in fine granularity.
We evaluate the effectiveness of the expert map against five expert pattern-tracking approaches as follows.
%
1) \textbf{Speculate}: speculative prediction used by Mixtral-Offloading~\cite{eliseev2023fast} and ProMoE~\cite{song2024promoe}, 
%
2) \textbf{Hit count}: request-level expert hit count used by MoE-Infinity~\cite{xue2024moe}, 
%
3) \textbf{Map (T)}: expert map with only trajectory similarity search,
4) \textbf{Map (T+S)}: expert map with both trajectory and semantic similarity search,
%
and
5) \textbf{Map (T+S+$\delta$)}: expert map with full features enabled, including trajectory and semantic similarity search (\S\ref{subsec:design-similarity-match}) and dynamic expert selection (\S\ref{subsec:design-expert-prefetch}).
%
We implement the above methods in \sys's Expert Map Matcher for a fair comparison.
Figure~\ref{fig:eval-expert-tracking} shows the expert hit rate of the above expert pattern tracking methods.
%
Speculative prediction is effective due to the widespread presence of residual connections in Transformer blocks. However, its effectiveness decreases drastically as prefetch distance increases~\cite{song2024promoe}.
%
The request-level expert activation count has the worst performance due to coarse granularity.
%
As features are incrementally restored to \sys's expert map, the expert hit rate gradually increases, demonstrating its effectiveness.

% \textbf{Effectiveness of asynchronous map matching.}




\begin{figure}[t]
  \centering
  \includegraphics[width=.9\linewidth]{figs/eval-cache-limit-arxiv.pdf}
  \vspace{-0.15in}
  \caption{Performance of \sys and other four baselines under varying expert cache limits.}
  \vspace{-0.1in}
  \label{fig:eval-cache-limit.pdf}
\end{figure}

\begin{figure}[!t]
    \centering
    \begin{subfigure}[t]{0.585\linewidth}
        \centering
        \includegraphics[width=\linewidth]{figs/eval-expert-tracking.pdf}
        \caption{Expert pattern tracking approaches.}
        \label{fig:eval-expert-tracking}
    \end{subfigure}
    % \hspace{0.02in}
    \begin{subfigure}[t]{0.385\linewidth}
        \centering
        \includegraphics[width=\linewidth]{figs/eval-prefetch-and-cache-arxiv.pdf}
        \caption{Prefetch and caching.}
        \label{fig:eval-prefetch-and-cache}
    \end{subfigure}
    \vspace{-0.1in}
    \caption{Ablation study of \sys.}
    \label{fig:eval-ablation}
    \vspace{-0.2in}
\end{figure}

\textbf{Effectiveness of expert prefetching and caching.}
We evaluate \sys's expert prefetching and caching against two caching algorithms:
1) \textbf{\LRU} used by Mixtral-Offloading~\cite{eliseev2023fast}
and 
2) \textbf{\LFU} used by MoE-Infinity~\cite{xue2024moe}.
%
Figure~\ref{fig:eval-prefetch-and-cache} depicts the expert hit rate of \sys and two baselines.
The results show that \LRU performs poorly in expert offloading scenarios. Though \LFU achieves a higher hit rate than \LRU, \sys surpasses both, achieving the highest expert hit rate.

\subsection{Sensitivity Analysis}
\label{subsec:eval-sensitivity}


\begin{figure}[t]
  \centering
  \includegraphics[width=.9\linewidth]{figs/eval-prefetch-distance.pdf}
  \vspace{-0.15in}
  \caption{Performance of \sys serving \MoE models with different prefetch distances.}
  \vspace{-0.1in}
  \label{fig:eval-prefetch-distance}
\end{figure}

% \begin{figure}[t]
%   \centering
%   \includegraphics[width=.9\linewidth]{figs/eval-store-capacity.pdf}
%   % \vspace{-0.15in}
%   \caption{Semantic and trajectory similarity lower bounds in \sys's serving with different Expert Map Store capacity.}
%   % \vspace{-0.25in}
%   \label{fig:eval-store-capacity}
% \end{figure}

\begin{figure}[t]
    \centering
    \begin{subfigure}[t]{0.55\linewidth}
        \centering
        \includegraphics[width=\linewidth]{figs/eval-store-capacity.pdf}
        \caption{Expert Map Store capacity.}
        \label{fig:eval-store-capacity}
    \end{subfigure}
    % \hspace{0.02in}
    \begin{subfigure}[t]{0.435\linewidth}
        \centering
        \includegraphics[width=\linewidth]{figs/eval-batch-size-arxiv.pdf}
        \caption{Inference batch size.}
        \label{fig:eval-batch-size}
    \end{subfigure}
    \vspace{-0.1in}
    \caption{Sensitivity analysis of \sys.}
    \vspace{-0.2in}
    \label{fig:eval-sensitivity}
\end{figure}


We analyze the sensitivity of three hyperparameters: prefetch distance of \MoE models, the capacity of Expert Map Store, and inference batch size.


\textbf{Prefetch distance of \MoE models.}
Figure~\ref{fig:eval-prefetch-distance} shows the \TTFT and \TPOT of \sys when serving three \MoE models with different prefetch distances.
%
We have demonstrated that the expert hit rate decreases when gradually increasing the prefetch distance (Figure~\ref{fig:bg-hit-distance}).
%
When the prefetch distance is small ($<3$), \sys cannot perfectly hide its system delay from the inference process, such as the map matching and expert prefetching, leading to the increase of inference latency.
%
With larger prefetch distances ($>3$), \sys has worse expert hit rates that also degrade the performance. 
Therefore, we set the prefetch distance $d$ to 3 for evaluating \sys.


\textbf{Capacity of Expert Map Store.}
We measure the mean semantic and trajectory similarity scores searched in \sys's expert map matching for \MoE model serving.
%
Figure~\ref{fig:eval-store-capacity} presents the mean semantic and trajectory similarity scores of \sys with different Expert Map Store capacity sizes.
%
Both semantic and trajectory similarity scores improve as the store capacity increases.
%
While the similarity scores exhibit a significant increase with capacities below 1K, further capacity expansion yields diminishing similarity gains. 
To minimize \sys's memory overhead, we set \sys's Expert Map Store capacity to 1K in evaluation.


\textbf{Inference batch size.}
We investigate the impact of inference batch size on \sys and three baselines using \mixtral with LMSYS-Chat-1M.
%
Figure~\ref{fig:eval-batch-size} presents the performance of \sys, Mixtral-Offloading, ProMoE, and MoE-Infinity as the batch size increases from one to four. \sys achieves the lowest \TTFT and \TPOT in most cases.


% \textbf{Inference batch size.}


% \subsection{Scalability}
% \label{subsec:eval-scalability}
% From one to six GPUs


\begin{figure}[t]
  \centering
  \includegraphics[width=.92\linewidth]{figs/eval-overhead-latency.pdf}
  \vspace{-0.15in}
  \caption{Latency breakdown of \sys's one inference iteration with three \MoE models.}
  \vspace{-0.1in}
  \label{fig:eval-overhead-latency.pdf}
\end{figure}





\subsection{System Overheads}
\label{subsec:eval-overhead}


\noindent \textbf{Latency overheads of \sys's operations.}
Figure~\ref{fig:eval-overhead-latency.pdf} shows the latency breakdown of one inference iteration in \sys when serving the three \MoE models.
We report any operations of \sys in \S\ref{subsec:eval-overall} that may incur a significant latency delay, including context collection, map matching, expert on-demand loading, expert prefetching, and map update after the iteration completes.
\qwen has lower end-to-end iteration latency than \mixtral and \phimoe because of significantly fewer parameters.
Note that expert prefetching, map matching, and map update tasks are executed asynchronously, aside from the inference process. Hence, they do not contribute to the end-to-end iteration latency.
Excluding three asynchronous tasks, the total delay incurred by other operations is consistently less than 30ms (5\% of the iteration) across three \MoE models, which is negligible compared to the inference latency.


\noindent \textbf{Memory overheads of \sys's Expert Map Store.}
Figure~\ref{fig:eval-overhead-memory.pdf} shows the CPU memory footprint of \sys's Expert Map Store when varying the store capacity from 1K to 32K maps.
The memory needed to store expert maps for \qwen is more than \mixtral and \phimoe because it has more experts per layer over the other two models, which increases the map shape.
Even for the largest capacity (32K), the Expert Map Store requires less than 200MB of memory to store the maps, which is trivial since modern GPU servers usually have abundant CPU memory (\eg, p4d.24xlarge on AWS EC2~\cite{aws-ec2} has over 1100 GB of CPU memory).
In the evaluation, \sys's map store capacity with 1K maps is sufficient for maintaining performance (\S\ref{subsec:eval-sensitivity}), resulting in minimal memory overhead.



\begin{figure}[t]
  \centering
  \includegraphics[width=.85\linewidth]{figs/eval-overhead-memory.pdf}
  % \vspace{-0.1in}
  \caption{CPU memory footprint of \sys's Expert Map Store with different capacity.}
  \vspace{-0.1in}
  \label{fig:eval-overhead-memory.pdf}
\end{figure}


To assess the performance of the coordination policy, the benchmark includes a comprehensive evaluation framework. The framework provides standardized tools for comparing algorithms across different domains and scenarios. Evaluation metrics include cumulative cost, task success rate, expert intervention frequency, and other domain-specific measures. These metrics are essential for understanding the trade-offs between autonomy and expert reliance. The evaluation framework ensures that comparisons between methods are fair, standardized, and meaningful. Central to this framework is the \texttt{Evaluator} class, which handles the execution of policies and the collection of metrics. The \texttt{eval.py} script leverages this class to perform standardized evaluation runs, as shown in \autoref{fig:eval}. It first loads the same configuration used for training, then instantiates the environment and the policy to be evaluated. If the evaluated algorithm requires a trained model, it is loaded from a specified checkpoint. The \texttt{eval.py} script then calls the \texttt{evaluator.eval()} method to evaluate the policy on the designated test split. The Evaluator class uses the evaluation metrics, providing a comprehensive assessment of the policy's performance and adhering to standardized metrics.



\subsection{Benchmark Dependencies}
Our benchmark implementation leverages several open-source repositories for environment implementations and algorithm baselines:

\begin{itemize}
\item \textbf{MiniGrid Environments}: We utilize the Farama Foundation's MiniGrid implementation \citep{MinigridMiniworld23} for grid-based navigation tasks. The environment wrapper and agent policies interface with the Gymnasium API provided by this repository. Codebase: \url{https://github.com/Farama-Foundation/Minigrid}
\item \textbf{Cliport Environments}: Robotic manipulation tasks are implemented using the CLIPort repository \citep{shridhar2021cliport}, which provides RGB-D observation spaces and physics-based manipulation challenges. Codebase: \url{https://github.com/cliport/cliport}
\item \textbf{Procgen Environments}: Procedurally generated environments are adapted from the ProcgenAISC fork, which maintains compatibility with asynchronous actor-critic algorithms. We use commit \texttt{7821f2c} for experiment reproducibility. Codebase: \url{https://github.com/JacobPfau/procgenAISC/tree/7821f2c00be9a4ff753c6d54b20aed26028ca812}
\item \textbf{OOD Detection}: The PyOD library \citep{zhao2019pyod} provides implementations of various outlier detection algorithms, including the Deep SVDD method used in our OOD-based policies. Codebase: \url{https://github.com/yzhao062/pyod}
\end{itemize}

All environments are wrapped using our custom \texttt{HelpEnvironment} class (described in \autoref{app:imp_details}) to enable standardized coordination policy evaluation. The PyOD implementations were particularly valuable for implementing the OOD detection-based policies. We modified the original repositories only to the extent required for policy coordination mechanics, preserving their core environment dynamics and observation spaces.



\subsection{Training Novice and Expert Policies}
The \ourMethod framework requires three acting policies for coordination policy training:
\begin{itemize}
    \item \textbf{Expert} ($\pi_e$): High-performing policy for test-time assistance
    \item \textbf{Novice} ($\pi_n$): Policy trained on $\mathcal{E}_{\text{train}}$
    \item \textbf{Weakened Novice} ($\pi_n^-$): Suboptimal novice policy
\end{itemize}

\textbf{Expert Policy Training}.
For MiniGrid and Procgen environments, we train $\pi_e$ using PPO on $\mathcal{E}_{\text{test}}$ until convergence \citep{huang2024open}. For CLIPort's robotic manipulation tasks, we use predefined rule-based oracles as experts, leveraging their guaranteed success rates through handcrafted logic.

\textbf{Novice Policy Training}. 
The novice $\pi_n$ is trained exclusively on $\mathcal{E}_{\text{train}}$. MiniGrid and Procgen novices are trained using PPO on $\mathcal{E}_{\text{train}}$ until convergence. CLIPort novices are taken from provided checkpoints trained on $100$ demonstrations, establishing baseline task proficiency.

\textbf{Weakened Novice Policy Training}.
We create $\pi_n^-$ by deliberately limiting training exposure. For MiniGrid and Procgen, we halve PPO training epochs while maintaining $\mathcal{E}_{\text{train}}$ exposure. CLIPort's $\pi_n^-$ uses checkpoints trained on only $10$ demonstrations, reflecting partial task mastery. This mimics test-time performance degradation while preserving training distribution familiarity.



\subsection{Training Coordination Policy}\label{app:coord_policy}

\subsubsection{\textsc{Logit-Based} Methods}
A widely used family of coordination policies in our benchmark is based on thresholding techniques applied to confidence scores computed from the novice's logits. The fundamental principle behind these methods is to quantify the model’s certainty using a specific metric and then compare this score against a threshold. If the score falls below the threshold, the policy delegates control to the expert; otherwise, it acts autonomously.

\textbf{Confidence Metrics}.  
We consider five different metrics for computing the confidence score of the novice:
\begin{itemize}
    \item \textsc{MaxLogit}: The maximum logit value is used directly as a confidence measure.
    \item \textsc{MaxProb}: The softmax function is applied to the logits, and the highest probability is selected.
    \item \textsc{Margin}: The difference between the top two probabilities in the softmax distribution is computed, where larger margins indicate higher confidence.
    \item \textsc{NegEntropy}: The negative entropy of the softmax probability distribution is used, with lower entropy (higher negative entropy) corresponding to more certainty.
    \item \textsc{NegEnergy}: The log-sum-exponential (logsumexp) of the logits is computed, offering an energy-based measure of certainty.
\end{itemize}

\textbf{Threshold Selection via Rollouts}.
To determine an optimal threshold, we conduct a grid search over a range of candidate threshold values. Specifically, we perform $64$ rollouts in the training environment, where the environment is set up to run $64$ parallel instances. From these rollouts, we generate a distribution of confidence scores given the environment's raw observations, from which candidate thresholds are computed as percentiles. The search space consists of percentiles ranging from $0$ to $100$, incremented in steps of $10$. 

This process is implemented in the \texttt{ThresholdAlgorithm} class, inherited from the \texttt{Algorithm} class. For each candidate threshold, the policy is evaluated on simulated and true validation splits, and records the candidate yielding the highest mean reward. During inference, the policy processes an input observation as follows: 
\begin{itemize}
    \item Computes the confidence score using the configured metric.
    \item Compares the score to the current threshold. Since a higher score corresponds to greater confidence, a score below the threshold triggers delegation, yielding control to the expert.
\end{itemize}

This threshold-based method enables a systematic, data-driven approach to determining delegation decisions. By evaluating different confidence metrics, it provides flexibility in choosing the most effective measure of certainty for a given environment. 




\subsubsection{\textsc{OOD-Detection} Methods}
The OOD detection methods in \ourMethod-Bench are built upon the Deep SVDD method \citep{pmlr-v80-ruff18a}. These methods aim to identify when the novice's observations fall outside the training distribution, thereby signaling that control should be delegated to the expert. In our implementation, the OOD detector is initialized by gathering rollouts from the training environment; specifically, we perform $64$ rollouts with $64$ parallel environment instances. The collected observations serve a dual purpose: they are used both to train the Deep SVDD model and to determine a suitable threshold for delegation via a grid search, similar to the \textsc{Logit-Based} methods.

The Deep SVDD algorithm minimizes the distance between feature representations and a pre-defined center. After training, the detector computes decision scores for a separate set of rollout observations. Candidate thresholds are then determined by linearly spacing values between the minimum and maximum decision scores, following the same procedure as the \textsc{Logit-Based} methods. Our implementation leverages the PyOD library, which provides a suite of OOD detection algorithms, including Deep SVDD \citep{zhao2019pyod}. All hyperparameters for Deep SVDD are set to their default values in PyOD, without additional tuning. Furthermore, with minimal modifications, other PyOD-based OOD detection methods can be seamlessly integrated to evaluate their effectiveness in the \ourMethod problem.

A key feature of our OOD-detection approach is its flexibility in the input feature space. The observation space may comprise raw observations, hidden features from the novice policy, or combinations thereof (e.g., \texttt{obs}, \texttt{hidden}, \texttt{hidden\_obs}, \texttt{dist}, \texttt{hidden\_dist}, \texttt{obs\_dist}, \texttt{obs\_hidden\_dist}). This design enables the OOD detector to leverage a richer set of features, potentially enhancing its ability to distinguish in-distribution inputs from OOD ones.

During inference, the OOD policy computes an anomaly score using the detector's decision function. A delegation decision is then made by comparing the anomaly score to the learned threshold: if the score is below the threshold, the policy yields control to the expert; otherwise, it retains control. 




\subsubsection{\textsc{RL-Based} Methods}
The RL-based coordination policy in our benchmark is trained using Proximal Policy Optimization (PPO), an on-policy actor–critic method that balances efficient policy updates with sample efficiency \citep{schulman2017proximal}. In our implementation, the coordination policy is parameterized via an actor–critic architecture, where the actor produces a probability distribution over actions and the critic estimates the corresponding state values.

During training, multiple parallel environments (e.g., $64$ instances) are run simultaneously to collect a batch of trajectories over a fixed number of steps. The observation space for the RL methods is flexible and can be configured to include raw observations, hidden features extracted by the novice agent, or combinations thereof (such as raw observations concatenated with hidden features or with action logits), similar to the \textsc{OOD-detection} methods. This flexibility allows the policy to leverage richer contextual information when making delegation decisions.

After collecting trajectories, advantage estimates are computed using Generalized Advantage Estimation \citep{Schulmanetal_ICLR2016}, with the critic bootstrapping the final state value to compute temporal-difference errors. These advantage estimates are typically normalized prior to being used in the policy update. The PPO update itself minimizes a surrogate objective that includes three key components: a clipped policy loss to restrict large updates, a value loss to improve the accuracy of the critic, and an entropy bonus to encourage exploration.

The underlying network architecture is based on an Impala model that extracts features from the input observations \citep{espeholt2018impala}. Depending on the chosen configuration, these features may be combined with latent representations from the novice or with softmax-transformed logits. A fully connected layer then projects the aggregated features to produce policy logits over the available actions.

Additional training techniques such as dynamic learning rate annealing and gradient clipping are employed to ensure stable convergence. Overall, the PPO-based method iteratively collects data, computes gradients on mini-batches, and updates the policy and value networks until the coordination policy converges.

This method can operate in two distinct modes: as a \textit{skyline approach} that utilizes access to the expert policy $\pi_e$ and test environment $\mathcal{E}_{\text{test}}$ during training to derive near-optimal policies, and as a \textit{baseline method} where such access is intentionally restricted during training. The latter configuration enables fair comparison with alternative coordination strategies by matching their practical constraints.


\clearpage

\section{Environment Details}\label{app:envs}

We evaluate coordination policies across three distinct domains, each containing multiple environments with carefully designed train-test splits to test policy generalization under distribution shifts. Below we describe the specific environments and their configurations.

\subsection{MiniGrid Environments}
The grid-based navigation domain contains three environment families with progressive complexity:

\begin{itemize}
\item \texttt{DistShift}: Training uses \textit{1-v0} (small grid), while testing uses \textit{2-v0} (expanded grid with longer trajectories)
\item \texttt{DoorKey}: Training on \textit{-5x5-v0} ($5\times5$ grid), testing on \textit{-8x8-v0} ($8\times8$ grid with more complex door-key relationships)
\item \texttt{LavaGap}: Training on \textit{S5-v0} (5-tile lava gap), testing on \textit{S7-v0} ($7$-tile gap requiring longer jumps)
\end{itemize}

All MiniGrid environments use partially observable grids with discrete actions. The test versions feature larger state spaces and more complex spatial relationships than their training counterparts.

\subsection{Procgen Environments}
The procedural generation suite includes $11$ distinct platformer games, each with two difficulty levels:

\begin{itemize}
\item \texttt{bossfight}: Combat-focused game with escalating enemies
\item \texttt{caveflyer}: Navigation through procedural caverns
\item \texttt{chaser}: Avoidance of pursuing enemies
\item \texttt{climber}: Vertical ascension challenge
\item \texttt{coinrun}: Collection-based platformer
\item \texttt{dodgebal}: Projectile avoidance game
\item \texttt{heist}: Stealth-based item retrieval
\item \texttt{jumper}: Precision jumping challenges
\item \texttt{maze}: Complex spatial navigation
\item \texttt{ninja}: Timing-based obstacle course
\item \texttt{plunder}: Resource gathering under threat
\end{itemize}

The \textit{easy} distribution (training/simulated evaluation) uses simplified dynamics and predictable patterns, while the \textit{hard} distribution (true evaluation/testing) introduces stochastic elements, and more complex terrain generation.

\subsection{Cliport Environments}
The robotic manipulation domain contains five tasks with object configuration splits:

\begin{itemize}
\item \texttt{Assembling-Kits-Seq}: Sequential object placement in kits
\item \texttt{Packing-Boxes-Pairs}: Object pairing and containerization
\item \texttt{Put-Block-in-Bowl}: Precise object-in-container placement
\item \texttt{Stack-Block-Pyramid-Seq}: Vertical structure assembly
\item \texttt{Separating-Piles}: Object sorting and segregation
\end{itemize}

The \textit{seen} split (training/simulated evaluation) uses a fixed set of object shapes and color configurations, while the \textit{unseen} split (testing) introduces novel object geometries and color combinations not encountered during training. All tasks require $6$-DOF control and pixel-level spatial reasoning.

The combination of these environments provides comprehensive coverage of key challenge domains: discrete vs continuous control, 2D vs 3D spatial reasoning, and symbolic vs pixel-based observations. \autoref{fig:envs} in the main text illustrates representative observations from each domain.


\clearpage

\section{Detailed Results}

\subsection{Performance of \textsc{RLOracle} Methods} \label{app:exp_rloracle}

We further analyze the performance of individual \textsc{RLOracle} algorithms across different environments, as shown in \autoref{fig:rl_performance}. This detailed breakdown reveals that the advantage of raw observation-based policies is more pronounced in Procgen and CLIPort environments, whereas it is less salient in the MiniGrid suite. This discrepancy can be attributed to the nature of the environments: Procgen and CLIPort feature visually rich, high-dimensional observation spaces where direct access to raw observations provides a clear advantage in learning nuanced coordination behaviors. In contrast, MiniGrid consists of low-dimensional, symbolic representations where the distinction between raw observations and the novice’s internal features is less significant. In such structured environments, the novice policy’s internal representations already capture most of the relevant task information, reducing the advantage of using raw observations.

% \khanh{can you explain why cliport seems to benefit less from observations than procgen?} \mohamad{my guess is that in cliport, the hidden representations coming from the novice are rich enough to holp the coord policy make an informed decision without the raw obs. while in procgen, the novice's hidden representations are not informative enough so giving the raw obs to the coord policy makes a bigger difference. after all this is just based on my intuition and bias. it should be investigated to verify whether the novice’s hidden representations are informative enough by analyzing feature distributions and running an ablation study.}

\begin{figure}[t]
    \centering
    \includegraphics[width=0.92\linewidth]{figures/rl_performance.pdf}
    \caption{Per-environment performance of \textsc{RLOracle} variants. \colorbox{boxsteelblue}{Observation-conditioned methods} show strongest advantages in high-dimensional environments, while MiniGrid environments show smaller differences due to their low-dimensional state representations.}
    \label{fig:rl_performance}
\end{figure}

\subsection{Near-Optimal Coordination Achievements} \label{app:exp_optimal}

\begin{figure}[t]
\centering
\includegraphics[width=\textwidth]{figures/env_performance.pdf}
\caption{Comprehensive performance comparison across all and methods. \colorbox{boxbrown}{Skyline} performance represents the oracle upper bound, with \colorbox{boxred}{logit} and \colorbox{boxpurple}{OOD} methods approaching this limit in structured environments. Gray backgrounds denotes such environments.}
\label{fig:env_perf}
\end{figure}

\autoref{fig:env_perf} illustrates the overall performance of each algorithm and input feature type across all environments studied in this paper. It reveals an interesting pattern: logit-based and OOD detection-based coordination policies achieve near-skyline performance in $3$ environments. We analyze these representative success cases:

\textbf{DoorKey (MiniGrid):} The $8\times8$ grid environment exhibits deterministic dynamics but requires precise multi-step sequencing (find key, then unlock door, then navigate to goal). The \textsc{MaxLogit} policy matches skyline performance matches skyline performance by interfering the novice's potentially flawed decision-making, preventing costly mistakes and ensuring efficient completion of the task.


\textbf{LavaGap (MiniGrid):} This environment's lethal consequences (falling into lava) create clean separation between high-confidence navigation actions and uncertainty ``cliff edges.'' The OOD-based method with \texttt{hidden-dist} features and Margin logit policy are the closest to skyline performance.

\textbf{Climber (Procgen):} Despite procedural generation, the logit-based methods are statistically the same as skyline methods. The policy successfully distinguishes between challenging-but-seen obstacles (handled by novice) and truly novel gap configurations (referred to expert), despite being trained solely on the \texttt{easy} distribution.
% \khanh{again, feel like a figure illustrating these scenarios would help}

Our analysis reveals significant performance gaps between \textsc{RLOracle} and other methods in certain scenarios. Notably, across all CLIPort manipulation tasks, no coordination policy approaches ``worst'' \textsc{RLOracle}'s performance. The most striking example occurs in the \texttt{packing-boxes-pairs} task: while the lowest-performing \textsc{RLOracle} variant (using only the novice's action probability distribution as input) achieves a performance of $0.83$, the best non-\textsc{RLOracle} methods (logit-based approaches) reach only $0.73$ - a $13.7\%$ relative performance gap. Other CLIPort tasks exhibit even wider disparities, with \textsc{RLOracle} outperforming alternatives by at least $30.7\%$ across \texttt{assembling-kits-seq}, $35.1\%$ across \texttt{put-block-in-bowl}, $40.3\%$ accross \texttt{stack-block-pyramid-seq}, and $20.9\%$ across \texttt{separating-piles} environments. These substantial gaps highlight fundamental limitations in current coordination strategies for high-dimensional manipulation tasks, suggesting urgent needs for improved policy architectures that better leverage both environmental observations and novice uncertainty signals. 
Moreover, in other environments, as shown in \autoref{fig:env_perf}, the \textsc{RLOracle} methods significantly outperform the other methods, showing the gap between the oracle method and baselines.