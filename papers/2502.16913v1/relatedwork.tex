\section{Related Work}
\textbf{GCN-based Methods.}\quad
The skeletal representation of humans, characterized by joints and their adjacent links, can be effectively modeled as a graph. This correlation has been extensively explored by researchers \cite{LTD,HRI,yao2024swift}. Graph Convolutional Networks (GCNs), specifically designed for graph data, offer a robust method for feature extraction in this context.
In \cite{DMGNN,SKEL}, a multi-scale graph was introduced that adeptly captures features across multiple scales, enabling the accurate prediction of future human motions. Concurrently, \cite{DBLP:conf/iccv/LiuS0SCH021} developed a semi-constrained graph that explicitly encodes skeletal connections and integrates prior knowledge. Building on this, they proposed a multi-scale spatial-temporal graph \cite{MSGNN} to achieve comprehensive modeling.
Differently, our work simulate the human visual system. We design a unique GCN structure that handles dynamic and static information separately, thereby eliminating the disruption of irrelevant data. Moreover, we encode the data hierarchically to obtain richer localized human information.

\textbf{TCN-based methods.} \quad
RNNs are inherently prone to error accumulation \cite{QN,SGRU,MM23_deepfake}. Conversely, Temporal Recurrent Networks (TCNs) mitigate common issues such as gradient explosion or vanishing gradients, as their back-propagation paths diverge from the temporal sequence direction. These strengths have led to the increasing application of TCNs in HMP.  
For instance, in \cite{MoP}, TCNs are employed to decode the dynamics of sub-motions and the spatial correlations within the entire motion sequence to predict future movements. Subsequently, \cite{TCGAN} propounds a residual TCN that is characterized by a minimalist design yet delivers high efficiency.
Diverging from conventional methods, we have developed a TCN-based model. This network harmoniously fuses the strengths of both RNNs and TCNs, creating a recurrent temporal convolutional network adept at predicting future movements. 

\textbf{GAN-based Methods.} \quad
GANs have emerged as a pivotal tool in addressing HMP challenges, not only for constructing novel networks but also for advancing learning algorithms. Recent advancements include specialized adaptations such as HP-GAN, which modifies the improved WGAN-GP for probabilistic HMP \cite{HPgan}. Additionally, Bi-GANs \cite{Bigan} and GAN-poser \cite{Gan-poser} introduce bi-directional structures to enhance prediction accuracy. An AMGAN \cite{AMGAN} targets kinematic chains.
Adversarial strategies in motion analysis also be leveraged for network training \cite{ARNET, AGED, yao2024swift}, demonstrating their effectiveness in various contexts. Notably, SGRU \cite{SGRU} employs GANs to simulate path integral, offering a distinct perspective on motion modeling.
In our investigation, we contribute a theoretical demonstration that dimension reduction significantly enhances the convergence of WGAN. Moreover, we utilize GANs for motion context modeling.
% recognizing the competitive dynamics among neurons, 
% In this work, we first present a HMP prediction system with human-like vision and inference that outperforms or is equivalent to current SOT prediction methods.

\begin{figure*}[t]
\begin{center}
\includegraphics[width=.9\linewidth]{Fig/F2.pdf}
\caption{HVIS including two components: Human-like vision module (HVM) and Human-like inference module (HIM).}
\label{fig:2}
\end{center}
\end{figure*}