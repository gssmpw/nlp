\documentclass{article}


\usepackage{arxiv}
% This must be in the first 5 lines to tell arXiv to use pdfLaTeX, which is strongly recommended.
\pdfoutput=1
% In particular, the hyperref package requires pdfLaTeX in order to break URLs across lines.


\usepackage[numbers]{natbib}
\usepackage{amsmath}
\usepackage{placeins}  % 在导言区加入
\usepackage{afterpage}  % 在导言区加入

% Change "review" to "final" to generate the final (sometimes called camera-ready) version.
% Change to "preprint" to generate a non-anonymous version with page numbers.
\usepackage{booktabs}
\usepackage{hyperref}
% Standard package includes
\usepackage{times}
\usepackage{latexsym}
\usepackage{multirow}
\usepackage{tabularx}
\usepackage{pifont}
\usepackage{tcolorbox}
\usepackage{float}


\usepackage{multirow}      % 用于多行合并
\usepackage{booktabs}      % 用于美化表格横线 (\toprule, \midrule, \bottomrule)
\usepackage{adjustbox}     % 如果需要在表格过宽时进行缩放
\usepackage{caption}       % 美化表格/图的标题
\usepackage{geometry}      % 若需要可自行调整页面边距
\usepackage{amssymb}
% \usepackage{minipage}

% For proper rendering and hyphenation of words containing Latin characters (including in bib files)
\usepackage[T1]{fontenc}
% For Vietnamese characters
% \usepackage[T5]{fontenc}
% See https://www.latex-project.org/help/documentation/encguide.pdf for other character sets

% This assumes your files are encoded as UTF8
\usepackage[utf8]{inputenc}

\usepackage{textcomp} % or fontenc, depending on your setup

% This is not strictly necessary, and may be commented out,
% but it will improve the layout of the manuscript,
% and will typically save some space.
\usepackage{microtype}

% This is also not strictly necessary, and may be commented out.
% However, it will improve the aesthetics of text in
% the typewriter font.
\usepackage{inconsolata}
\usepackage{xcolor}
%Including images in your LaTeX document requires adding
%additional package(s)
\usepackage{graphicx}
\usepackage{framed}
\usepackage{pifont}
\definecolor{shadecolor}{rgb}{0.92,0.92,0.92}

\PassOptionsToPackage{table}{xcolor} 
% If the title and author information does not fit in the area allocated, uncomment the following
%
%\setlength\titlebox{<dim>}
%
% and set <dim> to something 5cm or larger.
\usepackage[utf8]{inputenc} % allow utf-8 input
\usepackage[T1]{fontenc}    % use 8-bit T1 fonts
\usepackage{hyperref}       % hyperlinks
\usepackage{url}            % simple URL typesetting
\usepackage{booktabs}       % professional-quality tables
\usepackage{amsfonts}       % blackboard math symbols
\usepackage{nicefrac}       % compact symbols for 1/2, etc.
\usepackage{microtype}      % microtypography
\usepackage{lipsum}
\usepackage{graphicx}
\graphicspath{ {./images/} }


\title{Bridging the Editing Gap in LLMs: FineEdit for Precise and Targeted Text
Modifications}


\author{
 Yiming Zeng \\
  University of Connecticut\\
  \texttt{yiming.zeng@uconn.edu} \\
  %% examples of more authors
   \And
 Wanhao Yu \\
  University of North Carolina at Charlotte\\
  \texttt{wyu6@charlotte.edu} \\
  \And
 Zexin Li \\
  University of California, Riverside\\
  \texttt{zli536@ucr.edu} \\
\And
 Tao Ren \\
  University of Pittsburgh\\
  \texttt{tar118@pitt.edu} \\
\And
 Yu Ma \\
  Carnegie Mellon University \\
  \texttt{yuma13926@gmail.com} \\
\And
 Jinghan Cao \\
  San Francisco State University \\
  \texttt{jcao3@alumni.sfsu.edu} \\
\And
 Xiyan Chen \\
  University of Pittsburgh \\
  \texttt{xic130@pitt.edu} \\
\And
 Tingting Yu \\
  University of Connecticut \\
  \texttt{tingting.yu@uconn.edu} \\
  %% \AND
  %% Coauthor \\
  %% Affiliation \\
  %% Address \\
  %% \texttt{email} \\
  %% \And
  %% Coauthor \\
  %% Affiliation \\
  %% Address \\
  %% \texttt{email} \\
  %% \And
  %% Coauthor \\
  %% Affiliation \\
  %% Address \\
  %% \texttt{email} \\
}

\begin{document}
\maketitle
\begin{abstract}
Large Language Models (LLMs) have transformed natural language processing, yet they still struggle with direct text editing tasks that demand precise, context-aware modifications. While models like ChatGPT excel in text generation and analysis, their editing abilities often fall short, addressing only superficial issues rather than deeper structural or logical inconsistencies. In this work, we introduce a dual approach to enhance LLMs editing performance. First, we present InstrEditBench, a high-quality benchmark dataset comprising over 20,000 structured editing tasks spanning Wiki articles, LaTeX documents, code, and database Domain-specific Languages (DSL). InstrEditBench is generated using an innovative automated workflow that accurately identifies and evaluates targeted edits, ensuring that modifications adhere strictly to specified instructions without altering unrelated content. Second, we propose FineEdit, a specialized model trained on this curated benchmark. Experimental results demonstrate that FineEdit achieves significant improvements around {10\%}  compared with Gemini on direct editing tasks, convincingly validating its
effectiveness.
\end{abstract}


% keywords can be removed
%\keywords{First keyword \and Second keyword \and More}


\section{Introduction}
Large Language Models (LLMs) have revolutionized natural language processing, unlocking capabilities once thought unattainable. ChatGPT, for example, shows exceptional skills in text generation and logical reasoning~\cite{openai2023chatgpt}. Despite these impressive advancements, LLMs still face significant challenges in the underperformance of text editing tasks.~\cite{castillo2022chat} has mentioned that the use of ChatGPT for editing has obvious limitations, which might not accurately follow the editing task instructions and understand the author's intent, leading to changes that are not appropriate to the context of the text. Meanwhile, it effectively addresses surface-level issues, such as spelling and formatting, but cannot resolve complex challenges, such as editing in long text context or strict following task instructions.

% \begin{figure}[!tbp]
%     \centering
%     \includegraphics[width=0.5\textwidth]{figure/figure1_Edit.pdf}
%     \caption{Direct Editing Task. As illustrated, the ChatGPT processes the edit request (e.g., changing a label reference), modifies the original context, and generates the updated content.}
%     \vspace{-3mm}
%     \label{fig:edit-example}
% \end{figure}    


To address these limitations, researchers have developed methods to enhance the editing capabilities of LLMs, particularly under task-specific scenarios, e.g., editing in code, LaTeX, etc. However, LLMs’ general editing capabilities in task-specific settings often fall short~\cite{yao-etal-2023-editing, ma-etal-2024-robustness}. 
They tend to generate incorrect outputs and stray from the given editing instructions. This issue arises primarily because these models overly emphasize task-specific constraints and are susceptible to hallucinating extraneous information. 

In contrast, we notice that if narrowing the model's focus to just two factors: the precise location of the edit and the specific content to be changed, the edit task itself could be better accomplished. Per this intuition, we propose a dual approach consisting of a benchmark for editing tasks and a model named FineEdit. For the benchmark, we design an automated workflow that focuses on accurately identifying and evaluating structured text edits. This workflow identifies precise differences and ensures correct edits through quality control. By reducing noise and focusing on meaningful modifications, this process produces a benchmark that is both practical for training and robust for evaluation. It directly addresses limitations in existing methods and aligns better with the practical demands of real-world editing tasks.

Furthermore, we use part of the curated benchmark to train the model, focusing on direct editing tasks. FineEdit achieves an over 10\% improvement over Gemini 1.5 Flash and Gemini 2.0 Flash~\cite{google2024gemini}, and up to 30\% over Llama-3.2-3B~\cite{meta2024llama3_2} on diverse editing benchmarks, while outperforming Mistral-7B-OpenOrca~\cite{lian2023mistralorca1, mukherjee2023orca, longpre2023flan} over 40\% in direct editing tasks.

The main contributions of this work include:
\begin{itemize}
    \item \textbf{A high-quality benchmark dataset (InstrEditBench)}\footnote{We will release all datasets and the code to promote reproducibility on acceptance.}: We created a curated dataset with 20,000+ structured editing tasks across Wiki articles, LaTeX documents, Code, and Database DSL, providing a unified evaluation standard for structured text editing research.
    
    \item \textbf{An innovative automated dataset generation workflow}: We developed a comprehensive workflow that ensures the benchmark's quality by accurately identifying line numbers and applying rigorous criteria to filter meaningful and relevant edits.
    
    \item \textbf{The FineEdit model}: We introduce a specialized model designed for structured direct text editing, demonstrating superior performance across benchmarks compared with existing models.
\end{itemize}

\section{Background}
\subsection{Problem Formulation}
Each data point consists of an original structured text, \(T_{\text{orig}}\), and an editing instruction, \(I_{\text{edit}}\). The objective is to generate an edited text, \(T_{\text{edit}}\), that incorporates the modifications specified by \(I_{\text{edit}}\). Formally, this process is defined as
\begin{equation}
    T_{\text{edit}} = f\Bigl(T_{\text{orig}}, I_{\text{edit}}; \theta\Bigr)
\end{equation}
where \(\theta\) represents learned parameters and \(f\) denotes a function instantiated by a LLM that maps the original text \(T_{\text{orig}}\) and editing instruction \(I_{\text{edit}}\) to the edited text \(T_{\text{edit}}\).

The parameters \(\theta\) are learned from a dataset consisting of triples \(\{(T_{\text{orig}}^{(i)}, I_{\text{edit}}^{(i)}, T_{\text{edit}}^{(i)})\}_{i=1}^{N}\) during training, where the objective is to minimize the discrepancy between the generated output and the ground truth edited text.

Internally, \(f\) concatenates \(T_{\text{orig}}\) and \(I_{\text{edit}}\) into a single prompt and generates \(T_{\text{edit}}\) token by token in an autoregressive manner. Specifically, if 
\(T_{\text{edit}} = (y_1, y_2, \dots, y_t)\),
the probability of the edited text is factorized as
\begin{equation}
\begin{split}
        p(T_{\text{edit}} \mid T_{\text{orig}}, I_{\text{edit}}) =& \prod_{i=1}^{t} p\Bigl(y_i \mid T_{\text{orig}}, I_{\text{edit}}, \\
        & y_1, y_2, \dots, y_{i-1}\Bigr)
\end{split}
\end{equation}

For finetuning on the editing task, the prompt tokens (i.e., the original text and the editing instruction) are masked out in the loss function to ensure that the model focuses only on predicting the correct edited tokens. At inference time, the model processes the prompt and subsequently generates \(T_{\text{edit}}\).

The parameters \(\theta\) are fine-tuned on labeled examples \((T_{\text{orig}}, I_{\text{edit}}, T_{\text{edit}})\) by minimizing the negative log-likelihood of the target tokens with the loss:
\begin{equation}
    \mathcal{L}(\theta) = - \sum_{t=1}^{|T_{\text{edit}}|} \log P_{\theta}(y_t \mid T_{\text{orig}}, I_{\text{edit}}, y_{1:t-1})
\end{equation}
over all training samples in the dataset

\subsection{LLM Editing Tasks}

LLMs are increasingly recognized as versatile tools for automating and enhancing editing tasks across diverse domains. Previous studies have explored LLMs for editing tasks in areas such as natural language (e.g., wiki articles) and code. For instance, CoEdIT~\cite{raheja2023coedit} employs task-specific instruction tuning to achieve precise modifications, while other works fine-tune models like T5~\cite{raffel2020exploring} on pairs of original and edited texts~\cite{faltings2021leveraging, reid2022learning, mallinson2022edit5, du2022grit1, du2022grit2, kim2022towards}. However, many of these approaches rely on specialized techniques or focus narrowly on specific tasks, such as grammar correction~\cite{mallinson2022edit5, fang2023hierarchical}, text simplification~\cite{stajner2022simple}, paraphrase generation~\cite{chowdhury2022enhanced}, or style transfer~\cite{reif2022style}, which limits their generalizability across a broader range of editing scenarios. In the realm of code editing, Fan et al.~\cite{fan2024codechange} examined LLMs for code change tasks and identified weaknesses in generating accurate reviews and commit messages. While these studies offer valuable insights, they often fall short in providing unified benchmarks and robust solutions to address the full spectrum of editing challenges. Our work addresses these gaps by introducing a comprehensive, cross-scenario editing tasks benchmark that covers Wiki, code, DSL, and LaTeX.

\subsection{LLM Benchmarking}
LLM benchmarking is a crucial aspect of evaluating the diverse capabilities of LLM. Researchers have developed numerous benchmarks spanning multiple domains, including code~\cite{chen2021evaluating,austin2021program,jimenez2024swebench,yang2024swebenchmultimodal}, commonsense reasoning~\cite{bisk2020piqa,sap2019socialiqa,zellers2019hellaswag,sakaguchi2021winogrande}, reading comprehension~\cite{rajpurkar2018know,choi2018quac,clark2019boolq}, and language understanding~\cite{wang2018glue,wang2019superglue,xu2020clue}. However, only a few works benchmark the editing performance of LLMs. For example, GEM~\cite{xu2024benchmarking} introduces metrics for subjective tasks without gold standards, while CriticBench~\cite{lin2024criticbench} assesses iterative output refinement. Additionally, \cite{cassano2023can} explores that fine-tuning with curated training data significantly improves code editing performance. Some automated evaluation is also involved in LLM benchmarking. For instance, G-Eval~\cite{liu2023geval} is an automated evaluation framework that leverages large language models to assess text quality and model performance in generative tasks. Built on a Chain-of-Thought (CoT) prompting strategy \cite{wei2022chain}, G-Eval guides the model to articulate intermediate reasoning steps before reaching its final evaluation, leading to outputs that align closely with human judgments~\cite{liu2023geval}. However, these efforts focus on short-context, isolated tasks and do not systematically evaluate an LLM’s ability to locate and modify content within long contexts. Our work addresses this gap by introducing a comprehensive benchmark covering Wiki, code, DSL, and LaTeX, emphasizing long-context editing.

\begin{figure*}[htbp]
    \centering
    \includegraphics[width=0.8\textwidth]{figure/Figure2.pdf}
    \caption{Workflow of Generating High-quality FineEdit benchmark. The content difference is highlighted in red. }
    \label{fig:example-pdf}
\end{figure*}

\section{Method}

\subsection{Instruction categories}

We leverage four data sources to cover a wide range of representative text application scenarios: Wiki, Code, DSL, and LaTeX. The details of each categories are described as follows:

\begin{itemize}
    \item \textbf{Wiki}: Data is extracted from the WikiText language modeling dataset~\cite{merity2016pointer}, which contains over 100 million tokens from a dedicated subset of Wikipedia's Good articles~\cite{wikipedia_good_articles} and Wikipedia's Featured articles~\cite{wikipedia_featured_articles}. Specifically, sections from these articles are extracted and then contiguous segments are randomly selected to provide data points with various lengths.
    \item \textbf{Code}: Code samples are extracted from the CodeSearchNet corpus \cite{husain2019codesearchnet}, which contains about two million pairs of comments and code from GitHub projects. To make the edit task more challenging, each code sample in our benchmark is made up of several instead of one code segment because one single code segment is too short (about 10 lines).
    \item \textbf{DSL}: Database Domain Specific Language (DSL) is also considered in our benchmark. It consists of queries and schema definitions from multiple public repositories~\cite{hive,b-mc2_2023_sql-create-context,cassandra,chinookDatabase}.
    \item \textbf{LaTeX}: LaTeX data is extracted from the Latex2Poster dataset~\cite{latex2poster} that offers the LaTeX source code document of research papers along with metadata. Specifically, each data point in our benchmark consists of multiple subsections from each extracted document data.
\end{itemize}

\subsection{Instruction Generation}

Zero-shot instruction generation is efficient but often lacks diversity. To address this limitation, we build on the work of \cite{wang2022self,taori2023stanford} by leveraging ChatGPT-4o mini combined with in-context learning (ICL)~\cite{dong2024survey}. Our approach is designed to generate specific edit requests tailored to the structural characteristics of different data categories, as process \ding{192} in Figure~\ref{fig:example-pdf}. For Wiki, which primarily consists of clear structural text elements like headings and subheadings, we apply a zero-shot prompting strategy. In contrast, for more complex domains such as LaTeX, code, and DSL, we adopt ICL to improve the diversity and nuance of generated instructions. This category-specific strategy not only enriches the instruction sets but also enhances their ability to capture domain-specific editing challenges without compromising on precision and efficiency. We will describe prompt details in Appendix~\ref{sec:dataset_generation_prompts}.

% \zexin{this paragraph maybe wrong! we focus on edit itself, but not respect task-specific constraint.}
% By designing the ICL prompt, we cover different edit task scenarios in the prompts to help LLM improve the variety and accuracy of the generated editing instructions. For instance, the LaTeX editing process must respect a specific formatting standard. Accordingly, we concentrate on modifications relevant to LaTeX special formats—such as adjustments to environments, command modifications, citation corrections, and the handling of graphical elements, etc. This is more challenging than simply editing text without additional requirements. For Code, edited content should present proper format, including indentation, whitespace, variable, and function naming. Meanwhile, for database DSL, edited content should be legitimate query clauses (e.g., \texttt{SELECT}, \texttt{FROM}, \texttt{WHERE}, \texttt{JOIN}, etc.), consistent keyword casing, structured clause ordering with line breaks and indentation, and etc. \zexin{We will describe prompt details in Appendix~\ref{?}.}

\subsection{Instruction filtering}

After obtaining the edit instructions for each content, we apply them to the original text to produce an edited version as process \ding{193} in Figure~\ref{fig:example-pdf}. However, ensuring the quality of the edited content remains challenging. Although LLM generally follows the edit instructions, errors may occur---for example, targeting incorrect line numbers or misinterpreting the intended semantics~\cite{wang2025understandingcharacteristicscodegeneration, cassano2024editevaluatingabilitylarge}. To address this problem and improve data quality, we propose {DiffEval Pipeline}, which integrates G-Eval~\cite{liu2023geval} and Git-Diff as an automatic filter to improve data quality. 

% \zexin{this sounds others work, consider putting it into a background and only briefly describe in one sentence here.}
% G-Eval~\cite{liu2023geval} is an automated evaluation framework that leverages large language models to assess text quality and model performance in generative tasks. Built on a Chain-of-Thought (CoT) prompting strategy \cite{wei2022chain}, G-Eval guides the model to articulate intermediate reasoning steps before reaching its final evaluation, leading to outputs that align closely with human judgments~\cite{liu2023geval}.

Besides adopting G-Eval for automated assessment \citep{liu2023geval}, the DiffEval Pipeline also relies on \texttt{git} \citep{gitdiff}, a widely used version control system, to detect and classify textual modifications. Specifically, the command \texttt{git diff} specifies differences between the original and modified texts as process \ding{194} in Figure~\ref{fig:example-pdf}, categorizing changes into four types:

\begin{itemize}
\item \textbf{Replacements}: an original segment is transformed into a new form, indicated as \texttt{[original\_text -> modified\_text]}. This captures cases where an existing text portion is substituted with different content, which may alter meaning or style.
\item \textbf{Deletions}: a segment is removed entirely, shown as \texttt{[-original\_text-]}. Such removals can simplify the text or eliminate irrelevant or erroneous sections.
\item \textbf{Insertions}: new content is added, denoted as \texttt{[+modified\_text+]}. Insertions enrich the text with extra details, clarifications, or elaborations.
\item \textbf{Unchanged Text}: labeled as \texttt{equal: unchanged\_text}. This indicates portions that remain identical between the original and modified versions, providing a reference for what the model has chosen to retain.
\end{itemize}

By categorizing changes into these four types, the DiffEval Pipeline offers a structured view of how text is altered, enabling more precise evaluations when paired with G-Eval.

We make a concrete instance using data in the LaTeX category in Table~\ref{tab:edit-intentions}. If the edit request is to ``Remove the duplicate \texttt{\textbackslash{}begin\{abstract\}} at the beginning of the abstract environment," the diff output might display on Line~1:
\begin{verbatim}
\begin{abstract}[-\begin{abstract}-]
\end{verbatim}
This indicates that the duplicate has been successfully removed. 

Finally, process \ding{195} in Figure~\ref{fig:example-pdf} demonstrates that DiffEval carefully reviews the aggregated data (marked with red arrows) alongside the edit request to fully grasp the context, structure, and nuances of the text. It identifies discrepancies between the intended edits and the actual modifications, verifying whether the changes faithfully implement the edit instructions. By using the \texttt{git diff} output instead of the complete edited content, DiffEval can precisely locate modifications using supplementary information such as line numbers and structured differences. Moreover, \texttt{git diff} minimizes unnecessary noise and reduces computational overhead by significantly lowering the token count compared with the full edited content. Once all required data is gathered, the G-Eval analysis process evaluates the collected information to further enhance the dataset quality.

Specifically, the analysis process begins by parsing the structure of \texttt{git diff} outputs, categorizing changes as replacements, deletions, insertions, or unchanged segments. Next, it evaluates the semantic meaning of both the original content and the modifications to ensure that the changes are accurate and complete. This involves a thorough review of the original text, the edit request, and the resulting edits, applying predefined categorization rules, and assessing overall coherence.

Based on this analysis process, the DiffEval is able to assign a coherence score, \texttt{G-Score}, to the edited content, reflecting the semantic integrity and logical consistency of the modifications. This score is used to filter out output that does not meet the desired quality threshold $\alpha$.



\subsection{Data Statistics}

Our curated benchmark comprises 28,050 items spanning a diverse array of structured data types, including 8,366 LaTeX contexts, 7,712 code segments, 8,025 WikiText entries, and 3,947 database language samples, thereby reflecting both the generality and scale of real-world structured data. Table~\ref{tab:edit-intentions} shows the example across four categories. For each item, it has the following attributes: 

\begin{itemize}
    \item \texttt{Id:} a unique identifier for each entry.
    \item \texttt{Original content:} the content directly extracted from the data source. 
    \item \texttt{Edit request:} The editing instruction generated
through zero-shot or few-shot prompting based on the original content
    \item \texttt{Edited Content:} the output after applying edit request to the original content.
    \item \texttt{Difference:} the changed part between Edit content and original content. 
    \item \texttt{G-score:} evaluates the quality of the edited content based on its strict adherence to the edit request content. 
\end{itemize}


\begin{table*}[h!]
\centering
\resizebox{\textwidth}{!}{ % 自动调整表格宽度
\begin{tabular}{@{}p{2.5cm}p{5.5cm}p{5.5cm}p{5.5cm}p{6.0cm}p{1.0cm}@{}}
\toprule
\textbf{Data Category} & \textbf{Orignal Content} & \textbf{Edit Request} & \textbf{Edited Content} & \textbf{Difference} & \textbf{G-score}\\ \midrule
\textbf{WikiText} & ...As with previous <unk> Chronicles games, Valkyria Chronicles III is a tactical role @-@ playing game where players take control of a military unit... & Replace ``\textbackslash{}<unk>\textbackslash{}'' with ``Valkyria'' where it appears in the text.
 & ...As with previous Valkyria Chronicles games, Valkyria Chronicles III is a tactical role @-@ playing game where players take control of a military unit... & Line 2 differs: Differences: ...As with previous \textcolor{red}{[<un -> Val]k[> -> yria]} Chronicles games, Valkyria Chronicles III
is a tactical role @-@ playing game
where players take control of a mili-
tary unit... & 9 \\ \midrule 
\textbf{LaTex} & 
\textbackslash{}begin\{abstract\}\textbackslash{}n\textbackslash{}begin\{abstract\}\textbackslash {}n  \%\textbackslash{}mika\{\}, \textbackslash{}guandao\{\}, \textbackslash{}leo\{\}\textbackslash{}n  \textbackslash{}vspace\{-0.2cm\}\textbackslash{}n  Neural radiance fields (NeRF) rely on volume rendering to...
 & Remove the duplicate \textbackslash{}begin\{abstract\} at the beginning of the abstract environment. & 
\textbackslash{}begin\{abstract\}\textbackslash{}n  \%\textbackslash{}mika\{\}, \textbackslash{}guandao\{\},\textbackslash{}leo\{\}\textbackslash{}n  \textbackslash{}vspace\{-0.2cm\}\textbackslash{}n  Neural radiance fields (NeRF) rely on volume rendering to... & Line 1 differs: Differences: \textbackslash{}begin\{abstract\}\textcolor{red}{[- \textbackslash{}begin\{abstract\}-]} & 9
 \\ \midrule
\textbf{Code} & {...def yield\_nanopub(assertions, annotations, line\_num):\textbackslash{}n """Yield nanopub object""" if not assertions:...}
& Change the function definition from:

def yield\_nanopub(assertions, annotations, line\_num)

to include type annotations as:

def yield\_nanopub(assertions: list, annotations: dict, line\_num: int) -> dict
&...def yield\_nanopub(assertions: list, annotations: dict, line\_num: int) -> dict:
 """Yield nanopub object"""
if not assertions:... & Line 1 differs:
Differences: def yield\_nanopub({assertions\textcolor{red}{[+: list+]}, annotations\textcolor{red}{[+: dict+]}, line\_num\textcolor{red}{[+: int+]})\textcolor{red}{[+ -> dict+]}:} & 10
 \\ \midrule

\textbf{Database DSL} & 
...CREATE TABLE DB\_PRIVS\textbackslash{}n
(\textbackslash{}n
DB\_GRANT\_ID NUMBER NOT NULL,\textbackslash{}n
CREATE\_TIME NUMBER (10) NOT NULL,\textbackslash{}n
DB\_ID NUMBER NULL,\textbackslash{}n
)...
 & Rename the column \texttt{"CREATE\_TIME"} in the \texttt{DB\_PRIVS} table to \texttt{"CREATION\_TIMESTAMP"} & ...CREATE TABLE DB\_PRIVS\textbackslash{}n
(\textbackslash{}n
DB\_GRANT\_ID NUMBER NOT NULL,\textbackslash{}n
CREATION\_TIMESTAMP NUMBER (10) NOT NULL,\textbackslash{}n
DB\_ID NUMBER NULL,\textbackslash{}n
)... & Line 4 differs: Differences: CREATE\textcolor{red}{[E \texttt{->}ION]}\_TIME\textcolor{red}{[+STAMP+]} NUMBER (10) NOT NULL,
 & 9 \\ \bottomrule
\end{tabular}
}
\caption{Data examples of different data categories with all attributes (content, edit request, edited content, difference, and G-score).}
\label{tab:edit-intentions}
\end{table*}


\section{Evaluation}

\subsection{Experimental Setup}
In this section, we detail the experimental setups, including dataset splits, model variants, baselines, evaluation metrics, and implementation specifics.

\noindent \textbf{Dataset and Model Variants.} We evaluate FineEdit on our proposed InstrEditBench using a 90/10 train-test split. Additionally, we introduce three versions of FineEdit—FineEdit-L, FineEdit-XL, and FineEdit-Pro—fine-tuned from LLaMA-3.2-1B, LLaMA-3.2-3B, and Qwen2.5-3B-Instruct base models, respectively, to cover a wide spectrum of architectures and parameter scales.

\noindent \textbf{Baselines.} Our baselines include Gemini 1.5 Flash, Gemini 2.0 Flash, LLaMA-3.2-1B, LLaMA-3.2-3B, Qwen2.5-3B-Instruct, and Mistral-7B, spanning diverse architectures and sizes. We evaluate both zero-shot and few-shot prompting on the Gemini models, while open-source models are assessed using zero-shot prompting.

\noindent \textbf{Metrics.} Following established approaches~\cite{nakamachi2020text,shen2017style}, we use BLEU and ROUGE‑L metrics to assess the vocabulary and structural consistency between the edited and reference texts.

\noindent \textbf{Implementation details.} For existing models, we strictly adhere to configurations from their original papers. To manage fixed maximum token lengths \(L\), if the combined \(T_{\text{orig}}\) and \(I_{\text{edit}}\) exceed \(L\), we partition \(T_{\text{orig}}\) into chunks of size \(\leq L\), process each chunk independently with the same edit instruction, and concatenate the outputs to form the complete edited text. We fine-tune models using Low-Rank Adaptation (LoRA) \cite{hu2021lora} with \(r=8\), \(\alpha=32\), and a dropout rate of 0.05, employing the AdamW optimizer with a learning rate of \(2 \times 10^{-5}\), training for 2 epochs, an effective batch size of 1, and 4 gradient accumulation steps. During generation, we set the temperature to 0.7 and use top-p sampling with \(p=0.9\), merging outputs from all chunks to produce the final edited text. Additional hyperparameter configurations and training details are provided in Appendix~\ref{sec:appx_implementation_detail}.

\subsection{Performance of Existing Models}

We evaluated FineEdit against several state-of-the-art baselines on the InstrEditBench dataset across four data categories as presented in Table~\ref{tab:llm_comparison}.

\noindent \textbf{Comparison with Zero-shot Performance.} Among all baselines, Gemini 1.5 Flash achieved the highest overall scores, while Mistral-7B-OpenOrca recorded the lowest BLEU and ROUGE-L values. Although model size is often a crucial factor, Gemini 2.0 Flash did not surpass Gemini 1.5 Flash in overall effectiveness. For instance, despite having more parameters than LLaMA-3.2-1B, Mistral-7B-OpenOrca underperformed in both metrics, highlighting the significance of model architecture and training methods. Moreover, while Gemini 2.0 Flash shows superior semantic understanding in the Wiki category—achieving a BLEU score of 0.9133 and a ROUGE-L score of 0.9429—its overall performance remains below that of its counterpart.

FineEdit, and in particular its FineEdit-Pro variant, further outperforms all zero-shot baselines. FineEdit-Pro achieves an overall BLEU score of 0.9245, representing improvements of approximately 11.6\%, 57.7\%, and 184.7\% over Gemini 1.5 Flash (0.8285), LLaMA-3.2-3B (0.5862), and Mistral-7B-OpenOrca (0.3246), respectively. These gains are consistently observed across individual data categories—for example, FineEdit-Pro attains BLEU scores of 0.9521 and 0.9538 in the DSL and Code domains, respectively. These results underscore the effectiveness of FineEdit’s targeted fine-tuning strategy, which focuses on precise editing of location and content to preserve both structural and semantic integrity.

\noindent \textbf{Comparison with Few-shot Performance.} We further evaluated few-shot learning on the Gemini models. Although few-shot performance notably improved in some categories—for example, in the LaTeX domain, where Gemini 2.0 Flash exhibited a 20\% higher BLEU score than in the zero-shot setting—the overall few-shot results still lag behind FineEdit. In certain cases, such as the SQL category, few-shot learning made little difference, with BLEU and ROUGE-L scores of only 0.1600 and 0.1814, respectively. These findings reinforce the value of our curated benchmark in driving improvements in editing tasks.

\begin{minipage}{0.45\textwidth}
\begin{shaded}
    \noindent \textbf{Key Findings:} FindEdit demonstrates robust overall effectiveness across Wiki, Code, DSL, and LaTeX categories. These results not only position FineEdit as a competitive method for structured editing tasks but also provide valuable insights into how targeted training strategies can elevate model performance in diverse application scenarios.
\end{shaded}
\end{minipage}


\begin{table*}[!tbp]
    \centering
    \small
    \renewcommand{\arraystretch}{1.4}
    \begin{adjustbox}{width=\textwidth}
    \begin{tabular}{lcc|cc|cc|cc|cc|cc}
        \toprule
        \multirow{2}{*}{\textbf{Method}} &
        \multirow{2}{*}{\textbf{Model}} &
        \multirow{2}{*}{\textbf{Open-Source}} &
        \multicolumn{2}{c|}{\textbf{LaTeX}} &
        \multicolumn{2}{c|}{\textbf{DSL}} &
        \multicolumn{2}{c|}{\textbf{Wiki}} &
        \multicolumn{2}{c|}{\textbf{Code}} &
        \multicolumn{2}{c}{\textbf{Overall}} \\
        \cmidrule(lr){4-5} \cmidrule(lr){6-7}
        \cmidrule(lr){8-9} \cmidrule(lr){10-11} \cmidrule(l){12-13}
        & & & \textbf{BLEU} & \textbf{ROUGE-L}
              & \textbf{BLEU} & \textbf{ROUGE-L}
              & \textbf{BLEU} & \textbf{ROUGE-L}
              & \textbf{BLEU} & \textbf{ROUGE-L}
              & \textbf{BLEU} & \textbf{ROUGE-L} \\
        \midrule
        \multirow{6}{*}{\textbf{Zero-shot}}
            & Gemini 1.5 Flash  & \ding{55} & 0.8665 & 0.9150 & 0.8297 & 0.8555 & 0.7626 & 0.8361 & 0.8551 & 0.9073 & 0.8285 & 0.8819 \\
             
            & Gemini 2.0 Flash  & \ding{55} & 0.7413& 0.7951 & 0.4706 & 0.4964 & 0.9133 & 0.9429 & 0.1339 & 0.2737 & 0.5853 & 0.6519 \\
             
            & Llama-3.2-1B & {\checkmark} & 0.5088 & 0.6108 & 0.5564 & 0.6596 & 0.4413 & 0.5766 & 0.4742 & 0.6072 & 0.4867 & 0.6069 \\
             
            & Llama-3.2-3B & {\checkmark} & 0.5969 & 0.6925 & 0.5747 & 0.6821 & 0.5061 & 0.6384 & 0.6638 & 0.7727 & 0.5862 & 0.6976 \\
             
            & Qwen2.5-3B-Instruct & {\checkmark} & 0.5467 & 0.6712 & 0.4107 & 0.4991 & 0.4170 & 0.5699 & 0.3967 & 0.5390 & 0.4492 & 0.5816 \\
             
            & Mistral-7B-OpenOrca & {\checkmark} & 0.3782 & 0.5770 & 0.0361 & 0.1638 & 0.3608 & 0.5840 & 0.3763 & 0.6447 & 0.3246 & 0.5395 \\
        \midrule
        \multirow{2}{*}{\textbf{Few-shot}}
            & Gemini 1.5 Flash$_{(2-shot)}$   & \ding{55} & 0.8742 & 0.9324 & 0.0908 & 0.1190 & 0.8657 & 0.9139  & 0.7412 & 0.8302 & 0.7249 & 0.7845 \\
            & Gemini 2.0 Flash$_{(2-shot)}$   & \ding{55} & 0.9464 & 0.9723 & 0.1600 & 0.1814 & \textbf{0.9380} & \textbf{0.9665} & 0.8327 & 0.8698 & 0.8011 & 0.8302 \\
        \midrule

        \multirow{3}{*}{\textbf{FineEdit}}
            & FineEdit-L  & {\checkmark} & 0.9311 & 0.9697 & 0.9334 & 0.9615 & 0.8077 & 0.9036 & 0.9296 & 0.9725 & 0.8957 & 0.9504 \\
            & FineEdit-XL  & {\checkmark} & 0.8867 & 0.9502 & 0.9241 & 0.9552 & 0.8120 & 0.9056 & 0.9295 & 0.9720 & 0.8824 & 0.9441 \\
            & FineEdit-Pro & {\checkmark} & \textbf{0.9539} & \textbf{0.9821} &\textbf{ 0.9521} &\textbf{ 0.9710 } &  0.8521 & 0.9185 & \textbf{ 0.9538} & \textbf{0.9836} & \textbf{0.9245} & \textbf{0.9628} \\
        \bottomrule
    \end{tabular}
    \end{adjustbox}
    \caption{Comparison of LLMs on BLEU and ROUGE-L  for LaTeX, DSL, Wiki, Code. Overall data displays average performance among all data categories. The best results are highlighted in bold.}
    \label{tab:llm_comparison}
\end{table*}


\subsection{FineEdit: Supervised Finetuning}

Our FineEdit model is offered in three variants: FineEdit-L, FineEdit-XL, and FineEdit-Pro. Under zero-shot conditions, FineEdit-L consistently outperforms all baseline models in BLEU and ROUGE-L scores for LaTeX, DSL, Wiki, and Code tasks. For example, compared to Gemini 1.5 Flash, FineEdit-L improves overall BLEU scores by roughly 8\%, with even larger gains observed in specific categories. Notably, FineEdit-XL performs similarly to FineEdit-L, suggesting that increasing the parameter count from 1B to 3B using LLaMA does not yield a significant performance boost.

By leveraging the superior instruction-following capabilities of Qwen2.5-3B-Instruct, our final variant, FineEdit-Pro, further elevates performance. FineEdit-Pro achieves an overall BLEU score of 0.9245, which represents improvements of approximately 11.6\% over Gemini 1.5 Flash, and gains of around 14.7\% and 11.7\% in the DSL and Wiki tasks, respectively. These consistent improvements across multiple data categories underscore the effectiveness of our supervised fine-tuning strategy and highlight the importance of a strong instruction-tuned base model over merely increasing model size.

We also compared our models with Gemini's few-shot prompting approach in real-world scenarios. Although in-context learning (ICL) boosts Gemini’s performance in some cases—such as a 8\% higher BLEU score in Wiki dataset for Gemini 2.0 Flash—the overall results still lag behind FineEdit-Pro. This evidence confirms that our tailored supervised fine-tuning approach yields a more robust and generalizable solution for structured editing tasks.

\begin{minipage}{0.45\textwidth}
\begin{shaded}
    \noindent \textbf{Key Findings:} FineEdit's supervised fine-tuning markedly enhances performance. FineEdit-L surpasses zero-shot baselines and FineEdit-XL offers comparable gains, while FineEdit-Pro (built on Qwen2.5-3B-Instruct) achieves the highest scores. This highlights that robust instruction tuning is more effective than merely scaling model size.
\end{shaded}
\end{minipage}


\subsection{Qualitative Study}

To qualitatively assess the performance of FindEdit, we conduct several studies as shown in Figure~\ref{fig:qualitative_study}. This figure illustrates eight examples of how FineEdit-Pro and Gemini respond to diverse editing requests. In several cases, FineEdit-Pro accurately applies changes—such as adding new columns in DSL or adjusting environment commands—while Gemini often restates the instruction without implementing the intended modifications.

Specifically, both Gemini 1.5 Flash and 2.0 Flash perform well on LaTeX and Wiki tasks, yet they struggle with DSL and Code tasks. For example, as shown in Figure~\ref{fig:qualitative_study}, FineEdit-Pro correctly identifies the target table and appends a new column named \texttt{created\_at} with the data type \texttt{DEFAULT CURRENT\_TIMESTAMP}. In contrast, Gemini misinterprets the instruction, merely repeating the edit request rather than applying the intended change. These observations highlight the qualitative strengths of our proposed FineEdit approach.

Nonetheless, FineEdit is not without shortcomings. In the LaTeX example depicted in Figure~\ref{fig:qualitative_study}, Gemini accurately locates the \texttt{\/subsection\{Strengths\}} and updates it as specified. However, although FineEdit-Pro also identifies and modifies the correct location, it generates the correct response twice, which deviates from the direct editing requirement. This discrepancy suggests that FineEdit-Pro, though generally more reliable, can overapply modifications in specific cases.

Overall, these results illustrate FineEdit-Pro’s capacity to handle more complex edits, particularly for DSL and Code, while Gemini often fails to implement them. Nevertheless, occasional issues like duplicate outputs highlight the need for refinement, ensuring FineEdit-Pro consistently adheres to direct editing requirements without introducing redundant content. On the other hand, Gemini occasionally performs better in simpler tasks, such as LaTeX updates.

\begin{minipage}{0.45\textwidth}
\begin{shaded}
    \noindent \textbf{Key Findings:} FineEdit-Pro demonstrates superior handling of DSL and Code edits compared to Gemini, though minor issues such as duplicate outputs in LaTeX tasks remain. Overall, FineEdit's qualitative performance confirms its robust ability to interpret and execute complex editing instructions.
\end{shaded}
\end{minipage}

\begin{figure*}[!tbp]
    \centering
    \includegraphics[width=0.88\textwidth]{figure/figure3_v3.pdf}
    \caption{Comparison between Gemini and FindEdit Pro response.}
    \label{fig:qualitative_study}
\end{figure*}

\subsection{Human Evaluation}

To assess whether \texttt{DiffEval} enhances overall dataset quality, we conducted a human evaluation. Given that our dataset includes Code and DSL categories—areas closely tied to computer programming—we have three evaluators, each holding at least a Bachelor's degree in a Computer Science-related field. We established the following guidelines to ensure rigorous assessment:
(1) {Precise Observation:} Confirm that the updated content exactly corresponds to the segment specified by the edit request. (2) {No Unintended Modifications:} Verify that no other sections have been altered; any unexpected changes result in failure. (3) {Three-Round Procedure:} Two evaluators independently review each item, with a third evaluator resolving any discrepancies.

We examined 100 items per category and found that data processed through our DiffEval pipeline exhibited noticeably enhanced accuracy, as shown in Table~\ref{tab:human_eval}. The Wiki and Code datasets, in particular, demonstrated the most reliable outcomes, with edited content precisely matching the requested modifications. Notably, the DSL dataset experienced the greatest improvement, with quality increasing by over 24\% compared to data that did not meet DiffEval’s standards.

\begin{table}[!tbp]
\centering
\renewcommand\arraystretch{0.95}
\resizebox{0.45\textwidth}{!}{
\begin{tabular}{lcccc}
\hline
       & Wiki & LaTeX & DSL & Code \\
\hline
G-score $\geq$ 9  & 97\% & 93\% & 90\% & 97\% \\
G-score $<$ 9   & 87\% & 89\% & 66\% & 83\% \\
\hline
\end{tabular}
}
\caption{Sample performance based on the G-Score}
\label{tab:human_eval}
\end{table}

\begin{minipage}{0.45\textwidth}
\begin{shaded}
    \noindent \textbf{Key Findings:} The DiffEval pipeline significantly improves dataset quality, with Wiki and Code categories achieving high precision, and the DSL category showing over 24\% enhancement in quality.
\end{shaded}
\end{minipage}


\section{Conclusion}
This work addresses the critical gap in LLMs' ability to perform precise and targeted text modifications. We introduce {InstrEditBench}, a high-quality benchmark with 20,000+ structured editing tasks across Wiki articles, LaTeX documents, code, and database DSLs, enabling rigorous evaluation of direct editing capabilities. To further advance LLMs’ editing proficiency, we propose {FineEdit}, a specialized model trained on this benchmark. Extensive evaluations demonstrate that FineEdit outperforms state-of-the-art models, including GPT-4o, Gemini 2.0, and LLaMa-3.2, with up to 10\% improvement compared to Gemini in direct editing task performance.

\clearpage

\section{Limitations}

\noindent \textbf{Limited Deployment Scope.} Due to cost and hardware constraints, our evaluations were limited to large proprietary LLMs (e.g., Gemini), rather than large open-source models.

\noindent \textbf{Controlled Context Evaluation.} Our benchmark focuses on controlled evaluation context, where it does not yet encompass long-context chain-of-thought scenarios, as smaller LLMs are confined by limited context windows, even though such techniques could be effective in proprietary models
%\bibliographystyle{plain}%\bibliography{references}  %%% Remove comment to use the external .bib file (using bibtex).
%%% and comment out the ``thebibliography'' section.
\documentclass{MITstyle}

%\usepackage[table]{xcolor}
\usepackage{chngcntr}
\usepackage{hyperref}
\usepackage{microtype}

\title{A Lightweight and Extensible Cell Segmentation and Classification Model for Whole Slide Images}

\author{Nikita Shvetsov~$^{1, }$\footnote{Correspondence e-mail: nikita.shvetsov@uit.no}, Thomas K. Kilvaer~$^{2, 3}$, Masoud Tafavvoghi~$^{4}$, Anders Sildnes~$^{1}$, \\ Kajsa Møllersen~$^{4}$, Lill-Tove Rasmussen Busund~$^{5, 6}$, Lars Ailo Bongo~$^{1}$ \\
%
\vspace{1em} % Space between authors and afilliations
%
\normalfont{\small $^{1}$Department of Computer Science, UiT The Arctic University of Norway}\\
\normalfont{\small $^{2}$Department of Oncology, University Hospital of North Norway}\\
\normalfont{\small $^{3}$Department of Clinical Medicine, UiT The Arctic University of Norway}\\
\normalfont{\small $^{4}$Department of Community Medicine, UiT The Arctic University of Norway}\\
\normalfont{\small $^{5}$Department of Medical Biology, UiT The Arctic University of Norway} \\
\normalfont{\small $^{6}$Department of Clinical Pathology, University Hospital of North Norway} %\vspace{2em}
}

\begin{document}
\maketitle

\section*{Abstract}

% \begin{abstract}
% Developing clinically useful cell-level analysis tools in digital pathology remains challenging due to limitations in dataset granularity, inconsistent annotations, computational demands of advanced models, and difficulties in integrating new technologies into clinical workflows. To address these challenges, we propose a multi-faceted solution that enhances data quality, model performance, and usability to create a lightweight and extensible cell segmentation and classification model.

% First, we update data labels by employing a cross-relabeling process that refines the labels of two existing datasets, PanNuke and MoNuSAC, to create a new unified dataset with enhanced granularity, encompassing seven distinct cell types. Second, we leverage the H-Optimus foundation model as a fixed encoder to improve feature representation for simultaneous cell segmentation and classification tasks. Third, to address the computational demands of foundation models, we employ knowledge distillation to reduce model size and complexity while maintaining comparable performance. Finally, to facilitate integration into clinical workflows, we integrate the distilled model into the QuPath software, a widely used open-source platform in digital pathology.

% Our results demonstrate improvements in cell segmentation and classification performance using the H‑Optimus-based model compared to a CNN-based model. Specifically, the average $R^2$ improved from 0.575 to 0.871, and the average $PQ$ score improved from 0.450 to 0.492, indicating better alignment with actual cell counts and enhanced segmentation and classification quality. Furthermore, the distilled student model maintains performance comparable to the larger foundation model while reducing the parameter count by a factor of 48.
% Overall, by reducing computational complexity and integrating it into existing workflows, the proposed approach may significantly impact diagnostic processes, reduce the workload of pathologists, and contribute to improved patient outcomes. Though our approach shows potential enhancements in efficiency and usability of cell segmentation and classification models in digital pathology, extensive validation is needed to deploy these models in clinical practice.
% \end{abstract}

%%% shortened abstract
\begin{abstract}
Developing clinically useful cell-level analysis tools in digital pathology remains challenging due to limitations in dataset granularity, inconsistent annotations, high computational demands, and difficulties integrating new technologies into workflows. To address these issues, we propose a solution that enhances data quality, model performance, and usability by creating a lightweight, extensible cell segmentation and classification model. 

First, we update data labels through cross-relabeling to refine annotations of PanNuke and MoNuSAC, producing a unified dataset with seven distinct cell types. Second, we leverage the H-Optimus foundation model as a fixed encoder to improve feature representation for simultaneous segmentation and classification tasks. Third, to address foundation models' computational demands, we distill knowledge to reduce model size and complexity while maintaining comparable performance. Finally, we integrate the distilled model into QuPath, a widely used open-source digital pathology platform. 

Results demonstrate improved segmentation and classification performance using the H-Optimus-based model compared to a CNN-based model. Specifically, average $R^2$ improved from 0.575 to 0.871, and average $PQ$ score improved from 0.450 to 0.492, indicating better alignment with actual cell counts and enhanced segmentation quality. The distilled model maintains comparable performance while reducing parameter count by a factor of 48. By reducing computational complexity and integrating into workflows, this approach may significantly impact diagnostics, reduce pathologist workload, and improve outcomes. Although the method shows promise, extensive validation is necessary prior to clinical deployment.
\end{abstract}
\clearpage

\section{Introduction}
In digital pathology, accurate segmentation and classification of cells are crucial for many diagnostic, prognostic, and predictive analyses \cite{Jaber_Beziaeva_etal._2019,Lin_Pan_etal._2022,Park_Ock_etal._2022,Shen_Choi_etal._2024}. Nowadays, developments in computational pathology offer multiple solutions \cite{H._Qu_P._Wu_etal._2020,Javed_Mahmood_etal._2020} to utilize cell-level datasets to train machine learning models that solve these problems. The quality and specificity of training datasets are critical for robust and accurate models. Adhering to the principle of "garbage in, garbage out", it is essential to ensure that these datasets are extensively and accurately labeled with distinct classes that reflect the diverse biological characteristics of different cell types. Unfortunately, the number of open-source datasets comprising such high-quality annotations is limited. Existing cell segmentation datasets \cite{Gamper_Koohbanani_etal._2019,Graham_Vu_etal._2019,Verma_Kumar_etal._2021} may offer extensive annotations for certain cell types while providing more general labels for others. For example, in PanNuke, which is one of the largest open-source datasets comprising labeled cells, various types of morphologically and functionally different inflammatory cells like macrophages and lymphocytes are clustered in a broad "inflammatory" class. Consequently, these classes are frequently omitted from analyses or aggregated into broader meta-classes \cite{Gamper_Koohbanani_etal._2020} and likely interfere with other cell classes included in the dataset. This and similar inconsistencies in annotation granularity limit the ability of machine learning models to learn the comprehensive and nuanced features necessary for accurate cell segmentation and classification. To address these challenges, methods for refining and standardizing dataset annotations are essential to enhance the quality of training data.

A complementary approach to mitigate the absence of high-quality training data is the use of foundation models. Foundation models as encoders are defined as large-scale, versatile networks pre-trained on vast, diverse datasets using self-supervised learning, contrasting with convolutional neural network (CNN) pre-trained encoders that rely on supervised learning with labeled data. In practice, foundation models leverage enormous amounts of weakly or unlabeled data from millions of whole slide images (WSIs) and employ self-attention mechanisms to capture long-range dependencies and global context \cite{Chen_Ding_etal._2024,Saillard_Jenatton_etal._2024,Vorontsov_Bozkurt_etal._2024,Xu_Usuyama_etal._2024}. As a consequence, foundation models are able to produce transferable feature representations across different cell types and tissue environments. The feature representations can be leveraged by decoder networks to produce segmentation masks and pixel-level classifications. Because foundation models have comprehensive feature representations, they can be effectively fine-tuned using much smaller amounts of cell-level data compared to the large datasets needed to train models from scratch. Furthermore, foundation models incorporate adversarial training elements or contrastive learning \cite{Chen_Ding_etal._2024,Xu_Usuyama_etal._2024}, enhancing their resilience and adaptability by exposing them to challenging and varied scenarios during training. This may result in more generalizable models, often making them well-suited for diverse and complex tasks in digital pathology.

Despite the inherent advantages of foundation models, their deployment for practical use faces its own obstacles. In particular, they require substantial computational power, financial investments and rigorous testing to ensure reliability and efficacy for a given task \cite{Akkus_Dangott_etal._2022,Dragomir_Cocuz_etal._2022,Go_2022,Jafri_Farooqui_etal._2024}. Moreover, while foundation models enhance feature representation and performance, they depend on the quality of available annotations for decoder fine-tuning and, like any other model, cannot resolve existing inconsistencies or ambiguities in data labels. Therefore, there remains a critical need for solutions that address both data quality and practical deployment considerations.
Further, integrating new technologies into existing clinical workflows often encounters resistance, as it necessitates adjustments to established diagnostic processes. So, there is a need to develop solutions that could be integrated into current practices, minimizing the burden on medical professionals to adopt new tools \cite{King_Williams_etal._2023}.

Existing solutions \cite{Goldsborough_Philps_etal._2024,Hörst_Rempe_etal._2024}, while addressing some aspects of these challenges, fall short in providing a comprehensive approach. To address the data quality and clinical deployment issues, we propose a multi-faceted solution that encompasses data refinement, model optimization, and integration with existing pathology tools (\hyperref[fig:fig1]{Figure 1}). The outcome is a lightweight cell segmentation and classification model that can be integrated into digital pathology workflows for practical clinical use.

\begin{figure}[h!]
    \centering
    \includegraphics[width=\textwidth, height=0.82\textheight, keepaspectratio]{images/Figure_1.pdf}
    \caption{Overview of the proposed solution, including 1) Data refinement using cross-relabeling, 2) Teacher model development and fine tuning, 3) Student model optimization with knowledge distillation and 4) Student model and QuPath integration}
    \label{fig:fig1}
\end{figure}
\clearpage

Our approach begins with preparing the data for the fine-tuning and training of the machine learning models. We create a refined dataset, acquired via cross-relabeling two cell-level datasets, enhancing annotation specificity and consistency of the labeled data. Subsequently, we create a cell segmentation and classification model based on the foundation model. We leverage the foundation model as a fixed encoder and fine-tune a decoder using the refined dataset to improve generalization across diverse tissue- and cell types.
To ensure that the model remains lightweight and deployable in a possibly resource-constrained environment, we employ knowledge distillation to approximate the functionality of the foundation model. Finally, to facilitate the practical application of our model in digital pathology workflows, we integrate it with the QuPath \cite{Bankhead_Loughrey_etal._2017} application. Each methodological component contributes to the overarching goal of enhancing model performance, generalizability, and usability in clinical settings.

The primary contributions of this paper are:
\begin{enumerate}
    \item \textit{Data labels refinement through cross-relabeling:}
    
    We propose a new method for refining labels of cell-level datasets through cross-relabeling. This method employs classification models to re-label broad and ambiguous instances, resulting in a more diverse dataset. Our evaluation demonstrates that these classification models achieve high accuracy on test subsets, indicating the reliability of the method for label refinement.

    \item \textit{Enhanced model performance via foundation models:}
    
    We employ a foundation model as a feature extractor for the cell segmentation and classification task. In comparison with training a CNN model from scratch, the foundation model backbone only needs fine-tuning, which significantly reduces training time, computational resources and data requirements. We show that using a foundation model encoder leads to better performance in cell segmentation and classification networks than using a CNN-based encoder. This improvement may enable the model to generalize more effectively across various tissue types and imaging methods.
    
    \item \textit{Model optimization through knowledge distillation:}
    
    We show that a smaller student model trained using knowledge distillation on the refined dataset obtained via our cross-relabeling approach from a foundation model achieves comparable performance in cell segmentation and quantification tasks. As a result, this model is more suitable for deployment in environments without high-performance computing resources.
    
    \item \textit{Integration with QuPath:}
    
    We integrate the distilled cell segmentation and classification model into QuPath, a widely used open-source digital pathology platform, to accelerate clinical adaptation by enabling pathologists to more easily incorporate advanced computational tools into their existing workflows.
\end{enumerate}

Through these methodological steps, we aim to bridge the gap between advanced machine learning techniques and practical clinical applications, making accurate and efficient digital pathology accessible in a broader range of healthcare settings.

\section{Refining Existing Datasets Using Cross-Relabeling}
To address the limitations of sparse and ambiguous labeling of cell-level datasets, we propose a generalizable cross-relabeling strategy that can be applied to any dataset containing broadly categorized or imprecisely labeled cell types. This approach involves training and subsequently leveraging classification models to refine broad categories into more specific or biologically relevant classes.
When applied to cell-level data, the methodology includes extracting individual cell images from the dataset patches, preprocessing these images to standardize the size and accommodate partial cells, and then training deep learning classifiers capable of distinguishing between the finer cell subtypes within the coarser categories. 
To illustrate our approach, we focus on the PanNuke \cite{Gamper_Koohbanani_etal._2020, Gamper_Koohbanani_etal._2019} and MoNuSAC \cite{Verma_Kumar_etal._2021} datasets that we have used to train models for cell quantification in our previous works \cite{Shvetsov_Grønnesby_etal._2022,Shvetsov_Sildnes_etal._2024}. We find that for better cell differentiation we have to introduce more granular labels. PanNuke includes a broad classification of "inflammatory" cells, encompassing lymphocytes, macrophages, and neutrophils. Each cell type differs significantly in structure, function, and clinical relevance. Conversely, MoNuSAC uses the label "epithelial" for a class that comprises both benign epithelial cells and malignant neoplastic cells. This practice makes it challenging to differentiate between benign and malignant epithelial cells in the dataset, which is a critical distinction when identifying tumor areas within tissue samples. To address these issues, we implement a cross-relabeling strategy as shown in \hyperref[fig:fig2]{Figure 2}. The key components are two classification models: one is trained on singular cell images from PanNuke data to classify the epithelial meta-class into epithelial and neoplastic classes. The other is trained on MoNuSAC to refine the inflammatory class into lymphocytes, neutrophils, and macrophages.

\begin{figure}[h!]
    \centering
    \includegraphics[width=\textwidth]{images/Figure_2.pdf}
    \caption{Refined dataset generation via cross relabeling}
    \label{fig:fig2}
\end{figure}

The refining approach consists of three consecutive steps. The first is the preprocessing step, in which we extract individual cells from both datasets (\hyperref[fig:fig3]{Figure 3}). The specifics of PanNuke and MoNuSAC patch preparation before cell preprocessing are provided in \hyperref[chap:S1]{Appendix S1}.

\begin{figure}[h!]
    \centering
    \includegraphics[width=\textwidth]{images/Figure_3.pdf}
    \caption{Cell instances preprocessing including (1) cell map extraction, (2) bounding box delineation, (3) adjusting cell boxes and (4) cropping and resizing of cell images}
    \label{fig:fig3}
\end{figure}

During preprocessing, we extract cell type maps from the ground truth label mask and calculate bounding boxes around each cell instance. To accommodate partial cells at patch borders, a common issue in cropped patch images, we employ mirror padding and extend the field of view of the cell label by 15 pixels to capture adjacent cells. We then crop and resize the identified regions to $64 \times 64$ pixels using bicubic interpolation.

The preprocessed PanNuke dataset comprises 68,031 neoplastic and 23,207 epithelial cell images, while MoNuSAC comprises  33,104 lymphocytes, 1,252 neutrophils, and 1,695 macrophages, which we subsequently use in training cell classification models and classifying the cell image data \hyperref[fig:S2]{Appendix Figure S2 (1)}. 

The next step is to train two distinct ResNet50-based classifiers tailored to address the specific labeling challenges inherent in each dataset. We use ResNet50 for classification models due to its proven effectiveness for image classification tasks in histopathology \cite{pan2022reviewmachinelearningapproaches}, and its compatibility with small images. For the PanNuke dataset, we design the classifier, trained on MoNuSAC data, to disaggregate the heterogeneous "inflammatory" cell category into distinct subtypes: lymphocytes, macrophages, and neutrophils. Similarly, for the MoNuSAC dataset, the classifier is trained on PanNuke data and distinguishes between benign and malignant epithelial cells within the overarching "epithelial" label. By applying these targeted classifiers to their respective datasets, we assign more specific labels to individual cell instances, thus enabling us to create a unified dataset.
To ensure a balanced representation of classes, we train both models on datasets that had been equalized to match the size of the least represented class. Thus, we obtain datasets comprising 23,207 samples per class for PanNuke and 1,252 samples per class for MoNuSAC data. Next, we partition both of them into training (70\%), validation (20\%), and testing (10\%) subsets. To mitigate the risk of overfitting, we use a single dropout layer with a rate of p=0.5 in both models and data augmentation using randomized color perturbations, rotation, and horizontal and vertical flipping. We employ AdamW optimizer and the cross-entropy loss function for the training criterion.

To evaluate the two trained models, we measure the classification accuracy on the respective test subsets. The accuracies on the test subset for both classifiers are presented in \hyperref[tab:1]{Table 1}. The PanNuke model achieves an average accuracy of 93.57\%, with higher accuracy for neoplastic cells (96.06\%) compared to epithelial cells (86.26\%). The confusion matrix in Figure A3.1 shows that the model predominantly distinguishes accurately between epithelial and neoplastic tissues, with a substantial number of correct classifications and relatively few misclassifications. The MoNuSAC model demonstrates an average accuracy of 98.92\%, excelling in classifying lymphocytes (99.67\%) and macrophages (94.12\%), with lower performance for neutrophils (85.71\%). The confusion matrix in Figure A3.2 shows that the model identifies lymphocytes and performs reasonably well with macrophages and neutrophils.

\begin{table}[h!]
\renewcommand{\arraystretch}{1.5}
  \centering
  \caption{Cell classification results for PanNuke and MoNuSAC trained models (CI 95\%).}
  \label{tab:1}
  \begin{tabular}{|l|c|c|}
   \hline
   %\rowcolor{gray!30}
    Accuracy               & PanNuke model              & MoNuSAC model              \\
    \hline
    Average      & 0.936 (0.931--0.941)         & 0.989 (0.986--0.993)        \\
    \hline
    Neoplastic   & 0.961 (0.956--0.965)         & -                          \\
    \hline
    Epithelial   & 0.863 (0.849--0.877)         & -                          \\
    \hline
    Lymphocytes  & -                          & 0.997 (0.995--0.999)        \\
    \hline
    Neutrophils  & -                          & 0.857 (0.796--0.918)        \\
    \hline
    Macrophages  & -                          & 0.941 (0.906--0.976)        \\
    \hline
  \end{tabular}
\end{table}

Finally, during the last step, we use the model trained on PanNuke data for epithelial cells in MoNuSAC and the model trained on MoNuSAC for the inflammatory cells class in PanNuke. Specifically, we use classifier models to relabel epithelial cells in MoNuSAC and inflammatory cells in PanNuke data. Then we combine cells with refined labels and the rest of the cells in both datasets to create a refined dataset (\hyperref[fig:S2]{Appendix Figure S2 (2)}). The process of relabeling cells and visualizing them on a patch is shown in \hyperref[fig:fig4]{Figure 4}. The cell counts in the refined dataset are provided in \hyperref[tab:S4]{Appendix Table S4}.

\begin{figure}[h!]
    \centering
    \includegraphics[width=\textwidth, height=0.42\textheight, keepaspectratio]{images/Figure_4.pdf}
    \caption{Cell relabeling procedure for epithelial and inflammatory cell classes}
    \label{fig:fig4}
\end{figure}

%\hfill

Relabeling and combining datasets have been explored in a prior study \cite{Parulekar_Kanwat_etal._2023}, where consecutive fine-tuning on multiple datasets was employed to account for hierarchical class label structures. While the method presented in \cite{Parulekar_Kanwat_etal._2023} is intuitive, it often lacks consistency and requires multiple fine-tuning runs, which can be cumbersome and time-consuming. 
In contrast, cross-relabeling simplifies this process by using specialized classification models tailored to each dataset's specific labeling challenges. This approach provides better transparency and produces a unified dataset encompassing seven distinct cell types across multiple tissue samples, enhancing data diversity for further model training or fine-tuning.

Despite these improvements, cross-relabeling does not entirely resolve issues related to poor labeling quality or the amount of labeled data. Specifically, our results show lower accuracies persist for underrepresented classes, such as macrophages, which may stem from a limited sample availability and intrinsic challenges in distinguishing these cells based solely on H\&E staining. Furthermore, while our method enhances label specificity, it relies on the initial quality of the broad labels; thus, any fundamental inaccuracies in the original annotations can propagate through the relabeling process. Addressing the overall problem of limited data labels may require integrating additional data sources or utilizing complementary immunohistochemical staining methods.
Although the reported performance metrics are obtained from evaluations on the native test sets of each dataset, it is important to note that the primary application of these classifiers is to perform cross-relabeling, where a model trained on one dataset (e.g., PanNuke) is applied to another (e.g., MoNuSAC) and vice versa. We acknowledge that a more systematic evaluation of cross-dataset generalization is needed and could be performed in future work.

Overall, the refined dataset produced by our approach can enhance the supervised training or fine-tuning of cell segmentation and classification models, especially those that utilize pre-trained foundation models to improve feature extraction robustness. In addition, these models can detect nuanced classes that enable researchers to conduct more detailed analyses of biological processes in computational pathology.

\section{Foundation models for robust cell segmentation and classification}

Accurate cell segmentation and classification in digital pathology are hindered by limited labeled data and the fact that conventional CNNs are unable to capture global contextual information due to their local receptive field constraints \cite{Gheflati_Rivaz_2022,Yang_Marcus_etal.}. Traditional approaches in cell quantification have predominantly relied on CNN encoders, such as ResNet50, given their proven effectiveness in semantic segmentation tasks \cite{Deshmane_2023,Graham_Vu_etal._2019,Mukasheva_Koishiyeva_etal._2024,Stringer_Wang_etal._2021}. However, approaches that include fine-tuning of pretrained CNNs, data augmentation, and stain normalization to partially increase data variability and address staining differences often fail to achieve the necessary generalization and robustness across diverse tissue types and staining conditions \cite{G._Wang_W._Li_etal._2018,Gao_Bagci_etal._2018,Karim_El_Khoury_Martin_Fockedey_etal._2021}.

To overcome these challenges, we leverage an encoder-decoder network that uses a foundation model as the encoder and a CNN upsampling decoder (\hyperref[fig:fig5]{Figure 5}) for simultaneous cell segmentation and classification in 2D patches extracted from WSIs. Foundation models with transformer-based architectures are viable alternatives to CNN-based encoders \cite{Shamshad_Khan_etal._2023,Sourget_2023}. They enable the creation of more advanced architectures that can decode or transform learned features more effectively \cite{Chen_Duan_etal._2023,Cheng_Misra_etal._2022,Xie_Wang_etal._2021}.

\begin{figure}[h!]
    \centering
    \includegraphics[width=\textwidth]{images/Figure_5.pdf}
    \caption{UNETR-like model with foundational model as backbone}
    \label{fig:fig5}
\end{figure}

By utilizing a transformer-based encoder, we incorporate global contextual information into the feature extraction process, which is a key advantage of such architectures \cite{Chen_Lu_etal._2021}. This foundation model integration facilitates accurate pixel-wise segmentation and classification without the need for extensive encoder training, thereby potentially improving generalization across varied cellular structures and tissue types.
In our implementation, we employ a modified UNETR \cite{Hatamizadeh_Tang_etal._2021} architecture that combines a vision transformer (ViT) \cite{Dosovitskiy_Beyer_etal._2021} encoder with a CNN-based decoder. The encoder utilizes the pretrained H-Optimus foundation model, which contains 1.1 billion parameters and is trained on over 500,000 H\&E stained WSIs \cite{Saillard_Jenatton_etal._2024}. We extract outputs from four evenly spaced transformer blocks $Z_i$, where $i \in [1, 14, 26, 38]$, to serve as residual connections for the CNN decoder. We select these blocks based on our observation that features from non-adjacent levels of the encoder lead to better overall performance on the test subset.

The CNN decoder upsamples the feature representations, acquired from the transformer blocks, to generate an intermediate vector that is handled by two task-specific layers that generate cell segmentation and classification masks. The first task-specific layer is the ‘Cellpose head’,  which is used to delineate cell instances. The layer generates horizontal and vertical gradient maps to form vector fields that are refined through gradient tracking in a post-processing step using the Cellpose algorithm \cite{Stringer_Wang_etal._2021}, known for its efficacy in cell segmentation tasks and generalizability across multiple domains \cite{Pachitariu_Stringer_2022,Stringer_Pachitariu_2024}. The second task-specific layer is the "Cell type head", which assigns labels to individual pixels. In the post-processing step, we determine the output classification label of each segmented cell instance by majority voting over the labeled pixels that comprise the cell in the segmentation map.

To evaluate model performance and measure the impact of adding a foundation model as backbone, we compare it to a ResNet50-based model. ResNet50 is a widely used solution for encoders in segmentation architectures in the medical domain \cite{Deshmane_2023,Graham_Vu_etal._2019,Mukasheva_Koishiyeva_etal._2024,Stringer_Wang_etal._2021}. For the H-Optimus-based model, we utilize frozen weights for the encoder and only fine-tune the decoder to take advantage of the extensive pre-training of the foundation model. For the ResNet50-based model we start with ImageNet \cite{Deng_Dong_etal.} weights and train both encoder and decoder parts. Hyperparameters for the training step are set to be identical, where possible, for comparable evaluation. 
For this evaluation, we deliberately use the PanNuke dataset to provide a standardized and controlled comparison between the H‑Optimus and ResNet50-based models (\hyperref[fig:S2]{Appendix Figure S2 (3)}). Specifically, we use two of the default PanNuke dataset splits (66\%) for training and validation, and reserve the third split (33\%) for testing.

To address the challenge of cell class imbalance in the PanNuke dataset, which is a common characteristic in most cell-level H\&E patch datasets, both models’ training processes employ a weighted loss function comprising cross-entropy and focal loss \cite{Lin_Goyal_etal._2018}. The focal loss component is adjusted with coefficients derived from each cell class' instance frequency, emphasizing learning from underrepresented classes and enhancing the model's sensitivity to rare but significant cellular patterns. The cross-entropy loss is augmented with spectral decoupling regularization \cite{Pezeshki_Kaba_etal._2021,Pohjonen_Stürenberg_etal._2022} and spatially varying label smoothing \cite{Islam_Glocker_2021}, which potentially stabilizes training and improves generalization in case of complex tissue morphologies. For optimization, we employ the \textit{AdamW} \cite{Loshchilov_Hutter_2019} to counter unbalanced class scenarios, with cosine annealing learning rate scheduler.

We utilize the scikit-learn library \cite{Van_der_Walt_Schönberger_etal._2014} and HoVer-Net \cite{Graham_Vu_etal._2019} implementations of $R^2$ (the coefficient of determination) and $PQ$ (panoptic quality) to evaluate our experiments. Complete mathematical formulations and detailed explanations of these metrics are provided in \hyperref[chap:S5]{Appendix S5}. To compute confidence intervals, we use nonparametric bootstrapping, where after calculating the metric on the full sample, we generated 1000 bootstrap replicates by resampling with replacement and then determined the 95\% confidence intervals as the 2.5th and 97.5th percentiles of the resulting empirical distribution.

%\hfill

The model comparisons are summarized in \hyperref[tab:2]{Table 2}. The H‑Optimus-based model achieves higher $R^2$ across all cell classes compared to the ResNet50-based model, which means that its predictions are more closely aligned with the PanNuke cell counts, indicating a stronger correlation with the observed data. Notably, the improvement of $R^2_{dead}$ may be an indicator of better global contextual representations provided by the foundation model backbone. In terms of segmentation and classification quality combined, measured by the PQ score, the H‑Optimus-based model demonstrates notable improvements across most cell classes. Overall, the average $R^2$ improved from 0.575 to 0.871, while the average $PQ$ score improved from 0.450 to 0.492, demonstrating better performance of the H-Optimus-based model.

\begin{table}[h!]
\renewcommand{\arraystretch}{1.5}
  \centering
  \caption{Cell quantification metrics for baseline and proposed models (CI 95\%).}
  \label{tab:2}
  \begin{tabular}{|l|c|c|}
    \hline
    %\rowcolor{gray!30}
    Metric             & Resnet50-based            & H-optimus-based              \\
    \hline
    $R^2_{neoplastic}$    & 0.681 (0.576--0.769)       & \textbf{0.941 (0.917--0.960)} \\
    \hline
    $R^2_{inflammatory}$  & 0.863 (0.778--0.903)       & \textbf{0.949 (0.918--0.966)} \\
    \hline
    $R^2_{connective}$    & 0.600 (0.488--0.698)       & 0.609 (0.436--0.772)          \\
    \hline
    $R^2_{dead}$          & 0.097 (-11.389--0.669)     & 0.925 (0.404--0.982)          \\
    \hline
    $R^2_{epithelial}$    & 0.635 (0.490--0.747)       & \textbf{0.930 (0.886--0.964)} \\
    \hline
    $PQ_{neoplastic}$       & 0.517 (0.499--0.535)       & \textbf{0.589 (0.575--0.604)} \\
    \hline
    $PQ_{inflammatory}$     & 0.455 (0.429--0.482)       & \textbf{0.528 (0.507--0.549)} \\
    \hline
    $PQ_{connective}$       & 0.416 (0.400--0.431)       & \textbf{0.451 (0.436--0.465)} \\
    \hline
    $PQ_{dead}$             & 0.374 (0.342--0.408)       & 0.292 (0.209--0.365)          \\
    \hline
    $PQ_{epithelial}$       & 0.488 (0.460--0.519)       & \textbf{0.599 (0.579--0.618)} \\
    \hline
  \end{tabular}
\end{table}

Our results  show that integrating the H‑Optimus foundation model within the UNETR architecture enhances the model's ability to segment and classify cells across diverse tissues from PanNuke data. The pretrained transformer encoder provides robust feature representations, resulting in higher average $R^2$ and $PQ$ scores compared to the CNN-based model. This leads to more reliable cell quantification and more accurate downstream analysis. Additionally, the streamlined fine-tuning process reduces computational overhead and training time, making the model more adaptable for new data.

Despite these advancements, the foundation model-based approach does not fully resolve all challenges related to cell segmentation and classification. We observe lower metric scores for underrepresented classes in the training data. Furthermore, foundation models typically encompass billions of parameters, resulting in substantial computational and memory requirements. It therefore poses challenges for deployment in resource-constrained environments, limiting their practical applicability in certain clinical settings.

\section{Model optimization via Knowledge Distillation}

To address the limitations posed by the extensive size of foundation models, we implement knowledge distillation — a model compression technique that leverages the teacher-student paradigm \cite{Hinton_Vinyals_etal._2015}. By training a smaller, more efficient student model to replicate the output of a larger, pre-trained teacher model, we retain performance while significantly reducing the model's complexity and resource requirements (\hyperref[fig:fig6]{Figure 6}).

\begin{figure}[h!]
    \centering
    \includegraphics[width=\textwidth, height=0.45\textheight, keepaspectratio]{images/Figure_6.pdf}
    \caption{Knowledge distillation framework for training a student model using a pre-trained teacher}
    \label{fig:fig6}
\end{figure}

We employ knowledge distillation to compress the H‑Optimus-based teacher model into a more efficient student model. The teacher model is the modified UNETR architecture with the H‑Optimus foundation model described in the previous chapter. The student model is based on a UNet architecture augmented with residual connections and incorporates a smaller ViT encoder with 9 million parameters \cite{Steiner_Kolesnikov_etal._2022,Wightman_2019}. 

First, we fine-tune the teacher model using the refined dataset from the cross-relabeling procedure (Section 2). Initially we train the decoder of the teacher model while keeping the encoder weights frozen. We split the refined dataset into train (70\%), validation (20\%) and test (10\%) subsets (\hyperref[fig:S2]{Appendix Figure S2 (4)}). During fine-tuning, we use the train and validation subsets, while leaving the test subset for model evaluation. We set the training procedure and model hyperparameters to be identical to those that were used to demonstrate the utility of foundation models for the simultaneous cell segmentation and classification task.

Next, we perform knowledge distillation from teacher to student using the refined dataset used to fine-tune the teacher model. The student model is trained to replicate the teacher model's outputs. We utilize a specialized loss function that aligns the student's predicted probability distribution with the teacher's, incorporating the teacher's class probability distribution derived from the output. Following the methodology of Hinton et al. \cite{Hinton_Vinyals_etal._2015}, we experiment with various hyperparameter settings for the temperature ($T$) and the balancing coefficients ($\alpha$ and $\beta$) in the loss function. We vary $T$ from 1 to 20 and adjust $\alpha$ and $\beta$ to balance the distillation and student losses. Through iterative tuning and evaluation, we identify that setting $T=14$, $\alpha=0.3$, and $\beta=0.7$ yields a configuration that converges and closely approximates the teacher model's performance during training.

Finally, we assess the performance of both models using the $R^2$ and $PQ$ (defined in \hyperref[chap:S5]{Appendix S5}) on the test set of the refined dataset (\hyperref[tab:3]{Table 3}). We observe that the 95\% confidence intervals overlap for most cell types, so we cannot claim statistically significant performance differences between the teacher and student models. One exception appears in the neoplastic class. The teacher model produces an $R^2$ of 0.919, while the student model shows an $R^2$ of 0.852. In addition, the student model achieves higher $PQ$ values for the neoplastic and connective classes, though the confidence intervals show overlap.

\begin{table}[h!]
\renewcommand{\arraystretch}{1.5}
  \centering
  \caption{Cell quantification metrics for teacher and distilled student models (CI 95\%).}
  \label{tab:3}
  \begin{tabular}{|l|c|c|}
    \hline
    %\rowcolor{gray!30}
    Metric & Teacher & Student \\
    \hline
    $R^2_{neoplastic}$    & \textbf{0.919} (0.898--0.939) & 0.852 (0.800--0.891) \\
    \hline
    $R^2_{lymphocyte}$    & 0.969 (0.956--0.977)         & 0.969 (0.956--0.978) \\
    \hline
    $R^2_{connective}$    & 0.694 (0.548--0.809)         & 0.618 (0.469--0.741) \\
    \hline
    $R^2_{dead}$          & 0.755 (0.400--0.908)         & 0.424 (0.100--0.731) \\
    \hline
    $R^2_{epithelial}$    & 0.922 (0.870--0.958)         & 0.843 (0.738--0.917) \\
    \hline
    $R^2_{macrophage}$    & 0.384 (-0.369--0.724)        & 0.704 (0.352--0.859) \\
    \hline
    $R^2_{neutrofil}$     & 0.854 (0.578--0.929)         & 0.833 (0.502--0.925) \\
    \hline
    $PQ_{neoplastic}$       & 0.581 (0.569--0.593)         & 0.601 (0.588--0.613) \\
    \hline
    $PQ_{lymphocyte}$       & 0.536 (0.520--0.553)         & 0.563 (0.544--0.579) \\
    \hline
    $PQ_{connective}$       & 0.436 (0.421--0.451)         & 0.457 (0.441--0.474) \\
    \hline
    $PQ_{dead}$             & 0.272 (0.235--0.315)         & 0.279 (0.201--0.369) \\
    \hline
    $PQ_{epithelial}$       & 0.522 (0.500--0.545)         & 0.530 (0.506--0.555) \\
    \hline
    $PQ_{macrophage}$       & 0.524 (0.459--0.588)         & 0.474 (0.405--0.543) \\
    \hline
    $PQ_{neutrofil}$        & 0.541 (0.490--0.592)         & 0.565 (0.522--0.607) \\
    \hline
  \end{tabular}
\end{table}


We further decompose the $PQ$ metric into its $SQ$ and $DQ$ components (\hyperref[tab:S6]{Appendix Table S6}). Both models produce nearly identical $SQ$ values, which indicates that they predict instance boundaries with similar precision. Although the student model shows some improvement in $DQ$ scores for certain classes, the confidence intervals overlap and do not confirm a statistically significant difference.

We observe that the student and teacher models yield comparable detection performance despite the student model using a much smaller and simpler architecture. A model with fewer parameters reduces the risk of overfitting when training data are scarce relative to the model’s complexity \cite{Farias_Ludermir_etal._2022}. The knowledge distillation process also encourages the student model to focus on the most generalizable detection features learned from the teacher. These factors enable the student model to achieve similar detection performance across different cell types.

Additionally, considering the model sizes reported in \hyperref[tab:4]{Table 4}, the distilled model achieves a significant reduction compared to the teacher model, with a 48-fold decrease in parameter count and a 5.5-fold reduction in on-disk size. In inference mode, the teacher model requires 16 GB of VRAM for a batch size of 32, while the distilled model only needs 3 GB of VRAM for the same batch size. These reductions make the distilled model significantly more practical for fine-tuning and deployment in resource-constrained environments.

\begin{table}[h!]
\renewcommand{\arraystretch}{1.5}
  \centering
  \caption{Parameter counts and size of teacher and distilled model}
  \label{tab:4}
  \adjustbox{max width=\textwidth}{%
  \begin{tabular}{|l|c|c|c|}
    \hline
    %\rowcolor{gray!30}
    Metric & H-optimus-based (Teacher) & mobileViT-based (Student) & Magnitude of difference \\
    \hline
    Parameters count       & 1,158,917,906   & \textbf{24,093,393}   & \textbf{48x}  \\
    \hline
    Estimated Total Size (MB) & 87,912       & \textbf{15,935}    & \textbf{5.5x} \\
    \hline
  \end{tabular}%
}
\end{table}

%\hfill

With recent advancements in complex network architectures and the use of pretrained encoders to achieve state-of-the-art performance \cite{Baumann_Dislich_etal._2024,Hörst_Rempe_etal._2024} in cell segmentation and classification tasks, model size, computational complexity, and processing times have increased. This limits the scalability and accessibility of these models. As we demonstrate, this may be mitigated using knowledge distillation. Studies in the field of natural language processing have demonstrated the efficacy of knowledge distillation in retaining the capabilities of the teacher model while achieving significant reductions in size and complexity \cite{Huangpu_Gao_2024,Sun_Yu_etal.}. 

We demonstrate the feasibility of knowledge distillation in digital pathology, specifically for cell segmentation and classification tasks. Moreover, we achieve this performance while also significantly reducing the parameter count. In addressing the challenge of knowledge transfer, we found that distillation from a transformer-based model to a smaller transformer is more straightforward than attempting to map transformer features to CNN blocks. In our experiments, using a CNN-based network as a student results in worse cell quantification performance due to the structural constraints of CNN feature space dimensions. 

Although our primary approach relies on a transformer-based student model that performs well, it can be further optimized to incorporate advantages from CNN architectures. For example, employing alternative techniques such as using ViT adapters \cite{Chen_Duan_etal._2023} or $1 \times 1$ convolutions to adjust feature map sizes may be beneficial for harnessing CNN advantages like enhanced local feature extraction. Moreover, if additional performance improvements are desired, the process can be further enhanced by applying supplementary knowledge distillation techniques, such as self-distillation \cite{Zhang_Song_etal._2019} or online distillation \cite{Houyon_Cioppa_etal._2023}.

Despite these promising results, further validation on independent datasets is necessary to fully understand the model's limitations. Underrepresented classes may pose challenges when addressing complex cases. Pathologists need to validate these models to adopt them in clinical settings. While the distilled models are smaller and more deployable, a technological gap persists because pathologists traditionally rely on established methods for inspecting WSIs and diagnosing diseases. Addressing the complexities involved in deploying models for inference and supporting pathologists in adopting new tools is essential for integrating these models into clinical workflows.

\section{Model integration with QuPath}
Digital pathology tools with graphical user interfaces are essential for visualizing and analyzing WSIs. To make our student model useful in clinical pathology workflows, it needs to be integrated into a tool that enables inspecting regions, creating annotations, and providing quantitative analyses of biomarkers. Therefore, we integrate the trained student model from the previous chapter into the QuPath open‑source platform \cite{Bankhead_Loughrey_etal._2017}. QuPath provides the required annotation, visualization, and analysis tools to interpret complex histological data, including workflows for cell segmentation, classification, and quantification (\hyperref[fig:fig7]{Figure 7}). 

\begin{figure}[h!]
    \centering
    \includegraphics[width=\textwidth]{images/Figure_7.pdf}
    \caption{Visualization of model-generated cell quantification annotations (left) and the corresponding unannotated slide (right) in QuPath}
    \label{fig:fig7}
\end{figure}

To identify the regions in a WSI critical for prognosticating tumor development, such as specific tumor areas or border regions without overlapping healthy tissue, the pathologist uses QuPath to outline these regions. Then, the pathologist initiates a cell segmentation and classification script through the QuPath interface for the selected regions. The resulting annotations and quantified cell information are then directly overlaid onto the WSI in the QuPath interface. Additional design and implementation details are in \hyperref[chap:S7]{Appendix S7}. 

Two common approaches for integrating deep learning models into QuPath are Java‑based native QuPath extensions \cite{Goldsborough_Philps_etal._2024} and the execution of RESTful API requests to a model server coupled with handling the response via an extension, as demonstrated in the application of cell segmentation models applied to immunofluorescence images \cite{Sugawara_2023}. While the community is actively working on these integration strategies, there is currently no universal solution that fully addresses all integration and performance requirements.

Extensions may offer better integration with QuPath, allowing slightly improved performance and more widespread usage of the built-in QuPath models, but they lack the flexibility to customize models and modify their behavior. For example, the newest version of QuPath includes models such as StarDist \cite{Weigert_Schmidt} and InstanSeg \cite{Goldsborough_Philps_etal._2024} that can perform cell segmentation. Both models pose limitations when applied to simultaneous cell segmentation and classification. StarDist performs well only on convex, round shapes by design, whereas some neoplastic, inflammatory, and connective cells exhibit complex and non-convex shapes. InstanSeg provides only semantic segmentation without assigning classes to the segmented cells.

%\hfill

In contrast, our approach offers an alternative integration strategy. It utilizes the paquo library to directly interact with QuPath’s internal application programming interface from within Python. This enables data exchange and processing without the need for intermediate conversion steps and provides greater control over model customization, retraining, and the incorporation of custom processing steps.

The integration of our custom model with QuPath underscores its potential to significantly enhance the diagnostic process by reducing the time burden on pathologists and enabling them to focus on more complex interpretative tasks using familiar software. Leveraging a tool that is already well-established among pathologists increases the likelihood of its adoption into daily clinical workflows. The quantitative data generated through the automated workflow is critical for both clinical decision-making and research, facilitating more accurate biomarker analysis, enabling robust statistical evaluations, and supporting hypothesis generation and testing. Additionally, by streamlining cell segmentation and classification, the tool enhances the scalability and reproducibility of pathological assessments, ultimately contributing to improved diagnostic accuracy and patient outcomes.

\section{Conclusion and future work}

In this study, we address critical challenges in digital pathology and tackle the usability and deployment issues of the developed models in standard computing environments without the need for high-performance computing systems. Our multi-faceted approach encompasses data refinement through cross-relabeling, leveraging foundation models for robust cell segmentation and classification, optimizing model performance via knowledge distillation, and integrating the optimized model into the QuPath software for practical application. This approach is used to construct a capable, versatile, and adjustable model for cell segmentation and classification, with enhanced performance and usability.

\begin{sloppypar}
While our approach shows potential in the field of computational pathology, certain limitations persist. 
For example, our implementation currently exhibits lower performance in detecting macrophages. 
This serves as an instance of the broader challenge of accurately identifying complex cell types. In order to address this issue, extending our approach to incorporate additional data sources, exploring alternative modeling approaches, and integrating other imaging modalities such as immunohistochemical staining may help improve detection accuracy. Moreover, although the distilled model reduces computational demands, integrating advanced deep learning models into clinical practice requires addressing technological gaps and potential resistance to adopting new tools within established diagnostic processes.
\end{sloppypar}

Future work could focus on several key areas to refine the proposed approach and facilitate its adoption in clinical environments. Enhancing the cell-relabeling process with additional datasets \cite{Graham_Jahanifar_etal._2021} could improve the representation of underrepresented cell types and enhance overall model performance. Also, incorporating additional data sources, such as multi-modal imaging or complementary staining methods, may address limitations related to cell type differentiation and class imbalance. Exploring other foundation models \cite{Vorontsov_Bozkurt_etal._2024,Zimmermann_Vorontsov_etal._2024} or introducing additional modalities \cite{Ding_Wagner_etal._2024,Vaidya_Zhang_etal._2025} may provide alternative architectures better suited to specific tasks or offer improved efficiency. Implementing more complex knowledge distillation techniques \cite{Houyon_Cioppa_etal._2023,Zhang_Song_etal._2019} could further optimize the model's performance and adaptability. Additionally, deeper integration with QuPath or other digital pathology software could provide pathologists more control over cell quantification analysis directly within the QuPath interface, thereby increasing accessibility and usability. Such enhancements would not only refine model performance but also ensure greater adaptability and scalability within various clinical environments. Finally, extensive validation of the model by pathologists and benchmarking against independent datasets are essential steps toward establishing the model's reliability and fostering confidence in its clinical utility.

\section*{Acknowledgments} 
This work was funded in part by the Research Council of Norway grant no. 309439 SFI Visual Intelligence, and the North Norwegian Health Authority grant no. HNF1521-20.

\bibliographystyle{IEEEtran}
\begin{sloppypar}
\begin{thebibliography}{99}

\bibitem{chaplot2020neural} Chaplot, Devendra Singh, et al. "Neural topological slam for visual navigation." Proceedings of the IEEE/CVF conference on computer vision and pattern recognition. 2020.

\bibitem{maksymets2021thda} Maksymets, Oleksandr, et al. "Thda: Treasure hunt data augmentation for semantic navigation." Proceedings of the IEEE/CVF International Conference on Computer Vision. 2021.

\bibitem{mezghan2022memory} Mezghan, Lina, et al. "Memory-augmented reinforcement learning for image-goal navigation." 2022 IEEE/RSJ International Conference on Intelligent Robots and Systems (IROS). IEEE, 2022.

\bibitem{al2022zero} Al-Halah, Ziad, Santhosh Kumar Ramakrishnan, and Kristen Grauman. "Zero experience required: Plug \& play modular transfer learning for semantic visual navigation." Proceedings of the IEEE/CVF Conference on Computer Vision and Pattern Recognition. 2022.

\bibitem{ye2021auxiliary} Ye, Joel, et al. "Auxiliary tasks and exploration enable objectgoal navigation." Proceedings of the IEEE/CVF international conference on computer vision. 2021.

\bibitem{chaplot2020object} Chaplot, Devendra Singh, et al. "Object goal navigation using goal-oriented semantic exploration." Advances in Neural Information Processing Systems 33 (2020)

\bibitem{ramakrishnan2022poni} Ramakrishnan, Santhosh Kumar, et al. "Poni: Potential functions for objectgoal navigation with interaction-free learning." Proceedings of the IEEE/CVF Conference on Computer Vision and Pattern Recognition. 2022.

\bibitem{ramrakhya2022habitat} Ramrakhya, Ram, et al. "Habitat-web: Learning embodied object-search strategies from human demonstrations at scale." Proceedings of the IEEE/CVF Conference on Computer Vision and Pattern Recognition. 2022.

\bibitem{mousavian2019visual} Mousavian, Arsalan, et al. "Visual representations for semantic target driven navigation." 2019 International Conference on Robotics and Automation (ICRA). IEEE, 2019.

\bibitem{dhariwal2021diffusion} Dhariwal, Prafulla, and Alexander Nichol. "Diffusion models beat gans on image synthesis." Advances in neural information processing systems 34 (2021)

\bibitem{ho2022classifier} Ho, Jonathan, and Tim Salimans. "Classifier-free diffusion guidance." arXiv preprint arXiv:2207.12598 (2022).

\bibitem{nichol2021glide} Nichol, Alex, et al. "Glide: Towards photorealistic image generation and editing with text-guided diffusion models." arXiv preprint arXiv:2112.10741 (2021)

\bibitem{brooks2023instructpix2pix} Brooks, Tim, Aleksander Holynski, and Alexei A. Efros. "Instructpix2pix: Learning to follow image editing instructions." Proceedings of the IEEE/CVF Conference on Computer Vision and Pattern Recognition. 2023.

\bibitem{fu2023guiding} Fu, Tsu-Jui, et al. "Guiding instruction-based image editing via multimodal large language models." arXiv preprint arXiv:2309.17102 (2023).

\bibitem{geng2024instructdiffusion} Geng, Zigang, et al. "Instructdiffusion: A generalist modeling interface for vision tasks." Proceedings of the IEEE/CVF Conference on Computer Vision and Pattern Recognition. 2024.

\bibitem{zhou2024minedreamer} Zhou, Enshen, et al. "Minedreamer: Learning to follow instructions via chain-of-imagination for simulated-world control." arXiv preprint arXiv:2403.12037 (2024).

\bibitem{zhou2023esc} Zhou, Kaiwen, et al. "Esc: Exploration with soft commonsense constraints for zero-shot object navigation." International Conference on Machine Learning. PMLR, 2023.

\bibitem{yu2023l3mvn} Yu, Bangguo, Hamidreza Kasaei, and Ming Cao. "L3mvn: Leveraging large language models for visual target navigation." 2023 IEEE/RSJ International Conference on Intelligent Robots and Systems (IROS). IEEE, 2023.

\bibitem{gadre2023cows} Gadre, Samir Yitzhak, et al. "Cows on pasture: Baselines and benchmarks for language-driven zero-shot object navigation." Proceedings of the IEEE/CVF Conference on Computer Vision and Pattern Recognition. 2023.

\bibitem{shah2023navigation} Shah, Dhruv, et al. "Navigation with large language models: Semantic guesswork as a heuristic for planning." Conference on Robot Learning. PMLR, 2023.

\bibitem{cai2024bridging} Cai, Wenzhe, et al. "Bridging zero-shot object navigation and foundation models through pixel-guided navigation skill." 2024 IEEE International Conference on Robotics and Automation (ICRA). IEEE, 2024.

\bibitem{yu2023co} Yu, Bangguo, Hamidreza Kasaei, and Ming Cao. "Co-NavGPT: Multi-robot cooperative visual semantic navigation using large language models." arXiv preprint arXiv:2310.07937 (2023).

\bibitem{wu2024voronav} Wu, Pengying, et al. "Voronav: Voronoi-based zero-shot object navigation with large language model." arXiv preprint arXiv:2401.02695 (2024).

\bibitem{qin2023mp5} Qin, Yiran, et al. "Mp5: A multi-modal open-ended embodied system in minecraft via active perception." arXiv preprint arXiv:2312.07472 (2023).

\bibitem{du2024learning} Du, Yilun, et al. "Learning universal policies via text-guided video generation." Advances in Neural Information Processing Systems 36 (2024).

\bibitem{ajay2024compositional} Ajay, Anurag, et al. "Compositional foundation models for hierarchical planning." Advances in Neural Information Processing Systems 36 (2024).

\bibitem{liang2024skilldiffuser} Liang, Zhixuan, et al. "Skilldiffuser: Interpretable hierarchical planning via skill abstractions in diffusion-based task execution." Proceedings of the IEEE/CVF Conference on Computer Vision and Pattern Recognition. 2024.

\bibitem{heusel2017gans} Heusel, Martin, et al. "Gans trained by a two time-scale update rule converge to a local nash equilibrium." Advances in neural information processing systems 30 (2017).

\bibitem{zhang2018unreasonable} Zhang, Richard, et al. "The unreasonable effectiveness of deep features as a perceptual metric." Proceedings of the IEEE conference on computer vision and pattern recognition. 2018.

\bibitem{brown2020language} Brown, Tom B. "Language models are few-shot learners." arXiv preprint arXiv:2005.14165 (2020).

\bibitem{podell2023sdxl} Podell, Dustin, et al. "Sdxl: Improving latent diffusion models for high-resolution image synthesis." arXiv preprint arXiv:2307.01952 (2023).

\bibitem{brohan2022rt} Brohan, Anthony, et al. "Rt-1: Robotics transformer for real-world control at scale." arXiv preprint arXiv:2212.06817 (2022).

\bibitem{brohan2023rt} Brohan, Anthony, et al. "Rt-2: Vision-language-action models transfer web knowledge to robotic control." arXiv preprint arXiv:2307.15818 (2023).

\bibitem{li2024manipllm} Li, Xiaoqi, et al. "Manipllm: Embodied multimodal large language model for object-centric robotic manipulation." Proceedings of the IEEE/CVF Conference on Computer Vision and Pattern Recognition. 2024.

\bibitem{shah2023vint} Shah, Dhruv, et al. "ViNT: A foundation model for visual navigation." arXiv preprint arXiv:2306.14846 (2023).

\bibitem{liu2024visual} Liu, Haotian, et al. "Visual instruction tuning." Advances in neural information processing systems 36 (2024).

\bibitem{hu2021lora} Hu, Edward J., et al. "Lora: Low-rank adaptation of large language models." arXiv preprint arXiv:2106.09685 (2021).

\bibitem{qin2023supfusion} Qin, Yiran, et al. "SupFusion: Supervised LiDAR-camera fusion for 3D object detection." Proceedings of the IEEE/CVF International Conference on Computer Vision. 2023.

\bibitem{qin2024worldsimbench} Qin, Yiran, et al. "Worldsimbench: Towards video generation models as world simulators." arXiv preprint arXiv:2410.18072 (2024).

\bibitem{yu2025gamefactory} Yu, Jiwen, et al. "GameFactory: Creating New Games with Generative Interactive Videos." arXiv preprint arXiv:2501.08325 (2025).

\bibitem{zhou2024code} Zhou, Enshen, et al. "Code-as-Monitor: Constraint-aware Visual Programming for Reactive and Proactive Robotic Failure Detection." arXiv preprint arXiv:2412.04455 (2024).

\bibitem{zhang2024ad} Zhang, Zaibin, et al. "AD-H: Autonomous Driving with Hierarchical Agents." arXiv preprint arXiv:2406.03474 (2024).

\bibitem{wang2024toward} Wang, Chaoqun, et al. "Toward Accurate Camera-based 3D Object Detection via Cascade Depth Estimation and Calibration." arXiv preprint arXiv:2402.04883 (2024).

\bibitem{huang2024story3d} Huang, Yuzhou, et al. "Story3d-agent: Exploring 3d storytelling visualization with large language models." arXiv preprint arXiv:2408.11801 (2024).

\bibitem{savinov2018semi} Savinov, Nikolay, Alexey Dosovitskiy, and Vladlen Koltun. "Semi-parametric topological memory for navigation." arXiv preprint arXiv:1803.00653 (2018).

\bibitem{majumdar2022zson} Majumdar, Arjun, et al. "Zson: Zero-shot object-goal navigation using multimodal goal embeddings." Advances in Neural Information Processing Systems 35 (2022): 32340-32352.

\bibitem{yadav2023offline} Yadav, Karmesh, et al. "Offline visual representation learning for embodied navigation." Workshop on Reincarnating Reinforcement Learning at ICLR 2023. 2023.

\bibitem{yadav2023ovrl} Yadav, Karmesh, et al. "Ovrl-v2: A simple state-of-art baseline for imagenav and objectnav." arXiv preprint arXiv:2303.07798 (2023).

\bibitem{sun2024fgprompt} Sun, Xinyu, et al. "FGPrompt: fine-grained goal prompting for image-goal navigation." Advances in Neural Information Processing Systems 36 (2024).

\bibitem{zhu2017target} Zhu, Yuke, et al. "Target-driven visual navigation in indoor scenes using deep reinforcement learning." 2017 IEEE international conference on robotics and automation (ICRA). IEEE, 2017.

\bibitem{koh2024generating} Koh, Jing Yu, Daniel Fried, and Russ R. Salakhutdinov. "Generating images with multimodal language models." Advances in Neural Information Processing Systems 36 (2024).

\bibitem{krantz2022instance} Krantz, Jacob, et al. "Instance-specific image goal navigation: Training embodied agents to find object instances." arXiv preprint arXiv:2211.15876 (2022).

\bibitem{schulman2017proximal} Schulman, John, et al. "Proximal policy optimization algorithms." arXiv preprint arXiv:1707.06347 (2017).

\bibitem{anderson2018evaluation} Anderson, Peter, et al. "On evaluation of embodied navigation agents." arXiv preprint arXiv:1807.06757 (2018).

\bibitem{lin2024navcot} Lin, Bingqian, et al. "NavCoT: Boosting LLM-Based Vision-and-Language Navigation via Learning Disentangled Reasoning." arXiv preprint arXiv:2403.07376 (2024).

\bibitem{NavGPT} Zhou, Gengze, Yicong Hong, and Qi Wu. "Navgpt: Explicit reasoning in vision-and-language navigation with large language models." Proceedings of the AAAI Conference on Artificial Intelligence.

\bibitem{hahn2021no} Hahn, Meera, et al. "No rl, no simulation: Learning to navigate without navigating." Advances in Neural Information Processing Systems 34 (2021): 26661-26673.

\bibitem{li2025t2isafety} Li, Lijun, et al. "T2ISafety: Benchmark for Assessing Fairness, Toxicity, and Privacy in Image Generation." arXiv preprint arXiv:2501.12612 (2025).

\bibitem{an2024agfsync} An, Jingkun, et al. "AGFSync: Leveraging AI-Generated Feedback for Preference Optimization in Text-to-Image Generation." arXiv preprint arXiv:2403.13352 (2024).


\end{thebibliography}
\end{sloppypar}

\clearpage
\beginsupplement
\section*{Appendix}
\renewcommand{\thesubsection}{S\arabic{subsection}}

\subsection{\label{chap:S1}PanNuke and MoNuSAC preprocessing}
The PanNuke dataset comprises a set of 7,901 RGB patches, each with dimensions of $256 \times 256$ pixels, which we set as the standard patch size for our analysis. In contrast, the MoNuSAC dataset encompasses 294 images of heterogeneous dimensions. To standardize the MoNuSAC images with our experiments, we implement a standardization protocol. Specifically, for images exceeding the dimensions of $256 \times 256$ pixels, we segment them into equal-sized patches and apply mirror padding to the remaining portions to avoid information loss at the peripherals. Patches with dimensions less than $128 \times 128$ pixels are excluded from the dataset due to the insufficient resolution to capture relevant cellular details. For patches where either dimension falls between 128 and 256 pixels, we employ upsampling to achieve the standard patch size. As a result, we obtain a total of 2,823 RGB patches derived from the MoNuSAC dataset for subsequent analysis. For additional details on the MoNuSAC data preparation process, refer to the source code \cite{Shvetsov_2025a}.
\clearpage

\subsection{\label{chap:S2}Data usage for the methodology}

\counterwithin{figure}{subsection}
\renewcommand{\thefigure}{S\arabic{subsection}}

\begin{figure}[h!]
    \centering
    \includegraphics[width=\textwidth, height=0.85\textheight, keepaspectratio]{images/A2.pdf}
    \caption{Overview of the methodology for cross-labeling, dataset refinement, and model comparison. (1) Cross-relabeling - training and testing cell classification models, (2) Cross-relabeling - using cell classification models to create refined dataset, (3) Fine-tuning and training models for comparison, (4) Student knowledge distillation with refined dataset}
    \label{fig:S2}
\end{figure}
\clearpage

\subsection{\label{chap:S3}Confusion matrices for classification models}
\counterwithin{figure}{subsection}
\renewcommand{\thefigure}{S\arabic{subsection}.\arabic{figure}}

\begin{figure}[h!]
    \centering
    \includegraphics[width=\textwidth, height=0.4\textheight, keepaspectratio]{images/A3_1.pdf}
    \caption{Confusion matrix for PanNuke trained model}
    \label{fig:S3.1}
\end{figure}

\begin{figure}[h!]
    \centering
    \includegraphics[width=\textwidth, height=0.4\textheight, keepaspectratio]{images/A3_2.pdf}
    \caption{Confusion matrix for MoNuSAC trained model}
    \label{fig:S3.2}
\end{figure}

\clearpage

\subsection{\label{chap:S4}Datasets cell counts}

\counterwithin{table}{subsection}
\renewcommand{\thetable}{S\arabic{subsection}}

\begin{table}[h!]
\renewcommand{\arraystretch}{2.0}
\centering
\caption{\label{tab:S4}Cell counts for PanNuke, MoNuSAC and refined datasets. Numbers in parentheses indicate preprocessed cell counts for cell classifier models training and testing.}
%\adjustbox{max width=\textwidth}{%
\begin{tabular}{|l|c|c|c|}
\hline
%\rowcolor{gray!30}
Cell type & PanNuke & MoNuSAC & Refined \\
\hline
Neoplastic & 77,403 (68,031) & - & 105,451 \\
\hline
Epithelial & 26,572 (23,207) & - & 29,926 \\
\hline
Epithelial (benign and malignant) & - & 31,402 & - \\
\hline
Inflammatory & 32,276 & - & - \\
\hline
Lymphocytes & - & 37,045 (33,104) & 65,275 \\
\hline
Neutrophils & - & 1,355 (1,252) & 3,833 \\
\hline
Macrophage & - & 1,842 (1,695) & 3,410 \\
\hline
Dead & 2,908 & - & 2,908 \\
\hline
Connective & 50,585 & - & 50,585 \\
\hline
\end{tabular}
%
%}
\end{table}



\clearpage

\subsection{\label{chap:S5}Definition of validation metrics}
\counterwithin{equation}{subsection}
\renewcommand{\theequation}{\arabic{equation}}

\subsubsection{\label{chap:S5.1}R\textsuperscript{2}}
The coefficient of determination, denoted as $R^2$, is a statistical measure that represents the proportion of variance in the dependent variable that is predictable from the independent variables. In the context of cell quantification in pathology, $R^2$ is used to assess how well the predicted quantities of different cell types in a patch align with the actual quantities observed in the ground truth data, with higher values representing more accurate quantification. $R^2$ is defined as
\begin{equation*}
R^2 = 1 - \frac{\sum_{i=1}^n (y_i - \hat{y}_i)^2}{\sum_{i=1}^n (y_i - \bar{y})^2},
\end{equation*}
where $y_i$ represents the actual number of cells of a specific type in the $i$-th image, $\hat{y}_i$ represents the predicted number of cells of that type in the $i$-th image, $\bar{y}$ is the mean of the actual numbers across all images, and $n$ is the total number of images in the dataset.

The $R^2$ metric has a range of $(-\infty, 1]$. An $R^2$ of 1 indicates perfect prediction, where all predicted values exactly match the actual values. An $R^2$ of 0 suggests that the model explains none of the variability of the response data around its mean. If $R^2$ is negative, it indicates that the model performs worse than a model that simply predicts the mean of the actual values for all observations.

\subsubsection{\label{chap:S5.2}PQ}
Panoptic Quality ($PQ$) is a comprehensive metric used to evaluate the performance of segmentation models in tasks that require both instance segmentation and classification. $PQ$ provides a single score that encapsulates both the detection accuracy (i.e., how many objects were correctly identified) and the segmentation quality (i.e., how accurately the objects' boundaries were delineated). This metric is particularly useful in multiclass scenarios where each pixel is classified into distinct categories, such as different cell types in pathology images.

$PQ$ is calculated as the product of two terms: Detection Quality ($DQ$) and Segmentation Quality ($SQ$). It can be expressed as
\begin{equation*}
PQ = DQ \cdot SQ,
\end{equation*}
where
\begin{equation*}
DQ = \frac{TP}{TP + 0.5\, FP + 0.5\, FN},
\end{equation*}
\begin{equation*}
SQ = \frac{\sum_{(p, g) \in \mathcal{M}} IoU(p, g)}{TP}.
\end{equation*}
In these formulas, $TP$ denotes the number of correctly matched instances between ground truth and prediction, $FP$ denotes the predicted instances that have no corresponding ground truth, $FN$ denotes the ground truth instances that were not detected, $IoU(p, g)$ is the Intersection over Union for a pair of matched instances $p$ (prediction) and $g$ (ground truth), and $\mathcal{M}$ is the set of matched pairs.

The $PQ$ metric is calculated for each class and is averaged across classes to provide a global performance measure.

The $PQ$ score has a range of $[0, 1.0]$, where a higher score indicates better performance in both detecting and segmenting the instances correctly. A $PQ$ of 1 signifies perfect identification and segmentation of all instances, whereas a $PQ$ of 0 indicates that no instances were correctly identified and segmented.

\clearpage

\subsection{\label{chap:S6}Segmentation and Detection quality metrics for teacher and student models}

\begin{table}[h!]
\renewcommand{\arraystretch}{2.0}
\centering
\caption{Segmentation and detection quality for student and teacher models (CI 95\%)}
\label{tab:S6}
%\adjustbox{max width=\textwidth}{%
\begin{tabular}{|l|c|c|}
\hline
%\rowcolor{gray!30}
Metric & Teacher & Student \\
\hline
$SQ_{neoplastic}$ & 0.819 (0.815--0.823) & 0.824 (0.819--0.828) \\
\hline
$SQ_{lymphocyte}$ & 0.795 (0.788--0.802) & 0.790 (0.783--0.796) \\
\hline
$SQ_{connective}$ & 0.770 (0.762--0.776) & 0.780 (0.772--0.786) \\
\hline
$SQ_{dead}$ & 0.659 (0.623--0.688) & 0.657 (0.624--0.695) \\
\hline
$SQ_{epithelial}$ & 0.780 (0.770--0.790) & 0.788 (0.779--0.797) \\
\hline
$SQ_{macrophage}$ & 0.788 (0.760--0.810) & 0.757 (0.730--0.783) \\
\hline
$SQ_{neutrofil}$ & 0.782 (0.761--0.801) & 0.775 (0.759--0.792) \\
\hline
$DQ_{neoplastic}$ & 0.706 (0.692--0.719) & 0.727 (0.712--0.741) \\
\hline
$DQ_{lymphocyte}$ & 0.675 (0.656--0.698) & 0.713 (0.691--0.734) \\
\hline
$DQ_{connective}$ & 0.566 (0.546--0.584) & 0.583 (0.565--0.602) \\
\hline
$DQ_{dead}$ & 0.410 (0.361--0.465) & 0.435 (0.306--0.561) \\
\hline
$DQ_{epithelial}$ & 0.668 (0.639--0.694) & 0.673 (0.644--0.702) \\
\hline
$DQ_{macrophage}$ & 0.657 (0.583--0.727) & 0.615 (0.531--0.703) \\
\hline
$DQ_{neutrofil}$ & 0.691 (0.625--0.753) & 0.729 (0.679--0.778) \\
\hline
\end{tabular}
%
%}
\end{table}

\clearpage

\subsection{\label{chap:S7}QuPath integration method}
We adopt an integration strategy leveraging the paquo \cite{Bayer_AG} library, a Python package that enables direct interaction with QuPath’s internal API, thereby facilitating seamless data exchange without intermediate conversion steps. The data processing pipeline (\hyperref[fig:S7]{Appendix Figure S7}) begins with the acquisition of WSIs and their associated annotations from QuPath, which are represented as Shapely \cite{Gillies_Wel_etal._2024} polygons. Utilizing paquo, we directly read, create, and modify these annotations and detections within a QuPath project in the Python environment. Images are then cropped using these polygons and processed by cell segmentation and classification models employing standard vision processing toolkits such as OpenCV, pyvips, and PyTorch. Additionally, QuPath employs Groovy scripts to initiate a Python process that starts the entire pipeline from QuPath graphical interface: fetching polygons, extracting images from them, and running deep learning model inference on the cropped images. 
The results are returned to QuPath, leveraging paquo's Python bindings to manipulate QuPath data while minimizing the computational overhead typically associated with cross-environment communication.

\counterwithin{figure}{subsection}
\renewcommand{\thefigure}{S\arabic{subsection}}

\begin{figure}[h!]
    \centering
    \includegraphics[width=\textwidth]{images/A7.pdf}
    \caption{QuPath integration workflow using Python environment}
    \label{fig:S7}
\end{figure}

Compared to traditional workflows that involve exporting annotations as GeoJSON, classifying them in Python, and reimporting them into QuPath, our approach offers several advantages. We eliminate the need to switch between programming languages, providing a cohesive and streamlined development process entirely within QuPath software and removing the necessity to use other tools. Meanwhile, we avoid storing annotations as intermediate JSON files unless required for external use or archiving. By conducting the entire inference and post-processing workflow within the Python environment, we leverage the power and flexibility of Python libraries for image processing and machine learning. This approach also enables adjustments to any set of labels and models, thereby improving its applicability.

%\hfill

The distilled model and QuPath integration code are packaged into a Docker container, enabling streamlined execution with the Docker engine. Detailed integration code and deployment instructions can be found in the GitHub repository \cite{Shvetsov_2025b}.

Despite these benefits, we acknowledge that the paquo library is a proof‑of‑concept project in its early development stage and has not been tested across all versions of QuPath.

\clearpage

\subsection{\label{chap:S8}Data and code availability statement}
All datasets, models, and code used in this study are publicly available and can be obtained from the repositories listed below. 
The PanNuke \cite{Gamper_Koohbanani_etal._2019} and MoNuSAC \cite{Verma_Kumar_etal._2021} datasets are publicly accessible, and download information along with detailed descriptions can be found in their respective articles. Preprocessing scripts for PanNuke and MoNuSAC data, as well as individual cell extraction scripts, are available on GitHub \cite{Shvetsov_2025a}. The H-Optimus foundation model used in our experiments can be downloaded from the HuggingFace repository \cite{hoptimus2024}, and model information is available on GitHub \cite{Saillard_Jenatton_etal._2024}. In addition, the integration code for QuPath and the distilled model packaged in a Docker container are provided in the repository \cite{Shvetsov_2025b}, and paquo Python library is available from the authors GitHub repository \cite{Bayer_AG}.
\clearpage

\end{document}
  % 直接使用 bbl


%%% Comment out this section when you \bibliography{references} is enabled.


\appendix

\section{Dataset Generation Prompts}
\label{sec:dataset_generation_prompts}

We use the following prompts for dataset generation on each domain.
\begin{tcolorbox}
\small
\texttt{
user\_prompt = r'''Task: Gener
ate one precise editing request for the given LaTeX code, focusing exclusively on one detailed LaTeX-specific aspect.\\
\ \ \ \ \ 1. Analyze LaTeX Components: Examine the LaTeX code thoroughly, identifying elements such as commands, environments, packages, mathematical expressions, figures, tables, references, labels, and syntax structures.\\
\ \ \ \ \ 2. Target a Single LaTeX Issue: The editing request must address only one specific LaTeX-related issue such as commands, environments, packages, mathematical expressions, figures, tables, references, labels, and syntax structures.\\
\ \ \ \ \ 3. Clearly define the exact edit needed. The action should be definitive and unambiguous, avoiding any form of suggestion, optional language, or choices. Do not include reasons for the edit or any additional information beyond the request.\\
\ \ \ \ \ 4. Do not include reasons for the edit or any additional information beyond the edit request. The request should be a direct instruction.\\
\ \ \ \ \ The request examples are:\\
\ \ \ \ \ [Example 1]\\
\ \ \ \ \ <Edit Request>\\
\ \ \ \ \ Replace the \textbackslash begin\{equation\} ... \textbackslash end\{equation\} environment with a \textbackslash [ ...\textbackslash ] display math environment to present the equation.\\
\ \ \ \ \ </Edit Request>\\
\ \ \ \ \ [Example 2]\\
\ \ \ \ \ <Edit Request>\\
\ \ \ \ \ Remove the \textbackslash centering command inside the figure environment and insert \textbackslash centering immediately after \textbackslash begin\{figure\}.\\
\ \ \ \ \ </Edit Request>\\
\ \ \ \ \ [Example 3]\\
\ \ \ \ \ <Edit Request>\\
\ \ \ \ \ Change the citation command \textbackslash cite\{einstein\} to \textbackslash parencite\{einstein\} to display the citation in parentheses.\\
\ \ \ \ \ </Edit Request>\\
\ \ \ \ \ [Example 4]\\
\ \ \ \ \ <Edit Request>\\
\ \ \ \ \ Change the column specification in the tabular environment from \{l l l\} to \{l c r\} to adjust the alignment of the data columns.\\
\ \ \ \ \ </Edit Request>\\
\ \ \ \ \ [Example 5]\\
\ \ \ \ \ <Edit Request>\\
\ \ \ \ \ Replace the placeholder ??? in the reference text with \textbackslash ref\{sec:relwork\} to properly reference the “Related Work” section.\\
\ \ \ \ \ </Edit Request>\\
\ \ \ \ \ [Example 6]\\
\ \ \ \ \ <Edit Request>\\
\ \ \ \ \ Rename the macro \textbackslash vect to \textbackslash vecbold in both its definition and throughout the document.\\
\ \ \ \ \ </Edit Request>\\
}
\end{tcolorbox}
\begin{tcolorbox}
\small
\texttt{
\ \ \ \ \ [Example 7]\\
\ \ \ \ \ <Edit Request>\\
\ \ \ \ \ Add the optional width argument to \textbackslash includegraphics\{example-image\} as \textbackslash includegraphics[width=0.5\textbackslash textwidth] \\
\ \ \ \ \ \{example-image\} to scale the image.\\
\ \ \ \ \ </Edit Request>\\
\ \ \ \ \ [Example 8]\\
\ \ \ \ \ <Edit Request>\\
\ \ \ \ \ Remove the \textbackslash usepackage\{epsfig\} line and replace it with \textbackslash usepackage\{graphicx\} to handle graphics\\
\ \ \ \ \ </Edit Request>\\
\\
\ \ \ \ \ I will give you the content and then the editing request.\\ 
\ \ \ \ \ Please Edit the content based on the editing request. \\
\ \ \ \ \ While Editing, don't add other words like\\
\ \ \ \ \ modified or something. Just Edit directly. \\
\\
\ \ \ \ \ Content: \{original\_context\} \\
\ \ \ \ \ Editing Request: \{edit\_request\} \\
\ \ \ \ \ Please return the complete content after editing. \\
\ \ \ \ \ Don't skip the empty line and keep the original\\
\ \ \ \ \ apart from the editing part.
}
\end{tcolorbox}

We use the following prompt for doing G-eval.
% \FloatBarrier  % 防止浮动元素跨越 section
\begin{figure}[H]
    \centering
    \includegraphics[width=0.8\textwidth]{figure/figure2_v6.pdf}
    \vspace{-3mm}
    \label{fig:additional-edit-example}
\end{figure}

\section{Additional Implementation Details}
\label{sec:appx_implementation_detail}


\noindent \textbf{Chunking long context:}
Many large language models impose a fixed maximum token length \( L \) on their input (and sometimes output) sequences. Consequently, if the combination of \( T_{\text{orig}} \) and \( I_{\text{edit}} \) exceeds this limit, we divide the \( T_{\text{orig}} \) into smaller chunks of size \( \leq L \). Each chunk is then processed independently—paired with the same edit request and later concatenated to form the complete edited text. This approach ensures that every chunk fits within the model’s token budget, preventing overflow and reducing memory usage while preserving the overall structured editing behavior. 

\noindent \textbf{Fine-Tuning Strategy:}  
We use Low-Rank Adaptation (LoRA) \cite{hu2021lora} to efficiently adapt these models to our task, significantly reducing the number of trainable parameters while preserving their expressive power. In all LoRA configurations, We set the rank $r=8$ and scaling $\alpha=32$, and use a dropout probability of 0.05. For both Llama-based and Qwen-based models, we apply  LoRA to the attention's projection layers through trainable low-rank matrices. We used the AdamW optimizer with a learning rate of $2 \times 10^{-5}$, training for 2 epochs, and set the effective batch size of 1 with gradient accumulation steps of 4 due to device limits. This strategy not only reduces computational overhead but also enables rapid convergence on our structured editing tasks. Preliminary experiments guided the choice of hyperparameters across all three model variants. 

\noindent \textbf{Decoding and Inference:}  
During generation, we set the temperature to 0.7 and used top-p sampling with a probability of 0.9 to balance diversity and coherence. Greedy decoding is applied by default if without sampling setting. The final edited text is obtained by merging the edited outputs from all chunks.

% \section{Additional Data Example}

% We provide several additional qualitative results as displayed in Figure~\ref{fig:additional-edit-example}.

% \begin{figure}[!tbp]
%     \centering
%     \includegraphics[width=0.5\textwidth]{figure/Additional Figure.pdf}
%     \caption{FileEdit and Original LLMs differ in editing tasks, FileEdit can reduce LLM hallucinations and follow task instructions to complete tasks.}
%     \vspace{-3mm}
%     \label{fig:additional-edit-example}
% \end{figure} 





\end{document}
