\section{Related Works}
\label{sec:format}
Recent advancements in the RGB image domain have demonstrated the effectiveness of fully transformer-based pipelines for object tracking \cite{lin2022swintrack, cui2024mixformerv2}. These works clearly show that having a pretrained transformer-based backbone plays a crucial role in learning better feature representations for object tracking. In contrast, most previous methods for hyperspectral object tracking rely on CNN-based Siamese networks or hybrid networks that utilize CNNs for feature extraction \cite{MMF-Net, TransDAT}, and self-attention mechanisms of transformers are only used for feature fusion. Consequently, these approaches fail to leverage the full potential of transformers as backbones.

However, limited works that utilize transformers for hyperspectral object tracking \cite{TBR-Net, SPIRIT}. These methods often employ spectral dimensionality reduction techniques to reduce the number of bands to three, allowing pretrained RGB image weights of the transformer to be directly adopted. However, such methods do not fully exploit the rich spectral information inherent in hyperspectral data, and their performance primarily rely upon the effectiveness of the dimensionality reduction process.

In our approach, we address this limitation by inputting the full spectral data into the network, allowing it to adaptively learn salient spectral and spatial features in a learnable manner. This enables us to fully leverage the capabilities of transformers without sacrificing the spectral richness of hyperspectral images. Moreover, our framework adaptively learns across varying numbers of spectral bands, ensuring superior test performance even when some bands are missing during testing. 
\vspace{-0.1cm}