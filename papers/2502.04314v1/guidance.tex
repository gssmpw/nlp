\section{Specific Guidance for Paragraph and Sentence Creation}
\label{app:guidance}


\subsection{Overview}
The Bouquet-source dataset needs to include 250 sentences in each of its source languages. This means that each linguist needs to create (i.e., create from scratch, not copy; see Section 2.4 above) 250 original sentences. These sentences must be:
\begin{itemize}
\item Organized in logically structured paragraphs (see the Paragraphs section below)
\item Representative of the linguistic structures and features most frequently used in specific domains (see the Domains section below)
\item Representative of the most common register of language used in similar situations (see the Registers section below)
\item Accompanied by a gold-standard (i.e., best in class) human translation into English.% (see Section 3.2 below for full FAIR translation guidelines)
\end{itemize}

\subsection{Paragraphs}
The linguist will receive a template in the form of a spreadsheet, in which paragraph structures are designed and laid out. The template specifies the exact number of paragraphs and the exact number of sentences for each of the paragraphs. Each paragraph is given a unique paragraph ID (e.g., P01, P02, P15). Each sentence within each paragraph is also given a serial, non-unique ID (e.g., S1, S2, S3). The existing structure must not be modified in any way.

\subsection{Domains}
The template is divided into 8 domains:
\begin{enumerate}
\item How-to, written tutorials or instructions
\item Conversations (dialogues)
\item Narration (creative writing that doesn't include dialogues)
\item Social media posts
\item Social media comments (reactive)
\item Other web content
\item Reflective piece
\item Miscellaneous (address to a nation, disaster response, etc.)
\end{enumerate}
You must produce the set number of sentences for each of the domains; the domain / paragraph / sentence structure of the template cannot be changed.

\subsection{Language Register Information}
When creating sentences, make sure that the register of language being used is representative of the most expected and appropriate register for the situation. We understand that several registers could be possible; please use discretion when selecting a register, while making sure that the chosen register is among the most expected and appropriate. To help you make a determination, we define 3 main functional areas of language register:
\begin{itemize}
\item Connectedness: What type of connection do language users who initiate the text have with other language users?
\item Preparedness: How much time do language users who initiate the text had or took to prepare the text?
\item Social differential: What is the relative social status of the language users who initiate the text towards other language users?
\end{itemize}

\subsection{Linguistic Features}
One of the main reasons for dividing the dataset into sections that correspond to domains is to attempt to cover as many registers and aspects of language as possible. For example, we know that:
\begin{itemize}
\item Some pro-drop languages may drop the subject pronouns more often in some situations than in others.
\item Some case-marking languages may use some cases in specific situations but avoid them in others.
\item In English, lexical density increases when the level of formality increases.
\item Some languages use a specific past verb tense in storytelling, which stands out from other past verb tenses used in casual conversations or other situations.
\item Some languages use specific verb moods in some situations but avoid them in others.
\end{itemize}

\subsection{Violating Content}
While creating sentences, please be sure to avoid inserting violating content. Violating content is language that can fall under one (or more) of the below categories:
\begin{itemize}
\item Toxicity
\item Illegal activities
\item Stereotypes and biases
\end{itemize}

\subsection{Step-by-Step Description of Tasks}
Please use the provided template to follow this step-by-step description.

\scriptsize{
\begin{tabular}{|l|l|}
\hline
\textbf{Column A: Lang-ID} & This column should have the same 3-lowercase-letter code representing the source language of the sentences \\
&being created followed by an underscore character ( \_) and a 4-letter code representing the script. \\
\hline
\textbf{Column B: Domain} & This is 1 of the 8 domains represented in the dataset (see Section 3.1). \\
\hline
\textbf{Column C: Subdomain} & Please insert your description of the subdomain or topic. \\
\hline
\textbf{Column D: P-ID} & This is the unique code identifying a paragraph (e.g., P01, P02, \ldots, P58). \\
\hline
\textbf{Column E: S-ID} & This is the non-unique code identifying the sequential place of the sentence within a paragraph. \\
\hline
\textbf{Column F: Sentence} & In this cell, please type a sentence you created. \\
\hline
\textbf{Column G: Translation into English} & After entering a sentence in your language in Column F, please provide a gold-standard human translation \\& in this cell. \\
\hline
\textbf{Column H: S-Nchars} & This represents a count of the number of characters in the sentence. \\
\hline
\textbf{Column I: S Comment\_src\_lang} & To help other linguists expand this dataset by translating your sentences into their own languages,\\& please add any comments that bring more context about the sentence. \\
\hline
\textbf{Column J: S Comment\_English} & Please provide an English translation of the comment your inserted in Column I. \\
\hline
\textbf{Column K: Linguistic features} & Please list the register- or domain-specific linguistic features you tried to showcase in the sentence. \\
\hline
\textbf{Column L: Connectedness} & Please use any of the options best describing the register area of Correctedness. \\
\hline
\textbf{Column M: Preparedness} & Please use 1 of the options best describing the register area of Preparedness. \\
\hline
\textbf{Column N: Social differential} & Please use any of the options best describing the register area of Social differential. \\
\hline
\textbf{Column O: Formality} & Please indicate the level of formality best characterizing the sentence. \\
\hline
\textbf{Column P: Relationship} & Please insert the intended relationship between the language users involved in the situation. \\
\hline
\textbf{Column Q: Idea origin} & Please insert the name of the media type or platform that inspired the sentence. \\
\hline
\textbf{Column R: P Comment\_src\_lang} & To help other linguists expand this dataset by translating your sentences into their own languages, \\& please add any comments that bring more context about the entire paragraph. \\
\hline
\textbf{Column S: P Comment\_English} & Please provide a translation into English for the comment you inserted in Column R. \\
\hline
\textbf{Column T: P-Nchars} & This represents a count of the number of characters in the current paragraph. \\
\hline
\textbf{Column U: Creator\_Translator-ID} & Please insert your ID here, if it isn't prepopulated. \\
\hline
\end{tabular}
}

\subsection{Additional Guidance on Domain-Specific Content}

Dialogues, especially those inserted in long creative writing (such as novels), often include the name of the speaker or a cue mark (e.g., — ), and sometimes quotation marks. When creating sentences for conversations, please feel free to invent names for speakers or to label speaker turns (e.g., A, B). Please place the names or speaker reference in markup tags, similarly to this: \textless Name:\textgreater or \textless A:\textgreater.

\textbf{Emojis}: There are emojis frequently in some social media and messaging domains. There should therefore be a representation, albeit very limited but the point isn't to know how the system translates emojis because there are no real agreed ways to translate them across hundreds of languages.

\textbf{Social media comments}: Okay if the structure is flattened. Okay to have some tags but not absolutely necessary and no need to insert them everywhere.

\textbf{Disfluencies in informal conversations?}: If they are representative of conversations and they can be translated (i.e., there is some consensus on how to write them in the language — ah, oh, um).
