\documentclass[conference]{IEEEtran}
\IEEEoverridecommandlockouts

\usepackage{cite}
\usepackage{amsmath,amssymb,amsfonts}
\usepackage{algorithmic}
\usepackage{graphicx}
\usepackage{textcomp}
\usepackage{xcolor}
\usepackage{graphics} %
\usepackage{epsfig} %
\usepackage{mathptmx} %
\usepackage{times} %
\usepackage[misc,geometry]{ifsym}
\usepackage{multirow}
\usepackage{hyperref}
\def\BibTeX{{\rm B\kern-.05em{\sc i\kern-.025em b}\kern-.08em
    T\kern-.1667em\lower.7ex\hbox{E}\kern-.125emX}}
\begin{document}

\title{MedConv: Convolutions Beat Transformers on Long-Tailed Bone Density Prediction}


\author{Xuyin Qi$^{1,2,*,\dag}$,
Zeyu Zhang$^{1,3,*,\dag,\ddag}$, %
Huazhan Zheng$^{4,*}$, %
Mingxi Chen$^{5}$,  %
Numan Kutaiba$^6$, %
Ruth Lim$^{6,7}$, %
Cherie Chiang$^7$,\\ %
Zi En Tham$^8$, %
Xuan Ren$^{2}$, %
Wenxin Zhang$^{9}$, %
Lei Zhang$^{9}$, %
Hao Zhang$^{9}$, %
Wenbing Lv$^{10}$, %
Guangzhen Yao$^{11}$, %
Renda Han$^{12}$,\\ %
Kangsheng Wang$^{13}$, %
Mingyuan Li$^{14}$, %
Hongtao Mao$^{15}$, %
Yu Li$^{16}$, %
Zhibin Liao$^2$, %
Yang Zhao$^{8,\text{\Letter}}$, %
Minh-Son To$^1$ %
\thanks{$^{*}$Equal Contribution. $^{\text{\Letter}}$Corresponding author: \href{mailto:y.zhao2@latrobe.edu.au}{y.zhao2@latrobe.edu.au}}
\thanks{$^\dag$Work done while Zeyu Zhang is a researcher assistant at Flinders University.}
\thanks{$^\ddag$Project lead.}%
\thanks{$^1$Flinders University $^2$The University of Adelaide $^3$The Australian National University $^4$Zhejiang University of Technology $^5$Guangdong Technion – Israel Institute of Technology $^6$Austin Health $^{7}$The University of Melbourne $^{8}$La Trobe University $^9$University of Chinese Academy of Sciences $^{10}$Yunnan University $^{11}$Northeast Normal University $^{12}$Hainan University $^{13}$Univeristy of Science and Technology Beijing $^{14}$Hebei University of Technology $^{15}$Central China Normal University $^{16}$Hubei University}}





\maketitle

\begin{abstract}
Bone density prediction via CT scans to estimate T-scores is crucial, providing a more precise assessment of bone health compared to traditional methods like X-ray bone density tests, which lack spatial resolution and the ability to detect localized changes. However, CT-based prediction faces two major challenges: the high computational complexity of transformer-based architectures, which limits their deployment in portable and clinical settings, and the imbalanced, long-tailed distribution of real-world hospital data that skews predictions. To address these issues, we introduce \textbf{MedConv}, a convolutional model for bone density prediction that outperforms transformer models with lower computational demands. We also adapt Bal-CE loss and post-hoc logit adjustment to improve class balance. Extensive experiments on our AustinSpine dataset shows that our approach achieves up to \textbf{21\%} improvement in accuracy and \textbf{20\%} in ROC AUC over previous state-of-the-art methods.
Code will be available at \url{https://github.com/Richardqiyi/MedConv}.
\end{abstract}

\begin{IEEEkeywords}
Osteopenia, Osteoporosis, Long-Tailed Distributions, Bone Density, T-Score.
\end{IEEEkeywords}

\section{Introduction}
Bone health, crucial for mobility, fracture prevention, and overall well-being, is particularly important for aging populations or those with osteoporosis, a common skeletal disease that compromises bone strength by causing low bone mass and microarchitectural deterioration. This condition, increasing the risk of fragility fractures from low-energy impacts, often affects critical areas like the spine, hip, and wrist, significantly reducing quality of life ~\cite{lupsa2015bone}. Predicting bone density through CT scans to estimate T-scores offers a more precise and detailed assessment of bone health compared to traditional methods like X-ray bone density tests, which have lower spatial resolution and limited ability to detect localized bone changes. CT-based assessments can measure volumetric bone mineral density (BMD) and provide three-dimensional imaging, allowing for a comprehensive evaluation of bone quality. Studies have demonstrated that deep learning models applied to CT images can accurately predict BMD and T-scores, enhancing the detection and management of osteoporosis ~\cite{sato2022deep}. Additionally, quantitative computed tomography (QCT) has been shown to be a superior method for diagnosing osteoporosis and predicting fractures when compared to dual-energy X-ray absorptiometry (DXA) ~\cite{mori2024advancing}. These advancements highlight the potential of CT imaging in providing detailed insights into bone health, surpassing the capabilities of traditional X-ray-based methods. Recent advances in representation learning ~\cite{ji2024sine} and dense prediction \cite{zhang2023thinthick,wu2023bhsd,tan2024segstitch,ge2024esa,zhang2024meddet,cai2024msdet,tan2024segkan,zhang2025gamed}, particularly in the domain of medical imaging ~\cite{zhang2024jointvit, wu2024xlip,hiwase2025can,zhao2024landmark,cai2024medical,qi2025projectedex}, have significantly enhanced the accuracy and automation of osteoporosis detection. These advancements facilitate early diagnosis and timely intervention, providing a foundation for more effective personalized treatment and prevention strategies.

\begin{figure}[t]
    \centering
    \includegraphics[width=\columnwidth]{SAC_new.png}
    \caption{Visualization of segmentation results on CT images. 
    The first column shows the original images. 
    The second column represents the segmentation results from CTSpine1K~\cite{deng2021ctspine1k}. 
    The third column displays the segmentation results from TotalSegmentator~\cite{wasserthal2023totalsegmentator}. 
    Rows correspond to different anatomical planes: the sagittal plane (S) in the first row, the axial plane (A) in the second row, and the coronal plane (C) in the third row. 
    The region highlighted in red corresponds to the L5 vertebra, which plays a crucial role in diagnosing conditions like osteoporosis.}
    \label{fig:segmentation}
\end{figure}

However, predicting bone density from CT scans poses two significant challenges that hinder the effective application of advanced deep learning models.
First, transformer-based architectures, which have gained popularity in recent years for their superior performance in various domains, tend to suffer from quadratic complexity in their self-attention mechanisms. This results in substantial computational demands, particularly for high-resolution medical images like CT scans, where the input size can be extremely large. Such resource-intensive requirements make these models inefficient for deployment on portable or edge devices, which are increasingly sought after in modern healthcare for their potential to enable point-of-care diagnostics. Furthermore, the computational burden limits their feasibility in real-world clinical practice, where rapid processing and cost-effectiveness are critical.
Second, real-world hospital data often exhibits an imbalanced, long-tailed distribution, heavily skewed toward more prevalent cases of osteoporosis while containing significantly fewer samples of less common conditions, such as borderline or early-stage bone density anomalies. This data imbalance poses a considerable challenge for model training, as standard machine learning algorithms tend to prioritize the majority class, leading to suboptimal performance in predicting rare cases. Addressing this issue requires sophisticated techniques, such as class rebalancing strategies, data augmentation, or the use of domain-specific loss functions, to ensure that models can achieve robust and fair predictions across the full spectrum of cases ~\cite{johnson2019survey}.

To adress these problems, our paper presents three main contributions:

\begin{itemize}
    \item We introduce \textbf{MedConv}, a robust model that revisits convolutional approaches for bone density prediction on spinal CT scans, achieving superior performance over transformer-based models with reduced computational complexity.
    \item To address the long-tailed prediction challenge, we customize Bal-CE loss and post-hoc logit adjustment for improved class balance and accuracy.
    \item We evaluated our methods through extensive experiments on our AustinSpine dataset, applying various preprocessing techniques, which yielded improvements of up to 21\% in accuracy and 20\% in ROC AUC compared with previous state-of-the-art methods.
\end{itemize}

\section{Related Work}

\subsection{Deep Learning for Bone Mineral Density Prediction}

The prediction of bone mineral density (BMD) and the evaluation of fracture risk through the application of deep learning techniques \cite{zhang2024deep} have garnered increasing attention in recent years. A notable study by Hsieh et al. (2021) \cite{hsieh2021automated} introduced an innovative approach that utilizes deep learning models applied to plain radiographs for the automated prediction of BMD and fracture risk assessment. Their method demonstrated highly promising results, achieving area under precision-recall curve (AUPRC) scores of 0.89 for hip osteoporosis and 0.83 for spine osteoporosis prediction. Furthermore, their model exhibited an impressive accuracy of 91.7\% in estimating the risk of hip fractures. Leveraging a large dataset comprising pelvis and lumbar spine radiographs, the study underscored the potential of deep learning in addressing osteoporosis detection, particularly in scenarios where dual-energy X-ray absorptiometry (DXA) remains underutilized.

In another significant contribution, Yasaka et al. (2020) \cite{yasaka2020prediction} explored the use of CT imaging for BMD prediction, employing a convolutional neural network (CNN) specifically designed to estimate lumbar vertebrae BMD from unenhanced CT scans. Their findings demonstrated a strong correlation between CNN-predicted BMD values and those obtained through DXA, achieving area under the receiver operating characteristic curve (AUC) scores of 0.965 and 0.970 for internal and external validation datasets, respectively. This study laid the groundwork for using CT imaging as an effective alternative to DXA in BMD prediction, illustrating the capability of CNN-based models to accurately capture bone density-related features.

\begin{figure}[t]
    \centering
    \includegraphics[width=0.8\linewidth]{Comparison_of_models.png}
    \caption{
        Comparison between 3D ResNet and 2D ResNet architectures for volumetric medical data processing. 
        The upper pipeline illustrates the 3D ResNet-based MedConv model, which leverages three-dimensional convolutions to capture spatial and contextual information across volumetric CT scans. 
        The inclusion of Bal-CE Loss further refines the model's focus on imbalanced data distributions, ensuring accurate predictions for the L1 vertebra segmentation task.
        Conversely, the lower pipeline showcases the standard 2D ResNet approach, where slices are treated independently without spatial continuity across adjacent slices, potentially limiting performance in tasks requiring volumetric context. 
        This figure highlights the architectural and methodological differences, emphasizing the advantages of 3D ResNet for tasks that demand structural and contextual understanding of medical images.}
    \label{fig:comparison_models}
\end{figure}

Building upon this research, Dagan et al. (2019) \cite{dagan2020automated} developed a model aimed at predicting fracture risk based on routine CT scans, particularly when DXA-derived data is unavailable. Their CT-based method demonstrated superior AUC scores and sensitivity compared to the FRAX tool when BMD inputs were excluded, indicating that CT scans can serve as a reliable resource for assessing fracture risk. This approach suggests that CT imaging could effectively compensate for the underutilization of DXA in clinical settings.

In another noteworthy study, González et al. (2018) \cite{gonzalez2018deep} proposed a direct image-to-biomarker prediction approach. By employing a deep learning regression model, they predicted BMD directly from CT scans. Their results highlighted the effectiveness of a single convolutional neural network in simultaneously segmenting relevant anatomical regions and predicting BMD values with high accuracy. This streamlined approach provides an efficient alternative to traditional methods that rely on separate segmentation and prediction steps.

Lastly, Fang et al. (2020) \cite{fang2021opportunistic} demonstrated the potential of multi-detector CT imaging for opportunistic osteoporosis screening. By combining U-Net for vertebral segmentation with DenseNet-121 for BMD estimation, their method achieved a strong correlation with quantitative computed tomography (QCT) benchmarks. This fully automated pipeline showcased the feasibility of integrating CT-derived BMD analysis into routine clinical practice for opportunistic screening. Their study highlighted how deep learning can facilitate cost-effective and automated osteoporosis detection in diverse healthcare environments.


\subsection{Addressing Long-Tailed Distribution in Classification Tasks}

Long-tailed distributions, characterized by a few dominant classes and a large number of underrepresented classes, pose significant challenges in classification tasks. These challenges arise due to the imbalance in the data distribution, which can lead to biased model predictions favoring majority classes while neglecting minority ones. Two widely adopted strategies for addressing this issue are resampling methods and balanced augmentation (BalAug), both of which aim to mitigate the effects of data imbalance by adjusting the training process.

Resampling methods involve manipulating the class distribution in the training dataset. Oversampling techniques, such as random duplication or Synthetic Minority Over-sampling Technique (SMOTE), increase the representation of minority classes, thereby providing the model with more exposure to these underrepresented categories. However, these approaches may lead to overfitting on the minority classes due to repeated exposure to the same data points. On the other hand, undersampling methods reduce the number of majority class samples to balance the dataset, but this can result in a loss of valuable information from the majority classes, as noted in ~\cite{bellinger2020remix}. Consequently, while resampling methods are straightforward and often effective, they require careful tuning to avoid introducing new challenges.

Balanced augmentation (BalAug) offers an alternative approach by integrating data augmentation techniques with class balancing. Augmentation strategies such as rotation, cropping, flipping, and other transformations are selectively applied to the minority classes, enhancing the diversity of training data for these underrepresented categories. For instance, ~\cite{cui2019class} introduced a class-balanced loss that dynamically weights samples based on their effective number, ensuring that the model learns equitably from all classes. Furthermore, advanced techniques like class-aware sampling combined with augmentation, as proposed in ~\cite{liu2022long}, have demonstrated improved performance on long-tailed datasets by carefully balancing the sampling probabilities and incorporating diverse transformations. These methods not only enrich the training data but also help the model generalize better to unseen data.

In addition to data-focused strategies, training optimization methods have emerged as powerful tools for addressing long-tailed distributions. Foret et al. (2020) ~\cite{foret2020sharpness} introduced Sharpness-Aware Minimization (SAM), a novel optimization approach designed to enhance model generalization by simultaneously minimizing the loss value and the sharpness of the loss landscape. SAM identifies parameter regions with consistently low loss, effectively mitigating overfitting and improving generalization, particularly in overparameterized models. Through rigorous evaluation on benchmark datasets like CIFAR ~\cite{krizhevsky2009learning} and ImageNet ~\cite{deng2009imagenet}, SAM demonstrated superior performance, excelling in robustness to label noise and training stability, making it a valuable addition to the arsenal of techniques for long-tailed datasets.

Building on these ideas, Fang et al. (2023) ~\cite{defazio2024road} proposed a schedule-free optimization framework to address long-tailed distributions by replacing traditional learning rate schedules with momentum-driven primal averaging. Their approach dynamically balances gradient updates, avoiding the gradient collapse often observed in imbalanced datasets. This innovative method achieved state-of-the-art results across various tasks, including CIFAR-10 and ImageNet, by combining robust convergence properties with efficient generalization capabilities. By reducing reliance on extensive hyperparameter tuning, this approach offers a practical solution for training on long-tailed data.

In summary, addressing the challenges posed by long-tailed distributions typically requires a combination of data-level and training-level strategies. Data-level approaches, such as resampling and balanced augmentation, aim to correct the imbalance in the dataset, ensuring that all classes are adequately represented during training. Training-level techniques, like SAM and schedule-free optimization, focus on improving model generalization by optimizing the training process itself. When these methods are combined effectively, they can complement each other, leveraging the strengths of both data and training interventions to achieve robust and unbiased performance on long-tailed datasets.



\section{Methodology}

\subsection{Overview}

\begin{figure}[h]
    \centering
    \includegraphics[width=\columnwidth]{main-graph.png}
    \caption{Architecture of the proposed MedConv model, based on a 3D ResNet-50 backbone. 
    The model leverages the volumetric spatial representation capabilities of 3D convolutions, essential for accurate bone density estimation. 
    Key methodologies include the use of Balanced Cross-Entropy (Bal-CE) loss and post-hoc logit adjustment with hyperparameters $\tau_1 = 1$ and $\tau_2 = 0.5$, which enhance class balance and calibration.}
    \label{fig:main-graph}
\end{figure}

Our proposed model is built upon a 3D ResNet-50 backbone, selected for its superior ability to capture the spatial and contextual information embedded in volumetric medical data. Unlike conventional 2D convolutional neural networks (CNNs) that process individual image slices independently, thereby neglecting depth information, the 3D ResNet-50 employs three-dimensional convolutional operations. This design enables the model to effectively encode spatial continuity within volumetric datasets such as CT scans, a critical aspect for accurate bone density prediction.

The architecture leverages residual connections to address the vanishing gradient problem, facilitating the training of deep networks while maintaining representational efficiency. Additionally, the bottleneck structure within the 3D ResNet-50 reduces computational overhead without compromising its capacity to model complex patterns inherent in high-resolution medical images.

While transformer-based architectures excel in capturing long-range dependencies and global contextual features, their computational complexity grows quadratically with input size. This limitation poses significant challenges for processing high-resolution volumetric data in resource-constrained settings. By contrast, the 3D ResNet-50 achieves an effective trade-off between computational efficiency and representational power, making it a practical and scalable choice for clinical applications.

This backbone forms the foundation of our model, providing a framework that balances accuracy and efficiency for the analysis of volumetric medical data. Its ability to integrate three-dimensional spatial information ensures robust performance, particularly in tasks requiring detailed structural understanding, such as bone density prediction.


\subsection{Balanced Cross-Entropy (Bal-CE) Loss}

To address the inherent challenges of class imbalance in bone density prediction, we adopt a Balanced Cross-Entropy (Bal-CE) loss function. Medical imaging datasets often exhibit a long-tailed distribution, with underrepresented classes being critical for diagnosis. The Bal-CE loss function is designed to emphasize these minority classes by assigning class-specific weights, \( w_i \), during training. Its formulation remains as follows:

\[
\mathcal{L}_{\text{Bal-CE}} = - \frac{1}{N} \sum_{i=1}^{N} w_i \left( y_i \log(\hat{y}_i) + (1 - y_i) \log(1 - \hat{y}_i) \right),
\]

where \( y_i \) and \( \hat{y}_i \) represent the ground truth labels and predicted probabilities, respectively. The weight \( w_i \) is dynamically computed based on the inverse frequency of each class, ensuring greater emphasis on minority classes. This targeted adjustment helps the model avoid bias toward majority classes, leading to more balanced and reliable predictions.

\subsection{Post-Hoc Logit Adjustment}

To further enhance model calibration and refine class probabilities, we introduce a post-hoc logit adjustment technique. This method applies temperature scaling to logits, fine-tuning the relative contributions of majority and minority classes. The adjusted probabilities are calculated as:

\[
\hat{y}_i = \frac{e^{z_i / \tau_1}}{e^{z_i / \tau_1} + e^{z_j / \tau_2}},
\]

where \( z_i \) and \( z_j \) denote the logits for classes \( i \) and \( j \), respectively. The temperature parameters \( \tau_1 = 1 \) and \( \tau_2 = 0.5 \) are empirically chosen to achieve an effective balance. The lower value of \( \tau_2 \) amplifies the influence of minority class logits, while \( \tau_1 \) maintains the contribution of majority classes. This mechanism mitigates the impact of class imbalance by reshaping the probability distribution, allowing the model to produce well-calibrated predictions.

The combination of the Bal-CE loss and logit adjustment strategies ensures that our model effectively learns from imbalanced datasets while maintaining robustness in clinical scenarios. Together, these methods address the challenges posed by uneven class distributions and improve the reliability of the system for bone density prediction tasks.



\section{Dataset and Evaluation Matrices}
\subsection{AustinSpine Dataset}

\begin{figure}[h]
    \centering
    \includegraphics[width=0.5\linewidth]{dist.png}
    \caption{Long-tailed distribution of T-score classifications within the AustinSpine dataset.}
    \label{fig:dist}
\end{figure}

The AustinSpine dataset is a clinically curated collection of spinal CT scans, comprising imaging data from 389 patients, obtained with full ethical approval. Bone density for each scan is quantified using T-scores, a standardized metric widely employed to assess bone health. To ensure the reliability and consistency of annotations, each T-score underwent a thorough review by at least two expert radiologists, significantly enhancing the dataset's inter-rater reliability. Based on the World Health Organization (WHO) criteria for bone mineral density (BMD) \cite{who1994assessment}, the T-scores are categorized into three distinct classes, as detailed in Table \ref{tab:Tscore}. The dataset distribution, visualized in Figure \ref{fig:dist}, reveals a pronounced long-tailed pattern, highlighting the predominance of normal cases relative to the other classifications. This clinically enriched dataset offers a robust and reliable resource for the development and validation of automated bone density prediction models, particularly within real-world clinical settings where precise and consistent annotations are critical.

\begin{table}
\centering
\caption{World Health Organization (WHO) criteria for classification of patients with bone mineral density (BMD) ~\cite{who1994assessment}.}
\resizebox{\columnwidth}{!}{%
\begin{tabular}{c|c|c}
    \hline
    \textbf{T-score Range} & \textbf{Condition} & \textbf{Description} \\
    \hline
    -4 to -2.5 & Osteoporosis & Porous bone that can lead to fractures \\
    -2.5 to -1 & Osteopenia & Low Bone Density \\
    -1 and above & Normal & As compared to an average 30-year-old \\
    \hline
\end{tabular}%
}
\label{tab:Tscore}
\end{table}

\subsection{Evaluation Matrices}

For a fair comparison, we evaluated each method's overall classification performance on the test set using accuracy and ROC AUC scores. Additionally, we assessed sensitivity and specificity to understand how effectively each model handles both minority and majority classes within the long-tailed AustinSpine dataset.

\section{Experiment}

Our experiment is based on CT segmentation technology, utilizing two mainstream segmentation algorithms: CTSpine1K~\cite{deng2021ctspine1k} and TotalSegmentator \cite{wasserthal2023totalsegmentator}. CTSpine1K is a large-scale spinal CT dataset containing 1005 scans with over 11,100 labeled vertebrae, designed to advance research on spine-related image analysis tasks. TotalSegmentator is a deep learning segmentation model capable of automatically segmenting 104 major anatomical structures in CT images, including organs\cite{zhang2024segreg}, bones, muscles, and vessels, with robustness and high accuracy.

We use these algorithms to segment the lumbar vertebra L1 as input. The L1 vertebra, located at the top of the lumbar spine, serves as a critical load-bearing structure, supporting the upper body's weight while allowing flexibility and movement. Its position between the thoracic spine and the lower lumbar vertebrae makes it vital for both structural stability and mobility. Furthermore, the bone mineral density (BMD) of the L1 vertebra plays a crucial role in assessing overall bone health, serving as a key indicator in the diagnosis of osteoporosis and the evaluation of fracture risk ~\cite{ramschutz2024cervicothoracic}.

Through comparative experiments, we found that the segmentation results based on TotalSegmentator consistently outperformed those achieved by CTSpine1K in overall performance. Therefore, we selected the segmentation outputs of TotalSegmentator as the input for MedConv.

\begin{figure}[h]
    \centering
    \includegraphics[width=0.8\linewidth]{Comparison_of_segmentators.png}
    \caption{
        Experiment pipeline for evaluating segmentation methods and their impact on downstream tasks. 
        This flowchart illustrates the comparison between CTSpine1K~\cite{deng2021ctspine1k} and TotalSegmentator~\cite{wasserthal2023totalsegmentator}, two widely used segmentation algorithms. 
        Both methods segment the L1 vertebra from input CT images, with the outputs subsequently processed by the MedConv module, 
        followed by post-hoc logits optimized with balanced cross-entropy loss. 
        TotalSegmentator was identified as the superior model, producing more robust and accurate segmentation results, which were selected as inputs for the MedConv module.}
    \label{fig:experiment_flowchart}
\end{figure}

\subsection{Comparative Study}

\begin{table}[htbp]
\vspace{-0.5cm}
    \centering
    \caption{Comparative performance of various models on the given metrics.}
    \resizebox{\linewidth}{!}{%
    \begin{tabular}{l|c|c|c|c|c}
        \hline
        \textbf{Model} & \textbf{Accuracy} & \textbf{Sensitivity} & \textbf{Specificity} & \textbf{F1 Score} & \textbf{ROC AUC} \\
        \hline
        resnet10t.c3\_in1k+pretrain & 58.97 & 58.97 & 79.49 & 59.46 & 67.90 \\
        resnet14t.c3\_in1k+pretrain & 56.41 & 56.41 & 78.21 & 56.53 & 70.12 \\
        resnet18.a1\_in1k & 48.72 & 48.72 & 74.36 & 44.87 & 65.06 \\
        resnet18.a1\_in1k+windows & 47.44 & 47.44 & 73.72 & 44.81 & 63.63 \\
        resnet18.a1\_in1k+balaug & 47.44 & 47.44 & 73.72 & 42.75 & 64.67 \\
        resnet18.a1\_in1k+pretrain & 62.82 & 62.82 & 81.41 & 62.34 & 74.51 \\
        resnet18.a1\_in1k+pretrain+balce & 62.82 & 62.82 & 81.41 & 62.28 & 75.79 \\
        resnet18.a1\_in1k+pretrain+balaug & 57.69 & 57.69 & 78.85 & 55.47 & 75.02 \\
        resnet18.a1\_in1k+pretrain+windows & 56.41 & 56.41 & 78.21 & 54.92 & 69.53 \\
        resnet18.a1\_in1k+pretrain+balaug+windows & 58.97 & 58.97 & 79.49 & 58.64 & 77.71 \\
        resnet34.a1\_in1k & 48.72 & 48.72 & 74.36 & 46.77 & 64.47 \\
        resnet34.a1\_in1k+balaug & 46.15 & 46.15 & 73.08 & 46.41 & 66.15 \\
        resnet34.a1\_in1k+windows & 48.72 & 48.72 & 74.36 & 47.33 & 63.93 \\
        resnet34.a1\_in1k+pretrain & 58.97 & 58.97 & 79.49 & 57.45 & 83.01 \\
        resnet34.a1\_in1k+pretrain+balce & 57.69 & 57.69 & 78.85 & 56.54 & 74.51 \\
        resnet34.a1\_in1k+pretrain+balaug & 56.41 & 56.41 & 78.21 & 56.34 & 73.69 \\
        resnet50.a1\_in1k & 44.87 & 44.87 & 72.44 & 41.66 & 59.94 \\
        resnet50.a1\_in1k+pretrain & 57.69 & 57.69 & 78.85 & 55.89 & 76.87 \\
        resnet50.a1\_in1k+pretrain+balaug & 55.13 & 55.13 & 77.56 & 53.52 & 70.81 \\
        resnet50.a1\_in1k+pretrain+balce & 64.10 & 64.10 & 82.05 & 65.14 & 78.43 \\
        resnet50.a1\_in1k+pretrain+balce+schdulefree & 57.69 & 57.69 & 78.85 & 57.40 & 71.40 \\
        resnet50.a1\_in1k+pretrain+balce+balaug & 62.82 & 62.82 & 81.41 & 61.13 & 76.53 \\
        resnet50.a1\_in1k+pretrain+balce+resample & 61.54 & 61.54 & 80.77 & 60.61 & 73.25 \\
        resnet50.a1\_in1k+pretrain+sam & 55.13 & 55.13 & 77.56 & 54.36 & 73.10 \\
        resnet50.a1\_in1k+pretrain+balce+sam & 56.41 & 56.41 & 78.21 & 55.82 & 73.22 \\
        resnet50.a1\_in1k+trainParams+balaug & 53.85 & 53.85 & 76.92 & 53.13 & 74.58 \\
        mobilenetv2\_100.ra\_in1k+pretrain & 52.56 & 52.56 & 76.28 & 51.89 & 69.26 \\
        mobilenetv2\_100.ra\_in1k+pretrain+balce & 56.41 & 56.41 & 78.21 & 55.45 & 71.52 \\
        efficientnet\_b0.ra\_in1k+pretrain & 57.69 & 57.69 & 78.85 & 56.02 & 74.14 \\
        efficientnet\_b0.ra\_in1k+pretrain+balce & 60.26 & 60.26 & 80.13 & 58.73 & 78.06 \\
        resnext50\_32x4d.a1h\_in1k+pretrain & 60.26 & 60.26 & 80.13 & 60.77 & 76.06 \\
        resnext50\_32x4d.a1h\_in1k+pretrain+balce & 55.13 & 55.13 & 77.56 & 52.64 & 71.28 \\
        resnext50\_32x4d.a1h\_in1k+pretrain+balaug & 50.00 & 50.00 & 75.00 & 46.06 & 62.15 \\
        resnet101.a1\_in1k+pretrain & 55.13 & 55.13 & 77.56 & 54.78 & 73.30 \\
        resnet101.a1\_in1k+pretrain+balce & 58.97 & 58.97 & 79.49 & 55.27 & 73.30 \\
        resnet101.a1\_in1k+pretrain+balce+sam & 55.13 & 55.13 & 77.56 & 47.70 & 71.06 \\
        resnet152.tv\_in1k+pretrain & 50.00 & 50.00 & 75.00 & 47.91 & 69.06 \\
        resnet152.tv\_in1k+pretrain+balce & 57.69 & 57.69 & 78.85 & 58.30 & 70.76 \\
        resnet152.tv\_in1k+pretrain+balce+sam & 52.56 & 52.56 & 76.28 & 48.72 & 69.82 \\ 
        ViT+pretrain & 33.54 & 33.54 & 66.77 & 17.93 & 58.35 \\ 
        JointViT + pretrain & 41.03 & 41.03 & 76.92 & 53.85 & 60.78 \\
        JointViT +pretrain + balce & 43.59 & 43.59 & 71.79 & 34.60 & 56.81 \\ \hline
        \textbf{MedConv (Ours)} & \textbf{65.38} & \textbf{65.38} & \textbf{82.69} & \textbf{66.37} & \textbf{79.34} \\
        \hline
    \end{tabular}}
    \label{tab:comparative_results}
\end{table}

In this comparative experiment, we evaluated various models based on their performance metrics, including accuracy, sensitivity, specificity, F1 score, and ROC AUC. All models were tested using the segmentation outputs of TotalSegmentator, which were selected due to their superior performance in our preliminary ablation studies.

The results indicate that our proposed MedConv model achieved the highest accuracy of 65.38, surpassing other models such as resnet50.a1 in1k+pretrain+balce, which scored 64.10, and resnet34.a1 in1k+pretrain, which achieved an accuracy of 58.97. This demonstrates the effectiveness of MedConv in handling complex medical imaging data.

In terms of sensitivity and specificity, the MedConv model demonstrated remarkable results, with scores of 65.38 and 82.69, respectively. These metrics highlight MedConv's capability to accurately identify positive cases while minimizing the occurrence of false positives. In comparison, the next highest sensitivity was achieved by resnet50.a1 in1k+pretrain+balce with a score of 64.10, emphasizing MedConv's superior ability to distinguish true positives and true negatives with greater precision.

The F1 score for MedConv is 66.37, further establishing its robustness in balancing precision and recall. This metric is particularly critical in medical applications where both false positives and false negatives can significantly affect diagnostic reliability. MedConv's performance in this regard surpasses many other models, reinforcing its suitability for high-stakes scenarios where precise predictions are essential.

The ROC AUC for MedConv is 79.34, reflecting its overall performance across various classification thresholds. This metric is crucial for clinical applications, where decision-making often relies on evaluating a model's behavior across different thresholds. MedConv's high ROC AUC score highlights its reliability and effectiveness in real-world medical applications.

\begin{table}[htbp]
    \centering
    \caption{Comparative performance of CTspine1K and TotalSegmentator with different inputs.}
    \resizebox{\linewidth}{!}{%
    \begin{tabular}{l|l|c|c|c|c|c}
        \hline
        \textbf{Model} & \textbf{Input} & \textbf{Accuracy} & \textbf{Sensitivity} & \textbf{Specificity} & \textbf{F1 Score} & \textbf{ROC AUC} \\
        \hline
        \multirow{2}{*}{mobilenetv2\_100.ra\_in1k} 
            & CTspine1K        & 39.74 & 39.74 & 69.87 & 47.71 & 36.03 \\ 
            & TotalSegmentator & 52.56 & 52.56 & 76.28 & 51.89 & 69.26 \\ 
        \hline
        \multirow{2}{*}{resnet18.a1\_in1k} 
            & CTspine1K        & 46.15 & 46.15 & 73.08 & 41.30 & 53.43 \\ 
            & TotalSegmentator & 48.72 & 48.72 & 74.36 & 44.87 & 65.06 \\ 
        \hline
        \multirow{2}{*}{resnet34.a1\_in1k} 
            & CTspine1K        & 42.31 & 42.31 & 71.15 & 36.01 & 53.60 \\ 
            & TotalSegmentator & 48.72 & 48.72 & 74.36 & 46.77 & 64.47 \\ 
        \hline
        \multirow{2}{*}{resnet50.a1\_in1k} 
            & CTspine1K        & 44.87 & 44.87 & 72.44 & 37.97 & 59.32 \\ 
            & TotalSegmentator & 44.87 & 44.87 & 72.44 & 41.66 & 59.94 \\ 
        \hline
        \multirow{2}{*}{resnet101.a1\_in1k} 
            & CTspine1K        & 43.59 & 43.59 & 71.79 & 34.97 & 57.91 \\ 
            & TotalSegmentator & 55.13 & 55.13 & 77.56 & 54.78 & 73.30 \\ 
        \hline
        \multirow{2}{*}{resnet152.tv\_in1k} 
            & CTspine1K        & 42.31 & 42.31 & 71.15 & 36.10 & 54.29 \\ 
            & TotalSegmentator & 50.00 & 50.00 & 75.00 & 47.91 & 69.06 \\ 
        \hline
    \end{tabular}}
    \label{tab:comparative_results0}
\end{table}

To ensure the robustness of our experimental results, we performed additional evaluations using the same models but with segmentation outputs generated by CTspine1K instead of TotalSegmentator. CTspine1K, a specialized segmentation tool for spine imaging, serves as an alternative segmentation source. However, as shown in Table~\ref{tab:comparative_results0}, models consistently underperformed when using CTspine1K outputs compared to those using TotalSegmentator. Key metrics, including accuracy, sensitivity, specificity, F1 score, and ROC AUC, exhibited significant declines across all models.
Notably, the results demonstrate that TotalSegmentator’s high-quality and comprehensive segmentation is pivotal for achieving superior model performance. Furthermore, when TotalSegmentator outputs were used, model performance either improved or remained stable as model parameter counts increased. This trend highlights the richness of the information provided by TotalSegmentator, which facilitates more effective utilization of complex model architectures.
Based on these findings, we selected TotalSegmentator as the default segmentation input source for all subsequent experiments to ensure consistency and optimize the models' potential.

In summary, these results demonstrate that the MedConv model not only outperforms other tested alternatives but also represents a significant advancement in model architecture for medical imaging tasks. By leveraging the high-quality segmentation results from TotalSegmentator, MedConv has shown exceptional accuracy, sensitivity, specificity, and overall robustness. These findings underscore the potential of MedConv to enhance diagnostic accuracy and improve patient outcomes, making it a promising tool for clinical and medical research applications.

\subsection{Ablation Study}

This section presents a comprehensive analysis of the proposed approach through three separate experiments. The first experiment focuses on validating the effectiveness of BalCE loss across different backbone architectures, highlighting its role in addressing class imbalance and improving overall performance. The second experiment investigates the influence of the hyperparameter \(\tau_1\), which balances the loss contribution from positive and negative samples, on key performance metrics. This analysis aims to identify the optimal value of \(\tau_1\) for achieving stable and robust performance. Finally, the third experiment evaluates the sensitivity of the model to variations in the hyperparameter \(\tau_2\), which serves as an ad-hoc weighting parameter within the MedConv framework. These experiments collectively underscore the robustness, adaptability, and fine-tuning flexibility of the proposed method in addressing challenges associated with class imbalance in medical imaging.

\subsubsection{Impact of BalCE Loss on Different Backbones}

\begin{table}[h!]
\centering
\caption{Ablation Study: Comparison of BalCE Loss Across Different Backbones}
\resizebox{\columnwidth}{!}{
\begin{tabular}{l|c|c|c|c|c}
\hline
\textbf{Model}          & \textbf{Accuracy}       & \textbf{Sensitivity}    & \textbf{Specificity}    & \textbf{F1 Score}       & \textbf{ROC AUC}        \\ \hline
mobilenetv2 w/o         & 52.56                  & 52.56                  & 76.28                  & 51.89                  & 69.26                  \\
mobilenetv2 w/          & 56.41 \textcolor{green}{(+3.85)} & 56.41 \textcolor{green}{(+3.85)} & 78.21 \textcolor{green}{(+1.93)} & 55.45 \textcolor{green}{(+3.56)} & 71.52 \textcolor{green}{(+2.26)} \\\hline
efficientnet w/o        & 57.69                  & 57.69                  & 78.85                  & 56.02                  & 74.14                  \\
efficientnet w/         & 60.26 \textcolor{green}{(+2.57)} & 60.26 \textcolor{green}{(+2.57)} & 80.13 \textcolor{green}{(+1.28)} & 58.73 \textcolor{green}{(+2.71)} & 74.14                  \\\hline
resnet34 w/o            & 58.97                  & 58.97                  & 79.49                  & 57.45                  & 83.01                  \\
resnet34 w/             & 57.69 \textcolor{red}{(-1.28)} & 57.69 \textcolor{red}{(-1.28)} & 78.85 \textcolor{red}{(-0.64)} & 56.54 \textcolor{red}{(-0.91)} & 74.51 \textcolor{red}{(-8.50)} \\\hline
resnet50 w/o            & 57.69                  & 57.69                  & 78.85                  & 55.89                  & 76.87                  \\
\textbf{resnet50 w/}             & \textbf{64.10} \textcolor{green}{(+6.41)} & \textbf{64.10} \textcolor{green}{(+6.41)} & \textbf{82.05} \textcolor{green}{(+3.20)} & \textbf{65.14} \textcolor{green}{(+9.25)} & \textbf{78.43} \textcolor{green}{(+1.56)} \\\hline
resnet101 w/o           & 55.13                  & 55.13                  & 77.56                  & 54.78                  & 73.30                  \\
resnet101 w/            & 58.97 \textcolor{green}{(+3.84)} & 58.97 \textcolor{green}{(+3.84)} & 79.49 \textcolor{green}{(+1.93)} & 55.27 \textcolor{green}{(+0.49)} & 73.30 \textcolor{red}{(+0.00)} \\\hline
resnet152 w/o           & 50.00                  & 50.00                  & 75.00                  & 47.91                  & 69.06                  \\
resnet152 w/            & 57.69 \textcolor{green}{(+7.69)} & 57.69 \textcolor{green}{(+7.69)} & 78.85 \textcolor{green}{(+3.85)} & 58.30 \textcolor{green}{(+10.39)} & 70.76 \textcolor{green}{(+1.70)} \\\hline
\end{tabular}
}
\label{tab:ablation}
\end{table}

The results of integrating BalCE loss across different backbones are summarized in Table \ref{tab:ablation}. Consistent performance improvements are observed across most architectures, with significant gains in accuracy, sensitivity, and F1 score. Notably, ResNet50 exhibits the highest improvement, achieving a 6.41\% increase in accuracy and a 9.25\% increase in F1 score. These findings underscore the effectiveness of BalCE loss in addressing data imbalance, particularly in challenging medical imaging scenarios. However, a minor performance drop is noted in ResNet34, potentially due to overfitting or incompatibility between the backbone and loss function.

\subsubsection{Effect of \(\tau_1\) on Model Performance}

\begin{table}[h]
\centering
\caption{Ablation study of different \(\tau_1\) hyperparameter settings and their impact on model performance metrics.}
\resizebox{\columnwidth}{!}{
\begin{tabular}{c|c|c|c|c|c}
\hline
\textbf{$\tau_1$} & \textbf{Accuracy} & \textbf{Sensitivity} & \textbf{Specificity} & \textbf{F1} & \textbf{AUC} \\\hline
0     & 57.69   & 57.69   & 78.85   & 55.89   & 76.87   \\
0.25  & 58.97 {\color{green}(+1.28)} & 58.97 {\color{green}(+1.28)} & 79.49 {\color{green}(+0.64)} & 59.25 {\color{green}(+3.36)} & 75.42 {\color{red}(-1.45)} \\
0.5   & 57.69 {\color{green}(+0.00)} & 57.69 {\color{green}(+0.00)} & 78.85 {\color{green}(+0.00)} & 57.77 {\color{green}(+1.88)} & 78.33 {\color{green}(+1.46)} \\
0.65  & 58.97 {\color{green}(+1.28)} & 58.97 {\color{green}(+1.28)} & 79.49 {\color{green}(+0.64)} & 58.07 {\color{green}(+2.18)} & 70.69 {\color{red}(-6.18)} \\
0.75  & 61.54 {\color{green}(+3.85)} & 61.54 {\color{green}(+3.85)} & 80.77 {\color{green}(+1.92)} & 61.61 {\color{green}(+5.72)} & 77.12 {\color{green}(+0.25)} \\
0.85  & 61.54 {\color{green}(+3.85)} & 61.54 {\color{green}(+3.85)} & 80.77 {\color{green}(+1.92)} & 60.12 {\color{green}(+4.23)} & 74.11 {\color{red}(-2.76)} \\
0.9   & 58.97 {\color{green}(+1.28)} & 58.97 {\color{green}(+1.28)} & 79.49 {\color{green}(+0.64)} & 57.83 {\color{green}(+1.94)} & 74.38 {\color{red}(-2.49)} \\
0.92  & 52.56 {\color{red}(-5.13)} & 52.56 {\color{red}(-5.13)} & 76.28 {\color{red}(-2.57)} & 50.29 {\color{red}(-5.60)} & 70.88 {\color{red}(-5.99)} \\
0.95  & 61.54 {\color{green}(+3.85)} & 61.54 {\color{green}(+3.85)} & 80.77 {\color{green}(+1.92)} & 60.66 {\color{green}(+4.77)} & 74.46 {\color{red}(-2.41)} \\
0.96  & 56.41 {\color{red}(-1.28)} & 56.41 {\color{red}(-1.28)} & 78.21 {\color{red}(-0.64)} & 57.12 {\color{green}(+1.23)} & 73.40 {\color{red}(-3.47)} \\
0.97  & 57.69 {\color{green}(+0.00)} & 57.69 {\color{green}(+0.00)} & 78.85 {\color{green}(+0.00)} & 56.52 {\color{green}(+0.63)} & 69.30 {\color{red}(-7.57)} \\
0.99  & 53.85 {\color{red}(-3.84)} & 53.85 {\color{red}(-3.84)} & 76.92 {\color{red}(-1.93)} & 50.72 {\color{red}(-5.17)} & 70.09 {\color{red}(-6.78)} \\
0.995 & 58.97 {\color{green}(+1.28)} & 58.97 {\color{green}(+1.28)} & 79.49 {\color{green}(+0.64)} & 59.49 {\color{green}(+3.60)} & 72.61 {\color{red}(-4.26)} \\
0.999 & 60.26 {\color{green}(+2.57)} & 60.26 {\color{green}(+2.57)} & 80.13 {\color{green}(+1.28)} & 59.36 {\color{green}(+3.47)} & 78.43 {\color{green}(+1.56)} \\
\textbf{1}     & \textbf{64.10} {\color{green}(+6.41)} & \textbf{64.10} {\color{green}(+6.41)} & \textbf{82.05} {\color{green}(+3.20)} & \textbf{65.14} {\color{green}(+9.25)} & \textbf{78.43} {\color{green}(+1.56)} \\
1.1   & 57.69 {\color{green}(+0.00)} & 57.69 {\color{green}(+0.00)} & 78.85 {\color{green}(+0.00)} & 57.88 {\color{green}(+1.99)} & 74.98 {\color{red}(-1.89)} \\
1.5   & 56.41 {\color{red}(-1.28)} & 56.41 {\color{red}(-1.28)} & 78.21 {\color{red}(-0.64)} & 56.45 {\color{red}(-0.56)} & 75.76 {\color{red}(-1.11)} \\
2     & 52.56 {\color{red}(-5.13)} & 52.56 {\color{red}(-5.13)} & 76.28 {\color{red}(-2.57)} & 45.26 {\color{red}(-10.63)} & 74.73 {\color{red}(-2.14)} \\\hline
\end{tabular}
}
\label{tab:ablation_tau1}
\end{table}

The ablation study under the default condition of \(\tau_1 = \tau_2\) demonstrates that the model achieves its best performance when \(\tau_1 = 1\). This setting effectively balances the loss function, addressing the challenges posed by class imbalance and enhancing model robustness. The analysis confirms that \(\tau_1 = 1\) is the optimal choice for achieving a stable trade-off across performance metrics, providing a strong baseline for further exploration. Subsequently, additional ablations focus on varying \(\tau_2\) while keeping \(\tau_1\) fixed at its optimal value, allowing for a more comprehensive evaluation of the proposed approach.




\subsubsection{Effect of \(\tau_2\) on Model Performance}

We further evaluate the impact of varying the hyperparameter \(\tau_2\) on model performance. This experiment leverages segmentations generated by TotalSegmentator as input, exploring the sensitivity of key performance metrics to changes in \(\tau_2\).

\begin{table}[h]
\centering
\caption{Ablation study of different \(\tau_2\) hyperparameter settings and their impact on model performance metrics.}
\begin{tabular}{c|c|c|c|c|c}
\hline
\textbf{$\tau_2$} & \textbf{Accuracy} & \textbf{Sensitivity} & \textbf{Specificity} & \textbf{F1} & \textbf{AUC} \\\hline
1.0   & 0.6410 & 0.6410 & 0.8205 & 0.6514 & 0.7843 \\
0.9   & 0.6410 & 0.6410 & 0.8205 & 0.6514 & 0.7870 \\
0.8   & 0.6410 & 0.6410 & 0.8205 & 0.6514 & 0.7877 \\
0.7   & 0.6410 & 0.6410 & 0.8205 & 0.6514 & 0.7894 \\
0.6   & 0.6538 & 0.6538 & 0.8269 & 0.6637 & 0.7919 \\
\textbf{0.5}   & \textbf{0.6538} & \textbf{0.6538} & \textbf{0.8269} & \textbf{0.6637} & \textbf{0.7934} \\
0.4   & 0.6410 & 0.6410 & 0.8205 & 0.6493 & 0.7951 \\
0.3   & 0.6410 & 0.6410 & 0.8205 & 0.6493 & 0.7961 \\
0.2   & 0.6282 & 0.6282 & 0.8141 & 0.6359 & 0.7971 \\
0.1   & 0.6282 & 0.6282 & 0.8141 & 0.6337 & 0.7986 \\\hline
\end{tabular}
\label{tab:ablation_tau2}
\end{table}

\begin{figure}[h]
\centering
\includegraphics[width=\columnwidth]{tau2.png}
\caption{Ablation study showing the impact of different $\tau_2$ settings on model performance metrics. Each line represents a distinct metric: Accuracy, Sensitivity, Specificity, F1 Score, and AUC.}
\label{fig:ablation_tau2}
\end{figure}

As shown in Table \ref{tab:ablation_tau2} and Figure \ref{fig:ablation_tau2}, the model maintains stable performance for \(\tau_2\) values between 1.0 and 0.7, with accuracy, sensitivity, and specificity hovering around 64.10\%. A significant improvement is observed at \(\tau_2 = 0.6\) and \(\tau_2 = 0.5\), where accuracy rises to 65.38\% and F1 score reaches 66.37\%. This suggests that moderate \(\tau_2\) values balance precision and recall effectively.

When \(\tau_2\) is reduced further, a decline in performance becomes evident. At \(\tau_2 = 0.1\), accuracy drops to 62.82\%, with corresponding decreases in sensitivity and specificity. These findings highlight the importance of tuning \(\tau_2\) to achieve optimal results, emphasizing its role in improving model robustness and generalization.


\section{Conclusion}

In this study, we introduced MedConv, a convolutional neural network designed for bone density prediction via CT scans. MedConv outperforms transformer-based methods in accuracy, sensitivity, and specificity, while maintaining a significantly lower computational cost. By employing a 3D ResNet-50 backbone, the model effectively captures volumetric spatial information, which is critical for precise bone health assessment. This capability enables MedConv to be more suited for practical applications in clinical and resource-constrained settings compared to transformer models.

To address the inherent challenges of imbalanced and long-tailed datasets in real-world medical imaging, we adopted a Balanced Cross-Entropy (Bal-CE) loss function combined with post-hoc logit adjustment techniques. These strategies demonstrated robust improvements in classification accuracy and model calibration, as evidenced by the performance gains observed in our experiments on the AustinSpine dataset. Specifically, MedConv achieved a 21\% improvement in classification accuracy and a 20\% increase in ROC AUC compared to prior state-of-the-art methods, solidifying its position as a benchmark in this domain.

The ablation studies further emphasized the importance of hyperparameter tuning in optimizing model performance. For the logit adjustment hyperparameter $\tau_1$, the results indicate that the optimal setting of $\tau_1 = 1$ provides a balanced trade-off across various performance metrics, achieving the highest accuracy and F1 score. Similarly, an ad-hoc analysis of $\tau_2$ revealed that moderate values, particularly $\tau_2 = 0.5$, yielded significant performance gains. The model exhibited improved robustness and generalization at $\tau_2 = 0.5$, with accuracy increasing to 65.38\% and F1 score reaching 66.37\%. This suggests that $\tau_2$ plays a crucial role in calibrating the relative contributions of minority and majority classes, thereby enhancing overall performance.

Additionally, the study underscores the importance of high-quality segmentation tools such as TotalSegmentator, which played a pivotal role in enhancing the overall performance of MedConv. The segmentation outputs from TotalSegmentator provided superior input quality, enabling MedConv to better leverage the volumetric spatial information for accurate predictions.

MedConv’s success in balancing computational efficiency and predictive performance highlights its potential for broader applications in clinical settings, where timely and accurate diagnoses are imperative. Future work may explore extending MedConv to other imaging modalities and clinical tasks, as well as further refining its architecture to enhance versatility and scalability in diverse healthcare environments. By bridging the gap between advanced deep learning techniques and practical deployment, MedConv sets a promising foundation for improved diagnostic tools in the fight against osteoporosis and other bone health conditions.








































\documentclass[conference]{IEEEtran}
\IEEEoverridecommandlockouts
% The preceding line is only needed to identify funding in the first footnote. If that is unneeded, please comment it out.
%Template version as of 6/27/2024

\usepackage{cite}
\usepackage{amsmath,amssymb,amsfonts}
\usepackage{algorithmic}
\usepackage{graphicx}
\usepackage{textcomp}
\usepackage{xcolor}
\usepackage{multicol}
\usepackage{array, multirow}
% \usepackage[font=small, skip=5pt]{caption}
% \captionsetup{belowskip=2pt} % Set the space below captions to 2pt
\setlength{\tabcolsep}{9pt}
\renewcommand{\arraystretch}{1.3}
\def\BibTeX{{\rm B\kern-.05em{\sc i\kern-.025em b}\kern-.08em
    T\kern-.1667em\lower.7ex\hbox{E}\kern-.125emX}}
\begin{document}

\title{AAD-DCE: An Aggregated Multimodal Attention Mechanism for Early and Late Dynamic Contrast Enhanced Prostate MRI Synthesis.\\
% {\footnotesize \textsuperscript{*}Note: Sub-titles are not captured for https://ieeexplore.ieee.org  and
% should not be used}
% \thanks{Identify applicable funding agency here. If none, delete this.}
}

\author{\IEEEauthorblockN{Divya Bharti$^1{*}\quad$ Sriprabha Ramanarayanan$^{1,2}\quad$ Sadhana S$^1\quad$ Kishore Kumar M$^1\quad$ Keerthi Ram$^2\quad$ \\
Harsh Agarwal$^3\quad$ Ramesh Venkatesan$^3\quad$ Mohanasankar Sivaprakasam$^{1,2}\quad$}
\and 
\hspace{50mm}\textit{$^1$Indian Institute of Technology Madras (IITM), India}\\
\hspace{50mm}\textit{$^2$Healthcare Technology Innovation Centre (HTIC), India}\\
\hspace{50mm}\textit{$^3$GE HealthCare (GE), India}\\

\hspace{50mm}{\tt\small${*}$b13.divya@gmail.com}}


% \author{\IEEEauthorblockN{Author1$^1{*}\quad$ Author2 $^{1}\quad$ Author3$^1\quad$ Author4$^1\quad$ Author5$^2\quad$ \\
% Author6$^3\quad$ Author7$^3\quad$ Author8$^{1,2}\quad$}
% \and 
% \hspace{50mm}\textit{$^1$dept}\\
% \hspace{50mm}\textit{$^2$dept}\\
% \hspace{50mm}\textit{$^3$dept}\\}

% \hspace{50mm}{\tt\small${*}$b13.divya@gmail.com}}



% [12pt]
% $^1$Indian Institute of Technology Madras (IITM),
% India\\[4pt]
% $^2$Healthcare Technology Innovation Center (HTIC),
% India\\[4pt]
% $^3$GE HealthCare (GE), India \\ [4pt]

% \IEEEauthorblockA{\textit{Indian Institute of Technology, Madras} \\
% \textit{Indian Institute of Technology, Madras}\\
% City, Country \\
% email address or ORCID}
% \and
% \IEEEauthorblockN{Sriprabha Ramanarayanan$^{1}$}
% % \IEEEauthorblockA{\textit{dept. name of organization (of Aff.)} \\
% % \textit{name of organization (of Aff.)}\\
% % City, Country \\
% % email address or ORCID}
% \and
% \IEEEauthorblockN{Sadhana S$^1$}
% % \IEEEauthorblockA{\textit{dept. name of organization (of Aff.)} \\
% % \textit{name of organization (of Aff.)}\\
% % City, Country \\
% % email address or ORCID}
% \and
% \IEEEauthorblockN{Kishore Kumar M$^{1}$}
% \IEEEauthorblockA{\textit{dept. name of organization (of Aff.)} \\
% \textit{name of organization (of Aff.)}\\
% City, Country \\
% email address or ORCID}
% \and
% \IEEEauthorblockN{Keerthi Ram$^2$}
% % \IEEEauthorblockA{\textit{dept. name of organization (of Aff.)} \\
% % \textit{name of organization (of Aff.)}\\
% % City, Country \\
% % email address or ORCID}
% \and
% \IEEEauthorblockN{Harsh Agarwal$^{3}$}
% % \IEEEauthorblockA{\textit{dept. name of organization (of Aff.)} \\
% % \textit{name of organization (of Aff.)}\\
% % City, Country \\
% % email address or ORCID}
% \and 
% \IEEEauthorblockN{Ramesh Venkatesan$^{3}$}
% \and
% \IEEEauthorblockN{Mohanasankar Sivaprakasam$^{1,2}$}



\maketitle

\begin{abstract}
Dynamic Contrast-Enhanced Magnetic Resonance Imaging (DCE-MRI) is a medical imaging technique that plays a crucial role in the detailed visualization and identification of tissue perfusion in abnormal lesions and radiological suggestions for biopsy. However, DCE-MRI involves the administration of a Gadolinium-based (Gad) contrast agent, which is associated with a risk of toxicity in the body. Previous deep learning approaches that synthesize DCE-MR images employ unimodal non-contrast or low-dose contrast MRI images lacking focus on the local perfusion information within the anatomy of interest. We propose AAD-DCE, a generative adversarial network (GAN) with an aggregated attention discriminator module consisting of global and local discriminators. The discriminators provide a spatial embedded attention map to drive the generator to synthesize early and late response DCE-MRI images. Our method employs multimodal inputs - T2 weighted (T2W), Apparent Diffusion Coefficient (ADC), and T1 pre-contrast for image synthesis. Extensive comparative and ablation studies on the ProstateX dataset show that our model (i) is agnostic to various generator benchmarks and (ii) outperforms other DCE-MRI synthesis approaches with improvement margins of +0.64 dB PSNR, +0.0518 SSIM, -0.015 MAE for early response and +0.1 dB PSNR, +0.0424 SSIM, -0.021 MAE for late response, and (ii) emphasize the importance of attention ensembling. Our code is available at https://github.com/bhartidivya/AAD-DCE.
\end{abstract}

\begin{IEEEkeywords}
DCE-MRI, Aggregated Attention, Multimodal, Prostate Cancer
\end{IEEEkeywords}
\vspace{-3mm}
\section{Introduction}
Dynamic contrast-enhanced Magnetic Resonance Imaging (DCE-MRI) is a medical image scanning technique that quantifies tumor vasculature and perfusion characteristics \cite{b1}. The angiogenesis is captured by injecting a Gadolinium (Gad)-based contrast agent. This contrast agent exhibits a notable characteristic of rapid wash-in and wash-out kinetics occurring within a few seconds of Gad injection in the suspicious tissues, distinguishing it from normal healthy tissues. DCE-MRI plays a crucial role in prostate imaging, where the radiologists use a PI-RADS\footnote{https://www.acr.org/-/media/ACR/Files/RADS/Pi-RADS/PIRADS-v2-1.pdf} score (in the scale 1 to 5) to assess the cancer severity by visualizing non-contrast images (T2 or diffusion-weighted images (DWI)). A score of 3 suggests DCE-MRI acquisition, which helps to reduce unnecessary biopsies by 25\% \cite{b10}, avoiding over-diagnosis of insignificant cancers. However, Gad-based contrast agents cause patient discomfort and other contraindications like Nephrogenic Systemic Fibrosis \cite{b2}. Therefore, there is a need to decrease the dosage or refrain from using a Gad-based contrast agent in DCE-MRI.
% Furthermore, in prostate MRI, radiologists use PI-RADS\footnote{https://www.acr.org/-/media/ACR/Files/RADS/Pi-RADS/PIRADS-v2-1.pdf} scores (in the scale of 1-5) by visualizing non-contrast images (T2 or diffusion weighted images) to assess prostate cancer severity.
% \setlength{\belowcaptionskip}{-5mm}
% \begin{figure}[t]
% \vspace{-2mm}
%     \centering
%     \includegraphics[width=0.9\linewidth]{new_block_fig.pdf}
%     \caption{(a) Concept diagram of DCE-GAN. The multimodal non-contrast inputs, including Apparent Diffusion Coefficient (ADC), with embedded attention map, synthesize early and late response DCE-MR images. (b) Previous DCE-MRI synthesis approaches \cite{b6},\cite{b7} are based on adversarial training objectives lacking focus on important regions on the anatomy of interest. The proposed approach involves a discriminator-based global and local aggregated attention mechanism, further driving the generator.}
%     \label{fig:block diagram}
% \end{figure}

% \setlength{\belowcaptionskip}{-5mm}
% \begin{figure}[t]
% \vspace{-2mm}
%     \centering
%     \includegraphics[width=1\linewidth]{new_block_fig (1).pdf}
%     \caption{(a) Concept diagram of AAD-DCE. Multimodal non-contrast inputs, with aggregated attention maps, synthesize early and late response DCE-MRI. (b) Previous DCE-MRI synthesis methods use adversarial training without focusing on key anatomical regions. The proposed approach integrates a discriminator-based global and local attention mechanism to enhance the generator's performance.}
%     \label{fig:block diagram}
% \end{figure}

\setlength{\belowcaptionskip}{-5mm}
\begin{figure}[t]
\vspace{-2mm}
    \centering
    \includegraphics[width=0.9\linewidth]{new_block_fig_final.pdf}
    \caption{(a) Concept diagram of AAD-DCE. Multimodal non-contrast inputs, with aggregated attention maps, synthesize early and late response DCE-MRI. (b) Previous DCE-MRI synthesis methods use adversarial training without focusing on key anatomical regions. The proposed approach integrates a discriminator-based global and local attention mechanism to enhance the generator's performance.}
    \label{fig:block diagram}
\end{figure}


% Concept diagram of DCE-GAN. The multimodal non-contrast inputs, including Apparent Diffusion Coefficient (ADC), with embedded attention map, synthesize early and late response DCE-MR images. (b) Previous DCE-MRI synthesis approaches \cite{b6},\cite{b7} are based on adversarial training objectives lacking focus on important regions on the anatomy of interest. The proposed approach involves a discriminator-based global and local aggregated attention mechanism, further driving the generator.



% DCE-MRI plays a crucial role in prostate imaging, where the radiologists use a PI-RADS\footnote{https://www.acr.org/-/media/ACR/Files/RADS/Pi-RADS/PIRADS-v2-1.pdf} score (in the scale 1 to 5) to assess the cancer severity by visualizing non-contrast images (T2 or diffusion-weighted images (DWI)). A score of 3 suggests DCE-MRI acquisition, which helps to reduce unnecessary biopsies by 25\% \cite{b10}, avoiding over-diagnosis of insignificant cancers. 
% DCE-MRI is crucial when the score is 3, reducing unnecessary biopsies by 25\% \cite{b10}, avoiding over-diagnosis of insignificant cancers. 
% However, Gad-based contrast agents cause patient discomfort and other contraindications like Nephrogenic Systemic Fibrosis \cite{b2}. Therefore, there is a need to decrease the dosage or refrain from using a Gad-based contrast agent in DCE-MRI.

% DCE-MRI plays a crucial role in prostate imaging, where the radiologists use a PI-RADS\footnote{https://www.acr.org/-/media/ACR/Files/RADS/Pi-RADS/PIRADS-v2-1.pdf}





% For instance, in prostate DCE-MRI, a careful analysis of ProstateX dataset shows that of the patient data is assessed with an intermediate  score of 3.0 by the radiologist and DCE-MRI is utilized to upgrade the score from 3 to 4 or 5, affirming that DCE-MRI is an essential modality for diagnostic decision making.
% However, Gad-based contrast agents are expensive and cause patient discomfort and other contraindications like Nephrogenic Systemic Fibrosis \cite{b2}. Therefore, there is a need to decrease the dosage or refrain from using a Gadolinium-based contrast agent in DCE-MRI.
% \vspace{-5mm}

% \begin{figure}
% \vspace{-2mm}
%     \centering
%     \includegraphics[width=0.9\linewidth]{new_block_fig.pdf}
%     \caption{(a) Concept diagram of DCE-GAN. The multimodal non-contrast inputs, including Apparent Diffusion Constant (ADC), with aggregated attention map, synthesize early and later response DCE-MR images. (b) Previous DCE-MRI synthesis approaches \cite{b6},\cite{b7} are based on adversarial training objectives lacking focus on important regions on the anatomy of interest. The proposed approach involves discriminator-based global and local aggregated attention mechanism, further driving the generator.}
%     \label{fig:block diagram}
% \end{figure}

% \begin{figure}[t]
% \vspace{-2mm}
%     \centering
%     \includegraphics[width=0.9\linewidth]{new_block_fig.pdf}
%     \caption{(a) Concept diagram of DCE-GAN. The multimodal non-contrast inputs, including Apparent Diffusion Coefficient (ADC), with embedded attention map, synthesize early and late response DCE-MR images. (b) Previous DCE-MRI synthesis approaches \cite{b6},\cite{b7} are based on adversarial training objectives lacking focus on important regions on the anatomy of interest. The proposed approach involves discriminator-based global and local aggregated attention mechanism, further driving the generator.}
%     \label{fig:block diagram}
% \end{figure}


Deep learning methods have been explored for MRI reconstruction tasks (\cite{20},\cite{21}) and to reduce or eliminate the use of Gad-based DCE-MRI. The Generative Adversarial Network (GAN) \cite{b3} using a 3D U-Net-like generator produces 3D isotropic contrast-enhanced images from a 2D T2-flair image stack by adopting spatial pyramid pooling, enhanced residual blocks, and deep supervision. Retina U-Net \cite{b4} extracts the semantic features from non-contrast brain MR images and uses a synthesis module for contrast-enhanced images. The convolutional neural network (CNN) based network \cite{b5} generates full-dose late-response images from pre-contrast and low-contrast images for brain MRI. A U-Net-based conditional-GAN \cite{b6} with residual loss function synthesizes late-response images from early-response breast MRI images. DCE-Diff \cite{b16} is a diffusion-based generative model that maps non-contrast to contrast-enhanced prostate images.


% \begin{figure*}
%     \centering
%      {\includegraphics[width=1\linewidth]{concept_diagram_icassp_new (1).pdf}}
%     \caption{(a) Overall pipeline of the proposed AAD-DCE comprises a Generator and Aggregated Attention Discrimiator (AAD). (b) AAD  with local and global attention discriminators, $D_{LA}$ and $D_{GA}$, respectively, utilizing (c) Attention Discriminator (AD) architecture. (d) Embedded attention map $M_{x}$ with local attention map $M_{L}$ embedded on global attention map $M_{G}$.}
%     \label{fig:network architecture}
% \end{figure*}


% \begin{figure*}
%     \centering
%      {\includegraphics[width=1\linewidth]{concept_diagram_icassp_new_final.pdf}}
%     \caption{(a) Overall pipeline of the proposed AAD-DCE comprises a Generator and Aggregated Attention Discrimiator (AAD). (b) AAD  with local and global attention discriminators, $D_{LA}$ and $D_{GA}$, respectively, utilizes (c) Attention Discriminator (AD) architecture. (d) Embedded attention map $M_{x}$ with local attention map $M_{L}$ embedded on global attention map $M_{G}$.}
%     \label{fig:network architecture}
% \end{figure*}


% \begin{figure*}
%     \centering
%      {\includegraphics[width=1\linewidth]{concept_diagram_icassp_new_final (1).pdf}}
%     \caption{(a) Overall pipeline of the proposed AAD-DCE comprises a Generator and Aggregated Attention Discrimiator (AAD). (b) AAD  with local and global attention discriminators, $D_{LA}$ and $D_{GA}$, respectively, utilizes (c) Attention Discriminator (AD) architecture. (d) Embedded attention map $M_{x}$ with local attention map $M_{L}$ embedded on global attention map $M_{G}$.}
%     \label{fig:network architecture}
% \end{figure*}


% \begin{figure*}
%     \centering
%     {\includegraphics[width=0.95\linewidth]{concept_diagram_icassp_new_final (2).pdf}}
%     \caption{(a) AAD-DCE architecture with a generator and a Aggregated Attention Discrimiator (AAD) module. (b) AAD  with local and global attention discriminators, $D_{LA}$ and $D_{GA}$, respectively, utilizes (c) Attention Discriminator (AD) architecture. (d) Embedded attention map $M_{x}$.}
%     \label{fig:network architecture}
% \end{figure*}


\begin{figure*}
    \centering
     {\includegraphics[width=0.92\linewidth]{concept_diagram.pdf}}
    \caption{(a) AAD-DCE architecture with a generator and an Aggregated Attention Discriminator (AAD) module. (b) AAD  with local and global attention discriminators, $D_{LA}$ and $D_{GA}$, respectively, utilizes (c) Attention Discriminator (AD) architecture. (d) Embedded attention map $M_{x}$.}
    \label{fig:network architecture}
\end{figure*}





However, the former methods (\cite{b3},\cite{b4},\cite{b5},\cite{b6}) either depend on low-dose images or do not completely utilize the perfusion information from the non-contrast DCE-MRI data. In DCE-MRI, the Apparent Diffusion Coefficient (ADC) map computed using DWI carries useful information needed for generating the perfusion information. Secondly, the GAN-based methods adopt a global discriminator that is not specialized to consider the importance of local distribution related to the perfusion information within the anatomy of interest. DCE-Diff focuses on structural correlation without local context and suffers from higher computation costs for training and inference.

The goal of this work is to synthesize DCE-MRI early and late response images leveraging the complementary information from multimodal non-contrast MRI inputs and provide nuanced and detailed feature representation for perfusion (Fig. \ref{fig:block diagram}). Our method employs a GAN framework where the generator obtains additional guidance via an attention map computed and fed back from its discriminator. The attention map helps to focus more on the most discriminative areas between abnormal and healthy regions in the anatomy under study. Our method uses two discriminators to learn (i) the global structural correlation between the pre-contrast MRI and non-contrast T2 Weighted (T2W) MR images and (ii) local perfusion information using the ADC maps. In contrast, previous methods (TSGAN\cite{b7}, ReconGLGAN\cite{b8}) use two discriminators for optimization without attention guidance.
% with only an adversarial training perspective.
% The focus of this work is to synthesize DCE-MRI  early and late response images leveraging the complementary information from multimodal non-contrast MRI input and using two discriminators to learn (i) the global structural correlation between the pre-contrast MRI and non-contrast T2 MR images, and (ii) local perfusion information in ADC maps. Methods like TS-GAN \cite{b7} and Recon-GLGAN \cite{b8} use twin discriminators - local and global to capture the local and global context in DCE-MRI image generation and MRI reconstruction, respectively. However, they utilize the discriminators only from an adversarial training perspective, lacking local contextual attention.
% on suspicious regions.

% We go one step further to drive the generator using the two discriminators, which provide an embedded spatial attention map highlighting perfusion in the anatomy of interest 
% We propose a Trainable Attention Module (TAM) \cite{spatial_attn}  within the local and global discriminators, which provides an embedded spatial attention map.
% We propose a pair of discriminators consisting of an Aggregated Attention Module (AAM), for global and local processing of the region of interest. The discriminators provide an embedded spatial attention map highlighting perfusion in the anatomy of interest.
% This map, rich in local details, guides the generator for improved synthesis of post-contrast images. Note that our AAM differs from TAM\cite{b9}, wherein we ensemble the local and global attention maps to obtain the composed map that drives the generator.

% We propose a pair of discriminators consisting of an Aggregated Attention Module (AAM) for local and global processing of the region of interest. The discriminators provide spatial attention maps highlighting perfusion in the anatomy of interest. This map, rich in local detail, guides the generator for improved synthesis of post-contrast images. Unlike traditional approaches such as TAM\cite{b9}, our AAM uniquely integrates the local and global attention maps to create a composed map that effectively guides the generator. 
% Additionally, we use focal frequency loss \cite{ffl} and SSIM loss to capture structural details in the spectral and image domains, respectively.

We propose an Aggregated Attention-based discriminator (AAD) module consisting of two discriminators for global and local processing of the region of interest (ROI). The discriminators provide embedded spatial attention maps, each highlighting perfusion at fine and abstract levels. The two maps are aggregated and the ensembled map, rich in local details, guides the generator for improved synthesis of post-contrast images. Different from the previous attention-based discriminator \cite{b9}, the proposed Aggregated Attention-based discriminator introduces a composition of attention maps to drive the generator.
We summarize our contributions as follows:
1) An end-to-end trainable GAN for DCE-MR image synthesis from non-contrast multimodal MRI inputs, namely T2W, ADC, and T1 pre-contrast, with aggregated attention guidance to the generator to learn discriminative features that highlight hyper-intense abnormal regions in the prostate anatomy.
2) A dual-discriminator framework with aggregated attention modules that (i) enables adversarial learning to capture both global and local perfusion characteristics of the prostate anatomy and (ii) learns an aggregated attention map that highlights essential details in the region of interest to guide the generator. Our discriminator is agnostic to varying generator architectures.
3) Extensive experiments on the ProstateX dataset demonstrate that the proposed model predicts early and late response contrast-enhanced images with improvement margins (i)+0.64 dB PSNR, +0.0518 SSIM, -0.015 MAE for early response, and (ii) +0.1 dB PSNR, +0.0424 SSIM, -0.021 MAE for late response, surpassing the second best performing model, DCE-Diff. Our experiments highlight the importance of aggregated attention and ADC for accurate DCE-MRI synthesis.

% 3) We conducted extensive experiments on the ProstateX dataset, which demonstrates that the proposed model predicts early and late response DCE-MR images with improvement margins (i)+1.38 dB PSNR, +0.091 SSIM, -0.038 MAE, -13.92 FID for early response, and (ii) +0.95 dB PSNR, +0.069 SSIM, -0.040 MAE, -10.23 FID for late response, surpassing the second best performing model, ResViT. Our experiments highlight the importance of aggregated attention and ADC for accurate DCE-MRI synthesis.



% 1) An end-to-end trainable GAN for DCE-MRI image synthesis from non-contrast multimodal MRI inputs, namely T2, ADC, and T1 pre-contrast, with aggregated attention guidance from the discriminator to enhance the  of the generator.
% 2) A dual-discriminator framework that (i) enables adversarial learning to capture both global and local characteristics of the prostate anatomy, (ii) learns respective global and local attention maps based on their hidden representations, and (iii) composes the attention maps to obtain an aggregated attention map that highlights important regions in the anatomy of interest.
% 3) Extensive experiments on ProstateX dataset shows that the proposed model predicts the early and late response DCE-MR images with improvement margins of (i) +1.38 dB PSNR, +0.091 SSIM, -0.038 MAE, -13.92 FID for early response, and (ii) +0.95 dB PSNR, 0.069 SSIM, -0.040 MAE, -10.23 FID for late response, over the second best-performing model, ResViT.
% 3) We conducted extensive experiments on the ProstateX dataset, which demonstrates that the proposed model predicts early and late response DCE-MR images with improvement margins (i)+1.38 dB PSNR, +0.091 SSIM, -0.038 MAE, -13.92 FID for early response, and (ii) +0.95 dB PSNR, +0.069 SSIM, -0.040 MAE, -10.23 FID for late response, surpassing the second best performing model, ResViT. Additionally, an ablation study highlights the critical role of ADC in the model's performance.
% 1) We propose DCE-GLGAN, a generative adversarial model consisting of global and local discriminators and residual encoder-decoder-based CNN as the generator that synthesizes early and late DCE-MRI images from multimodal inputs - T2W, ADC, and T1 pre-contrast MRI.
% 2) The proposed GAN model has a Context Attention Discriminator (CAD) consisting of an aggregated attention module providing an embedded spatial attention map. The local attention map is embedded in the global attention map and guides the generator for enhanced image synthesis.
% 3) Extensive experimental results on the ProstateX dataset demonstrate that the proposed model performs better than the existing GAN, diffusion models, and image-to-image translation benchmarks, further highlighting the critical significance of ADC in improving diagnostic accuracy.
% 1) We propose DCE-GLGAN, an attention-guided generator with a Context Attention Discriminator (CAD) that synthesizes early and late DCE-MRI images from multimodal inputs.
% 2) We incorporate a combination of focal frequency loss and structural similarity loss to train our network.
% 3) Our extensive experimentation using two realistic prostate MRI datasets (ProstateX and Prostate-MRI) shows that the proposed discriminator-based attention performs better than other GAN models and image-to-image mapping benchmarks and also emphasizes the importance of ADC.

\section{Proposed Method}
Our method is based on GAN, where a generator (G) and a discriminator (D) compete with each other to synthesize ground truth-like images from multi-modal input images. The core concept of AAD-DCE is to employ embedded spatial attention maps generated by the Aggregated Attention Discriminator (AAD) to guide the generator. This approach drives the generator to focus more on ROI, thus enhancing the post-contrast images. The input $\boldsymbol{x}$ consists of T2W, ADC, and T1 pre-contrast images concatenated along the channel axis while $\boldsymbol{y}$ represents the early and late response DCE-MR images as seen in Fig. \ref{fig:network architecture}(a).
% predicted images.
% The adversarial approach is given by,
% \begin{equation}
%     G(\tilde{x})^*=\arg \min _G \max _D\left(G(x), D\left(y\right)\right) L_{G A N}
% \end{equation}

% \begin{table*}[t]
% \centering
% \caption{Quantitative comparison of the generated early and late response DCE-MRI images between DCE-GLGAN and other models for ProstateX dataset.}
% \vspace{-6mm}
% \label{tab:quantitative results}
% \begin{tabular}{cccclccccl}
%                                 &                           &                            & \multicolumn{1}{l}{}            &            &                           &                           &                            &                          &                           \\ \hline
% \multirow{2}{*}{\textbf{Model}} & \multicolumn{4}{c}{\textbf{Early Response}}                                                           &                           & \multicolumn{3}{c}{\textbf{Late Response}}                                        &                           \\ \cline{2-10} 
%                                 & \textbf{PSNR$\uparrow$}   & \textbf{SSIM$\uparrow$}    & \multicolumn{2}{c}{\textbf{MAE$\downarrow$}} & \textbf{FID $\downarrow$} & \textbf{PSNR$\uparrow$}   & \textbf{SSIM$\uparrow$}    & \textbf{MAE$\downarrow$} & \textbf{FID $\downarrow$} \\ \hline
% ConvLSTM                        & 14.92 $\pm$ 1.50          & 0.2397 $\pm$ 0.04          & \multicolumn{2}{c}{0.1393}                   & 118.706                   & 15.27 $\pm$ 2.56          & 0.2392 $\pm$ 0.06          & 0.135                    & 115.480                   \\
% CycleGAN                        & 18.61 $\pm$ 1.90          & 0.5134 $\pm$0.06           & \multicolumn{2}{c}{0.1281}                   &                           & 17.20 $\pm$ 2.40          & 0.4982 $\pm$ 0.06          & 0.129                    &                           \\
% Pix2Pix                         & 19.53 $\pm$ 1.97          & 0.5719 $\pm$ 0.05          & \multicolumn{2}{c}{0.0667}                   &                           & 19.35 $\pm$ 1.94          & 0.5546 $\pm$ 0.06          & 0.128                    &                           \\
% RegGAN                          & 20.56 $\pm$ 0.02          & 0.5966 $\pm$ 0.02          & \multicolumn{2}{c}{0.0571}                   & 23.7963                   & 19.89 $\pm$ 0.02          & 0.5803 $\pm$ 0.02          & 0.069                    & 22.6124                   \\
% TSGAN                           & 21.16 $\pm$ 3.50          & 0.6253 $\pm$ 0.10          & \multicolumn{2}{c}{0.0693}                   & 23.7533                   & 20.08 $\pm$ 2.64          & 0.5926 $\pm$ 0.09          & 0.074                    & 24.6665                   \\
% ResViT                          & 21.46 $\pm$ 1.61          & 0.6308 $\pm$ 0.05          & \multicolumn{2}{c}{0.0638}                   &                           & 20.88 $\pm$ 1.69          & 0.6231 $\pm$ 0.05          & 0.070                    &                           \\
% \textbf{DCE-GLGAN}                   & \textbf{22.74 $\pm$ 1.94} & \textbf{0.7218 $\pm$ 0.05} & \multicolumn{2}{c}{\textbf{0.0256}}          & \textbf{18.5352}          & \textbf{21.83 $\pm$ 2.15} & \textbf{0.6924 $\pm$ 0.06} & \textbf{0.0296}          & 19.8306                   \\ \hline
% \end{tabular}
% \end{table*}






% \begin{table*}[t]
% \centering
% \caption{Quantitative comparison of the generated early and late response DCE-MRI images between DCE-GLGAN and other models for ProstateX dataset.}
% \vspace{-6mm}
% \label{tab:quantitative results}
% \begin{tabular}{cccclccccc}
%                                 &                                      &                                     & \multicolumn{1}{l}{}            &            &                           &                                      &                                     &                          &                           \\ \hline
% \multirow{2}{*}{\textbf{Model}} & \multicolumn{4}{c}{\textbf{Early Response}}                                                                               &                           & \multicolumn{3}{c}{\textbf{Late Response}}                                                            &                           \\ \cline{2-10} 
%                                 & \textbf{PSNR$\uparrow$}              & \textbf{SSIM$\uparrow$}             & \multicolumn{2}{c}{\textbf{MAE$\downarrow$}} & \textbf{FID $\downarrow$} & \textbf{PSNR$\uparrow$}              & \textbf{SSIM$\uparrow$}             & \textbf{MAE$\downarrow$} & \textbf{FID $\downarrow$} \\ \hline
% ConvLSTM                        & 14.92 $\pm$ 1.50                     & 0.2397 $\pm$ 0.04                   & \multicolumn{2}{c}{0.1393}                   & 118.706                   & 15.27 $\pm$ 2.56                     & 0.2392 $\pm$ 0.06                   & 0.135                    & 115.480                   \\
% CycleGAN                        & 18.61 $\pm$ 1.90                     & 0.5134 $\pm$0.06                    & \multicolumn{2}{c}{0.1281}                   &                           & 17.20 $\pm$ 2.40                     & 0.4982 $\pm$ 0.06                   & 0.129                    &                           \\
% Pix2Pix                         & 19.53 $\pm$ 1.97                     & 0.5719 $\pm$ 0.05                   & \multicolumn{2}{c}{0.0667}                   &                           & 19.35 $\pm$ 1.94                     & 0.5546 $\pm$ 0.06                   & 0.128                    &                           \\
% RegGAN                          & 20.56 $\pm$ 0.02                     & 0.5966 $\pm$ 0.02                   & \multicolumn{2}{c}{0.0571}                   & 23.7963                   & 19.89 $\pm$ 0.02                     & 0.5803 $\pm$ 0.02                   & 0.069                    & 22.6124                   \\
% TSGAN                           & 21.16 $\pm$ 3.50                     & 0.6253 $\pm$ 0.10                   & \multicolumn{2}{c}{0.0693}                   & 23.7533                   & 20.08 $\pm$ 2.64                     & 0.5926 $\pm$ 0.09                   & 0.074                    & 24.6665                   \\
% ResViT                          & 20.12 $\pm$ 1.61                     & 0.6308 $\pm$ 0.05                   & \multicolumn{2}{c}{0.0638}                   &                           & 19.60 $\pm$ 1.69                     & 0.6153 $\pm$ 0.05                   & 0.070                    &                           \\
% \multicolumn{1}{l}{DEC-Diff}    & \multicolumn{1}{l}{22.10 $\pm$ 1.79} & \multicolumn{1}{l}{0.6722 $\pm$ 0.05} & \multicolumn{2}{c}{0.0415}                   & \textbf{10.59}            & \multicolumn{1}{l}{21.73 $\pm$ 1,95} & \multicolumn{1}{l}{0.6582 $\pm$ 0.06} & 0.05                     & \textbf{7.26}             \\
% \textbf{Ours}                   & \textbf{22.74 $\pm$ 1.94}            & \textbf{0.7218 $\pm$ 0.05}          & \multicolumn{2}{c}{\textbf{0.0256}}          & 18.5352                   & \textbf{21.83 $\pm$ 2.15}            & \textbf{0.6924 $\pm$ 0.06}          & \textbf{0.0296}          & 19.8306                   \\ \hline
% \end{tabular}
% \end{table*}



\begin{table*}[t]
\centering
\caption{Quantitative comparison of the generated early and late response DCE-MRI images between AAD-DCE and other models.}
\vspace{-6mm}
\label{tab:quantitative results}
\begin{tabular}{cccclccccc}
                                &                                      &                                     & \multicolumn{1}{l}{}            &            &                           &                                      &                                     &                          &                           \\ \hline
\multirow{2}{*}{\textbf{Model}} & \multicolumn{4}{c}{\textbf{Early Response}}                                                                               &                           & \multicolumn{3}{c}{\textbf{Late Response}}                                                            &                           \\ \cline{2-10} 
                                & \textbf{PSNR$\uparrow$}              & \textbf{SSIM$\uparrow$}             & \multicolumn{2}{c}{\textbf{MAE$\downarrow$}} & \textbf{FID $\downarrow$} & \textbf{PSNR$\uparrow$}              & \textbf{SSIM$\uparrow$}             & \textbf{MAE$\downarrow$} & \textbf{FID $\downarrow$} \\ \hline
ConvLSTM                        & 14.92 $\pm$ 1.50                     & 0.2397 $\pm$ 0.04                   & \multicolumn{2}{c}{0.139}                   & 118.706                   & 15.27 $\pm$ 2.56                     & 0.2392 $\pm$ 0.06                   & 0.135                    & 115.480                   \\
CycleGAN                        & 18.61 $\pm$ 1.90                     & 0.5134 $\pm$0.06                    & \multicolumn{2}{c}{0.128}                   & 40.4371                   & 17.20 $\pm$ 2.40                     & 0.4982 $\pm$ 0.06                   & 0.129                    & 41.6501                   \\
Pix2Pix                         & 19.53 $\pm$ 1.97                     & 0.5719 $\pm$ 0.05                   & \multicolumn{2}{c}{0.066}                   & 25.9182                   & 19.35 $\pm$ 1.94                     & 0.5546 $\pm$ 0.06                   & 0.128                    & 27.0816                   \\
RegGAN                          & 20.56 $\pm$ 0.02                     & 0.5966 $\pm$ 0.02                   & \multicolumn{2}{c}{0.057}                   & 23.7963                   & 19.89 $\pm$ 0.02                     & 0.5803 $\pm$ 0.02                   & 0.069                    & 22.6124                   \\
TSGAN                           & 21.16 $\pm$ 3.50                     & 0.6253 $\pm$ 0.10                   & \multicolumn{2}{c}{0.069}                   & 23.7533                   & 20.08 $\pm$ 2.64                     & 0.5926 $\pm$ 0.09                   & 0.074                    & 24.6665                   \\
ResViT                          & 20.12 $\pm$ 1.61                     & 0.6308 $\pm$ 0.05                   & \multicolumn{2}{c}{0.063}                   & 32.4624                   & 19.60 $\pm$ 1.69                     & 0.6153 $\pm$ 0.05                   & 0.070                    & 30.0611                   \\
\multicolumn{1}{l}{DCE-Diff}    & \multicolumn{1}{l}{22.10 $\pm$ 1.79} & \multicolumn{1}{l}{0.6700 $\pm$ 0.05} & \multicolumn{2}{c}{0.040}                   & \textbf{10.5900}            & \multicolumn{1}{l}{21.73 $\pm$ 1,95} & \multicolumn{1}{l}{0.6500 $\pm$ 0.06} & 0.050                     & \textbf{7.2600}             \\
\textbf{AAD-DCE}                   & \textbf{22.74 $\pm$ 1.94}            & \textbf{0.7218 $\pm$ 0.05}          & \multicolumn{2}{c}{\textbf{0.025}}          & 18.5352                   & \textbf{21.83 $\pm$ 2.15}            & \textbf{0.6924 $\pm$ 0.06}          & \textbf{0.029}          & 19.8306                   \\ \hline
\end{tabular}
\end{table*}




\subsection{Generator}\label{AA}
% The Generator (G) is an encoder-decoder based architecture with residual blocks.The encoder layer consists of series of convolutional layers to capture the hierarchy of localized features of source image with Residual CNN at the bottleneck layer. Followed by decoder layer consiting of CNN layers.

% The Generator (G) utilizes an encoder-decoder architecture with residual blocks. The encoder and decoder consists of convolution layer designed to capture hierarchial and localized features from the input image. At the bottleneck layer, residual convolution networks are employed to enhance the feature representation. The generator can be a any image-to-image mapping CNN or transformer network. Our discriminator is adaptle to any

% The Generator (G) can be any image-to-image mapping CNN or transformer-based network. Our discriminator is flexible, making it suitable for various network architectures. We utilized an encoder-decoder architecture with residual blocks. The encoder and decoder consists of convolution layer designed to capture hierarchial and localized features from the input image. At the bottleneck layer, residual convolution networks are employed to enhance the feature representation.

The Generator (G) can be any image-to-image mapping CNN or transformer-based network. Our discriminator is adaptable to various generator architectures (Section III B). The generator architecture in Fig. \ref{fig:network architecture}(a), is an encoder-decoder architecture with residual blocks.

% The encoder and decoder use convolutional layers to capture hierarchical and localized features from the input image, while residual convolution networks at the bottleneck enhance feature representation.



\subsection{Aggregated Attention Discriminator (AAD)}
The architecture of the AAD block is shown in Fig. \ref{fig:network architecture}(b). It consists of a Global Attention discriminator $D_{GA}$, a Local Attention discriminator $D_{LA}$, and a Classifier ($\Psi_C$). $D_{GA}$ and $D_{LA}$ are global and local feature extractors that employ a spatial guidance module named Attention Discriminator (AD). The input to $D_{GA}$ is the whole image ($H \times W$) whereas $D_{LA}$ operates only the ROI ($H^{\prime} \times W^{\prime}$).
The AD in Fig. \ref{fig:network architecture}(c) comprises two key components: the Attention branch and the Trunk branch inspired by RAM \cite{b19}. The Trunk branch made up of convolutional layers, extracts low-level features from the input $\boldsymbol{y}$ to produce the output T($\boldsymbol{y}$). Note that our trunk branch is much simpler than in RAM \cite{b19} while maintaining feature extraction capability. The Attention branch, using the bottom-up top-down structure \cite{b18}, learns an attention map A($\boldsymbol{y}$), which modulates the trunk branch's output by applying weights. The output of $D_{GA}$ and $D_{LA}$ that utilizes the AD module are shown in (\ref{equ:2}) and (\ref{equ:3}),



% AD, as seen in Figure \ref{fig:network architecture}(c), consists of an Attention branch and a Trunk branch. These are convolutional layers, where the initial layers extract the low-level features from the inputs and pass them through the subsequent branches. With the trunk branch generating output T($\boldsymbol{y}$) from input $\boldsymbol{y}$, the attention branch learns an attention map  A($\boldsymbol{y}$) that weights the output of the trunk. The output of $D_{GA}$ and $D_{LA}$ that utilizes the AD module are shown in (\ref{equ:2}) and (\ref{equ:3}),

% aatn branch modulates the featus
\vspace{-2mm}
\begin{equation}
\vspace{-2mm}
E_{G}= \left(A_{G}(y)+1\right) \times T_{G}(y)  
\label{equ:2}
\end{equation}
\begin{equation}
E_{L}= \left(A_{L}(y)+1\right) \times T_{L}(y)  
\label{equ:3}
\end{equation} 
% \vspace{-2mm}
where $A_{G}(y)$, $A_{L}(y)$ are the attention maps from attention branch and $T_{G}(y)$, $T_{L}(y)$ are trunk branch outputs for $D_{GA}$ and $D_{LA}$ respectively.
% and C is the set of channels.
The $N_{C}$ dimensional feature vectors $E_{G}$ and $E_{L}$ are concatenated into a $2N_{C}$ dimensional vector and passed to the classifier ($\Psi_C$) as shown in (\ref{equ:4}). This vector is then fed into a fully connected layer, followed by a sigmoid activation function to classify real or fake.
\vspace{-2mm}
\begin{equation}
D({y})=\Psi_C\left(E_{G}\|\ E_{L}\right)
\label{equ:4}
\end{equation}
% \vspace{-2mm}
% The attention map $M_{L}$, obtained from $D_{LA}$ is now embedded on attention map $M_{G}$ from $D_{GA}$, resulting in  $M_{x}$.
The mean value of the attention maps $M_{G}$ and $M_{L}$ obtained from the two discriminators, $D_{GA}$ and $D_{LA}$  are aggregated by embedding $M_{L}$ into $M_{G}$ to give $M_{x}$. $M_{x}$ is infused into the multimodal inputs ($x$) using Residual Hadamard Product (RHP) as seen in (\ref{equ:5}) and fed to the  Generator.
\vspace{-2mm}
\begin{equation}
    % \tilde{x}=x \oplus M_{x}=\left(g\left(M_{T_x} ; \theta\right)+1\right) \times x 
    % x^{\prime}=x \oplus M_{x}=\left(M_{x} +1\right) \times x 
    x^{\prime}=\left(M_{x} +1\right) \times x 
\label{equ:5}
\end{equation}
As the infusion of attention map is based on RHP, it is initialized to one at the start of the training. 
The adversarial GAN loss, generator loss, and the final objective function are: 
% (\ref{equ:6} \ref{equ:7},
\vspace{-1mm}
\begin{equation}
    \begin{split}
        L_{G A N}(G, D)= & \mathbb{E}_{y \sim \mathcal{Y}}[\log D(y)]+ \\
 & \mathbb{E}_{x \sim \mathcal{X}}\left[\log \left(1-D\left(G\left(x^{\prime}\right)\right)\right)\right] 
    \end{split}
\label{equ:6}
\end{equation}
\vspace{-2mm}
\begin{equation}
    L_{L 1}(G)=\mathbb{E}_{x, y}\left[\left\|y-G\left(x^{\prime}\right)\right\|_1\right]
\label{equ:7}
\end{equation}
\vspace{-2mm}
\begin{equation}
    \arg \min _G \max _D L_{G A N}(G, D)+\lambda L_{L 1}(G)
\label{equ:8}
\end{equation}
% where $x^{\prime}$ is computed from (\ref{equ:5}). 
% The generator loss is given as
% \vspace{-2mm}
% \begin{equation}
%     L_{L 1}(G)=\mathbb{E}_{x, y}\left[\left\|y-G\left(x^{\prime}\right)\right\|_1\right]
% \label{equ:7}
% \end{equation}
% \vspace{-2mm}
% The final objective function is:
% \vspace{-2mm}
% \begin{equation}
%     \arg \min _G \max _D L_{G A N}(G, D)+\lambda L_{L 1}(G)
% \label{equ:8}
% \end{equation}

% \vspace{-2mm}
% Here $g(;\theta)$ is the attention transfer block T, a small 3-layer convolution network that transfers the attention map to the corresponding pixel weight map.

% \begin{figure*}[!t]
%     \centering
%      {\includegraphics[width=0.85\linewidth]{final_793_prox_icassp.pdf}}
%     \caption{Enter Caption}
%     \label{fig:images_prosx}
% \end{figure*}


% \begin{figure*}[!t]
% \vspace*{-\baselineskip}
%     \centering
%      {\includegraphics[width=1\linewidth]{Copy of new_793_prox_icassp.pdf}}
%      \vspace{-2mm}
%     \caption{Visual results of early \& late response for ProstateX dataset with error maps (from left to right):Groundtruth, ConvLSTM, CycleGAN, Pix2Pix, RegGAN,
%  TSGAN, ResViT,and DCE-GAN(ours). The yellow bounding box represents the region of interest.}
%     \label{fig:image_posxl}
% \end{figure*}

% \begin{figure*}[!t]
% \vspace*{-\baselineskip}
%     \centering
%      {\includegraphics[width=1\linewidth]{Copy of new_793_prox_icassp.pdf}}
%      \vspace{-2mm}
%     \caption{Visual results of early \& late response for ProstateX dataset with error maps (from left to right):Groundtruth, ConvLSTM, CycleGAN, Pix2Pix, RegGAN,
%  TSGAN, ResViT,and DCE-GAN(ours). The yellow bounding box represents the region of interest.}
%     \label{fig:image_posxl}
% \end{figure*}


% \begin{figure*}[!t]
% \vspace*{-\baselineskip}
%     \centering
%      {\includegraphics[width=1\linewidth]{all_qualitative_result.pdf}}
%      \vspace{-4mm}
%     \caption{(a) Visual results of early \& late response for ProstateX dataset with error maps (from left to right).The yellow bounding box represents the region of interest. (b) The attention maps obtained by different ways of ensembling, illustrating that $D_{LA}$ embedded on $D_{GA}$ allows the model to focus more effectively on the suspicious regions pointed by the arrow and the bounding box in the ground truth DCE-MR image. (c) Incorporating ADC in the inputs yields a synthesis closer to the ground truth. The dark region highlighted by the arrow in the ADC image signifies an area where contrast enhancement is likely to occur following the contrast injection.}
%     \label{fig:image_posxl}
% \end{figure*}



% \begin{figure*}[!t]
%     \centering
%      {\includegraphics[width=1\linewidth]{all_qualitative_result (1).pdf}}
%     \caption{(a) Visual results of early and late response for the ProstateX dataset with error maps (left to right). The yellow bounding box marks the region of interest. (b) Attention maps from different ensembling methods. Attention embedding enables better focus on suspicious regions. (c) Ablative study with and without ADC maps.(d) Ablative study on various generator architectures.}
%     \label{fig:image_posxl}
% \end{figure*}




% \begin{figure*}[!t]
%     \centering
%      {\includegraphics[width=1\linewidth]{all_qualitative_result (2).pdf}}
%     \caption{(a) Visual results of early and late response for the ProstateX dataset with error maps (left to right). The yellow bounding box marks the region of interest. (b) Attention maps from different ensembling methods. Attention embedding enables better focus on suspicious regions. (c) Ablative study with and without ADC maps. (d) Ablative study on various generator architectures.}
%     \label{fig:image_posxl}
% \end{figure*}


% \begin{figure*}[!t]
%     \centering
%      {\includegraphics[width=0.96\linewidth]{Copy of all_qualitative_result.pdf}}
%     \caption{(a) Visual results of early and late response for the ProstateX dataset with error maps. The yellow bounding box marks the ROI. (b) Ablative study with and without ADC maps. (c) Attention maps from different ensembling methods. Attention embedding enables better focus on suspicious regions. }
%     \label{fig:image_posxl}
% \end{figure*}


\begin{figure*}[!t]
    \centering
     {\includegraphics[width=0.96\linewidth]{qualitative_result.pdf}}
    \caption{(a) Visual results of early and late response for the ProstateX dataset with error maps. The yellow bounding box marks the ROI. (b) Ablative study with and without ADC maps. (c) Attention maps from different ensembling methods. Attention embedding enables better focus on suspicious regions.}
    \label{fig:image_posxl}
\end{figure*}


% \vspace{-2mm}
% Including ADC in the inputs produces a synthesis closer to the ground truth, with the dark region in the ADC image (arrow) highlighting a potential site for contrast enhancement after injection


% Visual results of early and late response for the ProstateX dataset with error maps (left to right). The yellow bounding box marks the region of interest. (b) Attention maps from different ensembling methods show that embedding $D_{LA}$ into $D_{GA}$ enables the model to better focus on suspicious regions indicated by the arrow and bounding box in the ground truth DCE-MR image. (c) Including ADC in the inputs produces a synthesis closer to the ground truth, with the dark region in the ADC image (arrow) highlighting a potential site for contrast enhancement after injection.




% Visual results of early \& late response for ProstateX dataset with error maps (from left to right).The yellow bounding box represents the region of interest. (b) The attention maps obtained by different ways of ensembling, illustrating that $D_{LA}$ embedded on $D_{GA}$ allows the model to focus more effectively on the suspicious regions pointed by the arrow and the bounding box in the ground truth DCE-MR image. (c) Incorporating ADC in the inputs yields a synthesis closer to the ground truth. The dark region highlighted by the arrow in the ADC image signifies an area where contrast enhancement is likely to occur following the contrast injection.
% is able to pay more attention to the suspicious region pointed by the arrow in the ground truth DCE-MR image.







% \begin{figure*}[!t]
% \vspace*{-\baselineskip}
%     \centering
%      {\includegraphics[width=0.85\linewidth]{new_793_prox_icassp.pdf}}
%      \vspace{-2mm}
%     \caption{Visual results of early \& late response for ProstateX dataset with error maps (from left to right):Groundtruth, ConvLSTM, CycleGAN, Pix2Pix, RegGAN,
%  TSGAN, ResViT,and DCE-GAN(ours). The yellow bounding box represents the region of interest.}
%     \label{fig:image_posxl}
% \end{figure*}




\section{Experiments and Results}
\subsection{Dataset and Implementation Details}
We trained our model on the open-source ProstateX dataset \cite{b17} consisting of T2-Weighted, ADC, T1 pre-contrast, and DCE images. The dataset consists of 346 patient studies with 5520 images (4410 for training and 1104 for validation). In the DCE data, the early enhancement usually occurs within 10 seconds of the appearance of the injected contrast agent, and therefore, the early and late response time points are selected accordingly. The data is registered using SimpleITK rigid registration and is resampled to $H \times W \times 16$. Here $H \times W$ is $160\times160$ and $H^\prime \times W^\prime$ is $60\times60$. We set $N_{C}$ as 64.
% $160\times160\times16$.
% \vspace{-2mm}
\begin{table}[h]
\centering
\caption{Importance of ADC}
\vspace{-2mm}
\label{tab:adc_ablative}
\begin{tabular}{ccccc}
\hline
\textbf{DCE-MRI} & \multicolumn{2}{c}{\textbf{Early Response}} & \multicolumn{2}{c}{\textbf{Late Response}} \\ \hline
\textbf{ADC}     & w/o ADC          & w/ ADC                   & w/o ADC          & w/ ADC                  \\ \hline
\textbf{PSNR}    & 21.50            & \textbf{22.74}           & 20.43            & \textbf{20.64}          \\ \hline
\textbf{SSIM}    & 0.6841           & \textbf{0.7218}          & 0.6404           & \textbf{0.6924}         \\ \hline
\end{tabular}
\end{table}
% \vspace{-9mm}

The models are trained using Pytorch 2.0 on a 24GB RTX 3090 GPU. The Adam optimizer ($\beta_{1}=0.9$, $\beta_{2}=0.999$, learning rate 1e-3) is used for 200 epochs, batch size of 4, and $\lambda$ is 10. The evaluation metrics are Peak Signal-to-Nosie Ratio (PSNR), Structural Similarity Index (SSIM), Mean Absolute Error (MAE), and Frechet Inception Distance (FID).



% \vspace{}
\subsection{Results and Discussion}
\textbf{1. Comparative study with previous methods:} We compare AAD-DCE with baseline models, Pix2Pix \cite{b11}, CycleGAN \cite{b12}, ConvLSTM \cite{b13}, Reg-GAN \cite{b14}, ResViT \cite{b15}, TSGAN \cite{b7}, and DCE-Diff \cite{b16}. 
The quantitative results in Table \ref{tab:quantitative results} show that our model outperforms other GAN-based, transformer-based, and convLSTM methods in most cases. Our model outperforms DCE-Diff, by +0.64dB, +0.1 dB in PSNR, +0.0518, +0.0424 in SSIM, and -0.015, -0.021 in MAE for early and late response respectively, except in terms of FID due to diffusion models' ability to learn image distribution through probabilistic framework. Fig. \ref{fig:image_posxl}(a) illustrates the synthesis results highlighting the abnormal regions in the transition and peripheral zones (TZ and PZ) of the prostate. We observe that AAD-DCE is most comparable to the ground truth with fewer errors in the error map and better preserves the global and local details of the post-contrast images.
We believe that: (i) While $D_{GA}$ focuses on the overall anatomical features, $D_{LA}$ targets the suspicious regions within the prostate via the attention information. (ii) ADC provides perfusion information, while T2W and T1 pre-contrast MR images establish the structural correlations. Integrating these modalities enables learning the complementary information, resulting in more accurate and detailed structural representations and perfusion information in the predicted images.
% We believe the reason for this are: (i) $D_{LA}$ focuses on the ROI, particularly the prostate, and the embedded attention map captures and assess the critical contrast. (ii) ADC provides perfusion information, while T2W and T1 Pre-contrast MR images establish the structural correlations. Integrating these modalities enables the model to learn complementary information, resulting in more accurate and detailed structural representations in the output images.

% Table \ref{tab:quantitative results} shows that our method outperforms other methods in PSNR, SSIM and MAE, except DCE-Diff in terms of FID. Te 
% +1.58 dB in PSNR, +0.091 in SSIM and -0.0314 in MAE for early response and +1.75 dB in PSNR, +0.077 in SSIM and -0.0389 in MAE for late response.

% Fig.\ref{fig:image_posxl} illustrates the synthesis results on the ProstateX dataset. We observe that AAM-DCE is most comparable to the ground truth with fewer errors in the error map and better preserves the global and local details of the post-contrast images. We believe the reason for this are: (i) $D_{LA}$ focuses on the ROI, particularly the prostate, and the embedded attention map captures and assess the critical contrast. (ii) ADC provides perfusion information, while T2W and T1 Pre-contrast MR images establish the structural correlations. Integrating these modalities enables the model to learn complementary information, resulting in more accurate and detailed structural representations in the output images.

% \begin{figure*}[!t]
%     \centering
%      {\includegraphics[width=0.70\linewidth]{final_793_prox_icassp.pdf}}
%     \caption{Enter Caption}
%     \label{fig:images_prosx}
% \end{figure*}




% \begin{center}
% \begin{table}[]
% \centering
% \caption{Ablative study on G-TAM and L-TAM}
% \vspace{-2mm}
% \label{tab:discriminator_abalative}
% \begin{tabular}{llcc}
% \hline
% \textbf{Components}     & \textbf{PSNR} & \textbf{SSIM} \\ \hline
%  G-TAM                   & 22.0608       & 0.7013        \\ \hline 
%  L-TAM * G-TAM           & 21.6625       & 0.7014        \\ \hline 
%  L-TAM + G-TAM           & 22.3872       & 0.7038        \\ \hline 
%  L-TAM embedded on G-TAM & 22.7394       & 0.7218        \\ \hline 
% \end{tabular}
% \end{table}
% \end{center}

% \vspace{-6mm}
% \begin{table}[h]
% \centering
% \caption{Importance of ADC}
% \vspace{-2mm}
% \label{tab:adc_ablative}
% \begin{tabular}{llll}
% \hline
% \multicolumn{1}{c}{\textbf{DCE-MRI}} & \textbf{ADC}    & \textbf{PSNR}  & \textbf{SSIM}   \\ \hline
% Early Response                       & w/o ADC         & 21.50          & 0.6841          \\
%                                      & \textbf{w/ ADC} & \textbf{22.74} & \textbf{0.7218} \\ \hline
% Late Response                        & w/o ADC         & 20.43          & 0.6404          \\
%                                      & \textbf{w/ ADC} & \textbf{20.64} & \textbf{0.6924} \\ \hline
% \end{tabular}
% \end{table}


% \vspace{-4mm}
% \begin{table}[h]
% \centering
% \caption{Importance of ADC}
% \vspace{-2mm}
% \label{tab:adc_ablative}
% \begin{tabular}{ccccc}
% \hline
% \textbf{DCE-MRI} & \multicolumn{2}{c}{\textbf{Early Response}} & \multicolumn{2}{c}{\textbf{Late Response}} \\ \hline
% \textbf{ADC}     & w/o ADC          & w/ ADC                   & w/o ADC          & w/ ADC                  \\ \hline
% \textbf{PSNR}    & 21.50            & \textbf{22.74}           & 20.43            & \textbf{20.64}          \\ \hline
% \textbf{SSIM}    & 0.6841           & \textbf{0.7218}          & 0.6404           & \textbf{0.6924}         \\ \hline
% \end{tabular}
% \end{table}
% \vspace{-5mm}





% \begin{center}
% \begin{table}[]
% \centering
% \caption{Ablative study on $D_{LA}$ and $D_{GA}$}
% \vspace{-2mm}
% \label{tab:discriminator_abalative}
% \begin{tabular}{llc}
% \hline
% \textbf{Components}           & \textbf{PSNR} & \textbf{SSIM} \\ \hline
% $D_{GA}$                      & 22.0608       & 0.7013        \\ \hline
% $D_{LA}$ * $D_{GA}$           & 21.6625       & 0.7014        \\ \hline
% $D_{LA}$ + $D_{GA}$           & 22.3872       & 0.7038        \\ \hline
% $D_{LA}$ embedded on $D_{GA}$ & \textbf{22.7394}       & \textbf{0.7218}       \\ \hline
% \end{tabular}
% \end{table}
% \end{center}
\setlength{\tabcolsep}{4pt}
\begin{center}
\begin{table}[]
\centering
\caption{Ablative study on attention ensembling}
\vspace{-2mm}
\label{tab:discriminator_abalative}
\begin{tabular}{lcccc}
\hline
\textbf{Attention Maps} & $D_{GA}$ & $D_{LA}$*$D_{GA}$ & $D_{LA}+D_{GA}$ & \begin{tabular}[c]{@{}c@{}}$D_{LA}$ embed\\  on $D_{GA}$\end{tabular} \\ \hline
\textbf{PSNR}       & 22.06  & 21.66           & 22.38         & \textbf{22.73}                                                      \\ \hline
\textbf{SSIM}       & 0.7012   & 0.7014            & 0.7038          & \textbf{0.7218}                                                       \\ \hline
\end{tabular}
\end{table}
\end{center}
\vspace{-5mm}
% \vspace{-7mm}




% \begin{figure}
%     \centering
%     \includegraphics[width=1\linewidth]{ablative_gtam_ltam_739.pdf}
%     \caption{Enter Caption}
%     \label{fig:gtam_ablative}
% \end{figure}

% \begin{figure}
%     \centering
%     \includegraphics[width=1\linewidth]{dga_dla_ablative_study.pdf}
%     \caption{Enter Caption}
%     \label{fig:gl_ablative}
% \end{figure}


% \begin{figure}
% \vspace*{-\baselineskip}
%     \centering
%     \includegraphics[width=1\linewidth]{adc_attn_map_ablative_study.pdf}
%     \caption{(a) Qualitative result for the importance of ADC. (b) The attention maps obtained for different components and its seen that $D_{LA}$ embedded on $D_{GA}$ is able to pay more attention to the contrast region present in the ground truth DCE-MR image. }
%     \label{fig:ablative_study}
% \end{figure}




% \vspace{-6mm}
% \begin{table}[h]
% \centering
% \caption{Importance of ADC}
% \vspace{-2mm}
% \label{tab:adc_ablative}
% \begin{tabular}{llll}
% \hline
% \multicolumn{1}{c}{\textbf{DCE-MRI}} & \textbf{ADC}    & \textbf{PSNR}  & \textbf{SSIM}   \\ \hline
% Early Response                       & w/o ADC         & 21.50          & 0.6841          \\
%                                      & \textbf{w/ ADC} & \textbf{22.74} & \textbf{0.7218} \\ \hline
% Late Response                        & w/o ADC         & 20.43          & 0.6404          \\
%                                      & \textbf{w/ ADC} & \textbf{20.64} & \textbf{0.6924} \\ \hline
% \end{tabular}
% \end{table}


%only adc ablative study
% \begin{figure}
%     \centering
%     \includegraphics[width=1\linewidth]{adc_ablative_226_icassp.pdf}
%     \caption{Enter Caption}
%     \label{fig:enter-label}
% \end{figure}


\vspace{-3.5mm}
\textbf{2. Importance of ADC:}\label{AAA}
% \subsection{Ablative Studies}\label{AAA}
Table \ref{tab:adc_ablative} shows the importance of using ADC as inputs. The incorporation of ADC images containing perfusion information improves the post-contrast image synthesis. Fig. \ref{fig:image_posxl}(b) shows the abnormality in PZ that correlates with the tumor region findings in the ground truth (highlighted in yellow arrows).

\textbf{3. Ablative study on attention ensembling:} We study the effect of only the global attention and different ways of ensembling the global and local attention maps, namely additive, multiplicative, and embedding operations (Table \ref{tab:discriminator_abalative}). In Fig. \ref{fig:image_posxl}(c), the embedded attention map focuses more precisely on the regions with contrast uptake
with greater clarity. 

\textbf{4. Ablative study on various generator architectures:} We have evaluated the proposed AAD on various generator benchmarks - encoder-decoder CNN (U-Net) \cite{b11}, vision transformer \cite{b15}. A U-Net with AAD improves PSNR by +1.49 dB and SSIM by +0.0387, while for the transformer-based generator, AAD provides improvements of +1.77 dB in PSNR and +0.0432 in SSIM (Fig. \ref{fig:gen_ablative}).

\textbf{5. Model Parameters}: Comparing with the recent state-of-the-art, DCE-Diff, shows that our model has 19.93 million parameters with superior performance over DCE-Diff with 125.08 million parameters. 
% A UNet with a AAD gives an improvement of +1.49 dB PSNR and +0.0387 SSIM and a tranformer based gennerator with AAD gives +1.77 dB in PSNR and +0.0432 SSIM improvement.
% The contribution of each module of our model is studied individually as seen in Table \ref{tab:discriminator_abalative}. 
% Furthermore, we study the effect of only the global attention and different ways of ensembling the global and local attention maps, namely additive, multiplicative and embedding operations as seen in Table \ref{tab:discriminator_abalative}. In Fig. \ref{fig:image_posxl}(b), the embedded attention map focuses more precisely on the contrast region, capturing it with greater clarity.

% We try to understand the effect of local and global attention maps by adding, multiplying and embedding the local with the global attention maps. Fig.\ref{fig:ablative_study}(b). 
% Table \ref{tab:adc_ablative} shows the ablative study on using ADC as inputs. The incorporation of ADC images containing perfusion information significantly improves the results.
% \begin{figure}
%     \centering
%     \includegraphics[width=0.8\linewidth]{Untitled drawing.pdf}
%     \vspace{-2mm}
%     \caption{Ablative study on various generator architectures.}
%     \label{fig:gen_ablative}
% \end{figure}

\begin{figure}
    \centering
    \includegraphics[width=0.8\linewidth]{generator_ablative.pdf}
    \vspace{-2mm}
    \caption{Generator architecture with AAD enhances focus on the ROI.}
    \label{fig:gen_ablative}
\end{figure}



\section{Conclusion}
This work presents a GAN framework with aggregated attention feedback guidance to the generator from the discriminator to synthesize DCE-MR images from multimodal non-contrast MRI inputs. Extensive experimentation with comparative models and ablation studies with ADC, attention ensembling methods, and various generator architectures show that the proposed method can synthesize higher-quality DCE-MR images. We are currently aiming at clinical validation studies for practical use.


% \begin{figure*}[]
%     \centering
%      {\includegraphics[width=0.95\linewidth]{final_793_prox_icassp.pdf}}
%     \caption{Enter Caption}
%     \label{fig:enter-label}
% \end{figure*}






% \subsection{Identify the Headings}\label{ITH}
% Headings, or heads, are organizational devices that guide the reader through 
% your paper. There are two types: component heads and text heads.

% Component heads identify the different components of your paper and are not 
% topically subordinate to each other. Examples include Acknowledgments and 
% References and, for these, the correct style to use is ``Heading 5''. Use 
% ``figure caption'' for your Figure captions, and ``table head'' for your 
% table title. Run-in heads, such as ``Abstract'', will require you to apply a 
% style (in this case, italic) in addition to the style provided by the drop 
% down menu to differentiate the head from the text.

% Text heads organize the topics on a relational, hierarchical basis. For 
% example, the paper title is the primary text head because all subsequent 
% material relates and elaborates on this one topic. If there are two or more 
% sub-topics, the next level head (uppercase Roman numerals) should be used 
% and, conversely, if there are not at least two sub-topics, then no subheads 
% should be introduced.

% \subsection{Figures and Tables}\label{FAT}
% \paragraph{Positioning Figures and Tables} Place figures and tables at the top and 
% bottom of columns. Avoid placing them in the middle of columns. Large 
% figures and tables may span across both columns. Figure captions should be 
% below the figures; table heads should appear above the tables. Insert 
% figures and tables after they are cited in the text. Use the abbreviation 
% ``Fig.~\ref{fig}'', even at the beginning of a sentence.

% \begin{table}[htbp]
% \caption{Table Type Styles}
% \begin{center}
% \begin{tabular}{|c|c|c|c|}
% \hline
% \textbf{Table}&\multicolumn{3}{|c|}{\textbf{Table Column Head}} \\
% \cline{2-4} 
% \textbf{Head} & \textbf{\textit{Table column subhead}}& \textbf{\textit{Subhead}}& \textbf{\textit{Subhead}} \\
% \hline
% copy& More table copy$^{\mathrm{a}}$& &  \\
% \hline
% \multicolumn{4}{l}{$^{\mathrm{a}}$Sample of a Table footnote.}
% \end{tabular}
% \label{tab1}
% \end{center}
% \end{table}

% \begin{figure}[htbp]
% \centerline{\includegraphics{fig1.png}}
% \caption{Example of a figure caption.}
% \label{fig}
% \end{figure}

% Figure Labels: Use 8 point Times New Roman for Figure labels. Use words 
% rather than symbols or abbreviations when writing Figure axis labels to 
% avoid confusing the reader. As an example, write the quantity 
% ``Magnetization'', or ``Magnetization, M'', not just ``M''. If including 
% units in the label, present them within parentheses. Do not label axes only 
% with units. In the example, write ``Magnetization (A/m)'' or ``Magnetization 
% \{A[m(1)]\}'', not just ``A/m''. Do not label axes with a ratio of 
% quantities and units. For example, write ``Temperature (K)'', not 
% ``Temperature/K''.

% \section*{Acknowledgment}

% The preferred spelling of the word ``acknowledgment'' in America is without 
% an ``e'' after the ``g''. Avoid the stilted expression ``one of us (R. B. 
% G.) thanks $\ldots$''. Instead, try ``R. B. G. thanks$\ldots$''. Put sponsor 
% acknowledgments in the unnumbered footnote on the first page.

% \section*{References}

% Please number citations consecutively within brackets \cite{b1}. The 
% sentence punctuation follows the bracket \cite{b2}. Refer simply to the reference 
% number, as in \cite{b3}---do not use ``Ref. \cite{b3}'' or ``reference \cite{b3}'' except at 
% the beginning of a sentence: ``Reference \cite{b3} was the first $\ldots$''

% Number footnotes separately in superscripts. Place the actual footnote at 
% the bottom of the column in which it was cited. Do not put footnotes in the 
% abstract or reference list. Use letters for table footnotes.

% Unless there are six authors or more give all authors' names; do not use 
% ``et al.''. Papers that have not been published, even if they have been 
% submitted for publication, should be cited as ``unpublished'' \cite{b4}. Papers 
% that have been accepted for publication should be cited as ``in press'' \cite{b5}. 
% Capitalize only the first word in a paper title, except for proper nouns and 
% element symbols.

% For papers published in translation journals, please give the English 
% citation first, followed by the original foreign-language citation \cite{b6}.

\begin{thebibliography}{00}

\bibitem{b1} Mazaheri, Yousef, Oguz Akin, and Hedvig Hricak. "Dynamic Contrast-enhanced Magnetic Resonance Imaging of Prostate Cancer: A Review of Current Methods and Applications." World Journal of Radiology 9, no. 12, pp. 416-425, 2017.
\bibitem{b2} Moshe Rogosnitzky and Stacy Branch, “Gadolinium-based contrast agent toxicity: a review of known and proposed mechanisms,” Biometals, vol. 29, no. 3, pp.365–376, 2016.
\bibitem{b3} Wang, Y., Wu, W., Yang, Y., Hu, H., Yu, S., Dong, X., Chen, F., \& Liu, Q, "Deep learning-based 3D MRI contrast-enhanced synthesis from a 2D non-contrast T2Flair sequence", Medical Physics, 49(7), pp. 4478–4493, 2022.
\bibitem{b4} Xie, H., Lei, Y., Wang, T., Roper, J., Axente, M., Bradley, J. D., Liu, T., \& Yang, X, "Magnetic resonance imaging contrast enhancement synthesis using cascade networks with local supervision", Medical Physics, 49(5), pp. 3278–3287, 2022.
\bibitem{b5} Enhao Gong, M. John, Max Wintermark Pauly, and Greg Zaharchuk, “Deep learning enables reduced gadolinium dose for contrast-enhanced brain MRI,” in Journal of Magnetic Resonance Imaging, vol. 48, pp. 330–340, 2018
\bibitem{b6} J. C. Caicedo R. D. Fonnegra, M. Liliana Hernandez and G. M. Díaz, “Early-to-late prediction of DCE-MRI contrast-enhanced images in using generative adversarial networks,” in 2023 IEEE 20th International Symposium on Biomedical Imaging, pp. 1–5, 2023.
\bibitem{b7} E. Kim, H. -H. Cho, J. Kwon, Y. -T. Oh, E. S. Ko and H. Park, "Tumor-Attentive Segmentation-Guided GAN for Synthesizing Breast Contrast-Enhanced MRI Without Contrast Agents," in IEEE Journal of Translational Engineering in Health and Medicine, vol. 11, pp. 32-43, 2023
\bibitem{b8} Murugesan, B., S, V.R., Sarveswaran, K., Ram, K., Sivaprakasam, M, "Recon-GLGAN: A Global-Local context based Generative Adversarial Network for MRI Reconstruction," Machine Learning for Medical Image Reconstruction Workshop, MLMIR 2019, held in conjuntion with MICCAI, pp. 3-15, 2019.
\bibitem{b9} Lin, Y., Wang, Y., Li, Y., Gao, Y., Wang, Z. and Khan, L., 2021. Attention-based spatial guidance for image-to-image translation. In Proceedings of the IEEE/CVF Winter Conference on Applications of Computer Vision, pp. 816-825, 2021.
\bibitem{b10} S. G. Armato et al., “PROSTATEx Challenges for computerized classification of prostate lesions from multiparametric magnetic resonance images,” Journal of Medical Imaging, vol. 5, no. 04, p. 1, Nov. 2018, doi: 10.1117/1.jmi.5.4.044501.

% \bibitem{b11} Lin, Y., Wang, Y., Li, Y., Gao, Y., Wang, Z. and Khan, L., Attention-based spatial guidance for image-to-image translation. In Proceedings of the IEEE/CVF Winter Conference on Applications of Computer Vision, pp. 816-825, .

\bibitem{b11} P. Isola, J.-Y. Zhu, T. Zhou, and A. A. Efros, "Image-to-Image translation with conditional adversarial networks," 
In Proceedings of the IEEE Conference on Computer Vision and Pattern Recognition, pp. 1125-1134, 2017.
\bibitem{b12} J.-Y. Zhu, T. Park, P. Isola, and A. A. Efros, "Unpaired Image-to-Image Translation Using Cycle-Consistent Adversarial Networks", IEEE/ICCV,  pp.2223-2232, 2017.
\bibitem{b13}  Xingjian GulrajSHI, Zhourong Chen, Hao Wang, DitYan Yeung, Wai-kin Wong, and Wang-chun WOO, “Convolutional lstm network: A machine learning approach for precipitation nowcasting,” in Advances in Neural Information Processing Systems, vol. 28, 2015.
\bibitem{b14} L. Kong, C. Lian, D. Huang, Y. Hu, and Q. Zhou,
“Breaking the dilemma of medical image-to-image
translation,” in Advances in Neural Information Processing Systems, vol. 34, pp. 1964–1978, 2021.
\bibitem{b15} O. Dalmaz, M. Yurt, and T. Cukur, “RESVIT: Residual Vision Transformers for multimodal medical Image synthesis,” IEEE Transactions on Medical Imaging, vol. 41, no. 10, pp. 2598–2614, Oct. 2022.
\bibitem{b16} S. Ramanarayanan, A. Sarkar, MN. Gayathri, K. RAM, M. Sivaprakasam, "DCE-Diff: Diffusion Model for Synthesis of Early and Late Dynamic Contrast-Enhanced MR Images from Non-Contrast Multimodal Inputs," In Proceedings of the IEEE/CVF Conference on Computer Vision and Pattern Recognition, pp. 5174-5183, 2024.
\bibitem{b17} Oscar Debats Geert Litjens, Nico Karssemeijer
Jelle Barentsz, and Henkjan Huisman, “Prostatex
Challenge data,” in The Cancer Imaging Archive, 2017.
\bibitem{b18} A. Newell, K. Yang, and J. Deng. Stacked hourglass
networks for human pose estimation. arXiv preprint
arXiv:1603.06937, 2016.
\bibitem{b19} Fei Wang, Mengqing Jiang, Chen Qian, Shuo Yang, Cheng
Li, Honggang Zhang, Xiaogang Wang, and Xiaoou Tang.
"Residual attention network for image classification", In
CVPR, pp. 3156–3164, 2017.
\bibitem{20} S. Ramanarayanan, B. Murugesan, K. Ram, and M. Sivaprakasam, “DC-WCNN: A deep cascade of Wavelet based convolutional neural networks for MR image Reconstruction,” IEEE 19th International Symposium on Biomedical Imaging (ISBI), pp. 1069-1073, 2022.
\bibitem{21} Y. Beauferris et al., “Multi-Coil MRI Reconstruction Challenge—Assessing brain MRI reconstruction models and their generalizability to varying coil configurations,” Frontiers in Neuroscience, vol. 16, Jul. 2022.

% \bibitem{b20} J. Long, E. Shelhamer, and T. Darrell. Fully convolutional
% networks for semantic segmentation. In CVPR, 2015.



% \bibitem{b1} G. Eason, B. Noble, and I. N. Sneddon, ``On certain integrals of Lipschitz-Hankel type involving products of Bessel functions,'' Phil. Trans. Roy. Soc. London, vol. A247, pp. 529--551, April 1955.
% \bibitem{b2} J. Clerk Maxwell, A Treatise on Electricity and Magnetism, 3rd ed., vol. 2. Oxford: Clarendon, 1892, pp.68--73.
% \bibitem{b3} I. S. Jacobs and C. P. Bean, ``Fine particles, thin films and exchange anisotropy,'' in Magnetism, vol. III, G. T. Rado and H. Suhl, Eds. New York: Academic, 1963, pp. 271--350.
% \bibitem{b4} K. Elissa, ``Title of paper if known,'' unpublished.
% \bibitem{b5} R. Nicole, ``Title of paper with only first word capitalized,'' J. Name Stand. Abbrev., in press.
% \bibitem{b6} Y. Yorozu, M. Hirano, K. Oka, and Y. Tagawa, ``Electron spectroscopy studies on magneto-optical media and plastic substrate interface,'' IEEE Transl. J. Magn. Japan, vol. 2, pp. 740--741, August 1987 [Digests 9th Annual Conf. Magnetics Japan, p. 301, 1982].
% \bibitem{b7} M. Young, The Technical Writer's Handbook. Mill Valley, CA: University Science, 1989.
% \bibitem{b8} D. P. Kingma and M. Welling, ``Auto-encoding variational Bayes,'' 2013, arXiv:1312.6114. [Online]. Available: https://arxiv.org/abs/1312.6114
% \bibitem{b9} S. Liu, ``Wi-Fi Energy Detection Testbed (12MTC),'' 2023, gitHub repository. [Online]. Available: https://github.com/liustone99/Wi-Fi-Energy-Detection-Testbed-12MTC
% \bibitem{b10} ``Treatment episode data set: discharges (TEDS-D): concatenated, 2006 to 2009.'' U.S. Department of Health and Human Services, Substance Abuse and Mental Health Services Administration, Office of Applied Studies, August, 2013, DOI:10.3886/ICPSR30122.v2
% \bibitem{b11} K. Eves and J. Valasek, ``Adaptive control for singularly perturbed systems examples,'' Code Ocean, Aug. 2023. [Online]. Available: https://codeocean.com/capsule/4989235/tree
\end{thebibliography}

% \vspace{12pt}
% \color{red}
% IEEE conference templates contain guidance text for composing and formatting conference papers. Please ensure that all template text is removed from your conference paper prior to submission to the conference. Failure to remove the template text from your paper may result in your paper not being published.

\end{document}



\end{document}
