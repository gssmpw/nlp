\documentclass[conference]{IEEEtran}
\IEEEoverridecommandlockouts

\usepackage{cite}
\usepackage{amsmath,amssymb,amsfonts}
\usepackage{algorithmic}
\usepackage{graphicx}
\usepackage{textcomp}
\usepackage{xcolor}
\usepackage{graphics} %
\usepackage{epsfig} %
\usepackage{mathptmx} %
\usepackage{times} %
\usepackage[misc,geometry]{ifsym}
\usepackage{multirow}
\usepackage{hyperref}
\def\BibTeX{{\rm B\kern-.05em{\sc i\kern-.025em b}\kern-.08em
    T\kern-.1667em\lower.7ex\hbox{E}\kern-.125emX}}
\begin{document}

\title{MedConv: Convolutions Beat Transformers on Long-Tailed Bone Density Prediction}


\author{Xuyin Qi$^{1,2,*,\dag}$,
Zeyu Zhang$^{1,3,*,\dag,\ddag}$, %
Huazhan Zheng$^{4,*}$, %
Mingxi Chen$^{5}$,  %
Numan Kutaiba$^6$, %
Ruth Lim$^{6,7}$, %
Cherie Chiang$^7$,\\ %
Zi En Tham$^8$, %
Xuan Ren$^{2}$, %
Wenxin Zhang$^{9}$, %
Lei Zhang$^{9}$, %
Hao Zhang$^{9}$, %
Wenbing Lv$^{10}$, %
Guangzhen Yao$^{11}$, %
Renda Han$^{12}$,\\ %
Kangsheng Wang$^{13}$, %
Mingyuan Li$^{14}$, %
Hongtao Mao$^{15}$, %
Yu Li$^{16}$, %
Zhibin Liao$^2$, %
Yang Zhao$^{8,\text{\Letter}}$, %
Minh-Son To$^1$ %
\thanks{$^{*}$Equal Contribution. $^{\text{\Letter}}$Corresponding author: \href{mailto:y.zhao2@latrobe.edu.au}{y.zhao2@latrobe.edu.au}}
\thanks{$^\dag$Work done while Zeyu Zhang is a researcher assistant at Flinders University.}
\thanks{$^\ddag$Project lead.}%
\thanks{$^1$Flinders University $^2$The University of Adelaide $^3$The Australian National University $^4$Zhejiang University of Technology $^5$Guangdong Technion – Israel Institute of Technology $^6$Austin Health $^{7}$The University of Melbourne $^{8}$La Trobe University $^9$University of Chinese Academy of Sciences $^{10}$Yunnan University $^{11}$Northeast Normal University $^{12}$Hainan University $^{13}$Univeristy of Science and Technology Beijing $^{14}$Hebei University of Technology $^{15}$Central China Normal University $^{16}$Hubei University}}





\maketitle

\begin{abstract}
Bone density prediction via CT scans to estimate T-scores is crucial, providing a more precise assessment of bone health compared to traditional methods like X-ray bone density tests, which lack spatial resolution and the ability to detect localized changes. However, CT-based prediction faces two major challenges: the high computational complexity of transformer-based architectures, which limits their deployment in portable and clinical settings, and the imbalanced, long-tailed distribution of real-world hospital data that skews predictions. To address these issues, we introduce \textbf{MedConv}, a convolutional model for bone density prediction that outperforms transformer models with lower computational demands. We also adapt Bal-CE loss and post-hoc logit adjustment to improve class balance. Extensive experiments on our AustinSpine dataset shows that our approach achieves up to \textbf{21\%} improvement in accuracy and \textbf{20\%} in ROC AUC over previous state-of-the-art methods.
Code will be available at \url{https://github.com/Richardqiyi/MedConv}.
\end{abstract}

\begin{IEEEkeywords}
Osteopenia, Osteoporosis, Long-Tailed Distributions, Bone Density, T-Score.
\end{IEEEkeywords}

\section{Introduction}
Bone health, crucial for mobility, fracture prevention, and overall well-being, is particularly important for aging populations or those with osteoporosis, a common skeletal disease that compromises bone strength by causing low bone mass and microarchitectural deterioration. This condition, increasing the risk of fragility fractures from low-energy impacts, often affects critical areas like the spine, hip, and wrist, significantly reducing quality of life ~\cite{lupsa2015bone}. Predicting bone density through CT scans to estimate T-scores offers a more precise and detailed assessment of bone health compared to traditional methods like X-ray bone density tests, which have lower spatial resolution and limited ability to detect localized bone changes. CT-based assessments can measure volumetric bone mineral density (BMD) and provide three-dimensional imaging, allowing for a comprehensive evaluation of bone quality. Studies have demonstrated that deep learning models applied to CT images can accurately predict BMD and T-scores, enhancing the detection and management of osteoporosis ~\cite{sato2022deep}. Additionally, quantitative computed tomography (QCT) has been shown to be a superior method for diagnosing osteoporosis and predicting fractures when compared to dual-energy X-ray absorptiometry (DXA) ~\cite{mori2024advancing}. These advancements highlight the potential of CT imaging in providing detailed insights into bone health, surpassing the capabilities of traditional X-ray-based methods. Recent advances in representation learning ~\cite{ji2024sine} and dense prediction \cite{zhang2023thinthick,wu2023bhsd,tan2024segstitch,ge2024esa,zhang2024meddet,cai2024msdet,tan2024segkan,zhang2025gamed}, particularly in the domain of medical imaging ~\cite{zhang2024jointvit, wu2024xlip,hiwase2025can,zhao2024landmark,cai2024medical,qi2025projectedex}, have significantly enhanced the accuracy and automation of osteoporosis detection. These advancements facilitate early diagnosis and timely intervention, providing a foundation for more effective personalized treatment and prevention strategies.

\begin{figure}[t]
    \centering
    \includegraphics[width=\columnwidth]{SAC_new.png}
    \caption{Visualization of segmentation results on CT images. 
    The first column shows the original images. 
    The second column represents the segmentation results from CTSpine1K~\cite{deng2021ctspine1k}. 
    The third column displays the segmentation results from TotalSegmentator~\cite{wasserthal2023totalsegmentator}. 
    Rows correspond to different anatomical planes: the sagittal plane (S) in the first row, the axial plane (A) in the second row, and the coronal plane (C) in the third row. 
    The region highlighted in red corresponds to the L5 vertebra, which plays a crucial role in diagnosing conditions like osteoporosis.}
    \label{fig:segmentation}
\end{figure}

However, predicting bone density from CT scans poses two significant challenges that hinder the effective application of advanced deep learning models.
First, transformer-based architectures, which have gained popularity in recent years for their superior performance in various domains, tend to suffer from quadratic complexity in their self-attention mechanisms. This results in substantial computational demands, particularly for high-resolution medical images like CT scans, where the input size can be extremely large. Such resource-intensive requirements make these models inefficient for deployment on portable or edge devices, which are increasingly sought after in modern healthcare for their potential to enable point-of-care diagnostics. Furthermore, the computational burden limits their feasibility in real-world clinical practice, where rapid processing and cost-effectiveness are critical.
Second, real-world hospital data often exhibits an imbalanced, long-tailed distribution, heavily skewed toward more prevalent cases of osteoporosis while containing significantly fewer samples of less common conditions, such as borderline or early-stage bone density anomalies. This data imbalance poses a considerable challenge for model training, as standard machine learning algorithms tend to prioritize the majority class, leading to suboptimal performance in predicting rare cases. Addressing this issue requires sophisticated techniques, such as class rebalancing strategies, data augmentation, or the use of domain-specific loss functions, to ensure that models can achieve robust and fair predictions across the full spectrum of cases ~\cite{johnson2019survey}.

To adress these problems, our paper presents three main contributions:

\begin{itemize}
    \item We introduce \textbf{MedConv}, a robust model that revisits convolutional approaches for bone density prediction on spinal CT scans, achieving superior performance over transformer-based models with reduced computational complexity.
    \item To address the long-tailed prediction challenge, we customize Bal-CE loss and post-hoc logit adjustment for improved class balance and accuracy.
    \item We evaluated our methods through extensive experiments on our AustinSpine dataset, applying various preprocessing techniques, which yielded improvements of up to 21\% in accuracy and 20\% in ROC AUC compared with previous state-of-the-art methods.
\end{itemize}

\section{Related Work}

\subsection{Deep Learning for Bone Mineral Density Prediction}

The prediction of bone mineral density (BMD) and the evaluation of fracture risk through the application of deep learning techniques \cite{zhang2024deep} have garnered increasing attention in recent years. A notable study by Hsieh et al. (2021) \cite{hsieh2021automated} introduced an innovative approach that utilizes deep learning models applied to plain radiographs for the automated prediction of BMD and fracture risk assessment. Their method demonstrated highly promising results, achieving area under precision-recall curve (AUPRC) scores of 0.89 for hip osteoporosis and 0.83 for spine osteoporosis prediction. Furthermore, their model exhibited an impressive accuracy of 91.7\% in estimating the risk of hip fractures. Leveraging a large dataset comprising pelvis and lumbar spine radiographs, the study underscored the potential of deep learning in addressing osteoporosis detection, particularly in scenarios where dual-energy X-ray absorptiometry (DXA) remains underutilized.

In another significant contribution, Yasaka et al. (2020) \cite{yasaka2020prediction} explored the use of CT imaging for BMD prediction, employing a convolutional neural network (CNN) specifically designed to estimate lumbar vertebrae BMD from unenhanced CT scans. Their findings demonstrated a strong correlation between CNN-predicted BMD values and those obtained through DXA, achieving area under the receiver operating characteristic curve (AUC) scores of 0.965 and 0.970 for internal and external validation datasets, respectively. This study laid the groundwork for using CT imaging as an effective alternative to DXA in BMD prediction, illustrating the capability of CNN-based models to accurately capture bone density-related features.

\begin{figure}[t]
    \centering
    \includegraphics[width=0.8\linewidth]{Comparison_of_models.png}
    \caption{
        Comparison between 3D ResNet and 2D ResNet architectures for volumetric medical data processing. 
        The upper pipeline illustrates the 3D ResNet-based MedConv model, which leverages three-dimensional convolutions to capture spatial and contextual information across volumetric CT scans. 
        The inclusion of Bal-CE Loss further refines the model's focus on imbalanced data distributions, ensuring accurate predictions for the L1 vertebra segmentation task.
        Conversely, the lower pipeline showcases the standard 2D ResNet approach, where slices are treated independently without spatial continuity across adjacent slices, potentially limiting performance in tasks requiring volumetric context. 
        This figure highlights the architectural and methodological differences, emphasizing the advantages of 3D ResNet for tasks that demand structural and contextual understanding of medical images.}
    \label{fig:comparison_models}
\end{figure}

Building upon this research, Dagan et al. (2019) \cite{dagan2020automated} developed a model aimed at predicting fracture risk based on routine CT scans, particularly when DXA-derived data is unavailable. Their CT-based method demonstrated superior AUC scores and sensitivity compared to the FRAX tool when BMD inputs were excluded, indicating that CT scans can serve as a reliable resource for assessing fracture risk. This approach suggests that CT imaging could effectively compensate for the underutilization of DXA in clinical settings.

In another noteworthy study, González et al. (2018) \cite{gonzalez2018deep} proposed a direct image-to-biomarker prediction approach. By employing a deep learning regression model, they predicted BMD directly from CT scans. Their results highlighted the effectiveness of a single convolutional neural network in simultaneously segmenting relevant anatomical regions and predicting BMD values with high accuracy. This streamlined approach provides an efficient alternative to traditional methods that rely on separate segmentation and prediction steps.

Lastly, Fang et al. (2020) \cite{fang2021opportunistic} demonstrated the potential of multi-detector CT imaging for opportunistic osteoporosis screening. By combining U-Net for vertebral segmentation with DenseNet-121 for BMD estimation, their method achieved a strong correlation with quantitative computed tomography (QCT) benchmarks. This fully automated pipeline showcased the feasibility of integrating CT-derived BMD analysis into routine clinical practice for opportunistic screening. Their study highlighted how deep learning can facilitate cost-effective and automated osteoporosis detection in diverse healthcare environments.


\subsection{Addressing Long-Tailed Distribution in Classification Tasks}

Long-tailed distributions, characterized by a few dominant classes and a large number of underrepresented classes, pose significant challenges in classification tasks. These challenges arise due to the imbalance in the data distribution, which can lead to biased model predictions favoring majority classes while neglecting minority ones. Two widely adopted strategies for addressing this issue are resampling methods and balanced augmentation (BalAug), both of which aim to mitigate the effects of data imbalance by adjusting the training process.

Resampling methods involve manipulating the class distribution in the training dataset. Oversampling techniques, such as random duplication or Synthetic Minority Over-sampling Technique (SMOTE), increase the representation of minority classes, thereby providing the model with more exposure to these underrepresented categories. However, these approaches may lead to overfitting on the minority classes due to repeated exposure to the same data points. On the other hand, undersampling methods reduce the number of majority class samples to balance the dataset, but this can result in a loss of valuable information from the majority classes, as noted in ~\cite{bellinger2020remix}. Consequently, while resampling methods are straightforward and often effective, they require careful tuning to avoid introducing new challenges.

Balanced augmentation (BalAug) offers an alternative approach by integrating data augmentation techniques with class balancing. Augmentation strategies such as rotation, cropping, flipping, and other transformations are selectively applied to the minority classes, enhancing the diversity of training data for these underrepresented categories. For instance, ~\cite{cui2019class} introduced a class-balanced loss that dynamically weights samples based on their effective number, ensuring that the model learns equitably from all classes. Furthermore, advanced techniques like class-aware sampling combined with augmentation, as proposed in ~\cite{liu2022long}, have demonstrated improved performance on long-tailed datasets by carefully balancing the sampling probabilities and incorporating diverse transformations. These methods not only enrich the training data but also help the model generalize better to unseen data.

In addition to data-focused strategies, training optimization methods have emerged as powerful tools for addressing long-tailed distributions. Foret et al. (2020) ~\cite{foret2020sharpness} introduced Sharpness-Aware Minimization (SAM), a novel optimization approach designed to enhance model generalization by simultaneously minimizing the loss value and the sharpness of the loss landscape. SAM identifies parameter regions with consistently low loss, effectively mitigating overfitting and improving generalization, particularly in overparameterized models. Through rigorous evaluation on benchmark datasets like CIFAR ~\cite{krizhevsky2009learning} and ImageNet ~\cite{deng2009imagenet}, SAM demonstrated superior performance, excelling in robustness to label noise and training stability, making it a valuable addition to the arsenal of techniques for long-tailed datasets.

Building on these ideas, Fang et al. (2023) ~\cite{defazio2024road} proposed a schedule-free optimization framework to address long-tailed distributions by replacing traditional learning rate schedules with momentum-driven primal averaging. Their approach dynamically balances gradient updates, avoiding the gradient collapse often observed in imbalanced datasets. This innovative method achieved state-of-the-art results across various tasks, including CIFAR-10 and ImageNet, by combining robust convergence properties with efficient generalization capabilities. By reducing reliance on extensive hyperparameter tuning, this approach offers a practical solution for training on long-tailed data.

In summary, addressing the challenges posed by long-tailed distributions typically requires a combination of data-level and training-level strategies. Data-level approaches, such as resampling and balanced augmentation, aim to correct the imbalance in the dataset, ensuring that all classes are adequately represented during training. Training-level techniques, like SAM and schedule-free optimization, focus on improving model generalization by optimizing the training process itself. When these methods are combined effectively, they can complement each other, leveraging the strengths of both data and training interventions to achieve robust and unbiased performance on long-tailed datasets.



\section{Methodology}

\subsection{Overview}

\begin{figure}[h]
    \centering
    \includegraphics[width=\columnwidth]{main-graph.png}
    \caption{Architecture of the proposed MedConv model, based on a 3D ResNet-50 backbone. 
    The model leverages the volumetric spatial representation capabilities of 3D convolutions, essential for accurate bone density estimation. 
    Key methodologies include the use of Balanced Cross-Entropy (Bal-CE) loss and post-hoc logit adjustment with hyperparameters $\tau_1 = 1$ and $\tau_2 = 0.5$, which enhance class balance and calibration.}
    \label{fig:main-graph}
\end{figure}

Our proposed model is built upon a 3D ResNet-50 backbone, selected for its superior ability to capture the spatial and contextual information embedded in volumetric medical data. Unlike conventional 2D convolutional neural networks (CNNs) that process individual image slices independently, thereby neglecting depth information, the 3D ResNet-50 employs three-dimensional convolutional operations. This design enables the model to effectively encode spatial continuity within volumetric datasets such as CT scans, a critical aspect for accurate bone density prediction.

The architecture leverages residual connections to address the vanishing gradient problem, facilitating the training of deep networks while maintaining representational efficiency. Additionally, the bottleneck structure within the 3D ResNet-50 reduces computational overhead without compromising its capacity to model complex patterns inherent in high-resolution medical images.

While transformer-based architectures excel in capturing long-range dependencies and global contextual features, their computational complexity grows quadratically with input size. This limitation poses significant challenges for processing high-resolution volumetric data in resource-constrained settings. By contrast, the 3D ResNet-50 achieves an effective trade-off between computational efficiency and representational power, making it a practical and scalable choice for clinical applications.

This backbone forms the foundation of our model, providing a framework that balances accuracy and efficiency for the analysis of volumetric medical data. Its ability to integrate three-dimensional spatial information ensures robust performance, particularly in tasks requiring detailed structural understanding, such as bone density prediction.


\subsection{Balanced Cross-Entropy (Bal-CE) Loss}

To address the inherent challenges of class imbalance in bone density prediction, we adopt a Balanced Cross-Entropy (Bal-CE) loss function. Medical imaging datasets often exhibit a long-tailed distribution, with underrepresented classes being critical for diagnosis. The Bal-CE loss function is designed to emphasize these minority classes by assigning class-specific weights, \( w_i \), during training. Its formulation remains as follows:

\[
\mathcal{L}_{\text{Bal-CE}} = - \frac{1}{N} \sum_{i=1}^{N} w_i \left( y_i \log(\hat{y}_i) + (1 - y_i) \log(1 - \hat{y}_i) \right),
\]

where \( y_i \) and \( \hat{y}_i \) represent the ground truth labels and predicted probabilities, respectively. The weight \( w_i \) is dynamically computed based on the inverse frequency of each class, ensuring greater emphasis on minority classes. This targeted adjustment helps the model avoid bias toward majority classes, leading to more balanced and reliable predictions.

\subsection{Post-Hoc Logit Adjustment}

To further enhance model calibration and refine class probabilities, we introduce a post-hoc logit adjustment technique. This method applies temperature scaling to logits, fine-tuning the relative contributions of majority and minority classes. The adjusted probabilities are calculated as:

\[
\hat{y}_i = \frac{e^{z_i / \tau_1}}{e^{z_i / \tau_1} + e^{z_j / \tau_2}},
\]

where \( z_i \) and \( z_j \) denote the logits for classes \( i \) and \( j \), respectively. The temperature parameters \( \tau_1 = 1 \) and \( \tau_2 = 0.5 \) are empirically chosen to achieve an effective balance. The lower value of \( \tau_2 \) amplifies the influence of minority class logits, while \( \tau_1 \) maintains the contribution of majority classes. This mechanism mitigates the impact of class imbalance by reshaping the probability distribution, allowing the model to produce well-calibrated predictions.

The combination of the Bal-CE loss and logit adjustment strategies ensures that our model effectively learns from imbalanced datasets while maintaining robustness in clinical scenarios. Together, these methods address the challenges posed by uneven class distributions and improve the reliability of the system for bone density prediction tasks.



\section{Dataset and Evaluation Matrices}
\subsection{AustinSpine Dataset}

\begin{figure}[h]
    \centering
    \includegraphics[width=0.5\linewidth]{dist.png}
    \caption{Long-tailed distribution of T-score classifications within the AustinSpine dataset.}
    \label{fig:dist}
\end{figure}

The AustinSpine dataset is a clinically curated collection of spinal CT scans, comprising imaging data from 389 patients, obtained with full ethical approval. Bone density for each scan is quantified using T-scores, a standardized metric widely employed to assess bone health. To ensure the reliability and consistency of annotations, each T-score underwent a thorough review by at least two expert radiologists, significantly enhancing the dataset's inter-rater reliability. Based on the World Health Organization (WHO) criteria for bone mineral density (BMD) \cite{who1994assessment}, the T-scores are categorized into three distinct classes, as detailed in Table \ref{tab:Tscore}. The dataset distribution, visualized in Figure \ref{fig:dist}, reveals a pronounced long-tailed pattern, highlighting the predominance of normal cases relative to the other classifications. This clinically enriched dataset offers a robust and reliable resource for the development and validation of automated bone density prediction models, particularly within real-world clinical settings where precise and consistent annotations are critical.

\begin{table}
\centering
\caption{World Health Organization (WHO) criteria for classification of patients with bone mineral density (BMD) ~\cite{who1994assessment}.}
\resizebox{\columnwidth}{!}{%
\begin{tabular}{c|c|c}
    \hline
    \textbf{T-score Range} & \textbf{Condition} & \textbf{Description} \\
    \hline
    -4 to -2.5 & Osteoporosis & Porous bone that can lead to fractures \\
    -2.5 to -1 & Osteopenia & Low Bone Density \\
    -1 and above & Normal & As compared to an average 30-year-old \\
    \hline
\end{tabular}%
}
\label{tab:Tscore}
\end{table}

\subsection{Evaluation Matrices}

For a fair comparison, we evaluated each method's overall classification performance on the test set using accuracy and ROC AUC scores. Additionally, we assessed sensitivity and specificity to understand how effectively each model handles both minority and majority classes within the long-tailed AustinSpine dataset.

\section{Experiment}

Our experiment is based on CT segmentation technology, utilizing two mainstream segmentation algorithms: CTSpine1K~\cite{deng2021ctspine1k} and TotalSegmentator \cite{wasserthal2023totalsegmentator}. CTSpine1K is a large-scale spinal CT dataset containing 1005 scans with over 11,100 labeled vertebrae, designed to advance research on spine-related image analysis tasks. TotalSegmentator is a deep learning segmentation model capable of automatically segmenting 104 major anatomical structures in CT images, including organs\cite{zhang2024segreg}, bones, muscles, and vessels, with robustness and high accuracy.

We use these algorithms to segment the lumbar vertebra L1 as input. The L1 vertebra, located at the top of the lumbar spine, serves as a critical load-bearing structure, supporting the upper body's weight while allowing flexibility and movement. Its position between the thoracic spine and the lower lumbar vertebrae makes it vital for both structural stability and mobility. Furthermore, the bone mineral density (BMD) of the L1 vertebra plays a crucial role in assessing overall bone health, serving as a key indicator in the diagnosis of osteoporosis and the evaluation of fracture risk ~\cite{ramschutz2024cervicothoracic}.

Through comparative experiments, we found that the segmentation results based on TotalSegmentator consistently outperformed those achieved by CTSpine1K in overall performance. Therefore, we selected the segmentation outputs of TotalSegmentator as the input for MedConv.

\begin{figure}[h]
    \centering
    \includegraphics[width=0.8\linewidth]{Comparison_of_segmentators.png}
    \caption{
        Experiment pipeline for evaluating segmentation methods and their impact on downstream tasks. 
        This flowchart illustrates the comparison between CTSpine1K~\cite{deng2021ctspine1k} and TotalSegmentator~\cite{wasserthal2023totalsegmentator}, two widely used segmentation algorithms. 
        Both methods segment the L1 vertebra from input CT images, with the outputs subsequently processed by the MedConv module, 
        followed by post-hoc logits optimized with balanced cross-entropy loss. 
        TotalSegmentator was identified as the superior model, producing more robust and accurate segmentation results, which were selected as inputs for the MedConv module.}
    \label{fig:experiment_flowchart}
\end{figure}

\subsection{Comparative Study}

\begin{table}[htbp]
\vspace{-0.5cm}
    \centering
    \caption{Comparative performance of various models on the given metrics.}
    \resizebox{\linewidth}{!}{%
    \begin{tabular}{l|c|c|c|c|c}
        \hline
        \textbf{Model} & \textbf{Accuracy} & \textbf{Sensitivity} & \textbf{Specificity} & \textbf{F1 Score} & \textbf{ROC AUC} \\
        \hline
        resnet10t.c3\_in1k+pretrain & 58.97 & 58.97 & 79.49 & 59.46 & 67.90 \\
        resnet14t.c3\_in1k+pretrain & 56.41 & 56.41 & 78.21 & 56.53 & 70.12 \\
        resnet18.a1\_in1k & 48.72 & 48.72 & 74.36 & 44.87 & 65.06 \\
        resnet18.a1\_in1k+windows & 47.44 & 47.44 & 73.72 & 44.81 & 63.63 \\
        resnet18.a1\_in1k+balaug & 47.44 & 47.44 & 73.72 & 42.75 & 64.67 \\
        resnet18.a1\_in1k+pretrain & 62.82 & 62.82 & 81.41 & 62.34 & 74.51 \\
        resnet18.a1\_in1k+pretrain+balce & 62.82 & 62.82 & 81.41 & 62.28 & 75.79 \\
        resnet18.a1\_in1k+pretrain+balaug & 57.69 & 57.69 & 78.85 & 55.47 & 75.02 \\
        resnet18.a1\_in1k+pretrain+windows & 56.41 & 56.41 & 78.21 & 54.92 & 69.53 \\
        resnet18.a1\_in1k+pretrain+balaug+windows & 58.97 & 58.97 & 79.49 & 58.64 & 77.71 \\
        resnet34.a1\_in1k & 48.72 & 48.72 & 74.36 & 46.77 & 64.47 \\
        resnet34.a1\_in1k+balaug & 46.15 & 46.15 & 73.08 & 46.41 & 66.15 \\
        resnet34.a1\_in1k+windows & 48.72 & 48.72 & 74.36 & 47.33 & 63.93 \\
        resnet34.a1\_in1k+pretrain & 58.97 & 58.97 & 79.49 & 57.45 & 83.01 \\
        resnet34.a1\_in1k+pretrain+balce & 57.69 & 57.69 & 78.85 & 56.54 & 74.51 \\
        resnet34.a1\_in1k+pretrain+balaug & 56.41 & 56.41 & 78.21 & 56.34 & 73.69 \\
        resnet50.a1\_in1k & 44.87 & 44.87 & 72.44 & 41.66 & 59.94 \\
        resnet50.a1\_in1k+pretrain & 57.69 & 57.69 & 78.85 & 55.89 & 76.87 \\
        resnet50.a1\_in1k+pretrain+balaug & 55.13 & 55.13 & 77.56 & 53.52 & 70.81 \\
        resnet50.a1\_in1k+pretrain+balce & 64.10 & 64.10 & 82.05 & 65.14 & 78.43 \\
        resnet50.a1\_in1k+pretrain+balce+schdulefree & 57.69 & 57.69 & 78.85 & 57.40 & 71.40 \\
        resnet50.a1\_in1k+pretrain+balce+balaug & 62.82 & 62.82 & 81.41 & 61.13 & 76.53 \\
        resnet50.a1\_in1k+pretrain+balce+resample & 61.54 & 61.54 & 80.77 & 60.61 & 73.25 \\
        resnet50.a1\_in1k+pretrain+sam & 55.13 & 55.13 & 77.56 & 54.36 & 73.10 \\
        resnet50.a1\_in1k+pretrain+balce+sam & 56.41 & 56.41 & 78.21 & 55.82 & 73.22 \\
        resnet50.a1\_in1k+trainParams+balaug & 53.85 & 53.85 & 76.92 & 53.13 & 74.58 \\
        mobilenetv2\_100.ra\_in1k+pretrain & 52.56 & 52.56 & 76.28 & 51.89 & 69.26 \\
        mobilenetv2\_100.ra\_in1k+pretrain+balce & 56.41 & 56.41 & 78.21 & 55.45 & 71.52 \\
        efficientnet\_b0.ra\_in1k+pretrain & 57.69 & 57.69 & 78.85 & 56.02 & 74.14 \\
        efficientnet\_b0.ra\_in1k+pretrain+balce & 60.26 & 60.26 & 80.13 & 58.73 & 78.06 \\
        resnext50\_32x4d.a1h\_in1k+pretrain & 60.26 & 60.26 & 80.13 & 60.77 & 76.06 \\
        resnext50\_32x4d.a1h\_in1k+pretrain+balce & 55.13 & 55.13 & 77.56 & 52.64 & 71.28 \\
        resnext50\_32x4d.a1h\_in1k+pretrain+balaug & 50.00 & 50.00 & 75.00 & 46.06 & 62.15 \\
        resnet101.a1\_in1k+pretrain & 55.13 & 55.13 & 77.56 & 54.78 & 73.30 \\
        resnet101.a1\_in1k+pretrain+balce & 58.97 & 58.97 & 79.49 & 55.27 & 73.30 \\
        resnet101.a1\_in1k+pretrain+balce+sam & 55.13 & 55.13 & 77.56 & 47.70 & 71.06 \\
        resnet152.tv\_in1k+pretrain & 50.00 & 50.00 & 75.00 & 47.91 & 69.06 \\
        resnet152.tv\_in1k+pretrain+balce & 57.69 & 57.69 & 78.85 & 58.30 & 70.76 \\
        resnet152.tv\_in1k+pretrain+balce+sam & 52.56 & 52.56 & 76.28 & 48.72 & 69.82 \\ 
        ViT+pretrain & 33.54 & 33.54 & 66.77 & 17.93 & 58.35 \\ 
        JointViT + pretrain & 41.03 & 41.03 & 76.92 & 53.85 & 60.78 \\
        JointViT +pretrain + balce & 43.59 & 43.59 & 71.79 & 34.60 & 56.81 \\ \hline
        \textbf{MedConv (Ours)} & \textbf{65.38} & \textbf{65.38} & \textbf{82.69} & \textbf{66.37} & \textbf{79.34} \\
        \hline
    \end{tabular}}
    \label{tab:comparative_results}
\end{table}

In this comparative experiment, we evaluated various models based on their performance metrics, including accuracy, sensitivity, specificity, F1 score, and ROC AUC. All models were tested using the segmentation outputs of TotalSegmentator, which were selected due to their superior performance in our preliminary ablation studies.

The results indicate that our proposed MedConv model achieved the highest accuracy of 65.38, surpassing other models such as resnet50.a1 in1k+pretrain+balce, which scored 64.10, and resnet34.a1 in1k+pretrain, which achieved an accuracy of 58.97. This demonstrates the effectiveness of MedConv in handling complex medical imaging data.

In terms of sensitivity and specificity, the MedConv model demonstrated remarkable results, with scores of 65.38 and 82.69, respectively. These metrics highlight MedConv's capability to accurately identify positive cases while minimizing the occurrence of false positives. In comparison, the next highest sensitivity was achieved by resnet50.a1 in1k+pretrain+balce with a score of 64.10, emphasizing MedConv's superior ability to distinguish true positives and true negatives with greater precision.

The F1 score for MedConv is 66.37, further establishing its robustness in balancing precision and recall. This metric is particularly critical in medical applications where both false positives and false negatives can significantly affect diagnostic reliability. MedConv's performance in this regard surpasses many other models, reinforcing its suitability for high-stakes scenarios where precise predictions are essential.

The ROC AUC for MedConv is 79.34, reflecting its overall performance across various classification thresholds. This metric is crucial for clinical applications, where decision-making often relies on evaluating a model's behavior across different thresholds. MedConv's high ROC AUC score highlights its reliability and effectiveness in real-world medical applications.

\begin{table}[htbp]
    \centering
    \caption{Comparative performance of CTspine1K and TotalSegmentator with different inputs.}
    \resizebox{\linewidth}{!}{%
    \begin{tabular}{l|l|c|c|c|c|c}
        \hline
        \textbf{Model} & \textbf{Input} & \textbf{Accuracy} & \textbf{Sensitivity} & \textbf{Specificity} & \textbf{F1 Score} & \textbf{ROC AUC} \\
        \hline
        \multirow{2}{*}{mobilenetv2\_100.ra\_in1k} 
            & CTspine1K        & 39.74 & 39.74 & 69.87 & 47.71 & 36.03 \\ 
            & TotalSegmentator & 52.56 & 52.56 & 76.28 & 51.89 & 69.26 \\ 
        \hline
        \multirow{2}{*}{resnet18.a1\_in1k} 
            & CTspine1K        & 46.15 & 46.15 & 73.08 & 41.30 & 53.43 \\ 
            & TotalSegmentator & 48.72 & 48.72 & 74.36 & 44.87 & 65.06 \\ 
        \hline
        \multirow{2}{*}{resnet34.a1\_in1k} 
            & CTspine1K        & 42.31 & 42.31 & 71.15 & 36.01 & 53.60 \\ 
            & TotalSegmentator & 48.72 & 48.72 & 74.36 & 46.77 & 64.47 \\ 
        \hline
        \multirow{2}{*}{resnet50.a1\_in1k} 
            & CTspine1K        & 44.87 & 44.87 & 72.44 & 37.97 & 59.32 \\ 
            & TotalSegmentator & 44.87 & 44.87 & 72.44 & 41.66 & 59.94 \\ 
        \hline
        \multirow{2}{*}{resnet101.a1\_in1k} 
            & CTspine1K        & 43.59 & 43.59 & 71.79 & 34.97 & 57.91 \\ 
            & TotalSegmentator & 55.13 & 55.13 & 77.56 & 54.78 & 73.30 \\ 
        \hline
        \multirow{2}{*}{resnet152.tv\_in1k} 
            & CTspine1K        & 42.31 & 42.31 & 71.15 & 36.10 & 54.29 \\ 
            & TotalSegmentator & 50.00 & 50.00 & 75.00 & 47.91 & 69.06 \\ 
        \hline
    \end{tabular}}
    \label{tab:comparative_results0}
\end{table}

To ensure the robustness of our experimental results, we performed additional evaluations using the same models but with segmentation outputs generated by CTspine1K instead of TotalSegmentator. CTspine1K, a specialized segmentation tool for spine imaging, serves as an alternative segmentation source. However, as shown in Table~\ref{tab:comparative_results0}, models consistently underperformed when using CTspine1K outputs compared to those using TotalSegmentator. Key metrics, including accuracy, sensitivity, specificity, F1 score, and ROC AUC, exhibited significant declines across all models.
Notably, the results demonstrate that TotalSegmentator’s high-quality and comprehensive segmentation is pivotal for achieving superior model performance. Furthermore, when TotalSegmentator outputs were used, model performance either improved or remained stable as model parameter counts increased. This trend highlights the richness of the information provided by TotalSegmentator, which facilitates more effective utilization of complex model architectures.
Based on these findings, we selected TotalSegmentator as the default segmentation input source for all subsequent experiments to ensure consistency and optimize the models' potential.

In summary, these results demonstrate that the MedConv model not only outperforms other tested alternatives but also represents a significant advancement in model architecture for medical imaging tasks. By leveraging the high-quality segmentation results from TotalSegmentator, MedConv has shown exceptional accuracy, sensitivity, specificity, and overall robustness. These findings underscore the potential of MedConv to enhance diagnostic accuracy and improve patient outcomes, making it a promising tool for clinical and medical research applications.

\subsection{Ablation Study}

This section presents a comprehensive analysis of the proposed approach through three separate experiments. The first experiment focuses on validating the effectiveness of BalCE loss across different backbone architectures, highlighting its role in addressing class imbalance and improving overall performance. The second experiment investigates the influence of the hyperparameter \(\tau_1\), which balances the loss contribution from positive and negative samples, on key performance metrics. This analysis aims to identify the optimal value of \(\tau_1\) for achieving stable and robust performance. Finally, the third experiment evaluates the sensitivity of the model to variations in the hyperparameter \(\tau_2\), which serves as an ad-hoc weighting parameter within the MedConv framework. These experiments collectively underscore the robustness, adaptability, and fine-tuning flexibility of the proposed method in addressing challenges associated with class imbalance in medical imaging.

\subsubsection{Impact of BalCE Loss on Different Backbones}

\begin{table}[h!]
\centering
\caption{Ablation Study: Comparison of BalCE Loss Across Different Backbones}
\resizebox{\columnwidth}{!}{
\begin{tabular}{l|c|c|c|c|c}
\hline
\textbf{Model}          & \textbf{Accuracy}       & \textbf{Sensitivity}    & \textbf{Specificity}    & \textbf{F1 Score}       & \textbf{ROC AUC}        \\ \hline
mobilenetv2 w/o         & 52.56                  & 52.56                  & 76.28                  & 51.89                  & 69.26                  \\
mobilenetv2 w/          & 56.41 \textcolor{green}{(+3.85)} & 56.41 \textcolor{green}{(+3.85)} & 78.21 \textcolor{green}{(+1.93)} & 55.45 \textcolor{green}{(+3.56)} & 71.52 \textcolor{green}{(+2.26)} \\\hline
efficientnet w/o        & 57.69                  & 57.69                  & 78.85                  & 56.02                  & 74.14                  \\
efficientnet w/         & 60.26 \textcolor{green}{(+2.57)} & 60.26 \textcolor{green}{(+2.57)} & 80.13 \textcolor{green}{(+1.28)} & 58.73 \textcolor{green}{(+2.71)} & 74.14                  \\\hline
resnet34 w/o            & 58.97                  & 58.97                  & 79.49                  & 57.45                  & 83.01                  \\
resnet34 w/             & 57.69 \textcolor{red}{(-1.28)} & 57.69 \textcolor{red}{(-1.28)} & 78.85 \textcolor{red}{(-0.64)} & 56.54 \textcolor{red}{(-0.91)} & 74.51 \textcolor{red}{(-8.50)} \\\hline
resnet50 w/o            & 57.69                  & 57.69                  & 78.85                  & 55.89                  & 76.87                  \\
\textbf{resnet50 w/}             & \textbf{64.10} \textcolor{green}{(+6.41)} & \textbf{64.10} \textcolor{green}{(+6.41)} & \textbf{82.05} \textcolor{green}{(+3.20)} & \textbf{65.14} \textcolor{green}{(+9.25)} & \textbf{78.43} \textcolor{green}{(+1.56)} \\\hline
resnet101 w/o           & 55.13                  & 55.13                  & 77.56                  & 54.78                  & 73.30                  \\
resnet101 w/            & 58.97 \textcolor{green}{(+3.84)} & 58.97 \textcolor{green}{(+3.84)} & 79.49 \textcolor{green}{(+1.93)} & 55.27 \textcolor{green}{(+0.49)} & 73.30 \textcolor{red}{(+0.00)} \\\hline
resnet152 w/o           & 50.00                  & 50.00                  & 75.00                  & 47.91                  & 69.06                  \\
resnet152 w/            & 57.69 \textcolor{green}{(+7.69)} & 57.69 \textcolor{green}{(+7.69)} & 78.85 \textcolor{green}{(+3.85)} & 58.30 \textcolor{green}{(+10.39)} & 70.76 \textcolor{green}{(+1.70)} \\\hline
\end{tabular}
}
\label{tab:ablation}
\end{table}

The results of integrating BalCE loss across different backbones are summarized in Table \ref{tab:ablation}. Consistent performance improvements are observed across most architectures, with significant gains in accuracy, sensitivity, and F1 score. Notably, ResNet50 exhibits the highest improvement, achieving a 6.41\% increase in accuracy and a 9.25\% increase in F1 score. These findings underscore the effectiveness of BalCE loss in addressing data imbalance, particularly in challenging medical imaging scenarios. However, a minor performance drop is noted in ResNet34, potentially due to overfitting or incompatibility between the backbone and loss function.

\subsubsection{Effect of \(\tau_1\) on Model Performance}

\begin{table}[h]
\centering
\caption{Ablation study of different \(\tau_1\) hyperparameter settings and their impact on model performance metrics.}
\resizebox{\columnwidth}{!}{
\begin{tabular}{c|c|c|c|c|c}
\hline
\textbf{$\tau_1$} & \textbf{Accuracy} & \textbf{Sensitivity} & \textbf{Specificity} & \textbf{F1} & \textbf{AUC} \\\hline
0     & 57.69   & 57.69   & 78.85   & 55.89   & 76.87   \\
0.25  & 58.97 {\color{green}(+1.28)} & 58.97 {\color{green}(+1.28)} & 79.49 {\color{green}(+0.64)} & 59.25 {\color{green}(+3.36)} & 75.42 {\color{red}(-1.45)} \\
0.5   & 57.69 {\color{green}(+0.00)} & 57.69 {\color{green}(+0.00)} & 78.85 {\color{green}(+0.00)} & 57.77 {\color{green}(+1.88)} & 78.33 {\color{green}(+1.46)} \\
0.65  & 58.97 {\color{green}(+1.28)} & 58.97 {\color{green}(+1.28)} & 79.49 {\color{green}(+0.64)} & 58.07 {\color{green}(+2.18)} & 70.69 {\color{red}(-6.18)} \\
0.75  & 61.54 {\color{green}(+3.85)} & 61.54 {\color{green}(+3.85)} & 80.77 {\color{green}(+1.92)} & 61.61 {\color{green}(+5.72)} & 77.12 {\color{green}(+0.25)} \\
0.85  & 61.54 {\color{green}(+3.85)} & 61.54 {\color{green}(+3.85)} & 80.77 {\color{green}(+1.92)} & 60.12 {\color{green}(+4.23)} & 74.11 {\color{red}(-2.76)} \\
0.9   & 58.97 {\color{green}(+1.28)} & 58.97 {\color{green}(+1.28)} & 79.49 {\color{green}(+0.64)} & 57.83 {\color{green}(+1.94)} & 74.38 {\color{red}(-2.49)} \\
0.92  & 52.56 {\color{red}(-5.13)} & 52.56 {\color{red}(-5.13)} & 76.28 {\color{red}(-2.57)} & 50.29 {\color{red}(-5.60)} & 70.88 {\color{red}(-5.99)} \\
0.95  & 61.54 {\color{green}(+3.85)} & 61.54 {\color{green}(+3.85)} & 80.77 {\color{green}(+1.92)} & 60.66 {\color{green}(+4.77)} & 74.46 {\color{red}(-2.41)} \\
0.96  & 56.41 {\color{red}(-1.28)} & 56.41 {\color{red}(-1.28)} & 78.21 {\color{red}(-0.64)} & 57.12 {\color{green}(+1.23)} & 73.40 {\color{red}(-3.47)} \\
0.97  & 57.69 {\color{green}(+0.00)} & 57.69 {\color{green}(+0.00)} & 78.85 {\color{green}(+0.00)} & 56.52 {\color{green}(+0.63)} & 69.30 {\color{red}(-7.57)} \\
0.99  & 53.85 {\color{red}(-3.84)} & 53.85 {\color{red}(-3.84)} & 76.92 {\color{red}(-1.93)} & 50.72 {\color{red}(-5.17)} & 70.09 {\color{red}(-6.78)} \\
0.995 & 58.97 {\color{green}(+1.28)} & 58.97 {\color{green}(+1.28)} & 79.49 {\color{green}(+0.64)} & 59.49 {\color{green}(+3.60)} & 72.61 {\color{red}(-4.26)} \\
0.999 & 60.26 {\color{green}(+2.57)} & 60.26 {\color{green}(+2.57)} & 80.13 {\color{green}(+1.28)} & 59.36 {\color{green}(+3.47)} & 78.43 {\color{green}(+1.56)} \\
\textbf{1}     & \textbf{64.10} {\color{green}(+6.41)} & \textbf{64.10} {\color{green}(+6.41)} & \textbf{82.05} {\color{green}(+3.20)} & \textbf{65.14} {\color{green}(+9.25)} & \textbf{78.43} {\color{green}(+1.56)} \\
1.1   & 57.69 {\color{green}(+0.00)} & 57.69 {\color{green}(+0.00)} & 78.85 {\color{green}(+0.00)} & 57.88 {\color{green}(+1.99)} & 74.98 {\color{red}(-1.89)} \\
1.5   & 56.41 {\color{red}(-1.28)} & 56.41 {\color{red}(-1.28)} & 78.21 {\color{red}(-0.64)} & 56.45 {\color{red}(-0.56)} & 75.76 {\color{red}(-1.11)} \\
2     & 52.56 {\color{red}(-5.13)} & 52.56 {\color{red}(-5.13)} & 76.28 {\color{red}(-2.57)} & 45.26 {\color{red}(-10.63)} & 74.73 {\color{red}(-2.14)} \\\hline
\end{tabular}
}
\label{tab:ablation_tau1}
\end{table}

The ablation study under the default condition of \(\tau_1 = \tau_2\) demonstrates that the model achieves its best performance when \(\tau_1 = 1\). This setting effectively balances the loss function, addressing the challenges posed by class imbalance and enhancing model robustness. The analysis confirms that \(\tau_1 = 1\) is the optimal choice for achieving a stable trade-off across performance metrics, providing a strong baseline for further exploration. Subsequently, additional ablations focus on varying \(\tau_2\) while keeping \(\tau_1\) fixed at its optimal value, allowing for a more comprehensive evaluation of the proposed approach.




\subsubsection{Effect of \(\tau_2\) on Model Performance}

We further evaluate the impact of varying the hyperparameter \(\tau_2\) on model performance. This experiment leverages segmentations generated by TotalSegmentator as input, exploring the sensitivity of key performance metrics to changes in \(\tau_2\).

\begin{table}[h]
\centering
\caption{Ablation study of different \(\tau_2\) hyperparameter settings and their impact on model performance metrics.}
\begin{tabular}{c|c|c|c|c|c}
\hline
\textbf{$\tau_2$} & \textbf{Accuracy} & \textbf{Sensitivity} & \textbf{Specificity} & \textbf{F1} & \textbf{AUC} \\\hline
1.0   & 0.6410 & 0.6410 & 0.8205 & 0.6514 & 0.7843 \\
0.9   & 0.6410 & 0.6410 & 0.8205 & 0.6514 & 0.7870 \\
0.8   & 0.6410 & 0.6410 & 0.8205 & 0.6514 & 0.7877 \\
0.7   & 0.6410 & 0.6410 & 0.8205 & 0.6514 & 0.7894 \\
0.6   & 0.6538 & 0.6538 & 0.8269 & 0.6637 & 0.7919 \\
\textbf{0.5}   & \textbf{0.6538} & \textbf{0.6538} & \textbf{0.8269} & \textbf{0.6637} & \textbf{0.7934} \\
0.4   & 0.6410 & 0.6410 & 0.8205 & 0.6493 & 0.7951 \\
0.3   & 0.6410 & 0.6410 & 0.8205 & 0.6493 & 0.7961 \\
0.2   & 0.6282 & 0.6282 & 0.8141 & 0.6359 & 0.7971 \\
0.1   & 0.6282 & 0.6282 & 0.8141 & 0.6337 & 0.7986 \\\hline
\end{tabular}
\label{tab:ablation_tau2}
\end{table}

\begin{figure}[h]
\centering
\includegraphics[width=\columnwidth]{tau2.png}
\caption{Ablation study showing the impact of different $\tau_2$ settings on model performance metrics. Each line represents a distinct metric: Accuracy, Sensitivity, Specificity, F1 Score, and AUC.}
\label{fig:ablation_tau2}
\end{figure}

As shown in Table \ref{tab:ablation_tau2} and Figure \ref{fig:ablation_tau2}, the model maintains stable performance for \(\tau_2\) values between 1.0 and 0.7, with accuracy, sensitivity, and specificity hovering around 64.10\%. A significant improvement is observed at \(\tau_2 = 0.6\) and \(\tau_2 = 0.5\), where accuracy rises to 65.38\% and F1 score reaches 66.37\%. This suggests that moderate \(\tau_2\) values balance precision and recall effectively.

When \(\tau_2\) is reduced further, a decline in performance becomes evident. At \(\tau_2 = 0.1\), accuracy drops to 62.82\%, with corresponding decreases in sensitivity and specificity. These findings highlight the importance of tuning \(\tau_2\) to achieve optimal results, emphasizing its role in improving model robustness and generalization.


\section{Conclusion}

In this study, we introduced MedConv, a convolutional neural network designed for bone density prediction via CT scans. MedConv outperforms transformer-based methods in accuracy, sensitivity, and specificity, while maintaining a significantly lower computational cost. By employing a 3D ResNet-50 backbone, the model effectively captures volumetric spatial information, which is critical for precise bone health assessment. This capability enables MedConv to be more suited for practical applications in clinical and resource-constrained settings compared to transformer models.

To address the inherent challenges of imbalanced and long-tailed datasets in real-world medical imaging, we adopted a Balanced Cross-Entropy (Bal-CE) loss function combined with post-hoc logit adjustment techniques. These strategies demonstrated robust improvements in classification accuracy and model calibration, as evidenced by the performance gains observed in our experiments on the AustinSpine dataset. Specifically, MedConv achieved a 21\% improvement in classification accuracy and a 20\% increase in ROC AUC compared to prior state-of-the-art methods, solidifying its position as a benchmark in this domain.

The ablation studies further emphasized the importance of hyperparameter tuning in optimizing model performance. For the logit adjustment hyperparameter $\tau_1$, the results indicate that the optimal setting of $\tau_1 = 1$ provides a balanced trade-off across various performance metrics, achieving the highest accuracy and F1 score. Similarly, an ad-hoc analysis of $\tau_2$ revealed that moderate values, particularly $\tau_2 = 0.5$, yielded significant performance gains. The model exhibited improved robustness and generalization at $\tau_2 = 0.5$, with accuracy increasing to 65.38\% and F1 score reaching 66.37\%. This suggests that $\tau_2$ plays a crucial role in calibrating the relative contributions of minority and majority classes, thereby enhancing overall performance.

Additionally, the study underscores the importance of high-quality segmentation tools such as TotalSegmentator, which played a pivotal role in enhancing the overall performance of MedConv. The segmentation outputs from TotalSegmentator provided superior input quality, enabling MedConv to better leverage the volumetric spatial information for accurate predictions.

MedConv’s success in balancing computational efficiency and predictive performance highlights its potential for broader applications in clinical settings, where timely and accurate diagnoses are imperative. Future work may explore extending MedConv to other imaging modalities and clinical tasks, as well as further refining its architecture to enhance versatility and scalability in diverse healthcare environments. By bridging the gap between advanced deep learning techniques and practical deployment, MedConv sets a promising foundation for improved diagnostic tools in the fight against osteoporosis and other bone health conditions.








































\documentclass[conference]{IEEEtran}
%\documentclass[letterpaper, 10 pt, conference]{ieeeconf}
\IEEEoverridecommandlockouts
% The preceding line is only needed to identify funding in the first footnote. If that is unneeded, please comment it out.
%Template version as of 6/27/2024

\usepackage{cite}
\usepackage{amsmath,amssymb,amsfonts}
\usepackage{algorithmic}
\usepackage{graphicx}
\usepackage{textcomp}
\usepackage{xcolor}
\usepackage{array}
\usepackage{hyperref}
\usepackage{booktabs}
\def\BibTeX{{\rm B\kern-.05em{\sc i\kern-.025em b}\kern-.08em
    T\kern-.1667em\lower.7ex\hbox{E}\kern-.125emX}}
\begin{document}
\title{An Automated Machine Learning Framework for Surgical Suturing Action Detection under Class Imbalance\\
%{\footnotesize \textsuperscript{*}Note: Sub-titles are not captured for https://ieeexplore.ieee.org  and
%should not be used}

\thanks{This research was fully funded by EPSRC, UK. With the Grant Reference EP/Y017307/1.}
\thanks{* Corresponding Author}
}

\author{\IEEEauthorblockN{1\textsuperscript{st} Baobing Zhang*}
\IEEEauthorblockA{\textit{School of Engineering and Physical Sciences} \\
\textit{Heriot Watt University}\\
Edinburgh EH14 4AS, UK \\
B.Zhang@hw.ac.uk
}

\and
\IEEEauthorblockN{2\textsuperscript{st} Paul Sullivan}
\IEEEauthorblockA{\textit{School of Engineering and Physical Sciences} \\
	\textit{Heriot Watt University}\\
	Edinburgh EH14 4AS, UK \\
	P.Sulliva@hw.ac.uk
}

\and
\IEEEauthorblockN{3\textsuperscript{st} Benjie Tang}
\IEEEauthorblockA{\textit{Surgical Skills Centre, Dundee Institute for Healthcare
		Simulation} \\
\textit{Ninewells Hospital and Medical School, University
	of Dundee}\\
Dundee, UK \\
b.tang@dundee.ac.uk}

\and
\IEEEauthorblockN{4\textsuperscript{st} Ghulam Nabi}
\IEEEauthorblockA{\textit{Surgical Skills Centre, Dundee Institute for Healthcare
		Simulation} \\
	\textit{Ninewells Hospital and Medical School, University
		of Dundee}\\
	Dundee, UK \\
	g.nabi@dundee.ac.uk}
	
\and
\IEEEauthorblockN{5\textsuperscript{st} Mustafa Suphi Erden*}
\IEEEauthorblockA{\textit{School of Engineering and Physical Sciences} \\
	\textit{Heriot Watt University}\\
	Edinburgh EH14 4AS, UK \\
	@hw.ac.uk
}

%5\textsuperscript{st}

%\and
%\IEEEauthorblockN{6\textsuperscript{th} Given Name Surname}
%\IEEEauthorblockA{\textit{dept. name of organization (of Aff.)} \\
%\textit{name of organization (of Aff.)}\\
%City, Country \\
%email address or ORCID}
}

%\author{\IEEEauthorblockN{Anonymous Authors}}

\maketitle

\begin{abstract}
In laparoscopy surgical training and evaluation, real-time detection of surgical actions with interpretable outputs is crucial for automated and real-time instructional feedback and skill development. Such capability would enable development of machine guided training systems. This paper presents a rapid deployment approach utilizing automated machine learning methods, based on surgical action data collected from both experienced and trainee surgeons. The proposed approach effectively tackles the challenge of highly imbalanced class distributions, ensuring robust predictions across varying skill levels of surgeons. Additionally, our method partially incorporates model transparency, addressing the reliability requirements in medical applications. Compared to deep learning approaches, traditional machine learning models not only facilitate efficient rapid deployment but also offer significant advantages in interpretability. Through experiments, this study demonstrates the potential of this approach to provide quick, reliable and effective real-time detection in surgical training environments. 
\end{abstract}

\begin{IEEEkeywords}
Bayesian optimization, reliable, bayesian learning, probabilistic models, trustworthy.
\end{IEEEkeywords}

\section{INTRODUCTION}


Laparoscopic surgery, as a minimally invasive technique, is now considered a crucial choice in modern surgery due to its advantages, such as minimal trauma and faster recovery \cite{madhok2022safety, basunbul2022recent}. However, the unique perspective and operation mode of laparoscopic surgery place higher demands on surgeons' hand-eye coordination and spatial awareness, where the precision of the operation directly impacts surgical quality and patient recovery outcomes \cite{stulberg2020association, sanchez2020comparative}. To improve surgeons' laparoscopic skills, laparoscopic surgical training platforms have been developed. These platforms simulate real surgical environments, providing trainees with a safe space to practice \cite{hong2021simulation, van2020bimanual}. However, most existing platforms rely on fixed training steps and manual feedback, lacking intelligent real-time feedback mechanisms, and are unable to dynamically track trainees' movements. A surgical action recognition system, however, has the potential to monitor and analyze trainees' actions in real-time, providing immediate feedback during training, which can help trainees correct errors promptly and improve training efficiency and accuracy \cite{hashimoto2018artificial}. Such real-time feedback is crucial for provision of automated and real-time feedback, to prepare the next-generation laparoscopy training systems that eliminate the need of expert monitoring \cite{larsen2009effect}.

In practical applications, surgical action recognition systems face several major challenges. First, class imbalance is a common issue in medical data, especially in surgical data, where key actions often have fewer samples, while routine actions are abundant. This imbalance can lead to model bias toward majority classes, diminishing the recognition performance for minority classes \cite{krawczyk2016learning, he2009learning, johnson2019survey}. Second, model interpretability is essential in medical applications. Surgeons and trainers require not only a highly accurate model but also one with a transparent decision-making process to ensure the rationality and credibility of the model's outputs \cite{rudin2019stop, tjoa2020survey}. However, current deep learning \cite{lecun2015deep} models often have complex decision processes and lack interpretability and reliability, while statistical model-based decision making process which provides thorough interpretability and formal reliability guarantee \cite{zhang2024bayesian}. In addition, model stability and robustness are critical for the successful application of a surgical action recognition system. With the presence of both complexity and diversity in medical data, models may exhibit inconsistent performance across different data and environments, making it essential for the system to maintain stability and reliability across varied scenarios \cite{esteva2021deep, topol2019high}. Lastly, real-time feedback is indispensable in surgical training, as an efficient surgical action recognition system must process actions in real-time to provide immediate feedback on the quality of trainees' movements \cite{aggarwal2010training, sinha2023current}.

This study aims to propose an efficient surgical action recognition system to address the aforementioned challenges, with particular focus on handling class imbalance, enhancing model interpretability and stability, and achieving real-time feedback. To this end, we incorporate automated machine learning technology, by using automated machine learning to automatically construct and optimize models, thereby avoiding the complex process of manual tuning, while enhancing model stability and generalization through ensemble learning methods. In model construction, we combine techniques such as sample resampling and weighted classification to address class imbalance, ensuring the significance of minority classes in model training. Moreover, to improve system transparency and interpretability, we utilize traditional machine learning algorithms and interpretability techniques, making the decision process understandable for both surgeons and trainees. Finally, we designed and implemented a real-time surgical action recognition system capable of timely action recognition during training, which can be used in an intelligent and dynamic evaluation tool for laparoscopic surgical training platforms. 

This paper is structured in the following manner: Section 2 reviews prior research, focusing on advancements in surgical action recognition, class imbalance handling, interpretability, and real-time feedback. Section 3 details the methodology for laparoscopic surgical action recognition, including class imbalance handling, hyperparameter optimization, and ensemble learning. Section 4 provides an overview of the experimental configuration and findings, analyzing the system's accuracy, stability, and feedback efficiency on surgical training data. Section 5 concludes with a summary of the main contributions and discusses the potential applications and future directions of this system in surgical training.



\section{Related Work}


Laparoscopy surgical action recognition is a crucial research area in surgical training, for provision of real-time feedback to trainees during procedures. Lea et al \cite{lea2016segmental}. proposed Segmental Spatiotemporal Convolutional Neural Networks (CNNs) for capturing fine-grained temporal information to perform action segmentation, achieving significant results in surgical action recognition. Similarly, Twinanda et al \cite{twinanda2016endonet}. developed EndoNet, an advanced neural network architecture specifically designed for laparoscopic surgical videos, capable of performing multitask recognition, including action classification. The multitask structure of EndoNet effectively improves the recognition of complex actions, offering a novel approach for action detection in laparoscopic videos. In surgical data processing, class imbalance is a common phenomenon. Chawla et al \cite{chawla2002smote}. proposed SMOTE (Synthetic Minority Over-sampling Technique), which generates synthetic minority samples to balance the dataset distribution and improve recognition of minority class actions. Additionally, the survey by He and Garcia \cite{he2009learning} provides a comprehensive review of various techniques for handling class imbalance, including oversampling, undersampling, and weighted methods, which play a critical role in enhancing minority class recognition.

In medical applications, model interpretability is essential for machine learning model adoption. Rudin \cite{rudin2019stop} emphasized the necessity of using interpretable models in high-stakes domains and recommended prioritizing interpretable models over black-box models to ensure higher transparency for clinicians and trainers. Moreover, the work by Doshi-Velez and Kim \cite{doshi2017towards} explores various techniques for enhancing model interpretability, presenting a framework for applying interpretable models in medical applications, providing theoretical support for the use of interpretable models in clinical scenarios. Regarding model stability, Breiman's random forest algorithm \cite{breiman2001random}, as an ensemble learning method, enhances model stability and generalization by combining multiple decision trees, making it widely applicable to medical data processing where high stability is required. To further improve model stability and automate the optimization process, Feurer et al. \cite{feurer2015efficient} developed an automated machine learning system that combines automated model selection and hyperparameter tuning, effectively reducing the complexity of manual tuning and improving model adaptability and stability on surgical data. In surgical training, real-time feedback systems assist trainees by providing immediate feedback during procedures. Salvador et al \cite{salvador2024effects} investigated the application of real-time visual feedback in laparoscopic training, examining its impact on novices' learning curves. The results showed that trainees receiving real-time feedback demonstrated higher precision in tissue handling skills, significantly shortening their learning curves. The introduction of real-time feedback enabled trainees to master essential skills more quickly, improving training efficiency.


In summary, current laparoscopy surgical action recognition systems, class imbalance handling techniques, model interpretability methods, stability-enhancing techniques, and real-time feedback systems have shown progress in surgical training, yet limitations remain. Addressing these gaps, this study aims to optimize surgical action recognition systems using automated machine learning, combining multiple techniques to improve model performance in accuracy, interpretability, stability, and real-time functionality, providing a more intelligent and efficient feedback mechanism for surgical training.


\begin{figure*}[h!]
	\centering
	\includegraphics[width=\textwidth]{framework.png}
	\caption{Overall AutoML workflow including meta learning warmstart for bayesian optimization efficient model selection and ensemble building for laparoscopy surgical suturing action detection. 
	}
	\label{framework}
\end{figure*}



\section{Method}

%To address the challenges of processing complex data in the medical field, we utilized the automated machine learning method, which combines imbalance handling, hyperparameter optimization, and ensemble learning to construct a highly adaptive and reliable classification model.




%\section*{Functionality for Handling Class Imbalance}


%
%
%
%\section*{Hyperparameter Optimization: Bayesian Optimization}
%
%  Bayesian optimization is widely used in hyperparameter tuning for automatically selecting the optimal combination of hyperparameters. The core idea of Bayesian optimization is to use a surrogate model, typically a Gaussian process \cite{bergstra2011algorithms}, to approximate the hyperparameter space, allowing for efficient selection of the next optimal hyperparameter combination. In each iteration, the surrogate model models the predicted value and uncertainty of the current hyperparameter combination, gradually narrowing down the search space.
%
%In automated machine learning, Bayesian optimization combines the surrogate model with an acquisition function (e.g., Expected Improvement \cite{zhan2020expected}) to select hyperparameters by maximizing the acquisition function. The surrogate model, typically a Gaussian process, provides predicted mean and variance for each hyperparameter combination, effectively balancing exploration and exploitation.
%
%
%Suppose the objective function is \( f(\theta) \), where \( \theta \) represents the model's hyperparameters, Bayesian optimization aims to find the optimal hyperparameter combination \( \theta^* \) as:
%\begin{equation}
%	\theta^* = \arg \max_{\theta} f(\theta)
%\end{equation}
%The Gaussian process models \( f(\theta) \) as a normally distributed random variable with mean \( \mu(\theta) \) and variance \( \sigma^2(\theta) \). The acquisition function Expected Improvement can be expressed as:
%\begin{equation}
%	\text{EI}(\theta) = \mathbb{E}[\max(0, f(\theta) - f(\theta^+))]
%\end{equation}
%where \( f(\theta^+) \) is the objective function value of the current best solution.
%
%\section*{Model Selection and Ensemble Learning}
%
%In classification tasks, ensemble learning \cite{dong2020survey} combines multiple models to enhance overall model performance and stability. The model selection and ensemble mechanisms in automated machine learning automatically select top-performing models and combine them to form a weighted average or voting ensemble model, reducing bias and variance from individual models.
%
%Specifically, the weighted averaging method assigns different weights to each model based on its performance, with predictions summed according to their weights. In the voting method \cite{leon2017evaluating}, the final output is determined by selecting the most frequently predicted class across models. This ensemble approach shows strong generalization capabilities, especially when handling medical data.
%
%Suppose there are \( K \) models, with each model's prediction represented by \( \hat{y}_k \) and weight \( w_k \). The weighted average ensemble prediction \( \hat{y} \) is:
%\begin{equation}
%	\hat{y} = \sum_{k=1}^{K} w_k \cdot \hat{y}_k
%\end{equation}

The focus of this study is on surgical action recognition within the context of laparoscopy training and evaluation. This task aims to generate accurate predictions for surgical actions from trajectory data in real-time, leveraging automated machine learning techniques to minimize the need for human intervention. In practical applications, the computational budget is defined by the specific requirements of laparoscopy training systems, including constraints on CPU time, memory usage, and latency, ensuring compatibility with the real-time feedback needs of surgical training platforms. These constraints are critical for deploying models in training environments where quick and reliable action recognition is essential for effective evaluation and feedback. Specifically, the automated machine learning for surgical action recognition can be formulated in the following manner:

Let \( i = 1, \ldots, n + m \), where \( x_i \in \mathbb{R}^d \) denotes a feature vector and \( y_i \in Y \) represents a paired response label. Given a training dataset \( D_{\text{train}} = \{(x_1, y_1), \ldots, (x_n, y_n)\} \) and a test dataset \( D_{\text{test}} = \{(x_{n+1}, y_{n+1}), \ldots, (x_{n+m}, y_{n+m})\} \), where samples are obtained from a consistent data structure, the goal is to derive automated predictions \( \hat{y}_{n+1}, \ldots, \hat{y}_{n+m} \) for the test set, given a loss function \( \mathcal{L}(\cdot, \cdot) \) and a resource budget. The loss of the solution is defined as \( \frac{1}{m} \sum_{j=1}^{m} \mathcal{L}(\hat{y}_{n+j}, y_{n+j}) \).


%We employ the Auto-sklearn \cite{feurer2015efficient} method to accomplish the task of laparoscopic suturing action recognition. Our dataset comprises extensive two-hand coordinate data collected from both experienced surgeons and interns during suturing procedures. Inspired by the AutoML approach introduced in the Auto-sklearn framework, we utilize Bayesian optimization and automated model ensemble construction to improve model efficiency and robustness.

The formal definition of automated machine learning (AutoML) as a Combined Algorithm Selection and Hyperparameter optimization (CASH) problem was introduced by the AutoML approach in the AUTO-WEKA \cite{hall2009weka} system. In AutoML, a single machine learning algorithm may not always perform optimally across different datasets, making algorithm selection and hyperparameter optimization essential components of an AutoML system. The CASH problem integrates model selection and hyperparameter optimization into a single unified optimization task that can be solved using Bayesian optimization, aiming to identify the optimal algorithm and hyperparameter settings that minimize the loss function on a given dataset.

For the definition of CASH, let a set of algorithms \( \mathcal{A} = \{ A^{(1)}, \dots, A^{(R)} \} \), where each algorithm \( A^{(j)} \) has its own hyperparameter space \( \Lambda^{(j)} \). Given a training dataset \( D_{\text{train}} = \{ (x_1, y_1), \dots, (x_n, y_n) \} \), we partition it into \( K \) cross-validation folds, consisting of validation sets \( \{ D_{\text{valid}}^{(1)}, \dots, D_{\text{valid}}^{(K)} \} \) and corresponding training subsets \( \{ D_{\text{train}}^{(1)}, \dots, D_{\text{train}}^{(K)} \} \), where \( D_{\text{train}}^{(i)} = D_{\text{train}} \setminus D_{\text{valid}}^{(i)} \) for \( i = 1, \dots, K \). The loss function is donated as \( \mathcal{L}(A_{\lambda}^{(j)}, D_{\text{train}}^{(i)}, D_{\text{valid}}^{(i)}) \), representing the loss of algorithm \( A^{(j)} \) with hyperparameters \( \lambda \) when evaluated on \( D_{\text{valid}}^{(i)} \) and trained on \( D_{\text{train}}^{(i)} \). In the context of the CASH framework, the goal is to determine the best combination of model selection and hyperparameter tuning, aiming to reduce the average validation error across all cross-validation folds, formulated as:
\[
A^*, \lambda^* = \arg\min_{A^{(j)} \in \mathcal{A}, \lambda \in \Lambda^{(j)}} \frac{1}{K} \sum_{i=1}^{K} \mathcal{L}(A_{\lambda}^{(j)}, D_{\text{train}}^{(i)}, D_{\text{valid}}^{(i)}).
\]


The CASH problem is mathematically formulated as minimizing the average loss across all cross-validation folds. The optimal algorithm and hyperparameter settings are obtained by minimizing this loss function. Thornton et al  \cite{thornton2013auto} were among the earliest researchers to examine the CASH problem in the AUTO-WEKA system \cite{hall2009weka}, which utilized Bayesian optimization to solve the combined task of choosing the optimal learning algorithm and adjusting hyperparameters. Bayesian optimization \cite{brochu2010tutorial} employs a probabilistic model to estimate performance of different hyperparameter configurations, thereby achieving a balance between exploring new configurations and exploiting known optimal ones. Thornton et al. \cite{thornton2013auto} investigated Bayesian tree-based search algorithms, demonstrating that the SMAC framework utilizing random forests \cite{hutter2011sequential} performs better than its predecessor \cite{bergstra2011algorithms}. Within this study, SMAC is leveraged to optimize the CASH process. Besides leveraging random forests \cite{breiman2001random}, SMAC is distinguished by its ability to Enhance cross-validation efficiency through independent fold evaluation while filtering out ineffective hyperparameter configurations early on.

Class imbalance is a common issue in classification tasks, especially in medical datasets where minority class labels often lack sufficient samples. To better identify minority classes, automated machine learning provides functionality for handling imbalanced classes. The main idea is to assign different weights to each class so that the model places greater emphasis on minority class samples during training, thereby improving overall classifier performance.

\begin{figure*}[h!]
	\centering
	\includegraphics[width=\textwidth]{class.png}
	\caption{Class distribution across action categories shows a noticeable imbalance, with certain categories, such as Double throw and Set needle, containing significantly more samples compared to others like Grasp needle and Needle exits. This imbalance may lead to biased model performance, favoring well-represented classes while potentially underperforming on underrepresented ones.
	}
	\label{classdistribution}
\end{figure*}

In practice, AutoML combines two main approaches for handling class imbalance: \textit{sample resampling} and \textit{weighted classification}. Sample resampling includes oversampling (such as using SMOTE \cite{chawla2002smote} to produce additional data samples for minority classes) and undersampling (reducing the number of majority class samples). Weighted classification assigns higher weights to minority classes, increasing their penalty in the loss function and enhancing the model's sensitivity to these classes.

Suppose a dataset with \( C \) classes, where each class \( i \) has \( N_i \) samples, the weight for class \( i \), \( w_i \), is defined as:
\begin{equation}
	w_i = \frac{N}{C \cdot N_i}
\end{equation}
Here, \( N \) indicated total dataset cardinality. The weighted cross-entropy loss function is given by:
\begin{equation}
	L = - \sum_{i=1}^{C} w_i \cdot y_i \log(\hat{y}_i)
\end{equation}
where \( y_i \) represents the true label's one-hot encoding, and \( \hat{y}_i \) represents the model's predicted probability.

Auto-Sklearn integrates meta-learning \cite{hutter2011sequential} and automated ensemble \cite{guyon2010model, lacoste2014agnostic} techniques to accelerate model optimization and enhance overall performance. First, meta-learning leverages prior knowledge from previous tasks to predict configurations has a high probability of delivering reliable results on new data. It computes meta-features (such as dataset size, dimensionality, and class distribution) for several datasets to infer suitable algorithms for the new dataset. This process involves an offline phase and an online phase: in the offline phase, the system analyzes multiple datasets to identify high-performing configurations and stores them; when encountering a new dataset, it quickly identifies the most similar stored configuration based on meta-features as the starting point for Bayesian optimization, significantly reducing optimization time. Here we follow the Auto-Sklearn approach for our task.

Additionally, the automated ensemble method further improves robustness and performance during model optimization. Unlike traditional Bayesian optimization that seeks a single best model configuration, Auto-Sklearn retains multiple models trained during the search process and constructs an ensemble model from them. This approach avoids early convergence on a specific configuration, enhancing model robustness. The ensemble is constructed using a post-processing method that combines predictions from candidate models through weighted averaging, thereby optimizing overall performance. This method eliminates the need to fine-tune a single hyperparameter setting excessively, instead leveraging diverse model outputs to significantly boost generalization capabilities. The overall workflow for automated machine learning is as shown in Fig. \ref{framework}





\begin{figure}[ht]
	\centering
	\includegraphics[width=0.5\textwidth]{modelcomp.png}
	\caption{Model accuracy comparison, illustrating the impact of automated optimization versus conventional deep learning approaches.}
	\label{modelcomp}
\end{figure}


\begin{table*}[h!]
	\centering
	\caption{Trajectory Data Characteristics for Novice and Experienced Surgeons}
	\label{trajectorydatacharacteristics}
	\begin{tabular}{lccccc}
		\hline
		\textbf{Operation Type} & \textbf{Trajectory Length (data points)} & \textbf{Speed} & \textbf{Acceleration} & \textbf{Trajectory Range} & \textbf{Trajectory Deviation} \\
		\hline
		Novice Surgeon & 11716 & 14.59 & 113.84 & 4540.26 & 42.91 \\
		Experienced Surgeon & 4782 & 9.90 & 38.05 & 669.19 & 44.99 \\
		\hline
	\end{tabular}
\end{table*}







\begin{table*}[h]
	\centering
	\caption{Classifier Contribution Table}
	\label{tab:adjusted_classifier_contribution}
	\begin{tabular}{c|l|c|c|c|c}
		\hline
		\textbf{No.} & \textbf{Classifier} & \textbf{Ensemble Weight} & \textbf{Cost} & \textbf{Balancing Strategy} & \textbf{Validation Score} \\
		\hline
		1 & RandomForestClassifier (model 2) & 0.3 & 0.0737 & Weighting & 0.927 \\
		2 & RandomForestClassifier (model 3) & 0.06 & 0.8319 & Weighting & 0.85 \\
		3 & HistGradientBoostingClassifier (model 4) & 0.04 & 0.4264 & Weighting & 0.82 \\
		4 & KNeighborsClassifier (model 8) & 0.22 & 0.0997 & None & 0.9 \\
		5 & HistGradientBoostingClassifier (model 9) & 0.04 & 0.1113 & Weighting & 0.84 \\
		6 & RandomForestClassifier (model 13) & 0.34 & 0.0722 & Weighting & 0.93 \\
		\hline
	\end{tabular}
\end{table*}



\section{Experiments}


In this section, we utilize the laparoscopy surgical training data that we have collected from six professional and four novice laparoscopy surgeons using a laparoscopy training box. In average, the professional surgeons had 15
years’ expertise in general surgical practice and 133
hours in laparoscopic operations, and the trainee surgeons had 5 years’ surgical expertise and had gained 3 hours laparoscopic training. We recorded in total 10 videos of suturing exercise across all professional and novice participants. The experiment protocol for collection of this data was approved by the Ethics Committee of the Heriot-Watt University. The data is divided into two categories: one set was collected from operations performed by experienced surgeons, labeled as \texttt{Exp} files (6 files), while the other set was gathered from novice surgeons, labeled as \texttt{Nov} files (4 files). 
Trajectories were drawn from videos of the surgical tasks by calibrating the camera and tracking coloured markers, precisely positioned on each instrument shaft close to the instrument tip. Measurement of properties of the projected image of these markers allowed the 3D position of the markers and thus the instrument tip to be tracked across frames of the video. 
The videos had a frame rate of 25 fps and a resolution of 1920 by 1080 pixels.   
Comparison of with videos taken from a robotically controlled instrument as ground truth allowed an estimate of positioning accuracy of 5mm to be made.
These data not only capture the differences in precision and consistency but also reflect typical movement characteristics associated with different skill levels, providing a rich source of data for training and validating the surgical action recognition model. 
Our dataset comprises 11 categories of movement, as listed in Table \ref{tab:classification_report_novice}, each representing a specific surgical action. The 11 categories were created by combining surgical phase modelling \cite{lalys2014surgical} and heirarchical task analysis \cite{sarker2008constructing} to the suture task.
These activities are broadly composed of three phases:
Bringing the needle under control of the instrument and setting orientation (Seek needle, Grasp needle, Set Needle).
Passing the suture material through the tissue phantom (Seek proximal insertion, Proximal bite, Distal bite, Needle exits, Tension.
Forming a surgeon's knot (Create bight, Single Throw, Double throw).
This resulted in 11 distinct activities that must be performed in sequence to complete the task and as a result were conserved across the dataset. To fully illustrate the attributes of trajectory data, we offer a detailed computational table \ref{trajectorydatacharacteristics} of trajectory characteristics, where each attribute is calculated by averaging. Here, we don’t need to use dimensions for these attributes, because they are the same for both novices and experienced surgeons, making them easier to compare. Additionally, Fig. \ref{classdistribution} illustrates the data distribution across categories, clearly reflecting the class imbalance issue among different actions. Finally, a visualization of the trajectories in Fig. \ref{trajvisual} shows the spatial characteristics of the operations, displaying the paths taken by novice and experienced surgeons on the same task. These tables and figures help us gain an in-depth insight of the dataset, supporting subsequent model training and performance evaluation.


%Additionally, the data collection process adhered strictly to experimental protocols to ensure the authenticity and representativeness of the acquired data.

\begin{table}[h]
	\centering
	\caption{Classification Report on Test Dataset}
	\label{tab:classification_report_novice}
	\begin{tabular}{lcccc}
		\hline
		\textbf{Class} & \textbf{Precision} & \textbf{Recall} & \textbf{F1-score} & \textbf{Support} \\
		\hline
		Create Bight             & 0.90 & 0.90 & 0.90 & 1005 \\
		Distal bite              & 0.84 & 0.93 & 0.88 & 1861 \\
		Double throw             & 0.96 & 0.92 & 0.94 & 9604 \\
		Grasp needle             & 0.96 & 0.98 & 0.97 & 251  \\
		Needle exits             & 0.90 & 0.85 & 0.88 & 1076 \\
		Proximal bite            & 0.88 & 0.92 & 0.90 & 1754 \\
		Seek needle              & 0.99 & 0.96 & 0.97 & 684  \\
		Seek proximal insertion  & 0.85 & 0.86 & 0.85 & 630  \\
		Set needle               & 0.94 & 0.97 & 0.95 & 1876 \\
		Single throw             & 0.89 & 0.89 & 0.89 & 3259 \\
		Tension                  & 0.86 & 0.91 & 0.88 & 1432 \\
		\hline
		\textbf{Accuracy}        &       &       & 0.92 & 23432 \\
		\textbf{Macro avg}       & 0.91 & 0.92 & 0.91 & 23432 \\
		\textbf{Weighted avg}    & 0.92 & 0.92 & 0.92 & 23432 \\
		\hline
	\end{tabular}
\end{table}


\begin{figure}[h]
	\centering
	\includegraphics[width=0.5\textwidth]{modelsize.jpg}
	\caption{Analysis of prediction speed and accuracy relative to model size in ensemble architectures}
	\label{predictionvsmodelsize}
\end{figure}

The training platform used is the HP ZBook Fury 16 G10, featuring an Intel Core i9-13950HX processor, 64GB of DDR5 RAM, and an NVIDIA RTX 5000 Ada GPU with 16GB of dedicated memory. We use Auto-Sklearn \cite{feurer2015efficient} as the implementation. For data partitioning, considering the significant differences in data characteristics between the \texttt{Exp} and \texttt{Nov} datasets, we used all \texttt{Exp} data along with half of the \texttt{Nov} data as the training and validation set. The remaining half of the \texttt{Nov} data was designated as the test set. This devision ensures that the model is exposed to a diverse dataset while reserving part of the novice data for unbiased testing, highlighting the model’s capacity for adaptation across experience levels. 

\begin{figure*}[h!]
	\centering
	\includegraphics[width=0.8\textwidth]{Figure_1.png}
	\caption{Experienced surgeons' trajectories show consistent, controlled, and concentrated movements within a defined spatial range, reflecting precision and coordination. In contrast, novice surgeons' trajectories display irregular and dispersed patterns, suggesting a lack of control and coordination. The trajectories of novice surgeons also demonstrate frequent abrupt deviations and scattered points, indicating variability and more erratic hand movements. These inconsistencies and outliers in novice trajectories may introduce noise to the data, potentially impacting model training by skewing predictions toward these deviations.}
	\label{trajvisual}
\end{figure*}


\section{Results}

Fig. \ref{modelcomp} presents a comparison of various models’ effectiveness on the given dataset. AutoML achieves the highest accuracy at \textit{92\%}, significantly outperforming the classic deep learning based models. InceptionTime \cite{ismail2020inceptiontime} and FCN \cite{wang2017time} achieve comparable accuracies of \textit{67\%} and \textit{66\%}, respectively, demonstrating relatively strong performance. TapNet \cite{zhang2020tapnet} achieves an accuracy of \textit{56\%}, while CNN \cite{zhao2017convolutional} performs the worst with an accuracy of only \textit{48\%}. The superior performance of AutoML is likely due to its automated feature selection, hyperparameter optimization, and model ensembling techniques, whereas deep learning models like InceptionTime and FCN benefit from their specific adaptability to time-series data. CNN models, typically more suitable for image data, show limited modeling capacity for time-series tasks, which may explain their lowest accuracy. It is important to note that we used open-source implementations of deep learning models without any optimization tailored to our dataset. 

%such as handling class imbalance or adjusting model architectures.


The overall accuracy on the novice dataset is \textit{91.7\%}. Combined with the detailed metrics in the classification report (see Table \ref{tab:classification_report_novice}), we can gain insights into the model’s performance, stability, and reliability under conditions of class imbalance.

First, there is a significant disparity in the number of samples per class, with \texttt{Double throw} having a support count of 9604, while \texttt{Grasp needle} only has 251. Such imbalance usually leads to better performance on classes with larger sample sizes, potentially at the expense of smaller classes. By examining the \textit{Weighted avg} and \textit{Macro avg} metrics, we can better understand comprehensive effectiveness of this approach. In this dataset, the weighted average F1-score is \textit{0.92}, indicating that the model maintains stable performance across most classes, especially in those with larger sample sizes, which enhances overall performance on imbalanced data. Meanwhile, the macro average F1-score is \textit{0.91}, demonstrating that the model performs relatively consistently across classes without a strong bias towards larger classes, showing a degree of generalization.

The \textit{precision} and \textit{recall} values in the classification report further highlight the model's stability and reliability. For instance, the F1-score for \textit{Double throw} is \textit{0.94}, while \textit{Grasp needle} achieves an F1-score of \textit{0.97}. This suggests that the model achieves high precision and recall across both large and small classes, reliably identifying various surgical actions. Moreover, the high macro and weighted averages confirm the model's stable performance despite class imbalance.

Finally, \textit{support} provides the sample count for each class, clearly showing the distribution differences across classes. Despite the imbalance, the model still achieves a high overall F1-score and accuracy, demonstrating its robustness and adaptability. These metrics collectively validate the model's robust performance on the novice dataset, providing reliable support for surgical action recognition applications.





As shown in Table \ref{tab:adjusted_classifier_contribution}, notable variations exist in the contribution and performance of different classifiers within the ensemble model. In terms of interpretability, models like RandomForestClassifier and KNeighborsClassifier generally provide higher interpretability, which aids in analyzing the specific role of each model in the predictions. The table's Ensemble Weight attribute indicates contribution of each model to the ensemble; for example, RandomForestClassifier (model 13) has the highest weight (0.34), showing that it plays a dominant role in the final predictions. Additionally, the Balancing Strategy column shows the balancing strategy for each model, with most models using the Weighting strategy to address data imbalance issues, enhancing robustness and fairness in predictions. Finally, the Validation Score column reflects the models' performance on the validation set, where models with higher weights tend to have higher validation scores, further emphasizing their importance in the overall ensemble. Relationship between model size, prediction speed and prediction accuracy is as shown in Fig. \ref{predictionvsmodelsize}. This figure illustrates how the size of ensemble models (in MB) affects predictions per second and prediction accuracy. Smaller models exhibit significantly higher prediction speeds, with speeds exceeding 100 predictions per second for models under 25MB, making them highly suitable for scenarios with strict real-time requirements. However, as the model size increases, prediction speed drops significantly, while accuracy shows a gradual upward trend. Notably, the largest model (200MB) achieves a high accuracy close to 0.95 but suffers from a substantial decline in prediction speed, making it less viable for real-time applications. A closer look reveals that models within the 10MB to 25MB range strike the best balance between speed and accuracy. These models deliver prediction speeds exceeding 100 predictions per second while maintaining strong accuracy, making them ideal for scenarios that demand both real-time performance and reliable predictions. This range highlights a practical trade-off, ensuring sufficient speed without compromising predictive performance, making them highly suitable for scenarios with strict real-time requirements.



\section{CONCLUSIONS and Future Works}

In this study, we developed a traditional machine learning based and fast deployable laparoscopic surgical action recognition system, effectively addressing challenges related to data imbalance, model interpretability, stability, and reliability. By training our model on a high-performance platform and leveraging the automated machine learning framework, we achieved high accuracy and stable performance across multiple surgical action categories. Additionally, the classifier contribution table and model performance analysis provided valuable insights into model selection and optimization, validating the potential of our approach to enhance surgical action recognition and training efficiency. For upcoming studies, we plan to expand our dataset by collecting more surgical action data to improve model robustness. We also intend to explore the latest deep learning based detection techniques to further enhance the system's accuracy and real-time capabilities.


\addtolength{\textheight}{-12cm}   % This command serves to balance the column lengths
% on the last page of the document manually. It shortens
% the textheight of the last page by a suitable amount.
% This command does not take effect until the next page
% so it should come on the page before the last. Make
% sure that you do not shorten the textheight too much.

%%%%%%%%%%%%%%%%%%%%%%%%%%%%%%%%%%%%%%%%%%%%%%%%%%%%%%%%%%%%%%%%%%%%%%%%%%%%%%%%



%%%%%%%%%%%%%%%%%%%%%%%%%%%%%%%%%%%%%%%%%%%%%%%%%%%%%%%%%%%%%%%%%%%%%%%%%%%%%%%%



%%%%%%%%%%%%%%%%%%%%%%%%%%%%%%%%%%%%%%%%%%%%%%%%%%%%%%%%%%%%%%%%%%%%%%%%%%%%%%%%
%\section*{APPENDIX}
%
%Appendixes should appear before the acknowledgment.


\section*{Acknowledgment}

This research has been funded by the Engineering and Physical Sciences Research Council (EPSRC) of United Kingdom under Grant Reference EP/Y017307/1.


\bibliographystyle{IEEEtran}
\bibliography{IEEEabrv,ref}

%\begin{thebibliography}{00}
%\bibitem{b1} G. Eason, B. Noble, and I. N. Sneddon, ``On certain integrals of Lipschitz-Hankel type involving products of Bessel functions,'' Phil. Trans. Roy. Soc. London, vol. A247, pp. 529--551, April 1955.
%\bibitem{b2} J. Clerk Maxwell, A Treatise on Electricity and Magnetism, 3rd ed., vol. 2. Oxford: Clarendon, 1892, pp.68--73.
%\bibitem{b3} I. S. Jacobs and C. P. Bean, ``Fine particles, thin films and exchange anisotropy,'' in Magnetism, vol. III, G. T. Rado and H. Suhl, Eds. New York: Academic, 1963, pp. 271--350.
%\bibitem{b4} K. Elissa, ``Title of paper if known,'' unpublished.
%\bibitem{b5} R. Nicole, ``Title of paper with only first word capitalized,'' J. Name Stand. Abbrev., in press.
%\bibitem{b6} Y. Yorozu, M. Hirano, K. Oka, and Y. Tagawa, ``Electron spectroscopy studies on magneto-optical media and plastic substrate interface,'' IEEE Transl. J. Magn. Japan, vol. 2, pp. 740--741, August 1987 [Digests 9th Annual Conf. Magnetics Japan, p. 301, 1982].
%\bibitem{b7} M. Young, The Technical Writer's Handbook. Mill Valley, CA: University Science, 1989.
%\bibitem{b8} D. P. Kingma and M. Welling, ``Auto-encoding variational Bayes,'' 2013, arXiv:1312.6114. [Online]. Available: https://arxiv.org/abs/1312.6114
%\bibitem{b9} S. Liu, ``Wi-Fi Energy Detection Testbed (12MTC),'' 2023, gitHub repository. [Online]. Available: https://github.com/liustone99/Wi-Fi-Energy-Detection-Testbed-12MTC
%\bibitem{b10} ``Treatment episode data set: discharges (TEDS-D): concatenated, 2006 to 2009.'' U.S. Department of Health and Human Services, Substance Abuse and Mental Health Services Administration, Office of Applied Studies, August, 2013, DOI:10.3886/ICPSR30122.v2
%\bibitem{b11} K. Eves and J. Valasek, ``Adaptive control for singularly perturbed systems examples,'' Code Ocean, Aug. 2023. [Online]. Available: https://codeocean.com/capsule/4989235/tree
%\end{thebibliography}



\end{document}



\end{document}
