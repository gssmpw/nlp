\section{Related Work}
\emph{Syntactic} unification is unification modulo the empty
equational theory. There are many algorithms for syntactic unification, but there are
only a few which have been verified and formalized. The earliest formalization
is the algorithm from Manna and Waldinger \cite{manna1983deductive}, which was proved by Paulson \cite{paulson1985verifying} using LCF (Logic for Computable Functions). 
%@Yichi: Rose changed the authors here, please double-check my work.
This formalization is used as a basis for later research by Sternagel and Thiemann 
%This formalization is used as a basis for later research by Coen, Slind,
\cite{FirstOrderTermsAFP} in Isabelle. 
Urban, Pitts, and Gabbay \cite{urban2004nominal} also formalized first-order unification in Isabelle. 
A relatively recent formalization for syntactic unification is from Avelar, Galdino, deMoura, and
Ayala-Rincon \cite{avelar2014first} using PVS (Prototype Verification System).

\emph{E}-unification is unification modulo an equational theory.
Dougherty \cite{dougherty2019coq} has verified two algorithms for Boolean 
unification.  Ayala-Rinc{\'o}n \emph{et.\ al.}\ \cite{ayala2022certified} have
verified an AC(Associativity and Commutativity)-Unification algorithm using
PVS. For XOR unification, there are only a
few algorithms and no formalization. 
Tuengethal, Kusters and Turuani  \cite{tuengerthal2006implementing}
mentioned a relatively easy and intuitive way to design such an
algorithm by combining theories such that their overall output
satisfies the XOR properties. 
Guo, Narendran, and Wolfram \cite{guo1996unification} mentioned using Gaussian 
elimination over a Boolean ring to compute unifiers for XOR
unification. %
Liu and Lynch \cite{liu2011efficient}
give several terminating inference rules to solve XOR unification.  %
However, the above papers only give algorithms but not a formalization. Therefore 
in this paper, we decided to do a formalization over their work in Coq so we can be more confidence in the algorithm.