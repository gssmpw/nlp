\begin{figure}[h]
    \centering
    \begin{subfigure}[b]{0.3\textwidth}
        \centering
        \includegraphics[width=\textwidth]{paper/contents/images/training-data.pdf}
        \caption{}
        \label{fig:fig1}
    \end{subfigure}
    \hfill
    \begin{subfigure}[b]{0.3\textwidth}
        \centering
        \includegraphics[width=\textwidth]{paper/contents/images/test-data-strength_01.pdf}
        \caption{}
        \label{fig:fig2}
    \end{subfigure}
    \hfill
    \begin{subfigure}[b]{0.3\textwidth}
        \centering
        \includegraphics[width=\textwidth]{paper/contents/images/test-data-strength_02.pdf}
        \caption{}
        \label{fig:fig3}
    \end{subfigure}
    \caption{The figures illustrate the structure of the (a) training-time shifts and (b-c) test-time shifts for different perturbation strengths on the example of two covariates. Panel (a) shows the training data containing two environments--observational (blue) and shifted (orange) corresponding to the knockout of the gene ENSG00000089009. 
    Panels~(b) and~(c) show the training data in grey and test data from a previously unseen environment (green). 
    Panel~(b) depicts the top $10\%$ test data points closest to the training support (perturbation strength = $0.1$).
    Panel~(c) illustrates the full test data (perturbation strength = 1.0).
    }
    \label{fig:genes}
\end{figure}