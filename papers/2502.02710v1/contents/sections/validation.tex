% \subsection{Synthetic experiments}
%     \begin{wrapfigure}{R}{0.4\textwidth}
%     \centering
%     \includegraphics[width = 0.4\textwidth]{paper/contents/images/synthetic_plot_paper.pdf}
%     \caption{Mean-squared error on the worst-case test distribution for synthetic Gaussian data. }
%     \label{fig:synthetic-experiments}    
% \end{wrapfigure}
% \fy{as discussed in gdoc - i would put this plot right after section 3.2. and the experimental section would only be for semi-synthetic}
% First, we validate our findings on synthetic Gaussian data. Over multiple runs, we randomly generate the model-generating parameters as well as the shifts between training environments. We keep fixed across runs the condition number of the noise matrix $\Sigma$, the norm of $\betastar$, the number of observed training environments and the dimension of the identified subspace $\cS$.  We generate samples from all training distributions and use them to compute the least-squares, anchor, and identified causal estimators, as well as the PI-robust estimator. We generate samples from a test distribution which contains both previously observed as well as unobserved shifts, where the unobserved shifts are fixed to have a much lower magnitude. Our findings are presented in \prettyref{fig:synthetic-experiments}. At a small shift magnitude $\gamma$, OLS, anchor and PI-robust estimators all exhibit the same behaviour close to the lower bound in \prettyref{prop:pi-loss-lower-bound}. However, for larger $\gamma$, anchor and OLS estimators exhibit a suboptimal dependency on $\gamma$ compared to the lower bound. Both estimators have a similar test performance even if a small portion of unobserved shifts is present. The identified causal parameter $\betastarS$ shows correct asymptotic behaviour, but with a constant bias caused by discarding the information on the non-identified but stable subspace $\cSperp - R$. 

% \begin{wrapfigure}{r}{0.35\linewidth}
%     \vspace{-0.5cm}
%     \centering
%     \input{paper/contents/images/synthetic_plot_paper.pdf}
%     \caption{
% Mean-squared error on the worst-case test distribution for synthetic Gaussian data. In this plot, the number of covariates $X$ is $d = 10$, and six environments differing by mean shifts (including the reference environment) are observed. Thus, $\dim \cS = 5$. The test data differ by a Gaussian shift with variance $\Sigmatest = \gamma S S^\top + 0.1 \gamma R R^\top$, where both the identified subspace $S$ and the non-identified subspace $R$ are two-dimensional. The strength of shifts in the $R$-direction is set to be much lower. The solid lines show the means of the losses over 50 runs, and the shared regions denote $95\%$ t-test confidence intervals. \julia{quantiles instead? but they look bad. Maybe fix more parameters across experiments}
%     \label{fig:synthetic-experiments}
%     \vspace{-0.5cm}
% \end{wrapfigure}

\subsection{Semi-synthetic experiments}
