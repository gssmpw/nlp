\documentclass{article}


\PassOptionsToPackage{numbers,sort&compress}{natbib}
\usepackage[utf8]{inputenc} % allow utf-8 input
\usepackage[T1]{fontenc}    % use 8-bit T1 fonts
\usepackage{hyperref}       % hyperlinks

\usepackage{url}            % simple URL typesetting
\usepackage{booktabs}       % professional-quality tables
\usepackage{amsfonts}       % blackboard math symbols
\usepackage{nicefrac}       % compact symbols for 1/2, etc.
\usepackage{microtype}      % microtypography
\usepackage{xcolor}         % colors

% Recommended, but optional, packages for figures and better typesetting:
\usepackage{amsfonts}[mathscr]
\usepackage{amssymb}
\usepackage{amsmath}
\usepackage{mathtools}
\usepackage{microtype}
\usepackage{graphicx}
\usepackage{natbib}[numbers,sort&compress]
\usepackage{geometry}
\usepackage{booktabs} % for professional tables
\usepackage{graphicx} 
\usepackage{todonotes}
\usepackage{xcolor}
\usepackage{dsfont}
\usepackage{nicefrac}
\usepackage{tcolorbox}
\usepackage{tabularx}
\usepackage{pifont}
% \usepackage{enumitem}
\usepackage{multirow}
\usepackage{algorithm}
\usepackage{algpseudocode}
\usepackage{caption}
\usepackage{wrapfig}
\usepackage[capitalize,noabbrev]{cleveref}
\usepackage{mathrsfs}
\usepackage{algorithm}
\usepackage{algpseudocode}
\usepackage[export]{adjustbox}
\usepackage[list=true]{subcaption}

\usepackage{authblk,textcomp}

\usepackage[cal=euler]{mathalfa}
\usepackage{libertine}



\usepackage{titletoc}

% \newcommand\DoToC{%
%   \startcontents
% \hypersetup{colorlinks=true}
%   \printcontents{}{1}{\subsection*{\textbf{Table of contents}}}
%   \vskip3pt\vskip5pt
% }

%=====================================
% AMS package for theorem section
%-------------------------------------
\usepackage{amsthm}
\usepackage{thmtools, thm-restate}
\declaretheorem[numberwithin=section]{thm}
\declaretheorem[sibling=thm]{theorem}
\declaretheorem[sibling=thm]{lemma}
\declaretheorem[sibling=thm]{corollary}
\declaretheorem[numberwithin=section]{assumption}
\declaretheorem[numbered=no]{discussion}
\declaretheorem[]{challenged assumption}
\declaretheorem[]{definition}
\declaretheorem[]{proposition}
\declaretheorem[style=remark]{example}
%=====================================
% Commands for text
%-------------------------------------
\usepackage{xspace}
\DeclareRobustCommand{\eg}{e.g.,\@\xspace}
\DeclareRobustCommand{\ie}{i.e.,\@\xspace}
\DeclareRobustCommand{\wrt}{w.r.t.\@\xspace}


%=====================================
% Math commands
%-------------------------------------

%
\setlength\unitlength{1mm}
\newcommand{\twodots}{\mathinner {\ldotp \ldotp}}
% bb font symbols
\newcommand{\Rho}{\mathrm{P}}
\newcommand{\Tau}{\mathrm{T}}

\newfont{\bbb}{msbm10 scaled 700}
\newcommand{\CCC}{\mbox{\bbb C}}

\newfont{\bb}{msbm10 scaled 1100}
\newcommand{\CC}{\mbox{\bb C}}
\newcommand{\PP}{\mbox{\bb P}}
\newcommand{\RR}{\mbox{\bb R}}
\newcommand{\QQ}{\mbox{\bb Q}}
\newcommand{\ZZ}{\mbox{\bb Z}}
\newcommand{\FF}{\mbox{\bb F}}
\newcommand{\GG}{\mbox{\bb G}}
\newcommand{\EE}{\mbox{\bb E}}
\newcommand{\NN}{\mbox{\bb N}}
\newcommand{\KK}{\mbox{\bb K}}
\newcommand{\HH}{\mbox{\bb H}}
\newcommand{\SSS}{\mbox{\bb S}}
\newcommand{\UU}{\mbox{\bb U}}
\newcommand{\VV}{\mbox{\bb V}}


\newcommand{\yy}{\mathbbm{y}}
\newcommand{\xx}{\mathbbm{x}}
\newcommand{\zz}{\mathbbm{z}}
\newcommand{\sss}{\mathbbm{s}}
\newcommand{\rr}{\mathbbm{r}}
\newcommand{\pp}{\mathbbm{p}}
\newcommand{\qq}{\mathbbm{q}}
\newcommand{\ww}{\mathbbm{w}}
\newcommand{\hh}{\mathbbm{h}}
\newcommand{\vvv}{\mathbbm{v}}

% Vectors

\newcommand{\av}{{\bf a}}
\newcommand{\bv}{{\bf b}}
\newcommand{\cv}{{\bf c}}
\newcommand{\dv}{{\bf d}}
\newcommand{\ev}{{\bf e}}
\newcommand{\fv}{{\bf f}}
\newcommand{\gv}{{\bf g}}
\newcommand{\hv}{{\bf h}}
\newcommand{\iv}{{\bf i}}
\newcommand{\jv}{{\bf j}}
\newcommand{\kv}{{\bf k}}
\newcommand{\lv}{{\bf l}}
\newcommand{\mv}{{\bf m}}
\newcommand{\nv}{{\bf n}}
\newcommand{\ov}{{\bf o}}
\newcommand{\pv}{{\bf p}}
\newcommand{\qv}{{\bf q}}
\newcommand{\rv}{{\bf r}}
\newcommand{\sv}{{\bf s}}
\newcommand{\tv}{{\bf t}}
\newcommand{\uv}{{\bf u}}
\newcommand{\wv}{{\bf w}}
\newcommand{\vv}{{\bf v}}
\newcommand{\xv}{{\bf x}}
\newcommand{\yv}{{\bf y}}
\newcommand{\zv}{{\bf z}}
\newcommand{\zerov}{{\bf 0}}
\newcommand{\onev}{{\bf 1}}

% Matrices

\newcommand{\Am}{{\bf A}}
\newcommand{\Bm}{{\bf B}}
\newcommand{\Cm}{{\bf C}}
\newcommand{\Dm}{{\bf D}}
\newcommand{\Em}{{\bf E}}
\newcommand{\Fm}{{\bf F}}
\newcommand{\Gm}{{\bf G}}
\newcommand{\Hm}{{\bf H}}
\newcommand{\Id}{{\bf I}}
\newcommand{\Jm}{{\bf J}}
\newcommand{\Km}{{\bf K}}
\newcommand{\Lm}{{\bf L}}
\newcommand{\Mm}{{\bf M}}
\newcommand{\Nm}{{\bf N}}
\newcommand{\Om}{{\bf O}}
\newcommand{\Pm}{{\bf P}}
\newcommand{\Qm}{{\bf Q}}
\newcommand{\Rm}{{\bf R}}
\newcommand{\Sm}{{\bf S}}
\newcommand{\Tm}{{\bf T}}
\newcommand{\Um}{{\bf U}}
\newcommand{\Wm}{{\bf W}}
\newcommand{\Vm}{{\bf V}}
\newcommand{\Xm}{{\bf X}}
\newcommand{\Ym}{{\bf Y}}
\newcommand{\Zm}{{\bf Z}}

% Calligraphic

\newcommand{\Ac}{{\cal A}}
\newcommand{\Bc}{{\cal B}}
\newcommand{\Cc}{{\cal C}}
\newcommand{\Dc}{{\cal D}}
\newcommand{\Ec}{{\cal E}}
\newcommand{\Fc}{{\cal F}}
\newcommand{\Gc}{{\cal G}}
\newcommand{\Hc}{{\cal H}}
\newcommand{\Ic}{{\cal I}}
\newcommand{\Jc}{{\cal J}}
\newcommand{\Kc}{{\cal K}}
\newcommand{\Lc}{{\cal L}}
\newcommand{\Mc}{{\cal M}}
\newcommand{\Nc}{{\cal N}}
\newcommand{\nc}{{\cal n}}
\newcommand{\Oc}{{\cal O}}
\newcommand{\Pc}{{\cal P}}
\newcommand{\Qc}{{\cal Q}}
\newcommand{\Rc}{{\cal R}}
\newcommand{\Sc}{{\cal S}}
\newcommand{\Tc}{{\cal T}}
\newcommand{\Uc}{{\cal U}}
\newcommand{\Wc}{{\cal W}}
\newcommand{\Vc}{{\cal V}}
\newcommand{\Xc}{{\cal X}}
\newcommand{\Yc}{{\cal Y}}
\newcommand{\Zc}{{\cal Z}}

% Bold greek letters

\newcommand{\alphav}{\hbox{\boldmath$\alpha$}}
\newcommand{\betav}{\hbox{\boldmath$\beta$}}
\newcommand{\gammav}{\hbox{\boldmath$\gamma$}}
\newcommand{\deltav}{\hbox{\boldmath$\delta$}}
\newcommand{\etav}{\hbox{\boldmath$\eta$}}
\newcommand{\lambdav}{\hbox{\boldmath$\lambda$}}
\newcommand{\epsilonv}{\hbox{\boldmath$\epsilon$}}
\newcommand{\nuv}{\hbox{\boldmath$\nu$}}
\newcommand{\muv}{\hbox{\boldmath$\mu$}}
\newcommand{\zetav}{\hbox{\boldmath$\zeta$}}
\newcommand{\phiv}{\hbox{\boldmath$\phi$}}
\newcommand{\psiv}{\hbox{\boldmath$\psi$}}
\newcommand{\thetav}{\hbox{\boldmath$\theta$}}
\newcommand{\tauv}{\hbox{\boldmath$\tau$}}
\newcommand{\omegav}{\hbox{\boldmath$\omega$}}
\newcommand{\xiv}{\hbox{\boldmath$\xi$}}
\newcommand{\sigmav}{\hbox{\boldmath$\sigma$}}
\newcommand{\piv}{\hbox{\boldmath$\pi$}}
\newcommand{\rhov}{\hbox{\boldmath$\rho$}}
\newcommand{\upsilonv}{\hbox{\boldmath$\upsilon$}}

\newcommand{\Gammam}{\hbox{\boldmath$\Gamma$}}
\newcommand{\Lambdam}{\hbox{\boldmath$\Lambda$}}
\newcommand{\Deltam}{\hbox{\boldmath$\Delta$}}
\newcommand{\Sigmam}{\hbox{\boldmath$\Sigma$}}
\newcommand{\Phim}{\hbox{\boldmath$\Phi$}}
\newcommand{\Pim}{\hbox{\boldmath$\Pi$}}
\newcommand{\Psim}{\hbox{\boldmath$\Psi$}}
\newcommand{\Thetam}{\hbox{\boldmath$\Theta$}}
\newcommand{\Omegam}{\hbox{\boldmath$\Omega$}}
\newcommand{\Xim}{\hbox{\boldmath$\Xi$}}


% Sans Serif small case

\newcommand{\Gsf}{{\sf G}}

\newcommand{\asf}{{\sf a}}
\newcommand{\bsf}{{\sf b}}
\newcommand{\csf}{{\sf c}}
\newcommand{\dsf}{{\sf d}}
\newcommand{\esf}{{\sf e}}
\newcommand{\fsf}{{\sf f}}
\newcommand{\gsf}{{\sf g}}
\newcommand{\hsf}{{\sf h}}
\newcommand{\isf}{{\sf i}}
\newcommand{\jsf}{{\sf j}}
\newcommand{\ksf}{{\sf k}}
\newcommand{\lsf}{{\sf l}}
\newcommand{\msf}{{\sf m}}
\newcommand{\nsf}{{\sf n}}
\newcommand{\osf}{{\sf o}}
\newcommand{\psf}{{\sf p}}
\newcommand{\qsf}{{\sf q}}
\newcommand{\rsf}{{\sf r}}
\newcommand{\ssf}{{\sf s}}
\newcommand{\tsf}{{\sf t}}
\newcommand{\usf}{{\sf u}}
\newcommand{\wsf}{{\sf w}}
\newcommand{\vsf}{{\sf v}}
\newcommand{\xsf}{{\sf x}}
\newcommand{\ysf}{{\sf y}}
\newcommand{\zsf}{{\sf z}}


% mixed symbols

\newcommand{\sinc}{{\hbox{sinc}}}
\newcommand{\diag}{{\hbox{diag}}}
\renewcommand{\det}{{\hbox{det}}}
\newcommand{\trace}{{\hbox{tr}}}
\newcommand{\sign}{{\hbox{sign}}}
\renewcommand{\arg}{{\hbox{arg}}}
\newcommand{\var}{{\hbox{var}}}
\newcommand{\cov}{{\hbox{cov}}}
\newcommand{\Ei}{{\rm E}_{\rm i}}
\renewcommand{\Re}{{\rm Re}}
\renewcommand{\Im}{{\rm Im}}
\newcommand{\eqdef}{\stackrel{\Delta}{=}}
\newcommand{\defines}{{\,\,\stackrel{\scriptscriptstyle \bigtriangleup}{=}\,\,}}
\newcommand{\<}{\left\langle}
\renewcommand{\>}{\right\rangle}
\newcommand{\herm}{{\sf H}}
\newcommand{\trasp}{{\sf T}}
\newcommand{\transp}{{\sf T}}
\renewcommand{\vec}{{\rm vec}}
\newcommand{\Psf}{{\sf P}}
\newcommand{\SINR}{{\sf SINR}}
\newcommand{\SNR}{{\sf SNR}}
\newcommand{\MMSE}{{\sf MMSE}}
\newcommand{\REF}{{\RED [REF]}}

% Markov chain
\usepackage{stmaryrd} % for \mkv 
\newcommand{\mkv}{-\!\!\!\!\minuso\!\!\!\!-}

% Colors

\newcommand{\RED}{\color[rgb]{1.00,0.10,0.10}}
\newcommand{\BLUE}{\color[rgb]{0,0,0.90}}
\newcommand{\GREEN}{\color[rgb]{0,0.80,0.20}}

%%%%%%%%%%%%%%%%%%%%%%%%%%%%%%%%%%%%%%%%%%
\usepackage{hyperref}
\hypersetup{
    bookmarks=true,         % show bookmarks bar?
    unicode=false,          % non-Latin characters in AcrobatÕs bookmarks
    pdftoolbar=true,        % show AcrobatÕs toolbar?
    pdfmenubar=true,        % show AcrobatÕs menu?
    pdffitwindow=false,     % window fit to page when opened
    pdfstartview={FitH},    % fits the width of the page to the window
%    pdftitle={My title},    % title
%    pdfauthor={Author},     % author
%    pdfsubject={Subject},   % subject of the document
%    pdfcreator={Creator},   % creator of the document
%    pdfproducer={Producer}, % producer of the document
%    pdfkeywords={keyword1} {key2} {key3}, % list of keywords
    pdfnewwindow=true,      % links in new window
    colorlinks=true,       % false: boxed links; true: colored links
    linkcolor=red,          % color of internal links (change box color with linkbordercolor)
    citecolor=green,        % color of links to bibliography
    filecolor=blue,      % color of file links
    urlcolor=blue           % color of external links
}
%%%%%%%%%%%%%%%%%%%%%%%%%%%%%%%%%%%%%%%%%%%



\renewcommand{\X}{\mathcal{X}}
\newcommand{\Aspace}{\mathcal{A}}
\newcommand{\Sspace}{\mathcal{S}}
\newcommand{\Hspace}{\mathcal{H}}
\DeclareMathOperator*{\EV}{\mathbb{E}}
\DeclareMathOperator*{\Var}{\mathbb{V}ar}
\newcommand{\Reals}{\mathbb{R}}
\newcommand{\mdp}{\mathcal{M}}
\newcommand{\one}{\mathds{1}}
\DeclareMathOperator*{\argmax}{arg\,max}
\DeclareMathOperator*{\argmin}{arg\,min}
\DeclareMathOperator{\var}{VaR_{\alpha}}
\DeclareMathOperator{\cvar}{CVaR_{\alpha}}
\renewcommand{\F}{\mathcal{F}}
\newcommand{\J}{\mathcal{J}}
\newcommand{\cmdp}{\mathcal{CM}}
\renewcommand{\cmark}{\ding{51}}%
\renewcommand{\xmark}{\ding{55}}%
\newcommand{\emdp}{\mdp_\ell}
\newcommand{\eAspace}{\mathcal{A}_\ell}
\newcommand{\eSspace}{\mathcal{S}_\ell}
\newcommand{\eP}{P_\ell}
\newcommand{\emu}{\mu_\ell}
\newcommand{\er}{r_\ell}
\newcommand{\es}{s_\ell}
\newcommand{\ea}{a_\ell}
\newcommand{\pomdp}{\mathcal{P}\mdp}
\renewcommand{\R}{\mathcal{R}}
\DeclareMathOperator{\w}{\mathbf{w}}
%-------------------------------------

\usepackage{tikz}
\usetikzlibrary{shapes, arrows.meta, positioning}


\usepackage{enumerate}
\usepackage{wrapfig}

% Page Layout
\geometry{
a4paper,
left=20mm,
right=20mm,
top=20mm,
}

\hypersetup{
    colorlinks,
   linkcolor={pierCite},
    citecolor={pierCite},
    urlcolor={pierCite}
}



\title{Achievable distributional robustness when the robust risk is only partially identified}



\author[1]{Julia Kostin}
\author[2]{Nicola Gnecco\thanks{This work was conducted while NG was at the Gatsby Computational Neuroscience Unit, University College London.}}
\author[1]{Fanny Yang}
% new official EPFL format
\affil[1]{\small Department of Computer Science, ETH Zurich}
% \affil[2]
\affil[2]{\small Department of Mathematics, Imperial College London}



\begin{document}


\maketitle


\begin{abstract}

  In safety-critical applications, machine learning models should generalize well under worst-case distribution shifts, that is, have a small robust risk. Invariance-based algorithms 
  can provably take advantage of structural assumptions on the shifts when the training distributions are heterogeneous enough to identify the robust risk. However, in practice, such identifiability conditions are rarely
  satisfied -- a scenario so far underexplored in the theoretical literature. In this paper, we aim to fill the gap and propose to study the more general setting when the robust risk is only \emph{partially identifiable}. In particular, 
  we introduce the \emph{worst-case robust risk} as a new measure of robustness that is always well-defined regardless of identifiability.
  Its minimum corresponds to
  an algorithm-independent (population) minimax quantity that measures the \emph{best achievable robustness} under partial identifiability.
  While these concepts can be defined more broadly, 
  in this paper 
  we introduce and derive them explicitly for a linear model
  for concreteness of the presentation. 
  First, we show that existing robustness methods are provably suboptimal in the partially identifiable case.
  We then evaluate these methods and the minimizer of the (empirical) worst-case robust risk on real-world gene expression data and find a similar trend:
  the test error of existing robustness methods grows increasingly suboptimal as the 
  fraction of data from unseen environments increases, whereas accounting for partial identifiability allows for better generalization.
  \end{abstract}

  \section{Introduction}\label{sec:intro}
  % \nico{I write [mrw] to flash where I moved or merged related work into main text}

The success of machine learning methods typically relies on the assumption that the training and test data follow the same distribution. However, this assumption is often violated in practice. For instance, this can happen if the test data are collected at a different point in time or space, or using a different measuring device.
%- . This setting is generally known as \emph{distribution shift} \cite{quinonero2022dataset}. \fy{this is a century old problem - if we cite, need to cite carefully - here} 
%The task of creating models which perform well under distribution shifts, known as \emph{domain generalization}, remains one of the biggest challenges in modern machine learning. 
Without further assumptions on the test distribution, 
generalization under distribution shift is impossible.
%the domain generalization problem is, \fy{do we need to use this jargon here? } in general, ill-posed. 
However, practitioners often have partial information about the 
set of possible "shifts" 
% \julia{what are shifts outside of causality?} 
that may occur during test time, inducing a set of \emph{feasible test distributions} that the model should generalize to. We refer to the resulting set as the \emph{robustness set}. 
%strength or structure of possible distribution shifts during test time (e.g., the covariate shift assumption \cite{shimodaira2000improving} or bounded $f$-divergence \citep{ben2010theory, hu2018does}) \fy{here, I expected practical shifts rather than theoretical, probably would hold back with this in first paragraph}, which gives rise to a set of feasible test distributions called the \emph{robustness set} \fy{called by whom?}. 
When a probabilistic model for these possible test distributions is available or estimable, one may aim 
%This goal differs from the common objective of \emph{domain generalization}, which aims 
for good performance on a "typical" held-out distribution using a probabilistic framework. %\fy{add citation?}
When no extra information is given, one possibility is to find a model $\beta$ that has a small risk $\Loss(\beta;\prob)$ on the \emph{hardest} feasible test distribution. More formally, we aim to achieve a small
%With $\Loss(\beta;\prob)$ denoting the population risk of a model $\beta$ for a data distribution $\prob$, 
%the 
robust risk defined by
%can be written as 
\begin{equation}\label{eq:conventional-robust-risk}
   \Lossrob(\beta) \coloneqq \sup_{\prob \in \robset(\thetastar)} \Loss (\beta; \prob),
\end{equation}
where $\robset(\thetastar)$ corresponds to the robustness set which depends on 
%is partially characterized by \fy{dependent on}
%that we assume to be fully characterized by
some true parameter $\thetastar$.
% all possible test sets, and we assume that the robustness set is fully characterized by some true parameter $\thetastar$.
% In security applications this robustness set may, e.g., correspond to a maximal set for a malicious attack on $\Atest$ to remain unnoticed or require expert intervention. \fy{alternative sentence in comments}
In fact, this worst-case robustness aligns with security and safety-critical applications, where a small robust risk is necessary to confidently guard against possible malicious attacks.
%the goal is often to find a model %minimizer 
%that has good performance
%of the robust risk, i.e. a \emph{robust prediction model} that shows the best performance 
%on the \emph{worst-case} distribution out of the robustness set. 
% \fy{maybe we can cut this next sentence for space}
% Note that as robustness prioritizes 
% %safety, ensuring good performance on 
% worst-case distributions which might almost never occur in practice, evaluation often requires evaluation on partially synthetic adversarial benchmarks. 
% \fy{somewhere we should stress the following:} Note that this worst-case robustness notion asks for a different methodology and evaluation than domain generalization - \fy{in a nutshell, domain generalization would test on a random "real" held-out environment, whereas robustness prioritizes safety in environments that specifically do not occur naturally and hence requires artificially adversarially perturbed datasets}


%This estimator is commonly referred to as the \emph{robust predictor}. 
\par
% \fy{content-wise suggestion:}

To find a robust prediction model that minimizes \eqref{eq:conventional-robust-risk}, existing lines of work in distributional robustness assume a known robustness set, i.e., full knowledge of the robust risk objective. They then focus on how to minimize the resulting objective. For instance, in distributionally robust optimization \citep{bental2013robust, duchi2021learning}, or relatedly, adversarial robustness  \citep{goodfellow2014explaining, madry2018towards}, the robustness set is chosen to be some neighborhood (w.r.t. to a distributional distance notion) of the training distribution $\prob$. When multiple training distributions are available, related works aim to achieve robustness against the set of all convex combinations of the training distributions \citep{mansour2008, meinshausen2014, sagawa2019distributionally}.  
%\fy{we should add Mansour's mixture distribution here - they don't belong to either} 
%and $\thetastar$ corresponds to $\prob$ itself. 
In causality-oriented robustness  (see, e.g. \citep{buhlmann2020invariance,meinshausen2018causality,shen2023causalityoriented}) on the other hand, the robustness set is not explicitly given, but implicitly defined. Specifically, it is assumed that certain structural parameters (e.g. specific parts of the structural causal model) %of the model that 
remain invariant across distributions, whereas other parameters shift. 
%some structural parameters (like a graphical structure of the model) remain invariant across distributions, while other distributional parameters may vary. 
The robustness set may or may be not be fully known during training time, depending on the relationship of the varying parameters during training and shift time. 
%if the data , 
%For a given set of training distributions, there exist certain sets of varying parameters (``test shifts'') which render the robust risk identifiable. Similarly, for a given set of test shifts, a heterogeneous enough set of training distributions may identify the robust risk. 

%may then be identified and minimized if enough heterogeneous training environments are available. 

%(either to identify the invariant parameter or when test shifts are similar to training shifts)
%In practice, 
If the robust risk objective is known, methods can be derived to estimate its minimizer. However, in many scenarios such procedures suffer from ineffectiveness.
%compared to vanilla methods such as empirical risk minimization. 
For adversarial robustness for example, it is known that when the perturbations during training and test time differ, the robustness resulting from adversarial and standard training is comparable %\fy{could easily find more from later papers} 
(see, e.g. \cite{Tramer19,kang19}). Similarly, invariance-based methods such as \citep{peters2016causal,rojas2018invariant,arjovsky2020invariant,krueger2021out} often exhibit no advantage over vanilla empirical risk minimization \citep{ahuja2020empirical,ahuja2020invariant}.
In both cases, one of the main failure reasons is that the robust objective, which the final model is evaluated on, is not known during training.  For instance, invariance-based methods often fail on new environments if the true invariant predictor is not identifiable (e.g. \citep{kamath2021does,rosenfeld2020risks}). As a possible solution, in some recent works \citep{rothenhausler2021anchor,shen2023causalityoriented}, the set of feasible test distributions is chosen in a specific way which renders the robust risk objective computable despite the non-identifiability of the invariant parameters. 
%One of the reasons for the failure of invariance-based methods is the non-identifiability of the true invariant predictor (e.g. \citep{kamath2021does,rosenfeld2020risks}).  
%Empirically, this is observed when evaluated on real unseen environments, \citep{ahuja2020empirical,gulrajani2021in},
%Besides being effective only for very specific data-generating models \cite{ahuja2021invariance}, invariance-based methods generally are bound to fail when the heterogeneity of the training data is not enough for a given set of possible test shifts \citep{kamath2021does,rosenfeld2020risks}.
%While prior work has pointed out such non-identifiable scenarios as failure cases, 
%there have been no efforts to quantify algorithm-independent limits. Even though this issue of non-identifiability has been pointed out previously,
%\fy{currently this reads like we mean identifiable in terms of invariant predictor? instead of identifiability of the robust risk?}
In total, the theoretical analysis in prior work remains rather of ``binary'' nature: it either analyzes the fully identifiable case in which invariance-based methods are successful, or simply discards non-identifiable scenarios as failure cases without further quantification of the limits of robustness.
%that have not been previously quantified. 
%prior work so far was primarily satisfied with such a binary statement  - whether identifiability is given or not. \fy{failure when it's not identifiable and else success}
%\fy{well, at least gulrajani seems to say that when you take an average test shift ERM is not worse, does anybody actually evaluate on worst-case perturbed sets and claim it?}. 
%Many different reasons could underlie the latter observation, such as model misspecification or optimization issues [cite] \fy{rosenfeld2022online} \fy{what have people hypothesized so far?, in this sentence i would put the ones that are theoretical/provide analysis}. 
%One possible reason that has not yet been considered is the case when the robust risk simply cannot be identified. In this work we show that in that case, invariance-based methods are not effective. We also formalize how to quantify the best possible robustness for this partially identifiable setting
%In practice, causality- and invariance-based methods often result in wrong representations of the data  \citep{kamath2021does,rosenfeld2020risks} and end up
%Empirically, %it is observed that these methods 
%performing similarly to empirical risk minimization (ERM) that ignores the multi-environment information
%\citep{ahuja2020empirical,gulrajani2021in,rosenfeld2022online}. Many possible explanations for this observation have been proposed in the literature.
%\fy{still need to merge} In our work, we focus on examining/quantifying the impact of \emph{non-identifiability} failure scenario.
%In this work we show that in this case, invariance-based methods are not effective. 
%In particular, we aim to formalize how to quantify the best possible robustness for this partially identifiable setting.
% the scenario where the training data is not heterogeneous enough.
%We argue that the non-identifiable scenario warrants a more detailed discussion. 
%In particular, we extend the discussion of invariance-based methods to include the partially identifiable setting, where not only the causal parameter, but the robust risk \eqref{eq:conventional-robust-risk} is not determinable using training data either. 
In this paper, we aim to include the partially identifiable setting\footnote{Here, we mean partial identifiability of the robust risk, which is reminiscent of outputting uncertainty sets for a quantity of interest in the field of partial identification \cite{tamer2010partial, frake2023perfect}.} in our analysis and more specifically discuss the following question:

% A number of subfields in machine learning and optimization have addressed this problem. For example, in distributionally robust optimization (DRO) \citep{bental2013robust, duchi2021learning}, the parameter $\thetastar$ may be the training distribution $\prob$ and the robustness set the \emph{neighborhood} of $\prob$ in some probability distance metric \citep{kuhn2019wasserstein, mohajerin2018data, gao2022wasserstein, duchi2016variance}. Relatedly, adversarial robustness \citep{goodfellow2014explaining, madry2018towards} studies the risk on worst-case transformations of examples drawn from some distribution $\prob$ and can be seen as equivalent to distribution shift robustness \cite{sinha2017certifying}.
% DRO-type methods minimize the worst-case robustness against arbitrary distribution shifts in the neighborhood without structural assumptions. Although being assumption-agnostic can be viewed as a strength, it also has its caveat: even when available, prior knowledge about the structure of expected test shifts cannot be incorporated.
% %Both a caveat and strength of these methods is that they do not require nor allow incorporating structural assumptions on the distribution shifts, even when some prior knowledge about expected test shifts may be available. 
% In such cases, the robust model's prediction might be overly conservative, resulting in suboptimal performance when the test shifts are in fact more benign. \cite{sagawa2019distributionally}.
% %methods like group DRO \cite{sagawa2019distributionally}, in general, just enlarge their robustness set as the number of training environments increases. \fy{not so sure what you mean here by enlarge? aren't there a lot that talk about convex hull of training in different contexts?}

% In many practical scenarios, data from \emph{heterogeneous sources} is available at training time –  for example, data from different geographic locations or time ranges. 
% Due to the lack of modeling assumptions, multiple environments in the DRO setting cannot, in general, be leveraged to achieve better robustness in a given robustness set -- in those contexts, the 
% presence of multiple environments is usually argued to enable 
% robustness against a larger robustness set. Instead, domain experts can anticipate which aspects of the joint probability distribution of $(X,Y)$ are more likely to shift.
% %, thus allowing for incorporating structural assumptions. 
% Such prior structural information can, for example, be incorporated 
% %particularly well-formalized 
% through the framework of structural causal models (SCMs), via the approach of  \emph{causality-oriented robustness} \citep{meinshausen2018causality, buhlmann2020invariance}. Importantly, the literature in this area has so far focused on settings when the desired robust objective $\Lossrob$ is identifiable, i.e. computable from training data. Traditional causal learning and invariance-based methods aim to fully identify some underlying causal parameter of the SCM for robustness against \emph{all} (potentially infinite) interventions \citep{peters2016causal,rojas2018invariant,arjovsky2020invariant,krueger2021out}. However, the training data is often not heterogeneous enough to fully identify the causal parameter. %At the same time, the interventions during test time may neither happen in all \fy{directions} nor have infinite strength. 
% Thus, another line of work \citep{rothenhausler2021anchor,shi2022nonlinear,kook2022distributional} focuses on the scenario when
% %In such cases, it might not be necessary to identify the full causal parameter, but instead only 
% the causal parameter is not necessarily identifiable, but the test shifts only occur in training directions, rendering the robust risk \eqref{eq:conventional-robust-risk} identifiable. We provide an overview in \Cref{tab:rw} and an additional discussion of related work in \Cref{sec:apx-related_work}.
%previous works on robust generalization under heterogeneous training data sources all
%develop and evaluate algorithms for settings where 
% \fy{cite impossibility paper} \julia{don't get which impossibility paper} 



%It has been widely observed that when the training environments do not
%, in some sense, express 
%have sufficient variability or the setting is slightly misspecified, causality- and invariance-based methods can pick wrong representations of the data \citep{kamath2021does,rosenfeld2020risks}, and their robust performance degrades significantly, often even becoming worse than ERM \citep{ahuja2020empirical,gulrajani2021in,rosenfeld2022online}.  

% For most  causality- or invariance-based algorithms, there are also numerous papers pointing out that this \fy{maybe need to divide reference into causal parameter identification and r.r.id.?} identifiability breaks under slightly modified settings \fy{and/in} that they fail to improve on empirical risk minimization 
%(see \cite{kamath2021does}, \cite{rosenfeld2020risks}, Theorems 3 and 4 \fy{what is this referring to?}) and perform equally or worse than empirical risk minimization (ERM) 
% \citep{ahuja2020empirical, gulrajani2020search, kamath2021does, rosenfeld2020risks, rosenfeld2022online}.
% \fy{i compiled two lists of references together that were separate, check}
%Even minor violations of the identifiability assumptions can cause invariance- and causality-based methods to fail (Examples 4 and 5 in \cite{kamath2021does}, Theorem 3.1 in \cite{rosenfeld2020risks}, Theorems 3 and 4) and perform equally or worse than empirical risk minimization (ERM) \citep{ahuja2020empirical, gulrajani2020search, rosenfeld2022online}, which is oblivious to heterogeneity. 



%\fy{are there actually works that prove in some case there exist no algorithm that can identify? or they just show one particular method can't identify?}
%either finding the robust solution under the assumptions that the robustness set is identifiable from training data, or impossibility results when this is not the case. We summarize our classification of causality- and invariance-based robustness methods in \Cref{tab:rw}. 
% \fy{see~\Cref{tab:rw}- how important is bounded shifts actually?}

% \begin{table}[tbp]
%     \centering
%     \caption{Comparison of various distributional robustness frameworks and what kind of assumptions their analysis can account for (with an incomplete list of examples for each framework). % on identifiability.
%     }\label{tab:rw}
%     \vspace{1pt}
% \resizebox{.8\columnwidth}{!}{%
% % \setlength{\tabcolsep}{0.5pt}
% \begin{tabular}{@{}cccc@{}}
% \toprule
% Framework accounts for~ &
%   \begin{tabular}[c]{@{}c@{}}~bounded~~\\ shifts\end{tabular} & 
%   \begin{tabular}[c]{@{}c@{}}partial identifiability of\\  ~~model parameters ~~\end{tabular} &
%   \begin{tabular}[c]{@{}c@{}}partial identifiability of\\  ~~robustness set\end{tabular} \\ 
%   \toprule
% \begin{tabular}[c]{@{}c@{}}DRO\\ \citep{bental2013robust, duchi2021learning, sinha2017certifying, mohajerin2018data, sagawa2019distributionally} \end{tabular} & \cmark & $-$  & \xmark  \\ \midrule
% \begin{tabular}[c]{@{}c@{}}Infinite robustness \\
% \citep{peters2016causal, fan2023environment, magliacane2018domain, rojas2018invariant,arjovsky2020invariant, ahuja2020invariant,
% shi2021gradient, 
% xie2020risk, krueger2021out, ahuja2021invariance}\\

% \end{tabular}
% & \xmark  & \xmark & \xmark \\ \midrule
% \begin{tabular}[c]
% {@{}c@{}}Finite robustness \\
% \citep{rothenhausler2021anchor, jakobsen2022distributional, christiansen2021causal, kook2022distributional, shen2023causalityoriented}
% \end{tabular} &\cmark  & \cmark  & \xmark  \\ \midrule
% \begin{tabular}[c]
% {@{}c@{}}
% Partially id. robustness\\
% (this work)~~
% \phantom{~}
% \end{tabular} & \cmark & \cmark & \cmark \\ \bottomrule
% \end{tabular}}
% % \vspace{-15pt}
% \end{table} 
\begin{comment}


%\fy{bounded interventions does not have to do with identifying causal parameter?} 
Now a paragraph on fully identifiable causal -> fully identifiable robust risk
%how heterogeneity could be leveraged to find invariant mechanisms in all distributions, ultimately leading to predictive models with better robustness. 

When no prior information is available to describe the set of expected %"realistic" 
distribution shifts, a natural approach is to be robust in 
%When the robustness set is explicitly known \fy{such 
a neighborhood of the training distribution, % \fy{plural here?},
%with respect to some discrepancy measure, 
typically referred to as distributionally robust optimization (DRO), e.g., \citep{bental2013robust, duchi2021learning, sinha2017certifying, mohajerin2018data}. 
% \julia{add more work on DRO}
%DRO methods based on Wasserstein distance \citep{sinha2017certifying, mohajerin2018data} allow for test distributions outside of the training support, unlike the ones based on $f$-divergences \citep{bental2013robust, duchi2021learning}. Group DRO \citep{sagawa2019distributionally} 
%This is sth for which one can compute "optimally" robust estimator explicitly e.g. DRO \fy{maybe also covariate shift here?} 
% However, these robustness sets often need to be overly conservative to cover all possible shifts, resulting in large performance drawbacks "on average" \fy{maybe citation?}
%Considering all test distributions in a discrepancy ball can lead to overly 
However, enforcing good prediction performance in an entire neighborhood can be too conservative and lead to worse generalization on the actual test distribution \citep{hu2018does,frogner2019incorporating}. 
%\fy{clearer that prediction acc. suffers on test} 
%conservative predictions 
%Instead %realistic \fy{?} distribution shifts often exhibit strong structure or lie in a low-dimensional manifold \citep{belkin2003laplacian, block2022intrinsic}. 
%Alternatively, one can assume that some parts of the joint distribution of the data  do not change during test time (e.g., covariate shift \citep{shimodaira2000covariate, sugiyama2008direct} or label shift assumptions \citep{lipton2018detecting,garg2020unified}). Although covariate- and label shift based methods have been successfully applied in some scenarios, in many practical applications, their assumptions are violated \fy{need ref}. 
This paper considers an alternative scenario where partial information about future test distributions
%perturbations at test time 
is available at training time. 
In some settings, one might have access to the marginal test distribution of the covariates, reducing the problem to the domain adaptation setting \citep{pan2009survey, redko2020survey}. In other scenarios, one might have heterogeneous training data that share invariant properties that stay unchanged in all environments, including the test data. For example, suppose that we are conducting a long-term medical study, where data is collected from the same group of patients over the years to predict a health parameter $Y$ from a set $(X_1, ..., X_5)$ of covariates. During training time, we are given data from multiple past studies, and during test time, we want to predict $Y$ from newly collected data $(X_1,...,X_5) \sim \probtest$. Suppose that we have observed shifts of $X_1$ (e.g., age) across the training environments, and the distribution of the rest of the covariates has remained stable. In future data, that distribution might shift, however, since the data is collected in the same hospital, we believe that covariates $X_4, X_5$ will remain unshifted, and $X_2, X_3$ are particularly prone to distribution shift.
%For example, certain covariate distributions and dependencies of patients could be stable across different hospitals and regions due to invariant biological processes. \julia{here: real-world example 6 lines}
%Other times .... for different hospitals/regions some covariates stay unchanged while some may vary \fy{by some strength}

%Often some partial/structural information about the distribution shifts during test time is given. 
%This is the scenario we consider in this paper. 

Such prior structural information in the latter case can be particularly well-formalized through the framework of structural causal models (SCMs), via the approach of  \emph{causality-oriented robustness} \citep{meinshausen2018causality, buhlmann2020invariance}.
In this framework, the parameters  $\thetastar$ of the SCM (or parts of them) stay invariant, while distribution shifts can be modeled as (bounded or unbounded) shift interventions on the covariates. 
%where test shifts are induced by \emph{interventions} on the model, as typically studied in \emph{causality-oriented robustness} \citep{meinshausen2018causality, buhlmann2020invariance}. 
%\fy{merge} Some stuff invariant (thetastar) and  distribution shifts can be modeled as interventions on the covariates \citep{meinshausen2018causality, buhlmann2020invariance}. 
%In the end, both infinite and finite robustness methods aim to generalize to a set of distributions induced by the (invariant) causal data-generating process, characterized by some model parameters $\thetastar$, and a set of admissible (bounded or unbounded) test shifts. 
The goal is then to leverage (heterogeneous) training data to find a model that generalizes to a set of distributions induced by $\thetastar$ and a set of interventions. 
%the causal parameters  (invariant) causal data-generating process
This is informally captured by the
% \setlength{\abovedisplayskip}{3pt}
% \setlength{\belowdisplayskip}{3pt}
%This results in the 
following optimization problem:
\begin{equation}\label{eq:conventional-robust-risk}
   \betarob \coloneqq \argmin_{\beta} \sup_{\substack{\text{shift} \in \\ \text{shift set}}} \Loss (\beta; \prob^{\thetastar}_{\text{shift}}),
\end{equation}
where $\Loss(\beta; \prob^{\thetastar}_{\text{shift}})$ is the risk of an estimator $\beta$ on the distribution $\prob^{\thetastar}_{\text{shift}}$
%generated by the 
induced by the SCM with parameters $\thetastar$ and shifts from a given set of shifts/interventions.
%perturbed by a specific shift from the shift set. 
The minimizer of \cref{eq:conventional-robust-risk} ensures robustness with respect to the \emph{robustness set} $\{\prob^{\thetastar}_{\text{shift}}:\text{shift} \in  \text{shift set}\}$. However, in general this set is not fully known during training. %from training distribution. 
In prior works, the robustness set is computable 
% under the assumption
%by assuming 
% that 
if
the training distributions 
are heterogeneous enough
% contain enough information \fy{have enough heterogeneity} 
to identify the relevant parts of the causal model --
%for robust prediction 
this could involve finding a 
% which could either be a 
transformation of the covariates $X$ (also called representation) so that the conditional distribution of the labels given the transformed covariates is invariant \citep{peters2016causal, fan2023environment, magliacane2018domain, rojas2018invariant,arjovsky2020invariant, ahuja2020invariant,
shi2021gradient, 
xie2020risk, krueger2021out, ahuja2021invariance},
% \fy{cite also bottleneck bla}, 
or identifying the causal parameters themselves   \citep{angrist1996identification, hartford2017deep, singh2019kernel, bennett2019deep, muandet2020dual}. 
Most settings in both lines of work can achieve
% consider
generalization to test interventions of arbitrary strength 
% \citep{peters2016causal,arjovsky2020invariant} \fy{add IV citations}
% that 
and so we refer to them as \emph{infinite robustness methods}. 
Since it is more realistic to encounter bounded interventions in the real world, this approach would again be too conservative and pessimistic. This observation motivated \emph{finite robustness methods} \citep{rothenhausler2021anchor, jakobsen2022distributional, kook2022distributional, shen2023causalityoriented, christiansen2021causal} that trade off robustness strength against predictive power depending on the maximum expected strength of the test shifts. We refer to \cref{sec:related_work} for a more detailed discussion and comparison of these methods. 
% \fy{pointer to appendix for more detailed discussion}%, achieving \emph{finite robustness}. 
%invariance-based with infinite robustness methods \citep{peters2016causal, fan2023environment, magliacane2018domain, rojas2018invariant, achiam2017constrained}) or by restricting the shift set to directions observed during training time (e.g., finite robustness methods \citep{rothenhausler2021anchor,shen2023causalityoriented, jakobsen2022distributional, kook2022distributional,christiansen2021causal}).

%In general, the robust predictor \prettyref{eq:conventional-robust-risk} can be only computed if the corresponding robustness set $\{\prob^{\thetastar}_{\text{shift}}:\text{shift} \in  \text{shift set}\}$ is identified from training data. However, if the model parameter $\thetastar$ of the data-generating process is unknown, computability of the robustness set from training data cannot be ensured. 


% A recent line of \emph{invariance-based methods} \citep{peters2016causal, fan2023environment, magliacane2018domain, rojas2018invariant, arjovsky2020invariant} assumes the existence of an invertible mapping of the covariates which remains invariant across distribution shifts. In most of these methods, the invariances are formalized through the framework of structural causal models (SCMs) \citep{pearl2009causality}, where distribution shifts can be modeled as interventions on the covariates \citep{meinshausen2018causality, buhlmann2020invariance}. Given a sufficiently rich collection of shifted training environments, invariance-based methods can achieve \emph{infinite robustness}, i.e. identify an invariant prediction model which generalizes well to test interventions of arbitrary strength \citep{peters2016causal,arjovsky2020invariant}. In practice, considering shifts of arbitrary strength on all covariates can lead to overly conservative estimators. A related line of causality-oriented robustness methods \citep{rothenhausler2021anchor, jakobsen2022distributional, kook2022distributional, shen2023causalityoriented, christiansen2021causal} trades off robustness against predictive power depending on the maximum expected strength of the test shifts, achieving \emph{finite robustness}. 
%, and therefore, alternatives have been proposed in, e.g., the Group DRO literature \citep{sagawa2019distributionally, frogner2019incorporating, liu2022distributionally}. However, these methods cannot protect against perturbations larger than those seen during training time and do not provide a clear interpretation of the perturbation class.
% 
% \nico{This paper considers an alternative, more realistic scenario where partial information about the perturbations at test time is available at training time. }
%though the robustness set itself may not be fully given. 
% For example, for the same study conducted in different hospitals, one might anticipate/know which covariates will stay invariant and which ones will experience shifts.

% \julia{causality sentence}
% \nico{[mrw]: One way to model such prior information is to assume that the data are}
% % such prior information can be conveniently formalized if the data are assumed to be 
% generated by a structural causal model (SCM) \citep{pearl2009causality} and the test shifts are induced by \emph{interventions} on the model, as typically studied in \emph{causality-oriented robustness} \citep{meinshausen2018causality, buhlmann2020invariance}. 
% In this case, one can formulate the robustness problem and characterize the robustness set by leveraging the interventional structure of the training and test distributions, and the availability of multiple environments.
% % \julia{IRM sentence} 
% \nico{[mrw]:
% Depending on the strength and direction of interventions, we distinguish between \emph{infinite} and \emph{finite robustness} methods.
% }
% For example, t
% The line of work based on invariance \citep{peters2016causal, fan2023environment, magliacane2018domain, rojas2018invariant, arjovsky2020invariant} uses multiple training environments to identify the underlying "stable representation" of the data which does not change across environments. If the data heterogeneity is sufficient to fully identify the invariant representation, such methods are robust against arbitrary shifts in the covariates, \nico{and achieve \emph{infinite robustness}.}
% \nico{[mrw]:
% In real data, however, shifts of arbitrary direction and strength in the covariates are unrealistic. Thus, a different line of work \citep{rothenhausler2021anchor, jakobsen2022distributional, christiansen2021causal, kook2022distributional, shen2023causalityoriented} trades
% off robustness against predictive power to achieve what is known as \emph{finite robustness}.
% The main idea of finite robustness methods is to learn a function that is as predictive as possible while protecting against shifts up to some strength in the directions observed during training time.
% }
% \julia{anchor sentence}
% Similarly, a branch of methods around instrumental variable and anchor regression \citep{rothenhausler2021anchor, jakobsen2022distributional, kook2022distributional, shen2023causalityoriented, christiansen2021causal} exploits heterogeneity of training data to find a predictor which is robust against training-time shifts of bounded (but potentially larger) strength.  

% making the robustness a function of the training environments, making it identifiable by construction .     
% Prior work has focused on the case when the robust predictor is identifiable, i.e. the robustness set is known , and optimal robustness can be achieved by directly minimizing the robust risk $\sup_{\substack{\text{shift} \in \\ \text{shift set}} }\Loss (\prob^{\theta}_{\text{shift}} , \betahat)$. Or in a dual view, for a given estimator, we can find the robustness set for which it minimizes this loss \fy{cite anchor bla}.
% In \emph{causality-oriented robustness}, such prior information can be  well-formalized via the causality assumption and the notion of interventions. 
% \fy{For example, for different hospitals/regions some covariates stay unchanged but some may}
%This is the scenario we consider in this paper \julia{caution, makes it sound like we perturb single covariates}
%one may still be able to identify the robust predictor.

% In this case, the robustness set may still be identifiable from data by leveraging
% structural assumptions on the training and test shifts and availability of multiple heterogeneous environments.
% %when heterogeneous data from similar environments is available \fy{e.g. data from different hospitals (if you want to continue with self-driving cars - different regions)}. 
% \fy{IRM sentence} For example, in the setting of invariance-based methods, identifying
%often, more than one data distribution is available during training. For instance, in the medical setting, data for a particular study may be collected from different hospitals, 
%multiple environments (e.g., the same study conducted in different hospitals.
%In this case, the robust predictor can often be identified by exploiting \emph{invariances} across training distributions to identify which  \fy{do we want to introduce the term "invariance" here? there is tradeoff since many of these terms of overloaded}
% components of the data that result in a stable prediction across different training environments \citep{muandet2013domain, arjovsky2020invariant} can lead to robust solution against all arbitrary shifts.
% %\fy{not so clear or hard to guess what stable prediction means here}. 
% \fy{IV/anchor sentence} Or in the presence of latent confounding, IV/anchor setting, given shifts during training time, one can be robust against shifts 
% In all prior work so far, the provable robustness of the methods heavily rely on identifiability of the robustness set from the training distribution - which is "the same" as assuming similar shifts during training and test.  In particular, it is possible using training data alone to minimize $\sup_{\prob \in \text{rob. set} }\Loss (\prob, \beta)$
% \begin{wraptable}{r}{.5\columnwidth}
% \resizebox{.5\columnwidth}{!}{%
% \setlength{\tabcolsep}{0.5pt}
% \begin{tabular}{@{}c|c|c|c@{}}
% \toprule
% Framework accounts for~ &
%   \begin{tabular}[c]{@{}c@{}}~bounded~~\\ shifts\end{tabular} & 
%   \begin{tabular}[c]{@{}c@{}}partial id. of\\  ~~causal param.~~\end{tabular} &
%   \begin{tabular}[c]{@{}c@{}}partial id. of\\  ~~robustness set\end{tabular} \\ 
%   \toprule
% \begin{tabular}[c]{@{}c@{}}DRO\\ \citep{ben2013robust, duchi2021learning, sinha2017certifying, mohajerin2018data, sagawa2019distributionally} \end{tabular} & \cmark & $-$  & \xmark  \\ \midrule
% \begin{tabular}[c]{@{}c@{}}Inf. robustness methods\\
% \citep{peters2016causal, fan2023environment, magliacane2018domain, rojas2018invariant, arjovsky2020invariant}
% \end{tabular}
% & \xmark  & \xmark & \xmark \\ \midrule
% \begin{tabular}[c]
% {@{}c@{}}Finite robustness \\
% \citep{rothenhausler2021anchor, jakobsen2022distributional, christiansen2021causal, kook2022distributional, shen2023causalityoriented}
% \end{tabular} &\cmark  & \cmark  & \xmark  \\ \midrule
% \begin{tabular}[c]
% {@{}c@{}}\\
% \textbf{partially id. robustness}~~\\
% \phantom{~}
% \end{tabular} & \cmark & \cmark & \cmark \\ \bottomrule
% \end{tabular}%
% }
% \caption{Distributional robustness frameworks.}\label{tab:rw}
% \end{wraptable}
%However, in the real world
In practical applications, the training distributions may lack sufficient heterogeneity to ensure the identifiability of the robustness set, and thus, the robustness properties of the methods mentioned before would not be guaranteed.
% required by the aforementioned prior work under their respective assumptions.
%In practical applications, the heterogeneity of the training distributions might not be enough to ensure identifiability of the robustness set that all of the above prior work requires. \fy{not great sentence}
%test shifts are not guaranteed to be very similar to the training shifts, or other necessary assumptions do not hold, rendering the robustness set non-identifiable. 
%Equally, the assumptions necessary for the identification of the robustness set are not likely to hold exactly (for instance, if one only has access to few training environments).  -- this is kind of the same assumption as the one above?
Even minor violations of the identifiability assumptions can cause invariance- and causality-based methods to fail (Examples 4 and 5 in \cite{kamath2021does}, Theorem 3.1 in 
\cite{rosenfeld2020risks}, Theorems 3 and 4) and perform equally or worse than empirical risk minimization (ERM) \citep{ahuja2020empirical, gulrajani2020search, rosenfeld2022online}, which is oblivious to heterogeneity. 
Thus, the analysis of distributional robustness in the setting of \Cref{eq:conventional-robust-risk} is so far limited to either finding the robust solution under the assumptions that the robustness set is identified, or impossibility results when the robustness set is not known. \julia{Here: cite reviewer's impossibility results}


% %%%%%% this was ERM relation specific atttempt 2
% \fy{fanny attempt (content - to polish): (switch in order)} Empirically, invariance-based methods are often critiqued for not working well or better than ERM. 
% %of these methods when their motivating distributional assumptions do not hold, e.g. in the presence of latent confounding.
% \fy{shorten}
% In particular, such \emph{invariance-based} methods for the multi-environment setting are only proven to be beneficial (compared to ERM)
% %beneficial by proving robustness of the method output \fy{of certain causal parameters/latent variables} 
% for specific combinations of "underlying DAG" assumptions, the number of training environments and their "degree of variability" \citep{peters2016causal, arjovsky2020invariant} \fy{also anchor here}.
% %\fy{try 2:} On the constructive side, usually people started with assumptions on DAG or distribution heterogeneity and derived a method that recovers some causal parameters/latent variables. Alternatively, they derived robustness set for which some methods outputs the robust model \fy{discrete anchor}. 
% Failure is then explained by finding a set of similar assumptions/settings under which those methods fail to output these parameters/variables \fy{ICP ones}.
% %%%%%%%%%%%%%%%%%%%




%\fy{try 1:}
%In many real-world scenarios, however, these assumptions often do not hold and the theoretical analysis so far has been "content" with explaining failure by showing  assumptions/settings under which those methods do not identify these parameters/variables. Then they either 1) come up with a new method which can output the desired parameters or 2) derive robustness set/perturbation set for which the method 
%missing identifiability under some DAG assumptions (for icp style things) and "move on" or come up with another algorithm that can solve this assumption.

%\fy{well, they usually end when e.g. causal parameters can't be identified? which is even a stop beyond?}
%and at the same time the possible robustness set is bounded.  

%%%%%%%%%%%%%%%%
% \fy{previous ERM flow}
% However, \fy{changed:} such \emph{invariance-based} methods for the multi-environment setting are only provably beneficial when the number of training environments and their "degree of variability"  is large\citep{peters2016causal, arjovsky2020invariant}.
% When these assumptions are violated, invariance-based methods can fail to identify the robust predictor and
% %do not offer benefits compared to vanilla
% thus perform comparably to 
% empirical risk minimization (ERM) \citep{kamath2021does, rosenfeld2020risks}.\fy{these are icp ones - could be dnagerous?}
% This fact is also reflected in the empirical "belief" that ERM cannot be improved upon in generic settings \citep{gulrajani2021in,vedantam2021an}.
% \fy{the problem is that we still say that some sophisticated method could be better - i think one reason in practice is that they don't adversarially test (but just some random shift)} 
%%%%%%%%%%%%%%%%

%theoretical guarantees for identification of the robust predictor in a multi-environment setting often pose strict assumptions on the number of training environments and their "degree of variability" \citep{peters2016causal, arjovsky2020invariant}. 

%, and 
%In line with these observations, \fy{this is an empirical statement?} it has been argued that ERM might be 
%is the state-of-the-art domain generalization algorithm 
%\fy{can we add these to previous citation block?} 
%In most settings in prior work, even though some methods are proven to fail, others can still perfectly identify the robust solution from the training environments. For scenarios where this is not possible, the statement usually "stops at the impossibility". 

%So far in the literature, the focus has been on assumptions such that robustness set is identifiable. We want to also discuss robustness of any when this is not possible.

\end{comment}

%In practice however, not only the causal parameter is not identifiable, but we cannot even assume that the 


\begin{comment}
In practice, causality- and invariance-based methods often result in wrong representations of the data  \citep{kamath2021does,rosenfeld2020risks}, 
because the setting is slightly misspecified or the data is not heterogeneous enough. 
%In the latter case, the objective cannot be computed from data nor minimized by any algorithm.
%no algorithm would be able to learn the perfectly robust predictor. 
Empirically, %it is observed that 
these methods often perform at least as poorly as empirical risk minimization (ERM) that ignores the multi-environment information
\citep{ahuja2020empirical,gulrajani2021in,rosenfeld2022online}. Although many possible explanations have been proposed in the literature for this breakdown of invariance-based methods, in our work, we concentrate on a sptype of misspecification previously underexplored in the theoretical line of research -- \fy{when the robust risk is not identifiable?}
%lack of identifiability of the invariant representation. 
%In this paper, we propose to extend the discussion of invariance-based methods to include the partially identifiable setting, 
where not only the causal parameter, but the robust risk \eqref{eq:conventional-robust-risk} is not determinable using training data either.
\end{comment}

%In this paper, we argue that it is important to also consider a scenario that is likely to happen in practice -- where we cannot fully identify the robustness set but can still partially leverage invariances.

%crucial for practice to 
%consider the setting when not only the causal parameter is not identifiable but the robust risk \fy{and its minimizer?} in \Cref{eq:conventional-robust-risk} is not determinable using training data either.
% \fy{somewhere here we could consider putting the 5 covariate example??}\julia{probably not} 
%Specifically, we would like to 
%\begin{tcolorbox}[colframe=white!, top=2pt,left=2pt,right=2pt,bottom=2pt]
% \vspace{-5pt}
\begin{center}
\emph{
%\fy{only one of those} What is the inherent, algorithm-independent difficulty of a robust generalization problem given multiple training environments?
%how difficult is the robust generalization task? \fy{what does it depend on?} 
What is the optimal worst-case performance any model can have for given structural relationships between test and training distributions, and how do existing methods comparatively perform?}
%the partially identifiable setting?}  
\end{center}
% \vspace{-5pt}
% \julia{i feel like the questions became too general somehow}

When the robust risk is not identifiable from the collection of training distributions, 
%it means that instead of computing a single objective, 
we obtain a whole \emph{set} of possible objectives -- all compatible with the training distributions -- that includes 
%, containing, i.a., 
the true robust risk. 
% \Nicola{[here it sounds like a single true risk, later we say all true models. I understand what we mean, but perhaps more clear smth like:] ... of possible objectives that are compatible with the training distributions.} 
%We now want to evaluate
In this case, we are interested in the best achievable robustness for \emph{any algorithm}
that we capture in a quantity
%with the knowledge of this set of robust risks.
% \fy{How would be evaluate the best-achievable robustness in this case (for algorithms with this knowledge)? }
%In order to evaluate performance in the partially identifiable setting, 
%For this purpose, we introduce the concept of
called the \emph{\idRR}:
%As a metric 
%, we propose the notion of a \emph{identifiable robust risk} defined by
\begin{align}
\label{eq:identifiable-robust-risk}
    \Lossrobpi(\beta) := \sup_{\substack{\text{possible}\\ \text{true model $\thetastar$ }}}  \sup_{\prob \in \robset(\thetastar)} \Loss (\beta; \prob).
    %\text{shift bound} \fy{bound or set?}) = \sup_{\substack{\text{possible}\\ \text{true model $\theta$ }}} \sup_{\substack{\text{shift} \in \\ \text{shift set}} }\Loss (\beta; \prob^{\theta}_{\text{shift}} ).
\end{align}
Note that $\Lossrobpi(\beta)$ is well-defined %given the training distributions 
even when the standard robust risk is not identifiable -- 
it takes the supremum over the robust risks induced by 
%where the supremum is taken over all 
all model parameters $\thetastar$
that are consistent with the given set of training distributions. 
% \fy{not super happy with notation, here you don't see how different parts of the puzzle come in} \julia{we can change the inner sup to the robust risk, but i don't know what to change about the outer sup}
Furthermore, the minimal value of the worst-case robust risk corresponds to the optimal worst-case performance in the partially identifiable setting.
Spiritually, this \emph{minimax population quantity} is reminiscent of the 
%in settings where the robust risk cannot be identified.
algorithm-independent limits in classical statistical learning theory \cite{yu1997assouad}.\footnote{In particular, extending \eqref{eq:identifiable-robust-risk} to its finite-sample counterpart would introduce a more natural extension of the classical minimax risk statistical learning theory.  In this work, we focus on identifiability aspects instead of statistical rates.}
Even though our partial identifiability framework 
%is
%The identifiable robust risk~\eqref{eq:identifiable-robust-risk} is 
%defined generally and 
can be evaluated for arbitrary %restrictive or general 
modeling assumptions on the distribution shift (such as covariate/label shift, DRO, etc.), %However,
we present it
in \Cref{sec:setting} for a concrete linear setting for clarity of the exposition.
%the identifiable robust risk in a concrete instantiation of our partial identifiability framework~\eqref{eq:identifiable-robust-risk}.
Specifically, the setting is motivated by structural causal models (SCMs) with unobserved confounding (cf. \cref{sec:setting}), similar to the setting of instrumental variables (IV) and anchor regression \citep{rothenhausler2021anchor,saengkyongam2022exploiting}. Concurrent to our work, \cite{bellot2022partial} proposed a similar framework derived explicitly from structural causal models, with quantities that are closely related to our worst-case robust risk and its minimizer. 
%\fy{if we put this footnote or your original sentence in the main text this hinders the flow, but putting it in the footnote is also odd though? - leaving out more details would also be ok for me} \footnote{While we derive closed-form formulas on a continuous regression example, they analytically compute these values for specific discrete examples and empirically evaluate them on colored MNIST.}.
%framework for partial transportability which is conceptually related to our notion of \idRR. However, their approach leverages graphical assumptions, i.e., a priori knowledge about the structure of causal models during training and test time, whereas our focus is a more agnostic multi-environment setting. 
%In particular, 
%his instantiation allows us to explicitly compute the  identifiable robust risk and the optimal worst-case predictor and compare it with existing baselines. 


%For linear structural causal models, we first derive information-theoretic population lower bounds that provide a fine-grained characterization of robustness limitations for bounded additive test shifts and then provide a general method, called the \emph{identifiable robust prediction model}, that can achieve this lower bound.
%The \idRR~\eqref{eq:identifiable-robust-risk} quantifies the robust generalization performance of any predictor in the partially identifiable setting.
%represents a notion of algorithm-independent optimality for any combination of training and test shifts. 
In \Cref{sec:main-results}, we derive the \idRR~\eqref{eq:identifiable-robust-risk} and its minimum for the linear setting, and show theoretically and empirically that the ranking and optimality of different robustness methods change drastically in identifiable vs. partially identifiable settings.  Further, although the worst-case robust predictor derived in the paper is only provably optimal for the linear setting, experiments on real-world data in \Cref{sec:experiments} suggest that our estimator may significantly improve upon other invariance-based methods in more realistic scenarios. 
%Our experimental results 
Our experimental results provide evidence that evaluation and benchmarking on partially identifiable settings are important for determining the effectiveness of robustness methods.
%The same can be observed in experiments on real-world data. 
%in the presence of previously unobserved test shifts. 

%\fy{The framework instantiated for the particular linear SCM in this paper showcases the relevance of considering partial identifiability. The experimental results strongly suggest that benchmarking in general partial-identifiable settings is important to evaluate the robustness of a invariance-based method -- possibly beyond SCMs and for any distribution shift model. might also relevant for other multi-environment robustness models facing identifiability problems...}
%\fy{moved: Our analysis of partial identifiability in the SCM setting opens up avenues for extensions beyond this setting in future work.}

\begin{comment}
Our framework allows us to 
%We first 
theoretically benchmark robustness methods on this particular setting (cf. \Cref{sec:comp-with-finite-robustness-methods}). We find that in the partially identified setting, the OOD generalization of existing robustness methods is suboptimal, and the gap increases with the magnitude and "non-identifiability" of the test shifts. We validate our findings empirically, first on synthetic Gaussian data replicating our theoretical setting. Our experiments on real-world gene expression data suggest effectiveness of the identifiable robust predictor for more general additive shift scenarios. However, a more systematic evaluation on a diverse collection of datasets is needed. 


Based on our theoretical and empirical findings, we argue that our framework may be useful in practical OOD scenarios when some limited prior knowledge is available about the relationship between the distribution shifts during training and test time. Our findings highlight the importance of the partially identified setting for evaluating robustness methods in the future and open up avenues for extensions beyond the linear setting. \fy{moved: Our analysis of partial identifiability in the SCM setting opens up avenues for extensions beyond this setting in future work.}
\end{comment}
% we do  Benchmarking theoretically on particular setting  + empirically validated (on that setting) with (estimating (training shifts)) lower bound. as expected we see differences and hence we should pay attention to this setting.
% This id.rob. risk also naturally gives rise to a new finite-sample estimator. We study the irr minimizer of a specific setting (linear SCM assumption, specific M, Y unshifted) - 
% even though its optimal only under assumptions on the shift (M unknown, Y shifted) - trivially also confirmed by experiments - real-world experiments suggest effectiveness for more general linear shift scenarios (that would have to be evaluated more systematically).
% In \Cref{sec:setting}, we introduce the notion of an observationally equivalent set of parameters and the minimizer of the identifiable robust risk, as well as the formal definition 
% of the information-theoretic minimax quantity that is the minimum of \Cref{eq:identifiable-robust-risk}
% %induced by the risk \fy{refer to eq above?}.  
% \fy{following sentence does not fit flow:} We argue that our framework may be useful in practical scenarios when some limited prior knowledge is available about the relationship between the shifts during training and test time.
%In such cases, identifiable robust risk 
%In such cases, the notion of the identifiable robust risk offers the possibility to i) quantify/evaluate the difficulty of the robust generalization problem in terms of certain problem parameters and ii) can be directly used to obtain an optimal worst-case estimate \fy{if optimization is possible}.
% In \Cref{sec:main-results}, we then %initialize  \fy{what does initialize mean here? why not compute?}
% \fy{derive expressions for/compute} these quantities 
% %on the example of a linear SCM with unobserved confounding, similar to the setting of IV/anchor regression \cite{rothenhausler2021anchor}, 
% and compare the \emph{optimal} identifiable robust risk with the identifiable robust risk achieved by existing algorithms. 
%"We also compare the minimizer of the identifiable robust risk  (achieving optimality) to give a ballpark how far from optimal existing algorithms are in partial identifiable linear SCM setting"... 
% Finally, %in \Cref{sec:experiments}, 
% we also provide the corresponding empirical evaluation on synthetic Gaussian data as well as real-world gene expression data, and show that there is a significant gap in performance between the minimizer of the empirical identifiable risk and existing robust estimators in cases when the test data exhibits previously unobserved shifts. Thus, we provide a more refined explanation for failure of existing robustness algorithms in a partially identified setting, in particular when test data cannot be "interpolated" from training environments. 
% "Finally, the derived estimator also seems to work better on real data (where we don't do any adversarial evaluation)" \fy{here its important to admit that its trivial that an estimator thats made to minimize the id.rob.risk will be better than other methods, but question is more like how different - but I'm still not super happy with this current comparison cause  ideally: we'd also be comparing other algorithms on settings where only one algo is optimal for and the rest has some identifiability issue??}
\begin{comment}
In this paper, we introduce concepts that 
\paragraph{Our contributions}
In \Cref{sec:prop-invariant-set}, we introduce our framework of partially identified robustness in the context of linear structural causal models with latent confounding and additive distribution shifts. In particular, for any estimator $\beta$, we introduce a new risk measure called the \emph{identifiable robust risk}\footnote{Note that partial identifiability is commonly studied in the context of set-identifying the causal parameter \citep{tamer2010partial} (see \cref{sec:related_work} for a further discussion). Here, additionally, we use the term in the context of identifiability of the robustness set.} that is informally defined as
\begin{align*}
    \Lossrobpi(\beta; \text{shift bound} \fy{bound or set?}) = \sup_{\substack{\text{possible}\\ \text{true model $\theta$ }}} \sup_{\substack{\text{shift} \in \\ \text{shift set}} }\Loss (\beta; \prob^{\theta}_{\text{shift}} ).
\end{align*}
The minimal value of this quantity over all estimators $\beta$ then characterizes the hardness of the problem in the given partially identifiable setting. For example, it can offer guidance to the practitioner, such as whether to collect more data (connecting the result to active causal learning) or use a good algorithm. Since $\Lossrobpi$ is identifiable from data, one can also derive an approximate finite-sample estimator for better robustness in the partially identified setting.
%In line with the majority of works in this area, 
%\julia{ all the sections changed, redo}
%In \Cref{sec:PI-lower-bound}, we then apply the framework on linear structural causal models with latent confounding and additive distribution shifts. 
%The set of training and test distributions may differ in additive interventions on the covariates, while the causal mechanism and confounding stay invariant. 
In~\Cref{sec:identifiability-linear-SCM},  we discuss conditions for risk identifiability in our linear SCM setting and derive a lower bound 
%In a setup similar to anchor regression, we derive a lower bound 
for the identifiable robust risk. This lower bound is tight and achievable for some regimes of shift strength. 
In \cref{sec:comp-with-finite-robustness-methods},
we finally
evaluate the identifiable robust risk of prior robustness algorithms and show theoretically and empirically that the minimizer of the identifiable robust risk outperforms existing methods in partially identified settings.
\end{comment}
% \nico{cannot understand last sentence}
% \fy{sounds a little negative - could also write sth like "outperform"}
% that shows how existing methods can be far from optimal and close to ERM in robust performance in the partially identifiable setting. 
% Finally, in \Cref{sec:optimal-estimator}, based on 
% results in \Cref{sec:PI-lower-bound}, 
% we propose a new estimator that is optimal for bounded distribution shifts under partial identifiability.
%%%%%%%%%%%%%%%%%%% fanny try 1 %%%%%%%%%
% \fy{this somehow didn't fit in flow -maybe fits better to related work and when we actually introduce stuff} The case of unbounded interventions is studied in causality/IV literature, and unless the estimator is equal to the causal parameter, this risk can be infinity. Instead, in our fine-grained analysis, we focus on bounded shifts during test time for the worst-case risk to be well-defined and meaningful. 
% %In case the model may not be fully identified but no "unseen" test shifts are observed, the PI-robust risk reduces to the conventional robust risk and there exists an estimator that can actually reach zero robust risk (in population). We say that in this setting, the robust predictor is identifiable. Such settings have been studied before, like in anchor regression \cite{rothenhausler2021anchor} and distributionally robust gradients (DRIG) \cite{shen2023causalityoriented}.
% In contrast to previous works, (to the best of our knowledge,) we are the first to study distributional robustness under arbitrary bounded additive shifts 
% when the robust predictor may only be partially identifiable.  We first derive information-theoretic lower bounds that provide a fine-grained characterization of robustness limitations for different combinations of training environments and robustness sets during test time. For this purpose, we introduce a new minimax risk measure called the \emph{partially identified (PI) robust risk} ...
% ion to considering a broad class of distribution shifts, we do not require full identifiability of the model or a minimum number of training environments. Instead, we quantify the best possible robustness achievable for a given structure of training and test environments, even when few training environments are present and the causal structural model is not identifiable. To the best of our knowledge, we are the first to study distributional robustness under arbitrary bounded additive shifts in a partially identifiable setting. 
% In \prettyref{sec:prop-invariant-set}, we explicitly describe non-identifiability of the data-generating model in a multi-environment setting and connect it to the notion of \emph{observational equivalence} from the partial identifiability literature [cite]. As a consequence, we show that the robust predictor with respect to a specified robustness set can be computed from training data if and only if the robustness set does not contain "unseen" directions on which the model has not been identified \fy{that does not sound like the major contribution}. In \prettyref{sec:PI-lower-bound}, we introduce a new risk measure, which we call partially identified (PI)-robust loss. This minimax notion quantifies the worst-case robust risk under partial identifiability of the data-generating model. We derive a lower bound on this measure and show that no infinite robustness is possible for distribution shifts in \emph{any} direction not covered by the training data. In 
% \prettyref{sec:optimal-estimator}, we discuss the PI-robust loss and propose a new estimator which is optimal with respect to a bounded set distribution shifts under partial identifiability. Finally, in \prettyref{sec:experiments}, we conduct synthetic and semi-synthetic experiments that confirm our findings. 
%%%%%%%%%%%%%
%%%%%%%%%%%% example %%%%%%%%%%%
% \fy{hm my hunch for ml paper is that the contributions should come basically here}
% To illustrate our question, we use the following toy example: 
% \begin{example}\label{ex:multiple-studies}
%     Suppose that we are conducting a long-term medical study, where data is collected from the same group of patients over the years to predict a health parameter $Y$, e.g. the life expectancy \fy{but do you measure this label?}. We are given data $\{(X^e, Y^e)\}_{e \in \Ecaltrain}$ from multiple past studies, where $\Ecaltrain = \{2010, 2015, 2020 \}$. We assume that the data $(X, Y)$ are generated by an underlying causal model, which is unobserved \fy{here enough to say latent confounding? causality/causal lingo suddenly pops up without a warning ;)}. By observing multiple studies, we are able to partially identify the causal mechanism. We now want to train a model which generalizes best on the future study $(X^{2025}, Y^{2025})$. We expect this study to have a distribution shift compared to past studies, including both previously observed shifts (e.g. for the age variable, which shifts by 5 years with every study and retains the same distribution otherwise), and new, previously unobserved shifts (e.g. changes in blood parameters caused by the Covid-19 pandemic). One could discard all covariates with unpredictable shifts \fy{if its unpredictable how do you know which ones these are?}, however, this would severely diminish the predictive power of the model. The practitioner is left with two questions: 1) Do I have enough data to reliably generalize to the new study? 2) What is the optimal model to train on existing data which still has predictive power, but does not fail too severely even on unobserved test shifts? 
%     \fy{as we are not epxerts in clinical study this can easily look unrealistic and need to be very carefully constructed. at least it didn't read very convincingly to me (both the unpredictable shift and measuring/predicting which Y part)}
% \end{example}
%%%%%%%%%%%%%%%%%%%%%%%%%
% \fy{this seems too much detail at this point}
% In the following sections, we aim to provide a fine-grained analysis of robustness limitations in settings like \prettyref{ex:multiple-studies}. We consider data generated by a linear structural causal model, in which the response variable $Y$ is caused by covariates $X$. However, further correlations can be present through unobserved variables $H$, rendering the true causal relationship unidentifiable. As an input, we consider multiple training distributions $\{(X^e, Y^e)\}_{e \in \Ecaltrain}$, which differ by (bounded) additive shifts. Given this data, we are want to be robust to test distribution shifts of a specified (but arbirtary) strength and direction. It is known that \emph{infinite robustness} in arbitrary shift directions is generally impossible unless one can identify the true causal parameter, which, however, requires as many (sufficiently different) environments as there are covariates \cite{peters2016causal}. We find that existing \emph{finite robustness} methods only achieve robustness in directions on which the causal predictor is identified, leaving open the question what happens if the test shift occurs in a previously unobserved direction. To answer this question, we introduce a new minimax risk measure for an estimator $\betahat$, called the \emph{partially identified (PI) robust risk}:
% \begin{align*}
%     \Lossrobpi(\text{shift bound}, \betahat) = \sup_{\substack{\text{ground-truth}\\ \text{model $\theta$ }}} \sup_{\substack{\text{shift} \leq \\ \text{shift bound}} }\Loss (\prob^{\theta}_{\text{shift}} , \betahat).
% \end{align*}
% In case no "unseen" test shifts are observed or the model is fully identified, the PI-robust risk coincides with the conventional robust risk. Thus, our framework includes settings of anchor regression \cite{rothenhausler2021anchor} and distributionally robust gradients (DRIG) \cite{shen2023causalityoriented}. If these identifiability assumptions are not fullfilled, we find that the minimizer of the PI-robust risk, called the \emph{PI-robust estimator}, achieves better performance on unseen test shifts in partially identified settings than the existing finite robustness methods, which in turn behave similarly to ERM. This might potentially explain some of the previous findings of ERM outperforming domain generalization methods. 
% Depending on the assumptions on the test shift, our results could be interpreted in both domain generalization and domain adaptation settings. Moreover, in our derivation of the lower bound, we identify the most adversarial test shift directions under non-identifiability. This connects our findings to the field of \emph{active learning}: by collecting data corresponding to the most adversarial test shift, one maximally reduces the PI-robust risk and thus the non-identifiability of the robust predictor. 
% \fy{note to self: connection to abstention as per alex's question - to some extent we are doing abstention on the nonidentified directions to some extent (unless the shifts are unbounded where we want to utilize the spurious stuff)}
% We find that ex
% \begin{itemize}
%     \item In the following sections, we [our contribution]
%     \item In causality-oriented robustness methods, a method is frequently developed and then a robustness set for this method is derived.
%     \item We show by following this recipe, one only looks at cases where the robust loss is identifiable.
%     \item we will show that as soon as new directions appear, the methods fail.
%     \item We introduce the minimax quantity and the lower bound [informal]
%     \item we demonstrate how common methods are suboptimal w.r.t. lower bound theoretically and empirically
%     \item we empirically show that the minimizer of the PI-robust loss shows significantly improved robustness compared to methods that don't account for partial identifiability. 
%     \item Our results both useful for DG and DA
%     \item Our results could potentially be applied to active learning settings. 
% \end{itemize}
% In the following sections, we [...]. We show that for 
% If multiple varying data distributions are observed during training (e.g., studies from different hospitals), it is sometimes possible to exploit the heterogeneity to identify which components of the data result in a stable prediction across different environments. \cite{}
% In the recent years, a number of works have studied the setting of \emph{multi-environment}, or \emph{heterogeneous training data}, to identify components of the data, prediction on which remains stable across different distributions. 
% Computing the robust predictor is a challenging task and implicitly requires identifying components of the data, prediction on which remains stable across different distributions. 
% \begin{itemize}
%     \item X In safety-related fields, need for robust predictors 
%     \item X Current attempts: obtain these robust predictors from multi-environment training data (exploit heterogeneity)
%     \item However, in general not enough information in training data to compute the robust predictor. 
%     \item In the past, this has been seen in a binary way: "if not enough information, everything fails, if enough training information --> method to compute robust predictor"
%     \item In this paper, we challenge this binary view and instead aim to answer the question 
%     \item Fundamental question: what is the best possible robustness
%     \item Illustrative example
%     \item 
% \end{itemize}
% In the past years, various types structural assumptions have been explored, such as covariate shift [CITE], label shift [CITE] or similarity of distributions with respect to a probability discrepancy measure.

% \julia{include a simple, illustrative example along the lines of the following}:
% \begin{example}
%     Suppose that we are conducting a long-term medical study, where the same blood parameters are collected from the same group of patients over the years to predict a health parameter $Y$. We train our model on a number of past studies $\{S_{2005}, S_{2010}, S_{2015}, S_{2020}\}$. What determines how well this model can predict the results of the future study $S_{2025}$? Assuming a causal data-generating process, we can extract, for instance, how the variable $X_{age}$ affects the prediction, since in every new study, its mean shifts by 5 years, and the variance remains invariant. The same shift is expected for $S_{2025}$, and thus, we can reliably use  $X_{age}$ for prediction. However, some other covariates, previously stable, might shift distributions, for instance due to the Covid-19 pandemic. How can we decide whether to use those covariates for prediction, and if so, to which extent? 
% \end{example}


  \section{Setting}
  \label{sec:setting}We study (stochastic) gradient descent on the empirical risk
\begin{equation*}
\cL(w) = \frac{1}{n}\sum_{i=1}^n l(p_i(w))\, ,
\end{equation*}
where the loss function $l$ and the functions  $(p_i)_{i=1}^n$  are specified in the following assumptions. Note that the empirical risk for binary classification from Equation~\eqref{def:emp_risk_intro} is a special case of the above objective.

\begin{assumption}\label{hyp:loss_exp_log}\phantom{=}
  \begin{enumerate}[label=\roman*)]
    \item The loss is either the exponential loss, $l(q) = e^{-q}$, or the logistic loss, $l(q) = \log(1{+}e^{-q})$.
    \item There exists an integer $L \in \mathbb{N}^*$  such that, for all $1 \leq i \leq n$, the function $p_i$ is $L$-homogeneous\footnote{We recall that a mapping $f : \mathbb{R}^d \rightarrow \mathbb{R}$ is positively $L$-homogeneous if $f(\lambda w) = \lambda^L f(w)$ for all $w \in \mathbb{R}^d$ and $\lambda >0$.}, locally Lipschitz continuous and semialgebraic.
  \end{enumerate}
\end{assumption}
If the $p_i$'s were differentiable with respect to $w$, the chain rule would guarantee that
\begin{align*}
\nabla \mathcal{L}(w) = \frac{1}{n}\sum_{i=1}^n  l'(p_i(w)) \nabla p_i(w)\enspace.
\end{align*}
However, we only assume that the $p_i$'s are semialgebraic. While we could consider Clarke subgradients, the Clarke subgradient of operations on functions (e.g., addition, composition, and minimum) is only contained within the composition of the respective Clarke subgradients. This, as noted in Section~\ref{sec:cons_field}, implies that the output of backpropagation is usually not an element of a Clarke subgradient but a selection of some conservative set-valued field.
Consequently, for $1\leq i \leq n$, we consider $D_i : \bbR^d \rightrightarrows\bbR^d$, a conservative set-valued field of $p_i$, and a function $\sa_i : \bbR^d \rightarrow \bbR^d$ such that for all $w \in \bbR^d$, $\sa_i(w) \in D_i(w)$. Given a step-size $\gamma >0$, gradient descent (GD)\footnote{More precisely, this refers to conservative gradient descent. We use the term GD for simplicity, as conservative gradients behave similarly to standard gradients.} is then expressed as
\begin{equation*}\label{eq:gd_new}\tag{GD}
  w_{k+1} = w_k - \frac{\gamma}{n} \sum_{i=1}^n l'(p_i(w_k))\sa_i(w_k)\,.
\end{equation*}
For its stochastic counterpart, stochastic gradient descent (SGD), we fix a batch-size $1\leq n_b \leq n$. At each iteration $k \in \bbN$, we randomly and uniformly draw a batch $B_k \subset \{1, \ldots, n \}$ of size $n_b$. The update rule is then given by 
\begin{equation*}\label{eq:sgd_new}\tag{SGD}
  w_{k+1} = w_k -  \frac{\gamma}{n_b}\sum_{i\in B_k} l'(p_i(w_k)) \sa_i(w_k)\, .
\end{equation*}
The considered conservative set-valued fields will satisfy an Euler lemma-type assumption.
%\nic{Smoother transition}
\begin{assumption}\phantom{=}\label{hyp:conserv}
  For every $i \leq n$, $\sa_i$ is measurable and $D_i$ is semialgebraic. Moreover, for every $w \in \bbR^d$ and $\lambda \geq 0$, $\sa_i(w)  \in D_i(w)$,
  \begin{equation*}
    D_i(\lambda w) = \lambda^{L-1} D_i(w)\, , \textrm{ and } \quad   L p_i(w) = \scalarp{\sa_i(w)}{w}\, .
  \end{equation*}
\end{assumption}
%\nic{Smoother transition}
Having in mind the binary classification setting, in which $p_i(w) = y_i \Phi(x_i, w)$, we define the margin
\begin{equation}\label{def:marg}
  \sm: \bbR^d \rightarrow \bbR, \quad \sm(w) = \min_{1\leq i \leq n} p_i(w)\, .
\end{equation}
It quantifies the quality of a prediction rule $\Phi(\cdot, w)$. In particular,  the training data is perfectly separated when $\sm(w) >0$. A binary prediction for $x$ is given by the sign of $\Phi(x, w)$, and under the homogeneity assumption, it depends only on the normalized direction $w / \norm{w}$. Consequently, we will focus on the sequence of directions $u_k := w_k / \norm{w_k}$. Our final assumption ensures that the normalized directions $(u_k)$ have stabilized in a region where the training data is correctly classified.

\begin{assumption}\label{hyp:marg_lowb}
  Almost surely, $\liminf \sm(u_k) >0$.
\end{assumption}
Before presenting our main result, we comment on our assumptions.

\paragraph{On Assumption~\ref{hyp:loss_exp_log}.} As discussed in the introduction, the primary example we consider is when $p_i(w) = y_i \Phi(x_i;w)$ is the signed prediction of a feedforward neural network without biases and with piecewise linear activation functions on a labeled dataset $((x_i,y_i))_{i \leq n}$. In this case,
\begin{equation}\label{eq:NN}
 p_i(w) = y_i \Phi(w;x_i) = y_i V_L(W_L) \sigma(V_{L-1}(W_{L-1}) \sigma(V_{L-1}(W_{L-2}) \ldots \sigma(V_{1}(W_1 x_i))))\, ,
\end{equation}
where $w = [W_1, \ldots, W_L]$, $W_i$ represents the weights of the $i$-th layer, $V_i$ is a linear function in the space of matrices (with $V_i$ being the identity for fully-connected layers) and $\sigma$ is a coordinate-wise activation function such as $z \mapsto \max(0,z)$ ($\ReLU$), $z \mapsto \max(az, z)$ for a small parameter $a>0$ (LeakyReLu) or $z \mapsto z$. Note that the mapping $w \mapsto p_i(w)$ is semialgebraic and $L$-homogeneous for any of these activation functions. Regarding the loss functions, the logistic and exponential losses are among the most commonly studied and widely used. In Appendix~\ref{app:gen_sett}, we extend our results to a broader class of losses, including $l(q) = e^{-q^a}$ and $l(q) = \ln (1 + e^{-q^a})$ for any $a \geq 1$.

\paragraph{On Assumption~\ref{hyp:conserv}.} Assumption~\ref{hyp:conserv} holds automatically  if $D_i$ is the Clarke subgradient of $p_i$. Indeed, at any vector $w \in \bbR^d$, where $p_i$ is differentiable it holds that $p_i(\lambda w) = \lambda^{L} p_i(w)$. Differentiating relatively to $w$ and $\lambda$ (noting that $p_i$ remains differentiable at $\lambda w$ due to homogeneity), we obtain $\lambda \nabla p_i(\lambda w) = \lambda^{L} \nabla p_i(w)$ and $\scalarp{\nabla p_i(\lambda w)}{w} = L \lambda^{L-1} p_i(w)$. The expression for any element of the Clarke subgradient then follows from~\eqref{eq:def_clarke}. 

However, for an arbitrary conservative set-valued field, Assumption~\ref{hyp:conserv} does not necessarily hold. For instance, $D(x) = \mathds{1}(x \in \mathbb{N})$ is a conservative set-valued field for $p \equiv 0$, which does not satisfy Assumption~\ref{hyp:conserv}. Nevertheless, in practice, conservative set-valued fields naturally arise from a formal application of the chain rule. For a non-smooth but homogeneous activation function $\sigma$, one selects an element $e \in \partial \sigma (0)$, and computes $\sa_i(w)$ via backpropagation. Whenever a gradient candidate of $\sigma$ at zero is required (i.e., in~\eqref{eq:NN}, for some $j$, $V_j(W_j)$ contains a zero entry), it is replaced by $e$. 
Since $V_j(W_j)$ and $V_j(\lambda W_j)$ have the same zero elements, it follows that for every such $w$, $
\sa_i(\lambda w) = \lambda^L \sa_i(w)$. The conservative set-valued field $D_i$ is then obtained by associating to each $w$ the set of all possible outcomes of the chain rule, with $e$ ranging over all elements of $\partial \sigma(0)$. Thus, for such fields, Assumption~\ref{hyp:conserv} holds.


\paragraph{On Assumption~\ref{hyp:marg_lowb}.} Training typically continues even after the training error reaches zero.
Assumption~\ref{hyp:marg_lowb} characterizes this late-training phase, where our result applies. 
As noted earlier, since $\sm$ is $L$-homogeneous, the classification rule is determined by the direction of the  iterates $u_k=w_k/\norm{w_k}$. Assumption~\ref{hyp:marg_lowb} then states that, beyond some iteration, the normalized margin remains positive. 
This assumption is natural in the context of studying the implicit bias of SGD: we \emph{assume} that we reached the phase in which the dataset is correctly classified and \emph{then} characterize the limit points. A similar perspective was taken in  \cite{nacson2019lexicographic}, where the implicit bias of GF was analyzed under the assumption that the sequence of directions and the loss converge. However, unlike their approach, ours does not require assuming such convergence a priori.

Earlier works such as \cite{ji2020directional,Lyu_Li_maxmargin}, which analyze subgradient flow or smooth GD, establish convergence by assuming the existence of a single iterate $w_{k_0}$ satisfying $\sm(w_{k_0}) > \varepsilon$ and then proving that $\lim \sm(u_{k}) > 0$. Their approach relies on constructing a smooth approximation of the margin, which increases during training, ensuring that $\sm(u_k) > 0$ for all iterates with $k \geq k_0$. This is feasible in their setting, as they study either subgradient flow or GD with smooth $p_i$’s, allowing them to leverage the descent lemma.

In contrast, our analysis considers a nonsmooth and stochastic setting, in which, even if an iterate $w_{k_0}$ satisfying $\sm(w_{k_0}) > \varepsilon$ exists, there is no a priori assurance that subsequent iterates remain in the region where Assumption~\ref{hyp:marg_lowb} holds. From this perspective, Assumption~\ref{hyp:marg_lowb} can be viewed as a stability assumption, ensuring that iterates continue to classify the dataset correctly. Establishing stability for stochastic and nonsmooth algorithms is notoriously hard, and only partial results in restrictive settings exist \cite{borkar2000ode,ramaswamy2017generalization,josz2024global}.

%Finally, note that Assumption~\ref{hyp:marg_lowb} only needs to hold almost surely. Specifically, with probability 1, there exist $k_0$ and $\varepsilon$ such that for all $k \geq k_0$, $\sm(u_k) \geq \varepsilon > 0$. In the case of~\eqref{eq:sgd_new}, $k_0$ and $\delta$ are random variables and may take different values across different realizations. 

%\paragraph{On constant stepsizes.}
%We allow the step size to be a constant of arbitrary magnitude, subject to the stability Assumption~\ref{hyp:marg_lowb}. This may seem surprising in a nonsmooth and stochastic setting, where a vanishing step size is typically required to ensure convergence (see, e.g., \cite{majewski2018analysis, dav-dru-kak-lee-19, bolte2023subgradient, le2024nonsmooth}).
  \section{Theoretical results for 
  the linear setting}
  \label{sec:main-results}
  % Please add the following required packages to your document preamble:

% Beamer presentation requires \usepackage{colortbl} instead of \usepackage[table,xcdraw]{xcolor}
\begin{table*}[t]
\centering
\caption{Main Results. Eurus-2-7B-PRIME demonstrates the best reasoning ability.}
\label{tab:main_results}
\resizebox{\textwidth}{!}{
\begin{tabular}{lcccccc}
\toprule
\textbf{Model}                     & \textbf{AIME 2024}                           & \textbf{MATH-500} & \textbf{AMC}          & \textbf{Minerva Math} & \textbf{OlympiadBench} & \textbf{Avg.}          \\ \midrule
\textbf{GPT-4o}                    & 9.3                                          & 76.4              & 45.8                  & 36.8                  & \textbf{43.3}          & 43.3                   \\
\textbf{Llama-3.1-70B-Instruct}    & 16.7                                         & 64.6              & 30.1                  & 35.3                  & 31.9                   & 35.7                   \\
\textbf{Qwen-2.5-Math-7B-Instruct} & 13.3                                         & \textbf{79.8}     & 50.6                  & 34.6                  & 40.7                   & 43.8                   \\
\textbf{Eurus-2-7B-SFT}            & 3.3                                          & 65.1              & 30.1                  & 32.7                  & 29.8                   & 32.2                   \\
\textbf{Eurus-2-7B-PRIME}          & \textbf{26.7 {\color[HTML]{009901} (+23.3)}} & 79.2 {\color[HTML]{009901}(+14.1)}      & \textbf{57.8 {\color[HTML]{009901}(+27.7)}} & \textbf{38.6 {\color[HTML]{009901}(+5.9)}}  & 42.1 {\color[HTML]{009901}(+12.3) }          & \textbf{48.9 {\color[HTML]{009901}(+ 16.7)}} \\ \bottomrule
\end{tabular}
}
\end{table*}
  
  \section{Conclusion and future directions}
  \label{sec:conclusion}
  \section{Conclusion}
In this work, we propose a simple yet effective approach, called SMILE, for graph few-shot learning with fewer tasks. Specifically, we introduce a novel dual-level mixup strategy, including within-task and across-task mixup, for enriching the diversity of nodes within each task and the diversity of tasks. Also, we incorporate the degree-based prior information to learn expressive node embeddings. Theoretically, we prove that SMILE effectively enhances the model's generalization performance. Empirically, we conduct extensive experiments on multiple benchmarks and the results suggest that SMILE significantly outperforms other baselines, including both in-domain and cross-domain few-shot settings.

\section{Acknowledgements}
\label{sec:acknowledgements}
\section{Acknowledgements}

  
  \bibliography{biblio}
  \bibliographystyle{unsrt}
  % \bibliographystyle{plain} % !!! change to this before submission
  % \bibliographystyle{plainnat}


%%%%%%%%%%%%%%%%%%%%%%%%%%%%%%%%%%%%%%%%%%%%%%%%%%%%%%%%%%%%

\clearpage
\appendix
\hypersetup{
    colorlinks,
   linkcolor={pierCite},
    citecolor={pierCite},
    urlcolor={pierCite}
}
\section*{Appendix}
The following sections provide deferred discussions, proofs and experimental details.
% \DoToC
% \clearpage
% \hypersetup{
%     colorlinks,
%    linkcolor={pierLink},
%     citecolor={pierCite},
%     urlcolor={pierCite}
% }


\section{Extended related work}
\label{sec:apx-related_work}
\section{RELATED WORK}
\label{sec:relatedwork}
In this section, we describe the previous works related to our proposal, which are divided into two parts. In Section~\ref{sec:relatedwork_exoplanet}, we present a review of approaches based on machine learning techniques for the detection of planetary transit signals. Section~\ref{sec:relatedwork_attention} provides an account of the approaches based on attention mechanisms applied in Astronomy.\par

\subsection{Exoplanet detection}
\label{sec:relatedwork_exoplanet}
Machine learning methods have achieved great performance for the automatic selection of exoplanet transit signals. One of the earliest applications of machine learning is a model named Autovetter \citep{MCcauliff}, which is a random forest (RF) model based on characteristics derived from Kepler pipeline statistics to classify exoplanet and false positive signals. Then, other studies emerged that also used supervised learning. \cite{mislis2016sidra} also used a RF, but unlike the work by \citet{MCcauliff}, they used simulated light curves and a box least square \citep[BLS;][]{kovacs2002box}-based periodogram to search for transiting exoplanets. \citet{thompson2015machine} proposed a k-nearest neighbors model for Kepler data to determine if a given signal has similarity to known transits. Unsupervised learning techniques were also applied, such as self-organizing maps (SOM), proposed \citet{armstrong2016transit}; which implements an architecture to segment similar light curves. In the same way, \citet{armstrong2018automatic} developed a combination of supervised and unsupervised learning, including RF and SOM models. In general, these approaches require a previous phase of feature engineering for each light curve. \par

%DL is a modern data-driven technology that automatically extracts characteristics, and that has been successful in classification problems from a variety of application domains. The architecture relies on several layers of NNs of simple interconnected units and uses layers to build increasingly complex and useful features by means of linear and non-linear transformation. This family of models is capable of generating increasingly high-level representations \citep{lecun2015deep}.

The application of DL for exoplanetary signal detection has evolved rapidly in recent years and has become very popular in planetary science.  \citet{pearson2018} and \citet{zucker2018shallow} developed CNN-based algorithms that learn from synthetic data to search for exoplanets. Perhaps one of the most successful applications of the DL models in transit detection was that of \citet{Shallue_2018}; who, in collaboration with Google, proposed a CNN named AstroNet that recognizes exoplanet signals in real data from Kepler. AstroNet uses the training set of labelled TCEs from the Autovetter planet candidate catalog of Q1–Q17 data release 24 (DR24) of the Kepler mission \citep{catanzarite2015autovetter}. AstroNet analyses the data in two views: a ``global view'', and ``local view'' \citep{Shallue_2018}. \par


% The global view shows the characteristics of the light curve over an orbital period, and a local view shows the moment at occurring the transit in detail

%different = space-based

Based on AstroNet, researchers have modified the original AstroNet model to rank candidates from different surveys, specifically for Kepler and TESS missions. \citet{ansdell2018scientific} developed a CNN trained on Kepler data, and included for the first time the information on the centroids, showing that the model improves performance considerably. Then, \citet{osborn2020rapid} and \citet{yu2019identifying} also included the centroids information, but in addition, \citet{osborn2020rapid} included information of the stellar and transit parameters. Finally, \citet{rao2021nigraha} proposed a pipeline that includes a new ``half-phase'' view of the transit signal. This half-phase view represents a transit view with a different time and phase. The purpose of this view is to recover any possible secondary eclipse (the object hiding behind the disk of the primary star).


%last pipeline applies a procedure after the prediction of the model to obtain new candidates, this process is carried out through a series of steps that include the evaluation with Discovery and Validation of Exoplanets (DAVE) \citet{kostov2019discovery} that was adapted for the TESS telescope.\par
%



\subsection{Attention mechanisms in astronomy}
\label{sec:relatedwork_attention}
Despite the remarkable success of attention mechanisms in sequential data, few papers have exploited their advantages in astronomy. In particular, there are no models based on attention mechanisms for detecting planets. Below we present a summary of the main applications of this modeling approach to astronomy, based on two points of view; performance and interpretability of the model.\par
%Attention mechanisms have not yet been explored in all sub-areas of astronomy. However, recent works show a successful application of the mechanism.
%performance

The application of attention mechanisms has shown improvements in the performance of some regression and classification tasks compared to previous approaches. One of the first implementations of the attention mechanism was to find gravitational lenses proposed by \citet{thuruthipilly2021finding}. They designed 21 self-attention-based encoder models, where each model was trained separately with 18,000 simulated images, demonstrating that the model based on the Transformer has a better performance and uses fewer trainable parameters compared to CNN. A novel application was proposed by \citet{lin2021galaxy} for the morphological classification of galaxies, who used an architecture derived from the Transformer, named Vision Transformer (VIT) \citep{dosovitskiy2020image}. \citet{lin2021galaxy} demonstrated competitive results compared to CNNs. Another application with successful results was proposed by \citet{zerveas2021transformer}; which first proposed a transformer-based framework for learning unsupervised representations of multivariate time series. Their methodology takes advantage of unlabeled data to train an encoder and extract dense vector representations of time series. Subsequently, they evaluate the model for regression and classification tasks, demonstrating better performance than other state-of-the-art supervised methods, even with data sets with limited samples.

%interpretation
Regarding the interpretability of the model, a recent contribution that analyses the attention maps was presented by \citet{bowles20212}, which explored the use of group-equivariant self-attention for radio astronomy classification. Compared to other approaches, this model analysed the attention maps of the predictions and showed that the mechanism extracts the brightest spots and jets of the radio source more clearly. This indicates that attention maps for prediction interpretation could help experts see patterns that the human eye often misses. \par

In the field of variable stars, \citet{allam2021paying} employed the mechanism for classifying multivariate time series in variable stars. And additionally, \citet{allam2021paying} showed that the activation weights are accommodated according to the variation in brightness of the star, achieving a more interpretable model. And finally, related to the TESS telescope, \citet{morvan2022don} proposed a model that removes the noise from the light curves through the distribution of attention weights. \citet{morvan2022don} showed that the use of the attention mechanism is excellent for removing noise and outliers in time series datasets compared with other approaches. In addition, the use of attention maps allowed them to show the representations learned from the model. \par

Recent attention mechanism approaches in astronomy demonstrate comparable results with earlier approaches, such as CNNs. At the same time, they offer interpretability of their results, which allows a post-prediction analysis. \par



\section{Extension to the general additive shift setting}\label{sec:apx-extension}

% \begin{enumerate}
%     \item  What if there is no reference environment?
%     \item We show that instead betastar can be identified on $\cup_{e} (\mu_e - \mu_0)  (\mu_e - \mu_0)^\top + (\Sigma_e - \Sigma_0) $.
%     \item Reader can check: the results hold the same, just with new $\cS$
%     \item Highlight: $\noisecovxx$ cannot be identified, instead $\noisecovxx + \Sigma_0$
% \end{enumerate}

We discuss how our setting changes when we relax the assumptions on the existence of the reference environment. We consider the data-generating process in \cref{eqn:SCM}, where $\Ecaltrain = [m]$, $m \in \mathbb{N}$. If no environment $e$ exists with $\mue = 0$ and $\Sigmae = 0$, we first pick an arbitrary distribution $\PrefXY$ as the reference environment\footnote{In practice, it is useful to pick a distribution with the smallest covariance, i.e. $\trace \Cov(\Xref) \leq \trace \Cov(\Xe)$ for all $e$.} .
We denote $\noisecovxx' := \noisecovxxstar + \Sigmaref$. 

First, we show we can express the space $\cS$ of training additive shift directions defined in \cref{eqn:def-S} in the general case. We center all distributions by $\muref$  to obtain centered distributions $\tilde{\Pr}$ that $\EE_{X\sim \tilde{P}_e}[X]=0$. With respect to the arbitrary reference environment, we now define
\begin{align*}
    \tilde{\cS} \coloneqq \range \left[\bigcup_{e \in \Ecaltrain} \left(\Sigmae - \Sigmaref + (\mue - \muref)(\mue - \muref)^\top \right)\right]\subset \R^d.
\end{align*}
We now consider test shifts with respect to the environment $\PrefXY$\footnote{In other words, we require that the test distribution is a shifted version of the (arbitrarily) chosen reference distribution.}. We define the test shift upper bound $\Mtest = \gamma \Mseen + \gammaprime R R^\top$, where $\range(\Mseen) \subset \cS$ and $\range(R) \subset \cSperp$. Again, we can decompose the parameter $\betastar$ as $\betastar = \betastarS + \betastarperp$.
The projection $\betastarS$ of the causal parameter onto the relative training shifts induces the following observationally equivalent parameters corresponding to the reference distribution:
%\fy{still unsure if it should be in prop. shorter prop just make it seem less important as a result} also define the following set of parameters projected onto S ... 
\begin{align*}
    \thetastarS := (\betaS, \noisecovxx', \noisecovxyS, \noisecovyyS) = (\betastarS, \noisecovxx', \noisecovxystar + \noisecovxx' \betastarperp, \noisecovyystar + 2 \langle \noisecovxystar, \betastarperp\rangle + \langle \betastarperp, \noisecovxx' \betastarperp\rangle).
\end{align*}
Again, $\thetastarS$ can be identified from the training distributions and is referred to as the \emph{identified model parameters}. The following adapted version of \cref{prop:invariant-set} shows that assuming shifts on $\PrefXY$, the robust prediction model is only identifiable if the test shifts are in the direction of the relative training shifts:
\begin{proposition}[Identifiability of reference distribution parameters and robust prediction model]
\label{prop:invariant-set-general} Suppose that the set of training and test distributions is generated according to \Cref{eqn:SCM,eqn:testAbound}.
%, and the test distribution is generated under an unknown additive shift bounded by \cref{eqn:testAbound}. 
%$\betastarS$ induces \fy{didn't get "induce" here} a unique tuple of model parameters $\thetacS$ 
Then, $\thetacS$ is observationally equivalent
to $\thetastar$ and computable from training distributions.
%The parameters $\thetacS = (\betastarS, \SigmaS)$ are computable from training data, thus, we will refer to them as the \emph{identified model parameters}. 
Furthermore, it holds that
\begin{enumerate}[(a)]
 \item %\textbf{Identifiability of the model parameters.} 
 the model parameters %$\theta = (\beta, \noisecovxx, \noisecovxy, \noisecovyy)$ 
 generating the reference distribution can be identified up to the following \idset: 
\begin{align*}
   \Invset = \{ \betastarS + \alpha, \noisecovxx', \noisecovxyS - \noisecovxx' \alpha, \noisecovyyS - 2 \alpha^\top \noisecovxyS + \alpha^\top \noisecovxx' \alpha \colon \alpha \in \cSperp \}  \ni \thetastar 
\end{align*}
\item %\textbf{Identifiability of the robust predictor.}
the robust prediction model $\betarob$ as defined in \cref{eqn:formula-robust-predictor} is identified up to the set
    \begin{align*}
      \betastarS + (\gamma \projM + \noisecovxx')^{-1} \noisecovxyS + \{ (\gamma \projM + \noisecovxx')^{-1} \alpha\colon \, \alpha \in \range(R)  \} \ni \betarob 
    \end{align*}
\end{enumerate}
\end{proposition}
The proof is analogous to \cref{sec:apx-proof-invariant-set}. A version of \cref{thm:pi-loss-lower-bound} for perturbations on the reference environment follows accordingly. 


% As a last comment, another difference between this relaxed setting and the one presented in the main text is that $\EE(X_0X_0^\top) = \Sigma_0 + \noisecovxxstar$, and thus we can only identify $\Sigma_0 + \noisecovxxstar$. This, however, has no consequences on the results of \cref{sec:main-results}.


\section{Comparison to finite robustness methods continued}\label{sec:apx-anchor-connections}
\subsection{The setting of continuous anchor regression \cite{rothenhausler2021anchor}}\label{subsec:anchor}
In this section, we evaluate the \idRRs of the continuous anchor regression estimator.
    In the continuous anchor regression setting, during training we 
    observe the distribution %the training data are observed 
    according to the process $X = M A + \eta$; $Y = {\betastar}^\top X + \xi$, where $A$ is an observed $q$-dimensional anchor variable with mean $0$ and covariance $\Sigma_A$ and $M \in \R^{d \times q}$ is a known matrix. Note that in this setting, we do not have a reference environment, but, since the anchor variable is observed, the distribution of the additive shift $M A$ is known.  The test shifts are assumed to be bounded by $\Mtest = \gamma M \Sigma_A M^\top$. Since $\range(\Mtest )\subset \cS = \range(M)$, no new directions are observed during test time, in other words, $R = 0$. Thus, both the corresponding robust loss and the anchor regression estimator can be determined from training data. It holds that
\begin{align*}
    \betaa = \argmin_{\beta \in \R^d} \Lossrob(\beta; \thetastar,\gamma M \Sigma_A M^\top).
\end{align*}
Again, the pooled OLS estimator corresponds to $\betaa$ with $\gamma = 1$. Similar to the discrete anchor case, in case the test shifts are given by $\Mnew = \gamma M \Sigma_A M^\top + \gammaprime R R^\top$, the worst-case robust risk \eqref{eqn:PI-robust-loss} is given by
\begin{align*}
    \Lossrobpi(\beta; \Invset, \Mnew) = \gammaprime (\Cker + \| R^\top \beta \|_2)^2 + \Lossrob(\beta; \thetastar,\gamma M \Sigma_A M^\top) 
\end{align*}
and for the best worst-case robustness of the anchor estimator it holds 
\begin{align*}
    \Lossrobpi(\betaa, \Invset; \Mtest)&= (\Cker + \| R R^\top (\noisecovxxstar + \gamma M \Sigma_A M^\top )^{-1} \noisecovxyS \| )^2 \gammaprime + \text{const}; \\
    \lim_{\gammaprime \to 0} \Lossrobpi(\betaa, \Invset;\Mnew)/\gammaprime &= \lim_{\gammaprime \to 0} \minimaxPIlossarg{\Mnew}/ {\gammaprime}.
\end{align*}
The above results follow by analogy with \cref{sec:apx-proof-of-corollary}.

\subsection{Distributionally robust invariant gradients (DRIG) \cite{shen2023causalityoriented}}\label{subsec:drig}
DRIG \cite{shen2023causalityoriented} introduce a more general additive shift framework, where a collection of additive shifts $\Ae$ is given with moments $(\mue,\Sigmae)$. For each environment $e$, we observe data $(\Xe, \Ye)$ distributed according to the equations $\Xe= \Ae + \eta; \: \Ye = {\betastar}^\top \Xe + \xi$, where the noise is distributed like in \cref{eqn:SCM}. This DGP arises from the structural causal model assumption as described in \cref{fig:ex-scm}.  DRIG consider more a more general intervention setting, additionally allowing additive shifts of $Y$ and hidden confounders $H$. However, their identifiability results can only be shown for the case of interventions on $X$, and since identifiability of the causal parameter is a crucial part of our analysis, we only consider shifts on the covariates. DRIG assumes existence of a reference environment  $e = 0$ with $\mu_0 = 0$ and for which it is required that the second moment of the reference environment is dominated by the second moment of the training mixture: 
\begin{align*}
    \Sigma_0 \preceq \sum_{e \in [m]} w_e (\Sigma_e + \mu_e \mu_e^\top).
\end{align*}
This assumption allows \cite{shen2023causalityoriented} to derive the DRIG estimator which is robust against test shifts upper bounded by $\Mdrig :=  \gamma \sum_{e \in [m]} w_e (\Sigma_e - \Sigma_0 + \mu_e \mu_e^\top)$. The following lemma allows us to make further statements about $\Mdrig$:
\begin{lemma}\label{lm:span-inclusion}
    Let $A$ and $B$ be positive semidefinite matrices such that $B \preceq A$. Then it holds that $\range(B) \subset \range(A)$. 
\end{lemma}
\begin{proof}
    It suffices to show that $\ker(A) \subset \ker(B)$. ($\ker(A) \subset \ker (B)$ implies that $\range(A) = \ker (A)^\perp  \subset \ker(B)^\perp = \range(B)  $.) Consider $x \in \ker(A)$, $x \neq 0$. Then it holds that $x^\top (A - B) x = x^\top A x - x^\top B x = 0 - x^\top B x \geq 0$, from which it follows that $x^\top B x = 0$ and thus $x \in \ker(B)$. 
\end{proof}
Because of the assumption $\Sigma_0 \preceq \sum_{e \in [m]} w_e (\Sigma_e + \mue \mue^\top)$, by \cref{lm:span-inclusion} it follows that $\range(\Sigma_0) \subset \range \sum_{e \geq 1} ( \Sigma_e + \mu_e \mu_e^\top )$  and thus 
\begin{align*}
\range(\Mdrig) \subseteq \range\left( \sum_{e \geq 1} w_e (\Sigma_e + \mu_e \mu_e^\top) \right). 
\end{align*}
Hence, the robustness directions achievable by DRIG in the "dominated reference environment" setting are the same as the ones under the assumption $\Sigma_0 = 0$. \\
Again, we observe that the test shifts bounded by $\gamma \Mdrig$ are fully contained in the space of identified directions $\cS$. If the test shifts are instead bounded by $\Mnew := \gamma \Mdrig + \gammaprime R R^\top$,  including some unseen directions $\range(R) \subset \cSperp$, the robust risk in the DRIG setting is only partially identified. The worst-case robust risk \eqref{eqn:PI-robust-loss} is given by 
\begin{align*}
    \Lossrobpi(\beta; \Invset, \Mnew) = \gammaprime (\Cker + \| R^\top \beta \|_2)^2 + \Lossrob(\beta; \thetastar, \gamma \Mdrig),
\end{align*}
and again, the DRIG estimator is optimal for infinitesimal shifts $\gammaprime$ and suboptimal for larger $\gammaprime$:
\begin{equation*}
    \begin{aligned}
        \Lossrobpi(\betaDRIG; \Invset,\Mnew) &= (\Cker + \| R R^\top (\noisecovxxstar + \gamma \Mdrig)^{-1} \noisecovxyS \| )^2 \gammaprime + \text{const}; \\ 
\text{whereas }\frac{\minimaxPIlossarg{\Mnew}}{\gammaprime} &= \Cker^2, \: \text{ if } \gammaprime \geq \gammath; \\  \lim_{\gammaprime \to 0} \frac{\minimaxPIlossarg{\Mnew}}{\gammaprime} &= (\Cker + \| R R^\top (\noisecovxxstar + \gamma \Mdrig)^{-1} \noisecovxyS\| )^2.
    \end{aligned}
\end{equation*}
The above results follow by plugging $\Mnew$ with $M := \Mdrig$ into the proof of \cref{cor:estimators} in \cref{sec:apx-proof-of-corollary}.

% Thus, the anchor regression estimator is optimal in the limit of small unseen shifts. However, for larger shift strengths, the worst-case robust risk of both estimators significantly deviates from the best achievable robustness. Moreover, under some conditions on the covariance matrix\footnote{E.g., if $\noisecovxxstar$ is block-diagonalizable w.r.t. $\cS$ and $\cSperp$.}, pooled OLS and the anchor estimator achieve the same rate in $\gammaprime$, showcasing how a finite robustness method can perform similarly to empirical risk minimization if the assumptions on the robustness set are not met.




% We discuss how to extend our results to the general additive shift setting described in \cite{shen2023causalityoriented}. We consider a discrete training environment variable $\Etrain$, which takes values in $[m]$ with probability mass function $\prob(\{\Etrain = e\}) = w_e$ for some positive weights $w_e > 0$. Conditioned on the event $\Etrain = e$, 
% % We denote the data $(X, Y) \in \R^{d+1}$ conditioned on $\Etrain = e$ by $(\Xe, \Ye)$. For each environment $\Etrain = e$, 
% the data are generated according to the SCM~\eqref{eqn:SCM}.
% % \begin{align}\label{eqn:generalSCM}
% %     \Xe &= \Ae + \eta, \\
% %     \Ye &= \betastar^\top X^e + \xi,
% % \end{align}
% % where $\Ae \sim \cN(\mu_e, \Sigma_e)$ and $U = (\eta, \xi) \sim \cN(0, \Sigma_U)$.
% We now discuss how the assumption that there exists a reference environment $e = 0$ for which it holds that $\mu_0 = 0$ and $\Sigma_0 = 0$ relates to the conditions in \citep{shen2023causalityoriented}. First, notice that the first restriction is met by centering all data around $\mu_0$. The second condition is formulated weaker in \cite{shen2023causalityoriented}, where it is solely required that the second moment of the reference environment is dominated by the second moment of the training mixture: 
% \begin{align}
%     \Sigma_0 \preceq \sum_{e \in [m]} w_e (\Sigma_e + \mu_e \mu_e^\top).
% \end{align}

% This assumption allows \cite{shen2023causalityoriented} to derive the DRIG estimator which is robust against test shifts upper bounded by $ \gamma \sum_{e \in [m]} w_e (\Sigma_e - \Sigma_0 + \mu_e \mu_e^\top)$. However, since $\Sigma_0 \preceq \sum_{e \in [m]} w_e \Sigma_e$, it follows that $\range\ \Sigma_0 \subset \cup_{e \geq 1} \range\(\Sigma_e + \mu_e \mu_e^\top )$ (see \cref{lm:span-inclusion}) and thus 
% \begin{align*}
% \range\ \left( \sum_{e \in [m]} w_e (\Sigma_e - \Sigma_0 + \mu_e \mu_e^\top) \right) \subset \range\ \left( \sum_{e \geq 1} w_e (\Sigma_e + \mu_e \mu_e^\top) \right). 
% \end{align*}
% Hence, the robustness directions achievable by DRIG in the $\Sigma_0 = 0$ setting are the same as in the general reference environment setting. 

\section{Empirical estimation of the worst-case robust predictor}\label{sec:apx-empirical-estimation}


In this section, we discuss how to compute the worst-case robust loss and its minimizer from finite-sample multi-environment training data. We first describe the finite-sample setting and provide a high-level algorithm. We then discuss some parts of the algorithm in more detail. Finally, we show that the empirical worst-case robust loss is consistent under certain assumptions.  
Recall that we assume that $\Mtest = \gamma \PSMpop + \gammaprime R R^\top$, where $\gamma, \gammaprime \geq 0$, $\PSMpop $ is a PSD matrix satisfying $\range (\PSMpop) \subset \cS$ and $R$ is a semi-orthogonal matrix satisfying $\range (R) \subset \cSperp$. 

\subsection{Computing the worst-case robust loss}

\begin{algorithm}[h!]
    \caption{Computation of the worst-case robust loss} \label{alg:id-rob-loss}
    \begin{algorithmic}[1]
    \State \textbf{Input:} Multi-environment data $\cD \coloneqq \cup_{e \in \Ecaltrain} \cD_e$, test shift strengths $\gamma, \gammaprime > 0$, test shift directions $\Mtest \in \R^{d \times d}$, upper bound $C > 0$ on the norm of $\betastar$.
    % nuisances $S_0$, $R_0$, $\Cker$, $\betastarS$.
    
    \State \textbf{Step 1:} Estimate the training shift directions $\cShat(\cD)$, its  orthogonal complement $\cSperphat(\cD)$, and the identified linear parameter $\betaShat$.
      % \begin{align*}
    %     S(\cD) = \sum_{e = 1}^m ( \Cov(X^e)- \Cov(X^0) + \mu_e \mu_e^\top - \mu_0 \mu_0^\top)
    % \end{align*}
    \State \textbf{Step 2:} Estimate the identified and non-identified test shift upper bounds $\Mseenemp$, $\Rhat \Rhat^\top$, respectively, from $\Mtest$, $\cShat(\cD)$ and $\cSperphat(\cD)$.
 % \vphantom{$\betaShat$}
 %    \begin{align*}
 %        S_M, \PSM &\gets f(M, S); \quad S^{\perp}_M, \PSperpM \gets f(M, S^{\perp}). 
 %    \end{align*}
    \State \textbf{Step 3:} Estimate the maximum norm $\Ckerhat$ of the non-identified linear parameter.
    % \begin{align*}
    %     \Ckerhat \gets \sqrt{C^2 - \| \betaShat \|_2^2}.
    % \end{align*}
    % \State Compute the loss term on the reference environment:
    % \begin{align*}
    %    \LLref(\beta; \cD_0) &\gets \sum_{i = 1}^{n_0} (Y_{0,i} - \beta^\top X_{0,i})^2. 
    % \end{align*}
    % \State Compute the invariance penalty term:
    % \begin{align*}
    %    \LLinv(\beta; \betaShat, \PSM, \gamma) &\gets \gamma \| \PSM (\beta - \betastarS) \|^2_2. 
    % \end{align*}

    % \State Compute the non-identifiability penalty term \fy{probably you mean C-hat-ker}
    % \begin{align*}
    %    \LLid(\beta; \Cker, \PSperpM, \gamma) &\gets \gamma (\Cker + \| \PSperpM \beta \|_2)^2. 
    % \end{align*}
    \State \vphantom{$\hat{R}$}\textbf{Step 4:} Compute the worst-case robust loss function 
    \begin{align*}
        \LL_n(\beta; \betaShat, \PSM, \PSperpM) &\gets \underbrace{\LLref(\beta; \cD_0)}_{\text{reference loss}} + \underbrace{\LLinv(\beta; \betaShat, \PSM, \gamma)}_{\text{invariance penalty term}} + \underbrace{\LLid(\beta; \Ckerhat, \RRhat, \gammaprime)}_{\text{non-identifiability penalty term}}.
    \end{align*}
    \State \textbf{Return:}  worst-case robust predictor and the estimated minimax "hardness" of the problem: 
    \begin{align*}
        \betarobpihat &\gets \argmin_{\beta \in \R^d} \LL_n(\beta; \betaShat, \PSM, \PSperpM); \\ 
       \hat{\mathfrak{M}}(\cD,\gamma, \gammaprime, \Mtest) &\gets \min_{\beta \in \R^d} \LL_n(\beta; \betaShat, \PSM, \PSperpM).
    \end{align*}
    
        
    \end{algorithmic}
\end{algorithm}


\paragraph{Training data.} We observe data from $m + 1$ training environments indexed by $E \in \Ecaltrain = \{0,..., m\}$, where $E = 0$ represents the reference environment. We impose a discrete probability distribution $\prob^{E}$ on the training environment $E \in \Ecaltrain$, resulting in the joint distribution $(X, Y, E) \sim \prob^{X, Y \mid E} \times \prob^{E}$.  For each environment $E = e$, we observe the samples $\cD_e \coloneqq \{(X_{e,i}, Y_{e,i})\}_{i = 1}^{n_e}$, where $(X_{e, i}, Y_{e, i})$ are independent copies of $(X_e, Y_e) \sim \prob^{X, Y \mid E = e}$. Then, the resulting dataset is $\cD \coloneqq \cup_{e \in \Ecaltrain} \cD_e$ with $n \coloneqq n_0 + \cdots + n_m$. Furthermore, for each environment $E = e$, we define the weights $w_e \coloneqq n_e / n$. 

\paragraph{Computation of the worst-case robust loss.} In \cref{alg:id-rob-loss}, we present a high-level scheme for computing the worst-case robust loss from multi-environment data, which consists of multiple steps. First, nuisance parameters related to the training and test shift directions are estimated, which we describe in more detail below. Afterwards, the three terms of the loss are computed: the (squared) loss $\LLref(\beta; \cD_0)$ on the reference environment is computed as 
\begin{align*}
       \LLref(\beta; \cD_0) = \sum_{i = 1}^{n_0} (Y_{0,i} - \beta^\top X_{0,i})^2. 
    \end{align*}
The invariance penalty term $\LLinv(\beta; \betaShat, \PSM, \gamma)$ (which increasingly aligns any estimator $\beta$ in the direction of the estimated invariant predictor $\betaShat$ as $\gamma \to \infty$) can be computed as following in the linear setting:
    \begin{align*}
       \LLinv(\beta; \betaShat, \PSM, \gamma) = \gamma  (\beta - \betaShat)^\top \PSM (\beta - \betaShat) . 
    \end{align*}
Finally, the non-identifiability penalty term $\LLid(\beta; \Ckerhat, \PSperpM, \gamma)$ can be computed as follows:
    \begin{align*}
       \LLid(\beta; \Ckerhat, \PSperpM, \gammaprime) &= \gammaprime (\Cker + \| \PSperpM \beta \|_2)^2. 
    \end{align*}
The non-identifiability term, with increasing $\gammaprime$, penalizes any predictor $\beta$ towards zero on the subspace $\Rhat$ of non-identified test shift directions. In total, the worst-case robust loss (in the linear setting) equals
\begin{align*}
    \LL_n(\beta; \betaShat, \PSM, \PSperpM) = \sum_{i = 1}^{n_0} (Y_{0,i} - \beta^\top X_{0,i})^2 + \gamma (\beta - \betaShat)^\top \PSM (\beta - \betaShat) + \gammaprime (\Cker + \| \PSperpM \beta \|_2)^2, 
\end{align*}
where we suppress dependence on $C$, $\gamma$ and $\gammaprime$ and only leave the dependence on the nuisance parameters.
\paragraph{Choice/Estimation of nuisance parameters.} We now provide more details on the empirical estimation of the nuisance parameters $\cShat, \cSperphat, \Rhat$, $\PSM$, and $\betaShat$. 
\begin{itemize}
    \item The \textbf{constant} $C$ corresponds to the upper bound on the norm of the true causal parameter $\betastar$. Thus, the practitioner chooses $C$ in advance to ensure that (with high probability) $\| \betastar \|_2 \leq C$. 
    \item The \textbf{training shift directions} $\cShat$ can be computed via 
     \begin{align}\label{eq:S-estimation}
        \cShat(\cD) = \mathrm{range} \left[\sum_{e = 1}^m ( \Cov(X^e)- \Cov(X^0) + \mu_e \mu_e^\top - \mu_0 \mu_0^\top)\right],
    \end{align}
where for $e \in \Ecaltrain$, the matrix $\Cov(X^e)$ is the empirical covariance matrix estimated within the training environment $E = e$, and $\mu_e \in \R^d$ is the empirical mean of the covariates within the training environment $E = e$. Additionally, we compute the orthogonal complement $\cSperphat(\cD)$ of the space $\cShat(\cD)$\footnote{In general, $S(\cD)$ is a proper subspace of $\R^d$ and the RHS of \eqref{eq:S-estimation} corresponds to a sum of low-rank second moments. This can be consistently estimated if, for instance, the rank of each shift is known (e.g. in the mean shift setting), or the covariances have a spiked structure, allowing to cut off small eigenvalues.}. 
\item  Computation of the \textbf{seen and unseen test shift directions.} Multiple options are possible for the practitioner to compute the empirical test shift directions $\Mseenemp$ and $\Rhat \Rhat^\top$. One option is to choose $\Mseenemp = \sum_{e} w_e ( \Cov(X^e)- \Cov(X^0) + \mu_e \mu_e^\top - \mu_0 \mu_0^\top)$, where $w_e$ is the proportions of observations in environment $e \in \Ecaltrain$, akin to anchor regression \cite{rothenhausler2021anchor} and DRIG \cite{shen2023causalityoriented} with an appropriate shift magnitude $\gamma$. Afterwards, $\Rhat \Rhat^\top$ is chosen to be a projection onto an appropriate subspace of $\cSperphat$ (if additional information about test shift directions is available). If no information is given, we can choose $\Rhat \Rhat^\top = \Pi_{\cSperphat}$. Alternatively, if the information about potential test shift directions is given in form of a PSD matrix $\Mtest$, for instance, $\Mtest$ being a projection onto some subspace, we can decompose $\Mtest$ 
into identified and non-identified shift directions (and their corresponding projection matrices). 
Let $\PicShat$ and $\PicSperphat$ denote the projection matrices on $\cShat(\cD)$ and $\cSperphat(\cD)$, respectively. Consider the singular value decompositions $\PicShat \Mtest = U_{\cShat} \Sigma_{\cShat} V_{\cShat}^\top$ and $\PicSperphat \Mtest = U_{\cSperphat} \Sigma_{\cSperphat} V_{\cSperphat}^\top$ Then,  define
\begin{align*}
\Shat &= U_{\cShat}, \quad \Rhat = U_{\cSperphat}.
\end{align*}
The subspaces $\range (\PicShat \Mtest)$ and $\range( \PicSperphat \Mtest)$ are minimal subspaces contained in $\cShat$ and $\cSperphat$, respectively, such that $\range(\Mtest) \subset \range( \PicShat \Mtest) \oplus \range( \PicSperphat \Mtest)$. We can then take as $\PSM$ and $\PSperpM$ their corresponding projection matrices. 
\item The \textbf{identified parameter} $\betaShat$ (approximately) equals the true invariant parameter $\betastar$ on the space of training shift directions $\cShat$. 
% As known from the anchor regression/IV literature \citep{rothenhausler2021anchor,shen2023causalityoriented}, 
As conjectured in the anchor regression literature \citep{rothenhausler2021anchor,shen2023causalityoriented, jakobsen2022distributional} (see, for example, the discussion right after Theorem~3.4 in \citep{jakobsen2022distributional} and Appendix~H.3 therein)
for $\gamma \to \infty$, the estimators $\beta^{\gamma}_{\mathrm{anchor}}$ and $\beta^{\gamma}_{\mathrm{DRIG}}$ converge to the invariant parameter $\betastar$ on $\cS$. Thus, the identified parameter can be estimated as
\begin{align*}
    \betaShat \coloneqq \PicShat \beta^{\infty}_{\mathrm{anchor}} \quad \text{or} \quad \betaShat \coloneqq \PicShat \beta^{\infty}_{\mathrm{DRIG}}
\end{align*}
for the setting of mean or mean+variance shifts, respectively. 
% \nico{cite PULSE Theorem~3.4 and Figure H.6 in Appendix~H.3}
\end{itemize}

% \begin{algorithm}[t]
%     \caption{Computation of the worst-case robust loss} \label{alg:id-rob-loss}
%     \begin{algorithmic}[1]
%     \State \textbf{Input:} Multi-environment data $\cD \coloneqq \cup_{e \in \Ecaltrain} \cD_e$, test shift strength $\gamma > 0$, test shift directions $M \in \R^{d \times d}$, causal parameter upper bound $C > 0$.
%     % nuisances $S_0$, $R_0$, $\Cker$, $\betastarS$.
%     \State Estimate the training shift directions $S(\cD)$ and compute the orthogonal complement $S^{\perp}(\cD)$.
%     \State Estimate the identified causal parameter $\betaShat$ on $S$.
%     % \begin{align*}
%     %     S(\cD) = \sum_{e = 1}^m ( \Cov(X^e)- \Cov(X^0) + \mu_e \mu_e^\top - \mu_0 \mu_0^\top)
%     % \end{align*}
%     \State Estimate the identified and non-identified test shift directions and their projections:\vphantom{$\betaShat$}
%     \begin{align*}
%         S_M, \PSM &\gets f(M, S); \quad S^{\perp}_M, \PSperpM \gets f(M, S^{\perp}). 
%     \end{align*}
%     \State Estimate the norm of the non-identified causal parameter part:
%     \begin{align*}
%         \Ckerhat \gets \sqrt{C^2 - \| \betaShat \|_2^2}.
%     \end{align*}
%     \State Compute the loss term on the reference environment:
%     \begin{align*}
%        \LLref(\beta; \cD_0) &\gets \sum_{i = 1}^{n_0} (Y_{0,i} - \beta^\top X_{0,i})^2. 
%     \end{align*}
%     \State Compute the invariance penalty term:
%     \begin{align*}
%        \LLinv(\beta; \betaShat, \PSM, \gamma) &\gets \gamma \| \PSM (\beta - \betastarS) \|^2_2. 
%     \end{align*}

%     \State Compute the non-identifiability penalty term \fy{probably you mean C-hat-ker}
%     \begin{align*}
%        \LLid(\beta; \Cker, \PSperpM, \gamma) &\gets \gamma (\Cker + \| \PSperpM \beta \|_2)^2. 
%     \end{align*}
%     \State Compute the resulting loss function \fy{typo, should be perp in the id loss}
%     \begin{align*}
%         \LL_n(\beta; \betaShat, \PSM, \PSperpM, \Ckerhat, \gamma) &\gets \LLref(\beta; \cD_0) + \LLinv(\beta; \betaShat, \PSM, \gamma) + \LLid(\beta; \Ckerhat, \PSM, \gamma).
%     \end{align*}
%     \State \textbf{Return:}  worst-case robust predictor and the estimated minimax "hardness" of the problem: 
%     \begin{align*}
%         \betarobpihat &\gets \argmin_{\beta \in \R^d} \LL_n(\beta; \betaShat, \PSM, \PSperpM, \Ckerhat, \gamma); \\ 
%        \hat{\mathfrak{M}}(\cD,\gamma, M) &\gets \min_{\beta \in \R^d} \LL_n(\beta; \betaShat, \PSM, \PSperpM, \Ckerhat, \gamma).
%     \end{align*}
    
        
%     \end{algorithmic}
% \end{algorithm}




% \begin{enumerate}
%     \item \julia{put in the algorithm (very high level), then plug in all parts, then go over to the empirical estimation of our loss. High-level comments in the alg (as comments in the algorithm). cf Piers paper }

%     \item \nico{Fix argument for bound $MM^\top \preceq P_{S_{0}, M} + P_{R_{0}, M}$. Quick fix: no prior info on shift directions, i.e., $M = I$, so $MM^\top \preceq S_0S_0^\top + R_0R_0^\top$ trivially.}

%     \item \nico{Single $\gamma$ or $\gamma$, $\gammaprime$?}

%     \item \nico{No results in the literature for the consistency of the anchor estimator where $\gamma \to \infty$. High-level difficulty: need to define estimator where $\gamma_n \to \infty$ at the right speed.
%     Quick fix: Consider anchor setting like $X = MA + \eta$, where $M$ has distinct singular values [to discuss with Julia]}
% \end{enumerate}

% \paragraph{Nuisance parameters.}
% Let us fix the reference environment $0 \in \Ecaltrain$ and define the symmetric matrix
% \begin{align}\label{eqn:sigma-sim}
%     \Sigma \coloneqq
%     \sum_{e \in \Ecaltrain} w_e \left\{\Sigma_e - \Sigma_0 + (\mu_e - \mu_0)(\mu_e - \mu_0)^\top\right\} \in \R^{d \times d},
% \end{align}
% \julia{Sigma0 and mu0 are zero} where $w_e \coloneqq \prob^E(\{e\})$, $\mu_e \coloneqq \EE[X_e]$, and $\Sigma_e \coloneqq \EE[(X_e - \mu_e)(X_e - \mu_e)^\top]$.
% The symmetric matrix $\Sigma$ uniquely defines the space of training shift directions $\cS \coloneqq \range\(\Sigma)$.
% Let $S_0 \in \R^{d \times q}$ and $R_0 \in \R^{d \times (d - q)}$ denote orthonormal matrices whose columns form a basis for $\cS$ and $\cSperp$, respectively.
% Recall that we have access to some information about the anticipated test shift directions consisting of  a subspace $\cM \coloneqq \range\(M)$\footnote{If no information on the potential test shift directions is given, we choose $M = I_d$.}, where $M \in \R^{d \times r}$ is an orthonormal matrix, and test shift strength $\gamma > 0$. Then, by \cref{lm:upper-bound}, it holds that $MM^\top \preceq P_{S_{0}, M} + P_{R_{0}, M}$, where \julia{fix argument, remove Lemma F2}
% \begin{align}\label{eq:proj-to-check}
%     P_{S_{0}, M} \coloneqq S_0S_0^\top M M^\top S_0S_0^\top \in \R^{d \times d},\quad P_{R_{0},M} \coloneqq R_0R_0^\top M M^\top R_0R_0^\top \in \R^{d \times d}, 
% \end{align} 
% with $\range\(P_{S_{0, M}}) \subseteq \cS$ and $\range\(P_{R_{0, M}}) \subseteq \cSperp$.
% Moreover, define the projection of the causal parameter onto the training directions as
% \begin{align}
%     \betastarS \coloneqq S_0S_0^\top \betastar \in \R^d.
% \end{align}
% We call $\varphi_0 \coloneqq (P_{S_0, M}, \PSperpMpop, \betastarS)$ the true nuisance parameters. 


% \paragraph{Estimation of the nuisance parameters.}
% We now describe how to estimate the nuisance parameters $\varphi_0 = (P_{S_0, M}, \PSperpMpop, \betastarS)$ from the dataset $\cD$.
% First, let $\hat{\Sigma} \in \R^{d\times d}$ be the sample version of~\eqref{eqn:sigma-sim}, where the moments are replaced with sample averages and the environment probabilities $w_e$ with their corresponding sample frequencies.
% Suppose now that the dimension of $\cS$ is known\footnote{This assumption holds, for example, in the anchor regression setting where the anchor variable $A$ is observed (see \cref{subsec:anchor}).}.
% Then, the matrix $S_0 \in \R^{d \times q}$ can be estimated by computing the eigenvectors $\hat{S}$ of $\hat{\Sigma}$ corresponding to the $q$ largest eigenvalues. Similarly, the matrix $R_0 \in \R^{d \times (d - q)}$ can be estimated by computing the eigenvectors $\hat{R}$ of $\hat{\Sigma}$ corresponding to $d - q$ smallest eigenvalues.
% Given knowledge of test shift directions $\cM \coloneqq \range\(M)$, we can then estimate \nico{update this depending on~\eqref{eq:proj-to-check}}
% \begin{align}
%     \hat{P}_{S, M} \coloneqq \hat{S}\hat{S}^\top M M^\top \hat{S}\hat{S}^\top \in \R^{d \times d},\quad \hat{P}_{R,M} \coloneqq \hat{R}\hat{R}^\top M M^\top \hat{R}\hat{R}^\top \in \R^{d \times d}. 
% \end{align} 
% Moreover, the vector $\betastarS$ can be estimated as $\hat{\beta}^\cS\coloneqq $ \nico{...to clarify how}


% \paragraph{Robust identifiable risk and worst-case robust predictor.}
% For fixed constants $C > 0$ and $\gamma, \gammaprime > 0$, 
% the robust identifiable risk is defined for all parameters of interest $\beta \in \R^d$ and nuisance parameters $\varphi \coloneqq (P_S, P_R, b)$ as
% \begin{align}\label{eqn:rob-loss-ell}
%     \LL(\beta, \varphi)  \coloneqq 
%     \EE\left[\left(Y_0 - \beta^\top X_0 \right)^2 \right]
%     + \gamma  \norm{P_S(b - \beta)}_2^2 
%     +   \gammaprime \left(\sqrt{C - \norm{b}_2^2} 
% + \norm{P_R\beta}_2\right)^2,
% \end{align}
% \julia{add discussion on $C$} where $(X_0, Y_0) \sim \prob^{X, Y \mid E = 0}$ denote the predictor-response pair from the reference environment.
% For fixed $\varphi$, the worst-case robust loss is a strongly convex objective function in $\beta$, even though it is not differentiable at $\beta = 0$. The strong convexity implies the existence of a unique minimizer, which we call the worst-case robust predictor and is defined as
% \begin{align}
%     \betarobpi \coloneqq \argmin_{\beta \in \R^d} \LL(\beta, \varphi_0).
% \end{align}
% If $\gamma = 0$, the worst-case robust loss $\LL(\beta, \varphi_0)$ evaluated at the true nuisance parameters $\varphi_0 = (P_{S_0, M}, P_{R_0, M}, \betastarS)$ coincides with the squared loss on the reference environment. If no new shift directions are expected, i.e. $R_0 = 0$ (and correspondingly $\| b \|_2^2 = C$), the loss coincides with the anchor regression/DRIG loss \citep{rothenhausler2021anchor,shen2023causalityoriented}, with robustness set given by $C^\gamma = \{ \Atest \in \R^d: \EE[\Atest {\Atest}^\top] \preceq \gamma P_{S_{0, M}} \}$.
% \nico{$C^\gamma$ confusing with $C$}

% The second penalization term of the worst-case robust loss can be seen as a regularizer along the non-identifiable directions.
% If the training environments are rich enough to identify the causal parameter, i.e., $\cS = \R^d$, $\betastarS = \betastar$, then the loss corresponds to the squared loss on the reference environment with a causal regularization term $\gamma \| P_{S_{0, M}}(\betastarS - \beta) \|_2^2$ which penalizes the prediction models on test shift directions. 
% If the test shift strength $\gamma$ is large enough, the optimal estimator corresponds to the robust estimator constrained to the direct sum $\cS \oplus (\cSperp - \range\ (P_{R_{0, M}}))$ of the "identified" directions $\cS$ and "stable" unidentified directions $\cSperp - \range\ (P_{R_{0, M}})$.

% Thus, the worst-case robust estimator interpolates on two levels: on $\cS$,  it interpolates between the OLS and the causal predictor $\betastarS$ for  $\gamma \in [1, \infty]$\Nicola{$\gamma \in [1, \infty)$} – similarly to anchor regression, which can be seen as the 
% "finite robustness" axis. However, it also interpolates between the anchor estimator and the "abstaining" estimator~\eqref{eqn:abstaining} as the strength of non-identified shifts increases (the "non-identifiability" axis).

\subsection{Consistency of the worst-case robust predictor} 
For any estimator $\beta \in \R^d$ and given the estimated nuisance parameters $\hat\varphi \coloneqq (\PSM, \PSperpM, \betaShat)$, we define the sample worst-case robust risk as
\begin{align}\label{eqn:rob-loss-ell-sample}
\begin{split}
    \LL_n(\beta, \hat{\varphi})  \coloneqq &
    \frac{1}{n_0}
    \sum_{i \in \cD_0}\left(Y_{0,i} - \beta^\top X_{0,i} \right)^2
    + \gamma (\betaShat - \beta)^\top \PSM (\betaShat - \beta) 
    +   \gammaprime \left(\sqrt{C - \norm{\betaShat}_2^2} 
+ \norm{\PSperpM\beta}_2\right)^2.
\end{split}
\end{align}
% \nico{remove any gammaprime}
Correspondingly, we define the estimator of the worst-case robust predictor by
\begin{align}\label{eqn:betarobpi-consistent}
    \betarobpihat \coloneqq \argmin_{\beta \in \mathcal{B}} \LL_n(\beta, \hat{\varphi}),
\end{align}
where $\mathcal{B} \subseteq \R^d$ is some compact set whose interior contains the true parameter $\betarobpi$.


To show the consistency of~\eqref{eqn:betarobpi-consistent}, we first require consistency of the nuisance parameter estimators, which we state as an assumption.
\begin{assumption}\label{ass:consistency-nuisance}
    The estimated nuisance parameters $\hat\varphi \coloneqq (\PSM, \PSperpM, \betaShat)$ are consistent, that is, for $n \to \infty$,
    \begin{align*}
       \norm{\PSM - \PSMpop}_F \stackrel{\prob}{\to}0,
        \quad
        \norm{\PSperpM - \PSperpMpop}_F \stackrel{\prob}{\to}0,
        \quad
        \hat\beta^\cS \stackrel{\prob}{\to} \betastarS \coloneqq \Pi_{\cS} \betastar,
    \end{align*}
    where for any matrix $A \in \R^{m \times q}$, $\norm{A}_F = \sqrt{\trace(A^\top A)}$ denotes the Frobenius norm,  $\PSMpop$ is a PSD matrix with bounded eigenvalues with $\range(\PSMpop) \subset \cS$, and $\PSperpMpop$ is the corresponding population projection matrix onto $\cSperp$. 
\end{assumption}
 Depending on the assumptions of the data-generating process, Assumption~\ref{ass:consistency-nuisance} can be shown to hold. For example, in the anchor regression setting \cite{rothenhausler2021anchor}, the consistency of $\Mseen = \Manchor$, the projection
matrix $\PSperpM$, and $\PicShat$ holds if the dimension of $\cS$ is known (due to the mean shift structure).
The proof relies on the Davis--Kahan theorem (see, for example, \citep{yu2015useful}) and the consistency of the covariance matrix estimator.
Moreover, in the anchor regression setting, it is conjectured that the estimator $\beta^{\infty}_{\mathrm{anchor}}$ converges to its population counterpart (as discussed right after Theorem~3.4 in \citep{jakobsen2022distributional} and Appendix~H.3 therein) which implies that $\betaShat \coloneqq \PicShat \beta^{\infty}_{\mathrm{anchor}} $ consistently estimates $\betastarS = \Pi_{\cS} \betastar$.
% \nico{cite PULSE Theorem~3.4 and Figure H.6 in Appendix~H.3}

Under the assumption of the consistency of the nuisance parameter estimators, we can now show that~\eqref{eqn:betarobpi-consistent} is a consistent estimator of the worst-case robust predictor.

\begin{proposition}\label{prop:consistency-predictor}
  Consider the estimator  $\betarobpihat$ of the worst-case robust predictor defined in~\eqref{eqn:betarobpi-consistent}. Suppose the optimization problem is over a compact set $\cB \subseteq \R^d$ whose interior contains the true minimizer $\betarobpi$. 
  Assume that the covariance matrix $\EE[X_0X_0^\top] \succ 0$ with bounded eigenvalues and $\EE[Y_0^2] < \infty$.
  % \Nicola{Moreover, suppose that $\PSMpop \succeq 0$ with bounded eigenvalues.}
  Then, 
  under Assumption~\ref{ass:consistency-nuisance},
  $\betarobpihat$ is a consistent estimator of~$\betarobpi$.
\end{proposition}

\subsection{Proof of \Cref{prop:consistency-predictor}}

    For ease of notation define $\beta_0 \coloneqq \betarobpi$ and $\hat{\beta} \coloneqq \betarobpihat$.
    For any parameter of interest $\beta \in \cB$ and nuisance parameters $\varphi = (P_S, P_R, b)$,
   define the function 
  \begin{align}\label{eq:g-func}
      (x, y) \mapsto g_{\beta,\varphi}(x, y) 
      \coloneqq (y - \beta^\top x)^2 
      + \gamma \norm{P_S^{1/2}(b - \beta)}_2^2
      + \gamma \left(\sqrt{C - \norm{b}_2^2}  + \norm{P_R \beta}_2\right)^2.
  \end{align}
  Using~\eqref{eq:g-func}, the robust identifiable risk and its sample version defined in~\eqref{eqn:rob-loss-ell-sample} can be written, respectively as
  \begin{align*}
      \LL(\beta, \varphi) = \EE[g_{\beta, \varphi}(X_0, Y_0)],
      \quad
      \LL_n(\beta, \varphi) = \frac{1}{n_0}\sum_{i \in \cD_0} g_{\beta, \varphi}(X_{0,i}, Y_{0,i}).
  \end{align*}
  % can be written as 
  % Similarly, the sample robust identifiable risk defined  can be written as 
    % For any $\beta_1, \beta_2 \in \cB$, define $d(\beta_1, \beta_2) \coloneqq \norm{\beta_1 - \beta_2}_2$.
    % By slight abuse of notation, for any nuisance vector $\nu_1, \nu_2 \in \R^d \times \R^{d \times k}$ also define $d(\nu_1, \nu_2) = \norm{b_1 - b_2}_2 + \norm{S_1 - S_2}_2$.
    Our goal is to show that $\hat{\beta} \stackrel{\prob}{\to}\beta_0$. First, we show that the minimum of the loss is well-separated.

    \begin{lemma}\label{lm:well-separation}
    Suppose that $\EE[X_0X_0^\top] \succ 0$.
    Then, for all $\delta > 0$, it holds that
\begin{align}\label{eqn:well-sep}
    \inf\left\{\LL(\beta, \varphi_0) \colon \norm{\beta - \beta_0}_2 > \delta \right\} > \LL(\beta_0, \varphi_0).
\end{align}
\end{lemma}
    
   Fix $\delta > 0$.  From the well-separation of the minimum from Lemma~\ref{lm:well-separation}, there exists $\varepsilon > 0$ such that 
    \begin{align*}
        \left\{\norm{\hat\beta - \beta_0}_2 > \delta\right\} \subseteq
        \left\{\LL(\hat\beta, \varphi_0)- \LL(\beta_0, \varphi_0) > \varepsilon\right\}.
    \end{align*}
    Therefore,
    \begin{align}
        \prob&\left(\norm{\hat\beta - \beta_0}_2 > \delta \right) \leq 
        \prob\left(\LL(\hat\beta, \varphi_0)- \LL(\beta_0, \varphi_0) > \varepsilon\right) \nonumber\\
        &\quad= 
        \prob\left( 
        \LL(\hat\beta, \varphi_0)- \LL_n(\hat\beta, \varphi_0) + \LL_n(\hat\beta, \varphi_0)-
        \LL_n(\hat\beta, \hat\varphi)
        \right. \nonumber\\
        &\quad\quad\quad\quad
        \left. 
        + \LL_n(\hat\beta, \hat\varphi)
        - \LL_n(\beta_0, \hat\varphi)
        + \LL_n(\beta_0, \hat\varphi)
        -
        \LL(\beta_0, \varphi_0) > \varepsilon
        \right) \nonumber\\
        &\quad \leq
        \prob\left(
        \LL(\hat\beta, \varphi_0)- \LL_n(\hat\beta, \varphi_0) > \varepsilon/4
        \right)
        + \prob\left(
        \LL_n(\hat\beta, \varphi_0)-
        \LL_n(\hat\beta, \hat\varphi) > \varepsilon/4
        \right) \label{eq:consistency-1}\\
        &\quad\quad +
        \prob\left(
        \LL_n(\hat\beta, \hat\varphi)
        - \LL_n(\beta_0, \hat\varphi) > \varepsilon / 4
        \right)
        + \prob\left(
        \LL_n(\beta_0, \hat\varphi)
        -
        \LL(\beta_0, \varphi_0)>\varepsilon/4
        \right) \label{eq:consistency-2}.
    \end{align}
    We now want to prove convergence the four terms in ~\eqref{eq:consistency-1} and~\eqref{eq:consistency-2}. For this, we use the following statements proved in \Cref{sec:auxlemmaproofs}.

\begin{lemma}\label{lm:ulln-2}
    Suppose $\cB \subseteq \R^d$ is a compact set.
    Moreover, assume that the covariance matrix $\EE[X_0X_0^\top] \succ 0$ with bounded eigenvalues and $\EE[Y_0^2] < \infty$.
    Then, as $n, n_0 \to \infty$ it holds that 
    \begin{align}\label{eqn:ulln-2}
            \sup_{\beta \in \cB} |\LL_n(\beta, \varphi_0) - \LL(\beta, \varphi_0)| \stackrel{\prob}{\to} 0.
        \end{align}
\end{lemma}
\begin{lemma}\label{lm:lipschitz}
    As $n \to \infty$, it holds that
    \begin{align}\label{eqn:lipschitz}
            \sup_{\beta\in\mathcal{B}}|\LL_n(\beta, \hat\varphi) - \LL_n(\beta, \varphi_0)| \stackrel{\prob}{\to} 0.
        \end{align}
\end{lemma}
The two terms in~\eqref{eq:consistency-1} converge to 0 by \cref{lm:ulln-2} and \cref{lm:lipschitz}, respectively. The first term in~\eqref{eq:consistency-2} equals 0 since $\hat\beta$ minimizes $\beta \mapsto \LL_n(\beta, \hat\varphi)$. 
Finally, we observe that \begin{align}\label{eqn:ulln-1}
        \sup_{\beta \in \cB} |\LL_n(\beta, \hat\varphi) - \LL(\beta, \varphi_0)| \stackrel{\prob}{\to} 0,
        \end{align}
since we have that
    \begin{align*}
        \sup_{\beta \in \cB} |\LL_n(\beta, \hat\varphi) - \LL(\beta, \varphi_0)|
        \leq &\
        \sup_{\beta \in \cB} |\LL_n(\beta, \hat\varphi) - \LL_n(\beta, \varphi_0)|
        +
        \sup_{\beta \in \cB} |\LL_n(\beta, \varphi_0) - \LL(\beta, \varphi_0)|,
    \end{align*}
    where the first term converges in probability by Lemma~\ref{lm:lipschitz}, and the second term converges in probability by Lemma~\ref{lm:ulln-2}. This implies that the second term in~\eqref{eq:consistency-2} converges to zero. Since $\delta > 0$ was arbitrary, it follows that $\hat{\beta} \stackrel{\prob}{\to}\beta_0$.
    

\subsection{Proof of auxiliary lemmas}
\label{sec:auxlemmaproofs}
\subsubsection{Proof of \Cref{lm:well-separation}}
By definition,
\begin{align*}
    \LL(\beta, \varphi_0) = \EE[(Y_0 - \beta^\top X_0)^2]
    + \gamma \norm{\PSMpop^{1/2}(\betastarS - \beta)}_2^2
      + \gamma \left(\sqrt{C - \norm{\betastarS}_2^2}  + \norm{\PSperpMpop \beta}_2\right)^2.
\end{align*}
Since $\EE[X_0X_0^\top] \succ 0$, the first term is strongly convex in $\beta$.
Moreover, the second and third terms are convex in $\beta$. Therefore, $\LL(\beta, \varphi_0)$ is strongly convex in $\beta$. Since $\LL(\beta, \varphi_0)$ is also continuous in $\beta$, it follows that there exists a unique global minimum. Let $\beta_0$ denote the global minimizer of $\LL(\beta, \varphi_0)$.
% Fix $\delta > 0$ and define $A_\delta \coloneqq \{\LL(\beta, \varphi_0) \colon \norm{\beta - \beta_0}_2 > \delta\}$. We want to show that $\inf A_\delta > \LL(\beta_0, \varphi_0)$.
By the fact that $\LL(\beta_0, \varphi_0)$ is a global  minimum, and by definition of strong convexity, there exists a positive constant $m > 0$ such that, for all $\beta \in \cB$,
\begin{align}\label{eq:strong-conv}
    \LL(\beta, \varphi_0) \geq \LL(\beta_0, \varphi_0) + \frac{m}{2} \norm{\beta - \beta_0}_2^2.
\end{align}
Fix $\delta > 0$.  
Then, by~\eqref{eq:strong-conv},  for all $\beta \in \cB$ such that $\norm{\beta-\beta_0}_2 > \delta$ it holds that
\begin{align*}
    \LL(\beta, \varphi_0) \geq \LL(\beta_0, \varphi_0) + \frac{m\delta^2}{2} > \LL(\beta_0, \varphi_0). 
\end{align*}
Since the inequality holds for all $\beta \in \cB$ such that $\norm{\beta-\beta_0}_2 > \delta$, we conclude that
\begin{align*}
    \inf \{\LL(\beta, \varphi_0)  \colon \norm{\beta-\beta_0}_2 > \delta\} > \LL(\beta_0, \varphi_0).
\end{align*}
Since $\delta > 0$ was arbitrary, the claim follows.

\subsubsection{Proof of \Cref{lm:ulln-2}}
    Recall that for any $\beta \in \cB$
  \begin{align*}
      \LL(\beta, \varphi_0) = \EE[g_{\beta, \varphi_0}(X_0, Y_0)],
      \quad
      \LL_n(\beta, \varphi_0) = \frac{1}{n_0}\sum_{i \in \cD_0} g_{\beta, \varphi_0}(X_{0,i}, Y_{0,i}).
  \end{align*}
    To show the result, we must establish that the class of functions $\{g_{\beta, \varphi_0} \colon \beta \in \cB\}$ is Glivenko--Cantelli. From~\cite{van2000asymptotic}, a set of sufficient conditions for being a Glivenko--Cantelli class is that (i) $\cB$ is compact, (ii) $\beta \mapsto g_{\beta, \varphi_0}(x, y)$ is continuous for every $(x, y)$, and (iii) $\beta \mapsto g_{\beta, \varphi_0}$ is dominated by an integrable function.
    By assumption, (i) holds.
    Moreover, by~\eqref{eq:g-func}, it follows that $\beta \mapsto g_{\beta, \varphi_0}$ is continuous for all $(x, y)$ and thus (ii) holds. We now show that (iii) holds. 
    Since $\cB$ is compact we have that $\sup_{\beta\in\cB} \norm{\beta}_2 = C_1 < \infty$.
    For fixed $\gamma > 0$, and all $(x, y)$, we have that
    \begin{align}
    \label{eq:dominates-i}
    \begin{split}
        g_{\beta, \varphi_0}(x, y) 
        \leq &\ \sup_{\beta \in \cB} |g_{\beta, \varphi_0}(x, y)|\\
        \leq &\ \sup_{\beta \in \cB} (y - \beta^\top x)^2
        +
        2\gamma  \norm{\PSMpop^{1/2}}_F^2 \left(\norm{\betastarS}_2^2 + \sup_{\beta \in \cB} \norm{\beta}_2^2\right)\\
        &+   \gamma \left(\sqrt{C - \norm{\betastarS}_2^2} 
        + \norm{\PSperpMpop}_F \sup_{\beta\in\cB}\norm{\beta}_2\right)^2\\
        \leq &\
        2y^2 + 2C_1^2 \norm{x}_2^2 + K \eqqcolon G(x, y),
    \end{split}
    \end{align} 
    where $K < \infty$ is a finite constant not depending on $(x, y)$.
    Furthermore, we have that
    \begin{align}\label{eq:dominates-ii}
        \EE[G(X_0, Y_0)] = 2 \EE[Y_0^2] + 2C_1^2\ \trace (\EE[X_0X_0^\top]) + K < \infty,
    \end{align}
    since $\EE[Y^2] < \infty$ and $\EE[X_0X_0^\top]$has bounded eigenvalues by assumption. From~\eqref{eq:dominates-i} and~\eqref{eq:dominates-ii}, it follows that (iii) holds.

\subsubsection{Proof of \Cref{lm:lipschitz}}
For fixed $\gamma > 0$, we have that
\begin{align}
    \frac{1}{\gamma} & \sup_{\beta \in \cB}|\LL_n(\beta, \hat\varphi) - \LL_n(\beta, \varphi_0)|
    \leq
    \sup_{\beta \in \cB}\left|
    \norm{\PSM^{1/2}(\hat\beta^{\cS} - \beta)}_2^2
    - \norm{\PSMpop^{1/2}(\betastarS - \beta)}_2^2
    \right| \label{eq:lip-step-1}
    \\
    & + 
    \sup_{\beta \in \cB}\left |\left(\sqrt{C - \norm{\hat{\beta}^{\cS}}_2^2}  + \norm{\PSperpM \beta}_2\right)^2 
    - \left(\sqrt{C - \norm{\betastarS}_2^2}  + \norm{\PSperpMpop \beta}_2\right)^2\right|.
    \label{eq:lip-step-2}
\end{align}
We first show that ~\eqref{eq:lip-step-1} converges in probability to 0. 
\begin{align}
    &\sup_{\beta \in \cB}\left|
    \norm{\PSM^{1/2}(\hat{\beta}^{\cS} - \beta)}_2^2
    - \norm{\PSMpop^{1/2}(\betastarS - \beta)}_2^2
    \right| \notag\\
    = &\ 
    \sup_{\beta \in \cB}
    \left|(\hat{\beta}^{\cS} - \beta)^\top \PSM(\hat{\beta}^{\cS} - \beta) 
    - (\betastarS - \beta)^\top \PSMpop (\betastarS - \beta) \right| \notag\\
    = &\
    \sup_{\beta \in \cB}\left|(\hat{\beta}^{\cS} - \beta)^\top \PSM (\hat{\beta}^{\cS} - \betastarS)
    + (\hat{\beta}^{\cS} - \betastarS)^\top \PSM(\betastarS - \beta) \right. \notag \\
    & \left. \quad\quad + (\betastarS - \beta)^\top (\PSM - \PSMpop) (\betastarS - \beta) \right| \notag\\
    \stackrel{\clubsuit}{\leq}&\ 
     \sup_{\beta \in \cB} \norm{\hat{\beta}^{\cS} - \beta}_2\ \norm{\PSM}_F\ \norm{\hat{\beta}^{\cS} - \betastarS}_2
     + \sup_{\beta \in \cB} \norm{\betastarS - \beta}_2\ \norm{\PSM}_F\ \norm{\hat{\beta}^{\cS} - \betastarS}_2 \notag \\
     & \left. \quad\quad +
     \sup_{\beta \in \cB} \norm{\betastarS - \beta}_2^2\ \norm{\PSM - \PSMpop}_F\right. \label{eq:bound-1-1},
\end{align}
where $\clubsuit$ follows from the Cauchy--Schwarz inequality and from the fact that $\norm{A}_2 \leq \norm{A}_F$.
For any $\delta_1, \delta_2 > 0$, define the event
\begin{align*}
    A_n \coloneqq \left\{\norm{\betaShat - \betastarS} \leq \delta_1, \norm{\PSM - \PSMpop}_F \leq \delta_2\right\},
\end{align*}
and note that $\prob(A_n) \to 1$ as $n \to \infty$ from Assumption~\ref{ass:consistency-nuisance}.
On the event $A_n$, it holds
\begin{align*}
    \sup_{\beta \in \cB} \norm{\hat{\beta}^{\cS} - \beta}_2 \leq \norm{\hat{\beta}^{\cS} - \betastarS}_2 + \sup_{\beta \in \cB}  \norm{\betastarS - \beta}_2 \leq \delta_1 + C_1,\\
    \norm{\PSM}_F \leq \norm{\PSM - \PSMpop}_F + \norm{\PSMpop}_F \leq \delta_2 +  C_2,   
\end{align*}
where $C_1 < \infty$ follows from the compactness of~$\cB$ and $C_2$ follows from the fact that  $\PSMpop$ has bounded eigenvalues.
Therefore, on the event $A_n$, we can upper bound~\eqref{eq:bound-1-1} by
\begin{align}
    (\delta_1 + C_1) (\delta_2 + C_2)\ \norm{\hat{\beta}^{\cS} - \betastarS}_2 + (\delta_1 + C_1)\ \norm{\PSM - \PSMpop}_F.\label{eq:bound-1-2} 
\end{align}
From Assumption~\ref{ass:consistency-nuisance}, \eqref{eq:bound-1-2} converges to 0 in probability, and therefore,~\eqref{eq:lip-step-1} converges to 0 in probability as well.

Now, we can upper bound~\eqref{eq:lip-step-2} as follows,
\begin{align}
    \sup_{\beta \in \cB} &\left |\left(\sqrt{C - \norm{\hat{\beta}^{\cS}}_2^2}  + \norm{\PSperpM \beta}_2\right)^2 
    - \left(\sqrt{C - \norm{\betastarS}_2^2}  + \norm{\PSperpMpop \beta}_2\right)^2\right| \notag\\
    = &\ \sup_{\beta \in \cB} \left|
    C - \norm{\hat{\beta}^{\cS}}_2^2
    + \norm{\PSperpM \beta}_2^2 
    + 2 \sqrt{C - \norm{\hat{\beta}^{\cS}}_2^2}\ \norm{\PSperpM \beta}_2\right.\notag\\
    &\phantom{\sup_{\beta \in \cB}\ }\left.
    - C + \norm{\betastarS}_2^2 
    - \norm{\PSperpMpop \beta}_2^2
    - 2 \sqrt{C - \norm{\betastarS}_2^2}\  \norm{\PSperpMpop \beta}_2
    \right| \notag\\
    \leq &\ \left|\norm{\hat{\beta}^{\cS}}_2^2 - \norm{\betastarS}_2^2\right|
    + \sup_{\beta \in \cB} \left|\beta^\top (\PSperpM - \PSperpMpop) \beta\right|\notag\\ 
    &+2 \sup_{\beta \in \cB}\left|\sqrt{C - \norm{\hat{\beta}^{\cS}}_2^2}\ \norm{\PSperpM \beta}_2-\sqrt{C - \norm{\betastarS}_2^2}\ \norm{\PSperpMpop \beta}_2\right|\notag\\
    = &\ (I) + (II) + (III). \notag 
\end{align}
By Assumption~\ref{ass:consistency-nuisance}, $(I)$ converges in probability to zero. 
Regarding $(II)$, we have
\begin{align*}
    \sup_{\beta \in \cB} \left|\beta^\top (\PSperpM - \PSperpMpop) \beta\right| 
    \leq 
     \sup_{\beta \in \cB} 
     \norm{\beta}_2^2\
     \norm{\PSperpM - \PSperpMpop}_F
     \stackrel{\prob}{\to}0,
\end{align*}
where the inequality follows from Cauchy--Schwarz and that $\norm{A}_2 \leq \norm{A}_F$, and the convergence in probability follows from Assumption~\ref{ass:consistency-nuisance} along with the compactness of~$\cB$.
It remains to upper bound $(III)$. We have that
\begin{align}
% \begin{split}
    \frac{(III)}{2} 
 \leq 
 &\
 \sup_{\beta \in \cB}
 \left|
 \sqrt{C - \norm{\hat{\beta}^{\cS}}_2^2}\ \norm{\PSperpM \beta}_2
  - \sqrt{C - \norm{\betastarS}_2^2}\ \norm{\PSperpM \beta}_2
 \right| \notag\\
 &+
  \sup_{\beta \in \cB}
 \left|
 \sqrt{C - \norm{\betastarS}_2^2}\ \norm{\PSperpM \beta}_2
  - \sqrt{C - \norm{\betastarS}_2^2}\ \norm{\PSperpMpop \beta}_2
 \right| \notag\\
 \leq &\
 \left(\sup_{\beta \in \cB} \norm{\beta}_2\ \norm{\PSperpM}_F\right)\
 \left|
 \sqrt{C - \norm{\hat{\beta}^{\cS}}_2^2} 
 - 
 \sqrt{C - \norm{\betastarS}_2^2}
 \right|
 \notag\\
 &+\sup_{\beta \in \cB}
 \left|
 \sqrt{\beta^\top \PSperpM  \beta}
 -
 \sqrt{\beta^\top \PSperpMpop \beta}
 \right|
 \left(
 \sqrt{C - \norm{\betastarS}_2^2}\right)\notag\\
 \leq &\
 C_3 
 \left| \norm{\betastarS}_2^2 - \norm{\hat{\beta}^{\cS}}_2^2
 \right|^{1/2}
 + 
 \sqrt{C} 
 \sup_{\beta \in \cB} 
  \left|
 \beta^\top ( \PSperpM- \PSperpMpop) \beta
  \right|^{1/2}
  \label{eq:bound-sqrt-1}\\
  \leq &\
   C_3 
 \left| \norm{\betastarS}_2^2 - \norm{\hat{\beta}^{\cS}}_2^2
 \right|^{1/2}
 + 
 \sqrt{C} 
 \left(
    \sup_{\beta \in \cB} \norm{\beta}_2^2\ \norm{ \PSperpM- \PSperpMpop}_F
 \right)^{1/2} \stackrel{\prob}{\to} 0.\label{eq:bound-sqrt-2}
 % \end{split}
\end{align}
The inequality in~\eqref{eq:bound-sqrt-1} follows from the compactness of $\cB$, the fact that $\PSperpM$ has bounded eigenvalues, and that $|\sqrt{x} - \sqrt{y}| \leq |x - y|^{1/2}$ for all $x, y \geq 0$.
The inequality in~\eqref{eq:bound-sqrt-2} follows from Cauchy--Schwarz and that $\norm{A}_2 \leq \norm{A}_F$.
The convergence in probability follows form Assumption~\ref{ass:consistency-nuisance} and the compactness of~$\cB$.



% \begin{align}\label{eqn:well-sep}
%     \inf\left\{\LL(\beta, \varphi_0) \colon \norm{\beta - \beta_0}_2 > \delta \right\} > \LL(\beta_0, \varphi_0).
% \end{align}
% Moreover, by compactness of $\cB$, continuity of $\beta \mapsto g_{\beta, \varphi_0}(x, y)$ and the fact that $\EE[\sup_{\beta \in \cB} g_{\beta, \varphi_0}(\Xref, \Yref)] < \infty$, it follows from, e.g., \cite{van2000asymptotic} that
% \begin{align}\label{eqn:GC}
%     \sup_{\beta \in \cB} |\LL_n(\beta, \varphi_0) - \LL(\beta, \varphi_0)| \stackrel{\prob}{\to} 0.
% \end{align}
% Finally, by Lemma~\ref{lm:lipschitz-type}, there exists a positive $K < \infty$ such that for all $\varphi_1, \varphi_2$ it holds    \begin{align}\label{eqn:lipschitz}
%     \sup_{\beta \in \cB} |\LL(\beta, \varphi_1) - \LL(\beta, \varphi_2)| \leq K d(\varphi_1, \varphi_2).
% \end{align}
% From~\eqref{eqn:well-sep} and~\eqref{eqn:lipschitz} it follows that for any $\delta > 0$ there exist $\varepsilon_1 > 0$ and $\varepsilon_2 > 0$ such that
% \begin{align*}
%     \prob\left(d(\hat\beta, \beta_0) > \delta\right) 
%     \leq 
%     \prob \left(\left|\LL(\hat{\beta}, \hat{\varphi}) - \LL(\beta_0, \hat{\varphi})\right| > \varepsilon_1 \right) + \prob\left(d(\hat{\varphi}, \varphi_0) > \varepsilon_2\right).
% \end{align*}
% The second term on the right-hand side vanishes by the consistency of $\hat{\varphi}$ from Proposition~\ref{prop:consistency-nuisance}. It remains to show that $\LL(\hat\beta, \hat{\varphi}) - \LL(\beta_0, \hat{\varphi})$ converges in probability to zero.
% To do so, notice that
% \begin{align}
%     \left|\LL(\hat\beta, \hat\varphi) - \LL(\beta_0, \hat\varphi)\right| 
%     \leq &\ \left|\LL(\hat\beta, \varphi_0) - \LL(\beta_0, \varphi_0)\right|
%     \label{eqn:bound-1}\\
%     &\ + \left|\LL(\beta_0, \varphi_0) - \LL(\beta_0, \hat\varphi)\right| + \left|\LL(\hat\beta, \hat\varphi) - \LL(\hat\beta, \varphi_0)\right|.
%     \label{eqn:bound-2}
% \end{align}
% The terms in~\eqref{eqn:bound-2} vanish by~\eqref{eqn:lipschitz} and the consistency of $\hat\varphi$ by Proposition~\ref{prop:consistency-nuisance}.
% It remains to show that the term on the right-hand side of~\eqref{eqn:bound-1} vanishes. We have that
% \begin{align}
%   0 \leq &\ \LL(\hat\beta, \varphi_0) - \LL(\beta_0, \varphi_0) \nonumber \\
%   = &\ 
%   [\LL(\hat\beta, \varphi_0) - \LL(\beta_0, \varphi_0)] - [\LL_n(\hat\beta, \varphi_0) - \LL_n(\beta_0, \varphi_0)] + [\LL_n(\hat\beta, \varphi_0) - \LL_n(\beta_0, \varphi_0)] \label{eqn:bound-3}\\
%   \leq &\ [\LL(\hat\beta, \varphi_0) - \LL_n(\hat\beta, \varphi_0)] - [\LL(\beta_0, \varphi_0) - \LL_n(\beta_0, \varphi_0)] \stackrel{\prob}{\to} 0, \label{eqn:bound-4}
% \end{align}
% where the last inequality holds since the last term in~\eqref{eqn:bound-3} is at most zero, and the convergence in probability in~\eqref{eqn:bound-4} follows from~\eqref{eqn:GC}.
% }

% \Nicola{
% \begin{lemma}\label{lm:lipschitz-type}
% There exists a positive $K < \infty$ such that for all $\varphi_1, \varphi_2$ it holds    \begin{align}\label{eqn:lipschitz}
%     \sup_{\beta \in \cB} |\LL(\beta, \varphi_1) - \LL(\beta, \varphi_2)| \leq K d(\varphi_1, \varphi_2).
% \end{align}
% \Nicola{define metric $d$}
% \end{lemma}
% \begin{proof}
% We have that
% \begin{align}
%     \frac{1}{\gamma} & \sup_{\beta \in \cB}|\LL(\beta, \varphi_1) - \LL(\beta, \varphi_2)|
%     \leq
%     \sup_{\beta \in \cB}\left|
%     \norm{S_1^\top(b_1 - \beta)}_2^2
%     - \norm{P_{S_0, M}(b_2 - \beta)}_2^2
%     \right| \label{eq:lip-step-1}
%     \\
%     & + 
%     \sup_{\beta \in \cB}\left |\left(\sqrt{C - \norm{b_1}_2^2}  + \norm{R_1^\top \beta}_2\right)^2 
%     - \left(\sqrt{C - \norm{b_2}_2^2}  + \norm{R_2^\top \beta}_2\right)^2\right|
%     \label{eq:lip-step-2}
% \end{align}
% We can upper bound~\eqref{eq:lip-step-1} as follows,
% \begin{align}
%     &\sup_{\beta \in \cB}\left|
%     \norm{S_1^\top(b_1 - \beta)}_2^2
%     - \norm{P_{S_0, M}(b_2 - \beta)}_2^2
%     \right| \notag\\
%     = &\ 
%     \sup_{\beta \in \cB}
%     \left|(b_1 - \beta)^\top S_1S_1^\top (b_1 - \beta) 
%     - (b_2 - \beta)^\top P_{S_0, M}^2 (b_2 - \beta) \right| \notag\\
%     = &\
%     \sup_{\beta \in \cB}\left|(b_1 - \beta)^\top S_1S_1^\top (b_1 - b_2)
%     + (b_1 - b_2)^\top S_1S_1^\top (b_2 - \beta) \right. \notag \\
%     & \left. \quad\quad + (b_2 - \beta)^\top (S_1S_1^\top - P_{S_0, M}^2) (b_2 - \beta) \right| \notag\\
%     \leq&\ 
%     2 \sup_{\beta \in \cB} \norm{b_1 - \beta}_2\ \norm{S_1S_1^\top}_F\ \norm{b_1 - b_2}_2
%     + \sup_{\beta \in \cB} \norm{b_2 - \beta}_2^2\ \norm{S_1S_1^\top - P_{S_0, M}^2}_F \label{eq:bound-1-1}\\
%     \leq &\ C_1 \norm{b_1 - b_2}_2 + C_2 \norm{S_1S_1^\top - P_{S_0, M}^2}_F, \label{eq:bound-1-2}
% \end{align}
% where~\eqref{eq:bound-1-1} follows from the Cauchy--Schwarz inequality and that $\norm{A}_2 \leq \norm{A}_F$, and~\eqref{eq:bound-1-2} follows from compactness of $\cB$. Furthermore, we can upper bound~\eqref{eq:lip-step-2} as follows,
% \begin{align}
%     ...
% \end{align}
% \Nicola{
% Use that $|\sqrt{x} - \sqrt{y}| \leq |x - y|^{1/2}$.
% }
% \end{proof}

% % \Nicola{found an issue in the proof above. A quick fix: assume the nuisance parameters are known -- then the proof shortens quite a lot.
% % Alternative: try briefly to fix the issue.
% % }
% % \begin{proposition}
% % % Let $\Ecaltrain = \{0, \dots, K\}$ and denote by $e = 0$ the reference environment. For each environment $e \in \Ecaltrain$, define the samples $\cD_e \coloneqq \{(X_i, Y_i)\}_{i = 1}^{n_e}$ and $\cD \coloneqq \cup_{e \in \Ecaltrain} \cD_e$. Partition the sample as $\cD = \cD_\beta \cup \cD_\varphi$, where $\cD_\beta \subseteq \cD_0$ is a random subset of the reference sample used to estimate $\beta$ and $\cD_\varphi \coloneqq \cD \setminus \cD_\beta$.
% % For fixed constants $C > 0$ and $\gamma > 0$, 
% % define the robust identifiable risk for all parameters of interest $\beta \in \R^d$ and nuisance parameters $\varphi \coloneqq (P_S, P_R, b)$ as
% % \begin{align}\label{eqn:rob-loss-ell}
% %     \LL(\beta, \varphi)  \coloneqq 
% %     \EE_{\prob_0}\left[g_{\beta, \varphi}(X, Y)\right],
% % \end{align}
% % where $\prob_0$ denotes the joint distribution of $(X, Y)$ in the reference environment and the function $g_{\beta, \varphi}$ is defined as
% % \begin{align}
% %     g_{\beta,\varphi}(x, y) 
% %   \coloneqq (y - \beta^\top x)^2 
% %   + \gamma \norm{P_S^\top(b - \beta)}_2^2
% %   + \gamma \left(\sqrt{C - \norm{b}_2^2}  + \norm{P_R^\top \beta}_2\right)^2.
% % \end{align}
% % Define the true nuisance parameter $\varphi_0 \coloneqq (P_{S_0, M}, \PSperpMpop, \betastarS)$ and the worst-case robust predictor as
% % \begin{align}
% %     \betarobpi \coloneqq \argmin_{\beta \in \cB} \LL(\beta, \varphi_0),
% % \end{align}
% % where $\cB \subseteq \R^d$ is a compact set.
% % Denote by $\hat\varphi = (\hat{P}_{S,M}, \hat{P}_{R,M}, \hat{\beta}^{\cS})$ the estimated nuisance parameter obtained from the estimation sample $\cD\coloneqq \{(X_i, Y_i)\}_{i = 1}^{n}$ and define the empirical worst-case robust risk as
% % \begin{align}
% %     \LL_n(\beta, \hat\varphi) \coloneqq \frac{1}{|\cD_0 |} \sum_{i \in \cD_0}  g_{\beta, \hat\varphi}(X_i, Y_i),
% % \end{align}
% % where $\cD_0 \coloneqq \{(X_i, Y_i)\}_{i = 1}^{n_0} \subseteq \cD$ denotes the sample from the reference environment.
% % Moreover, define the estimator of the worst-case robust predictor as
% % \begin{align}
% %     \betarobpihat \coloneqq \argmin_{\beta \in \cB} \LL_n(\beta, \hat\varphi).
% % \end{align}
\section{Details on finite-sample experiments}\label{sec:apx-experiments}

In this section, we provide more details of the data generation for our synthetic finite-sample experiments as well as data processing for the real-world data experiments.
\subsection{Synthetic experiments}\label{sec:apx-synthetic-exps}

For the synthetic experiments, we generate a random SCM which satisfies our assumptions. For $d = 15$, we randomly sample the joint covariance $\Sigmastar$ of $(\eta,\xi)$, fixing its total variance and the eigenvalues. We consider 7 environments including the reference environment, and for each environment except the reference, we randomly generate mean shifts $\mue$ of fixed norm $1$. Since we have $6$ non-zero random Gaussian mean shifts, it holds a.s. that $\dim \cS = 6$. We then randomly generate an "initial guess" for $\betastar \in \R^d$ of fixed norm $C = 10$. Now, with respect to the space $\cS$ of the identifiable directions induced by the mean shifts, we choose the most "adversarial" causal parameter $\betaadv$ which is equal to $\betastar$ on $\cS$, but on $\cSperp$ has the opposite direction of the noise OLS estimator ${\noisecovxxstar}^{-1} \noisecovxystar$. We ensure that $\| \betaadv \|_2 = C$. Note that under the observed shifts, $\betastar$ and $\betaadv$ are observationally equivalent. We complete $\betaadv$ to the set $\thetaadv$ of observationally equivalent model parameters and generate the multi-environment training data according to $\thetaadv$ and the collection of mean shifts. 

For \cref{fig:synthetic-experiments} (left), we define the test shift upper bound as $\Manchor = \gamma \frac{1}{7} \sum_{e} \mu_e \mu_e^\top$. We vary $\gamma$ from $0$ to $10$, and for each $\gamma$, we compute the oracle anchor regression estimator by minimizing the discrete anchor regression loss with the correct $\gamma$. Additionally, we compute the pooled OLS estimator and the worst-case robust predictor $\betarobpi$ as described in \cref{sec:apx-empirical-estimation}. Finally, we generate test data with a Gaussian additive shift $\Atest \sim \cN(0, \Manchor)$. We evaluate the loss of $\betaOLS$, $\betaa$ and $\betarobpi$ on this test environment and include the population lower bound. 

For \cref{fig:synthetic-experiments} (right), we define the test shift upper bound as $\Mnew = \gamma \frac{1}{7} \sum_{e} \mu_e \mu_e^\top + \gammaprime R R^\top$, where $R$ is a 2-dimensional subspace of the space $\cSperp$. We fix the magnitude $\gamma$ of the ''seen'' test shift directions at $\gamma = 40$ and set vary $\gammaprime$ from $0$ to $2$ to showcase the effect of small unseen shifts compared to large identified shifts. We compute the oracle anchor regression estimator by minimizing the discrete anchor regression loss. Additionally, we compute the pooled OLS estimator and the worst-case robust predictor $\betarobpi$ as described in \cref{sec:apx-empirical-estimation}, for which we use the oracle $\gammaprime$, given $\Manchor$ and empirical estimates of the spaces $\cS$, $\cSperp$, $R$. \\
Finally, we generate test data with a Gaussian additive shift $\Atest \sim \cN(0, \Mnew)$. We evaluate the loss of $\betaOLS$, $\betaa$ and $\betarobpi$ on this test environment, plot the resulting test losses for different estimators and include the population lower bound. 

\subsection{Real-world data experiments}\label{sec:apx-real-world}

\begin{figure}[h]
    \centering
    \begin{subfigure}[b]{0.3\textwidth}
        \centering
        \includegraphics[width=\textwidth]{contents/images/training-data.pdf}
        \caption{}
        \label{fig:fig1}
    \end{subfigure}
    \hfill
    \begin{subfigure}[b]{0.3\textwidth}
        \centering
        \includegraphics[width=\textwidth]{contents/images/test-data-strength_01.pdf}
        \caption{}
        \label{fig:fig2}
    \end{subfigure}
    \hfill
    \begin{subfigure}[b]{0.3\textwidth}
        \centering
        \includegraphics[width=\textwidth]{contents/images/test-data-strength_02.pdf}
        \caption{}
        \label{fig:fig3}
    \end{subfigure}
    \caption{The figures illustrate the structure of the (a) training-time shifts and (b-c) test-time shifts for different perturbation strengths on the example of two covariates. Panel (a) shows the training data containing two environments--observational (blue) and shifted (orange) corresponding to the knockout of the gene ENSG00000089009. 
    Panels~(b) and~(c) show the training data in grey and test data from a previously unseen environment (green). 
    Panel~(b) depicts the top $10\%$ test data points closest to the training support (perturbation strength = $0.1$).
    Panel~(c) illustrates the full test data (perturbation strength = 1.0).
    }
    \label{fig:genes}
\end{figure}

We consider the K562 dataset from \cite{replogle2022mapping} and perform the preprocessing as done in \cite{chevalley2022causalbench}.
The resulting dataset consists of $n = 162,751$ single-cell observations over $d = 622$ genes collected from observational and several interventional environments. 
% We now consider the Causalbench single-cell dataset introduced by \citep{chevalley2022causalbench}.
% The dataset consists of single-cell observations of 622 genes.
The interventional environments arise by knocking down a single gene at a time using the CRISPR interference method \citep{qi2013repurposing}. Following \citep{schultheiss2024assessing}, we select only always-active genes in the observational setting, resulting in a smaller dataset of 28 genes. For each gene $j = 1, \ldots, 28$, we set $Y:= X_j$ as the target variable and select the three genes $X_{k_1}, \ldots, X_{k_3}$ most strongly correlated with $Y$ (using Lasso), resulting in a prediction problem over $Y, X_{k_1}, \ldots, X_{k_3}$.
Given this prediction problem, we construct the training and test datasets as follows. Let $\Iobs$ denote the 10,691 observations collected from the observational environment, and let $\Iint_{i}$ denote the observations collected from the interventional environment where the gene $k_i$ was knocked down. We will denote by $\Iint_{i,s}$ the $s \times 100$ percent of datapoints in $\Iint_{i}$ that are closest to the mean of gene $k_i$ in the observational environment $\Iobs$. 
For example, $\Iint_{i,0.1}$ consists of the 10\% of datapoints in $\Iint_{i}$ closest to the observational mean of gene $k_i$.
Thus, 
% $s$ acts as a corresponds to the fraction of test points closest to the observational mean
% These are the $s \times 100\%$ of datapoints with the \emph{weakest} shift compared to the observational mean
% of the gene $k_i$, and thus 
the parameter $s \in [0,1]$ acts as a proxy for the \emph{strength} of the shift. 
Denote by $\Iint_{i, s}^*$ a random sample of $\Iint_{i, s}$ of a certain size.
For each $i \in \{1, 2, 3\}$, we fit the methods on the training data  $\Dtrain_i \coloneqq \Iobs \cup \Iint_{i, 1}^*$, with $|\Iint_{i, 1}^*| = 20$. \cref{fig:genes}(a) illustrates an example of training data  $\Dtrain_i$.
Having fitted the methods on $\Dtrain_i$, we evaluate them on test datasets constructed as follows.
For each 
shift strength $s \in \{0.1, \dots, 0.9\}$ and proportion $\pi \in \{0, .33, .67, 1\}$, define the test dataset $\mathcal{D}_{\pi, s}^{\mathrm{test}}$ consisting of $\pi$ observations from $\cup_{\ell\neq i}\Iint_{\ell, s}$ and $1-\pi$ (out-of-training) observations from $\Iint_{i, s}$.
% $\Dtest_{j, s} \coloneqq \Iint_{j, s}^*$. If $j = i$, this corresponds to a shift seen during training of potentially differing strength. If $j \neq i$, the test data contains a previously unseen distribution shift. 
An example of a test dataset for different shift strengths $s$ and previously unseen directions (i.e., $\pi = 1$) is shown in \cref{fig:genes}(b-c).  
We compare our method Worst-case Rob., defined as the minimizer of the empirical worst-case robust risk \eqref{eqn:rob-loss-ell-sample}, with anchor regression \citep{rothenhausler2021anchor}, invariant causal prediction (ICP) \citep{peters2016causal},  Distributional Robustness via Invariant Gradients (DRIG) \citep{shen2023causalityoriented}, and OLS (corresponding to vanilla ERM).
We use the following parameters for Worst-case Rob.: $\gamma = 50$, $\Cker = 1.0$, and $M = \Id$. For anchor regression and DRIG, we select $\gamma = 50$. For ICP, we set the significance level for the invariance tests to $\alpha = 0.05$.

These numerical experiments are computationally light and can be run in $\approx 5$ minutes on a personal laptop.\footnote{We use a 2020 13-inch MacBook Pro with a 1.4 GHz Quad-Core Intel Core i5 processor, 8 GB of RAM, and Intel Iris Plus Graphics 645 with 1536 MB of graphics memory.}


\section{Proofs}
\label{sec:apx-proofs}
\subsection{Proof of \cref{prop:invariant-set}}\label{sec:apx-proof-invariant-set}
% \julia{update the proof to correspond to new notation}
For every environment $e \in \Ecaltrain$, we observe the first moments $\EE(X_e)$ and $\EE(Y_e)$,
and second moments $\EE(X_eX_e^\top)$, $\EE(Y_e^2)$ and $\EE(X_eY_e)$.
% $\Cov(X_e)$ of the covariates in all training environments, as well as the covariances $\Cov(X_e, Y_e)$. 
Since it holds by assumption that $\mu_0 = 0$ and $\Sigma_0 = 0$, we have that  $\EE(X_0X_0^\top) = \noisecovxxstar$, and so we can identify $\noisecovxxstar$ uniquely. Furthermore, it holds that
\begin{align}
    \EE(X_0Y_0) &= \noisecovxxstar \betastar + \noisecovxystar, \label{eqn:ident-1}\\
    \EE(X_eY_e) &= ( \Sigma_e + \mu_e \mu_e^\top + \noisecovxxstar) \betastar + \noisecovxystar.
    \label{eqn:ident-2}
\end{align}
By taking the difference between \cref{eqn:ident-2} and \cref{eqn:ident-1},
we can identify $(\Sigma_e + \mu_e \mu_e^\top) \betastar$.
Thus, 
% In other words,
the parameter $\betastar$ is identifiable on the subspace $\cS$ defined in \cref{eqn:def-S} and is not identifiable on its orthogonal complement $\cSperp$.
% the unions of the spans of $\Sigma_e$. Since it holds that $\range\ M = \cup_e \range\ \Sigma_e$, the causal parameter is identified on $\range\ M$ and non-identified on $\ker M = \Mperp$. 
Thus, for any vector $\alpha \in \cSperp$ , the vector $\beta = \betastar + \alpha$ is consistent with the data-generating process. It remains to compute the covariance parameters induced by an arbitrary $\tilde\beta \coloneqq \betastar + \alpha$, for $\alpha \in \cSperp$. For every environment $e \in \Ecaltrain$,  the second mixed moment between $X_e$ and $Y_e$ has to satisfy the following equality
\begin{align*}
    \EE(X_eY_e) = (\Sigma_e + \mu_e\mu_e^\top + \noisecovxxstar)\betastar + \noisecovxystar = (\Sigma_e + \mu_e\mu_e^\top + \noisecovxxstar) \tilde{\beta}+ \tilde{\Sigma}_{\eta, \xi},
\end{align*}
from which it follows that $\tilde{\Sigma}_{\eta, \xi}
 \coloneqq \noisecovxystar - \noisecovxxstar \alpha$. By computing $\EE(Y_e^2)$ and inserting $\tilde{\beta} = \betastar + \alpha$ and $\tilde{\Sigma}_{\eta, \xi}$, we similarly obtain 
\begin{align*}
    \tilde{\sigma}_{\xi}^{2} \coloneqq \noisecovyystar - 2 \alpha^\top \noisecovxystar + \alpha^\top \noisecovxxstar \alpha. 
\end{align*}
Thus, we obtain the following set of observationally equivalent model parameters consistent with $\probtrainarg{\thetastar}$:
\begin{align*}
    \Invset = \{ \betastar + \alpha, \noisecovxxstar, \noisecovxystar - \noisecovxxstar \alpha, \noisecovyystar - 2 \alpha^\top \noisecovxystar + \alpha^\top \noisecovxxstar \alpha \colon \alpha \in \cSperp \}. 
\end{align*}
Since the \idset is identifiable from the training distribution, but model parameters $\betastar$, $\noisecovxystar$, $\noisecovyystar$ are not, it is helpful to re-express the \idset through identifiable quantities. For this, we note that the "identifiable linear predictor" $\betastarS = \betastar - \betastarperp$ induces an observationally equivalent model given by 
% $(\betastarperp, \noisecovxxstar, \noisecovxystar + \noisecovxxstar \betastarker, \noisecovyystar + 2 \betastarker^\top \noisecovxystar + \betastarker^\top \noisecovxxstar \betastarker)$. 
\begin{align*}
    \thetastarS := (\betaS, \noisecovxxS, \noisecovxyS, \noisecovyyS) = (\betastarS, \noisecovxxstar, \noisecovxystar + \noisecovxxstar \betastarperp, \noisecovyystar + 2 \langle \noisecovxystar, \betastarperp\rangle + \langle \betastarperp, \noisecovxxstar \betastarperp\rangle).
\end{align*}
From this reparameterization, we infer the final form of the \idset:
\begin{align*}
   \Invset = \{ \betastarS + \alpha, \noisecovxx', \noisecovxyS - \noisecovxx' \alpha, \noisecovyyS - 2 \alpha^\top \noisecovxyS + \alpha^\top \noisecovxx' \alpha \colon \alpha \in \cSperp \}  \ni \thetastar 
\end{align*}
Therefore, \cref{eqn:def-invariant-set} follows.
To find the robust predictor $\betarob$, we write down the robust loss with respect to $\Mtest$ and any $\theta_\alpha$ from the \idset:
\begin{align*}
    \Lossrob(\beta;\theta_\alpha, \Mtest) &= (\betastarS + \alpha - \beta)^\top (\Mtest + \noisecovxxstar) (\betastarS + \alpha - \beta) \\ &+ 2 (\betastarS + \alpha - \beta)^\top (\noisecovxystar - \noisecovxxstar \alpha) + \noisecovyyS - 2 \alpha^\top \noisecovxyS + \alpha^\top \noisecovxxstar \alpha.
\end{align*}
inserting $\alpha \in \cSperp$ and rearranging, \cref{eqn:def-rob-pred-identif} follows.
% Denoting the latter two quantities by $\noisecovxyS$, $\noisecovyyS$ and reparameterizing we obtain the claim. 

\subsection{Proof of \cref{thm:pi-loss-lower-bound}}\label{sec:apx-proof-of-main-prop}

We structure the proof as follows: first, we quantify the non-identifiability of the robust risk by explicitly computing its supremum over the \idset of the model parameters (referred to as the \idRR). Second, we derive a lower bound for the \idRRs by considering two cases depending on how a predictor $\betabar$ interacts with the possible test shifts $\Mtest$. 
% In this proof, we use more general notation, with the test shifts bounded by a PSD matrix $\Mtest \preceq \gamma M + \gammaprime R R^\top$, which $\range M \subset \cS$ and $\range R \subset \cSperp$. The statement of the theorem follows by setting $\gamma = \gammaprime$. However, we believe that the more refined statement is useful, e.g., when one expects strong shifts in training directions and only weak "new" shifts.
\paragraph{Computation of the \idRR.} For any model-generating parameter $\theta = (\beta, \Sigma)$ it holds that the robust risk of the model \cref{eqn:SCM} under test shifts $\Mtest \succeq 0$ is given by 
\begin{align*}
%\label{eqn:apx-def-robust-loss}
    \Lossrob(\betabar;\theta, \Mtest) =  (\beta - \betabar)^\top(\Mtest + \noisecovxxstar)(\beta - \betabar) + 2(\beta - \betabar)^\top \noisecovxy + \noisecovyy. 
\end{align*}
We recall that the \idset of model parameters after observing the multi-environment training data \cref{eqn:SCM} is given by 
\begin{align}\label{eqn:apx-def-invariant-set}
     \Invset = \{ \betastarS + \alpha, \noisecovxxstar, \noisecovxyS - \noisecovxxstar \alpha, \noisecovyyS - 2 \alpha^\top \noisecovxyS + \alpha^\top \noisecovxx \alpha: \alpha \in \cSperp \},
\end{align}
where $\cS$ is the span of identified directions defined in \cref{eqn:def-S}. 
Moreover, we recall that by Assumption~\ref{as:bounded-betastar}, for any causal parameter $\beta$ it should hold that $\| \beta \|_2 = \| \betastarS + \alpha \|_2 \leq C$, which translates into the following constraint for the parameter $\alpha$:
\begin{align*}
    \| \alpha \|_2 \leq \sqrt{C^2 - \| \betastarS \|_2^2} =: \Cker. 
\end{align*}
Inserting \cref{eqn:apx-def-invariant-set} in \cref{eqn:PI-robust-loss}, we obtain
\begin{align*}
    \Lossrobpi(\betabar; \Invset, \Mtest) = \supalpha \Lossrob(\betabar; \theta_{\alpha}, \Mtest),
\end{align*}
where $\theta_\alpha$ is a short notation for $(\betastarS + \alpha, \noisecovxxstar, \noisecovxyS - \noisecovxxstar \alpha, \noisecovyyS - 2 \alpha^\top \noisecovxyS + \alpha^\top \noisecovxxstar \alpha)$. We now compute the supremum explicitly in case $\Mtest$ has the form $\Mtest = \gamma \Mseen + \gammaprime R R^\top$, where $\Mseen$ is a PSD matrix with $\range(M) \subseteq \cS$ and $R$ is a semi-orthogonal matrix with $\range(R) \subseteq \cSperp$. For any $\alpha \in \cSperp$, we write down the robust loss as
\begin{align*}
    \Lossrob(\betabar; \theta_\alpha, \Mtest) &= (\betastarS - \betabar)^\top (\Mtest + \noisecovxxstar) (\betastarS - \betabar) + 2 (\betastarS - \betabar)^\top \noisecovxyS + \noisecovyyS \\
    &+ \alpha^\top \Mtest \alpha + 2 \alpha^\top \Mtest(\betastarS -  \betabar ) \\
    &= \Lossrob(\betabar; \thetastarS, \Mtest) + \alpha^\top \Mtest \alpha + 2 \alpha^\top \Mtest(\betastarS -  \betabar ). 
\end{align*}
The first term is the robust risk of $\betabar$ under test shift $\Mtest$ and the identified model-generating parameter $\thetastarS$, thus it does not depend on $\alpha$. 
%We recall that $\Mtest = \gamma \cP_\cM = \gamma (S S^\top + R R^\top)$, where $\range\ S \subset \cS$ and $\range\ R \subset \cSperp$. 
By the structure of $\Mtest$, we obtain that 
\begin{align*}
    f(\alpha) := \alpha^\top \Mtest \alpha + 2 \alpha^\top \Mtest(\betastarS -  \betabar )  = \gammaprime \alpha^\top R R^\top \alpha - \gammaprime \alpha^\top R R^\top \betabar. 
\end{align*}
If $\gammaprime = 0$, i.e., the test shifts consist only of the identified directions, we have $f(\alpha) = 0$, independently of $\alpha$, and thus 
\begin{align*}
     \Lossrobpi(\betabar; \Invset, \Mtest) = \Lossrob(\betabar; \thetastarS, \Mtest).
\end{align*}
This implies the first statement of the theorem. 
\par
We now consider the case where $R \neq 0$, i.e., $R R^\top$ is a non-degenerate projection.
Our goal is to maximize $f(\alpha)$ subject to constraints $\alpha \in \cSperp$, $\| \alpha \|_2 \leq \Cker$. Let $\Rtilde$ be an orthonormal extension of $R$ such that $\range\ (R | \Rtilde) = \cSperp$. Then, we can parameterize $\alpha \in \cSperp$ as $\alpha = (R | \Rtilde) (\frac{w} {\wtilde})$ and the corresponding Lagrangian reads
\begin{align*}
    \mathcal{L}(\alpha, \lambda) &= \gammaprime \alpha^\top R R^\top \alpha - \gammaprime \alpha^\top R R^\top \betabar + \lambda(\Cker^2 - \| \alpha \|_2^2) \\ &= \gammaprime \| w \|_2^2 -\gammaprime w^\top R^\top \betabar + \lambda(\Cker^2 - \| (w, \wtilde) \|_2^2). 
\end{align*}
Differentiating with respect to $w, \wtilde$ yields
\begin{align*}
    w &= \frac{\gammaprime}{\gammaprime - \lambda} R^\top \betabar; \\
    \wtilde &= 0. 
\end{align*}
After differentiating w.r.t. $\lambda$, we obtain
$\frac{\gammaprime}{\gammaprime - \lambda} = \pm \frac{\Cker}{\| R^\top \betabar \|_2}$. By inserting in the objective function and comparing, we obtain the \textbf{value of the \idRR}: 
\begin{align}\label{eqn:proofs-id-robust-risk}
    \Lossrobpi(\betabar; \Invset, \Mtest) &= \gammaprime \Cker^2 + 2 \gammaprime \| R^\top \betabar \|_2 + \Lossrob(\betabar; \thetastarS, \Mtest) \\
    &= \gammaprime \Cker^2 + 2 \gammaprime \| R^\top \betabar \|_2 + \betabar^\top R R^\top \betabar + \gamma (\betastarS - \beta)^\top \Mseen (\betastarS - \beta) + \Loss_0 (\betabar,\thetastarS).
\end{align}
Putting together the two cases and simplifying, we obtain
\begin{align}\label{eqn:detailed-id-robust-risk}
\begin{split}
    \Lossrobpi(\betabar; \Invset, \Mtest) &= \gammaprime(\Cker + \| R^\top \betabar \|_2)^2 + \Lossrob(\betabar; \thetastarS, \gamma \Mseen) \\  &= \gammaprime  (\Cker + \| R^\top \betabar \|_2)^2 + \gamma (\betastarS - \betabar)^\top \Mseen (\betastarS-\betabar) + \Loss_0 (\betabar,\thetastarS), 
\end{split}
\end{align}
where $\Lossrob(\betabar; \thetastarS, \gamma \Mseen)$ is the robust risk of the estimator $\betabar$ w.r.t. the "identified" test shift $\gamma M$ and the identified model parameter $\thetastarS$, whereas $\Loss_0 (\betabar,\thetastarS)$ is the risk of $\betabar$ on the reference environment $e = 0$. 
\paragraph{Derivation of the lower bound for the \idRR.} Now that we have explicitly computed the \idRR, we devote ourselves to the computation of the lower bound for its best possible value
\begin{align*}
    \inf_{\betabar \in \R^d} \Lossrobpi(\betabar; \Invset, \Mtest). 
\end{align*}
In this part, we will only consider the case $R \neq 0$, since the case $R = 0$ corresponds to the (discrete) anchor regression-like setting, where both the robust risk and its minimizer are uniquely identifiable, and computable from training data. We will distinguish between two cases.
\paragraph{Case 1: $\| R^\top \betabar \|_2 = 0$.}  In this case, $\betabar$ is fully located in the orthogonal complement of $R$, which consists of $\cS$ and $\Rtilde$ (the orthogonal complement or $R$ in $\cSperp$). We will denote (the basis of) this subspace by $\Stot = \cS \oplus \Rtilde$. Thus, $\Stot$ is the "total" stable subspace consisting of identified directions in $\cS$ and non-identified, but unperturbed directions $\Rtilde$. We will parameterize $\betabar$ as $\betabar = \Stot w$. Thus, we are looking to solve the optimization problem 
\begin{align*}
   \betarobpi =  \argmin_{w} \, (\betastarS - \Stot w)^\top (\gamma \Mseen^\top + \noisecovxxstar) (\betastarS -  \Stot w) + 2 (\betastarS -  \Stot w)^\top \noisecovxyS + \noisecovyyS.
\end{align*}
Setting the gradient to zero yields the \emph{asymptotic worst-case robust estimator} 
\begin{equation}\label{eq:apx-pi-robust-formula}
\begin{aligned}
        \betarobpi &= \betastarS + \Stot [ \Stot^\top (\gamma \Mseen^\top + \noisecovxx) \Stot ]^{-1} \Stot^\top \noisecovxyS,
\end{aligned}
\end{equation}
which corresponds to the loss value of 
\begin{align*}
    \Lossrobpi(\betarobpi; \Invset, \Mtest) = \gammaprime \Cker^2 +  \noisecovyyS - 2 {\noisecovxyS}^\top \Stot [ \Stot^\top (\gamma \Mseen^\top + \noisecovxx) \Stot ]^{-1} \Stot^\top \noisecovxyS.
\end{align*}
As we observe, this quantity grows linearly in $\gammaprime$. However, as $\gamma \to \infty$, the quantity \emph{saturates} and is upper-bounded by $\noisecovyyS$.
\paragraph{Case 2:$\| R^\top \betabar \|_2 \neq 0$.} Since for $\| R^\top \betabar \|_2 \neq 0$, the objective function is differentiable, we compute its gradient to be
\begin{align*}
    \nabla  \Lossrobpi(\beta; \Invset, \Mtest) &= 2 \gammaprime R R^\top \beta / \| R R^\top \beta \| + 2 \gammaprime R R^\top \beta + \nabla \Lossrob(\beta; \thetastarS, \gamma \Mseen) \\ 
    &= 2 \gammaprime R R^\top \beta / \| R R^\top \beta \| + 2 \gammaprime R R^\top \beta  + 2(\noisecovxxstar + \gamma \Mseen) (\beta - \betastarS) - 2 \noisecovxyS. 
\end{align*}
This equation is, in general, not solvable w.r.t. $\beta$ in closed form. Instead, we provide the limit of the optimal value of the function when the strength of the unseen shifts is small, i.e. $\gammaprime \to 0$. We know that for $\gammaprime = 0$, the minimizer of the worst-case robust risk is given by the anchor estimator
\begin{align*}
    \betaa = \betastarS + (\noisecovxxstar + \gamma \Mseen)^{-1} \noisecovxyS. 
\end{align*}
Instead, we lower bound the non-differentiable term $2 \gammaprime \Cker \| R^\top \beta \|$ by the scalar product $2 \gammaprime \Cker \scalar{R^\top \beta}{R^\top \betaa}/ \| \betaa \|$ and expect it to be tight for small $\gammaprime$. After inserting this lower bound in \cref{eqn:proofs-id-robust-risk} we obtain the minimizer of the lower bound of form
\begin{align*}
    \beta_{LB} = \betastarS + (\noisecovxxstar + \gamma M + \gammaprime R R^\top)^{-1}(\noisecovxyS - \gammaprime \Cker R R^\top (\noisecovxxstar + \gamma M)^{-1} \noisecovxyS).
\end{align*}
We can now lower bound $\| R R^\top \beta_{LB} \|$ as 
\begin{equation}\label{eqn:small-gammaprime-lower-bound}
    \| R R^\top \beta_{LB} \| \geq \| R R^\top (\noisecovxxstar + \gamma M)^{-1} \noisecovxystar \| - \gammaprime \cdot \text{const}.
\end{equation}
Thus, the $\gammaprime$-rate of the \idRRs of $\beta_{LB}$ is at least $\gammaprime (\Cker + \| R R^\top (\noisecovxxstar + \gamma M)^{-1} \noisecovxystar \|)^2 + \mathcal{O}(\gammaprime^2)$,
from which the claim for small $\gammaprime$ follows. For \cref{sec:comp-with-finite-robustness-methods}, the lower bound directly implies optimality of the worst-case robust risk of the anchor estimator when the strength of the unseen shifts $\gammaprime$ is small. Additionally. if $\gamma = 0$, i.e. only unseen test shifts occur, we conclude that the OLS and anchor estimators have the same rates. 
\paragraph{Lower bound $\gammath$ for $\gammaprime$.}
Finally, we want to derive a lower bound on the shift strength  $\gammaprime$ such that for all $\gammaprime \geq \gammath$ Case 1 of our proof is valid, i.e. it holds that $\betarobpi$ is given by the closed form "abstaining" estimator \eqref{eq:apx-pi-robust-formula}. For this, we find $\gammath$ such that for all $\gammaprime \geq \gammath$ zero is contained in the subdifferential of$\Lossrobpi(\betarobpi;\Invset,\Mtest)$ at $\betarobpi$. Then the KKT conditions are met, and $\betarobpi$ is the unique minimizer of the worst-case robust risk due to strong convexity of the objective. We compute the subdifferential to be
\begin{align*}
    S = \gammaprime \Cker \{ R R^\top \beta: \| \beta \|_2 \leq 1 \} + \nabla \Lossrob(\betarobpi; \thetastarS, \gamma M).
\end{align*}
Since $\betarobpi$ is the minimizer of $ \Lossrob(\beta; \thetastarS, \gamma \Mseen)$ under the constraint $R^\top \beta = 0$, the gradient is zero in $R^\perp$ and it remains to show that 
\begin{align*}
    \| R R^\top \nabla \Lossrob(\betarobpi; \thetastarS, \gamma \Mseen) \| \leq \gammaprime \Cker,
\end{align*}
or 
\begin{align*}
    \gammaprime \geq \| R R^\top \nabla \Lossrob(\betarobpi; \thetastarS, \gamma \Mseen) \| / \Cker. 
\end{align*}
Via an upper bound on the projected gradient, we derive the stricter condition
\begin{align*}
    \gammaprime \geq \frac{\| R R^\top \noisecovxyS\| (1 + \kappa(\noisecovxxstar)) }{\Cker},
\end{align*}
where $\kappa(\noisecovxxstar)$ is the condition number of the covariance matrix. 

\subsection{Proof of \cref{cor:estimators}}\label{sec:apx-proof-of-corollary}
To obtain a new formulation for the \idRR, we start with \eqref{eqn:detailed-id-robust-risk} and expand 
\begin{align}
\begin{split}
    \Lossrobpi(\betabar; \Invset, \Mtest) &= \gammaprime  (\Cker + \| R^\top \betabar \|_2)^2 + \gamma (\betastarS - \betabar)^\top \Manchor (\betastarS-\betabar) + \Loss_0 (\betabar,\thetastarS) \\
    &= \gammaprime  (\Cker + \| R^\top \betabar \|_2)^2 + \gamma (\betastarS - \betabar)^\top \Manchor (\betastarS-\betabar) \\&+ (\betastarS - \betabar)^\top \noisecovxxstar (\betastarS-\betabar) + 2 (\betastarS - \beta)\noisecovxyS + \noisecovyyS \\ &= \gammaprime  (\Cker + \| R^\top \betabar \|_2)^2 + (\gamma - 1)(\betastarS - \betabar)^\top \Manchor (\betastarS-\betabar) + \Loss(\beta,\ptrain) \\
    &= \Lossrob(\beta, \thetastarS, \gamma \Manchor) + \gammaprime  (\Cker + \| R^\top \betabar \|_2)^2,
\end{split}
\end{align}
where we have used that the pooled second moment of $X$ equals to $\noisecovxxstar + \sum_e w_e (\mu_e \mu_e^\top) = \noisecovxxstar + \gamma \Manchor - (\gamma - 1) \Manchor$.
This reformulation shows that the \idRRs is equal to the anchor population loss (cf. \cite{rothenhausler2021anchor}) with an additional non-identifiability penalty term $\gammaprime  (\Cker + \| R^\top \betabar \|_2)^2$. 

We now want to evaluate the rates of the anchor and OLS estimators in terms of the magnitude $\gammaprime$ of unseen shift directions. We observe that only the non-identifiability term depends on $\gammaprime$, whereas the second term only depends on $\gamma$. First, we compute the closed-form anchor regression estimator, which reads 
\begin{equation}
    \betaa = \argmin_{\beta \in \R^d} \Lossrob(\beta, \thetastarS, \gamma \Manchor) = \betastarS + (\noisecovxxstar + \gamma \Manchor)^{-1} \noisecovxyS.
\end{equation}
Since $\betaOLS$ equals to the anchor estimator with $\gamma = 1$, we obtain 
\begin{equation*}
    \betaOLS = \betastarS + (\noisecovxxstar + \Manchor)^{-1} \noisecovxyS.
\end{equation*}
The claim of the corollary now follows by computing $\| R R^\top \betaa \|$ and $\| R R^\top \betaOLS \|$ and observing that the rest of the terms is constant in $\gammaprime$. Additionally, we observe that $\betaa$ is the minimizer of $\Lossrob(\beta, \thetastarS, \gamma \Manchor)$, and it is known (cf. e.g. \cite{rothenhausler2021anchor}) that $\Lossrob(\betaa, \thetastarS, \gamma \Manchor)$ is asymptotically constant in $\gamma$ and upper bounded by $\noisecovyyS$. On the other hand, the term $\Lossrob(\betaOLS, \thetastarS, \gamma \Manchor)$ is linear in $\gamma$. In total, we obtain 
\begin{equation*}
    \begin{aligned}
        \Lossrobpi(\betaa; \Invset, \Mtest) &= (\Cker + \| R R^\top \betaa \| )^2\gammaprime + c_1(\gamma); \\ 
        \Lossrobpi(\betaOLS; \Invset,\Mtest) &= (\Cker + \| R R^\top \betaOLS \| )^2\gammaprime + c_2(\gamma),
    \end{aligned}
\end{equation*}
where $c_1(\gamma) \leq \noisecovyyS$ and $c_2(\gamma) = \Omega(\gamma)$.
Comparing to the lower bound \eqref{eqn:small-gammaprime-lower-bound} for the minimax quantity for the case of $\gammaprime \to 0$, we observe that the anchor estimator is optimal (achieves the minimax rate) in the limit $\gammaprime \to 0$. Additionally, if $\gamma = 0$ (only new shifts occur during test time), anchor and OLS have identical rates in $\gammaprime$ and, in particular, OLS (corresponding to vanilla empirical risk minimization) is minimax-optimal in the limit of small unseen shifts. 
In the proof of \cref{thm:pi-loss-lower-bound} in \cref{sec:apx-proof-of-main-prop}, we show that for $\gammaprime \geq \gammath$, it holds that $R R^\top \betarobpi$ = 0, and thus the worst-case robust risk of the worst-case robust predictor equals 
\begin{equation*}
    \Lossrobpi(\betarobpi; \Invset, \Mtest) = \gammaprime \Cker^2 +  \noisecovyyS - o(\gamma) = \gammaprime \Cker^2 + c_3(\gamma),
\end{equation*}
where $c_3(\gamma) \leq \noisecovyyS$.
In total, we observe that the worst-case robust risk of \emph{all} considered prediction models grows linearly with the unseen shift strength $\gammaprime$, albeit with different rates. The terms $\| R R^\top \betaa \|$ and $\| R R^\top \betaOLS \|$ can be particularly large, for instance, when there is strong confounding aligned with the unseen shift directions which causes the empirical risk minimizer (OLS) to have a strong signal in these directions. The worst-case robust predictor $\betarobpi$, however, abstains in these directions, thus achieving a smaller rate.
% \begin{lemma}\label{lm:upper-bound}
    
% Let $\Mtest \coloneqq V \Lambda V^\top \succeq 0$ with 
% $V = [v_1, \dots, v_k] \in \R^{d \times k}$, $\Lambda = \diag(\lambda_1, \dots, \lambda_k)$, and define
% $\range\(\Mtest) = \cM$.
% Define $Q = (S, R) \in \R^{d \times q}$ such that $\range\(S) = \mathcal{S}$, 
% $\range\(R) \subset \cSperp$,
% and $\range\(Q) \supseteq \cM$. Furthermore define $\gamma := \max\{\lambda_j : j = 1, \dots, k\}$.
% Then, $\Mtest \preceq \gamma QQ^\top$.
% \end{lemma}
% \begin{proof}
%     Since $\range\(Q) \supseteq \cM$, we can write every eigenvector of $\Mtest$ as a linear combination of the columns of $Q$. That is, for every $j = 1, \dots, k$, there exists a vector $\tilde{v}_j \in \R^q$ such that $v_j = Q \tilde{v}_j$. 
%     Moreover, since $v_j \in \range\(Q)$, it follows that $QQ^\top v_j = v_j$, and by the orthonormality of the columns of $Q$, $\tilde{v}_j = Q^\top v_j$.
%     These two facts imply that $\norm{\tilde{v}_j}^2 = \tilde{v}_j^\top \tilde{v}_j = v_j^\top QQ^\top v_j = v_j^\top v_j = \norm{v_j}^2 = 1$.
%     Define $\tilde{V} \coloneqq [\tilde{v}_1, \dots, \tilde{v}_k] \in \R^{q \times k}$ and note that $\Mtest = V\Lambda V^\top = Q \tilde{V} \Lambda \tilde{V}^\top Q^\top$, so we can rewrite the matrix 
%     \begin{align}\label{eqn:psd-mat}
%     \gamma QQ^\top - \Mtest = Q (\gamma I_q - \tilde{V} \Lambda \tilde{V}^\top) Q.
%     \end{align}
    
%     We will now show that the matrix $\gamma QQ^\top - \Mtest \succeq 0$.
%     \begin{enumerate}[i)]
%         \item First, we establish that 
%     $\gamma I_q - \tilde{V} \Lambda \tilde{V}^\top \succeq 0$, by showing that its eigenvalues are non-negative.
%     Notice that the eigenvectors of $\gamma I_q - \tilde{V} \Lambda \tilde{V}^\top$ are $(\tilde{V}, \tilde{U}) \in \R^{q \times q}$ where $\tilde{U} \coloneqq [\tilde{u}_1, \dots, \tilde{u}_{q-k}]\in\R^{q\times (q - k)}$ with $\tilde{U}^\top \tilde{V} = 0$.
%     Let $\tilde{v}\in \R^q$ be an eigenvector of $\gamma I_q - \tilde{V} \Lambda \tilde{V}^\top$. If $\tilde{v}=\tilde{v}_j$ for some $j = 1, \dots, k$, we have that $(\gamma I_q - \tilde{V}\Lambda \tilde{V}^\top) \tilde{v} = (\gamma - \lambda_j)\tilde{v}$, which is associated to the eigenvalue $\gamma - \lambda_j \geq 0$.
%     If $\tilde{v} = \tilde{u}_j$, for some $j = 1, \dots, {q-k}$, we have that $(\gamma I_q - \tilde{V}\Lambda \tilde{V}^\top) \tilde{v} = \gamma \tilde{v}$, which is associated to the eigenvalue $\gamma \geq 0$.

%     \item Second, we establish that $\gamma QQ^\top - \Mtest = Q (\gamma I_q - \tilde{V} \Lambda \tilde{V}^\top) Q^\top \succeq 0$.
%     Fix $x \in \R^d$ and define $y \coloneqq Q^\top x \in \R^q$. Then
%     \begin{align*}
%         x^\top (\gamma QQ^\top - \Mtest )x = &\
%         x^\top Q (\gamma I_q - \tilde{V} \Lambda \tilde{V}^\top) Q^\top x = y^\top (\gamma I_q - \tilde{V} \Lambda \tilde{V}^\top) y \geq 0,
%     \end{align*}
%     where the first equality follows from \cref{eqn:psd-mat} and the last inequality follows from i).
%     \end{enumerate}
    
% \end{proof}





%%%%%%%%%%%%%%%%%%%%%%%%%%%%%%%%%%%%%%%%%%%%%%%%%%%%%%%%%%%%

\end{document}