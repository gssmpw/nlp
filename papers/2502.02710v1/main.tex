\documentclass{article}


\PassOptionsToPackage{numbers,sort&compress}{natbib}
\usepackage[utf8]{inputenc} % allow utf-8 input
\usepackage[T1]{fontenc}    % use 8-bit T1 fonts
\usepackage{hyperref}       % hyperlinks

\usepackage{url}            % simple URL typesetting
\usepackage{booktabs}       % professional-quality tables
\usepackage{amsfonts}       % blackboard math symbols
\usepackage{nicefrac}       % compact symbols for 1/2, etc.
\usepackage{microtype}      % microtypography
\usepackage{xcolor}         % colors

% Recommended, but optional, packages for figures and better typesetting:
\usepackage{amsfonts}[mathscr]
\usepackage{amssymb}
\usepackage{amsmath}
\usepackage{mathtools}
\usepackage{microtype}
\usepackage{graphicx}
\usepackage{natbib}[numbers,sort&compress]
\usepackage{geometry}
\usepackage{booktabs} % for professional tables
\usepackage{graphicx} 
\usepackage{todonotes}
\usepackage{xcolor}
\usepackage{dsfont}
\usepackage{nicefrac}
\usepackage{tcolorbox}
\usepackage{tabularx}
\usepackage{pifont}
% \usepackage{enumitem}
\usepackage{multirow}
\usepackage{algorithm}
\usepackage{algpseudocode}
\usepackage{caption}
\usepackage{wrapfig}
\usepackage[capitalize,noabbrev]{cleveref}
\usepackage{mathrsfs}
\usepackage{algorithm}
\usepackage{algpseudocode}
\usepackage[export]{adjustbox}
\usepackage[list=true]{subcaption}

\usepackage{authblk,textcomp}

\usepackage[cal=euler]{mathalfa}
\usepackage{libertine}



\usepackage{titletoc}

% \newcommand\DoToC{%
%   \startcontents
% \hypersetup{colorlinks=true}
%   \printcontents{}{1}{\subsection*{\textbf{Table of contents}}}
%   \vskip3pt\vskip5pt
% }

%=====================================
% AMS package for theorem section
%-------------------------------------
\usepackage{amsthm}
\usepackage{thmtools, thm-restate}
\declaretheorem[numberwithin=section]{thm}
\declaretheorem[sibling=thm]{theorem}
\declaretheorem[sibling=thm]{lemma}
\declaretheorem[sibling=thm]{corollary}
\declaretheorem[numberwithin=section]{assumption}
\declaretheorem[numbered=no]{discussion}
\declaretheorem[]{challenged assumption}
\declaretheorem[]{definition}
\declaretheorem[]{proposition}
\declaretheorem[style=remark]{example}
%=====================================
% Commands for text
%-------------------------------------
\usepackage{xspace}
\DeclareRobustCommand{\eg}{e.g.,\@\xspace}
\DeclareRobustCommand{\ie}{i.e.,\@\xspace}
\DeclareRobustCommand{\wrt}{w.r.t.\@\xspace}


%=====================================
% Math commands
%-------------------------------------
\newcommand{\thought}[1]{{\color[rgb]{0.2,0.39,0.66}(#1)}}
\newcommand{\todo}[1]{{\color[rgb]{1.0,0.0,0.0}(#1)}}
\newcommand{\hsh}[1]{{\color{green!50!black} Henrik: #1}}
\newcommand{\st}[1]{{\color{red!50!black} Sebastian: #1}}

\newcommand{\ulm}[1]{_{\scaleto{\mathrm{#1}}{3pt}}}
\newcommand\at[2]{\left.#1\right|_{#2}}











\newtheorem{assumption}{Assumption}

\DeclareMathOperator*{\argmax}{arg\,max}
\DeclareMathOperator*{\argmin}{arg\,min}

\newcommand{\swname}[1]{\texttt{#1}}
\newcommand{\ie}{i\/.\/e\/.,\/~}
\newcommand{\eg}{e\/.\/g\/.,\/~}
\newcommand{\cf}{cf\/.\/~}

\newcommand{\fig}{Fig\/.\/~}
\newcommand{\defn}{Def\/.\/~}
\newcommand{\sect}{Sec\/.\/~}
\newcommand{\tabl}{Tab\/.\/~}
\newcommand{\algo}{Algorithm~}
\newcommand{\theo}{Theorem~}

\newcommand{\bnnl}{3 hidden layers}
\newcommand{\bnnn}{50 neurons}
\newcommand{\bnna}{tanh activations}

\newcommand{\capt}[1]{\mdseries{\emph{#1}}}

\newcommand{\videolink}{at \url{https://youtu.be/_d7AqTRjz6g}}
\newcommand{\codelink}{\url{https://github.com/wheelbot/mini-wheelbot}}

\newcommand{\fakepar}[1]{\vspace{0mm}\noindent\textbf{#1.}}

\newcommand{\needref}{\textcolor{red}{[REF]}}

\newcommand{\plotfontsize}{9pt}


\renewcommand{\X}{\mathcal{X}}
\newcommand{\Aspace}{\mathcal{A}}
\newcommand{\Sspace}{\mathcal{S}}
\newcommand{\Hspace}{\mathcal{H}}
\DeclareMathOperator*{\EV}{\mathbb{E}}
\DeclareMathOperator*{\Var}{\mathbb{V}ar}
\newcommand{\Reals}{\mathbb{R}}
\newcommand{\mdp}{\mathcal{M}}
\newcommand{\one}{\mathds{1}}
\DeclareMathOperator*{\argmax}{arg\,max}
\DeclareMathOperator*{\argmin}{arg\,min}
\DeclareMathOperator{\var}{VaR_{\alpha}}
\DeclareMathOperator{\cvar}{CVaR_{\alpha}}
\renewcommand{\F}{\mathcal{F}}
\newcommand{\J}{\mathcal{J}}
\newcommand{\cmdp}{\mathcal{CM}}
\renewcommand{\cmark}{\ding{51}}%
\renewcommand{\xmark}{\ding{55}}%
\newcommand{\emdp}{\mdp_\ell}
\newcommand{\eAspace}{\mathcal{A}_\ell}
\newcommand{\eSspace}{\mathcal{S}_\ell}
\newcommand{\eP}{P_\ell}
\newcommand{\emu}{\mu_\ell}
\newcommand{\er}{r_\ell}
\newcommand{\es}{s_\ell}
\newcommand{\ea}{a_\ell}
\newcommand{\pomdp}{\mathcal{P}\mdp}
\renewcommand{\R}{\mathcal{R}}
\DeclareMathOperator{\w}{\mathbf{w}}
%-------------------------------------

\usepackage{tikz}
\usetikzlibrary{shapes, arrows.meta, positioning}


\usepackage{enumerate}
\usepackage{wrapfig}

% Page Layout
\geometry{
a4paper,
left=20mm,
right=20mm,
top=20mm,
}

\hypersetup{
    colorlinks,
   linkcolor={pierCite},
    citecolor={pierCite},
    urlcolor={pierCite}
}



\title{Achievable distributional robustness when the robust risk is only partially identified}



\author[1]{Julia Kostin}
\author[2]{Nicola Gnecco\thanks{This work was conducted while NG was at the Gatsby Computational Neuroscience Unit, University College London.}}
\author[1]{Fanny Yang}
% new official EPFL format
\affil[1]{\small Department of Computer Science, ETH Zurich}
% \affil[2]
\affil[2]{\small Department of Mathematics, Imperial College London}



\begin{document}


\maketitle


\begin{abstract}

  In safety-critical applications, machine learning models should generalize well under worst-case distribution shifts, that is, have a small robust risk. Invariance-based algorithms 
  can provably take advantage of structural assumptions on the shifts when the training distributions are heterogeneous enough to identify the robust risk. However, in practice, such identifiability conditions are rarely
  satisfied -- a scenario so far underexplored in the theoretical literature. In this paper, we aim to fill the gap and propose to study the more general setting when the robust risk is only \emph{partially identifiable}. In particular, 
  we introduce the \emph{worst-case robust risk} as a new measure of robustness that is always well-defined regardless of identifiability.
  Its minimum corresponds to
  an algorithm-independent (population) minimax quantity that measures the \emph{best achievable robustness} under partial identifiability.
  While these concepts can be defined more broadly, 
  in this paper 
  we introduce and derive them explicitly for a linear model
  for concreteness of the presentation. 
  First, we show that existing robustness methods are provably suboptimal in the partially identifiable case.
  We then evaluate these methods and the minimizer of the (empirical) worst-case robust risk on real-world gene expression data and find a similar trend:
  the test error of existing robustness methods grows increasingly suboptimal as the 
  fraction of data from unseen environments increases, whereas accounting for partial identifiability allows for better generalization.
  \end{abstract}

  \section{Introduction}\label{sec:intro}
  \section{Introduction}
Dark pools are private trading venues designed for institutional investors to execute large trades anonymously, concealing details such as price and identities until after the transaction. Orders, which include the trade direction (buy or sell), volume, and price, are matched by the operator when they have opposite directions and compatible bid and ask prices. While they help prevent price swings from large orders, there are trust concerns. Operators may engage in front running, using insider knowledge of upcoming trades to execute their own trades first and profit from the price movement. Additionally, dark pool operators, often large financial institutions, may prioritize their own trades over clients', creating a conflict of interest. The lack of transparency also makes it difficult to detect manipulative practices, raising concerns about fairness. Several dark pool operators have been fined for misconduct, including misleading investors and failing to maintain proper trading practices. Notable examples include Barclays (\$70 million) and Credit Suisse (\$84.3 million) in 2016 for misrepresenting their dark pool operations, Deutsche Bank (\$3.7 million in 2017) for similar issues, ITG (\$20.3 million in 2015) for conflicts of interest, and Citigroup (\$12 million in 2018) for misleading clients about trade execution~\cite{fines}.

While dark pools aim to prevent the leakage of large orders, operators still gain privileged access to clients' hidden orders, creating potential for conflicts of interest or misuse of sensitive data. Recent research has focused on cryptographically protecting order information. These systems allow users to submit orders in encrypted form, enabling dark pool operators to compare orders without revealing their contents, only unveiling them when matches occur. This approach ensures greater security and mitigates risks associated with operator access to sensitive trade information. 

Asharov et al.~\cite{AsharovBPV20} introduced a secure dark pool model using Threshold Fully Homomorphic Encryption (FHE), which combines two cryptographic techniques: FHE, allowing computations on encrypted data without decryption, and Threshold Cryptography, where a secret (such as a decryption key) is split among multiple parties. In this approach, data is encrypted with a public key, and computations are performed on the encrypted data by an untrusted party. Decryption requires a threshold number of participants to combine their key shares. In Asharov et al.~\cite{AsharovBPV20} model, orders are encrypted under a public key, and the operator matches them directly on the encrypted data. Once a match is found, the orders are decrypted by the clients using their decryption shares, ensuring that no single entity, including the operator, can access the sensitive order details. Throughout the process, the orders remain encrypted, preventing any single party from accessing both the data and the decryption key. However, FHE is known for being computationally heavy, and adding a threshold mechanism increases the complexity. Optimizing this for real-time or large-scale applications is still an active research area. Moreover, the need for multiple parties to collaborate on decryption and sometimes computation can introduce significant communication overhead. As a result, the process takes nearly a full second to complete a single match, significantly impacting performance.

A recent development, Prime Match~\cite{polychroniadou2023prime}, introduces a solution to protect the confidentiality of periodic auctions run by a market operator. Prime Match allows users to submit orders in an encrypted form, with the operator comparing these orders through encryption and only revealing them if a match occurs. The auctions capture the trade direction (buy or sell) and the desired volume, but exclude price. Prime Match represents the first financial tool based on secure multiparty computation (MPC). In the high-stakes, highly competitive financial sector, MPC is gaining significant traction as a crucial enabler of privacy, with J.P. Morgan successfully deploying Prime Match in production.


However, in continuous double auctions without excluding the price, such as those in dark pools, where the computational complexity surpasses that of simpler periodic auctions like Prime Match, secure computation techniques fall short. They cannot support high-frequency trading within acceptable timeframes, limiting the feasibility of these methods for enhancing privacy in dark pools and the broader financial sector, where speed and efficiency are critical.

In this work, we pose the question: Can we achieve a privacy-preserving solution to dark pools with efficiency comparable to non-private dark pool protocols? We propose a system that combines differential privacy with encryption, providing a more efficient alternative to secure MPC and FHE. Differential Privacy achieves privacy by adding noise to the results of queries or computations on datasets. The level of noise is determined by a privacy parameter, which quantifies the trade-off between privacy and accuracy. By leveraging differential privacy, our dark pool approach ensures that individual orders are obfuscated while still allowing for effective matching. This method conceals the most critical aspect of dark pools—the volumes of orders—thus preserving the primary objective of dark pools, which is to hide large trades. 

In summary, integrating differential privacy with encryption offers a streamlined, efficient solution that balances privacy and computational feasibility, making it an attractive practical alternative to impractical methods like MPC and FHE. 

\subsection{Our Contributions:}

\paragraph{Problem Statement:}

The dark pool consists of $n$ agents (clients), and an operator who receives the orders from the clients. Each order takes one of two forms: (1) Buy Order: $({\sf buy}, \price, \amount)$, where $\amount$ is the quantity/volume, and $\price$ is the highest price the buyer is willing to pay per share.
(2) Sell Order: $({\sf sell}, \price, \amount)$, where $\price$ is the lowest price the seller is willing to accept per share. A buy order can be matched to a sell order if the buying
price is at least the selling price. Our objective is to design matching protocols that maximize the total number of matches while preserving user data privacy. Specifically, we aim to conceal the quantity $\amount$ of the orders during the process. Our key contributions are as follows:
\begin{enumerate}
    \item {\bf Practical Dark Pool Solution:} We propose an efficient solution for continuous double auctions (such as dark pools) that conceals both bid and ask quantities, effectively reducing reliance on trusted auctioneers while ensuring privacy guarantees.
    \item {\bf Novel Privacy Concept:} We introduce indifferential privacy, a new extension of differential privacy tailored to this context, which can be of independent interest with potential applicability beyond auctions and dark pools.
    \item {\bf Maximum Matching:} Our new notion of indifferential privacy allows us to achieve the optimal maximum matching which is impossible to achieve under conventional differential privcy~\cite{DBLP:journals/siamcomp/HsuHRRW16}. 
     \item {\bf Efficiency and Implementation: } Our system significantly outperforms previous privacy-preserving auction models, which often struggled with practicality and hindered their adoption in production. We show that our solution rivals non-private auction protocols in terms of performance, making it viable for real-world deployment. 
\end{enumerate}

\paragraph{High-Level Idea of our techniques:} To preserve privacy while matching buy and sell orders, each order is viewed as containing 
$\amount$ units, with users submitting 
$\amount+ noise$  orders to the server based on indifferential privacy. The orders are represented as nodes in a bipartite graph, with sell orders from sellers $S_i$ on the left and buy orders from buyers $B_i$ on the right. See Figure~\ref{fig:auction} for an example. The server ranks the nodes based on price $p$, where higher prices for buyers and lower prices for sellers are considered more favorable, referred to as "extreme" prices. The algorithm then constructs a bipartite graph, with edges connecting buy and sell nodes if the buying price meets or exceeds the selling price. 


\begin{figure}[hbt!]
    \centering
    \includegraphics[width=0.6\columnwidth]{./auction.png}  % Replace with the path to your figure file
    \caption{\textbf{Maximum Matching Example.} \textnormal{The figure illustrates a bipartite graph and its maximum matching where nodes are first sorted by price, followed by the user's genuine orders and then their fake orders. This arrangement ensures optimal matching, maximizing the number of successful pairings according to algorithm~\ref{alg:matching}.}}
    \label{fig:auction}
\end{figure}





In particular, the algorithm constructs and maintains a bipartite graph, where edges exist between buy and sell nodes when their prices are compatible, i.e., the buying price is at least equal to the selling price. As matches are made, isolated nodes (the gray nodes in Figure~\ref{fig:auction})—those without neighbors, such as when their price cannot be met—are promptly removed from the graph. 
In the maximum matching problem, the goal is to identify a set of edges where no two edges share a node, thereby maximizing the total number of matches. As mentioned above, nodes with the most popular price on one side of the graph are naturally connected to nodes with the least popular price on the opposite side. These pairs, referred to as polar opposite nodes, form the basis of an iterative matching strategy that guarantees an optimal matching solution.

Our approach maintains this optimal matching even when users submit a number of noise-injected fake (the red nodes in Figure~\ref{fig:auction}) orders to obscure the true amounts of their order. To achieve this, we introduce  in-differential privacy (detailed in Section~\ref{sec:IDP}) and orders are not only sorted by price but also arranged such that each user’s true orders are followed by their corresponding fake orders. 

Unlike traditional differential privacy, our new in-differential privacy concept allows for selective disclosure of information post-match, aligning with realistic scenarios. For instance, in the context of a dark pool trading environment, it becomes acceptable to reveal the actual quantity of a trade once it has been fully matched, as it no longer poses a risk to privacy.
To establish this new privacy framework, we integrate graph refinement techniques, ensuring that the protocol not only protects user data but also facilitates an efficient and optimal matching process, even under the presence of noise.

\paragraph{Implementation:} We present an end-to-end implementation of our system. While previous FHE-based solutions processed fewer than one order per second, our system dramatically outperforms them, handling between 600 and 850 orders per second (according to Table~\ref{tab:comparison}), depending on the input volume. Furthermore, we provide an analysis of the overhead introduced by our privacy-preserving mechanism compared to the non-private version. While privacy inevitably incurs some cost and does not come for free, our system's overhead remains minimal and practical, making it highly suitable for high-frequency trading environments.

\subsection{Related Work}

The works of~\cite{CartlidgeSA19,CartlidgeSA21,MazloomDPB23} leverage secure multiparty computation (MPC) with multiple operators instead of a single one. While this method is computationally faster than fully homomorphic encryption (FHE), it comes with significant communication overhead due to the necessary interactions among multiple operators. Moreover, the practicality of this approach is limited, as most current dark pool systems operate with a single operator. Importantly, MPC can only guarantee the privacy of orders if the dark pool operators do not collude, raising concerns in scenarios where collusion is feasible. Given these challenges, it is crucial to focus on solutions that emphasize single-operator architectures, which can streamline communication and enhance privacy.

The work of Massacci et al.~\cite{massacci2018futuresmex} proposes a distributed market exchange for futures assets that features a multi-step functionality, including a dark pool component. Their experiments show that the system can support up to ten traders. Notably, their model does not conceal orders; instead, it discloses an aggregated list of all pending buy and sell orders, which sets it apart from our solution. Moreover, there are existing works proposing private dark pool constructions utilizing blockchain technology~\cite{bag2019seal,galal2021publicly,ngo2021practical}, our focus diverges from this area. Furthermore, all these solutions experience slowdowns due to the reliance on computationally intensive public key cryptographic mechanisms.


The work of Hsu et al.~\cite{DBLP:journals/siamcomp/HsuHRRW16} also considered a private matching problem under the notion of \emph{joint differential privacy}, where the view of the adversary consists of the output received by all users except the user whose privacy is concerned.
Since the notion is still based on the conventional approach of using divergence on the adversarial views for neighboring inputs, their privacy notion can only lead to an almost optimal matching. In contrast, our new notion can achieve the exact optimal matching.


The authors in~\cite{PolychroniadouC24} employ differential privacy in a distinct and simplified setting of volume matching~\cite{BalchDP20,GamaCPSA22,polychroniadou2023prime}, where prices are predetermined and fixed, to obfuscate aggregated client volumes and conceal the trading activity of concentrated clients. In their auction mechanism, the obfuscated aggregate volumes are published daily, enabling buyers to make informed matching decisions based on this publicly available inventory. 

  \section{Setting}
  \label{sec:setting}We study (stochastic) gradient descent on the empirical risk
\begin{equation*}
\cL(w) = \frac{1}{n}\sum_{i=1}^n l(p_i(w))\, ,
\end{equation*}
where the loss function $l$ and the functions  $(p_i)_{i=1}^n$  are specified in the following assumptions. Note that the empirical risk for binary classification from Equation~\eqref{def:emp_risk_intro} is a special case of the above objective.

\begin{assumption}\label{hyp:loss_exp_log}\phantom{=}
  \begin{enumerate}[label=\roman*)]
    \item The loss is either the exponential loss, $l(q) = e^{-q}$, or the logistic loss, $l(q) = \log(1{+}e^{-q})$.
    \item There exists an integer $L \in \mathbb{N}^*$  such that, for all $1 \leq i \leq n$, the function $p_i$ is $L$-homogeneous\footnote{We recall that a mapping $f : \mathbb{R}^d \rightarrow \mathbb{R}$ is positively $L$-homogeneous if $f(\lambda w) = \lambda^L f(w)$ for all $w \in \mathbb{R}^d$ and $\lambda >0$.}, locally Lipschitz continuous and semialgebraic.
  \end{enumerate}
\end{assumption}
If the $p_i$'s were differentiable with respect to $w$, the chain rule would guarantee that
\begin{align*}
\nabla \mathcal{L}(w) = \frac{1}{n}\sum_{i=1}^n  l'(p_i(w)) \nabla p_i(w)\enspace.
\end{align*}
However, we only assume that the $p_i$'s are semialgebraic. While we could consider Clarke subgradients, the Clarke subgradient of operations on functions (e.g., addition, composition, and minimum) is only contained within the composition of the respective Clarke subgradients. This, as noted in Section~\ref{sec:cons_field}, implies that the output of backpropagation is usually not an element of a Clarke subgradient but a selection of some conservative set-valued field.
Consequently, for $1\leq i \leq n$, we consider $D_i : \bbR^d \rightrightarrows\bbR^d$, a conservative set-valued field of $p_i$, and a function $\sa_i : \bbR^d \rightarrow \bbR^d$ such that for all $w \in \bbR^d$, $\sa_i(w) \in D_i(w)$. Given a step-size $\gamma >0$, gradient descent (GD)\footnote{More precisely, this refers to conservative gradient descent. We use the term GD for simplicity, as conservative gradients behave similarly to standard gradients.} is then expressed as
\begin{equation*}\label{eq:gd_new}\tag{GD}
  w_{k+1} = w_k - \frac{\gamma}{n} \sum_{i=1}^n l'(p_i(w_k))\sa_i(w_k)\,.
\end{equation*}
For its stochastic counterpart, stochastic gradient descent (SGD), we fix a batch-size $1\leq n_b \leq n$. At each iteration $k \in \bbN$, we randomly and uniformly draw a batch $B_k \subset \{1, \ldots, n \}$ of size $n_b$. The update rule is then given by 
\begin{equation*}\label{eq:sgd_new}\tag{SGD}
  w_{k+1} = w_k -  \frac{\gamma}{n_b}\sum_{i\in B_k} l'(p_i(w_k)) \sa_i(w_k)\, .
\end{equation*}
The considered conservative set-valued fields will satisfy an Euler lemma-type assumption.
%\nic{Smoother transition}
\begin{assumption}\phantom{=}\label{hyp:conserv}
  For every $i \leq n$, $\sa_i$ is measurable and $D_i$ is semialgebraic. Moreover, for every $w \in \bbR^d$ and $\lambda \geq 0$, $\sa_i(w)  \in D_i(w)$,
  \begin{equation*}
    D_i(\lambda w) = \lambda^{L-1} D_i(w)\, , \textrm{ and } \quad   L p_i(w) = \scalarp{\sa_i(w)}{w}\, .
  \end{equation*}
\end{assumption}
%\nic{Smoother transition}
Having in mind the binary classification setting, in which $p_i(w) = y_i \Phi(x_i, w)$, we define the margin
\begin{equation}\label{def:marg}
  \sm: \bbR^d \rightarrow \bbR, \quad \sm(w) = \min_{1\leq i \leq n} p_i(w)\, .
\end{equation}
It quantifies the quality of a prediction rule $\Phi(\cdot, w)$. In particular,  the training data is perfectly separated when $\sm(w) >0$. A binary prediction for $x$ is given by the sign of $\Phi(x, w)$, and under the homogeneity assumption, it depends only on the normalized direction $w / \norm{w}$. Consequently, we will focus on the sequence of directions $u_k := w_k / \norm{w_k}$. Our final assumption ensures that the normalized directions $(u_k)$ have stabilized in a region where the training data is correctly classified.

\begin{assumption}\label{hyp:marg_lowb}
  Almost surely, $\liminf \sm(u_k) >0$.
\end{assumption}
Before presenting our main result, we comment on our assumptions.

\paragraph{On Assumption~\ref{hyp:loss_exp_log}.} As discussed in the introduction, the primary example we consider is when $p_i(w) = y_i \Phi(x_i;w)$ is the signed prediction of a feedforward neural network without biases and with piecewise linear activation functions on a labeled dataset $((x_i,y_i))_{i \leq n}$. In this case,
\begin{equation}\label{eq:NN}
 p_i(w) = y_i \Phi(w;x_i) = y_i V_L(W_L) \sigma(V_{L-1}(W_{L-1}) \sigma(V_{L-1}(W_{L-2}) \ldots \sigma(V_{1}(W_1 x_i))))\, ,
\end{equation}
where $w = [W_1, \ldots, W_L]$, $W_i$ represents the weights of the $i$-th layer, $V_i$ is a linear function in the space of matrices (with $V_i$ being the identity for fully-connected layers) and $\sigma$ is a coordinate-wise activation function such as $z \mapsto \max(0,z)$ ($\ReLU$), $z \mapsto \max(az, z)$ for a small parameter $a>0$ (LeakyReLu) or $z \mapsto z$. Note that the mapping $w \mapsto p_i(w)$ is semialgebraic and $L$-homogeneous for any of these activation functions. Regarding the loss functions, the logistic and exponential losses are among the most commonly studied and widely used. In Appendix~\ref{app:gen_sett}, we extend our results to a broader class of losses, including $l(q) = e^{-q^a}$ and $l(q) = \ln (1 + e^{-q^a})$ for any $a \geq 1$.

\paragraph{On Assumption~\ref{hyp:conserv}.} Assumption~\ref{hyp:conserv} holds automatically  if $D_i$ is the Clarke subgradient of $p_i$. Indeed, at any vector $w \in \bbR^d$, where $p_i$ is differentiable it holds that $p_i(\lambda w) = \lambda^{L} p_i(w)$. Differentiating relatively to $w$ and $\lambda$ (noting that $p_i$ remains differentiable at $\lambda w$ due to homogeneity), we obtain $\lambda \nabla p_i(\lambda w) = \lambda^{L} \nabla p_i(w)$ and $\scalarp{\nabla p_i(\lambda w)}{w} = L \lambda^{L-1} p_i(w)$. The expression for any element of the Clarke subgradient then follows from~\eqref{eq:def_clarke}. 

However, for an arbitrary conservative set-valued field, Assumption~\ref{hyp:conserv} does not necessarily hold. For instance, $D(x) = \mathds{1}(x \in \mathbb{N})$ is a conservative set-valued field for $p \equiv 0$, which does not satisfy Assumption~\ref{hyp:conserv}. Nevertheless, in practice, conservative set-valued fields naturally arise from a formal application of the chain rule. For a non-smooth but homogeneous activation function $\sigma$, one selects an element $e \in \partial \sigma (0)$, and computes $\sa_i(w)$ via backpropagation. Whenever a gradient candidate of $\sigma$ at zero is required (i.e., in~\eqref{eq:NN}, for some $j$, $V_j(W_j)$ contains a zero entry), it is replaced by $e$. 
Since $V_j(W_j)$ and $V_j(\lambda W_j)$ have the same zero elements, it follows that for every such $w$, $
\sa_i(\lambda w) = \lambda^L \sa_i(w)$. The conservative set-valued field $D_i$ is then obtained by associating to each $w$ the set of all possible outcomes of the chain rule, with $e$ ranging over all elements of $\partial \sigma(0)$. Thus, for such fields, Assumption~\ref{hyp:conserv} holds.


\paragraph{On Assumption~\ref{hyp:marg_lowb}.} Training typically continues even after the training error reaches zero.
Assumption~\ref{hyp:marg_lowb} characterizes this late-training phase, where our result applies. 
As noted earlier, since $\sm$ is $L$-homogeneous, the classification rule is determined by the direction of the  iterates $u_k=w_k/\norm{w_k}$. Assumption~\ref{hyp:marg_lowb} then states that, beyond some iteration, the normalized margin remains positive. 
This assumption is natural in the context of studying the implicit bias of SGD: we \emph{assume} that we reached the phase in which the dataset is correctly classified and \emph{then} characterize the limit points. A similar perspective was taken in  \cite{nacson2019lexicographic}, where the implicit bias of GF was analyzed under the assumption that the sequence of directions and the loss converge. However, unlike their approach, ours does not require assuming such convergence a priori.

Earlier works such as \cite{ji2020directional,Lyu_Li_maxmargin}, which analyze subgradient flow or smooth GD, establish convergence by assuming the existence of a single iterate $w_{k_0}$ satisfying $\sm(w_{k_0}) > \varepsilon$ and then proving that $\lim \sm(u_{k}) > 0$. Their approach relies on constructing a smooth approximation of the margin, which increases during training, ensuring that $\sm(u_k) > 0$ for all iterates with $k \geq k_0$. This is feasible in their setting, as they study either subgradient flow or GD with smooth $p_i$’s, allowing them to leverage the descent lemma.

In contrast, our analysis considers a nonsmooth and stochastic setting, in which, even if an iterate $w_{k_0}$ satisfying $\sm(w_{k_0}) > \varepsilon$ exists, there is no a priori assurance that subsequent iterates remain in the region where Assumption~\ref{hyp:marg_lowb} holds. From this perspective, Assumption~\ref{hyp:marg_lowb} can be viewed as a stability assumption, ensuring that iterates continue to classify the dataset correctly. Establishing stability for stochastic and nonsmooth algorithms is notoriously hard, and only partial results in restrictive settings exist \cite{borkar2000ode,ramaswamy2017generalization,josz2024global}.

%Finally, note that Assumption~\ref{hyp:marg_lowb} only needs to hold almost surely. Specifically, with probability 1, there exist $k_0$ and $\varepsilon$ such that for all $k \geq k_0$, $\sm(u_k) \geq \varepsilon > 0$. In the case of~\eqref{eq:sgd_new}, $k_0$ and $\delta$ are random variables and may take different values across different realizations. 

%\paragraph{On constant stepsizes.}
%We allow the step size to be a constant of arbitrary magnitude, subject to the stability Assumption~\ref{hyp:marg_lowb}. This may seem surprising in a nonsmooth and stochastic setting, where a vanishing step size is typically required to ensure convergence (see, e.g., \cite{majewski2018analysis, dav-dru-kak-lee-19, bolte2023subgradient, le2024nonsmooth}).
  \section{Theoretical results for 
  the linear setting}
  \label{sec:main-results}
      
    
\begin{table*}[t!]
    \centering
    \small
    
    \scalebox{0.90}{
    \setlength{\tabcolsep}{1.0pt}
    \begin{tabular}{l c c c r | c c c c c c |c  c c }
    \toprule
    \multirow{1}{*}{Method} & \multirow{1}{*}{Recipe} & \multirow{1}{*}{Complexity} & \multirow{1}{*}{\# P.} & \multirow{1}{*}{\# T.P.}& MME & MMB &POPE & \multicolumn{1}{c} {SEED} & MMMU & MM-Vet& TQA & SQA-I  & \multicolumn{1}{c}{GQA} \\
    \midrule
    \rowcolor{gray!14}
    \multicolumn{14}{l}{\textbf{\textit{Encoder-based VLMs}}} \\ 
    OpenFlamingo~\cite{openflamingo} & \underline{PT, SFT}& Quadratic & 9B& 96.6\%  & - & 4.6 & - & - & - & - & 33.6 & - & - \\
    MiniGPT-4~\cite{minigpt} & \underline{PT, SFT}& Quadratic & 13B& 94.8\%  & 581.7 & 23.0 & - & - & -& 22.1 & - & - & 32.2  \\
    Qwen-VL~\cite{qwenvl} & \underline{PT, SFT}& Quadratic & 7B& 100.0\%  & - & 38.2 & - & 56.3 & - & - & 63.8 & 67.1 & 59.3\\ 
    LLaVA-Phi~\cite{llavaphi}  & \underline{PT, SFT}& Quadratic & 3B& 90.0\%  & 1335.1 & 59.8 & 85.0 & - & - & 28.9& 48.6 & 68.4 & - \\
    MobileVLM-3B~\cite{mobilevlm} & \underline{PT, SFT}& Quadratic & 3B& 90.0\%  & 1288.9 & 59.6 & 84.9 & - & - & - & 47.5 & 61.0 & 59.0  \\
    VisualRWKV~\cite{visualrwkv} & \underline{PT, SFT}&  \textbf{Linear} & 3B& 90.0\%  & 1369.2 & 59.5 & 83.1 & - & - & - & 48.7 & 65.3 & 59.6 \\
    VL-Mamba~\cite{vlmamba} & \underline{PT, SFT}&  \textbf{Linear} & 3B& 90.0\%  & 1369.6 & 57.0 & 84.4 & - & -& 32.6 & 48.9 & 65.4 & 56.2 \\
    Cobra~\cite{cobra} & \underline{PT, SFT}&  \textbf{Linear} & 3.5B& 82.6\%  & - & - & \textbf{88.4} & - & - & - & 58.2 & - & \textbf{62.3}\\
    \midrule
    \rowcolor{gray!14}
    \multicolumn{14}{l}{\textbf{\textit{Decoder-only VLMs}}} \\
    Fuyu-8B (HD)~\cite{fuyu} & \underline{PT, SFT}& Quadratic & 8B& 100.0\%  & 728.6 & 10.7 & 74.1 & - & - & 21.4 & - & - & -\\
    SOLO~\cite{solo} & \underline{PT, SFT}& Quadratic &  7B& 100.0\%   & 1001.3 & - & - & 64.4 & - & - & - & 73.3 & -   \\    
    Chameleon-7B~\cite{chameleon}  & \underline{PT, SFT}& Quadratic &  7B& 100.0\%   & 170 & 31.1 & - & 30.6 & 25.4 & 8.3 & 4.8 & 47.2 & -\\  
    EVE-7B~\cite{eve}  & \underline{PT, SFT}& Quadratic &  7B& 100.0\%  & 1217.3 & 49.5 & 83.6 & 61.3 & \underline{32.3} & 25.6& 51.9 & 63.0 & 60.8 \\
    Emu3~\cite{emu3} & \underline{PT, SFT}& Quadratic & 8B& 100.0\%  & - & 58.5 & 85.2 & \underline{68.2} & 31.6 & \underline{37.2} & \underline{64.7} & \underline{89.2} & 60.3\\
    HoVLE~\cite{hovle} & DT, PT, SFT & Quadratic & \textbf{2.6B}& 100.0\%  & \textbf{1433.5} & \textbf{71.9} & \underline{87.6} & \textbf{70.7} & \textbf{33.7} & \textbf{44.3} & \textbf{66.0} & \textbf{94.8} & \underline{60.9} \\
    \rowcolor{green!15}
    \name{} & \textbf{DT} & \textbf{Linear} & \underline{2.7B}& \underline{14.7\%}  &1303.5 & 57.2 & 85.2 & 62.9& 30.7  & 31.1 &47.7 & 79.2 & 57.4 \\
    \rowcolor{yellow!15}
    \name{} & \textbf{DT} & \underline{Hybrid} & \underline{2.7B}& \textbf{11.2\%}  & \underline{1371.1} & \underline{63.7} & 86.7 & 66.3 & \underline{32.3} & 36.9 & 55.1 & 86.9 & 59.3  \\
    
    \bottomrule
    \end{tabular}
    }
    \vspace{-1em}
    \caption{\textbf{Comparison with existing VLMs on general VLM benchmarks.} ``Recipe'' denotes the adopted training recipe. ``PT'', ``SFT'', and ``DT'' denote the pre-training, supervised fine-tuning, and distillation training, respectively. ``Complexity'' denotes the model computation complexity with respect to the number of tokens. ``\# P.'' denotes the number of total parameters. ``\# T.P.'' denotes the percentage of trainable parameters ($\frac{\text{trainable paramters}}{\text{total parameters}}$). The best performance is highlighted in \textbf{bold} and the second-best result is \underline{underlined}.}
    \label{tab:results_general}
    \end{table*}

  
  \section{Conclusion and future directions}
  \label{sec:conclusion}
  \section*{Conclusion}
This paper aims to enhance our understanding of the computational complexity of computing various Shapley value variants. We found that for various ML models --- including decision trees, regression tree ensembles, weighted automata, and linear regression --- both local and global interventional and baseline SHAP can be computed in polynomial time under HMM modeled distributions. This extends popular algorithms, such as TreeSHAP, beyond their empirical distributional scope. We also establish strict complexity gaps between the various SHAP variants (baseline, interventional, and conditional) and prove the intractability of computing SHAP for tree ensembles and neural networks in simplified scenarios. Overall, we present SHAP as a versatile framework whose complexity depends on four key factors: \begin{inparaenum}[(i)] \item model type, \item SHAP variant, \item distribution modeling approach, \item and local vs. global explanations\end{inparaenum}. We believe this perspective provides deeper insight into the computational complexity of SHAP, paving the way for future work.




%We believe that our framework provides a more intricate understanding of SHAP computation complexity across different models, distributions, and variants, paving the way for further research.

Our work opens promising directions for future research. First, expanding our computational analysis to other SHAP-related metrics, such as asymmetric SHAP~\citep{frye20} and SAGE~\citep{covert2020understanding}, would be valuable. Additionally, we aim to explore more expressive distribution classes and relaxed assumptions beyond those in Section \ref{sec:tractable} while maintaining tractable SHAP computation. Finally, when exact computation is intractable (Section \ref{sec:intractable}), investigating the approximability of SHAP metrics through approximation and parameterized complexity theory~\citep{downey2012parameterized} is an important direction.

%Our work opens several promising avenues for future research on the computational properties of explainable AI methods, with a particular focus on SHAP. First, it would be interesting to broaden the computational analysis conducted in this work to include other popular SHAP-related metrics in the literature, such as asymmetric SHAP \cite{frye20} and SAGE \cite{covert2020understanding}. Also, in the future, we aim to explore more expressive distribution classes and relaxed distributional assumptions—extending beyond those examined in Section \ref{sec:tractable} —that still yield tractable SHAP computation. Finally, when exact computation proves intractable (Section \ref{sec:intractable}), it is worthwhile to theoretically investigate the question of the approximability of computing the SHAP metrics across various configurations, through the lens of approximation and parametrized complexity theory \cite{arora2009computational}.

%This paper aims to deepen our understanding of the computational complexity involved in obtaining different Shapley value variants. We found that for a variety of ML models, including decision trees, tree ensembles for regression, weighted automata, and linear regression models — computing both local and global interventional and baseline SHAP can be done in polynomial time when distributions are modeled by HMMs. This extends the distributional scope of popular algorithms like TreeSHAP, which is limited to empirical distributions. Additionally, we demonstrate a strict complexity gap between SHAP variants, showing that interventional and baseline SHAP can be strictly easier to compute than conditional SHAP. Despite these positive results, we uncovered intractability for various SHAP variants in neural networks and tree ensembles. Finally, we provided generalized complexity relations across SHAP variants. We believe that our framework offers a deeper understanding of the complexity involved in computing SHAP across various variants, models, distributions, as well as in both local and global computations, laying the groundwork for future research.

\section{Acknowledgements}
\label{sec:acknowledgements}
\section*{Acknowledgments}
{\textcopyright}2025 All rights reserved. The research described in this paper was carried out at the Jet Propulsion Laboratory, California Institute of Technology, under a contract with the National Aeronautics and Space Administration (80NM0018D0004).
  
  \bibliography{biblio}
  \bibliographystyle{unsrt}
  % \bibliographystyle{plain} % !!! change to this before submission
  % \bibliographystyle{plainnat}


%%%%%%%%%%%%%%%%%%%%%%%%%%%%%%%%%%%%%%%%%%%%%%%%%%%%%%%%%%%%

\clearpage
\appendix
\hypersetup{
    colorlinks,
   linkcolor={pierCite},
    citecolor={pierCite},
    urlcolor={pierCite}
}
\section*{Appendix}
The following sections provide deferred discussions, proofs and experimental details.
% \DoToC
% \clearpage
% \hypersetup{
%     colorlinks,
%    linkcolor={pierLink},
%     citecolor={pierCite},
%     urlcolor={pierCite}
% }


\section{Extended related work}
\label{sec:apx-related_work}
\section{Related Work}
% \subsection{Vision Language Model}
% 시각장애인에서 상황을 설명할 DB가 없으니 만들었다. 그리고 이를 VLM에 튜닝했다.
\subsection{Technical approaches for assisting the visually-impaired}


\subsection{Datasets for visual instruction tuning}


\section{Extension to the general additive shift setting}\label{sec:apx-extension}

% \begin{enumerate}
%     \item  What if there is no reference environment?
%     \item We show that instead betastar can be identified on $\cup_{e} (\mu_e - \mu_0)  (\mu_e - \mu_0)^\top + (\Sigma_e - \Sigma_0) $.
%     \item Reader can check: the results hold the same, just with new $\cS$
%     \item Highlight: $\noisecovxx$ cannot be identified, instead $\noisecovxx + \Sigma_0$
% \end{enumerate}

We discuss how our setting changes when we relax the assumptions on the existence of the reference environment. We consider the data-generating process in \cref{eqn:SCM}, where $\Ecaltrain = [m]$, $m \in \mathbb{N}$. If no environment $e$ exists with $\mue = 0$ and $\Sigmae = 0$, we first pick an arbitrary distribution $\PrefXY$ as the reference environment\footnote{In practice, it is useful to pick a distribution with the smallest covariance, i.e. $\trace \Cov(\Xref) \leq \trace \Cov(\Xe)$ for all $e$.} .
We denote $\noisecovxx' := \noisecovxxstar + \Sigmaref$. 

First, we show we can express the space $\cS$ of training additive shift directions defined in \cref{eqn:def-S} in the general case. We center all distributions by $\muref$  to obtain centered distributions $\tilde{\Pr}$ that $\EE_{X\sim \tilde{P}_e}[X]=0$. With respect to the arbitrary reference environment, we now define
\begin{align*}
    \tilde{\cS} \coloneqq \range \left[\bigcup_{e \in \Ecaltrain} \left(\Sigmae - \Sigmaref + (\mue - \muref)(\mue - \muref)^\top \right)\right]\subset \R^d.
\end{align*}
We now consider test shifts with respect to the environment $\PrefXY$\footnote{In other words, we require that the test distribution is a shifted version of the (arbitrarily) chosen reference distribution.}. We define the test shift upper bound $\Mtest = \gamma \Mseen + \gammaprime R R^\top$, where $\range(\Mseen) \subset \cS$ and $\range(R) \subset \cSperp$. Again, we can decompose the parameter $\betastar$ as $\betastar = \betastarS + \betastarperp$.
The projection $\betastarS$ of the causal parameter onto the relative training shifts induces the following observationally equivalent parameters corresponding to the reference distribution:
%\fy{still unsure if it should be in prop. shorter prop just make it seem less important as a result} also define the following set of parameters projected onto S ... 
\begin{align*}
    \thetastarS := (\betaS, \noisecovxx', \noisecovxyS, \noisecovyyS) = (\betastarS, \noisecovxx', \noisecovxystar + \noisecovxx' \betastarperp, \noisecovyystar + 2 \langle \noisecovxystar, \betastarperp\rangle + \langle \betastarperp, \noisecovxx' \betastarperp\rangle).
\end{align*}
Again, $\thetastarS$ can be identified from the training distributions and is referred to as the \emph{identified model parameters}. The following adapted version of \cref{prop:invariant-set} shows that assuming shifts on $\PrefXY$, the robust prediction model is only identifiable if the test shifts are in the direction of the relative training shifts:
\begin{proposition}[Identifiability of reference distribution parameters and robust prediction model]
\label{prop:invariant-set-general} Suppose that the set of training and test distributions is generated according to \Cref{eqn:SCM,eqn:testAbound}.
%, and the test distribution is generated under an unknown additive shift bounded by \cref{eqn:testAbound}. 
%$\betastarS$ induces \fy{didn't get "induce" here} a unique tuple of model parameters $\thetacS$ 
Then, $\thetacS$ is observationally equivalent
to $\thetastar$ and computable from training distributions.
%The parameters $\thetacS = (\betastarS, \SigmaS)$ are computable from training data, thus, we will refer to them as the \emph{identified model parameters}. 
Furthermore, it holds that
\begin{enumerate}[(a)]
 \item %\textbf{Identifiability of the model parameters.} 
 the model parameters %$\theta = (\beta, \noisecovxx, \noisecovxy, \noisecovyy)$ 
 generating the reference distribution can be identified up to the following \idset: 
\begin{align*}
   \Invset = \{ \betastarS + \alpha, \noisecovxx', \noisecovxyS - \noisecovxx' \alpha, \noisecovyyS - 2 \alpha^\top \noisecovxyS + \alpha^\top \noisecovxx' \alpha \colon \alpha \in \cSperp \}  \ni \thetastar 
\end{align*}
\item %\textbf{Identifiability of the robust predictor.}
the robust prediction model $\betarob$ as defined in \cref{eqn:formula-robust-predictor} is identified up to the set
    \begin{align*}
      \betastarS + (\gamma \projM + \noisecovxx')^{-1} \noisecovxyS + \{ (\gamma \projM + \noisecovxx')^{-1} \alpha\colon \, \alpha \in \range(R)  \} \ni \betarob 
    \end{align*}
\end{enumerate}
\end{proposition}
The proof is analogous to \cref{sec:apx-proof-invariant-set}. A version of \cref{thm:pi-loss-lower-bound} for perturbations on the reference environment follows accordingly. 


% As a last comment, another difference between this relaxed setting and the one presented in the main text is that $\EE(X_0X_0^\top) = \Sigma_0 + \noisecovxxstar$, and thus we can only identify $\Sigma_0 + \noisecovxxstar$. This, however, has no consequences on the results of \cref{sec:main-results}.


\section{Comparison to finite robustness methods continued}\label{sec:apx-anchor-connections}
\subsection{The setting of continuous anchor regression \cite{rothenhausler2021anchor}}\label{subsec:anchor}
In this section, we evaluate the \idRRs of the continuous anchor regression estimator.
    In the continuous anchor regression setting, during training we 
    observe the distribution %the training data are observed 
    according to the process $X = M A + \eta$; $Y = {\betastar}^\top X + \xi$, where $A$ is an observed $q$-dimensional anchor variable with mean $0$ and covariance $\Sigma_A$ and $M \in \R^{d \times q}$ is a known matrix. Note that in this setting, we do not have a reference environment, but, since the anchor variable is observed, the distribution of the additive shift $M A$ is known.  The test shifts are assumed to be bounded by $\Mtest = \gamma M \Sigma_A M^\top$. Since $\range(\Mtest )\subset \cS = \range(M)$, no new directions are observed during test time, in other words, $R = 0$. Thus, both the corresponding robust loss and the anchor regression estimator can be determined from training data. It holds that
\begin{align*}
    \betaa = \argmin_{\beta \in \R^d} \Lossrob(\beta; \thetastar,\gamma M \Sigma_A M^\top).
\end{align*}
Again, the pooled OLS estimator corresponds to $\betaa$ with $\gamma = 1$. Similar to the discrete anchor case, in case the test shifts are given by $\Mnew = \gamma M \Sigma_A M^\top + \gammaprime R R^\top$, the worst-case robust risk \eqref{eqn:PI-robust-loss} is given by
\begin{align*}
    \Lossrobpi(\beta; \Invset, \Mnew) = \gammaprime (\Cker + \| R^\top \beta \|_2)^2 + \Lossrob(\beta; \thetastar,\gamma M \Sigma_A M^\top) 
\end{align*}
and for the best worst-case robustness of the anchor estimator it holds 
\begin{align*}
    \Lossrobpi(\betaa, \Invset; \Mtest)&= (\Cker + \| R R^\top (\noisecovxxstar + \gamma M \Sigma_A M^\top )^{-1} \noisecovxyS \| )^2 \gammaprime + \text{const}; \\
    \lim_{\gammaprime \to 0} \Lossrobpi(\betaa, \Invset;\Mnew)/\gammaprime &= \lim_{\gammaprime \to 0} \minimaxPIlossarg{\Mnew}/ {\gammaprime}.
\end{align*}
The above results follow by analogy with \cref{sec:apx-proof-of-corollary}.

\subsection{Distributionally robust invariant gradients (DRIG) \cite{shen2023causalityoriented}}\label{subsec:drig}
DRIG \cite{shen2023causalityoriented} introduce a more general additive shift framework, where a collection of additive shifts $\Ae$ is given with moments $(\mue,\Sigmae)$. For each environment $e$, we observe data $(\Xe, \Ye)$ distributed according to the equations $\Xe= \Ae + \eta; \: \Ye = {\betastar}^\top \Xe + \xi$, where the noise is distributed like in \cref{eqn:SCM}. This DGP arises from the structural causal model assumption as described in \cref{fig:ex-scm}.  DRIG consider more a more general intervention setting, additionally allowing additive shifts of $Y$ and hidden confounders $H$. However, their identifiability results can only be shown for the case of interventions on $X$, and since identifiability of the causal parameter is a crucial part of our analysis, we only consider shifts on the covariates. DRIG assumes existence of a reference environment  $e = 0$ with $\mu_0 = 0$ and for which it is required that the second moment of the reference environment is dominated by the second moment of the training mixture: 
\begin{align*}
    \Sigma_0 \preceq \sum_{e \in [m]} w_e (\Sigma_e + \mu_e \mu_e^\top).
\end{align*}
This assumption allows \cite{shen2023causalityoriented} to derive the DRIG estimator which is robust against test shifts upper bounded by $\Mdrig :=  \gamma \sum_{e \in [m]} w_e (\Sigma_e - \Sigma_0 + \mu_e \mu_e^\top)$. The following lemma allows us to make further statements about $\Mdrig$:
\begin{lemma}\label{lm:span-inclusion}
    Let $A$ and $B$ be positive semidefinite matrices such that $B \preceq A$. Then it holds that $\range(B) \subset \range(A)$. 
\end{lemma}
\begin{proof}
    It suffices to show that $\ker(A) \subset \ker(B)$. ($\ker(A) \subset \ker (B)$ implies that $\range(A) = \ker (A)^\perp  \subset \ker(B)^\perp = \range(B)  $.) Consider $x \in \ker(A)$, $x \neq 0$. Then it holds that $x^\top (A - B) x = x^\top A x - x^\top B x = 0 - x^\top B x \geq 0$, from which it follows that $x^\top B x = 0$ and thus $x \in \ker(B)$. 
\end{proof}
Because of the assumption $\Sigma_0 \preceq \sum_{e \in [m]} w_e (\Sigma_e + \mue \mue^\top)$, by \cref{lm:span-inclusion} it follows that $\range(\Sigma_0) \subset \range \sum_{e \geq 1} ( \Sigma_e + \mu_e \mu_e^\top )$  and thus 
\begin{align*}
\range(\Mdrig) \subseteq \range\left( \sum_{e \geq 1} w_e (\Sigma_e + \mu_e \mu_e^\top) \right). 
\end{align*}
Hence, the robustness directions achievable by DRIG in the "dominated reference environment" setting are the same as the ones under the assumption $\Sigma_0 = 0$. \\
Again, we observe that the test shifts bounded by $\gamma \Mdrig$ are fully contained in the space of identified directions $\cS$. If the test shifts are instead bounded by $\Mnew := \gamma \Mdrig + \gammaprime R R^\top$,  including some unseen directions $\range(R) \subset \cSperp$, the robust risk in the DRIG setting is only partially identified. The worst-case robust risk \eqref{eqn:PI-robust-loss} is given by 
\begin{align*}
    \Lossrobpi(\beta; \Invset, \Mnew) = \gammaprime (\Cker + \| R^\top \beta \|_2)^2 + \Lossrob(\beta; \thetastar, \gamma \Mdrig),
\end{align*}
and again, the DRIG estimator is optimal for infinitesimal shifts $\gammaprime$ and suboptimal for larger $\gammaprime$:
\begin{equation*}
    \begin{aligned}
        \Lossrobpi(\betaDRIG; \Invset,\Mnew) &= (\Cker + \| R R^\top (\noisecovxxstar + \gamma \Mdrig)^{-1} \noisecovxyS \| )^2 \gammaprime + \text{const}; \\ 
\text{whereas }\frac{\minimaxPIlossarg{\Mnew}}{\gammaprime} &= \Cker^2, \: \text{ if } \gammaprime \geq \gammath; \\  \lim_{\gammaprime \to 0} \frac{\minimaxPIlossarg{\Mnew}}{\gammaprime} &= (\Cker + \| R R^\top (\noisecovxxstar + \gamma \Mdrig)^{-1} \noisecovxyS\| )^2.
    \end{aligned}
\end{equation*}
The above results follow by plugging $\Mnew$ with $M := \Mdrig$ into the proof of \cref{cor:estimators} in \cref{sec:apx-proof-of-corollary}.

% Thus, the anchor regression estimator is optimal in the limit of small unseen shifts. However, for larger shift strengths, the worst-case robust risk of both estimators significantly deviates from the best achievable robustness. Moreover, under some conditions on the covariance matrix\footnote{E.g., if $\noisecovxxstar$ is block-diagonalizable w.r.t. $\cS$ and $\cSperp$.}, pooled OLS and the anchor estimator achieve the same rate in $\gammaprime$, showcasing how a finite robustness method can perform similarly to empirical risk minimization if the assumptions on the robustness set are not met.




% We discuss how to extend our results to the general additive shift setting described in \cite{shen2023causalityoriented}. We consider a discrete training environment variable $\Etrain$, which takes values in $[m]$ with probability mass function $\prob(\{\Etrain = e\}) = w_e$ for some positive weights $w_e > 0$. Conditioned on the event $\Etrain = e$, 
% % We denote the data $(X, Y) \in \R^{d+1}$ conditioned on $\Etrain = e$ by $(\Xe, \Ye)$. For each environment $\Etrain = e$, 
% the data are generated according to the SCM~\eqref{eqn:SCM}.
% % \begin{align}\label{eqn:generalSCM}
% %     \Xe &= \Ae + \eta, \\
% %     \Ye &= \betastar^\top X^e + \xi,
% % \end{align}
% % where $\Ae \sim \cN(\mu_e, \Sigma_e)$ and $U = (\eta, \xi) \sim \cN(0, \Sigma_U)$.
% We now discuss how the assumption that there exists a reference environment $e = 0$ for which it holds that $\mu_0 = 0$ and $\Sigma_0 = 0$ relates to the conditions in \citep{shen2023causalityoriented}. First, notice that the first restriction is met by centering all data around $\mu_0$. The second condition is formulated weaker in \cite{shen2023causalityoriented}, where it is solely required that the second moment of the reference environment is dominated by the second moment of the training mixture: 
% \begin{align}
%     \Sigma_0 \preceq \sum_{e \in [m]} w_e (\Sigma_e + \mu_e \mu_e^\top).
% \end{align}

% This assumption allows \cite{shen2023causalityoriented} to derive the DRIG estimator which is robust against test shifts upper bounded by $ \gamma \sum_{e \in [m]} w_e (\Sigma_e - \Sigma_0 + \mu_e \mu_e^\top)$. However, since $\Sigma_0 \preceq \sum_{e \in [m]} w_e \Sigma_e$, it follows that $\range\ \Sigma_0 \subset \cup_{e \geq 1} \range\(\Sigma_e + \mu_e \mu_e^\top )$ (see \cref{lm:span-inclusion}) and thus 
% \begin{align*}
% \range\ \left( \sum_{e \in [m]} w_e (\Sigma_e - \Sigma_0 + \mu_e \mu_e^\top) \right) \subset \range\ \left( \sum_{e \geq 1} w_e (\Sigma_e + \mu_e \mu_e^\top) \right). 
% \end{align*}
% Hence, the robustness directions achievable by DRIG in the $\Sigma_0 = 0$ setting are the same as in the general reference environment setting. 

\section{Empirical estimation of the worst-case robust predictor}\label{sec:apx-empirical-estimation}


In this section, we discuss how to compute the worst-case robust loss and its minimizer from finite-sample multi-environment training data. We first describe the finite-sample setting and provide a high-level algorithm. We then discuss some parts of the algorithm in more detail. Finally, we show that the empirical worst-case robust loss is consistent under certain assumptions.  
Recall that we assume that $\Mtest = \gamma \PSMpop + \gammaprime R R^\top$, where $\gamma, \gammaprime \geq 0$, $\PSMpop $ is a PSD matrix satisfying $\range (\PSMpop) \subset \cS$ and $R$ is a semi-orthogonal matrix satisfying $\range (R) \subset \cSperp$. 

\subsection{Computing the worst-case robust loss}

\begin{algorithm}[h!]
    \caption{Computation of the worst-case robust loss} \label{alg:id-rob-loss}
    \begin{algorithmic}[1]
    \State \textbf{Input:} Multi-environment data $\cD \coloneqq \cup_{e \in \Ecaltrain} \cD_e$, test shift strengths $\gamma, \gammaprime > 0$, test shift directions $\Mtest \in \R^{d \times d}$, upper bound $C > 0$ on the norm of $\betastar$.
    % nuisances $S_0$, $R_0$, $\Cker$, $\betastarS$.
    
    \State \textbf{Step 1:} Estimate the training shift directions $\cShat(\cD)$, its  orthogonal complement $\cSperphat(\cD)$, and the identified linear parameter $\betaShat$.
      % \begin{align*}
    %     S(\cD) = \sum_{e = 1}^m ( \Cov(X^e)- \Cov(X^0) + \mu_e \mu_e^\top - \mu_0 \mu_0^\top)
    % \end{align*}
    \State \textbf{Step 2:} Estimate the identified and non-identified test shift upper bounds $\Mseenemp$, $\Rhat \Rhat^\top$, respectively, from $\Mtest$, $\cShat(\cD)$ and $\cSperphat(\cD)$.
 % \vphantom{$\betaShat$}
 %    \begin{align*}
 %        S_M, \PSM &\gets f(M, S); \quad S^{\perp}_M, \PSperpM \gets f(M, S^{\perp}). 
 %    \end{align*}
    \State \textbf{Step 3:} Estimate the maximum norm $\Ckerhat$ of the non-identified linear parameter.
    % \begin{align*}
    %     \Ckerhat \gets \sqrt{C^2 - \| \betaShat \|_2^2}.
    % \end{align*}
    % \State Compute the loss term on the reference environment:
    % \begin{align*}
    %    \LLref(\beta; \cD_0) &\gets \sum_{i = 1}^{n_0} (Y_{0,i} - \beta^\top X_{0,i})^2. 
    % \end{align*}
    % \State Compute the invariance penalty term:
    % \begin{align*}
    %    \LLinv(\beta; \betaShat, \PSM, \gamma) &\gets \gamma \| \PSM (\beta - \betastarS) \|^2_2. 
    % \end{align*}

    % \State Compute the non-identifiability penalty term \fy{probably you mean C-hat-ker}
    % \begin{align*}
    %    \LLid(\beta; \Cker, \PSperpM, \gamma) &\gets \gamma (\Cker + \| \PSperpM \beta \|_2)^2. 
    % \end{align*}
    \State \vphantom{$\hat{R}$}\textbf{Step 4:} Compute the worst-case robust loss function 
    \begin{align*}
        \LL_n(\beta; \betaShat, \PSM, \PSperpM) &\gets \underbrace{\LLref(\beta; \cD_0)}_{\text{reference loss}} + \underbrace{\LLinv(\beta; \betaShat, \PSM, \gamma)}_{\text{invariance penalty term}} + \underbrace{\LLid(\beta; \Ckerhat, \RRhat, \gammaprime)}_{\text{non-identifiability penalty term}}.
    \end{align*}
    \State \textbf{Return:}  worst-case robust predictor and the estimated minimax "hardness" of the problem: 
    \begin{align*}
        \betarobpihat &\gets \argmin_{\beta \in \R^d} \LL_n(\beta; \betaShat, \PSM, \PSperpM); \\ 
       \hat{\mathfrak{M}}(\cD,\gamma, \gammaprime, \Mtest) &\gets \min_{\beta \in \R^d} \LL_n(\beta; \betaShat, \PSM, \PSperpM).
    \end{align*}
    
        
    \end{algorithmic}
\end{algorithm}


\paragraph{Training data.} We observe data from $m + 1$ training environments indexed by $E \in \Ecaltrain = \{0,..., m\}$, where $E = 0$ represents the reference environment. We impose a discrete probability distribution $\prob^{E}$ on the training environment $E \in \Ecaltrain$, resulting in the joint distribution $(X, Y, E) \sim \prob^{X, Y \mid E} \times \prob^{E}$.  For each environment $E = e$, we observe the samples $\cD_e \coloneqq \{(X_{e,i}, Y_{e,i})\}_{i = 1}^{n_e}$, where $(X_{e, i}, Y_{e, i})$ are independent copies of $(X_e, Y_e) \sim \prob^{X, Y \mid E = e}$. Then, the resulting dataset is $\cD \coloneqq \cup_{e \in \Ecaltrain} \cD_e$ with $n \coloneqq n_0 + \cdots + n_m$. Furthermore, for each environment $E = e$, we define the weights $w_e \coloneqq n_e / n$. 

\paragraph{Computation of the worst-case robust loss.} In \cref{alg:id-rob-loss}, we present a high-level scheme for computing the worst-case robust loss from multi-environment data, which consists of multiple steps. First, nuisance parameters related to the training and test shift directions are estimated, which we describe in more detail below. Afterwards, the three terms of the loss are computed: the (squared) loss $\LLref(\beta; \cD_0)$ on the reference environment is computed as 
\begin{align*}
       \LLref(\beta; \cD_0) = \sum_{i = 1}^{n_0} (Y_{0,i} - \beta^\top X_{0,i})^2. 
    \end{align*}
The invariance penalty term $\LLinv(\beta; \betaShat, \PSM, \gamma)$ (which increasingly aligns any estimator $\beta$ in the direction of the estimated invariant predictor $\betaShat$ as $\gamma \to \infty$) can be computed as following in the linear setting:
    \begin{align*}
       \LLinv(\beta; \betaShat, \PSM, \gamma) = \gamma  (\beta - \betaShat)^\top \PSM (\beta - \betaShat) . 
    \end{align*}
Finally, the non-identifiability penalty term $\LLid(\beta; \Ckerhat, \PSperpM, \gamma)$ can be computed as follows:
    \begin{align*}
       \LLid(\beta; \Ckerhat, \PSperpM, \gammaprime) &= \gammaprime (\Cker + \| \PSperpM \beta \|_2)^2. 
    \end{align*}
The non-identifiability term, with increasing $\gammaprime$, penalizes any predictor $\beta$ towards zero on the subspace $\Rhat$ of non-identified test shift directions. In total, the worst-case robust loss (in the linear setting) equals
\begin{align*}
    \LL_n(\beta; \betaShat, \PSM, \PSperpM) = \sum_{i = 1}^{n_0} (Y_{0,i} - \beta^\top X_{0,i})^2 + \gamma (\beta - \betaShat)^\top \PSM (\beta - \betaShat) + \gammaprime (\Cker + \| \PSperpM \beta \|_2)^2, 
\end{align*}
where we suppress dependence on $C$, $\gamma$ and $\gammaprime$ and only leave the dependence on the nuisance parameters.
\paragraph{Choice/Estimation of nuisance parameters.} We now provide more details on the empirical estimation of the nuisance parameters $\cShat, \cSperphat, \Rhat$, $\PSM$, and $\betaShat$. 
\begin{itemize}
    \item The \textbf{constant} $C$ corresponds to the upper bound on the norm of the true causal parameter $\betastar$. Thus, the practitioner chooses $C$ in advance to ensure that (with high probability) $\| \betastar \|_2 \leq C$. 
    \item The \textbf{training shift directions} $\cShat$ can be computed via 
     \begin{align}\label{eq:S-estimation}
        \cShat(\cD) = \mathrm{range} \left[\sum_{e = 1}^m ( \Cov(X^e)- \Cov(X^0) + \mu_e \mu_e^\top - \mu_0 \mu_0^\top)\right],
    \end{align}
where for $e \in \Ecaltrain$, the matrix $\Cov(X^e)$ is the empirical covariance matrix estimated within the training environment $E = e$, and $\mu_e \in \R^d$ is the empirical mean of the covariates within the training environment $E = e$. Additionally, we compute the orthogonal complement $\cSperphat(\cD)$ of the space $\cShat(\cD)$\footnote{In general, $S(\cD)$ is a proper subspace of $\R^d$ and the RHS of \eqref{eq:S-estimation} corresponds to a sum of low-rank second moments. This can be consistently estimated if, for instance, the rank of each shift is known (e.g. in the mean shift setting), or the covariances have a spiked structure, allowing to cut off small eigenvalues.}. 
\item  Computation of the \textbf{seen and unseen test shift directions.} Multiple options are possible for the practitioner to compute the empirical test shift directions $\Mseenemp$ and $\Rhat \Rhat^\top$. One option is to choose $\Mseenemp = \sum_{e} w_e ( \Cov(X^e)- \Cov(X^0) + \mu_e \mu_e^\top - \mu_0 \mu_0^\top)$, where $w_e$ is the proportions of observations in environment $e \in \Ecaltrain$, akin to anchor regression \cite{rothenhausler2021anchor} and DRIG \cite{shen2023causalityoriented} with an appropriate shift magnitude $\gamma$. Afterwards, $\Rhat \Rhat^\top$ is chosen to be a projection onto an appropriate subspace of $\cSperphat$ (if additional information about test shift directions is available). If no information is given, we can choose $\Rhat \Rhat^\top = \Pi_{\cSperphat}$. Alternatively, if the information about potential test shift directions is given in form of a PSD matrix $\Mtest$, for instance, $\Mtest$ being a projection onto some subspace, we can decompose $\Mtest$ 
into identified and non-identified shift directions (and their corresponding projection matrices). 
Let $\PicShat$ and $\PicSperphat$ denote the projection matrices on $\cShat(\cD)$ and $\cSperphat(\cD)$, respectively. Consider the singular value decompositions $\PicShat \Mtest = U_{\cShat} \Sigma_{\cShat} V_{\cShat}^\top$ and $\PicSperphat \Mtest = U_{\cSperphat} \Sigma_{\cSperphat} V_{\cSperphat}^\top$ Then,  define
\begin{align*}
\Shat &= U_{\cShat}, \quad \Rhat = U_{\cSperphat}.
\end{align*}
The subspaces $\range (\PicShat \Mtest)$ and $\range( \PicSperphat \Mtest)$ are minimal subspaces contained in $\cShat$ and $\cSperphat$, respectively, such that $\range(\Mtest) \subset \range( \PicShat \Mtest) \oplus \range( \PicSperphat \Mtest)$. We can then take as $\PSM$ and $\PSperpM$ their corresponding projection matrices. 
\item The \textbf{identified parameter} $\betaShat$ (approximately) equals the true invariant parameter $\betastar$ on the space of training shift directions $\cShat$. 
% As known from the anchor regression/IV literature \citep{rothenhausler2021anchor,shen2023causalityoriented}, 
As conjectured in the anchor regression literature \citep{rothenhausler2021anchor,shen2023causalityoriented, jakobsen2022distributional} (see, for example, the discussion right after Theorem~3.4 in \citep{jakobsen2022distributional} and Appendix~H.3 therein)
for $\gamma \to \infty$, the estimators $\beta^{\gamma}_{\mathrm{anchor}}$ and $\beta^{\gamma}_{\mathrm{DRIG}}$ converge to the invariant parameter $\betastar$ on $\cS$. Thus, the identified parameter can be estimated as
\begin{align*}
    \betaShat \coloneqq \PicShat \beta^{\infty}_{\mathrm{anchor}} \quad \text{or} \quad \betaShat \coloneqq \PicShat \beta^{\infty}_{\mathrm{DRIG}}
\end{align*}
for the setting of mean or mean+variance shifts, respectively. 
% \nico{cite PULSE Theorem~3.4 and Figure H.6 in Appendix~H.3}
\end{itemize}

% \begin{algorithm}[t]
%     \caption{Computation of the worst-case robust loss} \label{alg:id-rob-loss}
%     \begin{algorithmic}[1]
%     \State \textbf{Input:} Multi-environment data $\cD \coloneqq \cup_{e \in \Ecaltrain} \cD_e$, test shift strength $\gamma > 0$, test shift directions $M \in \R^{d \times d}$, causal parameter upper bound $C > 0$.
%     % nuisances $S_0$, $R_0$, $\Cker$, $\betastarS$.
%     \State Estimate the training shift directions $S(\cD)$ and compute the orthogonal complement $S^{\perp}(\cD)$.
%     \State Estimate the identified causal parameter $\betaShat$ on $S$.
%     % \begin{align*}
%     %     S(\cD) = \sum_{e = 1}^m ( \Cov(X^e)- \Cov(X^0) + \mu_e \mu_e^\top - \mu_0 \mu_0^\top)
%     % \end{align*}
%     \State Estimate the identified and non-identified test shift directions and their projections:\vphantom{$\betaShat$}
%     \begin{align*}
%         S_M, \PSM &\gets f(M, S); \quad S^{\perp}_M, \PSperpM \gets f(M, S^{\perp}). 
%     \end{align*}
%     \State Estimate the norm of the non-identified causal parameter part:
%     \begin{align*}
%         \Ckerhat \gets \sqrt{C^2 - \| \betaShat \|_2^2}.
%     \end{align*}
%     \State Compute the loss term on the reference environment:
%     \begin{align*}
%        \LLref(\beta; \cD_0) &\gets \sum_{i = 1}^{n_0} (Y_{0,i} - \beta^\top X_{0,i})^2. 
%     \end{align*}
%     \State Compute the invariance penalty term:
%     \begin{align*}
%        \LLinv(\beta; \betaShat, \PSM, \gamma) &\gets \gamma \| \PSM (\beta - \betastarS) \|^2_2. 
%     \end{align*}

%     \State Compute the non-identifiability penalty term \fy{probably you mean C-hat-ker}
%     \begin{align*}
%        \LLid(\beta; \Cker, \PSperpM, \gamma) &\gets \gamma (\Cker + \| \PSperpM \beta \|_2)^2. 
%     \end{align*}
%     \State Compute the resulting loss function \fy{typo, should be perp in the id loss}
%     \begin{align*}
%         \LL_n(\beta; \betaShat, \PSM, \PSperpM, \Ckerhat, \gamma) &\gets \LLref(\beta; \cD_0) + \LLinv(\beta; \betaShat, \PSM, \gamma) + \LLid(\beta; \Ckerhat, \PSM, \gamma).
%     \end{align*}
%     \State \textbf{Return:}  worst-case robust predictor and the estimated minimax "hardness" of the problem: 
%     \begin{align*}
%         \betarobpihat &\gets \argmin_{\beta \in \R^d} \LL_n(\beta; \betaShat, \PSM, \PSperpM, \Ckerhat, \gamma); \\ 
%        \hat{\mathfrak{M}}(\cD,\gamma, M) &\gets \min_{\beta \in \R^d} \LL_n(\beta; \betaShat, \PSM, \PSperpM, \Ckerhat, \gamma).
%     \end{align*}
    
        
%     \end{algorithmic}
% \end{algorithm}




% \begin{enumerate}
%     \item \julia{put in the algorithm (very high level), then plug in all parts, then go over to the empirical estimation of our loss. High-level comments in the alg (as comments in the algorithm). cf Piers paper }

%     \item \nico{Fix argument for bound $MM^\top \preceq P_{S_{0}, M} + P_{R_{0}, M}$. Quick fix: no prior info on shift directions, i.e., $M = I$, so $MM^\top \preceq S_0S_0^\top + R_0R_0^\top$ trivially.}

%     \item \nico{Single $\gamma$ or $\gamma$, $\gammaprime$?}

%     \item \nico{No results in the literature for the consistency of the anchor estimator where $\gamma \to \infty$. High-level difficulty: need to define estimator where $\gamma_n \to \infty$ at the right speed.
%     Quick fix: Consider anchor setting like $X = MA + \eta$, where $M$ has distinct singular values [to discuss with Julia]}
% \end{enumerate}

% \paragraph{Nuisance parameters.}
% Let us fix the reference environment $0 \in \Ecaltrain$ and define the symmetric matrix
% \begin{align}\label{eqn:sigma-sim}
%     \Sigma \coloneqq
%     \sum_{e \in \Ecaltrain} w_e \left\{\Sigma_e - \Sigma_0 + (\mu_e - \mu_0)(\mu_e - \mu_0)^\top\right\} \in \R^{d \times d},
% \end{align}
% \julia{Sigma0 and mu0 are zero} where $w_e \coloneqq \prob^E(\{e\})$, $\mu_e \coloneqq \EE[X_e]$, and $\Sigma_e \coloneqq \EE[(X_e - \mu_e)(X_e - \mu_e)^\top]$.
% The symmetric matrix $\Sigma$ uniquely defines the space of training shift directions $\cS \coloneqq \range\(\Sigma)$.
% Let $S_0 \in \R^{d \times q}$ and $R_0 \in \R^{d \times (d - q)}$ denote orthonormal matrices whose columns form a basis for $\cS$ and $\cSperp$, respectively.
% Recall that we have access to some information about the anticipated test shift directions consisting of  a subspace $\cM \coloneqq \range\(M)$\footnote{If no information on the potential test shift directions is given, we choose $M = I_d$.}, where $M \in \R^{d \times r}$ is an orthonormal matrix, and test shift strength $\gamma > 0$. Then, by \cref{lm:upper-bound}, it holds that $MM^\top \preceq P_{S_{0}, M} + P_{R_{0}, M}$, where \julia{fix argument, remove Lemma F2}
% \begin{align}\label{eq:proj-to-check}
%     P_{S_{0}, M} \coloneqq S_0S_0^\top M M^\top S_0S_0^\top \in \R^{d \times d},\quad P_{R_{0},M} \coloneqq R_0R_0^\top M M^\top R_0R_0^\top \in \R^{d \times d}, 
% \end{align} 
% with $\range\(P_{S_{0, M}}) \subseteq \cS$ and $\range\(P_{R_{0, M}}) \subseteq \cSperp$.
% Moreover, define the projection of the causal parameter onto the training directions as
% \begin{align}
%     \betastarS \coloneqq S_0S_0^\top \betastar \in \R^d.
% \end{align}
% We call $\varphi_0 \coloneqq (P_{S_0, M}, \PSperpMpop, \betastarS)$ the true nuisance parameters. 


% \paragraph{Estimation of the nuisance parameters.}
% We now describe how to estimate the nuisance parameters $\varphi_0 = (P_{S_0, M}, \PSperpMpop, \betastarS)$ from the dataset $\cD$.
% First, let $\hat{\Sigma} \in \R^{d\times d}$ be the sample version of~\eqref{eqn:sigma-sim}, where the moments are replaced with sample averages and the environment probabilities $w_e$ with their corresponding sample frequencies.
% Suppose now that the dimension of $\cS$ is known\footnote{This assumption holds, for example, in the anchor regression setting where the anchor variable $A$ is observed (see \cref{subsec:anchor}).}.
% Then, the matrix $S_0 \in \R^{d \times q}$ can be estimated by computing the eigenvectors $\hat{S}$ of $\hat{\Sigma}$ corresponding to the $q$ largest eigenvalues. Similarly, the matrix $R_0 \in \R^{d \times (d - q)}$ can be estimated by computing the eigenvectors $\hat{R}$ of $\hat{\Sigma}$ corresponding to $d - q$ smallest eigenvalues.
% Given knowledge of test shift directions $\cM \coloneqq \range\(M)$, we can then estimate \nico{update this depending on~\eqref{eq:proj-to-check}}
% \begin{align}
%     \hat{P}_{S, M} \coloneqq \hat{S}\hat{S}^\top M M^\top \hat{S}\hat{S}^\top \in \R^{d \times d},\quad \hat{P}_{R,M} \coloneqq \hat{R}\hat{R}^\top M M^\top \hat{R}\hat{R}^\top \in \R^{d \times d}. 
% \end{align} 
% Moreover, the vector $\betastarS$ can be estimated as $\hat{\beta}^\cS\coloneqq $ \nico{...to clarify how}


% \paragraph{Robust identifiable risk and worst-case robust predictor.}
% For fixed constants $C > 0$ and $\gamma, \gammaprime > 0$, 
% the robust identifiable risk is defined for all parameters of interest $\beta \in \R^d$ and nuisance parameters $\varphi \coloneqq (P_S, P_R, b)$ as
% \begin{align}\label{eqn:rob-loss-ell}
%     \LL(\beta, \varphi)  \coloneqq 
%     \EE\left[\left(Y_0 - \beta^\top X_0 \right)^2 \right]
%     + \gamma  \norm{P_S(b - \beta)}_2^2 
%     +   \gammaprime \left(\sqrt{C - \norm{b}_2^2} 
% + \norm{P_R\beta}_2\right)^2,
% \end{align}
% \julia{add discussion on $C$} where $(X_0, Y_0) \sim \prob^{X, Y \mid E = 0}$ denote the predictor-response pair from the reference environment.
% For fixed $\varphi$, the worst-case robust loss is a strongly convex objective function in $\beta$, even though it is not differentiable at $\beta = 0$. The strong convexity implies the existence of a unique minimizer, which we call the worst-case robust predictor and is defined as
% \begin{align}
%     \betarobpi \coloneqq \argmin_{\beta \in \R^d} \LL(\beta, \varphi_0).
% \end{align}
% If $\gamma = 0$, the worst-case robust loss $\LL(\beta, \varphi_0)$ evaluated at the true nuisance parameters $\varphi_0 = (P_{S_0, M}, P_{R_0, M}, \betastarS)$ coincides with the squared loss on the reference environment. If no new shift directions are expected, i.e. $R_0 = 0$ (and correspondingly $\| b \|_2^2 = C$), the loss coincides with the anchor regression/DRIG loss \citep{rothenhausler2021anchor,shen2023causalityoriented}, with robustness set given by $C^\gamma = \{ \Atest \in \R^d: \EE[\Atest {\Atest}^\top] \preceq \gamma P_{S_{0, M}} \}$.
% \nico{$C^\gamma$ confusing with $C$}

% The second penalization term of the worst-case robust loss can be seen as a regularizer along the non-identifiable directions.
% If the training environments are rich enough to identify the causal parameter, i.e., $\cS = \R^d$, $\betastarS = \betastar$, then the loss corresponds to the squared loss on the reference environment with a causal regularization term $\gamma \| P_{S_{0, M}}(\betastarS - \beta) \|_2^2$ which penalizes the prediction models on test shift directions. 
% If the test shift strength $\gamma$ is large enough, the optimal estimator corresponds to the robust estimator constrained to the direct sum $\cS \oplus (\cSperp - \range\ (P_{R_{0, M}}))$ of the "identified" directions $\cS$ and "stable" unidentified directions $\cSperp - \range\ (P_{R_{0, M}})$.

% Thus, the worst-case robust estimator interpolates on two levels: on $\cS$,  it interpolates between the OLS and the causal predictor $\betastarS$ for  $\gamma \in [1, \infty]$\Nicola{$\gamma \in [1, \infty)$} – similarly to anchor regression, which can be seen as the 
% "finite robustness" axis. However, it also interpolates between the anchor estimator and the "abstaining" estimator~\eqref{eqn:abstaining} as the strength of non-identified shifts increases (the "non-identifiability" axis).

\subsection{Consistency of the worst-case robust predictor} 
For any estimator $\beta \in \R^d$ and given the estimated nuisance parameters $\hat\varphi \coloneqq (\PSM, \PSperpM, \betaShat)$, we define the sample worst-case robust risk as
\begin{align}\label{eqn:rob-loss-ell-sample}
\begin{split}
    \LL_n(\beta, \hat{\varphi})  \coloneqq &
    \frac{1}{n_0}
    \sum_{i \in \cD_0}\left(Y_{0,i} - \beta^\top X_{0,i} \right)^2
    + \gamma (\betaShat - \beta)^\top \PSM (\betaShat - \beta) 
    +   \gammaprime \left(\sqrt{C - \norm{\betaShat}_2^2} 
+ \norm{\PSperpM\beta}_2\right)^2.
\end{split}
\end{align}
% \nico{remove any gammaprime}
Correspondingly, we define the estimator of the worst-case robust predictor by
\begin{align}\label{eqn:betarobpi-consistent}
    \betarobpihat \coloneqq \argmin_{\beta \in \mathcal{B}} \LL_n(\beta, \hat{\varphi}),
\end{align}
where $\mathcal{B} \subseteq \R^d$ is some compact set whose interior contains the true parameter $\betarobpi$.


To show the consistency of~\eqref{eqn:betarobpi-consistent}, we first require consistency of the nuisance parameter estimators, which we state as an assumption.
\begin{assumption}\label{ass:consistency-nuisance}
    The estimated nuisance parameters $\hat\varphi \coloneqq (\PSM, \PSperpM, \betaShat)$ are consistent, that is, for $n \to \infty$,
    \begin{align*}
       \norm{\PSM - \PSMpop}_F \stackrel{\prob}{\to}0,
        \quad
        \norm{\PSperpM - \PSperpMpop}_F \stackrel{\prob}{\to}0,
        \quad
        \hat\beta^\cS \stackrel{\prob}{\to} \betastarS \coloneqq \Pi_{\cS} \betastar,
    \end{align*}
    where for any matrix $A \in \R^{m \times q}$, $\norm{A}_F = \sqrt{\trace(A^\top A)}$ denotes the Frobenius norm,  $\PSMpop$ is a PSD matrix with bounded eigenvalues with $\range(\PSMpop) \subset \cS$, and $\PSperpMpop$ is the corresponding population projection matrix onto $\cSperp$. 
\end{assumption}
 Depending on the assumptions of the data-generating process, Assumption~\ref{ass:consistency-nuisance} can be shown to hold. For example, in the anchor regression setting \cite{rothenhausler2021anchor}, the consistency of $\Mseen = \Manchor$, the projection
matrix $\PSperpM$, and $\PicShat$ holds if the dimension of $\cS$ is known (due to the mean shift structure).
The proof relies on the Davis--Kahan theorem (see, for example, \citep{yu2015useful}) and the consistency of the covariance matrix estimator.
Moreover, in the anchor regression setting, it is conjectured that the estimator $\beta^{\infty}_{\mathrm{anchor}}$ converges to its population counterpart (as discussed right after Theorem~3.4 in \citep{jakobsen2022distributional} and Appendix~H.3 therein) which implies that $\betaShat \coloneqq \PicShat \beta^{\infty}_{\mathrm{anchor}} $ consistently estimates $\betastarS = \Pi_{\cS} \betastar$.
% \nico{cite PULSE Theorem~3.4 and Figure H.6 in Appendix~H.3}

Under the assumption of the consistency of the nuisance parameter estimators, we can now show that~\eqref{eqn:betarobpi-consistent} is a consistent estimator of the worst-case robust predictor.

\begin{proposition}\label{prop:consistency-predictor}
  Consider the estimator  $\betarobpihat$ of the worst-case robust predictor defined in~\eqref{eqn:betarobpi-consistent}. Suppose the optimization problem is over a compact set $\cB \subseteq \R^d$ whose interior contains the true minimizer $\betarobpi$. 
  Assume that the covariance matrix $\EE[X_0X_0^\top] \succ 0$ with bounded eigenvalues and $\EE[Y_0^2] < \infty$.
  % \Nicola{Moreover, suppose that $\PSMpop \succeq 0$ with bounded eigenvalues.}
  Then, 
  under Assumption~\ref{ass:consistency-nuisance},
  $\betarobpihat$ is a consistent estimator of~$\betarobpi$.
\end{proposition}

\subsection{Proof of \Cref{prop:consistency-predictor}}

    For ease of notation define $\beta_0 \coloneqq \betarobpi$ and $\hat{\beta} \coloneqq \betarobpihat$.
    For any parameter of interest $\beta \in \cB$ and nuisance parameters $\varphi = (P_S, P_R, b)$,
   define the function 
  \begin{align}\label{eq:g-func}
      (x, y) \mapsto g_{\beta,\varphi}(x, y) 
      \coloneqq (y - \beta^\top x)^2 
      + \gamma \norm{P_S^{1/2}(b - \beta)}_2^2
      + \gamma \left(\sqrt{C - \norm{b}_2^2}  + \norm{P_R \beta}_2\right)^2.
  \end{align}
  Using~\eqref{eq:g-func}, the robust identifiable risk and its sample version defined in~\eqref{eqn:rob-loss-ell-sample} can be written, respectively as
  \begin{align*}
      \LL(\beta, \varphi) = \EE[g_{\beta, \varphi}(X_0, Y_0)],
      \quad
      \LL_n(\beta, \varphi) = \frac{1}{n_0}\sum_{i \in \cD_0} g_{\beta, \varphi}(X_{0,i}, Y_{0,i}).
  \end{align*}
  % can be written as 
  % Similarly, the sample robust identifiable risk defined  can be written as 
    % For any $\beta_1, \beta_2 \in \cB$, define $d(\beta_1, \beta_2) \coloneqq \norm{\beta_1 - \beta_2}_2$.
    % By slight abuse of notation, for any nuisance vector $\nu_1, \nu_2 \in \R^d \times \R^{d \times k}$ also define $d(\nu_1, \nu_2) = \norm{b_1 - b_2}_2 + \norm{S_1 - S_2}_2$.
    Our goal is to show that $\hat{\beta} \stackrel{\prob}{\to}\beta_0$. First, we show that the minimum of the loss is well-separated.

    \begin{lemma}\label{lm:well-separation}
    Suppose that $\EE[X_0X_0^\top] \succ 0$.
    Then, for all $\delta > 0$, it holds that
\begin{align}\label{eqn:well-sep}
    \inf\left\{\LL(\beta, \varphi_0) \colon \norm{\beta - \beta_0}_2 > \delta \right\} > \LL(\beta_0, \varphi_0).
\end{align}
\end{lemma}
    
   Fix $\delta > 0$.  From the well-separation of the minimum from Lemma~\ref{lm:well-separation}, there exists $\varepsilon > 0$ such that 
    \begin{align*}
        \left\{\norm{\hat\beta - \beta_0}_2 > \delta\right\} \subseteq
        \left\{\LL(\hat\beta, \varphi_0)- \LL(\beta_0, \varphi_0) > \varepsilon\right\}.
    \end{align*}
    Therefore,
    \begin{align}
        \prob&\left(\norm{\hat\beta - \beta_0}_2 > \delta \right) \leq 
        \prob\left(\LL(\hat\beta, \varphi_0)- \LL(\beta_0, \varphi_0) > \varepsilon\right) \nonumber\\
        &\quad= 
        \prob\left( 
        \LL(\hat\beta, \varphi_0)- \LL_n(\hat\beta, \varphi_0) + \LL_n(\hat\beta, \varphi_0)-
        \LL_n(\hat\beta, \hat\varphi)
        \right. \nonumber\\
        &\quad\quad\quad\quad
        \left. 
        + \LL_n(\hat\beta, \hat\varphi)
        - \LL_n(\beta_0, \hat\varphi)
        + \LL_n(\beta_0, \hat\varphi)
        -
        \LL(\beta_0, \varphi_0) > \varepsilon
        \right) \nonumber\\
        &\quad \leq
        \prob\left(
        \LL(\hat\beta, \varphi_0)- \LL_n(\hat\beta, \varphi_0) > \varepsilon/4
        \right)
        + \prob\left(
        \LL_n(\hat\beta, \varphi_0)-
        \LL_n(\hat\beta, \hat\varphi) > \varepsilon/4
        \right) \label{eq:consistency-1}\\
        &\quad\quad +
        \prob\left(
        \LL_n(\hat\beta, \hat\varphi)
        - \LL_n(\beta_0, \hat\varphi) > \varepsilon / 4
        \right)
        + \prob\left(
        \LL_n(\beta_0, \hat\varphi)
        -
        \LL(\beta_0, \varphi_0)>\varepsilon/4
        \right) \label{eq:consistency-2}.
    \end{align}
    We now want to prove convergence the four terms in ~\eqref{eq:consistency-1} and~\eqref{eq:consistency-2}. For this, we use the following statements proved in \Cref{sec:auxlemmaproofs}.

\begin{lemma}\label{lm:ulln-2}
    Suppose $\cB \subseteq \R^d$ is a compact set.
    Moreover, assume that the covariance matrix $\EE[X_0X_0^\top] \succ 0$ with bounded eigenvalues and $\EE[Y_0^2] < \infty$.
    Then, as $n, n_0 \to \infty$ it holds that 
    \begin{align}\label{eqn:ulln-2}
            \sup_{\beta \in \cB} |\LL_n(\beta, \varphi_0) - \LL(\beta, \varphi_0)| \stackrel{\prob}{\to} 0.
        \end{align}
\end{lemma}
\begin{lemma}\label{lm:lipschitz}
    As $n \to \infty$, it holds that
    \begin{align}\label{eqn:lipschitz}
            \sup_{\beta\in\mathcal{B}}|\LL_n(\beta, \hat\varphi) - \LL_n(\beta, \varphi_0)| \stackrel{\prob}{\to} 0.
        \end{align}
\end{lemma}
The two terms in~\eqref{eq:consistency-1} converge to 0 by \cref{lm:ulln-2} and \cref{lm:lipschitz}, respectively. The first term in~\eqref{eq:consistency-2} equals 0 since $\hat\beta$ minimizes $\beta \mapsto \LL_n(\beta, \hat\varphi)$. 
Finally, we observe that \begin{align}\label{eqn:ulln-1}
        \sup_{\beta \in \cB} |\LL_n(\beta, \hat\varphi) - \LL(\beta, \varphi_0)| \stackrel{\prob}{\to} 0,
        \end{align}
since we have that
    \begin{align*}
        \sup_{\beta \in \cB} |\LL_n(\beta, \hat\varphi) - \LL(\beta, \varphi_0)|
        \leq &\
        \sup_{\beta \in \cB} |\LL_n(\beta, \hat\varphi) - \LL_n(\beta, \varphi_0)|
        +
        \sup_{\beta \in \cB} |\LL_n(\beta, \varphi_0) - \LL(\beta, \varphi_0)|,
    \end{align*}
    where the first term converges in probability by Lemma~\ref{lm:lipschitz}, and the second term converges in probability by Lemma~\ref{lm:ulln-2}. This implies that the second term in~\eqref{eq:consistency-2} converges to zero. Since $\delta > 0$ was arbitrary, it follows that $\hat{\beta} \stackrel{\prob}{\to}\beta_0$.
    

\subsection{Proof of auxiliary lemmas}
\label{sec:auxlemmaproofs}
\subsubsection{Proof of \Cref{lm:well-separation}}
By definition,
\begin{align*}
    \LL(\beta, \varphi_0) = \EE[(Y_0 - \beta^\top X_0)^2]
    + \gamma \norm{\PSMpop^{1/2}(\betastarS - \beta)}_2^2
      + \gamma \left(\sqrt{C - \norm{\betastarS}_2^2}  + \norm{\PSperpMpop \beta}_2\right)^2.
\end{align*}
Since $\EE[X_0X_0^\top] \succ 0$, the first term is strongly convex in $\beta$.
Moreover, the second and third terms are convex in $\beta$. Therefore, $\LL(\beta, \varphi_0)$ is strongly convex in $\beta$. Since $\LL(\beta, \varphi_0)$ is also continuous in $\beta$, it follows that there exists a unique global minimum. Let $\beta_0$ denote the global minimizer of $\LL(\beta, \varphi_0)$.
% Fix $\delta > 0$ and define $A_\delta \coloneqq \{\LL(\beta, \varphi_0) \colon \norm{\beta - \beta_0}_2 > \delta\}$. We want to show that $\inf A_\delta > \LL(\beta_0, \varphi_0)$.
By the fact that $\LL(\beta_0, \varphi_0)$ is a global  minimum, and by definition of strong convexity, there exists a positive constant $m > 0$ such that, for all $\beta \in \cB$,
\begin{align}\label{eq:strong-conv}
    \LL(\beta, \varphi_0) \geq \LL(\beta_0, \varphi_0) + \frac{m}{2} \norm{\beta - \beta_0}_2^2.
\end{align}
Fix $\delta > 0$.  
Then, by~\eqref{eq:strong-conv},  for all $\beta \in \cB$ such that $\norm{\beta-\beta_0}_2 > \delta$ it holds that
\begin{align*}
    \LL(\beta, \varphi_0) \geq \LL(\beta_0, \varphi_0) + \frac{m\delta^2}{2} > \LL(\beta_0, \varphi_0). 
\end{align*}
Since the inequality holds for all $\beta \in \cB$ such that $\norm{\beta-\beta_0}_2 > \delta$, we conclude that
\begin{align*}
    \inf \{\LL(\beta, \varphi_0)  \colon \norm{\beta-\beta_0}_2 > \delta\} > \LL(\beta_0, \varphi_0).
\end{align*}
Since $\delta > 0$ was arbitrary, the claim follows.

\subsubsection{Proof of \Cref{lm:ulln-2}}
    Recall that for any $\beta \in \cB$
  \begin{align*}
      \LL(\beta, \varphi_0) = \EE[g_{\beta, \varphi_0}(X_0, Y_0)],
      \quad
      \LL_n(\beta, \varphi_0) = \frac{1}{n_0}\sum_{i \in \cD_0} g_{\beta, \varphi_0}(X_{0,i}, Y_{0,i}).
  \end{align*}
    To show the result, we must establish that the class of functions $\{g_{\beta, \varphi_0} \colon \beta \in \cB\}$ is Glivenko--Cantelli. From~\cite{van2000asymptotic}, a set of sufficient conditions for being a Glivenko--Cantelli class is that (i) $\cB$ is compact, (ii) $\beta \mapsto g_{\beta, \varphi_0}(x, y)$ is continuous for every $(x, y)$, and (iii) $\beta \mapsto g_{\beta, \varphi_0}$ is dominated by an integrable function.
    By assumption, (i) holds.
    Moreover, by~\eqref{eq:g-func}, it follows that $\beta \mapsto g_{\beta, \varphi_0}$ is continuous for all $(x, y)$ and thus (ii) holds. We now show that (iii) holds. 
    Since $\cB$ is compact we have that $\sup_{\beta\in\cB} \norm{\beta}_2 = C_1 < \infty$.
    For fixed $\gamma > 0$, and all $(x, y)$, we have that
    \begin{align}
    \label{eq:dominates-i}
    \begin{split}
        g_{\beta, \varphi_0}(x, y) 
        \leq &\ \sup_{\beta \in \cB} |g_{\beta, \varphi_0}(x, y)|\\
        \leq &\ \sup_{\beta \in \cB} (y - \beta^\top x)^2
        +
        2\gamma  \norm{\PSMpop^{1/2}}_F^2 \left(\norm{\betastarS}_2^2 + \sup_{\beta \in \cB} \norm{\beta}_2^2\right)\\
        &+   \gamma \left(\sqrt{C - \norm{\betastarS}_2^2} 
        + \norm{\PSperpMpop}_F \sup_{\beta\in\cB}\norm{\beta}_2\right)^2\\
        \leq &\
        2y^2 + 2C_1^2 \norm{x}_2^2 + K \eqqcolon G(x, y),
    \end{split}
    \end{align} 
    where $K < \infty$ is a finite constant not depending on $(x, y)$.
    Furthermore, we have that
    \begin{align}\label{eq:dominates-ii}
        \EE[G(X_0, Y_0)] = 2 \EE[Y_0^2] + 2C_1^2\ \trace (\EE[X_0X_0^\top]) + K < \infty,
    \end{align}
    since $\EE[Y^2] < \infty$ and $\EE[X_0X_0^\top]$has bounded eigenvalues by assumption. From~\eqref{eq:dominates-i} and~\eqref{eq:dominates-ii}, it follows that (iii) holds.

\subsubsection{Proof of \Cref{lm:lipschitz}}
For fixed $\gamma > 0$, we have that
\begin{align}
    \frac{1}{\gamma} & \sup_{\beta \in \cB}|\LL_n(\beta, \hat\varphi) - \LL_n(\beta, \varphi_0)|
    \leq
    \sup_{\beta \in \cB}\left|
    \norm{\PSM^{1/2}(\hat\beta^{\cS} - \beta)}_2^2
    - \norm{\PSMpop^{1/2}(\betastarS - \beta)}_2^2
    \right| \label{eq:lip-step-1}
    \\
    & + 
    \sup_{\beta \in \cB}\left |\left(\sqrt{C - \norm{\hat{\beta}^{\cS}}_2^2}  + \norm{\PSperpM \beta}_2\right)^2 
    - \left(\sqrt{C - \norm{\betastarS}_2^2}  + \norm{\PSperpMpop \beta}_2\right)^2\right|.
    \label{eq:lip-step-2}
\end{align}
We first show that ~\eqref{eq:lip-step-1} converges in probability to 0. 
\begin{align}
    &\sup_{\beta \in \cB}\left|
    \norm{\PSM^{1/2}(\hat{\beta}^{\cS} - \beta)}_2^2
    - \norm{\PSMpop^{1/2}(\betastarS - \beta)}_2^2
    \right| \notag\\
    = &\ 
    \sup_{\beta \in \cB}
    \left|(\hat{\beta}^{\cS} - \beta)^\top \PSM(\hat{\beta}^{\cS} - \beta) 
    - (\betastarS - \beta)^\top \PSMpop (\betastarS - \beta) \right| \notag\\
    = &\
    \sup_{\beta \in \cB}\left|(\hat{\beta}^{\cS} - \beta)^\top \PSM (\hat{\beta}^{\cS} - \betastarS)
    + (\hat{\beta}^{\cS} - \betastarS)^\top \PSM(\betastarS - \beta) \right. \notag \\
    & \left. \quad\quad + (\betastarS - \beta)^\top (\PSM - \PSMpop) (\betastarS - \beta) \right| \notag\\
    \stackrel{\clubsuit}{\leq}&\ 
     \sup_{\beta \in \cB} \norm{\hat{\beta}^{\cS} - \beta}_2\ \norm{\PSM}_F\ \norm{\hat{\beta}^{\cS} - \betastarS}_2
     + \sup_{\beta \in \cB} \norm{\betastarS - \beta}_2\ \norm{\PSM}_F\ \norm{\hat{\beta}^{\cS} - \betastarS}_2 \notag \\
     & \left. \quad\quad +
     \sup_{\beta \in \cB} \norm{\betastarS - \beta}_2^2\ \norm{\PSM - \PSMpop}_F\right. \label{eq:bound-1-1},
\end{align}
where $\clubsuit$ follows from the Cauchy--Schwarz inequality and from the fact that $\norm{A}_2 \leq \norm{A}_F$.
For any $\delta_1, \delta_2 > 0$, define the event
\begin{align*}
    A_n \coloneqq \left\{\norm{\betaShat - \betastarS} \leq \delta_1, \norm{\PSM - \PSMpop}_F \leq \delta_2\right\},
\end{align*}
and note that $\prob(A_n) \to 1$ as $n \to \infty$ from Assumption~\ref{ass:consistency-nuisance}.
On the event $A_n$, it holds
\begin{align*}
    \sup_{\beta \in \cB} \norm{\hat{\beta}^{\cS} - \beta}_2 \leq \norm{\hat{\beta}^{\cS} - \betastarS}_2 + \sup_{\beta \in \cB}  \norm{\betastarS - \beta}_2 \leq \delta_1 + C_1,\\
    \norm{\PSM}_F \leq \norm{\PSM - \PSMpop}_F + \norm{\PSMpop}_F \leq \delta_2 +  C_2,   
\end{align*}
where $C_1 < \infty$ follows from the compactness of~$\cB$ and $C_2$ follows from the fact that  $\PSMpop$ has bounded eigenvalues.
Therefore, on the event $A_n$, we can upper bound~\eqref{eq:bound-1-1} by
\begin{align}
    (\delta_1 + C_1) (\delta_2 + C_2)\ \norm{\hat{\beta}^{\cS} - \betastarS}_2 + (\delta_1 + C_1)\ \norm{\PSM - \PSMpop}_F.\label{eq:bound-1-2} 
\end{align}
From Assumption~\ref{ass:consistency-nuisance}, \eqref{eq:bound-1-2} converges to 0 in probability, and therefore,~\eqref{eq:lip-step-1} converges to 0 in probability as well.

Now, we can upper bound~\eqref{eq:lip-step-2} as follows,
\begin{align}
    \sup_{\beta \in \cB} &\left |\left(\sqrt{C - \norm{\hat{\beta}^{\cS}}_2^2}  + \norm{\PSperpM \beta}_2\right)^2 
    - \left(\sqrt{C - \norm{\betastarS}_2^2}  + \norm{\PSperpMpop \beta}_2\right)^2\right| \notag\\
    = &\ \sup_{\beta \in \cB} \left|
    C - \norm{\hat{\beta}^{\cS}}_2^2
    + \norm{\PSperpM \beta}_2^2 
    + 2 \sqrt{C - \norm{\hat{\beta}^{\cS}}_2^2}\ \norm{\PSperpM \beta}_2\right.\notag\\
    &\phantom{\sup_{\beta \in \cB}\ }\left.
    - C + \norm{\betastarS}_2^2 
    - \norm{\PSperpMpop \beta}_2^2
    - 2 \sqrt{C - \norm{\betastarS}_2^2}\  \norm{\PSperpMpop \beta}_2
    \right| \notag\\
    \leq &\ \left|\norm{\hat{\beta}^{\cS}}_2^2 - \norm{\betastarS}_2^2\right|
    + \sup_{\beta \in \cB} \left|\beta^\top (\PSperpM - \PSperpMpop) \beta\right|\notag\\ 
    &+2 \sup_{\beta \in \cB}\left|\sqrt{C - \norm{\hat{\beta}^{\cS}}_2^2}\ \norm{\PSperpM \beta}_2-\sqrt{C - \norm{\betastarS}_2^2}\ \norm{\PSperpMpop \beta}_2\right|\notag\\
    = &\ (I) + (II) + (III). \notag 
\end{align}
By Assumption~\ref{ass:consistency-nuisance}, $(I)$ converges in probability to zero. 
Regarding $(II)$, we have
\begin{align*}
    \sup_{\beta \in \cB} \left|\beta^\top (\PSperpM - \PSperpMpop) \beta\right| 
    \leq 
     \sup_{\beta \in \cB} 
     \norm{\beta}_2^2\
     \norm{\PSperpM - \PSperpMpop}_F
     \stackrel{\prob}{\to}0,
\end{align*}
where the inequality follows from Cauchy--Schwarz and that $\norm{A}_2 \leq \norm{A}_F$, and the convergence in probability follows from Assumption~\ref{ass:consistency-nuisance} along with the compactness of~$\cB$.
It remains to upper bound $(III)$. We have that
\begin{align}
% \begin{split}
    \frac{(III)}{2} 
 \leq 
 &\
 \sup_{\beta \in \cB}
 \left|
 \sqrt{C - \norm{\hat{\beta}^{\cS}}_2^2}\ \norm{\PSperpM \beta}_2
  - \sqrt{C - \norm{\betastarS}_2^2}\ \norm{\PSperpM \beta}_2
 \right| \notag\\
 &+
  \sup_{\beta \in \cB}
 \left|
 \sqrt{C - \norm{\betastarS}_2^2}\ \norm{\PSperpM \beta}_2
  - \sqrt{C - \norm{\betastarS}_2^2}\ \norm{\PSperpMpop \beta}_2
 \right| \notag\\
 \leq &\
 \left(\sup_{\beta \in \cB} \norm{\beta}_2\ \norm{\PSperpM}_F\right)\
 \left|
 \sqrt{C - \norm{\hat{\beta}^{\cS}}_2^2} 
 - 
 \sqrt{C - \norm{\betastarS}_2^2}
 \right|
 \notag\\
 &+\sup_{\beta \in \cB}
 \left|
 \sqrt{\beta^\top \PSperpM  \beta}
 -
 \sqrt{\beta^\top \PSperpMpop \beta}
 \right|
 \left(
 \sqrt{C - \norm{\betastarS}_2^2}\right)\notag\\
 \leq &\
 C_3 
 \left| \norm{\betastarS}_2^2 - \norm{\hat{\beta}^{\cS}}_2^2
 \right|^{1/2}
 + 
 \sqrt{C} 
 \sup_{\beta \in \cB} 
  \left|
 \beta^\top ( \PSperpM- \PSperpMpop) \beta
  \right|^{1/2}
  \label{eq:bound-sqrt-1}\\
  \leq &\
   C_3 
 \left| \norm{\betastarS}_2^2 - \norm{\hat{\beta}^{\cS}}_2^2
 \right|^{1/2}
 + 
 \sqrt{C} 
 \left(
    \sup_{\beta \in \cB} \norm{\beta}_2^2\ \norm{ \PSperpM- \PSperpMpop}_F
 \right)^{1/2} \stackrel{\prob}{\to} 0.\label{eq:bound-sqrt-2}
 % \end{split}
\end{align}
The inequality in~\eqref{eq:bound-sqrt-1} follows from the compactness of $\cB$, the fact that $\PSperpM$ has bounded eigenvalues, and that $|\sqrt{x} - \sqrt{y}| \leq |x - y|^{1/2}$ for all $x, y \geq 0$.
The inequality in~\eqref{eq:bound-sqrt-2} follows from Cauchy--Schwarz and that $\norm{A}_2 \leq \norm{A}_F$.
The convergence in probability follows form Assumption~\ref{ass:consistency-nuisance} and the compactness of~$\cB$.



% \begin{align}\label{eqn:well-sep}
%     \inf\left\{\LL(\beta, \varphi_0) \colon \norm{\beta - \beta_0}_2 > \delta \right\} > \LL(\beta_0, \varphi_0).
% \end{align}
% Moreover, by compactness of $\cB$, continuity of $\beta \mapsto g_{\beta, \varphi_0}(x, y)$ and the fact that $\EE[\sup_{\beta \in \cB} g_{\beta, \varphi_0}(\Xref, \Yref)] < \infty$, it follows from, e.g., \cite{van2000asymptotic} that
% \begin{align}\label{eqn:GC}
%     \sup_{\beta \in \cB} |\LL_n(\beta, \varphi_0) - \LL(\beta, \varphi_0)| \stackrel{\prob}{\to} 0.
% \end{align}
% Finally, by Lemma~\ref{lm:lipschitz-type}, there exists a positive $K < \infty$ such that for all $\varphi_1, \varphi_2$ it holds    \begin{align}\label{eqn:lipschitz}
%     \sup_{\beta \in \cB} |\LL(\beta, \varphi_1) - \LL(\beta, \varphi_2)| \leq K d(\varphi_1, \varphi_2).
% \end{align}
% From~\eqref{eqn:well-sep} and~\eqref{eqn:lipschitz} it follows that for any $\delta > 0$ there exist $\varepsilon_1 > 0$ and $\varepsilon_2 > 0$ such that
% \begin{align*}
%     \prob\left(d(\hat\beta, \beta_0) > \delta\right) 
%     \leq 
%     \prob \left(\left|\LL(\hat{\beta}, \hat{\varphi}) - \LL(\beta_0, \hat{\varphi})\right| > \varepsilon_1 \right) + \prob\left(d(\hat{\varphi}, \varphi_0) > \varepsilon_2\right).
% \end{align*}
% The second term on the right-hand side vanishes by the consistency of $\hat{\varphi}$ from Proposition~\ref{prop:consistency-nuisance}. It remains to show that $\LL(\hat\beta, \hat{\varphi}) - \LL(\beta_0, \hat{\varphi})$ converges in probability to zero.
% To do so, notice that
% \begin{align}
%     \left|\LL(\hat\beta, \hat\varphi) - \LL(\beta_0, \hat\varphi)\right| 
%     \leq &\ \left|\LL(\hat\beta, \varphi_0) - \LL(\beta_0, \varphi_0)\right|
%     \label{eqn:bound-1}\\
%     &\ + \left|\LL(\beta_0, \varphi_0) - \LL(\beta_0, \hat\varphi)\right| + \left|\LL(\hat\beta, \hat\varphi) - \LL(\hat\beta, \varphi_0)\right|.
%     \label{eqn:bound-2}
% \end{align}
% The terms in~\eqref{eqn:bound-2} vanish by~\eqref{eqn:lipschitz} and the consistency of $\hat\varphi$ by Proposition~\ref{prop:consistency-nuisance}.
% It remains to show that the term on the right-hand side of~\eqref{eqn:bound-1} vanishes. We have that
% \begin{align}
%   0 \leq &\ \LL(\hat\beta, \varphi_0) - \LL(\beta_0, \varphi_0) \nonumber \\
%   = &\ 
%   [\LL(\hat\beta, \varphi_0) - \LL(\beta_0, \varphi_0)] - [\LL_n(\hat\beta, \varphi_0) - \LL_n(\beta_0, \varphi_0)] + [\LL_n(\hat\beta, \varphi_0) - \LL_n(\beta_0, \varphi_0)] \label{eqn:bound-3}\\
%   \leq &\ [\LL(\hat\beta, \varphi_0) - \LL_n(\hat\beta, \varphi_0)] - [\LL(\beta_0, \varphi_0) - \LL_n(\beta_0, \varphi_0)] \stackrel{\prob}{\to} 0, \label{eqn:bound-4}
% \end{align}
% where the last inequality holds since the last term in~\eqref{eqn:bound-3} is at most zero, and the convergence in probability in~\eqref{eqn:bound-4} follows from~\eqref{eqn:GC}.
% }

% \Nicola{
% \begin{lemma}\label{lm:lipschitz-type}
% There exists a positive $K < \infty$ such that for all $\varphi_1, \varphi_2$ it holds    \begin{align}\label{eqn:lipschitz}
%     \sup_{\beta \in \cB} |\LL(\beta, \varphi_1) - \LL(\beta, \varphi_2)| \leq K d(\varphi_1, \varphi_2).
% \end{align}
% \Nicola{define metric $d$}
% \end{lemma}
% \begin{proof}
% We have that
% \begin{align}
%     \frac{1}{\gamma} & \sup_{\beta \in \cB}|\LL(\beta, \varphi_1) - \LL(\beta, \varphi_2)|
%     \leq
%     \sup_{\beta \in \cB}\left|
%     \norm{S_1^\top(b_1 - \beta)}_2^2
%     - \norm{P_{S_0, M}(b_2 - \beta)}_2^2
%     \right| \label{eq:lip-step-1}
%     \\
%     & + 
%     \sup_{\beta \in \cB}\left |\left(\sqrt{C - \norm{b_1}_2^2}  + \norm{R_1^\top \beta}_2\right)^2 
%     - \left(\sqrt{C - \norm{b_2}_2^2}  + \norm{R_2^\top \beta}_2\right)^2\right|
%     \label{eq:lip-step-2}
% \end{align}
% We can upper bound~\eqref{eq:lip-step-1} as follows,
% \begin{align}
%     &\sup_{\beta \in \cB}\left|
%     \norm{S_1^\top(b_1 - \beta)}_2^2
%     - \norm{P_{S_0, M}(b_2 - \beta)}_2^2
%     \right| \notag\\
%     = &\ 
%     \sup_{\beta \in \cB}
%     \left|(b_1 - \beta)^\top S_1S_1^\top (b_1 - \beta) 
%     - (b_2 - \beta)^\top P_{S_0, M}^2 (b_2 - \beta) \right| \notag\\
%     = &\
%     \sup_{\beta \in \cB}\left|(b_1 - \beta)^\top S_1S_1^\top (b_1 - b_2)
%     + (b_1 - b_2)^\top S_1S_1^\top (b_2 - \beta) \right. \notag \\
%     & \left. \quad\quad + (b_2 - \beta)^\top (S_1S_1^\top - P_{S_0, M}^2) (b_2 - \beta) \right| \notag\\
%     \leq&\ 
%     2 \sup_{\beta \in \cB} \norm{b_1 - \beta}_2\ \norm{S_1S_1^\top}_F\ \norm{b_1 - b_2}_2
%     + \sup_{\beta \in \cB} \norm{b_2 - \beta}_2^2\ \norm{S_1S_1^\top - P_{S_0, M}^2}_F \label{eq:bound-1-1}\\
%     \leq &\ C_1 \norm{b_1 - b_2}_2 + C_2 \norm{S_1S_1^\top - P_{S_0, M}^2}_F, \label{eq:bound-1-2}
% \end{align}
% where~\eqref{eq:bound-1-1} follows from the Cauchy--Schwarz inequality and that $\norm{A}_2 \leq \norm{A}_F$, and~\eqref{eq:bound-1-2} follows from compactness of $\cB$. Furthermore, we can upper bound~\eqref{eq:lip-step-2} as follows,
% \begin{align}
%     ...
% \end{align}
% \Nicola{
% Use that $|\sqrt{x} - \sqrt{y}| \leq |x - y|^{1/2}$.
% }
% \end{proof}

% % \Nicola{found an issue in the proof above. A quick fix: assume the nuisance parameters are known -- then the proof shortens quite a lot.
% % Alternative: try briefly to fix the issue.
% % }
% % \begin{proposition}
% % % Let $\Ecaltrain = \{0, \dots, K\}$ and denote by $e = 0$ the reference environment. For each environment $e \in \Ecaltrain$, define the samples $\cD_e \coloneqq \{(X_i, Y_i)\}_{i = 1}^{n_e}$ and $\cD \coloneqq \cup_{e \in \Ecaltrain} \cD_e$. Partition the sample as $\cD = \cD_\beta \cup \cD_\varphi$, where $\cD_\beta \subseteq \cD_0$ is a random subset of the reference sample used to estimate $\beta$ and $\cD_\varphi \coloneqq \cD \setminus \cD_\beta$.
% % For fixed constants $C > 0$ and $\gamma > 0$, 
% % define the robust identifiable risk for all parameters of interest $\beta \in \R^d$ and nuisance parameters $\varphi \coloneqq (P_S, P_R, b)$ as
% % \begin{align}\label{eqn:rob-loss-ell}
% %     \LL(\beta, \varphi)  \coloneqq 
% %     \EE_{\prob_0}\left[g_{\beta, \varphi}(X, Y)\right],
% % \end{align}
% % where $\prob_0$ denotes the joint distribution of $(X, Y)$ in the reference environment and the function $g_{\beta, \varphi}$ is defined as
% % \begin{align}
% %     g_{\beta,\varphi}(x, y) 
% %   \coloneqq (y - \beta^\top x)^2 
% %   + \gamma \norm{P_S^\top(b - \beta)}_2^2
% %   + \gamma \left(\sqrt{C - \norm{b}_2^2}  + \norm{P_R^\top \beta}_2\right)^2.
% % \end{align}
% % Define the true nuisance parameter $\varphi_0 \coloneqq (P_{S_0, M}, \PSperpMpop, \betastarS)$ and the worst-case robust predictor as
% % \begin{align}
% %     \betarobpi \coloneqq \argmin_{\beta \in \cB} \LL(\beta, \varphi_0),
% % \end{align}
% % where $\cB \subseteq \R^d$ is a compact set.
% % Denote by $\hat\varphi = (\hat{P}_{S,M}, \hat{P}_{R,M}, \hat{\beta}^{\cS})$ the estimated nuisance parameter obtained from the estimation sample $\cD\coloneqq \{(X_i, Y_i)\}_{i = 1}^{n}$ and define the empirical worst-case robust risk as
% % \begin{align}
% %     \LL_n(\beta, \hat\varphi) \coloneqq \frac{1}{|\cD_0 |} \sum_{i \in \cD_0}  g_{\beta, \hat\varphi}(X_i, Y_i),
% % \end{align}
% % where $\cD_0 \coloneqq \{(X_i, Y_i)\}_{i = 1}^{n_0} \subseteq \cD$ denotes the sample from the reference environment.
% % Moreover, define the estimator of the worst-case robust predictor as
% % \begin{align}
% %     \betarobpihat \coloneqq \argmin_{\beta \in \cB} \LL_n(\beta, \hat\varphi).
% % \end{align}
\section{Details on finite-sample experiments}\label{sec:apx-experiments}

In this section, we provide more details of the data generation for our synthetic finite-sample experiments as well as data processing for the real-world data experiments.
\subsection{Synthetic experiments}\label{sec:apx-synthetic-exps}

For the synthetic experiments, we generate a random SCM which satisfies our assumptions. For $d = 15$, we randomly sample the joint covariance $\Sigmastar$ of $(\eta,\xi)$, fixing its total variance and the eigenvalues. We consider 7 environments including the reference environment, and for each environment except the reference, we randomly generate mean shifts $\mue$ of fixed norm $1$. Since we have $6$ non-zero random Gaussian mean shifts, it holds a.s. that $\dim \cS = 6$. We then randomly generate an "initial guess" for $\betastar \in \R^d$ of fixed norm $C = 10$. Now, with respect to the space $\cS$ of the identifiable directions induced by the mean shifts, we choose the most "adversarial" causal parameter $\betaadv$ which is equal to $\betastar$ on $\cS$, but on $\cSperp$ has the opposite direction of the noise OLS estimator ${\noisecovxxstar}^{-1} \noisecovxystar$. We ensure that $\| \betaadv \|_2 = C$. Note that under the observed shifts, $\betastar$ and $\betaadv$ are observationally equivalent. We complete $\betaadv$ to the set $\thetaadv$ of observationally equivalent model parameters and generate the multi-environment training data according to $\thetaadv$ and the collection of mean shifts. 

For \cref{fig:synthetic-experiments} (left), we define the test shift upper bound as $\Manchor = \gamma \frac{1}{7} \sum_{e} \mu_e \mu_e^\top$. We vary $\gamma$ from $0$ to $10$, and for each $\gamma$, we compute the oracle anchor regression estimator by minimizing the discrete anchor regression loss with the correct $\gamma$. Additionally, we compute the pooled OLS estimator and the worst-case robust predictor $\betarobpi$ as described in \cref{sec:apx-empirical-estimation}. Finally, we generate test data with a Gaussian additive shift $\Atest \sim \cN(0, \Manchor)$. We evaluate the loss of $\betaOLS$, $\betaa$ and $\betarobpi$ on this test environment and include the population lower bound. 

For \cref{fig:synthetic-experiments} (right), we define the test shift upper bound as $\Mnew = \gamma \frac{1}{7} \sum_{e} \mu_e \mu_e^\top + \gammaprime R R^\top$, where $R$ is a 2-dimensional subspace of the space $\cSperp$. We fix the magnitude $\gamma$ of the ''seen'' test shift directions at $\gamma = 40$ and set vary $\gammaprime$ from $0$ to $2$ to showcase the effect of small unseen shifts compared to large identified shifts. We compute the oracle anchor regression estimator by minimizing the discrete anchor regression loss. Additionally, we compute the pooled OLS estimator and the worst-case robust predictor $\betarobpi$ as described in \cref{sec:apx-empirical-estimation}, for which we use the oracle $\gammaprime$, given $\Manchor$ and empirical estimates of the spaces $\cS$, $\cSperp$, $R$. \\
Finally, we generate test data with a Gaussian additive shift $\Atest \sim \cN(0, \Mnew)$. We evaluate the loss of $\betaOLS$, $\betaa$ and $\betarobpi$ on this test environment, plot the resulting test losses for different estimators and include the population lower bound. 

\subsection{Real-world data experiments}\label{sec:apx-real-world}

\begin{figure}[h]
    \centering
    \begin{subfigure}[b]{0.3\textwidth}
        \centering
        \includegraphics[width=\textwidth]{contents/images/training-data.pdf}
        \caption{}
        \label{fig:fig1}
    \end{subfigure}
    \hfill
    \begin{subfigure}[b]{0.3\textwidth}
        \centering
        \includegraphics[width=\textwidth]{contents/images/test-data-strength_01.pdf}
        \caption{}
        \label{fig:fig2}
    \end{subfigure}
    \hfill
    \begin{subfigure}[b]{0.3\textwidth}
        \centering
        \includegraphics[width=\textwidth]{contents/images/test-data-strength_02.pdf}
        \caption{}
        \label{fig:fig3}
    \end{subfigure}
    \caption{The figures illustrate the structure of the (a) training-time shifts and (b-c) test-time shifts for different perturbation strengths on the example of two covariates. Panel (a) shows the training data containing two environments--observational (blue) and shifted (orange) corresponding to the knockout of the gene ENSG00000089009. 
    Panels~(b) and~(c) show the training data in grey and test data from a previously unseen environment (green). 
    Panel~(b) depicts the top $10\%$ test data points closest to the training support (perturbation strength = $0.1$).
    Panel~(c) illustrates the full test data (perturbation strength = 1.0).
    }
    \label{fig:genes}
\end{figure}

We consider the K562 dataset from \cite{replogle2022mapping} and perform the preprocessing as done in \cite{chevalley2022causalbench}.
The resulting dataset consists of $n = 162,751$ single-cell observations over $d = 622$ genes collected from observational and several interventional environments. 
% We now consider the Causalbench single-cell dataset introduced by \citep{chevalley2022causalbench}.
% The dataset consists of single-cell observations of 622 genes.
The interventional environments arise by knocking down a single gene at a time using the CRISPR interference method \citep{qi2013repurposing}. Following \citep{schultheiss2024assessing}, we select only always-active genes in the observational setting, resulting in a smaller dataset of 28 genes. For each gene $j = 1, \ldots, 28$, we set $Y:= X_j$ as the target variable and select the three genes $X_{k_1}, \ldots, X_{k_3}$ most strongly correlated with $Y$ (using Lasso), resulting in a prediction problem over $Y, X_{k_1}, \ldots, X_{k_3}$.
Given this prediction problem, we construct the training and test datasets as follows. Let $\Iobs$ denote the 10,691 observations collected from the observational environment, and let $\Iint_{i}$ denote the observations collected from the interventional environment where the gene $k_i$ was knocked down. We will denote by $\Iint_{i,s}$ the $s \times 100$ percent of datapoints in $\Iint_{i}$ that are closest to the mean of gene $k_i$ in the observational environment $\Iobs$. 
For example, $\Iint_{i,0.1}$ consists of the 10\% of datapoints in $\Iint_{i}$ closest to the observational mean of gene $k_i$.
Thus, 
% $s$ acts as a corresponds to the fraction of test points closest to the observational mean
% These are the $s \times 100\%$ of datapoints with the \emph{weakest} shift compared to the observational mean
% of the gene $k_i$, and thus 
the parameter $s \in [0,1]$ acts as a proxy for the \emph{strength} of the shift. 
Denote by $\Iint_{i, s}^*$ a random sample of $\Iint_{i, s}$ of a certain size.
For each $i \in \{1, 2, 3\}$, we fit the methods on the training data  $\Dtrain_i \coloneqq \Iobs \cup \Iint_{i, 1}^*$, with $|\Iint_{i, 1}^*| = 20$. \cref{fig:genes}(a) illustrates an example of training data  $\Dtrain_i$.
Having fitted the methods on $\Dtrain_i$, we evaluate them on test datasets constructed as follows.
For each 
shift strength $s \in \{0.1, \dots, 0.9\}$ and proportion $\pi \in \{0, .33, .67, 1\}$, define the test dataset $\mathcal{D}_{\pi, s}^{\mathrm{test}}$ consisting of $\pi$ observations from $\cup_{\ell\neq i}\Iint_{\ell, s}$ and $1-\pi$ (out-of-training) observations from $\Iint_{i, s}$.
% $\Dtest_{j, s} \coloneqq \Iint_{j, s}^*$. If $j = i$, this corresponds to a shift seen during training of potentially differing strength. If $j \neq i$, the test data contains a previously unseen distribution shift. 
An example of a test dataset for different shift strengths $s$ and previously unseen directions (i.e., $\pi = 1$) is shown in \cref{fig:genes}(b-c).  
We compare our method Worst-case Rob., defined as the minimizer of the empirical worst-case robust risk \eqref{eqn:rob-loss-ell-sample}, with anchor regression \citep{rothenhausler2021anchor}, invariant causal prediction (ICP) \citep{peters2016causal},  Distributional Robustness via Invariant Gradients (DRIG) \citep{shen2023causalityoriented}, and OLS (corresponding to vanilla ERM).
We use the following parameters for Worst-case Rob.: $\gamma = 50$, $\Cker = 1.0$, and $M = \Id$. For anchor regression and DRIG, we select $\gamma = 50$. For ICP, we set the significance level for the invariance tests to $\alpha = 0.05$.

These numerical experiments are computationally light and can be run in $\approx 5$ minutes on a personal laptop.\footnote{We use a 2020 13-inch MacBook Pro with a 1.4 GHz Quad-Core Intel Core i5 processor, 8 GB of RAM, and Intel Iris Plus Graphics 645 with 1536 MB of graphics memory.}


\section{Proofs}
\label{sec:apx-proofs}
\subsection{Proof of \cref{prop:invariant-set}}\label{sec:apx-proof-invariant-set}
% \julia{update the proof to correspond to new notation}
For every environment $e \in \Ecaltrain$, we observe the first moments $\EE(X_e)$ and $\EE(Y_e)$,
and second moments $\EE(X_eX_e^\top)$, $\EE(Y_e^2)$ and $\EE(X_eY_e)$.
% $\Cov(X_e)$ of the covariates in all training environments, as well as the covariances $\Cov(X_e, Y_e)$. 
Since it holds by assumption that $\mu_0 = 0$ and $\Sigma_0 = 0$, we have that  $\EE(X_0X_0^\top) = \noisecovxxstar$, and so we can identify $\noisecovxxstar$ uniquely. Furthermore, it holds that
\begin{align}
    \EE(X_0Y_0) &= \noisecovxxstar \betastar + \noisecovxystar, \label{eqn:ident-1}\\
    \EE(X_eY_e) &= ( \Sigma_e + \mu_e \mu_e^\top + \noisecovxxstar) \betastar + \noisecovxystar.
    \label{eqn:ident-2}
\end{align}
By taking the difference between \cref{eqn:ident-2} and \cref{eqn:ident-1},
we can identify $(\Sigma_e + \mu_e \mu_e^\top) \betastar$.
Thus, 
% In other words,
the parameter $\betastar$ is identifiable on the subspace $\cS$ defined in \cref{eqn:def-S} and is not identifiable on its orthogonal complement $\cSperp$.
% the unions of the spans of $\Sigma_e$. Since it holds that $\range\ M = \cup_e \range\ \Sigma_e$, the causal parameter is identified on $\range\ M$ and non-identified on $\ker M = \Mperp$. 
Thus, for any vector $\alpha \in \cSperp$ , the vector $\beta = \betastar + \alpha$ is consistent with the data-generating process. It remains to compute the covariance parameters induced by an arbitrary $\tilde\beta \coloneqq \betastar + \alpha$, for $\alpha \in \cSperp$. For every environment $e \in \Ecaltrain$,  the second mixed moment between $X_e$ and $Y_e$ has to satisfy the following equality
\begin{align*}
    \EE(X_eY_e) = (\Sigma_e + \mu_e\mu_e^\top + \noisecovxxstar)\betastar + \noisecovxystar = (\Sigma_e + \mu_e\mu_e^\top + \noisecovxxstar) \tilde{\beta}+ \tilde{\Sigma}_{\eta, \xi},
\end{align*}
from which it follows that $\tilde{\Sigma}_{\eta, \xi}
 \coloneqq \noisecovxystar - \noisecovxxstar \alpha$. By computing $\EE(Y_e^2)$ and inserting $\tilde{\beta} = \betastar + \alpha$ and $\tilde{\Sigma}_{\eta, \xi}$, we similarly obtain 
\begin{align*}
    \tilde{\sigma}_{\xi}^{2} \coloneqq \noisecovyystar - 2 \alpha^\top \noisecovxystar + \alpha^\top \noisecovxxstar \alpha. 
\end{align*}
Thus, we obtain the following set of observationally equivalent model parameters consistent with $\probtrainarg{\thetastar}$:
\begin{align*}
    \Invset = \{ \betastar + \alpha, \noisecovxxstar, \noisecovxystar - \noisecovxxstar \alpha, \noisecovyystar - 2 \alpha^\top \noisecovxystar + \alpha^\top \noisecovxxstar \alpha \colon \alpha \in \cSperp \}. 
\end{align*}
Since the \idset is identifiable from the training distribution, but model parameters $\betastar$, $\noisecovxystar$, $\noisecovyystar$ are not, it is helpful to re-express the \idset through identifiable quantities. For this, we note that the "identifiable linear predictor" $\betastarS = \betastar - \betastarperp$ induces an observationally equivalent model given by 
% $(\betastarperp, \noisecovxxstar, \noisecovxystar + \noisecovxxstar \betastarker, \noisecovyystar + 2 \betastarker^\top \noisecovxystar + \betastarker^\top \noisecovxxstar \betastarker)$. 
\begin{align*}
    \thetastarS := (\betaS, \noisecovxxS, \noisecovxyS, \noisecovyyS) = (\betastarS, \noisecovxxstar, \noisecovxystar + \noisecovxxstar \betastarperp, \noisecovyystar + 2 \langle \noisecovxystar, \betastarperp\rangle + \langle \betastarperp, \noisecovxxstar \betastarperp\rangle).
\end{align*}
From this reparameterization, we infer the final form of the \idset:
\begin{align*}
   \Invset = \{ \betastarS + \alpha, \noisecovxx', \noisecovxyS - \noisecovxx' \alpha, \noisecovyyS - 2 \alpha^\top \noisecovxyS + \alpha^\top \noisecovxx' \alpha \colon \alpha \in \cSperp \}  \ni \thetastar 
\end{align*}
Therefore, \cref{eqn:def-invariant-set} follows.
To find the robust predictor $\betarob$, we write down the robust loss with respect to $\Mtest$ and any $\theta_\alpha$ from the \idset:
\begin{align*}
    \Lossrob(\beta;\theta_\alpha, \Mtest) &= (\betastarS + \alpha - \beta)^\top (\Mtest + \noisecovxxstar) (\betastarS + \alpha - \beta) \\ &+ 2 (\betastarS + \alpha - \beta)^\top (\noisecovxystar - \noisecovxxstar \alpha) + \noisecovyyS - 2 \alpha^\top \noisecovxyS + \alpha^\top \noisecovxxstar \alpha.
\end{align*}
inserting $\alpha \in \cSperp$ and rearranging, \cref{eqn:def-rob-pred-identif} follows.
% Denoting the latter two quantities by $\noisecovxyS$, $\noisecovyyS$ and reparameterizing we obtain the claim. 

\subsection{Proof of \cref{thm:pi-loss-lower-bound}}\label{sec:apx-proof-of-main-prop}

We structure the proof as follows: first, we quantify the non-identifiability of the robust risk by explicitly computing its supremum over the \idset of the model parameters (referred to as the \idRR). Second, we derive a lower bound for the \idRRs by considering two cases depending on how a predictor $\betabar$ interacts with the possible test shifts $\Mtest$. 
% In this proof, we use more general notation, with the test shifts bounded by a PSD matrix $\Mtest \preceq \gamma M + \gammaprime R R^\top$, which $\range M \subset \cS$ and $\range R \subset \cSperp$. The statement of the theorem follows by setting $\gamma = \gammaprime$. However, we believe that the more refined statement is useful, e.g., when one expects strong shifts in training directions and only weak "new" shifts.
\paragraph{Computation of the \idRR.} For any model-generating parameter $\theta = (\beta, \Sigma)$ it holds that the robust risk of the model \cref{eqn:SCM} under test shifts $\Mtest \succeq 0$ is given by 
\begin{align*}
%\label{eqn:apx-def-robust-loss}
    \Lossrob(\betabar;\theta, \Mtest) =  (\beta - \betabar)^\top(\Mtest + \noisecovxxstar)(\beta - \betabar) + 2(\beta - \betabar)^\top \noisecovxy + \noisecovyy. 
\end{align*}
We recall that the \idset of model parameters after observing the multi-environment training data \cref{eqn:SCM} is given by 
\begin{align}\label{eqn:apx-def-invariant-set}
     \Invset = \{ \betastarS + \alpha, \noisecovxxstar, \noisecovxyS - \noisecovxxstar \alpha, \noisecovyyS - 2 \alpha^\top \noisecovxyS + \alpha^\top \noisecovxx \alpha: \alpha \in \cSperp \},
\end{align}
where $\cS$ is the span of identified directions defined in \cref{eqn:def-S}. 
Moreover, we recall that by Assumption~\ref{as:bounded-betastar}, for any causal parameter $\beta$ it should hold that $\| \beta \|_2 = \| \betastarS + \alpha \|_2 \leq C$, which translates into the following constraint for the parameter $\alpha$:
\begin{align*}
    \| \alpha \|_2 \leq \sqrt{C^2 - \| \betastarS \|_2^2} =: \Cker. 
\end{align*}
Inserting \cref{eqn:apx-def-invariant-set} in \cref{eqn:PI-robust-loss}, we obtain
\begin{align*}
    \Lossrobpi(\betabar; \Invset, \Mtest) = \supalpha \Lossrob(\betabar; \theta_{\alpha}, \Mtest),
\end{align*}
where $\theta_\alpha$ is a short notation for $(\betastarS + \alpha, \noisecovxxstar, \noisecovxyS - \noisecovxxstar \alpha, \noisecovyyS - 2 \alpha^\top \noisecovxyS + \alpha^\top \noisecovxxstar \alpha)$. We now compute the supremum explicitly in case $\Mtest$ has the form $\Mtest = \gamma \Mseen + \gammaprime R R^\top$, where $\Mseen$ is a PSD matrix with $\range(M) \subseteq \cS$ and $R$ is a semi-orthogonal matrix with $\range(R) \subseteq \cSperp$. For any $\alpha \in \cSperp$, we write down the robust loss as
\begin{align*}
    \Lossrob(\betabar; \theta_\alpha, \Mtest) &= (\betastarS - \betabar)^\top (\Mtest + \noisecovxxstar) (\betastarS - \betabar) + 2 (\betastarS - \betabar)^\top \noisecovxyS + \noisecovyyS \\
    &+ \alpha^\top \Mtest \alpha + 2 \alpha^\top \Mtest(\betastarS -  \betabar ) \\
    &= \Lossrob(\betabar; \thetastarS, \Mtest) + \alpha^\top \Mtest \alpha + 2 \alpha^\top \Mtest(\betastarS -  \betabar ). 
\end{align*}
The first term is the robust risk of $\betabar$ under test shift $\Mtest$ and the identified model-generating parameter $\thetastarS$, thus it does not depend on $\alpha$. 
%We recall that $\Mtest = \gamma \cP_\cM = \gamma (S S^\top + R R^\top)$, where $\range\ S \subset \cS$ and $\range\ R \subset \cSperp$. 
By the structure of $\Mtest$, we obtain that 
\begin{align*}
    f(\alpha) := \alpha^\top \Mtest \alpha + 2 \alpha^\top \Mtest(\betastarS -  \betabar )  = \gammaprime \alpha^\top R R^\top \alpha - \gammaprime \alpha^\top R R^\top \betabar. 
\end{align*}
If $\gammaprime = 0$, i.e., the test shifts consist only of the identified directions, we have $f(\alpha) = 0$, independently of $\alpha$, and thus 
\begin{align*}
     \Lossrobpi(\betabar; \Invset, \Mtest) = \Lossrob(\betabar; \thetastarS, \Mtest).
\end{align*}
This implies the first statement of the theorem. 
\par
We now consider the case where $R \neq 0$, i.e., $R R^\top$ is a non-degenerate projection.
Our goal is to maximize $f(\alpha)$ subject to constraints $\alpha \in \cSperp$, $\| \alpha \|_2 \leq \Cker$. Let $\Rtilde$ be an orthonormal extension of $R$ such that $\range\ (R | \Rtilde) = \cSperp$. Then, we can parameterize $\alpha \in \cSperp$ as $\alpha = (R | \Rtilde) (\frac{w} {\wtilde})$ and the corresponding Lagrangian reads
\begin{align*}
    \mathcal{L}(\alpha, \lambda) &= \gammaprime \alpha^\top R R^\top \alpha - \gammaprime \alpha^\top R R^\top \betabar + \lambda(\Cker^2 - \| \alpha \|_2^2) \\ &= \gammaprime \| w \|_2^2 -\gammaprime w^\top R^\top \betabar + \lambda(\Cker^2 - \| (w, \wtilde) \|_2^2). 
\end{align*}
Differentiating with respect to $w, \wtilde$ yields
\begin{align*}
    w &= \frac{\gammaprime}{\gammaprime - \lambda} R^\top \betabar; \\
    \wtilde &= 0. 
\end{align*}
After differentiating w.r.t. $\lambda$, we obtain
$\frac{\gammaprime}{\gammaprime - \lambda} = \pm \frac{\Cker}{\| R^\top \betabar \|_2}$. By inserting in the objective function and comparing, we obtain the \textbf{value of the \idRR}: 
\begin{align}\label{eqn:proofs-id-robust-risk}
    \Lossrobpi(\betabar; \Invset, \Mtest) &= \gammaprime \Cker^2 + 2 \gammaprime \| R^\top \betabar \|_2 + \Lossrob(\betabar; \thetastarS, \Mtest) \\
    &= \gammaprime \Cker^2 + 2 \gammaprime \| R^\top \betabar \|_2 + \betabar^\top R R^\top \betabar + \gamma (\betastarS - \beta)^\top \Mseen (\betastarS - \beta) + \Loss_0 (\betabar,\thetastarS).
\end{align}
Putting together the two cases and simplifying, we obtain
\begin{align}\label{eqn:detailed-id-robust-risk}
\begin{split}
    \Lossrobpi(\betabar; \Invset, \Mtest) &= \gammaprime(\Cker + \| R^\top \betabar \|_2)^2 + \Lossrob(\betabar; \thetastarS, \gamma \Mseen) \\  &= \gammaprime  (\Cker + \| R^\top \betabar \|_2)^2 + \gamma (\betastarS - \betabar)^\top \Mseen (\betastarS-\betabar) + \Loss_0 (\betabar,\thetastarS), 
\end{split}
\end{align}
where $\Lossrob(\betabar; \thetastarS, \gamma \Mseen)$ is the robust risk of the estimator $\betabar$ w.r.t. the "identified" test shift $\gamma M$ and the identified model parameter $\thetastarS$, whereas $\Loss_0 (\betabar,\thetastarS)$ is the risk of $\betabar$ on the reference environment $e = 0$. 
\paragraph{Derivation of the lower bound for the \idRR.} Now that we have explicitly computed the \idRR, we devote ourselves to the computation of the lower bound for its best possible value
\begin{align*}
    \inf_{\betabar \in \R^d} \Lossrobpi(\betabar; \Invset, \Mtest). 
\end{align*}
In this part, we will only consider the case $R \neq 0$, since the case $R = 0$ corresponds to the (discrete) anchor regression-like setting, where both the robust risk and its minimizer are uniquely identifiable, and computable from training data. We will distinguish between two cases.
\paragraph{Case 1: $\| R^\top \betabar \|_2 = 0$.}  In this case, $\betabar$ is fully located in the orthogonal complement of $R$, which consists of $\cS$ and $\Rtilde$ (the orthogonal complement or $R$ in $\cSperp$). We will denote (the basis of) this subspace by $\Stot = \cS \oplus \Rtilde$. Thus, $\Stot$ is the "total" stable subspace consisting of identified directions in $\cS$ and non-identified, but unperturbed directions $\Rtilde$. We will parameterize $\betabar$ as $\betabar = \Stot w$. Thus, we are looking to solve the optimization problem 
\begin{align*}
   \betarobpi =  \argmin_{w} \, (\betastarS - \Stot w)^\top (\gamma \Mseen^\top + \noisecovxxstar) (\betastarS -  \Stot w) + 2 (\betastarS -  \Stot w)^\top \noisecovxyS + \noisecovyyS.
\end{align*}
Setting the gradient to zero yields the \emph{asymptotic worst-case robust estimator} 
\begin{equation}\label{eq:apx-pi-robust-formula}
\begin{aligned}
        \betarobpi &= \betastarS + \Stot [ \Stot^\top (\gamma \Mseen^\top + \noisecovxx) \Stot ]^{-1} \Stot^\top \noisecovxyS,
\end{aligned}
\end{equation}
which corresponds to the loss value of 
\begin{align*}
    \Lossrobpi(\betarobpi; \Invset, \Mtest) = \gammaprime \Cker^2 +  \noisecovyyS - 2 {\noisecovxyS}^\top \Stot [ \Stot^\top (\gamma \Mseen^\top + \noisecovxx) \Stot ]^{-1} \Stot^\top \noisecovxyS.
\end{align*}
As we observe, this quantity grows linearly in $\gammaprime$. However, as $\gamma \to \infty$, the quantity \emph{saturates} and is upper-bounded by $\noisecovyyS$.
\paragraph{Case 2:$\| R^\top \betabar \|_2 \neq 0$.} Since for $\| R^\top \betabar \|_2 \neq 0$, the objective function is differentiable, we compute its gradient to be
\begin{align*}
    \nabla  \Lossrobpi(\beta; \Invset, \Mtest) &= 2 \gammaprime R R^\top \beta / \| R R^\top \beta \| + 2 \gammaprime R R^\top \beta + \nabla \Lossrob(\beta; \thetastarS, \gamma \Mseen) \\ 
    &= 2 \gammaprime R R^\top \beta / \| R R^\top \beta \| + 2 \gammaprime R R^\top \beta  + 2(\noisecovxxstar + \gamma \Mseen) (\beta - \betastarS) - 2 \noisecovxyS. 
\end{align*}
This equation is, in general, not solvable w.r.t. $\beta$ in closed form. Instead, we provide the limit of the optimal value of the function when the strength of the unseen shifts is small, i.e. $\gammaprime \to 0$. We know that for $\gammaprime = 0$, the minimizer of the worst-case robust risk is given by the anchor estimator
\begin{align*}
    \betaa = \betastarS + (\noisecovxxstar + \gamma \Mseen)^{-1} \noisecovxyS. 
\end{align*}
Instead, we lower bound the non-differentiable term $2 \gammaprime \Cker \| R^\top \beta \|$ by the scalar product $2 \gammaprime \Cker \scalar{R^\top \beta}{R^\top \betaa}/ \| \betaa \|$ and expect it to be tight for small $\gammaprime$. After inserting this lower bound in \cref{eqn:proofs-id-robust-risk} we obtain the minimizer of the lower bound of form
\begin{align*}
    \beta_{LB} = \betastarS + (\noisecovxxstar + \gamma M + \gammaprime R R^\top)^{-1}(\noisecovxyS - \gammaprime \Cker R R^\top (\noisecovxxstar + \gamma M)^{-1} \noisecovxyS).
\end{align*}
We can now lower bound $\| R R^\top \beta_{LB} \|$ as 
\begin{equation}\label{eqn:small-gammaprime-lower-bound}
    \| R R^\top \beta_{LB} \| \geq \| R R^\top (\noisecovxxstar + \gamma M)^{-1} \noisecovxystar \| - \gammaprime \cdot \text{const}.
\end{equation}
Thus, the $\gammaprime$-rate of the \idRRs of $\beta_{LB}$ is at least $\gammaprime (\Cker + \| R R^\top (\noisecovxxstar + \gamma M)^{-1} \noisecovxystar \|)^2 + \mathcal{O}(\gammaprime^2)$,
from which the claim for small $\gammaprime$ follows. For \cref{sec:comp-with-finite-robustness-methods}, the lower bound directly implies optimality of the worst-case robust risk of the anchor estimator when the strength of the unseen shifts $\gammaprime$ is small. Additionally. if $\gamma = 0$, i.e. only unseen test shifts occur, we conclude that the OLS and anchor estimators have the same rates. 
\paragraph{Lower bound $\gammath$ for $\gammaprime$.}
Finally, we want to derive a lower bound on the shift strength  $\gammaprime$ such that for all $\gammaprime \geq \gammath$ Case 1 of our proof is valid, i.e. it holds that $\betarobpi$ is given by the closed form "abstaining" estimator \eqref{eq:apx-pi-robust-formula}. For this, we find $\gammath$ such that for all $\gammaprime \geq \gammath$ zero is contained in the subdifferential of$\Lossrobpi(\betarobpi;\Invset,\Mtest)$ at $\betarobpi$. Then the KKT conditions are met, and $\betarobpi$ is the unique minimizer of the worst-case robust risk due to strong convexity of the objective. We compute the subdifferential to be
\begin{align*}
    S = \gammaprime \Cker \{ R R^\top \beta: \| \beta \|_2 \leq 1 \} + \nabla \Lossrob(\betarobpi; \thetastarS, \gamma M).
\end{align*}
Since $\betarobpi$ is the minimizer of $ \Lossrob(\beta; \thetastarS, \gamma \Mseen)$ under the constraint $R^\top \beta = 0$, the gradient is zero in $R^\perp$ and it remains to show that 
\begin{align*}
    \| R R^\top \nabla \Lossrob(\betarobpi; \thetastarS, \gamma \Mseen) \| \leq \gammaprime \Cker,
\end{align*}
or 
\begin{align*}
    \gammaprime \geq \| R R^\top \nabla \Lossrob(\betarobpi; \thetastarS, \gamma \Mseen) \| / \Cker. 
\end{align*}
Via an upper bound on the projected gradient, we derive the stricter condition
\begin{align*}
    \gammaprime \geq \frac{\| R R^\top \noisecovxyS\| (1 + \kappa(\noisecovxxstar)) }{\Cker},
\end{align*}
where $\kappa(\noisecovxxstar)$ is the condition number of the covariance matrix. 

\subsection{Proof of \cref{cor:estimators}}\label{sec:apx-proof-of-corollary}
To obtain a new formulation for the \idRR, we start with \eqref{eqn:detailed-id-robust-risk} and expand 
\begin{align}
\begin{split}
    \Lossrobpi(\betabar; \Invset, \Mtest) &= \gammaprime  (\Cker + \| R^\top \betabar \|_2)^2 + \gamma (\betastarS - \betabar)^\top \Manchor (\betastarS-\betabar) + \Loss_0 (\betabar,\thetastarS) \\
    &= \gammaprime  (\Cker + \| R^\top \betabar \|_2)^2 + \gamma (\betastarS - \betabar)^\top \Manchor (\betastarS-\betabar) \\&+ (\betastarS - \betabar)^\top \noisecovxxstar (\betastarS-\betabar) + 2 (\betastarS - \beta)\noisecovxyS + \noisecovyyS \\ &= \gammaprime  (\Cker + \| R^\top \betabar \|_2)^2 + (\gamma - 1)(\betastarS - \betabar)^\top \Manchor (\betastarS-\betabar) + \Loss(\beta,\ptrain) \\
    &= \Lossrob(\beta, \thetastarS, \gamma \Manchor) + \gammaprime  (\Cker + \| R^\top \betabar \|_2)^2,
\end{split}
\end{align}
where we have used that the pooled second moment of $X$ equals to $\noisecovxxstar + \sum_e w_e (\mu_e \mu_e^\top) = \noisecovxxstar + \gamma \Manchor - (\gamma - 1) \Manchor$.
This reformulation shows that the \idRRs is equal to the anchor population loss (cf. \cite{rothenhausler2021anchor}) with an additional non-identifiability penalty term $\gammaprime  (\Cker + \| R^\top \betabar \|_2)^2$. 

We now want to evaluate the rates of the anchor and OLS estimators in terms of the magnitude $\gammaprime$ of unseen shift directions. We observe that only the non-identifiability term depends on $\gammaprime$, whereas the second term only depends on $\gamma$. First, we compute the closed-form anchor regression estimator, which reads 
\begin{equation}
    \betaa = \argmin_{\beta \in \R^d} \Lossrob(\beta, \thetastarS, \gamma \Manchor) = \betastarS + (\noisecovxxstar + \gamma \Manchor)^{-1} \noisecovxyS.
\end{equation}
Since $\betaOLS$ equals to the anchor estimator with $\gamma = 1$, we obtain 
\begin{equation*}
    \betaOLS = \betastarS + (\noisecovxxstar + \Manchor)^{-1} \noisecovxyS.
\end{equation*}
The claim of the corollary now follows by computing $\| R R^\top \betaa \|$ and $\| R R^\top \betaOLS \|$ and observing that the rest of the terms is constant in $\gammaprime$. Additionally, we observe that $\betaa$ is the minimizer of $\Lossrob(\beta, \thetastarS, \gamma \Manchor)$, and it is known (cf. e.g. \cite{rothenhausler2021anchor}) that $\Lossrob(\betaa, \thetastarS, \gamma \Manchor)$ is asymptotically constant in $\gamma$ and upper bounded by $\noisecovyyS$. On the other hand, the term $\Lossrob(\betaOLS, \thetastarS, \gamma \Manchor)$ is linear in $\gamma$. In total, we obtain 
\begin{equation*}
    \begin{aligned}
        \Lossrobpi(\betaa; \Invset, \Mtest) &= (\Cker + \| R R^\top \betaa \| )^2\gammaprime + c_1(\gamma); \\ 
        \Lossrobpi(\betaOLS; \Invset,\Mtest) &= (\Cker + \| R R^\top \betaOLS \| )^2\gammaprime + c_2(\gamma),
    \end{aligned}
\end{equation*}
where $c_1(\gamma) \leq \noisecovyyS$ and $c_2(\gamma) = \Omega(\gamma)$.
Comparing to the lower bound \eqref{eqn:small-gammaprime-lower-bound} for the minimax quantity for the case of $\gammaprime \to 0$, we observe that the anchor estimator is optimal (achieves the minimax rate) in the limit $\gammaprime \to 0$. Additionally, if $\gamma = 0$ (only new shifts occur during test time), anchor and OLS have identical rates in $\gammaprime$ and, in particular, OLS (corresponding to vanilla empirical risk minimization) is minimax-optimal in the limit of small unseen shifts. 
In the proof of \cref{thm:pi-loss-lower-bound} in \cref{sec:apx-proof-of-main-prop}, we show that for $\gammaprime \geq \gammath$, it holds that $R R^\top \betarobpi$ = 0, and thus the worst-case robust risk of the worst-case robust predictor equals 
\begin{equation*}
    \Lossrobpi(\betarobpi; \Invset, \Mtest) = \gammaprime \Cker^2 +  \noisecovyyS - o(\gamma) = \gammaprime \Cker^2 + c_3(\gamma),
\end{equation*}
where $c_3(\gamma) \leq \noisecovyyS$.
In total, we observe that the worst-case robust risk of \emph{all} considered prediction models grows linearly with the unseen shift strength $\gammaprime$, albeit with different rates. The terms $\| R R^\top \betaa \|$ and $\| R R^\top \betaOLS \|$ can be particularly large, for instance, when there is strong confounding aligned with the unseen shift directions which causes the empirical risk minimizer (OLS) to have a strong signal in these directions. The worst-case robust predictor $\betarobpi$, however, abstains in these directions, thus achieving a smaller rate.
% \begin{lemma}\label{lm:upper-bound}
    
% Let $\Mtest \coloneqq V \Lambda V^\top \succeq 0$ with 
% $V = [v_1, \dots, v_k] \in \R^{d \times k}$, $\Lambda = \diag(\lambda_1, \dots, \lambda_k)$, and define
% $\range\(\Mtest) = \cM$.
% Define $Q = (S, R) \in \R^{d \times q}$ such that $\range\(S) = \mathcal{S}$, 
% $\range\(R) \subset \cSperp$,
% and $\range\(Q) \supseteq \cM$. Furthermore define $\gamma := \max\{\lambda_j : j = 1, \dots, k\}$.
% Then, $\Mtest \preceq \gamma QQ^\top$.
% \end{lemma}
% \begin{proof}
%     Since $\range\(Q) \supseteq \cM$, we can write every eigenvector of $\Mtest$ as a linear combination of the columns of $Q$. That is, for every $j = 1, \dots, k$, there exists a vector $\tilde{v}_j \in \R^q$ such that $v_j = Q \tilde{v}_j$. 
%     Moreover, since $v_j \in \range\(Q)$, it follows that $QQ^\top v_j = v_j$, and by the orthonormality of the columns of $Q$, $\tilde{v}_j = Q^\top v_j$.
%     These two facts imply that $\norm{\tilde{v}_j}^2 = \tilde{v}_j^\top \tilde{v}_j = v_j^\top QQ^\top v_j = v_j^\top v_j = \norm{v_j}^2 = 1$.
%     Define $\tilde{V} \coloneqq [\tilde{v}_1, \dots, \tilde{v}_k] \in \R^{q \times k}$ and note that $\Mtest = V\Lambda V^\top = Q \tilde{V} \Lambda \tilde{V}^\top Q^\top$, so we can rewrite the matrix 
%     \begin{align}\label{eqn:psd-mat}
%     \gamma QQ^\top - \Mtest = Q (\gamma I_q - \tilde{V} \Lambda \tilde{V}^\top) Q.
%     \end{align}
    
%     We will now show that the matrix $\gamma QQ^\top - \Mtest \succeq 0$.
%     \begin{enumerate}[i)]
%         \item First, we establish that 
%     $\gamma I_q - \tilde{V} \Lambda \tilde{V}^\top \succeq 0$, by showing that its eigenvalues are non-negative.
%     Notice that the eigenvectors of $\gamma I_q - \tilde{V} \Lambda \tilde{V}^\top$ are $(\tilde{V}, \tilde{U}) \in \R^{q \times q}$ where $\tilde{U} \coloneqq [\tilde{u}_1, \dots, \tilde{u}_{q-k}]\in\R^{q\times (q - k)}$ with $\tilde{U}^\top \tilde{V} = 0$.
%     Let $\tilde{v}\in \R^q$ be an eigenvector of $\gamma I_q - \tilde{V} \Lambda \tilde{V}^\top$. If $\tilde{v}=\tilde{v}_j$ for some $j = 1, \dots, k$, we have that $(\gamma I_q - \tilde{V}\Lambda \tilde{V}^\top) \tilde{v} = (\gamma - \lambda_j)\tilde{v}$, which is associated to the eigenvalue $\gamma - \lambda_j \geq 0$.
%     If $\tilde{v} = \tilde{u}_j$, for some $j = 1, \dots, {q-k}$, we have that $(\gamma I_q - \tilde{V}\Lambda \tilde{V}^\top) \tilde{v} = \gamma \tilde{v}$, which is associated to the eigenvalue $\gamma \geq 0$.

%     \item Second, we establish that $\gamma QQ^\top - \Mtest = Q (\gamma I_q - \tilde{V} \Lambda \tilde{V}^\top) Q^\top \succeq 0$.
%     Fix $x \in \R^d$ and define $y \coloneqq Q^\top x \in \R^q$. Then
%     \begin{align*}
%         x^\top (\gamma QQ^\top - \Mtest )x = &\
%         x^\top Q (\gamma I_q - \tilde{V} \Lambda \tilde{V}^\top) Q^\top x = y^\top (\gamma I_q - \tilde{V} \Lambda \tilde{V}^\top) y \geq 0,
%     \end{align*}
%     where the first equality follows from \cref{eqn:psd-mat} and the last inequality follows from i).
%     \end{enumerate}
    
% \end{proof}





%%%%%%%%%%%%%%%%%%%%%%%%%%%%%%%%%%%%%%%%%%%%%%%%%%%%%%%%%%%%

\end{document}