\subsection{Ablation Studies}
We conduct an ablation study to showcase the efficacy of integrating CLIP with a facial recognition system, generating 600 photos for quantitative analysis.

% \subsection{Impact of Facial Recognizer}
% The facial recognizer (FR) is explicitly designed to capture intricate characteristics of an individual's identification. To evaluate its impact, we conduct experiments both with and without it, employing two metrics: facial reidentification models(Re-ID) \cite{king2009dlib}, and Perceptual Similarity (LPIPS) \cite{zhang2018unreasonable} metric. These metrics evaluate the degree to which the generated photographs consistently maintain the subject's identity, even when there are changes in pose, expression, or other qualities. 
% \begin{table}[ht]
%     \centering
%     \resizebox{\linewidth}{!}{%
%     \begin{tabular}{lcc}
%         \toprule
%         Method & LPIPS $\downarrow$ & Re-ID $\uparrow$ \\
%         \midrule
%         \\
%         DiffusionRig* & 0.4547 & \textbf{96.55} \\
%         \\
%         \textbf{Ours (insightface)} & \textbf{0.3870} & 96.59 \\
%         \\
%         \bottomrule
%     \end{tabular}%
%     }
%     % \caption{Novel expression synthesis. Quantitative evaluation for novel expression synthesis on the held-out "jaw\_right" expression. When evaluating against ground truth images with a white background, we convert the background pixels to white for all baselines using the ground truth masks. Our method outperforms the baselines on all metrics.}
%     \caption{pose rigging}
%     \label{tab:novel_expression_synthesis}
% \end{table}

% \subsection{Impact of Facial Recognizer}
\textbf{Impact of Facial Recognizer.} The facial recognizer (FR) is designed to capture critical identity-specific features. To assess its impact, we conduct ablation studies with and without the FR using the Re-identification (Re-ID) metric \cite{king2009dlib} to measure identity consistency across pose, expression, and lighting variations. As shown in Table~\ref{tab:facial_recognizer_ablation}, while the improvement may appear minor, it is crucial, as the FR plays a significant role in preserving essential facial features, such as the eyes, nose, and mouth, which are key to maintaining the subject's identity.

% \begin{table}[ht]
%     \centering
%     \caption{Quantitative evaluation for identity retention}
%     \label{tab:ablation_1}

%     \begin{subtable}[t]{\linewidth}
%         \centering
%         \caption{Re-identification Accuracy $\uparrow$}
%         \label{table:1a}
%         \begin{tabular}{lcccc}
%             \toprule
%             Method & Pose & Expression & Light \\
%             \midrule
%             Base + CLIP & 96.20 & 97.00 & 96.96 \\
%             \textbf{Base + CLIP + FR} & \textbf{96.72} & \textbf{97.28} & \textbf{97.12} \\
%             \bottomrule
%         \end{tabular}
%     \end{subtable}

%     \vspace{0.5cm} % Adds some vertical space between the sub-tables

%     \begin{subtable}[t]{\linewidth}
%         \centering
%         \caption{Perceptual Similarity Metric $\uparrow$}
%         \label{table:1a}
%         \begin{tabular}{lcccc}
%             \toprule
%             Method & Pose & Expression & Light \\
%             \midrule
%             Base + CLIP & 0.3715 & 0.0865 & 0.1407 \\
%             \textbf{Base + CLIP + FR} & \textbf{0.3990} & \textbf{0.0925} & \textbf{0.1512} \\
%             \bottomrule
%         \end{tabular}
%     \end{subtable}

% \end{table}



% \begin{table}[]
% \centering
% \begin{tabular}{lcccc}
% \toprule
% Method & Pose & Expression & Light \\
% \midrule
% Base + CLIP & 96.20 & 97.00 & 96.96 \\
% \textbf{Base + CLIP + FR} & \textbf{96.72} & \textbf{97.28} & \textbf{97.12} \\
% \bottomrule
% \end{tabular}
% \caption{Quantitative evaluation for identity retention: Re-identification Accuracy ($\uparrow$)}
% \label{tab:ablation_1}
% \end{table}

\textbf{Impact of CLIP Encoder.} Our approach integrates the CLIP encoder to provide an accurate representation of the input image and maintain its original distribution. To evaluate its effectiveness, we use the Fréchet Inception Distance (FID) \cite{heusel2017gans} and Structural Similarity Index Measure (SSIM) \cite{wang2004image} to quantify image realism and fidelity. As shown in Table~\ref{tab:clip_encoder_ablation}, the CLIP encoder significantly improves fidelity, especially for pose variations, while also preserving important background details such as the neck, hair, and accessories. 

% Our approach integrates the CLIP encoder to provide an accurate representation and maintain the original distribution of the primary input image. To assess its efficacy, we perform tests applying the Fréchet Inception Distance (FID) \cite{heusel2017gans} and Structural Similarity Index Measure (SSIM) \cite{wang2004image} to quantify the extent to which the produced images differ from the original input in terms of their realism and fidelity. This experiment aims to analyze the impact of the CLIP encoder on the model's training by comparing the results obtained with and without its use. The objective is to emphasize the CLIP encoder's importance in maintaining the generated images' overall quality.

% \begin{table}[]
%     \centering
%     \caption{Quantitative evaluation for image fidelity and realism}
%     \label{tab:ablation_1}

%     \begin{subtable}[t]{\linewidth}
%         \centering
%         \caption{Fréchet Inception Distance (FID) $\downarrow$}
%         \label{table:1a}
%         \begin{tabular}{lcccc}
%             \toprule
%             Method & Pose & Expression & Light \\
%             \midrule
%             Base + FR & 70.435 & 56.84 & 53.133 \\
%             \textbf{Base + CLIP + FR} & \textbf{58.20} & \textbf{46.685} & \textbf{44.66} \\
%             \bottomrule
%         \end{tabular}
%     \end{subtable}

%     \vspace{0.5cm} % Adds some vertical space between the sub-tables

%     \begin{subtable}[t]{\linewidth}
%         \centering
%         \caption{Structural Similarity Index Measure (SSIM) $\uparrow$}
%         \label{table:1a}
%         \begin{tabular}{lcccc}
%             \toprule
%             Method & Pose & Expression & Light \\
%             \midrule
%             Base + FR & 0.5766 & 0.8730 & 0.7660 \\
%             \textbf{Base + FR + CLIP } & \textbf{0.5879} & \textbf{0.8920} & \textbf{0.7850} \\
%             \bottomrule
%         \end{tabular}
%     \end{subtable}

% \end{table}


% \begin{table}[!htbp]
% \centering
% \caption{Quantitative evaluation for ablation studies: Identity retention, image fidelity, and realism.}
% \setlength{\tabcolsep}{4pt} % Adjust column spacing for better readability
% \renewcommand{\arraystretch}{1.1} % Adjust row height for better readability

% % Table for identity retention
% \begin{subtable}[t]{\columnwidth}
%     \centering
%     \caption{Re-identification Accuracy (\(\uparrow\))}
%     \begin{tabular}{lccc}
%         \toprule
%         Method & Pose & Expression & Light \\
%         \midrule
%         Base + CLIP & 96.20 & 97.00 & 96.96 \\
%         Base + CLIP + FR & \textbf{96.72} & \textbf{97.28} & \textbf{97.12} \\
%         \bottomrule
%     \end{tabular}
%     \label{tab:identity_retention}
% \end{subtable}
% \vspace{0.1cm} % Reduce space between sub-tables

% % Table for FID
% \begin{subtable}[t]{\columnwidth}
%     \centering
%     \caption{Fr\'echet Inception Distance (FID \(\downarrow\))}
%     \begin{tabular}{lccc}
%         \toprule
%         Method & Pose & Expression & Light \\
%         \midrule
%         Base + FR & 70.435 & 56.84 & 53.133 \\
%         Base + CLIP + FR & \textbf{58.20} & \textbf{46.685} & \textbf{44.66} \\
%         \bottomrule
%     \end{tabular}
%     \label{tab:fid}
% \end{subtable}
% \vspace{0.1cm} % Reduce space between sub-tables

% % Table for SSIM
% \begin{subtable}[t]{\columnwidth}
%     \centering
%     \caption{Structural Similarity Index Measure (SSIM \(\uparrow\))}
%     \begin{tabular}{lccc}
%         \toprule
%         Method & Pose & Expression & Light \\
%         \midrule
%         Base + FR & 0.5766 & 0.8730 & 0.7660 \\
%         Base + FR + CLIP & \textbf{0.5879} & \textbf{0.8920} & \textbf{0.7850} \\
%         \bottomrule
%     \end{tabular}
%     \label{tab:ssim}
% \end{subtable}
% \vspace{-0.2cm} % Further reduce space after tables

% \label{tab:ablation_studies}
% \end{table}

% \begin{table}[!htbp]
% \centering
% \caption{Quantitative evaluation for ablation studies: Identity retention, image fidelity, and realism.}
% \setlength{\tabcolsep}{4pt} % Adjust column spacing for better readability
% \renewcommand{\arraystretch}{1.1} % Adjust row height for better readability
% \small % Make the font size smaller for the table content
% \begin{tabular}{lccc}
%     \toprule
%     Method & Pose & Expression & Light \\
%     \midrule
%     \multicolumn{4}{l}{\textbf{Re-identification Accuracy (\(\uparrow\))}} \\
%     Base + CLIP & 96.20 & 97.00 & 96.96 \\
%     Base + CLIP + FR & \textbf{96.72} & \textbf{97.28} & \textbf{97.12} \\
%     \midrule
%     \multicolumn{4}{l}{\textbf{Fr\'echet Inception Distance (FID \(\downarrow\))}} \\
%     Base + FR & 70.435 & 56.84 & 53.133 \\
%     Base + CLIP + FR & \textbf{58.20} & \textbf{46.685} & \textbf{44.66} \\
%     \midrule
%     \multicolumn{4}{l}{\textbf{Structural Similarity Index Measure (SSIM \(\uparrow\))}} \\
%     Base + FR & 0.5766 & 0.8730 & 0.7660 \\
%     Base + CLIP + FR & \textbf{0.5879} & \textbf{0.8920} & \textbf{0.7850} \\
%     \bottomrule
% \end{tabular}
% \vspace{-0.3cm} % Reduce space after tables
% \label{tab:ablation_studies}
% \end{table}

% \begin{table}[!htbp]
% \centering
% \caption{Quantitative evaluation for ablation studies: Identity retention, image fidelity, and realism.}
% \setlength{\tabcolsep}{6pt} % Adjust column spacing for better readability
% \renewcommand{\arraystretch}{1.1} % Adjust row height for better readability
% \small % Make the font size smaller for the table content
% \begin{tabular*}{\columnwidth}{@{\extracolsep{\fill}}lccc@{}}
%     \toprule
%     Method & Pose & Expression & Lighting \\
%     \midrule
%     \multicolumn{4}{@{}l}{\textbf{\footnotesize Re-identification Accuracy (Re-ID\(\uparrow\))}} \\
%     Base + CLIP & 96.20 & 97.00 & 96.96 \\
%     Base + CLIP + FR & \textbf{96.72} & \textbf{97.28} & \textbf{97.12} \\
%     \midrule
%     \midrule
%     \multicolumn{4}{@{}l}{\textbf{\footnotesize Fr\'echet Inception Distance (FID \(\downarrow\))}} \\
%     Base + FR & 70.435 & 56.84 & 53.133 \\
%     Base + FR + CLIP & \textbf{58.20} & \textbf{46.685} & \textbf{44.66} \\
%     \midrule
%     \multicolumn{4}{@{}l}{\textbf{\footnotesize Structural Similarity Index Measure (SSIM \(\uparrow\))}} \\
%     Base + FR & 0.5766 & 0.8730 & 0.7660 \\
%     Base + FR + CLIP & \textbf{0.5879} & \textbf{0.8920} & \textbf{0.7850} \\
%     \bottomrule
% \end{tabular*}
% \vspace{-0.3cm} % Reduce space after tables
% \label{tab:ablation_studies}
% \end{table}
\vspace{-0.1cm} % Reduce space after tables
\begin{table}[!htbp]
    \centering
    \caption{Quantitative evaluation for ablation studies: Identity retention, image fidelity, and realism.}
    \setlength{\tabcolsep}{6pt} % Adjust column spacing for better readability
    \renewcommand{\arraystretch}{1.1} % Adjust row height for better readability
    \small % Make the font size smaller for the table content

    % Subtable (a): Facial Recognizer Ablation
    \begin{subtable}[t]{\columnwidth}
        \centering
        \caption{ Facial Recognizer Ablation}
        \label{tab:facial_recognizer_ablation}
        \begin{tabular*}{\columnwidth}{@{\extracolsep{\fill}}lccc@{}}
            \toprule
            Method & Pose & Expression & Lighting \\
            \midrule
            \multicolumn{4}{@{}l}{\textbf{\footnotesize Re-identification Accuracy (Re-ID\(\uparrow\))}} \\
            Base + CLIP & 96.20 & 97.00 & 96.96 \\
            Base + CLIP + FR & \textbf{96.72} & \textbf{97.28} & \textbf{97.12} \\
            \bottomrule
        \end{tabular*}
    \end{subtable}
    
    \vspace{0.2cm} % Space between subtables

    % Subtable (b): CLIP Encoder Ablation
    \begin{subtable}[t]{\columnwidth}
        \centering
        \caption{ CLIP Encoder Ablation}
        \label{tab:clip_encoder_ablation}
        \begin{tabular*}{\columnwidth}{@{\extracolsep{\fill}}lccc@{}}
            \toprule
            Method & Pose & Expression & Lighting \\
            \midrule
            \multicolumn{4}{@{}l}{\textbf{\footnotesize Fr\'echet Inception Distance (FID \(\downarrow\))}} \\
            Base + FR & 70.435 & 56.84 & 53.133 \\
            Base + FR + CLIP & \textbf{58.20} & \textbf{46.685} & \textbf{44.66} \\
            \midrule
            \multicolumn{4}{@{}l}{\textbf{\footnotesize Structural Similarity Index Measure (SSIM \(\uparrow\))}} \\
            Base + FR & 0.5766 & 0.8730 & 0.7660 \\
            Base + FR + CLIP & \textbf{0.5879} & \textbf{0.8920} & \textbf{0.7850} \\
            \bottomrule
        \end{tabular*}
    \end{subtable}

    \vspace{-0.5cm} % Reduce space after tables
    % \label{tab:ablation_studies}
\end{table}



% \subsection{Rigging Quality Evaluation}
% In this experiment, we aim to evaluate the rigging quality of our model. We generate 1200 photographs to examine how accurately the model conforms to the desired position, expression, and shape. In contrast to prior studies, we do not randomly select photos for this examination. Similar to DiffusionRig\cite{ding2023diffusionrig}, our model includes an extra fine-tuning phase. We contend that this stage involves a compromise between safeguarding the primary characteristics of the image and retaining authority over the intended properties. Specifically, when fine-tuning the parameters of an album or a single image, there is a possibility of overfitting the primary image, leading to a minor neglect of the conditional inputs. To verify this, we conduct an experiment in which we train DiffusionRig using only one image while keeping the same parameters outlined in their paper. Furthermore, we train our model using only one picture and then measure the DECA\cite{li2017learning} re-inference error to compare the results.






% \begin{table}[!htbp]
% \centering
% \begin{tabular}{lcc}
% \toprule
% \textbf{Method} & Pose & Expression \\
% \midrule
% Caphuman & 23.51 mm & 7.37 mm \\
% DiffusionRig & 11.32 mm & \textbf{3.37 mm} \\
% \textbf{Ours}  & \textbf{9.37 mm} & 5.14 mm \\
% \bottomrule
% \end{tabular}
% \caption{Deca re-inference error}
% \end{table}


% \subsection{Discussion}

% There are notable disparities in control capability between our approach and DiffusionRig, especially when adjusting parameters for a single image. Our model exhibits far better pose control, with an inaccuracy of only 9.37 mm, in contrast to DiffusionRig's 11.32 mm. The main reason for this difference is the distinct approaches that each model employs in managing control inputs. DiffusionRig incorporates the controls by combining them with the primary input picture. When the model is accurately tuned for a single image, this method often leads to the model becoming too specialized for the pose of that particular image, resulting in a considerable error. Figure ~\ref{fig:pose_rigging_quality} illustrates the pose rigging quality of both DiffusionRig and our method during single-image fine-tuning. On the other hand, our model separates the control maps from the primary image, guaranteeing that the controls continue to be functional even while making minor adjustments. This strategy of disentanglement prevents the model from being excessively impacted by the input image during the fine-tuning process, enabling more precise control over the position.

% DiffusionRig demonstrates superior expression control with an error of 3.37 mm, outperforming our model's error of 5.14 mm. It is crucial to acknowledge that DiffusionRig's method frequently results in artifacts near the mouth area. This phenomenon arises due to the model's attempt to maintain the integrity of pixels from the primary input image, leading to an overlap that gives rise to conspicuous artifacts. However, while our approach may have a slightly reduced precision in regulating expression, it effectively avoids these undesirable effects, leading to cleaner and more genuine results. Figure ~\ref{fig: expression rigging quality} shows the expression rigging quality of DiffusionRig and our method.





% \begin{figure}[ht]
%     \centering
%     \includegraphics[width=\linewidth]{Sections/figures/rigging_quality_pose.png} % Include your image here
%     \caption{Pose rigging quality of DiffusionRig and our method during single-image fine-tuning. The input image is expected to follow the pose of the target image.}
%     \label{fig:pose_rigging_quality}
% \end{figure}

% \begin{figure}[ht]
%     \centering
%     \includegraphics[width=\linewidth]{Sections/figures/rigging_quality_exp.png} % Include your image here
%     \caption{Expression rigging quality of DiffusionRig and our method. The target image's expression is expected to be replicated by the input image.}
%     \label{fig: expression rigging quality}
% \end{figure}

% Now write though diffusionrig we expression better, but sekhaneo overlap ase and amader kisu example o oi related provide korte hobe.

% Then, amader oi background er kahini tao dekhaite hobe if needed.

