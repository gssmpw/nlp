\section{Related Work}
We review prior research on design parameters for Tactons in haptics, studies on vibrotactile perception, sensation, and emotion, and computational models for haptic stimuli.


\subsection{Design Parameters and Libraries for Tactons}
Over the decades, extensive research have explored myriad design parameters for Tactons to effectively convey and communicate information and emotions in various haptic user interfaces~\cite{maclean2008foundations}.
Prior studies have investigated low-level parameters of Tactons, such as carrier frequency~\cite{hwang2010perceptual, israr2006frequency}, envelope frequency~\cite{park2011perceptual, lim2023can}, duration~\cite{kwon2023can, yoo2015emotional}, and amplitude~\cite{hwang2010perceptual, israr2006frequency}, as well as superposition of multiple sinusoids~\cite{yoo2022perceived, hwang2017perceptual} and combinations of multiple parameters~\cite{yoo2015emotional, lim2023can}.
Other studies have proposed high-level parameters of Tactons, such as rhythmic structure~\cite{ternes2008designing, brown2006multidimensional, abou2022vibrotactile}, which include the evenness of pulses and note length, interval between vibrations~\cite{tan2019user}, and sound waveform or timbre~\cite{brown2007tactons, brown2005first}.
Designers can create Tactons by systematically varying these parameters (i.e., parameter-based design approach).
For constructing a haptic dataset, we select four common low-level Tacton parameters -- carrier frequency, envelope frequency, duration, and amplitude -- and one high-level parameter, rhythmic structure.


In addition to creating Tactons from scratch, haptic designers can create new Tacotns by transforming libraries from other modalities into vibration libraries or by modifying template Tactons from existing vibration libraries~\cite{schneider2016studying}.
Past studies have proposed various vibration libraries together with their associated user feelings or metaphors (i.e., metaphor-based design approach).
For example, van Erp and M.A. Spap\'e created 59 Tactons by transforming auditory melodies into vibrations and examined their perceptual impacts~\cite{van2003distilling}.
Disney Research introduced FeelEffects, a library of over 40 Tactons, and investigated the semantic and parametric spaces of these Tactons~\cite{israr2014feel}.
Seifi et al. proposed VibViz, a library of 120 Tactons with subjective ratings and descriptive tags on their physical, sensory, emotional, usage, and metaphoric attributes~\cite{seifi2015vibviz}.
These libraries consist of Tactons with complex waveforms, such as intricate rhythmic structures in the time domain, varying frequency spectra over time, and various durations.
We include 40 Tactons by modifying existing Tactons from the open-source vibration library VibViz for rendering on iPhones, to enhance the diversity of our haptic dataset.


\subsection{Studies on Vibrotactile Perception, Sensation, and Emotion}

Investigating how humans perceive vibrations is a fundamental aspect of the haptics field.
Prior research has studied the neurophysiological processes underlying tactile perception, identifying four types of mechanoreceptors in human skin that contribute to the perception of touch~\cite{goldstein1989sensation}.
These four mechanoreceptive tactile channels have different characteristics, such as perception properties, spectral sensitivities, and sensory adaptation rates~\cite{kandel2000principles, choi2012vibrotactile}.
Among these four mechanoreceptive channels, Meissner Corpuscle (RA1) and Pacinian Corpuscle (RA2) are primarily activated by vibrotactile stimuli compared to Merkel Disk (SA1) and Ruffini Ending (SA2).
Drawing on the properties of the two mechanoreceptive channels (RA1 and RA2) most related to vibrations and their spectral sensitivities, we propose a mechanoreceptive processing approach that converts vibration signals into two-channel spectrograms.


Previous studies have also explored the perceptibility and discriminability of vibrations.
Psychophysics researchers have examined the detection thresholds, or Absolute Limens (AL), for vibrations and uncovered a U-shaped threshold curve across frequencies, with the highest sensitivity occurring around 200\,Hz (between 150\,Hz and 300\,Hz)~\cite{gescheider2013psychophysics, ryu2010psychophysical}.
Other studies have investigated the discrimination thresholds, or Just Noticeable Differences (JND), between vibrations~\cite{israr2006frequency, franzen1975vibrotactile, goble1994vibrotactile, goff1967differential}.
JNDs for vibration intensity generally range from 10\% to 30\%, while JNDs for vibration frequency typically fall between 15\% and 30\%~\cite{choi2012vibrotactile}.
Some studies have also explored the relative impact of these parameters; increasing the duration of vibrations has been shown to decrease the JNDs for vibration intensity~\cite{gescheider1996effects, jones2006human}.
We use these established findings on detection and discrimination thresholds for vibrations to augment our vibration signals while ensuring that the augmented vibrations remain perceptually indistinguishable to users.


The sensations and emotions elicited by haptic stimuli are also important research topics in HCI.
Past research has developed sensory and emotional lexicons associated with haptic stimuli~\cite{holliins1993perceptual, hollins2000individual, soufflet2004comparison, tiest2006analysis}.
Researchers have collected user ratings based on Russell's circumplex model of affect, which defines two key emotional dimensions: valence and arousal~\cite{russell1980circumplex}.
Guest et al. identified 26 sensory attributes and 14 emotional attributes for describing touch and demonstrated that valence and arousal are primary factors in haptic emotional experience~\cite{guest2011development}.
% Among sensory attributes, researchers have found that roughness and hardness are key dimensions for capturing the perceptual properties of real textures or materials, with low variability among individual users.
% However, users often face challenges in associating hardness with Tactons~\cite{seifi2013first, seifi2015vibviz}.
Among sensory attributes, researchers identified roughness as the primary dimension for capturing the perceptual properties of real textures or materials, with low variability among individual users.
Studies on Tactons reported analogous findings: participants associated Tactons with roughness effectively but often faced challenges in associating hardness with Tactons, and sensory attributes like wet/dry or hot/cold were not relevant to vibrations~\cite{seifi2013first, seifi2015vibviz}.
Additionally, while temporal attributes such as tempo, energy, and rhythm are relevant for describing certain vibration patterns, they are not considered primary sensory dimensions, as they are context-dependent and less generalizable across different tactile scenarios.
Building on these established frameworks, we select roughness as the primary sensory dimension for Tactons and valence and arousal as the two primary emotional dimensions.


To inform the design of Tactons that convey specific sensations and emotions, previous studies have sought to derive sensory and emotional spaces for Tactons and identify effective design parameters.
After creating one or more sets of Tactons, designers typically recruit participants to gather subjective responses to these Tactons.
The sensory and emotional spaces are then visualized using averaged ratings for each Tacton, or statistical tests are applied to determine which design parameters significantly impact the sensations and emotions.
Seifi and Maclean designed 14 Tactons and demonstrated that rhythmic structure influenced all three dimensions (roughness, valence, and arousal) and that carrier frequency affected arousal~\cite{seifi2013first}.
Yoo et al. explored emotional spaces with three different Tacton sets (25, 36, and 24 patterns) and provided design guidelines for four sinusoidal parameters~\cite{yoo2015emotional}.
Seifi et al. collected roughness, valence, and arousal ratings for 120 Tactons and developed the VibViz visualization to assist haptic designers~\cite{seifi2015vibviz}.
While these studies have examined emotional and sensory spaces through controlled laboratory user studies (i.e., offline studies), recent research has explored the efficacy of crowdsourcing user studies (i.e., online studies) to reduce the effort and time needed to collect sensory and emotional ratings of Tactons~\cite{schneider2016hapturk, lim4785071emotional}. 
% This approach used proxy Tactons~\cite{schneider2016hapturk} or the same Tactons used in lab studies~\cite{lim4785071emotional} and employed participants' smartphones equipped with vibrotactile actuators.
However, these studies still rely on the efforts of researchers and designers in evaluating Tactons, which limits their scalability.
To address this limitation, we construct sensory and emotional ratings of 154 Tactons, the largest Tacton set ever studied, and develop a model that predicts roughness, valence, and arousal ratings to enable the rapid prototyping of effective Tacton candidates.


\subsection{Computational Models for Haptic Stimuli}

Computational models that predict subjective evaluations of haptic stimuli or compare different haptic stimuli can help designers and researchers enhance user experiences and accelerate the design process in various applications. 
Previous research has developed models to predict haptic perceptions and sensations for objects or textured surfaces.
These studies used robots or rigid tools to collect haptic data generated from physical interactions, such as accelerations, forces, and speeds.
The resulting models predict sensory attributes elicited from objects~\cite{chu2015robotic, richardson2020learning, gao2016deep}, textured surfaces~\cite{awan2023predicting, ito2021model}, or perceptual similarities between textured surfaces~\cite{richardson2022learning}.
While these models primarily focus on real objects or textured surfaces, we propose a model that predicts the sensations and emotions elicited by Tactons conveyed through a single vibrotactile actuator in haptic interfaces, which generates 1D accelerations.


Past studies have also proposed computational models for predicting subjective evaluations for vibrations rendered on a vibrotactile actuator.
Park and Kuchenbecker introduced algorithms to transform three-axes accelerations collected from human interactions with textured surfaces into one-axis accelerations that can be played on a vibrotactile actuator~\cite{park2019objective}.
They then developed a model to assess the perceptual similarities between the haptic stimuli generated from human interactions with textured surfaces and the converted one-axis accelerations.
Others researchers proposed models for evaluating and comparing perceptual qualities between original and compressed vibration signals~\cite{hassen2019subjective, muschter2021perceptual, noll2022automated}.
Recent work introduced a model to predict perceptual dissimilarities between Tactons by simulating the neural transmission from mechanoreceptors in the skin to the brain~\cite{lim2023can}.
% This model used two parallel streams for simulation: the first stream captured temporal features of vibrations, especially rhythmic structures, by applying Dynamic Time Warping (DTW)~\cite{berndt1994using} to calculate the distance between two waveforms in the time domain.
% The second stream aimed to capture spectral features of vibrations by converting vibrations into sets of neural spikes and processing them with mechanoreceptive filters.
Previous studies primarily focused on developing models to predict perceptual distinguishability between real-world haptic stimuli and vibrations or between different Tactons.
In contrast, we propose a computational model that predicts sensations and emotions elicited by Tactons.
We note that unless perceptual differences between Tactons are extremely subtle (i.e., Tactons are not easily distinguishable), users may perceive their sensory and emotional attributes differently. %these differences can significantly influence variations in roughness and emotions.
This highlights the need for vibrotactile sensation and emotion prediction models, as perceptual distinguishability models alone may not capture the nuanced mappings between vibrations and user-rated sensation and emotions.
%Furthermore, relying solely on perceptual similarity assumes a linear relationship with emotions, which is often not the case.
Our approach enables efficient prediction for new and unique Tactons that may not closely resemble existing vibrations in the dataset.


\begin{figure*}[t]
  \centering
    \includegraphics[width=\textwidth]{Figure/Approach.png}
    \caption{
    An overview diagram illustrating the Tacton design, user study to construct a haptic dataset, and our computational framework.
    % The dataset consists of acceleration signals and corresponding sensory and emotional ratings for 154 vibrations.
    % The proposed framework includes haptic data augmentation, mechanoreceptive processing, and a neural network model (VibNet) to predict user ratings.
    % We verify the performance of VibNet using 48 unseen vibrations.
    }
  \label{fig:overview}
  % \Description{}
\end{figure*}