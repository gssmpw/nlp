\section{Literature Review}
Credit risk assessment is a critical aspect of financial decision-making, central to the lending practices of financial institutions. Traditional methods, including logistic regression and decision trees, have long served as the foundation for credit risk assessment models. These models depend heavily on historical data and predefined risk categories for classifying borrowers into specific risk groups. While these traditional methods have been somewhat effective, they struggle to adapt to the dynamic and complex nature of modern financial markets. This limitation often results in inaccuracies in risk predictions \cite{Marqués20131384}.\\\\In recent years, there has been a notable shift toward exploring more advanced techniques for credit risk assessment. Machine learning, particularly deep learning algorithms, have gained prominence due to their ability to capture intricate patterns and relationships in data. Researchers have investigated the use of neural networks, random forests, and support vector machines for credit scoring \cite{GOLBAYANI2020101251}. These machine learning approaches have shown promise, offering improved predictive accuracy compared to traditional methods \cite{thakkar2024improved} \cite{schuld2019quantum}. However, these models are still limited by their reliance on large amounts of labeled data and can struggle to generalize when faced with novel or unseen scenarios, particularly in dynamic financial environments.\\\\Moreover, quantum computing has emerged as a transformative technology with the potential to reshape various industries, including finance. Quantum computing's ability to perform complex calculations at unprecedented speeds has raised significant interest in its application to financial problems, including credit risk assessment. Quantum algorithms, such as Shor's and Grover's, show great potential for addressing complex financial optimization challenges \cite{Shor2002} \cite{Grover1996}. These quantum algorithms, while powerful, are still in the early stages of development and often require substantial computational resources, which presents a significant challenge for their practical application in real-world financial systems.\\\\ \gls{qml}, an emerging field, investigates the integration of quantum computing with machine learning techniques \cite{jha2020quantum} \cite{farhi2018classification}. QML has the potential to enhance the training and prediction capabilities of machine learning models, offering improved pattern recognition and predictive accuracy in financial applications \cite{lloyd2013quantum} \cite{Schuld2019}. This intersection of quantum computing and machine learning offers exciting opportunities to develop innovative credit risk assessment models. However, the adoption of QML techniques in financial risk models is still limited by issues such as the scalability of quantum systems and the high cost of quantum hardware, making large-scale implementation challenging.\\\\ QDL is an interdisciplinary field that combines the principles of quantum computing with deep learning techniques to develop more powerful machine learning models. This emerging area explores how quantum computing hardware and algorithms can be leveraged to enhance the training and execution of deep neural networks. QDL has the potential to address certain complex problems more efficiently than classical deep learning methods and classical machine learning methods, especially in areas where quantum computation provides an advantage \cite{9528698}. The ability of QDL to handle large datasets and improve convergence rates in deep learning models has been demonstrated in several domains; however, its integration into financial models, particularly for credit risk assessment, remains an area of ongoing research and experimentation.\\\\ In the pursuit of more accurate and efficient credit risk assessment, the integration of quantum computing and classical deep neural networks has garnered attention. This hybrid model combines the computational advantages of quantum computing with the interpretability and stability of traditional classical models \cite{biamonte2017quantum} \cite{havlicek2019supervised}. This synergy between quantum computing and machine learning holds the potential to revolutionize credit risk assessment by harnessing quantum power while addressing the intricacies of diverse loan types. However, most existing research on hybrid quantum-classical models for credit risk assessment lacks a focus on Row-Type Dependent Predictive Analysis (RTDPA), which recognizes that different loan types exhibit distinct characteristics and risk profiles. Our proposed framework aims to fill this gap by leveraging RTDPA within the quantum-powered model to enable more accurate and granular risk assessments for various loan types.\\\\ In conclusion, the literature reviewed here highlights the evolving landscape of credit risk assessment, from traditional methods to advanced machine learning techniques, quantum computing, and innovative approaches like RTDPA. While existing studies have explored various aspects of quantum computing and machine learning for credit risk assessment, our research stands out by integrating quantum computing with RTDPA to provide a more adaptive, precise, and effective framework for credit risk analysis. By presenting a comprehensive overview of these developments and their potential implications, this review lays the groundwork for our proposed research on Quantum-Powered Credit Risk Assessment, which aims to bridge the gap between quantum computing and financial risk assessment.