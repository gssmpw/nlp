
\section{Preliminaries}
\label{sec:pre}

In this section, we introduce the structure of legal case documents, which guides the design of our task. 
Specifically, unlike documents in general domain, legal case documents typically have a more structured format. Following the definition by Li et al.,~\cite{li2023sailer}, legal case documents generally consist of five parts: \textbf{Procedure}, \textbf{Trail Fact}, \textbf{Reasoning}, \textbf{Judgment} and \textbf{Tail}.
The Procedure section includes claims, defense statements, and the evidence lists submitted by both parties.
The Trial Fact section presents the verified events as determined by the court.
The Reasoning section explains how the court analyzes disputed issues, selects relevant legal rules, and applies them to the case facts.
The Judgment section includes the court's final ruling and relevant legal provisions.
The Tail section contains details such as the court's name, judge information, and other procedural formalities.


Each section of a legal case document follows distinct writing logic and evaluation criteria. For example, the Trial Fact section prioritizes a complete evidence chain and an accurate timeline, while the Reasoning section focuses on identifying key issues and applying legal rules correctly.
These differences impose distinct demands on the LLM’s understanding and reasoning abilities. Generating a complete case document in one step fails to properly evaluate the LLM's performance in generating each structural component.
Therefore, we design the multi-stage generation task that aligns with the writing logic of legal case documents.
This approach not only enables more precise evaluation of LLMs, but also provides a more reliable solution for practical legal AI applications.





