\section{Related Work}
\subsection{Memory for LLM Agents}
Prior works on LLM agent memory systems have explored various mechanisms for memory management and utilization~\cite{aios,agentlite,mem0,memorybank}. Some approaches complete interaction storage, which maintains comprehensive historical records through dense retrieval models~\cite{memorybank} or read-write memory structures~\cite{modarressi2023ret}. Moreover, MemGPT~\cite{memgpt} leverages cache-like architectures to prioritize recent information. Similarly, SCM~\cite{wang2023enhancing} proposes a Self-Controlled Memory framework that enhances LLMs' capability to maintain long-term memory through a memory stream and controller mechanism.
However, these approaches face significant limitations in handling diverse real-world tasks. While they can provide basic memory functionality, their operations are typically constrained by predefined structures and fixed workflows. These constraints stem from their reliance on rigid operational patterns, particularly in memory writing and retrieval processes. Such inflexibility leads to poor generalization in new environments and limited effectiveness in long-term interactions. Therefore, designing a flexible and universal memory system that supports agents' long-term interactions remains a crucial challenge.


\subsection{Retrieval-Augmented Generation}
Retrieval-Augmented Generation (RAG) has emerged as a powerful approach to enhance LLMs by incorporating external knowledge sources~\cite{rag1,borgeaud2022improving,gao2023retrieval}. The standard RAG~\cite{yu2023chain,wang2023learning} process involves indexing documents into chunks, retrieving relevant chunks based on semantic similarity, and augmenting the LLM's prompt with this retrieved context for generation. Advanced RAG systems~\cite{lin2023ra,ilin2023advanced} have evolved to include sophisticated pre-retrieval and post-retrieval optimizations.
Building upon these foundations, recent researches has introduced agentic RAG systems that demonstrate more autonomous and adaptive behaviors in the retrieval process. These systems can dynamically determine when and what to retrieve~\cite{asai2023self,jiang2023active}, generate hypothetical responses to guide retrieval, and iteratively refine their search strategies based on intermediate results~\cite{trivedi2022interleaving,shao2023enhancing}. 

However, while agentic RAG approaches demonstrate agency in the retrieval phase by autonomously deciding when and what to retrieve~\cite{asai2023self,jiang2023active,yu2023augmentation}, our agentic memory system exhibits agency at a more fundamental level through the autonomous evolution of its memory structure. Inspired by the Zettelkasten method, our system allows memories to actively generate their own contextual descriptions, form meaningful connections with related memories, and evolve both their content and relationships as new experiences emerge. This fundamental distinction in agency between retrieval versus storage and evolution distinguishes our approach from agentic RAG systems, which maintain static knowledge bases despite their sophisticated retrieval mechanisms.