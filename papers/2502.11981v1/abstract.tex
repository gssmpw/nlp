
Decades of research in machine learning have given us powerful tools for making accurate predictions.
But when used in social settings and on human inputs,
% on human inputs
better accuracy does not immediately translate to better social outcomes.
This may not be surprising given that conventional learning frameworks
are not designed to express societal preferences---let alone promote them.
This position paper argues that machine learning is currently missing,
and can gain much from incorporating,
a proper notion of \emph{social welfare}.
% But how can we reason about 
% which social outcomes are preferable, and how do we promote them in practice?
% This position paper argues that machine learning can greatly benefit from integrating a proper notion of \emph{social welfare}.
% We propose to draw on the general ideas and principles of welfare economics, and adapt them to our purpose.
% the use of machine learning in modern social contexts.
The field of  welfare economics asks:
how should we allocate limited resources to self-interested agents in a way that maximizes social benefit?
We argue that this perspective applies to %, and can be useful for,
many modern applications of machine learning in social contexts,
and advocate for its adoption.
Rather than disposing of prediction,
we aim to leverage this forte of machine learning
for promoting social welfare.
We demonstrate this idea by proposing a conceptual framework that gradually transitions from accuracy maximization (with awareness to welfare)
to welfare maximization (via accurate prediction). 
We detail applications and use-cases for which our framework can be effective,
identify technical challenges and practical opportunities,
and highlight future avenues worth pursuing.
% \squeeze



% We believe this presents an indispensable resource that can be used by the learning community to design welfare-aware learning approaches in an informed and disciplined manner.
% This position paper aims to establish a concrete connection between machine learning and the field of welfare economics.