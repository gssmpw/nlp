\section{Alternative Views} \label{sec:alt_views}
While it is easy to agree that learning systems should be designed to enable the promotion of social good, there will likely be disagreement as to how.
We propose to adopt the perspective of welfare economics,
but there certainly exist alternative viewpoints and complementary approaches.
\squeeze

\paragraph{Give accuracy time.}
One perspective is that if we give data enough time to accumulate and new methods enough time to improve, then machine learning will organically overcome the challenges we discussed.
One example to draw on is how despite many advances in optimization, the simple gradient descent algorithm still drives most modern tools.
Another is how large language models have demonstrated that simply predicting the next word with sufficient accuracy and on enough data 
gives rise to emergent phenomena far beyond this basic task.
Our position is that limited resources is an inherent problem of any social system, whether technology-driven or not.%
\extended{Our point here would be that since even 50 thousand years of cultural evolution have not `solved' the problem of scarce resource allocation, 
it is unlikely to just go away.}
% nor has this been dismissed in biology or ecology.
We believe that scarcity should be addressed explicitly---%
but of course we may be proven wrong.
\squeeze

\paragraph{Divide and conquer.}
Even if  machine learning as a standalone solution does not suffice,
one could argue that an economic approach can be applied on top of existing learning tools, rather than integrated within them as we propose.
Hence, learning and policy can be advanced independently and combined only later.
This is reasonable, and independent efforts and application will likely be required
regardless of whether direct integration works or not.
But there is increasing evidence that this will not suffice;
in fact, the field of fair machine learning rose in response to the clear need for embedding social considerations within the learning objective itself.
Advances in the study of fairness in learning have also shown that fairness constraints alone cannot guarantee equity,
such as when learning effects accumulate over time \citep[e.g.,][]{liu2018delayed}
or as a result of strategic user behavior
\citep{horowitz2024classification}.
We take these to suggest that the novelty in the interface between learning and economics requires a wholistic approach
specialized for this intersection.%
\extended{that operates on a deep, joint understanding of how both fields can intersect.}
\squeeze

\paragraph{Welfare without welfare.}
Welfare economics is not the only approach for reasoning about and facilitating welfare,
nor is it free from issues and limitations in itself.
% For example, welfare economics has been criticized for
Criticism includes
its subjective nature;
the need to measure and compare utility across individuals; 
the emphasis on cardinal rather than ordinal utilities;%
\extended{(which are more plausible but harder to work with);}
the reliance on assumptions of rational behavior; 
the susceptibility to externalities and other sources of market failure; 
the need for a centralized social planner entity;
and challenges in policy evaluation.
Other schools of thought in economics offer alternatives:
For example, Sen's \emph{capabilities approach}
\citep{sen1999commodities}
focuses on ensuring people are capable of achieving what they seek,
rather than the value of what they obtain.
\priority{Another example is behavioral economics which stresses the nature of human-decision making as the key to improving social outcomes.}
Within machine learning, there been have calls for alternative approaches as well,
such as to `democratize' the issue of alignment using social choice theory
\citep{conitzer2024position,ge2024axioms,fish2024generative}.
We view these as complementary to ours,
and believe there is merit advancing welfare in machine learning
simultaneously along several fronts.
\squeeze

% and while some approaches may prove more beneficial than others in practice, until this becomes clear, we believe there is merit in exploring ideas along multiple fronts.



% 1. maybe standard ml is ok. maybe fairness suffices. hasn't worked well.
% 2. can just apply econ consideration on top of existing stuff. but - this was supposedly possible until now, but wasn't done.
% 3. even if integrate welfare into ml, there is crit for welf econ itself. there are alternatives 
% 4. possible alternatives... not instead, but alongside. welfml not only option.

% I think one opposition is that as model/data become sufficiently large and accuracy improves, they will automatically address all our proposed questions, so our questions do not need to be studied separately….

% - maybe accuracy *will* just solve it all? might be better to leave ML intact and in isolation (maybe with eg fairness constraints) and wrap with policy, rather than plugging econ modeling inside. may lose more than gain by integrating

% - not only supervised learning: online, rl, generative (LLMs)

% - other welfare econ crits: subjectivity in defining welfare and equity; comparing utilities across individuals (ordinal,  cardinal); rational modeling; externalities.
%   -- alternatives: Capabilities Approach, behavioral econ

% - centralized control of societal prefs; vs social choice, deliberative democracy
 % -- Social Choice Should Guide AI Alignment in Dealing with Diverse Human Feedback / conitzer et al, ICML24 position paper!
 % -- Generative Social Choice / procaccia
 % -- Axioms for AI Alignment from Human Feedback / procaccia [NeurIPS 2024]

% - measurement and eval - can't really estimate utility (before and after)



\extended{%
\bbox
\textbf{Desideratum:}
\blue{Understand limitations, but also benefits of naive learning: when it works, when it doesn't, and why.}
\ebox
}
