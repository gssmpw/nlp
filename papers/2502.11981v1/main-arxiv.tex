%%%%%%%% ICML 2025 EXAMPLE LATEX SUBMISSION FILE %%%%%%%%%%%%%%%%%

\documentclass{article}

% Recommended, but optional, packages for figures and better typesetting:
\usepackage{microtype}
\usepackage{graphicx}
\usepackage{subfigure}
\usepackage{booktabs} % for professional tables

% hyperref makes hyperlinks in the resulting PDF.
% If your build breaks (sometimes temporarily if a hyperlink spans a page)
% please comment out the following usepackage line and replace
% \usepackage{icml2025} with \usepackage[nohyperref]{icml2025} above.
\usepackage{hyperref}
\hypersetup{
    colorlinks,
    linkcolor={red!50!black},
    citecolor={blue!50!black},
    urlcolor={blue!80!black}
}

% Attempt to make hyperref and algorithmic work together better:
\newcommand{\theHalgorithm}{\arabic{algorithm}}

% Use the following line for the initial blind version submitted for review:
%\usepackage{icml2025}

% If accepted, instead use the following line for the camera-ready submission:
% \usepackage[accepted]{icml2025}
\usepackage[preprint]{neurips_2024} 

% For theorems and such
\usepackage{amsmath}
\usepackage{amssymb}
\usepackage{mathtools}
\usepackage{amsthm}

% if you use cleveref..
\usepackage[capitalize,noabbrev]{cleveref}



%%%%%%%%%%%%%%%%%%%%%%%%%%%%%%%%
% THEOREMS
%%%%%%%%%%%%%%%%%%%%%%%%%%%%%%%%
\theoremstyle{plain}
\newtheorem{theorem}{Theorem}[section]
\newtheorem{proposition}[theorem]{Proposition}
\newtheorem{lemma}[theorem]{Lemma}
\newtheorem{corollary}[theorem]{Corollary}
\theoremstyle{definition}
\newtheorem{definition}[theorem]{Definition}
\newtheorem{assumption}[theorem]{Assumption}
\theoremstyle{remark}
\newtheorem{remark}[theorem]{Remark}

% Todonotes is useful during development; simply uncomment the next line
%    and comment out the line below the next line to turn off comments
%\usepackage[disable,textsize=tiny]{todonotes}
% \usepackage[textsize=tiny]{todonotes}


% The \icmltitle you define below is probably too long as a header.
% Therefore, a short form for the running title is supplied here:
% \icmltitlerunning{Submission and Formatting Instructions for ICML 2025}


% our packages:
\usepackage{bbm, dsfont, bm}
\usepackage{enumitem}
\usepackage[hang,flushmargin]{footmisc}
\usepackage{relsize}
\usepackage{tcolorbox}
\newcommand{\bbox}{\begin{tcolorbox}[colback=red!5!white,colframe=red!75!black,boxsep=0mm]}
\newcommand{\ebox}{\end{tcolorbox}}


%
\setlength\unitlength{1mm}
\newcommand{\twodots}{\mathinner {\ldotp \ldotp}}
% bb font symbols
\newcommand{\Rho}{\mathrm{P}}
\newcommand{\Tau}{\mathrm{T}}

\newfont{\bbb}{msbm10 scaled 700}
\newcommand{\CCC}{\mbox{\bbb C}}

\newfont{\bb}{msbm10 scaled 1100}
\newcommand{\CC}{\mbox{\bb C}}
\newcommand{\PP}{\mbox{\bb P}}
\newcommand{\RR}{\mbox{\bb R}}
\newcommand{\QQ}{\mbox{\bb Q}}
\newcommand{\ZZ}{\mbox{\bb Z}}
\newcommand{\FF}{\mbox{\bb F}}
\newcommand{\GG}{\mbox{\bb G}}
\newcommand{\EE}{\mbox{\bb E}}
\newcommand{\NN}{\mbox{\bb N}}
\newcommand{\KK}{\mbox{\bb K}}
\newcommand{\HH}{\mbox{\bb H}}
\newcommand{\SSS}{\mbox{\bb S}}
\newcommand{\UU}{\mbox{\bb U}}
\newcommand{\VV}{\mbox{\bb V}}


\newcommand{\yy}{\mathbbm{y}}
\newcommand{\xx}{\mathbbm{x}}
\newcommand{\zz}{\mathbbm{z}}
\newcommand{\sss}{\mathbbm{s}}
\newcommand{\rr}{\mathbbm{r}}
\newcommand{\pp}{\mathbbm{p}}
\newcommand{\qq}{\mathbbm{q}}
\newcommand{\ww}{\mathbbm{w}}
\newcommand{\hh}{\mathbbm{h}}
\newcommand{\vvv}{\mathbbm{v}}

% Vectors

\newcommand{\av}{{\bf a}}
\newcommand{\bv}{{\bf b}}
\newcommand{\cv}{{\bf c}}
\newcommand{\dv}{{\bf d}}
\newcommand{\ev}{{\bf e}}
\newcommand{\fv}{{\bf f}}
\newcommand{\gv}{{\bf g}}
\newcommand{\hv}{{\bf h}}
\newcommand{\iv}{{\bf i}}
\newcommand{\jv}{{\bf j}}
\newcommand{\kv}{{\bf k}}
\newcommand{\lv}{{\bf l}}
\newcommand{\mv}{{\bf m}}
\newcommand{\nv}{{\bf n}}
\newcommand{\ov}{{\bf o}}
\newcommand{\pv}{{\bf p}}
\newcommand{\qv}{{\bf q}}
\newcommand{\rv}{{\bf r}}
\newcommand{\sv}{{\bf s}}
\newcommand{\tv}{{\bf t}}
\newcommand{\uv}{{\bf u}}
\newcommand{\wv}{{\bf w}}
\newcommand{\vv}{{\bf v}}
\newcommand{\xv}{{\bf x}}
\newcommand{\yv}{{\bf y}}
\newcommand{\zv}{{\bf z}}
\newcommand{\zerov}{{\bf 0}}
\newcommand{\onev}{{\bf 1}}

% Matrices

\newcommand{\Am}{{\bf A}}
\newcommand{\Bm}{{\bf B}}
\newcommand{\Cm}{{\bf C}}
\newcommand{\Dm}{{\bf D}}
\newcommand{\Em}{{\bf E}}
\newcommand{\Fm}{{\bf F}}
\newcommand{\Gm}{{\bf G}}
\newcommand{\Hm}{{\bf H}}
\newcommand{\Id}{{\bf I}}
\newcommand{\Jm}{{\bf J}}
\newcommand{\Km}{{\bf K}}
\newcommand{\Lm}{{\bf L}}
\newcommand{\Mm}{{\bf M}}
\newcommand{\Nm}{{\bf N}}
\newcommand{\Om}{{\bf O}}
\newcommand{\Pm}{{\bf P}}
\newcommand{\Qm}{{\bf Q}}
\newcommand{\Rm}{{\bf R}}
\newcommand{\Sm}{{\bf S}}
\newcommand{\Tm}{{\bf T}}
\newcommand{\Um}{{\bf U}}
\newcommand{\Wm}{{\bf W}}
\newcommand{\Vm}{{\bf V}}
\newcommand{\Xm}{{\bf X}}
\newcommand{\Ym}{{\bf Y}}
\newcommand{\Zm}{{\bf Z}}

% Calligraphic

\newcommand{\Ac}{{\cal A}}
\newcommand{\Bc}{{\cal B}}
\newcommand{\Cc}{{\cal C}}
\newcommand{\Dc}{{\cal D}}
\newcommand{\Ec}{{\cal E}}
\newcommand{\Fc}{{\cal F}}
\newcommand{\Gc}{{\cal G}}
\newcommand{\Hc}{{\cal H}}
\newcommand{\Ic}{{\cal I}}
\newcommand{\Jc}{{\cal J}}
\newcommand{\Kc}{{\cal K}}
\newcommand{\Lc}{{\cal L}}
\newcommand{\Mc}{{\cal M}}
\newcommand{\Nc}{{\cal N}}
\newcommand{\nc}{{\cal n}}
\newcommand{\Oc}{{\cal O}}
\newcommand{\Pc}{{\cal P}}
\newcommand{\Qc}{{\cal Q}}
\newcommand{\Rc}{{\cal R}}
\newcommand{\Sc}{{\cal S}}
\newcommand{\Tc}{{\cal T}}
\newcommand{\Uc}{{\cal U}}
\newcommand{\Wc}{{\cal W}}
\newcommand{\Vc}{{\cal V}}
\newcommand{\Xc}{{\cal X}}
\newcommand{\Yc}{{\cal Y}}
\newcommand{\Zc}{{\cal Z}}

% Bold greek letters

\newcommand{\alphav}{\hbox{\boldmath$\alpha$}}
\newcommand{\betav}{\hbox{\boldmath$\beta$}}
\newcommand{\gammav}{\hbox{\boldmath$\gamma$}}
\newcommand{\deltav}{\hbox{\boldmath$\delta$}}
\newcommand{\etav}{\hbox{\boldmath$\eta$}}
\newcommand{\lambdav}{\hbox{\boldmath$\lambda$}}
\newcommand{\epsilonv}{\hbox{\boldmath$\epsilon$}}
\newcommand{\nuv}{\hbox{\boldmath$\nu$}}
\newcommand{\muv}{\hbox{\boldmath$\mu$}}
\newcommand{\zetav}{\hbox{\boldmath$\zeta$}}
\newcommand{\phiv}{\hbox{\boldmath$\phi$}}
\newcommand{\psiv}{\hbox{\boldmath$\psi$}}
\newcommand{\thetav}{\hbox{\boldmath$\theta$}}
\newcommand{\tauv}{\hbox{\boldmath$\tau$}}
\newcommand{\omegav}{\hbox{\boldmath$\omega$}}
\newcommand{\xiv}{\hbox{\boldmath$\xi$}}
\newcommand{\sigmav}{\hbox{\boldmath$\sigma$}}
\newcommand{\piv}{\hbox{\boldmath$\pi$}}
\newcommand{\rhov}{\hbox{\boldmath$\rho$}}
\newcommand{\upsilonv}{\hbox{\boldmath$\upsilon$}}

\newcommand{\Gammam}{\hbox{\boldmath$\Gamma$}}
\newcommand{\Lambdam}{\hbox{\boldmath$\Lambda$}}
\newcommand{\Deltam}{\hbox{\boldmath$\Delta$}}
\newcommand{\Sigmam}{\hbox{\boldmath$\Sigma$}}
\newcommand{\Phim}{\hbox{\boldmath$\Phi$}}
\newcommand{\Pim}{\hbox{\boldmath$\Pi$}}
\newcommand{\Psim}{\hbox{\boldmath$\Psi$}}
\newcommand{\Thetam}{\hbox{\boldmath$\Theta$}}
\newcommand{\Omegam}{\hbox{\boldmath$\Omega$}}
\newcommand{\Xim}{\hbox{\boldmath$\Xi$}}


% Sans Serif small case

\newcommand{\Gsf}{{\sf G}}

\newcommand{\asf}{{\sf a}}
\newcommand{\bsf}{{\sf b}}
\newcommand{\csf}{{\sf c}}
\newcommand{\dsf}{{\sf d}}
\newcommand{\esf}{{\sf e}}
\newcommand{\fsf}{{\sf f}}
\newcommand{\gsf}{{\sf g}}
\newcommand{\hsf}{{\sf h}}
\newcommand{\isf}{{\sf i}}
\newcommand{\jsf}{{\sf j}}
\newcommand{\ksf}{{\sf k}}
\newcommand{\lsf}{{\sf l}}
\newcommand{\msf}{{\sf m}}
\newcommand{\nsf}{{\sf n}}
\newcommand{\osf}{{\sf o}}
\newcommand{\psf}{{\sf p}}
\newcommand{\qsf}{{\sf q}}
\newcommand{\rsf}{{\sf r}}
\newcommand{\ssf}{{\sf s}}
\newcommand{\tsf}{{\sf t}}
\newcommand{\usf}{{\sf u}}
\newcommand{\wsf}{{\sf w}}
\newcommand{\vsf}{{\sf v}}
\newcommand{\xsf}{{\sf x}}
\newcommand{\ysf}{{\sf y}}
\newcommand{\zsf}{{\sf z}}


% mixed symbols

\newcommand{\sinc}{{\hbox{sinc}}}
\newcommand{\diag}{{\hbox{diag}}}
\renewcommand{\det}{{\hbox{det}}}
\newcommand{\trace}{{\hbox{tr}}}
\newcommand{\sign}{{\hbox{sign}}}
\renewcommand{\arg}{{\hbox{arg}}}
\newcommand{\var}{{\hbox{var}}}
\newcommand{\cov}{{\hbox{cov}}}
\newcommand{\Ei}{{\rm E}_{\rm i}}
\renewcommand{\Re}{{\rm Re}}
\renewcommand{\Im}{{\rm Im}}
\newcommand{\eqdef}{\stackrel{\Delta}{=}}
\newcommand{\defines}{{\,\,\stackrel{\scriptscriptstyle \bigtriangleup}{=}\,\,}}
\newcommand{\<}{\left\langle}
\renewcommand{\>}{\right\rangle}
\newcommand{\herm}{{\sf H}}
\newcommand{\trasp}{{\sf T}}
\newcommand{\transp}{{\sf T}}
\renewcommand{\vec}{{\rm vec}}
\newcommand{\Psf}{{\sf P}}
\newcommand{\SINR}{{\sf SINR}}
\newcommand{\SNR}{{\sf SNR}}
\newcommand{\MMSE}{{\sf MMSE}}
\newcommand{\REF}{{\RED [REF]}}

% Markov chain
\usepackage{stmaryrd} % for \mkv 
\newcommand{\mkv}{-\!\!\!\!\minuso\!\!\!\!-}

% Colors

\newcommand{\RED}{\color[rgb]{1.00,0.10,0.10}}
\newcommand{\BLUE}{\color[rgb]{0,0,0.90}}
\newcommand{\GREEN}{\color[rgb]{0,0.80,0.20}}

%%%%%%%%%%%%%%%%%%%%%%%%%%%%%%%%%%%%%%%%%%
\usepackage{hyperref}
\hypersetup{
    bookmarks=true,         % show bookmarks bar?
    unicode=false,          % non-Latin characters in AcrobatÕs bookmarks
    pdftoolbar=true,        % show AcrobatÕs toolbar?
    pdfmenubar=true,        % show AcrobatÕs menu?
    pdffitwindow=false,     % window fit to page when opened
    pdfstartview={FitH},    % fits the width of the page to the window
%    pdftitle={My title},    % title
%    pdfauthor={Author},     % author
%    pdfsubject={Subject},   % subject of the document
%    pdfcreator={Creator},   % creator of the document
%    pdfproducer={Producer}, % producer of the document
%    pdfkeywords={keyword1} {key2} {key3}, % list of keywords
    pdfnewwindow=true,      % links in new window
    colorlinks=true,       % false: boxed links; true: colored links
    linkcolor=red,          % color of internal links (change box color with linkbordercolor)
    citecolor=green,        % color of links to bibliography
    filecolor=blue,      % color of file links
    urlcolor=blue           % color of external links
}
%%%%%%%%%%%%%%%%%%%%%%%%%%%%%%%%%%%%%%%%%%%



\title{Machine Learning Should Maximize Welfare,  Not (Only) Accuracy}

\author{%
   Nir Rosenfeld\thanks{ 
  Faculty of Computer Science, 
  Technion -- Israel Institute of Technology, 
  \texttt{nirr@technion.ac.il}. }  
  \And Haifeng Xu\thanks{ 
  Department of Computer Science, 
  University of Chicago, 
  \texttt{haifengxu@uchicago.edu}. Haifeng is supported by   the AI2050 program at Schmidt Sciences (Grant G-24-66104), Army Research Office Award W911NF-23-1-0030 and NSF Award CCF-2303372.  } 
}


\begin{document}

% \twocolumn[ 
% \icmltitle{Machine Learning Should Maximize Welfare,
% Not (Only) Accuracy}

% % It is OKAY to include author information, even for blind
% % submissions: the style file will automatically remove it for you
% % unless you've provided the [accepted] option to the icml2025
% % package.

% % List of affiliations: The first argument should be a (short)
% % identifier you will use later to specify author affiliations
% % Academic affiliations should list Department, University, City, Region, Country
% % Industry affiliations should list Company, City, Region, Country

% % You can specify symbols, otherwise they are numbered in order.
% % Ideally, you should not use this facility. Affiliations will be numbered
% % in order of appearance and this is the preferred way.
% \icmlsetsymbol{equal}{*}

% \begin{icmlauthorlist}
% \icmlauthor{Firstname1 Lastname1}{equal,yyy}
% \icmlauthor{Firstname2 Lastname2}{equal,yyy,comp}
% \icmlauthor{Firstname3 Lastname3}{comp}
% \icmlauthor{Firstname4 Lastname4}{sch}
% \icmlauthor{Firstname5 Lastname5}{yyy}
% \icmlauthor{Firstname6 Lastname6}{sch,yyy,comp}
% \icmlauthor{Firstname7 Lastname7}{comp}
% %\icmlauthor{}{sch}
% \icmlauthor{Firstname8 Lastname8}{sch}
% \icmlauthor{Firstname8 Lastname8}{yyy,comp}
% %\icmlauthor{}{sch}
% %\icmlauthor{}{sch}
% \end{icmlauthorlist}

% \icmlaffiliation{yyy}{Department of Computer, University of YYY, Location, Country}
% \icmlaffiliation{comp}{Company Name, Location, Country}
% \icmlaffiliation{sch}{School of ZZZ, Institute of WWW, Location, Country}

% \icmlcorrespondingauthor{Firstname1 Lastname1}{first1.last1@xxx.edu}
% \icmlcorrespondingauthor{Firstname2 Lastname2}{first2.last2@www.uk}

% % You may provide any keywords that you
% % find helpful for describing your paper; these are used to populate
% % the "keywords" metadata in the PDF but will not be shown in the document
% \icmlkeywords{Machine Learning, ICML}

% \vskip 0.3in
% ]

% this must go after the closing bracket ] following \twocolumn[ ...

% This command actually creates the footnote in the first column
% listing the affiliations and the copyright notice.
% The command takes one argument, which is text to display at the start of the footnote.
% The \icmlEqualContribution command is standard text for equal contribution.
% Remove it (just {}) if you do not need this facility.

%\printAffiliationsAndNotice{}  % leave blank if no need to mention equal contribution
% \printAffiliationsAndNotice{\icmlEqualContribution} % otherwise use the standard text.

\maketitle


\begin{abstract}
\begin{abstract}  
Test time scaling is currently one of the most active research areas that shows promise after training time scaling has reached its limits.
Deep-thinking (DT) models are a class of recurrent models that can perform easy-to-hard generalization by assigning more compute to harder test samples.
However, due to their inability to determine the complexity of a test sample, DT models have to use a large amount of computation for both easy and hard test samples.
Excessive test time computation is wasteful and can cause the ``overthinking'' problem where more test time computation leads to worse results.
In this paper, we introduce a test time training method for determining the optimal amount of computation needed for each sample during test time.
We also propose Conv-LiGRU, a novel recurrent architecture for efficient and robust visual reasoning. 
Extensive experiments demonstrate that Conv-LiGRU is more stable than DT, effectively mitigates the ``overthinking'' phenomenon, and achieves superior accuracy.
\end{abstract}  
\end{abstract}

\section{Introduction}


\begin{figure}[t]
\centering
\includegraphics[width=0.6\columnwidth]{figures/evaluation_desiderata_V5.pdf}
\vspace{-0.5cm}
\caption{\systemName is a platform for conducting realistic evaluations of code LLMs, collecting human preferences of coding models with real users, real tasks, and in realistic environments, aimed at addressing the limitations of existing evaluations.
}
\label{fig:motivation}
\end{figure}

\begin{figure*}[t]
\centering
\includegraphics[width=\textwidth]{figures/system_design_v2.png}
\caption{We introduce \systemName, a VSCode extension to collect human preferences of code directly in a developer's IDE. \systemName enables developers to use code completions from various models. The system comprises a) the interface in the user's IDE which presents paired completions to users (left), b) a sampling strategy that picks model pairs to reduce latency (right, top), and c) a prompting scheme that allows diverse LLMs to perform code completions with high fidelity.
Users can select between the top completion (green box) using \texttt{tab} or the bottom completion (blue box) using \texttt{shift+tab}.}
\label{fig:overview}
\end{figure*}

As model capabilities improve, large language models (LLMs) are increasingly integrated into user environments and workflows.
For example, software developers code with AI in integrated developer environments (IDEs)~\citep{peng2023impact}, doctors rely on notes generated through ambient listening~\citep{oberst2024science}, and lawyers consider case evidence identified by electronic discovery systems~\citep{yang2024beyond}.
Increasing deployment of models in productivity tools demands evaluation that more closely reflects real-world circumstances~\citep{hutchinson2022evaluation, saxon2024benchmarks, kapoor2024ai}.
While newer benchmarks and live platforms incorporate human feedback to capture real-world usage, they almost exclusively focus on evaluating LLMs in chat conversations~\citep{zheng2023judging,dubois2023alpacafarm,chiang2024chatbot, kirk2024the}.
Model evaluation must move beyond chat-based interactions and into specialized user environments.



 

In this work, we focus on evaluating LLM-based coding assistants. 
Despite the popularity of these tools---millions of developers use Github Copilot~\citep{Copilot}---existing
evaluations of the coding capabilities of new models exhibit multiple limitations (Figure~\ref{fig:motivation}, bottom).
Traditional ML benchmarks evaluate LLM capabilities by measuring how well a model can complete static, interview-style coding tasks~\citep{chen2021evaluating,austin2021program,jain2024livecodebench, white2024livebench} and lack \emph{real users}. 
User studies recruit real users to evaluate the effectiveness of LLMs as coding assistants, but are often limited to simple programming tasks as opposed to \emph{real tasks}~\citep{vaithilingam2022expectation,ross2023programmer, mozannar2024realhumaneval}.
Recent efforts to collect human feedback such as Chatbot Arena~\citep{chiang2024chatbot} are still removed from a \emph{realistic environment}, resulting in users and data that deviate from typical software development processes.
We introduce \systemName to address these limitations (Figure~\ref{fig:motivation}, top), and we describe our three main contributions below.


\textbf{We deploy \systemName in-the-wild to collect human preferences on code.} 
\systemName is a Visual Studio Code extension, collecting preferences directly in a developer's IDE within their actual workflow (Figure~\ref{fig:overview}).
\systemName provides developers with code completions, akin to the type of support provided by Github Copilot~\citep{Copilot}. 
Over the past 3 months, \systemName has served over~\completions suggestions from 10 state-of-the-art LLMs, 
gathering \sampleCount~votes from \userCount~users.
To collect user preferences,
\systemName presents a novel interface that shows users paired code completions from two different LLMs, which are determined based on a sampling strategy that aims to 
mitigate latency while preserving coverage across model comparisons.
Additionally, we devise a prompting scheme that allows a diverse set of models to perform code completions with high fidelity.
See Section~\ref{sec:system} and Section~\ref{sec:deployment} for details about system design and deployment respectively.



\textbf{We construct a leaderboard of user preferences and find notable differences from existing static benchmarks and human preference leaderboards.}
In general, we observe that smaller models seem to overperform in static benchmarks compared to our leaderboard, while performance among larger models is mixed (Section~\ref{sec:leaderboard_calculation}).
We attribute these differences to the fact that \systemName is exposed to users and tasks that differ drastically from code evaluations in the past. 
Our data spans 103 programming languages and 24 natural languages as well as a variety of real-world applications and code structures, while static benchmarks tend to focus on a specific programming and natural language and task (e.g. coding competition problems).
Additionally, while all of \systemName interactions contain code contexts and the majority involve infilling tasks, a much smaller fraction of Chatbot Arena's coding tasks contain code context, with infilling tasks appearing even more rarely. 
We analyze our data in depth in Section~\ref{subsec:comparison}.



\textbf{We derive new insights into user preferences of code by analyzing \systemName's diverse and distinct data distribution.}
We compare user preferences across different stratifications of input data (e.g., common versus rare languages) and observe which affect observed preferences most (Section~\ref{sec:analysis}).
For example, while user preferences stay relatively consistent across various programming languages, they differ drastically between different task categories (e.g. frontend/backend versus algorithm design).
We also observe variations in user preference due to different features related to code structure 
(e.g., context length and completion patterns).
We open-source \systemName and release a curated subset of code contexts.
Altogether, our results highlight the necessity of model evaluation in realistic and domain-specific settings.






\section{Welfare economics and what it can teach us} \label{sec:welfecon}

Welfare economics is the subfield of economics that is concerned with the 
characterization, evaluation, and maximization of social welfare in economies and societies.
The main principles of welfare economics date back to \citet{smith1759moral,smith1776wealth},
but its formal foundations were laid out only a century later by
notable neoclassic economists such as
\citet{edgeworth1881psychics},
\citet{marshall1890principles},
\citet{pareto1906manual},
and \citet{pigou1920welfare}.
Modern welfare economists include influential figures such as
\citet{hicks1939value},
\citet{arrow1951social},
\citet{sen1970collective},
and \citet{stiglitz2012inequality}---all of which received the Nobel Prize for their contributions to this field.
% \todo{short list; noble laureates?}.
Given its rich history,
we believe that machine learning has much to gain from adopting ideas and perspectives
from this well-established discipline.

\subsection{Welfare economics: crash course}
\paragraph{The distribution of wealth.}
The main question welfare economics asks is:
how should wealth be distributed across individuals in the economy?
% Towards this,
Under the working hypothesis that resources (and therefore wealth) are limited,
the main object of interest in welfare economics is the \emph{Pareto front}---%
the set of all possible economic states in which no individual can be made better off without making things worse for another \citep[see, e.g.,][]{johansson1991introduction}.
In terms of welfare, there are two main considerations for policymakers:
\squeeze
\begin{enumerate}[leftmargin=1.7em,topsep=0em,itemsep=0.3em]
\item \textbf{Efficiency:}
How do we reach a Pareto state?

\item \textbf{Equity:}
Of all Pareto states, which are preferable? 
% Which Pareto states are preferable to others?
\end{enumerate}
\emph{Efficiency} requires an ability to optimize economic outcomes 
to a point where social benefit is maximal (in the Pareto sense).
Markets are a classic example of how the actions of many self-interested agents can combine to produce  efficient outcomes \citep{arrow1954existence}.
However, there are typically many states 
that are maximally beneficial,
% that are equally efficient,
% with high overall benefit,
% but nonetheless admit very different distributions 
but that differ in how benefits are distributed across individuals;
in regard to this, markets are mostly silent.
As such, \emph{equity} makes a statement about the relative preference ordering over all possible states, and considers means for steering towards preferable ones.
A canonical example is income distribution:
all governments likely seek higher overall income (efficiency),
but may disagree about whether high inequality should be permitted or suppressed
(equity).

\priority{%
\todo{add pareto curve graphic? general econ and/or acc-vs-welfare?}
}

%\hf{add 2-3 sentences to explain connection to fairness study, which misses people's reaction to predictions. Then point readers to appendix for more discussion. }

% adopt ideas from welfare economics:
% - pareto front - main object of interest
% 1. efficient
% 2. choose point


\paragraph{Social welfare functions.}
In welfare economics,
the primary tool for defining and promoting equity 
is the \emph{social welfare function},
which ranks or scores all possible economic states.
Our focus will be on the common choice of cardinal (i.e., real-valued) welfare functions that take the form of an expectation over weighted individual utilities:%
\footnote{The literature on social welfare functions is of course rich and diverse. Here we present a basic and simplified formulation which suits our purposes. Curious readers are referred to \citep{adler2019measuring} for a more thorough discussion of possible alternatives.}
% other possibilities in this space.}
\begin{equation}
\label{eq:welfare}
\welf(\policy;w) = \expect{(x,y)\sim \dist}{w(x) u(x,y;\policy)}
\end{equation}
% \hfr{
where $\policy$ is a \emph{policy}
(e.g., rules deciding which resources are allocated to whom)
% which individuals to accept or how much resource each individual gets)
% $\dist$ is the population distribution about individuals characterized by (feature,label) pairs $(x,y)$ 
and $u(x,y;\policy)$ is the utility of user $x$ under policy $\pi$, given $y$.
Notably, $w(x)$ is a weight function that the planner \emph{chooses}:
% as such, it provides a means to control
% the social welfare function
this defines
the desired direction for overall improvement (i.e., efficiency)
by balancing the importances of individual outcomes (i.e., equity).%
% Such weights have been adopted to prioritize certain subgroups \cite{björkegren2022machinelearningpoliciesvalue}  or balance different objectives \cite{rolf2020balancing}.
\footnote{More generally, weights $w(x)$ can even depend on the utility $u$ of $x$. This allows to express welfare using broader functional forms
(e.g., $\min$ instead of sum) or relative measures of inequality (e.g., Gini or Theil). For our purposes, the simpler form $w(x)$ suffices.}
% }
%Where \nir{...temporary notation, need to revise}. \hf{  $w(x)$ is determined by planner, not agent; agent decides $u$ utility. 
%In this way, maximizing welfare works towards improving overall outcomes (i.e., efficiency) while balancing individual outcomes (i.e., equity) as prescribed by weights $w(x)$. 
When the policy $\policy=\policy_h$ is guided by a predictive model $h$,
we will write $\welf(h)$ to mean $\welf(\policy_h)$.

 

\paragraph{The social planner.}
Equity is inherently a subjective notion that requires making value judgements.
Social welfare functions make it possible to formally express these by setting appropriate weights.
In welfare economics, weights are designated by a \emph{social planner}---either a real or fictitious entity that represents societal preferences.
A social planner can set weights to aid low income individuals;
implement affirmative action towards some social group;
or ensure that all individuals obtain some minimal level of utility.
Concrete examples include using weights to prioritize certain subgroups \citep{bjorkegren2022machinelearningpoliciesvalue} or balance different objectives \citep{rolf2020balancing}.
The simplest weighing scheme is of course using uniform weights, i.e., $w(x) = 1$ for any $x$.
But note even this makes a statement, which is that individuals should be weighted by their utility;
this is known as `utilitarian welfare'.
Hence, from the perspective of welfare economics, any objective that optimizes a (non-weighted) average---such as accuracy in machine learning---is in effect making a statement about how value should be distributed.

\todo{ref SWF examples in appendix?}

\paragraph{Human agency.}
% \hfr{
Welfare economics makes explicit the idea that individuals have \emph{agency}.
Intuitively, this states that individuals
(i) \emph{want} things,
(ii) \emph{know} things,
and (iii) \emph{act} -- to get what they want, using what they know.
These notions are formally accounted for by modeling
utility functions (want),
private information (know),
and decision-making, e.g., rational or behavioral (act).
Any policy that aims to advance welfare must take these into account.
Often this requires the planner to make additional efforts,
such as to elicit preferences,
create incentives for truthful reporting,
or infer how users will respond to different policy choices.
These are challenging, but give the planner power:
if incentives can be aligned, then it becomes possible
to harness the willingness of \emph{users} to invest effort for improving outcomes for all; consider public goods and services, crowdfunding platforms, open-source software, and collaborative knowledge bases.
\squeeze



% \todo{check flow now that connections sections moved to appx}

\subsection{Connections to machine learning.}
We believe that machine learning has much to gain from adopting a welfare perspective:
when inputs represent humans, it becomes 
possible to promote overall social benefit (efficiency),
and imperative to consider its distribution across the population (equity).
Current tools already push forward on these ideas, but only to a limited extent.
One reason is that standard learning objectives are notoriously underspecified;
this has implications on e.g. robustness \citep{d2022underspecification},
explainability \citep[see][]{rudin2019stop},
and fairness \citep[e.g.,][]{rodolfa2020case,coston2021characterizing,black2022model}.
% as some examples.
But underspecification also presents an opportunity for \emph{steering} outcomes toward socially beneficial states.
Consider how the idea of \emph{model multiplicity}
\citep{breiman2001statistical,marx2020predictive,hsu2022rashomon},
i.e., that there is typically a large set of (approximately) optimal models,
connects to the notion of an efficient Pareto front.%
\priority{since by definition, any deviation that increases accuracy for some users will necessarily come at the cost of reduced accuracy for others.}
The challenge lies in how to provide a social planner effective means to choose a preferred operating point.
In some cases, this will be diffucult;
in others, it may be achievable with tools as simple as regularization---if chosen appropriately \citep{levanon2021strategic}.
In terms of limited resources, machine learning already offers many 
relevant tools, such as constraints on cardinality (e.g., top-$k$ prediction)
or error rates (e.g., precision and recall).
Also relevant are tools from cost-sensitive \citep[e.g.,][]{elkan2001foundations}
and decision-focused learning \citep{mandi2024decision}.
But to be effective, these must be adapted to account for agency:
how users report information,
how they act in response to the learned classifier,
and how they interact with each other.
This idea will be key to our framework, as discussed next.
% \squeeze


% -- resources: [top-k]

% -- welfare: [acc objective]

% -- planner:
% if learning seeks to promote welfare, then it needs to confront these issues

% -- agency:
% Conventional learning frameworks have no notion of agency.
% This makes them unable to contend with the challenges that agency presents,
% nor to capitalize on the prospects it offers.
% This point will be key in our transition from welfare economics to machine learning,
% as we discuss next.

% -- Pareto: [multiplicity/rashomon]

% each individual's ``agency properties'' by accounting for: (a) what they \emph{want} through utility functions; (b) what they \emph{know} through modeling private information; and (c) what they \emph{do} through rational decision making.
% This perspective significantly enriches the planer's design choices such as eliciting private information from individuals, and using incentives or information to influence their behaviors. Meanwhile, it also necessitates the considerations of individual's incentives and responses to adopted policies.
% These novel aspects crucially differ from the static data distribution assumption in classic machine learning setups, which is appropriate for object recognition style of tasks. With the increasing demand of using ML technology to aid decisions and policies in human-involved contexts, it becomes important to integrate welfare economics into learning algorithm design, as we elaborate next.
% }

% Much of the power of machine learning stems from considering inputs as abstract features (e.g., vectors).
% In our setting, features $x$ represent actual human beings,
% and predictions $\hat{y}$ are made about them or for them.
% The crux is that humans are not your conventional input: they have goals, beliefs, and aspirations, and take action to promote their own self-interests.
% % In other words, humans are inputs that `behave'.
% A key aspect of our setting is that we will endow users with agency;
% intuitively, this means that we model users as acting on the basis of what they know or belief to pursue their objectives.
% Throughout we will mostly consider rational agents which act to maximize utility, but there is certainly room for other models of human behavior.
% Rational modeling is useful since it makes explicit the interests of users,
% as they compare to those of the learner, and possibly of society.
% As we will show, the question of incentives and how they relate has many implications on learning.

% \todo{define agency: want (utility, welfare), know (information, type, reporting, truthfulness), do (actions). want should also include externalities - negative (i want to be relatively better than others) and positive (i care about fairness and equity)}

% - social welfare function
% - social planner
% -- allows to monitor, audit, regulate, subsidize
  


% 1. learning objective underspecified
%   -- expectation (standard goal) vs. variation (=distribution of value)
% 2. social welfare function as weighted average (over utility from outcomes)


% \paragraph{Useful notions.}
% Because outcomes depend also on the interconnected behavior of users,
% reasoning about welfare may require notions beyond those in the standard machine learning toolkit.
% Luckily, many useful tools can be adopted from game theory, mechanism design, information design, and behavioral economics.
% These include:
% \begin{itemize}[leftmargin=1em,topsep=0em,itemsep=0.3em]
% \item
% equilibrium: various notions

% \item
% dynamics

% \item
% behavior: strategic/rational, boundedly-rational, non-rational (behavioral)
% \end{itemize}


% adopt notions from game theory:
% - strategic behavior; vs non-rational behavior
% - equilibrium
% - dynamics


% The field of machine learning spans many settings, formulations, and objectives.
% Our position paper focuses on the simplest and most fundamental of these,
% namely supervised discriminative learning:
% given training data of labeled pairs $(x,y)$ sampled iid from some unknown distribution,
% learn a function $f$ from a class $F$ whose predictions $\hat{y}=f(x)$ are accurate in expectation.
% This choice is for several reasons.
% First, although simple, supervised learning includes the fundamental considerations that pertain to all learning tasks: modeling, optimization, and statistics. This makes it appropriate as a first step towards welfare-aware learning.
% %  and the basis form more complex settings.
% Second, many other forms of learning are now based on predictive learning;
% this includes reinforcement learning (e.g., Q-learning, imitation learning (?)),
% generative learning (e.g., LLMs, diffusion models (?)),
% unsupervised learning (e.g., self-supervised learning),
% and even causal ([ref athey papers])
% and statistical inference ([ref zrnic papers]).
% Third, supervised learning is the hallmark of machine learning: 
% it is popular and well-known, the first to be taught, is in widespread use, and is highly supported in terms of software and code packages.
% Even if it is not the right tool---we argue that it is, and will likely continue to be, the main go-to approach used in practice.

% Even if our goal is to promote welfare, we will insist that accuracy remains an integral part of the learning objective.






% \section{Preliminaries}
\label{sec:prelim}
\label{sec:term}
We define the key terminologies used, primarily focusing on the hidden states (or activations) during the forward pass. 

\paragraph{Components in an attention layer.} We denote $\Res$ as the residual stream. We denote $\Val$ as Value (states), $\Qry$ as Query (states), and $\Key$ as Key (states) in one attention head. The \attlogit~represents the value before the softmax operation and can be understood as the inner product between  $\Qry$  and  $\Key$. We use \Attn~to denote the attention weights of applying the SoftMax function to \attlogit, and ``attention map'' to describe the visualization of the heat map of the attention weights. When referring to the \attlogit~from ``$\tokenB$'' to  ``$\tokenA$'', we indicate the inner product  $\langle\Qry(\tokenB), \Key(\tokenA)\rangle$, specifically the entry in the ``$\tokenB$'' row and ``$\tokenA$'' column of the attention map.

\paragraph{Logit lens.} We use the method of ``Logit Lens'' to interpret the hidden states and value states \citep{belrose2023eliciting}. We use \logit~to denote pre-SoftMax values of the next-token prediction for LLMs. Denote \readout~as the linear operator after the last layer of transformers that maps the hidden states to the \logit. 
The logit lens is defined as applying the readout matrix to residual or value states in middle layers. Through the logit lens, the transformed hidden states can be interpreted as their direct effect on the logits for next-token prediction. 

\paragraph{Terminologies in two-hop reasoning.} We refer to an input like “\Src$\to$\brga, \brgb$\to$\Ed” as a two-hop reasoning chain, or simply a chain. The source entity $\Src$ serves as the starting point or origin of the reasoning. The end entity $\Ed$ represents the endpoint or destination of the reasoning chain. The bridge entity $\Brg$ connects the source and end entities within the reasoning chain. We distinguish between two occurrences of $\Brg$: the bridge in the first premise is called $\brga$, while the bridge in the second premise that connects to $\Ed$ is called $\brgc$. Additionally, for any premise ``$\tokenA \to \tokenB$'', we define $\tokenA$ as the parent node and $\tokenB$ as the child node. Furthermore, if at the end of the sequence, the query token is ``$\tokenA$'', we define the chain ``$\tokenA \to \tokenB$, $\tokenB \to \tokenC$'' as the Target Chain, while all other chains present in the context are referred to as distraction chains. Figure~\ref{fig:data_illustration} provides an illustration of the terminologies.

\paragraph{Input format.}
Motivated by two-hop reasoning in real contexts, we consider input in the format $\bos, \text{context information}, \query, \answer$. A transformer model is trained to predict the correct $\answer$ given the query $\query$ and the context information. The context compromises of $K=5$ disjoint two-hop chains, each appearing once and containing two premises. Within the same chain, the relative order of two premises is fixed so that \Src$\to$\brga~always precedes \brgb$\to$\Ed. The orders of chains are randomly generated, and chains may interleave with each other. The labels for the entities are re-shuffled for every sequence, choosing from a vocabulary size $V=30$. Given the $\bos$ token, $K=5$ two-hop chains, \query, and the \answer~tokens, the total context length is $N=23$. Figure~\ref{fig:data_illustration} also illustrates the data format. 

\paragraph{Model structure and training.} We pre-train a three-layer transformer with a single head per layer. Unless otherwise specified, the model is trained using Adam for $10,000$ steps, achieving near-optimal prediction accuracy. Details are relegated to Appendix~\ref{app:sec_add_training_detail}.


% \RZ{Do we use source entity, target entity, and mediator entity? Or do we use original token, bridge token, end token?}





% \paragraph{Basic notations.} We use ... We use $\ve_i$ to denote one-hot vectors of which only the $i$-th entry equals one, and all other entries are zero. The dimension of $\ve_i$ are usually omitted and can be inferred from contexts. We use $\indicator\{\cdot\}$ to denote the indicator function.

% Let $V > 0$ be a fixed positive integer, and let $\vocab = [V] \defeq \{1, 2, \ldots, V\}$ be the vocabulary. A token $v \in \vocab$ is an integer in $[V]$ and the input studied in this paper is a sequence of tokens $s_{1:T} \defeq (s_1, s_2, \ldots, s_T) \in \vocab^T$ of length $T$. For any set $\mathcal{S}$, we use $\Delta(\mathcal{S})$ to denote the set of distributions over $\mathcal{S}$.

% % to a sequence of vectors $z_1, z_2, \ldots, z_T \in \real^{\dout}$ of dimension $\dout$ and length $T$.

% Let $\mU = [\vu_1, \vu_2, \ldots, \vu_V]^\transpose \in \real^{V\times d}$ denote the token embedding matrix, where the $i$-th row $\vu_i \in \real^d$ represents the $d$-dimensional embedding of token $i \in [V]$. Similarly, let $\mP = [\vp_1, \vp_2, \ldots, \vp_T]^\transpose \in \real^{T\times d}$ denote the positional embedding matrix, where the $i$-th row $\vp_i \in \real^d$ represents the $d$-dimensional embedding of position $i \in [T]$. Both $\mU$ and $\mP$ can be fixed or learnable.

% After receiving an input sequence of tokens $s_{1:T}$, a transformer will first process it using embedding matrices $\mU$ and $\mP$ to obtain a sequence of vectors $\mH = [\vh_1, \vh_2, \ldots, \vh_T] \in \real^{d\times T}$, where 
% \[
% \vh_i = \mU^\transpose\ve_{s_i} + \mP^\transpose\ve_{i} = \vu_{s_i} + \vp_i.
% \]

% We make the following definitions of basic operations in a transformer.

% \begin{definition}[Basic operations in transformers] 
% \label{defn:operators}
% Define the softmax function $\softmax(\cdot): \real^d \to \real^d$ over a vector $\vv \in \real^d$ as
% \[\softmax(\vv)_i = \frac{\exp(\vv_i)}{\sum_{j=1}^d \exp(\vv_j)} \]
% and define the softmax function $\softmax(\cdot): \real^{m\times n} \to \real^{m \times n}$ over a matrix $\mV \in \real^{m\times n}$ as a column-wise softmax operator. For a squared matrix $\mM \in \real^{m\times m}$, the causal mask operator $\mask(\cdot): \real^{m\times m} \to \real^{m\times m}$  is defined as $\mask(\mM)_{ij} = \mM_{ij}$ if $i \leq j$ and  $\mask(\mM)_{ij} = -\infty$ otherwise. For a vector $\vv \in \real^n$ where $n$ is the number of hidden neurons in a layer, we use $\layernorm(\cdot): \real^n \to \real^n$ to denote the layer normalization operator where
% \[
% \layernorm(\vv)_i = \frac{\vv_i-\mu}{\sigma}, \mu = \frac{1}{n}\sum_{j=1}^n \vv_j, \sigma = \sqrt{\frac{1}{n}\sum_{j=1}^n (\vv_j-\mu)^2}
% \]
% and use $\layernorm(\cdot): \real^{n\times m} \to \real^{n\times m}$ to denote the column-wise layer normalization on a matrix.
% We also use $\nonlin(\cdot)$ to denote element-wise nonlinearity such as $\relu(\cdot)$.
% \end{definition}

% The main components of a transformer are causal self-attention heads and MLP layers, which are defined as follows.

% \begin{definition}[Attentions and MLPs]
% \label{defn:attn_mlp} 
% A single-head causal self-attention $\attn(\mH;\mQ,\mK,\mV,\mO)$ parameterized by $\mQ,\mK,\mV \in \real^{{\dqkv\times \din}}$ and $\mO \in \real^{\dout\times\dqkv}$ maps an input matrix $\mH \in \real^{\din\times T}$ to
% \begin{align*}
% &\attn(\mH;\mQ,\mK,\mV,\mO) \\
% =&\mO\mV\layernorm(\mH)\softmax(\mask(\layernorm(\mH)^\transpose\mK^\transpose\mQ\layernorm(\mH))).
% \end{align*}
% Furthermore, a multi-head attention with $M$ heads parameterized by $\{(\mQ_m,\mK_m,\mV_m,\mO_m) \}_{m=1}^M$ is defined as 
% \begin{align*}
%     &\Attn(\mH; \{(\mQ_m,\mK_m,\mV_m,\mO_m) \}_{m\in[M]}) \\ =& \sum_{m=1}^M \attn(\mH;\mQ_m,\mK_m,\mV_m,\mO_m) \in \real^{\dout \times T}.
% \end{align*}
% An MLP layer $\mlp(\mH;\mW_1,\mW_2)$ parameterized by $\mW_1 \in \real^{\dhidden\times \din}$ and $\mW_2 \in \real^{\dout \times \dhidden}$ maps an input matrix $\mH = [\vh_1, \ldots, \vh_T] \in \real^{\din \times T}$ to
% \begin{align*}
%     &\mlp(\mH;\mW_1,\mW_2) = [\vy_1, \ldots, \vy_T], \\ \text{where } &\vy_i = \mW_2\nonlin(\mW_1\layernorm(\vh_i)), \forall i \in [T].
% \end{align*}

% \end{definition}

% In this paper, we assume $\din=\dout=d$ for all attention heads and MLPs to facilitate residual stream unless otherwise specified. Given \Cref{defn:operators,defn:attn_mlp}, we are now able to define a multi-layer transformer.

% \begin{definition}[Multi-layer transformers]
% \label{defn:transformer}
%     An $L$-layer transformer $\transformer(\cdot): \vocab^T \to \Delta(\vocab)$ parameterized by $\mP$, $\mU$, $\{(\mQ_m^{(l)},\mK_m^{(l)},\mV_m^{(l)},\mO_m^{(l)})\}_{m\in[M],l\in[L]}$,  $\{(\mW_1^{(l)},\mW_2^{(l)})\}_{l\in[L]}$ and $\Wreadout \in \real^{V \times d}$ receives a sequence of tokens $s_{1:T}$ as input and predict the next token by outputting a distribution over the vocabulary. The input is first mapped to embeddings $\mH = [\vh_1, \vh_2, \ldots, \vh_T] \in \real^{d\times T}$ by embedding matrices $\mP, \mU$ where 
%     \[
%     \vh_i = \mU^\transpose\ve_{s_i} + \mP^\transpose\ve_{i}, \forall i \in [T].
%     \]
%     For each layer $l \in [L]$, the output of layer $l$, $\mH^{(l)} \in \real^{d\times T}$, is obtained by 
%     \begin{align*}
%         &\mH^{(l)} =  \mH^{(l-1/2)} + \mlp(\mH^{(l-1/2)};\mW_1^{(l)},\mW_2^{(l)}), \\
%         & \mH^{(l-1/2)} = \mH^{(l-1)} + \\ & \quad \Attn(\mH^{(l-1)}; \{(\mQ_m^{(l)},\mK_m^{(l)},\mV_m^{(l)},\mO_m^{(l)}) \}_{m\in[M]}), 
%     \end{align*}
%     where the input $\mH^{(l-1)}$ is the output of the previous layer $l-1$ for $l > 1$ and the input of the first layer $\mH^{(0)} = \mH$. Finally, the output of the transformer is obtained by 
%     \begin{align*}
%         \transformer(s_{1:T}) = \softmax(\Wreadout\vh_T^{(L)})
%     \end{align*}
%     which is a $V$-dimensional vector after softmax representing a distribution over $\vocab$, and $\vh_T^{(L)}$ is the $T$-th column of the output of the last layer, $\mH^{(L)}$.
% \end{definition}



% For each token $v \in \vocab$, there is a corresponding $d_t$-dimensional token embedding vector $\embed(v) \in \mathbb{R}^{d_t}$. Assume the maximum length of the sequence studied in this paper does not exceed $T$. For each position $t \in [T]$, there is a corresponding positional embedding  







% \section{Proposed framework} \label{sec:framework}
\section{Three Orders of Welfare-Maximizing ML} \label{sec:orders}

%========================================================================
\begin{figure*}[t!]
\centering
\includegraphics[width=0.97\textwidth]{illust.v5-crop.pdf}
% \vspace{-2mm}
\caption{
\textbf{Proposed framework:} The three orders of welfare-maximizing machine learning.
}
\label{fig:illust}
\end{figure*}
%========================================================================


% We are now ready to present our proposed framework of welfare-maximizing machine learning and its three orders.

% \paragraph{From accuracy to welfare.}
% \paragraph{From maximizing accuracy to maximizing welfare.}

Our framework presents a hierarchy of problem formulations for welfare-maximizing learning,
organized into three `orders' of gradually increasing complexity.
The orders are built in a bottom-up fashion:
The lowest order (Order 0)
coincides with the standard objective of maximizing predictive accuracy,
but provides a novel welfare perspective suited for social tasks.
Each higher order then builds on and generalizes the one below it,
this by adding to the objective another layer of economic complexity:
first focusing on system decisions (Order 1),
and then on user choices (Order 2).
As such, our framework is based on accuracy maximization at its core,
but enables---and in fact requires---to explicitly model resources, allocations, and agency.
It also requires to specify the role predictions play in shaping social outcomes.
An illustration of the framework and its orders is given in Fig.~\ref{fig:illust}.
\squeeze

We begin with a welfare interpretation of supervised learning,
and then proceed to discuss the three framework orders.
Further prospects and challenges are discussed in Appx.~\ref{appx:beyond}.
\squeeze

\todo{agency: orders 0+1 - info/report, order 2 - actions}

% \squeeze
% \begin{itemize}[leftmargin=1em,topsep=0em,itemsep=0.3em]
% \item
% \textbf{\underline{Order 0}: Accuracy as a resource.}
% When prediction is provided as a service \emph{for humans},
% and when those humans benefit from accurate predictions,
% accuracy itself becomes a scarce resource
% that requires careful allocation.
% Examples include
% medical diagnosis, financial risk prediction, career advice,
% and personalized recommendations.

% \item
% \textbf{\underline{Order 1}: Predictions as decision aids}.
% Predictions can help inform decisions made \emph{about us}.
% But when decisions must be made collectively about a group of people, 
% this often entails constraints on individual decisions.
% Examples include resume screening, school admissions,
% loans approval, insurance claims, and medical program eligibility.
% \squeeze

% \item
% \textbf{\underline{Order 2}: Predictions that empower choices.}
% A primary use of predictions in social settings is to facilitate and improve the choices made \emph{by us}.
% This makes us, in some sense, a finite resource over which platforms,
% service providers, sellers, and content creators contend.
% Examples include recommendation systems,
% online market platforms,
% and job applications and hiring.

% \end{itemize}


\paragraph{Supervised learning from a welfare perspective.}
Given labeled training data of pairs $(x,y)$ sampled iid from an unknown distribution $\dist$,
the goal in supervised learning is to find a function $h$ from a chosen class $H$ whose predictions $\hat{y}=h(x)$ are accurate
on future examples in expectation:
\begin{equation}
\label{eq:accuracy_objective}
\argmax\nolimits_{h \in H} \expect{\dist}{\one{y = h(x)}}
\end{equation}
Since in our setting examples $(x,y)$ are associated with individuals
% \hf{May need to think a bit about terminology, shall we call user or agent?}
in the population $\dist$,
then from a welfare perspective,
we can think of Eq.~\eqref{eq:accuracy_objective}
as prescribing a particular social welfare function---%
a special case of Eq.~\eqref{eq:welfare}---%
% then from the welfare perspective of Eq.~\eqref{eq:welfare}, 
% what Eq.~\eqref{eq:accuracy_objective} is advancing is a welfare function
in which:
% \niradd{
(i) utility to users derives from accurate predictions,
(ii) all users share the same utility function,
$u(x,y;\policy_h) = \one{y = h(x)}$,
(iii) possible outcomes include correct and incorrect predictions,
% and so depend directly on the classifier, ,
hence (iv) the policy is degenerate, $u(x,y;\policy_h)=u(x,y;h)$),
and finally (v) weights are uniform, $w(x) \equiv 1$.
% }
% (i) all individuals are assumed to share the same utility $u(x,y,\policy) = \one{y = h(x)}$; (ii) each individual's utility is independent of policy $\pi$ or outcomes,
% % $u(x)=\one{y = h(x)}$, 
% and (iii) weights $w(x) \equiv 1$ are uniform. 
\squeeze


\paragraph{From accuracy to welfare maximization.}
Despite an apparent `neutrality', 
the above reveals that
accuracy maximization in fact makes an (implicit)
statement about social preferences.
This entails a particular notion of equity---%
one that derives operationally from the predictive task at hand,
rather than from a planned or designated social goal.
But learning is of course not bound to maximizing objectives only of the form of Eq.~\eqref{eq:accuracy_objective}.
If our true goal is to maximize welfare, then we should modify the objective to support this goal.
For example, if users differ in how much they benefit from accurate predictions,
then we can plug in an appropriate utility function $u$;
or if there are exogenous constraints on predictions,
then we can express these via the policy $\policy_h$.
Such steps are certainly possible, but introduce challenges beyond those
found in standard machine learning tasks.
% The three orders of our framework, presented next,
Towards this, we pave a path that
gradually transforms Eq.~\eqref{eq:accuracy_objective} to support increasingly complex economic considerations.
% \squeeze


% This reveals two limitations of applying conventional learning tools in societal settings.
% First, despite an apparent `neutrality', 
% accuracy maximization in fact makes an (implicit)
% statement about social preferences.
% This entails a particular notion of equity---%
% one that derives operationally from the predictive task at hand,
% rather than from a planned or designated social goal.
% % easier to pred are up-weighted
% Second, while useful for common prediction tasks over `static' inputs 
% (e.g., object recognition in images),
% accuracy alone
% % \hfr{%
% % While this classic formulation is natural for ML tasks such as object recognitions, it 
% fails to capture the `active' nature of users:
% their autonomy, preferences, and interactions.
% This means that even if maximizing the objective in
% Eq.~\eqref{eq:accuracy_objective}
% is what we want,
% it will not necessarily be what we will get.
% % This renders it inadequate for guaranteeing promoting welfare.
% % welfare-maximizing machine learning. 
% % }

% \paragraph{From accuracy to welfare maximization.}
% Learning is of course not bound to maximizing objectives only of the form of Eq.~\eqref{eq:accuracy_objective}.
% If our true goal is to maximize welfare, then we should modify the objective to support this goal.
% For example, if users differ in how much they benefit from accurate predictions,
% then we can plug in an appropriate utility function $u$;
% or if there are exogenous constraints on predictions,
% then we can express these via the policy $\policy_h$.
% Such steps are certainly possible, but introduce challenges beyond those
% found in standard machine learning tasks.
% The three orders of our framework, presented next,
% gradually transform Eq.~\eqref{eq:accuracy_objective} to support increasingly complex economic settings.
% \squeeze

% \hfr{In this section, we describe a general   framework with three orders of increasing richness that significantly generalizes the above classic framework in order to cast various learning tasks in societal and economic setups. Most learning tasks in these setups exhibit challenges of these different orders. For concreteness, we illustrate these orders using an example of making purchase recommendations on digital markets such as Amazon, Wayfair, etc.; more application examples can be found in Section \ref{sec:agenda} when we elaborate on research tasks.   }


% \hf{I ended up using "making purchase recommendations on digital markets" as an illustrative domain, since I think job hiring via ML is a less mature application at this moment (most hires are arguably still through human decisions). }

\priority{%
\todo{consider adding table of orders with resources, agency (want, know, do), decisions, policy, outcomes, solution concepts (stack, stack-nash, nash-stack), etc.}
}

% \todo{add notation - user value/utility $v_h(x)$ ($=v(x;\yhat)$?)}  


\subsection{Order 0: Allocating Accuracy} \label{sec:order0}
% Accuracy is beneficial to users in any setting where predictions are given to them as a service.
% \hfr{
Accuracy will be beneficial to users in any a platform or service
% that are driven by predictions; 
that relies on predictions;
consider recommendation systems, online market platforms,
financial aid tools, or diagnistic health services.
% examples include various prediction-driven recommendations for products, contents, financial service, etc.
% } 
% ; consider \todo{examples}.
We argue that in such cases,
and as long as
learning is prone to some level of error
% perfect accuracy is unattainable
(which is plausible), then
\emph{accuracy itself is a limited resource}---%
simply because not all users can obtain
the same level of accuracy.%
\extended{But since there can be many models with similarly high overall accuracy but that differ in who they are accurate on,
these create points along the \emph{Pareto frontier}---since by definition, 
any deviation that increases accuracy for some users will necessarily come at the cost of reduced accuracy for others.}
% \hfr{the same level of  prediction accuracies  --- on the Pareto frontier of the (imperfect) prediction accuracy for different users, the accuracy increase for one group of users   comes at the cost of decreasing the accuracy   of another group.} 
% accurate predictions.
% In this sense,
Thus, the choice of classifier becomes, in effect, a decision on which users will be `allocated' more accurate predictions, and which less. %will not.
This then begs the question:
who \emph{should} be allocated an accurate prediction?
% \hf{an accurate, or a more accurate? I guess no one will be perfectly accurate?} 
% \nir{i think this is a question of classification vs regression. the loss we considered so far is the 0-1 loss, so the discussion is geared towards classification. but i guess the point holds also more generally for regression.}
One answer is users who most need it, or who benefit mostly.
In the reasonable case where there is natural variation in individual utilities,
we can to optimize a utility-weighted accuracy objective:
% \hf{a potential issue is that this cannot capture $v$'s dependence on $y$...}
% \squeeze
\begin{equation}
\label{eq:order0_objective}
\argmax\nolimits_{h \in H} \expect{\dist}{u(x,y; \hat{y}) \cdotp \one{y = h(x)}}
\end{equation}
% \nir{should $u$ depend also on $y$? otherwise it can't be a function of the prediction being correct}
where $u(x,y; \hat{y})$ is the utility for user $x$ given prediction $\hat{y}$.%
\priority{
\footnote{\hfr{This modeling assumes that $x, \hat{y}$ fully captures the individual's preference. More generally, if there are factors that cannot be captured by these two, a more general model of individual's utility function  $u(x; \hat{y}) + \epsilon(x)$ for some noise term $\epsilon$  (possibly depending on $x$) that captures hidden factors' effects.} }
}
%\hf{plan is to change $v(x)$ to $u(x; \hat{y})$. More generally, we can model $u$ as $u(x; \hat{y}) + \epsilon(x)$ to capture un-modeled components. }
This transforms Eq.~\eqref{eq:accuracy_objective} into an explicit utilitarian welfare objective in which individualized utilities derive from prediction correctness:
$u(x;\yhat)$ if $\yhat=y$, and zero if not.
Conditioning $u$ also on predictions $\yhat$ enables to express utilities that depend on predictions themselves (beyond their correctness);
for example, if users benefit more from positive predictions,
as in loans, hiring, admissions, etc.
% and takes one step towards Eq.~\eqref{eq:welfare} by incorporating individual importance weights.

\paragraph{Agency.}
While apparently simple, optimizing Eq.~\eqref{eq:order0_objective} poses significant challenges
\emph{once we accept that users have agency}.
If users seek to promote their own interests, and have control of information that is important to the system, then their choices regarding how and what to report can significantly affect learning outcomes.
For example, consider that if we simply ask people to report their utilities,
then each individual will be incentivized to report the highest possible value,
making all reports completely uninformative.
Another example is that users may be prone to manipulate their features if this helps them secure favorable predictions,
as is the case of strategic classification \citep{hardt2016strategic}.
% Another example is that, even if we have $v$,
% users can still misreport or manipulate their $x$ to influence predictions in their favor \tocitec{sc?}.
% \blue{We discuss some ideas for how such problems can be reconciled 
% in Sec.~\toref.}

\paragraph{Challenge: Optimizing welfare objectives.}
Maximizing welfare through learning presents new algorithmic challenges.
One task is to identify learners that optimize a given social welfare function---%
intrinsically, a question of identifying a desirable point on the Pareto front.
Another task is to generate \emph{all} points on the frontier \citep[see, e.g.,][]{navon2021learning}.
One possible approach is to cast
Eq.~\eqref{eq:order0_objective} as a problem of distribution shift.
Here shift can stem from several factors, including
the welfare function weights,
user utilities, and user inputs.
But agency means that utilities can be misreported and features manipulated;
as a result, the way in which the distribution shifts becomes
\emph{dependent on the choice of classifier},
indirectly through how users respond.
This makes Eq.~\eqref{eq:order0_objective} a challenging instance of
\emph{model-dependent distribution shift} \citep{drusvyatskiy2023stochastic}.
Here the interaction between the system and its users 
can be modeled as a Stackelberg game \citep{chen2020learning}.
% Some advances have been made on this front,
This perspective has been useful for questions on learnability,
e.g. via generalization bounds
that rely on \emph{strategic VC} analysis
\citep{zhang2021incentive,sundaram2023pac}
or that accommodate feature-dependent utilities \citep{pardeshi2024learning}.
Nonetheless, many important open questions remain.
\squeeze

% Even if we have faithful access to user utilities,
% the algorithmic question of how to optimize welfare in practice remains.




% \paragraph{Challenge \#2: Learnability and generalization.}


% \paragraph{Challenge \#1: the optimization problem -- which learning algorithm optimizes a given welfare objective?} It is widely observed that there often exist multiple predictors that all achieve optimal or close-to-optimal accuracy. Indeed, this phenomenon has been widely known as \emph{model multiplicity} \cite{black2022model,d2022underspecification} or ``Rashomon effect'' \cite{semenova2022existence,rudin2024amazing},  coined by Leo Breiman. \cite{d2022underspecification} specifically attributes this phenomenon to underspecification of the learning pipeline.
% \nir{i added model multiplicity/underspecification to the end of sec 2, so probably safe to reference these ideas more loosely here}
% A natural question thus is, among the predictors that have high accuracies, which one would maximize a given welfare function? This new challenge  is intrinsically about identifying algorithms that  achieve  the Pareto frontier   concerning various parties' utilities, since the optimal solution under any welfare function must be at the Pareto frontier and conversely, any point at the frontier   must correspond to the optimal solution of some welfare functions.   %  by allowing varied  utility functions, the performance distinction will almost necessarily show up. This hence raises the important machine learning question  of studying which predictor is  best-suited for which welfare objective  and, sometimes, the need of designing new predictors that are optimal under new objectives.  

% \paragraph{Challenge 2: the statistical question -- how does agency change learnability?} The second fundamental question is how classic learnability theory established based on accuracy maximization (or risk minimization) will change under the new welfare objective. Two key factors will affect the development of a new theory of learnability for welfare-maximizing ML: (1) users' agency behaviors such as strategically disguising or altering their private data from the learner; (2) the shift from equal-weights of each data point  to varied and possibly feature-dependent utilities in the learning objective. Recent study of \cite{sundaram2023pac} developed a generalized theory for Probabilistic Approximately Correct (PAC) learning  to address factor (1) assuming known user utility function, whereas \cite{pardeshi2024learning} addresses factor (2) for a particular class of welfare functions called weighted power mean functions. More general welfare functions as well as the combined situation with presence of both factor (1) and (2) have not been examined thus far. Among others, one factor that  will affect learning is the variance among different users' utility values. \hf{add some citations here about similar result on effective sample size that depends on the variance of propensity weights.  }  This also raises an interesting societal question of how inequality could affect welfare-maximizing learning. 



% \paragraph{Challenge \#1: the optimization problem -- which learning algorithm optimizes a given welfare objective?} It is widely observed that there often exist multiple predictors that all achieve optimal or close-to-optimal accuracy. Indeed, this phenomenon has been widely known as \emph{model multiplicity} \cite{black2022model,d2022underspecification} or ``Rashomon effect'' \cite{semenova2022existence,rudin2024amazing},  coined by Leo Breiman. \cite{d2022underspecification} specifically attributes this phenomenon to underspecification of the learning pipeline.
% \nir{i added model multiplicity/underspecification to the end of sec 2, so probably safe to reference these ideas more loosely here}
% A natural question thus is, among the predictors that have high accuracies, which one would maximize a given welfare function? This new challenge  is intrinsically about identifying algorithms that  achieve  the Pareto frontier   concerning various parties' utilities, since the optimal solution under any welfare function must be at the Pareto frontier and conversely, any point at the frontier   must correspond to the optimal solution of some welfare functions.   %  by allowing varied  utility functions, the performance distinction will almost necessarily show up. This hence raises the important machine learning question  of studying which predictor is  best-suited for which welfare objective  and, sometimes, the need of designing new predictors that are optimal under new objectives.  

% \paragraph{Challenge 2: the statistical question -- how does agency change learnability?} The second fundamental question is how classic learnability theory established based on accuracy maximization (or risk minimization) will change under the new welfare objective. Two key factors will affect the development of a new theory of learnability for welfare-maximizing ML: (1) users' agency behaviors such as strategically disguising or altering their private data from the learner; (2) the shift from equal-weights of each data point  to varied and possibly feature-dependent utilities in the learning objective. Recent study of \cite{sundaram2023pac} developed a generalized theory for Probabilistic Approximately Correct (PAC) learning  to address factor (1) assuming known user utility function, whereas \cite{pardeshi2024learning} addresses factor (2) for a particular class of welfare functions called weighted power mean functions. More general welfare functions as well as the combined situation with presence of both factor (1) and (2) have not been examined thus far. Among others, one factor that  will affect learning is the variance among different users' utility values. \hf{add some citations here about similar result on effective sample size that depends on the variance of propensity weights.  }  This also raises an interesting societal question of how inequality could affect welfare-maximizing learning. 



% \hf{Does agency only mean response and report? Agency inlcudes   (want, know, do):  information witheheld, action taken, welfare/utility (including potential  externality to capture fairness, envy)   }
 
% This is a direct result of the fact that \emph{users have agency}---a notion which is typically conveniently abstracted away.
% One possible solution is to augment the learning objective with an appropriate mechanism for eliciting true valuations.
% We present this idea and other alternatives in Sec.~\ref{sec:order1}.
% % \squeeze
 
\subsection{Order 1: Incorporating System Decisions} \label{sec:order1}
Generally, predictions are useful for the system if they aid in making better decisions about uncertain outcomes. 
Since decisions in social settings often directly prescribe how to allocate limited resources to users,
a useful next step is to encode them explicitly into the objective.
We formalize this idea by modeling a system that makes decisions about users
(e.g., who to hire)
through predictions (e.g., resume screening).
Consider a user $x$ with label $y$, and denote by $a$ the action the system takes (e.g., hire or not).
Let $\reward(a,y)$ be the reward for the system on this user given action $a$
% \hf{maybe unify the reference to "decisions". }
(e.g., the quality of the candidate, if hired).
The direct dependence of $\reward$ on $y$ means that if we know $y$, then we can write the optimal action as $a^*=\policy(y)$ for some policy $\policy$.
Since $y$ is generally unknown, the common approach is to replace $y$ in $\policy$ with a prediction $\yhat = h(x)$,
denoted $a = \policy(h(x)) = \policy_h(x)$,
in hopes that better predictions translate to better decisions. %
\priority{\footnote{An alternative approach is to directly learn the mapping from $x$ to $a$, which also known as decision-focused learning \citep{wilder2019melding,kotary2021end,dontidc3}. However, this approach has thus far been mostly studying learning optimization decisions without any consideration of agency and welfare effects.}}
Such policies,
referred to as \emph{prediction policies} \citep{kleinberg2015prediction},
are appropriate when the uncertainty in $y$ is a stronger factor for outcomes than potential causal effects (we discuss this distinction further in Appendix~\ref{appx:causal}).
In this setting, 
choosing to work with prediction policies
gives reason for the system to optimize $h$ for accuracy (Eq.~\eqref{eq:accuracy_objective}).
\squeeze
% Such policies are referred to as a .
% \hf{emphasize this more, as this is subtle yet importantly different from our method. } 

% even though deeper thinking reveals that this may not be the right thing to proceed.    

% \hf{need to explain why we separate $h, \pi$ though they could have been comibned. Causal effect (pi may affect $y$), and practically done like this }

\paragraph{From predictions to actions.}
We consider cases where there is a global restriction on the set of all actions $a$. These can express, for example,
a limited number of available jobs (via cardinality constraints),
% ($\sum_i a_i \le k, \, a_i \in \{0,1\}$),
a total sum of funds that a bank can lend (knapsack constraints),
% ($\sum_i a_i \le K, \, a_i \in \R_+$),
regulation on the amount of financial risk an insurer can take (bounded expected risk),
or a limit on the number of posts a social platform can block to still enable free speech (lower-bounded rates).
% or a lower bound on the number of posts a social platform must allow to enable free speech ($\sum_i a_i \ge k$).
Given a set $A$ of feasible actions,
the objective can be written as:
\begin{equation}
\label{eq:order1_objective}
\argmax\nolimits_{h \in H} \expect{\dist}{\reward(\policy_h(x),y)}
\quad \text{s.t.} \quad
\policy_h(\dist) \in A
\end{equation}
where $\policy_h(\dist)$ is the set of actions over the entire population.  



\paragraph{Agency.}
Machine learning has many tools for coping with constraints.
For example, if there is a quota on the number of available interview slots,
then top-$k$ classification \citep[e.g.][]{lapin2015top,petersen2022differentiable} or ranking \citep[e.g.][]{cao2007learning}
% or learning-to-rank algorithms 
seem like adequate solutions.
The crux, however, is that such methods do not account for agency:
since users seek to be `allocated' (e.g., get the interviews),
they will likely try to present themselves as relevant
(e.g. by tweaking their resume).
% widely recognized and studied in economics---but not much in machine learning.
% But has not been carefully investigated in learning algorithm design.  
% In effect, user agency combined with limited resources creates competition between users---which learning must account for in order to achieve its goals.
% In Sec.~\toref\ we discuss how such challenges manifest in the learning objective, and what can be done about it.
A key point is that since users contend over resources,
this introduces dependencies into their actions and subsequent outcomes.
Such behaviors are ubiquitous in economics
(e.g., costly signaling, adverse selection),
but have been underexplored in machine learning.
% Such 
% research competition
% behavior, known as ``adverse selection'' \cite{cohen2010testing,greenwald2018adverse}, 
% in machine learning, it has yet to receive the appropriate attention it requires.
% To achieve its goals, learning must account for these.
\squeeze

% \nir{revise:}
% A key point is that in either case,
% users will have incentive to steer outcomes in their favor,
% which they can achieve for example by manipulating their features $x$ to influence predictions \tocitec{sc?}.
% \blue{In Sec.~\ref{sec:order1} we discuss the challenges that arise when
% such behavior interacts with the limitation on resources,
% and how to approach this.} 

% - top-k loss and why it doesn't suffice (but still helps and should be built on)

\paragraph{Social welfare.}
In terms of welfare outcomes,
we make a distinction between two settings of interest:
\begin{itemize}[leftmargin=1em,topsep=0em,itemsep=0.2em]
\item
An \emph{aligned} setting in which $\reward=\welf$,
and so the interests of the system align with societal preferences,
e.g., as for government and nonprofit organizations. % organizations, and international agencies).
Note Eq.~\eqref{eq:order0_objective} becomes a special case of
Eq.~\eqref{eq:order1_objective} when utility derives from accuracy
and $A$ does not impose restrictions. 

\item
A \emph{misaligned} setting in which $\reward$ can be at odds with welfare,
for example if it concerns revenue or user engagement,
as is more common in commercial settings.
Here a social planner is needed to incentivize the learner to account for welfare outcomes, as we detail in Sec.~\ref{sec:regulation}.
\end{itemize}


% - accuracy -- implicit? as constraint? weighted in objective? \\
% - users can change x \\
% - welfare, valuations


% \subsection{Challenges at Order 1 } 






% Under Order 1, the learner looks to optimize   decisions, often subject to resource constraints. Learning such constrained decisions   differs from   directly learning the welfare function itself as at Order 0.  While each individual have similar importance in the welfare function, the sensitivity of  the welfare-maximizing   decision boundary to each individual can differ dramatically. Consider the , the real problem here is actually not just an accuracy maximization problem as --- instead, it is actually a problem at Order 1.\footnote{In fact, it could even be placed to Order 2 since students in turn have their own choices after getting offers. However, such choices have been less considered during university admissions due to relatively stable acceptance rate.}  This is because each university has its own admission limit each year, hence their true task is not to assign an uncorrelated binary label to acceptance/reject to every applicant as modeled by strategic classification, but rather to pick the best possible $k$ candidates for some admission quota $k$.  Hence as widely observed in practice, most of the effort spent on  deciding admissions are often those boundary applicants, to whom the optimal decisions are significantly more sensitive compared applicants who are at very top or bottom. Such  unequal influence of individuals to the optimal decision also show up in other applications such as approving loan applicants where the decisions are even more complex since it is not only about approving a loan or not but rather about how much loan amount (possibly $0$, meaning declining) to approve for each individual, typically  subject to total  available loan amount. Very little previous studies have gone beyond  classification or regression to study such Order-1   questions of optimally learning the welfare-maximizing decisions.   



% \paragraph{Challenge \#1: Decisions under limited resources correlate user behavior.}  
\paragraph{Challenge \#1: Decisions and externalities.}  
At Order 0, user responses are generally made independently,
and so the learning objective decomposes over examples.%
\footnote{Note this is a key assumption in standard strategic classification.}
But once the system makes decisions under constrained resources,
user behavior can become intricately dependent.
Consider for example admissions under a limited quota:
here applicants cannot simply pass a fixed acceptance threshold,
but instead, must surpass other applicants---i.e., the bar adjusts according to competition.
The tension for both decision-makers and applicants tends
to focus on `marginal' students;
complex gaming behavior among such candidates is well-documented in 
the education literature \citep{bound2009playing}.
In economics, inter-user dependencies are known as \emph{externalities}.
One way to model how learning creates externalities and coordinates user behavior
is by casting the problem as a Stackelbeg-Nash game \citep{liu1998stackelberg},
where the system plays first, and users respond collectively by playing an induced simultaneous game (with externalities).
Not much work has studied
learning in this challenging setting.
% \squeeze
\priority{scext, scgnn; recdiv?}

% \paragraph{Challenge \#1: Complex inter-dependent agency behaviors under resource-constrained decisions.}  
% To see how learning differs at Order 0 and Order 1, let us illustrate with the problem of learning university
% admission decisions, which is the introductory example that
% populates the study of strategic classification \cite{hardt2016strategic}. Previous works treat it as an Order-0 problem, under which there is no admission limit hence each individual responds independently to a university's admission rules. However, this problem is in fact at  Order 1 since admissions in practice are constrained decisions with  a limited quota; consequently, a student's strategic behavior is   not  simply to cross the classification boundary as modeled  in previous works, but rather to surpass other similar peers in order to get admitted, introducing \emph{externalities} among students, particularly those at admission thresholds known as ``marginal students'' in the literature (complicate gaming among marginal students are well-documented in education research literature  \cite{bound2009playing}). Such inter-dependent agent actions lead to a  \emph{Stackelberg-Nash} style game.  Similar issues happen in loan applications where limited loan opportunities also introduce externality among applicants who   compete for limited loan resources while not merely for passing the bar. Little previous studies have considered  such Order-1   questions of  learning the welfare-maximizing decisions.   

% bound2009playing


\paragraph{Challenge \#2:  Misaligned system and user objectives.}  
As we note, generally we cannot expect the reward for the system to align with user interests.
If no external forces intervene, 
% then the system faces a
% Stackelberg-Nash game with a single leader and many followers.
then the challenge is to 
% use samples of past system-user interactions to
learn a decision policy that is effective at the Stackelberg-Nash equilibrium.
This necessitates learning objectives that can express and anticipate
how users will collectively respond to a deployed model.
Contrarily, if there is regulation (e.g., on welfare or risk),
then the system's decision space becomes much more limited.
An interesting question then is \emph{how} to align incentives through regulation.
For example, would requiring the system to optimize a weighted sum of accuracy and welfare help to promote equity? And if so, which weights, and by what means?
\squeeze

% \paragraph{Challenge \#2:  Misaligned system and user objectives.}  
% %The third fundamental challenge that exhibits under Order 1 is the mis-aligned objectives between learner and users. While in some situations, agents and the system have aligned objectives (e.g., high education potential as in school admission), in others their objectives can be different and sometimes conflicting. For instance, during loan application, the system   tends to maximize profit subject to loan availability, whereas applicants maximizes the amount of approved loan. 
% There are two situations concerning misaligned objectives.
% First, if no external force intervenes in the system, the system then faces a Stackelberg-Nash game with a single leader and many followers. The challenge is to effectively  learn the optimal system equilibrium strategy from sampled system-user interactions. Second, if there are regulations that impose risk   and welfare controls, this often helps to mitigate the conflicting system-individual interest. This   challenges   the social planner to learn  effective regulation policies.  For example, would enforcing the system to optimize a weighted sum of revenue and welfare help to achieve better equity, and if so which weighing scheme is the best?  

 

 
\subsection{Order 2: Enabling User Choices} \label{sec:order2}
Even if a system has the capacity to make decisions about users,
% outcomes are typically also the product of user choices.
often those very users will also have some say regarding final outcomes.
% In fact, 
% Even in straightforward 
For example, even in hiring and admissions,
candidates much first \emph{choose} to apply, and if selected, \emph{choose} to accept.
Thus, both reward for the system and utility to users depend on outcomes that result from the interplay of system decisions and user choices.
Similarly to system decisions, we will model user choices as also depending on predictions, or more generally on the learned model $h$.
Let $\choice_{\policy_h}(x)$ be the choice of user $x$ in response to $h$.
Incorporating user choices into the system's objective gives:
\squeeze
% \begin{align}
% \label{eq:order2_objective}
% \argmax_{h \in H} & \expect{\dist}{\reward(\policy_h(x),\choice_h(x), y)} \\
% \text{s.t.} 
% & \quad \choice_h(x) = \red{\argmax v?} \,\,\, \forall x \nonumber \\
% & \quad \policy_h(\dist) \in A \nonumber
% \end{align}
\begin{equation}
\label{eq:order2_objective}
\argmax_{h \in H} \expect{\dist}{\reward(\policy_h(x),\choice_{\policy_h}(x),y)}
\quad \text{s.t.} \quad
\policy_h(\dist) \in A
\end{equation}
where $\reward$ now depends jointly on system decisions $\policy_h$
and user choices $\choice_{\policy_h}$ through the learned predictor $h$.

\paragraph{Agency and resources.}
When users make choices, 
this makes them a part of the allocation process---but also the reason for scarcity.
Here we outline three scenarios of interest:
\squeeze
\squeeze
\begin{itemize}[leftmargin=1em,topsep=0em,itemsep=0.2em]
\item 
\textbf{Self-selection:}
For most systems that makes decisions about users,
those users must first choose to join the system;
consider hiring, admissions, medical programs, educational aid,
and welfare benefits.
In econometrics, the choice of users to participate (or not)
based on the utility they expect to obtain
is known as \emph{self-selection} \citep{roy1951some}.
Since self-selection in aggregate determines the user population,
once user decisions depend on the learned model,
learning obtains the power to shape the population's composition---%
a power which should not be taken lightly.%
\priority{\tocitec{evoML}}
Although self-selection has a long history in economics,
it is only now beginning to draw attention in machine learning research \citep{zhang2021classification,cherapanamjeri2023makes,horowitz2024classification}.

% In some tasks where the system selects users,
% those users must also choose the system.
% Examples include
% hiring, admissions, medical programs, educational aid,
% and welfare benefits,
% in which users must agree to participate, or even apply in the first place.
% Such behavior, known in econometrics as \emph{self-selection} \tocite,
% has so far received little attention in machine learning \tocite.



\item
\textbf{Matching:}
Prime examples in which user choices drive economic outcomes are recommendation systems and online market platforms.
These typically include a single learner whose role is
to match users with relevant items \citep{jagadeesan2023learning};
once matched, it is the choices of users that ultimately determine allocations.
Since matchings are often based on predictions,
their quality determines outcomes for users, suppliers, and the system. % \tocitec{ML}.
This makes them unique in relation to conventional \emph{two-sided markets}  \citep{rochet2003platform}.
In modern human-facing systems, key resources primarily include the attention, time, and budgets of users.
This calls for careful and responsible modeling of user choice behavior
\citep{kleinberg2024challenge}.%
\priority{\citep{saig2023learning}}

%\vspace{-2mm}

\item
\textbf{Competition:}
For many tasks and services of interest, users can now choose between multiple platforms or providers.
This creates competition:
once providers aim to maximize their market share,
users themselves become the scarce resource,
over which providers compete \citep{ben2019regression,jagadeesan2023competition,yao2023bad,yao2024user}.
% A natural extension of self-selection is to allow users to choose among multiple  alternatives, such as between competing platforms or service providers \tocite.
When the benefit of users depends on the quality of predictions,
competition will revolve around which provider learns more accurate models---and for which users.
This creates a novel `market for accuracy',
in which learning can have much impact on efficiency and equity.%
\priority{\tocitec{cmpt}}
% require deliberation.
% \squeeze

% \hf{This case might have multiple providers as well; this will lead to a competitive game. Shall we consider this? Maybe leave it to the last as this is the most complicate. }

% how users make such choices.

% Since the quality of matchings will determine 
% outcomes for users, suppliers, and the system,
% Hence, this requires careful modeling of user behavior.

% The quality of matchings, as well as their diversity,
% are likely to determine outcomes for users, suppliers, and the system.
\squeeze

\end{itemize}

 
% The above suggest that Eq.~\eqref{eq:order2_objective} can benefit from encoding structure in the form of a bi-partite graph between users and service providers (for competition) or producers (for matching). We elaborate on this further in Sec.~\toref.
% This draws a strong connection to the economic literature on \emph{matching markets} \tocite.
 

% - bipartite graphs

\paragraph{Social welfare.}
A canonical property of classic markets is that they coordinate the behavior of many self-interested agents in a way which can naturaly lead to welfare maximization \citep[e.g.,][]{arrow1954existence,shapley1971assignment}.
A key question in our context is whether this emerges also in 
markets where coordination is mediated by predictions
\citep{nahum2024decongestion}.
Given the growing concerns regarding how recommendation systems 
may drive polarization, echo chambers, informational barriers, economic inequity, and unhealthy usage patterns---the answer is likely negative.
Welfare economics may be helpful in posing the question of \emph{why} 
this happens as one of \emph{market failure}.
This provides tools for uncovering the mechanisms underlying failure,
such as negative externalities,
public goods,
information asymmetry,%
\priority{(e.g., moral hazard),}
market power and control (e.g., by monopolies),
or collusion,
as they manifest through learning.%
\priority{This raises interesting questions regarding how these manifest in learning settings \citep{hardt2022performative},
and what can be done about them---e.g., via regulation by a social planner
(see Sec.~\ref{sec:regulation}).}
\squeeze 


% \subsection{Challenges at Order 2 } 
\paragraph{Challenge \#1: Escaping echo chambers.}
Since its early days, % of recommendation systems,
the task of recommending content has been treated as a pure prediction problem.
But accumulating evidence of its likely ill effects
% as to the possible ill affects of this approach
has motivated a search for more viable alternatives.
A major issue is the reinforcing nature of accuracy-driven recommendation:
users that are recommended certain items become `locked in' on similar content
through a positive feedback loop of indefinite
model retraining, choice behavior, and recommendation.
One aspect of this loop that is often overlooked, and relevant to our context,
is that recommendations also incentivize the \emph{creation} of new content
by exposure-seeking creators.
Recommendation is essentially a problem of matching
demand (of users) and supply (of new content) over a bi-partite user-content graph;
thus, if we wish to prevent the formation of echo chambers and filter bubbles,
we should find ways to exploit this structure to promote socially-beneficial outcomes.
Some examples include incentivizing exploration \citep{mansour2020bayesian}
or the creation of diverse content \citep{eilat2023performative,yao2024user},
but many other paths are possible.
\squeeze


% - reinforce behavior: observed data, recommended content, incentivizing creators
% - limited resources
% - bip graph

% - accuracy-first/centric
% - incentivize diversity, exploration



% \paragraph{Challenge 1: escaping from the echo chamber in recommendation learning.}
% A prominent problem at  Order 2 is the  \emph{recommendation system}  (RS).  An RS's  can be viewed as  a \emph{bipartite graph} where one side is the contents and the other side is users (i.e., content consumers).  The system learns the relevance score for each (content, user) pair, based on which the most relevant few contents are  recommended to each user. At a first glance, this may appear  a standard learning problem,  but  the fact that users make selections from limited choices, either due to limited user attention or limited system supplies, causes the issue of echo chamber -- that is, the system learns users' preferences over a limited choices, then repeatedly reinforces this learned preference by repeating similar recommendations, and ultimately stuck within an ``information bubble'' always with similar contents. What's worse, this problematic content consumption behavior is further reinforced by increasing supply of the similar contents, driven by content creators' traffic-seeking behaviors. 
% %give rise to the following important new challenges:  (1) reinforcing users' preferences ; (2) strategic reactions from content creators.  For (1), the system   uses a user's past behavior trajectory to learn new recommendations, which mostly are similar contents, which then reinforces similar user behavior. For (2), traffic-seeking content creators further intensify this issue by constantly creating   contents that were popular in the past. These two factors together builds ``content bubbles'' for users who continuously see very similar contents, leading to the a famousely known echo chamber issue.  This not only leads to unhealth user content consumption behavior but also drive the entire content ecosystem towards high-concentrated content themes with little diversity. 
% Addressing this pressing challenge requires us to go beyond traditional accuracy-first thinking, and design more explorative learning algorithms to account for users' choices and promote content diversity \cite{mansour2020bayesian,eilat2023performative,yao2024user}.  


 
% WE PROBABLY DO NOT HAVE ROOM FOR FOLLOWING POINTS  ANY MORE, BUT IT IS ALSO IMPORTANT. Strategic behavior and stable marketing, this is when one sided supply is limited. so not true for recommendation, but true for job positions like Linkedin (online labor market, upwork)

\paragraph{Challenge \#2: Learning to compete.} 
Once predictions are given as a service,
it is only  natural that multiple providers will compete over who
can give users better predictions.
One example are housing market platforms such as Zillow and Redfin which compete over who
provides better pricing recommendations. % (e.g., Zillow, Redfin).
To some extent, another example are LLMs who compete over better text generation.
Competition is driven by the fact that users are free to choose a provider,
but that choices are costly (e.g., time, premium features, opportunity costs).
Thus, although users gain from accuracy,
optimizing expected accuracy is likely \emph{not} a good strategy for providers.
As a result, welfare could suffer.
% even if overall accuracy is high, 
Welfare in such markets will be high if the market is efficient.
This requires a planner that can steer competing 
learners towards a favorable \emph{Nash-Stackelberg} equilibrium,
where users follow the simultaneous moves of multilpe systems. 



% \paragraph{Challenge 2: learner competitions.}  Our discussions thus far have focused on  a single learner/system. To account for user choices, we need to zoom  out and consider situations with multiple systems; a notable   example is the house hunting applications with competing systems \emph{Zillow} and \emph{Redfine}. Users can choose from these   systems, and users' opportunity costs such as time and payment for premium features induce systems' competition. Users  always prefer higher accuracy, but no single  system  can  achieve the highest accuracy across the entire user population. Hence traditional learning approach that simply maximizes accuracy can fail miserably ---   it may produce a learning algorithm that has \emph{reasonably high accuracy for every user but not the best for any user}. Welfare-maximizing machine learning in such competitive situations   features complex tradeoff between what  model performances are obtainable and what user population on the market needs better accuracy. Zooming out further, from a social planner's perspective, it is valuable to understand what market equilibrium outcome can arrive and  whether the equilibrium will be efficient. Such situation with users responding to multiple  competing systems features a novel \emph{Nash-Stackelberg} style of interactions, featured with  learning algorithms as system strategies. 


% \todo{give examples of matching markets and how and why they need help/prices don't suffice?}

% - we know it doesn't in recsys \\
% - as market failure \\
% - monopolies and market power \\
% - matching markets - known to be `different' \\
% - things to do about it

\extended{%
\nir{most applications can fit all three orders. [think of example: only predictions, +decisions, +choices; maybe hiring].
it is therefore a question of how and to what resolution we want to model the setting, and what aspects are most significant.
actually, a problem with the field of recsys is that they've been thinking about it only as order 0 (or less), but we need to think about it as order 2 for things to work well in terms of welfare.}

% If users gain from accuracy, and if they \emph{can exercise their agency},
% then each user will likely to act to secure accurate predictions for herself.

% - learning affects behavior affects learning outcomes
% - outcomes for users become dependent through model
% - replace uniform $u$ with true valuations $v$

\blue{%
\underline{checklist}: \\
- resources \\
- policy \\
- choices \\
- agency, behavior \\
- utility
}
}

% \blue{%
% \paragraph{Hierarchy.}
% Eq.~\eqref{eq:order2_objective} reduces to Eq.~\eqref{eq:order1_objective}
% when $r=\welf$ (utilitarian),
% utility derives from accuracy,
% and $A$ does not impose restrictions. 
% }


% % !TEX root = main.tex

\section{Beyond Markov}
\label{sec:beyond}
% Here we focus on Mamba with non-Markovian data.

\subsection{Switching Markov model}
\label{sec:switch}
A key component of \ref{eq:mamba_block} enabling selectivity is the state-transition factor $a_t$, that controls the flow of information from the past state $H_{t-1}$ to the current $H_t$: if $a_t=1$, the past information is fully utilized in computing the current state, and hence the output, whereas $a_t=0$ completely ignores the past. In the Markovian setting considered so far, the role of $a_t$ has largely been dormant: $a_t \approx 1$ for all $t\geq 1$, as the optimal Laplacian predictor requires counts of all transitions, demanding the use of full past (\prettyref{sec:main_thm}). To better highlight this selectivity mechanism, we consider a non-Markovian process, where the role of $a_t$ becomes fundamental. Specifically, we focus on the \emph{switching Markov process}, where we add a \emph{switch} token to the binary alphabet, \ie we consider $\calX = \{0,1,S\}$. The key difference here compared to the random Markov generation in \prettyref{sec:markov_background} is that until we hit switch token, the sequence follows the same binary Markov chain, but once the switch state is reached, the next symbols follow a newly sampled Markov chain, independent of the previous one. The switch tokens are sampled according to a parallel \iid Bernoulli process with probability $p_{\rm switch}$ ($0.01$ in our experiments). More formally, the process consists of the following steps:
\begin{enumerate}
    \item Initialize $t=1$.
    \vspace{-0.5em}
    \item Draw a binary Markov kernel $P$ with each conditional distribution $P_{i_1^k}$ sampled \iid from $\dir{\beta}$.
    \vspace{-0.5em}
    \item Let $x_t = S$ with probability $p_{\rm switch}$, or sample $x_{t} \sim P_{x_{t-k+1}^t}$ with probability $1 - p_{\rm switch}$ (the first $k$ samples after each switch token are sampled from $\unif{\binary}$).
    \vspace{-0.5em}
    \item If $x_t = S$, set $t = t+1$ and go to step 2; if $x_t \neq S$, set $t = t+1$ and go to step 3.
\end{enumerate}

\begin{figure*}[t]
\captionsetup[sub]{}
\centering
\begin{subfigure}{0.49\textwidth}
\centering
\includegraphics[width=\textwidth]{figures/switch-estimator.pdf} 
    \put(-125,-7){\fontsize{9}{3}\selectfont $t: x_t = 0$}
      \put(-242,86.5){\rotatebox[origin=t]{90}{\fontsize{8}{3}\selectfont Predicted probability $\mathbb{P}_{\btheta}\pth{{x_{t+1}=1 \mid x_1^t}}$}}
      \put(-76,157){\fontsize{7.5}{3}\selectfont $1$-layer Mamba}
      \put(-76,146){\fontsize{7.5}{3}\selectfont Optimal estimator}
\caption{Predicted probabilities}
\label{fig:switch-estimator}
\end{subfigure}
\hfill
\begin{subfigure}{0.49\textwidth}
\centering
\includegraphics[width=\textwidth]{figures/switch-at.pdf} 
    \put(-132,-7){\fontsize{9}{3}\selectfont Position $t$}
      \put(-238,88){\rotatebox[origin=t]{90}{\fontsize{10}{3}\selectfont $a_t$}}
\caption{Value of $a_t$ across positions}
\label{fig:switch-at}
\end{subfigure}
\caption{Single-layer Mamba on data generated from the switching Markov process with $p_{\rm switch} = 0.01$. The red vertical lines mark the positions of switch tokens. Figure (a) shows that the model's prediction follows very precisely that of the optimal estimator also in this more complex scenario. Figure (b) highlights the selectivity process of the model: every time a switch token appears, the model erases all information about the past by setting $a_t=0$.}
\label{fig:mamba-switch}
\end{figure*}

\noindent {\bf Mamba learns the optimal estimator.} With this data model, the optimal prediction strategy is to use the add-$\beta$ estimator in between two switch tokens, and reset the transition counts every time a switch occurs. Indeed, \prettyref{fig:mamba-switch} illustrates that Mamba implements precisely this strategy, closely tracking the switching events via the transition factor $a_t$: it sets $a_t$ to be zero whenever $x_t = S$ and to one otherwise. This enables the model to zero out the transition counts at every switch event, so that it can estimate the statistics of the new chain from scratch.


\subsection{Natural language modeling}
\label{sec:english}
To test the generality of our finding that convolution plays key role on Markovian data (\prettyref{fig:conv}), we conduct experiments on the language modeling task using the WikiText-103 dataset with a sequence length of $256$. We use a standard 2-layer Transformer consisting of attention and MLP modules with a model dimension of $256$. To ensure a comparable parameter count, we stack the Mamba-$2$ cell across four layers, following the design in \cite{dao2024transformers}. By adding or removing convolution in both these models, we obtain the results shown in \autoref{tab:lm_result}. The results illustrate that while convolution significantly enhances the performance of Mamba-2, reducing perplexity from $30.68$ to $27.55$, the improvement for the Transformer is marginal. This highlights the fundamental role of convolution even on the complex language modeling tasks. 
% These results on language modeling further demonstrate the effectiveness of convolution in Mamba models.

\begin{table}[h]
\centering
\caption{Perplexity results on the WikiText-$103$ dataset for both models. (w/o conv) denotes the absence of convolution, while (w/ conv) indicates its use.}
\begin{tabular}{lll}
\toprule
\bf Model & \bf \# Params. (M) &\bf Perplexity ($\downarrow$)\\
\midrule
Mamba-$2$ (w/o conv)& $ 14.53 $ & $ 30.68 $ \\ 
Mamba-$2$ (w/conv)& $14.54$ &  $\bf{27.55}$ \\
\midrule
Transformer (w/o conv) & $14.46$ & $29.28$ \\
Transformer (w/ conv) & $14.46$ &  $\bf{28.67}$  \\
\bottomrule
\end{tabular}
\label{tab:lm_result}
\end{table}


\section{Alternative Views} \label{sec:alt_views}
While it is easy to agree that learning systems should be designed to enable the promotion of social good, there will likely be disagreement as to how.
We propose to adopt the perspective of welfare economics,
but there certainly exist alternative viewpoints and complementary approaches.
\squeeze

\paragraph{Give accuracy time.}
One perspective is that if we give data enough time to accumulate and new methods enough time to improve, then machine learning will organically overcome the challenges we discussed.
One example to draw on is how despite many advances in optimization, the simple gradient descent algorithm still drives most modern tools.
Another is how large language models have demonstrated that simply predicting the next word with sufficient accuracy and on enough data 
gives rise to emergent phenomena far beyond this basic task.
Our position is that limited resources is an inherent problem of any social system, whether technology-driven or not.%
\extended{Our point here would be that since even 50 thousand years of cultural evolution have not `solved' the problem of scarce resource allocation, 
it is unlikely to just go away.}
% nor has this been dismissed in biology or ecology.
We believe that scarcity should be addressed explicitly---%
but of course we may be proven wrong.
\squeeze

\paragraph{Divide and conquer.}
Even if  machine learning as a standalone solution does not suffice,
one could argue that an economic approach can be applied on top of existing learning tools, rather than integrated within them as we propose.
Hence, learning and policy can be advanced independently and combined only later.
This is reasonable, and independent efforts and application will likely be required
regardless of whether direct integration works or not.
But there is increasing evidence that this will not suffice;
in fact, the field of fair machine learning rose in response to the clear need for embedding social considerations within the learning objective itself.
Advances in the study of fairness in learning have also shown that fairness constraints alone cannot guarantee equity,
such as when learning effects accumulate over time \citep[e.g.,][]{liu2018delayed}
or as a result of strategic user behavior
\citep{horowitz2024classification}.
We take these to suggest that the novelty in the interface between learning and economics requires a wholistic approach
specialized for this intersection.%
\extended{that operates on a deep, joint understanding of how both fields can intersect.}
\squeeze

\paragraph{Welfare without welfare.}
Welfare economics is not the only approach for reasoning about and facilitating welfare,
nor is it free from issues and limitations in itself.
% For example, welfare economics has been criticized for
Criticism includes
its subjective nature;
the need to measure and compare utility across individuals; 
the emphasis on cardinal rather than ordinal utilities;%
\extended{(which are more plausible but harder to work with);}
the reliance on assumptions of rational behavior; 
the susceptibility to externalities and other sources of market failure; 
the need for a centralized social planner entity;
and challenges in policy evaluation.
Other schools of thought in economics offer alternatives:
For example, Sen's \emph{capabilities approach}
\citep{sen1999commodities}
focuses on ensuring people are capable of achieving what they seek,
rather than the value of what they obtain.
\priority{Another example is behavioral economics which stresses the nature of human-decision making as the key to improving social outcomes.}
Within machine learning, there been have calls for alternative approaches as well,
such as to `democratize' the issue of alignment using social choice theory
\citep{conitzer2024position,ge2024axioms,fish2024generative}.
We view these as complementary to ours,
and believe there is merit advancing welfare in machine learning
simultaneously along several fronts.
\squeeze

% and while some approaches may prove more beneficial than others in practice, until this becomes clear, we believe there is merit in exploring ideas along multiple fronts.



% 1. maybe standard ml is ok. maybe fairness suffices. hasn't worked well.
% 2. can just apply econ consideration on top of existing stuff. but - this was supposedly possible until now, but wasn't done.
% 3. even if integrate welfare into ml, there is crit for welf econ itself. there are alternatives 
% 4. possible alternatives... not instead, but alongside. welfml not only option.

% I think one opposition is that as model/data become sufficiently large and accuracy improves, they will automatically address all our proposed questions, so our questions do not need to be studied separately….

% - maybe accuracy *will* just solve it all? might be better to leave ML intact and in isolation (maybe with eg fairness constraints) and wrap with policy, rather than plugging econ modeling inside. may lose more than gain by integrating

% - not only supervised learning: online, rl, generative (LLMs)

% - other welfare econ crits: subjectivity in defining welfare and equity; comparing utilities across individuals (ordinal,  cardinal); rational modeling; externalities.
%   -- alternatives: Capabilities Approach, behavioral econ

% - centralized control of societal prefs; vs social choice, deliberative democracy
 % -- Social Choice Should Guide AI Alignment in Dealing with Diverse Human Feedback / conitzer et al, ICML24 position paper!
 % -- Generative Social Choice / procaccia
 % -- Axioms for AI Alignment from Human Feedback / procaccia [NeurIPS 2024]

% - measurement and eval - can't really estimate utility (before and after)



\extended{%
\bbox
\textbf{Desideratum:}
\blue{Understand limitations, but also benefits of naive learning: when it works, when it doesn't, and why.}
\ebox
}
 
\subsection{Lloyd-Max Algorithm}
\label{subsec:Lloyd-Max}
For a given quantization bitwidth $B$ and an operand $\bm{X}$, the Lloyd-Max algorithm finds $2^B$ quantization levels $\{\hat{x}_i\}_{i=1}^{2^B}$ such that quantizing $\bm{X}$ by rounding each scalar in $\bm{X}$ to the nearest quantization level minimizes the quantization MSE. 

The algorithm starts with an initial guess of quantization levels and then iteratively computes quantization thresholds $\{\tau_i\}_{i=1}^{2^B-1}$ and updates quantization levels $\{\hat{x}_i\}_{i=1}^{2^B}$. Specifically, at iteration $n$, thresholds are set to the midpoints of the previous iteration's levels:
\begin{align*}
    \tau_i^{(n)}=\frac{\hat{x}_i^{(n-1)}+\hat{x}_{i+1}^{(n-1)}}2 \text{ for } i=1\ldots 2^B-1
\end{align*}
Subsequently, the quantization levels are re-computed as conditional means of the data regions defined by the new thresholds:
\begin{align*}
    \hat{x}_i^{(n)}=\mathbb{E}\left[ \bm{X} \big| \bm{X}\in [\tau_{i-1}^{(n)},\tau_i^{(n)}] \right] \text{ for } i=1\ldots 2^B
\end{align*}
where to satisfy boundary conditions we have $\tau_0=-\infty$ and $\tau_{2^B}=\infty$. The algorithm iterates the above steps until convergence.

Figure \ref{fig:lm_quant} compares the quantization levels of a $7$-bit floating point (E3M3) quantizer (left) to a $7$-bit Lloyd-Max quantizer (right) when quantizing a layer of weights from the GPT3-126M model at a per-tensor granularity. As shown, the Lloyd-Max quantizer achieves substantially lower quantization MSE. Further, Table \ref{tab:FP7_vs_LM7} shows the superior perplexity achieved by Lloyd-Max quantizers for bitwidths of $7$, $6$ and $5$. The difference between the quantizers is clear at 5 bits, where per-tensor FP quantization incurs a drastic and unacceptable increase in perplexity, while Lloyd-Max quantization incurs a much smaller increase. Nevertheless, we note that even the optimal Lloyd-Max quantizer incurs a notable ($\sim 1.5$) increase in perplexity due to the coarse granularity of quantization. 

\begin{figure}[h]
  \centering
  \includegraphics[width=0.7\linewidth]{sections/figures/LM7_FP7.pdf}
  \caption{\small Quantization levels and the corresponding quantization MSE of Floating Point (left) vs Lloyd-Max (right) Quantizers for a layer of weights in the GPT3-126M model.}
  \label{fig:lm_quant}
\end{figure}

\begin{table}[h]\scriptsize
\begin{center}
\caption{\label{tab:FP7_vs_LM7} \small Comparing perplexity (lower is better) achieved by floating point quantizers and Lloyd-Max quantizers on a GPT3-126M model for the Wikitext-103 dataset.}
\begin{tabular}{c|cc|c}
\hline
 \multirow{2}{*}{\textbf{Bitwidth}} & \multicolumn{2}{|c|}{\textbf{Floating-Point Quantizer}} & \textbf{Lloyd-Max Quantizer} \\
 & Best Format & Wikitext-103 Perplexity & Wikitext-103 Perplexity \\
\hline
7 & E3M3 & 18.32 & 18.27 \\
6 & E3M2 & 19.07 & 18.51 \\
5 & E4M0 & 43.89 & 19.71 \\
\hline
\end{tabular}
\end{center}
\end{table}

\subsection{Proof of Local Optimality of LO-BCQ}
\label{subsec:lobcq_opt_proof}
For a given block $\bm{b}_j$, the quantization MSE during LO-BCQ can be empirically evaluated as $\frac{1}{L_b}\lVert \bm{b}_j- \bm{\hat{b}}_j\rVert^2_2$ where $\bm{\hat{b}}_j$ is computed from equation (\ref{eq:clustered_quantization_definition}) as $C_{f(\bm{b}_j)}(\bm{b}_j)$. Further, for a given block cluster $\mathcal{B}_i$, we compute the quantization MSE as $\frac{1}{|\mathcal{B}_{i}|}\sum_{\bm{b} \in \mathcal{B}_{i}} \frac{1}{L_b}\lVert \bm{b}- C_i^{(n)}(\bm{b})\rVert^2_2$. Therefore, at the end of iteration $n$, we evaluate the overall quantization MSE $J^{(n)}$ for a given operand $\bm{X}$ composed of $N_c$ block clusters as:
\begin{align*}
    \label{eq:mse_iter_n}
    J^{(n)} = \frac{1}{N_c} \sum_{i=1}^{N_c} \frac{1}{|\mathcal{B}_{i}^{(n)}|}\sum_{\bm{v} \in \mathcal{B}_{i}^{(n)}} \frac{1}{L_b}\lVert \bm{b}- B_i^{(n)}(\bm{b})\rVert^2_2
\end{align*}

At the end of iteration $n$, the codebooks are updated from $\mathcal{C}^{(n-1)}$ to $\mathcal{C}^{(n)}$. However, the mapping of a given vector $\bm{b}_j$ to quantizers $\mathcal{C}^{(n)}$ remains as  $f^{(n)}(\bm{b}_j)$. At the next iteration, during the vector clustering step, $f^{(n+1)}(\bm{b}_j)$ finds new mapping of $\bm{b}_j$ to updated codebooks $\mathcal{C}^{(n)}$ such that the quantization MSE over the candidate codebooks is minimized. Therefore, we obtain the following result for $\bm{b}_j$:
\begin{align*}
\frac{1}{L_b}\lVert \bm{b}_j - C_{f^{(n+1)}(\bm{b}_j)}^{(n)}(\bm{b}_j)\rVert^2_2 \le \frac{1}{L_b}\lVert \bm{b}_j - C_{f^{(n)}(\bm{b}_j)}^{(n)}(\bm{b}_j)\rVert^2_2
\end{align*}

That is, quantizing $\bm{b}_j$ at the end of the block clustering step of iteration $n+1$ results in lower quantization MSE compared to quantizing at the end of iteration $n$. Since this is true for all $\bm{b} \in \bm{X}$, we assert the following:
\begin{equation}
\begin{split}
\label{eq:mse_ineq_1}
    \tilde{J}^{(n+1)} &= \frac{1}{N_c} \sum_{i=1}^{N_c} \frac{1}{|\mathcal{B}_{i}^{(n+1)}|}\sum_{\bm{b} \in \mathcal{B}_{i}^{(n+1)}} \frac{1}{L_b}\lVert \bm{b} - C_i^{(n)}(b)\rVert^2_2 \le J^{(n)}
\end{split}
\end{equation}
where $\tilde{J}^{(n+1)}$ is the the quantization MSE after the vector clustering step at iteration $n+1$.

Next, during the codebook update step (\ref{eq:quantizers_update}) at iteration $n+1$, the per-cluster codebooks $\mathcal{C}^{(n)}$ are updated to $\mathcal{C}^{(n+1)}$ by invoking the Lloyd-Max algorithm \citep{Lloyd}. We know that for any given value distribution, the Lloyd-Max algorithm minimizes the quantization MSE. Therefore, for a given vector cluster $\mathcal{B}_i$ we obtain the following result:

\begin{equation}
    \frac{1}{|\mathcal{B}_{i}^{(n+1)}|}\sum_{\bm{b} \in \mathcal{B}_{i}^{(n+1)}} \frac{1}{L_b}\lVert \bm{b}- C_i^{(n+1)}(\bm{b})\rVert^2_2 \le \frac{1}{|\mathcal{B}_{i}^{(n+1)}|}\sum_{\bm{b} \in \mathcal{B}_{i}^{(n+1)}} \frac{1}{L_b}\lVert \bm{b}- C_i^{(n)}(\bm{b})\rVert^2_2
\end{equation}

The above equation states that quantizing the given block cluster $\mathcal{B}_i$ after updating the associated codebook from $C_i^{(n)}$ to $C_i^{(n+1)}$ results in lower quantization MSE. Since this is true for all the block clusters, we derive the following result: 
\begin{equation}
\begin{split}
\label{eq:mse_ineq_2}
     J^{(n+1)} &= \frac{1}{N_c} \sum_{i=1}^{N_c} \frac{1}{|\mathcal{B}_{i}^{(n+1)}|}\sum_{\bm{b} \in \mathcal{B}_{i}^{(n+1)}} \frac{1}{L_b}\lVert \bm{b}- C_i^{(n+1)}(\bm{b})\rVert^2_2  \le \tilde{J}^{(n+1)}   
\end{split}
\end{equation}

Following (\ref{eq:mse_ineq_1}) and (\ref{eq:mse_ineq_2}), we find that the quantization MSE is non-increasing for each iteration, that is, $J^{(1)} \ge J^{(2)} \ge J^{(3)} \ge \ldots \ge J^{(M)}$ where $M$ is the maximum number of iterations. 
%Therefore, we can say that if the algorithm converges, then it must be that it has converged to a local minimum. 
\hfill $\blacksquare$


\begin{figure}
    \begin{center}
    \includegraphics[width=0.5\textwidth]{sections//figures/mse_vs_iter.pdf}
    \end{center}
    \caption{\small NMSE vs iterations during LO-BCQ compared to other block quantization proposals}
    \label{fig:nmse_vs_iter}
\end{figure}

Figure \ref{fig:nmse_vs_iter} shows the empirical convergence of LO-BCQ across several block lengths and number of codebooks. Also, the MSE achieved by LO-BCQ is compared to baselines such as MXFP and VSQ. As shown, LO-BCQ converges to a lower MSE than the baselines. Further, we achieve better convergence for larger number of codebooks ($N_c$) and for a smaller block length ($L_b$), both of which increase the bitwidth of BCQ (see Eq \ref{eq:bitwidth_bcq}).


\subsection{Additional Accuracy Results}
%Table \ref{tab:lobcq_config} lists the various LOBCQ configurations and their corresponding bitwidths.
\begin{table}
\setlength{\tabcolsep}{4.75pt}
\begin{center}
\caption{\label{tab:lobcq_config} Various LO-BCQ configurations and their bitwidths.}
\begin{tabular}{|c||c|c|c|c||c|c||c|} 
\hline
 & \multicolumn{4}{|c||}{$L_b=8$} & \multicolumn{2}{|c||}{$L_b=4$} & $L_b=2$ \\
 \hline
 \backslashbox{$L_A$\kern-1em}{\kern-1em$N_c$} & 2 & 4 & 8 & 16 & 2 & 4 & 2 \\
 \hline
 64 & 4.25 & 4.375 & 4.5 & 4.625 & 4.375 & 4.625 & 4.625\\
 \hline
 32 & 4.375 & 4.5 & 4.625& 4.75 & 4.5 & 4.75 & 4.75 \\
 \hline
 16 & 4.625 & 4.75& 4.875 & 5 & 4.75 & 5 & 5 \\
 \hline
\end{tabular}
\end{center}
\end{table}

%\subsection{Perplexity achieved by various LO-BCQ configurations on Wikitext-103 dataset}

\begin{table} \centering
\begin{tabular}{|c||c|c|c|c||c|c||c|} 
\hline
 $L_b \rightarrow$& \multicolumn{4}{c||}{8} & \multicolumn{2}{c||}{4} & 2\\
 \hline
 \backslashbox{$L_A$\kern-1em}{\kern-1em$N_c$} & 2 & 4 & 8 & 16 & 2 & 4 & 2  \\
 %$N_c \rightarrow$ & 2 & 4 & 8 & 16 & 2 & 4 & 2 \\
 \hline
 \hline
 \multicolumn{8}{c}{GPT3-1.3B (FP32 PPL = 9.98)} \\ 
 \hline
 \hline
 64 & 10.40 & 10.23 & 10.17 & 10.15 &  10.28 & 10.18 & 10.19 \\
 \hline
 32 & 10.25 & 10.20 & 10.15 & 10.12 &  10.23 & 10.17 & 10.17 \\
 \hline
 16 & 10.22 & 10.16 & 10.10 & 10.09 &  10.21 & 10.14 & 10.16 \\
 \hline
  \hline
 \multicolumn{8}{c}{GPT3-8B (FP32 PPL = 7.38)} \\ 
 \hline
 \hline
 64 & 7.61 & 7.52 & 7.48 &  7.47 &  7.55 &  7.49 & 7.50 \\
 \hline
 32 & 7.52 & 7.50 & 7.46 &  7.45 &  7.52 &  7.48 & 7.48  \\
 \hline
 16 & 7.51 & 7.48 & 7.44 &  7.44 &  7.51 &  7.49 & 7.47  \\
 \hline
\end{tabular}
\caption{\label{tab:ppl_gpt3_abalation} Wikitext-103 perplexity across GPT3-1.3B and 8B models.}
\end{table}

\begin{table} \centering
\begin{tabular}{|c||c|c|c|c||} 
\hline
 $L_b \rightarrow$& \multicolumn{4}{c||}{8}\\
 \hline
 \backslashbox{$L_A$\kern-1em}{\kern-1em$N_c$} & 2 & 4 & 8 & 16 \\
 %$N_c \rightarrow$ & 2 & 4 & 8 & 16 & 2 & 4 & 2 \\
 \hline
 \hline
 \multicolumn{5}{|c|}{Llama2-7B (FP32 PPL = 5.06)} \\ 
 \hline
 \hline
 64 & 5.31 & 5.26 & 5.19 & 5.18  \\
 \hline
 32 & 5.23 & 5.25 & 5.18 & 5.15  \\
 \hline
 16 & 5.23 & 5.19 & 5.16 & 5.14  \\
 \hline
 \multicolumn{5}{|c|}{Nemotron4-15B (FP32 PPL = 5.87)} \\ 
 \hline
 \hline
 64  & 6.3 & 6.20 & 6.13 & 6.08  \\
 \hline
 32  & 6.24 & 6.12 & 6.07 & 6.03  \\
 \hline
 16  & 6.12 & 6.14 & 6.04 & 6.02  \\
 \hline
 \multicolumn{5}{|c|}{Nemotron4-340B (FP32 PPL = 3.48)} \\ 
 \hline
 \hline
 64 & 3.67 & 3.62 & 3.60 & 3.59 \\
 \hline
 32 & 3.63 & 3.61 & 3.59 & 3.56 \\
 \hline
 16 & 3.61 & 3.58 & 3.57 & 3.55 \\
 \hline
\end{tabular}
\caption{\label{tab:ppl_llama7B_nemo15B} Wikitext-103 perplexity compared to FP32 baseline in Llama2-7B and Nemotron4-15B, 340B models}
\end{table}

%\subsection{Perplexity achieved by various LO-BCQ configurations on MMLU dataset}


\begin{table} \centering
\begin{tabular}{|c||c|c|c|c||c|c|c|c|} 
\hline
 $L_b \rightarrow$& \multicolumn{4}{c||}{8} & \multicolumn{4}{c||}{8}\\
 \hline
 \backslashbox{$L_A$\kern-1em}{\kern-1em$N_c$} & 2 & 4 & 8 & 16 & 2 & 4 & 8 & 16  \\
 %$N_c \rightarrow$ & 2 & 4 & 8 & 16 & 2 & 4 & 2 \\
 \hline
 \hline
 \multicolumn{5}{|c|}{Llama2-7B (FP32 Accuracy = 45.8\%)} & \multicolumn{4}{|c|}{Llama2-70B (FP32 Accuracy = 69.12\%)} \\ 
 \hline
 \hline
 64 & 43.9 & 43.4 & 43.9 & 44.9 & 68.07 & 68.27 & 68.17 & 68.75 \\
 \hline
 32 & 44.5 & 43.8 & 44.9 & 44.5 & 68.37 & 68.51 & 68.35 & 68.27  \\
 \hline
 16 & 43.9 & 42.7 & 44.9 & 45 & 68.12 & 68.77 & 68.31 & 68.59  \\
 \hline
 \hline
 \multicolumn{5}{|c|}{GPT3-22B (FP32 Accuracy = 38.75\%)} & \multicolumn{4}{|c|}{Nemotron4-15B (FP32 Accuracy = 64.3\%)} \\ 
 \hline
 \hline
 64 & 36.71 & 38.85 & 38.13 & 38.92 & 63.17 & 62.36 & 63.72 & 64.09 \\
 \hline
 32 & 37.95 & 38.69 & 39.45 & 38.34 & 64.05 & 62.30 & 63.8 & 64.33  \\
 \hline
 16 & 38.88 & 38.80 & 38.31 & 38.92 & 63.22 & 63.51 & 63.93 & 64.43  \\
 \hline
\end{tabular}
\caption{\label{tab:mmlu_abalation} Accuracy on MMLU dataset across GPT3-22B, Llama2-7B, 70B and Nemotron4-15B models.}
\end{table}


%\subsection{Perplexity achieved by various LO-BCQ configurations on LM evaluation harness}

\begin{table} \centering
\begin{tabular}{|c||c|c|c|c||c|c|c|c|} 
\hline
 $L_b \rightarrow$& \multicolumn{4}{c||}{8} & \multicolumn{4}{c||}{8}\\
 \hline
 \backslashbox{$L_A$\kern-1em}{\kern-1em$N_c$} & 2 & 4 & 8 & 16 & 2 & 4 & 8 & 16  \\
 %$N_c \rightarrow$ & 2 & 4 & 8 & 16 & 2 & 4 & 2 \\
 \hline
 \hline
 \multicolumn{5}{|c|}{Race (FP32 Accuracy = 37.51\%)} & \multicolumn{4}{|c|}{Boolq (FP32 Accuracy = 64.62\%)} \\ 
 \hline
 \hline
 64 & 36.94 & 37.13 & 36.27 & 37.13 & 63.73 & 62.26 & 63.49 & 63.36 \\
 \hline
 32 & 37.03 & 36.36 & 36.08 & 37.03 & 62.54 & 63.51 & 63.49 & 63.55  \\
 \hline
 16 & 37.03 & 37.03 & 36.46 & 37.03 & 61.1 & 63.79 & 63.58 & 63.33  \\
 \hline
 \hline
 \multicolumn{5}{|c|}{Winogrande (FP32 Accuracy = 58.01\%)} & \multicolumn{4}{|c|}{Piqa (FP32 Accuracy = 74.21\%)} \\ 
 \hline
 \hline
 64 & 58.17 & 57.22 & 57.85 & 58.33 & 73.01 & 73.07 & 73.07 & 72.80 \\
 \hline
 32 & 59.12 & 58.09 & 57.85 & 58.41 & 73.01 & 73.94 & 72.74 & 73.18  \\
 \hline
 16 & 57.93 & 58.88 & 57.93 & 58.56 & 73.94 & 72.80 & 73.01 & 73.94  \\
 \hline
\end{tabular}
\caption{\label{tab:mmlu_abalation} Accuracy on LM evaluation harness tasks on GPT3-1.3B model.}
\end{table}

\begin{table} \centering
\begin{tabular}{|c||c|c|c|c||c|c|c|c|} 
\hline
 $L_b \rightarrow$& \multicolumn{4}{c||}{8} & \multicolumn{4}{c||}{8}\\
 \hline
 \backslashbox{$L_A$\kern-1em}{\kern-1em$N_c$} & 2 & 4 & 8 & 16 & 2 & 4 & 8 & 16  \\
 %$N_c \rightarrow$ & 2 & 4 & 8 & 16 & 2 & 4 & 2 \\
 \hline
 \hline
 \multicolumn{5}{|c|}{Race (FP32 Accuracy = 41.34\%)} & \multicolumn{4}{|c|}{Boolq (FP32 Accuracy = 68.32\%)} \\ 
 \hline
 \hline
 64 & 40.48 & 40.10 & 39.43 & 39.90 & 69.20 & 68.41 & 69.45 & 68.56 \\
 \hline
 32 & 39.52 & 39.52 & 40.77 & 39.62 & 68.32 & 67.43 & 68.17 & 69.30  \\
 \hline
 16 & 39.81 & 39.71 & 39.90 & 40.38 & 68.10 & 66.33 & 69.51 & 69.42  \\
 \hline
 \hline
 \multicolumn{5}{|c|}{Winogrande (FP32 Accuracy = 67.88\%)} & \multicolumn{4}{|c|}{Piqa (FP32 Accuracy = 78.78\%)} \\ 
 \hline
 \hline
 64 & 66.85 & 66.61 & 67.72 & 67.88 & 77.31 & 77.42 & 77.75 & 77.64 \\
 \hline
 32 & 67.25 & 67.72 & 67.72 & 67.00 & 77.31 & 77.04 & 77.80 & 77.37  \\
 \hline
 16 & 68.11 & 68.90 & 67.88 & 67.48 & 77.37 & 78.13 & 78.13 & 77.69  \\
 \hline
\end{tabular}
\caption{\label{tab:mmlu_abalation} Accuracy on LM evaluation harness tasks on GPT3-8B model.}
\end{table}

\begin{table} \centering
\begin{tabular}{|c||c|c|c|c||c|c|c|c|} 
\hline
 $L_b \rightarrow$& \multicolumn{4}{c||}{8} & \multicolumn{4}{c||}{8}\\
 \hline
 \backslashbox{$L_A$\kern-1em}{\kern-1em$N_c$} & 2 & 4 & 8 & 16 & 2 & 4 & 8 & 16  \\
 %$N_c \rightarrow$ & 2 & 4 & 8 & 16 & 2 & 4 & 2 \\
 \hline
 \hline
 \multicolumn{5}{|c|}{Race (FP32 Accuracy = 40.67\%)} & \multicolumn{4}{|c|}{Boolq (FP32 Accuracy = 76.54\%)} \\ 
 \hline
 \hline
 64 & 40.48 & 40.10 & 39.43 & 39.90 & 75.41 & 75.11 & 77.09 & 75.66 \\
 \hline
 32 & 39.52 & 39.52 & 40.77 & 39.62 & 76.02 & 76.02 & 75.96 & 75.35  \\
 \hline
 16 & 39.81 & 39.71 & 39.90 & 40.38 & 75.05 & 73.82 & 75.72 & 76.09  \\
 \hline
 \hline
 \multicolumn{5}{|c|}{Winogrande (FP32 Accuracy = 70.64\%)} & \multicolumn{4}{|c|}{Piqa (FP32 Accuracy = 79.16\%)} \\ 
 \hline
 \hline
 64 & 69.14 & 70.17 & 70.17 & 70.56 & 78.24 & 79.00 & 78.62 & 78.73 \\
 \hline
 32 & 70.96 & 69.69 & 71.27 & 69.30 & 78.56 & 79.49 & 79.16 & 78.89  \\
 \hline
 16 & 71.03 & 69.53 & 69.69 & 70.40 & 78.13 & 79.16 & 79.00 & 79.00  \\
 \hline
\end{tabular}
\caption{\label{tab:mmlu_abalation} Accuracy on LM evaluation harness tasks on GPT3-22B model.}
\end{table}

\begin{table} \centering
\begin{tabular}{|c||c|c|c|c||c|c|c|c|} 
\hline
 $L_b \rightarrow$& \multicolumn{4}{c||}{8} & \multicolumn{4}{c||}{8}\\
 \hline
 \backslashbox{$L_A$\kern-1em}{\kern-1em$N_c$} & 2 & 4 & 8 & 16 & 2 & 4 & 8 & 16  \\
 %$N_c \rightarrow$ & 2 & 4 & 8 & 16 & 2 & 4 & 2 \\
 \hline
 \hline
 \multicolumn{5}{|c|}{Race (FP32 Accuracy = 44.4\%)} & \multicolumn{4}{|c|}{Boolq (FP32 Accuracy = 79.29\%)} \\ 
 \hline
 \hline
 64 & 42.49 & 42.51 & 42.58 & 43.45 & 77.58 & 77.37 & 77.43 & 78.1 \\
 \hline
 32 & 43.35 & 42.49 & 43.64 & 43.73 & 77.86 & 75.32 & 77.28 & 77.86  \\
 \hline
 16 & 44.21 & 44.21 & 43.64 & 42.97 & 78.65 & 77 & 76.94 & 77.98  \\
 \hline
 \hline
 \multicolumn{5}{|c|}{Winogrande (FP32 Accuracy = 69.38\%)} & \multicolumn{4}{|c|}{Piqa (FP32 Accuracy = 78.07\%)} \\ 
 \hline
 \hline
 64 & 68.9 & 68.43 & 69.77 & 68.19 & 77.09 & 76.82 & 77.09 & 77.86 \\
 \hline
 32 & 69.38 & 68.51 & 68.82 & 68.90 & 78.07 & 76.71 & 78.07 & 77.86  \\
 \hline
 16 & 69.53 & 67.09 & 69.38 & 68.90 & 77.37 & 77.8 & 77.91 & 77.69  \\
 \hline
\end{tabular}
\caption{\label{tab:mmlu_abalation} Accuracy on LM evaluation harness tasks on Llama2-7B model.}
\end{table}

\begin{table} \centering
\begin{tabular}{|c||c|c|c|c||c|c|c|c|} 
\hline
 $L_b \rightarrow$& \multicolumn{4}{c||}{8} & \multicolumn{4}{c||}{8}\\
 \hline
 \backslashbox{$L_A$\kern-1em}{\kern-1em$N_c$} & 2 & 4 & 8 & 16 & 2 & 4 & 8 & 16  \\
 %$N_c \rightarrow$ & 2 & 4 & 8 & 16 & 2 & 4 & 2 \\
 \hline
 \hline
 \multicolumn{5}{|c|}{Race (FP32 Accuracy = 48.8\%)} & \multicolumn{4}{|c|}{Boolq (FP32 Accuracy = 85.23\%)} \\ 
 \hline
 \hline
 64 & 49.00 & 49.00 & 49.28 & 48.71 & 82.82 & 84.28 & 84.03 & 84.25 \\
 \hline
 32 & 49.57 & 48.52 & 48.33 & 49.28 & 83.85 & 84.46 & 84.31 & 84.93  \\
 \hline
 16 & 49.85 & 49.09 & 49.28 & 48.99 & 85.11 & 84.46 & 84.61 & 83.94  \\
 \hline
 \hline
 \multicolumn{5}{|c|}{Winogrande (FP32 Accuracy = 79.95\%)} & \multicolumn{4}{|c|}{Piqa (FP32 Accuracy = 81.56\%)} \\ 
 \hline
 \hline
 64 & 78.77 & 78.45 & 78.37 & 79.16 & 81.45 & 80.69 & 81.45 & 81.5 \\
 \hline
 32 & 78.45 & 79.01 & 78.69 & 80.66 & 81.56 & 80.58 & 81.18 & 81.34  \\
 \hline
 16 & 79.95 & 79.56 & 79.79 & 79.72 & 81.28 & 81.66 & 81.28 & 80.96  \\
 \hline
\end{tabular}
\caption{\label{tab:mmlu_abalation} Accuracy on LM evaluation harness tasks on Llama2-70B model.}
\end{table}

%\section{MSE Studies}
%\textcolor{red}{TODO}


\subsection{Number Formats and Quantization Method}
\label{subsec:numFormats_quantMethod}
\subsubsection{Integer Format}
An $n$-bit signed integer (INT) is typically represented with a 2s-complement format \citep{yao2022zeroquant,xiao2023smoothquant,dai2021vsq}, where the most significant bit denotes the sign.

\subsubsection{Floating Point Format}
An $n$-bit signed floating point (FP) number $x$ comprises of a 1-bit sign ($x_{\mathrm{sign}}$), $B_m$-bit mantissa ($x_{\mathrm{mant}}$) and $B_e$-bit exponent ($x_{\mathrm{exp}}$) such that $B_m+B_e=n-1$. The associated constant exponent bias ($E_{\mathrm{bias}}$) is computed as $(2^{{B_e}-1}-1)$. We denote this format as $E_{B_e}M_{B_m}$.  

\subsubsection{Quantization Scheme}
\label{subsec:quant_method}
A quantization scheme dictates how a given unquantized tensor is converted to its quantized representation. We consider FP formats for the purpose of illustration. Given an unquantized tensor $\bm{X}$ and an FP format $E_{B_e}M_{B_m}$, we first, we compute the quantization scale factor $s_X$ that maps the maximum absolute value of $\bm{X}$ to the maximum quantization level of the $E_{B_e}M_{B_m}$ format as follows:
\begin{align}
\label{eq:sf}
    s_X = \frac{\mathrm{max}(|\bm{X}|)}{\mathrm{max}(E_{B_e}M_{B_m})}
\end{align}
In the above equation, $|\cdot|$ denotes the absolute value function.

Next, we scale $\bm{X}$ by $s_X$ and quantize it to $\hat{\bm{X}}$ by rounding it to the nearest quantization level of $E_{B_e}M_{B_m}$ as:

\begin{align}
\label{eq:tensor_quant}
    \hat{\bm{X}} = \text{round-to-nearest}\left(\frac{\bm{X}}{s_X}, E_{B_e}M_{B_m}\right)
\end{align}

We perform dynamic max-scaled quantization \citep{wu2020integer}, where the scale factor $s$ for activations is dynamically computed during runtime.

\subsection{Vector Scaled Quantization}
\begin{wrapfigure}{r}{0.35\linewidth}
  \centering
  \includegraphics[width=\linewidth]{sections/figures/vsquant.jpg}
  \caption{\small Vectorwise decomposition for per-vector scaled quantization (VSQ \citep{dai2021vsq}).}
  \label{fig:vsquant}
\end{wrapfigure}
During VSQ \citep{dai2021vsq}, the operand tensors are decomposed into 1D vectors in a hardware friendly manner as shown in Figure \ref{fig:vsquant}. Since the decomposed tensors are used as operands in matrix multiplications during inference, it is beneficial to perform this decomposition along the reduction dimension of the multiplication. The vectorwise quantization is performed similar to tensorwise quantization described in Equations \ref{eq:sf} and \ref{eq:tensor_quant}, where a scale factor $s_v$ is required for each vector $\bm{v}$ that maps the maximum absolute value of that vector to the maximum quantization level. While smaller vector lengths can lead to larger accuracy gains, the associated memory and computational overheads due to the per-vector scale factors increases. To alleviate these overheads, VSQ \citep{dai2021vsq} proposed a second level quantization of the per-vector scale factors to unsigned integers, while MX \citep{rouhani2023shared} quantizes them to integer powers of 2 (denoted as $2^{INT}$).

\subsubsection{MX Format}
The MX format proposed in \citep{rouhani2023microscaling} introduces the concept of sub-block shifting. For every two scalar elements of $b$-bits each, there is a shared exponent bit. The value of this exponent bit is determined through an empirical analysis that targets minimizing quantization MSE. We note that the FP format $E_{1}M_{b}$ is strictly better than MX from an accuracy perspective since it allocates a dedicated exponent bit to each scalar as opposed to sharing it across two scalars. Therefore, we conservatively bound the accuracy of a $b+2$-bit signed MX format with that of a $E_{1}M_{b}$ format in our comparisons. For instance, we use E1M2 format as a proxy for MX4.

\begin{figure}
    \centering
    \includegraphics[width=1\linewidth]{sections//figures/BlockFormats.pdf}
    \caption{\small Comparing LO-BCQ to MX format.}
    \label{fig:block_formats}
\end{figure}

Figure \ref{fig:block_formats} compares our $4$-bit LO-BCQ block format to MX \citep{rouhani2023microscaling}. As shown, both LO-BCQ and MX decompose a given operand tensor into block arrays and each block array into blocks. Similar to MX, we find that per-block quantization ($L_b < L_A$) leads to better accuracy due to increased flexibility. While MX achieves this through per-block $1$-bit micro-scales, we associate a dedicated codebook to each block through a per-block codebook selector. Further, MX quantizes the per-block array scale-factor to E8M0 format without per-tensor scaling. In contrast during LO-BCQ, we find that per-tensor scaling combined with quantization of per-block array scale-factor to E4M3 format results in superior inference accuracy across models. 
 
\section{Discussion of Assumptions}\label{sec:discussion}
In this paper, we have made several assumptions for the sake of clarity and simplicity. In this section, we discuss the rationale behind these assumptions, the extent to which these assumptions hold in practice, and the consequences for our protocol when these assumptions hold.

\subsection{Assumptions on the Demand}

There are two simplifying assumptions we make about the demand. First, we assume the demand at any time is relatively small compared to the channel capacities. Second, we take the demand to be constant over time. We elaborate upon both these points below.

\paragraph{Small demands} The assumption that demands are small relative to channel capacities is made precise in \eqref{eq:large_capacity_assumption}. This assumption simplifies two major aspects of our protocol. First, it largely removes congestion from consideration. In \eqref{eq:primal_problem}, there is no constraint ensuring that total flow in both directions stays below capacity--this is always met. Consequently, there is no Lagrange multiplier for congestion and no congestion pricing; only imbalance penalties apply. In contrast, protocols in \cite{sivaraman2020high, varma2021throughput, wang2024fence} include congestion fees due to explicit congestion constraints. Second, the bound \eqref{eq:large_capacity_assumption} ensures that as long as channels remain balanced, the network can always meet demand, no matter how the demand is routed. Since channels can rebalance when necessary, they never drop transactions. This allows prices and flows to adjust as per the equations in \eqref{eq:algorithm}, which makes it easier to prove the protocol's convergence guarantees. This also preserves the key property that a channel's price remains proportional to net money flow through it.

In practice, payment channel networks are used most often for micro-payments, for which on-chain transactions are prohibitively expensive; large transactions typically take place directly on the blockchain. For example, according to \cite{river2023lightning}, the average channel capacity is roughly $0.1$ BTC ($5,000$ BTC distributed over $50,000$ channels), while the average transaction amount is less than $0.0004$ BTC ($44.7k$ satoshis). Thus, the small demand assumption is not too unrealistic. Additionally, the occasional large transaction can be treated as a sequence of smaller transactions by breaking it into packets and executing each packet serially (as done by \cite{sivaraman2020high}).
Lastly, a good path discovery process that favors large capacity channels over small capacity ones can help ensure that the bound in \eqref{eq:large_capacity_assumption} holds.

\paragraph{Constant demands} 
In this work, we assume that any transacting pair of nodes have a steady transaction demand between them (see Section \ref{sec:transaction_requests}). Making this assumption is necessary to obtain the kind of guarantees that we have presented in this paper. Unless the demand is steady, it is unreasonable to expect that the flows converge to a steady value. Weaker assumptions on the demand lead to weaker guarantees. For example, with the more general setting of stochastic, but i.i.d. demand between any two nodes, \cite{varma2021throughput} shows that the channel queue lengths are bounded in expectation. If the demand can be arbitrary, then it is very hard to get any meaningful performance guarantees; \cite{wang2024fence} shows that even for a single bidirectional channel, the competitive ratio is infinite. Indeed, because a PCN is a decentralized system and decisions must be made based on local information alone, it is difficult for the network to find the optimal detailed balance flow at every time step with a time-varying demand.  With a steady demand, the network can discover the optimal flows in a reasonably short time, as our work shows.

We view the constant demand assumption as an approximation for a more general demand process that could be piece-wise constant, stochastic, or both (see simulations in Figure \ref{fig:five_nodes_variable_demand}).
We believe it should be possible to merge ideas from our work and \cite{varma2021throughput} to provide guarantees in a setting with random demands with arbitrary means. We leave this for future work. In addition, our work suggests that a reasonable method of handling stochastic demands is to queue the transaction requests \textit{at the source node} itself. This queuing action should be viewed in conjunction with flow-control. Indeed, a temporarily high unidirectional demand would raise prices for the sender, incentivizing the sender to stop sending the transactions. If the sender queues the transactions, they can send them later when prices drop. This form of queuing does not require any overhaul of the basic PCN infrastructure and is therefore simpler to implement than per-channel queues as suggested by \cite{sivaraman2020high} and \cite{varma2021throughput}.

\subsection{The Incentive of Channels}
The actions of the channels as prescribed by the DEBT control protocol can be summarized as follows. Channels adjust their prices in proportion to the net flow through them. They rebalance themselves whenever necessary and execute any transaction request that has been made of them. We discuss both these aspects below.

\paragraph{On Prices}
In this work, the exclusive role of channel prices is to ensure that the flows through each channel remains balanced. In practice, it would be important to include other components in a channel's price/fee as well: a congestion price  and an incentive price. The congestion price, as suggested by \cite{varma2021throughput}, would depend on the total flow of transactions through the channel, and would incentivize nodes to balance the load over different paths. The incentive price, which is commonly used in practice \cite{river2023lightning}, is necessary to provide channels with an incentive to serve as an intermediary for different channels. In practice, we expect both these components to be smaller than the imbalance price. Consequently, we expect the behavior of our protocol to be similar to our theoretical results even with these additional prices.

A key aspect of our protocol is that channel fees are allowed to be negative. Although the original Lightning network whitepaper \cite{poon2016bitcoin} suggests that negative channel prices may be a good solution to promote rebalancing, the idea of negative prices in not very popular in the literature. To our knowledge, the only prior work with this feature is \cite{varma2021throughput}. Indeed, in papers such as \cite{van2021merchant} and \cite{wang2024fence}, the price function is explicitly modified such that the channel price is never negative. The results of our paper show the benefits of negative prices. For one, in steady state, equal flows in both directions ensure that a channel doesn't loose any money (the other price components mentioned above ensure that the channel will only gain money). More importantly, negative prices are important to ensure that the protocol selectively stifles acyclic flows while allowing circulations to flow. Indeed, in the example of Section \ref{sec:flow_control_example}, the flows between nodes $A$ and $C$ are left on only because the large positive price over one channel is canceled by the corresponding negative price over the other channel, leading to a net zero price.

Lastly, observe that in the DEBT control protocol, the price charged by a channel does not depend on its capacity. This is a natural consequence of the price being the Lagrange multiplier for the net-zero flow constraint, which also does not depend on the channel capacity. In contrast, in many other works, the imbalance price is normalized by the channel capacity \cite{ren2018optimal, lin2020funds, wang2024fence}; this is shown to work well in practice. The rationale for such a price structure is explained well in \cite{wang2024fence}, where this fee is derived with the aim of always maintaining some balance (liquidity) at each end of every channel. This is a reasonable aim if a channel is to never rebalance itself; the experiments of the aforementioned papers are conducted in such a regime. In this work, however, we allow the channels to rebalance themselves a few times in order to settle on a detailed balance flow. This is because our focus is on the long-term steady state performance of the protocol. This difference in perspective also shows up in how the price depends on the channel imbalance. \cite{lin2020funds} and \cite{wang2024fence} advocate for strictly convex prices whereas this work and \cite{varma2021throughput} propose linear prices.

\paragraph{On Rebalancing} 
Recall that the DEBT control protocol ensures that the flows in the network converge to a detailed balance flow, which can be sustained perpetually without any rebalancing. However, during the transient phase (before convergence), channels may have to perform on-chain rebalancing a few times. Since rebalancing is an expensive operation, it is worthwhile discussing methods by which channels can reduce the extent of rebalancing. One option for the channels to reduce the extent of rebalancing is to increase their capacity; however, this comes at the cost of locking in more capital. Each channel can decide for itself the optimum amount of capital to lock in. Another option, which we discuss in Section \ref{sec:five_node}, is for channels to increase the rate $\gamma$ at which they adjust prices. 

Ultimately, whether or not it is beneficial for a channel to rebalance depends on the time-horizon under consideration. Our protocol is based on the assumption that the demand remains steady for a long period of time. If this is indeed the case, it would be worthwhile for a channel to rebalance itself as it can make up this cost through the incentive fees gained from the flow of transactions through it in steady state. If a channel chooses not to rebalance itself, however, there is a risk of being trapped in a deadlock, which is suboptimal for not only the nodes but also the channel.

\section{Conclusion}
This work presents DEBT control: a protocol for payment channel networks that uses source routing and flow control based on channel prices. The protocol is derived by posing a network utility maximization problem and analyzing its dual minimization. It is shown that under steady demands, the protocol guides the network to an optimal, sustainable point. Simulations show its robustness to demand variations. The work demonstrates that simple protocols with strong theoretical guarantees are possible for PCNs and we hope it inspires further theoretical research in this direction.
\section*{Impact statement}

This paper proposes that machine learning can and should be used to maximize social welfare. In principle, and by construction, the impact of our proposed framework on society aims to be positive. But our paper also points to the inherent difficulties of identifying, and making formal, what `good for society' is. We lean on the field of welfare economics, which has for decades contended with this challenge, for ideas on how the learning community can begin to approach this daunting task.
However, even if these ideas are conceptually appealing,
the path to practical welfare improvement presents many challenges---%
some expected, others unforseen.
% and will likely include many ups and downs.
For example, we may specify incorrect social welfare functions;
or we may specify them correctly but be unable to optimize them appropriately;
or we may be able to optimize but find that 
our assumptions are wrong, that theory differs from practice,
or that there were other considerations and complexities that we did not take into account.
For this we can look to other related fields---%
such as fairness, privacy, and alignment in machine learning---%
which have taken (and are still taking) similar journeys,
and learn from both their success and mistakes.
% and hope that ours will be similar.

Any discipline that seeks to affect policy should do so with much deliberation and care. Whereas welfare economics was designed with the explicit purpose of supporting (and influencing) policymakers,
machine learning has found itself in a similar position, but likely without any planned intent.
On the one hand, adjusting machine learning to support notions, such as social welfare,
that it was not designed to support initially can prove challenging.
However, and as we argue throughout, we believe that building on top of existing machinery is a more practical approach than to begin from scratch.
The necessity of confronting with welfare consideration can also
be an opportunity---as we can leverage these novel challenges
to make machine learning practice more informed, transparent, responsible, and socially aware.


% At the same time, the novelty of the challenges that welfare considerations present to the field make this an opportunity---%
% for chaning the role of machine learning in society for the better in a manner that is informed, transparent, and aware.

 

\extended{\todo{acks: david, students}}

\bibliographystyle{plainnat}
\bibliography{refs}

% \newpage
% \appendix
% \onecolumn



\end{document}


% This document was modified from the file originally made available by
% Pat Langley and Andrea Danyluk for ICML-2K. This version was created
% by Iain Murray in 2018, and modified by Alexandre Bouchard in
% 2019 and 2021 and by Csaba Szepesvari, Gang Niu and Sivan Sabato in 2022.
% Modified again in 2023 and 2024 by Sivan Sabato and Jonathan Scarlett.
% Previous contributors include Dan Roy, Lise Getoor and Tobias
% Scheffer, which was slightly modified from the 2010 version by
% Thorsten Joachims & Johannes Fuernkranz, slightly modified from the
% 2009 version by Kiri Wagstaff and Sam Roweis's 2008 version, which is
% slightly modified from Prasad Tadepalli's 2007 version which is a
% lightly changed version of the previous year's version by Andrew
% Moore, which was in turn edited from those of Kristian Kersting and
% Codrina Lauth. Alex Smola contributed to the algorithmic style files.
