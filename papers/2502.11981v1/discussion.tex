\section{Concluding Discussion} \label{sec:discussion}

Economics studies the tensions that arise from having limited means but unlimited wants.
Our main thesis is that while the appetite for what machine learning can give us only grows,
we do not yet have the tools to integrate the limitations inherent to any social system
in a way that supports informed policy choices.
Welfare economics provides general guidelines for designing systems that enable a social planner to express and promote outcomes that align with societal preferences, as they relate to scarcity in resources.
We have identified ways in which we believe this perspective applies to, and can be useful for, the design of learning systems that operate in social contexts and with human agency.
Scarcity in learned systems is often far more illusive than in classic economic settings that consider tangible goods.
But nonetheless, it exists, and its implications bear much significance for the value that we can hope to obtain from using machine learning as individuals.
% as a society.
%  in the social sphere.
Our proposal to augment conventional learning frameworks with a notion of social welfare
serves as a first step towards this goal. But the road ahead is long and challenging,
and requires careful deliberation of the societal role we envision for machine learning.
 


% % - the main lesson to learn from econ is that we can't just write down whatever we want as an objective and expect to get it. because others have interests in outcomes and will act to get them. 

% \subsection{Rationale of focus and design choices} \label{sec:rationale}
% % [short motivation for here?]
% % - supervised discriminative learning
% % - users as input
% % - prediction policies


 

