\section{Task Description}
The goal of this research is to develop models capable of performing semantic segmentation on legal documents by identifying and classifying rhetorical roles (RR) within the text. Let \( D = \{d_1, d_2, \dots, d_n\} \) represent a collection of legal documents, where \( d_i \in D \) consists of a sequence of sentences \( S_i = \{s_{i1}, s_{i2}, \dots, s_{im}\} \), with \( m \) representing the number of sentences in document \( d_i \). The task is to assign a rhetorical role label \( y_{ij} \in Y \) to each sentence \( s_{ij} \), where $Y$ is the predefined set of 7 rhetorical role labels.

Formally, the task can be described as:
\[
f: S_i \rightarrow Y
\]
\[
Y = \left\{
\begin{aligned}
&\text{Facts}, \, \text{Issue}, \, \text{Arguments of Petitioner}, \\
&\text{Arguments of Respondent}, \, \text{Reasoning},  \\
&\text{Decision}, \, \text{None}
\end{aligned}
\right\}
\]
where \( f \) is a function that maps each sentence \( s_{ij} \) in a document \( d_i \) to its corresponding rhetorical role label \( y_{ij} \). Thus, the goal is to find:
\[
f(s_{ij}) = y_{ij}, \quad \forall s_{ij} \in S_i, \quad y_{ij} \in Y
\]
The input to the system is a legal document \( d_i \), and the output is a sequence of rhetorical role labels corresponding to each sentence in the document:
\[
f(S_i) = \{y_{i1}, y_{i2}, \dots, y_{im}\}, \quad y_{ij} \in Y
\]




% \[
% \mathcal{L} = \left\{
% \begin{aligned}
% &\text{Facts}, \, \text{Issue}, \, \text{Arguments of Petitioner}, \\
% &\text{Arguments of Respondent}, \, \text{Reasoning},  \\
% &\text{Decision}, \, \text{None}
% \end{aligned}
% \right\}
% \]
