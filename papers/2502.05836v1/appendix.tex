\newpage
% \section{Appendix}
\section{Experimental Setup and Hyper-parameters}
We conducted experiments across several models, utilizing different architectures and training techniques tailored to rhetorical role classification tasks. Below, we provide an overview of the key experimental setups and hyper-parameters used.

\subsection{\texttt{RhetoricLLaMA} Training Procedure}
\texttt{RhetoricLLaMA}, built on the LLaMA-2-7B model, was fine-tuned with Bfloat16 precision using a single A100 GPU with 40GB memory. Given the computational constraints, the model was optimized for efficiency, with training lasting 48 hours. A maximum token length of 1000 was used, and Low-Rank Adaptation (LoRA) was employed with a rank of 16, alpha set to 64, and a dropout rate of 0.1. The model leveraged flash-attention 2 for faster training. We applied a Paged Adam optimizer with a learning rate of 1e-4 and a cosine learning rate scheduler, along with gradient accumulation steps of 4. The model trained for 52,617 steps, corresponding to 3 epochs.

\subsection{Transformers Training Hyper-parameters}
For the Role-Aware Transformers, built upon the InLegalBERT model, pre-training involved self-supervised tasks such as Masked Language Modeling (MLM) with role embeddings added. The model processed a maximum sequence length of 512 tokens with a batch size of 4, running for 20 epochs. The learning rate was set to 2e-5, using the AdamW optimizer. Class weights were applied to handle the imbalance in rhetorical roles, and early stopping was used to prevent overfitting.

\subsection{Graph Neural Networks (GNN) with Document Context}
We utilized a Graph Neural Network (GNN) architecture to model sentence relationships within legal documents. A two-layer Graph Convolutional Network (GCN) processed sentence embeddings from InLegalBERT. The first and second GCN layers both had output dimensions of 128, using ReLU activations. The model was trained for 10 epochs with a learning rate of 1e-4 and employed a Cross-Entropy Loss function. Graphs were constructed with edges between consecutive sentences, capturing both sequential and semantic relationships.

\subsection{Incorporating Previous Sentence and Actual Label}
In this method, the input to the model combined the current sentence with the previous sentence and its actual rhetorical role label. The model used InLegalBERT with a maximum sequence length of 512 tokens, trained for 5 epochs with a learning rate of 2e-5. This approach provided explicit sequential context and utilized Cross-Entropy Loss with class weights to manage class imbalance.

\subsection{Incorporating Previous Sentence and Predicted Label}
Extending the previous approach, this variant incorporated the predicted label of the previous sentence, simulating real-world conditions. The same configuration was used as in the previous model, but the predicted label replaced the actual label during both training and inference.

\subsection{Common Settings Across Models}
All models were evaluated using Accuracy, Precision, Recall, F1 Score, and Matthews Correlation Coefficient (MCC). The experiments utilized PyTorch and Hugging Face Transformers libraries, with PyTorch Geometric handling the graph data in the GNN method. All models were trained on machines with NVIDIA GPUs for parallel computation.

%%%%%%%%%%%%%%%%%%%%%%%%%%%%%%%%%%%%
\begin{table*}[h!]
\centering
\resizebox{\textwidth}{!}{%
\begin{tabular}{|l|l|}
\hline
\textbf{Label}                   & \multicolumn{1}{|c|}{\textbf{{Sentences}}}
% \begin{tabular}[c]{@{}c@{}}\textbf{Sentences}    \end{tabular}       
\\ \hline
Fact                    & \begin{tabular}[c]{@{}l@{}}For the sake of convenience, we are referring to the facts of Civil Appeal No.1328 of 2021.\end{tabular}                                              \\ \hline
Fact                    & \begin{tabular}[c]{@{}l@{}}At the time of the assessment proceedings, the Assessee submitted a revised computation \\of income by revising its claim of deduction under Section 80IA of the Act.\end{tabular}                                                                                                                                                   \\ \hline
Issue   & \begin{tabular}[c]{@{}l@{}}The Income Tax Appellate Tribunal (hereinafter the Tribunal), upheld the decision of the \\Appellate Authority on the issue of deduction under Section 80IA.\end{tabular}                                                                                                                                                          \\ \hline
Issue   & \begin{tabular}[c]{@{}l@{}}The High Court refused to interfere with the Tribunals order as far as the issue on deduction \\under Section 80IA is concerned.\end{tabular}                                                                                      \\ \hline
\begin{tabular}[c]{@{}l@{}}Arguments of \\Petitioner (AoP)  \end{tabular}              & \begin{tabular}[c]{@{}l@{}}Mr. Arijit Prasad, learned Senior Counsel appearing on behalf of the Revenue, submitted \\that Section 80AB of the Act contemplates deductions in respect of incomes against income \\of the nature specified in the relevant section.\end{tabular}                                                        \\ \hline
\begin{tabular}[c]{@{}l@{}}Arguments of \\Petitioner (AoP)  \end{tabular}                 & \begin{tabular}[c]{@{}l@{}}According to him, the phrase derived from in subsection (1) of Section 80IA of the Act \\indicates that the computation of deduction is restricted only to the profits and gains from the \\eligible business.\end{tabular} \\ \hline

\begin{tabular}[c]{@{}l@{}}Arguments of \\Respondent (AoR)  \end{tabular}                & \begin{tabular}[c]{@{}l@{}}In response, the Assessee supported the order passed by the Appellate Authority which was \\upheld by the Tribunal and the High Court.\end{tabular}                                                                                                                        \\ \hline

\begin{tabular}[c]{@{}l@{}}Arguments of \\Respondent (AoR)  \end{tabular}                 & \begin{tabular}[c]{@{}l@{}}He submitted that there is no indication in subsection (5) of Section 80IA that the deduction \\under subsection (1) is restricted to business income only.\end{tabular}                                                                                                                             \\ \hline

Reasoning  & \begin{tabular}[c]{@{}l@{}}As stated above, Section 80AB was inserted in the year 1981 to get over a judgment of this \\Court in Cloth Traders (P) Ltd. (supra).\end{tabular}                                                                                                     \\ \hline
Reasoning   & \begin{tabular}[c]{@{}l@{}}On the question of existence of vacancies, although learned counsel for the appellant \\submitted that vacancies are still lying there, which submission however has been refuted by \\the learned counsel for the State of Rajasthan.\end{tabular}                                                                                                                                                       \\ \hline
Decision & \begin{tabular}[c]{@{}l@{}}For the aforementioned reasons, the Appeal is dismissed qua the issue of the extent of \\deduction under Section 80IA of the Act.\end{tabular}                                                                                        \\ \hline
Decision & \begin{tabular}[c]{@{}l@{}}The assets of the Corporate Debtor shall be managed strictly in terms of the provisions of \\the IBC.\end{tabular}                                                                                                                                                                              \\ \hline
None                    & Clause 11(b) reads as follows 11.                                                                                                                                                                                                                                                                                                                 \\ \hline
None                    & The clause reads thus 12 Miscellaneous.                                                                                                                                                                                                                                                                                  \\ \hline

\end{tabular}
}
\caption{Example sentences for each label.}
\label{tab:rr-examples}
\end{table*}
%%%%%%%%%%%%%%%%%%%%%%%%%%%%%%%%%%%%%%%%%

%%%%%%%%%%%%%%%%%%%%%%%%%%%%%%%%%%%%%%%%%
\begin{table*}[h]
\centering
\resizebox{\textwidth}{!}{%
\begin{tabular}{|c|l|}
\hline
\multicolumn{2}{|c|}{\textbf{\textcolor{blue}{Instruction Sets}}} \\ \hline

\multicolumn{1}{|c|}{1}  &  \begin{tabular}[c]{@{}l@{}}Analyze the given legal sentence and predict its rhetorical role as a number: None-0, Facts-1, Issue-2, \\Arguments of Petitioner-3, Arguments of Respondent-4, Reasoning-5, Decision-6.\end{tabular} \\ \hline

\multicolumn{1}{|c|}{2}  &  \begin{tabular}[c]{@{}l@{}}Determine the rhetorical function of this sentence from a court case and provide its corresponding number: \\None-0, Facts-1, Issue-2, Arguments of Petitioner-3, Arguments of Respondent-4, Reasoning-5, Decision-6.\end{tabular} \\ \hline

\multicolumn{1}{|c|}{3}  &  \begin{tabular}[c]{@{}l@{}}Based on the content of the following legal text, classify its rhetorical role by selecting the appropriate number: \\None-0, Facts-1, Issue-2, Arguments of Petitioner-3, Arguments of Respondent-4, Reasoning-5, Decision-6.\end{tabular} \\ \hline

\multicolumn{1}{|c|}{4}  &  \begin{tabular}[c]{@{}l@{}}Identify the rhetorical category of this legal statement and provide the number that represents it: None-0, \\Facts-1, Issue-2, Arguments of Petitioner-3, Arguments of Respondent-4, Reasoning-5, Decision-6.\end{tabular} \\ \hline

\multicolumn{1}{|c|}{5}  &  \begin{tabular}[c]{@{}l@{}}Evaluate the rhetorical purpose of the provided legal sentence and label it with the correct number: None-0, \\Facts-1, Issue-2, Arguments of Petitioner-3, Arguments of Respondent-4, Reasoning-5, Decision-6.\end{tabular} \\ \hline

\multicolumn{1}{|c|}{6}  &  \begin{tabular}[c]{@{}l@{}}Assign a number to the rhetorical role of this sentence from a legal case, choosing from: None-0, Facts-1, \\Issue-2, Arguments of Petitioner-3, Arguments of Respondent-4, Reasoning-5, Decision-6.\end{tabular} \\ \hline

\multicolumn{1}{|c|}{7}  &  \begin{tabular}[c]{@{}l@{}}Review the legal statement and predict its rhetorical function using the corresponding number: None-0, \\Facts-1, Issue-2, Arguments of Petitioner-3, Arguments of Respondent-4, Reasoning-5, Decision-6.\end{tabular} \\ \hline

\multicolumn{1}{|c|}{8}  &  \begin{tabular}[c]{@{}l@{}}Examine this legal text and determine its rhetorical role by outputting the appropriate number: None-0, Facts-1, \\Issue-2, Arguments of Petitioner-3, Arguments of Respondent-4, Reasoning-5, Decision-6.\end{tabular} \\ \hline

\multicolumn{1}{|c|}{9}  &  \begin{tabular}[c]{@{}l@{}}Categorize the rhetorical purpose of the following sentence from a court proceeding with a number: None-0, \\Facts-1, Issue-2, Arguments of Petitioner-3, Arguments of Respondent-4, Reasoning-5, Decision-6.\end{tabular} \\ \hline

\multicolumn{1}{|c|}{10} &  \begin{tabular}[c]{@{}l@{}}Analyze the provided legal sentence and classify it into its rhetorical role, outputting only the number: None-0,\\ Facts-1, Issue-2, Arguments of Petitioner-3, Arguments of Respondent-4, Reasoning-5, Decision-6.\end{tabular} \\ \hline

\multicolumn{1}{|c|}{11} &  \begin{tabular}[c]{@{}l@{}}Determine the appropriate number for the rhetorical category of this legal text: None-0, Facts-1, Issue-2, \\Arguments of Petitioner-3, Arguments of Respondent-4, Reasoning-5, Decision-6.\end{tabular} \\ \hline

\multicolumn{1}{|c|}{12} &  \begin{tabular}[c]{@{}l@{}}Assign a numerical label to the rhetorical role of this statement in a legal case: None-0, Facts-1, Issue-2, \\Arguments of Petitioner-3, Arguments of Respondent-4, Reasoning-5, Decision-6.\end{tabular} \\ \hline

\multicolumn{1}{|c|}{13} &  \begin{tabular}[c]{@{}l@{}}Predict the number that corresponds to the rhetorical function of the following legal sentence: None-0, Facts-1, \\Issue-2, Arguments of Petitioner-3, Arguments of Respondent-4, Reasoning-5, Decision-6.\end{tabular} \\ \hline

\multicolumn{1}{|c|}{14} &  \begin{tabular}[c]{@{}l@{}}Identify the number that represents the rhetorical role of this legal text: None-0, Facts-1, Issue-2, Arguments of \\Petitioner-3, Arguments of Respondent-4, Reasoning-5, Decision-6.\end{tabular} \\ \hline

\multicolumn{1}{|c|}{15} &  \begin{tabular}[c]{@{}l@{}}Analyze this legal statement and assign the number that best matches its rhetorical function: None-0, Facts-1, \\Issue-2, Arguments of Petitioner-3, Arguments of Respondent-4, Reasoning-5, Decision-6.\end{tabular} \\ \hline

\multicolumn{1}{|c|}{16} &  \begin{tabular}[c]{@{}l@{}}Classify the following sentence from a court case by selecting its rhetorical role number: None-0, Facts-1, \\Issue-2, Arguments of Petitioner-3, Arguments of Respondent-4, Reasoning-5, Decision-6.\end{tabular} \\ \hline
\end{tabular}
}
\caption{Instruction Sets for Predicting the Roles}
\label{tab:instruction_sets}
\end{table*}
%%%%%%%%%%%%%%%%%%%%%%%%%%%%%%%%%

% % %%%%%%%%%%%%%%%%%%%%%%%%%%%%%%%%%%%
% \begin{figure}[ht]
%     \centering
%     \includegraphics[width=\linewidth]{figures/Hier_BiLSTM CRF.pdf}
%     \caption{Confusion matrix for rhetorical role classification using Hierarchical BiLSTM-CRF model.}
%     \label{fig:hier_bilstm_crf}
% \end{figure}
% % %%%%%%%%%%%%%%%%%%%%%%%%%%%%%%%%%%%%

% %%%%%%%%%%%%%%%%%%%%%%%%%%%%%%%%%%%
\begin{figure}[ht]
    \centering
    \includegraphics[width=\linewidth]{figures/MTL.pdf}
    \caption{Confusion matrix for rhetorical role classification using the Multi-Task Learning (MTL) model.}
\label{fig:mtl}

\end{figure}
% %%%%%%%%%%%%%%%%%%%%%%%%%%%%%%%%%%%%

% %%%%%%%%%%%%%%%%%%%%%%%%%%%%%%%%%%%
\begin{figure}[ht]
    \centering
    \includegraphics[width=\linewidth]{figures/GNN.pdf}
    \caption{Confusion matrix for rhetorical role classification using GNN.}
\label{fig:gnn}
\end{figure}
% %%%%%%%%%%%%%%%%%%%%%%%%%%%%%%%%%%%%


% %%%%%%%%%%%%%%%%%%%%%%%%%%%%%%%%%%%
\begin{figure}[ht]
    \centering
    \includegraphics[width=\linewidth]{figures/ToInLegalBERT.pdf}
    \caption{Confusion matrix for rhetorical role classification using TransformerOverInLegalBERT (ToInLegalBERT).}
\label{fig:toinlegalbert}
\end{figure}
% %%%%%%%%%%%%%%%%%%%%%%%%%%%%%%%%%%%%

% %%%%%%%%%%%%%%%%%%%%%%%%%%%%%%%%%%%
\begin{figure}[ht]
    \centering
    \includegraphics[width=\linewidth]{figures/RhetoricLLaMA.pdf}
    \caption{Confusion matrix for rhetorical role classification using \texttt{RhetoricLLaMA}, an instruction-tuned large language model.}
\label{fig:rhetoricllama}
\end{figure}
% %%%%%%%%%%%%%%%%%%%%%%%%%%%%%%%%%%%%

% %%%%%%%%%%%%%%%%%%%%%%%%%%%%%%%%%%%
\begin{figure}[ht]
    \centering
    \includegraphics[width=\linewidth]{figures/InLegalBERT_i.pdf}
    \caption{Confusion matrix for rhetorical role classification using InLegalBERT model with the current sentence (i) as input.}
    \label{fig:inlegalbert_i}
\end{figure}
% %%%%%%%%%%%%%%%%%%%%%%%%%%%%%%%%%%%%

% %%%%%%%%%%%%%%%%%%%%%%%%%%%%%%%%%%%
\begin{figure}[ht]
    \centering
    \includegraphics[width=\linewidth]{figures/InLegalBERT_i_minus_1_i.pdf}
    \caption{Confusion matrix for rhetorical role classification using InLegalBERT model with the current sentence (i) and the previous sentence (i-1) as input.}
\label{fig:inlegalbert_i-1_i}

\end{figure}
% %%%%%%%%%%%%%%%%%%%%%%%%%%%%%%%%%%%%

% %%%%%%%%%%%%%%%%%%%%%%%%%%%%%%%%%%%
\begin{figure}[ht]
    \centering
    \includegraphics[width=\linewidth]{figures/InLegalBERT_i_minus_2_i_minus_1_i.pdf}
    \caption{Confusion matrix for rhetorical role classification using InLegalBERT model with the previous-to-previous sentence (i-2), previous sentence (i-1), and the current sentence (i) as input.}
\label{fig:inlegalbert_i-2_i-1_i}

\end{figure}
% %%%%%%%%%%%%%%%%%%%%%%%%%%%%%%%%%%%%

% %%%%%%%%%%%%%%%%%%%%%%%%%%%%%%%%%%%
\begin{figure}[ht]
    \centering
    \includegraphics[width=\linewidth]{figures/InLegalBERT_i_minus_1_i_i_plus_1.pdf}
    \caption{Confusion matrix for rhetorical role classification using InLegalBERT model with the current sentence (i), previous sentence (i-1), and next sentence (i+1) as input.}
\label{fig:inlegalbert_i-1_i_i+1}

\end{figure}
% %%%%%%%%%%%%%%%%%%%%%%%%%%%%%%%%%%%%

% %%%%%%%%%%%%%%%%%%%%%%%%%%%%%%%%%%%
\begin{figure}[ht]
    \centering
    \includegraphics[width=\linewidth]{figures/InLegalBERT_i_minus1_label_t_i.pdf}
    \caption{Confusion matrix for rhetorical role classification using InLegalBERT model with the true label of the previous sentence (i-1) and the current sentence (i) as input.}
\label{fig:inlegalbert_label_t}

\end{figure}
% %%%%%%%%%%%%%%%%%%%%%%%%%%%%%%%%%%%%

% %%%%%%%%%%%%%%%%%%%%%%%%%%%%%%%%%%%
\begin{figure}[ht]
    \centering
    \includegraphics[width=\linewidth]{figures/InLegalBERT_i_minus_1_label_p_i.pdf}
    \caption{Confusion matrix for rhetorical role classification using InLegalBERT model with predicted label of the previous sentence (i-1) and the current sentence (i) as input.}
\label{fig:inlegalbert_label_p}

\end{figure}
% %%%%%%%%%%%%%%%%%%%%%%%%%%%%%%%%%%%%
% %%%%%%%%%%%%%%%%%%%%%%%%%%%%%%%%%%%%
% \begin{figure}[ht] % [h] 
%     \centering 
%     \includegraphics[width=\linewidth]{figures/rhetorical_roles_pie_chart.pdf} 
%     \caption{Distribution of Rhetorical Roles within the Dataset.}
%     \label{pie-chart}
% \end{figure}
% %%%%%%%%%%%%%%%%%%%%%%%%%%%%%%%%%%%%