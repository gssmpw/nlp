\section*{Limitations}
While this study makes significant strides in rhetorical role classification for legal documents, a few areas remain where further refinement could enhance the approach. These areas are opportunities for future work rather than major limitations and are not expected to diminish the contribution of this research.

The \texttt{LegalSeg} dataset, while being the largest and most comprehensive of its kind for Indian legal judgments, is understandably specialized in the context of the Indian judiciary system. This focus provides unique insights into this particular legal domain. However, it is acknowledged that the models may require adaptation to handle legal documents from different jurisdictions. This does not limit the validity of our findings but opens a path for future research into cross-jurisdictional generalization using transfer learning techniques or domain adaptation strategies, which are common challenges in domain-specific NLP.

The class imbalance in the dataset, which is inherent in most real-world legal corpora, reflects the natural distribution of rhetorical roles in judgments. While some roles like Issue and Decision are less frequent, this mirrors their actual occurrence in legal texts. We have taken steps to mitigate this issue through advanced modeling techniques such as label shift prediction and the incorporation of contextual information. Future work could explore further enhancements, such as data augmentation or more refined class-weighting techniques, to boost performance on the less frequent roles.

Additionally, the computational requirements of models like ToInLegalBERT and \texttt{RhetoricLLaMA} are justified given the complexity and the high accuracy they provide. These models are aligned with state-of-the-art practices in NLP, which involve significant computational demands. While this may pose a challenge for deployment in low-resource environments, it is important to note that high-performance models are typically developed on powerful infrastructures and then optimized for more practical use cases through techniques such as model pruning, quantization, or distillation, which can be addressed in future work.

The overlap in rhetorical roles, such as between Facts and Reasoning, is an inherent challenge in legal discourse due to the intertwined nature of legal arguments and fact presentation. The models already handle these overlaps competently, and our use of sequential and contextual information improves performance. However, we recognize that future refinements, such as more sophisticated context-aware mechanisms or hybrid models that integrate symbolic reasoning with machine learning, could offer even greater differentiation between closely related roles.

In conclusion, the challenges discussed here are not insurmountable and represent common issues in the evolving field of legal NLP. This work provides a strong foundation for addressing these aspects, and we are confident that the solutions proposed will inspire future innovations and improvements. This manuscript significantly advances the state of the art in rhetorical role classification, and any remaining opportunities for refinement will only serve to further enhance the impact of this research.