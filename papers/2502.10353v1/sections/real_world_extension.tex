\section{Extending Policies to Real-World Considerations}
\label{sec:real_world}
We extend our model to incorporate three real-world matching phenomena: 1) batch-offering assortments, 2) patient cognitive load, and 3) fairness considerations. 

\subsection{Batch Offering Patients}
\label{sec:batching}
Thus far, we have assumed that all menus are constructed up front, in a single ``batch.'' Healthcare administrators can sacrifice logistical ease to improve match quality by making offers to groups of patients in batches rather than all at once.
By offering assortments in batches, administrators increase control over the response order but increase the total response time for all patients to respond. 
By assumption, the response time for each patient is exponentially distributed (see Section~\ref{sec:model}), and the expected maximum of $N$ exponential variables is proportional to $\log(N)$~\citep{max_exponential}.
By offering $L$ batches, our total response time increases from $\log(N)$ to $L \log(N/L)$. 

We aim to construct batches, $b_{1},b_{2},\ldots,b_{L} \subseteq [N]$, so that the resulting response order maximizes match quality. 
Formally, let $S(b_{1},b_{2},\ldots,b_{L})$ be a distribution over response orders so that any patient $i \in b_{k}$ is before any patient $i' \in b_{k'}$ with $k < k'$. 
Next, let $\sigma^{*}$ be the optimal ordering for some fixed policy $\pi$: $\sigma^{*} = \argmax_{\sigma} \frac{1}{N} \sum_{t=1}^{N} \sum_{t=1}^{N}  (f_{\sigma_{t}}\left(\pi(\theta,\sigma)_{\sigma_{t}} \odot \mathbf{y}^{(t)}\right) \cdot \theta_{\sigma_{t}})
$. 
We aim to set $b_{1},b_{2},\ldots,b_{L}$ so that $\sigma \sim S(b_{1},b_{2},\ldots,b_{L})$ and $\sigma^{*}$ are similar. 

To do so, we first characterize $\sigma^{*}$ for a policy $\pi^{A}(\theta,\sigma)$, which augments the pairwise policy using knowledge of $\sigma$, $\pi^{A}(\theta,\sigma)_{i,j} \geq \pi^{P}(\theta)_{i,j} \forall i,j$. 
Recall that $v_{i} = j$ if $\pi^{P}(\theta)_{i,j} = 1$, and let $v^{-1}_{j} = i$ if $v_{i}=j$. 
Then, $\pi^{A}(\theta)_{i,j} = 1$ if $\theta_{i,j} \geq \theta_{i,v_{i}}$ and $\sigma^{-1}_{i} \geq \sigma^{-1}_{v^{-1}_{j}}$, where $\sigma^{-1}_{i} = t$ if $\sigma_{t} = i$. 
Essentially, $\pi^{A}(\theta)$ augments the pairwise policy $\pi^{P}(\theta)$ to add pairs $i,v_{i'}$ for patients $i'$ preceding $i$ in $\sigma$. 
We focus on such a policy because it represents a natural extension of the pairwise policy to incorporate the order $\sigma$. 
We characterize $\sigma^{*}$ for the policy $\pi^{A}(\theta)$ then use this to construct the batches $b_{k}$: 
\begin{restatable}{lemma2}{thmordering} 
    Let $G=(V,E)$ be a graph with $N$ nodes such that node $i$ is connected to $i'$ if $\theta_{i,v_{i}} \leq \theta_{i,v_{i'}}$ or $v_{i} = -1$ and $v_{i'} \neq -1$. 
    Let $\sigma^{*}$ be the optimal ordering for the policy $\pi^{A}$.
    Then, traversing the nodes defined by $\sigma^{*}$ is a reverse topological ordering on $G$. 
\end{restatable}
We demonstrate this by considering a pair of patients $i$ and $i'$, so that $i$ precedes $i'$ in any reverse topological order, but $i$ precedes $i'$ in $\sigma^{*}$. 
We then use case work to show that we can either a) show that there is a cycle in the graph $G$, leading to a contradiction, b) recurse on a smaller pair of patients, $w,i$ or $i',w$, or c) show that swapping $i$ and $i'$ in $\sigma$ can only increase the match quality. 

We next construct $b_{1},b_{2},\ldots,b_{L}$, so $\sigma \sim S(b_{1},b_{2},\ldots,b_{L})$ is close to the optimal ordering $\sigma^{*}$. 
The idea is to partition $\sigma^{*}$ into $L$ batches; then we know that all patients in $b_{1}$ precede $b_{2}$. 
To do this, we first demonstrate that if no pairs of patients within a batch share a common descendant, then such a batching  preserves the match quality of $\sigma^{*}$:  
\begin{restatable}{theorem}{optimalordering}
    Let $\sigma^{*}$ be an optimal ordering.
    Let $G=(V,E)$ be a graph with $N$ nodes such that node $i$ is connected to $i'$ if $\theta_{i,v_{i}} \leq \theta_{i,v_{i'}}$ or $v_{i} = -1$ and $v_{i'} \neq -1$. 
    Suppose that there exists a partition of $\sigma^{*}$ into $K$ batches, $b_{1},b_{2},\ldots,b_{L}$, such that for any partition $b_{k}$, no $i, i' \in b_{k}$ have a common descendant in $G=(V,E)$. 
    Then $\pi^{A}$ achieves the same match quality under partition $g$  as under the optimal ordering: 
    \begin{equation}
        \mathbb{E}_{\sigma \sim S(b_{1},b_{2},\ldots,b_{L})}[\frac{1}{N} \sum_{t=1}^{N}  (f_{\sigma_{t}}\left(\pi^{A}(\theta)_{\sigma_{t}} \odot \mathbf{y}^{(t)}\right) \cdot \theta_{\sigma_{t}})] =\frac{1}{N}  \sum_{t=1}^{N}  (f_{\sigma^{*}_{t}}\left(\pi^{A}(\theta)_{\sigma_{t}} \odot \mathbf{y}^{(t)}\right) \cdot \theta_{\sigma_{t}}) 
    \end{equation}
\end{restatable}
This holds because the selection made by patient $i$ depends only on the descendants of patient $i$ on the graph G. 
If $b_{1},b_{2},\ldots,b_{K}$ ensures that any pair of descendants are in different batches, then the order of descendants is maintained, so patients select the same providers in $\sigma$ and $\sigma^{*}$. 

In practice, we compute the batches $b_{k}$ through the following: 
\begin{enumerate}
    \item Computing the graph $G=(V,E)$ with $i$ connected to $i'$ if $\theta_{i,v_{i'}} > \theta_{i,v_{i}}$.
    \item Computing a reverse topological ordering $\sigma^{*}$.
    \item Selecting batches with the minimum number of edges within a batch (via dynamic programming).
\end{enumerate}
We note that this is an approximation of optimal batches for two reasons: 1) while $\sigma^{*}$ is a reverse topological ordering, there could be multiple reverse topological orderings, and 2) minimizing the number of edges within a batch does not guarantee that the order of descendants is maintained. 

\subsection{Maximum Assortment Sizes}
\label{sec:choice_overload}
To incorporate patient cognitive load~\citep{choice_overload}, we incorporate restrictions on the maximum assortment size into our model.
Large assortments might be undesirable or practically infeasible because patients only consider a subset of the options they are presented due to logistical or time constraints. 
Formally, we suppose that patients only consider a random $d$-subset $\mathbf{z}^{(d)}$ of the options they are presented at random, $z^{(d)}_{j} \leq X_{i,j} y^{(t)}_{j}$ with $\lVert \mathbf{z}^{(d)} \rVert_{1} \leq d$. 
Patients make decisions according to $f_{i}(\mathbf{z}^{(d)})$ rather than $f_{i}(\pi(\theta) \odot \mathbf{y}^{(t)})$. 
When $d$ is small, policies should offer small tailored assortments, as large assortments lead to patients ignoring many options. 

\subsection{Fair Matches}
\label{sec:fair}
Policies should aim to achieve fair matches to encourage participation and maintain equity. 
We construct examples of fairness metrics through the prior literature on fair matches and list these below~\citep{fairness_optimization,fair_minimax}. 
\begin{itemize}
    \item \textbf{Minimum Match Quality} - One natural objective is to maximize the minimum match quality across all matched patients. Such an objective is natural when administrators aim to maintain a certain level of care across patients. 
    \item \textbf{Variance in Match Quality} - Another approach minimizes variance in match quality to ensure that the discrepancy in match quality between patients is small. 
    \item \textbf{Match Quality Range} - One final approach to match quality fairness is to minimize the absolute difference between the maximum and minimum match quality for matched patients. 
\end{itemize}
Each notion of fairness has different implications for the optimal policy. 
For example, optimizing for the minimum match quality might lead policies to only offer assortments to high match quality patient-provider pairs, whereas minimizing the variance in match quality might yield many low quality matches.  
We explore tradeoffs in match quality and fairness in Section~\ref{sec:real_world_considerations}. 