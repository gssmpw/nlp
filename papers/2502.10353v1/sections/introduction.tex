\section{Introduction}
\label{sec:intro}
Primary care providers are essential to the healthcare ecosystem because they are the first point of contact for many patients~\citep{physician_trust,physician_trust_adherence}. 
Patients rely on primary care providers for routine checkups and referrals to specialists. 
Moreover, care continuity can instill trust and improve medication uptake rates and patient health~\citep{physician_trust_adherence}. 

Unfortunately, high provider turnover rates frequently lead to patients without an assigned provider~\citep{pcp_turnover}. 
Provider turnover disrupts patient care and leads to worse care~\citep{pcp_turnover}. 
In principle, healthcare administrators reassign unmatched patients to other providers; however, in practice, the process takes months due to provider scarcity and the logistical burden of rematching and coordinating patient matches~\citep{finding_new_provider}.
While many patients find their new provider quickly, others have to wait years to find a new provider due to large numbers of patients, high turnover rates, and provider scarcity~\citep{finding_new_provider,provider_burnout}.

Algorithms that automatically match patients and providers can reduce logistical hassle but require balancing patient autonomy and system-wide utility. 
For example, while automatically assigning each patient to a provider would decrease wait times, it also reduces patient autonomy because patients cannot select their provider~\citep{patient_autonomy,gaynor_free_2016}. 
In contrast, offering patients full autonomy could lead to suboptimal matches; for example, patients who rematch quickly might prevent better matches for late-responding patients. 
In addition, granting patients full independence can be overwhelming and delay decisions~\citep{bate_choice_2005}. 
% 
We propose to match patients and providers through an assortment framework~\citep{assortment_school,assortment_dating,assortment_mnl}. 
Assortments are menus of options given to customers who select an option from such a menu~\citep{assortment_optimization}. 
Assortments are used in e-commerce settings where platforms give customers a suggested list of products and have them select from this list~\citep{e_commerce_assortment}. 
In an assortment setting, each patient receives a menu of providers and selects a provider from such a menu. 
These curated offer sets allow administrators to balance patient autonomy and system-wide match utility. 

We devise policies for offering assortments in a patient-provider matching scenario. 
To do so, we first develop a model of patient-provider matching then construct a set of policies for offering assortments. 
We characterize how policy performance depends on problem characteristics such as patient-to-provider ratio, and we empirically detail the regimes under which each of our policies performs best.  
We then evaluate our policies in a real-world scenario constructed on top of a Medicare dataset to assess the performance of our policies in a real-world example. 
\footnote{We include all code at \url{https://github.com/naveenr414/patient-provider}}

Our key question is the following: \textit{How should healthcare administrators design provider menus for patients to optimize system-wide match rates and quality?}

\subsection{Contributions}
We develop a model of patient-provider matching using assortment offerings and analyze its impact on match rate and match quality. 
We characterize well-performing policies when varying properties such as the patient/provider ratio.
We then empirically analyze these policies in a synthetic and real-world dataset built using a Medicare dataset. 
We use the insights from these studies to propose recommendations for designing patient-provider matching systems in practice. 

\paragraph{Developing a Model of Patient-Provider Matching}
We develop a model of patient-provider matching where administrators offer assortments upfront, then patients sequentially respond in a random order and select providers one by one. 
We do so because it captures two critical elements: 1) patients have autonomy in selecting providers and 2) administrators have some control over the matching process by varying assortments offered to patients. 
We model patient decisions through a choice model that captures how the set of providers offered impacts the final provider chosen. 
The selection of choice model is key to the fidelity of our model, so we analyze various choice models, including the uniform choice model and the multinomial logit choice model.
Finally, we capture patient heterogeneity through a match quality matrix, which captures differences in match quality between pairs of patients and providers. 
Match quality can encompass geographical proximity, provider specialties, and concordance along language, race, and gender~\citep{race_concordance,gender_concordance,language_concordance}. 
Patients tend to match with higher quality providers, though the exact match probabilities are dictated by their choice model. 

\paragraph{Constructing Assortment Policies}
We characterize the optimal policy in the single-provider scenario and use these insights to construct a set of policies for assortment optimization. 
We first construct a baseline policy, greedy, which offers every provider to every patient. 
We demonstrate that the greedy policy can do arbitrarily poorly, so we build upon this by developing three new policies that adaptively vary assortment sizes: pairwise, group-based, and gradient descent.
The pairwise policy offers each patient at most one provider, and we demonstrate an instance-dependent approximation guarantee for this policy. 
We then develop the group-based policy by augmenting the pairwise policy so we offer assortments to groups of patients. 
We demonstrate that such group-based maintains the match rate of the pairwise policy while potentially improving the match quality. 
For the gradient descent policy, we construct an objective function that provably lower bounds the match quality, and we use gradient descent to optimize this quantity. 

\paragraph{Characterizing Policy Performance}
We analyze the performance of our policies in a synthetic setting and empirically characterize the conditions under which each policy performs best. 
We demonstrate that the best-performing policy depends on two factors: the match probability and the ratio of patients to providers. 
When patients outnumber providers, as is common in many real-world health systems, the gradient descent policy performs best because the objective function is closer to the lower bound. 
When patient and provider counts are balanced, we show that the greedy policy achieves a high match quality when patients have low match probabilities, while the group-based policy achieves a high match quality when patients have high match probabilities. 

We evaluate policies in a real-world scenario constructed via Medicare data~\citep{Medicare} and we show here that the gradient descent policy performs best. 
We extend this scenario to real-world considerations such as fairness and cognitive overload. 

\paragraph{Contribution Summary}
Overall, we make three contributions to the patient-provider matching and assortment optimization literature: we i) develop a model of patient-provider matching under assortments, ii) characterize the performance of assortment policies, and detail how the best-performing policy depends on problem characteristics such as patient match probability and patient/provider ratio, and iii) make recommendations for the implementation of a real-world patient-provider matching system based on the results of our theoretical and empirical analysis.  
\begin{figure*}
    \centering 
    \includegraphics[width=\textwidth]{figures/patient_provider.pdf}
    \caption{A) In this work, we match patients with providers B) by offering a set of assortments to patients, where each assortment consists of a set of providers. C) Patients then respond sequentially and either select a provider from their assortments or abstain. D) This results in matches between patients and providers.}
    \label{fig:pull}
\end{figure*}

\subsection{Related Work}
\textbf{Matching Patients and Providers} 
Care continuity is critical for patient health, as it allows for better patient communication, lowers operating costs, and improves provider teamwork~\citep{lower_turnover}. 
Despite this, provider turnover rates are high, with the average healthcare system reporting a 7\% turnover rate year-to-year~\citep{lower_turnover}. 
Moreover, such issues are exacerbated recently due to high rates of provider burnout~\citep{provider_burnout} and the Covid-19 pandemic~\citep{provider_burnout_covid19}. 
Once a provider drops out, it is difficult for patients to find a new one; while 54\% of patients find their new provider within 12 months, 6\% fail to find a new provider even after 36 months~\citep{finding_new_provider}. 
High levels of provider turnover can worsen primary care because patients lack access to routine checkups or reminders to take medication~\citep{pcp_turnover,importance_primary_care,race_concordance_adherence,concordance_race_cardio}. 

While provider turnover impacts patient satisfaction, quantifying ``match quality'' is hard. 
Patients care mainly about geographic proximity~\citep{patient_decisions}, with concordance along race~\citep{race_concordance}, gender~\citep{gender_concordance}, and language~\citep{language_concordance} playing a secondary role. 
Communication style alignment can improve match quality~\citep{personal_factor_concordance}. These factors build trust and lead patients to more truthfully report symptoms~\citep{race_concordance_lung}. 

\textbf{Algorithms for Patient-Provider Matching}
Existing work in patient-provider matching uses either a one-shot matching framework or a genetic programming-based approach. 
Within the one-shot framework, linear programs can manage provider panel sizes~\citep{panel_size_lp}, the deferred acceptance framework can perform two-stage matching~\citep{two_stage_matching}, and scheduling algorithms are used to model optimal rates of on-demand scheduling~\citep{on_demand_percent}. 

A second approach to patient-provider matching builds matches through genetic programming. 
Genetic programming algorithms learn matches over time through a fitness function. 
Within this framework, one line of work uses genetic programming to find fair matches between patients and providers~\citep{two_stage_matching}, while another approach uses genetic programming to balance workloads between providers~\citep{genetic_non_dominated}. 
We build on these lines of work by incorporating patient autonomy into the process, which motivates the need for an assortment-based approach. 

\textbf{Assortment Optimization} 
Our work analyzes patient-provider matching through assortment optimization, where retailers offer a set of menus to customers. 
In assortment optimization, retailers construct assortments and customers make decisions based on offered options~\citep{assortment_optimization}. 
Retailers first construct models of customer decisions, such as the multinomial logit model~\citep{assortment_mnl} and nested logit models~\citep{assortment_nested_logit}, both of which use logit probabilities to model customer decisions. 
Assortments are employed in domains including school choice~\citep{assortment_school}, dating markets~\citep{assortment_dating}, and e-commerce~\citep{e_commerce_assortment}. 

Assortment optimization algorithms come in offline and online variants. 
In offline assortment optimization, assortments are offered one-shot and customers make decisions simultaneously and independently~\citep{assortment_optimization}. 
In online assortment optimization, customers arrive sequentially, and options are offered for each~\citep{assortment_online}. 
Our setting resides between offline and online, as we offer assortments offline, but patients respond online. 

\textbf{Matching under Random Ordering} 
Patient-provider matching is challenging from an optimization perspective because patients make decisions in a random order. 
Randomized response orders are found in variants of online selection problems such as secretary problems~\citep{secretary_problem} and online learning~\citep{online_learning}. 
One key difference is the lack of flexibility in our scenario; changing assortments online for the patient-provider matching setting would involve considerable logistical hassle, so we offer assortments in a one-shot manner. 