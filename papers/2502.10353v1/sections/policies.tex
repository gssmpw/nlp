\section{Constructing Matching Policies}
\label{sec:policies}
We introduce policies $\pi(\theta)$ to construct assortments. 
We first introduce a baseline greedy policy and discuss why such a policy can perform poorly. 
We then propose three new policies: pairwise, which assigns each provider a single patient, group-based, which improves upon the weaknesses of the pairwise policy, and gradient descent, which optimizes for a lower bound on the match quality. 

\subsection{Greedy Policy}
\label{sec:greedy}
The greedy policy offers all providers to each patient and grants patients full autonomy. 
Formally, the greedy policy is $\pi^{R}(\theta)_{i,j} = 1$, for every patient $i$ and provider $j$. 
The greedy policy maximizes the match rate under the uniform choice model because $\mathbf{X}_{\sigma_{t}} = \mathbf{1}$, and so $\lVert f_{\sigma_{t}}(\mathbf{X}_{\sigma_{t}} \odot \mathbf{y}^{(t)})\rVert_{1} = \lVert f_{\sigma_{t}}(\mathbf{y}^{(t)}) \rVert_{1}$.

For match quality, the greedy policy considers assortments myopically and ignores the structure of the match quality matrix, which can lead to poor performance. 
For example, if $M=1$ and $\theta_{1,1} >> \theta_{1,2}$, the greedy policy gives both patients equal chances of matching, whereas the optimal policy offers $j=1$ only to patient $i=1$. 
Formally, let $\pi^{*}$ be the optimal policy, $\pi^{*}(\theta) = \argmax_{\mathbf{X}} \mathbb{E}_{\sigma}[\frac{1}{N} \sum_{t=1}^{N} (f_{\sigma_{t}}\left(\mathbf{X}_{\sigma_{t}} \odot \mathbf{y}^{(t)}\right)  \cdot \theta_{\sigma_{t}})]$. 
We first prove that $\pi^{R}$ is an $\epsilon$-approximation: 

\begin{restatable}{theorem}{thmgreedy}
    Let $f_{i}$ be the uniform choice model with match probability $p$. 
    For any $p$ and $\epsilon$, there exists a $\theta$ such that 
    \begin{equation}
        \mathbb{E}_{\sigma}[\frac{1}{N} \sum_{t=1}^{N} (f_{\sigma_{t}}\left(\pi^{R}(\theta)_{\sigma_{t}} \odot \mathbf{y}^{(t)}\right)  \cdot \theta_{\sigma_{t}})] \leq \epsilon \mathbb{E}_{\sigma}[\frac{1}{N} \sum_{t=1}^{N} (f_{\sigma_{t}}\left(\pi^{*}(\theta)_{\sigma_{t}} \odot \mathbf{y}^{(t)}\right)  \cdot \theta_{\sigma_{t}})]
    \end{equation}
\end{restatable}

We prove this by generalizing the example where $\theta_{1,1} >> \theta_{2,1}$ to arbitrary $N$ and $M$. 
Here, the optimal policy is to offer each provider to only one patient. 
However, the greedy policy offers all providers to all patients, which leads to sub-optimal matches for each provider $j$. 
Moreover, we show that we can construct scenarios with $N=M=\frac{3}{\epsilon p}$ where the greedy policy is an $\epsilon$-approximation. 
For large $p$, even small $M$ and $N$ lead to situations where the greedy policy performs poorly. 

\subsection{Pairwise Policy}
\label{sec:pairwise}
The pairwise policy inverts the greedy policy by offering each patient at most one provider. 
This strategy reduces patient choices while better incorporating match quality structure. 
The pairwise policy, $\pi^{P}(\theta)$, pairs patients and providers by solving the weighted bipartite matching problem: 
\begin{equation}
   \max\limits_{\pi^{P}(\theta), \sum_{i=1}^{N} \pi^{P}(\theta)_{i,j} \leq 1, \sum_{j=1}^{M} \pi^{P}(\theta)_{i,j} \leq 1} \sum_{i=1}^{N} \sum_{j=1}^{M} \pi^{P}(\theta)_{i,j} \theta_{i,j}
\end{equation}
Under a uniform choice model, the choice of $p$ impacts the pairwise policy performance: 
\begin{restatable}{theorem}{thmlp}
\label{thm:lp}
    Let $f_{i}$ be the uniform choice model with match probability $p$. 
    Then 
    \begin{equation}
        \mathbb{E}_{\sigma}[\frac{1}{N} \sum_{t=1}^{N} (f_{\sigma_{t}}\left(\pi^{P}(\theta)_{\sigma_{t}} \odot \mathbf{y}^{(t)}\right)  \cdot \theta_{\sigma_{t}})] \geq p \mathbb{E}_{\sigma}[\frac{1}{N} \sum_{t=1}^{N} (f_{\sigma_{t}}\left(\pi^{*}(\theta)_{\sigma_{t}} \odot \mathbf{y}^{(t)}\right)  \cdot \theta_{\sigma_{t}})]
    \end{equation}

\end{restatable}
To prove this, we first upper bound the optimal policy, $\pi^{*}(\theta)$, with the value of the underlying bipartite matching problem. 
We then show that the pairwise policy corresponds to the bipartite matching problem, where edges correspond to pairs of patients and providers, and each edge exists with probability $p$. 
Summing over edges gives that the pairwise policy is a $p$-approximation to the bipartite matching problem and is, therefore, a $p$-approximation to $\pi^{*}$. 

Beyond the uniform choice model, we can show a similar guarantee in the MNL scenario. 
For the MNL choice model, the exit option $\gamma$ is analogous to $p$ in controlling policy performance.
First, let $v_{i}=j$ if $\pi^{P}(\theta)_{i,j}=1$ and let $v_{i} = -1$ if $\sum_{j=1}^{M} \pi^{P}(\theta)_{i,j} = 0$. 
Then the MNL choice model is equivalent to the uniform choice model with $p_{i}=\frac{\mathbbm{1}[v_{i} \geq 0] \exp(\theta_{i,v_{i}})}{\exp(\gamma) + \mathbbm{1}[v_{i} \geq 0] \exp(\theta_{i,v_{i}})}$. 
We can then upper bound optimal match quality as $\frac{1}{N} \sum_{i=1}^{N} \mathbbm{1}[v_{i} \geq 0] \theta_{i,v_{i}}$, while the pairwise policy achieves $ \sum_{i=1}^{N} \frac{p_{i}}{N} \mathbbm{1}[v_{i} \geq 0] \theta_{i,v_{i}}$. 
Let $a_{i} = \mathbbm{1}[v_{i} \geq 0] \theta_{i,v_{i}}$, then the pairwise policy achieves an approximation ratio of: 
\begin{equation}
    \frac{\sum_{i=1}^{N} a_{i} \frac{\exp(a_{i})}{\exp(\gamma)+\exp(a_{i})}}{\sum_{i=1}^{N} a_{i}} \geq \frac{\exp(\frac{1}{N} \sum_{i=1}^{N} a_{i}) (\frac{1}{N}\sum_{i=1}^{N} a_{i})}{(\exp(\gamma)+\exp(\frac{1}{N}\sum_{i=1}^{N} a_{i}))(\frac{1}{N}\sum_{i=1}^{N} a_{i})} \geq \frac{\exp(\frac{1}{N} \sum_{i=1}^{N} a_{i})}{\exp(\gamma)+\exp(\frac{1}{N}\sum_{i=1}^{N} a_{i})}  
\end{equation}
Small $\gamma$ improves pairwise performance, corresponding to patients more inclined to match. 

\subsection{Group-Based Policy}
\label{sec:grouping}
The pairwise policy performs poorly for small $p$, so we develop the group-based policy. 
When $p$ is small, administrators should offer patients larger assortments because a patient's top preference is more likely to be available. 
For example, consider a scenario with $N$ patients and $M$ providers. 
We demonstrate a performance gap between greedy and pairwise policy dependent on $p$: 
\begin{restatable}{proposition2}{thmgrouping}
\label{thm:grouping}
Let $f_{i}$ be the uniform choice model with match probability $p$. 
If $\theta_{i,j} \sim U(0,1)$ and $M \leq N$ then
\begin{equation}
    \frac{\mathbb{E}_{\sigma,\theta}[\frac{1}{N} \sum_{t=1}^{N}  (f_{\sigma_{t}}\left(\pi^{R}(\theta)_{\sigma_{t}} \odot \mathbf{y}^{(t)}\right) \cdot \theta_{\sigma_{t}})]}{\mathbb{E}_{\sigma,\theta}[\frac{1}{N} \sum_{t=1}^{N}  (f_{\sigma_{t}}\left(\pi^{P}(\theta)_{\sigma_{t}} \odot \mathbf{y}^{(t)}\right) \cdot \theta_{\sigma_{t}})]} \geq \frac{1-(1-p)^{N/M}}{2p}
\end{equation}

\end{restatable}
We demonstrate this by showing that the pairwise policy achieves a match quality of $Mp$. 
We then compute the match probability for the greedy policy as $1-(1-p)^{N/M}$ for each of the $M$ providers, and noting that the match quality for each matched provider is at least $\frac{1}{2}$ in expectation. 
The pairwise policy performs poorly for small $p$, so we develop a new policy that modulates between the pairwise policy (for large $p$) and the greedy policy (for small $p$). 
To do this, we add a provider $j'$ to the assortment of patient $i$ if $\theta_{i,j'} \geq \theta_{i,v_{i}}$; that is, patients should be assigned preferred providers. 
However, such a policy could decrease match rates because some patients have empty assortments. 
\begin{example}
    \label{ex:group_based}
    Let $\theta$ be the following: 
    \begin{equation} 
        \theta =
        \begin{bmatrix}
        0.6 & 0.7 \\
        0.3 & 0.6 \\
        \end{bmatrix}
    \end{equation}
    Here, $\pi^{P}(\theta)_{1,1} = \pi^{P}(\theta)_{2,2} = 1$. 
    We add provider $j=2$ to patient $i=1$, so $X_{i,2} = 1$. 
    Let $\sigma=[1,2]$. 
    Then, with probability $p$, provider $j=2$ is selected by patient $i=1$ and $\mathbf{y}^{(2)} = [1,0]$ so patient $i=2$ has no options. 
    This reduces the match rate from $p$ to $p-\frac{p^2}{2}$. 
\end{example}

We overcome this issue by developing the group-based policy, $\pi^{G}(\theta)$. 
The group-based policy constructs groups so that each group receives the same assortment. 
In Example~\ref{ex:group_based}, this would mean that $\mathbf{X}_{1}=\mathbf{X}_{2} = [1,1]$. 
We demonstrate that such a grouping strategy is the only way to maintain a match rate while building upon the pairwise policy: 
\begin{restatable}{theorem}{thmgroupingmatch}
    Let $\pi$ be a policy that augments the pairwise policy; $\pi(\theta)_{i,j} \geq \pi^{P}(\theta)_{i,j}$ for any $\theta$ for all $i,j$. 
    Let $G = (V,E)$ be a directed graph with $N$ nodes such that nodes $i$ and $i'$ are connected if $\pi(\theta)_{i,v(\theta)_{i'}} = 1$ . 
    Here, $v(\theta)_{i} = j$ if $\pi^{P}(\theta)_{i,j} = 1$.
    If $N=M$, then 
    \begin{equation}
        \mathbb{E}_{\sigma}[\frac{1}{N} \sum_{t=1}^{N} \lVert f_{\sigma_{t}} \left(\pi(\theta)_{\sigma_{t}} \odot \mathbf{y}^{(t)} \right) \rVert_{1}] = \mathbb{E}_{\sigma}[\frac{1}{N}\sum_{t=1}^{N} \lVert f_{\sigma_{t}} \left(\pi^{P}(\theta)_{\sigma_{t}} \odot \mathbf{y}^{(t)} \right) \rVert_{1}] = p
    \end{equation}
    if and only if each component in $G$ is a complete digraph.      
\end{restatable}
We prove this by showing that when $N=M$, all patients can match with a provider with probability $p$, leading to an overall match rate of $p$. 
To construct groups, we first compute edge weights for each pair of patients in $G$, denoted as $\alpha_{i,i'}$.
Here, $\alpha_{i,i'}$ captures the benefit when placing $(i,v_{i})$ and $(i',v_{i'})$ in the same assortment. 
Next, we construct groups to maximize the sum of pairwise edges within each group. 
We do so through a linear program, with $z_{i,j}$ representing whether we include edge $(i,j)$ and $q_{i}$ representing whether we include node $i$ in the group. 
We place details on computing $\alpha_{i, i'}$ and constructing the assortments in Algorithm~\ref{alg:grouping}. 

\begin{algorithm}[h]
   \caption{Group-based policy ($\pi^{G}$)}
\begin{algorithmic}[1]
    \STATE {\bfseries Input:} Pairwise policy, $\pi^{P}$ and match quality $\theta$
    \STATE {\bfseries Output:} Assortment, $\mathbf{X}$
    \STATE Initialize $\mathbf{X} = \mathbf{X}^{P} = \pi^{P}(\theta)$, $v_{i}=j$ if $X_{i,j}^{P}=1$, and $\mathbf{c} = \mathbf{1}$
    \FORALL{$(i,i') \subseteq [N]$}
        \STATE Let $\mathbf{X}^{\prime} = \mathbf{X}^{P}$ 
        \STATE Let ${X}^{\prime}_{i,v_{i'}}=1$ and $X^{\prime}_{i',v_{i}}=1$ 
        \STATE Let $\alpha_{i,i'} =\mathbb{E}_{\sigma \sim [\{i,i'\},\{i',i\}]}[\sum_{t=1}^{2} (f_{\sigma_{t}}\left(\mathbf{X}^{\prime}_{\sigma_t} \odot \mathbf{y}^{(t)}\right) \cdot \theta_{\sigma_{t}} - f_{\sigma_{t}}\left(\mathbf{X}^{P}_{\sigma_t} \odot \mathbf{y}^{(t)}\right) \cdot \theta_{\sigma_{t}})]$
    \ENDFOR

    \STATE Let $s=\infty$ and $r=\{1,\ldots,N\}$
    \WHILE{s > 0}
        \STATE Let $s = \max\limits_{\mathbf{z}} \sum_{i \in r} \sum_{i' \in r} z_{i,j} \alpha_{i,i'}$, subject to $z_{i,j} = q_{i} q_{j}$, and let $\mathbf{z}$ be the corresponding solution
        \FOR{$i,i' \in r$ with $z_{i,i'}=1$}
            \STATE Let $X_{i,v_{i'}}=X_{i',v_{i}}=1$
            \STATE Remove $i,i'$ from $r$ 
        \ENDFOR 
    \ENDWHILE
\end{algorithmic}
\label{alg:grouping}
\end{algorithm}

\subsection{Gradient Descent Policy}
\label{sec:lower_bound}
The group-based policy makes local improvements to the pairwise policy but can fail because it only considers pairwise interactions. 
To fix this, we develop the gradient descent policy by considering the global structure in $\theta$. 
Our gradient descent policy uses gradient descent to optimize an objective function $g(f(\mathbf{X}))$, which represents a lower bound on the match quality of an assortment $\mathbf{X}$ under the uniform choice model. 
Here, $f(\mathbf{X})_{i,j}$ lower bounds the probability that provider $j$ is available, while $g(f(\mathbf{X}))_{i,j} \in [0,1]^{N \times M}$ lower bounds the probability that provider $j$ is the top choice for patient $i$: $g(f(\mathbf{X})) \leq \mathrm{Pr}[j = \argmax_{j'} \theta_{i,j'} \mathbf{y}^{(t)}_{j'}]$.
We first detail how to compute $f$ and $g$, then prove that this lower bounds the match quality (see Algorithm~\ref{alg:provider}): 
\begin{enumerate}
    \item $f(\mathbf{X})$ lower bounds the availability probability for provider $j$. We first note that $f(\mathbf{X})_{i,j}=0$ if $X_{i,j} = 0$. Next, availability depends on the response order $\sigma$. For example, if $\sigma_{1}=i$, then provider $j$ is available if $X_{i,j}=1$. Next, if $\sigma_{t}=i$, then under a uniform choice model, there is at least a $(1-p)^{t-1}$ availability probability. This holds because, in the worst case, the $t-1$ preceding patients also prefer provider $j$, so the probability none of the $t-1$ patients select $j$ is $(1-p)^{t-1}$. 
    Next, if $n$ out of $N$ patients have provider $j$ on their assortment, then the probability that a given patient $i' \neq i$ matches to $j$ is $p \frac{n-1}{N-1}$. 
    That is, if patient $i$ is in the t-th position, then the availability probability is at least $(1-p \frac{n-1}{N-1})^{t-1}$. Averaging over all $N$ potential values for $t$ gives: 
    \begin{equation}
       f(\mathbf{X})_{i,j} = X_{i,j} h(\lVert \mathbf{X}_{*,j} \rVert_{1})  , h(n) = \frac{1}{N} \sum_{t=1}^{N} (1-p \frac{n-1}{N-1})^{t-1}
    \end{equation}
    
    \item $g(f(\mathbf{X}))$ computes the probability that provider $j$ is the top available provider.  
    Let $u_{i,1}, u_{i,2},\ldots,u_{i,N}$ be coefficients so $\theta_{i,u_{1}} \geq \theta_{i,u_{2}} \cdots \theta_{i,u_{M}}$. 
    Then provider $u_{i,1}$ is the top option whenever it is available; $g(f(\mathbf{X}))_{i,u_{i,1}}) = f(\mathbf{X})_{i,u_{i,1}}$. 
    Similarly, $u_{i,2}$ is only selected if $u_{i,1}$ is not selected; we can estimate $g(f(\mathbf{X}))_{i,u_{i,2}}) = f(\mathbf{X})_{i,u_{i,2}} (1-f(\mathbf{X})_{i,u_{i,1}})$. 
    We can generalize this as: 
    \begin{equation}
        g(f(\mathbf{X}))_{i,u_{i,k}} = f(\mathbf{X})_{i,u_{i,k}} \prod_{k'=1}^{k-1} (1-f(\mathbf{X})_{i,u_{i,k'}})
    \end{equation}
    The expected match quality is then $p(g(f(\mathbf{X})) \cdot \theta)$.  
\end{enumerate} 
% \begin{figure*}
%     \centering 
%     \includegraphics[width=\textwidth]{figures/gradient_descent.pdf}
%     \caption{We compute the objective for gradient descent in two steps. First, given an assortment $\mathbf{X}$, we compute the number of patients with provider $j$ on their assortment. We compute $h(n)$ for each these values, and set $f(\mathbf{X})_{i,j} = h(\lVert \mathbf{X}_{*,j} \rVert_{1})$. Next, we use the results from $f(\mathbf{X})$ to compute $g(f(\mathbf{X}))$, by taking the products of each row of $f(\mathbf{X})$ from the most preferred to least preferred provider.} 
%     \label{fig:lower_bound}
% \end{figure*}


\begin{algorithm}[h]
   \caption{Gradient Descent Objective}
\begin{algorithmic}
    \STATE {\bfseries Input:} Match quality $\theta$ and Assortment $\mathbf{X}$
    \STATE {\bfseries Output:} Estimated total match quality
    \STATE Let $h(n) = \frac{1}{N} \sum_{t=1}^{N} (1-p \frac{n-1}{N-1})^{t-1}$ 
    \STATE Let $f(\mathbf{X})_{i,j} = h(\lVert \mathbf{X}_{*,j} \rVert_{1}) \mathbf{X}_{i,j}$ 
    \STATE Let $u_{i,1},u_{i,2},\ldots,u_{i,M}$ be a permutation of $[M]$, so $\theta_{i,u_{i,1}} \geq \theta_{i,u_{i,1}} \cdots \theta_{i,u_{i,M}}$
    \STATE Let $g(f(\mathbf{X}))_{i,u_{k}} = f(\mathbf{X})_{i,u_{k}} \prod_{k'=1}^{k-1} (1-f(\mathbf{X})_{i,u_{k'}})$
    \RETURN $p*g(f(\mathbf{X})) \cdot \theta$
\end{algorithmic}
\label{alg:provider}
\end{algorithm}

\begin{restatable}{theorem}{thmlowerbound}
    
    The following holds for any $\mathbf{X}$:
    \begin{equation}
        p*(g(f(\mathbf{X})) \cdot \theta) \leq \mathbb{E}_{\sigma}[\sum_{t=1}^{N} (f_{\sigma_{t}}\left(\mathbf{X}_{\sigma_{t}} \odot \mathbf{y}^{(t)}\right)  \cdot \theta_{\sigma_{t}})]  
    \end{equation}
\end{restatable}

Moreover, we demonstrate tightness when using the pairwise policy: 
\begin{restatable}{lemma}{thmtightlowerbound}
    When $\mathbf{X} = \pi^{P}(\theta)$, then 
    \begin{equation}
        p*(g(f(\mathbf{X})) \cdot \theta) = \mathbb{E}_{\sigma}[\sum_{t=1}^{N} (f_{\sigma_{t}}\left(\mathbf{X}_{\sigma_{t}} \odot \mathbf{y}^{(t)}\right)  \cdot \theta_{\sigma_{t}})]  
    \end{equation}
\end{restatable}

We demonstrate this by showing that for any $i$, $\sum_{j=1}^{m} p g(f(\mathbf{X}))_{i,j}$ is an underestimation for the probability that any of the top-$m$ providers are available. 
We then show that because $\theta_{i,u_{i,j}}$ is sorted from high to low, we can similarly prove that $\sum_{j=1}^{m} p (g(f(\mathbf{X}))_{i,j} \theta_{i,j}$ is also an underestimation for each patient $i$. 
In the scenario where $\mathbf{X} = \pi^{P}(\theta)$, we can explicitly compute $g(f(\mathbf{X}))$ because each patient matches with at most one provider. 

Key to our policy is the tightness of the lower bound. 
When each patient receives more providers the lower bound is looser so the gradient descent policy performs worse. 
Conversely, when $N>>M$, each patient can only receive a few providers, so our policy performs better. 
Essentially, when there is low ``inter-provider interference'', or patients receiving large assortments (relative to the number of providers), our policy performs better. 
The gradient descent policy can also naturally extend to optimize match rate if $\theta_{i,j}$ is constant; in that scenario, we optimize for $\sum_{i=1}^{N} \sum_{j=1}^{M} g(f(\mathbf{X}))_{i,j}$. 

\subsection{Worked Example with Policies}
\label{sec:example_policies}
We motivate the need for different policies beyond greedy through a simple example with $N=3$ patients and $M=1$ providers.
\begin{example}
    Consider a scenario with $p=0.75$ and $\theta = [0.7,0.7,0.1]$. 
    According to Theorem~\ref{thm:one_m}, the optimal assortment in such a scenario is $\mathbf{X} = [1,1,0]$, which achieves a match quality of $0.22$. 
    We next compare our four policies: 
    \begin{enumerate}
        \item \textbf{Greedy} offers $\mathbf{X} = [1,1,1]$, which leads to an average match quality of $\frac{(1-(1-p)^{N}) \frac{1}{N} \sum_{i=1}^{N} \theta_{i,1}}{N} = 0.16$. Greedy achieves a low match quality because it offers a provider to patient $i=3$, which drags down the average match quality.
        \item \textbf{Pairwise} offers $\mathbf{X} = [1,0,0]$, which achieves a match quality of $0.18$. While the pairwise policy improves upon the greedy policy, the symmetry between patients one and two implies that it should offer a provider to patient $i=2$, as it can improve the match probability. 
        \item \textbf{Group-Based} first computes $\alpha_{1,2} = 0.13$ and $\alpha_{1,3} = -0.15$ and constructs a group with patients 1 and 2. It then offers $\mathbf{X} = [1,1,0]$, the optimal assortment for match quality. 
        \item \textbf{Gradient Descent} optimizes for $g(f(\mathbf{X})) \cdot \theta = f(\mathbf{X}) \cdot \theta = \sum_{i=1}^{N} X_{i,1} \theta_{i,1} h(\lVert \mathbf{X}_{*,j} \rVert_{1})$, which is optimizes at $\mathbf{X} = [1,1,0]$ and is the optimal assortment for match quality.  
    \end{enumerate}
\end{example}
The greedy policy fails because it offers the provider to patients with a low match quality, while the pairwise policy fails because it offers providers to too few patients. 
The group-based and gradient descent policies rectify this situation by adapting assortment size based on the match quality and correctly offering the provider to patients 1 and 2. 