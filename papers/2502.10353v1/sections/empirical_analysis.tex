\section{Empirical Analysis}
\label{sec:empirical}
We analyze policies in synthetic and real-world settings to characterize how assortment policy impacts match quality and rate. 
First, we demonstrate that gradient descent is the best policy when patients outnumber providers. 
We then show that group-based and greedy policies perform well for large and small values of match probability $p$, respectively.   
Our results demonstrate that assortment policies should be selected based on problem specifics. 

\subsection{Experimental Details}
We compare the four policies from Section~\ref{sec:policies} (greedy, pairwise, group-based, and gradient descent) and a random baseline with $\pi(\theta)_{i,j} \sim \text{Ber}(\frac{1}{2})$. 
For a fixed problem setup, we randomly sample $T$ permutations of patients and compute the match quality and match rate for the corresponding patient order ($\sigma$). We average runs across 15 seeds and $T=100$ trials and plot the standard error across these seeds. 
% For each policy, we randomly sample $\sigma$ for $T$ trials and computing the match quality and match rate for each $\sigma$. 
Because the scale of match quality and rate might vary between experiments, we report the normalized match quality and match rate (norm. MQ and norm. MR for short), which divides each quantity by that of the random policy.  
% We plan to release code + data after reviewing.

\subsection{Synthetic Examples}
\label{sec:synthetic}
We construct synthetic examples to gain insight into our policies across different scenarios. 
For each example, we specify values for $M$, $N$, $f_{i}$, and $\theta$. 

\begin{example}
\label{ex:small}
\begin{figure*}
    \centering 
    \includegraphics[width=\textwidth]{figures/small_patient_provider.pdf}
    \caption{We find the optimal policy and compute the approximation ratio for match quality compared to optimal. While all non-random policies achieve approximation ratios $\geq 0.9$ when $M \geq N$ (left), the pairwise policy performs poorly and achieves a low approximation ratio when $N \geq M$ (right).} 
    \label{fig:small}
\end{figure*}
\textbf{Few patients and providers} - We first analyze policies with a small number of patients $N$ and number of providers $M$. 
Small $N$ and $M$ allow us to compare to the optimal policy $\pi^{*}$ because we can compute $\pi^{*}$ by enumerating all $2^{NM}$ assortments and selecting the one that maximizes match quality. 
We run two experiments: the first fixes $N=2$ while varying $M$ from $2$ to $5$, and the second fixes $M=2$ while varying $N$ from $2$ to $5$. 
For both, we uniformly distribute $\theta$, $\theta_{i,j} \sim U(0,1)$, and let the choice model be uniform choice with $p=1/2$.
We compute the match quality approximation ratio to understand how close each policy is to the optimal. 

In Figure~\ref{fig:small}, we demonstrate that when $M \geq N$ (left), all non-random policies perform well and achieve within 3.5\% of optimal, while for $N \geq M$ (right), the pairwise policy performs poorly. 
When $M \geq N$, providers are abundant so each patient can match with their most preferred provider. 
However, when $N \geq M$, the pairwise policy performs up to 28\% worse than optimal. Providers are scarce in this scenario, so policies should offer the same provider to multiple patients. 
The pairwise policy only offers each provider one patient, which is suboptimal and leads to poor match quality.
\end{example}


\begin{example}
\label{ex:patient_provider}
\begin{figure*}
    \centering 
    \includegraphics[width=\textwidth]{figures/vary_providers_patients.pdf}
    \caption{When there are more patients than providers, the gradient descent policy performs best, both for uniformly and normally distributed $\theta$.}
    \label{fig:patient_providers}
\end{figure*}

\textbf{Varied patient/provider ratio} - 
In real-world situations, patients outnumbers providers~\citep{healthcare_shortage}, so we analyze how the patient/provider ratio impacts policy performance. 
We fix the number of providers $M=25$, while varying the number of patients $N$ from $25$ up to $200$. 
We compare policies under two match quality distributions: the first uniformly distributes $\theta \sim U(0,1)$, while the second normally distributes $\theta_{i,j} \sim \mathcal{N}(\mu_{j},0.1)$. 
The former corresponds to heterogeneous preferences where patient preferences are non-correlated, while the latter corresponds to homogenous preferences, where patient preferences are influenced by some signal of provider quality $\mu_{j}$. 
For both experiments, we fix $p=0.5$ for the uniform choice model. 

In Figure~\ref{fig:patient_providers}, we demonstrate that the gradient descent policy performs best when patients outnumber providers. 
When $N/M=8$, the gradient descent policy performs 5\% better than alternatives for uniform $\theta$ and 9\% for normal $\theta$. 
The gradient descent policy performs best for large $N/M$ because $g(f(\mathbf{X}))$ better approximates match quality when $N>M$; see Section~\ref{sec:lower_bound} for details.
Both greedy and group-based policies perform similarly, while the pairwise policy performs worst, mirroring trends from Example~\ref{ex:small}. 
In Appendix~\ref{sec:more_experiments}, we similarly compare policies according to the match rate and find that the gradient descent and greedy policies both maximize the match rate. 
\end{example}

\begin{example}
\label{ex:comparison}
\begin{figure*}
    \centering 
    \includegraphics[width=\textwidth]{figures/policy_comparison.pdf}
    \caption{The greedy policy performs poorly when $p=0.9$ compared to smaller $p$, for both uniform and normal $\theta$, while the pairwise policy performs worse for smaller $p$, especially when $\theta$ is normally distributed. The group-based policy gets around these issues and performs well when $p$ is large and small.} 
    \label{fig:comparison}
\end{figure*}
\textbf{Impact of $p$ and $\theta$} - We assess the impact of match quality $\theta$ and match probability $p$ on policy performance with balanced patient and provider counts. 
We fix $N=M=25$, vary $p \in \{0.1,0.25,0.5,0.75,0.9\}$, and distribute $\theta$ either uniformly or normally. 

In Figure~\ref{fig:comparison}, we demonstrate that the group-based policy achieves high match quality. 
As noted in Proposition~\ref{thm:grouping}, the group-based policy can improve upon the pairwise policy when $p$ is small. 
The pairwise policy performs especially poorly when $\theta$ is normally distributed because patient preferences are homogeneous, as all patients prefer some provider $j^{*} = \argmax_{j} \mu_{j}$. 
The pairwise policy only places $j^{*}$ in one patient's assortment, whereas $j^{*}$ should optimally be in multiple assortments due to its preferability. 
We also note the greedy and group-based policies perform well across $p$ and $\theta$; both policies are always within 6\% of the best-performing policy. 
In Appendix~\ref{sec:more_experiments}, we demonstrate that policies achieve similar match rates across $\theta$ and $p$. 

\begin{figure*}
    \centering 
    \subfloat[\centering Uniform]{\includegraphics[width=0.45 \textwidth]{figures/phase_diagram.pdf}}
    \subfloat[\centering Normal]{\includegraphics[width=0.45 \textwidth]{figures/phase_diagram_normal.pdf}}

    \caption{We characterize which policy maximizes match quality when varying patient/provider ratio and match probability $p$ when $\theta$ is uniformly (left) and normally (right) distributed. The group-based policy performs best for large $p$, the greedy policy performs best for small $p$, and the gradient descent policy performs best when the patient/provider ratio is large.}
    \label{fig:phase}
\end{figure*}
To understand which policy performs best across $p$, $\theta$, and $N/M$, we plot the policy which maximizes match quality when varying $p \in \{0.1,0.3,0.5,0.7,0.9\}$ and $N/M \in \{1,2,3,4,5,6,7,8\}$ with $M=10$, and distributing $\theta$ uniformly (Figure~\ref{fig:phase}a) and normally (Figure~\ref{fig:phase}b). 
The gradient descent policy performs best for large patient/provider ratios, which matches the results from Example~\ref{ex:patient_provider}. 
The group-based policy performs well for small $p$ because the problem more closely resembles bipartite matching (see Theorem~\ref{thm:lp}), while the greedy policy performs well for small $p$ because it becomes advantageous to offer larger assortments (see Proposition~\ref{thm:grouping}).
We finally note the random policy performs best when $p=0.1$ and $N/M=1$ because policies have high variance in that situation, making it difficult to ascertain which policy performs best. 
\end{example}

\begin{example}
\label{ex:choice_model}
    \textbf{Varied Choice Model} - 
We evaluate the impact of choice models on policy performance by investigating the MNL and threshold choice models. 
We fix $N=M=25$, let $\theta$ be distributed uniformly, and vary $\gamma \in \{0,0.1,0.25,0.5\}$ for the MNL choice model and vary  $\alpha \in \{0,0.1,0.25,0.5,0.75\}$ with $p=0.5$ for the threshold choice model. . 

\begin{figure*}
    \centering 
    \includegraphics[width=\textwidth]{figures/other_choice_models.pdf}
    \caption{We compare policy performances when varying the exit option $\gamma$ for the MNL choice model, and the threshold $\alpha$ for the threshold choice model. 
    For the MNL choice model, when $\gamma$ is small, the greedy policy performs poorly, while for large $\gamma$, all non-random policies perform similarly. For the threshold choice model, all policies perform similarly across choices of $\alpha$.}
    \label{fig:other_choice}
\end{figure*}

In Figure~\ref{fig:other_choice}, we show that $\gamma$ determines the performance of the greedy policy for the MNL choice model, while policies perform similarly across $\alpha$ for the threshold choice model. 
As noted in Section~\ref{sec:pairwise}, the choice of $\gamma$ is inversely related to the setting of $p$, with small $\gamma$ analogous to large $p$. Similar to the trends in Figure~\ref{fig:comparison}, we find that the greedy policy performs 22\% worse than the best policy when $\gamma=0.1$. 
For the threshold choice model,  $\alpha$ has little impact on the best policy, as greedy, pairwise, and gradient descent perform similarly across $\alpha$. 
In Appendix~\ref{sec:more_experiments}, we explore variants of the uniform choice model where the match probability $p$ is misspecified.  

\end{example}

\subsection{Real-World Dataset}
We work with a partner healthcare organization to assess the performance of our policies in a simulation that mimics situations faced in practice. 

\subsubsection{Constructing Data from Healthcare Providers}
\label{sec:semi_synthetic_data}
We work with a large healthcare provider in Connecticut to develop a simulation that reflects their system dynamics. 
We select values of $N$, $M$, $\theta$, and $f_{i}$: 
\begin{enumerate}
    \item \textbf{$N$ and $M$} - We estimate $N=1225$ and $M=700$ from the average panel size and provider count of the organization.
    \item \textbf{Choice Model $f_{i}$} - We select the choice model based on previous work which shows that patients are low-effort decision makers who primarily make decisions based on geographic proximity~\citep{choosing_doctor,distance_rural}.
    We incorporate these two factors and let $f_{i}$ be a threshold model, with $\alpha$ representing the maximum distance patients are willing to travel. 
    \item \textbf{Match Quality $\theta$} - We construct $\theta$ to reflect  geographic proximity and the presence of comorbidities. 
    Geographic proximity is the top factor impacting patient match quality~\citep{washington_distance,distance_rural}, while patients with comorbidities are best served by providers with the corresponding specialized training. 
    Formally, for a patient $i$ and provider $j$, let $d_{i,j}$ be the distance between the patient and provider, and let $\beta_{i,j}$ denote whether patient $i$'s comorbidity and $j$'s specialty match. 
    For example, if patient $i$ has a heart-related comorbidity and provider $j$ has a cardiology speciality, then $\beta_{i,j}=1$. 
    We manually match patient comorbidities to provider specialties. 
    We therefore let match quality $\theta_{i,j} = \alpha + (1-\alpha)(\Delta \beta_{i,j} + (1-\Delta)(\frac{\bar{d}}{d_{i,j}}-1)$. 
    Here, $\Delta$ weights between comorbidities and distance and $\alpha$ sets the threshold match quality for a patient at distance $d_{i,j}=\bar{d}$.  
\end{enumerate}
We compute $d_{i,j}$ by sampling patient locations from Connecticut zip codes and obtaining provider locations from a Medicare dataset~\citep{Medicare}. 
We obtain $\beta_{i,j}$ from prior work on comorbidity rates ~\citep{age_comorbidity} and using information on provider specialties from a Medicare dataset~\citep{Medicare}. 
We let $p=0.75$ because most patients are low-effort, and $\Delta=\alpha=\frac{1}{2}$ to balance proximity and comorbidities. Finally, we set $\bar{d}=20.2$ since prior work demonstrates the average distance threshold for patients is 20.2 miles~\citep{washington_distance}. 

\subsubsection{Real-World Simulation Results}
\label{sec:semi_synthetic_results}
\begin{figure*}
    \centering 
    \subfloat[\centering Comparison]{\includegraphics[width=0.45\textwidth]{figures/semi_synthetic_comparison.pdf}}
    \subfloat[\centering Comorbidities]{\includegraphics[width=0.45\textwidth]{figures/comorbidity_comparison.pdf}} \\ 
    \subfloat[\centering Match Rate vs. Population]{\includegraphics[width=0.45\textwidth]{figures/scatter_ct.pdf}}
    \subfloat[\centering Match Rate by Zipcode]{\includegraphics[width=0.45\textwidth]{figures/ct_map.pdf}}
    \caption{(a) We construct a real-world scenario with $N=1225$ and $M=700$. We find that the gradient descent policy performs best because of the large patient/provider ratio. 
    (b) The gradient descent policy achieves a higher match quality for patients with comorbidities because these patients match with specialist providers. (c) Geographically, the gradient descent policy tends to match patients from more populated zip codes, (d) with the highest match rates occurring in cities such as Hartford and New Haven.} 
    \label{fig:semi_synthetic}
\end{figure*}

In Figure~\ref{fig:semi_synthetic}a, we show that the gradient descent policy achieves the highest match quality and outperforms alternatives by 13\%. 
This mirrors Figure~\ref{fig:phase}, where gradient descent is the best policy for large patient/provider ratios.  
Next, we analyze \textit{which} patients achieve high-quality matches under the gradient descent policy. 
We find that patients with comorbidities achieve higher match quality ($p<10^{-17}$) because they tend to match with specialist providers (Figure~\ref{fig:semi_synthetic}b).
Additionally, patients from more populated zip codes achieve higher match rates (Figure~\ref{fig:semi_synthetic}c), potentially due to the lack of providers in rural areas~\citep{rural_health_care}. 
Pictorially, this corresponds to high match rates in urban areas such as Hartford and New Haven (Figure~\ref{fig:semi_synthetic}d). 

\subsection{Real-World Considerations}
\label{sec:real_world_considerations}
We extend our scenario to explore the real-world considerations introduced in Section~\ref{sec:real_world}: patient batching, maximum assortment sizes, and fairness considerations. 

\subsubsection{Batched Patients}
\label{sec:batched_patients}
\begin{figure*}
    \centering 
    \subfloat[\centering Batch Size]{\includegraphics[height=0.18\textwidth]{figures/batch_size.pdf}} 
    \subfloat[\centering Max Assortment Size]{\includegraphics[height=0.18\textwidth]{figures/max_menu_size.pdf}}
    \subfloat[\centering Fairness]{\includegraphics[height=0.18\textwidth]{figures/fairness_comparison.pdf}}
    
    \caption{(a) Offering patients in two batches improves match quality by 7\% while increasing response time by 90\%. Beyond two batches, the increases in match quality are diminishing. (b) We simulate cognitive overload by assuming patients consider up to $d$ options. Smaller $d$ favors the gradient descent policy because it offers carefully tailored assortments without being too large. (c) While the gradient descent policy maximizes match quality, it achieves worse fairness because it neglects the worst-off patient. In contrast, pairwise and group-based policies achieve higher fairness by only suggesting high-quality matches.} 
    \label{fig:batch_menu_size}
\end{figure*}
Offering assortments to batches of patients rather than one-shot allows administrators to exercise some control over patient response orders. 
We apply the batching algorithm from Section~\ref{sec:batching} and analyze the impact of the number of batches, $L$, on the match quality in our real-world scenario. 
In Figure~\ref{fig:batch_menu_size}a, we demonstrate that batch offerings allow administrators to trade off logistical burdens for match quality. 
While offering assortments in two batches can improve match quality by 8\% compared to the single-batch setting, the total response time also increases by 74\%. 
Moreover, larger batch sizes lead to diminishing increases in match quality with a near linear increase in the time taken. 

\subsubsection{Varied Assortment Sizes}
\label{sec:menu_size}
Patients might only consider a subset of providers offered due to choice overload~\citep{choice_overload}. 
To simulate this, we incorporate the maximum assortment size, $d$, to simulate choice overload (see Section~\ref{sec:choice_overload} for details). 
In Figure~\ref{fig:batch_menu_size}b, we vary $d \in \{5,10,15,20,25\}$ and find that the gradient descent policy is the best policy across values of $d$, especially for small $d$. 
This occurs because the gradient descent policy offers smaller assortments compared to the greedy policy, and so restricting assortment size has less of an impact. 

\subsubsection{Fairness Considerations}
To understand how policies trade-off between fairness and match quality, we plot the minimum match quality against the average match quality in Figure~\ref{fig:batch_menu_size}c (see Section~\ref{sec:fair} for a more thorough discussion of different notions of fairness). 
In our fairness computation we exclude patients who fail to match, because including them would make the fairness metric always 0 and would not reflect the difference in fairness across policies.  
We find that the fairest policy, pairwise, also achieves the lowest average match quality. 
Fairer policies maximize tend to match fewer patients, which keeps the minimum match quality high but lowers the average match quality. 
If healthcare administrators are focused on fairness as opposed to match quality, then they can select a fairer policy such as the pairwise policy.
We compare policies across one definition of fairness for simplicity, but we note that in reality, there are multiple dimensions of fairness that need to be considered (see Section~\ref{sec:fair}). 