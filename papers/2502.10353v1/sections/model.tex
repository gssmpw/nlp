\section{Problem Formulation}
\label{sec:problem}

\subsection{Formal Model of Patient-Provider Matching}
\label{sec:model}
We focus on matching $N$ patients to $M$ providers. 
We present each patient $i$ with an assortment $\mathbf{X}_{i} \in \{0,1\}^{M}$ upfront.
Here, $X_{i,j}$ denotes whether patient $i$ has provider $j$ in their assortment. 
Patients then select providers from their assortment sequentially by responding in an order $\sigma$, and can abstain from provider selection. 
Here, $\sigma_{1} \in [N]$ represents the first patient in the order. 
We assume patient response times are i.i.d. and exponential; prior work demonstrates this can model real-world wait times~\citep{queueing_theory}. 
Under this assumption, $\sigma$ is a random permutation of $[N]$. 

Each patient selects a provider from their assortment according to a choice model $f_{i}: \{0,1\}^{M} \rightarrow \{0,1\}^{M}$. 
The choice model takes as input a 0-1 vector $\mathbf{z}$ and outputs a random 0-1 vector representing the provider selected by patient $i$. 
Formally, let $\mathbf{y}^{(t)}$ represent the set of unselected providers at time $t$. 
Then, patient $\sigma_{t}$ selects providers according to $f_{\sigma_{t}}(\mathbf{X}_{\sigma_{t}} \odot \mathbf{y}^{(t)})$, where $\odot$ is the element-wise product ($\mathbf{a} \odot \mathbf{b} = \mathbf{c} \rightarrow a_{k} b_{k} = c_{k} \forall k$). 
$\mathbf{X}_{\sigma_{t}} \odot \mathbf{y}^{(t)}$ represents the set of providers that are on patient $\sigma_{t}$'s assortment and unselected by previous patients. 
If $f_{\sigma_{t}}(\mathbf{X}_{\sigma_{t}} \odot \mathbf{y}^{(t)}) = \mathbf{0}$, then no provider is selected, while if $f_{\sigma_{t}}(\mathbf{X}_{\sigma_{t}} \odot \mathbf{y}^{(t)})_{j} = 1$, then provider $j$ is selected. 
We note that a patient can match with at most one provider: $\lVert f_{\sigma_{t}}(\mathbf{X}_{\sigma_{t}} \odot \mathbf{y}^{(t)})\rVert_{1} \leq 1$

Patients are more likely to select providers with a higher match quality, $\theta_{i,j} \in [0,1]$, where $\theta_{i,j}$ is analogous to a reward for patient $i$ when selecting provider $j$. 
Match quality can encompass various factors, including demographic concordance~\citep{race_concordance,gender_concordance,language_concordance}, physical proximity, and patient needs.
We assume administrators know $\theta_{i,j}$. Prior work demonstrates providers are able to successfully identify new high-match quality providers for patients~\citep{physician_referral}, which implies that we can predict match quality from patient data.

We present examples of choice models below: 
\begin{enumerate}
    \item \textbf{Uniform} - Patients select their most preferred provider with probability $p$ and otherwise abstain. 
    Let $\mathbf{e}_{k}$ denote the $k$-th standard basis vector, then: 
    
    \begin{equation}
    f_{i}(\mathbf{z}) =\left\{ \begin{array}{ll}
                \mathbf{e}_{\argmax_{j} (\theta_{i} \odot \mathbf{z})_{j}} & \text{with prb. p}\\
                \mathbf{0} &\text{otherwise}
            \end{array} \right.
    \end{equation}

    \item \textbf{Threshold} - Patients follow the uniform choice model but only select providers with match quality above some threshold $\alpha$. If match quality corresponds to geographic proximity, $\alpha$ corresponds to maximum travel distance. 
Formally: 
    \begin{equation}
    f_{i}(\mathbf{z}) =\left\{ \begin{array}{ll}
                \mathbf{e}_{\argmax_{j} (\theta_{i} \odot \mathbf{z})_{j}} & \text{with prb. p if } \max \theta_{i} \odot \mathbf{z} \geq \alpha \\
                \mathbf{0} &\text{otherwise}
            \end{array} \right.
    \end{equation}
    \item \textbf{Multinomial Logit (MNL)} - Patients select providers according to $\theta$, and have an exit option $\gamma$ which represents selecting no provider. Under the MNL choice model~\citep{choice_models_textbook}, this is:
    \begin{equation}
            f_{i}(\mathbf{z}) =\left\{ \begin{array}{ll}
                \mathbf{e}_{j} & \text{with prb. } \frac{z_{j}\exp(\theta_{i,j})}{\exp(\gamma) + \sum_{j'} z_{j'} \exp(\theta_{i,j'})}\\\\
                \mathbf{0} &\text{with prb. } \frac{\exp(\gamma)} {\exp(\gamma) + \sum_{j'} z_{j'} \exp(\theta_{i,j'})}
            \end{array} \right.
    \end{equation}
\end{enumerate}
% Patients select providers according to a true match quality $\theta^{*}$, but we only have knowledge of $\theta_{i,j} = \theta^{*} + \epsilon$, where $\epsilon$ is Gumbel distributed noise~\citep{choice_models_textbook}. 
    % Patients either select a provider $j$ or can abstain from selecting any provider, which we represent through an exit option with $\gamma$: 

We aim to find a policy $\pi(\theta)$ that constructs an assortment $\mathbf{X}$, given the match quality matrix $\theta$. 
We evaluate policies through two metrics: match quality and match rate.
\begin{enumerate}
    \item \textbf{Match Quality} measures $\theta$ across selected patient-provider pairs. Formally, it aggregates $\theta_{\sigma_{t}}$ across provider selections $f_{\sigma_{t}}(\pi(\theta) \odot \mathbf{y}^{(t)})$ for all $N$ patients (= $N$ time periods):
    \begin{equation}
       \mathbb{E}_{\sigma}[\frac{1}{N} \sum_{t=1}^{N} f_{\sigma_{t}}(\pi(\theta) \odot \mathbf{y}^{(t)}) \cdot \theta_{\sigma_{t}}]
    \end{equation}   
    \item \textbf{Match Rate} is the proportion of patients who are matched. Formally, it computes if patient $\sigma_{t}$ matches with a provider: $\lVert f_{\sigma_{t}}(\mathbf{X}_{\sigma_{t}} \odot \mathbf{y}^{(t)}) \rVert_{1} \geq 0$: 
    \begin{equation}
        \mathbb{E}_{\sigma}[\frac{1}{N} \sum_{t=1}^{N} \lVert f_{\sigma_{t}}(\pi(\theta) \odot \mathbf{y}^{(t)}) \rVert_{1}]
    \end{equation}
\end{enumerate}

\begin{example}
    We quantify the example in Figure~\ref{fig:pull}. 
    Suppose $N=M=3$, and let $\theta$ and $\mathbf{X}$ be 
    \begin{equation} 
        \theta =
        \begin{bmatrix}
        0.5 & 0.6 & 0.9 \\
        0.7 & 0.5 & 0.4 \\
        0.6 & 0.2 & 0.8
        \end{bmatrix}, \hspace{0.5cm}
        \mathbf{X} =
        \begin{bmatrix}
        0 & 0 & 1 \\
        1 & 1 & 0 \\
        1 & 0 & 1
        \end{bmatrix}
    \end{equation}
    Suppose $\sigma = [2,1,3]$. 
    Then one trial could be the following: patient $\sigma_{1}=2$ abstains from selecting a provider, patient $1$ selects provider $3$ and patient $3$ selects provider $1$: $f_{2}(\mathbf{y}^{(1)} \odot \mathbf{X}_{2}) = \mathbf{0}$, $f_{1}(\mathbf{y}^{(2)} \odot \mathbf{X}_{1}) = \mathbf{e}_{3}$, and $f_{3}(\mathbf{y}^{(3)} \odot \mathbf{X}_{3}) = \mathbf{e}_{1}$. 
    Here, $\mathbf{y}^{(1)} = [1,1,1]$, $\mathbf{y}^{(2)} = [1,1,1]$, and $\mathbf{y}^{(3)} = [1,1,0]$. The resulting match quality is $\frac{1}{3}(\theta_{1,3} + \theta_{3,1}) = 0.5$ and match rate is $\frac{2}{3}$
\end{example}


\subsection{Model Fidelity}
We discuss our model fidelity for real-world patient-provider matching:
\begin{enumerate}
    \item \textbf{Patient autonomy} is a critical part of our model, as assortments allow patients to select between providers while allowing administrators some control. Prior work demonstrated that people are willing to trade autonomy for efficiency, including in online markets~\citep{autonomy_efficiency} and domestic life~\citep{india_efficiency_autonomy}. 
    We find similar ideas during our conversations with our healthcare partners, where patients might prefer a few high-quality choices to an overwhelmingly large number of options~\citep{choice_overload}. 
    \item \textbf{Match quality and match rate} underlie assortment computation. We focus on match quality because it reflects the benefit patients get from matching, while match rate reflects the fraction of patients with access to providers. For example, if patients primarily care about proximity, match quality should be based on the distance between patients and providers. We discuss different notions of match quality in Section~\ref{sec:semi_synthetic_data}. 
    \item \textbf{Choice models} reflect how patients select providers. Choice models are critical to our setting, so we analyze various choice models in Section~\ref{sec:policies} and experiment with different selections in Section~\ref{sec:synthetic}. 
    We focus on choice models that are clinically motivated or ubiquitous in prior work. 
    For example, the uniform choice model reflects low-effort patient decisions~\citep{choosing_doctor}, while the MNL choice model is commonly seen in prior work~\citep{assortment_mnl}. 
\end{enumerate}

\subsection{Analysis in the Single Provider Scenario}
\label{sec:greedy_provider}
To gain insight into the optimal match quality policy structure, we analyze the $M=1$ scenario under the uniform choice model. 
We first explicitly detail the optimal policy: 
\begin{restatable}{theorem}{thmmone}
\label{thm:one_m}
    Let $f_{i}$ be the uniform choice model with probability $p$.
    Suppose $M=1$, and let $u_{1},u_{2},\ldots,u_{N}$ be a set of coefficients such that $\theta_{u_{1}} \geq \theta_{u_{2}} \cdots \theta_{u_{N}}$.
    Let $s$ be defined as follows:
    \begin{equation}
       s = \argmax_{s} (1-(1-p)^{s}) \frac{\sum_{i=1}^{N} \theta_{u_{i},1}}{s}
    \end{equation}
    Then the policy which maximizes match quality is $\mathbf{X}_{u_{1},1} = \mathbf{X}_{u_{2},1} \cdots \mathbf{X}_{u_{s},1} = 1$, where $\mathbf{X}_{i,1}$ is 0 otherwise. 
\end{restatable}
When $M=1$, we can decompose the match quality into the probability any patient matches and the average match quality given matching. 
If we offer the single provider to $s$ patients, then the match probability is $ (1-(1-p)^{s})$.
For fixed $s$, we offer provider $j=1$ to the top-$s$ patients by match quality. 
Note that the optimal match rate is achieved with $s=N$, while the optimal match quality might arise from $s < N$ (as we'll show in Section~\ref{sec:example_policies}). 
We include all full proofs in Appendix~\ref{sec:proofs}. 
