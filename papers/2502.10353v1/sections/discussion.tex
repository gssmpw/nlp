\section{Conclusion and Discussion}
\label{sec:discussion}
The patient-provider relationship is key to quality healthcare, yet provider turnover frequently leads to patients being without providers. 
Our work studies how to re-match patients in such a scenario through assortment optimization. 
Our study proposes algorithmic approaches for patient-provider matching systems and offers analytical and empirical insights to inform system design.
We further offer contributions to one-shot matching under random response order, an understudied phenomenon within the assortment literature.
We conclude by providing recommendations for deploying assortment-based systems into practice and discussing extensions of our work. 

\paragraph{Recommendations for Deployment}
We translate our takeaways into real-world implications: 
\begin{enumerate}
    \item  Larger assortments become increasingly useful as patients become more selective because patients are more likely to reject matches, making it important to give more options. 
    Moreover, assortment sizes should be tailored to patient behavior; the choice model and match probability impact which policy performs best. 
    \item The level of heterogeneity impacts the amount of work needed to design a good algorithm. 
    For example, patients with diverse preferences require less work to find the ideal match, and even relatively simple baselines would work well. 
    However, homogeneous scenarios (e.g., a small city with few providers) require care when matching patients and providers. For example, when patients all prefer the same provider, simple solutions such as the pairwise policy can perform poorly. 
    \item Healthcare administrators should be aware of the underlying factors that dictate \textit{which} patients get matched. For example, providers are scarce in rural areas, so assortments might naturally lead to low match rates in rural areas. Algorithm designers should work with administrators during development to minimize these biases.   
\end{enumerate}

\paragraph{Model Extensions}
We detail several extensions of this model that better capture real-world patient-provider matching. 
As noted in Section~\ref{sec:model}, we assume that the response time for each patient is i.i.d. 
A natural extension would be to develop policies that work well independent of the response order distribution. 
Our experiments also assume that the choice model, $f_{i}$, is constant across patients. 
However, in reality, patients might have heterogeneous choice models. 
For example, in the uniform choice model, the value of $p$ could vary across patients. 
It would be interesting to see how extensions of our policies would perform under heterogeneous choice models.  

\paragraph{Extensions to Other Domains}
Future work can extend our framework to other scenarios with random response orders. 
Examples include matching between students and courses, where universities could offer each student an assortment of courses, and between schools and children, where school choice might lead counties to offer each child multiple options~\citep{school_choice}. 