In the US healthcare system, patients match with providers or specialists, who provide them with information and care throughout their health journey. 
The quality of this match is integral to maintaining patient health; prior work has demonstrated that improved patient-provider racial concordance is correlated with improved risk perception~\cite{race_concordance_lung_cancer}, while gendered concordance can improve patient outcomes~\cite{gender_concordance}. 
Additionally, providers who speak the same language as patients can potentially lead to decreased medicine omission rates~\cite{language_concordance}. 

Despite the importance of patient-provider relationships, provider turnover rates have increased since 2010~\cite{physician_turnover}. 
Loss of a primary care provider leads to worse patient experiences~\cite{pcp_turnover}, more spending, and more emergency room visits~\cite{pcp_turnover_health_outcome}. 

Prior work has tackled the physician-provider matching problem from a variety of perspectives. 
These works have primarily been concerned with solving a matching problem between providers and patients through techniques such as genetic programming~\cite{genetic_algorithm}, deferred acceptance~\cite{deferred_acceptance}, and knowledge rules~\cite{knowledge_rules}. 
We build on such works to consider the matching problem under patient heterogeneity and preferences. 
In particular, we aim to find an assignment of patients to providers, where patients have autonomy in rejecting or accepting decisions, which models real-world scenarios.

To tackle this problem, we first model patients through a choice model, then develop assortment planning algorithms for this scenario. 
Assortment planning considers the set of options offered to each patient, and better reflects reality, as patient matching decisions cannot be made unilaterally. 
We offer a set of algorithms for one specific choice model, then generalize this across patient choice models. 
We then demonstrate how such algorithms can be extended to account for real-world constraints, such as uncertainty in preferences, and provider-side constraints. 
Our contributions are as follows: 1) we propose an assortment planning model of the patient-provider matching problem, 2) we demonstrate how we can find efficient assortments, both for the uniform choice model, and for general choice models, 3) we extend such a model to account for constraints on both the provider and patient sides, and 4) we show how assortment planning algorithms can account for uncertaint over patient prferences. 