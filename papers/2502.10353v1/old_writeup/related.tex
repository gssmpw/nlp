\section{Related Works}
\subsection{Patient-Provider Matching}
Various works have investigated and developed algorithms for different aspects of the patient-provider matching system. 
For example, prior work has leveraged deferred acceptance algorithms to overcome long wait times~\cite{deferred_acceptance}.  
Other techniques include using genetic algorithms  to balance between provider balance and patient match quality, and knowledge rules in a two-stage approach~\cite{knowledge_rules}. 
Our work builds on this by considering both the learning and optimization problem, solving each by explicitly modeling the choices made by patients, and giving guarantees on these results. 

\subsection{Match Quality}
We detail prior work which quantifies the set of factors which impact provider-patient match quality. 
In particular, prior work demonstrates the impact of race~\cite{race_concordance_lung_cancer,race_medication,race_press_ganey} and gender~\cite{gender_concordance} concordance, and its impact on downstream patient health~\cite{race_medication}. 
Such studies typically measure metrics such as patient happiness, through the use of surveys~\cite{race_press_ganey}, or downstream behavioral outcomes, such as medication uptake \cite{race_medication} and preventative screening rates~\cite{race_diversity_downstream}. 
Such studies demonstrate how we can learn $\theta_{i,j}$ values from such data sources, which can inform the types of assortments provided. 

\subsection{Assortment Planning} 
Assortment planning aims to study the set of options given to consumers while aiming to maximize utility~\cite{assortment_mnl}. 
Such a setting is frequently seen in economics, where many matching markets need to optimize for the products shown~\cite{assortment_mnl}. 
Within assortment planning, solutions are studied either in an online~\cite{online_assortment} or offline~\cite{assortment_mnl} fashion. 
Applications include matching markets, which are commonly used by rideshare and dating platforms~\cite{assortment_matching,platform_dating}.
For both, linear-programming based solutions are typically used, where online solutions incorporate an additional regularization term so decisions are non-myopic. 
