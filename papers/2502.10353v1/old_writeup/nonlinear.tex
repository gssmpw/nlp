\section{Balancing Patients}
\subsection{Defining Balance}
Patient-provider matching frequently involves objectives beyond match quality due to the complex interactions on both sides of the market. 
We focus on a general class of objectives here: the need to balance patient ``workload'' between different providers. 
Formally, we let each patient correspond to a workload quantity, which we denote $\beta_{i}$. 
This workload can refer to quantities such as their age, level of commodities, and estimated visits per month. 
If we then define each provider to have a workload, $\hat{\beta}_{j}$, we aim to ensure equal distribution of new patients across providers. 
That is, we aim to minimize the variance in workload across providers: 
\begin{equation}
    \min\limits_{X_{i,j}} \frac{1}{P} \sum_{j=1}^{P} |\sum_{i=1}^{N} X_{i,j} \beta_{i} + \hat{\beta}_{j} - \bar{\beta}| 
\end{equation}
Here, $\bar{\beta}$ is the average workload across providers: $\bar{\beta} = \frac{1}{P} \sum_{j=1}^P \sum_{i=1}^{N} X_{i,j} \beta_{i} + \hat{\beta_{j}}$. 
We note that multiple values of $\beta$ are possible, which corresponds to the different notions of workload references earlier. 

We note that such an objective can be written through a linear program as follows: 
\begin{equation}
    \min\limits_{v_{j}} \frac{1}{P} \sum_{j=1}^{P} v_{j}
\end{equation}
So that $-v_{j} \leq \sum_{i=1}^{N} X_{i,j} \beta_{i} + \hat{\beta}_{j} - \bar{\beta}\leq v_{j}$ and $\bar{\beta}$ is a linear function in $X_{i,j}$ as defined earlier. 
Such a formulation allows for easy incorporation into our model.

\subsection{Optimizing for Balance}
By incorporating the need for patient balance across providers, our overall objective function becomes
\begin{equation}
    \max  \sum_{i=1}^{N} \sum_{j=1}^{P} \theta_{i,j} X_{i,j} - \frac{\lambda}{P} \sum_{j=1}^{P} v_{j}
\end{equation}
Here, $\lambda$ is a hyperparameter which balances the need for balancing patients with improving match quality. 
We note that optimizing such a problem with approximation guarantees becomes difficult when $\lambda >> N$, because matching any patients worsens the balance. 
We prove this formally as follows: 
\begin{lemma}
Consider the following Linear Program: 
\begin{equation}
    \max \sum_{i=1}^{N} \sum_{j=1}^{P} \theta_{i,j} X_{i,j} - \frac{\lambda}{P} \sum_{j=1}^{P} v_{j}
\end{equation}
When $\lambda >> N$, any solution with $X_{i,j} = 1$ for some $i,j$ results in an unbounded approximation ratio compared to $\mathrm{OPT}$. 
\end{lemma}

Motivated by this, we instead focus on scenarios where $\lambda$ is small. 
We display results in one-shot assortment decision scenarios with known and random ordering. 
We first show that, with known ordering, our previous algorithms can be used, with little impact on the approximation guarantees, assuming that $\lambda$ is small. 
\begin{lemma}
Suppose we select $X_{i,j}$ to optimize the following Linear Program: 
\begin{equation}
    \max \sum_{i=1}^{N} \sum_{j=1}^{P} \theta_{i,j} X_{i,j} - \frac{\lambda}{P} \sum_{j=1}^{P} v_{j}
\end{equation}
Where $-v_{j} \leq \sum_{i=1}^{N} X_{i,j} \beta_{i} + \hat{\beta}_{j} - \bar{\beta}\leq v_{j}$ for some known $\beta_{i}, \hat{\beta}_{j}$. 
Suppose the Linear Programming solution to this allocates provider $\pi_{i}$ to patient $i$. 
If $\lambda \leq \frac{\theta_{i,j}-\theta_{i,\pi_{i}}}{\beta_{i}}$ for all $i,j$ such that $\theta_{i,j}-\theta_{i,\pi_{i}} \geq 0$, then the Stacked Linear Programming solution is optimal. 
\end{lemma}
We informally note that, in scenarios where such a condition does not hold, a strategy is to only allocate provider $j$, after all patients with $\pi_{i}=j$ have been matched or declined. 
Such a scenario differs from the traditional Stacked Linear Programming scenario, because providers can be matched with multiple patients. 

We next detail two approxation bounds for the known-ordering and random-ordering scenarios: 
\begin{lemma}
Suppose we select $X_{i,j}$ to optimize the following Linear Program: 
\begin{equation}
    \max \sum_{i=1}^{N} \sum_{j=1}^{P} \theta_{i,j} X_{i,j} - \frac{\lambda}{P} \sum_{j=1}^{P} v_{j}
\end{equation}
Where $-v_{j} \leq \sum_{i=1}^{N} X_{i,j} \beta_{i} + \hat{\beta}_{j} - \bar{\beta}\leq v_{j}$ for some known $\beta_{i}, \hat{\beta}_{j}$. 
Suppose that $\beta_{i} \leq \frac{1}{P}$, and let the order of patient responses be known.  
Let the LP solution to this be denoted $\mathrm{ALG}$, and note that prior results showed $\mathrm{ALG} = \mathrm{OPT}$. 
Then using Stacked Linear Programming results in an approximation ratio of $\frac{\mathrm{ALG}}{\mathrm{ALG} + \lambda}$
\end{lemma}

\begin{lemma}
Suppose we select $X_{i,j}$ to optimize the following Linear Program: 
\begin{equation}
    \max \sum_{i=1}^{N} \sum_{j=1}^{P} \theta_{i,j} X_{i,j} - \frac{\lambda}{P} \sum_{j=1}^{P} v_{j}
\end{equation}
Where $-v_{j} \leq \sum_{i=1}^{N} X_{i,j} \beta_{i} + \hat{\beta}_{j} - \bar{\beta}\leq v_{j}$ for some known $\beta_{i}, \hat{\beta}_{j}$. 
Suppose that $\beta_{i} \leq \frac{1}{P}$, and let the order of patient responses be random uniform.  
Let the LP solution to this be denoted $\mathrm{OPT}$. 
Then allocating only the providers from the LP solution to each patient results in an approximation ratio of $\frac{p \mathrm{OPT}}{\mathrm{OPT} + \lambda}$
\end{lemma}
These demonstrate the ability to re-use our previous solutions, while adapting to the varying constraints placed by provider balancing. 
