\section{Matching Algorithms}
\subsection{Analysis of small provider scenario}
We provide an analysis of the problem scenario when the number of providers, $P=1$, while we vary the number of patients $N$. 
This allows us to understand intuitions for the problem and allows us to derive the optimal policy. 

\begin{lemma}
    When $P=1$, for any value of $N$, if the ordering of patients, $\pi$, is known, then the optimal policy can be solved through backwards induction. 
\end{lemma}

\begin{lemma}
    When $P=1$, any deterministic algorithm can achieve at most $\frac{p}{p+({1-p})^2}$-fraction of the optimal policy for any $p$. 
\end{lemma}
\subsection{Known Ordering}
We next consider the scenario where patients response orderings are known, but all assortment decisions need to be made in a one-shot manner.
Making menu decisions one-shot is difficult because declinations by patients during their selection cannot be incorporated into future assortments. 
For example, if patient $i$ declines to be matched, then patient $i'$ might have the opportunity to be matched with patient $i$'s top choice. 
We take advantage of the known ordering in this scenario to propose the following algorithm: 
% \begin{definition}
%     \textbf{Stacked Linear Programming} - Consider a known ordering $\pi$ of agents, and without loss of generality, let $\pi = \{1,2,\ldots,N\}$. 
%     Then, for patient $i$, let $X_{i,j} = 1$ for any $j = v_{i'}$ and $i' \leq i$. 
% \end{definition}
% Under the stacked Linear Programming algorithm, patients are given successively larger menus, which accounts for prior patients declining their option. 
% We demonstrate that such a strategy is optimal:
% \begin{lemma}
%     Let $\pi$ be known a-priori, so that we aim to maximize: 
%     \begin{equation}
%         \max_{M_{1} \cdots M_{N}} \sum_{i=1}^{N} \sum_{j=1}^{P} \theta_{\pi_{i},j} Y_{\pi_{i},j}(\mathbf{Z}_{i})
%     \end{equation}
%     Under a delayed response scenario, where $M_{i}$ are all chosen initially, a stacked Linear Programming solution maximizes our objective function. 
% \end{lemma}

Moreover, we demonstrate that we can compute the optimal ordering of patients, $\pi^{*}$ using information on the pairwise preference of patients. 
Using Linear Programming solutions, each patient $i$ is assigned to a provider $v_{i}$. 
We construct an ordering, $\pi^{*}$, so that patients, $i$, who prefer providers $v_{i'}$ to their own, go after $i'$ in $\pi^{*}$. 
We do this, so if $i'$ declines their provider, patient $i$ has the opportunity to be matched with $v_{i'}$. 
We demonstrate this formally as follows: 
\begin{lemma}
    Consider the following optimization problem: 
    \begin{equation}
        \max_{\pi,M_{1} \cdots M_{N}} \sum_{i=1}^{N} \sum_{j=1}^{P} \theta_{\pi_{i},j} Y_{\pi_{i},j}(\mathbf{Z}_{i})
    \end{equation}
    Consider the bipartite matching problem: 
    \begin{equation}
        \sum_{i=1}^{N} \sum_{j=1}^{P} \theta_{i,j} X_{i,j}
    \end{equation}
    With 1-1 matching constraints on nodes. 
    Let the optimal allocation to this problem, for patient $i$, be $v_{i} \in \{0,1,2\ldots,P\}$.  
    Let $G = (V,E)$ be a directed graph, where an edge from $i$ to $i'$ indicates that $\theta_{i,v_{i'}} \geq \theta_{i,v_{i}}$. 
    Let $\pi^{*}$ be any reversed topological ordering of nodes in $G$, so that edges only flow backwards. 
    Then $\pi^{*}$ is optimal for our objective function: 
    \begin{equation}
        \max_{\pi,M_{1} \cdots M_{N}} \sum_{i=1}^{N} \sum_{j=1}^{P} \theta_{\pi_{i},j} Y_{\pi_{i},j}(\mathbf{Z}_{i}) = \max_{M_{1} \cdots M_{N}} \sum_{i=1}^{N} \sum_{j=1}^{P} \theta_{\pi^{*}_{i},j} Y_{\pi^{*}_{i},j}(\mathbf{Z}_{i})
    \end{equation}

\end{lemma}

\begin{question}
    What is the optimal policy for the known ordering situation? 
\end{question}

\subsection{Known Ordering with Budgets}
Assortments or menus in real-world have finite sizes; the last patient, $N$, cannot have an $N$-sized menu. 
To tackle this, we consider the presence of a budget $B$, so that $\sum_{j} X_{i,j} \leq B$ for all $i$. 
Such a scenario does not impact the menus for the first $B$ patients, as their menus are naturally smaller than $B$. 

For all other patients, we consider the following strategy to select menus under this constrained setting. 
We know that the menu should be a subset of the providers $\pi(1) \cdots \pi(b)$ where $b \geq B$. 
We aim to solve the following optimization problem to determine which menu choices should be shown: 
\begin{equation}
    \max\limits_{Y_{i}} \sum_{i=1}^{b} \prod_{j=1}^{i-1} (1-\alpha_{b,\pi(j)}Y_{j}) \alpha_{b,\pi(i)} Y_{i} \theta_{b,\pi(i)}
\end{equation}
Here, $\alpha_{b,\pi(j)}$ represents the probability that provider $\pi(j)$ is available for the $b^{\mathrm{th}}$ patient. 
We view $\mathbf{\alpha}_{b}$ as a vector, and compute this recursively through dynamic programming, noting that $\mathbf{\alpha}_{0} = \mathbf{1}$. 

Because such an optimization problem is intractable, due to the product of $b$ continuous variables, we solve this problem through the use of McCormick envelopes. 
Such a technique allows for us to develop new auxiliary variables which represent these products of continuous variables. 
From there, we reduce the problem down to a product of continuous and binary variable,s which can be solved efficiently. 

\begin{question}
    What is the optimal policy for the known ordering situation? 
\end{question}

\subsection{Random Ordering}
When $\pi$ is unknown, deciding optimal menus is more difficult. 
As a baseline, we first show that offering only $v_{i}$ to patient $i$ achieves a $p$-approximation. 
\begin{lemma}
    Let $\pi$ be unknown, so that we aim to maximize: 
    \begin{equation}
        \max_{M_{1} \cdots M_{N}} \mathbb{E}_{\pi}[\sum_{i=1}^{N} \sum_{j=1}^{P} \theta_{\pi_{i},j} Y_{\pi_{i},j}(\mathbf{Z}_{i})]
    \end{equation}
    Under a delayed response scenario, where $M_{i}$ are all chosen initially, letting $X_{i,j} = \mathbbm{1}[j = v_{i}]$ results in a $p$-approximation to the problem. 
    That is, when $X_{i,j} = \mathbbm{1}[j = v_{i}]$, $\mathrm{ALG} = \mathbb{E}_{\pi}[\sum_{i=1}^{N} \sum_{j=1}^{P} \theta_{\pi_{i},j} Y_{\pi_{i},j}(\mathbf{Z}_{i})]$, and $\mathrm{OPT} = \max_{M_{1} \cdots M_{N}} \mathbb{E}_{\pi}[\sum_{i=1}^{N} \sum_{j=1}^{P} \theta_{\pi_{i},j} Y_{\pi_{i},j}(\mathbf{Z}_{i})]$, then $\mathrm{ALG} \geq p \mathrm{OPT}$. 
    Moreover, such analysis is tight, as there exists $Y_{i,j}$ such that $\mathrm{ALG} = p \mathrm{OPT}$
\end{lemma}

\begin{lemma}
    Greedy algorithms on the random ordering problem can perform arbitrarily bad, for any fixed $p$. 
\end{lemma}


To go beyond this, we consider adding providers preferred by patient $i$. 
Adding all preferred providers is suboptimal; under some scenarios, the cost for swapping providers is high. 

To account for this, we add pairs of patient-providers, so that the pairs are disjoint, while maximizing the expected improvement in match quality. 
We do so through manual computation of the expected match quality then determining which pairs of patient-providers can be selected to improve match quality. 
Graphically, we select a subset of edges in the graph $G=(V,E)$, where nodes represent patients, and an edge from patient $i$ to patient $i'$ indicates that $\theta_{i,\pi(i')} \geq \theta_{i,\pi(i)}$, where $\pi$ is the assignment made through the linear program. 
From such a graph, we select pairs of patients, say $i$ and $i'$, then add $\pi(i)$ and $\pi(i')$ to each assortment. 

We first note that such a strategy still results in a valid set of assortments
\begin{lemma}
    Consider an initial set of assortments, denoted by $X_{i,j}$, so that $X_{i,\pi(i)} = 1$ for some function $\pi: \{1,2,\cdots,N\} \rightarrow \{1,2,\cdots,P\}$. 
    Next, consider an algorithm that augments such an assortment through the following: at each step, a clique is chosen, so that for any $i,i'$ in the clique, $X_{i,\pi(i')} = X_{i',\pi(i)} = 1$. Then, for any patient $i$, for any random ordering, the resulting menu, $\mathbf{Z}_{i}$ is non-empty. 
\end{lemma}

Based on this, we propose methods to augment the assortments offered so they build upon the assortment where $X_{i,j} = \mathbbm{1}[j=v_{i}]$. 
Such an assortment is useful in the scenario when $p<<1$; this occurs because many more providers are available, increasing the utility of offering additional providers. 
Formally, we demonstrate the following: 
\begin{corollary}
    Suppose whether each patient successfully matched with some provider is chosen a prior. Then the simple choice algorithm is optimal for the match rate. 
\end{corollary}
However, when $p<<1$, we augment assortments as follows. 
We add a pair, $i,i'$ under the situation where $(1-\frac{p}{2})(\theta_{i,\pi(i')} + \theta_{\theta_{i',\pi(i')}}) > \theta_{i,\pi(i)} + \theta_{i',\pi(i')}$. 
\nrcomment{Switch $\pi$ to $v$}
\nrcomment{Switch $\pi$ to $v$}
Formally, we propose the following algorithm
\begin{definition}
    \textbf{Augmented Simple Choice} - Consider a scenario with unknown, random ordering of patients, and consider the simple assortment, where $X_{i,j} = \mathbbm{v_{i}=j}$. Next, consider an augmentation of such an assortment by selecting a set $S = \{(i,i')\}$, which contains pairs of providers no more than once, and maximizes the following: 
    \begin{equation}
        \sum_{(i,i') \in S} (1-\frac{p}{2})(\theta_{i,\pi(i')} + \theta_{\theta_{i',\pi(i')}}) - \theta_{i,\pi(i)} - \theta_{i',\pi(i')}
    \end{equation}
    Then we let $X_{i,\pi(i')} = 1$ and $X_{i',\pi(i)} = 1$ for all $(i,i') \in S$. 
\end{definition}

Under such an algorithm, we demonstrate that this can only improve upon the baseline assortment
\begin{theorem}
    Let $ALG$ be the value of the following objective when using the augmented simple choice algorithm: 
    \begin{equation}
     \sum_{i=1}^{N} \sum_{j=1}^{P} \theta_{\pi_{i},j} Y_{\pi_{i},j}(\mathbf{Z}_{i})
    \end{equation}
    Let $\mathrm{OPT} = \max_{M_{1} \cdots M_{N}} \mathbb{E}_{\pi}[\sum_{i=1}^{N} \sum_{j=1}^{P} \theta_{\pi_{i},j} Y_{\pi_{i},j}(\mathbf{Z}_{i})]$, then $\mathrm{ALG} \geq p \mathrm{OPT}$.
\end{theorem}
We suspect that adding pairs of patients, $(i,i') \notin S$, but in the graph $G = (V,E)$ of patient preferences can be beneficial for matching. 
While such a claim is not formally proven here, we propose the following conjecture: 
% \begin{conjecture}
%     Consider the bipartite matching problem: 
%     \begin{equation}
%         \sum_{i=1}^{N} \sum_{j=1}^{P} \theta_{i,j} X_{i,j}
%     \end{equation}
%     With 1-1 matching constraints on nodes. 
%     Let the optimal allocation to this problem, for patient $i$, be $v_{i} \in \{0,1,2\ldots,P\}$.  
%     Let $G = (V,E)$ be a directed graph, where an edge from $i$ to $i'$ indicates that $\theta_{i,v_{i'}} \geq \theta_{i,v_{i}}$. 
%     Then letting $X_{i,j} = \mathbbm{1}[j = v_{i} \lor ((i,i') \in E \land f(i,i'))]$, where $v_{i'} = j$, is optimal. 
% \end{conjecture}

\subsection{Accounting for skewed markets}
Finally, we detail how adding unmatched providers an be beneficial. 
We first note that adding such a provider adds no cost, assuming that we do not care about provider balance. 
This is because such providers are initially unmatched; adding them to a menu does not negatively impact any other patient: 
\begin{lemma}
    Consider an unmatched provider $j$, so that $j \neq \pi(i)$ for any $i$, and $X_{i,j} = 0$ for all $i$.  
    Next, consider some patient, $i$, such that $\theta_{i,\pi(i)} \leq \theta_{i,j}$. 
    Then letting $X_{i,j} = 1$ monotonically improves the total match quality. 
\end{lemma}
Deciding which unassigned providers to allocate to which patients provides a challenge. 
In the scenario where we are agnotstic to provider balance, we can simply adopt the strategy of greedily allocating providers to menus, until we reach the maximum menu budget $B$. 
However, when we care about provider balance, we note that such an allocation can be solved through the following linear program: 
\begin{equation}
    \max_{X_{i,j}} \sum_{i,j} \frac{X_{i,j}}{B} \theta_{i,j}
\end{equation}
We add constraints for provider balance as needed. 
We note that with such a linear program, there are no one-to-one constraints, with the only limit being the budget constraint: 
$\sum_{j} X_{i,j} \leq B$, and the menu choices already made. 
\begin{question}
    What is the optimal algorithm for matching excess providers? 
\end{question}

On the other side, when the number of providers is less than the number of patients, we prove the following: 
\begin{lemma}
    In the Linear Program, iteratively matching at most $\frac{1}{p}$ providers to each patient is optimal; assuming regularity conditions on the values for $\theta$. 
\end{lemma}

\subsection{System Analysis}
\begin{lemma} 
    Under the Augmented simple choice model, we achieve a match rate of $p$. 
\end{lemma}

\begin{question} 
    What is the sum of the utilities, in expectation, when using the simple choice model? 
\end{question}