\section{Simulation Experiments}
\subsection{Simulator Outline}
We design a simulator to understand the performance of our algorithms using synthetic data. 
In particular, we consider how varying the number of patients, $N$, and providers, $P$, impacts the both the match quality and the provider balance. 
We study this across scenarios, including those where the patient ordering is known and those where the ordering is random. 

We construct the scenario by randomly distributing patient preferences, $\theta_{i,j}$, patient workloads, $\beta_{i}$, and incorporating provider capacities, $B_{j}$ and workloads, $\beta_{j}$. 
\nrcomment{How should we randomly distribute these}. 
We aim to answer the following questions: 
\begin{enumerate}
    \item How does system parameters impact the performance of our policies? In particular, how does the choice of $N$ and $P$ impact the performance of our policies? 
    \item What happens if our choice model is misspecified? Do our policies still perform well? 
    \item How does learning work when there's underlying prefereneces? Can we learn from multiple matching instances? 
    \item What values of $\lambda$ can actually lead to provider balance? 
    \item What happens if there is uncertainty or errors in the preference model? How much error can our algorithm tolerate? 
    \item What happens as we increase the number of excess providers; does this improve the quality of matches? 
    \item What is the impact of the budget, $B$, on the aggregate match quality
    \item What happens if the value of $p$ is misspecified? 
    \item How tight are the approximation bounds in practice? 
    \item What happens if preferences arise from a simple linear contextual framework? 
\end{enumerate}