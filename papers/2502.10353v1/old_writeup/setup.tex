\section{Model Setup and Preliminaries}
\subsection{Patient-Provider Matching}
We consider a scenario where patients are matched with providers. 
Such a scenario occurs when providers retire or switch positions, causing a sudden increase in the number of unmatched patients. 
In such a scenario, each patient is given a set of options, which we call ``menus.''
Each menu consists of a set of providers, and patients make decisions based on such a menu. 
Decisions are made by the hospital or health agency, who have authority over the contents of the menus offered. 

Formally, we consider $N$ patients who need to be matched to $P$ providers. 
Here, we consider $P$ to represent the combination of providers and time availabilities, so that no two patients can be matched with the same provider. 

\subsection{Patient Preferences}
We consider patients to have preferences between providers based on the match quality. 
For patient $i$, we denote their score for provider $j$ as $\theta_{i,j}$. 
While observing the raw $\theta$ values is infeasible, in Section~\ref{sec:learning}, we detail a situation where we learn approximate values of $\theta$, which we denote $\hat{\theta}_{i,j}$. 
For each patient, we let $r_{i,1}$ denote the most favored provider for a patient, and in general, let $r_{i,j}$ denote the $j^{\mathrm{th}}$ most favored provider for a patient. 
That is, $r_{i,1} = \argmax_{j} \theta_{i,j}$. 

\subsection{Assortment Planning Problem}
The goal of the assortment planning problem is to offer a series of menus, $M_{1} \cdots M_{N} \in \{0,1\}^{P}$, where each menu $M_{i}$ represents the set of providers offered to patient $i$. 
We represent $M_{i}$ as a set of indicator variables, $X_{i,1}, X_{i,2}, \cdots, X_{i,P}$. 
Let $Y_{i,j}(\mathbf{Z}_{t})$ be the probability that, for patient $i$, provider $j$ is selected, given that $\mathbf{Z}_{t} = \{Z_{t,1} \cdots Z_{t,P}\}$ represents the availability of providers at time $t$. 
We note that $\sum_{j=1}^{i} Y_{i,j}(\mathbf{Z}) \leq 1 \forall \mathbf{Z}$. 
Our problem is then to optimize the following: 
\begin{equation}
    \max_{M_{1} \cdots M_{N}} \mathbb{E}_{\pi}[\sum_{i=1}^{N} \sum_{j=1}^{P} \theta_{\pi_{i},j} Y_{\pi_{i},j}(\mathbf{Z}_{i})]
\end{equation}
Here, patients make decisions according to the choice model $Y_{i,j}(\mathbf{Z}_{t})$, and $\mathbf{Z}_{t}$ is updated after each patient makes a selection based on their selections. 
We assume that all menus are limited by a budget, $B$, so that $\sum_{j=1}^{P} X_{i,j} \leq B$ for all $i$. 