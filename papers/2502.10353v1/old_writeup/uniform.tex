\subsection{Defining the Uniform Choice Model}
We introduce optimal menus first through a simple scenario where choice models have uniform probability of selection, with perfect information on preferences. 
We define a choice model through the function $Y_{i,j}(\mathbf{Z}_{t})$. 
First, consider a parameter, $0 \leq p \leq 1$. 
This parameter dictates the probability that a patient matches one of the providers in the system. 
We consider $p$ to be the same across all patients, and $p=0$ indicates that we achieve poor objectives in any scenario. 

Next, consider $Y_{i,j}(\mathbf{Z}_{t}) = Y_{i,j}(\{Z_{t,1}, \cdots, Z_{t,P})$. 
Recall that $r_{i,1} = \argmax_{j} \theta_{i,j}$. 
Similarly, define $s_{i,t} = \argmax_{j} \theta_{i,j} Z_{t,j}$. 
Then $Y_{i,j}(\{Z_{t,1}, \cdots, Z_{t,P}) = p$ if $j = s_{i,t}$, and is $0$ otherwise. 
Such a scenario represents the simplest choice model, where all patients select their highest available provider. 

