Patient-provider turnover is prevalent due to the frequent departure and entry of providers across healthcare systems. 
Such turnover adversely impacts patients due to inadequate access to healthcare, leading to negative clinical outcomes. 
To combat this problem, we detail a system for patient reassignment under provider turnover. 
We design such a system under the lens of assortment planning, where patients receive options for potential providers, and patients decide patients from this list of options. 
Such a system reduces the friction present in the current patient reassignment system. 
We introduce this framework by detailing how we tackle two forms of challenges: optimization-based and learning-based. 
On the optimization front, we detail algorithms for assortment planning under the uniform choice model and demonstrate performance guarantees for both fixed and random patient response orderings. 
We generalize these algorithms to general choice models through the use of sampling-based techniques and demonstrate finite sample bounds. 
Beyond optimizing for match quality, we demonstrate how we can design systems that accommodate both patient-side and provider-side constraints. 
For example, we show that our algorithms can be extended so patients are balanced between different providers, reducing overall provider burden. 
Finally, we show how our model can be adapted to situations where patient preferences are unknown or noisy through the use of machine learning techniques. 
Our study serves as a framework for understanding patient-provider matching under real-world clinical and operational constraints. 