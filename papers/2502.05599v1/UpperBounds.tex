%\vspace{-0.1in}
\section{Algorithm with regret $O(\sqrt{T \log T})$ in the Repeated Identical Auction}
Note that from Lemma \ref{lem:lbuniv}, we know that even the constant $v$ case ($v_t=v >0 \ \forall \ t$) is not trivial for any $\cA$ to solve \eqref{eq:prob} that strictly satisfies the ROSC. 
To belabour the point, consider a simple algorithm $\cA_b$ that satisfies the ROSC on a sample path basis, i.e., $CCV(T)\le 0$, by  bidding $b_t = v_t+\text{Margin}(t-1),$ where $\text{Margin}(t-1)$ has been defined in Definition \ref{defn:ccv}. 


Using Example \ref{exm:1}, we next show that $\cR_{\cA_b}(T)=\Omega(T)$. Let the allocation function be of type \eqref{def:thresholdfunc}, where $\cA_b$ wins a slot if $b_t \ge \theta_t$. 
Since $v_t$ is a constant, any algorithm is just trying to maximize the number of slots it can win while satisfying the ROSC. 
First we describe the $\opt$. For the input defined in Example \ref{exm:1},  slots with $\theta_t=0.4$ are won for `free' since $\opt$ always bids $b_t\ge v=1/2$, and provide with a positive contribution towards ROSC-S of $0.1$ in each such slot. %Hence, the total expected accrued valuation of $\opt$ is $T v \cdot 0.1/3$.
Since $v_t=v \ \forall \ t$, among slots where $\theta_t>v$, $\opt$ only wins those slots where $\theta_t=0.6$ (negative contribution towards ROSC-S of $0.1$) by bidding $b_t=0.6$ to ensure satisfying the ROSC, since $\theta_t$ takes values $0.3$ and $0.6$ with probability $1/3$. Thus, $b_t^\opt=0.6$ for all $t$, and $V_\opt = T\cdot v \cdot 0.1 \cdot (2/3)$.


  \begin{figure*}
        \includegraphics[scale=.4]{SimulationFigurewide.eps}
        \caption{Bid Profiles for algorithm $\cA_b$ and $\cA_c$ for Example \ref{exm:1}.}
        \label{fig:1}
    \end{figure*}
%    \end{centering}
%\includegraphics[scale=.45]{} 

The bid profile of $\cA_b$ is, however, different than $\opt$, as can be seen  from Fig. \ref{fig:1}, where it wins slots with both $\theta=0.6$ and $\theta=0.7$ because of large variability in its bid values $b_t$. Winning a single slot with $\theta=0.7$ precludes winning two slots with $\theta=0.6$, and since the objective is to maximize the total number of won slots, $\cA_b$ falls behind $\opt$ and has linear regret with respect to $\opt$ since it wins slots having $\theta=0.7$ with non-zero probability. This can be shown rigorously by using the same results on random walks as done in Lemma \ref{lem:lbApd}.

Thus, an algorithm that hopes to achieve a sub-linear regret has 
to inherently learn the maximum bid $\opt$ will make.
Towards that end, we propose algorithm $\cA_c$ that bids
 \begin{equation}\label{eq:bidprocess}
b_t = v+ \frac{1}{\sqrt{T}}\text{Margin}(t-1)
\end{equation}
 where the Margin process is as defined in Definition \ref{defn:ccv}.
%updates as follows, 
% \begin{equation}\label{defn:marginprocessgen}
%\text{Margin}(t) = \max\{\text{Margin}(t-1) + (v x_t(b_t) - p_t), 0\},
%\end{equation}
%with $\text{Margin}(0)=0$.
Similar to $\cA_b$, $\cA_c$ also satisfies the ROSC  on a sample path basis.

%We describe in detail the algorithm and the associated result for allocation functions of type  \eqref{def:thresholdfunc}. How to extend the result to allocation functions satisfying the Myerson's condition \eqref{eq:myersontruthfulcond} is described in Remark  \ref{rem:gendefmu}.

For allocations functions satisfying \eqref{def:thresholdfunc}, the Margin process  simplifies to 
% \begin{equation}\label{eq:bidprocess}
%b_t = v+ \frac{1}{\sqrt{T}}\text{Margin}(t-1),
%\end{equation}
   \begin{equation}\label{defn:marginprocess}
\text{Margin}(t) = \begin{cases} \text{Margin}(t-1) + (v-\theta_t) & \text{if} \ b_t \ge \theta_t 
\\ 
 \text{Margin}(t-1), & \text{otherwise}.
 \end{cases}
\end{equation}
%since $\text{Margin}(t)\ge 0$ for all $t$.
%\vspace{-0.1in}
%In \eqref{defn:marginprocess}, we have defined $\cA_c$ for . For general allocation functions, the `if' condition in \eqref{defn:marginprocess} will be replaced by  %Algorithm $\cA_c$ wins slot $t$ only if $\frac{1}{\sqrt{T}}\text{Margin}(t-1) \ge \theta_t - v_t$. 
$\cA_c$ is a conservative algorithm that overbids by an amount exactly equal to its non-negative slack in ROSC-S until that slot scaled down by $\sqrt{T}$. 
Motivation behind $\cA_c$ can be best understood by following the execution of $\cA_c$ again over Example \ref{exm:1}.
%First we identify the behaviour of $\opt$. Since $v_t$ is constant, an algorithm is just trying to maximize the number of slots it can win while satisfying \eqref{eq:prob}. Thus, among slots where $\theta_t>v$, $\opt$ will win only those slots where $\theta_t=0.6$ by bidding $b_t=0.6$. Since slots with $\theta_t=0.4$ are won for 'free', the total  accrued valuation of  $\opt$ is $T v \cdot 2/3$.
$\cA_c$, in contrast to $\opt$, initially does not win any slots with $\theta_t \in \{0.6, 0.7\}$, but once its bid value `converges' (as can be seen in Fig. \ref{fig:1}), wins all slots $t$ with $\theta_t =0.6$ and no slot $t$ with $\theta_t=0.7$. With $\cA_c$, $b_t$ crosses the optimal maximum bid value of $0.6$ of $\opt$ in $O(\sqrt{T})$ time and stays very close to it throughout, thereby avoiding winning any slots with $\theta_t=0.7$. 
%Thus, the accrued valuation of $\cA_c$ is $T v \cdot 2/3 - O(\sqrt{T})$, and the $1/\sqrt{T}$ scaling is critical in getting a regret better than that of $\cA_b$.

%Why $1/\sqrt{T}$ scaling matters can be understood by considering an alternate algorithm $\cA_b$ (that also ) that just bids $b_t = v + \text{Margin}(t-1)$ and 
%satisfies ROSC . 
%as a result for $T= 1000000$,  $\cA_b$ wins 
%a slot with $\theta=0.7$  with probability $0.07$, and that is the primary reason why its accrued valuation is $0.59 T \cdot v$  instead of $.6650 T  \cdot v$ as achieved by $\cA_c$.

In general, $\cA_c$ lets its bid process converge to a value close to the maximum bid that $\opt$ will make, and then operates its bid profile in a narrow band around that value which avoids winning slots with large values of $\theta$ (that $\opt$ avoids), while ensuring that enough slots that $\opt$ wins are also won by it. To achieve this, $\cA_c$ sacrifices a regret of $O(\sqrt{T})$ in the initial phase where its bid is very low because of scaling down by $1/\sqrt{T}$.
%concentrated around $\bbE\{v}+\mu_L$, where $\mu_L$ is the `left' mean. Thereafter if there is sufficient mass of $\theta$ around $v_t+\mu_L$ with right mean $\mu_R \sim \mu_L$, 
%then $\cA_c$ wins all those slots. In case, there is not sufficient mass of $\theta$ around $v_t+\mu_L$ and $\mu_R> \mu_L$, then the $\text{Margin}$ process 
%drifts upwards towards concentrating around the right mean, and wins all slots with probability $p$ such that $p \mu_R = \mu_L$.
Next, we make this intuition formal, and the {\bf main result} of this section is as follows.

\begin{theorem}\label{thm:ubequalcase}
 When $v_t=v >0, \ \forall \ t$, the regret of $\cA_c$ is at most $O(\sqrt{T \log T})$ when allocation functions are of type \eqref{def:thresholdfunc} and Assumptions \ref{assump:drift} and \ref{assump:updown} \footnote{Defined on next page after stating some definitions.} holds.
\end{theorem}
%\newpage
%proof of Theorem \ref{thm:ubequalcase} extends to all allocation functions in a straightforward manner, see Remark \ref{rem:gendefmu}.
Combining Theorem \ref{thm:ubequalcase} with  Lemma \ref{lem:lbuniv}, we conclude that algorithm $\cA_c$ achieves optimal regret up to logarithmic terms. Even though Theorem \ref{thm:ubequalcase} is applicable only for allocation functions of type \eqref{def:thresholdfunc} and under some assumptions, it is an important (first) step towards finding algorithms with optimal regret that strictly satisfies the ROSC.

Full proof of Theorem \ref{thm:ubequalcase} is provided in Appendix \ref{app:thm:ubequalcase}, with the following proof outline.
As assumed, let $v_t=v>0, \ \forall \ t$. 
%This helps to present the idea crisply. We will remove this restriction later. 
  We first define a few quantities.
 For fixed $v$, let  $\mu_L = \bbE\{1_{v \ge \theta} \cdot(v-\theta)\}$, while $\mu_R = \bbE\{1_{v< \theta}\cdot (v-\theta)\}$. Note that $\mu_L$ and $\mu_R$ 
 are expected per-slot positive and negative contributions to the ROSC, respectively, if all slots are won, e.g. by bidding $b_t=1 \ \forall \ t$.
 Let $\mu_L + \mu_R \le 0$. The other case is easier to handle since then the per-slot expected contribution to ROSC is non-negative, and an algorithm can win all slots by bidding $b_t=1$ while 
satisfying the ROSC.
 
 \begin{rem}\label{rem:degenerate} Algorithm $\cA_{pd}$ of \cite{Feng} is shown to satisfy the ROSC when $\mu_L + \mu_R > 0$. But in this setting the problem is trivial, since bidding $b_t=1$ always 
 guarantees zero regret while satisfying the ROSC  in the repeated identical auction setting.
 \end{rem}
 
Given that $v_t=v$ for all $t$, the objective is to maximize the total number of won slots while satisfying the ROSC. Essentially, the identity of which slots are won is immaterial. Therefore, for $\opt$, there exists a $\theta^\star$ and $0<\pi^\star\le 1$ such that it bids 
$b_t = \theta_t$ whenever $\theta_t<\theta^\star$ and wins all those slots, while bids 
$b_t=\theta^\star$ with probability $\pi^\star$ when $\theta_t=\theta^\star$ (wins slots with $\theta_t=\theta^\star$ with probability $\pi^\star$) such 
that the ROSC is satisfied. In particular, 
\begin{equation}\label{defn:thetastar}
\pi^\star \bbE\{1_{v= \theta^\star}\cdot (v-\theta^\star)\} + \bbE\{1_{v< \theta^\star}\cdot (v-\theta^\star)\}+ \mu_L = 0.
\end{equation}
 
% Given the context, it is clear that we are referring to a fixed $v$, thus we drop the $(v)$ in the respective quantities as long as the meaning is clear.
%With abuse of notation,  let $\b1_{\theta}=1$ with probability $p$ and $0$ otherwise. Then we define, 
%  %\vspace{-0.1in}
%\begin{equation}\label{defn:thetastar}
%\theta^\star = \max \{\theta : \exists p > 0 \ \text{s.t.}\   \bbE\{\b1_{\theta}1_{v< \theta}\cdot (v-\theta)\} + \mu_L \ge 0\}.
%\end{equation}
%Since $v_t=v>0, \ \forall \ t$, $\opt$ is just trying to maximize the number of slots it wins.
%Thus, $\theta^\star$ is the highest bid value that $\opt$ makes when $\mu_L + \mu_R \le 0$ to satisfy the ROSC, i.e.  $\opt$ cannot improve its expected accrued valuation by bidding $b_t>\theta^\star$
%while satisfying the ROSC.
%Since $\bbE\{\b1_{\theta}1_{v< \theta}\cdot (v-\theta)\} $ is a monotone decreasing function of $p$, we also get $\pi^\star$ such that  
%\begin{equation}\label{defnpstar}
%\bbE\{\b1_{\theta^\star}1_{v< \theta^\star}\cdot (v-\theta^\star)\} + \mu_L = 0,
%\end{equation} where $\b1_{\theta^\star}=1$ with probability $\pi^\star$ and $0$ otherwise. Thus, the highest bid  $\opt$ makes is $b_t=\theta^\star$ with probability $\pi^\star$.
Let ${\underline \mu}_R^{\theta^\star} = 
\bbE\{\b1_{v< \theta}\cdot (v-\theta) | \theta<\theta^\star\}$ and $\bar {\mu}_R^{\theta^\star} = 
\bbE\{\b1_{v< \theta}\cdot (v-\theta) | \theta \ge \theta^\star\}$.  
  \begin{definition}\label{defn:mindrift}
For $\cA_c$, the increments (positive jumps) to the Margin process \eqref{defn:marginprocess} have mean $\mu_L$, while the mean of the decrements (negative jumps) to the Margin process \eqref{defn:marginprocess} is  $\underline {\mu}_R^{\theta^\star}$ (a negative quantity) until $b_t<\theta^\star$. Thus, the $\text{Margin}(t)$ process \eqref{defn:marginprocess} has  drift $\Delta = \mu_L +\underline {\mu}_R^{\theta^\star}$ until $b_t<\theta^\star$.
%Let this minimum positive drift be denoted as $$\Delta = \mu_L + \max_{\theta<\theta^\star}\mu_R^{\theta}.$$ 
Following \eqref{defn:thetastar}, $\Delta>0$. 
%$\Delta$ is the minimum positive drift to the $\text{Margin}(t)$ process \eqref{defn:marginprocess} of $\cA_c$ for all $t$ until $b_t< \theta^\star$.
\end{definition}
%Hereafter, we will refer to $\theta^\star(v)$ as $\theta^\star$ as long 
% as it is being referred for a fixed $v$ is clear. 
 \begin{assumption}\label{assump:drift} $\Delta$ does not depend on $T$.
 \end{assumption}
  \begin{assumption}\label{assump:updown} $|\bar {\mu}_R^{\theta^\star}| \le |\underline {\mu}_R^{\theta^\star}|$.
 \end{assumption}
Assumption \ref{assump:updown} implies that the negative drift to $\text{Margin}(t)$ process \eqref{defn:marginprocess} obtained by winning slots with $\theta_t > v$ is weaker after the bid process crosses $\theta^{\star}$ compared to before crossing $\theta^{\star}$.
 It is worth noting that Assumption \ref{assump:updown} does not make the problem trivial, and is needed for technical reasons in the proof.
 %Essentially Assumption \ref{assump:drift} entails that the distribution of $\theta_t$ does not depend on $T$, which is a reasonable assumption to make.
 
 
  \begin{definition}
Let $\tau_{\min}= \min\{t: b_t \ge \theta^\star\}$ be the earliest time at which $b_t$ process \eqref{eq:bidprocess} crosses $\theta^\star$. 
\end{definition}
\begin{definition}\label{defn:slotindex}Let the set of all $T$ slots be denoted by $\cT$. For $t\ge  \tau_{\min} $ the subset of slots where $\theta=\theta^\star$ be denoted by $\cT_{\theta^\star}$, while where $\theta> \theta^\star$ by $\cT^+_{\theta^\star}$ and $\theta< \theta^\star$ by $\cT^{-}_{\theta^\star}$. 
\end{definition}

Using Definition \ref{defn:slotindex} we can write the regret for $\cA_c$ as 
\begin{equation}\nn
\cR_{\cA_c}(T)= \cR_{\cA_c}([1:\tau_{\min}]) +   \cR_{\cA_c}( \cT_{\theta^\star})  +  \cR_{\cA_c}(\cT^+_{\theta^\star})+ \cR_{\cA_c}(\cT^{-}_{\theta^\star}). 
\end{equation}
From definition \eqref{defn:thetastar}, $\opt$ does not win any slots of $\cT^+_{\theta^\star}$, thus, 
$\cR_{\cA_c}(\cT^+_{\theta^\star})\le 0$.
%Let the set of all $T$ slots be denoted by $\cT$. For $t\ge  \tau_{\min} $, the subset of slots where $\theta=\theta^\star$ be denoted by $\cT_{\theta^\star}$, while where $\theta> \theta^\star$ by $\cT^+_{\theta^\star}$, and $\theta< \theta^\star$ by $\cT^{-}_{\theta^\star}$. 
Under Assumption \ref{assump:drift}, the rest of the {\bf proof of Theorem \ref{thm:ubequalcase} is divided in three parts}, i) we show that $\bbE\{\tau_{\min}\} = O(\sqrt{T})$ which implies that $\cR_{\cA}([1:\tau_{\min}])=O(\sqrt{T})$, ii) $\cR_{\cA_c}(\cT^+_{\theta^\star})\le 0$  by showing that $\cA_c$ and $\opt$ win equal number of slots in expectation when $\theta = \theta^\star$, and finally iii) $\cR_{\cA_c}(\cT^{-}_{\theta^\star}) = O(\sqrt{T \log T})$ using a result from \cite{Koksal} stated in Lemma \ref{lem:RAstar} that shows that bid process \eqref{eq:bidprocess} concentrates very close to $\theta^\star$, and $\cA_c$ wins any slots having $\theta<\theta^\star$ with probability 
$\Omega(1-1/\sqrt{T})$.

%Theorem \ref{thm:ubequalcase} can be extended to all $x_t$'s, $p_t$'s satisfying the Myerson's condition as follows.

%%\begin{rem}\label{rem:gendefmu} For  functions $x_t$ satisfying the Myerson's condition \eqref{eq:myersontruthfulcond},  let 
% $b^\star$ be such that for $b=b^\star$, \eqref{eq:myersontruthfulcond} holds with an equality. Recall from Section \ref{sec:warmup} that $b^\star \ge v$. Thus,  the counterparts of $\mu_L$ and $\mu_R$ (defined for proving Theorem \ref{thm:ubequalcase}) for general allocation functions satisfying \eqref{eq:myersontruthfulcond} are
%  $\mu_L^{\text{gen}} = \sum_{b, b \le b^\star} \bbE\{(v x_t(b)-p_t(b))\} \ge 0$ and $\mu_R^{\text{gen}} =  \sum_{b, b >  b^\star} \bbE\{ (v x_t(b)-p_t(b))\} < 0$.  
% [XXX:The counterpart of $\theta^\star$ needs to be identified.] 
%  %We prove Theorem \ref{thm:ubequalcase} for allocation  functions \eqref{def:thresholdfunc}, but the result for general allocation functions follows by replacing $\mu_L$ and $\mu_R$ respectively with $\mu_L^{\text{gen}}$ and $\mu_R^{\text{gen}}$.
% \end{rem}

With the case $v_t=v \ \forall \ t$ resolved, we next move on to the more challenging and interesting case when $v_t\ne v, \ \forall \ t$ and derive a $1/2$-competitive algorithm.

