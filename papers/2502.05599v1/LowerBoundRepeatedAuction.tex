\section{Repeated Identical Auction Setting $v_t=v  >0\ \forall \ t.$}
\subsection{Lower Bound on Regret for Any Online Algorithm $\cA$}
We derive a general result connecting the regret and the $\text{CCV}$ for any online algorithm $\cA$ that solves \eqref{eq:prob} when $v_t=v$ for all $t$, as follows.

\begin{theorem}\label{thm:lbuniv}
Even when $v_t=v, \ \forall \ t$, for any  $\cA$ that solves \eqref{eq:prob}, we have that 
  $\cR_\cA(T) + \bbE\{\text{CCV}_\cA(T)\} = \Omega(\sqrt{T}).$
\end{theorem}

As a corollary of Theorem \ref{thm:lbuniv}, we get that the following  {\bf main result} of this section.
\begin{lemma}\label{lem:lbuniv}
Even when $v_t=v, \ \forall \ t$, $\cR_\cA(T)= \Omega(\sqrt{T})$ for any $\cA$ that solves \eqref{eq:prob} and strictly satisfies the ROSC.
\end{lemma}
The proof of Theorem \ref{thm:lbuniv} is provided in Appendix \ref{app:lbequal}, where the main idea is similar to that of Theorem \ref{thm:lbunivGenV}
 of constructing two inputs 
with small K-L divergence (making them difficult to distinguish for any $\cA$) but for which the behaviour of $\opt$ is very different to ensure that the 
ROSC is satisfied. Compared to Theorem \ref{thm:lbunivGenV}
 there is less freedom since $v_t=v$ for all $t$, and hence we get a weaker lower bound. 
 %Then $\cA$'s limitation in not knowing the input beforehand is utilized to derive the result. 

Lemma \ref{lem:lbuniv} exposes the inherent difficulty in solving Problem \ref{eq:prob}, that is even if $v_t=v$ is known, the problem remains 
challenging, and either the regret or the ROSC violation has to scale as $\Omega(\sqrt{T})$. 

Currently, the best known algorithm \cite{Feng} for solving Problem \eqref{eq:prob} when Myerson's condition \eqref{eq:myersontruthfulcond} is satisfied has both regret and ROSC violation guarantee of $O(\sqrt{T})$. Thus, Theorem \ref{thm:lbuniv} does not imply a lower bound of $\Omega(\sqrt{T})$ on the regret of algorithm \cite{Feng}. In the next section, we separately show that the regret of algorithm \cite{Feng} is $\Omega(\sqrt{T})$ irrespective of its constraint violation even in the repeated identical auction setting. This is accomplished by connecting the regret of algorithm \cite{Feng} to the hitting probability of a reflected random walk. 
