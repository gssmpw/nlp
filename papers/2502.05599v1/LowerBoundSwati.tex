\section{Lower Bound on Regret for algorithm \cite{Feng} }


The primal-dual based algorithm $\cA_{pd}$ constructed in \cite{Feng}  for minimizing the regret \eqref{defn:regret}, chooses
$b_t = v_t+\frac{v_t}{\lambda_t},$ where 
$\lambda_t = \exp\left(\alpha \ \text{CCV}(t-1)\right)$ with $\lambda_1=1$ and $\alpha=1/\sqrt{T}$.
%Let  denote this algorithm.
From \cite{Feng}, we have $\cR_{\cA_{pd}}(T) =O(\sqrt{T})$ and ROSC-S of $\text{CCV}(T) = O(\sqrt{T\log T})$. 
%, however, does not 
%necessarily satisfy the ROSC and has been shown to have a $\bbE\{\text{CCV}(T)\}$ of  $$  without any extra assumptions. 
Next, we show that $\cR_{\cA_{pd}}(T) =\Omega(\sqrt{T})$.
\begin{lemma}\label{lem:lbApd} $\cR_{\cA_{pd}}(T) =\Omega(\sqrt{T})$ even if $v_t=v \ \forall \ t$ and the Myerson's condition \eqref{eq:myersontruthfulcond} is satisfied.
\end{lemma}
\begin{proof}
{\bf Input:} Let $v_t=v=1/2 \ \forall \ t$. Let the allocation function be $x_t^{\theta_t}$ \eqref{def:thresholdfunc} in any slot, where $\theta_t = \{0,1\}$ with equal probability which satisfies the Myerson's condition \eqref{eq:myersontruthfulcond}. Since $\theta_t$ is symmetric around $v_t=1/2$,  $\opt$ can bid $b_t=\theta_t$ for each slot $t$, and win every slot resulting in its total accrued valuation of $vT = T/2$, while clearly satisfying  the ROSC. 





Since $v_t=1/2 \ \forall \ t$, the  total accrued valuation of algorithm $\cA_{pd}$ is 
$\sum_{t=1}^T \b1_{\cA_{pd}}(t) v = \sum_{t=1}^T \b1_{\cA_{pd}}(t) \frac{1}{2}.$
Note that because of the choice of $v_t$ and $\theta_t$, $\text{CV}(t) = p_t^{\theta_t}(b_t) - v_t x_t^{\theta_t}(b_t)$ takes only two values $\left\{-\frac{1}{2}, \frac{1}{2}\right\}$ 
for a slot that is {\bf won} by $\cA_{pd}$. For a slot that is {\bf lost}, $\text{CV}(t)=0$.

By the choice of the algorithm $\cA_{pd}$, whenever $\text{CCV}(t-1) = 1/2$, $\lambda_t= \exp\left(\alpha \ \text{CCV}(t-1)\right)\ge 1$ and its bid $b_t = v_t + \frac{v_t}{\lambda_t}< 1$ and $\cA_{pd}$ wins that slot $t$ only if $\theta_t=0$.
Therefore the CCV process for $\cA_{pd}$ evolves as follows, 
\begin{equation}\label{defn:CCVprocessFeng}
\text{CCV}(t)=\begin{cases}\text{CCV}(t-1) \pm 1/2 \   \text{w.p. $\{1/2, 1/2\}$ if} \  
\text{CCV}(t-1) < 1/2, \\
\text{CCV}(t-1) -1/2 \ \ \text{if}\ \theta_t=0 \ \ \ \text{and} \
\text{CCV}(t-1) =1/2,
\\
\text{CCV}(t-1)\ \quad \quad \quad \text{if}\ \theta_t=1\ \ \   \text{and} \  
\text{CCV}(t-1) = 1/2. \end{cases}
\end{equation}
Thus, the $\text{CCV}(t)$ process is a simple reflected (backwards at $1/2$) random walk with increments $\pm 1/2$ with equal probability. Using classical results on reflected random walks we proceed further.

\begin{definition}\label{defn:randomwalk}\cite{port1965}
Consider a simple random walk process $\cW$ with i.i.d. increments  $X_i$ (a random variable) having $\bbE\{X_1\}=0$ and $\bbE\{X_1^2\}<\infty$  reflected at zero, i.e. $\cW_n = \max\{\cW_{n-1}+X_n,0\}$, where 
$\cW_0=0$.
 Let $\bbP_n(x,y)$ be the probability that $\cW_n=y$ given $\cW_0=x$.
\end{definition}

\begin{prop}\label{prop:classicalrandomwalkresult1}\cite{port1965} 
For Definition \ref{defn:randomwalk},  
$\lim_{n\rightarrow \infty} \bbP_n(0,0)\sqrt{n} = c,$ for constant $c>0$. Moreover, $\lim_{n\rightarrow \infty} \frac{\bbP_{n+1}(x,0)}{\bbP_n(0,0)} =1, \ \forall \ x. $
\end{prop}
%\begin{prop}\label{prop:classicalrandomwalkresult2}\cite{port1965}

%\end{prop}

Using the two parts of Proposition \ref{prop:classicalrandomwalkresult1} for the CCV process \eqref{defn:CCVprocessFeng}, and noting that \eqref{defn:CCVprocessFeng} is reflected at $1/2$, we have 
$\bbP(\text{CCV}(t) = 1/2)\ge c/ \sqrt{t}$ for $t$ large.
%probability that the CCV process \eqref{defn:CCVprocessFeng} is at $1/2$ at time $t$ is $\ge c/ \sqrt{t}$ for $t$ large. 
Whenever, $\text{CCV}(t) =1/2$,  $\cA_{pd}$'s bid is $<1$ and it wins a slot $t$ only if $\theta_t=0$, which happens with probability $1/2$.
Therefore, using linearity of expectation, we get that the expected number of  slots {\bf lost} by $\cA_{pd}$ is 
$\ge \frac{1}{2} \sum_{t=T/2}^T \frac{c}{\sqrt{t}} = \Omega(\sqrt{T}),$
where we are accounting only for $t\ge T/2$ since Proposition \ref{prop:classicalrandomwalkresult1} applies for large $t$. 
Recall that $\opt$ wins all the slots, thus the regret of $\cA_{pd}$ is at least as much as its number of lost slots.
% $=\Omega(\sqrt{T})$. 
%the regret of algorithm $\cA_{pd}$ is at least $\Omega(\sqrt{T})$.
\end{proof}

%To prove Lemma \ref{lem:lbApd} ROS  constraint \eqref{eq:prob} was not enforced  since $\cA_{pd}$ violates it. 
As far as we know there is no known algorithm even in the repeated auction setting $v_t=v \ \forall \ t $  for Problem \ref{eq:prob} that strictly satisfies the ROSC and achieves the lower bound of Lemma \ref{lem:lbuniv} unless one considers a trivial setting \cite{Feng} to be described in Remark \ref{rem:degenerate}. 
In the next section, we construct one such algorithm when the allocation and payment function are of the threshold type \eqref{def:thresholdfunc} under some mild assumptions.  %Since algorithm $\cA_{pd}$ violates ROSC vIn light of Lemma \ref{} it is not clear whether there exists 
%
%An emergent question that Lemma \ref{lem:lbApd} poses is whether the regret of any online algorithm for solving Problem \eqref{eq:prob} when $v_t=v \ \forall \ t$ is $\Omega(\sqrt{T})$ or it is just the limitation of $\cA_{pd}$, and is answered  next. 