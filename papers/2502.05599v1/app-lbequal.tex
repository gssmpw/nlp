%\vspace{-0.1in}
\section{Proof of Theorem \ref{thm:lbuniv}}\label{app:lbequal}
\begin{proof}
%\vspace{-0.05in}
Consider Input 1: $v_t=\frac{1}{2}$ $\forall \ t$, and allocation functions are $x_t^{\theta_t}$  \eqref{def:thresholdfunc}, where $\theta_t$ takes four possible values $A=\frac{1}{2}-u, B=\frac{1}{2}-w, C=\frac{1}{2} + m_1 - \delta, D=\frac{1}{2} + m_1$, with equal probability of $1/4$ where $0< w< u < \frac{1}{2}$, $m_1 = \frac{\frac{1}{2}-u + \frac{1}{2}-w}{2}$ (the conditional mean of $\theta_t$ given $\theta_t \in \{A,B\}$) and $\delta=\frac{1}{\sqrt{T}}$.

Similarly let Input 2: $v_t=\frac{1}{2}$ $\forall \ t$, and allocation functions are $x_t^{\theta_t}$ \eqref{def:thresholdfunc}, where $\theta_t$ takes the same four values as in Input 1, $A,B,C,D$, but with perturbed probabilities. 
In particular, $\bbP(\theta_t\in \{A, B\}) =\frac{1}{2}$, but the marginal probabilities of $\{A, B\}$ are chosen such that the conditional mean of $\theta_t$ given $\theta_t \in \{A,B\}$ is $m_1+\delta$, while 
$\theta_t\in\{C, D\}$ are chosen with probability $1/4$ each as with Input 1. 

See Fig. \ref{fig:inputrepauction} for a pictorial description where quantities with blue and red color correspond to same sign on their ROSC contribution.
\begin{figure}
\includegraphics[width=10cm,keepaspectratio,angle=0]{FigRepeatedAuction.png}
\caption{Input used to prove Theorem \ref{thm:lbuniv}.}
\label{fig:inputrepauction}
\end{figure}

With both Input 1 and Input 2, $(v_t=1/2, \theta_t)$ is realized i.i.d. in each slot $t$ with distribution as described above. 

Consider the behaviour of $\opt$. If the input is Input 1, $\opt$ bids $b_t = D$ for all slots and wins all of them while satisfying the ROSC, since the expected positive (negative) per-slot contribution to ROSC when $\theta \in \{A,B\}$ ($\theta \in \{C,D\}$) is $m_1$ ($\ge -m_1$). 
%Bidding $b_t = D$, $\opt$ wins all slots, and 
Consequently, the total  accrued valuation by $\opt$ is $v\cdot T = T/2$, while satisfying the ROSC.

 In contrast, if the input is Input 2, then consider an algorithm $\cB$ (that need not be $\opt$) that always bids $b_t = C$. Clearly, $\cB$ satisfies the ROSC, thus so can $\opt$.

Given that the input is either Input 1 and Input 2, WLOG, we let any $\cA$ only bid either $b_t = C$ or 
$b_t=D$ for all $t$. 
%\vspace{-0.165in}
%Let $\text{CCV}(t)= p_t - v_tx_t(b_t)$ and recall that $\text{CCV}=\sum_t \text{CCV}(t)$.
\begin{lemma}\label{lem:ccvC}
For any slot $t$, if  $\cA$ bids $b_t=C$, then under  Input 2, $$\bbE\{\text{CV}_{\cA}(t)\}= -\left(\frac{1}{4}-\frac{3m_1}{4}-\frac{\delta}{4}\right).$$
With $m_1=\frac{1}{4}$, $\bbE\{\text{CV}_{\cA}(t)\}=-(\frac{1}{16}-\frac{\delta}{4})$.
\end{lemma}
%\vspace{-0.1in}
\begin{proof}
Whenever  $\cA$ bids $b_t=C$ and let under Input 2. if $\theta_t =D$, then  $\text{CV}_{\cA}(t)=0$ since $\cA$ loses that slot and $p_t^{\theta_t}=0, v_tx_t^{\theta_t}(b_t)=0$.
Thus, the only cases to consider are when $\theta_t \in \{A,B,C\}$, and for which we get $\bbE\{\text{CV}_{\cA}(t)\} =$
%\begin{align}\nn
$ - \frac{1}{4} \left(\frac{1}{2} - C\right) - \frac{1}{2}\left(\frac{1}{2}-(m_1+\delta)\right) = -\left(\frac{1}{4}-\frac{3m_1}{4}-\frac{\delta}{4}\right)$,
%\end{align}
where the last equality is obtained by plugging in the value for $C$. 
%Given that $m_1 < \frac{1}{2}$ and $\delta$ is small, $\frac{1}{4}-\frac{3m_1}{4}-\frac{\delta}{4} >0$.  
Let $m_1=\frac{1}{4}$, then $\bbE\{\text{CV}_{\cA}(t)\}=-(\frac{1}{16}-\frac{\delta}{4})$ when $\cA$ chooses $b_t=C$ for any slot $t$.
\end{proof}


\begin{lemma}\label{lem:ccvD}
For any slot $t$, if $\cA$ bids $b_t=D$, and then under Input 2, then $\bbE\{\text{CV}_{\cA}(t)\} =\frac{\delta}{4}.$
\end{lemma}
\begin{proof}
With Input 2, if $\cA$ chooses $b_t=D$ in slot $t$, all possible values of $\theta_t \in \{A,B,C,D\}$ result in non-trivial CV. 
We compute the expected $\text{CV}$ as follows.  
\begin{align}\nn
\bbE\{\text{CV}_{\cA}(t)\} &= -\frac{1}{4} \left(\frac{1}{2} - C\right) - \frac{1}{4} \left(\frac{1}{2} - D\right) -   \frac{1}{2}\left(\frac{1}{2}-(m_1+\delta)\right) = \frac{\delta}{4}.
\end{align}
%where the last equality is obtained by plugging in the values for $C$ and $D$.
\end{proof}

\begin{lemma}\label{lem:ccvlb}
If the number of slots for which $\cA$ chooses $b_t=D$ is at least as much as $T-\sqrt{T}$, then if the input is Input 2, 
$\bbE\{\text{CCV}_{\cA}(T)\} \ge \frac{\sqrt{T}}{16} - \frac{T^{1/4}}{4}$ for $\delta=\frac{1}{\sqrt{T}}$.

\end{lemma}
\begin{proof}  $\bbE\{\text{CCV}_{\cA}(T)\} = \sum_{t=1}^{T} \bbE\{\text{CV}_{\cA}(t)\}$
$$\stackrel{(a)}\ge -\left(\frac{1}{16}-\frac{\delta}{4}\right)\sqrt{T} + \frac{\delta}{4} (T-\sqrt{T}) \stackrel{(b)} \ge  \frac{\sqrt{T}}{16} - \frac{T^{1/4}}{4},$$
where $(a)$ follows by combining Lemma \ref{lem:ccvC} and \ref{lem:ccvD} if the number of slots for which $b_t=D$ is at least as much as $T-\sqrt{T}$, while $(b)$ follows by plugging in $\delta=\frac{1}{\sqrt{T}}$.
\end{proof}

Let the probability measure or expectation under Input $i, \ i=1,2$ be denoted as $\bbP^i$ or $\bbE^i$.
Let 
$$\cE = \{\# \text{of slots $t$ for which} \ b_t=C \ \text{by} \ \cA > T-\sqrt{T}\}.$$
Recall that with Input $1$, $\theta_t=D$ with probability $1/4$. Therefore all slots, where $\cA$ chooses $b_t=C$ and for which $\theta_t=D$, are lost by $\cA$. Recall that with Input 1, the total accrued reward of $\opt$ is $T/2$. 
Thus, the regret of $\cA$ is at least 
$$\cR_{\cA}(T) \ge \frac{\sqrt{T}}{4} \bbP^1(\cE).$$

As pointed before, any online algorithm $\cA$ will only bid $b_t = C$ or 
$b_t=D$ for all $t$ given the definition of Input 1 and 2, i.e., either event $\cE$ happens or its complement $\cE^c$. 
Thus, from Lemma \ref{lem:ccvlb}, the constraint violation for $\cA$ is 
$$\bbE\{\text{CCV}_\cA(T)\} \ge \left(\frac{\sqrt{T}}{16} - \frac{T^{1/4}}{4}\right)\bbP^2(\cE^c).$$

From the Bretagnolle-Huber inequality \cite{lattimore2020bandit}
 we have 
$$\bbP^1(\cE) + \bbP^2(\cE^c) \ge \frac{1}{2} \exp\left(-\textsf{KL}(\bbP^1|| \bbP^2)\right) \stackrel{(a)}\ge \frac{1}{2} \exp\left(- 2T \delta^2\right),$$
%\end{equation}
where $(a)$ follows from the choice of Input 1 and Input 2.

Therefore, we get 
$$\cR_{\cA}(T) + \bbE\{\text{CCV}_{\cA}(T)\} \ge \frac{\sqrt{T}}{16} \exp\left(- 2T \delta^2\right) - o(T).$$
Since $\delta=1/\sqrt{T}$, we get the result.
\end{proof}
%at 
%$\cR_\cA(T) + \bbE\{\text{CCV}_\cA(T)\} \ge \Omega(\sqrt{T}).$