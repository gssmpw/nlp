\section{Proof of Theorem \ref{thm:ubequalcase}}\label{app:thm:ubequalcase}
{\bf Case I $\mu_L + \mu_R\le 0$}
\begin{prop}\label{prop:crossingtime}
$\bbE\{\tau_{\min}\}=O(\sqrt{T})$ if
$\Delta = \mu_L+ \underline {\mu}_R^{\theta^\star} = \mu_L+ \bbE\{\b1_{v< \theta}\cdot (v-\theta) | \theta<\theta^\star\}>0$ does not depend on $T$.
\end{prop}
\begin{proof}
From Definition \ref{defn:mindrift}, $\text{Margin}(t)$ process  for $\cA_c$ has  drift $\Delta >0$ until $b_t<\theta^\star$ (and consequently the $b_t$ process has drift $\Delta/\sqrt{T}$). Moreover, since $\theta^\star\le 1$, following a standard 
result from random walks with positive drift \cite{feller1991introduction}, we have $\bbE\{\tau_{\min}\}=O(\frac{\sqrt{T}}{\Delta}) = O(\sqrt{T})$.
%  Therefore, the Margin process has positive drift with expectation $\Delta$ till $\text{Margin}(t)<\theta^\star$. Moreover, since $\theta^\star<1$, thus,  with probability 
%  at least $1-1/\sqrt{T}$, $b(t) \ge \theta^\star$ in $O(\sqrt{T})$ slots.
\end{proof}

%In general, we get the following result that bounds the regret of $\cA_c$ by time $\tau_{\min}$.
%We also get the following result.
%\begin{proposition}\label{prop:iddist}
 %For some constant $c>1$, for $t\ge c \sqrt{T}$,  the bid process $b_t$ is identically distributed across time $t$.
%\end{proposition}
Proposition \ref{prop:crossingtime} implies the following Lemma.
 \begin{lemma}\label{lem:regretcrossingtime}
$\cR_{\cA_c}([1:\tau_{\min}])=O(\sqrt{T})$ when $\Delta$ does not depend on $T$.
\end{lemma}
%\begin{proof}
%If $\tau_{\min} = O(\sqrt{T})$, then $\cR_{\cA_c}([1:\tau_{\min}]) =O(\sqrt{T})$ trivially.
%%For the claim to be false, $\tau_{\min} = \Omega(\sqrt{T})$ since otherwise trivially, 
%Thus, let $\tau_{\min} = \Theta(T^{1/2 + \alpha})$ for some $\alpha>0$. Moreover, let the claim that $\cR_{\cA_c}(\tau_{\min})=O(\sqrt{T})$ be false. Thus, by time $\tau_{\min}$, there are $\Theta(T^{1/2+ \beta})$ for some $0 < \beta \le \alpha$ slots that $\opt$ wins but 
%$\cA_{c}$ does not. Note that by definition of $\cA_c$, the slots it does not win are necessarily those have $\theta_t < v$. 
%Let $T_{\text{last}}$ be the last slot that $\opt$ wins but not $\cA_{c}$ before time $\tau_{\min}$, and at time $T_{\text{last}}$, $\opt$ satisfies 
%the ROSC.\footnote{Without loss of generality, since in each slot $(v_t, \theta_t)$ is i.i.d., we let $\opt$ satisfy the ROSC in each slot.}
%
%Recall that the Margin process of $\cA_c$ for all time slots remains non-negative. Moreover, since $\opt$ wins $\Theta(T^{1/2+ \beta})$ slots that $\cA_c$ does not, we have $\text{Margin}_{\cA_c}(T_{\text{last}}) - \text{Margin}_\opt(T_{\text{last}})\ge - \left(\sum_{t=1}^{\Theta(T^{1/2+ \beta})} \b1_{\theta_t > v}(v-\theta_t))\right) > 0,$ where we have renumbered the slots where $\opt$ wins but not $\cA_{c}$ as $1, \dots, \Theta(T^{1/2+ \beta})$ until time $T_{\text{last}}$. 
%Moreover, $\text{Margin}_\opt(T_{\text{last}})\ge0$, since at time $T_{\text{last}}$, $\opt$ satisfies 
%the ROSC.   Therefore, $\text{Margin}_{\cA_c}(T_{\text{last}}) \ge -{\hat \mu}_R^{\theta^\star} \Theta(T^{1/2+ \beta})$, 
%where ${\hat \mu}_R^{\theta^\star} = \frac{1}{\Theta(T^{1/2+ \beta})}\sum_{t=1}^{\Theta(T^{1/2+ \beta})} \b1_{\theta_t > v}(v- \theta_t))$ and ${\hat \mu}_R^{\theta^\star}\rightarrow \mu_R^{\theta^\star}< 0$.
%Recall that the bid made by $\cA_c$ at time $T_{\text{last}}$ is  $v + \frac{1}{T^{1/2}} \text{Margin}_{\cA_c}(T_{\text{last}}).$ Given that $\theta_t\le 1$ for all $t$, 
%and $\text{Margin}_{\cA_c}(T_{\text{last}}) \ge -{\hat \mu}_R^{\theta^\star} \Theta(T^{1/2+ \beta}) > 0$ and $\beta>0$,
%this means that $\cA$ wins (if $v + \frac{1}{T^{1/2}} \text{Margin}_\cA(T_{\text{last}}) > \theta_t$) the slot $T_{\text{last}}$, thus  contradicting the hypothesis that $\cR_{\cA_c}([1:\tau_{\min}])=O(\sqrt{T})$ is false.
%\end{proof} 


Next, we characterize  $\cR_{\cA_c}(\cT^{-}_{\theta^\star})$ 
by establishing a concentration bound for $b(t)$ as follows.
\begin{lemma}\label{lem:coupling}
For $\cA_c$, at any $t\ge \tau_{\min}$,  $\bbP(b_t < (1-\delta)\theta^\star) \le \exp\left(-  \delta \Delta \theta^\star \sqrt{T} c\right)$, where $c$ is a constant that depends on the distribution of allocation function \eqref{def:thresholdfunc}.
\end{lemma} 



\begin{proof}
Consider the following  process
\begin{equation}\label{eq:alg2}
N(t) = \max\{N(t-1) + A(t) - B(t),0\}
\end{equation} and 
\vspace{-0.1in}
\begin{equation}\label{eq:alg1}
B(t) = \begin{cases} 
\mu - \nu^{-} & \text{if} \  N(t) \le U/2,\\
 \mu + \nu^{+} &\text{if} \ N(t) > U/2,
\end{cases}
\end{equation} 
where $A(t)\ge 0$ is a ``well-behaved" arrival process\footnote{Asymptotic semi-invariant log moment generating function exists for $(-\inf, r)$ for some $r>0$.} with $\bbE\{A(t)\}=\mu$ and $U >0$ is some threshold, and $0< \nu^+<\nu^-  < \mu$.  
Essentially, $N(t)$ can be thought of as a queue length process where arrivals $A(t)$ are exogenous, and departures $B(t)$ are regulated by a controller with an objective of keeping  $N(t)$ close to a threshold of $U/2$ as much as possible. To achieve this, the control $B(t)$ is chosen to be little more (less) than the expected arrival rate $\mu$ whenever $N(t)$ is above (below) the threshold $U/2$. The upward drift whenever $N(t) \le U/2$ in \eqref{eq:alg1} is $\nu^{-}$ and downward drift is $\nu^+$ when $N(t) \le U/2$.


In \cite[Lemma 2]{Koksal} show that 
\begin{equation}\label{eq:resultkoksal}
\bbP(N(t) =0) \le \exp\left(- \nu^{-} U c\right),
\end{equation} for any  $U$, and where $c$ depends on the 
distribution of $A(t)$.

%Compared to this, the margin process for $\cA_c$ evolves as \eqref{defn:marginprocess} and bid process $b_t$ as \eqref{eq:bidprocess}. 
%$\text{Margin}(t) $
% \begin{equation}\label{defn:marginprocessapp}
%= \begin{cases} \text{Margin}(t-1) + (v_t-\theta_t) & \text{if} \ v_t+ \frac{1}{\sqrt{T}}\text{Margin}(t-1) \ge \theta_t, 
%\\ 
% \text{Margin}(t-1), & \text{otherwise},
% \end{cases}
%\end{equation}
%with $\text{Margin}(0)=0$, and decrement of $v_t-\theta_t $ when $\theta_t>v_t$ is made to the  \text{Margin} process only if  $\text{Margin}(t-1) \ge \sqrt{T}(\theta_t - v_t)$.
%Moreover, the bid process for $\cA_c$ evolves as
% \begin{equation}\label{eq:bidprocessrecursive}
%b_t = v + \frac{1}{\sqrt{T}}\text{Margin}(t-1). 
%\end{equation}
To derive our result we make the following connection between $\text{Margin}(t)$ process \eqref{defn:marginprocess} and \eqref{eq:alg2}.
The $\text{Margin}(t)$ process \eqref{defn:marginprocess} (that drives the bid process \eqref{eq:bidprocess}) is similar to \eqref{eq:alg2}, where the possible increments/decrements at slot $t$ are 
$(v_t-\theta_t)$ (which can be positive/negative) only if $\text{Margin}(t-1)\ge \sqrt{T}(v_t-\theta_t)$. Thus, $\sqrt{T} \theta^\star$ in \eqref{defn:marginprocess} is analogous to the threshold $U/2$ in \eqref{eq:alg2}, and the increments/decrements $(v_t-\theta_t)$  are analogous 
to $A(t)-B(t)$ in \eqref{eq:alg2}, and the enforced drift $\nu^{-}$ in \eqref{eq:alg1} below $U/2$  is now $\mu_L+ \underline {\mu}_R^{\theta^\star}=\Delta > 0$ when the $\text{Margin}$ process is below $\sqrt{T} \theta^\star$ for \eqref{eq:bidprocess}) and $\nu^+$ is $\mu_L+\bar {\mu}_R^{\theta^\star}$ when the $\text{Margin}$ process is above $\sqrt{T} \theta^\star$.
From Assumption \ref{assump:updown}, $|\mu_L+\bar {\mu}_R^{\theta^\star}| < |\Delta|$. Thus,  following an identical proof to \cite[Lemma 2]{Koksal} used to derive \eqref{eq:resultkoksal}, 
%that decomposes the bad event $(b(t) < (1-\delta)\theta^\star)$ into union over $t'$ of bad events $\cE(t')$ (starting from the last time $t'$ where $b_{t'}\ge \theta^\star$ at time $t$, $(b(t) < (1-\delta)\theta^\star)$ and upper bounding the probability of each bad event using the Chernoff bound, 
we get
$$\bbP(b(t) < (1-\delta)\theta^\star) \le \exp\left(-  \delta \Delta \theta^\star \sqrt{T} c\right).$$
\end{proof} 
%$\bbP(b(t)< (1-\delta)\theta^\star\  \forall \ t \ge \tau_{\min}) \le \sum_{t=\tau_{\min}}^T \exp\left(- \delta \Delta \sqrt{T} c\right) \le T  \exp\left(- \delta \Delta \sqrt{T} c\right)$.
%Let event $\cF = \{b(t)\ge (1-\delta)\theta^\star\  \forall \ t \ge \tau_{\min}\}$. Under $\cF$, we want to show that 


%From time $\tau_{\min}$ onwards, the $b(t)$ process \eqref{defn:marginprocess} has positive drift $c_1/\sqrt{T}>0$ when it is below $\theta^\star$ and negative drift ($c_2/\sqrt{T}<0$) when  $b(t)\ge \theta^\star$, for some constant $c_1,c_2$. 
%\vspace{-0.1in}
Using Lemma \ref{lem:coupling}, next, we upper bound  $ \cR_{\cA_c}(\cT^{-}_{\theta^\star})$.
%Among $\cT \backslash \{\cT^+_{\theta^\star}, \cT_{\theta^\star}\}$, we discount the first $\tau_{\min}$ slots, where we do not have the concentration result of Lemma \ref{lem:coupling}. 
%Let $\cT^- = \cT \backslash \{\cT^+_{\theta^\star}, \cT_{\theta^\star}\}$ for notational convenience. 
Choose $\delta=\sqrt{\frac{\log (T)}{T}}$. Lemma \ref{lem:coupling} implies that $\cA_c$ wins any slot belonging to $\cT^{-}$ for which $\theta < (1-\delta)\theta^\star$ with probability at least $1-\frac{c_1}{\sqrt{T}}$ when $\Delta$ is independent of $T$.
%\footnote{If there is any probability mass of $\theta$ in the interval $((1-\delta)\theta^\star, \theta^\star)$ for $\delta=\log (T)/\sqrt{T}$, then move that mass to $\theta = (1-\delta)\theta^\star$ which can only increase the accrued valuation for the $\opt$.} 
Thus, using the linearity of expectation, the expected number of slots that $\cA_c$ wins among $\cT^{-}_{\theta^\star}$ for which $\theta < (1-\delta)\theta^\star$ is 
$\left(1-\frac{c_1}{\sqrt{T}}\right)|\cT^{-}_{\theta^\star}|$. Moreover, for the interval, $[(1-\delta)\theta^\star, \theta^\star)$, let $T$ be large enough such that 
the total probability mass of $\theta \in  [(1-\delta)\theta^\star, \theta^\star)$ be at most $\delta=\sqrt{\frac{\log (T)}{T}}$, since $\theta^\star \le 1$. Thus,  
the expected number of slots that $\cA_c$ can lose among $\cT^{-}_{\theta^\star}$ for which $[(1-\delta)\theta^\star, \theta^\star)$ is 
$T \cdot \delta = T \sqrt{\frac{\log (T)}{T}} = \sqrt{T \log T}$.
Thus, 
\begin{equation}\label{eq:dummy11}
\cR_{\cA_c}(\cT^{-}_{\theta^\star}) = O(\sqrt{T \log T}).
\end{equation}

 Next, we identify the rate at which the $b_t$ process of $\cA_c$ crosses the threshold $\theta^\star$ \eqref{defn:thetastar} as a function of $\pi^\star$ \eqref{defn:thetastar}.
  \begin{lemma}\label{lem:hittingthetastar}
For $t\ge \tau_{\min} $, %\footnote{$c>1$ needed for Proposition \ref{prop:iddist} to hold},   
for $b_t$ \eqref{eq:bidprocess}, $\bbP(b_t\ge \theta^\star) \ge \rho^\star$, where $\rho^\star \bbP(\theta=\theta^\star) = \pi^\star$.
\end{lemma}
\begin{proof}
%Since from Proposition \ref{prop:iddist}, for $t\ge c \sqrt{T}$,  the bid process $b_t$ is identically distributed across time $t$, we consider any $t\ge c \sqrt{T}$ to prove the claim.
Let the claim be false, i.e, $\bbP(b_t\ge \theta^\star)< \rho^\star$. Then the probability of winning any slot by $\cA_c$ when $\theta=\theta^\star$ is   $\bbP(b_t\ge \theta^\star)\bbP(\theta=\theta^\star)< \rho^\star \bbP(\theta=\theta^\star) < \pi^\star$. 
But from the definition \eqref{defn:thetastar} of $\theta^\star$ and $\pi^\star$ when $\bbP(b_t\ge \theta^\star)\bbP(\theta=\theta^\star)< \pi^\star$, the $\text{Margin}(t)$ process \eqref{defn:marginprocess} and effectively the $b_t$ process \eqref{eq:bidprocess} of $\cA_c$ has positive drift\footnote{The expected  increase minus the expected decrease in any slot.}
 even when $b_t=\theta^\star$. Thus, similar to Lemma \ref{lem:coupling}, we will get that with high probability $b_t > \theta^\star$ for $t\ge \tau_{\min}$.  However, given that the algorithm $\cA_c$ satisfies the ROSC on a sample path basis,  this contradicts the definition of $\theta^\star$, since $\theta^\star$ is the highest bid that $\opt$ can make with non-zero probability while satisfying the ROSC. 
 Thus, we get a contradiction to the claim that $\bbP(b_t\ge \theta^\star)< \rho^\star$.
\end{proof}
%$\bbP(b_t\ge \theta^\star) < \rho^\star$ for $o(T)$ slots
%Let $N^\star$ be the total number of slots where $b_t\ge \theta^\star$, and let $\bbE\{N^\star\}  < \rho^\star T$. 
%Then $\rho^\star \bbP(\theta=\theta^\star)< \pi^\star$. But from the definition of $\theta^\star$ (and consequently $\pi^\star$) when $\rho^\star \bbP(\theta=\theta^\star)< \pi^\star$, $b_t$ has positive drift and $b_t$ will cross  $\theta^\star$ more times on average than $\bbE\{N^\star\}$.
 

Next, using Lemma \ref{lem:hittingthetastar}, we show that $\cR_{\cA_c}( \cT_{\theta^\star})=0$.
\begin{lemma}\label{lem:RAstar}
  $\cR_{\cA_c}( \cT_{\theta^\star})=0.$
\end{lemma}
\begin{proof}\nonumber
\begin{align} \cR_{\cA_c}( \cT_{\theta^\star})  & = \bbE\{\#\text{slots won by} \ \opt \ \text{in} \ \cT_{\theta^\star}\}    -  \bbE\{\#\text{slots won by} \  \cA_c \ \text{in} \ \cT_{\theta^\star} \}, \\
&  \stackrel{(a)}\le  0,
\end{align}
where $(a)$ follows from Lemma \ref{lem:hittingthetastar} that shows that both $\cA_c$ and $\opt$ win slots with $\theta=\theta^\star$ with probability $\pi^\star$.
\end{proof}
\vspace{-0.1in}
%Lemma \ref{lem:hittingthetastar} establishes that $\cA_c$ wins the same expected number of slots as $\opt$ when $\theta=\theta^\star$. This is in fact an easy task to accomplish for an algorithm. The more delicate question is what happens to slots for which $\theta <\theta^\star$ all of which are won by $\opt$. Essentially, any algorithm has to win almost all such slots, together with winning the same expected number of slots as $\opt$ does when $\theta=\theta^\star$.


%Hence we get that 
%\begin{align}\nn
%  \cR_{\cA_c}(\cT^-)&=  \bbP(\tau_{\min}= O(\sqrt{T}))\left(\cR_{\cA_c}(1:\tau_{\min} | \tau_{\min}= O(\sqrt{T}))+ \cR_{\cA_c}(\cT^-\backslash \{[1:\tau_{\min}]\} | \tau_{\min}= O(\sqrt{T})) \right) \\ \nn
%  &  \quad + 
%  \bbP(\tau_{\min}= \Omega(\sqrt{T}))\cR_{\cA_c}(\cT^- | \tau_{\min}= \Omega(\sqrt{T})),  \\ \nn
%  & \stackrel{(a)}\le  (1-1/\sqrt{T})\left(O(\sqrt{T})  + O(\sqrt{T})\right) + \frac{1}{\sqrt{T}} . T, \\ \label{}
%  & =  O(\sqrt{T}),
%\end{align}
%where $(a)$ follows from \eqref{eq:dummy11}.
\begin{proof}[Proof of Theorem \ref{thm:ubequalcase}]
%When $\tau_{\min} = \Omega(T)$, then $\cR_{\cA}(T) = O(\sqrt{T})$ directly from Lemma \ref{lem:regretcrossingtime}. Otherwise, 
When  $\Delta$ does not depend on $T$, combining  Lemma \ref{lem:regretcrossingtime},  Lemma \ref{lem:RAstar}, and \eqref{eq:dummy11} and recalling that  $\cR_{\cA_c}(\cT^+_{\theta^\star})\le 0$, we get 
$$\cR_{\cA}(T) = \cR_{\cA}([1:\tau_{\min}]) + \cR_{\cA_c}(\cT^+_{\theta^\star}) + \cR_{\cA_c}( \cT_{\theta^\star})+  \cR_{\cA_c}(\cT^{-}_{\theta^\star}) \le O(\sqrt{T \log T}).$$ 
%since 
%$\cR_{\cA_c}(\cT^+_{\theta^\star})\le 0$, $\cR_{\cA_c}( \cT_{\theta^\star})=0$, and $\cR_{\cA_c}(\cT^-)= O(\sqrt{T})$.


%Recall that $\opt$ bids at most $b_t=\theta^\star$ in any slot. 
%Thus, from Lemma \ref{lem:coupling}, we get that 
%$$\bbP(\cA_c \ \text{loses slot} \ t\  \text{which} \ \opt \ \text{wins}) \le \frac{c}{\sqrt{t}}.$$
%
%
%Using linearity of expectation, thus, we get that 
%$$\bbE\{\text{Number of lost slots after $\tau_{\min}$ by} \ \cA_c \ \text{but won by} \ \opt\} \le \sum_{t=\tau_{\min}}^T\frac{c}{\sqrt{t}} \le O(\sqrt{T}).$$
% 
% Since the regret of any algorithm is at most the number of lost slots compared to $\opt$,  the regret of $\cA_c$ is at most $O(\sqrt{T})$ during time slot $[\tau_{\min}+1:T]$. Thus, we get 
%
%%Since $\tau_{\min}= O(\sqrt{T})$, with probability at least $1-1/\sqrt{T}$, we get that 
%\begin{align}\nn
%  \cR_{\cA_c}&=  \bbP(\tau_{\min}= O(\sqrt{T}))\left(\cR_{\cA_c}(1:\tau_{\min})+ \cR_{\cA_c}(\tau_{\min}+1:T)\right) +\bbP(\tau_{\min}= \Omega(\sqrt{T}))\cR_{\cA_c}(1:T),  \\ \nn
%  & \le  (1-1/\sqrt{T})\left(O(\sqrt{T})  + O(\sqrt{T})\right) + \frac{1}{\sqrt{T}} . T, \\ \label{}
%  & =  O(\sqrt{T}).
%\end{align}

%For $t\ge \tau_{\min}$, we next define a new process $Z(t)$ that is related to the  
% $\text{Margin}(t)$ process as follows.
%
%For $t\ge \tau_{\min}$, let
%\begin{equation}\label{defn:Zprocess}
%Z(t) =  \max\{Z(t-1) + \frac{1}{\sqrt{T}}(v-\theta_t), \theta^\star\},  
%\end{equation}
%where $Z(\tau_{\min})=\theta^\star$.
%Increments to the margin process and $Z(t)$ process are identical, however, the decrements are identical only if $v_t+ \frac{1}{\sqrt{T}}\text{Margin}(t-1) \ge \theta_t$ and there is no decrement to the $\text{Margin}(t)$ process otherwise. Thus, $\text{Margin}(t)\ge Z(t)$.
%
%An important point to note is that the $Z(t)$ process is a reflected random walk at $\theta^\star$ with increments $(v-\theta_t)$ with  $\bbE\{(v-\theta_t)\}=0$ 
%since $t\ge \tau_{\min}$.
%
%\begin{lemma}\label{lem:coupling} For $t\ge \tau_{\min}$, $\cA_c$ does not win slot $t$ which $\opt$ does only if $Z(t)=\theta^\star$.
%\end{lemma}
%\begin{proof}
%By definition $Z(t) \le \text{Margin}(t)$ for all $t$.
%Let $Z(t) > \theta^\star$ and $v-\theta_t <0$. This means that $Z(t-1)>\frac{1}{\sqrt{T}}(\theta_t - v) +\theta^\star$ implying that 
%$\text{Margin}(t-1) > \frac{1}{\sqrt{T}}(\theta_t - v) + \theta^\star$. Recall that $\opt$ bids at most $b_t=\theta^\star$ in any slot. 
%Thus, if $Z(t) > \theta^\star$ and $\opt$ wins that slot, then so does $\cA_c$.
%The case when $Z(t) > 0$ and $v_t-\theta_t \ge 0$ follows similarly.
%\end{proof}



%Thus, to upper bound the probability that $\cA_c$ loses a slot which $\opt$ wins it is sufficient to lower bound the probability that $Z(t)=\theta^\star$.

%For the problem \eqref{eq:prob} to be feasible, the constraint has to be satisfied, which for the case when all 
%allocations functions are of type \eqref{def:thresholdfunc}, takes the form $\bbE\{v_t - \theta_t\} \ge 0$ for all $t$ since the input is generated i.i.d.  

%Note that since $Z(t)$ process is defined for time $t\ge \tau_{\min}$, and the increment-decrement process $v - \theta_t$ for $t\ge \tau_{\min}$ has  
%$\bbE\{v - \theta_t\} = 0$, the process $Z(t)$ is identical to 
% random walk that is reflected at  $\theta^\star$ with expected increment of zero. Thus, we can use the result of Proposition \ref{prop:classicalrandomwalkresult1} to get that $ \bbP(Z(t)=\theta^\star) \le \frac{c}{\sqrt{t}}$.
%
%With our assumption, we have $\bbE\{v_t - \theta_t\} \ge 0$.
%Since we are interested in upper bounding the probability that $Z(t)=0$, we 
%consider the case when $\bbE\{v_t - \theta_t\} = 0$ which will give the largest upper  bound.
%Thus, appealing to the classical result stated in Proposition \ref{prop:classicalrandomwalkresult1} for a  random walk that is reflected at zero with expected increment of zero, we get that 


%Finally, now we remove the restriction that $v_t=v, \ \forall \ t$. We define 
%$\mu_L = \bbE\{\mu_L(v)\}, \mu_R = \bbE\{\mu_R(v)\}$ and  $\theta^\star = \bbE\{\theta^\star(v)\}$. Since $(v_t,x_t)$ is i.i.d., the proof is identical to the case 
%when $v_t=v$.


{\bf Case II  $\mu_L + \mu_R>0$} 
In this case, $\opt$ wins all slots since satisfying the ROSC  is trivial by bidding $b_t^\opt=1$ for all $t$.
For $\cA_c$, let $\tau'_{\min}= \min\{t: b_t \ge 1\}.$ Since $\mu_L + \mu_R>0$, the $\text{Margin}(t)$ process \eqref{defn:marginprocess} for $\cA_c$ has a positive drift for all slots. Thus, similar to Lemma \ref{lem:regretcrossingtime}, $\bbE\{\tau'_{\min}\} = O(\sqrt{T})$, and similar to 
Lemma \ref{lem:coupling}, with high probability $b_t \ge 1$ for $t\ge \tau'_{\min}$. Thus, similar to Case I, we get  $\cR_{\cA_c}(T) = O(\sqrt{T \log T}).$
%and hence $b_t \ge 1$ for $t\ge \tau'_{\min}$, where  
%Given that $\theta_t\le1, \ \forall t$, thus $\cA_c$ wins all slots after $\tau'_{\min}$. Thus,.
%\vspace{-0.1in}
\end{proof}


%\begin{comment}
%
%\begin{proof}
%  Recall the definition of winning a slot from Definition \ref{defn:slotwin}.
%  Algorithm $\cA_c$ wins slot $t$ only if $\text{Margin}(t-1) \ge \theta_t - v_t$, and consequently the Margin process updates as follows.
%  \begin{equation}\label{}
%\text{Margin}(t) = \begin{cases} \text{Margin}(t-1) + (v_t-\theta_t) & \text{if} \ v_t+ \text{Margin}(t-1) \ge \theta_t, 
%\\ 
% \text{Margin}(t-1), & \text{otherwise},
% \end{cases}
%\end{equation}
%with $\text{Margin}(0)=0$.
%To derive an upper bound on the regret of algorithm $\cA_c$ we want to lower bound the probability that $\cA_c$ wins a slot. Towards that end, consider another process that will couple the Margin process from below as follows. Let
%  \begin{equation}\label{defn:Zprocess}
%Z(t) =  \max\{Z(t-1) + (v_t-\theta_t), 0\},  
%\end{equation}
%where $Z(0)=0$.
%The $Z(t)$ process is a reflected random walk at zero with increments $ (v_t-\theta_t)$.
%
%\begin{lemma}\label{lem:coupling} $\cA_c$ does not win slot $t$ only if $Z(t)=0$.
%\end{lemma}
%\begin{proof}
%By definition $Z(t) \le \text{Margin}(t)$ for all $t$.
%Let $Z(t) > 0$ and $v_t-\theta_t <0$. This means that $Z(t-1)>\theta_t - v_t$ implying that 
%$\text{Margin}(t-1) > \theta_t - v_t$. Thus, if $Z(t) > 0$ then $\cA_c$ wins slot $t$.
%The case when $Z(t) > 0$ and $v_t-\theta_t \ge 0$ follows similarly.
%\end{proof}
%
%
%
%Thus, to upper bound the probability that $\cA_c$ loses a slot it is sufficient to lower bound the probability that $Z(t)=0$.
%
%%For the problem \eqref{eq:prob} to be feasible, the constraint has to be satisfied, which for the case when all 
%%allocations functions are of type \eqref{def:thresholdfunc}, takes the form $\bbE\{v_t - \theta_t\} \ge 0$ for all $t$ since the input is generated i.i.d.  
%
%
%
%With our assumption, we have $\bbE\{v_t - \theta_t\} \ge 0$.
%Since we are interested in upper bounding the probability that $Z(t)=0$, we 
%consider the case when $\bbE\{v_t - \theta_t\} = 0$ which will give the largest upper  bound.
%Thus, appealing to the classical result stated in Proposition \ref{prop:classicalrandomwalkresult1} for a  random walk that is reflected at zero with expected increment of zero, we get that 
%$ \bbP(Z(t)=0) \le \frac{c}{\sqrt{t}}$. From Lemma \ref{lem:coupling}, we get that 
%$$\bbP(\cA_c \ \text{loses slot} \ t) \le  \bbP(Z(t)=0) \le \frac{c}{\sqrt{t}}.$$
%
%
%Using linearity of expectation, thus, we get that 
%$$\bbE\{\text{Number of lost slots by} \ \cA_c\} \le \sum_{t=1}^T\frac{c}{\sqrt{t}} \le O(\sqrt{T}).$$
%Since the regret of any algorithm is at most the number of lost slots,  the regret of $\cA_c$ is at most $O(\sqrt{T})$.
%\end{proof}
%
%In light of Lemma \ref{lem:lbuniv}, thus, as long as each constraint function is of type defined in \eqref{def:thresholdfunc} and $\bbE\{v_t-\theta_t\}\ge 0$, algorithm $\cA_c$ is optimal.
%
%
%
%\subsection{Removing  the assumption $\bbE\{v_t-\theta_t\}\ge 0$}
%
%Simpler attempt.
%
%Let $\mu_L = \bbE\{1_{v_t\ge \theta_t} \cdot(v_t-\theta_t)\}$, while $\mu_R = \bbE\{1_{v_t< \theta_t}\cdot (v_t-\theta_t)\}$. If $\mu_R\le \mu_L$, we already have the result.
%So let $\mu_R> \mu_L$. Also, let both $\mu_L$ and $\mu_R$ be known to the algorithm (we will learn this eventually). 
%
%Then, consider a modification of algorithm $\cA_c$, defined by $\cA_{clr}$ that bids  $$b_t = v_t+ \b1_{t}\max\{\text{Margin}(t-1), 0\},$$
%where $\b1_t= 1$ with probability $\frac{\mu_L}{\mu_R}$ and $0$ otherwise. We are essentially scaling down the bids on the right side of $v_t$ so that the realized $\mu_R$ matches $\mu_L$.
%
%
%
%As before,  recall the definition of winning a slot from Definition \ref{defn:slotwin}.
%  Algorithm $\cA_{clr}$ wins slot $t$ only if $\text{Margin}(t-1) \ge \theta_t - v_t$, and consequently the Margin process updates as follows.
%  \begin{equation}\label{}
%\text{Margin'}(t) = \begin{cases} \text{Margin'}(t-1) + (v_t-\theta_t) & \text{if} \ v_t+ \text{Margin}(t-1) \ge \theta_t \ \text{and} \ \b1_t= 1, 
%\\ 
% \text{Margin'}(t-1), & \text{otherwise},
% \end{cases}
%\end{equation}
%with $\text{Margin}(0)=0$.
%
%Similar to $Z(t)$ process \eqref{defn:Zprocess}, we define a new process
%
%  \begin{equation}\label{defn:newZprocess}
%{\tilde Z}(t) =  \max\{{\tilde Z}(t-1) + \b1_t \cdot (v_t-\theta_t), 0\},  
%\end{equation}
%where ${\tilde Z}(0)=0$.
%The $Z(t)$ process is a reflected random walk at zero with increments $ \b1_t\cdot (v_t-\theta_t)$.
%and more importantly, the $\bbE\{ \b1_t\cdot (v_t-\theta_t)\} =0$ by our choice of $\b1_t$.
%
%Since the connection between the \text{Margin} process and $Z(t)$ and \text{Margin'} process and ${\tilde Z}(t)$ are identical, we get the following result.
%\begin{lemma}\label{lem:couplingnew} $\cA_{clr}$ does not win slot $t$ only if ${\tilde Z}(t)=0$.
%\end{lemma}
%
%Identical to the proof of Lemma \ref{lem:coupling}. Moreover, since the increments $ \b1_t\cdot (v_t-\theta_t)$ have $\bbE\{ \b1_t\cdot (v_t-\theta_t)\} =0$, and hence using the classical result 
%on a reflected random walk with mean zero increments, we get
%
%$$\bbP(\cA_{clr} \ \text{loses slot} \ t) \le  \bbP({\tilde Z}(t)=0) \le \frac{c}{\sqrt{t}}.$$
%
%
%Using linearity of expectation, thus, we get that 
%$$\bbE\{\text{Number of lost slots by} \ \cA_{clr}\} \le \sum_{t=1}^T\frac{c}{\sqrt{t}} \le O(\sqrt{T}).$$
%Since the regret of any algorithm is at most the number of lost slots,  the regret of $\cA_{clr}$ is at most $O(\sqrt{T})$.
%Thus, as long as we can learn $\mu_L$ and $\mu_R$ accurately, we can get the desired result.
%
%Possible strategy to learn $\mu_L$ and $\mu_R$ is for the first $\sqrt{T}$ slots, bid $b_t=v_t$ always, and estimate $\mu_L$ and $\mu_R$. 
%
%\subsubsection{Other alternative algorithms}
%
%Let the empirical mean of the increments to the $\text{Margin}$ process when $\theta_t < v_t$, be 
%$${\hat \mu}(t) = \sum_{\theta: \theta=\theta_t < v_t} f_{\theta} (v_t-\theta),$$
%where $f_{\theta}(t) = N_\theta/t$ and $N_\theta$ is the number of times $\theta_\tau=\theta$ for $1\le \tau\le t$.
%
%{\bf Algorithm $\cA_m$:} 
%
%At time $t$, algorithm $\cA_m$ bids  $$b_t = v_t+{\hat \mu}(t).$$
%
%Problem Input: $\theta=1$ with substantial probability mass, and the $\opt$ will pick $b_t=1$ with probability $\mu$ (true left-mean).
%
%As a fix if $$b_t = v_t+a_t,$$
%where $a_t = 1-v_t$ with probability ${\hat \mu}(t)$ and zero otherwise. Problem Input: Let $v_t=v$, and 
%$\theta=v+\mu$ with substantial probability mass, and the $\opt$ will pick $b_t=v+\mu$  always.
%
%
%
%
%
%
%%\begin{comment}
%\begin{rem}\label{rem:removeass}
%The assumption $\bbE\{v_t-\theta_t\}\ge 0$ can be avoided as follows. For allocation functions satisfying \eqref{def:thresholdfunc}, consider 
%$\opt$ for solving \eqref{eq:prob}. Clearly, since $\opt$ knows the input, it always bids $v_t$ whenever $\theta_t \le v_t$. Only non-trivial decision for $\opt$ is whether to 
%bid $b_t^\opt =\theta_t$ at slot $t$ when $\theta_t > v_t$ or not. Let $\b1^\opt_t = 1$ if $\opt$ 
%bids $b_t^\opt =\theta_t$ at slot $t$ when $\theta_t > v_t$, and zero otherwise. 
%
%For the $\opt$, let us  define a new process $Y(t)$ that tracks it Margin accrued  in slot $t$.
% \begin{equation}\label{}
%Y_t = \begin{cases} (v_t-\theta_t) & \text{if} \ v_t-\theta_t > 0, \\ 
% \b1^\opt_t(v_t-\theta_t) , & \text{otherwise}.
% \end{cases}
%\end{equation}
%
%
%Recall that $\opt$ also has to satisfy the ROSC. Thus, $\bbE\{Y_t\}\ge 0$ for any $t$. 
%
%Next, we define a process ${\tilde Z}(t)$ analogous to \eqref{defn:Zprocess}, as follows.
%\begin{equation}\label{}
%{\tilde Z}(t) =  \max\{{\tilde Z}(t-1) + Y_t, 0\},  
%\end{equation}
%where $Z(0)=0$. 
%We need to connect the $\text{Margin}(t)$ process of $\cA_c$ to process ${\tilde Z}(t)$. 
%
%Let $$U(t) = \sum_{\tau=1}^t \b1_{v_\tau\ge \theta_\tau} \cdot (v_\tau -  \theta_\tau)$$ be the  total positive margin seen till slot $t$, while 
%$$D(t) = \sum_{\tau=1}^t \b1_{\tau}^\opt \b1_{v_\tau< \theta_\tau}\cdot (v_\tau -  \theta_\tau)$$ be the total 
%negative margin accrued by $\opt$ till slot $t$. By the definition of the $\opt$, $\bbE\{U(t)\} = \bbE\{D(t)\}$.
%
%
%If at time $t$, $\text{Margin}(t) \ge {\tilde Z}(t)$, then if $\opt$ wins slot $t$ but not $\cA_c$, then neccesarily ${\tilde Z}(t)=0$, and we can use the classical result.
%The other possibility is that $\text{Margin}(t) < {\tilde Z}(t)$. In this case if $\opt$ wins slot $t$ but not $\cA_c$, we have no clear bound. So we instead attempt to upper bound the 
%probability that $\bbP(\text{Margin}(t) < {\tilde Z}(t))$.
%
%Consider $\cE$ represent the event $\text{Margin}(t) < {\tilde Z}(t)$. Recall that the upticks for both the $\text{Margin}(t)$ and ${\tilde Z}(t)$ are identical. The downticks can be different, and 
%$\cA_c$ is generally more conservative than $\opt$, however, its possible that at some slot $\tau < t$, $v_\tau + \text{Margin}(\tau) > \theta_\tau$ making $\cA_c$ win slot $t$ which was not won by $\opt$.
% This means that $U(t) > D(t) + \delta$ for some $\delta>0$. 
%
%
%%\begin{lemma}\label{lem:couplinOpt} $\bbP(\text{Margin}(t)>x) \ge \bbP({\tilde Z}(t)>x)$ for any $x$.
%%\end{lemma}
%%\begin{proof}
%%By the definition of $\opt$, 
%%\begin{equation}\label{eq:winprobAcOpt}
%%\bbP( \opt \ \text{wins slot} \ t) \ge \bbP( \cA_c \ \text{wins slot} \ t)\ \quad \text{for any} \  t,
%%\end{equation}
%% since both $\opt$ and $\cA$ satisfy the ROSC.
%%The upticks (where the value increases) for both the $\text{Margin}(t)$ and ${\tilde Z}(t)$ are identical. 
%%In particular, the uptick is equal to $v_t-\theta_t$ when 
%%$v_t-\theta_t > 0$. The downticks for $\text{Margin}(t)$ and ${\tilde Z}(t)$ are, however, different.
%%But, given that \eqref{eq:winprobAcOpt} is true this implies that 
%%\begin{equation}\label{lem:MarginTildeZcoupling}
%%\bbP(\text{Margin}(t)>x) \ge \bbP({\tilde Z}(t)>x) \ \text{for any} \ x.
%%\end{equation}
%%\end{proof}
%%
%%\begin{lemma} $\bbP( \cA_c \ \text{loses slot} \ t) \le \bbP({\tilde Z}(t)=0)$.
%%\end{lemma}
%%\begin{proof}
%%Let ${\tilde Z}(t) > 0$, $v_t-\theta_t <0$ and $\b1^\opt_{t}=1$. This means that ${\tilde Z}(t-1)>\theta_t - v_t$. From Lemma \ref{lem:MarginTildeZcoupling}, we get that  
%%$\bbP(\text{Margin}(t-1) > \theta_t - v_t) \ge \bbP({\tilde Z}(t) > 0)$. Thus, $\bbP( \cA_c \ \text{loses slot} \ t) \le \bbP({\tilde Z}(t)=0)$.
%%%if $Z(t) > 0$ then $\cA_c$ wins slot $t$.
%%%The case when $Z(t) > 0$ and $v_t-\theta_t \ge 0$ follows similarly.
%%\end{proof}
%%
%%
%%Now compared to having $\text{Margin}(t) \ge Z(t)$, we should be able to prove $$\bbP(\text{Margin}(t)>x) \ge \bbP({\tilde Z}(t)>x), \ \forall \ x.$$ Since $\bbE\{(v_t-\theta_t)\b1_{\theta_t\le \beta}\}\ge 0$, we will get the same result as above as long as $\bbP(\text{Margin}(t)>x) \ge \bbP({\tilde Z}(t)>x)$.
%\end{rem}
%
%
%
%$\bbE\{v_t-\theta_t\}\ge 0$ may not be true, but there exists a threshold $\beta$ such that $\bbE\{(v_t-\theta_t)\b1_{\theta_t\le \beta}\}\ge 0$ since the ROSC is satisfied by bidding $b_t=v_t$ always. With $\bbE\{(v_t-\theta_t)\b1_{\theta_t\le \beta}\}\ge 0$, thus compared to \eqref{defn:Zprocess}, we will get  
%
%
%
%
%
%\end{comment}
%
%




