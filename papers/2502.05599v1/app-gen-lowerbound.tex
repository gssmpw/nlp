\section{Proof of Theorem \ref{thm:lbunivGenV}}\label{app:lbnotequal}
\begin{proof}
Let $v_t$ take two values $\{\sfv_1, \sfv_2\} = \{0.5, 0.9\}$ with equal probability for all $t$.  The allocation functions are $x_t^{\theta_t}$ (of type \eqref{def:thresholdfunc}) where $\theta_t$ are distributed as follows.

Input 1: If for slot $t$, $v_t=\sfv_1$, then $\theta_t \in \{v_t\pm \epsilon\}$ with equal probability, where $0< \epsilon$ small. Moreover, if  for slot $t$,  $v_t=\sfv_2$, then 
$\theta_t = v_t+X$ where $X \in \{\sfa,\sfb\}$ with probability $r, 1-r$, respectively, and $\bbE\{X\} = r\sfa+(1-r)\sfb = \epsilon$ with $\sfa,\sfb \ge 0, \sfa\le \sfb , \sfv_2+\sfb \le 1$ and  $\sfa>\frac{2\epsilon}{3}$, $r<1-r$, i.e., $r< \frac{1}{2}$.

Input 2: If for slot $t$, $v_t=\sfv_1$, then $\theta_t \in \{\sfv_1\pm \epsilon\}$ with equal probability and if for slot $t$ $v_t=\sfv_2$, then 
$\theta_t = v_t+X'$ where $X'  \in \{\sfa,\sfb\}$ with probability $r-\delta , 1-r + \delta $, respectively with $r-\delta <1-r+\delta$. Let the choice of $\sfa,\sfb,\delta$ be such that $\bbE\{X'\} = 4\epsilon$.
For example, $\delta=\frac{3}{\sfb-\sfa}\epsilon$ suffices. Essentially, we need $\delta=\Theta(\epsilon)$.

See Fig. \ref{fig:inputgen} for a pictorial description where quantities with blue and red color correspond to the values of $v_t$ and $\theta_t$ that appear together.
\begin{figure}
\includegraphics[width=10cm,keepaspectratio,angle=270]{FigInputGen.pdf}
\caption{Input used to prove Theorem \ref{thm:lbunivGenV}.}
\label{fig:inputgen}
\end{figure}

Note that with both Input 1 and  2, the only case where $\theta_t< v_t$ is when $v_t=\sfv_1$. Thus, given the distribution, the total expected per-slot positive  contribution to ROSC is $\frac{\epsilon}{4}$ with both Input 1 and  2 which can be used 
by $\opt$ or any algorithm to win slots with $\theta_t>v_t$.


{\bf $\opt$'s actions:} By the choice of Input 1, the expected per-slot negative contribution to ROSC by winning a slot with $v_t=\sfv_1$ and $\theta_t=v_t+\epsilon$ is $-\epsilon/4$ while the expected per-slot negative contribution to ROSC by winning a slot with $v_t=\sfv_2$ and $\theta_t=v_t+X$ is $-\epsilon/2$. 
Moreover, the expected value from winning a slot with $v_t=\sfv_1$ and $\theta_t=v_t+\epsilon$ is $\sfv_1/4$ while by winning a slot with $v_t=\sfv_2$ and $\theta_t=v_t+X$ is $\sfv_2/2$.
Thus, since $\sfv_2>\sfv_1$, with Input 1, consider an algorithm $\cB$ that spends all the available 
total expected per-slot positive contribution to ROSC of $\frac{\epsilon}{4}$ in winning the maximum number slots with $v_t=\sfv_2$ while satisfying the ROSC. Thus, $\cB$'s bidding profile on input 1 is:  bid $b_t=v_t$ whenever $v_t=\sfv_1$, and bid $b_t=1$ with probability $1/2$ whenever $v_t=\sfv_2$. 
Thus, $\cB$ satisfies the ROSC exactly, and its expected accrued valuation is $\sfv_1T/4+\sfv_2 T/4$. Since $\opt$ is at least as good as any other algorithm while satisfying the ROSC exactly, $\opt$'s expected accrued valuation is $\ge \sfv_1T/4+\sfv_2 T/4$.
%Too see this, note that when $v_t=\sfv_1$,  by bidding $b_t=v_t$, $\opt$ gets an expected per-slot positive contribution of $\epsilon/4$ towards the ROSC, since $v_t=\sfv_1, \theta_t=\sfv_1-\epsilon$ with probability $1/4$. Moreover, since $\sfv_2>\sfv_1$, $\opt$ uses this positive contribution to  exclusively win slots when $v_t=\sfv_2=0.9$, by bidding $b_t=1$ with probability $1/2$ whenever $v_t=\sfv_2$ which results in a negative per-slot contribution of $-\epsilon$ to ROSC.  Since $v_t=\sfv_2$ with probability $1/2$,  this strategy of $\opt$ satisfies the ROSC and the maximum accrued valuation of $\opt$ is $\sfv_1T/4+\sfv_2 T/4$.

%On input 2, 
%$\opt$ will bid $b_t=v_t$ whenever $v_t=\sfv_1$ and $\theta_t=v_t-\epsilon$. Recall that the contribution of any slot that is won to the ROSC is $v_t - \theta_t$.
%  the constraint of \eqref{eq:prob} occurs only when $v_t=\sfv_1$ and $\theta_t=v_t-\epsilon$. Given that 
 %$v_t=\sfv_1$ and $\theta_t=v_t-\epsilon$ occur together with probability $1/4$, the total expected positive contribution per-slot made to \eqref{eq:prob} that can be used to win slots with $\theta_t>v_t$ is $\frac{\epsilon}{4}$. 
%Moreover, 
%$\opt$ will bid $b_t=v_t+\epsilon$ whenever $v_t=\sfv_1$ and $\theta_t=v_t+\epsilon$ with probability $p$ and 
%$\opt$ will bid $b_t=v_t$ whenever $v_t=\sfv_2$ and $\theta_t=v_t+a$ with probability $1-p$ such that 
%$$p \frac{1}{4} \epsilon + (1-p) a \frac{1}{2} \left(r-\delta\right) \epsilon = \frac{\epsilon}{4}.$$
%to satisfy the constraint.

Next, we characterize the structure of $\opt$ with Input 2.
%\vspace{-0.2in}
\begin{definition}\label{defn:p3}
Let any algorithm bid $b_t=v_t+\epsilon$ whenever $v_t=\sfv_1$ and $\theta_t=\sfv_1+\epsilon$, with probability $g_1$,  
bid $b_t=v_t+\sfa$ whenever $v_t=\sfv_2$ and $\theta_t=\sfv_2+\sfa$, with probability $g_2$, and bid $b_t=v_t+\sfb$ whenever $v_t=\sfv_2$ and $\theta_t=\sfv_2+\sfb$, with probability $g_3$.
\end{definition}
%\vspace{-0.2in} 
Following Definition \ref{defn:p3}, the expected accrued valuation per-slot by $\opt$ with Input 2 is
%\vspace{-0.1in} 
\begin{equation}\label{eq:accvalLBGen}
\max_{g_1, g_2, g_3} \ \ \frac{1}{4}\sfv_1 + \frac{1}{4}g_1 \sfv_1 + g_2 \sfv_2 \frac{1}{2} \left(r-\delta\right) + g_3 \sfv_2 \frac{1}{2} \left(1-r+\delta\right)
\end{equation}
subject to satisfying the ROSC that translates to
%\vspace{-0.15in}  
\begin{equation}\label{eq:consLBGen}
g_1 \frac{1}{4} \epsilon + g_2 \sfa \frac{1}{2} \left(r-\delta\right)  + g_3 \sfb \frac{1}{2} \left(1-r+\delta\right) \le \frac{\epsilon}{4}.
\end{equation}
%\vspace{-0.2in}
\begin{lemma}\label{lem:accvalLBGen} With Input 2, let $\bp^\star = (g_1^\star, g_2^\star, g_3^\star)$ be the $\opt$'s solution to maximize \eqref{eq:accvalLBGen} satisfying \eqref{eq:consLBGen}. Then $g_3^\star=0$.
\end{lemma}
%\vspace{-0.1in} 
\begin{proof}
Let $\opt$'s solution be $\bg_1 = (g_1, g_2, g_3)$ to maximize \eqref{eq:accvalLBGen} satisfying \eqref{eq:consLBGen}, where $g_3>0$.  To prove the result, we will perturb 
$\bg_1$ to create a new solution $\bg_1'$ that has $g_3'=0$ and a higher valuation than that of $\bg_1$, as follows.
It is easy to see that for $\opt$, $g_2\ge g_3$, since the expected per-slot negative contribution to ROSC with the two choices which are chosen with probability $g_2$ and $g_3$ are $-\sfa \frac{1}{2} \left(r-\delta\right)$ and $-\sfb \frac{1}{2} \left(1-r+\delta\right)$, respectively, where $\sfa<\sfb, r-\delta <1-r+\delta$, while the accrued value is $\sfv_2$ in both cases. 
Also for $\opt$, the constraint in \eqref{eq:consLBGen} will be tight. Thus, \eqref{eq:consLBGen} is 
\begin{align}\nonumber
g_1 \frac{1}{4} \epsilon + g_2 \sfa \frac{1}{2} \left(r-\delta\right)  + g_3 \sfb \frac{1}{2} \left(1-r+\delta\right) &= \frac{\epsilon}{4}, \\ \nonumber
 g_1 \frac{1}{4} \epsilon + (g_2-g_3) \sfa \frac{1}{2} \left(r-\delta\right) + \frac{g_3}{2} \left(\sfa  \left(r-\delta\right)  + \sfb  \left(1-r+\delta\right)\right) &\stackrel{(a)}= \frac{\epsilon}{4}, \\\label{eq:dummy11x1}
g_1 \frac{1}{4} \epsilon + (g_2-g_3) \sfa \frac{1}{2} \left(r-\delta\right) + \frac{g_3}{2} 4 \epsilon &\stackrel{(b)}= \frac{\epsilon}{4},
\end{align}
where $(a)$ follows by adding and subtracting $g_3 \sfa \frac{1}{2} \left(r-\delta\right)$ to the L.H.S. while (b) follows since $\bbE\{X'\} = \sfa  \left(r-\delta\right) + \sfb  \left(1-r+\delta\right) = 4 \epsilon$.
Compared to $\bg_1 = (g_1, g_2, g_3)$, consider another solution $\bg_1' = (g_1+8g_3, g_2-g_3, 0)$ that also satisfies \eqref{eq:dummy11x1}. Moreover,
the valuation  \eqref{eq:accvalLBGen} with $\bg_1'$ is larger than $\bg_1$, since $r-\delta\le 1$, $(1-r+\delta) \le 1$, and $\sfv_2/\sfv_1=0.9/0.5=1.8$.
Hence, $\bg_1$ cannot be optimal when $g_3>0$.
%because of $g_3>0$ is $\sfv_2 g_3 \frac{1}{2} \left(1-r+\delta\right)$, while with $g_1$ is  
%$\sfv_1 g_1 \frac{1}{4}$, and $\sfv_2/\sfv_1=0.9/0.5<2$. Since by definition $\left(1-r+\delta\right)\le 1$, thus, we get that $\bg_1' = (g_1+8g_3, g_2-g_3, 0)$ has higher valuation \eqref{eq:accvalLBGen} than $\bg_1 = (g_1, g_2, g_3)$ and satisfies \eqref{eq:dummy11x1}.
\end{proof}
To summarize, with Input 2, $\opt$ never bids $b_t >  v_t+\sfa$ whenever $v_t=\sfv_2$.


%\vspace{-0.05in}
The following corollary is immediate.
\begin{corollary}\label{cor:regretLB}
 With Input 2, any online algorithm $\cA$ that uses $g_3>\sfc>0$ ($\sfc$ is a constant) has regret of $\Omega(T)$ or $\cA$ violates the ROSC.
\end{corollary}
%\vspace{-0.1in}
Winning slots with option i) $v_t=\sfv_2, \theta_t=\sfv_2+\sfb$ compared to option ii) $v_t=\sfv_1,\theta_t=\sfv_1+\epsilon$ gives higher accrued value but a disproportionately larger per-slot expected negative contribution to ROSC.
Thus, one way to understand Corollary \ref{cor:regretLB} is that for any $\cA$ that exactly satisfies the  ROSC and has $g_3>\sfc$, must suffer a regret of $\Omega(T)$ with Input 2, since 
 with Input 2, having  $g_3>\sfc$, winning slots with $v_t=\sfv_2$ and $\theta_t=\sfv_2+\sfb$ comes at a disproportionate (ROSC) cost of losing out on winning slots with $v_t=\sfv_1$ and $\theta_t=\sfv_1+\epsilon$ given that $\sfv_2/\sfv_1<2$.
% 
%
%Given that $\sfv_2/\sfv_1<2$, Corollary \ref{cor:regretLB} is saying that with Input 2, this tradeoff is in favour of winning slots with $v_t=\sfv_1,\theta_t=\sfv_1+\epsilon$ that has less value and less negative contribution to ROSC. Thus, an algorithm that uses $g_3>\sfc>0$ and exactly satisfies the ROSC ends up losing out on winning $\Omega(T)$ more slots of type $v_t=\sfv_1,\theta_t=\sfv_1+\epsilon$ 
%
%the reason 
%Essentially, Corollary \ref{cor:regretLB} says that if an algorithm uses $g_3>\sfc>0$ and strictly satisfies the ROSC, then its better to 
%
%then its accrued valuation is $\Omega(T)$ less than an algorithm uses $g_3=0$ and  reduces 
%the number of slots won by an algorithm of type $v_t=\sfv_2$ and $\theta_t=\sfv_2+\sfb$
%
%, increase the probability $g_1$ by $g_1+8\sfc$ which will result in an addi accrued valuation since 
%$\sfv_2/\sfv_1<2$
%
%the competition is between winning slots $v_t=\sfv_2$ and $\theta_t=\sfv_2+\sfb$ or $v_t=\sfv_1$ and $\theta_t=\sfv_1+\epsilon$. Keeping $g_3>\sfc>0$, the accrued valuation of $\cA$ from slots $v_t=\sfv_2$ and $\theta_t=\sfv_2+\sfb$ is $c T x$ 
%
%
%allows an algorithm to win $g_3. x number of slots $v_t=\sfv_2$ and $\theta_t=\sfv_2+\sfb$ (with value accrued in each slot being $0.9$) but consumes a disproportionately large fraction of positive ROSC contribution of $\epsilon/4$
%
%Essentially, Corollary \ref{cor:regretLB} says that if $g_3>\sfc>0$, then $\cA$ loses out on the opportunity of winning slots with $v_t = \sfv_1$ in proportion more than 
%
%
%Winning a slot with $v_t=\sfv_2$ and $\theta_t=\sfv_2+\sfb$ gives a value of $0.9$ whil compared to a slots with $v_t=\sfv_1$ and $\theta_t=\sfv_1+\epsilon$.
%
%
% 
%  the increased amount of negative contribution to ROSC by winning slots having $v_t=\sfv_2$ and $\theta_t=\sfv_2+\sfb$ compared to when $v_t=\sfv_1$ and $\theta_t=\sfv_1+\epsilon$ limits the probability with which $\cA$ can obtain increased valuation of $\sfv_2>\sfv_1$ given $\sfv_2/\sfv_1<2$.

%Another way to understand Corollary \ref{cor:regretLB} is that for any $\cA$ that satisfies the  ROSC and has $g_3>\sfc$, must suffer a regret of $\Omega(T)$ since 
% with Input 2,  the increased amount of negative contribution to ROSC by winning slots having $v_t=\sfv_2$ and $\theta_t=\sfv_2+\sfb$ compared to when $v_t=\sfv_1$ and $\theta_t=\sfv_1+\epsilon$ limits the probability with which $\cA$ can obtain increased valuation of $\sfv_2>\sfv_1$ given $\sfv_2/\sfv_1<2$.




%Essentially, we are discounting the fact that $\opt$ ever bids $b_t=v_t+b$ when $v_t=\sfv_2$. This is true, since 
%if $\opt$ bids $b_t=v_t+b$ when $v_t=\sfv_2$ with probability $g_3$, where $\opt$  bids $b_t=v_t+\epsilon$ whenever $v_t=\sfv_1$ and $\theta_t=v_t+\epsilon$ with probability $g_1$ and 
%$\opt$ bids $b_t=v_t$ whenever $v_t=\sfv_2$ and $\theta_t=v_t+a$ with probability $g_2$,
% then the following equation has to be 
%satisfied
%
%Using the fact that $\bbE\{X'\} = a  \left(r-\delta\right) + b  \left(1-r+\delta\right) = 4 \epsilon$, it is immediate that the solution $(g_1, g_2, g_3)$ with $g_3>0$ comes at the expense of decreasing $g_1$ by a factor of $4$. However, since the two values of $v_t$ are $\{\sfv_1, \sfv_2\}$ the increased valuation is only twice. Thus, $g_3=0$ gives the optimal solution.
%
%
%we get that 
% 
%$b_t=v_t+\epsilon$ whenever $v_t=\sfv_1$ with probability, and bid nothing when $v_t=\sfv_2$. This is because of the following reason. With the above strategy, the value accrued by $\opt$ is $(\sfv_1/2)T$.
%Compared to this, consider any other algorithm $\cB$. By definition, value accrued in a slot is $\sfv_2$ only if $b_t =\sfv_2+X'$. However, to satisfy the constraint, the maximum probability $p$ with which any algorithm can bid $\sfv_2+X'$ when $v_t=\sfv_2$ is such that $$p\cdot\frac{1}{2}\cdot X' = \frac{1}{4}\epsilon,$$ under the condition that $\cB$ makes bid $b_t=\sfv_1$ whenever $v_t=\sfv_1$ thus never winning a slot where $\theta_t=v_t+\sfv_1$.
%Given $X'=10\epsilon$, we get that $p = \frac{1}{20}$. 
%Thus, the expected value accrued by $\cB$ is at most $(\sfv_1*\frac{1}{4} + \sfv_2*\frac{1}{40})T\le(\sfv_1/2)T$.
%Lemma \ref{lem:accvalLBGen} implies that  $\opt$ never 
%uses $g_3>0$ with Input 2.
%Therefore, the behaviour of $\opt$ with Input 1 and 2 is entirely different. In particular, 

Let the probability measure or expectation under Input $i, \ i=1,2$ be denoted as $\bbP^i$ or $\bbE^i$.
%Recall that with Input 1, the total accrued reward of $\opt$ is $\sfv_1T/4+\sfv_2 T/4$.  
Consider any online algorithm $\cA$. Let the number of slots for which $\cA$ bids $b_t \ge \sfv_2+\sfb$ when $v_t = \sfv_2$ be $N_\cA$. Recall Definition \ref{defn:p3} for $g_1,g_2, g_3$. Let $\cE$ be the event that $N_\cA = o(T)$, i.e. $g_3=\frac{o(T)}{T}$.


Similar to \eqref{eq:consLBGen}, for $\cA$ to satisfy the ROSC with Input 1, we have
$$g_1 \frac{1}{4} \epsilon + g_2 \sfa \frac{1}{2} \left(r\right)  + g_3 \sfb \frac{1}{2} \left(1-r\right) \le \frac{\epsilon}{4}.$$
Given that $g_3=\frac{o(T)}{T}$, we have 
$$g_1 \frac{1}{4} \epsilon + g_2 \sfa \frac{1}{2} \left(r\right)  +  \frac{o(T)}{T}\sfb \frac{1}{2} \left(1-r\right) \le \frac{\epsilon}{4}.$$
Moreover, by definition $\sfa \ge \frac{2\epsilon}{3}$. Thus, we get 
\begin{equation}\label{eq:input1efrosc}
g_1 \frac{1}{4} \epsilon + g_2 \frac{2\epsilon}{3} \frac{1}{2} r  \le \frac{\epsilon}{4},
\end{equation}
since 
$\frac{o(T)}{T}\sfb \frac{1}{2} \left(1-r\right)\ge 0$.
Moreover, the expected accrued valuation of $\cA$ with Input 1 is 
$$\left(\frac{\sfv_1 T}{4} + \frac{g_1\sfv_1 T}{4} + \frac{\sfv_2 g_2 T}{2} + \sfv_2 (1-r) o(T)\right).$$
Enforcing \eqref{eq:input1efrosc} and recalling that $r< \frac{1}{2}$ implies that the 
the expected accrued valuation of $\cA$ with Input 1 is at most
$$\left(\frac{\sfv_1 T}{4} + \frac{c \sfv_2T}{4} + \sfv_2 (1-r) o(T)\right),$$
since $\sfv_2=0.9$ and $\sfv_1=0.5$ for some constant $c<1$.

%Let $\cA$ bid $v_t = \sfv_2+ \sfa$ when $v_t=\sfv_2$  with probability $g_2$. Then with Input 1, the per-slot ROSC translates to 
%$\frac{1}{2} r g_2 \sfa + \frac{1}{2} (1-r) \sfb \frac{o(T)}{T} \le \frac{\epsilon}{4}.$ 
%Given that $\sfa > \frac{\epsilon}{2}$, we get $r g_2 < 1/2$. Thus, with Input 1, $\cA$ can win a slot where $v_t=\sfv_2$ and $\theta_t=\sfv_2+ \sfa$ with probability at most $c/2$ for some 
%$c < 1$.

%Thus, when event $\cE$ happens ($N_\cA = o(T)$),  the total accrued valuation for $\cA$ with Input 1 is 
%$$\le \left(\frac{\sfv_1 T}{4} + \frac{\sfv_2 c T}{4} + \sfv_2 (1-r) o(T)\right),$$ where $0<c<1$ is a constant, and the three terms correspond to winning slots with $v_t=\sfv_1$ and $\theta_t = \sfv_1-\epsilon$, $v_t=\sfv_2$ and $\theta_t=\sfv_2+\sfa$, and $v_t=\sfv_2$ and $\theta_t=\sfv_2+\sfb$, respectively.



Recall that with Input 1, the total accrued valuation for $\opt$ is at least
$\frac{\sfv_1 T}{4} + \frac{\sfv_2 T}{4}$.
Thus, when event $\cE$ happens ($N_\cA = o(T)$),
$$\cR_\cA^1(T) \ge \left(\frac{\sfv_2 (1-c) T}{4} \right) \bbP^1(\cE) > \Omega(T)\bbP^1(\cE).$$



However, if $\cE^c$ is true, then $g_3>\sfc_1$ (for some constant $\sfc_1$) for $\cA$, in which case Corollary \ref{cor:regretLB} implies that either the ROSC is violated by $\cA$ or the 
regret of $\cA$ is at least $\Omega(T)$. Thus, for $\cA$ that satisfies the ROSC, we have $$\cR^2_\cA(T) \ge \Omega(T) \bbP^2(\cE^c).$$
%as argued in Lemma \ref{lem:accvalLBGen} since instead of bidding $b_t\ge \theta_t$ when $v_t = \sfv_2$ for $N_\cA$ slots, 
%$\cA$ could have bid $b_t= \sfv_1+\epsilon$ in $4N_\cA$ slots where $v_t=\sfv_1$ and $\theta_t = \sfv_1+\epsilon$, and $\Omega(T)$ more accrued valuation.

% the  $\delta = 1/\sqrt{T}$, from Lemma \ref{lem:ccvlb}, the constraint violation for $\cA$ with Input 2 is 
%$$\bbE\{\text{CCV}\} \ge \left(\frac{\sqrt{T}}{16} - \frac{T^{1/4}}{4}\right)\bbP^2(\cE^c).$$
From the Bretagnolle-Huber inequality \cite{lattimore2020bandit}
 we have 
%\begin{equation}\label{eq:BHineq}
$$\bbP^1(\cE) + \bbP^2(\cE^c) \ge \frac{1}{2} \exp\left(-\textsf{KL}(\bbP^1|| \bbP^2)\right) \stackrel{(a)}\ge \frac{1}{2} \exp\left(- 2T c_2\delta^2\right),$$
%\end{equation}
where $(a)$ follows from simple computation given the choice of Input 1 and Input 2 and $c_2$ is a constant.

Therefore, we get 
$$\cR_\cA^1(T) + \cR_\cA^2(T) \ge \Omega(T) \exp\left(- 2c_2T \delta^2\right).$$
Choosing $\epsilon =1/\sqrt{T}$ and since $\delta=\Theta(\epsilon)$, we get that 
$\cR_\cA^1(T) + \cR_\cA^2(T) \ge \Omega(T).$
\end{proof}