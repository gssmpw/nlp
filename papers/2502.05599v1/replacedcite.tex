\section{Prior Work}
Single learner auto-bidding problem has been considered quite extensively in literature, but one allowance that is almost universally made is accepting some violation of the ROSC. For example, in ____, an algorithm with $\cR(T) = O(\sqrt{T})$ is proposed but it violates the ROSC by an amount  $O(\sqrt{T\log T})$. Prior to this, ____ derived an algorithm with $\cR(T) =O(T^{3/4})$ together with $O(T^{3/4})$ ROSC violation. For a slightly different objective function of $(v_t-s_t)$ at time $t$, where $s_t$ is the second largest bid at time $t$ from other bidders, and both $v_t,s_t$ are generated i.i.d., ____ derives a threshold based algorithm with $\cR(T) = O(T^{2/3})$ and satisfies the ROSC. In addition to ROSC, additional constraint on total payment made 
has also been considered in ____. 
Related to auto-bidding problem is the online allocation problem ____, that only has budget constraints and no ROSC.
%where requests arrive sequentially during a finite horizon
%and, for each request, a decision maker needs to
%choose an action that consumes a certain amount
%of resources and generates revenue. The revenue
%function and resource consumption of each request are drawn i.i.d. with an independently and at random
% unknown distribution. The objective is to maximize
%cumulative revenues subject to a constraint on the
%total consumption of resources.


More complicated auto-bidding models ____ 
have also been studied where multiple agents compete directly, however, always following a second-price auction, for which $O(\sqrt{T})$ regret and ROSC satisfaction with high probability are known. 


\vspace{-0.1in}