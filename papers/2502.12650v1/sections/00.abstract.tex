\begin{abstract}

Read disturbance in modern DRAM is an important \agy{0}{robustness (security, safety, and reliability)} problem, where repeatedly accessing (hammering) a row of DRAM cells (DRAM row) induces bitflips in other physically nearby DRAM rows.
Shrinking technology node size exacerbates \agy{0}{DRAM} read disturbance over generations.
To help mitigate read disturbance, the latest DDR5 \agy{0}{specifications (as of April 2024)} introduced a new RowHammer mitigation framework, called \gls{prac}. \gls{prac}
1)~enables the DRAM chip to accurately track row activations by allocating an activation counter per row and
2)~provides the DRAM chip with \om{2}{the} necessary time window to perform RowHammer-preventive refreshes by introducing a new back-off signal.
Unfortunately, no prior work rigorously studies \gls{prac}'s security guarantees and overheads.
\agy{0}{In this paper, we}
1)~present the first rigorous security, performance, energy, and cost analyses of \gls{prac} and
2)~propose \X{}, \ous{0}{a new mechanism that} addresses \gls{prac}'s two major weaknesses.

Our analysis shows that \gls{prac}'s system performance overhead on benign applications is non-negligible for \ous{0}{modern} DRAM chips and prohibitively large for future DRAM chips that are more vulnerable to read disturbance.
We identify \param{two} weaknesses of \gls{prac} that cause these overheads.
First, \gls{prac} increases critical DRAM access latency parameters due to the additional time required to increment activation counters.
Second, \gls{prac} performs a constant number of preventive refreshes at a time,
making it vulnerable to an adversarial access pattern, known as the wave attack, and consequently requiring it to be configured for significantly smaller activation thresholds.

To address \gls{prac}'s two weaknesses, we propose a new on-DRAM-die RowHammer mitigation mechanism, \X{}. \X{} 
1)~updates row activation counters concurrently \ous{5}{while} serving accesses by separating counters from the data and
2)~prevents the wave attack by dynamically controlling the number of preventive refreshes performed.
Our performance analysis shows that \X{}'s system performance overhead is near-zero for modern DRAM chips and very low for future DRAM chips.
\X{} outperforms \param{three} variants of \gls{prac} and \param{three} other state-of-the-art \om{2}{read disturbance} solutions.
We discuss \X{}'s and \gls{prac}'s implications \ous{0}{for future systems} and foreshadow future research directions.
To aid future research, we open-source our \X{} implementation at \url{https://github.com/CMU-SAFARI/Chronus}.\ouscomment{1}{Repository will be ready after artifact evaluation results}

\end{abstract}
