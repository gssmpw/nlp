\section{Security Analysis \om{4}{of \X{}}}

We analyze the security of \X{} against read disturbance bitflips.
\ous{5}{We base our analysis on \X{}'s \param{three} key properties:
$P1$) \X{} accurately tracks the activation count of all rows (as it employs per row activation counters)\om{6}{;}
$P2$) \X{} can trigger a back-off at any time (as it does \emph{not} enforce a delay period)\om{6}{;} and
$P3$) \X{} back-offs remain in effect until all rows that reach the back-off threshold have been refreshed}.\ouscomment{5}{Properties are defined here}

Let $A(i)$ denote the activation count of row $i$ and $N_{BO}$ denote the back-off threshold.
We assume that a read disturbance bitflip is only possible if a row can be activated at least \gls{nrh} times \ous{0}{before \om{4}{its} victims are preventively refreshed}.
Therefore, a system is secure if and only if \param{$A(i) < \nrh{}$} \ous{0}{holds} for all rows $i$ at all times.
We prove that \X{} is secure against read disturbance bitflips by investigating the system state in four phases:
1) before a back-off is triggered,
2) during the window of normal traffic,
3) during the recovery period, and
4) after the recovery period.

First, before a back-off is triggered, the maximum number of activations \om{4}{to} a single row \om{4}{can be at most} $N_{BO} - 1$ (\om{6}{following from} property $P1$ and \ous{5}{property} $P2$).
Second, \ous{0}{during the window of normal traffic (after a back-off is triggered)}, a single row \ous{0}{can be activated} at most \ous{4}{\param{$A_{normal} = \SI{180}{\nano\second}$/$\trc{}$}}\ouscomment{4}{Anormal calculation is defined here} times before the first preventive refresh is issued.
Therefore, the total number of activations \om{4}{to} a row $i$ before the \ous{0}{memory controller issues the} first preventive refresh is $A(i) = N_{BO} + A_{normal}$ \ous{0}{(because $N_{BO}$th activation triggers the back-off)}.
Third, during the recovery period, \X{} preventively refreshes the potential victims and resets the activation counts of all rows $i$ where \param{$A(i) \geq N_{BO}$} (\om{6}{following from} \ous{5}{property $P1$} and \ous{5}{property $P3$}).\omcomment{6}{Sounds odd. By P3 meaning?}\ouscomment{6}{I was just referring to the properties that ensure the sentence. For example, \X{} tracks all activation correctly (P1) and refreshes any row reaching the back-off threshold during the recovery period (P3). Therefore, each back-off refreshes all rows $i$ where \param{$A(i) \geq N_{BO}$}.}\omcomment{6}{Is this all described?}\ouscomment{6}{I think properties themselves are clear. We could be more verbose within the text but it would mostly be redundant to what the properties say.}
Fourth, after the recovery period, a row \ous{0}{has} at most $N_{BO} - 1$ activations \ous{0}{(which is the same as step one)}.

These four phases show that \X{} performs preventive refreshes in a way that \param{$A(i) \leq N_{BO} + A_{normal}$} for all $i$ at all times.
Therefore, \X{} is secure against read disturbance bitflips \ous{0}{in configurations where} \param{$N_{BO} < \nrh{} - A_{normal}$}.

\head{Determining a Secure Aggressor Tracking Table Size}
An attacker can try to overwhelm the \emph{\om{4}{Aggressor} Tracking Table} (ATT, see \secref{sec:briefsummary}) by forcing many rows to reach $N_{BO}$ activations.
Our proof shows that an attacker can perform at most $A_{normal}$ additional activations after triggering a back-off.
The attacker can maximize the number of rows that require a refresh in three steps.
First, the attacker activates $A_{normal} + 1$ rows $N_{BO} - 1$ times each \emph{without} triggering a back-off.
Second, the attacker activates one of the rows for the $N_{BO}$th time, which triggers a back-off.
Third, the attacker activates the other $A_{normal}$ rows that did \emph{not} trigger the back-off for the $N_{BO}$th time during the window of normal traffic.
By doing so, an attacker can force \emph{at most} $A_{normal} + 1$ rows to reach $N_{BO}$ activations.
Therefore, ATT should be able to hold at least $A_{normal} + 1$ entries \ous{4}{(e.g., $\lfloor{} \SI{180}{\nano\second}/\trc{} \rfloor{} + 1 = 4$ entries for a $\trc{}$ of \SI{47}{\nano\second})}.\omcomment{4}{How is this calculation done?}\ouscomment{4}{changed to match the Anormal definition above}