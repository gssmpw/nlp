\section{Related Work}
\label{sec:relatedwork}

This is the first work that
1) rigorously analyzes the security and performance and
2) solves the major problems of \gls{prac}, a key feature introduced in the latest JEDEC DDR5 DRAM specification~\cite{jedec2024jesd795c} \om{5}{to prevent read disturbance bitflips~\cite{kim2014flipping, luo2023rowpress}}.
\om{4}{An earlier version of this paper}~\cite{canpolat2024understanding} analyzes the security and performance benefits of \gls{prac}, \om{5}{but~\cite{canpolat2024understanding}} does \emph{not} propose or evaluate a mechanism that solves \gls{prac}'s major weaknesses.

We propose \X{}, which addresses the \om{5}{major} weaknesses of \gls{prac} \om{5}{by} 1) \om{5}{reducing the latency of its counter maintenance operations} and 2) preventing adversarial access patterns.
\ous{5}{\secref{sec:evaluation} and \secref{sec:evaluation_dos}} \om{4}{already} qualitatively and quantitatively compare \X{} to \gls{prac} and \ous{4}{three prominent read disturbance} mitigation mechanisms: \ous{4}{Graphene~\cite{park2020graphene}, PARA~\cite{kim2014flipping}, and Hydra~\cite{qureshi2022hydra}.
We demonstrate that \X{} outperforms these evaluated mechanisms for \om{5}{modern} and future DRAM chips.
In this section, we discuss other read disturbance mitigation techniques}.

\noindent
\textbf{Per-Row Activation Tracking.}\omcomment{4}{Start with the decade discussion, then Panopticon, and finally Hydra.}\ouscomment{4}{Acknowledged. Reordered.}
\om{4}{Per}-row activation counters were already discussed close to a decade ago in the original RowHammer paper~\cite{kim2014flipping} and other works~\cite{kim2014architectural,yaglikci2021security,bains2016distributed,bains2016row}.
Building on these \om{5}{works}, our paper implements these counters \emph{without} inducing high performance and hardware complexity overheads to modern and future DRAM chips.
Panopticon~\cite{bennett2021panopticon} proposes implementing separate per-row counters in a \emph{counter mat} and using the \texttt{alert\_n} signal to request time for preventive refreshes.
Panopticon does \emph{not} discuss performance, energy, or hardware complexity overhead.
On the other hand, \X{} replicates an existing DRAM bank structure (i.e., a subarray) and quantifies the performance, energy, and hardware complexity overheads.
Hydra~\cite{qureshi2022hydra}, a memory controller-based mechanism, proposes storing per-row tracking entries in DRAM and caching the entries in the memory controller.
Hydra does \emph{not} perfectly track per-row activations and does \emph{not} propose a way to update in-DRAM counters \ous{6}{concurrently while} serving memory requests.
\om{5}{In \secref{sec:evaluation}, we show that \X{} greatly outperforms Hydra.}\omcomment{5}{Does Hydra need this much discussion?}\ouscomment{5}{Chopped Hydra a bit}

\noindent
\textbf{Other \om{4}{On-DRAM}-\ous{4}{Die} Mitigation Techniques.}
DRAM manufacturers implement read disturbance mitigation techniques, also known as Target Row Refresh (TRR)~\cite{jedec2020jesd795,jedec2017ddr4,hassan2021utrr,frigo2020trrespass}, in commercial DRAM chips.
The specific designs of these techniques are not openly disclosed.
Recent research shows that custom attacks can bypass these mechanisms~\cite{frigo2020trrespass, hassan2021utrr, jattke2022blacksmith, deridder2021smash, van2016drammer, saroiu2022price} and cause read disturbance bitflips.

\noindent
\textbf{Hardware-based Mitigation Techniques.}
Prior works propose hardware-based mitigation techniques~\citeHardwareBasedMitigations{} to prevent read disturbance bitflips.
\om{5}{Some} of these works~\cite{kim2014flipping, you2019mrloc, son2017making, wang2021discreet,yaglikci2022hira,saroiu2022configure,kim2021hammerfilter} \om{5}{propose} probabilistic preventive refresh mechanisms to mitigate read disturbance at low area cost.
\om{5}{These} mechanisms do not provide deterministic read disturbance prevention and thus \om{5}{cause} high system performance overhead as \gls{nrh} decreases \om{5}{(as they need to generate many more preventive refreshes)}.
\om{5}{Three} prior works~\cite{joardar2022learning, joardar2022machine, naseredini2022alarm} propose machine-learning-based mechanisms.
\ous{6}{These mechanisms are \emph{not} fully secure \ous{7}{because they have} increasing bitflip probability \om{8}{with lower} read disturbance thresholds~\cite{joardar2022learning, joardar2022machine} or \om{7}{they require} error-correction codes to correct a small number of bitflips~\cite{naseredini2022alarm}}.
Another group of prior works~\cite{seyedzadeh2017cbt, seyedzadeh2018cbt, kang2020cattwo, lee2019twice, saileshwar2022randomized, saxena2022aqua, kim2022mithril, marazzi2022protrr, park2020graphene, woo2022scalable, olgun2024abacus} propose using the Misra-Gries frequent item counting algorithm~\cite{misra1982finding}.
\ous{4}{Misra-Gries-based mechanisms use} a large number of counters implemented with \ous{5}{content-addressable memory} for low \gls{nrh} values~\cite{olgun2024abacus, bostanci2024comet}, thereby inducing high \om{5}{hardware} area overheads \om{5}{(as we showed for Graphene~\cite{park2020graphene} in \secref{sec:evaluation})}.
In contrast, \X{} provides \om{6}{\emph{deterministic}} security guarantees \om{5}{at} low \gls{nrh} values \om{5}{with} \om{6}{\emph{low system performance and hardware overheads}}.

\noindent
\textbf{Software-based Mitigation Techniques.}
Several software-based read disturbance mitigation techniques~\cite{konoth2018zebram, van2018guardion, brasser2017can, bock2019riprh, aweke2016anvil, zhang2022softtrr, enomoto2022efficient} propose to avoid hardware-level modifications. However, these works cannot monitor \textit{all} memory requests and thus, {many} of them are shown to be defeated by recent attacks~\cite{qiao2016new, gruss2016rowhammer, gruss2018another, cojocar2019eccploit, zhang2019telehammer, kwong2020rambleed, zhang2020pthammer}.

\noindent
\textbf{Integrity-based Mitigation Techniques.}
Another set of mitigation techniques~\integrityBasedMitigationsAllCitations{} implements integrity check mechanisms that identify and correct potential bitflips.
However, it is either impossible or too costly to address all read disturbance bitflips with these mechanisms.

%\atb{6}{There \om{7}{are various other mitigation} mechanisms that can be implemented in the memory controller~\mcBasedRowHammerMitigations{} or in the DRAM chip~\inDRAMRowHammerMitigations{}. We leave \om{7}{a} rigorous \om{8}{comparison} of \gls{prac} to this broader set of RowHammer mitigation \om{7}{techniques to} future work.}}}

% This section \atb{6}{briefly describes many other mitigation mechanisms in \param{three}\atbcomment{5}{FILL ME} categories.
% We leave rigorous comparison of \gls{prac} to this broader set of RowHammer mitigation mechanisms for future work.}}}
% \iffalse
% \ous{0}{\head{Perfect RowHammer Tracking}
% Read disturbance vulnerability of modern DRAM chips gets worse as manufacturing technology scales to smaller feature sizes~\cite{mutlu2017rowhammer, mutlu2019rowhammer, frigo2020trrespass, cojocar2020rowhammer, kim2020revisiting, kim2014flipping, luo2023rowpress}.
% The performance and area overheads of read disturbance mitigation mechanisms significantly increase as they need to track row activations and perform preventive actions more aggressively~\cite{olgun2024abacus, bostanci2024comet, canpolat2024leveraging}.
% Prior works discuss the use of per-row activation counters to detect how many times each row in DRAM is activated within a refresh interval~\cite{kim2014flipping,kim2014architectural,bennett2021panopticon,kim2023ddr5,yaglikci2021security} to perfectly track aggressor rows and reduce performance overheads by performing preventive actions \emph{only} when necessary, i.e., when a row activation counter gets close to \gls{nrh}.}

% \copied{BreakHammer}{\ous{0}{\head{On-die RowHammer Mitigations}
% DRAM manufacturers implement RowHammer mitigation techniques, also known as Target Row Refresh (TRR), in commercial DRAM chips~\cite{jedec2020jesd795,jedec2017ddr4}, but they do not openly disclose the specific designs.
% Recent research shows that custom attack patterns can bypass these mechanisms~\cite{frigo2020trrespass, hassan2021utrr, jattke2022blacksmith, deridder2021smash, van2016drammer, saroiu2022price}.
% With worsening read disturbance vulnerability, these mechanisms need to perform an increasing number of preventive refresh operations.
% To provide these mechanisms with the necessary time window, new DRAM communication protocols (e.g., DDR5~\cite{jedec2024jesd795c}) require the memory controller to issue refresh management (RFM) commands.}}

% \copied{BreakHammer}{\ous{0}{\head{Hardware-based RowHammer Mitigations}
% Prior works propose a set of hardware-based mitigations~\mitigatingRowHammerAllCitations{} to prevent RowHammer bitflips.
% A subset of these works~\cite{kim2014flipping, you2019mrloc, son2017making, wang2021discreet, yaglikci2022hira, saroiu2022configure} proposes probabilistic preventive refresh mechanisms to mitigate RowHammer at low area cost.
% In a similar vein, three prior works~\cite{joardar2022learning,joardar2022machine,naseredini2022alarm} propose machine-learning-based mechanisms that are not fully secure for all \gls{nrh} values.
% Two prior works~\cite{kim2015architectural, qureshi2022hydra} propose implementing per-row counters to accurately track row activations.
% Another group of prior works~\cite{seyedzadeh2017cbt, seyedzadeh2018cbt, kang2020cattwo, lee2019twice, saileshwar2022randomized, saxena2022aqua, kim2022mithril, marazzi2022protrr, park2020graphene, woo2022scalable} propose using frequent item counting algorithm Misra-Gries~\cite{misra1982finding}.
% Another set of prior works~\cite{yaglikci2021blockhammer, kim2014flipping, greenfield2012throttling} proposes throttling mechanisms that delay memory requests to prevent RowHammer bitflips.}}
% \fi

% There is a large body of research on mitigating DRAM read disturbance on-DRAM-die~\inDRAMRowHammerMitigations{} and from within the memory controller~\mcBasedRowHammerMitigations{}.
% As DRAM chips become more vulnerable to read disturbance, these prior works suffer from the limitations of DRAM interface more.

% To aid this problem, a recent DDR5 protocol update~\cite{saroiu2024ddr5, jedec2024jesd795c} introduces a new on-DRAM-die mechanism\gf{5}{,} \gls{prac}\gf{5}{,} without a rigorous evaluation of this mechanism. 
% This work investigates \gls{prac}'s different implementation possibilities, and qualitatively evaluates their impacts in terms of security, performance, energy, and cost.

% This section discusses the most relevant prior works. 

% \head{In-DRAM read disturbance mitigation mechanisms}
% \inDRAMRowHammerMitigations{}
% \outline{fill in}

% \head{Memory controller-based read disturbance mitigation mechanisms}
% \mcBasedRowHammerMitigations{}
% \outline{fill in}