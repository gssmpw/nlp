\section{Future Research Directions}
\label{sec:implications}
% Giray: We can also discuss how ABACUS can be integrated with Chronus to make the area overhead even lower.

Although \X{} addresses major weaknesses of \gls{prac}, \om{5}{we believe there is still significant room for improvement in read disturbance solutions (including \X{} and \gls{prac})}.
\ous{5}{In this section, we discuss \param{five} directions to improve read disturbance solutions}.
% we still identify \ous{4}{\param{five}}\ouscomment{4}{Added SMD as a new direction. Any reordering suggestions between the directions?} directions to improve \X{} and \gls{prac}.

A first direction is to determine read disturbance thresholds more accurately.
Read disturbance mitigation mechanisms provide security under the assumption that a safe \gls{nrh} value is known.
However, determining a safe \gls{nrh} value to cover all rows is not easy due to the need to identify the lowest \gls{nrh} in the presence of \gls{nrh} variation across \om{7}{data patterns~\cite{kim2014flipping, kim2020revisiting, luo2023rowpress, yaglikci2022understanding, olgun2025variable}, temperature~\cite{kim2020revisiting, orosa2022spyhammer}, access patterns~\cite{luo2023rowpress, orosa2021deeper, kogler2022half, luo2024experimental}, voltage~\cite{yaglikci2022understanding}, physical row locations~\cite{orosa2021deeper, yaglikci2024spatial}, and time~\cite{olgun2025variable}}, as shown by multiple works~\cite{kim2014flipping, orosa2021deeper, luo2023rowpress, kim2020revisiting, saroiu2022configure, olgun2023understanding, zhou2023threshold, olgun2025variable, tugrul2025understanding, luo2024experimental, orosa2022spyhammer, yaglikci2022understanding}.
A second direction to explore is to implement \X{} and \gls{prac} counters more efficiently.
Future work can improve the efficiency with
$i$) counter update policies against different read disturbance attacks (e.g., RowPress~\cite{luo2023rowpress})\footnote{One way to mitigate RowPress is to configure RowHammer solutions at lower thresholds (e.g., $<$500)~\cite{luo2023rowpress}. We already show that \X{} outperforms state-of-the-art \om{5}{RowPress} solutions~\cite{park2020graphene,qureshi2022hydra,kim2014flipping,jedec2024jesd795c} at such low thresholds.} and
$ii$) different architectures that improve area and DRAM energy overheads of both \X{} and \gls{prac}.
A third direction is to leverage the significant variation in read disturbance vulnerability across rows to avoid overprotecting the vast majority of the rows~\cite{yaglikci2024spatial,orosa2021deeper}.
For example, Svärd~\cite{yaglikci2024spatial} enhances existing RowHammer solutions to become spatial RowHammer threshold aware.
\X{} could similarly be combined with Svärd to improve system performance by extending the counter subarray to store the necessary meta-data about the \ous{4}{read disturbance vulnerability}\omcomment{4}{Defined?}\ouscomment{4}{revised, the original paper uses "vulnerability bin"} (e.g., very low, low, average, high) of each row.
\X{} already has relatively low performance overheads (e.g., $<$0.1\%) \emph{without} spatial variation awareness even at very low RowHammer thresholds (e.g., $<$100) and we expect \X{}’ performance overheads would \om{5}{further reduce} with spatial variation awareness.
A fourth direction is to defend against malicious attackers that exploit preventive refreshes.
Attackers can trigger increasing amounts of preventive refreshes as \gls{nrh} decreases, allowing a new attack vector to conduct memory performance attacks~\cite{mutlu2007memory}.
Preventing these performance attacks may be possible by accurately detecting and throttling workloads that trigger many preventive refreshes~\cite{yaglikci2021blockhammer, canpolat2024breakhammer}.
\ous{4}{A fifth direction is to explore a more flexible DRAM interface design and protocol \om{6}{(as in~\cite{hassan2024self})}.
Improving the flexibility of the interface and intelligently dividing the work between the memory controller and DRAM by separating \om{5}{the responsibility of handling/controlling} access operations (e.g., reading and writing data) and maintenance operations (e.g., DRAM refresh and read disturbance mitigation, \om{5}{like \X{}}) can significantly improve system performance and energy, \om{6}{as shown by Self-Managing DRAM}~\cite{hassan2024self}}.