\section{A Brief Summary of RFM and PRAC}
\label{sec:briefsummary}

This section briefly explains the \gls{rfm} command, \gls{prac} mechanism, and assumptions we use for our evaluations.

\head{RFM Command}
\gls{rfm} is a DRAM command that provides the DRAM chip with a time window (e.g.,~\param{\SI{195}{\nano\second}}~\cite{jedec2024jesd795c}) so that the DRAM chip \om{3}{can} preventively refresh potential victim rows.
The DRAM chip is responsible for identifying and preventively refreshing potential victim rows, and the memory controller is responsible for issuing \gls{rfm} commands.

\head{\gls{prac} Overview}
\gls{prac}~\cite{jedec2024jesd795c}, \om{3}{as described in the JEDEC DDR5 standard} \om{4}{updated in April 2024}~\cite{jedec2024jesd795c}, implements an activation counter for each DRAM row, and thus accurately measures the activation counts of \emph{all} rows.
When a row's activation count reaches a threshold, the DRAM chip asserts a back-off signal which forces the memory controller to issue an RFM command.
The DRAM chip preventively refreshes potential victim rows upon receiving an RFM command. 

\head{Assumptions About \gls{prac}}
We make two assumptions:
1)~\gls{prac} always refreshes victims of the row with the maximum activation count during each \gls{rfm} command\ouscomment{4}{Some PRAC implementations (Moin's MOAT) only refresh rows that exceed a threshold. Doing so accelerates the wave attack.}\omcomment{4}{Do they show the negative effect on the wave attack?}\ouscomment{4}{I triple-checked all analyses (both our's and moin's) and MOAT's approach does not affect the wave attack (because wave attack starts from $N_{BO}$ activations, which is always higher than "MOAT-style" refresh threshold). We only need to assume that PRAC refreshes the row with max activations. If we still have the ``why is this a good assumption'' question on the currently remaining part: I am not sure if we can mathematically analyze the security if we do not know if PRAC correctly refreshes victims of a row with maximal activation count.}
and 2)~physically-adjacent DRAM rows can experience bitflips when a DRAM row is activated \ous{0}{at least \gls{nrh} times}.
% (even if the activation count is \emph{not} close to \gls{nrh})\footnote{The specification does \emph{not} enforce refreshing the victims of the aggressor with the maximum activation count~\cite{jedec2024jesd795c}. We assume a preventive refresh is performed with each \gls{rfm} to help reduce counter values as the \ous{3}{memory controller already does \emph{not} access the bank for the command's duration ($\trfm{}$)}.}

\head{\gls{prac}'s Operation and Parameters}
\gls{prac} increments a row's activation count \ous{3}{during a precharge command}.
\ous{2}{When a bank receives a precharge command, the bank internally reads, modifies, and writes the open row's counter before de-asserting the wordline.
As such, \gls{prac}'s counter update} affects several DRAM timing parameters \ous{2}{around row accesses}.
Table~\ref{tab:practiming}\ouscomment{4}{TODO: fix table overextending to the right} summarizes DRAM timing parameter changes \om{4}{when \gls{prac} is enabled} for the DRAM 3200AN~\cite{jedec2024jesd795c} speed bin.

\begin{table}[ht]
    \centering
    \footnotesize
    \vspace{1em}
    \caption{\gls{prac}'s DRAM Timing Parameter Changes \om{4}{with \gls{prac}}}
    \begin{tabular}{l|l||@{\hspace{2pt}}c@{\hspace{3pt}}c}
    \makecell[b]{Parameter} & \makecell[bl]{Description} & \makecell[b]{\om{3}{DDR5} \\ \om{3}{without \gls{prac}}} & \makecell[b]{\om{4}{DDR5} \\ \om{4}{with \gls{prac}}} \\
    \hline
    \hline
    $t_{RAS}$ & \makecell[l]{minimum time for a $PRE$ \\ after an $ACT$ to the same bank} & \SI{32}{\nano\second} & \SI{16}{\nano\second} \\ 
    \hline
    $t_{RP}$ & \makecell[l]{minimum time for an $ACT$ \\ after a $PRE$ to the same bank} & \SI{15}{\nano\second} & \SI{36}{\nano\second} \\ 
    \hline
    $t_{RC}$ & \makecell[l]{minimum time for two $ACT$s \\ to the same bank} & \SI{47}{\nano\second} & \SI{52}{\nano\second} \\ 
    \hline
    $t_{RTP}$ & \makecell[l]{minimum time for a $PRE$ \\ after a $RD$ to the same bank} & \SI{7.5}{\nano\second} & \SI{5}{\nano\second} \\ 
    \hline
    $t_{WR}$ & \makecell[l]{minimum time for a $PRE$ \\ after a $WR$ to the same bank} & \SI{30}{\nano\second} & \SI{10}{\nano\second} \\ 
    \hline
    \end{tabular}
    \label{tab:practiming}
\end{table}

\ous{3}{We make two observations from Table~\ref{tab:practiming}.
First, the $\trp{}$ and $\trc{}$ timing parameters increase because of the additional time needed to update the counter before the wordline is de-asserted.
Second, the $\tras{}$, $\trtp{}$, and $\twr{}$ timing parameters likely reduce to account for the additional time the row stays open as the counter is updated.
We note that \gls{prac} timing parameters can also be better utilizing the aggressive guardbands existing in modern DRAM chips as shown in~\cite{lee2015adaptive, chang2016understanding, chang2017understanding, chang2017understandingphd, kim2018solar, yaglikci2022understanding, mathew2017using, tugrul2025understanding} to reduce one or more of the \om{4}{latter three} timing parameters.}
% \omcomment{3}{Why do these parameters reduce? Can there be another mechanism at play?}\ouscomment{3}{We cant know for certain. Added a sentence to note the possibility.}

% Because of this delay, 1) $\trp$ increases by \param{\SI{21}{\nano\second}} (\param{+140\%}) and 2) $\tras{}$, $\trtp{}$, and $\twr{}$ reduces by \param{\SI{16}{\nano\second}} (\param{-50\%}), \param{\SI{2.5}{\nano\second}} (\param{-33\%}), and \param{\SI{20}{\nano\second}} (\param{-67\%})~\cite{jedec2024jesd795c}.
% \agycomment{0}{I removed the part about tRTP amd tWR, as they are not defined.}
% \agycomment{0}{is this 67\%?}\ouscomment{0}{30 to 10, so yes rounding to 67 is better}
% Combined effect of these timing parameters result in a $\trc$ increase of \param{\SI{5}{\nano\second}} (\param{+10\%}) for DDR5-3200AN speed bin~\cite{jedec2024jesd795c}.}

The DRAM chip asserts the back-off signal when a row's activation count reaches a fraction of \gls{nrh}, denoted as \gls{aboth}, where the fraction can be configured to either 70\%, 80\%, 90\%, or 100\%~\cite{jedec2024jesd795c}.
The memory controller receives the back-off signal shortly after (e.g., \param{$\approx$\SI{5}{\nano\second}}~\cite{jedec2024jesd795c}) issuing a \gls{pre} command.
% between the time after a command that closes rows (e.g., precharge or refresh) is issued and a small latency after the same command's completion (e.g., \param{$\approx$\SI{5}{\nano\second}}~\cite{jedec2024jesd795c}).
The memory controller and the DRAM chip go through three phases when the back-off signal is asserted.
First, during \gls{taboact}~\cite{jedec2024jesd795c}, the memory controller has a limited time window (e.g., \param{\SI{180}{\nano\second}}~\cite{jedec2024jesd795c}) to serve requests after receiving the back-off signal.
A DRAM row can receive up to \gls{taboact}/\gls{trc} activations in this window.
Second, during the \emph{recovery period}~\cite{jedec2024jesd795c}, the memory controller issues a number of \gls{rfm} commands, which we denote as $\bonrefs{}$ (e.g., 1, 2 or 4~\cite{jedec2024jesd795c}).
An \gls{rfm} command can further increment the activation count of a row before its potential victims are refreshed.
Third, during the \emph{delay period} or \gls{tbodelay}~\cite{jedec2024jesd795c}, the DRAM chip \emph{cannot} reassert the back-off signal until it receives a number of activate commands, which we denote as $\bonacts{}$  (e.g., 1, 2 or 4~\cite{jedec2024jesd795c}).\footnote{Current DDR5 specification~\cite{jedec2024jesd795c} notes that $\bonrefs$ and $\bonacts$ always have the same value. To comprehensively assess PRAC's security guarantees, we also consider different values in our security analysis (\secref{sec:configurationandsecurity}).}
\om{3}{Taking into account} these three phases, \secref{sec:configurationandsecurity} calculates the highest achievable activation count to any row in a \gls{prac}-protected system.

\head{Determining Rows to Refresh During an RFM Command}
\ous{3}{\gls{prac} maintains a large number of counters in each DRAM bank}.
As such, it is \emph{not} \ous{4}{practical} for the DRAM bank to search for the row with the maximum activation count during an RFM command.
To track the rows with the highest activation counts, our \gls{prac} implementation uses a relatively small \ous{2}{per-bank} table \ous{2}{(e.g., 4 \om{3}{entries} \ous{3}{store enough rows for each \gls{rfm} command to refresh during the recovery period}) that} we call \om{3}{the} \emph{Aggressor Tracking Table} (ATT).
The table starts empty (i.e., all entries are invalid).
When a row is precharged, the table is updated \ous{3}{with the precharged row's counter value} if
1) the precharged row exists in the table,
2) an entry in the table is invalid, or
3) the precharged row's counter value exceeds the table entry with the \emph{lowest} activation count.
When \ous{2}{a DRAM bank receives} an RFM command, the bank invalidates and refreshes the potential victims of the table entry with the \emph{maximum} activation count.

\head{\gls{rfm} and \gls{prac} Implementations}
We analyze \param{three} different \gls{rfm} and \gls{prac} implementations:
% of in-DRAM read disturbance mitigation mechanisms: 
1) \emph{\gls{prfm}}, where the memory controller issues an \gls{rfm} command \emph{periodically} when the total number of activations to a bank reaches a predefined threshold called \ous{3}{\gls{rfmth}}\ouscomment{3}{defined $\rfmth{}$ here} with \emph{no} back-off signal from the DRAM chip, as described in early DDR5 standards~\cite{jedec2020jesd795};
2) \emph{\gls{prac}-N}, where the memory controller issues N back-to-back \gls{rfm} commands \emph{only} after receiving a back-off signal from the DRAM chip, as described in the latest JEDEC DDR5 standard~\cite{saroiu2024ddr5, jedec2024jesd795c};
3) \emph{\gls{prac}+\gls{prfm}}, where the memory controller issues an \gls{rfm} command when $i$) the total number of activations to a bank reaches \om{3}{$\rfmth{} = 75$} or $ii$) it receives a back-off signal from the DRAM chip.\footnote{The \gls{rfmth} of 75 \om{4}{is} provided \om{4}{in} the example \gls{prac}+\gls{prfm} configuration in the latest (as of April 2024) JEDEC DDR5 standard~\cite{jedec2024jesd795c}.}\omcomment{3}{Is RFMth defined in the footnote?}
\om{3}{As shown in \secref{sec:configurationandsecurity}}, \gls{prac}-N implementations are \emph{not} secure at \gls{nrh} values lower than \param{20}.
Therefore, combining \gls{prac} and \gls{prfm} enables security at lower \gls{nrh} values at the cost of potentially refreshing the victims of aggressor rows whose activation counts are \emph{not} close to \gls{nrh}.