\section{Conclusion}
\label{sec:conclusion}

We presented the first rigorous security, performance, energy, and cost analyses of \gls{prac} and proposed a new mechanism, \X{}, which addresses \gls{prac}'s two major weaknesses.
Our analyses show that \gls{prac} increases the critical DRAM access latency parameters due to the additional time required to increment activation counters and performs a \ous{4}{fixed} number of preventive refreshes at a time, making it vulnerable to an adversarial access pattern, known as the \emph{the wave attack} or \emph{the feinting attack}.
These two weaknesses of \gls{prac} cause significant performance \om{5}{overheads} at current and future DRAM chips.
Our mechanism, \X{}, \om{6}{solves} \gls{prac}'s two \om{5}{major} problems \om{6}{by} \ous{5}{1)} updating row activation counters concurrently \ous{5}{while} serving accesses by physically separating counters from the data and
\ous{5}{2)} dynamically controlling the number of preventive refreshes performed.
Our evaluation shows that \X{} outperforms \ous{5}{\gls{prac} and \gls{prfm}, the \om{6}{state-of-the-art} industry solutions to read disturbance}, and \param{three} other state-of-the-art \om{5}{read disturbance mitigation} proposals in terms of \om{6}{both} system performance and DRAM energy, \om{6}{especially for future DRAM chips with higher read disturbance vulnerability}.
\om{6}{We believe \X{} provides a robust and efficient solution to read disturbance \ous{6}{for current and future DRAM chips at low area, performance, and energy costs}.
We hope that future research continues to improve read disturbance solutions with new ideas, interfaces, and even more efficient techniques.}

% We conclude that more research is needed to improve \gls{prac} by
% $i$) reducing the high system performance and DRAM energy overheads due to increased DRAM timing parameters,
% $ii$) solving the exacerbated performance impact as \gls{nrh} decreases, and
% $iii$) stopping its preventive refreshes from being exploited by memory performance attacks.

% \agy{0}{In this work, we present the first rigorous security, performance, energy, and cost analyses of
% the% state-of-the-art 
% on-DRAM-die read disturbance mitigation method, widely known as \gls{prac}\gf{5}{,} with respect to its description in the updated \gf{5}{JEDEC} DDR5 specifications.}
% % and performance analyses of \agy{0}{a key feature of state-of-the-art DRAM standards:} \gls{prac}\agy{0}{, which provides the DRAM chip with the necessary time window to perform preventive refreshes when necessary. \gls{prac} is important for practical and low-overhead read disturbance mitigation for modern and future DRAM chips.}
% % To do this analysis, we 
% % 1)~formally define an adversarial access pattern which represents the worst-case for \gls{prac}, 
% % 2)~investigate \gls{prac}'s different configurations and their security guarantees, and 
% % 3)~evaluate \gls{prac}'s performance, energy, and cost overheads.
% We show that \gls{prac}
% 1)~can provide robust operation for a hammer threshold as low as \param{10} activations per aggressor row and 
% 2)~incurs less than \param{22.3\%} and an average of \param{17.4\%} performance overhead across \param{60} randomly chosen benign workload mixes for modern and future DRAM chips.