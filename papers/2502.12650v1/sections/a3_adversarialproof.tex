\section{Worst-Case Access Pattern Analysis}
\label{apx:adversarialproof}

We prove that the access pattern used in \secref{sec:evaluation_dos} yields the maximum theoretical DRAM bandwidth consumption \ous{10}{of preventive refreshes} in a \X{}- or \gls{prac}-protected system.

\head{Properties}
We build our proof on \param{three} properties of \gls{prac} and \X{}:
$P1$)~a DRAM row's activation count increases \om{10}{by one} when the row is opened and closed;
$P2$)~a back-off is triggered when a row's activation count reaches \gls{aboth};
$P3$)~a back-off refreshes $i$)~$\bonrefs{}$ rows with the highest activation count in \gls{prac} and $ii$)~all rows that exceed \gls{aboth} in \X{} for each bank.

% Flow outline of the proof, everything is a placeholder except the proof's logical flow
\head{Maximum Bandwidth Consumption \ous{10}{of Preventive Refreshes}}
\agy{9}{For a given access pattern, we calculate the fraction of DRAM bandwidth \ous{9}{consumed} for performing preventive refreshes.
Our \ous{9}{\emph{DRAM Bandwidth Consumption} (\dbc{})} function is formally defined as $\dbc{}: P \rightarrow [0, 1]$, where $P$ denotes the set of all possible access patterns and $[0,1]$ is the fraction of DRAM bandwidth consumed by preventive refreshes.
For example, the worst-case access pattern evaluated in \secref{sec:evaluation_dos} triggers a preventive refresh operation $\bonrefs{}$ times and each refresh operation takes $\trfm{}$ (i.e., $\bonrefs{}\times\trfm{}$).
% \agycomment{9}{Is this true? Double check}\ouscomment{9}{yes, it is true}
To trigger a refresh, this access pattern includes \gls{aboth} row activations, each of which takes at least $\trc{}$ (\ous{9}{i.e.}, $\aboth{}\times{}\trc{}$)}.
As such, the adversarial pattern evaluated in \secref{sec:evaluation_dos} (i.e., $\advanal{}$) has the DRAM bandwidth consumption presented in Expression~\ref{eqn:dosanal}.
\begin{equation}
\label{eqn:dosanal}
\dbc{}(\advanal{})
=
\frac{
(\bonrefs{}\times{}\trfm{})
}{
(\bonrefs{}\times{}\trfm{})+(\aboth{}\times{}\trc{})
}
\end{equation}

\head{Proof Overview}
\ous{9}{We prove that $\advanal{}$ yields the maximum DRAM bandwidth consumption in \param{four} steps \agy{9}{using proof-by-contradiction}~\cite{rosen2019discrete}.
% \agycomment{9}{Check BlockHammer and cite if we cited there or if you find a good textbook}\ouscomment{9}{BlockHammer doesn't cite it but added a nice textbook.}
First, we assume that there exists an adversarial pattern $\advhyp{}$ (i.e., hypothetical adversarial pattern) that yields a greater DRAM bandwidth consumption than $\advanal{}$ within an arbitrary time window $T$.\footnote{We choose an arbitrary $T$ with no assumptions. Therefore, the following steps of the proof apply for all $T$ by the \emph{Universal Introduction} rule~\cite{enderton2001mathematical}.}
Second, we calculate the number of back-offs triggered by $\advanal{}$ within $T$ as a lower bound for $\advhyp{}$.
Third, we calculate the time needed to perform the preventive refreshes of back-offs caused by $\advhyp{}$.
Fourth, we show that time remaining after $\advhyp{}$'s preventive refreshes \emph{cannot} be used to trigger $\advhyp{}$'s back-offs within $T$.
Therefore, a pattern that yields greater DRAM bandwidth availability \emph{cannot} exist within the constraints of \gls{prac} and \X{}}.
% \agycomment{9}{Better to go as first, second, third, fourth after this point instead of Proof.}
% \ouscomment{9}{does the following "First" appear out of no where now? or is it fine?}

\head{Step 1: Adversarial memory access pattern}
% \ous{9}{First}, let's assume 
Assume there exists an adversarial access pattern $\advhyp{}$ that yields a greater DRAM bandwidth consumption than $\advanal{}$ within an arbitrary time window $T$ (i.e., $\dbc{}(\advhyp{}) > \dbc{}(\advanal{})$).
Since $\advhyp{}$ consumes more DRAM bandwidth than $\advanal{}$, $\advhyp{}$ must trigger at least one more back-off than $\advanal{}$.
\ous{9}{For a given access pattern $P$, we calculate the \emph{Back-Offs Triggered} (\anbot{}) within a time window of length $T$.
To do so, we find the DRAM bandwidth consumed by access pattern $P$ (i.e., $T \times{} \dbc{}(P)$) and divide it by the duration of a back-off (i.e., $\bonrefs{}\times{}\trfm{}$) as shown in Expression~\ref{eqn:botdef}}.
% \agycomment{9}{P seems like a function or a multiplier in the expression. Add an (i.e.,}
% \ouscomment{9}{added}
\begin{equation}
\label{eqn:botdef}
\begin{aligned}
\anbot{}(P, T) = T \times{} \dbc{}(P) / (\bonrefs{}\times{}\trfm{})
\end{aligned}
\end{equation}

\head{Step 2: Lower bound of the number of back-offs}
\agy{9}{Based on the assumption in Step~1, $\anbot{}(\advhyp{}, T)$ should be larger than $\anbot{}(\advanal{}, T)$.}
% \ous{9}{Given Expression~\ref{eqn:botdef}, we calculate a lower bound for the number of back-offs triggered by $\advhyp{}$ in Expression~\ref{eqn:hypbotlowerbound}}.
% \agycomment{9}{You call all these equations. However, some of them (e.g., \ref{eqn:hypbotlowerbound}) are not equations. They are inequalities. Instead, you can call all of them expressions.}\agycomment{9}{It would be simpler to make $\geq$ $>$ and drop $+1$.}\ouscomment{9}{Acknowledged. Changed to expression and dropped the +1 everywhere}
\agy{9}{By placing Expressions~\ref{eqn:dosanal} and~\ref{eqn:botdef} in the restriction $\anbot{}(\advhyp{}, T) > \anbot{}(\advanal{}, T)$, we \om{10}{derive} Expression~\ref{eqn:hypbotlowerbound}.}
\begin{equation}
\label{eqn:hypbotlowerbound}
\begin{aligned}
\anbot{}(\advhyp{}, T) > \frac{ T }{ \bonrefs{}\times{}\trfm{}+\aboth{}\times{}\trc{} }
\end{aligned}
\end{equation} 

% \vspace{-4pt}
% \begin{equation}
% \label{eqn:hypbotlowerbound}
% \begin{aligned}
% \anbot{}(\advhyp{}, T) & \geq \anbot{}(\advanal{}, T) + 1 \\
% & \geq \frac{ T \times{} \dbc{}(\advanal{}) }{ \bonrefs{}\times{}\trfm{} } + 1 \\
% & \geq \frac{ T }{ \bonrefs{}\times{}\trfm{}+\aboth{}\times{}\trc{} } + 1
% \end{aligned}
% \end{equation}

\head{Step 3: Time taken to perform preventive refreshes}
\ous{9}{By knowing the lower bound for the number of back-offs triggered by $\advhyp{}$, we calculate} the time taken to perform the preventive refreshes of $\advhyp{}$'s back-offs (i.e., $\textit{PR}_{HYP}$) by multiplying the number of back-offs (i.e., $\anbot{}(\advhyp{}, T)$) with the duration of a back-off (i.e., $\bonrefs{}\times{}\trfm{}$) \ous{9}{in Expression~\ref{eqn:timetorefresh}}.
\begin{equation}
\label{eqn:timetorefresh}
\begin{aligned}
\textit{PR}_{HYP} & > \frac{T\times{}\bonrefs{}\times{}\trfm{}}{\bonrefs{}\times{}\trfm{} + \aboth{}\times{}\trc{}}
\end{aligned}
\end{equation}
% \[\geq \frac{T\times{}(\bonrefs{}\times{}\trfm{} +\aboth{}\times{}\trc{}) - T\times{}\aboth{}\times{}\trc{}}{\bonrefs{}\times{}\trfm{} + \aboth{}\times{}\trc{}} + \bonrefs{}\times{}\trfm{} \]
% \[\geq T - \frac{T\times{}\aboth{}\times{}\trc{}}{\bonrefs{}\times{}\trfm{} + \aboth{}\times{}\trc{}} + \bonrefs{}\times{}\trfm{} \]

\head{Step 4: Remaining time after the preventive refreshes}
\ous{9}{The time taken to both trigger and perform the preventive refreshes of $\advhyp{}$ should fit within $T$}.
We \agy{9}{calculate} the remaining time after performing the preventive refreshes of $\advhyp{}$'s back-offs (i.e., $\textit{RT}_{HYP}$).
\ous{9}{Expression~\ref{eqn:timetorefresh} provides a lower bound for $\textit{PR}_{HYP}$.
Therefore, subtracting $\textit{PR}_{HYP}$ from $T$ yields an upper bound for $\textit{RT}_{HYP}$ \agy{9}{(i.e., $\textit{RT}_{HYP} < T - \textit{PR}_{HYP}$).
Solving this expression for $\textit{RT}_{HYP}$, we} \ous{10}{derive} Expression~\ref{eqn:remainingtime}}.\agycomment{9}{Exp. 6 gives a lower bound for PRHYP, but we need the value of PRHYP in Exp 7.}
\begin{equation}
\label{eqn:remainingtime}
\begin{aligned}
\textit{RT}_{HYP}<\frac{T\times{}\aboth{}\times{}\trc{}}{\bonrefs{}\times{}\trfm{} + \aboth{}\times{}\trc{}}
\end{aligned}
\end{equation}


% \vspace{-8pt}
% \begin{equation}
% \label{eqn:remainingtime}
% \begin{aligned}
% \textit{RT}_{HYP} & < T - \textit{PR}_{HYP} \\
% & < T - \frac{T\times{}\bonrefs{}\times{}\trfm{}}{\bonrefs{}\times{}\trfm{} + \aboth{}\times{}\trc{}} \\
% & < \frac{T\times{}\aboth{}\times{}\trc{}}{\bonrefs{}\times{}\trfm{} + \aboth{}\times{}\trc{}}
% \end{aligned}
% \end{equation}

Triggering a back-off with a single bank takes $\aboth{}\times{}\trc{}$ (following from $P1$ and $P2$) and concurrent aggressors \emph{cannot} be used to trigger a back-off more quickly (following from $P3$).
Given $P1$, $P2$, and $P3$, \ous{9}{we calculate} time needed to trigger $\advhyp{}$'s back-offs in a \gls{prac} or \X{} protected system ($\textit{TBO}_{HYP}$).
\ous{9}{Expression~\ref{eqn:hypbotlowerbound} presents a lower bound for $\anbot{}(\advhyp{}, T)$.
Multiplying $\anbot{}(\advhyp{}, T)$ with the time taken to trigger a single back-off yields a lower bound for $\textit{TBO}_{HYP}$ (i.e.,~$\textit{TBO}_{HYP}>\anbot{}(\advhyp{}, T)\times{}\aboth{}\times{}\trc{}$)}.
Solving this expression for $\textit{TBO}_{HYP}$, we evaluate Expression~\ref{eqn:timetotrigger}.
\begin{equation}
\label{eqn:timetotrigger}
\begin{aligned}
\textit{TBO}_{HYP} & > \frac{ T\times{}\aboth{}\times{}\trc{} }{ \bonrefs{}\times{}\trfm{}+\aboth{}\times{}\trc{} }
\end{aligned}
\end{equation}

\ous{9}{By comparing Expressions~\ref{eqn:remainingtime} and~\ref{eqn:timetotrigger}}, we see that the time necessary to trigger $\advhyp{}$'s preventive refreshes ($\textit{TBO}_{HYP}$) exceeds the time remaining after $\advhyp{}$'s preventive refreshes are performed ($\textit{RT}_{HYP}$), i.e, $\textit{RT}_{HYP} < \textit{TBO}_{HYP}$.
This means that \emph{no} $\advhyp{}$ exists that obey the three properties of \gls{prac} and \X{} (i.e., $P1$, $P2$, and $P3$).
Therefore, $\advanal{}$ \agy{9}{(as used in \secref{sec:evaluation_dos})} yields the maximum DRAM bandwidth consumption, \agy{9}{and thus represents the worst possible access pattern}.