\section{Adversarial Access Pattern: The Wave Attack}
\label{sec:adversarial}

\ous{6}{We describe} our threat model and the adversarial access pattern, known as \emph{the wave attack}~\cite{yaglikci2021security, devaux2021method} or \emph{the feinting attack}~\cite{marazzi2022protrr}.

\head{Threat Model}
We assume that the attacker
1)~knows the physical layout of DRAM rows (as in~\cite{yaglikci2021blockhammer}),
2)~accurately detects when a row is internally refreshed (preventively or periodically, as in U-TRR~\cite{hassan2021utrr}), and
3)~precisely times \emph{all} DRAM commands except \gls{ref} and \gls{rfm} commands (as in~\cite{yaglikci2021blockhammer, hassan2021utrr}). 

\head{Overview}
The wave attack aims to achieve the highest activation count for a given row in an \gls{rfm} and \gls{prac}-protected DRAM chip by overwhelming the mitigation mechanism using a number of decoy rows.
% \agy{0}{\footnote{\agy{0}{This adversarial access pattern is called \emph{the wave attack}~\cite{yaglikci2021security, devaux2021method} or \emph{the feinting attack}~\cite{marazzi2022protrr} in the literature.}}} 
% similar to the \emph{wave attack}~\cite{yaglikci2021security, devaux2021method} and \emph{feinting attack}~\cite{marazzi2022protrr}.
\figref{fig:waveattack} visualizes the buildup of a wave attack against a periodic read disturbance mitigation \nb{0}{mechanism} (e.g., \gls{prfm}) that refreshes the potential victims of an aggressor for every three row activations.\omcomment{4}{Cannot see the figure.}\ouscomment{4}{Replaced with a PNG. Is it fixed?}
% \vspace{-1em}

\begin{figure}[h]
\centering
\includegraphics[width=\linewidth]{figures/waveattack.png}
\caption{Wave attack buildup visualization}
\label{fig:waveattack}
\end{figure}

% \vspace{-1em}
In this access pattern, the attacker hammers several rows in a balanced manner, such that the mitigation mechanism can perform preventive refreshes \emph{only} for a small subset of the hammered rows when a preventive refresh is issued.
When an aggressor row's victims are refreshed, the attacker excludes the aggressor row in the next round of activations.
By doing so, this adversarial access pattern achieves the highest possible activation count for the row whose victims are preventively refreshed last.
\ous{3}{More details and analyses of the attack can be found in~\cite{yaglikci2021security, devaux2021method, marazzi2022protrr}}.