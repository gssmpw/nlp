\appendix
\section{Decrementer Circuit of \X{}}
\label{apx:decrementer}

\ous{4}{
\X{} uses custom circuitry to update row activation counters.
To do so, we implement a circuit that decrements an 8-bit number (i.e., size of a \X{} activation counter, as explained in~\secref{sec:mechanism}) by 1.
\tabref{tab:decrementer} shows the pseudo hardware description of the decrementer and the resources to implement the circuit.
In the table, $x$ and $y$ respectively show the 8-bit row activation counter as input and 8-bit updated value as output of the circuit, where $y=x-1$.
The subscripted numbers denote the bit index of the input and the output, e.g., $x_1$ and $y_1$ respectively show the first bit of the 8-bit input and 8-bit output of the circuit.
Rows of the table show
1) the logical expression to obtain each bit of $y$,
2) the logic gates needed to implement the expression, and
3) the number of transistors needed to implement the logic gates.}

%
\begin{table}[ht]
    \centering
    \footnotesize
    \vspace{1em}
    \caption{Gate-level implementation of the circuitry that decrements an 8-bit number by 1}
    \begin{tabular}{l@{\hspace{2pt}}|@{\hspace{2pt}}ccccr}
    Logical expression & NOT & MUX & NAND & NOR & \#Ts\\
    \hline
    $y_{0}$ = $\overline{x_{0}}$                                               & 1 & 0 & 0 & 0 &  2 \\
    $y_{1}$ = \ous{0}{${x_{0}}$} ? $x_{1}$ : $\overline{x_{1}}$                & 1 & 1 & 0 & 0 & 10 \\
    $y_{2}$ = nor($x_{0}$, $x_{1}$) ? $\overline{x_{2}}$ : $x_{2}$             & 1 & 1 & 0 & 1 & 14 \\
    for $i = 3\rightarrow7$: & & & & & \\
    \hspace*{0.2cm} $y_{i}$ = nand($y_{i-1}$, $\overline{x_{i-1}}$) ? $x_{i}$ : $\overline{x_{i}}$ & 1 & 1 & 1 & 0 & 14 \\
    \hline
    \multicolumn{1}{r}{Total:} & 8 & 7 & 5 & 1 & 96 \\

    \end{tabular}
    \label{tab:decrementer}
\end{table}