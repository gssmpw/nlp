\section{Related Work}
Classical work in off-road navigation ____ has focused on creating the costmap 
of the environment from sensors' data to represent navigation cost associated
with the various types of visual and geometric features of the environment. 
Earlier approaches ____
relied on feature engineering while later approaches
____ relied on deep learning based 
semantic segmentation to represent the visual and geometric features of the surrounding terrain. These later approaches 
train the semantic segmentation pipeline from scratch and use hand designed cost functions for the planner. 
Although, these costmaps provide rich information for the downstream planning tasks, 
%subsequent planning algorithms,
tuning them to capture the complex dynamic interactions while navigating on various terrain types is extremely challenging and requires significant domain expertise.  


Recent advances in Deep learning 
have inspired researchers in robotics community to develop end-to-end learning algorithms____ that 
directly learn the mapping from sensor information to control commands thereby bypassing 
the need for manual costmap creation and tuning. Despite the promises of these end-to-end approaches, they still require a large amount of data and show poor generalisation to different settings (domain adaptation). Moreover, the black box nature of these approaches make them very hard to debug and deploy on real systems.

To deal with these challenges, recent research has focused on learning-based algorithms that combine the strengths of classical and more recent end to end approaches. These efforts aim towards an acceptable trade off between domain expertise, explain-ability and training data requirements. 

% To deal with these challenges, In the recent past, the researchers in the off-road navigation community has proposed 
% learning based algorithms that aim to combine the attractive features of both the classical approaches and more recent 
% end-to-end methods while trying to find an acceptable tradeoff between the domain expertise, explainability and the
% amount of data required to train these algorithms. 
Recent works like ____
leverage human driving data to either directly learn the costmap of the environment or learn the traversability of various terrain types. 
% These costmaps are then subsequently used to solve an optimal control problem 
% to generate vehicle controls. Although these methods capture
% the tire terrain interactions of the vehicle, they do not consider the high level reasoning (negotiating between various terrain types) that is required  
% in order to successfully navigate in challenging off-road environments. Moreover, these work solely relay on real world 
% data to learn the costmap/traversability while in our work we show that how a simulation platform can be used to 
% learn a local planning algorithm that can successfully negotiate between various terrain types in a real world offroad 
% setting.
These costmaps are crucial to solve the optimal control problem and generate appropriate vehicle controls. While these methods effectively model the tire-terrain interactions of the vehicle, they often overlook the higher level reasoning required to navigate in challenging off-road environments. Additionally, these methods primarily depend on real-world data to learn costmaps and traversability. In contrast, our research demonstrates the potential of simulation platforms to learn a local planning algorithm. 

%This algorithm is capable of successfully navigating in real off-road settings, while choosing a path by negotiating between various terrain types showing the advantage of simulation based learning from human driving data for complex navigation tasks.
%\textcolor{red}{Negotiating between various terrain type is not clear. Choosing the most suitable terrain similar to learned human %behaviours based on the given goal..maybe something like this}
%