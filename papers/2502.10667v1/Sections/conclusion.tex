% \section{Conclusions and Future Work}
% \sj{update this}
% In this paper, we propose DQuaG, a novel approach for data quality validation. 
% DQuaG offers significant advantages in automated data quality validation. 
% By leveraging GNNs and VAEs, DQuaG eliminates the need for expert-defined constraints and manual intervention, making the process accessible to non-expert useres, as well as more efficient and scalable. 
% It excels not only in detecting ordinary data quality issues but also in identifying hidden errors that existing methods often miss. 
% Additionally, DQuaG can pinpoint the exact features contributing to data quality problems, providing a granular understanding of the issues. 


% Our future work will focus on extending our approach to encompass post-validation tasks, such as data cleaning and data selection. 
% We also plan to enhance the interpretability of our models by optimizing the GNN and VAE frameworks. 


\section{Conclusions and Future Work}
We proposed DQuaG, a novel multi-task learning framework that combines a GNN encoder with dual decoders for data quality validation and repair. 
Our approach effectively detects and repairs data quality issues without expert-defined constraints or manual interventions. By utilizing a dual-decoder structure, our model independently optimizes data quality validation and repair tasks, with a specially designed validation decoder loss that enhances the detection of erroneous data. 
We evaluated our approach using real-world datasets containing inherent data quality issues, where it performed well. However, in extreme scenarios—such as datasets with only a few thousand samples where only a single erroneous instance exists—our method may lack sensitivity. In such cases, traditional statistical methods often prove to be more accurate, as they can better identify isolated anomalies within smaller datasets. 
We aim to further refine our approach, improving its robustness in extreme conditions.
Additionally, we plan to extend our approach to encompass post-validation tasks, such as data cleaning and data selection. 
To improve the interpretability of our models, we will focus on optimizing the GNN frameworks.
%We will also extend our experimental analysis to develop a full paper.




%This involves developing techniques to better understand the relationships and dependencies identified by the models, thereby providing more actionable insights for improving data quality.



% \section{Future Work}

% Future work will focus on extending our approach to encompass post-validation tasks, such as data cleaning and data selection. We also plan to enhance the interpretability of our models by optimizing the GNN and VAE frameworks. This will involve developing techniques to better understand the relationships and dependencies identified by the models, thereby providing more actionable insights for improving data quality.

% \section{Conclusions}

% Our proposed method offers significant advantages in automated data quality verification. By leveraging GNNs and VAEs, our approach eliminates the need for expert-defined constraints and manual intervention, making the process more efficient and scalable. It excels not only in detecting conventional data quality issues but also in identifying hidden errors that traditional methods often miss. Additionally, our method can pinpoint the exact features contributing to data quality problems, providing a granular understanding of the issues. This comprehensive capability enhances the accuracy and reliability of data-driven decision-making.
