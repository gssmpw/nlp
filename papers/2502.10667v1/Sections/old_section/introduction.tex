\section{Introduction}
The emergence of Big Data has marked a transformative phase in the digital age, presenting both novel opportunities and intricate challenges. Among these challenges, ensuring data quality is paramount, especially as data proliferates at an unprecedented rate in terms of volume, variety, and velocity. High-quality data is essential for reliable data-driven decision-making, particularly in machine learning applications where the integrity of training and test datasets directly impacts model performance. This paper explores the critical issue of data quality verification, focusing on innovative approaches that address the limitations of traditional methods.

\noindent\textbf{Background:} 
 In the context of Big Data, quality-aware systems have become indispensable tools. These systems aim to ensure that data adheres to specific quality benchmarks, regardless of its scale or diversity. Traditional quality-aware systems typically begin with data profiling, a process that generates metadata to describe various quality attributes such as missing values, data dependency rules, and other essential facets. This metadata forms the foundation for assessing data quality against established metrics and dimensions. However, traditional methods exhibit several significant limitations. They often rely heavily on expert-defined constraints, which are resource-intensive and time-consuming to develop and maintain. These constraints require substantial domain knowledge and manual effort for continuous monitoring and adjustment, making the process inefficient and prone to human error.

Moreover, some approaches attempt to automate constraint generation by learning from the dataset’s characteristics and data types. Despite these advancements, automatically set constraints are typically incomplete and may be either too stringent or too lenient. Strict constraints can lead to numerous false positives, flagging acceptable data as problematic, while lenient constraints may miss critical quality issues, allowing poor-quality data to pass through. Consequently, even automated systems require significant expert intervention to fine-tune and validate the constraints, limiting their practical effectiveness.

Additionally, traditional methods often fail to detect hidden relationships and dependencies within the data. Data quality is not only about individual data points but also about understanding the complex interrelationships among them. Traditional systems lack the capability to automatically recognize these latent connections, which can lead to the oversight of subtle yet crucial data quality issues. Experts, too, may overlook these hidden relationships, further compromising the data quality verification process.
% Quality-aware systems have risen to prominence as indispensable tools in the Big Data context. Their primary objective is to ensure that data, regardless of its scale or diversity, adheres to specific quality benchmarks. The vastness and complexity of Big Data amplify the need for such systems, as they serve as the first line of defense against potential quality discrepancies and anomalies that might otherwise go unnoticed in such a massive data landscape.

% A cornerstone of Quality-aware systems is their capability to model and measure data quality meticulously. This process is initiated with data profiling, a sophisticated technique that generates metadata. This metadata is not just a simple descriptor; it's a treasure trove of essential quality attributes, capturing nuances like missing values, data dependency rules, and other quality-related facets. Once armed with this rich metadata, the system is poised to conduct a thorough assessment, evaluating the data against a suite of established quality metrics and dimensions.

% Despite the advances in data quality verification, traditional quality-aware systems exhibit several significant limitations. Firstly, these methods often involve calculating various data quality dimensions and metrics or rely heavily on expert-defined constraints. This reliance on experts can be resource-intensive and time-consuming, as it demands substantial domain knowledge and manual effort to set appropriate constraints. Experts must continuously monitor and adjust these constraints to maintain data quality, which is not always feasible.

% Moreover, some approaches attempt to automate the constraint-setting process by learning from the dataset’s characteristics and data types. However, these automatically set constraints are typically not comprehensive enough. They either fail to cover all necessary aspects of data quality or set constraints that are too strict or too lenient. Strict constraints can lead to an excessive number of false positives, flagging acceptable data as problematic, while lenient constraints may miss critical quality issues, allowing poor-quality data to pass through. Consequently, these automated systems still require significant expert intervention to fine-tune and validate the constraints, limiting their practical effectiveness.

% Additionally, these traditional methods often fail to detect hidden relationships and dependencies within the data. Data quality is not only about individual data points but also about understanding the complex interrelationships among them. Traditional systems lack the capability to automatically recognize these latent connections, leading to potential oversight of subtle yet crucial data quality issues. Experts might also overlook these hidden relationships, further compromising the data quality verification process.

\noindent\textbf{Motivation:} The need for robust data quality verification has never been more critical in the era of Big Data. Traditional methods, while foundational, exhibit significant limitations that impede their effectiveness. These methods often depend heavily on expert-defined constraints, which can be incomplete, overly strict, or too lenient, requiring substantial domain knowledge and manual effort for continuous monitoring and adjustment. This dependence on experts not only makes the process resource-intensive and time-consuming but also introduces the risk of human error and oversight.

Moreover, automated approaches that attempt to learn constraints from the data itself often fall short. These automatically generated constraints frequently lack comprehensiveness, failing to cover all necessary aspects of data quality. They might also be miscalibrated, either too stringent, leading to numerous false positives, or too permissive, allowing poor-quality data to slip through. Consequently, these automated systems still necessitate significant expert intervention to fine-tune and validate the constraints, limiting their practical effectiveness.

A particularly challenging issue is the detection of hidden relationships and dependencies within the data. Traditional systems and even many automated approaches are not equipped to recognize these latent connections, which are crucial for identifying subtle yet significant data quality issues. Experts, too, may overlook these hidden relationships, further compromising the verification process.

These limitations underscore the urgent need for more advanced, automated methods that can provide comprehensive and accurate data quality verification without heavy reliance on expert input. Such methods must be capable of automatically detecting and leveraging hidden data relationships and dependencies, adapting to the diverse and growing scale of Big Data environments. This motivates the development of a novel approach that can address these critical challenges and enhance the overall efficacy and efficiency of data quality verification.

\noindent\textbf{Challenge:} Developing an advanced data quality verification system that addresses the limitations of traditional methods involves several significant challenges:
\textbf{Constraint-Free Identification:} Traditional methods rely on predefined constraints, which can be inaccurate. Our approach needs to automatically identify problematic data without such constraints, providing a more flexible and accurate solution.
\textbf{Hidden Relationship Detection:} Identifying latent relationships and dependencies within data is critical. Our method must effectively uncover and leverage these hidden patterns to ensure comprehensive data quality assessment.
\textbf{Reducing Expert Intervention:} Traditional systems require ongoing expert input for constraint management. Our solution aims to minimize this dependency, making the process more efficient and scalable.
\textbf{Scalability and Robustness:} The solution must handle large and complex datasets efficiently, maintaining performance and accuracy as data volume grows.

These challenges require a novel approach that combines advanced techniques for automatic data quality assessment, hidden pattern recognition, and scalability in Big Data environments.



\noindent\textbf{Our Solution:} In response to these challenges, we propose a novel approach for tabular data quality verification:
\begin{enumerate}[leftmargin=*]
    \item \textbf{Knowledge Graph Generation:} Our method starts by automatically generating a knowledge graph from the original dataset. This graph captures intricate relationships and dependencies within the data that traditional methods might overlook.
    \item \textbf{Embedding Knowledge Graph:} The generated knowledge graph is then embedded with the tabular data. This embedding forms a comprehensive representation that integrates both the original data and its relational information.
    \item \textbf{Training the VAE:} We train a Variational Autoencoder (VAE) using the embedded data. The VAE learns to differentiate between problematic and non-problematic datasets by modeling the normal data distribution.
    \item \textbf{Error Identification:} By integrating the knowledge graph into the VAE framework, our approach can effectively identify specific samples that are likely to contain errors. This process enhances the detection of hidden data relationships and reduces the need for expert intervention.
\end{enumerate}
This novel approach addresses the key challenges by combining advanced techniques for automatic data quality assessment, hidden pattern recognition, and scalability in Big Data environments.


\noindent\textbf{Contributions:} The contributions of this paper are as follows:
\begin{enumerate}[leftmargin=*]
    \item We propose a novel approach that integrates automatically generated knowledge graphs with tabular data for enhanced data quality verification.
    \item We develop a VAE-based model that utilizes this combined representation to accurately identify and locate problematic data samples.
    \item We demonstrate that our method significantly reduces the reliance on expert-defined constraints and enhances the detection of hidden relationships within the data.
    \item We validate our approach through extensive experiments, showcasing its effectiveness and scalability in real-world Big Data scenarios.
\end{enumerate}

In the following sections, we delve deeper into the details of our approach, beginning with a comprehensive review of related work, followed by the methodology, experimental results, and conclusion.

