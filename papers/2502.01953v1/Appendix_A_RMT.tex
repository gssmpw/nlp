\newpage
\section{Random matrix theory: the asymptotics of the Hessian}
\label{sec:RMT}
The goal of this section is to study the asymptotics of the spectrum of the Hessian $\bH$
originally defined in Eq.~\eqref{eq:bH_def} of Section~\ref{sec:pf_thm1}, and recalled bellow for convenience.
The section will culminate in the proof of Proposition~\ref{prop:uniform_convergence_lipschitz_test_functions} of Section~\ref{sec:pf_thm1}.

Recall the definitions
\begin{equation*}
\bH(\bTheta,\bbV;\bw) = \bH_0(\bbV; \bw) + n \grad^2 \rho(\bTheta), \quad\quad
    \bH_0(\bbV;\bw) = \left(\bI_k \otimes \bX\right)^\sT \bSec(\bbV;\bw)\left(\bI_k \otimes \bX\right)
\end{equation*}
where
\begin{equation*}
   \bSec(\bbV;\bw) = \begin{pmatrix}
\bSec_{i,j}(\bbV;\bw)
   \end{pmatrix}_{i,j \in[k]}
,\quad
    \bSec_{i,j}(\bbV;\bw)= \Diag\left\{\left(\frac{\partial^2}{\partial {v_i}\partial v_j}\ell(\bbV;\bw)\right)\right\}\in\R^{n\times n}, \quad i,j\in[k].
\end{equation*}
%uniformly for $(\bbV,\bTheta)\in\cM$.
%We state the main result of this section in the following proposition. 
%Let us make some definitions before giving the statement.
%
%Fix $z\in\bbH^+$. For $\nu\in\cuP(\R^{k+k_0+1}),\bS\in\C^{k\times k}$, define 
%\begin{equation}
%\label{eq:fp_eq}
%    \bF_z(\bS;\nu) := \big(\E_{(\bv,\bv_0,w)\sim\nu}\big[(\bI + \grad^2_\bv\ell(\bv,\bv_0,w)\bS)^{-1} \grad^2_\bv\ell(\bv,\bv_0,w)\big]  -z\bI\big)^{-1}.
%\end{equation}
%Let $\bS_\star(z;\nu)$ be the unique solution of the fixed point equation
%\begin{equation}
%    \bF_z(\bS_\star;\nu) = \alpha^{-1} \bS_\star.
%\end{equation}
%Define $\mu_{\MP}(\nu)$ as the measure whose Stieltjes transform is given by 
%\begin{equation}
%   s_{\MP}(z;\nu) := \frac1k \Tr(\bS_\star(z;\nu)).
%\end{equation}
%For $\mu\subseteq\cuP(\R^{k+k_0})$,
%let
%\begin{equation}
%\mu_\rho(\mu) := \bigcup_{j=1}^k \spec(\rho_{0\# \mu_{\{j\} }}).
%\end{equation}
%(Observe that $\mu_\rho(\mu)$ only depends on the marginal of the first $k$ coordinates, but we write it as such to simplify notation). 
%Finally, define
%\begin{equation}
%    \mu_\star(\mu,\nu) := \mu_{\MP}(\nu) \boxplus \mu_{\rho}(\mu).
%\end{equation}
%
%\bns{Write final result}
%\begin{proposition}
%\label{prop:uniform_convergence_lipschitz_test_functions}
%Under Assumption~\ref{ass:regime},~\ref{ass:loss},~\ref{ass:regularizer} and~\ref{ass:sets}, we have
%  for any Lipschitz function $f:\R\to\R$,
%  \begin{equation}
%  \lim_{\substack{n\to\infty\\n/d \to \alpha}}
%  \sup_{(\bbV,\bTheta)\in\cM(\cuA,\cuB)}\left|\frac1{dk}\E\left[\Tr \,f\left(\frac1n\bH(\bbV) + \grad^2\rho(\bTheta) \right)\right]
%      - \int f(\lambda) \mu_\star(\hnu_\bbV,\hmu_\bTheta)(\de \lambda)
%      \right| = 0.
%  \end{equation}
%\end{proposition}

As always, we'll often find it convenient to suppress the dependence on $\bbV$ in the notation. For example, we write $\bH$ for the matrix 
$\bH(\bbV)$.
Additionally, we will sometimes write $\bH(\hnu)$ when we want to emphasize the dependence of $\bH$ on $\hnu = \hnu_{\bbV}$.

The main object of analysis will be the empirical matrix-valued Stieltjes transform defined for $z\in \bbH^+$ and $\hnu \in\cuP_n(\R^{k+k_0+1})$ by
\begin{equation}
\label{eq:Sn_def}
\bS_n(z;\hnu) :=  (\bI_k \otimes \Tr)(\bH(\hnu) - z n \bI_{dk})^{-1}.
\end{equation}
%The goal of this section is to prove the following proposition.
%\begin{proposition}
%For $z \in \mathbb{H}_+$, let $\bS_\star \in\bbH_+^k$ be the unique solution to
%\begin{equation}
%\label{eq:ST_FP}
%\bS = \frac1{\alpha_n}\bF_z(\bS).
%\end{equation}
%Then \begin{equation}
%    \norm{\bS_n - \bS_\star}_{2} \le \dots
%\end{equation}
%with probability $\dots$.
%\end{proposition}

%\bns{I'll rewrite the following paragraph.}
%In Section~\notate{ref} below, we first show that $\bS_n$ satisfies the fixed point~\eqref{eq:ST_FP} approximately. In Section~\notate{ref}, we prove uniquness of this solution on $\bbH_+^k$. Then finally in Section~\notate{ref}, we show the claimed convergence $\bS_n$ to the solution $\bS_\star$.
%We defer technical lemmas to Section~\notate{ref}.

\subsection{Preliminary results}
This subsection summarizes some preliminary results useful for proofs of this section. 
\subsubsection{Some general properties of  \texorpdfstring{${(\id \otimes \Tr)},\Re$ and $\Im$}{(I x Tr),Re and Im}}


Let us present some properties of the operators $\Re,\Im$ and $\id \otimes \Tr$ (or $\bI\otimes \Tr)$ that will be useful for this section. Some proofs are deferred to~\ref{section:RMT_appendix_technical_results}.
\begin{lemma}[Properties of $\Re$ and $\Im$.]
\label{lemma:re_im_properties}
Let $\bZ \in\bbH^+_k$. Then,
\begin{enumerate}
\item 
$\bZ$ is invertible,
\begin{equation*}
    \Im(\bZ^{-1}) = - \bZ^{-1} \Im(\bZ) \bZ^{*-1}\prec\bzero, \quad\quad
\|\bZ^{-1}\|_\op \le \|\Im(\bZ)^{-1}\|_\op, \quad\textrm{and}\quad
\norm{\Im(\bZ)}_\op \le \norm{\bZ}_\op.
\end{equation*}

\item For any $\bW$ self-adjoint, we have
\begin{align*}
    \Im((\bI+\bW \bZ)^{-1}\bW ) &= -((\bI + \bW \bZ)^{-1}\bW)\Im(\bZ)((\bI + \bW \bZ)^{-1}\bW)^*,
\end{align*}
and
\begin{align*}
   \norm{(\bI + \bW\bZ)^{-1}\bW}_\op &\le \norm{\Im(\bZ)^{-1}}_\op.
\end{align*}
\end{enumerate}
\end{lemma}
The proof of the lemma above is deferred to Section~\ref{sec:proof_lemma_re_im_properties}.
\begin{lemma}[Properties of $(\bI\otimes\Tr)$]
\label{lemma:tensor_trace_norm_bounds}
\label{lemma:tensor_trace_properties}
Let $\bM \in\C^{dk\times dk}$. Then the following hold.
\begin{enumerate}[(1.)]
    \item We have the bounds
\begin{align*}
    \norm{(\bI_k \otimes \Tr)\bM}_{\Fnorm} \le  \sqrt{d}\norm{\bM}_\Fnorm
    \quad\textrm{and}\quad
    \norm{(\bI_k \otimes \Tr)\bM}_\op \le  d\norm{\bM}_\op.
\end{align*}
\item If $\bM^*  =\bM  \succ  \bzero$, then $(\bI_k \otimes \Tr)\bM \succ\bzero.$
The same statement holds if we replace both strict relations 
$(\succ)$ with non-strict ones $(\succeq)$.
\item If $\bM^* = \bM \succeq \bzero$, then
\begin{equation*}
    \lambda_{\min}\left( (\bI_k \otimes \Tr)\bM \right) \ge d \lambda_{\min}(\bM).
\end{equation*}
\item We have
\begin{equation*}
   \Im\left(
   (\bI_k \otimes \Tr) \bM
   \right) =  (\bI_k \otimes \Tr) \Im(\bM).
\end{equation*}
\end{enumerate}
\end{lemma}
The proof is deferred to Section~\ref{sec:proof_lemma_tensor_trace_properties}.

\subsubsection{Definitions, relevant norm bounds, and algebraic identities}
We give some definitions that will be used throughout this section. For $j\in [n]$, let
\begin{equation*}
    \bW_j := \grad^2 \ell(\bv_j, \bu_j, w_j) \in \R^{k\times k},
\quad\bxi_j := (\bI_k \otimes \bx_j)\in\R^{dk\times k}.
%\widetilde\bxi_i =  \bxi_i\bG_i \in \R^{dk\times k}.
\end{equation*}
Let $\sfK := \sup_{\bv,\bv_0,w} \norm{\grad^2 \ell(\bv,\bv_0,w)}_{\op}$.
With this notation,  we can write 
   $\bH =  \sum_{j=1}^n \bxi_j \bW_j \bxi_j^{\sT}.$
For $i\in[n],$ let $\bH_i := \sum_{j\neq i} \bxi_j \bW_j \bxi_j^\sT$,
and define the (normalized) resolvent and the leave-one-out resolvent as
\begin{equation*}
\bR(z) := \left(\bH - z n \bI_{dk}\right)^{-1}\quad
\textrm{and}\quad
\bR_i(z) := \left(\bH_i  - z n \bI_{dk}\right)^{-1},
\end{equation*}
respectively.
We present the following algebraic identities that will be used in the leave-one-out approach we follow in Section~\ref{sec:approx_ST_FP}. The proof of these is deferred to Section~\ref{sec:proof_lemma_algebra_lemma}.
\begin{lemma}[Woodbury and algebraic identities]
\label{lemma:algebra_lemma}
    For all $i\in[n]$ and $z\in\bbH_+$, we have
    \begin{equation}
\label{eq:alg_id1}
        \left(\bI_k \otimes \Tr\right)\bxi_i \bW_i \bxi_i^\sT \bR(z)  =  \left( \bI_k + \bW_i \bxi_i^\sT \bR_i(z) \bxi_i\right)^{-1}
        \bW_i\bxi_i^\sT \bR_i(z)\bxi_i,
    \end{equation}
and
\begin{equation}
\label{eq:alg_id2}
    \left(\bI_k \otimes \Tr\right) \left(\bR_i(z)-  \bR(z)\right)  =  \bxi_i^\sT \bR_i(z) (\bW_i \otimes \bI_d) \bR(z)\bxi_i.
\end{equation}

%   and
%   \begin{equation}
%\label{eq:alg_id2}
%      (\bI \otimes \Tr)\left(\bM_i^{-1} \bz_i (\bI_k + \widetilde \bz_i^\sT \bM_i^{-1}\bz_i)^{-1}\widetilde \bz_i^{\sT}\bM_i^{-1} \right)= 
%     \bz_i^\sT \bM_i^{-1} (\grad^2 \rho_i \otimes \bI_k) \bM^{-1} \bz_i.
%   \end{equation}
\end{lemma}

 The next lemma summarizes some \emph{a priori} bounds on the matrices
involved in the upcoming proofs.
The proof is deferred to Section~\ref{sec:proof_lemma_as_norm_bounds}.
%
\begin{lemma}[Deterministic norm bounds]
\label{lemma:as_norm_bounds}
For all $i\in[n]$ and $z\in\bbH_+$, we have, 
\begin{align}
\label{eq:det_norm_bound_lemma_eq123}
     \norm{\bR(z)}_\Fnorm^2 \vee \norm{\bR_i(z)}_\Fnorm^2 \le \frac{dk}{n^2} \frac{1}{\Im(z)^2},
     \quad
  \norm{\bR(z)}_\op \vee\norm{\bR_i(z)}_\op  \le  \frac{1}{n} \frac1{\Im(z)},
  \quad
  \norm{\bSec}_\op \le \sfK,
   \quad
   \norm{\bH}_\op \le \sfK \norm{\bX}_\op^2.
\end{align}
Further, for $z \in\bbH_+$, we have
\begin{equation}
\label{eq:det_norm_bound_lemma_eq4}
    \|\Im((\bI_k\otimes\Tr)\bR(z))^{-1}\|_\op 
    %\le  \frac1{\Im(z)}\frac{1}{\sigma_{\min}((\bH - z\bI)^{-1})^2} 
    \le   \frac{1}{\Im(z)} \left(\frac{1}{n}\norm{\bH}_\op  + |z|\right)^2
\end{equation}
and
\begin{equation}
\label{eq:det_norm_bound_lemma_eq5}
    \|\Im(\bxi_i^\sT\bR_i(z)\bxi_i)^{-1}\|_\op \le \frac{n}{\norm{\bx_i}_2^2} \frac{1}{\Im(z)} \left(\frac{1}{n}\norm{\bH_i}_\op  + |z|\right)^2.
\end{equation}
%\begin{align}
%    \norm{\widetilde \bz_i^\sT \bM_i^{-1} \bz_i} = O(1), 
%\norm{\grad^2 \rho_i (\bI\otimes \Tr) \bM_i^{-1}} = O(1)\\
%    \norm{(\bI+\widetilde \bz_i^\sT \bM_i^{-1} \bz_i)^{-1}} = O(1), 
%\norm{(\bI+\grad^2 \rho_i (\bI\otimes \Tr) \bM_i^{-1})^{-1}} = O(1).\\
%\end{align}
\end{lemma}
%

We also recall the following textbook fact 
for future reference.
\begin{lemma}[Operator norm bounds for Gaussian matrices
\cite{BaiSilverstein}]
\label{lemma:standard_norm_bounds}
 Let
 $$\Omega_0 := \{\norm{\bX}_\op \le 2 d^{1/2}(1 + \sqrt{\alpha_n})),\; \norm{\bx_i}_2 \in [d^{1/2}/{2}, 2d^{1/2}] \quad\textrm{for all}\quad i\in[n]\}.$$
Then 
\begin{equation}
    \P(\Omega_0^c ) \le C \exp\{- c d\}
\end{equation}
for some universal $C,c>0$.
\end{lemma}

\subsubsection{Concentration of tensor quadratic forms}

Finally, we end this section with the following consequence of the Hanson-Wright inequality for the concentration of quadratic forms of (sub)Gaussian random variables.
%\begin{lemma}
%\label{lemma:hanson-wright}
%Let $\bx \sim \cN(0,\bI_d), \bxi := (\bI_k \otimes \bx)$.
%   Let $\bM \in\C^{dk\times {dk}}$ be independent of $\bx$. Then for any $t>0$, 
%\begin{equation}
%    \P\left(\norm{\bxi^\sT \bM  \bxi -  (\bI_k \otimes \Tr)\bM}_F \ge t \right) \le
%    2 k^2\exp\left\{ -\frac{c t^2 }{k^2 \norm{\bM}_F^2 + k t \norm{\bM}_\op}  \right\}
%\end{equation}
%where $c>0$ is some universal constant.
%\end{lemma}
%
%\begin{proof}
%The statement follows from an element-wise application of Hanson-Wright. Namely, let $\bM_{j,l}\in\C^{d\times d}$  for $j,l\in[k]$ denote the blocks of the matrix $\bM$, 
%and let 
%$F^2_{jl}:=  \norm{\bM_{j,l}}_F^2$
%and
%$O_{jl}:=  \norm{\bM_{j,l}}_\op$. Then Hanson-Wright~\notate{cite} gives, for any $t>0$ and some fixed universal constant $c>0$,
%\begin{align}
%\P\left( \left|\bx^\sT \bM_{l,j} \bx - \Tr(\bM_{l,j}) \right| \ge t\right) 
%\le 2 \exp\left\{ - \frac{c \, t^2}{
%F_{jl}^2+
% O_{jl} \, t 
%}\right\}.
%\end{align}
%Then via a union bound, we obtain
%\begin{align}
%    \P\left(\norm{ \bxi^\sT \bM \bxi -  (\bI_k \otimes \Tr)\bM }_F   \ge t \right)
%    &\le 
%     \P\left(\norm{\bxi^\sT \bM \bxi -  (\bI_k \otimes \Tr)\bM }_\infty   \ge t/k \right)\\
%    &\le 2 k^2  
%      \exp\left\{ - \frac{c \, t^2/k^2}{
%\max_{j,l}\left\{ F_{jl}^2\right\}+
% \max_{j,k }\left\{O_{jl} \right\}\, t /k
%}\right\}.
%\end{align}
%Now what remains is to use the bounds
%\begin{equation}
%   \max_{jl}  F_{jl}^2 \le \norm{\bM}_F^2 
%   \quad\textrm{and}\quad
%\max_{jl}O_{jl} 
%\le \norm{\bM}_\op.
%\end{equation}
%\end{proof}
%
\begin{lemma}
\label{lemma:hanson-wright}
Let $\bx \sim \normal(\bzero,\bI_d), \bxi := (\bI_k \otimes \bx)$.
   Let $\bM \in\C^{dk\times {dk}}$ be independent of $\bx$, and 
   set $k_+(d):= k\vee \log d$. Then,
   for any $L\ge 1$, we have
\begin{equation*}
    \norm{\bxi^\sT \bM  \bxi -  (\bI_k \otimes \Tr)\bM}_\op \le
     C L \left( k_+(d)^{1/2} d^{1/2} \norm{\bM}_\op \vee k_+(d) \norm{\bM}_\op \right)
\end{equation*}
with probability at least
\begin{equation*}
    1 - 2\min\Big(e^{-cLk}, d^{-cL}\Big)\,
\end{equation*}
where $C,c > 0$ are universal constants.
\end{lemma}

\begin{proof}
Let $\cN$ be a minimal $1/4$-net of the unit ball in $\C^k$.
Then we have
\begin{equation*}
   \|\bxi^\sT\bM \bxi - (\bI_k \otimes\Tr) \bM\|_\op \le
2 \sup_{\bu,\bv\in\cN}  \bu^\sT\left(\bxi^\sT\bM \bxi - (\bI_k \otimes\Tr) \bM\right)\bv.
\end{equation*}
Meanwhile, for any fixed $\bu,\bv\in\cN$, note that $(\bI_k\otimes\bx)\bu = (\bu \otimes\bI_d)\bx$ so that Hanson-Wright gives
    \begin{align*}
        \bu^\sT\bxi^\sT\bM\bxi\bv -  \bu^\sT(\bI_k \otimes \Tr)\bM\bv
        &=
        \bu^\sT\bxi^\sT\bM\bxi\bv -
\E_\bx\left[\bu^\sT\bxi^\sT\bM\bxi\bv \right]
        \\
        &= \bx^\sT(\bu \otimes\bI_d)^\sT \bM (\bv \otimes\bI_d) \bx
        - \Tr\left((\bu \otimes\bI_d)^\sT \bM (\bv \otimes\bI_d)\right) \\
        &\le s
    \end{align*}
with probability larger than 
\begin{equation*}
1 - 2\exp\left\{  -c_0\left(\frac{s^2}{\norm{\bM}_\op^2 d} \wedge 
\frac{s}{\norm{\bM}_\op}
\right)
\right\},
\end{equation*}
where we used that $\norm{\bu\otimes\bI_d}_\op \le \norm{\bu}_2 \le1$ (and same for $\bv$) to deduce
\begin{equation*}
    %\norm{(\bu \otimes \bI_d)^\sT \bM (\bv\otimes \bI_d)}_F \le 
%\norm{\bM}_F, \quad
    \|(\bu \otimes \bI_d)^\sT \bM (\bv\otimes \bI_d)\|_F \le 
     \sqrt{d}
\norm{\bM}_\op, \quad
    \|(\bu \otimes \bI_d)^\sT \bM (\bv\otimes \bI_d)\|_\op \le 
\norm{\bM}_\op.
\end{equation*}
A standard result \cite{vershynin2018high} gives that the size of $\cN$ is at most $C_0^k$ for some $C_0 > 0$. Then taking 
\begin{equation*}
%    s =
%\left(\frac{2 \log(C_0) k  \norm{\bM}_F^2}{c_0}\right)^{1/2}
%    \vee   \frac{2 \log(C_0) k \norm{\bM}_\op}{c_0},
    s = 
 \left( L^{1/2} k_+(d)^{1/2} d^{1/2} \norm{\bM}_\op \vee L k_+(d) \norm{\bM}_\op \right),
\end{equation*}
we obtain via a union bound
\begin{align*}
    \P\left(\|\bxi^\sT\bM\bxi - (\bI_k\otimes\Tr)\bM\|_\op \ge 2 s \right)
    &\le 
    \P\left( \sup_{\bv,\bu\in\cN} \bu^\sT(\bxi^\sT\bM\bxi - (\bI\otimes\Tr)\bM) \bv \ge s  \right)\\
    &\le 2C_0^k \big (e^{-c L k}\wedge d^{-cL}\big)
\end{align*}
%
for some constant $c>0$. The claim follows by taking $L$ a sufficiently large universal constant. 
Redefining the universal constants $c$ and $C_0$ allow us to take $L\ge 1$ as in the statement.

\end{proof}

\subsection{Approximate solution to FP}
\label{sec:approx_ST_FP}
We begin with a concentration result.
\begin{lemma}[Concentration of the leave-one-out quadratic forms]
\label{lemma:concentration_loo_quad_form}
There exist absolute constant $c,C$, such that the following holds.
Let $k_+(d):= k\vee \log d$ and define the event (for $L\ge 1$)
\begin{equation}
\label{eq:concentration_loo_quad_form_2}
   \Omega_1(L) := \left\{\norm{\bxi_i^\sT \bR_i \bxi_i - (\bI_k \otimes \Tr)\bR}_\op
   \le  C L\sqrt{\frac{k_+(d)}{n \alpha_n}} \frac1{\Im(z)} 
   +  \frac{\sfK\norm{\bx_i}_2^2}{n^2 \Im(z)^2}\quad\textrm{for all}\quad i\in[n]\right\}.
\end{equation}
Then for  $n\ge \alpha_n k_+(d)$, $n\le d^{10}$ we have
for some universal constant $c>0$.
\begin{equation}
\nonumber
    \P(\Omega_1(L)^c) \le 2 (e^{- cLk}\vee d^{-cL}).
\end{equation}
\end{lemma}
\begin{proof}

For all $i\in[n]$, we have by Lemma~\ref{lemma:hanson-wright} and the bounds on the norms of $\bR_i$ in Lemma~\ref{lemma:as_norm_bounds}, 
\begin{equation}
\nonumber
    \norm{\bxi_i^\sT\bR_i \bxi_i - \left(\bI_k \otimes\Tr\right) \bR_i}_\op \le 
    %C \left(\frac{k }{\sqrt{n\alpha_n}} \frac1{\Im(z)} \vee \frac{k}{n} \frac1{\Im(z)}\right)
    C L\left(
    \sqrt{\frac{k_+(d)}{\alpha_n n}}
     \vee \frac{k_+(d)}{n} \right)
    \frac1{\Im(z)}
\end{equation}
%
with probability  at least $1-2 (e^{- cLk}\vee d^{-cL})$.
Meanwhile, we have by
Lemma~\ref{lemma:algebra_lemma} and bound of Lemma~\ref{lemma:as_norm_bounds} once again that
    \begin{align}
    \nonumber
        \norm{(\bI_k \otimes \Tr)\left(\bR_i - \bR\right) }_\op = 
\norm{\bxi_i^\sT \bR_i (\bW_i \otimes \bI_d) \bR\bxi_i}_\op
\le \sfK \norm{\bx_i}_2^2 \norm{\bR}_\op\norm{\bR_i}_\op
\le \sfK \frac{\norm{\bx_i}^2}{n^2}\frac1{\Im(z)^2}.
%
    \end{align}
A triangle inequality and union bound gives the result.
\end{proof}

We are ready to prove an approximate 
fixed point equation for $\bS = \bS_n$ defined in Eq.~\eqref{eq:Sn_def}.
\begin{lemma}[Fixed point equation for the Stieltjis transform]
\label{lemma:fix_point_rate}
Let $\Omega_0,\Omega_1(L)$ be the events of Lemmas~\ref{lemma:standard_norm_bounds} and~\ref{lemma:concentration_loo_quad_form} respectively.
For any empirical distribution $\hnu\in\cuP(\R^{k+k_0+1}),$
$z \in\bbH_+$, $L\ge 1$, we have on $\Omega_0 \cap \Omega_1(L)$,
\begin{align}
    \norm{
\frac1{\alpha_n}\bI_k - \bF_z(\bS_n(z);\hnu)^{-1}
\bS_n(z)
    }_\op
&\le   
\Err_{\FP}(z; n,k)
\end{align}
where, letting $k_+(d) = k\vee \log d$,
\begin{align}
\Err_{\FP}(z;n,k) :=  C(\sfK) \frac{(1+|z|^4)}{\Im(z)^4} 
\left( L\sqrt{\frac{ k_+(d) }{n}}  
   +  \frac{1}{ n \Im(z)}\right)
%\omega_{\textrm{FP}}(z,n,k,\alpha_n):=C(\sfD)(\alpha_n+\alpha_n^{-1})
%     \left( \frac{1 + |z|^2}{\Im(z)} \right)
%    \left( \frac{1 + |z|^2}{\Im(z)}
%    + 1\right)
%\left(  L\sqrt{\frac{k_+(d)}{n}}\frac1{\Im(z)} 
%   +  \frac{1}{n \Im(z)^2}\right).
\end{align}
for some $C(\sfK)>0$ depending only on $\sfK$.
%with probability at least
%\begin{equation}
%% 1 - 2 nk^2 \exp\left\{-ck\right\} - C n \exp\left\{ -c d\right\}.
%\end{equation}
\end{lemma}

\begin{proof}
The proof proceeds as in the usual scalar case.
Namely, suppressing the argument $z \in\bbH_+$, 
we write
\begin{align}
\label{eq:decomposition_resolvent_eq}
\frac{d}{n} \cdot \bI_k 
= \frac1n\left(\bI_k \otimes \Tr\right)
\bR^{-1} \bR
= \left(\bI_k \otimes \Tr\right)
\left(\frac1n\sum_{i=1}^n \bxi_i \bW_i\bxi_i^\sT \bR 
 -z  \bR\right)
&= 
\left(\frac1n\sum_{i=1}^n 
\left(\bI_k \otimes \Tr\right)\left(
\bxi_i \bW_i\bxi_i^\sT \bR \right)
 - z  \bS_n  \right).
\end{align}
%Via a leave-one-out argument, we show that for each $i\in[n]$,
%$$\left(\bI_k \otimes \Tr\right)\bxi_i \bSec_i \bxi_i^\sT \bR  
%\approx
%\grad^2 \rho_i(\bI_k \otimes \Tr)\bR(\bI_k + \grad^2 \rho_i(\bI_k \otimes \Tr)\bR)^{-1}$$
%in an appropriate sense.
%Taking expectations with respect to $\widehat \nu_{\bV,\bU}$ will then allow us to conclude
%\begin{equation}
%    \frac{d}{n} \bI_k  \approx \E[\bSec \bQ(\bI + \bSec \bQ)^{-1}] - z \bQ.
%\end{equation}

Letting $\bA_i = \bxi_i^\sT \bR_i\bxi_i$ and recalling the definition $\bS_n = (\bI_k \otimes \Tr)\bR$, we bound
\begin{align*}
        \bDelta_i&:=
        \left(\bI_k \otimes \Tr\right)\bxi_i \bW_i \bxi_i^\sT \bR - \bW_i\bS_n(\bI_k + \bW_i\bS_n)^{-1} \\
        &=\left( \bI_k + \bW_i\bA_i\right)^{-1}\bW_i\bA_i
        - (\bI_k + \bW_i\bS_n)^{-1}\bW_i\bS_n\\
        &=
        \left( \bI_k + \bW_i\bA_i\right)^{-1}\bW_i\bA_i
        - (\bI_k + \bW_i\bS_n)^{-1}\bW_i\bA_i
       +\left( \bI_k + \bW_i\bS_n\right)^{-1}\bW_i\bA_i
        - (\bI_k + \bW_i\bS_n)^{-1}\bW_i\bS_n\\
&= (\bI_k +\bW_i\bA_i)^{-1}\bW_i (\bS_n -\bA_i) (\bI + \bW_i \bS_n)^{-1} \bW_i \bA_i
+ (\bI_k + \bW_i\bS_n)^{-1}\bW_i (\bA_i -\bS_n).
\end{align*}
where the first equality follows from Lemma~\ref{lemma:algebra_lemma}.

Lemma~\ref{lemma:concentration_loo_quad_form} above provides a bound for $\norm{\bA_i -\bS_n}_\op$ on the event $\Omega_1(L)$. Meanwhile, by Lemma~\ref{lemma:tensor_trace_properties}, $\Im(\bS_n) = (\bI_k \otimes \Tr)\Im(\bR) \succ\bzero $ since $\Im(\bR) \succ\bzero$. So we have by Lemmas~\ref{lemma:re_im_properties} and~\ref{lemma:as_norm_bounds} on $\Omega_0$
\begin{equation}
\nonumber
    \norm{(\bI + \bW_i\bS_n)^{-1}\bW_i}_\op 
    \le \norm{\Im(\bS_n)^{-1}}_\op
    \le   \frac{1}{\Im(z)} \left( \frac{\sfK}{n} \norm{\bX}_\op^2 + |z|\right)^2 \le \frac{C_1}{\Im(z)} (\sfK^2 + |z|^2)
\end{equation}
and similarly and by the same lemmas, we conclude on $\Omega_0$
\begin{equation}
\nonumber
    \norm{(\bI + \bW_i\bA_i)^{-1}\bW_i}_\op 
    \le   \frac{1}{\Im(z)} \frac{n}{\norm{\bx_i}_2^2} \left(\frac{\sfK}{n} \norm{\bX}_\op^2 + |z|\right)^2 
    \le \frac{C_2 \alpha_n}{\Im(z)} (\sfK^2 + |z|^2).
\end{equation}
%
Finally, on $\Omega_0$,
\begin{equation}
\nonumber
   \norm{\bA_i}_\op {\le} \norm{\bx_i}_2^2 \norm{\bR_i}_\op \le \frac{C_3}{\alpha_n \Im(z)}.
\end{equation}

Combining these bounds along with the one in Lemma~\ref{lemma:concentration_loo_quad_form} gives on $\Omega_0 \cap\Omega_1(L)$,
\begin{align*}
    \norm{\bDelta_i}_F
  &\le\norm{(\bI_k + \bW_i \bS_n)^{-1}\bW_i}_\op \norm{\bA_i -\bS_n}_\op \left(\norm{(\bI_k + \bW_i \bA_i)^{-1}\bW_i}_\op\norm{\bA_i}_\op + 1\right)
   \\ 
  &\stackrel{(a)}{\le}
  C_4(\sfK) \frac{1}{\Im(z)} (1 + |z|^2) 
\left( \frac{L\sqrt{ k_+(d)} }{\sqrt{n\alpha_n}} \frac1{\Im(z)} 
   +  \frac{\sfK}{\alpha_n n \Im(z)^2}\right)
   \frac{1}{\Im(z)^2}(\sfK^2 + |z|^2)
  \\
  &\stackrel{(b)}{\le} 
  C_5(\sfK) \frac{(1+|z|^4)}{\Im(z)^4} 
\left( \frac{L \sqrt{k_+(d)} }{\sqrt{n}}  
   +  \frac{1}{ n \Im(z)}\right)\\
   &\equiv \Err_{\FP}(z; n, k)
%&\le\frac{C}{\Im(z)}
%    \left(\frac{1}{\Im(z)} \left( \sfK^2 + |z|^2 \right)\right)
%    \left( \frac{1}{ \Im(z)} \left( \sfK^2 + |z|^2 \right) 
%    + 1\right)
%\left( \frac{L k_+(d) }{\sqrt{n\alpha_n}} \frac1{\Im(z)} 
%   +  \frac{\sfK}{\alpha_n n \Im(z)^2}\right)
%\le\omega_{\textrm{FP}}(z,n,k,\alpha_n)
\end{align*}
where in $(a)$ we used that $(\sfK + |z|^2)/\Im(z)^2+1 \le C_6(\sfK) (1+ |z|^2)/\Im(z)^2$ for some $C_6>0,$ and in $(b)$ we used that $\alpha_n \ge 1.$
%
%\am{I think there is a term $\|\bA_i\|_{\op}$ missing. Please, double check. Also, some factors $\alpha_n$ seem off. I get
%\begin{align}
%    \norm{\bDelta_i}_F&\le
%    \left(\frac{2}{\Im(z)} \left( 9\sfK^2 + |z|^2 \right)\right)
%    \left( \frac{8\alpha_n}{ \Im(z) } \left( 9\sfK^2 + |z|^2 \right) 
%    \|\bA_i\|_{\op}+ 1\right)\alpha_n
%\left(C L\frac{k_+(d) }{\sqrt{n\alpha_n}} \frac1{\Im(z)} 
%   +  \frac{\sfK}{\alpha_n n \Im(z)^2}\right)
%\end{align}}
Since this holds for all $i\in[n]$. using Eq.~\eqref{eq:decomposition_resolvent_eq} we conclude
%\begin{align}
%\frac1{\alpha_n} \bI_k  + z\bQ_n
%= 
%\frac1n\sum_{i=1}^n 
%\left(\bI_k \otimes \Tr\right)\left(
%\bz_i \widetilde\bz_i^\sT \bR \right),
%\end{align}
on $\Omega_0\cap\Omega_1$ we have
\begin{align}
\nonumber
&\norm{\frac1\alpha_n\bI_k + z \bS_n - \frac1n \sum_{i=1}^n\bW_i \bS_n(\bI_k + \bW_i \bS_n)^{-1} }_\op
\le
 \frac1n\sum_{i=1}^n \norm{\bDelta_i}_\op\le 
\Err_{\FP}(z;n,k).
\end{align}
\end{proof}




%As a corollary of the above and Hanson-Wright, the following lemma states that the quadratic forms $\bxi_i^\sT\bR_i\bxi_i$ concentrate.
%\begin{lemma}[Concentration of the leave-one-out quadratic forms]
%\label{lemma:concentration_loo_quad_form}
%For all $i\in[n]$ and $t>0$, we have
%\begin{equation}
%\label{eq:concentration_loo_quad_form_1}
%   \P\left( \norm{\bxi_i^\sT \bR_i \bxi_i - (\bI_k \otimes \Tr)\bR_i }_F  \ge  t \right) \le 
%2 k^2  
%      \exp\left\{ - \frac{c \Im(z)\, t^2 \, n}{
% \frac{dk^3}{ n \Im(z)}
%  +  tk 
%}\right\}
%\end{equation}
%for some universal constant $c >0$.
%Consequently, for any $s>0$, we have
%\begin{equation}
%\label{eq:concentration_loo_quad_form_2}
%   \norm{\bxi_i^\sT \bR_i \bxi_i - (\bI_k \otimes \Tr)\bR}_F
%   \le \frac{k^{3/2+s}}{\sqrt{n}}\frac1{\Im(z)} +  \frac{\sfK\norm{\bx_i}_2^2}{\alpha_n n^2 \Im(z)^2}
%\end{equation}
%with probability at least
%\begin{equation}
%    1 - 2 k^2\exp\left\{ -  \frac{ c k^{3+2s} \alpha_n }{ k^3 + k^{5/2+s} \alpha_n^{1/2}d^{-1/2} }\right\}. %- C \exp\left\{ - cd\right\}.
%\end{equation}
%\end{lemma}
%\begin{proof}
%The tail bound follows directly from Lemma~\ref{lemma:hanson-wright} after appying the bounds of Lemma~\ref{lemma:as_norm_bounds} on $\norm{\bR_i}_F$ and $\norm{\bR_i}_\op$.
%Meanwhile, for Eq.~\eqref{eq:concentration_loo_quad_form_2},
%Lemma~\ref{lemma:algebra_lemma} and the operator norm bound of Lemma~\ref{lemma:as_norm_bounds} gives
%    \begin{align}
%        \norm{(\bI_k \otimes \Tr)\left(\bR_i - \bR\right) }_\op = 
%\norm{\bz_i^\sT \bR_i (\bSec_i \otimes \bI_d) \bR\bz_i}_\op
%\le \sfK \norm{\bx}_2^2 \norm{\bR}_\op\norm{\bR_i}_\op
%\le \sfK \frac{\norm{\bx}_i^2}{n^2}\frac1{\Im(z)^2}.
%%\norm{\grad^2 \rho_i (\bI_k \otimes \Tr)\left(\bM_i^{-1} \bz_i (\bI_k + \widetilde \bz_i^\sT \bM_i^{-1} \bz_i)^{-1} \widetilde \bz_i^\sT \bM_i^{-1}\right) }_F\\
%%&\le \frac{2\sqrt{ndk}\sfK^2}{n^2 \Im(z)^2}
%    \end{align}
%%where the last equality holds 
%%with probability at least $1- Ce^{- c d}$.
%Taking $t = k^{3/2+s}n^{-1/2}/\Im(z)$ gives the result.
%%\begin{equation}
%%      %1 -  2k^2 \exp\left\{-  \frac{c n\sqrt{d}  }{ \sqrt{d} k^{3/2}  +  \sqrt{n k} }\right\}.
%%\end{equation}
%%Combining with the concentration bound of Eq.~\eqref{eq:concentration_loo_quad_form_1} for $t = k^{3/2+s}n^{-1/2}/\Im(z)$ gives the result.
%\end{proof}
%




%\begin{lemma}[High probability norm bounds]
%\label{lemma:hp_norm_bounds}
%For all $i \in[n]$, we have, 
%\begin{equation}
% \norm{
%   (\bI_k \otimes \Tr)
%\bM_i^{-1}\bz_i
%\left(\bI_k + \widetilde\bz_i^\sT \bM_{i}^{-1} \bz_i\right)^{-1} \widetilde\bz_i^\sT \bM_i^{-1} 
%}_F \le 
%\frac{ 2\sqrt{n d k } \sfK}{n^2 \Im(z)^2}
%\end{equation}
%with probability at least 
%\begin{equation}
%      1 -  2k^2 \exp\left\{-  \frac{c n\sqrt{d}  }{ \sqrt{d} k^{3/2}  +  \sqrt{n k} }\right\}
%\end{equation}
%for some universal constant $c>0$.
%\end{lemma}
%\begin{proof}
%Let $\bA := \bM_i^{-1} (\grad^2 \rho_i \otimes \bI_k) \bM^{-1}$ and for $j,l\in[k]$ let $(\bA)_{jl}\in\R^{d}$ be the $j,l$th block of $\bA$. Then by Lemma~\ref{lemma:algebra_lemma}, we have
%\begin{align}
%   (\bI_k \otimes \Tr)
%   \bM_i^{-1}
%\bz_i
%\left(\bI_k + \widetilde\bz_i^\sT \bM_{i}^{-1} \bz_i\right)^{-1} \widetilde\bz_i^\sT \bM_i^{-1} 
%&= 
%     \bz_i^\sT \bA \bz_i.
%\end{align}
%
%For all $t>0$ and some universal $c>0$, we have by Hanson-Wright and a union bound,
%\begin{align}
%\P\left(
%    \norm{(\bI_k \otimes \Tr)(\bz_i^\sT \bA \bz_i) -  (\bI_{k}\otimes \Tr) \bA }_F \ge t
%\right)
%&\le    \P\left(\sum_{j,l}\left|\bx_i^\sT \bA_{jl}\bx_i - \Tr(\bA_{jl})\right|^2  > t^2\right)\\
%&\le    \P\left(\max_{j,l}\left|\bx_i^\sT \bA_{jl}\bx_i - \Tr(\bA_{jl})\right|^2  > \frac{t^2}{k^2}\right)\\
%&\le 2 k^2 \exp\left\{ \frac{-c (t/k)^2}{ \max_{jl}F_{jl}^2 + (t/k)\max_{jl}O_{jl} }\right\}\\
%&\le 2 k^2 \exp\left\{ \frac{-c (t/k)^2}{ \norm{\bA}_F^2 + (t/k) \norm{\bA}_\op }\right\}.
%\end{align}
%
%Now note that
%\begin{equation}
%    \norm{\bA}_\op \le \norm{\bM_{i}^{-1}}_\op
%    \norm{\grad^2\rho_i}_\op
%    \norm{\bM^{-1}}_\op\le  \frac{\sfK}{n^2} \frac1{\Im(z)^2}
%\end{equation}
%and 
%\begin{equation}
%    \norm{\bA}_F^2 \le \norm{\bM_{i}^{-1}}_F^2
%    \norm{\grad^2\rho_i}_\op^2
%    \norm{\bM^{-1}}_\op^2\le  \frac{d k \sfK^2}{n^4} \frac1{\Im(z)^4}.
%\end{equation}
%Taking $t = (\sqrt{ndk} \sfK)/n^2 \Im(z)^2$, we obtain that with probability larger than 
%\begin{equation}
%   1 -  2k^2 \exp\left\{- \frac{ nd \sfK^2 }{ d k^{3/2} \sfK^2 + \sfK^2 \sqrt{nd k} }\right\}
%\end{equation}
%we have, using Lemma~\notate{ref}
%\begin{align}
%\norm{(\bI_k \otimes \Tr) (\bz_i^\sT\bA\bz_i)}_F &\le \norm{(\bI_k \otimes \Tr)\bA}_F  + t
%\le \frac{ 2\sqrt{n d k } \sfK}{n^2 \Im(z)^2}.
%\end{align}
%\end{proof}
\subsection{The asymptotic matrix-valued ST}
\label{sec:AsymptoticST}

\subsubsection{Free probability preliminaries}

In this section we collect some relevant free probability background. Most of what follows can be found in
\cite{nica2006lectures,mingo2017free}.
Let $(\cA,\tau)$ be a $C^*$-probability space.
An element $M\in\cA$ is said to have a free Poisson distribution with rate $\alpha_0$ if the moments of $M$ under $\tau$ correspond to the moments of the Marchenko-Pastur law with aspect ratio $\alpha_0$. For $M,T \in\cA$, if $M$ is a free Poisson element and $T$ is self-adjoint, then $M^{1/2} T M^{1/2}$ has distribution given by the multiplicative free convolution of the free Poisson distribution and the distribution of $T$. 


Given a distribution $\nu_{0}$ on $(\bv,\bu, w) \in\R^{k+k_0+1}$, 
consider a sequence (indexed by $m$)
of collections of deterministic real diagonal matrices $\left\{\bar\bSec_{a,b}^\up{m}\right\}_{a,b\in[k]}$ with
$\bar\bSec_{a,b}^\up{m}\in \R^{m\times m}$,
$\bar\bSec_{a,b}^\up{m} = \diag((\bar K_i)_{a,b}:\; i\le m)$
such that the following hold.
%\bns{Why uniformly bounded? I am using $\sfK$ here for $\|\grad^2\ell\|_\op$. What is $K_i$?}
The entries of these matrices are uniformly bounded and
letting $\sP_{\nabla^2 \ell}$ denote the probability distribution
of $\nabla^2\ell(\bv,\bu, w)$ when $(\bv,\bu, w) \sim\nu_0$,
we have
%
\begin{align}
\nonumber
\frac{1}{m}\sum_{i=1}^m\delta_{\bar K_i} \Rightarrow 
\sP_{\nabla^2 \ell}\, .
\end{align}
%
Equivalently, for any set of pairs 
$\cP= \{ (a_1,b_1),(a_2,b_2),\dots, (a_L,b_L)\}$, 
\begin{equation}
\label{eq:K_empirical_limit}
    \lim_{m\to\infty} \frac1m \Tr\left( \prod_{(i,j) \in \cP} 
   \bar \bSec_{i,j}^\up{m}
    \right) = \int \prod_{(i,j)\in\cP}  \partial_{i,j}\ell(\bv,\bu, w)  \, \, \de\nu_{0}(\bv,\bu,w).
\end{equation}
Now define $T_{i,j}\in(\cA,\tau)$ for each $i,j\in [k]$
so that for any $\cP$,
%
\begin{equation}
\label{eq:taus_of_prods_of_D}
     \tau\left( \prod_{(i,j) \in \cP} 
   T_{i,j}
    \right) = \int \prod_{(i,j)\in\cP}  \partial_{i,j}\ell(\bv,\bu,w) \,  \de\nu_{0}(\bv,\bu, w)
\end{equation}
and that $\{T_{i,j}\}_{i,j\in [k]}\cup \{M\}$ are free.

Let $\{\bX^\up{m}\}_{m\ge 1}$ be a sequence of $m\times p$ matrices of i.i.d standard normal entries so that
$m/p \to \alpha_0$.
Then for any noncommutative polynomial $Q$ on $n$ variables, and any $(i_1,j_1),\dots (i_m,j_m) \in [k]\times [k]$,
\begin{equation}
\label{eq:moment_convergence}
    \lim_{m\to\infty}  \E\left[\frac1m\Tr\; Q( ( m^{-1}\bX^{\up{m}\sT} \bar \bSec_{i_l,j_l}^\up{m}\bX^\up{m})_{l\in[m]})\right] =  \tau\left( Q\left( M^{1/2} T_{i_l,j_l} M^{1/2}\right)_{l\in[m]} \right).
\end{equation}
 The quantity on the right hand side of the last equation is completely determined by the moments of the 
 form~\eqref{eq:taus_of_prods_of_D} by freeness of $\{T_{i,j}\}_{i,j} \cup \{M\}$.

%Now let $\bSec_{i,j}$ for $i,j\in[k]$ be the diagonal matrices defined in~\notate{ref}. 
%For each $i,j \in[k]$, let $\nu_{i,j}$ be the empirical spectral distribution of $\bSec_{i,j}$ and define the sequence $\left\{\bSec_{i,j}^{(n)}\right\}_{n}$ in such a way so that for all $n$, the diagonal matrix $\bSec_{i,j}^{(n)}$ has spectral measure $\nu_{i,j}$.
\subsubsection{Characterization of the asymptotic matrix-valued Stieltjis transform}
%\label{sec:AsymptoticST}

Let $\cM^{k\times k}(\cA)$ denote the set of $k\times k$ matrices with values in $\cA$.
For $z\in\bbH_+$, $\nu_0\in\cuP(\R^{k+k_0+1})$, and $\alpha_0 >1$, define 
$\bH_{*}(\alpha_0,\nu_0),\bR_\star(z; \alpha_0, \nu_0) \in 
\cM^{k\times k}(\cA)$ via
\begin{align}
\label{eq:def_H_star}
\bH_\star(\alpha_0, \nu_0) &:= (M^{1/2}T_{i,j} M^{1/2})_{i,j\in[k]} \, ,\\
   \bR_\star(z; \alpha_0, \nu_0) &:= \left(
   \bH_\star(\alpha_0, \nu_0)- z (\bI_k \otimes \aid )\right)^{-1} ,
\end{align}
where, for $i,j\in[k]$, $T_{i,j}\in(\cA,\tau)$ satisfy Eq.~\eqref{eq:taus_of_prods_of_D} for $\nu_0$, 
and are free; here $M \in(\cA,\tau)$ is
a free Poisson element with rate $\alpha_0$. 
Further, let
\begin{equation}
\label{eq:def_S_star}
    \bS_\star(z; \alpha_0, \nu_0) := \left(\bI_k \otimes \tau\right) \bR_\star(z; \alpha_0, \nu_0) \in\C^{k\times k}.
\end{equation}
The following lemma shows that $\bS_\star$ satisfies 
the fixed point equation for the Stieltjes transform, as one might expect. Note however the utility of this result: it allows one to decouple the asymptotics of the Gaussian matrix $\bX$ from that of the measure $\nu_0.$
%
\begin{lemma}[Asymptotic solution of the fixed point equation]
\label{lemma:asymp_ST}
Fix $\nu_0\in\cuP(\R^{k+k_0+1})$.
Assume $\|\grad^2 \rho(\bv,\bu,w)\|_{\op}\le \sfK$ with 
probability one under $\nu_0$.
For any fixed positive integer $k$, $z\in \bbH_+$,  
and $\alpha_0 >1$, we have
    \begin{equation}
        \alpha_0 \bS_\star(z; \alpha_0, \nu_0) = 
        \bF_z(\bS_\star(z; \alpha_0, \nu_0);\nu_0).
    \end{equation}
\end{lemma}
\begin{proof}
Fix $k$ throughout and let $\bT \in\cM^{k\times k}(\cA)$ be defined by $\bT := ( T_{i,j})_{i,j\in[k]}$,
where, for $i,j\in[k]$, $T_{i,j}\in(\cA,\tau)$ satisfy Eq.~\eqref{eq:taus_of_prods_of_D} for $\nu_0$. 

Our goal is to use Lemma~\ref{lemma:fix_point_rate} to show that $\bS_\star(z;\alpha_0,\nu_0)$ satisfies the desired fixed point equation. Since this lemma is stated in terms of empirical measures, we define 
$\{\hnu_{0,m}\}_m$,
$\hnu_{0,m}\in\cuP_m(\R^{k+k_0+1})$
to be a sequence of empirical measures satisfying $\hnu_{0,m}\Rightarrow \nu_0$.
These in turn define  a sequence $\bar \bSec^\up{m} = \left(\bar \bSec^\up{m}_{i,j}\right)_{i,j \in[k]}$
of deterministic matrices 
satisfying~\eqref{eq:K_empirical_limit} for the given $\nu_0$.

Write $\bS_\star(z)$ via its power expansion:
We have, by boundedness of $\bT$,
for $|z|$ sufficiently large
%\am{Need to use the fact that
%$\max_{ij}\|\bar D_{i,j}\|<\infty$ or similar?}
\begin{align}
\nonumber
\bS_\star(z) :=
\bS_\star(z; \alpha_0,\nu_0)
&= (\bI_k \otimes \tau)\left[ \frac1{z}\left( z^{-1}(\bI_k \otimes M^{1/2})\bT(\bI_k \otimes M^{1/2} )   - (\bI_k \otimes \id) \right)^{-1}\right]
\\
&= \sum_{a=0}^{\infty} (-z)^{-{(a+1)}}(\bI_k \otimes \tau)\left[\left( 
(\bI_k \otimes M^{1/2})\bT (\bI_k \otimes M^{1/2} )
\right)^a\right].\label{eq:ExpSstar}
\end{align}
Then, Eq.~\eqref{eq:moment_convergence} gives
\begin{equation}
\nonumber
    (\bI_k \otimes \tau)\left[\left( 
(\bI_k \otimes M^{1/2})\bT (\bI_k \otimes M^{1/2} )
\right)^a\right] = \lim_{m\to\infty} 
    (\bI_k \otimes \frac1m \Tr)\left[\left( 
(\bI_k \otimes \bX^\sT)\bar\bSec^\up{m} (\bI_k \otimes \bX )
\right)^a\right].
\end{equation}
Hence, there exists $r_0$, possibly dependent on $k$, such that for $|z| > r_0$, 
\begin{align*}
    \bS_\star(z) &= \lim_{m\to\infty}
\sum_{a=0}^{\infty}
    (-z)^{-(a+1)}(\bI_k \otimes \frac1m \Tr)\left[\left( 
(\bI_k \otimes \bX^\sT)\bar\bSec^\up{m} (\bI_k \otimes \bX )
\right)^a\right]
= \lim_{m\to\infty}\bS_m(z, \hnu_{0,m})
\end{align*}
element-wise.
It follows again from  Eq.~\eqref{eq:ExpSstar}
that $\|\bS\|_{\op}\le C/|z|$ for $|z|\ge r_1$.
Finally, for $|z|\le r_2$,
$\bS\mapsto \bF_z(\bS;\nu_0)$ is continuous 
on $\|\bS\|_{\op}\le \eta$ 
(for suitable constants $r_1,r_2,\eta$). 
Eventually increasing $r_0$,
Lemma~\ref{lemma:fix_point_rate} implies that, for $|z|\ge r_0$
\begin{equation} 
\nonumber
    \bF_z(\bS_\star(z);\nu_0) = \lim_{m\to\infty}  \bF_z(\bS_m(z);\hnu_{0,m})
     =  \lim_{m\to\infty} \alpha_0\bS_m(z; \hnu_{0,m}) = \alpha_0\bS_\star(z).
\end{equation}
Since $z\mapsto \bF_z(\bS_\star(z))$ and $z\mapsto \bS_\star(z)$ are both analytic on $\bbH_+$ 
%($\bF_z(\bS)$ is analytic for all $\bS \in\bbH_+^k$),
we conclude the claim by analytic continuation.
\end{proof}

We now give a lower bound on the minimum singular value of $\bS_\star$. This will be necessary to establish that $\bS_\star$ is the unique solution of the Stieltjes transform fixed point equation.
\begin{lemma}
\label{lemma:smallest_singular_value_Sstar}
For any $z\in\bbH_+$, $\nu_0\in\cuP(\R^{k+k_0+1})$ and $\alpha_0 >1$, we have
\begin{equation}
\nonumber
    \norm{\Im(\bS_\star(z; \alpha_0, \nu_0))^{-1}}_\op \le \frac1{\Im(z)} \left(\sfK (1+ \alpha_0^{-1/2})^2 + |z|\right)^2.
\end{equation}
\end{lemma}


\begin{proof}
By the Gelfand-Naimark-Segal construction
\cite[Lecture 7]{nica2006lectures}, there exists a Hilbert space $\cH$ and a $*$-representation of $\cA$, $\pi :\cA \to\cB(\cH)$ and some $\psi_0 \in \cH$ with $\norm{\psi_0}_\cH = 1$ such that for any $A \in\cA$, 
\begin{equation}
    \tau(A) = \inner{\psi_0,\pi(A) \psi_0}_\cH.
\end{equation}
Let us identify $A$ with $\pi(A)$ in the notation below.
Recall the product space $\cH^{k} := \cH \times \cdots \times \cH$ with Hilbert inner product given by
   \begin{equation}
        \inner{
        (\xi_1,\cdots,\xi_k),(\bar\xi_1,\cdots,\bar\xi_k)}_{\cH^k} = \sum_{i=1}^k \inner{\xi_i, \bar \xi_i}_\cH
   \end{equation}
for $\xi_i,\bar \xi_i \in\cH$ for $i\in[k]$.
Now we can bound 
$\sigma_{\min}(\Im(\bS_\star))$ by the variational characterization 
\begin{align}
\sigma_{\min}(\Im(\bS_\star)) 
%&= \lambda_{\min}(\Im(\bS_\star))\\ 
&= 
   \lambda_{\min}\left((\bI_k \otimes \tau)\Im(\bR_\star)\right)
   =\hspace{-2mm} \inf_{\substack{\bu\in\C^k\\\norm{\bu}_2 = 1}}  \bu^* 
   \left(\bI_k\otimes \psi_0\right)^* \Im(\bR_\star)(\bI_k \otimes \psi_0) \bu
  \nonumber 
   \\
&= \hspace{-2mm}\inf_{\substack{\bu\in\C^k\\\norm{\bu}_2 = 1}} \psi_0^* (\bu\otimes \aid)^* 
    \Im(\bR_\star)(\bu \otimes \aid) \psi_0
    \nonumber
   \\
&\ge  \norm{\psi_0}_\cH^2 \norm{\bu \otimes \aid}_{\cH^k}^2 
\lambda_{\min}\left(\Im(\bR_\star)\right) 
= 
\lambda_{\min}\left(\Im(\bR_\star)\right).
\nonumber
\end{align}
So by Lemma~\ref{lemma:re_im_properties}, 
\begin{align}
\sigma_{\min}(\Im(\bS_\star)) 
&= \Im(z)\lambda_{\min}\left(\bR_\star^* \bR_\star\right)
\ge  \Im(z) \left( \norm{(\bI_k \otimes M^{1/2}) \bar \bSec (\bI_k \otimes M^{1/2})}_{\cB(\cH^k)} + |z| \right)^{-2}
\label{eq:lb_on_Sstar}
\end{align}
%where $(a)$  follows from $\Im(\bS_\star) \succ\bzero$
To bound the norm in the term above, 
note that for any nonnegative integer $p$, we have
   \begin{equation}
   \nonumber
       \left(\frac1k \Tr \otimes \tau\right)\left(|\bar \bSec|^{2p}\right) 
\stackrel{(a)}{=}
       \left(\frac1k \Tr \otimes \tau\right)\left(\bar \bSec^{2p}\right) 
= 
\lim_{n\to\infty}\left(\frac1k \Tr \otimes \frac1n\Tr\right)\left((\bar \bSec^\up{n})^{2p}\right) \stackrel{(b)}{\le} \lim_{n\to\infty}\norm{\bar \bSec^\up{n}}_\op^{2p} \stackrel{(c)}{\le} \sfK^{2p}
   \end{equation}
where $(a)$ follows from self-adjointness, $(b)$ follows by monotonicity of $L^p$ norms and $(c)$ follows from Lemma~\ref{lemma:as_norm_bounds}. Now taking both sides to the power $1/(2p)$ and sending $p\to\infty$ gives the bound
\begin{equation}
\nonumber
    \norm{\bar \bSec}_{\cB(\cH^k)} \le \sfK.
\end{equation}
Meanwhile, by definition of the free Poisson element $M$, we have
\begin{equation}
\nonumber
\norm{\bI_k \otimes M^{1/2}}_{\cH^k}^2 \le (1+ \alpha_0^{-1/2})^2.
\end{equation}
Combining the previous two displays with Eq.~\eqref{eq:lb_on_Sstar} gives the claim.
\end{proof}
As a summary of this section, we have the following corollary.
\begin{corollary}
\label{cor:S_star_min_singular_value_bound}
For any $z\in\bbH_+,\nu\in\cuP(\R^{k+k_0+1}),$ any positive integer $k$ and any $\alpha_n \in\Q$ with $\alpha_n > 1$, the fixed point equation 
\begin{equation}
    \alpha_n \bS = \bF_z(\bS; \nu)
\end{equation}
has a solution $\bS_\star(z; \alpha_n, \nu) \in \bbH_+^k$ satisfying
\begin{equation}
    \norm{\Im(\bS_\star(z; \alpha_n, \nu))^{-1}}_\op \le \frac1{\Im(z)} \left(\sfK (1+ \alpha_n^{-1/2})^2 + |z|\right)^2.
\end{equation}
Furthermore, $\mu_\star(\mu,\nu)$ is compactly supported uniformly in $\mu,\nu$.

%and 
%\begin{equation}
%   \norm{\bS_\star(z; \alpha_n, \widehat \nu_{\bV,\bU,\bw})}_\op \le  .
%\end{equation}
%\bns{Provide a proof of the operator norm bound on $\bS_\star$ by the same proof as above using GNS.}
\end{corollary}
\begin{proof}
  The first part of the statement follows directly from Lemma~\ref{lemma:smallest_singular_value_Sstar}.
  That $\mu_\star(\mu,\nu)$ is compactly supported follows from the definition of $\mu_\star$ as the measure whose Stieltjes transform is $\bS_\star$ of Eq.~\eqref{eq:def_S_star} and the definition in Eq.~\eqref{eq:def_H_star} and uniform bounds on the operator norm of $\bH_\star.$
\end{proof}

%
%   \begin{align}
%       \left(\frac1k \Tr \otimes \tau\right)\left(|(\bI_k \otimes M^{1/2})\widetilde \bSec (\bI_k \otimes M^{1/2})|^{2p}\right) 
%&\stackrel{(a)}{=}
%\lim_{n\to\infty}\left(\frac1k \Tr \otimes \frac1n\Tr\right)\left(
%\left((\bI_k\otimes n^{-1/2}\bX)^{\sT}\widetilde \bSec^\up{n} (\bI_k \otimes n^{-1/2}\bX)\right)^{2p}\right) \\
%&\stackrel{(b)}{\le} \lim_{n\to\infty}\norm{\widetilde \bSec^\up{n}}_\op^{2p} \stackrel{(c)}{\le} \sfK^{2p}
%   \end{align}
%where $(a)$ follows from self-adjointness and Eq.~\eqref{eq:moment_convergence}, and $(b)$ follows by monotonicity of $L^p$ norms and $(c)$ follows from Lemma~\ref{lemma:as_norm_bounds}. Now taking both sides to the power $1/(2p)$ and sending $p\to\infty$ gives the claim.





   





%%%Old existence and uniquness proof.
%\subsection{Existence and uniqueness of the solution of FP equation}
%For this section, it'll be convenient to define $\bG :\bbH_k^+ \mapsto \C$ by 
%\begin{equation}
%   \bG(\bS;\bSec) := (\bI + \bSec \bS)^{-1}\bSec
%\end{equation}
%Our goal is to prove the following.
%\begin{proposition}
%\label{prop:existence_unqiuness}
%For any $z\in \bbH^+$, the equation $\alpha\bS = \bF_z(\bS)$ has a unique solution $\bS_\star$ in $\bbH_k^+$.
%\end{proposition}
%This method of proof is due to~\notate{cite}
%for proving existence and uniquness of matrix quadratic equations (See Theorem 2.1). We apply the Earle-Hamilton fixed point theorem specialized below to our setting
%\bns{change $\bbH_k^+$ to $\bbH_+^k$: makes more sense.}
%\begin{theorem}[Reformulation of Earle-Hamilton Theorem]
%Let $\cH \subseteq \bbH_+^k$ be a connected open subset, and $\bF :\cH \mapsto \C^{k\times k}$ holomorphic.  If the image of $\bF(\cH)$ is bounded, and $\bF(\cH)$ lies strictly inside $\cH$, then $\bF$ has a unique fixed point $\bS$ in $\cH$.
%\end{theorem}
%
%\bns{need lower bound on singular value of $\bF(\bQ)$ unformly over all $\bQ\in\bbH_+^k$.}





%To apply this theorem in the proof of Proposition~\ref{prop:existence_unqiuness}, it's sufficient to show the following lemma. 
%For $R,r>0$, let
%\begin{equation}
%   \cH(R,r) :=   \left\{\bS: \Im(\bS)\succ r\bI_k,\; \norm{\bS}_\op < R\right\}.
%\end{equation}
%Given $z$, let $R(z),r(z) > 0$ satisfy
%\begin{equation}
%    R(z) = \frac2{\alpha\Im(z)},
%    \quad\quad
%    r\left(|z| + r\right)^2 = \frac{\Im(z)}{2\alpha},
%\end{equation}
%respectively. 
%\begin{lemma}
%Fix $z \in \bbH_+$. Then for all $R\ge R(z)$, $r < r(z)$, the image of $\cH(R,r)$ under $\alpha^{-1}\bF_z$ lies strictly inside $\cH(R,r)$.
%\end{lemma}
%\begin{proof}
%Let us suppress the dependence on $\bSec$ in what follows in the expectations.
%For $\bS \in\bbH_+^k$ and $z\in\bbH_+$,
%we have by Lemma~\ref{lemma:re_im_properties},
%\begin{equation}
%\label{eq:imagine_part_F}
%    \Im(\bF_z(\bS)) = -\bF_z(\bS) \left(\Im(\E[\bG(\bS)]) - \Im(z) \bI  \right)\bF_z(\bS)^* =
%\bF_z(\bS) \left(\E[\bG(\bS)\Im(\bS)\bG(\bS)^*] + \Im(z) \bI  \right)\bF_z(\bS)^* 
%\succ \bzero.
%\end{equation}
%Hence, by the bounds in this same lemma,
%\begin{equation}
%   \norm{\frac{1}{\alpha}\bF_z(\bS)}_\op \le \norm{\frac{1}{\alpha}\Im\left(\E[\bG(\bS;\bSec)] -z\bI\right)^{-1}}_\op 
%   = \norm{\frac{1}{\alpha}(\E[\bG(\bS)\Im(\bS) \bG(\bS)] + z\bI)^{-1}}_\op \le \frac1{\alpha\Im(z)} = \frac{R(z)}{2}.
%\end{equation}
%So the image of $\cH(R,r)$ is contained strictly in $\{\norm{\bS}_\op < R\}$ for all $R \ge R(z).$
%We show now that it's contained strictly in $\{\bS \succ r\}$ for any $r\le r(z).$
%    %As an illustration, let us first consider the case of $k=1$.
%    %Write $F_z$ and $\eta$ for the functions involved in this case.
%    %Then we can write
%    %\begin{align}
%    %    \Im( F_z) = \Im\left(\frac{F_z \overline F_z}{\overline F_z}\right)
%    %    =  |F_z|^2 \Im\left(\frac{1}{\overline F_z } \right)
%    %    =|F_z|^2 \Im \left(\overline z - \overline \eta\right) \le - |F_z|^2 \Im(z).
%    %\end{align}
%    %Now, we have 
%    %\begin{equation}
%    %|F_z|^2 \Im(z) = |(z - \eta)^{-1}|^2  |\Im(z)| \ge 
%    %\frac{|\Im(z)|}{\left(|z| + \sup_{q \in \cH(C)}|\eta(q)|\right)^2}.
%    %\end{equation}
%    %This gives a sufficient bound.
%    %We write now
%    %\begin{align}
%    %    \Im(\bF_z) = \frac1{2i}\left(\bF_z - \bF_z^*\right) = \bF_z^* \left( \frac{\bF_z^{* -1} - \bF_z^{-1}}{2i}\right)\bF_z = \bF_z^* \Im(\bF_z^{*-1}) \bF_z = \bF_z^* \left(\Im(\overline{z}\bI)- \Im(\boldeta^*)\right) \bF_z.
%    %\end{align}
%    %Now  since $-\Im(\boldeta^*) = \Im(\boldeta) \preceq \bzero$ by assumption, we have the bound
%First, we bound the lowest singular value of $\bF_z$ using Lemma~\ref{lemma:re_im_properties}:
%    \begin{align}
%    \norm{\bF_z^{*-1}}_\op \le \norm{\E[\bG(\bS)] - z\bI}_\op
%\le
%    \norm{\Im(\bS)^{-1}}_\op  + |z|
%    \le  \eps + |z|.
%    \end{align}
%Consequently, 
%by Eq.~\eqref{eq:imagine_part_F} once again, we have
%    \begin{equation}
%        \Im(\alpha^{-1}\bF_z) \succeq  \frac{\Im(z)}{\alpha}\bF_z\bF_z^*
%\succeq\frac{\Im(z)}{\alpha(|z| + r(z))^2} \bI_k = 2 r(z) \bI \succ r(z) \bI,
%\end{equation}
%which gives the desired conclusion.
%\end{proof}
%

%\begin{theorem}[Theorem 2.1, \notate{Cite}]
%Assume $\boldeta$ is holomorphic and satisfies $\Im(\boldeta(\bZ))\preceq\bzero$ for $\bZ\in\bbH_k^+$. 
%Then 
%\begin{equation}
%\bQ = \frac{1}{\alpha} \bF_z(\bQ)    
%\end{equation}
%has a unique solution in $\bbH^-_k$.
%\end{theorem}



%{Cite: Speicher; Operator-valued semicircular elements: solving a quadrtic matrix} for proving uniqueness of the fixed point equation for the operator valued semi-circular law.
%Given $\boldeta:\R^{k\times k} \mapsto \R^{k\times k}$ and $z \in \C$, define
%    \begin{equation}
%        \bF_z(\bQ) := \left( z\bI_{k} - \boldeta(\bQ)\right)^{-1}.
%    \end{equation}
%The goal is to show that $\bQ \mapsto \bF_z(\bQ)$ has exactly one solution in $\bQ \in \bbH_k^-$, for any $z \in \bbH^+$.

%\begin{proof}[Proof of Proposition~\ref{prop:existence_unqiuness}] 
%Fix any $z \in \bbH_+$. Then for any $R > R(z), r<r(z)$, $\alpha^{-1}\bF_z$ has a unique fixed point in $\cH(R,r)$ be Theorem~\notate{ref}. Uniqueness of this solution on $\bbH_+^k$ follows from observing that $\cH(R_p,r_p) \uparrow \bbH_+^k$ as $p\to\infty$ for $R_p := p R(z), r_p := r(z)/p$.
%\end{proof}


    

%\paragraph{Verification of the properties of $\eta$}
%
%For a symmetric matrix $\bSec\in\R^{k\times k}$, let $\bSec^{1/2}$ denote its square root in $\C^{k\times k}$.
%Define
%\begin{equation}
%    \boldeta(\bQ) := \E_\bSec[(\bI -\bSec\bQ)^{-1}\bSec]
%\end{equation}
%where the expectation is over $\bSec$ supported on real symmetric matrices.
%We show the following.
%\begin{lemma}
%The map $\boldeta$ is well-defined and holomorphic on $\bbH_k^-$. Furthermore, $\Im(\boldeta(\bQ))\preceq\bzero$ for $\bQ\in\bbH_k^{-}$.
%\end{lemma}
%\begin{proof}
%Fix $\bQ \in\bbH_k^-$.
%Let us first show that $(\bI-\bSec\bQ)$ is invertible for any fixed $\bSec$ real and symmetric.
%Namely, we'll show that for all non-zero $\bv \in\C^k$, $\bv^*(\bI - \bSec\bQ) \neq \bzero^*$.
%First note that if non-zero $\bv$ satisfies $\bSec\bv =\bzero$, then this is clear.
%Meanwhile, to see this for the case of $\bv$ with $\bu := \bSec\bv \neq \bzero,$ note that $\bSec,\Re(\bQ)$ and $\Im(\bQ)$ are all self-adjoint and hence
%\begin{equation}
%    \bv^* (\bI - \bSec \bQ)\bu =  \bv^*\bSec \bv - \bv^* \bSec \Re(\bQ) \bSec\bv  -i\bv^* \bSec \Im(\bQ) \bSec\bv = \bv^*\bSec \bv - \bu^* \Re(\bQ) \bu - i \bu^* \Im(\bQ) \bu
%\end{equation}
%is of the form $ a + ib$ for $a,b\in\R$ and $b = \bu^* \Im(\bQ) \bu  < 0$ by assumption on $\bQ$. Hence $\bv^*(\bI - \bSec\bQ) \neq \bzero^*$.
%
%This now implies that $\boldeta(\bQ)$ is holomorphic and well defined on $\bbH_k^-$. What remains is to show is that $\Im(\bQ) \preceq \bzero$ if $\bQ\in\bbH_k^-$ .
%
%%Let $\cG = \{\, \bSec\,\, \textrm{invertible} \,\}$.
%%On $\cG$,  we have $(\bI -\bSec\bQ)^{-1}\bSec = (\bSec^{-1} - \bQ)^{-1}$.
%%Since $\Im((\bSec^{-1} - \bQ)) = -\Im(\bQ) \succ\bzero$, we conclude by Lemma~\ref{lemma:re_im_properties} that $(\bSec^{-1} - \bQ)^{-1} \in\bbH_k^-$ on $\cG$.
%
%%Now for $\cG^c$,
%For fixed $\bSec$, let $\sigma_{\min} \equiv \sigma_{\min}(\bSec) \in\R$, and for $\eps \in( 0,1)$ define
%\begin{equation}
%    \bSec_\eps := \bSec + \eps \sigma_{\min} \sign(\sigma_{\min}) \bI_k
%\end{equation}
%The matrix $\bSec_\eps$ is then invertible, real and symmetric and hence 
%\begin{equation}
%\Im((\bI - \bSec_\eps\bQ)^{-1}\bSec_\eps) 
%=\Im((\bSec_\eps^{-1} - \bQ)^{-1})
%\prec \bzero
%\end{equation}
%so that for any non-zero $\bv \in\C^k$, $\eps \mapsto \bv^* \Im(
%(\bI - \bSec_\eps\bQ)^{-1}\bSec_\eps)\bv$ is strictly negative for all $\eps \in(0,1)$.
%Furthermore, it is continuous (see for instance the explicit form in Eq.~\eqref{eq:im_invs}). So taking $\eps \to 0$ shows that $\Im(
%(\bI - \bSec\bQ)^{-1}\bSec) \preceq \bzero$.
%Finally, note that
%    $\Im(\boldeta(\bQ))= \E\left[\Im((\bI - \bSec\bQ)^{-1}\bSec)\right] \preceq \bzero.$
%\end{proof}
%
%\begin{corollary}
%For any $z \in \mathbb{H}_+,$
%\begin{equation}
%   \frac1\alpha \bI - \E_{\widehat\nu}[(\bI + \grad^2 \rho \bQ)^{-1} \grad^2 \rho \bQ] + z \bQ = 0
%\end{equation}
%has a unique solution satisfying $\Im(\bQ(z)) \succ \bzero.$
%\end{corollary}
%



%Let $\bQ \in\bbH^{-1}$.
%Let's first show that $(\bI - \bSec^{1/2}\bQ\bSec^{1/2}) $
%is invertible. 
%Namely, we show that for any $\bv \in \C^k$, $\bv^* (\bI - \bSec^{1/2}\bQ \bSec^{1/2})$
%
%
%
%
%
%Let $\bA$  denote this matrix.
%We'll show that for all non-zero $\bv \in\C^{k}$, $\bv^*\bA \bv\neq 0$.
%
%We have
%\begin{equation}
%    \Re(\bA) = \bI - \bSec^{1/2}\Re(\bQ) \bSec^{1/2},\quad
%    \Im(\bA) 
%    = - \bSec^{1/2} \Im(\bQ) \bSec^{1/2}.
%\end{equation}
%So for any $\bv \in\C^k$, we have
%\begin{equation}
%   \bv^*\bA \bv = \bv^*\Re(\bA)\bv + i \bv^*\Im(\bA)\bv= 
%   \begin{cases}
%    \norm{\bv}_2^2,  & \bSec^{1/2}\bv  = 0\\
%    a + i b & \bSec^{1/2}\bv \neq 0
%   \end{cases}
%\end{equation}
%where $a,b\in\R$ with $b>0$ since $\bu := \bSec^{1/2}\bv \neq 0$ implies
%    $\bv^* \Im(\bA) \bv = -\bu \Im(\bQ) \bu  >0$
%by assumption on $\bQ$.
%
%
%





%Once again, let's start with $k=1$. Then we have
%\begin{equation}
%\Im(\eta(q)) = \Im\left(\E\left[\frac{d}{1 - dq}\right]\right)
%= \E\left[\frac{ d^2 \Im(q)}{|1 -dq|^2}\right] < 0.
%\end{equation}
%
%Consider the imaginary part of the term inside the expectation. We have
%\begin{align}
%    \Im((\bI - \bSec \bQ )^{-1}\bSec) &= 
%    \frac1{2i}\left((\bI - \bSec \bQ )^{-1}\bSec)-
%    (\bI - \bSec \bQ^* )^{-1}\bSec)
%    \right)\\
%   & = \frac1{2i} \left((\bI-\bSec\bQ)^{-1} \bSec(\bQ - \bQ^*)(\bI-\bSec\bQ^*)^{-1} \bSec\right)\\
%   &= \left((\bI-\bSec\bQ)^{-1}\bSec\right) \Im(\bQ) \left((\bI-\bSec\bQ)^{-1} \bSec\right)^* \\
%   &\preceq \bzero.
%\end{align}
%\bns{More work to be done here. Need assumption guaranteeing that $\P(\bSec \succ \bzero)\neq 0$ and need to bound the "modulus" properly.}

%\subsection{Convergence of $\mathbf{S}_n$ to $\mathbf{S}_\star$}
%The goal of this section is to prove the following Lemma.
%\begin{claim}
%For $z \in \mathbb{H}_+$, let $\bS_\star$ be the unique solution to
%\begin{equation}
%   \frac1\alpha_n \bI - \E_\nu[(\bI + \grad^2 \rho \bS)^{-1} \grad^2 \rho \bS] + z \bS = 0.
%\end{equation}
%Then \begin{equation}
%    \norm{\bS_n - \bS_\star}_{F} \le \dots
%\end{equation}
%with probability $\dots$.
%\end{claim}
%Let us introduce the notation
%\begin{equation}
%        \bG(\bS, \bSec) := (\bI + \bSec\bS)^{-1}\bSec.
%\end{equation}
%Throughout this section, let 
%\begin{equation}
%\bA := \Re(\bS_\star),\quad\bB := \Im(\bS_\star),
%\quad\bA_n :=\Re(\bS_n),\quad\bB_n := \Im(\bS_n).
%\end{equation}
%Note that $\bB,\bB_n \succ\bzero$.
%Consider the tensor $\bT_n:\C^{k\times k}\mapsto \C^{k \times k}$
%defined by
%\begin{equation}
%    \bT(\bM) := 
%     \alpha \bF_z(\bS_\star) \E[\bG(\bS_\star,\bSec) \bM  \bG(\bS_n,\bSec) ] \bF_z(\bS_n).
%\end{equation}
%Our goal is to write 
%\begin{equation}
%    \bS_n - \bS_\star  \approx \bT(\bS_n - \bS_\star)
%\end{equation}
%then bound the operator norm $\norm{\bT}_\op := \norm{\bT}_{F\to F}$ to obtain a bound on $\norm{\bS_n - \bS_\star}_F$.
%The following observation will simplify some computations: the matrix $\bS_\star$, and hence $\bG(\bS_\star,\bSec)$ are symmetric. To see this, it's sufficient to note that  for $\Im(z)>0$, if $\Im(\bS_*)\succ \bzero$ and solves $\bF_z(\bS) = \bS$, then $\Im(\bS_*^\sT)\succ\bzero$ and also solves the fixed point equation. Uniqueness (Lemma~\notate{ref}) then proves $\bS_\star = \bS_\star^\sT$.
%
%
%Our first lemma gives a deterministic bound on $\Tr(|\bT|^p)$ for all positive integer $p$.
%\begin{lemma}
%For any $\bM \in\C^{k\times k}$ and any integer $p>0$, we have
%\begin{align}
%    \norm{\bT^p(\bM)}_F^2&\le 
%    \alpha^{2p} 
%    \norm{\bM}_F^2 
%\left(\norm{\E\left[\bB^{-1/2} \bF \bG \bB\bG^* \bF^*\bB^{-1/2}\right]}_\op^{2p-1}
%\norm{\bB}_\op
%\norm{\bB^{-1/2}\bF^* \bF \bB^{-1/2}}_\op
%\norm{
%\E[ \bG^*
%\bB\bG]}_\op
%\norm{\bB_n^{-1}}_\op\right)^{1/2}
%\nonumber\\
%&
%    \quad\left(\norm{\E\left[\bB_n^{-1/2} \bF_n \bG_n \bB_n\bG_n^* \bF_n^*\bB_n^{-1/2}\right]}_\op^{2p-1}
%\norm{\bB_n}_\op
%\norm{\bB_n^{-1/2}\bF_n^* \bF_n \bB_n^{-1/2}}_\op
%\norm{
%\E[ \bG_n^*
%\bB_n\bG_n]}_\op
%\norm{\bB^{-1}}_\op\right)^{1/2}
%\end{align}
%where 
%$\bT^p(\bA) := \bT(\bT^{p-1}(\bA))$
%and
%\begin{equation}
%\bF_n := \bF_z(\bS_n),\quad \bF:=\bF_z(\bS_\star),\quad \bG_n \equiv\bG_n(\bSec) :=  \bG(\bS_n, \bSec),\quad \bG \equiv \bG(\bSec) := \bG(\bS_\star,\bSec).
%\end{equation}
%\end{lemma}
%
%\begin{proof}
%In what follows, 
%let $\bSec_1,\dots,\bSec_p, \widetilde \bSec_1,\dots,\widetilde\bSec_p$ be i.i.d. copies of $\bSec$.
%We use the shorthand $\bG_{n,i} \equiv \bG(\bQ_n;\bSec_i)$ and 
%$\widetilde\bG_{n,i} \equiv \bG(\bQ_n;\widetilde\bSec_i)$.
%Now we write
%\bns{Fix $\bF$ to $\bF_n$. Fix $\bB$ to $\bB^{-1}$.}
%\begin{align}
%\frac1{\alpha^{2p}}\frac1k\Tr\left( |
%\bT^p(\bM)|^2\right)
%&= \frac1k \Tr\left(
%\E\left[
%\prod_{i=p}^1 (\bF \bG_i)\bM \prod_{i=1}^p(\bG_{n,i} \bF)
%\prod_{i=p}^1 ( \bF^*\widetilde \bG_{n,i}^*)
%\bM^*
%\prod_{i=1}^p ( \widetilde \bG_{i}^*\bF^*)
%\right]
%\right)
%\\
%&= 
%\frac1k 
%\E\left[
%\Tr\left(
%\bB^{1/2}\bM \prod_{i=1}^p(\bG_{n,i} \bF)
%\prod_{i=p}^1 ( \bF^*\widetilde \bG_{n,i}^* )
%\bB_n^{-1/2}\bB_n^{1/2}
%\bM^*
%\prod_{i=1}^p ( \widetilde \bG_{i}^*\bF^*)
%\prod_{i=p}^1 (\bF \bG_i)\bB^{-1/2}
%\right)
%\right]
%\\
%&\le 
%\frac1k 
%\E\left[
%\Tr\left(
%\left|
%\bB^{1/2}
%\bM \prod_{i=1}^p(\bG_{n,i} \bF)
%\prod_{i=p}^1 ( \bF^*\widetilde \bG_{n,i}^* )
%\bB_n^{-1/2}
%\right|^2\right) \right]^{1/2}\\
%&\hspace{10mm}
%\E\left[\Tr\left(
%\left|
%\bB_n^{1/2}
%\bM^*
%\prod_{i=1}^p ( \widetilde \bG_{i}^*\bF^*)
%\prod_{i=p}^1 (\bF \bG_i)
%\bB^{-1/2}
%\right|^2
%\right)
%\right]^{1/2}
%\end{align}
%where the inequality follows by H\"older for $L_p(S^p)$ norms.
%We bound the second expectation as 
%\begin{align}
%&\E\left[\Tr\left(
%\left|
%\bB_n^{1/2}
%\bM^*
%\prod_{i=1}^p ( \widetilde \bG_{i}^*\bF^*)
%\prod_{i=p}^1 (\bF \bG_i)
%\bB^{-1/2}
%\right|^2
%\right)
%\right]\\
%&=
%\E\left[\Tr\left(
%\bB_n^{1/2}
%\bM^*
%\prod_{i=1}^p ( \widetilde \bG_{i}^*\bF^*)
%\prod_{i=p}^1 (\bF \bG_i)
%\bB^{-1}
%\prod_{i=1}^p (\bG_i^*\bF^* )
%\prod_{i=p}^1 (\bF \widetilde \bG_{i})
%\bM
%\bB_n^{1/2}
%\right)
%\right]\\
%&\le
%\E\left[\Tr\left(
%\bB_n^{1/2}
%\bM^*
%\prod_{i=1}^p ( \widetilde \bG_{i}^*\bF^*)
%\bB^{-1}
%\prod_{i=p}^1 (\bF\widetilde \bG_{i})
%\bM
%\bB_n^{1/2}
%\right)
%\right]
%\norm{\E\left[\bB^{1/2} \bF \bG \bB^{-1}\bG^* \bF^*\bB^{1/2}\right]}_\op^p\\
%&\le
%\E\left[\Tr\left(
%\bB_n^{1/2}
%\bM^*
%\widetilde \bG_1^*
%\prod_{i=2}^p (\bF^* \widetilde \bG_{i}^*)
%\bB^{-1}
%\prod_{i=p}^2 (\widetilde \bG_{i}\bF) \widetilde \bG_1
%\bM
%\bB_n^{1/2}
%\right)
%\right]
%\norm{\E\left[\bB^{1/2} \bF \bG \bB^{-1}\bG^* \bF^*\bB^{1/2}\right]}_\op^p
%\norm{\bB^{-1}}_\op
%\norm{\bB^{1/2}\bF^* \bF \bB^{1/2}}_\op
%\\
%&\le 
%\E\left[\Tr\left(
%\bB_n^{1/2}
%\bM^*
%\widetilde \bG_1^*
%\bB^{-1}
%\widetilde\bG_1
%\bM
%\bB_n^{1/2}
%\right)
%\right]
%\norm{\E\left[\bB^{1/2} \bF \bG \bB^{-1}\bG^* \bF^*\bB^{1/2}\right]}_\op^{2p-1}
%\norm{\bB^{-1}}_\op
%\norm{\bB^{1/2}\bF^* \bF \bB^{1/2}}_\op\\
%&\le 
%\E\left[\Tr\left(
%\bM^*
%\bM
%\right)
%\right]
%\norm{\E\left[\bB^{1/2} \bF \bG \bB^{-1}\bG^* \bF^*\bB^{1/2}\right]}_\op^{2p-1}
%\norm{\bB^{-1}}_\op
%\norm{\bB^{1/2}\bF^* \bF \bB^{1/2}}_\op
%\norm{
%\E[ \bG^*
%\bB^{-1}\bG]}_\op
%\norm{\bB_n}_\op.
%\end{align}
%A similar computation gives the bound on the first expectation 
%\begin{align}
%    &\E\left[
%\Tr\left(
%\left|
%\bB^{1/2}
%\bM \prod_{i=1}^p(\bG_{n,i} \bF)
%\prod_{i=p}^1 ( \bF^*\widetilde \bG_{n,i}^* )
%\bB_n^{-1/2}
%\right|^2\right) \right]\\
%&\le
%\E\left[\Tr\left(
%\bM^*
%\bM
%\right)
%\right]
%\norm{\E\left[\bB_n^{1/2} \bF_n \bG_n \bB_n^{-1}\bG_n^* \bF_n^*\bB_n^{1/2}\right]}_\op^{2p-1}
%\norm{\bB_n^{-1}}_\op
%\norm{\bB_n^{1/2}\bF_n^* \bF_n \bB_n^{1/2}}_\op
%\norm{
%\E[ \bG_n^*
%\bB_n^{-1}\bG_n]}_\op
%\norm{\bB}_\op
%\end{align}
%where we used that all matrices involved are symmetric.
%\end{proof}
%
%
%\begin{lemma}
%    We have the bounds
%    \begin{equation}
%\norm{\E\left[\bB^{-1/2} \bF \bG \bB\bG^* \bF^*\bB^{-1/2}\right]}_\op \le 
%1 -\frac{\alpha \Im(z)}{2} \lambda_{\min}(\bB^{-1/2} \bS_\star \bS_\star^* \bB^{-1/2}),
%\quad
%\norm{\E[ \bG^*
%\bB\bG]}_\op \le
%    \norm{\bS_\star^{-1}\bB\bS_\star^{*-1}}_\op.
%    \end{equation}
%and 
%    \begin{equation}
%\frac1\alpha\norm{\E\left[\bB_n^{-1/2} \bF_n \bG_n \bB_n\bG_n^* \bF_n^*\bB_n^{-1/2}\right]}_\op 
%    \le \bI -  \frac{\Im(z)}{\alpha}\lambda_{\min} \left(\bB_n^{-1/2} \bF_n\bF_n^* \bB_n^{-1/2}\right) 
%,\quad
%\norm{\E[ \bG_n^*
%\bB_n\bG_n]}_\op \le \norm{\Im(\bF_n^{-1})}_\op
%    \end{equation}
%with probability
%\begin{equation}
%\dots
%\end{equation}
%for $n> n_0(z)$ where ..
%\end{lemma}
%
%\begin{proof}
%By Lemma~\ref{lemma:re_im_properties}, we have
%$\Im(\bG_\star) = -\bG_\star\bB \bG_\star^*$, and $\Im(\bS_\star^{-1}) = - \bS_\star^{-1} \bB\bS_{\star}^{*-1}.$
%So rewriting the fixed point for $\bS_\star$ as
%    $z\bI = \E[\bG_\star] - \frac1\alpha \bS_\star$ and taking the imaginary parts
%gives
%\begin{equation}
%\bzero\prec \Im(z) \bI = - \E[\bG_\star \bB \bG_\star^*] + \frac1\alpha\bS_\star^{-1}\bB\bS_\star^{*-1}.
%\end{equation}
%This implies
%\begin{equation}
%    \alpha \bB^{-1/2} \bS_\star\E[\bG_\star \bB\bG_\star^*] \bS_\star^* \bB^{-1/2} \prec 
%    \bI - \frac{\alpha\Im(z)}{2} \bB^{-1/2}\bS_\star\bS^*_\star \bB^{-1/2},\quad
%   \alpha \E[\bG_\star\bB\bG_\star^*]  \preceq \bS_\star^{-1}\bB\bS_\star - \Im(z)\bI,
%\end{equation}
%giving the bounds desired.
%%\begin{equation}
%%    \norm{
%%\alpha \bB^{-1/2} \bS_\star\E[\bG_\star \bB\bG_\star^*] \bS_\star^* \bB^{-1/2} }_\op < 1 -\frac{\alpha \Im(z)}{2} \lambda_{\min}(\bB^{-1/2} \bS_\star \bS_\star^* \bB^{-1/2}),
%%\end{equation}
%
%For the remaining two bounds, once again let us write, by definition of $\bF_n$,
%   $z\bI = \E[\bG_n] - \bF_n^{-1},$
%which then gives, after multiplying to the left by $\bF_n$ and to the right by $\bF_n^*$,
%\begin{equation}
%    z \bF_n \bF_n^* =  \E[\bF_n \bG_n \bF_n^*] - \bF_n^*.
%\end{equation}
%Taking the imaginary part using Lemma~\ref{lemma:re_im_properties} then gives
%\begin{equation}
%    \Im(z) \bF_n \bF_n^* = - \E[\bF_n \bG_n\bB_n \bG_n^* \bF_n^*]  + \Im ( \bF_n)
%\end{equation}
%so that
%\begin{equation}
%    \E[\bF_n \bG_n \bB_n \bG_n^* \bF_n^*] = \Im(\bF_n - \alpha\bQ_n)  + \alpha\bB_n - \Im(z) \bF_n\bF_n^*.
%\end{equation}
%Letting $\bE_n := (\alpha^{-1}\bI  -\bF_n^{-1} \bQ_n)$,
%\begin{align}
%    \norm{\Im(  \bF_n^{-1} (\bF_n - \alpha\bQ_n) \bF_{n}^{*-1})}_\op
%    &=
%   \alpha \norm{\Im(   \bE_n \bF_{n}^{*-1})}_\op
%   \le  
%    \alpha\norm{\bE_n}_\op\norm{\bF_n^{-1}}_\op
%    \le \norm{\bE_n}_\op \norm{\bQ_n^{-1}}_\op \norm{\bI - \bE_n}_\op\\
%    &\le \frac2{\Im(z)} \left(\frac{1}{n^2}\norm{\bH}_\op^2  + |z|^2\right) \omega_{\textrm{FP}}(z;n,k) (1 + \omega_{\textrm{FP}}(z; n, k))
%\end{align}
%and taking $n \le n_0(z)$ so that 
%\bns{Fix this}
%\begin{equation}
%     \frac{10(\sfK^2 + |z|)}{\alpha\Im(z)^2}\omega_{\textrm{FP}}(z;n,k)(1 + \omega_{\textrm{FP}}(z;n,k))  \le  1,
%\end{equation}
%we have on the event $\cG_0$ of Lemma~\notate{ref} that
%\begin{equation}
%    \frac1{\alpha}\norm{\E[\bB_n^{-1/2}\bF_n \bG_n \bB_n \bG_n^* \bF_n^* \bB_n^{-1/2}]  }_\op
%    \le \bI -  \frac{\Im(z)}{\alpha}\lambda_{\min} \left(\bB_n^{-1/2} \bF_n\bF_n^* \bB_n^{-1/2}\right),
%\end{equation}
%and similarly,
%\begin{equation}
%\norm{\E[\bG_n \bB_n\bG_n^*]}_\op \le \norm{\Im(\bF_n^{-1})}_\op.
%\end{equation}
%\end{proof}


\subsection{Uniqueness of \texorpdfstring{$\mathbf{S}_\star$}{S*} and convergence of \texorpdfstring{$\mathbf{S}_n$}{Sn} to \texorpdfstring{$\mathbf{S}_\star$}{S*}}
\label{sec:UniquenessSstar}
The goal of this section is to show that the asymptotic Stieltjes transform defined 
in Section \ref{sec:AsymptoticST} is the unique solution of the fixed point equation of Eq.~\eqref{eq:fp_eq}, and to derive a bound on the difference of this quantity and the empirical Stieltjes transform.
%\begin{claim}
%For $z \in \mathbb{H}_+$, there exists a unique solution $\bS_\star$
%\begin{equation}
%   \frac1\alpha_n \bI - \E_\nu[(\bI + \grad^2 \rho \bS)^{-1} \grad^2 \rho \bS] + z \bS = 0,
%\end{equation}
%and
%\begin{equation}
%    \norm{\bS_n - \bS_\star}_{F} \le \dots
%\end{equation}
%with probability $\dots$.
%\end{claim}
Our approach is to study a certain linear operator whose invertibility implies the uniqueness of the solution of~\eqref{eq:fp_eq}.
To define this operator, first introduce the notation
\begin{equation}
        \boldeta(\bS, \bW) := (\bI + \bW\bS)^{-1}\bW.
\end{equation}
%Throughout this section, let 
%\begin{equation}
%\bA_\star := \Re(\bS_\star),\quad\bB := \Im(\bS_\star),
%\quad\bA_n :=\Re(\bS_n),\quad\bB_n := \Im(\bS_n).
%\end{equation}
For a given $\bS\in\C^{k\times k}$ with $\Im(\bS)\succeq \bzero$, $z\in\bbH_+$ $\alpha >1, \nu\in\cuP(\R^{k+k_0+1})$,
define $\bT_\bS(\;\cdot\;; z,\alpha, \nu):\C^{k\times k}\to \C^{k \times k}$ 
\begin{equation}
\label{eq:def_T}
    \bT_\bS(\bDelta;
    z,\alpha,\nu)
    := \bF_z(\bS_\star;\nu) \E[\boldeta(\bS_\star,\bW) \bDelta  \boldeta(\bS,\bW) ] \bF_z(\bS;\nu),
\end{equation}
where
$\bS_\star = \bS_\star(z; \alpha,\nu)$,
%\begin{equation}
%\bS_\star :=\begin{cases}
%     \bS_\star(z; \alpha,\nu) & \Im(z) > 0\\
%     \lim_{\eps \to0}
%         \bS_\star(z + i\eps; \alpha,\nu) & \Im(z) = 0
%    \end{cases}.
%\end{equation}
%whenever
%the right-hand-side is finite (notice that it will be finite for any $z\in\bbH_+$ and $\Im(\bS) \succ\bzero$, which is the setting of this section. We state it more generally for its utility later on in the analysis.)
Now, the significance of this operator is highlighted by the following relation: 
letting $\bS_\star$ be as defined above and suppressing the dependence on $z,\alpha,\nu$,
we have for any $\bS \in \bbH_+^k$
\begin{align*}
   \bF_z(\bS_\star) - \bF_z(\bS) &=
        \left(\E[\bfeta(\bS_\star,\bW)] - z\bI\right)^{-1}
        -
        \left(\E[\bfeta(\bS, \bW)] - z\bI\right)^{-1}\\
        &=
        -\left(\E[\bfeta(\bS_\star,\bW)] - z\bI\right)^{-1}
      \E[\bfeta(\bS_\star, \bW) - 
      \bfeta(\bS,\bW)] 
        \left(\E[\bfeta(\bS, \bW)] - z\bI\right)^{-1}\\
        &=
        \bF_z(\bS_\star) 
      \E[\bfeta(\bS_\star, \bW)(\bS_\star - \bS) 
      \bfeta(\bS,\bW))] 
     \bF_z(\bS) \\
   &=\bT_{\bS}(\bS_\star - \bS).
\end{align*}
Summarizing, we have
\begin{align}
\label{eq:relation_F_T}
   \bF_z(\bS_\star) - \bF_z(\bS) =\bT_{\bS}(\bS_\star - \bS).
\end{align}

To prove uniqueness of $\bS_\star$, we'll consider $\bS_0$ to be any solution of $\alpha_n \bS = \bF_z(\bS)$ in $\bbH_+^k$ and show that $\bT_0 := \bT_{\bS_0}$ has a convergent Neumann series. 
Similarly we will derive the rate of convergence of $\bS_n$ to $\bS_\star$ by bounding the norm of $(\id - \bT_{\bS_n})^{-1}$.

Our first lemma of this section gives a deterministic bound on $\norm{\bT_\bS^p}_{\op\to\op}$ for all positive integer $p$, which will later allow us to assert the convergence of the Neumann series of $\bT_\bS$.
Here, $\bT^p$ is the $p$-fold composition of $\bT$,
namely $\bT^p(\bA) = \bT(\bT^{p-1}(\bA))$.
\begin{lemma}
\label{lemma:op_norm_bound_power_T}
Fix $\bS\in\C^{k\times k}$ with $\Im(\bS)\succeq \bzero,$ $\Im(z) \ge 0,\alpha >1$, and $\nu\in\cuP(\R^{k+k_0+1})$ such that $\bT_\bS(\;\cdot\;, z,\alpha,\nu)$ of Eq.~\eqref{eq:def_T} is defined.
We have for any
$\bB,\bB_\star \succ \bzero$,
and integer $p>0$, we have
\begin{align}
    \norm{\bT_\bS^p}_{\op \to\op}&\le 
    \left(
 \norm{\E[\bB^{-1/2} \bF\bfeta \bB\bfeta^* \bF^* \bB^{-1/2}]}_\op^p
\norm{\bB^{-1}}_\op
\norm{\bB}_\op
\right)^{1/2}
\nonumber
\\
&\hspace{5cm}\left(
 \norm{\E[\bB_\star^{-1/2} \bF_\star\bfeta_\star \bB_\star\bfeta_\star^* \bF_\star^* \bB_\star^{-1/2}]}_\op^{p}
\norm{\bB_\star^{-1}}_\op
\norm{\bB_\star}_\op
\right)^{1/2},
\nonumber
\end{align}
where $\bS_\star$ is as defined in Eq.~\eqref{eq:def_T} and 
\begin{equation}
\bF_\star := \bF_z(\bS_\star;\nu),\quad \bF:=\bF_z(\bS;\nu),\quad \bfeta_\star  :=\bfeta(\bS_\star, \bW),\quad \bfeta := \bfeta(\bS,\bW).
%,\quad\bB := \Im(\bS),\quad\bB_\star:=\Im(\bS_\star).
\end{equation}

\end{lemma}

\begin{proof}
In what follows, 
let $\bW_1,\dots,\bW_p, \widetilde \bW_1,\dots,\widetilde\bW_p$ be i.i.d. copies of $\bW$.
We use the shorthand $\bfeta_{\star,i} \equiv \bfeta(\bS_\star;\bG_i)$ and 
$\widetilde\bfeta_{\star,i} \equiv \bfeta(\bS_\star;\widetilde\bG_i)$. Similarly define $\bfeta_i, \widetilde\bfeta_i$ for $\bS$ replacing $\bS_\star$.
Fix any $\bv,\bu\in \C^{k}$ and $\bDelta \in \C^{k\times k}$ and write
\begin{align*}
\left|\bv^* \bT^p(\bDelta) \bu\right|^2
&=  \bu^* \bT^p(\bDelta)^* \bv \bv^*  \bT^p(\bDelta) \bu\\
&=  \Tr\left(
\bu^*\E\left[
\prod_{i=p}^1 (\bF \bfeta_i)\bDelta \prod_{i=1}^p(\bfeta_{\star,i} \bF_\star)
\bv\bv^*
\prod_{i=p}^1 ( \bF_\star^*\widetilde \bfeta_{\star,i}^*)
\bDelta^*
\prod_{i=1}^p ( \widetilde \bfeta_{i}^*\bF^*)
\right]\bu
\right)
\\
&= 
\E\left[
\Tr\left(
\bDelta \prod_{i=1}^p(\bfeta_{\star,i} \bF_\star)
\bv\bv^*
\prod_{i=p}^1 ( \bF_\star^*\widetilde \bfeta_{\star,i}^* )
\bDelta^*
\prod_{i=1}^p ( \widetilde \bfeta_{i}^*\bF^*)
\bu\bu^*
\prod_{i=p}^1 (\bF \bfeta_i)
\right)
\right]
\\
&\le 
\E\left[
\Tr\left(
\left|
\bDelta \prod_{i=1}^p(\bfeta_{\star,i} \bF_\star)
\bv\bv^*
\prod_{i=p}^1 ( \bF_\star^*\widetilde \bfeta_{\star,i}^* )
\right|^2\right) \right]^{1/2}\\
&\hspace{10mm}
\E\left[\Tr\left(
\left|
\bDelta^*
\prod_{i=1}^p ( \widetilde \bfeta_{i}^*\bF^*)\bu\bu^*
\prod_{i=p}^1 (\bF \bfeta_i)
\right|^2
\right)
\right]^{1/2}
\end{align*}
where the inequality follows by Cauchy-Schwarz for (random) 
matrices.
We bound the second expectation as 
\begin{align*}
&\E\left[\Tr\left(
\left|
\bDelta^*
\prod_{i=1}^p ( \widetilde \bfeta_{i}^*\bF^*)
\bu\bu^*
\prod_{i=p}^1 (\bF \bfeta_i)
\right|^2
\right)
\right]\\
&=
\E\left[\Tr\left(
\bDelta^*
\prod_{i=1}^p ( \widetilde \bfeta_{i}^*\bF^*)
\bu\bu^*
\prod_{i=p}^1 (\bF \bfeta_i)
\prod_{i=1}^p (\bfeta_i^*\bF^* )
\bu\bu^*
\prod_{i=p}^1 (\bF \widetilde \bfeta_{i})
\bDelta
\right)
\right]\\
&=
\E\left[
\bu^*
\prod_{i=p}^1 (\bF \widetilde \bfeta_{i})
\bDelta
\bDelta^*
\prod_{i=1}^p ( \widetilde \bfeta_{i}^*\bF^*)
\bu
\right]
\E\left[
\bu^*
\prod_{i=p}^1 (\bF \bfeta_i)
\prod_{i=1}^p (\bfeta_i^*\bF^* )
\bu
\right]
\\
&\le
\E\left[
\bu^*
\prod_{i=p}^1 (\bF \widetilde \bfeta_{i})
\bDelta
\bDelta^*
\prod_{i=1}^p ( \widetilde \bfeta_{i}^*\bF^*)
\bu
\right]
\norm{\bB^{-1}}_\op \norm{\E[\bB^{-1/2} \bF\bfeta \bB\bfeta^* \bF^* \bB^{-1/2}]}_\op^p
\norm{\bB}_\op \norm{\bu}_2^2\\
&\le
\norm{\bDelta}_\op^2
 \norm{\E[\bB^{-1/2} \bF\bfeta \bB\bfeta^* \bF^* \bB^{-1/2}]}_\op^{2p}
\norm{\bB^{-1}}_\op^2
\norm{\bB}_\op^2 \norm{\bu}_2^4,
\end{align*}
where the last two inequalities can be proven by induction over $p$.
A similar computation gives the bound on the first expectation 
\begin{align*}
    \E\left[
\Tr\left(
\left|
\bDelta \prod_{i=1}^p(\bfeta_{\star,i} \bF_\star)
\bv\bv^*
\prod_{i=p}^1 ( \bF_\star^*\widetilde \bfeta_{\star,i}^* )
\right|^2\right) \right]
\le
\norm{\bDelta}_\op^2
 \norm{\E[\bB_\star^{-1/2} \bF_\star\bfeta_\star \bB_\star\bfeta_\star^* \bF_\star^* \bB_\star^{-1/2}]}_\op^{2p}
\norm{\bB_\star^{-1}}_\op^2
\norm{\bB_\star}_\op^2 \norm{\bv}_2^4.
\end{align*}
Taking supremum over $\bv,\bu$ of unit norm gives that
\begin{align*}
    \frac{\norm{\bT^p(\bDelta)}_\op}{\norm{\bDelta}_\op} &\le
    \Bigg(
 \norm{\E[\bB_\star^{-1/2} \bF_\star\bfeta_\star \bB_\star\bfeta_\star^* \bF_\star^* \bB_\star^{-1/2}]}_\op^p
\norm{\bB_\star^{-1}}_\op
\norm{\bB_\star}_\op\\
&\hspace{60mm}\cdots\norm{\E[\bB^{-1/2} \bF\bfeta \bB\bfeta^* \bF^* \bB^{-1/2}]}_\op^{p}
\norm{\bB^{-1}}_\op
\norm{\bB}_\op
\Bigg)^{1/2}
\end{align*}
for all $\bDelta$. Taking supremum over $\norm{\bDelta}_\op = 1$ gives the result.
\end{proof}

\begin{lemma}
\label{lemma:op_norm_bound_for_sols_fp}
Fix $z \in\bbH_+$, $\hnu \in \cuP_n(\R^{k+k_0+1})$.
Let $\bS_0$ be any solution of $\alpha_n \bS = \bF_z(\bS;\hnu)$ in    $\bbH_+^k$, and let $\bS_n(z;\hnu)$ be the quantity defined in Eq.~\eqref{eq:Sn_def}.
Use the notation
\begin{equation}
\bF_0 := \bF_z(\bS_0;\hnu),\quad \bF_n:=\bF_z(\bS_n;\hnu),\quad \bfeta_0  :=\bfeta(\bS_0, \bW),\quad \bfeta_n := \bfeta(\bS_n,\bW),\quad\bB_n := \Im(\bS_n),\quad\bB_0:=\Im(\bS_0),
\end{equation}
where $\bW\sim \grad^2\ell_{\# \hnu}$.
Then we have the bound
    \begin{equation}
    \nonumber
\frac1\alpha_n\norm{\E\left[\bB_0^{-1/2} \bF_0 \bfeta_0 \bB_0\bfeta_0^* \bF_0^*\bB_0^{-1/2}\right]}_\op \le 
1 -\frac{\alpha_n \Im(z)}{2} \lambda_{\min}(\bB_0^{-1/2} \bS_0 \bS_0^* \bB_0^{-1/2}).
    \end{equation}
    
Further, if 
\begin{equation}
\label{eq:n_n(z)}
     \frac{10(\sfK^2 + |z|^2)}{\Im(z)^2}\Err_{\FP}(z;n,k)(1 + \alpha_n\Err_{\FP}(z;n,k))  \le  \frac12
 \end{equation}
then the following holds,
for any $L\ge 1$,
on the event $\Omega_0\cap\Omega_1(L)$  of 
Lemmas \ref{lemma:standard_norm_bounds},
\ref{lemma:concentration_loo_quad_form}
    \begin{equation}
\frac1\alpha_n\norm{\E\left[\bB_n^{-1/2} \bF_n \bfeta_n \bB_n\bfeta_n^* \bF_n^*\bB_n^{-1/2}\right]}_\op 
    \le 1 -  \frac{\Im(z)}{2\alpha_n}\lambda_{\min} \left(\bB_n^{-1/2} \bF_n\bF_n^* \bB_n^{-1/2}\right)\, .
    \label{eq:SecondBoundFp}
    \end{equation}
    \end{lemma}

\begin{proof}
By Lemma~\ref{lemma:re_im_properties}, we have
$\Im(\bfeta_0) = -\bfeta_0\bB_0 \bfeta_0^*$, and $\Im(\bS_0^{-1}) = - \bS_0^{-1} \bB_0\bS_{0}^{*-1}.$
So rewriting the fixed point for $\bS_0$ as
    $z\bI = \E[\bfeta_0] - \alpha_n^{-1} \bS_0^{-1}$ and taking the imaginary parts
gives
\begin{equation}
\nonumber
\bzero\prec \Im(z) \bI = - \E[\bfeta_0 \bB_0 \bfeta_0^*] + \frac1\alpha_n\bS_0^{-1}\bB_0\bS_0^{*-1}.
\end{equation}
This implies
\begin{equation}
\nonumber
    \alpha_n \bB_0^{-1/2} \bS_0\E[\bfeta_0 \bB_0\bfeta_0^*] \bS_0^* \bB^{-1/2} \prec 
    \bI - \frac{\alpha_n\Im(z)}{2} \bB_0^{-1/2}\bS_0\bS^*_0 \bB_0^{-1/2}
\end{equation}
giving the first bound after substituiting $\bF_0 = \alpha_n 
\bS_0$.
%\begin{equation}
%    \norm{
%\alpha \bB^{-1/2} \bS_\star\E[\bG_\star \bB\bG_\star^*] \bS_\star^* \bB^{-1/2} }_\op < 1 -\frac{\alpha \Im(z)}{2} \lambda_{\min}(\bB^{-1/2} \bS_\star \bS_\star^* \bB^{-1/2}),
%\end{equation}

For the bound \eqref{eq:SecondBoundFp}, once again let us write by definition of $\bF_n$,
   $z\bI = \E[\bfeta_n] - \bF_n^{-1}$
which gives after multiplying to the left by $\bF_n$ and to the right by $\bF_n^*$
\begin{equation}
\nonumber
    z \bF_n \bF_n^* =  \E[\bF_n \bfeta_n \bF_n^*] - \bF_n^*.
\end{equation}
Taking the imaginary part using Lemma~\ref{lemma:re_im_properties} then gives
\begin{equation}
\nonumber
    \Im(z) \bF_n \bF_n^* = - \E[\bF_n \bfeta_n\bB_n \bfeta_n^* \bF_n^*]  + \Im ( \bF_n)
\end{equation}
so that
\begin{equation}\label{eq:Fn-relation}
    \E[\bF_n \bfeta_n \bB_n \bfeta_n^* \bF_n^*] = \Im(\bF_n - \alpha_n\bS_n)  + \alpha_n\bB_n - \Im(z) \bF_n\bF_n^*.
\end{equation}
Letting $\bE_n := (\alpha_n^{-1}\bI  -\bF_n^{-1} \bS_n)$,
\begin{align*}
    \norm{\Im(  \bF_n^{-1} (\bF_n - \alpha_n\bS_n) \bF_{n}^{*-1})}_\op
    &=
   \alpha_n \norm{\Im(   \bE_n \bF_{n}^{*-1})}_\op
   \le  
    \alpha_n\norm{\bE_n}_\op\norm{\bF_n^{-1}}_\op\\
    &
    \le \norm{\bE_n}_\op \norm{\bS_n^{-1}}_\op \norm{\bI - \alpha_n\bE_n}_\op\\
    &\stackrel{(a)}\le \frac1{\Im(z)} \left(\frac{1}{n^2}\norm{\bH}_\op^2  + |z|^2\right) \Err_{\FP}(z;n,k) (1 + \alpha_n\Err_{\FP}(z; n, k))\\
    &\stackrel{(b)}\le \frac{10}{\Im(z)} \left(\sfK^2  + |z|^2\right) \Err_{\FP}(z;n,k) (1 + \alpha_n\Err_{\FP}(z; n, k))\\
    &\stackrel{(c)}\le \frac{\Im(z)}{2}
\end{align*}
on $\Omega_0 \cap\Omega_1(L)$,
where $(a)$ follows from Lemma~\ref{lemma:fix_point_rate}
and Lemma \ref{lemma:as_norm_bounds}, $(b)$ follows from Lemma~\ref{lemma:standard_norm_bounds}, and $(c)$ follows from the assumption in Eq.~\eqref{eq:n_n(z)}.
We conclude that
\begin{equation}
\nonumber
    \Im(\bF_n - \alpha_n\bS_n) - \frac{\Im(z)}{2}\bF_n\bF_n^* \preceq \bzero\, ,
\end{equation}
and therefore, using Eq.~\eqref{eq:Fn-relation},
and the fact that $\Im(z)>0$,
\begin{equation}
\nonumber
    \frac1{\alpha_n}\norm{\E[\bB_n^{-1/2}\bF_n \bfeta_n \bB_n \bfeta_n^* \bF_n^* \bB_n^{-1/2}]  }_\op
    \le 1 -  \frac{\Im(z)}{2\alpha_n}\sigma_{\min} \left(\bB_n^{-1/2} \bF_n\bF_n^* \bB_n^{-1/2}\right)
\end{equation}
as desired.
\end{proof}


\subsubsection{Uniqueness of \texorpdfstring{$\mathbf{S_\star}$}{S*}}
\label{app:sec:UniquenessSstar}
We are now ready to prove uniqueness of the solution $\bS_\star$.
\begin{lemma}
\label{lemma:uniqueness_ST}
    For any $z\in\bbH_+$, $\alpha_0 >1$, $\nu\in\cuP(\R^{k+k_0+1})$, the solution $\bS_\star(z;\alpha_0,\nu)$ of Eq.~\eqref{eq:def_S_star} is the unique solution to $\alpha_0 \bS = \bF_z(\bS;\nu)$ on $\bbH_+^k$.
\end{lemma}
\begin{proof}


Let $\bS_0 \in\bbH_+^k$ be any solution to this fixed point equation.  Then by Eq.~\eqref{eq:relation_F_T}
    \begin{equation}
\label{eq:diff_two_sols}
0 = \bS_0 - \bS_\star - \frac1{\alpha_0} (\bF_z(\bS_0) -\bF_z(\bS_\star)) = \left(\id - \frac1{\alpha_0} \bT_{\bS_0}\right)\left(\bS_0 -\bS_\star\right).
    \end{equation}
So to conclude uniqueness, it's sufficient to show that $\left(\id - \alpha_0^{-1} \bT_{\bS_0}\right)$ is invertible.
Using Lemmas~\ref{lemma:op_norm_bound_power_T} and~\ref{lemma:op_norm_bound_for_sols_fp}, let
\begin{equation}
\nonumber
    \delta_0 := \frac{\alpha_0\Im(z)}{2} \lambda_{\min}(\bB_0^{-1/2}\bS_0 \bS_0^* \bB_0^{-1/2}),
    \quad
    \delta_\star := \frac{\alpha_0\Im(z)}{2} \lambda_{\min}(\bB_\star^{-1/2}\bS_\star \bS_\star^* \bB_\star^{-1/2}).
\end{equation}
Since $\bS_0,\bS_\star \in\bbH_+^k$, we have
by Lemma~\ref{lemma:re_im_properties} that $\delta_0,\delta_\star > 0$.
Hence by Lemmas~\ref{lemma:op_norm_bound_power_T} and~\ref{lemma:op_norm_bound_for_sols_fp} 
\begin{align*}
    \norm{\sum_{p}\alpha_0^{-p} \bT_{\bS_0}^p}_{\op\to\op}  &\le \sum_{p} (1-\delta_0)^{p/2}(1-\delta_\star)^{p/2} \left(\norm{\bB_0}_\op \norm{\bB_0^{-1}}_\op  \norm{\bB_\star}_\op \norm{\bB_\star^{-1}}_\op\right)^{1/2} \\&\le 
    \left( \frac1{\delta_0 \delta_\star}\norm{\bB_0}_\op \norm{\bB_0^{-1}}_\op  \norm{\bB_\star}_\op \norm{\bB_\star^{-1}}_\op\right)^{1/2}  
\end{align*}
%The latter quantity is bounded by Lemma~\ref{lemma:re_im_properties} and since $\bS_0,\bS_\star \in\bbH_+^k$. 
implying convergence of the Neumann series, and in turn, the desired invertibility.
\end{proof}

\subsubsection{Rate of convergence}

\begin{lemma}
\label{lemma:rate_matrix_ST}
Whenever
\begin{equation}
\label{eq:n_n_0_2}
     \frac{10(\sfK^2 + |z|^2)}{\Im(z)^2}\Err_{\FP}(z;n,k)(1 + \alpha_n\Err_{\FP}(z;n,k))  \le  \frac1{2\alpha_n},
 \end{equation}
we have 
\begin{equation}
\nonumber
    \sup_{\hnu \in\cuP_n(\R^{k+k_0+1})}\norm{\bS_\star(z;\alpha_n, \hnu) -\bS_n(z;\hnu)}_\op \le C(\sfK)
    %(\alpha_n+\alpha_n^{-1})^2
    \frac{1 + |z|^4}{\Im(z)^5} 
\Err_{\FP}(z; n, k)
\end{equation}
on the event $\Omega_0 \cap\Omega_1(L)$ of Lemmas \ref{lemma:standard_norm_bounds},
\ref{lemma:concentration_loo_quad_form}, for $L\ge 1$.
\end{lemma}
\begin{proof}
Recalling (for $\bF_{\star} = \bF_z(\bS_{\star})$,
$\bF_{n} = \bF_z(\bS_{n})$) that $\alpha_n^{-1}\bF_\star = \bS_\star$ and 
$\alpha_n^{-1}\bF_n = \bS_n +\bF_n\bE_n$ where $\bE_n := \alpha_n^{-1}\bI - \bF_n^{-1}\bS_n$,
we have by Eq.~\eqref{eq:relation_F_T},
\begin{equation}
    \bS_\star - \bS_n =
    \alpha_n^{-1}(\bF_\star - \bF_n)  + \bF_n\bE_n =
    \frac1\alpha_n \bT_{\bS_n}(\bS_\star -\bS_n) +  \bF_n\bE_n.
\end{equation}
Letting
\begin{equation}
\nonumber
    \delta_\star := \frac{\alpha_n\Im(z)}{2} \lambda_{\min}(\bB_\star^{-1/2} \bS_\star\bS_\star^* \bB_\star^{-1/2}),\quad
    \delta_n := \frac{\Im(z)}{2\alpha_n} \lambda_{\min}(\bB_n^{-1/2} \bF_n\bF_n^* \bB_n^{-1/2}),
\end{equation}
an argument similar to that of Lemma~\ref{lemma:uniqueness_ST} (making use of Lemmas~\ref{lemma:op_norm_bound_power_T} and~\ref{lemma:op_norm_bound_for_sols_fp}) 
implies that $(\id- \alpha_n^{-1}\bT_{\bS_n})$ is invertible and so
\begin{align}
\nonumber
    \norm{\bS_* - \bS_n}_\op 
    &= \norm{\left(\bI - \alpha_n^{-1}\bT_{\bS_n}\right)^{-1} \bF_n \bE_n}\\
\nonumber
    %&\le\norm{\sum_{p=0}^\infty \frac1{\alpha^p}\bT^p  }_\op \norm{\bF_n}_\op\norm{\bE_n}_\op\\
    &\le \sum_{p=0}^\infty \alpha_n^{-p}\norm{\bT_{\bS_n}^p}_{\op\to\op} \norm{\bF_n\bE_n}_\op\\
\nonumber
    &\le \sum_{p=0}^\infty (1-\delta)^{p/2}(1-\delta_n)^{p/2} 
    \left(\norm{\bB_n^{-1}}_\op \norm{\bB_n}_\op 
    \norm{\bB}_\op \norm{\bB^{-1}}_\op\right)^{1/2} \norm{\bF_n}_\op\norm{\bE_n}_\op \\
    &\le \left(
    \frac{1}{\delta}
    \frac{1}{\delta_n}
    \norm{\bB_n^{-1}}_\op \norm{\bB_n}_\op 
    \norm{\bB}_\op \norm{\bB^{-1}}_\op
    \right)^{1/2}
    \norm{\bF_n}_\op \norm{\bE_n}_\op.
    \label{eq:S_Sn_rate_expansion}
\end{align}

Now we collect the bounds appearing on the right-hand side of this equation. On the event $\Omega_0$ of Lemma~\ref{lemma:standard_norm_bounds}, we have Lemma~\ref{lemma:as_norm_bounds} and  Corollary~\ref{cor:S_star_min_singular_value_bound}
(recall that $\bB_n=\Im(\bS_n)$, $\bB_{\star}=\Im(\bS_{\star})$): 
\begin{equation}
\nonumber
    \norm{\bB_n^{-1}}_\op \le  \frac{C_1}{\Im(z)} \left( \sfK^2 + |z|^2 \right),\quad\textrm{and}\quad
    \norm{\bB_\star^{-1}}_\op \le \frac{C_2}{\Im(z)}\left(\sfK^2 + |z|^2\right),
\end{equation}
respectively.
Furthermore, by Lemma~\ref{lemma:re_im_properties}, then Lemma~\ref{lemma:as_norm_bounds} and Corollary~\ref{cor:S_star_min_singular_value_bound} respectively, we have the bounds
\begin{equation}
\nonumber
   \norm{\bB_n}_\op \le \frac1{\Im(z)},\quad  \norm{\bB_\star}_\op \le\frac1{\Im(z)}.
\end{equation}
Meanwhile, to bound the norm of $\bF_n$, we  can observe that 
    $\bF_n = \alpha_n (\bF_n \bE_n + \bS_n)$.
Since the assumption guarantees that
\begin{equation}
\label{eq:bound_En}
\norm{\bE_n}_\op \equiv \Err_{\FP} \le \frac1{2\alpha_n}  \frac1{1+\alpha_n \Err_{\FP}} \frac{|z|^2}{10(\sfK^2 + |z|^2)} \le \frac1{2\alpha_n},
\end{equation}
we conclude that 
\begin{equation}
\nonumber
    \norm{\bF_n}_\op \le 2\alpha_n \norm{\bS_n}_\op \stackrel{(a)}{\le} \frac{2\alpha_n}{\alpha_n \Im(z)} = \frac{2}{\Im(z)},
\end{equation}
where $(a)$ follows from Lemma~\ref{lemma:as_norm_bounds}.
Further, we have
\begin{align}
\label{eq:lb_delta_star}
\delta_{\star} &= \frac{\alpha_n \Im(z)}{2} \lambda_{\min}\left(  
(\bB_\star^{-1/2}\bA_\star\bB_\star^{-1/2} + i \bI)\bB_\star (\bB_\star^{-1/2}\bA_\star\bB_\star^{-1/2} - i\bI)
\right) \\
&\stackrel{(a)}{\ge} \frac{\alpha_n\Im(z)}{2} \lambda_{\min}\left(\bB_\star\right) \ge 
 C_3\frac{\alpha_n\Im(z)^2}{\sfK^2 + |z|^2} 
 \nonumber
\end{align}
where $(a)$ holds since the spectrum of $\bB_{\star}^{-1/2}\bA_{\star}\bB_{\star}^{-1/2}$ is real.
Finally, we lower bound $\delta_n$ by writing
%\delta_{n} \ge \frac{\Im(z)}{2 \alpha} \sigma_{\min}(\bB_n^{-1}) \sigma_{\min}(\bF_n)^2 \ge C  
%\frac{\Im(z)}{\alpha(1 + \alpha \omega_n)} \sigma_{\min}(\bB_n^{-1}) \sigma_{\min}(\bB_n)^2
%\ge 
%C  
%\frac{\Im(z)^4}{\alpha(1 + \alpha \omega_n)} \sigma_{\min}(\bB_n)^2
\begin{align*}
    \delta_n  &= \frac{\Im(z)}{2\alpha_n} \lambda_{\min}(\bB_n^{-1/2} \bS_n \bS_n^{-1}\bF_n \bF_n^*\bS_n^{*-1}\bS_n^*\bB_n^{-1/2})
    \ge \frac{\Im(z)}{2\alpha_n} \sigma_{\min}(\bS_n^{-1} \bF_n) \lambda_{\min}\left( \bB_n^{-1/2} \bS_n \bS_n^* \bB_n^{-1/2}\right).
\end{align*}
Noting that
as a consequence of Eq.~\eqref{eq:bound_En} we have
    $\norm{\bS_n \bF_n^{-1}} = \norm{\alpha_n^{-1}+\bE_n}_\op \le  3/(2\alpha_n)$ gives us the lower bound on $\sigma_{\min}(\bS_n^{-1}\bF_n) \ge 2\alpha_n/3$.
This along with the decomposition of Eq.~\eqref{eq:lb_delta_star} applied to the display above gives 
\begin{equation}
\nonumber
    \delta_n \ge \frac{\Im(z)}{3}\lambda_{\min}(\bB_n^{-1/2} \bS_n\bS_n^* \bB_n^{-1/2}) \ge \frac{\Im(z)}{3} \lambda_{\min}(\bB_n) \ge
C_4\frac{\Im(z)^2}{\sfK^2 + |z|^2}.
\end{equation}
Using these bounds in Eq.~\eqref{eq:S_Sn_rate_expansion} above gives the claim. 
\end{proof}


\subsection{Uniform convergence under test functions: proof of Proposition~\ref{prop:uniform_convergence_lipschitz_test_functions}
}
%Given a function $f :\R\to\R$ and $B\ge 0 $, define the restricted Lipschitz norm
%\begin{equation}
%\norm{f}_{\Lip(B)} := \norm{f\cdot\one_{[-B, B]} + f(B)\one_{(B,\infty)} + f(-B)\one_{(-\infty,-B)}}_\Lip.
%\end{equation}
First, note that Eq.~\eqref{eq:ST_convergence_in_P_seq_measures} of Proposition~\ref{prop:uniform_convergence_lipschitz_test_functions}
can be deduced directly from Lemma~\ref{lemma:rate_matrix_ST}.

The following lemma allows us to deduce convergence of the expectation of a bounded Lipschitz function from convergence of the Stieltjes transform. This result, and its proof are fairly standard. The proof is included in Section~\ref{sec:proof_lemma_f_bound_st} for the sake of completion.
\begin{lemma}
\label{lemma:f_bound_st}
Let $f:\R\to\R$ be continuous. Let $\mu_1,\mu_2$ be two probability measures on $\R$ with support in $[-A,A]$, let $s_1,s_2$ denote their Stieltjes transforms, respectively.
Then for any $\gamma \in (0,1)$, we have
    \begin{align*}
         \bigg|\int f(x_0)\de\mu_1(x_0) - \int f(x_0)\de\mu_2(x_0)\bigg| &\le
        \frac1\pi \norm{f}_{\infty,A} \int_{-2A}^{2A} \Big|s_1(x+i\gamma)-s_2(x+i\gamma)\Big|\de x\\
&+
\gamma\left(
2\norm{f}_{\Lip,A} \log(16 A^2+ 1)
        + \frac{2 \norm{f}_{\infty,A}}{A}\right),
     \end{align*}
where $\norm{f}_{\Lip,A}$ and $\norm{f}_{\infty,A}$ are the Lipschitz constant and $\ell_\infty$ norm, respectively, of the function 
$$x \mapsto f(-A)\one_{\{x < -A\}} + f(x)\one_{\{x\in[-A,A]\}} +  f(A)\one_{\{x > A\}}.$$
\end{lemma}
We'll apply this lemma to our setting. 
For $z\in\bbH_+$, $\hnu\in\cuP(\R^{k+k_0+1)}$, let
\begin{equation}
    s_n(z;\hnu) := \frac1k \Tr\Big( \bS_n(z; \hnu)\Big),\quad
    s_\star(z ;\hnu, \alpha_n) := \frac1k \Tr\Big( \bS_\star(z; \hnu, \alpha_n\Big),
\end{equation}
%
where we recall that $\bS_n$ is 
defined in Eq.~\eqref{eq:Sn_def} and
$\bS_{\star}$ is 
defined by Eq.~\eqref{eq:def_S_star}. 
Recall the definition of $\mu_{\MP} := \mu_{\MP}(\hnu,\alpha_n)$ whose Stieltjes transform is $s_\star$, and let $\mu_n = \mu_n(\widehat \nu_{\bV,\bU,w}, \alpha_n)$ be the ESD of $\bH/n$. Note that by definition, $s_n$ is the Stieltjes transform of $\mu_n$.
%\begin{lemma}
%\label{lemma:supp_bound_mu_star}
%There exist a constant $A_0(\alpha_n,\sfK) > 0$ depending only on $\alpha_n$ and $\sfK$ such that
%\begin{equation}
%    \supp\left(\mu_\star(\nu, \alpha_n)\right) \subseteq [-A_0,A_0].
%\end{equation}
%\end{lemma}
%\begin{proof}
%\end{proof}
Since $\hmu_{\sqrt{d}\bTheta}$ is supported on $[-\sfA_\bR,\sfA_\bR]$ for $\bTheta$ in the range of interest, and $\rho_0''$ is continuous and hence bounded on this range,
to deduce the first statement of Proposition~\ref{prop:uniform_convergence_lipschitz_test_functions}, 
it's sufficient to prove the following lemma.
\begin{lemma}
\label{lemma:LP_bound}
For any Lipschitz function $f:\R\to\R$, we have
\begin{equation}
\nonumber
    \limsup_{n\to\infty} \sup_{\hnu\in\cuP_n(\bR^{k+k_0+1})} 
    \left|
    \frac1{dk} \E\left[\Tr\;f \left(\frac1n \bH(\hnu)\right)\right] - \int f(\lambda) \mu_{\MP}(\hnu,\alpha_n)(\de\lambda)\right| = 0.
\end{equation}
\end{lemma}
\begin{proof}
We apply Lemma~\ref{lemma:f_bound_st}.
By definition of $\mu_\star$,  for $\alpha_n >1$, there exists a constant $A_0(\sfK) > 0$ such that
\begin{equation}
\nonumber
    \supp\left(\mu_\star(\nu, \alpha_n)\right) \subseteq [-A_0(\sfK),A_0(\sfK)].
\end{equation}
Furthermore, on the event $\Omega_0$ of Lemma~\ref{lemma:standard_norm_bounds}, we have the bound
(for a similarly bounded constant $A_1$)
\begin{equation}
\nonumber
    \frac1n\norm{\bH}_\op 
    %\le C \sfK  \left( 1 + \alpha_n^{-1}\right) 
    \le A_1(\sfK).
    \end{equation}
%for some universal $C>0.$
%
Taking  $A := A_1(\sfK) \vee A_0(\sfK)$,
we have by Lemma~\ref{lemma:f_bound_st} that 
for any $\gamma \in (0,1)$,
denoting
\begin{equation}
\nonumber
    \hat I_n(\hnu) := 
    \frac1{dk} \Tr\;f \left(\frac1n \bH(\hnu)\right), \quad
    I_\star(\hnu) :=  \int f(\lambda) \mu_{\MP}(\hnu,\alpha_n)(\de\lambda),
\end{equation}
\begin{align*}
\left|
    \hat I_n(\hnu)-  I_\star(\hnu)\right|
&\le C_4(\sfK)
\bigg(
\norm{f}_{\infty,A(\sfK)}
 \sup_{x \in [-2A,2A]}\left|s_n(x + i \gamma;\hnu) - s_\star(x+ i\gamma;\hnu,\alpha_n)\right|+\gamma \left(\norm{f}_{\Lip}  + \norm{f}_{\infty,A(\sfK)}  \right)  \bigg).
\end{align*}
Meanwhile, on $\Omega_0\cap\Omega_1(1)$ (choosing $L=1$ in the definition of $\Omega_1$), we have by Lemma~\ref{lemma:rate_matrix_ST},
\begin{align*}
  \sup_{x\in[-2A,2A]} |s_n(x + i \gamma) - s_\star(x+i\gamma)| &\le 
  \sup_{x\in [-A,A]}\norm{\bS_n(x+i \gamma) - \bS_\star(x + i\gamma)}_\op\\
  &\le
   C_0(\sfK) \frac{1 + |A|^4+  |\gamma|^4}{\gamma^5} 
   \Err_{\FP}( 2A + i\gamma;n,k)
\end{align*}
whenever Eq.~\eqref{eq:n_n_0_2} is satisfied. 
So choosing $\gamma := \gamma_n \to 0$ slow enough so that Eq.~\eqref{eq:n_n_0_2} is satisfied uniformly for all $z$ with $\Im(z) \in [-2A,A]$, and $\Err_{\FP}(2A + i\gamma_n; n, k) \to 0$   as $n\to\infty$
shows that 
\begin{equation}
\label{eq:hatI_diff_Istar}
\lim_{n\to\infty}\sup_{\hnu\in\cuP_n(\R^{k+k_0+1})}\left|
    \hat I_n(\hnu)-  I_\star(\hnu)\right| = 0
\end{equation}
on $\Omega_0 \cap \Omega_1(1)$. 
So
\begin{align*}
   \left|\E[\hat I_n(\hnu)] -  I_\star(\hnu)\right| \le 
 \E\left[\left|\hat I_n(\hnu) -  I_\star(\hnu)\right| \one_{\Omega_0 \cap \Omega_1(1)}\right] 
 +
 2\|f\|_{\infty,A}  \left(\P\left(\Omega_0^c \right) + \P\left(
 \Omega_1^c\right)\right).
\end{align*}
Taking supremum over $\nu$ then sending $n\to\infty$ and using 
Lemmas~\ref{lemma:standard_norm_bounds} and~\ref{lemma:concentration_loo_quad_form} to bound the probability along with \eqref{eq:hatI_diff_Istar} gives the result.
\end{proof}



%\begin{lemma}
%\label{lemma:LP_bound}
%Let $\mu_\star = \mu_\star(\widehat\nu_{\bV,\bU,\bw},\alpha_n)$
%and 
%$\mu_n = \mu_n(\widehat\nu_{\bV,\bU,\bw},\alpha_n)$.
%   For a Lipschitz function $f:\R\to\R$, 
% define 
% \begin{equation}
% \omega_{\textrm{BL}}(n,d,k, f;\gamma) := 
%C(\sfK)
%\norm{f}_{\infty,A(\sfK)}
%\frac{k}{\gamma^9}\left(L\sqrt{\frac{k_+(d)}{n}} + \frac{1}{n \gamma}\right)
%+ 
% k\gamma( \norm{f}_{\Lip} + \norm{f}_{\infty,A(\sfK)} )
% \end{equation}
% where $A(\sfK)>0$ is a constant depending only on $\sfK$.
%For any $\gamma\in(0,1)$, if
%$\omega_{\textrm{BL}}(n,d,k; f,\gamma) < \alpha_n^{-1}$, 
%  we have on the event $\Omega_0\cap\Omega_1(L)$ defined in
%   Lemmas \ref{lemma:standard_norm_bounds},
%\ref{lemma:concentration_loo_quad_form},
%   \begin{equation}
% k \left|
%    \int f(\lambda) \de \mu_n(\lambda) - \int f(\lambda) \de \mu_\star(\lambda)
%\right|
%\le 
%\omega_{\textrm{BL}}(n,d,k; f,\gamma).
%   \end{equation}
%Consequently, 
%there exists universal constant $c>0$ such that if $d > c$, we have for any $\eps > 0$,
%\begin{align}
% k \left|
% \E\left[\int \log(\lambda \vee \eps) \mu_n(\de \lambda)\right]- \int \log(\lambda \vee \eps) \mu_\star(\de\lambda)
%\right| \le  
%\omega_{\LP}(n,d,k; \gamma, \eps)
%\end{align}
%as long as $\omega_{\LP} < \alpha_n^{-1}$, 
%where
%\begin{equation}
%\omega_{\LP}(n,d,k; \eps):=
%\inf_{\gamma\in(0,1)}
%C(\sfK) \left(\frac{k}{\gamma^9} \left(L \sqrt{\frac{k_+(d)}{n}} + \frac1{n\gamma}\right) + \frac{k\gamma }{\eps}  \right) +  \frac{k}{d}.
%\end{equation}
%\end{lemma}
%\begin{proof}
%On $\Omega_0\cap\Omega_1(L)$, we have by Lemma~\ref{lemma:rate_matrix_ST}, for $z \in\bbH_+$,
%\begin{equation}
%   |s_n(z) - s_\star(z)| \le \norm{\bS_n(z) - \bS_\star(z)}_\op \le
%   C_0(\sfK) \frac{\left(1 + |z|\right)^4}{\Im(z)^5} 
%   \omega_{\textrm{FP}}(z,n,k,\alpha_n)
%\end{equation}
%whenever 
%\begin{equation}
%     \frac{10(\sfK^2 + |z|^2)}{\Im(z)^2}
%     \omega_{\textrm{FP}}(z,n,k,\alpha_n)(1 + \alpha_n\omega_{\textrm{FP}}(z,n,k,\alpha_n))  \le  \frac1{2\alpha_n}.
%\end{equation}
%Now for $z = x+ i\gamma$, $\gamma\in(0,1)$, $|x| \le 2A$, we have,
%\begin{align}
%\omega_{\textrm{FP}}(z,n,k,\alpha_n)&=C_1(\sfK)
%     \left( \frac{1 + |z|^4}{\Im(z)^4} \right)
%\left(  L\sqrt{\frac{k_+(d)}{n}} 
%   +  \frac{1}{n \Im(z)}\right)\\
%   &\le  C_2(\sfK) \frac{|A|^4}{\gamma^4}\left(L \sqrt{\frac{k_+(d)}{n}} + \frac1{n\gamma}\right)
%\end{align}
%as long as $A \ge 1$.
% %will denote some constant that is bounded uniformly over $\alpha_n$ bounded away
%%from $0$ and $\infty$, and $A,\sfK$ bounded.
%So for some $C_3(\sfK) >0$ sufficiently large, when
%%\begin{equation}
%%     \frac{C_3(\sfK) |A|^4}{\gamma^5}\left(L\sqrt{\frac{k_+(d)}{n}} + \frac1{n\gamma}\right)  \le  \frac1{\alpha_n}, 
%% \end{equation}
%for any $A > 1$ and $\gamma  > 0$, we have the bound
%\begin{equation}
%\sup_{x \in [-2A,2A]}\left|s_n(x + i \gamma) - s_\star(x+ i\gamma)\right| \le 
% \frac{C_3(\sfK)  |A|^8}{\gamma^9}\left( L\sqrt{\frac{k_+(d)}{n}} + \frac1{n\gamma}\right)
%\end{equation}
%whenever the quantity on the right is bounded by $\alpha_n^{-1}$.
%Now by definition of $\mu_\star$,  for $\alpha_n >1$, there exists a constant $A_0(\sfK) > 0$ such that
%\begin{equation}
%    \supp\left(\mu_\star(\nu, \alpha_n)\right) \subseteq [-A_0(\sfK),A_0(\sfK)].
%\end{equation}
%Furthermore, on the event $\Omega_0$ of Lemma~\ref{lemma:standard_norm_bounds}, we have the bound
%(for a similarly bounded constant $A_1$)
%\begin{equation}
%    \frac1n\norm{\bH}_\op 
%    %\le C \sfK  \left( 1 + \alpha_n^{-1}\right) 
%    \le A_1(\sfK).
%    \end{equation}
%%for some universal $C>0.$
%%
%Taking  $A(\sfK) := A_1(\sfK) \vee A_0(\sfK)$,
%we have by Lemma~\ref{lemma:f_bound_st} that 
%for any $\gamma \in (0,1)$,
%%\am{The formula below does not match Corollary~\ref{cor:f_bound_st}. RHS is nonlinear in $f$!}
%\begin{align}
%\left|
%    \int f(\lambda) \de \mu_n(\lambda) - \int f(\lambda) \de \mu_\star(\lambda)
%\right|
%&\le C_4(\sfK)
%\bigg(
%\norm{f}_{\infty,A(\sfK)}
% \sup_{x \in [-2A(\sfK),2A(\sfK)]}\left|s_n(x + i \gamma) - s_\star(x+ i\gamma)\right|
% \\
%&\quad\quad\quad+\gamma \left(\norm{f}_{\Lip}  + \norm{f}_{\infty,A(\sfK)}  \right)  \bigg)
%\\
%&\le C_5(\sfK)
%\norm{f}_{\infty,A(\sfK)}
%\frac{1}{\gamma^9}\left(L\sqrt{\frac{k_+(d)}{n}} + \frac{1}{n \gamma}\right)
%+ 
%\gamma( \norm{f}_{\Lip} + \norm{f}_{\infty,A(\sfK)} )
%\end{align}
%whenever this quantity is bounded by $\alpha_n^{-1}$.
%
%Now note that $\lambda \mapsto \log(\lambda \vee \eps)$ has a Lipschitz constant equal to $\eps^{-1}$, and recalling the bounds on $\P(\Omega_0^c)$ and $\P(\Omega_1(L)^c)$ for $L\ge1$, we obtain
%\begin{align}
%&\left|\E\left[\int \log(\lambda\vee \eps) \mu_n(\de \lambda)\right] - k \int \log(\lambda\vee \eps)  \mu_\star(\widehat\nu_{\tilde\bV}) (\de \lambda) \right|
%\le \E\left[\left|\int \log(\lambda\vee \eps) \mu_n(\de \lambda) - k \int \log(\lambda\vee \eps)  \mu_\star(\widehat\nu_{\tilde\bV}) (\de \lambda) \right|\right]\\
%&\quad\quad\le 
%C(\sfK)
%\|{\log^\up{\eps}}\|_{\infty,A(\sfK)}
%\frac{k}{\gamma^9}\left(L\sqrt{\frac{k_+(d)}{n}} + \frac{1}{n \gamma}\right)
%+ 
% k\gamma( \|\log^\up{\eps}\|_{\Lip} + \|{\log^\up{\eps}}\|_{\infty,A(\sfK)} ) 
% + k \log(A(\sfK)) (\P\left(\Omega_0^c\right) + \P(\Omega_1^c(L)))\\
% &\quad\quad\le C_1(\sfK)\left( \frac{k}{\gamma^9} \left(L \sqrt{\frac{k_+(d)}{n}} + \frac1{n\gamma}\right) + \frac{k\gamma }{\eps}  
% + k \left( e^{-c_1d } + e^{- c_2 L k} \vee d^{-c_2 L }\right)\right).
%\end{align}
%%Observe that whenever $k/(\gamma^{9} \sqrt{n}) <1$, we have $k/(n\gamma^{10}) <1$. 
%With the choice 
%\begin{equation}
%    L \equiv L(d) :=\begin{cases}
%        \frac1{c_2} \vee 1& k \ge \log(d)\\ 
%        \frac1{c_2}\log(d)  \vee 1& k < \log(d),
%    \end{cases}
%\end{equation}
%we have for $d > 1/{c_1}$,
%\begin{equation}
%    k \left( e^{-c_1d } + e^{- c_2 L k} \vee d^{-c_2 L }\right) \le \frac{k}{d}.
%\end{equation}
%This concludes the bound.
%
%\end{proof}

\subsection{Proofs of technical results of this section.}
\label{section:RMT_appendix_technical_results}

\subsubsection{Proof of Lemma~\ref{lemma:re_im_properties}}
\label{sec:proof_lemma_re_im_properties}
For $\bZ\in\bbH^+_k$, let $\bA = \Re(\bZ)$ and $\bB =\Im(\bZ)$. Since $\bB \succ \bzero$ we can write
\begin{equation}
\label{eq:bA_decomp}
   \bZ = (\bA + i \bB)  =  \bB^{1/2} \left(\bB^{-1/2}\bA \bB^{-1/2} + i \bI\right) \bB^{1/2}.
\end{equation}
The spectrum of $\bB^{-1/2}\bA\bB^{-1/2}$ is real since it's self-adjoint, and hence its perturbation by $i\bI$ does not contain $0$, 
proving invertibility of $\bZ$.
Now by Eq.~\eqref{eq:bA_decomp},
   \begin{align*}
       \Im(\bZ^{-1}) 
       &=
       \Im\left(
       \bB^{-1/2}(\bB^{-1/2}\bA\bB^{-1/2} + i\bI)^{-1}\bB^{-1/2}
       \right)\\
       &=
       \Im\left(
       \bB^{-1/2}(\bB^{-1/2}\bA\bB^{-1/2} + i\bI)^{-1}
       (\bB^{-1/2}\bA\bB^{-1/2} - i\bI)^{-1}
       (\bB^{-1/2}\bA\bB^{-1/2} - i\bI)\bB^{-1/2}
       \right)\\
       &=
       -\bB^{-1/2}(\bB^{-1/2}\bA\bB^{-1/2} + i\bI)^{-1}
       (\bB^{-1/2}\bA\bB^{-1/2} - i\bI)^{-1}
       \bB^{-1/2}\\
       &= 
       -(\bA + i\bB)^{-1}
       \bB
   (\bA - i\bB)^{-1}\\
&= - \bZ^{-1} \Im(\bZ) \bZ^{*-1}\\
&\prec \bzero,
   \end{align*}
where in the last line we used that $\Im(\bZ)\succ\bzero.$
To prove the bounds in Item~\textit{1},
note that, for any vector $\bx\in \C^k$,
\begin{align}
\label{eq:modulus_is_positive}
\|(\bB^{-1/2}\bA\bB^{-1/2} + i\bI)\bx\|_2 \ge \|\bx\|_2,
\end{align}
whence, for any $\bx\in \C^k$,
$\|(\bB^{-1/2}\bA\bB^{-1/2} + i\bI)^{-1}\bx\|_2 \le \|\bx\|_2$.
Therefore, taking inverses of both sides of Eq.~\eqref{eq:bA_decomp}
we conclude that $\|\bZ^{-1}\bx\|_2 \le \|\bB^{-1}\bx\|_2 
=\|\Im(\bZ)^{-1}\bx\|_2$ as desired for the bound on $\norm{\bZ^{-1}}_\op$. Using Eq.~\eqref{eq:modulus_is_positive} once again we conclude that 
$\norm{\bZ\bx}_2 \ge \norm{\bB\bx}_2$ giving the desired bound on $\Im(\bZ).$

To prove Item~\textit{2}, first consider the case where $\bW$ is invertible. In this case, we can write
   $(\bI + \bW\bZ)^{-1} \bW = (\bW^{-1} + \bZ)^{-1}.$
Noting that $\Im(\bW^{-1} + \bZ) = \Im(\bZ) \succ \bzero$, we see that an application of Item \textit{1} gives both claims. 
For non-invertible $\bW$, let $s_{\min} := \lambda_{*}(\bW)$
be the non-zero eigenvalue of $\bW$ with the smallest absolute value,
and define $\bW_\eps := \bW + \eps |s_{\min}|$ for $\eps \in (0,1)$.
We have by the previous argument that the statement holds for $\bW$ replaced with $\bW_\eps.$ Taking $\eps\to 0$ proves it in the general non-invertible case.
\qed

\subsubsection{Proof of Lemma~\ref{lemma:tensor_trace_properties}}
\label{sec:proof_lemma_tensor_trace_properties}
Let $\bM_{i,j}\in\C^{d\times d}$ be the blocks of $\bM$ for $i,j \in[k]$.
We obtain the first bound in \textit{1} by writing
\begin{align}
\nonumber
    \norm{(\bI_k \otimes \Tr)\bM}_F^2
    &= \sum_{i,j \in[k]} \Tr(\bM_{ij})^2
    \le d \sum_{i,j\in[k]} \norm{\bM_{ij}}_F^2
    = d \norm{\bM}_F^2.
\end{align}
Now let $\bx\in\R^{d}$ be a random variable distributed uniformly on the sphere of radius $\sqrt{d}$.
For any $\bv,\bu \in \C^{k}$ we have
\begin{equation}
\label{eq:tensor_tr_to_E_sphere}
    \bu^* \left(\left(\bI_k \otimes \Tr\right)\bM \right)\bv 
   = \sum_{i,j} \overline{u_i} v_j \Tr(\bM_{i,j}) 
  =  \sum_{i,j} \overline{u_i} v_j \E[\bx^\sT\bM_{i,j} \bx] = \E[(\bu\otimes \bx)^* \bM (\bv\otimes \bx)].
\end{equation}
Optimizing over $\bv,\bu$ of unit norm gives
\begin{align}
\nonumber
   \norm{(\bI_k \otimes \Tr)\bM}_\op  
%   &= \max_{\norm{\bv}=\norm{\bu} = 1} \sum_{i,j} u_i v_j \E[\bx^\sT\bM_{i,j}\bx]\\
   %&\le \max_{\norm{\bv} = \norm{\bu} = 1} \E[(\bu \otimes \bx)^\sT \bM (\bv \otimes \bx)]
    \le 
\max_{\norm{\bv}_2 = \norm{\bu}_2 = 1}\norm{\bM}_\op  \E[\norm{\bx}_2^2] \norm{\bu}\norm{\bv}
 = d \norm{\bM}_\op
\end{align}
giving the second bound in Item \textit{1}.
%
For the claim in \textit{2},
take $\bv = \bu$ in Eq.~\eqref{eq:tensor_tr_to_E_sphere} to conclude the (strict) positivity of $(\bI_k\otimes \Tr)\bM$ from that of $\bM$.
For \textit{3}, once consider Eq.~\eqref{eq:tensor_tr_to_E_sphere} and minimize over $\bv = \bu$ with unit norm and use \textit{2} to write
\begin{align}
\nonumber
   \lambda_{\min}\left((\bI_k \otimes \Tr)\bM\right)
   %&= \min_{\norm{\bv}=1}\left| \sum_{i,j} v_i v_j \E[\bx^\sT\bM_{i,j}\bx ] \right|
   = \min_{\norm{\bv} = 1} \E[(\bv \otimes \bx)^* \bM (\bv \otimes \bx)]
   \ge \min_{\norm{\bv} = 1} \lambda_{\min}(\bM) \norm{\bv \otimes \bx}_2^2
   %&= \min_{\norm{\bv} = 1} \sigma_{\min}(\bA) \norm{\bv \otimes \bx}_2^2\\
    = d \lambda_{\min}(\bM),
\end{align}
giving the claim.
Finally, \textit{4} follows by linearity of the involved operators.
%For \textit{3.}, defining $(\overline \Tr)(\bA) := \overline{\Tr(\bA)} = \Tr(\bA^*),$ we have
%\begin{align} 
%\Im\left( (\bI_k \otimes \Tr) \bA\right) 
%&= \frac1{2i}\left((\bI_k \otimes \Tr)(\bA) - (\bI_k \otimes \overline\Tr )(\bA)\right)\\
%&= (\bI_k \otimes \Tr)\left(\frac1{2i} \left(\bA - \bA^*\right) \right)\\
%&= (\bI_k \otimes \Tr)\Im(\bA).
%\end{align}
\qed
\subsubsection{
Proof of Lemma~\ref{lemma:algebra_lemma}}
\label{sec:proof_lemma_algebra_lemma}
Suppress the argument $z$ in what follows.
By Woodbury, we have for each $i\in[n],$
\begin{equation}
\nonumber
\bR  = \bR_i - \bR_i\bxi_i
\left(\bI_k + \bW_i\bxi_i^\sT \bR_i \bxi_i\right)^{-1} \bW_i\bxi_i^\sT \bR_i.
\end{equation}
So
\begin{align*}
 \bxi_i \bW_i\bxi_i^\sT \bR 
&= 
 \bxi_i \bW_i\bxi_i^\sT  \bR_i -
\bxi_i \bW_i\bxi_i^\sT
\bR_i\bxi_i
\left(\bI_k + \bW_i\bxi_i^\sT \bR_i \bxi_i\right)^{-1} \bW_i\bxi_i^\sT \bR_i\\
&= 
 \bxi_i \bW_i\bxi_i^\sT  \bR_i -
\bxi_i
\left(
\bI_k - 
\left(\bI_k +\bW_i \bxi_i^\sT \bR_i \bxi_i\right)^{-1}
\right)
\bW_i\bxi_i^\sT \bR_i\\
&=
\bxi_i
\left(\bI_k + \bW_i\bxi_i^\sT \bR_i \bxi_i\right)^{-1}
\bW_i\bxi_i^\sT \bR_i.
\end{align*}

To prove the identity in~Eq.~\eqref{eq:alg_id1},
for any $\bA \in\R^{k\times k}$, note that 
we have
\begin{align*}
\left(\bI_k \otimes \Tr\right) \bxi_i \bA  \bW_i\bxi_i^\sT \bR_i
   &=\left(\bI_k \otimes \Tr\right) (\bI_k \otimes \bx_i)
   \bA  
\bW_i
   (\bI_k \otimes \bx_i)^\sT \bR_i\\
   &= \left(\bI_k \otimes \Tr\right) \left(\sum_{a\in[k]} 
   (\bA \bW_i)_{j,a}
   \bx_i\bx_i^\sT  (\bR_i)_{a,l}
   \right)_{j,l \in[k]}
   \\
   &= \left( \sum_{a \in [k]}
   (\bA \bW_i)_{j,a}
   \Tr\left(\bx_i\bx_i^\sT  \left(\bR_i\right)_{a,l}
   \right)\right)_{j,l \in [k]}\\
   &= \left( \sum_{a \in [k]}
   (\bA \bW_i)_{j,a}
   \bx_i^\sT  \left(\bR\right)_{a,l} \bx_i
   \right)_{j,l \in [k]}\\
   &= 
   (\bA \bW_i)
   \bxi_i^\sT \bR_i\bxi_i.
\end{align*}
Using this for 
    $\bA :=\left( \bI_k + \bW_i\bxi_i^\sT \bR_i \bxi_i\right)^{-1}$
gives the result.
To prove the identity of Eq.~\eqref{eq:alg_id2}, we write
\begin{align*}
   \left(\bI_k \otimes \Tr \right)(\bR_i - \bR) &= \left(\bI_k \otimes \Tr \right)\bR_i \bxi_i \bW_i\bxi_i^\sT \bR\\
   &= \left(\Tr\left( \sum_{b,c \in [k]} (\bR_i)_{a,b} \bx_i(\bW_i)_{b,c}\bx_i^\sT \bR_{c,d}\right)  \right)_{a,d \in [k]}\\
   &= \left( \sum_{b,c \in [k]}  \bx_i^\sT  (\bR_i)_{a,b}^\sT (\bW_i)_{b,c}\bR_{c,d}^\sT \bx_i  \right)_{a,c \in [k]}\\
   &=  \bxi_i^\sT \bR_i^\sT (\bW_i \otimes \bI_d) \bR^\sT \bxi_i.
\end{align*}
Symmetry of the matrices $\bR_i$ and $\bR$ gives the conclusion.
%Now for the identity in Eq.~\eqref{eq:alg_id2}, by~Eq.~\eqref{eq:woodbury_cor}, we have
%\begin{align}
%    \bM_i^{-1}  \bz_i (\bI_k + \widetilde \bz_i^\sT \bM_i^{-1} \bz_i)^{-1} \widetilde \bz_i^\sT \bM_i^{-1}
%    &= \bM_i^{-1} \bz_i \widetilde \bz_i^\sT \bM^{-1}\\
%    &= \bM_i^{-1} \left(
%\sum_{a\in[k]} (\grad^2 \rho_{i})_{j,a} \bx_i \bx_i^\sT (\bM^{-1})_{a,l}
%    \right)_{j,l \in[k]}\\
%    &= \left(\sum_{j,a \in[k]} 
%    (\bM_i^{-1})_{b,j} 
%  (\grad^2 \rho_{i})_{j,a} \bx_i \bx_i^\sT (\bM^{-1})_{a,l}    
%    \right)_{b,l \in[k]}.
%\end{align}
%Taking $(\bI \otimes \Tr)$ and noting that $\bM_i,\grad^2 \rho_i$ and $\bM$ are symmetric, we have
%\begin{align}
%(\bI \otimes \Tr)
%    \left(\bM_i^{-1}  \bz_i (\bI_k + \widetilde \bz_i^\sT \bM_i^{-1} \bz_i)^{-1} \widetilde \bz_i^\sT \bM_i^{-1}\right)
%&= \left(\sum_{j,a \in[k]} 
%    \bx_i^\sT (\bM^{-1})_{l,a}   
%   (\grad^2 \rho_{i})_{a,j}
%(\bM_i^{-1})_{j,b}
%\bx_i
%    \right)_{b,l \in[k]}\\
%    &= 
%   \left((\bI_k \otimes \bx_i)^\sT 
%   \bM^{-1}(\grad^2\rho_i \otimes  \bI_k) \bM_{i}^{-1} (\bI_k \otimes \bx_i)\right)^{\sT}
%\end{align}
%which gives the result.
\qed

\subsubsection{Proof of Lemma~\ref{lemma:as_norm_bounds}}
\label{sec:proof_lemma_as_norm_bounds}
Since the eigenvalues $\left\{\lambda_j\right\}_{j\in[dk]}$ of $\bH_i$ are  real, we have
for the first two bounds of Eq.~\eqref{eq:det_norm_bound_lemma_eq123}
\begin{align}
\nonumber
   \norm{\bR_i}_{F}^2 
   &= \sum_{j=1}^{dk} \frac{1}{|\lambda_j - z n|^2 }
\le \frac{dk}{n^2} \frac1{\Im(z)^2}\quad\textrm{and}\quad
   \norm{\bR_i}_{\op}
   = \max_{i \in [dk]} \frac{1}{|\lambda_i - z n| }
\le \frac{1}{n} \frac1{\Im(z)}.
\end{align}
The bounds on $\norm{\bR}_{F}^2$, $\norm{\bR}_{\op}$
follow similarly.

For the third bound of Eq.~\eqref{eq:det_norm_bound_lemma_eq123}, given a vector $\ba\in\R^{nk}\simeq \R^{n}\otimes \R^k$, 
we write its entries as $\ba = (a_{i,l}: \; i\in[n],
l\in [k])$.
Then write
\begin{align}
\nonumber
    %\norm{(\Diag(\partial_{l,j}\ell_i)_{i\in [n]})_{l,j\in[k]}}_\op
    %\norm{\bW}_\op
    %&\le 
    \<\ba,\bSec\ba\> 
= \sum_{i=1}^n \< \ba_{i,\cdot}, \nabla^2_{\bv}\ell(\bv_i,\bu_i,\eps_i) \ba_{i,\cdot}\> 
    =  \sum_{i=1}^n \< \ba_{i,\cdot}, \bW_i \ba_{i,\cdot}\>  
    \le\sfK  \sum_{i=1}^n \norm{\ba_{i,\cdot}}_2^2\, .
\end{align}
%
Optimizing over $\norm{\ba}_2 = 1$ gives $\norm{\bW}_\op \le \sfK$ as desired.

The inequality $\norm{\bH}_\op \le\sfK \norm{\bX}_\op^2$ follows directly fro the previous one.

For the bound in~\eqref{eq:det_norm_bound_lemma_eq4}
recall that $\bH$ is self-adjoint, and hence for $z\in\bbH_+$, 
$\Im(\bH/n - z\bI)^{-1} \succ \bzero$. So by Lemma~\ref{lemma:tensor_trace_properties} and Lemma~\ref{lemma:re_im_properties}, we can bound
\begin{align*}
   \lambda_{\min}(\Im((\bI\otimes\Tr)\bR)) &= \lambda_{\min}\left( 
    \frac1n \left( \bI_k \otimes \Tr\right) \Im\left(\left(\bH/n - z \bI_{nk}\right)^{-1}\right)\right)
    &\ge  \lambda_{\min}\left(\Im((\bH/n - z\bI)^{-1})\right)\\
&=  \Im(z)\lambda_{\min}\left(
(\bH/n -z\bI)^{-1}(\bH/n -z^*\bI)^{-1}
\right)
&\ge  \Im(z)(\norm{\bH/n}_\op  + |z|)^{-2}.
\end{align*}
The conclusion now readily follows.

Finally, for the bound in~\eqref{eq:det_norm_bound_lemma_eq5} note that
\begin{equation}
    \lambda_{\min}(\Im(\bxi_i^\sT\bR_i \bxi_i)) = 
    \lambda_{\min}(\bxi_i^\sT\Im(\bR_i) \bxi_i) = \frac{\sigma_{\min}(\bxi_i^\sT\bxi_i)}{n} \lambda_{\min}
    \big(\Im( (\bH_i/n - z\bI_{n,k})^{-1})\big)
\end{equation}
and that  $\lambda_{\min}(\bxi_i^\sT\bxi_i) = \lambda_{\min}(\bI_k \otimes \bx_i^\sT\bx_i)= \norm{\bx_i}_2^2$ to derive the conclusion.
\qed



\subsubsection{Proof of Lemma~\ref{lemma:f_bound_st}}
\label{sec:proof_lemma_f_bound_st}
Lemma~\ref{lemma:f_bound_st} is a direct corollary of the two lemmas of this section.
Let
\begin{equation}
    \rho(x; x_0, \gamma) := \frac1\pi \Im\left( \frac{1}{x - (x_0 + i\gamma)} \right)
\end{equation}
be the density of a Cauchy distribution with location $x_0\in\R$ and scale $\gamma>0$. Recall that we have, for any continuous bounded function $f$,
$$\lim_{\gamma \to 0}\int f(x) \rho(x; x_0, \gamma) \, \de x = f(x_0).$$
The next lemma gives a quantitative version of this
fact.

\begin{lemma}
\label{lemma:quant_dirac_integral}
Fix positive reals $B > A > 0$. Define for $x_0 \in\R$,
\begin{equation}
    \Delta_{f,B,\gamma}(x_0):= f(x_0) - \int_{-B}^B  f(x)\rho(x;x_0,\gamma)\, \de x.
\end{equation}
We have the bounds
\begin{equation}
\nonumber
        \sup_{x_0 \in [-A,A]}\left|\Delta_{ f,B,\gamma}(x_0)\right|\leq \norm{f}_{\Lip} \gamma\log(4B^2+\gamma^2)
        +\norm{f}_\infty \frac{\gamma}{2} \left(\frac1{B-A} + \frac1{B+A}\right)
\end{equation}
and
\begin{equation}
\nonumber
    \sup_{x_0 \in \R \setminus [-A,A]}\left|\Delta_{ f,B,\gamma}(x_0)\right|
    \leq 
    2\norm{f}_\infty.
\end{equation}
\end{lemma}



\begin{proof}
The second bound is immediate since $\rho$ is a density. 
To show the first, fix $x_0\in[-B,B]$. We have
    \begin{align}
    \nonumber
        \Delta_{ f,B,\gamma}(x_0) =&  f(x_0)\int_{-\infty}^\infty\rho(x;x_0,\gamma)\de x
        -\int_{-B}^B  f(x) \rho(x;x_0,\gamma)\de x\\
        \nonumber
        =&\int_{-B}^B \left(f(x_0)-f(x) \right) \rho(x;x_0,\gamma)\de x +  f(x_0)\left(1-\int_{-B}^{B} \rho(x;x_0,\gamma)\de x\right)\\
        \leq & \norm{f}_{\Lip}\left(\int_{-B}^B |x_0-x|\rho(x;x_0,\gamma)\de x\right)
        +\norm{f}_\infty\left(1-\int_{-B}^B \rho(x;x_0,\gamma)\de x_0\right).
        \label{eq:last_eq_in_DeltafB_bound}
    \end{align}
    By a change of variable, the first integral above can be bounded as
    \begin{align*}
        \int_{-B}^B |x_0-x|\rho(x;x_0,\gamma)\de x = 
        \int_{-B-x_0}^{B-x_0} |x|\rho(x;0,\gamma)\de x
        \leq& \frac{2}{\pi}\int_{0}^{2B} x \frac{\gamma}{x^2+\gamma^2}\de x
        = \frac{\gamma}{\pi}\log\left(\frac{4B^2}{\gamma^2} + 1\right)
    \end{align*}
where we used the even symmetry of the integrand and that $|x_0 | \le B$ to deduce the inequality.
%\bns{This can be strengthened if necessary by bounding by $A$ instead but probably not needed.}
Meanwhile, the second integral in Eq.~\eqref{eq:last_eq_in_DeltafB_bound} is bounded as
    \begin{align*}
        1-\int_{-B}^B \rho(x;x_0,v)\de x
        %\le  1 - \int_{0}^{2B} \rho(x;0,\gamma) \de x
        &= 1- \frac{1}{\pi }\left[\arctan \left( \frac{B-x_0}{\gamma}\right) - \arctan \left( \frac{-B-x_0}{\gamma}\right)\right]\\
        &\le  1- \frac{1}{\pi }\left[\arctan \left( \frac{B-A}{\gamma}\right) + \arctan \left( \frac{B+A}{\gamma}\right)\right]\\
        &\le
        \frac12 \left( \frac{\gamma}{B-A}  + \frac{\gamma}{B+A}\right),
    \end{align*}
   where in the last line we used that $1 -2\pi^{-1}\arctan(t) \le t^{-1}$. This concludes the proof.
\end{proof}
\begin{lemma}
Let $f:\R\to\R$ be continuous. Let $\mu_1,\mu_2$ be two probability measures on $\R$ and let $s_1,s_2$ denote their corresponding Stieltjes transforms, respectively.
Then for any positive reals $B > A \ge 0$ and $\gamma  > 0$, we have
    \begin{align*}
         \bigg|\int f(x_0)\de\mu_1(x_0) - \int f(x_0)\de\mu_2(x_0)\bigg| &\le
        \frac1\pi \norm{f}_\infty \int_{-B}^B \Big|s_1(x+i\gamma)-s_2(x+i\gamma)\Big|\de x\\
&+2\norm{f}_{\Lip} \gamma\log(4B^2+\gamma^2)
        +\norm{f}_\infty \gamma \left(\frac1{B-A} + \frac1{B+A}\right)\\
&+2 \norm{f}_\infty \big(\mu_1\left(\R \setminus [-A,A]\right) + \mu_2\left(\R \setminus [-A,A]\right)\big).
     \end{align*} 
\end{lemma}


\begin{proof}
Rewriting $f$ in terms of the quantity $\Delta_{f,B,\gamma}$ defined in Lemma~\ref{lemma:quant_dirac_integral}, we have
\begin{align}
\nonumber
    \int f(x_0) \left(\de \mu_1(x_0) - \de \mu_2(x_0)\right) 
    &= \int\left( \int_{-B}^B f(x) \rho(x;x_0, \gamma) \de x  + \Delta_{f,B,\gamma}(x_0) \right) 
     \left(\de \mu_1(x_0) - \de \mu_2(x_0)\right) \\
     \nonumber
    &= 
     \int_{-B}^B f(x)\left(\int  \rho(x;x_0, \gamma) 
     \left(\de \mu_1(x_0) - \de \mu_2(x_0)\right) \right)\de x\\
     &\quad\quad+
\int \Delta_{f,B,\gamma}(x_0)
    \left(\de \mu_1(x_0) - \de \mu_2(x_0)\right)
     \label{eq:decomp_Ex_diff}
\end{align}
where the change of order of integration is justified by integrability of the continuous $f$ over $[-B,B]$.
Noting that
for $j\in\{1,2\}$,
\begin{equation}
\nonumber
     \int \rho(x;x_0,\gamma) \de \mu_j(x_0)=
     \frac1{\pi}\Im(s_j(x+i\gamma)),
\end{equation}
%\am{I actually get (using the Wikipedia convention for ST):
%\begin{equation}
%     \int \rho(x;x_0,\gamma) \de \mu_j(x_0)=
%     \frac1{\pi}\Im(s_j(x+i\gamma)),
%\end{equation}
%Please double check and propagate below
%}
the first term in Eq.~\eqref{eq:decomp_Ex_diff} is bounded as
\begin{align}
\nonumber
     \int_{-B}^B f(x)\left(\int  \rho(x;x_0, \gamma) 
     \left(\de \mu_1(x_0) - \de \mu_2(x_0)\right) \right)\de x 
     &=
     \frac1\pi\int_{-B}^B f(x) \left(
    \Im\left(s_1(x + i \gamma) - s_2(x+ i\gamma)\right)
     \right)
     \de x\\
     &\le \frac1\pi \norm{f}_\infty \int_{-B}^B \left|s_1(x + i \gamma) - s_2(x+ i\gamma)\right| \de x.
     \label{eq:decom_Ex_diff_bound_1}
\end{align}

To bound the second term in Eq.~\eqref{eq:decomp_Ex_diff}, 
for each $j\in\{1,2\}$
we have
\begin{align*}
   \int \Delta_{f,B,\gamma}(x_0) \de \mu_j(x_0)  
   &\le  \int_{-A}^A \left|\Delta_{f,B,\gamma}(x_0)\right| \de \mu_j(x_0)+ \int_{\R\setminus[-A,A]} \left|\Delta_{f,B,\gamma}(x_0) \right|\de \mu_j(x_0)\\
   &\le  
\sup_{x_0 \in [-A,A]} \left|\Delta_{f,B,\gamma}(x_0)\right|
        +
        \sup_{x_0 \in \R\setminus [-A,A]} \left|\Delta_{f,B,\gamma}(x_0)\right| \cdot
        \mu_j\left(\R \setminus [-A,A]\right).
\end{align*} 
Applying Lemma~\ref{lemma:quant_dirac_integral} and combining with Eq.~\eqref{eq:decomp_Ex_diff} and Eq.~\eqref{eq:decom_Ex_diff_bound_1} gives the desired bound.


\end{proof}



