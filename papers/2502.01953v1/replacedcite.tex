\section{Related work}
A substantial line of work studies the existence and properties
of local minima of the empirical risk for a variety of statistics 
and machine learning problems, see e.g. 
____. However,
concentration techniques used in these works typically require $n/d\to\infty$.

The Kac-Rice approach has a substantial history in statistical
physics where it has been used to characterize the landscape of 
mean field spin glasses 
____. 
We refer in particular to the seminal works
of Fyodorov ____ and Auffinger, Ben Arous, Cern\`y ____, as well as to the recent papers
____ and references therein.
It has also a history in statistics, where it has been used for statistical analysis of Gaussian fields ____.

Within high-dimensional statistics, the Kac-Rice approach was first used
in ____ to characterize 
topology of the likelihood function for Tensor PCA. 
In particular, these authors showed that the expected number of modes of the likelihood can grow exponential in the dimension, hence 
making optimization intractable.
The Kac-Rice approach was used in ____ to show that Bayes-optimal estimation 
in high dimension can be perforemd for $\integers_{2}$-synchronization via minimization of the so-called Thouless-Anderson-Palmer (TAP) free energy. These results  were substantially strengthened in ____ which proved local convexity of TAP free energy in a neighborhood of
the global minimum. 

None of the above papers studied the ERM problem that is the focus of the present paper.
The crucial challenge to apply the Kac-Rice formula to
the empirical risk of Eq.~\eqref{eq:erm_obj_0} lies in the fact that
the empirical risk function $\hR_n(\,\cdot\,)$ is not a Gaussian process
(even for Gaussian covariates $\bx_i$).
In contrast, 
____
treat problems  for which $\hR_n$ is itself Gaussian.
In the recent review paper ____, Fyodorov and Ros 
mention non-Gaussian landscapes as an outstanding challenge even at the non-rigorous theoretical physics

While in principle Kac-Rice formula can be extended to non-Gaussian processes
(provided the gradient of $\hR_n$ has a density), this creates technical difficulties.  
In a notable paper, Maillard, Ben Arous, Biroli ____ 
followed this route to study (a special case of) the ERM problem
of Eq.~\eqref{eq:erm_obj_0}. They derive an upper bound
of the form \eqref{eq:Summary}, albeit with a different function 
$\Phi(\mu,\nu)$. However, it is unknown whether their bound has
the rate trivialization property \eqref{eq:Trivialization},
even when applied to convex ERM problems\footnote{Strictly speaking, ____ does not apply to any non-quadratic convex risk function, but replicating their proof in a more general settings leads us to this conclusion.}.

Our approach is quite different from earlier works.
We apply Kac-Rice formula to a process in $(n+d)k$ dimensions,
that we refer to as the `gradient process.'
The gradient process extends the gradient $\nabla \hR_n(\bTheta)$ and has two important additional properties: it is Gaussian;
zeros of the gradient process (with certain additional conditions) correspond to local minima of the empirical risk.

We finally note that, in the case of convex losses, an alternative proof technique is available,
which is based on approximate message passing (AMP) algorithms. This approach was initially developed to
analyze the Lasso ____ and subsequently refined and extended, see e.g.
____.
While our main motivation is to move beyond convexity, our approach recovers and unifies,
with some distinctive advantages.