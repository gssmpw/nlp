%%%%%%Arxiv submission format:
\documentclass[11pt]{article}
\usepackage{ifthen}
\usepackage{tabulary}
\newboolean{arxiv}
\setboolean{arxiv}{true}



%%%%%%Colt submission format:
%\documentclass[anon,12pt]{colt2022} % Anonymized submission
%\documentclass[final,12pt]{colt2022} % Include author names
%\newboolean{arxiv}
%\setboolean{arxiv}{false}



\usepackage{ifthen}

%%%%Import packages according to submission
\ifthenelse{\boolean{arxiv}}{
\usepackage{arxiv_def}
}
{
\usepackage{times}
\usepackage{colt_def}
}


\usepackage[toc,page]{appendix}
\usepackage{xr}
\usepackage{mathtools}
\usepackage{yfonts}  
\usepackage{subcaption} % loads the caption package




%\usepackage{tgheros}
\DeclareSymbolFont{Greekletters}{OT1}{iwona}{m}{n}
\DeclareSymbolFont{greekletters}{OML}{iwona}{m}{it}
\DeclareMathSymbol{\salpha}{\mathord}{greekletters}{"0B}
\DeclareMathSymbol{\sbeta}{\mathord}{greekletters}{"0C}
\DeclareMathSymbol{\sgamma}{\mathord}{greekletters}{"0D}
\DeclareMathSymbol{\sOmega}{\mathord}{Greekletters}{"0A}
\DeclareMathSymbol{\smu}{\mathord}{greekletters}{"16}
\DeclareMathSymbol{\svarepsilon}{\mathord}{greekletters}{"22}
\DeclareMathSymbol{\svarrho}{\mathord}{greekletters}{"25}
\DeclareMathSymbol{\svarphi}{\mathord}{greekletters}{"27}

%\definecolor{cc}{RGB}{1,11,111}

\newcommand{\am}[1]{\textcolor{red}{[AM: #1]}}
\newcommand{\bns}[1]{\textcolor{orange}{[BS: #1]}}
\newcommand{\kas}[1]{\textcolor{brown}{[KA: #1]}}





\makeatletter
\newcommand{\vast}{\bBigg@{3}}
\newcommand{\Vast}{\bBigg@{4}}
\makeatother



\begin{document}

\title{Local minima of the empirical risk in high dimension:\\
General theorems and convex examples}


\author{Kiana Asgari\thanks{Department of Management Science and Engineering, Stanford University} \;\;
\and\;\;
Andrea Montanari\thanks{Department of Statistics and Department of Mathematics, Stanford University} 
	%
	\and 
	%
	Basil Saeed\thanks{Department of Electrical Engineering, Stanford University}
	%
}

\maketitle

\begin{abstract}
We consider a general model for high-dimensional empirical risk minimization whereby the data $\bx_i$ are $d$-dimensional isotropic Gaussian vectors, the model is parametrized by $\bTheta\in \reals^{d\times k}$, and the loss depends on the data via the projection $\bTheta^{\sT}\bx_i$. This setting covers as special cases 
classical statistics methods (e.g. multinomial regression and other generalized linear models), but also two-layer fully connected neural networks with $k$ hidden neurons.

We use the Kac-Rice formula from Gaussian process theory to derive a bound on the expected number of local minima of this empirical risk,
under the proportional asymptotics in which $n,d\to\infty$, with $n\asymp d$. Via Markov's inequality, this bound allows to determine the positions of these minimizers (with exponential deviation bounds) and hence derive sharp asymptotics on the estimation and prediction error.

In this paper, we apply our characterization to convex losses,
where high-dimensional asymptotics were not (in general) rigorously established for $k\ge 2$. We show that our approach is tight and
allows to prove previously conjectured results. In addition, we characterize the spectrum of the Hessian at the minimizer. A companion paper applies our general result to non-convex examples.
\end{abstract}


\tableofcontents

%\section{Facts}
%
%If $X$ is a random vector with a density $p$ in $\reals^d$ and $h$ is a continuous function in $\reals^d$,
%and $M=\{x:g(x)=0\}$ is a regular manifold ($g:\reals^d\to\reals^k$ with $\rank(Dg(x))=k$ on $M$)
%then 
%%
%\begin{align}
%    \int_M\, h(x)\cdot p(x) \, \de_M x = \E[ h(X)|g(X)=0]\, P_{g(X)}(0)\, .
%\end{align}




\section{Introduction}
\label{sec:Intro}

Empirical risk minimization (ERM) is by far the most popular parameter
estimation technique in high-dimensional statistics and 
machine learning. While empirical process theory provides a
fairly accurate picture of its behavior when the sample size is sufficiently large \cite{van1998asymptotic,vershynin2018high}, a wealth of new phenomena arise when the number of 
model parameters per sample becomes of order one \cite{montanari2018mean}.
Among the most interesting of such phenomena:
exact or weak recovery phase transitions \cite{DMM09,BayatiMontanariLASSO,lelarge2019fundamental,barbier2019optimal,mignacco2020role}; 
information-computation gaps \cite{celentano2022fundamental,schramm2022computational}; 
benign overfitting and double descent
\cite{hastie2022surprises}. 
%(Of course, the number of parameters per
%sample is not always a good measure of complexity, and a more accurate
%statement would involve, for instance, Radamacher complexity.)

The proportional regime is increasingly of interest because of the adoption of
ever-more expressive families of statistical models by practitioners.
For instance, scaling laws in AI development imply that optimal predictive performances are achieved when he number of model parameters scales roughly proportionally to the number of samples \cite{kaplan2020scaling,hoffmann2022training}.

The theoretical toolbox at our disposal to understand ERM in the
proportional regime is far less robust and general than textbook 
asymptotic statistics. Approximate message passing (AMP) algorithms 
admit a general high-dimensional characterization
\cite{bayati2011dynamics,BayatiMontanariLASSO,donoho2016high},
and gives access to certain local 
minimizers of the empirical risk, but not all of them, 
and require case-by-case technical analysis. Gaussian 
comparison inequalities provide a very elegant route \cite{ThrampoulidisOyHa15,thrampoulidis2018precise}, but 
give two-sided bounds only for convex problems, and only succeed 
when a simple comparison process exists that yields a sharp 
inequality. Leave-one out techniques \cite{el2018impact} can be powerful,  but 
also crucially rely on the condition that perturbing the ERM problem by leaving out one datapoint does not affect significantly the minimizer. This in turn is challenging to guarantee without convexity. 

\subsection{Setting}

The main objective of this paper is to develop a new approach 
for ERM analysis that is based on the celebrated Kac-Rice formula for the expected number of zeros of a Gaussian process
\cite{Kac1943,Rice1945}.
We consider a general empirical risk of the form
\begin{equation}
\label{eq:erm_obj_0}
    \widehat R_n(\bTheta) := \frac1n \sum_{i=1}^n L(\bTheta^\sT \bx_i, \by_i) + \frac1d\sum_{j=1}^d \sum_{i=1}^k\rho(\sqrt{d}\Theta_{i,j})
\end{equation}
where
    $\bx_i\in\R^d$, $\by_i\in\R^{q},$ 
    $\bTheta = (\Theta_{i,j})_{i\in[d],j\in[k]} \in \R^{d\times k}$, while the loss function $L : \R^{k+q} \to \R$, $(\bu,\by)\mapsto L(\bu,\by)$, and regularizer
$\rho :\R \to \R$ are differentiable and independent of $n,d$.

Sharp asymptotics for the ERM problem \eqref{eq:erm_obj_0}
are generally unknown, even when $L$ is convex, although conjectures can be derived
using statistical physics methods \cite{engel2001statistical,montanari2024friendly}. Even when
$L$, $\rho$ are convex, Gaussian comparison inequalities fail to provide acharacterization for 
$k\ge 2$.

We assume the covariate vectors $(\bx_i:i\le n)$ to be i.i.d.
with $\bx_i\sim\normal(0,\bI_d)$ and the response variable to be distributed according to a general multi-index model.
Namely
%
\begin{align}
\nonumber
\P(\by_i\in S|\bx_i) = \rP(S |\bTheta_0^{\sT}\bx_i)\, ,
\end{align}
%
where $\bTheta_0^{d\times k_0}$ is a fixed matrix of coefficients and  $\rP:\cB_{\reals^q}\times\reals^{k_0}\to [0,1]$ is a probability kernel. More concretely (and equivalently)
$\by_i = \bphi(\bTheta_0^{\sT}\bx_i,w_i)$ for some  measurable function $\bphi:\reals^{k_0}\times\reals\to\reals^q$ and
$w_i\sim \P_{w}$ independent of $\bx_i$. 

It might be useful to point out a couple of examples.


\paragraph{Multinomial regression.} We define labels via one-hot encoding, i.e. 
$\by_i\in \reals^{k+1}$ takes value in 
$\{\be_0=\bzero,\be_1,\dots,\be_k\}$, $q=k+1$. Further, we assume a well specified model whereby
$\P(\by_i=\be_j|\bx_i) = p_j(\bTheta_0^{\sT}\bx_i)$
and for $j\in\{1,\dots,k\}$, $\bv_0 \in\R^{k}$,
\begin{equation}
   p_j(\bv_0)  := \frac{\exp\{ v_{0,j}\} }{ 1  + \sum_{j'=1}^k \exp\{ v_{0,j'}\}},\quad\quad 
   p_{0}(v_0)  := \frac{1 }{ 1  + \sum_{j'=1}^k \exp\{ v_{0,j'}\}}. \label{eq:MultiNomialDef}
\end{equation}

The maximum likelihood estimator is obtained using cross-entropy loss
%
\begin{align}
\nonumber
L(\bv, \by) := -\sum_{j=0}^k y_j\log p_j(\bv)\, .
\end{align}
%
Despite its simplicity, high-dimensional asymptotics for this model are well understood only for the case of 
two classes, i.e. $k=1$
\cite{sur2019modern,montanari2019generalization,deng2022model}.
We will apply our general theory to this model in Section
\ref{sec:Multinomial}. 

\paragraph{Two-layer neural networks.}
Consider, to be definite, a binary classification problem
whereby $y_i\in\{0,1\}$, with $\P(y_i=1|\bx_i) = \varphi(\bTheta_0^{\sT}\bx_i)$. It makes sense to fit 
a two-layer  neural network,  with $k\ge k_0$ (for $\bTheta_{\cdot,i}$ the $i$-th column of $\bTheta$):
%
\begin{align}
\nonumber
f(\bx;\bTheta) = \sum_{i=1}^ka_i \sigma(\bTheta_{\cdot,i}^{\sT}\bx)\, .
\end{align}
%
For simplicity, we can think of the second layer weights $(a_i)$
as fixed. The ERM problem can be recast in the form \eqref{eq:erm_obj_0}
by setting:
%
\begin{align}
\nonumber
L(\bv, y) := - y f_0(\bv) + \log (1+e^{f_0(\bv)})\, ,\;\;\;\;\;
f_0(\bv):= \sum_{i=1}^k a_i\sigma(v_i)\, .
\end{align}
%
We will use our general theory to treat this example in a companion paper \cite{OursInPreparation}.


\subsection{Summary of results: Number of local minima and topological trivialization}


The main results of this paper are:
%
\begin{enumerate}
\item A general upper bound on the exponential growth rate
of the expected number of local minima of $ \hR_n(\bTheta)$ 
in any specified domain of the parameter space.
\item An analysis of this upper bound in the case of convex losses,
showing that it implies sharp asymptotics on the properties of the empirical risk minimizer. 
\item A demonstration of how these general results can be applied to yield concrete predictions on specific problems.
We specialize to exponential families, and even more explicitly to multinomial regression.
\end{enumerate}
%
Analyzing our general 
upper bound (point 1 above) 
in the case of non-convex losses is somewhat more challenging but does still provide sharp results
in several cases. We will present those results in a separate paper \cite{OursInPreparation}. 

Omitting several technical details, let
$\hmu_{\sqrt{d}[\bTheta,\bTheta_0]}$ be the empirical distribution of the  
rows of the matrix $\sqrt{d}[\bTheta,\bTheta_0]\in\reals^{d\times (k+k_0)}$ and $\hnu_{\bX\bTheta,\by}$ be the empirical distribution of the  
rows of the matrix $[\bX\bTheta,\by]\in\reals^{k\times q}$. 
For any set of probability distributions $\cuA\subseteq \cuP(\R^{k+k_0}),\cuB\subseteq\cuP(\R^{k+q})$ (we denote by $\cuP(\Omega)$ the set
of probability measures on the Borel space $\Omega$), define
%
\begin{align}
\nonumber
\cZ_n(\cuA, \cuB):=\Big\{\mbox{ local minima of }  \hR_n(\bTheta)
\mbox{ s.t.  }  \hmu_{\sqrt{d}[\bTheta,\bTheta_0]}\in\cuA, \hnu_{\bX\bTheta,\by}\in\cuB \Big\}\, .
\end{align}
%
We consider the asymptotics $n,d\to\infty$ with $n/d\to\alpha\in(0,\infty)$, with $k,k_0$, $\bw$, $\rP$, $L$, $\rho$
fixed.
Our main result is an inequality of the form (below $|\cS|$ denotes the cardinality of set $\cS$):
%
\begin{align}
%
    \lim_{n,d\to\infty}\frac{1}{n}\log \E |\cZ_n(\cuA, \cuB)|
    \le -\inf_{\mu\in\cuA,\nu\in\cuB}\Phi(\mu,\nu)\, .\label{eq:Summary}
\end{align}
%
Via Markov inequality, such a bound allows to
localize the minimizers. To be concrete, for a test function 
$\psi:\reals^k\times\reals^{k_0}\to \reals$,
and any local minimizer $\hbTheta$
(with $\hbtheta_i$ , $\btheta_{0,i}$ the $i$-th rows of $\hbTheta$, $\bTheta_0$), we have
%
\begin{align}
\nonumber
\P\left\{\frac{1}{d}\sum_{i=1}^d\psi(\sqrt{d}\hbtheta_{i},\sqrt{d}\btheta_{0,i})
\in I\right\} \le \exp\left\{-n\inf_{\mu\in\cuA(I,\psi), \nu}
\Phi(\mu,\nu)+o(n)\right\}\, ,
\end{align}
%
where $\cuA(I,\psi): = \{\mu: \int \psi(\bt,\bt_0)\mu(\de\bt,\de\bt_0)\in I\}$.

We expect Eq.~\eqref{eq:Summary} hold with equality, although we do not attempt to prove. More crucially, we observe that in a number of
setting of interest, our bound allows to identify 
the precise limit of $\hmu_{\sqrt{d}[\bTheta,\bTheta_0]}$ and $\hnu_{\bX\bTheta,\by}$. Namely,
in many cases of interest there exist $(\mu_\star,\nu_\star)$ such that:
%
\begin{equation}\label{eq:Trivialization}
\begin{split}
    &\Phi(\mu,\nu)\ge 0 \;\;\;\forall \mu,\nu\, ,\\
    &\Phi(\mu,\nu) = 0 \;\;\Leftrightarrow  (\mu,\nu) = (\mu_\star,\nu_\star)\, ,
    \end{split}
\end{equation}
%
which implies that $\mu_\star,\nu_\star$ is the unique limit:
%
\begin{align}
\nonumber
\hmu_{\sqrt{d}[\bTheta,\bTheta_0]}\weakc \mu_\star\,,\;\;\;\;\;\;
\hnu_{\bX\bTheta,\by}\weakc\nu_\star\, .
\end{align}
We will refer to Eq.~\eqref{eq:Trivialization} as the \emph{rate trivialization property}. 
In particular, we will prove that rate trivialization
takes place for strictly convex ERM problems. 
Although strictly convex ERM problems have a unique minimizer,
rate trivialization is far from obvious because of the inequality 
in Eq.~\eqref{eq:Summary}. 
Hence, convex examples provide an important test of our general theory, 
and also an important domain of application.

In our companion paper we will characterize regimes in which 
\eqref{eq:Trivialization} holds for non-convex losses hence determining the asymptotics in those problems as well.

\subsection{Related work}

A substantial line of work studies the existence and properties
of local minima of the empirical risk for a variety of statistics 
and machine learning problems, see e.g. 
\cite{mei2018landscape,sun2018geometric,soltanolkotabi2018theoretical}. However,
concentration techniques used in these works typically require $n/d\to\infty$.

The Kac-Rice approach has a substantial history in statistical
physics where it has been used to characterize the landscape of 
mean field spin glasses 
\cite{bray1980metastable}. 
We refer in particular to the seminal works
of Fyodorov \cite{fyodorov2004complexity} and Auffinger, Ben Arous, Cern\`y \cite{auffinger2013random}, as well as to the recent papers
\cite{subag2017complexity,ros2023high} and references therein.
It has also a history in statistics, where it has been used for statistical analysis of Gaussian fields \cite{adler2009random}.

Within high-dimensional statistics, the Kac-Rice approach was first used
in \cite{arous2019landscape} to characterize 
topology of the likelihood function for Tensor PCA. 
In particular, these authors showed that the expected number of modes of the likelihood can grow exponential in the dimension, hence 
making optimization intractable.
The Kac-Rice approach was used in \cite{fan2021tap} to show that Bayes-optimal estimation 
in high dimension can be perforemd for $\integers_{2}$-synchronization via minimization of the so-called Thouless-Anderson-Palmer (TAP) free energy. These results  were substantially strengthened in \cite{celentano2023local} which proved local convexity of TAP free energy in a neighborhood of
the global minimum. 

None of the above papers studied the ERM problem that is the focus of the present paper.
The crucial challenge to apply the Kac-Rice formula to
the empirical risk of Eq.~\eqref{eq:erm_obj_0} lies in the fact that
the empirical risk function $\hR_n(\,\cdot\,)$ is not a Gaussian process
(even for Gaussian covariates $\bx_i$).
In contrast, 
\cite{arous2019landscape,fan2021tap,celentano2023local}
treat problems  for which $\hR_n$ is itself Gaussian.
In the recent review paper \cite{ros2023high}, Fyodorov and Ros 
mention non-Gaussian landscapes as an outstanding challenge even at the non-rigorous theoretical physics

While in principle Kac-Rice formula can be extended to non-Gaussian processes
(provided the gradient of $\hR_n$ has a density), this creates technical difficulties.  
In a notable paper, Maillard, Ben Arous, Biroli \cite{maillard2020landscape} 
followed this route to study (a special case of) the ERM problem
of Eq.~\eqref{eq:erm_obj_0}. They derive an upper bound
of the form \eqref{eq:Summary}, albeit with a different function 
$\Phi(\mu,\nu)$. However, it is unknown whether their bound has
the rate trivialization property \eqref{eq:Trivialization},
even when applied to convex ERM problems\footnote{Strictly speaking, \cite{maillard2020landscape} does not apply to any non-quadratic convex risk function, but replicating their proof in a more general settings leads us to this conclusion.}.

Our approach is quite different from earlier works.
We apply Kac-Rice formula to a process in $(n+d)k$ dimensions,
that we refer to as the `gradient process.'
The gradient process extends the gradient $\nabla \hR_n(\bTheta)$ and has two important additional properties: it is Gaussian;
zeros of the gradient process (with certain additional conditions) correspond to local minima of the empirical risk.

We finally note that, in the case of convex losses, an alternative proof technique is available,
which is based on approximate message passing (AMP) algorithms. This approach was initially developed to
analyze the Lasso \cite{BayatiMontanariLASSO} and subsequently refined and extended, see e.g.
\cite{donoho2016high,sur2019modern,loureiro2021learning,ccakmak2024convergence}.
While our main motivation is to move beyond convexity, our approach recovers and unifies,
with some distinctive advantages. 

\subsection*{Notations}
We denote by $\cuP(\Omega)$ the set of (Borel) probability measure on 
 $\Omega$ (which will always be Polish equipped with its Borel $\sigma$-field). 
Additionally, we denote by $\cuP_n(\Omega)\subset \cuP(\Omega)$ the set of empirical measures on $\Omega$ of $n$ atoms.
 Throughout we assume $\cuP(\Omega)$ is endowed with the topology induced by the Lipschitz bounded metric, defined by
\begin{equation}
\nonumber
\dBL(\mu,\nu) = \sup_{\vartheta\in \cF_{\textrm{LU}}} 
    \left|\int \vartheta(\omega) \,\mu(\de\omega) - \int \vartheta(\omega) \,\nu(\de\omega)\right|
    \, ,
\end{equation}
where $\cF_{\mathrm{LU}}$ is the class of Lipschitz continuous functions $\vartheta:\Omega\to \R$ with Lipschitz constant at most 1 and uniformly  bounded by 1.
We will also use $W_2(\mu,\nu)$ to denote the Wasserstein 2-distance.

etting $\Omega = \Omega_1\times\Omega_2$, $\ba_2 $ a generic point in $\Omega_2$, and $\mu\in \cuP(\Omega)$, we denote by $\mu_{\cdot|\ba_2}\in\cuP(\Omega_1)$ to be the regular conditional distribution of $\mu$ given $\ba_2$.  
We also denote by $\mu_{(\ba_2)}\in\cuP(\Omega_2)$  the restriction of $\mu$ to $\Omega_2$ (refer to  Section D.3 \cite{dembo2009large} for more about product spaces).

We let $\sS^k$ be the
set of $k\times k$ symmetric matrices with entries in $\R$, and 
$\sS^k_{\succeq 0},\sS^k_{\succ 0}$ the subsets of positive semidefinite, positive definite matrices, respectively.
For a matrix $\bZ \in\C^{k\times k}$, define
\begin{equation}
\nonumber
\Re(\bZ) =  \frac12 \left(\bZ + \bZ^*\right),\quad
    \Im(\bZ) := \frac1{2i } \left(\bZ - \bZ^*\right)\, ,
\end{equation}
and let 
\begin{equation}
\nonumber
\bbH_+^k := \left\{\bZ \in\C^{k\times k} : {\Im(\bZ)} \succ \bzero\right\},
\quad 
\bbH_-^k := \left\{\bZ \in\C^{k\times k} : \Im(\bZ) \prec \bzero\right\}.
\end{equation}
Note that $\Re(\bZ)$ and $\Im(\bZ)$ are self-adjoint for any $\bZ$.
Given a self-adjoint matrix $\bA\in\C^{n\times n}$, we denote by $\lambda_1(\bA)\ge,
\dots,\ge \lambda_n(\bA)\in\reals$ its  eigenvalues, and by
$\lambda_{\min}(\bA), \lambda_{\max}(\bA)$ the minimum and  maximum
eigenvalues, respectively. 
For $m\le n$, we use $\sigma_1(\bA)\ge \dots \ge\sigma_{m}(\bA)$ for the singular values of a matrix $\bA\in\C^{m\times n}$, and $\sigma_{\max}=\sigma_1,\sigma_{\min}=\sigma_m$ to denote the minimum and maximum of these.
We use $\grad_\ba f \in \R^m,\grad_\ba^2 f \in\R^{m\times m}$ to denote the Euclidean gradient and Hessian of $f:\R^m \to \R$ with respect to $\ba\in\R^m$, for some $m$. Furthermore, for function $f :\R^{m\times n} \to \R$, $\grad^2_\bA f(\bA)  \in\R^{mn\times mn}$, 
in the Hessian defined by identifying $\R^{m\times n}$ with $\R^{mn}$.
Similarly, for a function $\boldf :\R^{m} \to \R^{n},$  we denote by $\bJ_\ba \boldf \in\R^{n \times m}$ its Jacobian matrix with respect to $\ba\in\R^m$. For $\bF:\R^{n\times m} \to \R^{k\times p},$ 
we use $\bJ_{\bA} \bF \in\R^{kp \times nm}$ to denote the Jacobian with respect to $\bA \in\R^{n\times m}$ after vectorizing the input and output by concatenating the columns. We'll often omit the argument in the subscript when it is clear from context.

We use $|S|$ to denote the cardinality of set $S$.
We denote the Euclidean ball of radius $r$ and center $\ba$ in $\reals^d$ by $\Ball^d_r(\ba)$. We similarly use $\Ball^{n\times d}_r(\bA)$ to denote the Frobenius norm ball of matrices centered at $\bA\in\R^{n\times d}$.
%
For two distributions $\nu,\mu$, we use
$\KL(\nu\|\mu)$ to denote their KL-divergence.
%*****************************************************************
%*****************************************************************
%
\section{Main results I: General empirical risk minimization}
\subsection{Definitions}
\label{sec:definitions}

As  already stated, we assume $\by_i = \bphi(\bTheta_0^\sT\bx_i,w_i)$
for $w_i$ independent of $\bx_i\sim\normal(0,\bI_d)$.
In our general treatment, we will explicate the dependence of $\by_i$ on $\bTheta_0^\sT,\bx_i,w_i$ and write for each $i\in[n]$,
\begin{equation}
\nonumber
    \ell(\bTheta^\sT \bx_i, \bTheta_0{^\sT} \bx_i, w_i) = L(\bTheta^\sT\bx_i, \bphi(\bTheta_0^\sT\bx_i,w_i)).
\end{equation}
Hence, from a mathematical viewpoint, the empirical risk \eqref{eq:erm_obj_0} is equivalent 
to
\begin{equation}
\label{eq:erm_obj}
    \widehat R_n(\bTheta) = \frac1n \sum_{i=1}^n \ell(\bTheta^\sT \bx_i, \bTheta_0{^\sT} \bx_i, w_i) + \frac1d\sum_{j=1}^d \sum_{i=1}^k\rho(\sqrt{d}\Theta_{i,j})\, .
\end{equation}
(Of course, from the statistical viewpoint, estimation proceeds by minimizing \eqref{eq:erm_obj_0}, without knowledge of $\bTheta_0$.)

Here
    $\bx_i\in\R^d,
    \bTheta = (\Theta_{i,j})_{i\in[d],j\in[k]} \in \R^{d\times k}$, $\bTheta_0\in \R^{d \times k_0}$,
    $\bw:= (w_1,\dots,w_n) \in \R^n$, $\ell : \R^{k+k^*+1} \to \R$,
     $(\bu,\bv,w)\mapsto \ell(\bu,\bv,w)$, and
$\rho :\R \to \R$.
Recall that
$\hmu_{\sqrt{d}[\bTheta,\bTheta_0]}$ denotes the empirical distribution of
rows of $\sqrt{d}\begin{bmatrix}\bTheta,\bTheta_0\end{bmatrix}\in \R^{d\times (k+k_0)}$. We further define 
%
\begin{align}
\bR(\hmu_{\sqrt{d}[\bTheta,\bTheta_0]}) :=\begin{bmatrix}
\bTheta^{\sT}\bTheta & \bTheta^{\sT}\bTheta_0\\
\bTheta_0^{\sT}\bTheta & \bTheta_0^{\sT}\bTheta_0\\
\end{bmatrix}=\int \bt\bt^{\sT} \hmu_{\sqrt{d}[\bTheta,\bTheta_0]}(\de\bt)\, .
\end{align}
%
Given block matrix $\bR\in \sS_{k+k_0}$ we define the Schur complement of  the $k_0\times k_0$ block as:
%
\begin{align}
\nonumber
\bR = \begin{bmatrix}
\bR_{11} & \bR_{10}\\
\bR_{01} & \bR_{00}\\
\end{bmatrix}
\;\;\Rightarrow\;\; \bR/\bR_{00} = \bR_{11}-\bR_{10}\bR_{00}^{-1}
\bR_{01}\, .
\end{align}
%

We will consider local minima of the ERM problem satisfying a set of
constraints  specified by the parameters:
%
\begin{equation}
\nonumber
    \sPi := (\sfsigma_{\bH}(n), \sfsigma_{\bG}(n),
    \sfsigma_{\bL},\sfsigma_\bV,\sfsigma_\bR,\sfA_{\bR},\sfA_\bV)\, . %\quad\textrm{for}\quad
    %\sfA_{\bR},\sfA_\bV \ge 1, 
\end{equation}
Namely,  for $\cuA\subseteq \cuP(\R^{k+k_0}),\cuB\subseteq\cuP(\R^{k+k_0}\times \R)$, 
we define 
%
\begin{align}
\label{eq:number_of_zeros_main}
& Z_n(\cuA,\cuB,\bTheta_0,\bw,\sPi) :=  |\cZ_n(\cuA,\cuB,\bTheta_0,\bw,\sPi)|\, ,
\end{align}
where
%
\begin{align}
\label{eq:set_of_zeros_main}
\cZ_n&(\cuA,\cuB,\bTheta_0,\bw,\sPi):=\\
&\Big\{\bTheta\in \R^{d\times k}:\; 
\nabla \hR_n(\bTheta)=\bzero\,, \hmu_{\sqrt{d}[\bTheta,\bTheta_0]} 
\in \cuA,\;\hnu:=\hnu_{\bX\bTheta,\bX\bTheta_0,\bw} \in\cuB,\;
\nabla^2 \hR_n(\bTheta)\succeq \sfsigma_\bH,
\nonumber \\
& 
\sfA_{\bR} \succ \bR(\hmu_{\sqrt{d}[\bTheta,\bTheta_0]}) \succ\sfsigma_\bR,
\sfA_\bV \succ \E_{\hnu}[\bv\bv^\sT] \succ \sfsigma_{\bV}, \;
\E_{\hnu}[\grad\ell \grad\ell^\sT] \succ \sfsigma_{\bL},\;
\sigma_{\min}\left( \bJ_{(\bV,\bV_0,\bTheta)} \bG\right) > n\,\sfsigma_{\bG}(n)
\Big\}\,,\nonumber
\end{align}
where 
%
\begin{enumerate}
%
\item $\hnu_{\bX\bTheta,\bX\bTheta_0,\bw}$ denotes the empirical distribution of rows of $[\bX\bTheta,\bX\bTheta_0,\bw]\in\R^{n\times(k+k_0+1)}$.
\item For $\bV^\sT = (\bv_j^\sT)_{j\in[k]} \in \R^{k\times d}, \bV_0^\sT = (\bv_{0,j}^\sT)_{j\in[k_0]} \in\R^{k_{0}\times d}$ and $\bw\in\R^n$, defining
\begin{equation}
\label{eq:def_bL_bRho}
  \bL(\bV,\bV_0;\bw)=\left(\frac{\partial}{\partial_{v_j}} \ell(\bv_i,\bv_{0,i}, w_i)\right)_{i\in[n],j\in[k]}\quad
  \textrm{and}
  \quad\quad
  \bRho(\bTheta) :=  \frac1{\sqrt{d}}\left(\rho'(\sqrt{d} \Theta_{i,j})\right)_{i\in[d],j\in[k]},
\end{equation}

\begin{equation}
\label{eq:def_G}
\bG(\bV,\bV_0,\bTheta;\bw) := \frac1n \bL(\bV,\bV_0;\bw)^\sT[\bV,\bV_0] + \bRho(\bTheta)^\sT[\bTheta,\bTheta_0] \in\R^{k\times(k+k_0)}\, ,
\end{equation}
and $\bJ_{(\bV,\bV_0,\bTheta)} \bG$ denotes the Jacobian matrix of this map.
\end{enumerate}

For $z\in\C$ with $\Im(z) >0$, $\bS\in\reals^{k\times k}$ a symmetric matrix
and $\nu\in\cuP^{\R^{k+k_0+1}}$,
%a probability distribution over the  
%$k\times k$ symmetric matrix $\bD$, 
we let  $\bF_z(\bS; \nu)\in\C^{k\times k}$
\begin{equation}
     \bF_z(\bS; \nu) := \Big( \E_\nu[(\bI + \grad^{2}\ell(\bv,\bv_0,w)\bS )^{-1}
     \grad^{2}\ell(\bv,\bv_0,w)
     ] - z \bI \Big)^{-1}\,.
\end{equation}
%
We let $\bS_\star(\nu;z)$ be the the unique solution to 
\begin{equation}
\label{eq:fp_eq}
\bF_z(\bS_\star;\nu )= \alpha\bS_\star,
\end{equation}
%
and define $\bS_0(\nu) = \lim_{\eps\to 0} \bS_\star(\nu;i\eps)$.
Existence and uniqueness of $\bS_*(\nu;z)$ and  $\bS_0(\nu)$ are proven in 
Appendix~\ref{sec:RMT}, which extends classical results on the Marchenko-Pastur law. 
Appendix~\ref{sec:RMT} also proves that the function
\begin{equation}
\nonumber
    z \mapsto \frac1k \Tr(\alpha\bS_\star(\nu;z))
\end{equation}
%
is the  Stieltjes transform on a unique measure on $\reals$ which we denote by
$\mu_{\MP}(\nu)\in\cuP(\R)$.

Observe that for any fixed $\bTheta$, the empirical spectral distribution  of the Hessian of the regularization term
(as an element of $\R^{dk \times dk}$) is given by
\begin{equation}
\nonumber
    \mu_{\rho}(\bTheta) :=  \frac1{dk}\sum_{j=1}^k \sum_{i=1}^d \delta_{\rho''(\sqrt{d}\Theta_{i,j})}
    = \frac1k \sum_{j=1}^k \rho''_{\#} \hmu_{j} ,\quad\quad\textrm{where}\quad\quad
    \hmu_j := \hmu_{\sqrt{d}\bTheta_{\cdot, j}}.
\end{equation}
We will often abuse notation and denote this by $\mu_{\rho}( \hmu_{\sqrt{d}\bTheta})$ since this is purely a function of the empirical distribution of $\bTheta.$
We let 
\begin{equation}
\nonumber
    \mu_\star(\nu,\mu) := \mu_{\MP}(\nu) \boxplus \mu_{\rho}(\mu) 
\end{equation}
where $\boxplus$ denotes the free additive convolution.

Finally, we recall the definition of the multivariate proximal operator. 
For $\bz \in\R^k, \bv_0 \in\R^{k_0}, \bS\in\R^{k\times k}, w\in\R,$  and $f : \R^{k+k_0+ 1} \to\R,$ let
\begin{equation}
\nonumber
    \Prox_{f(\cdot, \bv_0, w)}(\bz;\bS):=\arg\min_{\bx\in\R^k}\left( \frac12(\bx-\bz)^\bT\bS^{-1}(\bx-\bz) + f(\bx,\bv_0,w)\right)\in\R^k.
\end{equation}






\subsection{General assumptions}

\label{sec:assumptions}

We will make the following assumptions.
\begin{assumption}[Regime]
\label{ass:regime}
We assume the proportional asymptotics, i.e., $d := d_n$ such that
   \begin{equation}
       \alpha_n := \frac{n}{d_n} \to \alpha \in (1, \infty), \quad\quad\textrm{and} \quad\quad k,k_0 \le C
   \end{equation}
   for some universal constant $C$ independent of $n$.
Furthermore, for all $n,$ we have $n>d_n + k$.
\end{assumption}

\begin{assumption}[Loss]
\label{ass:loss}
\label{ass:density}
Assume that the following partial derivatives of $\ell:\R^{k+k_0+ 1} \to\R$ exist and are Lipschitz:
%
\begin{equation}
\nonumber
    \frac{\partial}{\partial u_l}\ell(\bu),\quad\quad
    \frac{\partial^2}{\partial {u_j} \partial{u_l}} \ell(\bu),\quad\quad
    \frac{\partial^3}{\partial u_i \partial u_j \partial u_l}\ell(\bu), \quad\quad\textrm{for}\quad\quad i,j\in[k+k_0], l\in[k], 
\end{equation}
%
Furthermore, for any $\bw\in\supp(\P_\bw)^n, \bTheta \in\R^{d\times k}$, $\bTheta_0 \in\R^{d \times k_0}$ and $l\in[k]$, the random variable
    \begin{equation}
\nonumber
        \frac{\partial}{\partial{u_l}}\ell(\bTheta^\sT\bx, \bTheta_0^\sT \bx, \bw),\quad 
        \bx\sim \cN(0,\bI_d)
    \end{equation}
    has a (bounded) density in a neighborhood of $0$, for all  
    $\bTheta, \bTheta_0$ such that $\sfsigma_\bR \prec \bR(\bTheta,\bTheta_0) \prec  \sfA_\bR$. 
\end{assumption}
\begin{assumption}[Regularizer]
\label{ass:regularizer}
The regularizer $\rho:\R \to \R$ 
is three times continuously differentiable. 
\end{assumption}

\begin{assumption}[Constraint sets]
\label{ass:sets}
The constraint sets $\cuA\subseteq \cuP(\R^{k+k_0}),\cuB\subseteq \cuP(\R^{k+k_0+1})$ are measurable, and 
\begin{equation}
\nonumber
    \{\bTheta :
    \in\R^{d\times k} : \hmu_{\sqrt{d}[\bTheta,\bTheta_0]} \in \cuA\},\quad\quad
    \{\bbV \in\R^{n\times (k+k_0)} : \hnu_{\bbV,\bw} \in\cuB\} 
\end{equation}
are open for any fixed $\bw \in \R^n$.
%open in the topology induced by the bounded Lipschitz metric.
\end{assumption}

\begin{assumption}[Data generation]\label{ass:Data}
The covariates/noise pairs $((\bx_i,w_i):i\le n)$ are i.i.d. with $(\bx_i,w_i)\sim\cN(\bzero,\bI)\otimes \P_w$. 
\end{assumption}

\begin{assumption}[Distribution of $\bTheta_0$]
\label{ass:theta_0}
There exists a constant $c$ independent of $n$ such that 
$\sigma_{\min}(\bTheta_0) \succ c \bI$.
   Furthermore, for some $\mu_0 \in\cuP(\R^{k_0})$, we have the convergence 
   \begin{equation}
\nonumber
       \hmu_{\sqrt{d}\bTheta_0} \stackrel{W_2}{\Rightarrow} \mu_0.
   \end{equation}
\end{assumption}


\begin{assumption}[The parameters $\sPi$]
\label{ass:params}
 The parameters $\sfsigma_{\bL},\sfsigma_\bV,\sfsigma_\bR,\sfA_{\bR},\sfA_\bV$ are fixed independent of $n,d$ in the proportional asymptotics. Parameters $\sfsigma_{\bH}(n)$, $\sfsigma_{\bG}(n)$ can depend on $n$, subject to
 $\sfsigma_{\bH}(n) = e^{-o(n)}$, $\sfsigma_{\bG}(n) = e^{-o(n)}$.
\end{assumption}

%\bns{Add the technical bound so that $\sfsigma \le 1$ and $\sfA \ge 1$} 



\subsection{General variational upper bound on the number of critical points}
%
Our main theorem gives a general formula that upper bounds the number of critical points of the risk~\eqref{eq:erm_obj}. 
To state it, let $\cuV := \cuV(\cuA,\cuB,\sPi,\P_\bw,\smu_{0})$ be defined by
\begin{align}
\nonumber
 \cuV(\cuA,\cuB,\sPi,\P_\bw,\smu_{0}) := \Big\{(\mu,\nu) \in \cuA\times \cuB &:\;
\sfA_{\bR} \succ \bR(\mu) \succ\sfsigma_\bR,\;
\sfA_\bV \succ \E_{\nu}[\bv\bv^\sT] \succ \sfsigma_{\bV},\;
\E_{\nu}[\grad\ell \grad\ell^\sT] \succ \sfsigma_{\bL},\;\\
&\quad\quad\nonumber
\E_\nu[\grad \ell(\bv,\bv_0,w)(\bv,\bv_0)^\sT + 
     \E_\mu[\rho'(\btheta) (\btheta, \btheta_0)^\sT] =   \bzero_{k\times (k+k_0)},\\
    &\quad\quad
\mu_{(\btheta_0)} = \smu_{0},\; \nu_{(w)} = \P_w,\;
\mu_{\star}(\mu,\nu)((-\infty,0)) = 0
\Big\}.\label{eq:GeneralSet}
\end{align}
%
Further, define $\Phi_\gen:\cuP(\R^{k}\times \R^{k_0})\times \cuP(\R^{k+k_*+1})\times\sS^{k+k_0}\to\reals $
via
\begin{align}
\Phi_\gen(\mu,\nu,\bR)&:=-\frac{k}{2\alpha}\log(\alpha)
-\frac{k}{\alpha}
\int \log(\lambda ) \mu_{\star}(\nu,\mu)(\de \lambda)
  + \frac{1}{2\alpha}\log \det\left( \E_{\nu}[\grad \ell\grad\ell^\sT]\right)
  - \frac{1}{2\alpha} \Tr\left(\bR_{11}\right) 
 \nonumber \\
  &\quad+ \frac{1}{2\alpha}\log \det\left( \E_{\nu}[\grad \ell\grad\ell^\sT]\right)
+\frac1{2}\log\det(\bR)
 - \frac12 \Tr\left((\bI_{k+k_0} - \bR^{-1})\E_\nu[[\bv^\sT,\bv_0^\sT]^\sT [\bv^\sT,\bv_0^\sT]]\right)  
\nonumber\\
   &\quad
+\frac1{2}
\Tr\left( (\E_\nu[\grad\ell\grad\ell^\sT])^{-1}\E_\mu[\grad \rho \grad \rho^\sT]\right)\label{eq:PhiGen} \\
  &\quad- \frac1{2}\Tr\left(\E_\nu\left[[\bv^\sT,\bv_0^\sT]^\sT\grad\ell^\sT\right] (\E_\nu[\grad\ell\grad\ell^\sT])^{-1}\E_\nu[\grad\ell[\bv^\sT,\bv_0^\sT]^\sT] \bR^{-1}\right)\nonumber\\
%
    &\quad
   +\KL( \nu_{\cdot |w} \|\cN(0,\bI_{k+k_0}) ) + \frac1\alpha\KL(\mu_{\cdot | {\btheta_0}}\| \cN(\bzero, \bI_{k})).\nonumber
\end{align}


\begin{theorem}
\label{thm:general}
For $\delta>0$, define
\begin{equation}
\nonumber
    \cG_\delta := 
    \{\bw \in\Ball_{\sfA_{w}\sqrt{n}}^n (\bzero)  : \dBL(\hnu_{\bw}, \P_w) < \delta
    \}\, .
\end{equation}
Then under  Assumptions \ref{ass:regime} to \ref{ass:params} of Section~\ref{sec:assumptions}, we have
\begin{align}
\nonumber
 \lim_{\delta\to0}\lim_{n\to\infty}\frac{1}{n}\log\E_{\bX,\bw}\left[Z_n(\cuA,\cuB,\bTheta_*,\bw,\sPi) \one_{\cG_\delta}\right]
    \le- \inf_{(\mu,\nu) \in \cuV} \Phi_\gen(\mu,\nu, \bR(\mu)).
\end{align}
%
%
\end{theorem}




\section{Main results II: Convex empirical risk minimization}


\subsection{Assumptions for convex empirical risk minimization}

We specialize our treatment to the case of convex losses and ridge regularization, 
making the following additional assumptions.
%
\begin{assumption}[Convex loss and ridge regularizer]
\label{ass:convexity}    
We have
$\grad_{\bv}^2 \ell(\bv,\bv_0,w) \succeq\bzero$ for all
$(\bv,\bv_0,w)\in\reals^{k+k_0+1}$. 
%the function $\bv \mapsto \ell(\bv,\bv_0,w)$ is convex 
%for almost every $(\bv,\bv_0,w) \sim \nu$, and every $\nu\in \cuB$. 
Furthermore, for some $\lambda\ge 0$ fixed, we take 
$$\rho(t) = \frac{\lambda}{2}t^2.$$
\end{assumption}

\begin{assumption}[Distribution of $\bw$]
\label{ass:noise}
The noise variables $w_i \stackrel{i.i.d.}{\sim} \P_{w}$ 
are subgaussian with subgaussian constant bounded by a constant $C>0$ independent of $n$.
\end{assumption}


\subsection{Simplified variational upper bound for convex losses and ridge regularizer}
We specialize our result to the case of convex loss and ridge regularization.  
In doing so, our approach recovers and extends results obtained with alternative technique.
We provide a comparison in Remark \ref{rmk:ComparisonCVX}. Most importantly, within our treatment,
the asymptotics for convex losses is a special case of a more general result.

The next definition plays a crucial role in what follows.
%
\begin{definition}
\label{def:opt_FP_conds}
   We say that the pair $(\mu^\opt,\nu^\opt)\in\cuP(\R^{k+k_0}) \times \cuP(\R^{k+k_0+1})$ satisfies the \emph{critical point optimality condition} if the following holds. The pair $(\bR,\bS) \in \R^{(k+k_0)\times (k+k_0)} \times \R^{k\times k}$ satisfy the following set of equations
%\
\begin{align}
\label{eq:opt_fp_eqs}
    & \alpha \E_{\nu^\opt}[\grad\ell(\bv,\bv_0,W)\grad\ell(\bv,\bv_0,w)^\sT]
    =\bS^{-1} (\bR/\bR_{00}) \bS^{-1}
    \\
    &\E_{\nu^\opt}\left[ \grad \ell(\bv,\bv_0,w) (\bv,\bv_0)^\sT\right] + \lambda (\bR_{11},\bR_{10}) = \bzero_{k\times (k+k_0)},
\end{align}
where 
%
   \begin{align}
&\nu^\opt = \mathrm{Law}(
\Prox( \bg; \bS, \bg_0, w),\bg_0,w),\quad\quad\textrm{where}\quad\quad
(\bg^\sT,\bg_0^\sT)^\sT \sim \cN\left( \bzero_{k+k_0},\bR\right)\;\indep \;w\sim\P_W,
\\
&\mu^\opt_{(\btheta_0)} = \mu_{0},\quad  \mu^\opt_{\cdot|\btheta_0}= \cN(\bR_{10}({\bR_{00}})^{-1}\btheta_0,\bR/\bR_{00}), \quad\quad\textrm{for}\quad\quad\btheta_0\in\R^{k_0},\label{eq:muopt}
    %&\E_{\nu}\left[ \grad \ell(\bv,\bv_0,W) (\bv,\bv_0)^\sT\right] + \lambda (\bR_{11},\bR_{10}) = \bzero_{k\times (k+k_0)}.
\end{align}
and we introduced the notation $\Prox( \bg; \bS, \bg_0, w) = 
\Prox_{\ell(\,\cdot\,,\bg_0,w)}( \bg; \bS)$.
   %$\bR = \bR(\mu)$,
\end{definition}


\begin{theorem}[Rate function under convexity]
\label{thm:convexity}
Consider the setting of Theorem \ref{thm:general}.
Let Assumptions \ref{ass:regime} to \ref{ass:convexity} hold, and
define 
%
\begin{align}
\Phi_\cvx(\nu,\mu,\bS,\bR)
&:=
    -\lambda\Tr(\bS)  -
    \E_{\nu}[\log\det(\bI + \grad^2 \ell(\bv,\bu, w)\bS) ]  +\frac1\alpha \log\det(\bS) + \frac{k}{2\alpha} \log(\alpha) +\frac{k}{\alpha}\\
  &\quad+ \frac{1}{2\alpha}\log \det\left( \E_{\nu}[\grad \ell\grad\ell^\sT]\right)
- \frac{1}{2\alpha} \Tr\left(\bR_{11}\right)
% &\quad-\frac{\lambda^2}{2\alpha}
%\Tr\left( (\E_\nu[\grad\ell\grad\ell^\sT])^{-1}\bR_{11}\right) + \frac{\lambda^2}{2\alpha}\Tr\left( (\E[\grad\ell\grad\ell^\sT])^{-1} [\bR_{11},\bR_{00}] \bR^{-1}[\bR_{11},\bR_{00}]^\sT\right).
+  \KL(\nu_{\cdot|w} \| \cN(\bzero, \bR)) + \frac1\alpha \KL ( \mu_{\cdot| \btheta_0}\| \cN(\bzero,\bI_k)).\nonumber
\end{align}
Further, define the sets 
\begin{align}
\nonumber
\cuV(\bR, \cuB) := \Big\{\nu \in  \cuB &:
\sfA_\bV \succ \E_{\nu}[\bv\bv^\sT] \succ \sfsigma_{\bV},\;
\E_{\nu}[\grad\ell \grad\ell^\sT] \succ \sfsigma_{\bL},\\
&\E_\nu[\grad \ell(\bv,\bv_0,w)(\bv,\bv_0)^\sT] + 
     \lambda (\bR_{11},\bR_{1,0}) =   \bzero,\;
 \nu_{(w)} = \P_w\; \Big\}
\end{align}
and
\begin{equation}
   \cuT(\cuA) := \{\mu \in\cuA : \mu_{(\btheta_0)} = \mu_{0}, \; 
   \sfA_{\bR} \succ \bR(\mu) \succ\sfsigma_\bR,\;
   \}.
\end{equation}
Then the following hold.
\begin{enumerate}
    \item Upper bound on the rate function:
\begin{align}
\nonumber
   &\limsup_{\delta\to 0 }\limsup_{n\to\infty}
   \frac1n\log (\E[Z_n(\cuA,\cuB, \sPi,\bw) \one_{\bw\in\cG_\delta}] )
   \le 
-\inf_{\mu\in \cuT(\cuA)}
   \inf_{\nu\in\cuV(\bR(\mu),\cuB)}\sup_{\bS\succ\bzero}
   \Phi_\cvx(\nu,\mu,\bS,\bR(\mu)).
\nonumber
\end{align}
\item For any $\mu\in\cuT(\cuA)$, $\nu \in\cuV(\bR(\mu),\cuB)$, 
\begin{equation}
\nonumber
    \sup_{\bS \succ\bzero} \Phi_\cvx(\mu,\nu, \bR(\mu),\bS) \ge 0
\end{equation}
with equality if and only if $(\mu,\nu)$ satisfy the critical point optimality condition 
of Definition~\ref{def:opt_FP_conds}.
\end{enumerate}
\end{theorem}


\begin{remark}[Comparison with related work]\label{rmk:ComparisonCVX}
As mentioned in the introduction, several earlier papers analyzed convex ERM in settings analogous to the one considered
here using AMP algorithms \cite{bayati2010combinatorial,donoho2016high,sur2019modern}. The idea is to construct an AMP algorithm minimize the empirical risk and establish sharp asymptotics for the minimzer in two steps: $(1)$~Characterize the high-dimensional asymptotics of AMP via state evolution (for any constant number of iterations); 
$(2)$~Prove that AMP converges to the minimizer in $O(1)$ iterations. 
For the case $k>1$, this approach was initiated in \cite{loureiro2021learning}, with important steps completed in \cite{ccakmak2024convergence}, albeit under the assumption of strong convexity. 

Even in the convex case, the Kac-Rice approach has some important advantages. Most notably, it can be used to characterizes \emph{all}
potential minimizers, 
via point 2 of Theorem \ref{thm:convexity}, as all solutions of the critical point optimality conditions of Definition \ref{def:opt_FP_conds}. 
Indeed, while Theorem \ref{thm:convexity} focuses on second-order strict minimizers, other minimizers can be accessed by a simple perturbation argument.
Further, this result emerges as a special case of a more general one, which is the main point of the paper.
\end{remark}
%
%
%*****************************************************************
%
\subsection{Rate trivialization under convexity}

\begin{theorem}
\label{thm:global_min}
With the definitions of Theorem~\ref{thm:convexity}, 
let Assumptions~\ref{ass:regime},~\ref{ass:loss},~\ref{ass:Data},~\ref{ass:theta_0},~\ref{ass:convexity},~\ref{ass:noise}
hold.

Assume there exists unique $\mu^\opt,\nu^\opt$ that satisfy 
 the critical point optimality condition of Definition~\ref{def:opt_FP_conds}, 
and 
there exist constants $C,c>0$ and (for each $n$) a critical point $\hat\bTheta_n\in\reals^{d\times k}$
such that, with high probability, 
%
\begin{align}
\hat\bTheta_n \in \cE(\bTheta_0) := \Big\{
\bTheta\in\R^{k\times d} &: 
\nabla^2 \hR_n(\bTheta)\succeq e^{-o(n)},
\,
 \bR(\hmu_{\sqrt{d}[\bTheta,\bTheta_0]})\prec C\bI,\; %\succ c,
%C \succ \E_{\hnu}[\bv\bv^\sT] \succ c, \;
\E_{\hnu}[\grad\ell \grad\ell^\sT] \succ c\bI
\Big\}\, ,\label{eq:SetUniqueness}
\end{align} 
%
where the expectation $\E_{\hnu}[\, \cdot\,]$ is with respect to $\hnu=\hnu_{\bX\hbTheta,\bX\bTheta_0,\bw}$.
(In particular, by convexity $\hat\bTheta_n$ is the unique empirical risk minimizer.)
Then the following hold:
\begin{enumerate}
    \item We have, in probability, 
    \begin{equation}
    \hmu_{\sqrt{d} [\hat\bTheta_n,\bTheta_0 ]} \stackrel{W_2}{\Rightarrow} \mu^\opt ,\quad\quad
    \hnu_{(\bX\hat\bTheta_n,\bX\bTheta_0,\bw)} \stackrel{W_2}{\Rightarrow} \nu^\opt\, .
    \end{equation}
    %
    \item The empirical spectral distribution of the Hessian at the minimizer converges weakly to $\mu_\star(\nu^\opt)$ in probability. Equivalently,  for all $z \in\bbH_+$, the following limit holds in probability:
    \begin{equation}
    \lim_{n\to\infty}\frac1{dk} (\bI_k \otimes \Tr) \left(\grad^2\hat R_n(\hat\bTheta_n) - z \bI_{dk} \right)^{-1} = 
      \alpha \, \bS_\star(\nu^\opt, z -\lambda )\, .
    \end{equation}
    \item Finally  a sufficient condition for  the solution to the critical point optimality condition  
    to be unique is stated in Theorem \ref{prop:simple_critical_point_variational_formula} below.
\end{enumerate}
\end{theorem}

\begin{remark}
   Notice that the set in Eq.~\eqref{eq:SetUniqueness} does not include any constraint on the
 Jacobian of $\bG$, which instead enters the set \eqref{eq:GeneralSet} of Theorem~\ref{thm:general}. 
 Indeed, a key step in the proof Theorem \ref{thm:global_min}  is to lower bound the minimum singular value of the Jacobian of $\bG$ in terms of the minimum singular values of $[\bTheta,\bTheta_0]$ and the Hessian of $\hR_n$.
 (See Lemma~\ref{lemma:jacobian_lb} in the appendix, or the statement of its result in Section~\ref{sec:pf_convex_results} )
\end{remark}

We finally show that the the solution of the critical point optimality condition of 
Definition~\ref{def:opt_FP_conds}  is unique for strictly convex ERM problems. 
A crucial observation is that, under Assumption~\ref{ass:convexity}
(with $\cuB = \cuP(\R^{k+k_0+1})$),  $\bR^\opt,\bS^\opt$ solve the fixed point equations in Definition~\ref{def:opt_FP_conds}, 
if they are a stationary point of the following min-max problem:
   \begin{equation}
   \label{eq:min_max_critical_pts}
       \min_{\bR \succeq \bzero} \max_{\bS\succeq\bzero} \left\{
       \E\left[\More_{\ell(\cdot, \bg_0,w)}(\bg;\bS)\right] - \frac1{2\alpha}\Tr(\bS^{-1}(\bR/\bR_{00})) 
       + \frac{\lambda}{2} \Tr(\bR_{11})\right\}
   \end{equation}
   where 
   \begin{align}
   \label{eq:moreau_def}
       \More_{\ell(\cdot, \bg_0,w)}(\bg;\bS)
       := \min_{\bx \in\R^{k}}\left\{ \frac12(\bx - \bg)^\sT\bS^{-1}(\bx - \bg) + \ell(\bx,\bg_0 ,w) \right\}
   \end{align}
   is the \emph{Moreau envelope}, and  the expectation is over $(\bg,\bg_0) \sim \cN(\bzero,\bR), w\sim \P_w$. (See the proof of
   Theorem \ref{prop:simple_critical_point_variational_formula} for a proof of this claim.)
   
 The next theorem shows that $\bR^{\opt}$ also corresponds to the minimizer of a convex program
 which is unique under strict convexity.
 %
\begin{theorem}
\label{prop:simple_critical_point_variational_formula}
\label{thm:simple_critical_point_variational_formula}
Under Assumption~\ref{ass:convexity}, any solution $\bR^\opt$ of critical point optimality condition of 
Definition~\ref{def:opt_FP_conds}   corresponds to a minimizer of the following
convex program. 
Let $(\Omega,\cF,\P)$ be a probability space with independent random variables 
$\bz_0\sim\normal(\bzero,\bI_{k_0})$,
$\bz_1\sim\normal(0,\bI_k)$ and
$w\sim\P_w$ defined on it, and additional independent randomness $Z\sim \Unif([0,1))$,
and $L^2 = L^2(\Omega\to\R^k,\cF,\P)$. Define
\begin{equation}
\nonumber
    %\cS(\bSigma) :=
    %\left\{(\bu, \bB)  \in L_2 \times  \sfS^k : 
    %\E[\bu\bu^\sT]  \preceq \bB,\;
    %\bB \bSigma^{-1} \bB \succeq \alpha \bI
    %\right\}\,.
    \cS(\bK) :=
    \left\{\bu  \in L_2 : 
    \E[\bu\bu^\sT]  \preceq \alpha^{-1}\bK^2
    %\bB \bSigma^{-1} \bB \succeq \alpha \bI
    \right\}\,.
\end{equation}
%
%\bns{I think the formula without $\bB$ that doesn't really seem convex is probably clearner to present. Thoughts?}
Further define the function $F:   \sfS^k_{\ge}\times \R^{k\times k_0}\to \R$ by
%
\begin{align}
%F(\bSigma,\bM):= \inf_{(\bu,\bB)\in\cS(\bSigma)} \E[\ell(\bu + \bSigma \bz_1 + \bM \bz_0, \bR_{00}^{1/2}\bz_0, W)]  + \lambda(\bSigma^2 + \bM\bM^\sT)\, .
F(\bK,\bM):= \inf_{\bu\in\cS(\bK)} \E[\ell(\bu + \bK \bz_1 + \bM \bz_0, \bR_{00}^{1/2}\bz_0, w)]  + \frac{\lambda}{2} \Tr(\bK^2 + \bM\bM^\sT) \, .\label{eq:FKM_Def}
\end{align}
%
Then 
\begin{enumerate}
\item $F$ is convex on $\sfS^k_{\succeq}\times \R^{k\times k_0}$.
\item The minimizers of $F$ are in one-to-one correspondence with the solutions $\bR^\opt$ of the
critical point optimality condition of  Definition~\ref{def:opt_FP_conds}, via 
%
\begin{equation}
\nonumber
 (\bR^{\opt}/\bR_{00}, \bR^\opt_{10} \bR_{00}^{-1/2}) =  ((\bK^\opt)^2,\bM^\opt)\, 
\end{equation}
%
where $\bK^\opt,\bM^\opt$ are the minimizers of $F(\bK,\bM).$
%
\item  $F$ is strictly convex if
either of the following conditions hold:
\begin{enumerate}
    \item $\lambda>0$. In this case the minimizer $(\bK,\bM)$ exists uniquely.
    \item $\lambda =0$ and $\ell(\, \cdot\, , \bv_0, w)$ is strictly convex for all $\bv_0,w$.
    In this case either the minimizer $(\bK,\bM)$ exists uniquely, or any minimizing sequence diverges.
\end{enumerate}
\end{enumerate}
\end{theorem}

%
%***********************************************************
%

\section{Application: Exponential families and multinomial regression}
We demonstrate how to apply the theory in the last section to regularizad maximum-likelihood
estimation (MLE) in exponential families. A large number of earlier works
studied the case of exponential families with a single parameter, a prominent example being logistic
regression \cite{sur2019modern,candes2020phase,montanari2019generalization,deng2022model,zhao2022asymptotic}. 


Here we obtain sharp high-dimensional asymptotics in the general case. 

\subsection{General exponential families}

Fix $m$ and $k = k_0$.
%and an exponential family with sufficient statistics $\bt$ and
%a reference measure $\nu_0$.
Given $\bt:\R^{m} \to \R^{k}$, and a reference measure $\nu_0$ on $\reals^m$, 
define the probability distribution on $\R^m$ 
%
\begin{align}
 \rP(\de\by|\bfeta)=  e^{\bfeta^\sT \bt(\by_i) - a(\bfeta)}
 \nu_0(\de\by)
 \, ,\;\;\;\;
 a(\bfeta) := \log\left\{\int e^{\bfeta^\sT \bt(\by_i)}
 \nu_0(\de\by)\right\}\, .\label{eq:ExpoDef1}
 \end{align}
 
We  assume to be given i.i.d. samples $(\bx_i,\by_i) \in \R^d\times \R^{m}$  for $i\le n$,
where $\bx_i\sim\normal(\bzero,\bI_d)$ and 
%
\begin{align}
    \P(\by_i\in S|\bx_i) = \rP(S|\bTheta_0^{\sT}\bx_i)\, .\label{eq:ExpoDef2}
\end{align}
%
We then consider the regularized  MLE defined as the minimizer of
%
\begin{equation}
     \hR_n(\bTheta) := \frac1n \sum_{i=1}^n \left\{ a(\bTheta^\sT \bx_i) -
      \<\bTheta^{\sT}\bx_i  ,\bt(\by_i)\>
      \right\} + \frac{\lambda}{2} \|\bTheta\|_F^2\, ,\label{eq:RiskExpo}
\end{equation}
%
for $\lambda \ge 0$.

In this case, the critical point optimality conditions of Definition \ref{def:opt_FP_conds}
reduce to the following set of equations for  $\bR\in\sfS^{2k}_{\succeq \bzero}$ 
and $\bS \in \sfS^k_{\succ \bzero}$:
   \begin{align}
    &\alpha \; \E[(\grad a(\bv) - \bt(\by)) (\grad a(\bv) -\bt(\by))^\sT]  = \bS^{-1}(\bR/ \bR_{00}) \bS^{-1}\, ,\label{eq:ExpoFP1}\\
    &\E\left[ (\grad a(\bv) - \bt(\by))(\bv^{\sT},\bg_0^\sT)\right] + \lambda (\bR_{11},\bR_{10}) = \bzero_{k\times (k+k)},\label{eq:ExpoFP2}
\end{align}
where, letting $\ell(\bv,\by):= a(\bv) - \<\bv, \bt(\by)\>$,
\begin{align}
\bv  = \Prox_{\ell(\,\cdot\,,\by)}( \bg; \bS),\quad\quad
\by \sim \rP(\by | \bfeta=\bg_0),\quad
[\bg^\sT,\bg_0^\sT]^\sT \sim \cN\left( \bzero_{k+k},\bR\right). \label{eq:ExpoFP3}
\end{align}

Here we state a simple general result under a strong convexity assumption.
In the next section we show that our general results can also be applied when strong 
convexity fails, by considering the case of multinomial regression.
%
\begin{proposition}\label{propo:Exponential}
Consider the exponential family of Eqs.~\eqref{eq:ExpoDef1}, \eqref{eq:ExpoDef2}, 
and assume that  $a_1\bI_m\preceq \nabla^2 a(\bfeta) \preceq a_2\bI_m$
for some constant $0<a_1\le a_2$, and that $n,d\to\infty$ with $n/d\to\alpha\in(1,\infty)$.
Further assume that $\bTheta^{\sT}_0\bTheta_0/d\to\bR_{00}\in\R^{k\times k}$ as $n,d\to\infty$.
Let $\hbTheta_n$ the the regularized  MLE  \eqref{eq:RiskExpo} (almost surely 
unique for all $n$ large enough).

Then \eqref{eq:ExpoFP1}, \eqref{eq:ExpoFP2} admit a unique solution $\bR^\opt, \bS^\opt$.
Further, if for
some $a_3 > 0$ independent of $n$; we have either
\begin{enumerate}
    \item 
    $\lambda_{\min}(\E_{\hnu}[\grad\ell \grad \ell^\sT])\ge a_3$ ; or
   \item  
   $\alpha > k$ and
   $\lambda_{\min}(\hbTheta_n^{\sT}\hbTheta_n/d)\ge a_3$,
\end{enumerate}
%$\lambda_{\min}(\hbTheta_n^{\sT}\hbTheta_n/d)\ge e^{-o(n)}$,
then we have, in probability, 
    \begin{equation}
    \hnu_{(\bX\hat\bTheta_n,\bY)} \stackrel{W_2}{\Rightarrow} \nu^\opt\, ,
    \end{equation}
    %
   where $\nu^\opt={\rm Law}(\bv,\by)$,
 and $\bv,\by$ are the random variables in Eq.~\eqref{eq:ExpoFP3}.
 
     If  in addition $\hmu_{\sqrt{d} \bTheta_0 } \stackrel{W_2}{\Rightarrow} \mu_0$,
    then the following holds, with   $\mu^\opt$  determined by $\bR^\opt$ as per Eq.~\eqref{eq:muopt},
     \begin{equation}
    \hmu_{\sqrt{d} [\hat\bTheta_n,\bTheta_0 ]} \stackrel{W_2}{\Rightarrow} \mu^\opt .
     \end{equation}
    %
\end{proposition}


\subsection{Revisiting multinomial regression}
\label{sec:Multinomial}

As a special case of exponential family, we consider multinomial regression 
with $k+1$ classes labeled $\{0,\dots,k\}$. 
The labels distribution 
and loss function have been already defined in Section \ref{sec:Intro}.

For $j\in\{1,\dots, k\}$,  $\be_j$ denotes the canonical basis vector in $\R^{k}$ 
and we let $\be_0 = \bzero_k$.  We encode class labels by letting 
$\by_i\in \{\be_0,\dots,\be_k\}$.
The regularized MLE minimizes the following risk function:
%
\begin{align}
\hR_n(\bTheta) = 
\frac1n \sum_{i=1}^n \bigg\{ -  \<\bTheta^\sT\bx_i,\by_i\>+
   \log\Big( 1+\sum_{j=1}^k e^{\<\be_j,\bTheta^{\sT}\bx_i\>}\Big)  
    \bigg\} +\frac{\lambda}{2}\|\bTheta\|_F^2\, .\label{eq:RiskMultinomial}
\end{align}
%
We also define the moment generating function $a:\reals^k\to\reals$ via
\begin{align}
    a(\bv) := \log\Big( 1+\sum_{j=1}^k e^{v_j}\Big)  \, .
\end{align}


For multinomial regression, Eqs.~\eqref{eq:ExpoFP1}, \eqref{eq:ExpoFP2}  take the even more explicit form
\begin{align}
\label{eq:FP_multinomial}
   \alpha \; \E[   (\bp(\bv)- \by )(\bp(\bv) - \by)^\sT ]   &=  \bS^{-1}(\bR/ \bR_{00})\bS^{-1},\\
   \E[   (\bp(\bv)- \by )(\bv^\sT,\bg_0^\sT)] &= \bzero\, ,\nonumber
   %\kas{=-\lambda \bR_{10}}
   %,\\
   %\E[   (\bp(\bv)- \by )\bv^\sT ] &= \bzero_{k\times k}
   %\kas{ \E[(I + \bS \bJ \bp(\bv))^{-1}] = (1-\frac1\alpha)\bI_{k\times k}+2\lambda \bS} ,
\end{align}
where $\bp(\bv) := \big( p_{j}(\bv) \big)_{j\in[k]}$ for $p_j$ defined in Eq.~\eqref{eq:MultiNomialDef}, and the random variables $\bv, \by$ 
have joint distribution defined by
\begin{align}
\bv  = \Prox_{a(\, \cdot\, )}( \bg + \bS \by; \bS),\quad
   \P\left(\by = \be_j\right)  = p_{j}(\bg_0),\; j\in\{0,\dots,k\},\quad
[\bg^\sT,\bg_0^\sT]^\sT \sim \cN\left( \bzero_{k+k_0},\bR\right)\, .\label{eq:FP_Multi3}
  \end{align}
  %


\begin{proposition}
\label{prop:multinomial}
Consider multinomial regression under the model of Eqs.~\eqref{eq:MultiNomialDef} with risk function \eqref{eq:RiskMultinomial} for $\lambda=0$. 
Assume that $n,d\to\infty$ with $n/d\to\alpha\in(0,\infty)$
and that $\bTheta^{\sT}_0\bTheta_0/d\to\bR_{00}\in\R^{k\times k}$ as $n,d\to\infty$, with 
$\bR_{00}\succ \bzero$ strictly.

For any $\alpha >1$ the following hold:
\begin{enumerate}
    \item 
If the system~\eqref{eq:FP_multinomial} has a solution $(\bR^\opt,\bS^\opt)$, then 
\begin{enumerate}
    \item  $(\bR^\opt,\bS^\opt)$  is
the unique solution of Eq.~\eqref{eq:FP_multinomial}.
\item We have,
for  $\hbTheta:= \argmin \hR_n(\bTheta)$ and some finite $C>0$, 
\begin{equation}
\label{eq:mle_exists_condition}
    \lim_{n\to\infty } \P\left( \hbTheta  \;\mbox{\rm exists},\; \|\hat\bTheta\|_F < C\right) = 1.
\end{equation}
\item Letting $\mu^\opt$ be  determined by $\mu_0,\bR^\opt$ as per Eq.~\eqref{eq:muopt},
and $\nu^\opt={\rm Law}(\bv,\by)$ with $\bv,\by$ defined by Eq.~\eqref{eq:FP_Multi3},
we have
    \begin{equation}
    \hmu_{\sqrt{d} [\hat\bTheta_n,\bTheta_0 ]} \stackrel{W_2}{\Rightarrow} \mu^\opt ,\quad\quad
    \hnu_{(\bX\hat\bTheta_n,\bX\bTheta_0,\bw)} \stackrel{W_2}{\Rightarrow} \nu^\opt\, .
    \end{equation}
    (For the first limit we assume $\hmu_{\sqrt{d} \bTheta_0 } \stackrel{W_2}{\Rightarrow} \mu_0$.)
    %
    \item The empirical spectral distribution of the Hessian at the minimizer 
    converges weakly to $\mu_\star(\nu^\opt)$ in probability. Equivalently, for all $z \in\bbH_+$,
    \begin{equation}
    \lim_{n\to\infty}\frac1{dk} (\bI_k \otimes \Tr) \left(\grad^2\hat R_n(\hat\bTheta_n) - z \bI_{dk} \right)^{-1} = 
      \alpha \, \bS_\star(\nu^\opt, z)
    \end{equation}
    in probability.
    \end{enumerate}
\item Conversely, if the system~\eqref{eq:FP_multinomial} does not have a solution, then, for all $C>0$,
\begin{equation}
\lim_{n\to\infty } \P\left( \hat\bTheta  \;\mbox{\rm exists},\; \|\hat\bTheta\|_F < C\right) = 0.
\end{equation}
\end{enumerate}
\end{proposition}

\begin{remark}
    A detailed study of high-dimensional asymptotics in multinomial regression was recently carried out in 
    \cite{tan2024multinomial}. However the techniques of \cite{tan2024multinomial} only allows to characterize
    the distribution of the MLE on null covariates. 
\end{remark}


\begin{figure}[t]
    \centering
     \begin{subfigure}[t]{0.45\textwidth}
        \includegraphics[width=\textwidth]{figures/multinomial/regularized/reg_misclassification_test_errors_error_k2_k02.pdf}
    \end{subfigure}
    % Subfigure (a)
    \begin{subfigure}[t]{0.45\textwidth}
        \includegraphics[width=\textwidth]{figures/multinomial/regularized/reg_test_errors_error_k2_k02.pdf}
    \end{subfigure}
    \begin{subfigure}[t]{0.45\textwidth}
        \centering
        \includegraphics[width=\textwidth]{figures/multinomial/regularized/reg_train_errors_error_k2_k02.pdf}
    \end{subfigure}
    \begin{subfigure}[t]{0.45\textwidth}
       \centering
        \includegraphics[width=\textwidth]{figures/multinomial/regularized/reg_norms_error_k2_k02.pdf}
    \end{subfigure}

    
    \caption{Train/test error (log loss), estimation error, and classification error
    of ridge regularized multinomial regression, for $(k+1)=3$ symmetric classes,
    as a function of the regularization parameter $\lambda$ for several values of $\alpha$. 
    Empirical results are averaged over 100 independent trials,  with $d = 250$. 
    Continuous lines are theoretical predictions obtained by solving numerically the system \eqref{eq:FP_multinomial}.
   }
    \label{fig:regularized_error}
\end{figure} 

\begin{figure}[t]
    \centering
    
     \begin{subfigure}[t]{0.45\textwidth}
        \includegraphics[width=\textwidth]{figures/multinomial/error_vs_alpha/misclassification_errors_vs.pdf}
    \end{subfigure}
    % Subfigure (a)
    \begin{subfigure}[t]{0.45\textwidth}
        \includegraphics[width=\textwidth]{figures/multinomial/error_vs_alpha/test_errors_vs_alpha.pdf}
    \end{subfigure}
    % Subfigure (b)
    \begin{subfigure}[t]{0.45\textwidth}
       \centering
        \includegraphics[width=\textwidth]{figures/multinomial/error_vs_alpha/train_errors_vs_alpha_k2_k02.pdf}
    \end{subfigure}
    % Subfigure (c)
    \begin{subfigure}[t]{0.45\textwidth}
        \centering
        \includegraphics[width=\textwidth]{figures/multinomial/error_vs_alpha/F_norm_vs_alpha.pdf}
    \end{subfigure}
    % Subfigure (d)
    
    \caption{Train/test error (log loss), estimation error, and classification error
    of multinomial regression, for $(k+1)=3$ symmetric classes, as a function of $\alpha$ for different values of $\bR_{00}$ specified in the text.
%    $
%    \bR_{00}^{(1)} 
%    =\bR_{00}^s(1,1/2), 
%    = \begin{bmatrix}
%        1&1/2\\1/2&1
%    \end{bmatrix},
      %\bR_{00}^{(2)}  = \bR_{00}^s(1, 0.9), 
     % = \begin{bmatrix}
     %   1&0.9\\0.9&1
     % \end{bmatrix},
      %\bR_{00}^{(3)} =  \bR_{00}^s(1,-1/2)$.
    %  \begin{bmatrix}
    %    1&-1/2\\-1/2&1
    %\end{bmatrix}
    Empirical results are averaged over 100 independent trials,  with $d = 250$. 
   }
    \label{fig:error_vs_alpha}
\end{figure} 


\begin{figure}[t]
    \centering
    % Subfigure (a)
    \begin{subfigure}[t]{0.45\textwidth}
        \includegraphics[width=\textwidth]{figures/multinomial/esd/esd3.pdf}
        \label{fig:subfig1}
    \end{subfigure}
    % Subfigure (b)
    \begin{subfigure}[t]{0.45\textwidth}
       \centering
        \includegraphics[width=\textwidth]{figures/multinomial/esd/esd5.pdf}
        \label{fig:subfig2}
    \end{subfigure}
    % Subfigure (c)
    \begin{subfigure}[t]{0.45\textwidth}
        \centering
        \includegraphics[width=\textwidth]{figures/multinomial/esd/esd10.pdf}
  %      \label{fig:subfig3}
    \end{subfigure}
    % Subfigure (d)
    \begin{subfigure}[t]{0.45\textwidth}
   %    \centering
        \includegraphics[width=\textwidth]{figures/multinomial/esd/esd20.pdf}
        \label{fig:subfig4}
    \end{subfigure}
    
    \caption{Histograms of the empirical spectral distribution of the Hessian at the MLE
    for multinomial regression with three symmetric classes, in $d=250$ dimensions, aggregated over $100$ 
    independent realizations.
    From left to right, $\alpha = 3$, $\alpha = 5$, $\alpha=10$, and $\alpha = 20$. Blue lines represent the theoretical distribution derived from Proposition \ref{prop:multinomial}.}
    \label{fig:Spectrum}
\end{figure} 



\section{Numerical experiments}

In this section we compare the predictions of our theory to numerical simulations 
for ridge-regularized multinomial regression, presented in Section \ref{sec:Multinomial}.
We present the results of two types of experiments:
\begin{enumerate}
\item Experiments with synthetic data, distributed according to the model used in our analysis.
In this case, we observe very close agreement between experiments and asymptotic predictions already
when $d\gtrsim 250$. 
\item Experiments with the image-classification dataset Fashion-MNIST \cite{xiao2017fashion}. In this case, we construct the feature vectors $\bx_i$ by passing the images through a one-layer random neural network \cite{rahimi2008random}. Of course, the resulting vectors $\bx_i$ are non-Gaussian, but we nevertheless observe encouraging agreement with the predictions.
\end{enumerate}


\subsection{Synthetic data}

In Fig.~\ref{fig:regularized_error}, we consider the case of $(k+1)= 3$ classes which
are completely symmetrical (the optimal decision regions are congruent). A simple 
calculation reveals that this corresponds to the case $\bR_{00}= \bR_{00}^s(c)$
for some $c>0$,
where
\begin{align}
\bR_{00}^s(c) :=
\begin{bmatrix}
        c & c/2 \\ 
        c/2 & c
    \end{bmatrix}\,.
%\begin{bmatrix}
%        c & b\,c/2 \\ 
%        b\,c/2 & c
%    \end{bmatrix}\,.
\end{align}
In the simulations  of Fig.~\ref{fig:regularized_error}, we use $c=1$.
    We compare empirical results
    for estimation error, test error and train error (in log-loss). We observe that theory matches well with the numerical simulations even for moderate dimensions.

In Fig.~\ref{fig:error_vs_alpha}, we once again compare the same empirical and theoretical quantities, for different values of the ground truth parameters (encoded in $\bR_{00}$) as a function of $\alpha$. We consider 3 different values of $\bR_{00}$:
\begin{equation}
    \bR_{00}^{(1)} 
    := \begin{bmatrix}
        1&1/2\\1/2&1
    \end{bmatrix},
    \quad\quad
      \bR_{00}^{(2)}  
      := \begin{bmatrix}
        1&0.9\\0.9&1
      \end{bmatrix},\quad\quad
      \bR_{00}^{(3)} := 
      \begin{bmatrix}
        1&-1/2\\-1/2&1
    \end{bmatrix}.
\end{equation}


%log-loss test and train errors, classification test error, and estimation error
%on the geometry of the ground truth parameters (encoded in $\bR_{00}$), for different values of $\alpha$.



In Fig.~\ref{fig:Spectrum}, we consider the first setting above with $\bR_{00} = \bR_{00}^s(1)$, $\lambda =0$
and plot the empirical spectral distribution of the Hessian $\nabla^2\hR_n(\hbTheta)$
at the MLE $\hbTheta$. We compare this with the prediction of Proposition \ref{prop:multinomial}.
Again, the agreement is excellent. 

We observe that the spectrum structure changes significantly around $\alpha\gtrsim 5$,
developing two `bumps.' This is related to the structure of the population Hessian,
which is easy to derive:
%
\begin{align}
\nabla^2 R(\bTheta)\big|_{\bTheta=\bTheta_0} = \E[\bA(\bTheta_0^{\sT}\bx)]\otimes \bI_d
+\sum_{a,b=1}^k\E[\partial^2_{a,b} \bA(\bTheta_0^{\sT}\bx)]\otimes \bTheta_{\cdot,a}
\bTheta_{\cdot,b}^{\sT}\, ,\label{eq:HessianDecomposition}
%
\end{align}
%
where  $\bA:\R^k\to\R^{k\times k}$ is defined by
%
\begin{align}
 \bA(\bv) := \nabla_{\bv}^2\ell(\bv,\bv_{0},w)\big|_{\bv_0= \bv}\, .
\end{align}
%
Hence, neglecting the second term in Eq.~\eqref{eq:HessianDecomposition}
(which is low rank and contributes at most $k^2$ outlier eigenvalues), the eigenvalues should concentrate (for large $n$) around  the $k$ eigenvalues of $\E[\bH(\bTheta_0^{\sT}\bx)]$.
The actual structure of $\grad^2\hat R_n(\hat\bTheta)$ 
in Fig.~\ref{fig:Spectrum} is significantly different: eigenvalues do not concentrate, but 
we begin seeing a $k$-mode structure emerging for $\alpha\gtrsim 10$.


We summarize a few qualitative findings that emerge from these experiments and our theory:
\begin{enumerate}
\item In the noisy regime considered here,
unregularized multinomial regression is substantially suboptimal
even when the number of samples per dimension is quite large ($\alpha=10$). 
See Fig.~\ref{fig:regularized_error}.
\item Similarly, the structure of the Hessian and the empirical risk minimizer, is very different from classical theory, even when $\alpha = 20$. See Fig.~\ref{fig:Spectrum}.
\end{enumerate}

\subsection{Fashion-MNIST data}
\label{sec:Fashion-MNIST}


\begin{figure}[htbp]
    \centering

    % Column titles with smaller font
    {\tiny  % Adjust font size (can also use \tiny if needed)
    \makebox[0.32\textwidth]{ 250 features}  % First column title
    \makebox[0.32\textwidth]{ 350 features}  % Second column title
    \makebox[0.32\textwidth]{ 500 features}  % Third column title
    }

    \vspace{0.15cm}  % Space between titles and first row

    \begin{tabular}{@{}c@{} c@{} c@{}}   % Create a table-like structure for labels and images
        % First row: Classification
        \begin{subfigure}[b]{0.335\textwidth}
            \centering
            \includegraphics[width=\textwidth]{figures/multinomial/tanh/250/classification_tanh_250.pdf}
        \end{subfigure} &
        \begin{subfigure}[b]{0.335\textwidth}
            \centering
            \includegraphics[width=\textwidth]{figures/multinomial/tanh/350/classification_tanh_350.pdf}
        \end{subfigure} &
        \begin{subfigure}[b]{0.335\textwidth}
            \centering
            \includegraphics[width=\textwidth]{figures/multinomial/tanh/500/classification_tanh_500.pdf}
        \end{subfigure} \\

        \vspace{0.3cm}  

        % Second row: Test Errors
        \begin{subfigure}[b]{0.335\textwidth}
            \centering
            \includegraphics[width=\textwidth]{figures/multinomial/tanh/250/test_tanh_250.pdf}
        \end{subfigure} &
        \begin{subfigure}[b]{0.335\textwidth}
            \centering
            \includegraphics[width=\textwidth]{figures/multinomial/tanh/350/test_tanh_350.pdf}
        \end{subfigure} &
        \begin{subfigure}[b]{0.335\textwidth}
            \centering
            \includegraphics[width=\textwidth]{figures/multinomial/tanh/500/test_tanh_500.pdf}
        \end{subfigure} \\

        \vspace{0.3cm}  

        % Third row: Train Errors
        \begin{subfigure}[b]{0.335\textwidth}
            \centering
            \includegraphics[width=\textwidth]{figures/multinomial/tanh/250/train_tanh_250.pdf}
        \end{subfigure} &
                \begin{subfigure}[b]{0.335\textwidth}
            \centering
            \includegraphics[width=\textwidth]{figures/multinomial/tanh/350/train_tanh_350.pdf}
        \end{subfigure} &
        \begin{subfigure}[b]{0.335\textwidth}
            \centering
            \includegraphics[width=\textwidth]{figures/multinomial/tanh/500/train_tanh_500.pdf}
        \end{subfigure} \\

    \end{tabular}

    \caption{Performance of multinomial regression on the Fashion-MNIST dataset 
    for $(k+1)=3$ classes, as a function of $\alpha$. We construct feature vectors $\bx_i$ using a random one-layer neural network, as discussed in 
    Section \ref{sec:Fashion-MNIST}.}
    \label{fig:mnist_tanh}
\end{figure}



\begin{figure}[!ht]
    \centering
    \begin{tabular}{c@{}c@{}}  % Remove spacing between columns
        % Subfigure (a)
        \begin{subfigure}[t]{0.52\textwidth}
            \centering
            \includegraphics[width=\textwidth]{figures/multinomial/tanh/esd/esd_350_tanh_alpha10.0.pdf}
        \end{subfigure} &
        % Subfigure (b)
        \begin{subfigure}[t]{0.52\textwidth}
            \centering
            \includegraphics[width=\textwidth]{figures/multinomial/tanh/esd/esd_350_tanh_alpha30.0.pdf}
        \end{subfigure} 
    \end{tabular}

    \caption{Histograms: Empirical spectral distribution of the Hessian at the MLE. Here we use use Fashion-MNIST data for $(k+1)=3$ classes, and construct feature vectors $\bx_i$ using a random one-layer neural network, as discussed in 
    Section \ref{sec:Fashion-MNIST}.}
    \label{fig:ESD_mnist}
\end{figure}











In Fig.~\ref{fig:mnist_tanh}, we report the result of our experiments on the Fashion MNIST
dataset \cite{xiao2017fashion}. This consists of overall $70000$ grayscale images of dimension $28\times 28$ belonging to $10$
classes (split into $60000$ images for training and $10000$ images for testing).
We select $k+1=3$ classes (pullovers, coats, and shirts), for a total of $N=18000$ training images. We standardize the entries of these images to get 
vectors $\{\bz_i\}_{i\le N}$, $\bz_i\in\reals^{d_0}$, $d_0=784$.
We then construct feature vectors $\bx_i$ by using a one-layer neural network with random weights. Namely
%
\begin{align}
\obx_i = \sigma(\bW\bz_i)\, ,
\end{align}
%
where $\bW= (W_{jl})_{j\le d,l\le d_0}$ is a matrix with i.i.d. entries $W_{jl}\sim\normal(0,1/d_0)$,
and $\sigma:\R\to\R$ is a (nonlinear) activation function which acts on vectors entrywise.
We then construct vectors $(\bx_i)_{i\le N}$ by `whitening' the $\{\obx_i\}_{i\le N}$. 
Namely, letting $\hbSigma:= N^{-1}\sum_{i\le N}\obx_i\obx_i^{\sT}$,
we define $\bx_i = \hbSigma^{-1/2}\obx_i$.
In Fig.~\ref{fig:mnist_tanh} we use $\sigma(x) = \tanh(x)$, but similar results are obtained with other activations.

For several values of $n$, we subsample $n$ out of the $N$ samples and fit multinomial regression and
average the observed test/train error and classification error to obtain the empirical data
in Fig.~\ref{fig:mnist_tanh}.

Since we do not know the ground truth, we fit multinomial regression to the whole 
$N$ samples $\{(y_i,\bx_i)\}_{i\le N}$ thus constructed, and assume that the resulting estimate coincides with
$\bTheta_0$. We then extract $\bR_{00} = \bTheta_0^{\sT}\bTheta_0$, which is the only unknown quantity
to evaluate the predictions of Proposition \ref{prop:multinomial}.


In Fig.~\ref{fig:ESD_mnist} we plot the empirical spectral distribution of the Hessian
at the MLE, and compare it with the theoretical prediction of Proposition \ref{prop:multinomial}.

These experiments suggest the following conclusions:
%
\begin{itemize}
    \item Figures \ref{fig:mnist_tanh} and  \ref{fig:ESD_mnist} shows reasonable quantitative agreement between theoretical predictions and experiments with real data. Given  the Gaussian covariates of Proposition \ref{prop:multinomial}, such agreement is surprising. While recent universality results \cite{hu2022universality,montanari2022universality,pesce2023gaussian} points in this direction, there is still much unexplained in this agreement.
    \item We constructed isotropic feature vectors $\bx_i$ through  the `whitening'
     step $\bx_i = \hbSigma^{-1/2}\obx_i$. We expect it to be  possible to generalize 
     our results to non-isotropic feature vectors, but leave it for future work.
\end{itemize}










%
%*****************************************************************
%*****************************************************************
%
\section{General empirical risk minimization: Proof of Theorem~\ref{thm:general}}
%\subsection{The Kac-Rice integral and its limit: Proof of Theorem~\ref{thm:general}}
\label{sec:pf_thm1}
\subsection{Applying the Kac-Rice integral formula}
\label{sec:pf_thm1_kr_integral}
As discussed in the introduction, the proof of Theorem~\ref{thm:general} relies on the Kac-Rice equation for counting the number of zeros of a Gaussian process. 
We recall the generic Kac-Rice formula in the following theorem, which is an adaptation of Theorem 6.2 from \cite{azais2009level}. The only modification is the introduction of the process $\bh$ and the associated assumptions and event $\{\bh \in\cO\}$. 
Its proof is essentially the same as the one in  \cite{azais2009level}.
\begin{theorem}[Modification of Theorem 6.2, \cite{azais2009level}]
\label{thm:kac_rice}
Let $\cT$ be an open subset of $\R^m$, and $\cO$ be an open subset of $\R^M$.
Let $\bz: \cT \to \R^m$ 
and $\bh:\cT  \to \R^M$ be random fields.
Assume that 
\begin{enumerate}
    \item  $\bz,\bh$ are jointly Gaussian.
    \item Almost surely the function $\bt\mapsto \bz(\bt)$ is of class $C^1$.
    \item For each $\bt\in \cT$, $\bz(\bt)$ has a nondegenerate distribution (i.e., positive definite covariance).
    \item We have 
    $\P(\exists \bt \in \cT, \bz(\bt) = \bzero, \bh(\bt) \in \partial\cO) = 0$.
    \item We have $\P(\exists \bt \in \cT, \bz(\bt) = \bzero, \det(\bJ_\bt \bz(\bt)) = \bzero) = 0$.
\end{enumerate}
Then for every Borel set $\cB$ contained in $\cU$, denoting 
$N_0(\cB) := \left|\left\{ \bt : \bz(\bt) =  \bzero, \; \bh(\bt) \in \cO \right\}\right|,$
we have
\begin{equation}
\E[N_0(\cB)]  = \int_{\cB} \E\left[|\det \bJ_\bt \bz(\bt)| \one_{\bh(\bt) \in \cO} \big| \bz(t) = \bzero\right] p_{\bz(t)}(\bzero) \de \bt,
\end{equation}
where $p_{\bz(\bt)}$ denotes the density of $\bz(\bt)$.
\end{theorem}
Our goal is to apply this formula to a Gaussian process whose zeros correspond to the critical points of the ERM problem of~\eqref{eq:erm_obj}.
Before we introduce this process, let us fix some notation to streamline the exposition:
For $j\in[k]$, let $\bell_j(\bV,\bV_0;\bw) \in\R^n$
and $\rho_j(\bTheta) \in\R^{d}$ 
be the columns of the matrices $\bL(\bV,\bV_0;\bw)$ and $\bRho(\bTheta)$
of~\eqref{eq:def_bL_bRho}, respectively.
Recall that we use $\bV \in\R^{n\times k},\bV_0\in\R^{n\times k_0}$ for the matrices whose columns are $\bv_j, \bv_{0,l}$, respectively, for $j\in[k],l\in[k_0]$.
For convenience, in what follows we'll often use the notation $\bbV := [\bV,\bV_0]$, and suppress the dependence on the arguments in the notation whenever it does not cause confusion.
Furthermore, it'll often be convenient to work with the empirical distributions $\hmu_{\sqrt{d}[\bTheta,\bTheta_0]}$ of $\sqrt{d}[\bTheta,\bTheta_0]$ and $\hnu_{[\bbV,\bw]}$ of $[\bbV,\bw]$.
We will use the less cumbersome notation $\hmu,\hnu$ for these throughout.

Let $m_n := dk +nk +nk_0$. For fixed $\bw\in\R^n$, define the \emph{gradient process}
$\bzeta(\,\cdot\,;\bw) : \R^{m_n} \to \R^{m_n},$ 
\begin{equation}
\nonumber
    \bzeta(\bTheta,\bV,\bV_0;\bw) :=
    %\boldf(\btheta_1,\dots,\btheta_k, \bv_1,\dots,\bv_k , \bu_1,\dots,\bu_{k_0}):=
    \begin{bmatrix}
     \bX^\sT\bell_1(\bV,\bV_0,\bw) + n \brho_1(\bTheta)\\
     \vdots\\
     \bX^\sT \bell_k(\bV,\bV_0,\bw) + n \brho_k(\bTheta)\\
     \bX\btheta_1 - \bv_1\\
     \vdots\\
     \bX\btheta_k - \bv_k\\
     \bX\btheta_{0,1} - \bv_{0,1}\\
     \vdots\\
     \bX\btheta_{0,k_0} - \bv_{0,k_0}
    \end{bmatrix}.\label{eq:GradProcess}
\end{equation}
%
From the definition of $\bzeta$, it's easy to note that for any $\bw$,
\begin{equation}
    \{(\bTheta,\bbV) : \bzeta(\bTheta,\bbV; \bw) = \bzero \} = \{(\bTheta,\bbV): \grad_\bTheta \hat R_n(\bTheta) = \bzero, \bbV = \bX [\bTheta,\bTheta_0]\}.
\end{equation}
Let
%
\begin{equation}
\label{eq:bH_def}
\bH(\bTheta,\bbV;\bw) := \bH_0(\bbV; \bw) + n \grad^2 \rho(\bTheta), \quad\quad
    \bH_0(\bbV;\bw) := \left(\bI_k \otimes \bX\right)^\sT \bSec(\bbV;\bw)\left(\bI_k \otimes \bX\right)\, ,
\end{equation}
\begin{equation}
   \bSec(\bbV;\bw) := \begin{pmatrix}
\bSec_{i,j}(\bbV;\bw)
   \end{pmatrix}_{i,j \in[k]}
,\quad
    \bSec_{i,j}(\bbV;\bw):= \Diag\left\{(\partial_{i,j}\ell(\bbV;\bw))\right\}\in\R^{n\times n}, \quad i,j\in[k]\, .
    \label{eq:SecDef}
\end{equation}
Recall the notation used here for the hessian of the regularizer
 $\grad^2\rho(\bTheta) = \Diag\{\rho''(\sqrt{d}\Theta_{i,j})\}_{i,j \in [d]\times [k]} \in\R^{dk\times dk}$.
% 
Observe the relation between these quantities and the Hessian of the empirical risk at the points $\bzeta = \bzero$. We have
%
\begin{align}
\label{eq:inclusion_1_f=0}
\{\bzeta =\bzero \} \;\; \subseteq \;\;
\{\bbV = \bX(\bTheta,\bTheta_0) \} \;\;\subseteq\;\;
\left\{\nabla^2\hR_n(\bTheta) = \frac1n \bH(\bTheta,\bbV;\bw) \right\}\,.
\end{align}
%

In order to study the expected size of $Z_n$ defined in Eq.~\eqref{eq:number_of_zeros_main}, we
would like to apply
Theorem~\ref{thm:kac_rice} to $\bzeta$ with the constraint $\bH_0/n + \grad^2\rho \succ \sfsigma_\bH$ on the Hessian, along with the additional constraints of $\cZ_n$ defined in~\eqref{eq:set_of_zeros_main} on the index set.
However, the process $\bzeta$ is \emph{degenerate}: as we show in Appendix~\ref{section:kac_rice}, its covariance has rank $m_n -r_k$ for $r_k := k(k+k_0)$
while Theorem~\ref{thm:kac_rice} requires the dimension of the index set to be the same as the rank of the covariance of the process. On the other hand, by the KKT conditions, the points $(\bTheta,\bbV)$ corresponding to critical points of $\hat R_n$ belong 
to the $m_n -r_k$ dimensional manifold 
$\cM_0 := \{\bG(\bTheta,\bbV) = 0\}$
where $\bG$ was defined in~\eqref{eq:def_G}.
Furthermore, with some algebra (Lemma~\ref{lemma:eig_vecs_NS_Sigma}), one can show that the mean $\bmu(\bTheta,\bbV)$ of the process $\bzeta(\bTheta,\bbV)$ is orthogonal to the nullspace of the covariance 
 at any point $(\bTheta,\bbV)$ in this manifold, so that restricting to this manifold gives a process of dimension $m_n - r_k$ to which we apply Theorem~\ref{thm:kac_rice} to.
%, the index set of the gradient process
%becomes $(m_n - r_k)$-dimensional.
The following lemma provides the necessary extension of
this theorem to our setting.
%
\begin{lemma}[Kac-Rice on the manifold]
\label{prop:kac_rice_manifold}
\label{lemma:kac_rice_manifold}
Fix $\bw \in\R^n$ (suppressed in the notation).
For $\cuA,\cuB$ as in Assumption~\ref{ass:sets}, define the \emph{parameter manifold}
\begin{align}
\label{eq:param_manifold_def}
\cM(\cuA,\cuB, \sPi) := \Big\{
   (\bTheta,\bbV) &: \hmu\in\cuA,\;
   \hnu\in\cuB,\;
   \bG(\bbV,\bTheta) =  \bzero,\;
   \sfA_\bR\succ \bR(\hmu_{\sqrt{d}[\bTheta,\bTheta_0]}) \succ\sfsigma_\bR,\;\\
   &\quad 
\sfA_\bV \succ \E_{\hnu}[\bv\bv^\sT] \succ \sfsigma_{\bV}, \;
\E_{\hnu}[\grad\ell \grad\ell^\sT] \succ \sfsigma_{\bL},\;
\sigma_{\min}\left( \bJ_{(\bbV,\bTheta)} \bG^\sT\right) > n\,\sfsigma_{\bG}
    \Big\}.
\end{align}
Let $\bB_{\bSigma}(\bTheta,\bbV)$ be a basis matrix for the column space of the covariance of $\bzeta$ at $(\bTheta,\bbV)$ (defined in Corollary~\ref{cor:proj}), and 
$\bz(\bTheta,\bbV) := \bB_{\bSigma}(\bTheta,\bbV)^\sT \bzeta(\bTheta,\bbV).$
Let $p_{\bTheta,\bbV}(\bzero)$ be the density of $\bz(\bTheta,\bbV)$, and finally, let
\begin{align}
\nonumber
Z_{0,n}(\cuA,\cuB,\sPi) := \Big|\Big\{
   (\bTheta,\bbV) &: \hmu\in\cuA,\;
   \hnu\in\cuB,\;
   \bzeta = \bzero,\;
   \bG(\bbV,\bTheta) =  \bzero,\;
   \frac1n \bH \succ \sfsigma_\bH,\; 
   \sfA_\bR\succ \bR(\hmu_{\bTheta,\bTheta_0}) \succ\sfsigma_\bR,\;\\
   &\quad 
\sfA_\bV \succ \E_{\hnu}[\bv\bv^\sT] \succ \sfsigma_{\bV}, \;
\E_{\hnu}[\grad\ell \grad\ell^\sT] \succ \sfsigma_{\bL},\;
\sigma_{\min}\left( \bJ_{(\bbV,\bTheta)} \bG^\sT\right) > n\,\sfsigma_{\bD}
    \Big\}\Big|.
\end{align}
Then under Assumptions~\ref{ass:loss},~\ref{ass:regularizer},~\ref{ass:sets}, we have 
\begin{align}
\label{eq:kr_eq_manifold}
\E[Z_{0,n}(\cuA,\cuB,\sPi)|\bw] 
    &=\int_{(\bTheta,\bbV) \in \cM(\cuA,\cuB,\sPi)}  \E\left[\left| \det (\de \bz(\bTheta,\bbV) )\right|
    \one_{\bH \succ n\sfsigma_\bH}
    \big| \bz  = \bzero, \bw\right] p_{\bTheta,\bbV}(\bzero)  \de_\cM V
\end{align}
where the latter is an integral over the manifold $\cM$ 
(with the volume element denoted by $\de_{\cM}V$), and $\de \bz(\bTheta,\bbV) : T_{(\bTheta,\bbV)}\cM \mapsto \R^{m_n - r_k}$ is the differential which we identify with a $\R^{(m_n - r_k)\times (m_n - r_k)}$ matrix.
\end{lemma}
The proof of this lemma is deferred to Section~\ref{sec:proof_of_kac_rice_on_manifold}.
Directly from the definitions above, we have
\begin{equation}
\label{eq:inclusion_2_f=0}
    \{\grad \widehat R_n(\bTheta) = \bzero \} \;\; \subseteq\;\; \{\bzeta =\bzero,\; \bG(\bbV,\bTheta) = \bzero\}.
\end{equation}
This inclusion along with the one of 
Eq.~\eqref{eq:inclusion_1_f=0}
implies then that for $\cuA,\cuB$
in their respective domains, we have 
\begin{equation}
      Z_n(\cuA,\cuB, \sPi)
      \le Z_{0,n}(\cuA,\cuB, \sPi)
\end{equation}
for $Z_n$ as in Eq.~\eqref{eq:number_of_zeros_main}.
So to derive the bound of Theorem~\ref{thm:general}, we will study the asymptotics (up to first order in the exponent) of the integrand in Eq.~\eqref{eq:kr_eq_manifold}.

\subsection{Integration over the manifold}
To control the integral in Eq.~\eqref{eq:kr_eq_manifold} over the parameter manifold, we'll upper bound the integral by a volume integral over the \emph{$\beta$-blow up} of $\cM$ defined by
\begin{equation}
    \cM^\up{\beta}(\cuA,\cuB,\sPi) := \{\bu\in\R^{m_n}: \exists\;\bu_0\in\cM(\cuA,\cuB,\sPi),\quad \norm{\bu-\bu_0}_2\le \beta\}
\end{equation}
for some $\beta >0$. Since we are interested in the asymptotics of the integral in Eq.~\eqref{eq:kr_eq_manifold},  we would like to choose $\beta$
independent of $n$ but small enough so that the 
volume integral is a good approximation of the manifold integral.

The needed regularity of  $\cM(\cuA,\cuB,\sPi)$ is guaranteed by 
Assumption~\ref{ass:params}, which states that  minimum singular value of the Jacobian of the constraint $\bG$ defining $\cM$ is lower bounded by a constant $\sfsigma_\bG$ independent of $n$. The estimate of manifold integrals by volume integrals is formalized  by
the following lemma.
%
\begin{lemma}[Manifold integral lemma]
\label{lemma:manifold_integral} 
Let $r_k := k(k+k_0)$, $m_n:= nk + nk_0 + dk$.
Let $f: \cM^\up{1} \subseteq \R^{m_n}\rightarrow \R$ be a nonnegative continuous function. 
There exists a constant $C = C(\sfA_{\bV},\sfA_{\bR})>0$ that depends only on $(\sfA_{\bV},\sfA_{\bR})$, such that for positive
\begin{equation}
   \beta_n  \le \frac{C(\sfA_{\bV},\sfA_{\bR}) \sfsigma_{\bG}(n)^3}{r_k^6} %\wedge 1,
\end{equation}
we have
\begin{equation}
        \int_{(\bTheta,\bbV)\in \cM} f(\bTheta,\bbV) \de_\cM V
        \le
        \Err_{\sblowup}(n)
         \,
        e^{\beta_n\;\norm{\log f}_{\Lip,\cM^{(1)}}}
        \int_{(\bTheta,\bbV)\in \cM^{(\beta_n)}} f(\bTheta,\bbV)\de(\bTheta,\bbV),
\end{equation}
where the multiplicative error $\Err_{\sblowup}(n)$ is given explicitly in Lemma~\ref{lem:intg-tube} and satisfies
\begin{equation}
    \lim_{n\to\infty}  \frac1n \log 
    \Err_{\sblowup}(n)
 = 0.
\end{equation}
%\begin{equation}
    %E(\beta,n,k) := \left(\frac1{1 - \beta\;r_k^2 C }\right)^{(m-r_k)/2}
        %\left(\frac{r_k^{5/2} C}{\beta(\sfsigma_{\bG} - \beta C r_k^2)}\right)^{r_k}
%\end{equation}
\end{lemma}
The proof of this lemma is deferred to Section~\ref{section:manifold_integration} of the appendix.

The determinant term appearing in Eq.~\eqref{eq:kr_eq_manifold}
is that of the differential $\de \bz(\bTheta,\bbV)$ defined on the tangent space of $\cM$. 
We relate this to the Euclidean Jacobian of $\bzeta(\bTheta,\bbV)$ which will be defined on $\cM^\up{1} \subseteq R^{m_n}$.
Let $\bB_{\bT(\bTheta,\bbV)}$ and $\bB_{\bSigma(\bTheta,\bbV)}$ be a basis for the tangent space of $\cM$ and the column space of $\bSigma(\bTheta,\bbV)$ at $(\bTheta,\bbV)$, respectively. Further let
$\bB_{\bT^c(\bTheta,\bbV)}$ and $\bB_{\bSigma^c(\bTheta,\bbV)}$ be basis matrices for the complement of these spaces.
Suppressing $(\bTheta,\bbV)$ in the arguments, we have in this notation $\det(\de\bz)  = \det\left(\bB_{\bSigma}^\sT \bJ \bzeta \bB_\bT\right)$.
Since the codimension of  the tangent space of $\cM$, and of the column space of $\bSigma$ are $r_k=O(1)$, we expect that $\log |\det(\de \bz(\bTheta,\bbV))| = \log \det|\bJ \bzeta(\bTheta,\bbV)|+O(1)$ for large $n$. In turn, we can directly compute
    \begin{equation}
        \bJ \bzeta = \begin{bmatrix}
            n\grad^2 \rho& (\bI_k\otimes\bX)^\sT\bSec &
            (\bI_k\otimes \bX)^\sT\tilde\bSec\\
            \bI_k\otimes\bX& -\bI&\bzero\\
            \bzero &\bzero &-\bI
        \end{bmatrix}
\end{equation}
to see that $|\det(\bJ \bzeta)| =  \det\left(\bH \right)$,
where $\bH$ was defined in~\eqref{eq:bH_def}.
Characterizing the asymptotic spectral density of $\bH$ will allows us to determine the asymptotics $\E[\det(\bH)]$ using Gaussian concentration. 
Of course, we actually need to modify such calculation to correctly account for the 
conditioning on $\bz=\bzero$.

We will formalize this analysis in the below, while deferring most technical details to the appendix.


\paragraph{Relating $\det(\de \bz)$ to $\det(\bH)$.}
As noted previously (and stated in Lemma~\ref{lemma:eig_vecs_NS_Sigma} of the appendix), the projection of $\bzeta$ onto the nullspace of $\bSigma$ vanishes,
so that
$\bB_{\bSigma^c}^\sT \bJ\bzeta\bB_{\bT} = \bzero$ and hence
\begin{equation}
\label{eq:det_projection}
\det(\de\bz)  = \det\left(\bB_{\bSigma}^\sT \bJ \bzeta \bB_\bT\right)  =\frac{ \det\left( \bJ \bzeta\right)}
   {\det( \bB_{{\bSigma}^c}^\sT \bJ \bzeta \bB_{\bT^c})}\, .
\end{equation}
Since the dimensions of the tangent space and the nullspace of $\bSigma$ are $r_k$, we then have
\begin{equation}
\label{eq:det_to_min_singular_value}
\big|\det( \bB_{{\bSigma}^c}^\sT \bJ \bzeta \bB_{\bT^c})\big|
\ge
   {\sigma_{\min}(\bB_{{\bSigma}^c}^\sT \bJ \bzeta \bB_{\bT^c})^{r_k}}
   \ge 
   {\sigma_{\min}(\bJ \bzeta)^{r_k}}.
\end{equation}
hence, we can upper bound the determinant of $\de \bz$ given a lower bound on the minimum singular value of $\bJ\bzeta$. The following lemma furnishes the latter in terms of the minimum singular value of $\bH$.

\begin{lemma}[Lower bound on the singular values of $\bJ_{(\bTheta,\bbV)} \boldf $]
\label{lemma:lb_singular_value_Df}
Under Assumption~\ref{ass:loss} and~\ref{ass:regularizer}, we have the bound
   \begin{equation}
\sigma_{\min}(\bJ_{(\bTheta,\bbV)} \boldf) \ge \;\frac{\sigma_{\min}(\bH)}{\Err_\sigma(\bX)}
   \end{equation}
   where the multiplicative error is given by
\begin{equation}
\Err_\sigma(\bX) :=
C(\sfA_\bR)
\left(
\|(\bX^\sT\bX)^{-1}\|_\op^{3/2} +1
\right)
\left(\|\bX^\sT\bX\|_\op^{7/2}+ n^3\right)
%    E(\bX^\sT\bX;\sfK,\tilde\sfK) := C(\|\bX^\sT\bX\|_\op^{5/2} \|(\bX^\sT\bX)^{-1}\|_\op^{3/2} (\sfK^2 + \tilde{\sfK}^2 + 1 ))
\end{equation}
for some constant $C>0$ depending only on $\sfA_{\bR}$.
%and $\sOmega$.
\end{lemma}

When combined with equations~\eqref{eq:det_projection} and~\eqref{eq:det_to_min_singular_value}, this lemma then gives the bound 
\begin{equation}
\label{eq:simplfied_det_bound}
\E\left[\left| \det (\de \bz(\bTheta,\bbV) )\right|
    \one_{\bH\succ n\sfsigma_\bH}
    \big| \bz  = \bzero\right] 
    \le 
\frac{1}{(n \sfsigma_{\bH})^{r_k}}
\E\left[\left| \det (
\bH)\right|
    \one_{\bH \succ n\sfsigma_\bH} 
    \Err_\sigma(\bX)^{r_k}
    \big| \bz = \bzero\right].
\end{equation}

Now letting
\begin{equation}
\label{eq:integrand_of_interest}
  f(\bTheta,\bbV)  :=
\frac{1}{(n \sfsigma_{\bH})^{r_k}}
\E\left[\left| \det (
\bH)\right|
    \one_{\bH \succ n\sfsigma_\bH} 
    \Err_\sigma(\bX)^{r_k}
    \big| \bz = \bzero\right]
%\E\left[\left| \det (\de \bz(\bTheta,\bbV) )\right|
%    \one_{\bH \succ n\sfsigma_\bH}
%    \big| \bz  = \bzero, \bw\right]
    p_{\bTheta,\bbV}(\bzero),
\end{equation}
the function $f$ bounds the integrand in Eq.~\eqref{eq:kr_eq_manifold}. 
We'll apply Lemma~\ref{lemma:manifold_integral} to obtain an integral over the blow-up $\cM^\up{1}$. First, we'll 
an upper estimate for the density term $p_{\bTheta,\bbV}(\bzero)$ in the next section.

   % which will allow us to upper bound $\tilde f$ with some function $f$ defined and Lipschitz on the blow-up $\cM^\up{1}$, such that $f \ge \tilde f$  on $\cM$.
   % 
   % and study the asymptotics 
   % (up to first order in the exponent) of this bound. This is done in the next section.
%In the next section, we begin this task by giving an upper estimate for the density term $p_{\bTheta,\bbV}(\bzero)$.

%The integrand of interest here is of course
%%we must now study the asymptotics (up to first order in the exponent) of the function
%\begin{equation}
%\label{eq:integrand_of_interest}
%  \tilde f(\bTheta,\bbV)  :=
%\E\left[\left| \det (\de \bz(\bTheta,\bbV) )\right|
%    \one_{\bH \succ n\sfsigma_\bH}
%    \big| \bz  = \bzero, \bw\right] p_{\bTheta,\bbV}(\bzero)
%\end{equation}
%appearing in Eq.~\eqref{eq:kr_eq_manifold}, which is defined on $\cM$.
%%and bound the Lipschitz constant of the logarithm of this function.
%To use this lemma to obtain an upper bound on the integral of $\tilde f$, we'll first upper bounding the determinant term
%$\E\left[\left| \det (\de \bz(\bTheta,\bbV) )\right|
%    \one_{\bH \succ n\sfsigma_\bH}
%    \big| \bz  = \bzero, \bw\right]$ by a term defined on $\cM^\up{1}\subseteq \R^{m_n}$. 



\subsection{Computing and bounding the density term}
For any  
$(\bTheta,\bbV)$, the term $p_{\bTheta,\bbV}(\bzero)$ corresponds to the density function of a Gaussian  random variable at $\bzero$.
With some algebra, this can be shown to be  (cf. Lemma~\ref{lemma:density})
\begin{equation}\label{eq:DensityAtZero}
  p_{\bTheta,\bbV}(\bzero) 
    :=
\frac{
    \exp\left\{-\frac{1}2\left(n^2
\Tr\left(\bRho (\bL^\sT\bL)^{-1}\bRho\right) + \Tr(\bbV \bR^{-1}\bbV) + n\Tr\left(\bRho (\bL^\sT\bL)^{-1}\bL^\sT \bbV \bR^{-1}(\bTheta,\bTheta_0)^\sT\right)
%    \Tr\left(\bbV
%\bR^{-1}(\bTheta)
%    \bbV^\sT\right)
    \right)\right\}}
    {
\det^*(2\pi\bSigma(\bTheta,\bbV))^{1/2}
}
    \, ,
\end{equation}
with $\det^*$ denoting the product of the non-zero eigenvalues, and the covariance $\bSigma(\bTheta,\bbV)$ is given by
\begin{equation}
\bSigma:= 
   \begin{bmatrix}
       \bL^\sT \bL \otimes \bI_{d}  & \bM & \bM_0\\
       \bM^\sT  & \bTheta^\sT \bTheta \otimes \bI_n & \bTheta^\sT\bTheta_0 \otimes \bI_n\\
       \bM_0^{\sT} & \bTheta_0^\sT\bTheta  \otimes \bI_n& \bTheta_0^{\sT}\bTheta_0 \otimes \bI_n
   \end{bmatrix} ,
\end{equation}
where

\begin{equation}
    \bM :=  \begin{bmatrix}
        \btheta_1\bell_1^\sT & \dots  & \btheta_k \bell_1^\sT\\
        \vdots  &   & \vdots \\
        \btheta_1\bell_k^\sT & \dots  & \btheta_k \bell_k^\sT\\
    \end{bmatrix}
    \in \R^{d k\times n k},\quad
    \bM_0 :=
    \begin{bmatrix}
        \btheta_{0,1}\bell_1^\sT & \dots  & \btheta_{0,k_0} \bell_1^\sT\\
        \vdots  &   & \vdots \\
        \btheta_{0,1}\bell_k^\sT & \dots  & \btheta_{0,k_0} \bell_k^\sT\\
        \end{bmatrix} \in \R^{d{k} \times n k_0 }.
\end{equation}

From the low-rank structure of $\bM$ and $\bM_0$, one would expect that 
$\det^*(\bSigma(\bTheta,\bbV))$ is approximately given by $\det(\bL^\sT\bL \otimes \bI_d)\cdot\det(\bR(\bTheta) \otimes \bI_n)$ for large $n$ and fixed $k,k_0$. 
If we use this heuristics in Eq.~\eqref{eq:DensityAtZero} rewrite
various quantities 
in terms of the empirical measures $\hmu$ and $\hnu$, we reach the conclusion
of the following lemma. (We refer to Appendix~\ref{sec:density_bound} for its proof.)
%This in turn allows us to upper bound the density term by the following expression.
 \begin{lemma}[Bounding the density]
\label{lemma:density_bounds}
Define the Gaussian densities
\begin{align}
    p_{1}(\bTheta) := \frac{d^{dk/2}}{(2\pi)^{dk/2}} \exp\left\{-\frac{d}{2} \Tr\left(\bTheta^\sT\bTheta\right)\right\},\quad\quad
    p_{2}(\bbV) := 
    \frac1{(2\pi)^{n(k+k_0)/2}}
\exp\left\{-\frac12
    \Tr\left(\bbV^\sT
    \bbV\right)
    \right\}.
\end{align}
Under Assumptions \ref{ass:regime} to \ref{ass:params} of Section~\ref{sec:assumptions}, there exist constants $n_0, C >0$ depending only on 
%$\sOmega$ and 
$k,k_0$, such that for all $(\bTheta,\bbV) \in\cM(\cuA,\cuB)$ and $n> n_0,$
\begin{equation}
   p_{\bTheta,\bbV}(\bzero)\le C(k,k_0) \,
   (\alpha_n^{1/2} d)^{-dk}
   e^{n h_0(\hmu,\hnu;\alpha_n)}  \;
    p_1(\bbV) p_2(\bTheta)
\end{equation}
where $\alpha_n:=n/d$ and 
%
\begin{align}
\nonumber
h_0(\mu,\nu;\alpha) :=& - \frac{1}{2\alpha} \log \det(\E_\hnu[\grad\ell\grad\ell^\sT]) 
+\frac{1}{2 \alpha}\Tr(\bR_{11}(\hmu))
-
\frac{1}2 \Tr(\E_\hmu[\grad \rho\grad\rho^\sT] \E_\hnu[\grad\ell\grad\ell^\sT]^{-1})
- \frac{1}2 \log\det (\bR(\hmu))
\\
&+\frac{1}2 \Tr\big(\E_\hnu[\bbv\grad\ell^\sT] \E_\hnu[\grad\ell\grad\ell^\sT]^{-1} 
\E_\nu[\grad\ell\bbv^\sT]
\bR(\hmu)^{-1}
\big)  + \frac{1}{2}\Tr\big((\bI_k - \bR(\hmu)^{-1}) \E_\hnu[\bbv\bbv^\sT]\big).
\end{align}
 \end{lemma} 

In the above lemma, we deliberately wrote the bound in terms of product of a density over $(\bTheta,\bbV)$  (the term $p_1(\bbV) p_2(\bTheta)$), 
and a term that depends on $(\bTheta,\bbV)$ only through their empirical distributions.
 Recalling $f$ defined in Eq.~\eqref{eq:integrand_of_interest}, when the bound on the density is combined with Lemma~\ref{lemma:manifold_integral}, and 
assuming we can show that $\|\log f\|_{\Lip}\le C n$ for some constant $C>0$ independent of $n$,
then we have
\begin{align}
\label{eq:expectation_over_bV_bTheta}
    \lim_{n\to\infty} \frac1n &\log \E[Z_{0,n}| \bw] \\
    &\le
    \lim_{n\to\infty} \frac1n\log
    \E_{\substack{\bTheta\sim p_1\\ \bbV\sim p_2}}\Big[
\E\left[\left| \det (\bH )\right|
\Err_\sigma(\bX)^{r_k}
    \one_{\bH \succ n\sfsigma_\bH}
    \big| \bz  = \bzero, \bw\right] (\alpha_n^{1/2} d)^{-dk} e^{n h_0(\hmu,\hnu;\alpha_n)} \, \one_{(\bTheta,\bbV) \in \cM^\up{\beta}}
    \Big]\, .\nonumber
\end{align}
%
The details of bounding the Lipschitz constant are left to Appendix~\ref{sec:kr_asymptotics}.
We will continue our analysis by estimating the right-hand side of Eq.~\eqref{eq:expectation_over_bV_bTheta}.
In particular, in the next section we will consider 
the conditional expectation involving the determinant.




\subsection{Analysis of the determinant term}


\paragraph{Asymptotic spectral density of $\bH$.}
Recall now the definition of $\mu_\star(\nu,\mu)$ introduced in Section~\ref{sec:definitions}.
The following proposition affirms that $\mu_\star$ is the limit of the empirical spectral distribution of $\bH$, uniformly over $\bbV,\bTheta.$ We recall that $\bH_0=\bH_0(\bbV;\bw)$ is the Hessian of the loss part of
the risk, cf. Eq.~\eqref{eq:bH_def}. We also use the notation $\bH_0=\bH_0(\hnu_n)$, since this depends only on
the empirical distribution of the rows of $[\bbV,\bw]$.
%
\begin{proposition}
\label{prop:uniform_convergence_lipschitz_test_functions}
Under Assumptions of Section~\ref{sec:assumptions}, we have for any Lipschitz function $g:\R\to\R$,
  \begin{equation}
  \lim_{\substack{n\to\infty\\n/d \to \alpha}}
  \sup_{\substack{\bw\in\R^n\\(\bbV,\bTheta)\in\cM(\cuA,\cuB)}}\left|\frac1{dk}\E\left[\Tr \,g\left(\frac1n\bH(\bTheta,\bbV;\bw)\right)\right]
      - \int g(\lambda) \mu_\star(\hnu,\hmu)(\de \lambda)
      \right| = 0.
  \end{equation}
%
  Moreover, if $\hnu \Rightarrow \nu$ weakly in probability for some $\nu\in\cuP(\R^{k+k_0+1})$,
  we have for any fixed $z\in\bbH_+$ the convergence in probability
\begin{equation}
\label{eq:ST_convergence_in_P_seq_measures}
     \frac1{dk} (\bI_k \otimes \Tr) \left(\bH_0(\hnu_n) - z\bI_{dk}\right)^{-1} \to \alpha \bS_\star(\nu,z)\, ,
\end{equation}
where for any $z\in \bbH_+$,  $\bS_{\star}(\nu,z)$ is defined as the unique solution of  Eq.~\eqref{eq:fp_eq}.
\end{proposition}
%
The proof of this proposition is deferred to Appendix~\ref{sec:RMT},
which also proves uniqueness of  $\bS_\star(\nu,z)$ in Appendix \ref{app:sec:UniquenessSstar}. There, we analyze the empirical Steiltjes transform of $\bH_0$  
via a leave-one-out approach that is similar to the one used in deriving the asymptotic density of Wishart matrices $\bX^\sT\bX$~\cite{BaiSilverstein}. Of course the difference here is that the Stieltjes transform is an element of $\C^{k\times k}$ (often called the operator-valued Stieltjes transform in the free probability literature~\cite{speicher2019non}). As a result, the analysis requires additional  care compared to the scalar case to deal with the additional complications arising from non-commutativity. 
Finally, the empirical spectral distribution of $\bH$ frollows from the one of $\bH_0$
via a free probability argument.
See Appendix~\ref{sec:RMT} for details.

\paragraph{Conditioning on $\bz=\bzero$ and concentration of the determinant.}
We outline the main steps in bounding the conditional expectation of the determinant appearing in the right-hand side of Eq.~\eqref{eq:simplfied_det_bound}.
We leave most technical details to Appendix~\ref{sec:determinant_bound}.

First, note that
since the mean of $\bzeta$ is in the column space of $\bSigma(\bTheta,\bbV)$ for any $(\bTheta,\bbV)$, conditioning on $\bz = \bzero$ is equivalent to conditioning 
on $\bzeta = \bzero$. 
The latter meanwhile is  to conditioning on
$\{\bL^\sT\bX = - n\bRho^\sT,\; \bX[\bTheta,\bTheta_0] = \bbV\}$. 
So letting $\bP_{\bTheta}, \bP_\bL$ be the projections onto the columns spaces of $[\bTheta,\bTheta_0],\bL$ respectively, we have on $\{\bzeta = 0\}$
\begin{equation}
    \bX = \bP_\bL^\perp \bX \bP_\bTheta^\perp - \bL(\bL^\sT\bL)^{-1} \bRho^\sT \bP_{\bTheta}^\perp  + \bbV\bR^{-1} [\bTheta,\bTheta_0]^\sT.
\end{equation}
%

Hence for any measurable function $g$,
\begin{equation}
\label{eq:conditioning_generic}
    \E[g(\bX) | \bzeta = 0] = \E[g(\bX + \bDelta_{0,k})]
\end{equation}
for some matrix $\bDelta_{0,k} = \bDelta_{0,k}(\bTheta,\bbV)$ 
satisfying
%\begin{align}
%    &\rank(\bDelta_{0,k}) \le r_k,\\
%    &\norm{\bDelta_{0,k}}_\op \le \norm{\bX}_\op.
%\end{align}
\begin{align}
    &\rank(\bDelta_{0,k}) \le 4(k+k_0),\\
    &\norm{\bDelta_{0,k}}_\op \le C\sqrt{n} \max\left(\frac{\norm{\bX}_\op}{\sqrt{n}} ,
    \frac{\sfA_{\bV}}{\sfsigma_{\bR}^{1/2}} + \frac{\sfA_{\bR}}{ \sfsigma_{\bL}^{1/2}}\right)
\end{align}
for some constant $C>0$ independent of $n$.
%only on $\sOmega$.

The identity \eqref{eq:conditioning_generic} allows us to bound the expectation in
Eq.~\eqref{eq:expectation_over_bV_bTheta}. Namely,
recalling
that $\bH = (\bI_k \otimes \bX)^\sT\bSec(\bI_k \otimes \bX) + n \grad^2\rho$, we have
\begin{align}
&
\E\left[\left| \det (\bH )\right|
\Err_\sigma(\bX)^{r_k}
    \one_{\bH \succ n\sfsigma_\bH}
    \big| \bz  = \bzero, \bw\right]
%\E[|\det \big(\de \bz\big)|\one_{\bH \succ n\sfsigma_\bH} \big| \bzeta = 0 ]
{\le}
%\frac{1}{(n\sfsigma_\bH)^{r_k}}
\E\left[  \big|\det\left(\bH + \bDelta_{1,k}\right)\big|    \one_{\{\bH + \bDelta_{1,k} \succ n\sfsigma_\bH\}}
\Err_{\sigma}(\bX + \bDelta_{0,k})^{r_k}
\right]
\label{eq:concentration_decomp_0_main}
\end{align}
for some $\bDelta_{1,k}$ of rank at most some $r'_k = O(r_k)$. 
The effect of the error term $\Err_\sigma$ will be bounded in Appendix~\ref{sec:determinant_bound} where we show that it contributes at most a multiplicative factor that is polynomial in $n$, and is thereby exponentially trivial.
Meanwhile, the interlacing theorem implies that, for any $i\ge dk-r'_k$,
\begin{equation}
\label{eq:interlacing}
    \lambda_{i+r'_k}(\bH+  \bDelta_{1,k})\leq \lambda_{i}(\bH),\quad \lambda_i(\bH + \bDelta_{1,k})\leq \|\bH + \bDelta_{1,k}\|_\op.
\end{equation}
As a consequence, the deteminant involving $\bH$ in Eq.~\eqref{eq:concentration_decomp_0_main} can be estimated as follows.
For any $\tau_1 >0$,
  \begin{align}
\label{eq:log_det_to_log_eps}
      \det((\bH  + \bDelta_{1,k})/n)=& \exp\left\{\sum_{i=1}^{dk}\log 
      (\lambda_i(\bH+  \bDelta_{1,k} )/n)\right\}\\
      %=& \sum_{i=1}^{dk}\log^{(\eps_\bH)}
      %(\lambda_i(\bH+\bE_k))\\
      \leq& \exp\left\{\sum_{i=1}^{dk -r'_k}\log \lambda_i(\bH /n) +
      r_k'\log\left( \frac{\|\bH + \bDelta_{1,k}\|_\op}{n}\right)\right\}\\
      %\leq & \sum_{i=1}^{dk} \log^{(\eps)} \lambda_i(\bH/n) - 3r_k\log\eps + 3r_k\log(2\norm{\bH}_\op)\\
      \leq& \exp\left\{\Tr\left(\log^{(\tau_1)}(\bH /n) \right)\right\} \cdot
\left( \frac{\|\bH+ \bDelta_{1,k}\|_\op}{n \tau_1}\right)^{r'_k},
  \end{align}
  where $\log^\up{\tau_1}(t) := \log(\tau_1 \vee t).$
As we show in Section~\ref{sec:pf_lemma_CE_bound} the appendix, 
the term $\|\bH+ \bDelta_{1,k}\|_\op$ is at most polynomial in $n$ uniformly over $(\bTheta,\bbV) \in \cM$ with high probability.
Therefore, for the indicator involving the minimal singular value of $\bH$
in Eq.~\eqref{eq:concentration_decomp_0_main}, we note that 
$$\{\bH + \bDelta_{1,k} \succ n \sfsigma_\bH\}\subseteq
    \left\{
    \big|\left\{ \lambda \in \spec\left(\bH /n \right) : \lambda \leq 0\right\} \big| < r'_k
    \right\}= \{ \hmu_{\bH}(-\infty,0) < r'_k/n\}$$
    where $\hmu_{\bH}$ is the empirical spectral measure of $\bH/n$.
    
With the proper formalization of the above, along with a concentration argument showing that Lipschitz functions of $\bH$ concentrate super-exponentially, we reach the following lemma which summarizes the results of analyzing the determinant.
We'll use the bound in Eq.~\eqref{eq:simplfied_det_bound} to state this more concisely.
%\bns{Add blurb about concentration and perturbation explaining the proof of the following lemma, and how we get the hard constraint on the ASD}
%\bns{Remark about passing to hard constraint.}

\begin{lemma}[Bounding the conditional expectation of the determinant]
 \label{lemma:CE_bound}
Fix $\tau_0,\tau_1 \in (0,1)$, and $\bw$ satisfying $\|\bw\|_2 \le \sfA_{\bw}\sqrt{n}$.
Then under Assumptions \ref{ass:regime} to \ref{ass:params} of Section~\ref{sec:assumptions}, there exist constants $C,c>0$,
%depending only on $\sOmega$
and $C_0(\tau_0)$ depending on $\tau_0$, both independent of $n$, such that for all $n > C_0(\tau_0)$, 
\begin{enumerate}
\item For any $(\bbV,\bTheta) \in\cM(\cuA,\cuB)$ satisfying $\mu_{\star}(\hnu,\hmu)((-\infty, -\tau_0)) < \tau_0,$ we have
\begin{align}
&\E[|\det \big(\de \bz\big)|\one_{\bH\succ n\sfsigma_\bH} \big|\bzeta=\bzero,\bw]
\le
n^{dk}
\Bigg(\exp\left\{
\E\left[\Tr\log^{(\tau_1)}\left(\bH/n\right)\Big| \bw\right] +  \frac{C n^{1-1/4}}{\tau_1}
\right\}
+ \exp\left\{ -n^{5/4}\right\}
\Bigg)
\end{align}

\item For $(\bbV,\bTheta)\in\cM$ satisfying 
$\mu_{\star}(\hnu,\hmu)((-\infty, -\tau_0)) \ge \tau_0,$
\begin{align}
\E[|\det \big(\de \bz\big)|\one_{\bH \succ n\sfsigma_\bH} \big|\bzeta = 0 ,\bw]
& \le C \exp \left\{ 
    -c 
    \tau_0^2
    n^{3/2}
    \right\}.
\end{align}
\end{enumerate}
\end{lemma}



\subsection{Asymptotics of the integral}
Finally, the analysis of the previous three sections can be applied to upper bound the asymptotics of the integral of Eq.~\eqref{eq:kr_eq_manifold}.
By recalling that $Z_n \le Z_{0,n}$, and combining the bound of Eq.~\eqref{eq:expectation_over_bV_bTheta} along with Lemma~\ref{lemma:CE_bound} to bound the conditional expectation, we arrive at the following lemma.

\begin{lemma}[Upper bound on the Kac-Rice integral]
\label{prop:asymp_1}
\label{lemma:asymp_1}
Fix $\tau_0,\tau_1\in (0,1)$. Let $\cG\subseteq \Ball_{\sfA_{\bw}\sqrt{n}}^n(\bzero)$ be any measurable subset.
Define 
\begin{equation}
\nonumber
\phi_{\tau_1}(\nu,\mu; \alpha)
:=
 \frac{k}{2\alpha}\log(\alpha)+
\frac{k}{\alpha}
\int \log^{(\tau_1)}(\lambda) \mu_{\star}(\nu,\mu)(\de \lambda) + h_0(\mu,\nu; \alpha),\quad\quad\textrm{and}
\end{equation}
%
\begin{equation}
\nonumber
    \cuM^{(\beta)}(\cuA,\cuB,\sPi) := \left\{ (\mu,\nu) :  \exists\;  (\mu_0,\nu_0) \in \cuM(\cuA,\cuB,\sPi) \; \textrm{s.t.} \;  W_2(\nu,\nu_0) < \beta, W_2(\mu,\mu_0) < \beta \right\},\quad\quad\textrm{where}
\end{equation}
\begin{align}
\nonumber
\cuM(\cuA,\cuB,\sPi) := \Big\{(\mu,\nu) \in \cuA\times \cuB &:\;
\sfA_{\bR} \succ \bR(\mu) \succ\sfsigma_\bR,\;
\sfA_\bV \succ \E_{\nu}[\bv\bv^\sT] \succ \sfsigma_{\bV},\;
\E_{\nu}[\grad\ell \grad\ell^\sT] \succ \sfsigma_{\bL},\;\\
&\quad\quad\quad
\E_\nu[\grad \ell(\bv,\bv_0,w)(\bv,\bv_0)^\sT]+ 
     \E_\mu[\rho'(\btheta) (\btheta, \btheta_0)^\sT] =   \bzero_{k\times (k+k_0)}
\Big\}.
\nonumber
\end{align}
%\begin{align}
%\phi_{\tau_1}(\nu,\mu)
%&:=
% \frac{k}{2\alpha}\log(\alpha)+
%\frac{k}{\alpha}
%\int \log^{(\tau_1)}(\lambda) \mu_{\star}(\nu,\mu)(\de \lambda)
%  - \frac{1}{2\alpha}\log \det\left( \E_{\nu}[\grad \ell\grad\ell^\sT]\right)
%+ \frac{1}{2\alpha} \Tr\left(\bR_{11}(\mu)\right) 
%-\frac1{2}\log\det(\bR(\mu))
%\\
%   &\quad-\frac1{2\alpha}
%\Tr\left( (\E_\nu[\grad\ell\grad\ell^\sT])^{-1}\E_\mu[\grad \rho \grad \rho^\sT]\right) + \frac1{2\alpha}\Tr\left(\E\left[\bbv\grad\ell^\sT\right] (\E[\grad\ell\grad\ell^\sT])^{-1}\E[\grad\ell\bbv^\sT] \bR(\mu)^{-1}\right) \\
%&\quad+ \frac12 \Tr\left((\bI_k - \bR(\mu)^{-1})\E[\bbv\bbv^\sT]\right).
%\end{align}
Under Assumptions~\ref{ass:regime},\ref{ass:loss},\ref{ass:regularizer} and~\ref{ass:sets}, there exists a constant $C(\sfA_{\bR},\sfA_{\bV})>0$ depending only on $(\sfA_{\bR},\sfA_{\bV})$, such that
for any $\beta\in(0, 1)$,
we have
\begin{align}
\limsup_{n\to\infty}\frac1n\log\E[Z_n(\cuA,\cuB) \one_{\bw\in\cG}]
&\le
   \limsup_{n\to\infty}\frac1n\log
   \E_\bw\E_{\substack{\bTheta\sim p_1\\ \bbV\sim p_2}}\left[e^{n\phi_{\tau_1}(\hnu,\hmu)}
   \one_{\{\mu_{\star}(\hmu,\hnu)((-\infty, -\tau_0]  ) < \tau_0\} \cap \cuM^{(\beta)}}
   \right] \one_{\bw \in\cG }.
\end{align}
\end{lemma}

The technical details of the proof are left to Appendix~\ref{sec:proof_prop_asymp_1}.

At this point, 
the statement of Theorem~\ref{thm:general} follows from this lemma as a consequence of Sanov's theorem and Varadhan's lemma, from which the $\KL$-divergence terms in the formula \eqref{eq:PhiGen} for  $\Phi_{\gen}$ 
appear as the rate function for large deviations of the empirical measures  $\hmu$ and $\hnu$.
The final details of this are left to Appendix~\ref{sec:proof_thm1_large_deviations}.







\section{Convex empirical risk minimization: Proofs of Theorems \ref{thm:convexity}, \ref{thm:global_min}}
\label{sec:pf_convex_results}

\subsection{Proof of Theorem~\ref{thm:convexity}}

\subsubsection{Simplified variational upper bound under convexity: Proof of point 
\textit{1}}
The upper bound  of Theorem~\ref{thm:convexity} is a special case of Theorem~\ref{thm:general} under the additional assumption of convexity.
In order to derive $\phi_\cvx$ from $\phi_\gen$ of Theorem~\ref{thm:general} 
under Assumption~\ref{ass:convexity}, we begin with the following identity
easily verifiable from the definition of the $\KL$-divergence: denoting $\bbv^\sT := [\bv^\sT,\bv_0^\sT],$  we have
\begin{align*}
   -\frac12 \log \det(\bR)  + \frac12 \Tr((\bI_{k+k_0} - \bR^{-1}) \E_{\nu}[\bbv\bbv^\sT]) -
   \KL(\nu_{\cdot|w} \| \cN(\bzero, \bI_{k+k_0}))
= - \KL(\nu_{\cdot|w} \| \cN(\bzero, \bR)).
\end{align*}
%
Meanwhile, for $\rho(t) = \lambda t^2/2$, so that $\grad_{\btheta}\rho(\btheta) = \lambda \btheta,$  one obtains after using the constraint $\E[\grad\ell \bbv^\sT + \grad \rho [\btheta^\sT,\btheta_0^\sT]] = \bzero$  that
\begin{align*}
   &-\frac1{2\alpha}
\Tr\left( (\E_\nu[\grad\ell\grad\ell^\sT])^{-1}\E_\mu[\grad \rho \grad \rho^\sT]\right) + \frac1{2\alpha}\Tr\left(\E_{\nu}\left[\bbv\grad\ell^\sT\right] (\E_\nu[\grad\ell\grad\ell^\sT])^{-1}\E_\nu[\grad\ell\bbv^\sT] \bR(\mu)^{-1}\right)\\
&=
   -\frac{\lambda^2}{2\alpha}
\Tr\left( (\E_\nu[\grad\ell\grad\ell^\sT])^{-1}\bR_{11}(\mu)\right) + \frac{\lambda^2}{2\alpha}\Tr\left((\E_\nu[\grad\ell\grad\ell^\sT])^{-1}
 [\bR_{11}(\mu),\bR_{10}(\mu)] \bR(\mu)^{-1} [\bR_{11}(\mu),\bR_{10}(\mu)]^\sT
\right)\\
&= 0\, ,
\end{align*} where the last equality follows by noting that $[\bR_{11},\bR_{10}] \bR^{-1} [\bR_{11},\bR_{10}]^\sT = \bR_{11}$.
Comparing $\Phi_\cvx$ with $\Phi_\gen$, we see that what remains now is to obtain the following bound on the logarithmic potential.

\begin{lemma}[Variational principle for the log potential]
\label{lemma:variational_log_pot}
Let
\begin{equation}
    \mu_{\star,\lambda}(\nu) :=
    \mu_{\MP}(\nu) \boxplus  \delta_{\lambda}.
\end{equation}
   Under Assumption~\ref{ass:convexity} , for any $\nu\in\cuB$, and $\lambda \ge0$,
   we have
    \begin{equation}
    \label{eq:variational_log_pot}
        k\int\log(\zeta ) \mu_{\star,\lambda}(\nu) (\de\zeta)
\le \inf_{\bS\succ\bzero} K_{-\lambda}(\bS;\nu),
    \end{equation}  
    where
\begin{equation}
    K_z(\bQ;\nu):= -\alpha z \Tr(\bQ) + \alpha \E_{\nu}[\log\det(\bI + \grad^2 \ell(\bv,\bu, w)\bQ) ]  - \log\det(\bQ) - k (\log(\alpha) + 1).
\end{equation}
\end{lemma}
Since $\rho''(t) = \lambda$ under Assumption~\ref{ass:convexity}, note that for any $(\nu,\mu) \in\cuP(\R^{k+k_0+1})\times \cuP(\R^{k+k_0})$, $\mu_\star(\mu,\nu) = \mu_{\star,\lambda}(\nu).$
This finally shows $\phi_\gen(\nu,\mu,\bR) \le \sup_{\bS\succ\bzero}\phi_\cvx(\nu,\mu,\bR,\bS)$ as claimed, giving the claim of point~\textit{1} of the theorem.

\begin{remark}
Note that for $z\in\bbH^+$, by directly differentiating $K_z$ of Lemma~\ref{lemma:variational_log_pot} with respect to $\bQ$ we can easily see that $\bS_\star$ defined by~Eq.~\eqref{eq:fp_eq} is a critical point of $K_z(\bQ;\nu)$.
In the proof of Lemma~\ref{lemma:variational_log_pot} which is deferred to 
Section~\ref{sec:log_pot_proof} of the appendix,
 we show that under the convexity assumption of Assumption~\ref{ass:convexity} this critical point is the  minimizer of $K_z(\bQ).$
\end{remark}

\subsubsection{The critical point optimality condition: Proof of point~\texorpdfstring{$\textit{2}$}{2}}
Let $\bC = \bC(\bR) := \bR/\bR_{00}$ for ease of notation.
Let us decompose $\Phi_\cvx$ as $\Phi_\cvx = \Phi_{\cvx,1} + \Phi_{\cvx,2}$ for
\begin{align}
\Phi_{\cvx,1}(\mu)
&:=\frac{k}{2\alpha}
- \frac{1}{2\alpha} \Tr\left(\bR_{11}(\mu)\right)
+ \frac1{2\alpha} \log \det(\bC(\bR(\mu)))
 + \frac1\alpha \KL ( \mu_{\btheta| \btheta_0}\| \cN(\bzero,\bI_k)).\\
    \Phi_{\cvx,2}(\nu,\bR,\bS) &:=
    -\lambda\Tr(\bS)  +
    \frac{1}{2\alpha} \log\det\left(\E_\nu\left[ \grad \ell\grad\ell^\sT\right]\right)
+ 
\frac1{\alpha}\log\det\bS 
- \E_{\nu}\left[\log \det \left(\bI_k + \grad^2 \ell^{1/2} \bS \grad^2 \ell^{1/2}\right)\right]\nonumber\\
&\quad+\frac{k}{2\alpha} 
+ \frac{k}{2\alpha} \log(\alpha)
-\frac1{2\alpha} \log\det\left(
\bC(\bR)
\right)   + \KL\left(\nu_{\bv,\bv_0 | w}\| \cN(\bzero, \bR)\right).
\end{align}
%
We'll treat each of $\Phi_{\cvx,1}$ and $\Phi_{\cvx,2}$ separately, and show that they are both strictly positive, unless $(\mu,\nu) = (\mu^\opt,\nu^\opt)$, which case both are identically equal to $0$.

\noindent\textbf{Lower bounding $\Phi_{\cvx,1}$:}
For  fixed $\bR$, consider  the following minimization problem
\begin{align}
    \inf_{\mu : \bR(\mu) = \bR}  \frac1{\alpha}\KL(\mu_{\cdot| \btheta_0} \| \cN(\bzero,\bI_k)) \, ,
\end{align}
%
where it is understood that the ``outer" expectation in the conditional divergence is taken with respect to measure $\mu_{(\btheta_0)} = \mu_0$. By maximum entropy property of Gaussian measures, we obtain that the minimizer is such that $\mu_{\cdot|\btheta_0}=
\normal(\bA\btheta_0,\bB)$ for certain matrices $\bA$, $\bB$. Enforcing the 
constraint  $\bR(\mu) = \bR$ we obtain that the minimizer is in fact
$\mu^\opt$ 
of Definition~\ref{def:opt_FP_conds}.
We can then directly compute for the fixed $\bR$,
\begin{align}
     \frac1{\alpha}\KL( \mu^\opt_{\cdot|{\btheta_0}} \| \cN(\bzero,\bI_k))  
     &= 
     \frac1{2\alpha}\E_{\btheta_0\sim\mu_0}\left[ -\log\det(\bC(\bR)) - k 
     + \Tr(\bR_{11} - \bR_{10} \bR_{00}^{-1} \bR_{10})+
      \btheta_0^\sT \bR_{00}^{-1}\bR_{01}\bR_{10}\bR_{00}^{-1} \btheta_0\right]
      \nonumber\\
      &=- \frac1{2\alpha}\log\det(\bC(\bR)) - \frac{k}{2\alpha} + \frac{1}{2\alpha}\Tr(\bR_{11}).
\end{align}
Consequently, for any $\mu$ as in the statement of the theorem, we have
\begin{align}
    \Phi_{\cvx,1}(\mu) &\ge \inf_{\mu \in\cuP(\R^{k+k_0})} \Phi_{\cvx,1}(\mu) = 
    \Phi_{\cvx,1}(\mu^\opt)\\
    &=
    \inf_{\bR\succeq \bzero} \left\{
\frac{k}{2\alpha}
- \frac{1}{2\alpha} \Tr\left(\bR_{11}(\mu)\right)
+ \frac1{2\alpha} \log \det(\bC(\bR(\mu))) 
- \frac1\alpha \KL(\mu^\opt \| \cN(\bzero,\bI_k))
    \right\} = 0,
\end{align}
with equality if and only if $\mu = \mu^\opt$.

\noindent \textbf{Lower bounding $\Phi_{\cvx,2}$:}
To lower bound $\Phi_{\cvx,2}$, we'll rewrite the divergence term as a divergence involving the distribution $\nu^\opt$ of the proximal operator in the Definition~\ref{def:opt_FP_conds}.
To derive the density of $\nu^\opt$, it is useful to observe that for $f(\bv,\bv_0, w)$ convex in $\bv$ for fixed $\bv_0,w$,
 the map $\bz \mapsto \Prox_{f(\cdot, \bv_0, w)}(\bz; \bS)$  is invertible  for any $\bS \succeq \bzero_{k\times k}$, with inverse given by
\begin{equation}
\Prox_{f(\cdot,\bv_0,w)}^{-1}(\bv; \bS) = \bS\grad f(\bv,\bv_0,w)+\bv,
\end{equation}
which can be derived from the first order conditions
\begin{align}
    %\rho_{\bQ,\bv_0,w}(\bv)=&\min_{\bx\in\R^k}\left( \frac12 (\bx-\bv)^\bT\bQ^{-1}(\bx-\bv)+\rho(\bx,\bv_0,w)\right)\in\R\\
     \bS\grad f(\Prox_{f(\cdot;\bv_0,w)}(\bz;\bS),\bv_0,w)=\bz- \Prox_{f(\cdot,\bv_0,w)}(\bz; \bS)
\end{align}
where $\grad f \in\R^{k}$  is the gradient of $f$ with respect to the first $k$ variables.

In what follows, we use $(\bv,\bv_0,w)$, $\bv\in\R^{k},\bv_0\in\R^{k_0},w\in\R$ to denote random variables whose distribution is $\nu^\opt.$
To that end, let $\bg,\bg_0$ be jointly Gaussian as in Definition~\ref{def:opt_FP_conds}, and $w\sim\P_w$. 
For any $\bS,\bR \succ\bzero$,
denoting
$p^\opt_{\bS,\bR}(\bv| \bv_0, w)$ the conditional density of 
$\Prox(\bg; \bS, \bg_0, w) = \Prox_{\ell(\,\cdot\,;\bg_0,w)}(\bg; \bS)$ 
given $w$,$\bv_0=\bg_0$,
we find that
\begin{align}
    p^\opt_{\bS,\bR}(\bv|\bv_0, w) =& \exp\left\{ -\frac12 (\bS\grad\ell(\bv,\bv_0, w )+\bv-\bmu(\bv_0,\bR))^\sT\bC(\bR)^{-1}(\bS\grad\ell(\bv,\bv_0, w )+\bv-\bmu(\bv_0,\bR))\right\}\nonumber\\
    &\quad\quad(2\pi)^{-k/2}\det(\bC(\bR))^{-1/2}
     \det\left(\bI_k+\grad^2\ell(\bv,\bv_0,w)^{1/2}\bS \grad^2\ell(\bv,\bv_0,w)^{1/2}\right),
\end{align}
where $\bmu := \bmu(\bv_0, \bR) := \bR_{10}\bR_{00}^{-1}\bv_0$.
Using this identity, the KL
divergence of the conditional measure $\nu_{\bv|\bv_0,w}$ with respect to $\cN(\bmu,\bC)$ can be written as
\begin{align*}
    \KL\left(\nu_{\cdot|\bv_0,w}\|  \cN(\bmu,\bC)\right) 
    &=\KL\left(\nu_{\cdot|\bv_0,w}\Big\|  p^\opt_{\bS,\bR}(\cdot | \bv_0, w) \right) 
    +
     \E_\nu\left[\log\det\left(\bI_k+\grad^2\ell^{1/2}\bS \grad^2\ell^{1/2}\right)\right]\\
     &\quad\quad-\frac12\E_\nu\left[\grad \ell^\sT \bS \bC(\bR)^{-1} \bS\grad \ell\right]
     -\E_\nu\left[
    \grad \ell^\sT \bS \bC(\bR)^{-1} \left(\bv - \bmu\right)
     \right].
\end{align*}
Recall that $\nu\in\cuV(\bR)$
implies that $\E[\grad \ell \cdot (\bv^\sT,\bv_0^\sT)] + \lambda (\bR_{00},\bR_{01}) = \bzero,$ whence
\begin{align*}
\E_\nu\left[
    \grad \ell^\sT \bS \bC(\bR)^{-1} \left(\bv - \bmu\right)
     \right]  &=  \Tr\left( \bS \bC(\bR)^{-1} \E_{\nu}[\bv \grad\ell^\sT - \bR_{10}\bR_{00}^{-1}\bv_0\grad\ell^\sT]\right)  \\
&=
-\lambda\Tr\left( \bS 
(\bR_{00} - \bR_{10}\bR_{00}^{-1}\bR_{01})^{-1}
(\bR_{00} - \bR_{10}\bR_{00}^{-1}\bR_{01})\right) \\
&= -\lambda \Tr(\bS).
\end{align*}
This, along with the chain rule for the KL-divergence and the expansion of the conditional KL above gives
\begin{align}
\nonumber
  \KL\left(\nu_{\bv, \bv_0|w}\|  \cN(\bzero,\bR)\right) &= 
\KL\left(\nu_{\bv|\bv_0,w}\|  p_{\bS,\bR}^\opt(\cdot | \bv_0, w) \right) 
    +
     \E_\nu\left[\log\det\left(\bI_k+\grad^2\ell^{1/2}\bS \grad^2\ell^{1/2}\right)\right]\\
  &\quad\quad
  -\frac12\E_\nu\left[\grad \ell^\sT \bS \bC(\bR)^{-1} \bS\grad \ell\right]
  +\KL\left(\nu_{\bv_0|w}\|  \cN(\bzero,\bR_{00})\right) + \lambda \Tr(\bS).
\end{align}
By substituting this equality for the KL term into $\Phi_{\cvx,2}$ and carrying out the appropriate cancellations, 
this shows that for any $\mu,\nu$ as in the statement, 
\begin{align}
   \sup_{\bS\succ\bzero}\Phi_{\cvx,2}(\nu,\bR(\mu),\bS) 
     &=
    \sup_{\bS\succ\bzero} \bigg\{\frac1{2\alpha} \log\det \left(
    \E_\nu[\grad \ell \grad \ell^\sT ] \bS^2 \bC(\bR(\mu))^{-1}
    \right)-\frac12\E_\nu[\grad\ell^\sT \bS \bC(\bR(\mu))^{-1}\bS \grad\ell]\nonumber\\
    &\hspace{15mm}+\frac k{2\alpha}\log(\alpha e) +\KL(\nu_{\bv|\bv_0,w}\|p^\opt_{\bS,\bR(\mu)})+\KL(\nu_{\bv_0|W}\|\cN(\bzero,\bR(\mu)_{00}))\bigg\}\\
&\stackrel{(a)}{\ge}
    \sup_{\bS\succ\bzero} \left\{M(\bS;\nu, \bR(\mu)) \right\}\, ,\nonumber
\end{align}
where
\begin{equation}
    M(\bS;\nu, \bR) = \frac1{2\alpha} \log\det \left(
    \E_{\nu}[\grad \ell \grad \ell^\sT ] \bS^2 \bC(\bR)^{-1}
    \right)-\frac12\E_{\nu}[\grad\ell^\sT \bS \bC(\bR)^{-1}\bS \grad\ell]+\frac k{2\alpha}\log(\alpha e).
\end{equation}
The inequality in $(a)$ follows from non-negativity of the KL-divergence and holds with equality if and only if $\nu_{\bv,\bv_0| w} = \nu_{\bv,\bv_0|w}^\opt$, the measure induced by the density $p^\opt_{\bS,\bR}$ defined above.
Since $\E_\nu[\grad\ell\grad\ell^\sT]\succ \bzero,  \bR(\mu) \succ\bzero$ for such measures, one can check that $M(\bS;\nu,\bR)$ is strictly concave in $\bS$ and is uniquely maximized at 
\begin{equation}
    \bS= \bS^\opt(\nu,\bR) =\frac1{\sqrt\alpha}\bC(\bR)^{1/2}\left(\bC(\bR)^{-1/2}\E_{\nu}[\grad\bell\grad\bell^\sT]^{-1} \bC(\bR)^{-1/2}\right)^{1/2}\bC(\bR)^{1/2},
\end{equation}
with $M(\bS^\opt(\nu,\bR);\nu, \bR) = 0$.
\newline

\noindent\textbf{Concluding:}
Using the lower bounds above, for any $\mu,\nu$ as in the statement, we have by design
\begin{align}
   \sup_{\bS\succ\bzero}  
   \Phi_\cvx(\mu,\nu,\bR(\mu),\bS) &=
    \Phi_{\cvx,1}(\mu)  
    +\sup_{\bS\succ\bzero}\Phi_{\cvx,2}(\nu,\bR(\mu),\bS)\\
    &= \inf_{\mu_0 : \bR(\mu_0) = \bR(\mu)}
  \left\{
    \Phi_{\cvx,1}(\mu_0) + \sup_{\bS\succ\bzero} \Phi_{\cvx,2}(\nu^\opt,\bR(\mu), \bS) 
    \right\}\nonumber\\
    &= \inf_{\mu_0 : \bR(\mu_0) = \bR(\mu)}
    \Phi_{\cvx,1}(\mu_0) + \sup_{\bS\succ\bzero} \Phi_{\cvx,2}(\nu^\opt,\bR(\mu), \bS) 
    \nonumber\\
    &\stackrel{(a)}{\ge} 0 + \sup_{\bS\succ\bzero} \Phi_{\cvx,2}(\nu^\opt,\bR(\mu), \bS)
    \nonumber\\
    &\stackrel{(b)}{\ge} 0 + 
\sup_{\bS\succ\bzero}  M(S; \nu,\bR(\mu)) = 0.\nonumber
\end{align}
where in $(a)$ we used that for any $\bR \succ\bzero$, the Gaussian measure $\mu^\opt = \mu^\opt(\bR)$ chosen previously satisfies $\Phi_{\cvx,1}(\mu^\opt) = 0$.
By the previous steps, $(a)$ and $(b)$ hold with equality if and only $(\mu,\nu)$ are as given in Definition~\ref{def:opt_FP_conds}.

%Combining with Eq.~\eqref{eq:phi_1_LB} shows that $\sup_{\bS}\Phi \ge0$, with equality if and only if the inequality in $(a)$ and the inequality in Eq.~\eqref{eq:phi_1_LB} hold with equality, i.e., if and only if $\nu$ and $\mu$ satisfy the equations in Definition~\ref{def:opt_FP_conds}.

\subsection{Proof of Theorem~\ref{thm:global_min}}
%\paragraph{Convergence of the empirical distributions: proof of item~\textit{1.}}
To prove point \textit{1}, we'll look at critical points $\hat\bTheta_n$ of the 
empirical risk $\hat R_n$ such that $\hmu_{\sqrt{d}[\hbTheta_n,\bTheta_0]}$
and $\hnu_{[\bX\hbTheta_n,\bX\bTheta_0,\bw]}$ belong to the complement  of the sets
%
\begin{equation}
    \cuA_\eps:=  \{\mu : W_2(\mu,\mu^\opt) \le\eps \},\quad
    \quad\quad
    \cuB_\eps := 
     \{\nu : W_2(\nu,\nu^\opt) \le\eps \},
\end{equation}
for fixed $\eps >0$. We will then apply Theorem~\ref{thm:convexity} to deduce that the probability that there exist such critical points vanishes under the high-dimensional asymptotics.

Let $\Omega_0 := \{\widehat\bTheta_n \in \cE(\bTheta_0)\}$, and $\Omega_1 := \{ n C_0(\alpha) \succ \bX^\sT\bX\succ nc_0(\alpha)\}$.
We cite two results from Appendix~\ref{sec:simplifying_constraint_set} allowing to simplify the set in Eq.~\eqref{eq:set_of_zeros_main} to the set $\cE$ defined in Theorem~\ref{thm:convexity}:
First, 
under the conditions of the theorem, Lemma~\ref{lemma:jacobian_lb} gives the deterministic bound
\begin{equation}
    \sigma_{\min}\left( \bJ_{(\bbV,\bTheta)} \bG^\sT\right) \ge 
     \frac{\sigma_{\min}(\grad^2 \hat R_n(\bTheta))
     \sigma_{\min}([\bTheta,\bTheta_0])
     }{(1 + \|\bX\|_\op )}.
\end{equation}
%
%
%\begin{equation}
%    \sigma_{\min}\left( \bJ_{(\bbV,\bTheta)} \bG^\sT\right) \ge 
%    C \frac{\sigma_{\min}(\grad^2 \hat R_n(\bTheta))}{(1 + \lambda \|\grad^2\hat R_n(\bTheta)\|_\op) } \frac{\sigma_{\min}([\bTheta,\bTheta_0])}{ \|\bX\|_\op \vee 1}.
%\end{equation}
Since Assumption~\ref{ass:loss} guarantees that $\|\grad^2\hat R_n(\bTheta)\|_\op =O(1)$ on $\Omega_1$, we have $\sigma_{\min}(\bJ_{(\bbV,\bTheta)} \bG) = e^{-o(n)}$ on this event.
Further, under the same conditions, Lemma~\ref{lemma:min_sv_Theta} of Appendix shows that for any $C,c$, there exists $\delta>0$ sufficient small so that
\begin{equation}
\Omega_2 := \left\{ \textrm{for all}\; \bTheta 
\;\textrm{with}\;
\|\bTheta\|_F \le C\;
\textrm{and}\;
\grad \hat R_n(\bTheta)  = \bzero,
\;\textrm{if}\;
\sigma_{\min}(\bL) \ge  c\; 
\;\textrm{then}\;
\sigma_{\min}([\bTheta,\bTheta_0]) \ge \delta,\;  
\right\}
\end{equation}
is a high probability event. 
Hence, on $\Omega_0 \cap\Omega_1\cap\Omega_2$, we have $\hat\bTheta_n \in \cZ_n$ of Eq.~\eqref{eq:set_of_zeros_main} for some choice of $\sPi$ satisfying Assumption~\ref{ass:params}.

Then, by Theorem~\ref{thm:convexity} 
there exists some $c_0(\eps) >0$ such that
for any $\mu \in\cuT(\cuA^c_\eps),\nu \in\cuV(\bR(\mu),\cuB^c_\eps)$, $\sup_{\bS\succ\bzero}\Phi_\cvx(\mu,\nu,\bR(\mu)) > c(\eps)$ uniformly. 
So
using the shorthand 
$\hmu:=\hat\mu_{\sqrt{d}[\hat\bTheta_n,\bTheta]}$ and 
$\hnu := \hat\nu_{[\bX\hat\bTheta_n,\bX\bTheta]}$,
 we can bound for any $\delta>0$,
\begin{align*}
    \P\left( \left\{ W_2(\hmu, \mu^\opt) > \eps\right\}
\cup
\left\{ W_2(\hnu, \nu^\opt) > \eps\right\}
    \right)
    &\le  \P\left(\{(\hmu,\hnu) \in\cuA_\eps^c \times \cuB_\eps^c\} \cap \{\bw \in\cG_\delta\}\cap \Omega_0\cap\Omega_1\right)\\
&\quad\quad + \P(\Omega_0^c)
+ \P(\Omega_1^c)+ \P(\cG_\delta^c)\\
&\le \P\left(\one_{\hat\bTheta_n \in \cZ_n(\cuA_\eps^c,\cuB_\eps^c, \sPi)} \one_{\bw \in\cG_\delta}\right)
+ \P(\Omega_0^c)
+ \P(\Omega_1^c)
+ \P(\cG_\delta^c).
\end{align*}
Taking $n\to\infty$ and noting that
\begin{equation}
    \lim_{n\to\infty }
( \P(\Omega_0^c)
+ \P(\Omega_1^c)
+ \P(\cG_\delta^c)) = 0
\end{equation}
by the assumption on $\hat\bTheta_n$ and Assumption~\ref{ass:noise},
we conclude by Theorem~\ref{thm:convexity} that for all $\eps>0$
\begin{equation}
    \lim_{n\to\infty}\P\left( \left\{W_2(\hmu, \mu^\opt) > \eps\right\}
\cup
\left\{ W_2(\hnu, \nu^\opt) > \eps\right\}
    \right)
    \le 
\lim_{n\to\infty}
\E[\cZ_{n}(\cuA_\eps^c, \cuB_\eps^c, \sPi)\one_{\{\bw\in\cG_\delta\}} ] \le \lim_{n\to\infty} e^{- n c(\eps)} = 0
\end{equation}
giving the statement of \textit{1} of the theorem.

Claim \textit{2} now follows from the convergence in point~\textit{1} and Proposition~\ref{prop:uniform_convergence_lipschitz_test_functions}. This concludes the proof of the theorem.

\subsection*{Acknowledgments}
This work was supported by the NSF through award DMS-2031883, the Simons Foundation through
Award 814639 for the Collaboration on the Theoretical Foundations of Deep Learning, 
and the ONR grant N00014-18-1-2729.

%\paragraph{Convergence of the distribution of the hessian: proof of item~\textit{2.}}



%by Lemma~\ref{lemma:rate_matrix_ST} along with an argument similar to that of Lemma~\ref{lemma:asymp_ST}, we can deduce that for any $\hnu \Rightarrow \nu$ in probability,
%\begin{equation}
%     \frac1{dk} (\bI_k \otimes \Tr) \left(\bH(\hnu_n) - z\bI_{dk}\right)^{-1} \to \alpha \bS^\opt(\nu,z)
%\end{equation}
%in probability. The claim now follows by from \textit{1.} after recalling the definition of $\bH$.


%\subsection{Proof of Proposition~\ref{prop:simple_critical_point_variational_formula}}
%\bns{Do we add this here?}




%
%********************************************************
%
%\section{Notation}
%\begin{table}[htbp]\caption{Table of common notational shorthands used}
%\centering % to have the caption near the table
%\begin{tabular}{r c p{10cm} }
%$\bV \in\R^{n\times k},\bV_0 \in\R^{n\times k_0},\bbV\in\R^{n\times(k+k_0)}$ &$\:=$& indices of gradient process, $\bbV = (\bV,\bV_0)$\\
%$\bRho \in\R^{d\times k}$ &$:=$ & Jacobian of $\rho(\bTheta)$\\
%$\bSec\:= \bSec(\bbV,\bw)\in\R^{dk\times dk}$ &$\:=$& matrix with blocks $\bK_{i,j}:=\Diag(\partial_{i,j}\ell(\bbV)),i,j\in[k]$\\
%$\bH\equiv \bH(\bbV,\bw)$& $:=$&$\left(\bI_k \otimes \bX\right)^\sT \bSec(\bbV) (\bI_k\otimes\bX)$\\
%$\bG(\bbV,\bTheta) $ &$:=$ & the KKT constraint function $\bL^\sT\bbV + \bRho(\bTheta)^\sT\bTheta $\\
%$\hnu,\hmu$ & $:=$ & empirical measures of $(\bV,\bU,\bw), \sqrt{d}(\bTheta,\bTheta_0)$, respectively.\\
%$\cuP(\cK),\cK\subseteq\R^{m}$ &$\equiv$& set of probability measures with support in $\cK$.\\
%$\alpha_n$ &$:=$& $n/d$\\
%$r_k$ &$:=$& $(k^2 + kk_0).$\\
%$k_+(d)$&$:=$& $k \vee \log(d)$\\
%$\sfD$ & $:=$ & $\sup_{\bv,\bu,w} \norm{\grad^2\ell(\bv,\bu,w)}_\op$\\
%$\sfS_{\bbV}$ & $:=$ & $\frac1{\sqrt{n}}\sup_{\cB} \norm{(\bV,\bU)}_\op$\\
%$\sfA_{\bbV}$ & $:=$ & $\frac1{\sqrt{n}}\sup_{\cB} \norm{(\bV,\bU)}_F$\\
%$\sfS_{\bL}$ & $:=$ & $\frac1{\sqrt{n}}\sup_{\cB} \norm{\bL(\bV,\bU,\bw)}_\op$\\
%$\sfs_{\bL}$ & $:=$ & $\frac1{\sqrt{n}} \inf\sigma_{\min}(\bL(\bV,\bU,\bw))$\\
%$\sfS_{\bRho}$ & $:=$ & $\frac1{n} \sup_{\cA}\norm{\bRho}_\op$\\
%$\sfS_{\grad^2\rho}$ & $:=$ & $\frac1{n}\sup \norm{\grad^2\rho(\bTheta)}_\op$\\
%%$\sfR_n$ & $:=$ & $\sup_{\bTheta} \|\bR(\bTheta,\bTheta_0)\|_\op$\\
%%$\sfV_n$ & $:=$ & $\sup_{\bV,\bU} \|(\bV,\bU)\|_\op/\sqrt{n}$\\
%%$\sfL_n$ & $:=$ & $\sup_{\bV,\bU} \|\bL(\bV,\bU)\|_\op/\sqrt{n}$\\
%$\bF_z(\bS; \nu)$ &$:=$&$\left( \E_\nu[(\bI + \grad^2\ell\bS )^{-1}\grad^2\ell] - z \bI \right)^{-1}.$\\
%%\bF(\bQ ;\nu )$ &$:=$& $\alpha^{-1} \bQ^{-1} - \E_\nu[(\bI_{k} + \grad^2\ell \bQ)^{-1} \grad^2 \ell]\in\C^{k\times k}$.\\
%$\mu_\cI$ for $\mu\in\cuP(\R^{m}),\cI\subseteq [m]$ &$\equiv$&  the marginal of $\mu$ on coordinates indexed by $\cI$.\\
%$\pi_\cI(\cS),\cS \subseteq\R^m, \cI \subseteq [m]$ &$\equiv$&
%the projection of $\cS$ in the coordinates indexed by $\cI$.\\
%$\bS_n(z)\equiv \bS_n(z;\bbV,\bw)$& $:=$&$(\bI_k \otimes \Tr)(\bH - z n \bI_{dk})^{-1}.$\\
%$\cM \equiv \cM(\cA,\cB)$,$\cA\subseteq \cuP(\R^{k+k_0)},\cB\subseteq\cuP(\R^{k+k_0+1})$&$\equiv$& Parameter manifold defined in Eq.~\eqref{eq:param_manifold_def}.\\
%$\cM_0 \equiv \cM_0(\cA,\cB)$,$\cA\subseteq \cuP(\R^{k+k_0)},\cB\subseteq\cuP(\R^{k+k_0+1})$&$\equiv$& (without the $\bG = 0$ constraint)
%\end{tabular}
%\label{tab:notation}
%\end{table}
\newpage
\appendix

\newpage
\section{Random matrix theory: the asymptotics of the Hessian}
\label{sec:RMT}
The goal of this section is to study the asymptotics of the spectrum of the Hessian $\bH$
originally defined in Eq.~\eqref{eq:bH_def} of Section~\ref{sec:pf_thm1}, and recalled bellow for convenience.
The section will culminate in the proof of Proposition~\ref{prop:uniform_convergence_lipschitz_test_functions} of Section~\ref{sec:pf_thm1}.

Recall the definitions
\begin{equation*}
\bH(\bTheta,\bbV;\bw) = \bH_0(\bbV; \bw) + n \grad^2 \rho(\bTheta), \quad\quad
    \bH_0(\bbV;\bw) = \left(\bI_k \otimes \bX\right)^\sT \bSec(\bbV;\bw)\left(\bI_k \otimes \bX\right)
\end{equation*}
where
\begin{equation*}
   \bSec(\bbV;\bw) = \begin{pmatrix}
\bSec_{i,j}(\bbV;\bw)
   \end{pmatrix}_{i,j \in[k]}
,\quad
    \bSec_{i,j}(\bbV;\bw)= \Diag\left\{\left(\frac{\partial^2}{\partial {v_i}\partial v_j}\ell(\bbV;\bw)\right)\right\}\in\R^{n\times n}, \quad i,j\in[k].
\end{equation*}
%uniformly for $(\bbV,\bTheta)\in\cM$.
%We state the main result of this section in the following proposition. 
%Let us make some definitions before giving the statement.
%
%Fix $z\in\bbH^+$. For $\nu\in\cuP(\R^{k+k_0+1}),\bS\in\C^{k\times k}$, define 
%\begin{equation}
%\label{eq:fp_eq}
%    \bF_z(\bS;\nu) := \big(\E_{(\bv,\bv_0,w)\sim\nu}\big[(\bI + \grad^2_\bv\ell(\bv,\bv_0,w)\bS)^{-1} \grad^2_\bv\ell(\bv,\bv_0,w)\big]  -z\bI\big)^{-1}.
%\end{equation}
%Let $\bS_\star(z;\nu)$ be the unique solution of the fixed point equation
%\begin{equation}
%    \bF_z(\bS_\star;\nu) = \alpha^{-1} \bS_\star.
%\end{equation}
%Define $\mu_{\MP}(\nu)$ as the measure whose Stieltjes transform is given by 
%\begin{equation}
%   s_{\MP}(z;\nu) := \frac1k \Tr(\bS_\star(z;\nu)).
%\end{equation}
%For $\mu\subseteq\cuP(\R^{k+k_0})$,
%let
%\begin{equation}
%\mu_\rho(\mu) := \bigcup_{j=1}^k \spec(\rho_{0\# \mu_{\{j\} }}).
%\end{equation}
%(Observe that $\mu_\rho(\mu)$ only depends on the marginal of the first $k$ coordinates, but we write it as such to simplify notation). 
%Finally, define
%\begin{equation}
%    \mu_\star(\mu,\nu) := \mu_{\MP}(\nu) \boxplus \mu_{\rho}(\mu).
%\end{equation}
%
%\bns{Write final result}
%\begin{proposition}
%\label{prop:uniform_convergence_lipschitz_test_functions}
%Under Assumption~\ref{ass:regime},~\ref{ass:loss},~\ref{ass:regularizer} and~\ref{ass:sets}, we have
%  for any Lipschitz function $f:\R\to\R$,
%  \begin{equation}
%  \lim_{\substack{n\to\infty\\n/d \to \alpha}}
%  \sup_{(\bbV,\bTheta)\in\cM(\cuA,\cuB)}\left|\frac1{dk}\E\left[\Tr \,f\left(\frac1n\bH(\bbV) + \grad^2\rho(\bTheta) \right)\right]
%      - \int f(\lambda) \mu_\star(\hnu_\bbV,\hmu_\bTheta)(\de \lambda)
%      \right| = 0.
%  \end{equation}
%\end{proposition}

As always, we'll often find it convenient to suppress the dependence on $\bbV$ in the notation. For example, we write $\bH$ for the matrix 
$\bH(\bbV)$.
Additionally, we will sometimes write $\bH(\hnu)$ when we want to emphasize the dependence of $\bH$ on $\hnu = \hnu_{\bbV}$.

The main object of analysis will be the empirical matrix-valued Stieltjes transform defined for $z\in \bbH^+$ and $\hnu \in\cuP_n(\R^{k+k_0+1})$ by
\begin{equation}
\label{eq:Sn_def}
\bS_n(z;\hnu) :=  (\bI_k \otimes \Tr)(\bH(\hnu) - z n \bI_{dk})^{-1}.
\end{equation}
%The goal of this section is to prove the following proposition.
%\begin{proposition}
%For $z \in \mathbb{H}_+$, let $\bS_\star \in\bbH_+^k$ be the unique solution to
%\begin{equation}
%\label{eq:ST_FP}
%\bS = \frac1{\alpha_n}\bF_z(\bS).
%\end{equation}
%Then \begin{equation}
%    \norm{\bS_n - \bS_\star}_{2} \le \dots
%\end{equation}
%with probability $\dots$.
%\end{proposition}

%\bns{I'll rewrite the following paragraph.}
%In Section~\notate{ref} below, we first show that $\bS_n$ satisfies the fixed point~\eqref{eq:ST_FP} approximately. In Section~\notate{ref}, we prove uniquness of this solution on $\bbH_+^k$. Then finally in Section~\notate{ref}, we show the claimed convergence $\bS_n$ to the solution $\bS_\star$.
%We defer technical lemmas to Section~\notate{ref}.

\subsection{Preliminary results}
This subsection summarizes some preliminary results useful for proofs of this section. 
\subsubsection{Some general properties of  \texorpdfstring{${(\id \otimes \Tr)},\Re$ and $\Im$}{(I x Tr),Re and Im}}


Let us present some properties of the operators $\Re,\Im$ and $\id \otimes \Tr$ (or $\bI\otimes \Tr)$ that will be useful for this section. Some proofs are deferred to~\ref{section:RMT_appendix_technical_results}.
\begin{lemma}[Properties of $\Re$ and $\Im$.]
\label{lemma:re_im_properties}
Let $\bZ \in\bbH^+_k$. Then,
\begin{enumerate}
\item 
$\bZ$ is invertible,
\begin{equation*}
    \Im(\bZ^{-1}) = - \bZ^{-1} \Im(\bZ) \bZ^{*-1}\prec\bzero, \quad\quad
\|\bZ^{-1}\|_\op \le \|\Im(\bZ)^{-1}\|_\op, \quad\textrm{and}\quad
\norm{\Im(\bZ)}_\op \le \norm{\bZ}_\op.
\end{equation*}

\item For any $\bW$ self-adjoint, we have
\begin{align*}
    \Im((\bI+\bW \bZ)^{-1}\bW ) &= -((\bI + \bW \bZ)^{-1}\bW)\Im(\bZ)((\bI + \bW \bZ)^{-1}\bW)^*,
\end{align*}
and
\begin{align*}
   \norm{(\bI + \bW\bZ)^{-1}\bW}_\op &\le \norm{\Im(\bZ)^{-1}}_\op.
\end{align*}
\end{enumerate}
\end{lemma}
The proof of the lemma above is deferred to Section~\ref{sec:proof_lemma_re_im_properties}.
\begin{lemma}[Properties of $(\bI\otimes\Tr)$]
\label{lemma:tensor_trace_norm_bounds}
\label{lemma:tensor_trace_properties}
Let $\bM \in\C^{dk\times dk}$. Then the following hold.
\begin{enumerate}[(1.)]
    \item We have the bounds
\begin{align*}
    \norm{(\bI_k \otimes \Tr)\bM}_{\Fnorm} \le  \sqrt{d}\norm{\bM}_\Fnorm
    \quad\textrm{and}\quad
    \norm{(\bI_k \otimes \Tr)\bM}_\op \le  d\norm{\bM}_\op.
\end{align*}
\item If $\bM^*  =\bM  \succ  \bzero$, then $(\bI_k \otimes \Tr)\bM \succ\bzero.$
The same statement holds if we replace both strict relations 
$(\succ)$ with non-strict ones $(\succeq)$.
\item If $\bM^* = \bM \succeq \bzero$, then
\begin{equation*}
    \lambda_{\min}\left( (\bI_k \otimes \Tr)\bM \right) \ge d \lambda_{\min}(\bM).
\end{equation*}
\item We have
\begin{equation*}
   \Im\left(
   (\bI_k \otimes \Tr) \bM
   \right) =  (\bI_k \otimes \Tr) \Im(\bM).
\end{equation*}
\end{enumerate}
\end{lemma}
The proof is deferred to Section~\ref{sec:proof_lemma_tensor_trace_properties}.

\subsubsection{Definitions, relevant norm bounds, and algebraic identities}
We give some definitions that will be used throughout this section. For $j\in [n]$, let
\begin{equation*}
    \bW_j := \grad^2 \ell(\bv_j, \bu_j, w_j) \in \R^{k\times k},
\quad\bxi_j := (\bI_k \otimes \bx_j)\in\R^{dk\times k}.
%\widetilde\bxi_i =  \bxi_i\bG_i \in \R^{dk\times k}.
\end{equation*}
Let $\sfK := \sup_{\bv,\bv_0,w} \norm{\grad^2 \ell(\bv,\bv_0,w)}_{\op}$.
With this notation,  we can write 
   $\bH =  \sum_{j=1}^n \bxi_j \bW_j \bxi_j^{\sT}.$
For $i\in[n],$ let $\bH_i := \sum_{j\neq i} \bxi_j \bW_j \bxi_j^\sT$,
and define the (normalized) resolvent and the leave-one-out resolvent as
\begin{equation*}
\bR(z) := \left(\bH - z n \bI_{dk}\right)^{-1}\quad
\textrm{and}\quad
\bR_i(z) := \left(\bH_i  - z n \bI_{dk}\right)^{-1},
\end{equation*}
respectively.
We present the following algebraic identities that will be used in the leave-one-out approach we follow in Section~\ref{sec:approx_ST_FP}. The proof of these is deferred to Section~\ref{sec:proof_lemma_algebra_lemma}.
\begin{lemma}[Woodbury and algebraic identities]
\label{lemma:algebra_lemma}
    For all $i\in[n]$ and $z\in\bbH_+$, we have
    \begin{equation}
\label{eq:alg_id1}
        \left(\bI_k \otimes \Tr\right)\bxi_i \bW_i \bxi_i^\sT \bR(z)  =  \left( \bI_k + \bW_i \bxi_i^\sT \bR_i(z) \bxi_i\right)^{-1}
        \bW_i\bxi_i^\sT \bR_i(z)\bxi_i,
    \end{equation}
and
\begin{equation}
\label{eq:alg_id2}
    \left(\bI_k \otimes \Tr\right) \left(\bR_i(z)-  \bR(z)\right)  =  \bxi_i^\sT \bR_i(z) (\bW_i \otimes \bI_d) \bR(z)\bxi_i.
\end{equation}

%   and
%   \begin{equation}
%\label{eq:alg_id2}
%      (\bI \otimes \Tr)\left(\bM_i^{-1} \bz_i (\bI_k + \widetilde \bz_i^\sT \bM_i^{-1}\bz_i)^{-1}\widetilde \bz_i^{\sT}\bM_i^{-1} \right)= 
%     \bz_i^\sT \bM_i^{-1} (\grad^2 \rho_i \otimes \bI_k) \bM^{-1} \bz_i.
%   \end{equation}
\end{lemma}

 The next lemma summarizes some \emph{a priori} bounds on the matrices
involved in the upcoming proofs.
The proof is deferred to Section~\ref{sec:proof_lemma_as_norm_bounds}.
%
\begin{lemma}[Deterministic norm bounds]
\label{lemma:as_norm_bounds}
For all $i\in[n]$ and $z\in\bbH_+$, we have, 
\begin{align}
\label{eq:det_norm_bound_lemma_eq123}
     \norm{\bR(z)}_\Fnorm^2 \vee \norm{\bR_i(z)}_\Fnorm^2 \le \frac{dk}{n^2} \frac{1}{\Im(z)^2},
     \quad
  \norm{\bR(z)}_\op \vee\norm{\bR_i(z)}_\op  \le  \frac{1}{n} \frac1{\Im(z)},
  \quad
  \norm{\bSec}_\op \le \sfK,
   \quad
   \norm{\bH}_\op \le \sfK \norm{\bX}_\op^2.
\end{align}
Further, for $z \in\bbH_+$, we have
\begin{equation}
\label{eq:det_norm_bound_lemma_eq4}
    \|\Im((\bI_k\otimes\Tr)\bR(z))^{-1}\|_\op 
    %\le  \frac1{\Im(z)}\frac{1}{\sigma_{\min}((\bH - z\bI)^{-1})^2} 
    \le   \frac{1}{\Im(z)} \left(\frac{1}{n}\norm{\bH}_\op  + |z|\right)^2
\end{equation}
and
\begin{equation}
\label{eq:det_norm_bound_lemma_eq5}
    \|\Im(\bxi_i^\sT\bR_i(z)\bxi_i)^{-1}\|_\op \le \frac{n}{\norm{\bx_i}_2^2} \frac{1}{\Im(z)} \left(\frac{1}{n}\norm{\bH_i}_\op  + |z|\right)^2.
\end{equation}
%\begin{align}
%    \norm{\widetilde \bz_i^\sT \bM_i^{-1} \bz_i} = O(1), 
%\norm{\grad^2 \rho_i (\bI\otimes \Tr) \bM_i^{-1}} = O(1)\\
%    \norm{(\bI+\widetilde \bz_i^\sT \bM_i^{-1} \bz_i)^{-1}} = O(1), 
%\norm{(\bI+\grad^2 \rho_i (\bI\otimes \Tr) \bM_i^{-1})^{-1}} = O(1).\\
%\end{align}
\end{lemma}
%

We also recall the following textbook fact 
for future reference.
\begin{lemma}[Operator norm bounds for Gaussian matrices
\cite{BaiSilverstein}]
\label{lemma:standard_norm_bounds}
 Let
 $$\Omega_0 := \{\norm{\bX}_\op \le 2 d^{1/2}(1 + \sqrt{\alpha_n})),\; \norm{\bx_i}_2 \in [d^{1/2}/{2}, 2d^{1/2}] \quad\textrm{for all}\quad i\in[n]\}.$$
Then 
\begin{equation}
    \P(\Omega_0^c ) \le C \exp\{- c d\}
\end{equation}
for some universal $C,c>0$.
\end{lemma}

\subsubsection{Concentration of tensor quadratic forms}

Finally, we end this section with the following consequence of the Hanson-Wright inequality for the concentration of quadratic forms of (sub)Gaussian random variables.
%\begin{lemma}
%\label{lemma:hanson-wright}
%Let $\bx \sim \cN(0,\bI_d), \bxi := (\bI_k \otimes \bx)$.
%   Let $\bM \in\C^{dk\times {dk}}$ be independent of $\bx$. Then for any $t>0$, 
%\begin{equation}
%    \P\left(\norm{\bxi^\sT \bM  \bxi -  (\bI_k \otimes \Tr)\bM}_F \ge t \right) \le
%    2 k^2\exp\left\{ -\frac{c t^2 }{k^2 \norm{\bM}_F^2 + k t \norm{\bM}_\op}  \right\}
%\end{equation}
%where $c>0$ is some universal constant.
%\end{lemma}
%
%\begin{proof}
%The statement follows from an element-wise application of Hanson-Wright. Namely, let $\bM_{j,l}\in\C^{d\times d}$  for $j,l\in[k]$ denote the blocks of the matrix $\bM$, 
%and let 
%$F^2_{jl}:=  \norm{\bM_{j,l}}_F^2$
%and
%$O_{jl}:=  \norm{\bM_{j,l}}_\op$. Then Hanson-Wright~\notate{cite} gives, for any $t>0$ and some fixed universal constant $c>0$,
%\begin{align}
%\P\left( \left|\bx^\sT \bM_{l,j} \bx - \Tr(\bM_{l,j}) \right| \ge t\right) 
%\le 2 \exp\left\{ - \frac{c \, t^2}{
%F_{jl}^2+
% O_{jl} \, t 
%}\right\}.
%\end{align}
%Then via a union bound, we obtain
%\begin{align}
%    \P\left(\norm{ \bxi^\sT \bM \bxi -  (\bI_k \otimes \Tr)\bM }_F   \ge t \right)
%    &\le 
%     \P\left(\norm{\bxi^\sT \bM \bxi -  (\bI_k \otimes \Tr)\bM }_\infty   \ge t/k \right)\\
%    &\le 2 k^2  
%      \exp\left\{ - \frac{c \, t^2/k^2}{
%\max_{j,l}\left\{ F_{jl}^2\right\}+
% \max_{j,k }\left\{O_{jl} \right\}\, t /k
%}\right\}.
%\end{align}
%Now what remains is to use the bounds
%\begin{equation}
%   \max_{jl}  F_{jl}^2 \le \norm{\bM}_F^2 
%   \quad\textrm{and}\quad
%\max_{jl}O_{jl} 
%\le \norm{\bM}_\op.
%\end{equation}
%\end{proof}
%
\begin{lemma}
\label{lemma:hanson-wright}
Let $\bx \sim \normal(\bzero,\bI_d), \bxi := (\bI_k \otimes \bx)$.
   Let $\bM \in\C^{dk\times {dk}}$ be independent of $\bx$, and 
   set $k_+(d):= k\vee \log d$. Then,
   for any $L\ge 1$, we have
\begin{equation*}
    \norm{\bxi^\sT \bM  \bxi -  (\bI_k \otimes \Tr)\bM}_\op \le
     C L \left( k_+(d)^{1/2} d^{1/2} \norm{\bM}_\op \vee k_+(d) \norm{\bM}_\op \right)
\end{equation*}
with probability at least
\begin{equation*}
    1 - 2\min\Big(e^{-cLk}, d^{-cL}\Big)\,
\end{equation*}
where $C,c > 0$ are universal constants.
\end{lemma}

\begin{proof}
Let $\cN$ be a minimal $1/4$-net of the unit ball in $\C^k$.
Then we have
\begin{equation*}
   \|\bxi^\sT\bM \bxi - (\bI_k \otimes\Tr) \bM\|_\op \le
2 \sup_{\bu,\bv\in\cN}  \bu^\sT\left(\bxi^\sT\bM \bxi - (\bI_k \otimes\Tr) \bM\right)\bv.
\end{equation*}
Meanwhile, for any fixed $\bu,\bv\in\cN$, note that $(\bI_k\otimes\bx)\bu = (\bu \otimes\bI_d)\bx$ so that Hanson-Wright gives
    \begin{align*}
        \bu^\sT\bxi^\sT\bM\bxi\bv -  \bu^\sT(\bI_k \otimes \Tr)\bM\bv
        &=
        \bu^\sT\bxi^\sT\bM\bxi\bv -
\E_\bx\left[\bu^\sT\bxi^\sT\bM\bxi\bv \right]
        \\
        &= \bx^\sT(\bu \otimes\bI_d)^\sT \bM (\bv \otimes\bI_d) \bx
        - \Tr\left((\bu \otimes\bI_d)^\sT \bM (\bv \otimes\bI_d)\right) \\
        &\le s
    \end{align*}
with probability larger than 
\begin{equation*}
1 - 2\exp\left\{  -c_0\left(\frac{s^2}{\norm{\bM}_\op^2 d} \wedge 
\frac{s}{\norm{\bM}_\op}
\right)
\right\},
\end{equation*}
where we used that $\norm{\bu\otimes\bI_d}_\op \le \norm{\bu}_2 \le1$ (and same for $\bv$) to deduce
\begin{equation*}
    %\norm{(\bu \otimes \bI_d)^\sT \bM (\bv\otimes \bI_d)}_F \le 
%\norm{\bM}_F, \quad
    \|(\bu \otimes \bI_d)^\sT \bM (\bv\otimes \bI_d)\|_F \le 
     \sqrt{d}
\norm{\bM}_\op, \quad
    \|(\bu \otimes \bI_d)^\sT \bM (\bv\otimes \bI_d)\|_\op \le 
\norm{\bM}_\op.
\end{equation*}
A standard result \cite{vershynin2018high} gives that the size of $\cN$ is at most $C_0^k$ for some $C_0 > 0$. Then taking 
\begin{equation*}
%    s =
%\left(\frac{2 \log(C_0) k  \norm{\bM}_F^2}{c_0}\right)^{1/2}
%    \vee   \frac{2 \log(C_0) k \norm{\bM}_\op}{c_0},
    s = 
 \left( L^{1/2} k_+(d)^{1/2} d^{1/2} \norm{\bM}_\op \vee L k_+(d) \norm{\bM}_\op \right),
\end{equation*}
we obtain via a union bound
\begin{align*}
    \P\left(\|\bxi^\sT\bM\bxi - (\bI_k\otimes\Tr)\bM\|_\op \ge 2 s \right)
    &\le 
    \P\left( \sup_{\bv,\bu\in\cN} \bu^\sT(\bxi^\sT\bM\bxi - (\bI\otimes\Tr)\bM) \bv \ge s  \right)\\
    &\le 2C_0^k \big (e^{-c L k}\wedge d^{-cL}\big)
\end{align*}
%
for some constant $c>0$. The claim follows by taking $L$ a sufficiently large universal constant. 
Redefining the universal constants $c$ and $C_0$ allow us to take $L\ge 1$ as in the statement.

\end{proof}

\subsection{Approximate solution to FP}
\label{sec:approx_ST_FP}
We begin with a concentration result.
\begin{lemma}[Concentration of the leave-one-out quadratic forms]
\label{lemma:concentration_loo_quad_form}
There exist absolute constant $c,C$, such that the following holds.
Let $k_+(d):= k\vee \log d$ and define the event (for $L\ge 1$)
\begin{equation}
\label{eq:concentration_loo_quad_form_2}
   \Omega_1(L) := \left\{\norm{\bxi_i^\sT \bR_i \bxi_i - (\bI_k \otimes \Tr)\bR}_\op
   \le  C L\sqrt{\frac{k_+(d)}{n \alpha_n}} \frac1{\Im(z)} 
   +  \frac{\sfK\norm{\bx_i}_2^2}{n^2 \Im(z)^2}\quad\textrm{for all}\quad i\in[n]\right\}.
\end{equation}
Then for  $n\ge \alpha_n k_+(d)$, $n\le d^{10}$ we have
for some universal constant $c>0$.
\begin{equation}
\nonumber
    \P(\Omega_1(L)^c) \le 2 (e^{- cLk}\vee d^{-cL}).
\end{equation}
\end{lemma}
\begin{proof}

For all $i\in[n]$, we have by Lemma~\ref{lemma:hanson-wright} and the bounds on the norms of $\bR_i$ in Lemma~\ref{lemma:as_norm_bounds}, 
\begin{equation}
\nonumber
    \norm{\bxi_i^\sT\bR_i \bxi_i - \left(\bI_k \otimes\Tr\right) \bR_i}_\op \le 
    %C \left(\frac{k }{\sqrt{n\alpha_n}} \frac1{\Im(z)} \vee \frac{k}{n} \frac1{\Im(z)}\right)
    C L\left(
    \sqrt{\frac{k_+(d)}{\alpha_n n}}
     \vee \frac{k_+(d)}{n} \right)
    \frac1{\Im(z)}
\end{equation}
%
with probability  at least $1-2 (e^{- cLk}\vee d^{-cL})$.
Meanwhile, we have by
Lemma~\ref{lemma:algebra_lemma} and bound of Lemma~\ref{lemma:as_norm_bounds} once again that
    \begin{align}
    \nonumber
        \norm{(\bI_k \otimes \Tr)\left(\bR_i - \bR\right) }_\op = 
\norm{\bxi_i^\sT \bR_i (\bW_i \otimes \bI_d) \bR\bxi_i}_\op
\le \sfK \norm{\bx_i}_2^2 \norm{\bR}_\op\norm{\bR_i}_\op
\le \sfK \frac{\norm{\bx_i}^2}{n^2}\frac1{\Im(z)^2}.
%
    \end{align}
A triangle inequality and union bound gives the result.
\end{proof}

We are ready to prove an approximate 
fixed point equation for $\bS = \bS_n$ defined in Eq.~\eqref{eq:Sn_def}.
\begin{lemma}[Fixed point equation for the Stieltjis transform]
\label{lemma:fix_point_rate}
Let $\Omega_0,\Omega_1(L)$ be the events of Lemmas~\ref{lemma:standard_norm_bounds} and~\ref{lemma:concentration_loo_quad_form} respectively.
For any empirical distribution $\hnu\in\cuP(\R^{k+k_0+1}),$
$z \in\bbH_+$, $L\ge 1$, we have on $\Omega_0 \cap \Omega_1(L)$,
\begin{align}
    \norm{
\frac1{\alpha_n}\bI_k - \bF_z(\bS_n(z);\hnu)^{-1}
\bS_n(z)
    }_\op
&\le   
\Err_{\FP}(z; n,k)
\end{align}
where, letting $k_+(d) = k\vee \log d$,
\begin{align}
\Err_{\FP}(z;n,k) :=  C(\sfK) \frac{(1+|z|^4)}{\Im(z)^4} 
\left( L\sqrt{\frac{ k_+(d) }{n}}  
   +  \frac{1}{ n \Im(z)}\right)
%\omega_{\textrm{FP}}(z,n,k,\alpha_n):=C(\sfD)(\alpha_n+\alpha_n^{-1})
%     \left( \frac{1 + |z|^2}{\Im(z)} \right)
%    \left( \frac{1 + |z|^2}{\Im(z)}
%    + 1\right)
%\left(  L\sqrt{\frac{k_+(d)}{n}}\frac1{\Im(z)} 
%   +  \frac{1}{n \Im(z)^2}\right).
\end{align}
for some $C(\sfK)>0$ depending only on $\sfK$.
%with probability at least
%\begin{equation}
%% 1 - 2 nk^2 \exp\left\{-ck\right\} - C n \exp\left\{ -c d\right\}.
%\end{equation}
\end{lemma}

\begin{proof}
The proof proceeds as in the usual scalar case.
Namely, suppressing the argument $z \in\bbH_+$, 
we write
\begin{align}
\label{eq:decomposition_resolvent_eq}
\frac{d}{n} \cdot \bI_k 
= \frac1n\left(\bI_k \otimes \Tr\right)
\bR^{-1} \bR
= \left(\bI_k \otimes \Tr\right)
\left(\frac1n\sum_{i=1}^n \bxi_i \bW_i\bxi_i^\sT \bR 
 -z  \bR\right)
&= 
\left(\frac1n\sum_{i=1}^n 
\left(\bI_k \otimes \Tr\right)\left(
\bxi_i \bW_i\bxi_i^\sT \bR \right)
 - z  \bS_n  \right).
\end{align}
%Via a leave-one-out argument, we show that for each $i\in[n]$,
%$$\left(\bI_k \otimes \Tr\right)\bxi_i \bSec_i \bxi_i^\sT \bR  
%\approx
%\grad^2 \rho_i(\bI_k \otimes \Tr)\bR(\bI_k + \grad^2 \rho_i(\bI_k \otimes \Tr)\bR)^{-1}$$
%in an appropriate sense.
%Taking expectations with respect to $\widehat \nu_{\bV,\bU}$ will then allow us to conclude
%\begin{equation}
%    \frac{d}{n} \bI_k  \approx \E[\bSec \bQ(\bI + \bSec \bQ)^{-1}] - z \bQ.
%\end{equation}

Letting $\bA_i = \bxi_i^\sT \bR_i\bxi_i$ and recalling the definition $\bS_n = (\bI_k \otimes \Tr)\bR$, we bound
\begin{align*}
        \bDelta_i&:=
        \left(\bI_k \otimes \Tr\right)\bxi_i \bW_i \bxi_i^\sT \bR - \bW_i\bS_n(\bI_k + \bW_i\bS_n)^{-1} \\
        &=\left( \bI_k + \bW_i\bA_i\right)^{-1}\bW_i\bA_i
        - (\bI_k + \bW_i\bS_n)^{-1}\bW_i\bS_n\\
        &=
        \left( \bI_k + \bW_i\bA_i\right)^{-1}\bW_i\bA_i
        - (\bI_k + \bW_i\bS_n)^{-1}\bW_i\bA_i
       +\left( \bI_k + \bW_i\bS_n\right)^{-1}\bW_i\bA_i
        - (\bI_k + \bW_i\bS_n)^{-1}\bW_i\bS_n\\
&= (\bI_k +\bW_i\bA_i)^{-1}\bW_i (\bS_n -\bA_i) (\bI + \bW_i \bS_n)^{-1} \bW_i \bA_i
+ (\bI_k + \bW_i\bS_n)^{-1}\bW_i (\bA_i -\bS_n).
\end{align*}
where the first equality follows from Lemma~\ref{lemma:algebra_lemma}.

Lemma~\ref{lemma:concentration_loo_quad_form} above provides a bound for $\norm{\bA_i -\bS_n}_\op$ on the event $\Omega_1(L)$. Meanwhile, by Lemma~\ref{lemma:tensor_trace_properties}, $\Im(\bS_n) = (\bI_k \otimes \Tr)\Im(\bR) \succ\bzero $ since $\Im(\bR) \succ\bzero$. So we have by Lemmas~\ref{lemma:re_im_properties} and~\ref{lemma:as_norm_bounds} on $\Omega_0$
\begin{equation}
\nonumber
    \norm{(\bI + \bW_i\bS_n)^{-1}\bW_i}_\op 
    \le \norm{\Im(\bS_n)^{-1}}_\op
    \le   \frac{1}{\Im(z)} \left( \frac{\sfK}{n} \norm{\bX}_\op^2 + |z|\right)^2 \le \frac{C_1}{\Im(z)} (\sfK^2 + |z|^2)
\end{equation}
and similarly and by the same lemmas, we conclude on $\Omega_0$
\begin{equation}
\nonumber
    \norm{(\bI + \bW_i\bA_i)^{-1}\bW_i}_\op 
    \le   \frac{1}{\Im(z)} \frac{n}{\norm{\bx_i}_2^2} \left(\frac{\sfK}{n} \norm{\bX}_\op^2 + |z|\right)^2 
    \le \frac{C_2 \alpha_n}{\Im(z)} (\sfK^2 + |z|^2).
\end{equation}
%
Finally, on $\Omega_0$,
\begin{equation}
\nonumber
   \norm{\bA_i}_\op {\le} \norm{\bx_i}_2^2 \norm{\bR_i}_\op \le \frac{C_3}{\alpha_n \Im(z)}.
\end{equation}

Combining these bounds along with the one in Lemma~\ref{lemma:concentration_loo_quad_form} gives on $\Omega_0 \cap\Omega_1(L)$,
\begin{align*}
    \norm{\bDelta_i}_F
  &\le\norm{(\bI_k + \bW_i \bS_n)^{-1}\bW_i}_\op \norm{\bA_i -\bS_n}_\op \left(\norm{(\bI_k + \bW_i \bA_i)^{-1}\bW_i}_\op\norm{\bA_i}_\op + 1\right)
   \\ 
  &\stackrel{(a)}{\le}
  C_4(\sfK) \frac{1}{\Im(z)} (1 + |z|^2) 
\left( \frac{L\sqrt{ k_+(d)} }{\sqrt{n\alpha_n}} \frac1{\Im(z)} 
   +  \frac{\sfK}{\alpha_n n \Im(z)^2}\right)
   \frac{1}{\Im(z)^2}(\sfK^2 + |z|^2)
  \\
  &\stackrel{(b)}{\le} 
  C_5(\sfK) \frac{(1+|z|^4)}{\Im(z)^4} 
\left( \frac{L \sqrt{k_+(d)} }{\sqrt{n}}  
   +  \frac{1}{ n \Im(z)}\right)\\
   &\equiv \Err_{\FP}(z; n, k)
%&\le\frac{C}{\Im(z)}
%    \left(\frac{1}{\Im(z)} \left( \sfK^2 + |z|^2 \right)\right)
%    \left( \frac{1}{ \Im(z)} \left( \sfK^2 + |z|^2 \right) 
%    + 1\right)
%\left( \frac{L k_+(d) }{\sqrt{n\alpha_n}} \frac1{\Im(z)} 
%   +  \frac{\sfK}{\alpha_n n \Im(z)^2}\right)
%\le\omega_{\textrm{FP}}(z,n,k,\alpha_n)
\end{align*}
where in $(a)$ we used that $(\sfK + |z|^2)/\Im(z)^2+1 \le C_6(\sfK) (1+ |z|^2)/\Im(z)^2$ for some $C_6>0,$ and in $(b)$ we used that $\alpha_n \ge 1.$
%
%\am{I think there is a term $\|\bA_i\|_{\op}$ missing. Please, double check. Also, some factors $\alpha_n$ seem off. I get
%\begin{align}
%    \norm{\bDelta_i}_F&\le
%    \left(\frac{2}{\Im(z)} \left( 9\sfK^2 + |z|^2 \right)\right)
%    \left( \frac{8\alpha_n}{ \Im(z) } \left( 9\sfK^2 + |z|^2 \right) 
%    \|\bA_i\|_{\op}+ 1\right)\alpha_n
%\left(C L\frac{k_+(d) }{\sqrt{n\alpha_n}} \frac1{\Im(z)} 
%   +  \frac{\sfK}{\alpha_n n \Im(z)^2}\right)
%\end{align}}
Since this holds for all $i\in[n]$. using Eq.~\eqref{eq:decomposition_resolvent_eq} we conclude
%\begin{align}
%\frac1{\alpha_n} \bI_k  + z\bQ_n
%= 
%\frac1n\sum_{i=1}^n 
%\left(\bI_k \otimes \Tr\right)\left(
%\bz_i \widetilde\bz_i^\sT \bR \right),
%\end{align}
on $\Omega_0\cap\Omega_1$ we have
\begin{align}
\nonumber
&\norm{\frac1\alpha_n\bI_k + z \bS_n - \frac1n \sum_{i=1}^n\bW_i \bS_n(\bI_k + \bW_i \bS_n)^{-1} }_\op
\le
 \frac1n\sum_{i=1}^n \norm{\bDelta_i}_\op\le 
\Err_{\FP}(z;n,k).
\end{align}
\end{proof}




%As a corollary of the above and Hanson-Wright, the following lemma states that the quadratic forms $\bxi_i^\sT\bR_i\bxi_i$ concentrate.
%\begin{lemma}[Concentration of the leave-one-out quadratic forms]
%\label{lemma:concentration_loo_quad_form}
%For all $i\in[n]$ and $t>0$, we have
%\begin{equation}
%\label{eq:concentration_loo_quad_form_1}
%   \P\left( \norm{\bxi_i^\sT \bR_i \bxi_i - (\bI_k \otimes \Tr)\bR_i }_F  \ge  t \right) \le 
%2 k^2  
%      \exp\left\{ - \frac{c \Im(z)\, t^2 \, n}{
% \frac{dk^3}{ n \Im(z)}
%  +  tk 
%}\right\}
%\end{equation}
%for some universal constant $c >0$.
%Consequently, for any $s>0$, we have
%\begin{equation}
%\label{eq:concentration_loo_quad_form_2}
%   \norm{\bxi_i^\sT \bR_i \bxi_i - (\bI_k \otimes \Tr)\bR}_F
%   \le \frac{k^{3/2+s}}{\sqrt{n}}\frac1{\Im(z)} +  \frac{\sfK\norm{\bx_i}_2^2}{\alpha_n n^2 \Im(z)^2}
%\end{equation}
%with probability at least
%\begin{equation}
%    1 - 2 k^2\exp\left\{ -  \frac{ c k^{3+2s} \alpha_n }{ k^3 + k^{5/2+s} \alpha_n^{1/2}d^{-1/2} }\right\}. %- C \exp\left\{ - cd\right\}.
%\end{equation}
%\end{lemma}
%\begin{proof}
%The tail bound follows directly from Lemma~\ref{lemma:hanson-wright} after appying the bounds of Lemma~\ref{lemma:as_norm_bounds} on $\norm{\bR_i}_F$ and $\norm{\bR_i}_\op$.
%Meanwhile, for Eq.~\eqref{eq:concentration_loo_quad_form_2},
%Lemma~\ref{lemma:algebra_lemma} and the operator norm bound of Lemma~\ref{lemma:as_norm_bounds} gives
%    \begin{align}
%        \norm{(\bI_k \otimes \Tr)\left(\bR_i - \bR\right) }_\op = 
%\norm{\bz_i^\sT \bR_i (\bSec_i \otimes \bI_d) \bR\bz_i}_\op
%\le \sfK \norm{\bx}_2^2 \norm{\bR}_\op\norm{\bR_i}_\op
%\le \sfK \frac{\norm{\bx}_i^2}{n^2}\frac1{\Im(z)^2}.
%%\norm{\grad^2 \rho_i (\bI_k \otimes \Tr)\left(\bM_i^{-1} \bz_i (\bI_k + \widetilde \bz_i^\sT \bM_i^{-1} \bz_i)^{-1} \widetilde \bz_i^\sT \bM_i^{-1}\right) }_F\\
%%&\le \frac{2\sqrt{ndk}\sfK^2}{n^2 \Im(z)^2}
%    \end{align}
%%where the last equality holds 
%%with probability at least $1- Ce^{- c d}$.
%Taking $t = k^{3/2+s}n^{-1/2}/\Im(z)$ gives the result.
%%\begin{equation}
%%      %1 -  2k^2 \exp\left\{-  \frac{c n\sqrt{d}  }{ \sqrt{d} k^{3/2}  +  \sqrt{n k} }\right\}.
%%\end{equation}
%%Combining with the concentration bound of Eq.~\eqref{eq:concentration_loo_quad_form_1} for $t = k^{3/2+s}n^{-1/2}/\Im(z)$ gives the result.
%\end{proof}
%




%\begin{lemma}[High probability norm bounds]
%\label{lemma:hp_norm_bounds}
%For all $i \in[n]$, we have, 
%\begin{equation}
% \norm{
%   (\bI_k \otimes \Tr)
%\bM_i^{-1}\bz_i
%\left(\bI_k + \widetilde\bz_i^\sT \bM_{i}^{-1} \bz_i\right)^{-1} \widetilde\bz_i^\sT \bM_i^{-1} 
%}_F \le 
%\frac{ 2\sqrt{n d k } \sfK}{n^2 \Im(z)^2}
%\end{equation}
%with probability at least 
%\begin{equation}
%      1 -  2k^2 \exp\left\{-  \frac{c n\sqrt{d}  }{ \sqrt{d} k^{3/2}  +  \sqrt{n k} }\right\}
%\end{equation}
%for some universal constant $c>0$.
%\end{lemma}
%\begin{proof}
%Let $\bA := \bM_i^{-1} (\grad^2 \rho_i \otimes \bI_k) \bM^{-1}$ and for $j,l\in[k]$ let $(\bA)_{jl}\in\R^{d}$ be the $j,l$th block of $\bA$. Then by Lemma~\ref{lemma:algebra_lemma}, we have
%\begin{align}
%   (\bI_k \otimes \Tr)
%   \bM_i^{-1}
%\bz_i
%\left(\bI_k + \widetilde\bz_i^\sT \bM_{i}^{-1} \bz_i\right)^{-1} \widetilde\bz_i^\sT \bM_i^{-1} 
%&= 
%     \bz_i^\sT \bA \bz_i.
%\end{align}
%
%For all $t>0$ and some universal $c>0$, we have by Hanson-Wright and a union bound,
%\begin{align}
%\P\left(
%    \norm{(\bI_k \otimes \Tr)(\bz_i^\sT \bA \bz_i) -  (\bI_{k}\otimes \Tr) \bA }_F \ge t
%\right)
%&\le    \P\left(\sum_{j,l}\left|\bx_i^\sT \bA_{jl}\bx_i - \Tr(\bA_{jl})\right|^2  > t^2\right)\\
%&\le    \P\left(\max_{j,l}\left|\bx_i^\sT \bA_{jl}\bx_i - \Tr(\bA_{jl})\right|^2  > \frac{t^2}{k^2}\right)\\
%&\le 2 k^2 \exp\left\{ \frac{-c (t/k)^2}{ \max_{jl}F_{jl}^2 + (t/k)\max_{jl}O_{jl} }\right\}\\
%&\le 2 k^2 \exp\left\{ \frac{-c (t/k)^2}{ \norm{\bA}_F^2 + (t/k) \norm{\bA}_\op }\right\}.
%\end{align}
%
%Now note that
%\begin{equation}
%    \norm{\bA}_\op \le \norm{\bM_{i}^{-1}}_\op
%    \norm{\grad^2\rho_i}_\op
%    \norm{\bM^{-1}}_\op\le  \frac{\sfK}{n^2} \frac1{\Im(z)^2}
%\end{equation}
%and 
%\begin{equation}
%    \norm{\bA}_F^2 \le \norm{\bM_{i}^{-1}}_F^2
%    \norm{\grad^2\rho_i}_\op^2
%    \norm{\bM^{-1}}_\op^2\le  \frac{d k \sfK^2}{n^4} \frac1{\Im(z)^4}.
%\end{equation}
%Taking $t = (\sqrt{ndk} \sfK)/n^2 \Im(z)^2$, we obtain that with probability larger than 
%\begin{equation}
%   1 -  2k^2 \exp\left\{- \frac{ nd \sfK^2 }{ d k^{3/2} \sfK^2 + \sfK^2 \sqrt{nd k} }\right\}
%\end{equation}
%we have, using Lemma~\notate{ref}
%\begin{align}
%\norm{(\bI_k \otimes \Tr) (\bz_i^\sT\bA\bz_i)}_F &\le \norm{(\bI_k \otimes \Tr)\bA}_F  + t
%\le \frac{ 2\sqrt{n d k } \sfK}{n^2 \Im(z)^2}.
%\end{align}
%\end{proof}
\subsection{The asymptotic matrix-valued ST}
\label{sec:AsymptoticST}

\subsubsection{Free probability preliminaries}

In this section we collect some relevant free probability background. Most of what follows can be found in
\cite{nica2006lectures,mingo2017free}.
Let $(\cA,\tau)$ be a $C^*$-probability space.
An element $M\in\cA$ is said to have a free Poisson distribution with rate $\alpha_0$ if the moments of $M$ under $\tau$ correspond to the moments of the Marchenko-Pastur law with aspect ratio $\alpha_0$. For $M,T \in\cA$, if $M$ is a free Poisson element and $T$ is self-adjoint, then $M^{1/2} T M^{1/2}$ has distribution given by the multiplicative free convolution of the free Poisson distribution and the distribution of $T$. 


Given a distribution $\nu_{0}$ on $(\bv,\bu, w) \in\R^{k+k_0+1}$, 
consider a sequence (indexed by $m$)
of collections of deterministic real diagonal matrices $\left\{\bar\bSec_{a,b}^\up{m}\right\}_{a,b\in[k]}$ with
$\bar\bSec_{a,b}^\up{m}\in \R^{m\times m}$,
$\bar\bSec_{a,b}^\up{m} = \diag((\bar K_i)_{a,b}:\; i\le m)$
such that the following hold.
%\bns{Why uniformly bounded? I am using $\sfK$ here for $\|\grad^2\ell\|_\op$. What is $K_i$?}
The entries of these matrices are uniformly bounded and
letting $\sP_{\nabla^2 \ell}$ denote the probability distribution
of $\nabla^2\ell(\bv,\bu, w)$ when $(\bv,\bu, w) \sim\nu_0$,
we have
%
\begin{align}
\nonumber
\frac{1}{m}\sum_{i=1}^m\delta_{\bar K_i} \Rightarrow 
\sP_{\nabla^2 \ell}\, .
\end{align}
%
Equivalently, for any set of pairs 
$\cP= \{ (a_1,b_1),(a_2,b_2),\dots, (a_L,b_L)\}$, 
\begin{equation}
\label{eq:K_empirical_limit}
    \lim_{m\to\infty} \frac1m \Tr\left( \prod_{(i,j) \in \cP} 
   \bar \bSec_{i,j}^\up{m}
    \right) = \int \prod_{(i,j)\in\cP}  \partial_{i,j}\ell(\bv,\bu, w)  \, \, \de\nu_{0}(\bv,\bu,w).
\end{equation}
Now define $T_{i,j}\in(\cA,\tau)$ for each $i,j\in [k]$
so that for any $\cP$,
%
\begin{equation}
\label{eq:taus_of_prods_of_D}
     \tau\left( \prod_{(i,j) \in \cP} 
   T_{i,j}
    \right) = \int \prod_{(i,j)\in\cP}  \partial_{i,j}\ell(\bv,\bu,w) \,  \de\nu_{0}(\bv,\bu, w)
\end{equation}
and that $\{T_{i,j}\}_{i,j\in [k]}\cup \{M\}$ are free.

Let $\{\bX^\up{m}\}_{m\ge 1}$ be a sequence of $m\times p$ matrices of i.i.d standard normal entries so that
$m/p \to \alpha_0$.
Then for any noncommutative polynomial $Q$ on $n$ variables, and any $(i_1,j_1),\dots (i_m,j_m) \in [k]\times [k]$,
\begin{equation}
\label{eq:moment_convergence}
    \lim_{m\to\infty}  \E\left[\frac1m\Tr\; Q( ( m^{-1}\bX^{\up{m}\sT} \bar \bSec_{i_l,j_l}^\up{m}\bX^\up{m})_{l\in[m]})\right] =  \tau\left( Q\left( M^{1/2} T_{i_l,j_l} M^{1/2}\right)_{l\in[m]} \right).
\end{equation}
 The quantity on the right hand side of the last equation is completely determined by the moments of the 
 form~\eqref{eq:taus_of_prods_of_D} by freeness of $\{T_{i,j}\}_{i,j} \cup \{M\}$.

%Now let $\bSec_{i,j}$ for $i,j\in[k]$ be the diagonal matrices defined in~\notate{ref}. 
%For each $i,j \in[k]$, let $\nu_{i,j}$ be the empirical spectral distribution of $\bSec_{i,j}$ and define the sequence $\left\{\bSec_{i,j}^{(n)}\right\}_{n}$ in such a way so that for all $n$, the diagonal matrix $\bSec_{i,j}^{(n)}$ has spectral measure $\nu_{i,j}$.
\subsubsection{Characterization of the asymptotic matrix-valued Stieltjis transform}
%\label{sec:AsymptoticST}

Let $\cM^{k\times k}(\cA)$ denote the set of $k\times k$ matrices with values in $\cA$.
For $z\in\bbH_+$, $\nu_0\in\cuP(\R^{k+k_0+1})$, and $\alpha_0 >1$, define 
$\bH_{*}(\alpha_0,\nu_0),\bR_\star(z; \alpha_0, \nu_0) \in 
\cM^{k\times k}(\cA)$ via
\begin{align}
\label{eq:def_H_star}
\bH_\star(\alpha_0, \nu_0) &:= (M^{1/2}T_{i,j} M^{1/2})_{i,j\in[k]} \, ,\\
   \bR_\star(z; \alpha_0, \nu_0) &:= \left(
   \bH_\star(\alpha_0, \nu_0)- z (\bI_k \otimes \aid )\right)^{-1} ,
\end{align}
where, for $i,j\in[k]$, $T_{i,j}\in(\cA,\tau)$ satisfy Eq.~\eqref{eq:taus_of_prods_of_D} for $\nu_0$, 
and are free; here $M \in(\cA,\tau)$ is
a free Poisson element with rate $\alpha_0$. 
Further, let
\begin{equation}
\label{eq:def_S_star}
    \bS_\star(z; \alpha_0, \nu_0) := \left(\bI_k \otimes \tau\right) \bR_\star(z; \alpha_0, \nu_0) \in\C^{k\times k}.
\end{equation}
The following lemma shows that $\bS_\star$ satisfies 
the fixed point equation for the Stieltjes transform, as one might expect. Note however the utility of this result: it allows one to decouple the asymptotics of the Gaussian matrix $\bX$ from that of the measure $\nu_0.$
%
\begin{lemma}[Asymptotic solution of the fixed point equation]
\label{lemma:asymp_ST}
Fix $\nu_0\in\cuP(\R^{k+k_0+1})$.
Assume $\|\grad^2 \rho(\bv,\bu,w)\|_{\op}\le \sfK$ with 
probability one under $\nu_0$.
For any fixed positive integer $k$, $z\in \bbH_+$,  
and $\alpha_0 >1$, we have
    \begin{equation}
        \alpha_0 \bS_\star(z; \alpha_0, \nu_0) = 
        \bF_z(\bS_\star(z; \alpha_0, \nu_0);\nu_0).
    \end{equation}
\end{lemma}
\begin{proof}
Fix $k$ throughout and let $\bT \in\cM^{k\times k}(\cA)$ be defined by $\bT := ( T_{i,j})_{i,j\in[k]}$,
where, for $i,j\in[k]$, $T_{i,j}\in(\cA,\tau)$ satisfy Eq.~\eqref{eq:taus_of_prods_of_D} for $\nu_0$. 

Our goal is to use Lemma~\ref{lemma:fix_point_rate} to show that $\bS_\star(z;\alpha_0,\nu_0)$ satisfies the desired fixed point equation. Since this lemma is stated in terms of empirical measures, we define 
$\{\hnu_{0,m}\}_m$,
$\hnu_{0,m}\in\cuP_m(\R^{k+k_0+1})$
to be a sequence of empirical measures satisfying $\hnu_{0,m}\Rightarrow \nu_0$.
These in turn define  a sequence $\bar \bSec^\up{m} = \left(\bar \bSec^\up{m}_{i,j}\right)_{i,j \in[k]}$
of deterministic matrices 
satisfying~\eqref{eq:K_empirical_limit} for the given $\nu_0$.

Write $\bS_\star(z)$ via its power expansion:
We have, by boundedness of $\bT$,
for $|z|$ sufficiently large
%\am{Need to use the fact that
%$\max_{ij}\|\bar D_{i,j}\|<\infty$ or similar?}
\begin{align}
\nonumber
\bS_\star(z) :=
\bS_\star(z; \alpha_0,\nu_0)
&= (\bI_k \otimes \tau)\left[ \frac1{z}\left( z^{-1}(\bI_k \otimes M^{1/2})\bT(\bI_k \otimes M^{1/2} )   - (\bI_k \otimes \id) \right)^{-1}\right]
\\
&= \sum_{a=0}^{\infty} (-z)^{-{(a+1)}}(\bI_k \otimes \tau)\left[\left( 
(\bI_k \otimes M^{1/2})\bT (\bI_k \otimes M^{1/2} )
\right)^a\right].\label{eq:ExpSstar}
\end{align}
Then, Eq.~\eqref{eq:moment_convergence} gives
\begin{equation}
\nonumber
    (\bI_k \otimes \tau)\left[\left( 
(\bI_k \otimes M^{1/2})\bT (\bI_k \otimes M^{1/2} )
\right)^a\right] = \lim_{m\to\infty} 
    (\bI_k \otimes \frac1m \Tr)\left[\left( 
(\bI_k \otimes \bX^\sT)\bar\bSec^\up{m} (\bI_k \otimes \bX )
\right)^a\right].
\end{equation}
Hence, there exists $r_0$, possibly dependent on $k$, such that for $|z| > r_0$, 
\begin{align*}
    \bS_\star(z) &= \lim_{m\to\infty}
\sum_{a=0}^{\infty}
    (-z)^{-(a+1)}(\bI_k \otimes \frac1m \Tr)\left[\left( 
(\bI_k \otimes \bX^\sT)\bar\bSec^\up{m} (\bI_k \otimes \bX )
\right)^a\right]
= \lim_{m\to\infty}\bS_m(z, \hnu_{0,m})
\end{align*}
element-wise.
It follows again from  Eq.~\eqref{eq:ExpSstar}
that $\|\bS\|_{\op}\le C/|z|$ for $|z|\ge r_1$.
Finally, for $|z|\le r_2$,
$\bS\mapsto \bF_z(\bS;\nu_0)$ is continuous 
on $\|\bS\|_{\op}\le \eta$ 
(for suitable constants $r_1,r_2,\eta$). 
Eventually increasing $r_0$,
Lemma~\ref{lemma:fix_point_rate} implies that, for $|z|\ge r_0$
\begin{equation} 
\nonumber
    \bF_z(\bS_\star(z);\nu_0) = \lim_{m\to\infty}  \bF_z(\bS_m(z);\hnu_{0,m})
     =  \lim_{m\to\infty} \alpha_0\bS_m(z; \hnu_{0,m}) = \alpha_0\bS_\star(z).
\end{equation}
Since $z\mapsto \bF_z(\bS_\star(z))$ and $z\mapsto \bS_\star(z)$ are both analytic on $\bbH_+$ 
%($\bF_z(\bS)$ is analytic for all $\bS \in\bbH_+^k$),
we conclude the claim by analytic continuation.
\end{proof}

We now give a lower bound on the minimum singular value of $\bS_\star$. This will be necessary to establish that $\bS_\star$ is the unique solution of the Stieltjes transform fixed point equation.
\begin{lemma}
\label{lemma:smallest_singular_value_Sstar}
For any $z\in\bbH_+$, $\nu_0\in\cuP(\R^{k+k_0+1})$ and $\alpha_0 >1$, we have
\begin{equation}
\nonumber
    \norm{\Im(\bS_\star(z; \alpha_0, \nu_0))^{-1}}_\op \le \frac1{\Im(z)} \left(\sfK (1+ \alpha_0^{-1/2})^2 + |z|\right)^2.
\end{equation}
\end{lemma}


\begin{proof}
By the Gelfand-Naimark-Segal construction
\cite[Lecture 7]{nica2006lectures}, there exists a Hilbert space $\cH$ and a $*$-representation of $\cA$, $\pi :\cA \to\cB(\cH)$ and some $\psi_0 \in \cH$ with $\norm{\psi_0}_\cH = 1$ such that for any $A \in\cA$, 
\begin{equation}
    \tau(A) = \inner{\psi_0,\pi(A) \psi_0}_\cH.
\end{equation}
Let us identify $A$ with $\pi(A)$ in the notation below.
Recall the product space $\cH^{k} := \cH \times \cdots \times \cH$ with Hilbert inner product given by
   \begin{equation}
        \inner{
        (\xi_1,\cdots,\xi_k),(\bar\xi_1,\cdots,\bar\xi_k)}_{\cH^k} = \sum_{i=1}^k \inner{\xi_i, \bar \xi_i}_\cH
   \end{equation}
for $\xi_i,\bar \xi_i \in\cH$ for $i\in[k]$.
Now we can bound 
$\sigma_{\min}(\Im(\bS_\star))$ by the variational characterization 
\begin{align}
\sigma_{\min}(\Im(\bS_\star)) 
%&= \lambda_{\min}(\Im(\bS_\star))\\ 
&= 
   \lambda_{\min}\left((\bI_k \otimes \tau)\Im(\bR_\star)\right)
   =\hspace{-2mm} \inf_{\substack{\bu\in\C^k\\\norm{\bu}_2 = 1}}  \bu^* 
   \left(\bI_k\otimes \psi_0\right)^* \Im(\bR_\star)(\bI_k \otimes \psi_0) \bu
  \nonumber 
   \\
&= \hspace{-2mm}\inf_{\substack{\bu\in\C^k\\\norm{\bu}_2 = 1}} \psi_0^* (\bu\otimes \aid)^* 
    \Im(\bR_\star)(\bu \otimes \aid) \psi_0
    \nonumber
   \\
&\ge  \norm{\psi_0}_\cH^2 \norm{\bu \otimes \aid}_{\cH^k}^2 
\lambda_{\min}\left(\Im(\bR_\star)\right) 
= 
\lambda_{\min}\left(\Im(\bR_\star)\right).
\nonumber
\end{align}
So by Lemma~\ref{lemma:re_im_properties}, 
\begin{align}
\sigma_{\min}(\Im(\bS_\star)) 
&= \Im(z)\lambda_{\min}\left(\bR_\star^* \bR_\star\right)
\ge  \Im(z) \left( \norm{(\bI_k \otimes M^{1/2}) \bar \bSec (\bI_k \otimes M^{1/2})}_{\cB(\cH^k)} + |z| \right)^{-2}
\label{eq:lb_on_Sstar}
\end{align}
%where $(a)$  follows from $\Im(\bS_\star) \succ\bzero$
To bound the norm in the term above, 
note that for any nonnegative integer $p$, we have
   \begin{equation}
   \nonumber
       \left(\frac1k \Tr \otimes \tau\right)\left(|\bar \bSec|^{2p}\right) 
\stackrel{(a)}{=}
       \left(\frac1k \Tr \otimes \tau\right)\left(\bar \bSec^{2p}\right) 
= 
\lim_{n\to\infty}\left(\frac1k \Tr \otimes \frac1n\Tr\right)\left((\bar \bSec^\up{n})^{2p}\right) \stackrel{(b)}{\le} \lim_{n\to\infty}\norm{\bar \bSec^\up{n}}_\op^{2p} \stackrel{(c)}{\le} \sfK^{2p}
   \end{equation}
where $(a)$ follows from self-adjointness, $(b)$ follows by monotonicity of $L^p$ norms and $(c)$ follows from Lemma~\ref{lemma:as_norm_bounds}. Now taking both sides to the power $1/(2p)$ and sending $p\to\infty$ gives the bound
\begin{equation}
\nonumber
    \norm{\bar \bSec}_{\cB(\cH^k)} \le \sfK.
\end{equation}
Meanwhile, by definition of the free Poisson element $M$, we have
\begin{equation}
\nonumber
\norm{\bI_k \otimes M^{1/2}}_{\cH^k}^2 \le (1+ \alpha_0^{-1/2})^2.
\end{equation}
Combining the previous two displays with Eq.~\eqref{eq:lb_on_Sstar} gives the claim.
\end{proof}
As a summary of this section, we have the following corollary.
\begin{corollary}
\label{cor:S_star_min_singular_value_bound}
For any $z\in\bbH_+,\nu\in\cuP(\R^{k+k_0+1}),$ any positive integer $k$ and any $\alpha_n \in\Q$ with $\alpha_n > 1$, the fixed point equation 
\begin{equation}
    \alpha_n \bS = \bF_z(\bS; \nu)
\end{equation}
has a solution $\bS_\star(z; \alpha_n, \nu) \in \bbH_+^k$ satisfying
\begin{equation}
    \norm{\Im(\bS_\star(z; \alpha_n, \nu))^{-1}}_\op \le \frac1{\Im(z)} \left(\sfK (1+ \alpha_n^{-1/2})^2 + |z|\right)^2.
\end{equation}
Furthermore, $\mu_\star(\mu,\nu)$ is compactly supported uniformly in $\mu,\nu$.

%and 
%\begin{equation}
%   \norm{\bS_\star(z; \alpha_n, \widehat \nu_{\bV,\bU,\bw})}_\op \le  .
%\end{equation}
%\bns{Provide a proof of the operator norm bound on $\bS_\star$ by the same proof as above using GNS.}
\end{corollary}
\begin{proof}
  The first part of the statement follows directly from Lemma~\ref{lemma:smallest_singular_value_Sstar}.
  That $\mu_\star(\mu,\nu)$ is compactly supported follows from the definition of $\mu_\star$ as the measure whose Stieltjes transform is $\bS_\star$ of Eq.~\eqref{eq:def_S_star} and the definition in Eq.~\eqref{eq:def_H_star} and uniform bounds on the operator norm of $\bH_\star.$
\end{proof}

%
%   \begin{align}
%       \left(\frac1k \Tr \otimes \tau\right)\left(|(\bI_k \otimes M^{1/2})\widetilde \bSec (\bI_k \otimes M^{1/2})|^{2p}\right) 
%&\stackrel{(a)}{=}
%\lim_{n\to\infty}\left(\frac1k \Tr \otimes \frac1n\Tr\right)\left(
%\left((\bI_k\otimes n^{-1/2}\bX)^{\sT}\widetilde \bSec^\up{n} (\bI_k \otimes n^{-1/2}\bX)\right)^{2p}\right) \\
%&\stackrel{(b)}{\le} \lim_{n\to\infty}\norm{\widetilde \bSec^\up{n}}_\op^{2p} \stackrel{(c)}{\le} \sfK^{2p}
%   \end{align}
%where $(a)$ follows from self-adjointness and Eq.~\eqref{eq:moment_convergence}, and $(b)$ follows by monotonicity of $L^p$ norms and $(c)$ follows from Lemma~\ref{lemma:as_norm_bounds}. Now taking both sides to the power $1/(2p)$ and sending $p\to\infty$ gives the claim.





   





%%%Old existence and uniquness proof.
%\subsection{Existence and uniqueness of the solution of FP equation}
%For this section, it'll be convenient to define $\bG :\bbH_k^+ \mapsto \C$ by 
%\begin{equation}
%   \bG(\bS;\bSec) := (\bI + \bSec \bS)^{-1}\bSec
%\end{equation}
%Our goal is to prove the following.
%\begin{proposition}
%\label{prop:existence_unqiuness}
%For any $z\in \bbH^+$, the equation $\alpha\bS = \bF_z(\bS)$ has a unique solution $\bS_\star$ in $\bbH_k^+$.
%\end{proposition}
%This method of proof is due to~\notate{cite}
%for proving existence and uniquness of matrix quadratic equations (See Theorem 2.1). We apply the Earle-Hamilton fixed point theorem specialized below to our setting
%\bns{change $\bbH_k^+$ to $\bbH_+^k$: makes more sense.}
%\begin{theorem}[Reformulation of Earle-Hamilton Theorem]
%Let $\cH \subseteq \bbH_+^k$ be a connected open subset, and $\bF :\cH \mapsto \C^{k\times k}$ holomorphic.  If the image of $\bF(\cH)$ is bounded, and $\bF(\cH)$ lies strictly inside $\cH$, then $\bF$ has a unique fixed point $\bS$ in $\cH$.
%\end{theorem}
%
%\bns{need lower bound on singular value of $\bF(\bQ)$ unformly over all $\bQ\in\bbH_+^k$.}





%To apply this theorem in the proof of Proposition~\ref{prop:existence_unqiuness}, it's sufficient to show the following lemma. 
%For $R,r>0$, let
%\begin{equation}
%   \cH(R,r) :=   \left\{\bS: \Im(\bS)\succ r\bI_k,\; \norm{\bS}_\op < R\right\}.
%\end{equation}
%Given $z$, let $R(z),r(z) > 0$ satisfy
%\begin{equation}
%    R(z) = \frac2{\alpha\Im(z)},
%    \quad\quad
%    r\left(|z| + r\right)^2 = \frac{\Im(z)}{2\alpha},
%\end{equation}
%respectively. 
%\begin{lemma}
%Fix $z \in \bbH_+$. Then for all $R\ge R(z)$, $r < r(z)$, the image of $\cH(R,r)$ under $\alpha^{-1}\bF_z$ lies strictly inside $\cH(R,r)$.
%\end{lemma}
%\begin{proof}
%Let us suppress the dependence on $\bSec$ in what follows in the expectations.
%For $\bS \in\bbH_+^k$ and $z\in\bbH_+$,
%we have by Lemma~\ref{lemma:re_im_properties},
%\begin{equation}
%\label{eq:imagine_part_F}
%    \Im(\bF_z(\bS)) = -\bF_z(\bS) \left(\Im(\E[\bG(\bS)]) - \Im(z) \bI  \right)\bF_z(\bS)^* =
%\bF_z(\bS) \left(\E[\bG(\bS)\Im(\bS)\bG(\bS)^*] + \Im(z) \bI  \right)\bF_z(\bS)^* 
%\succ \bzero.
%\end{equation}
%Hence, by the bounds in this same lemma,
%\begin{equation}
%   \norm{\frac{1}{\alpha}\bF_z(\bS)}_\op \le \norm{\frac{1}{\alpha}\Im\left(\E[\bG(\bS;\bSec)] -z\bI\right)^{-1}}_\op 
%   = \norm{\frac{1}{\alpha}(\E[\bG(\bS)\Im(\bS) \bG(\bS)] + z\bI)^{-1}}_\op \le \frac1{\alpha\Im(z)} = \frac{R(z)}{2}.
%\end{equation}
%So the image of $\cH(R,r)$ is contained strictly in $\{\norm{\bS}_\op < R\}$ for all $R \ge R(z).$
%We show now that it's contained strictly in $\{\bS \succ r\}$ for any $r\le r(z).$
%    %As an illustration, let us first consider the case of $k=1$.
%    %Write $F_z$ and $\eta$ for the functions involved in this case.
%    %Then we can write
%    %\begin{align}
%    %    \Im( F_z) = \Im\left(\frac{F_z \overline F_z}{\overline F_z}\right)
%    %    =  |F_z|^2 \Im\left(\frac{1}{\overline F_z } \right)
%    %    =|F_z|^2 \Im \left(\overline z - \overline \eta\right) \le - |F_z|^2 \Im(z).
%    %\end{align}
%    %Now, we have 
%    %\begin{equation}
%    %|F_z|^2 \Im(z) = |(z - \eta)^{-1}|^2  |\Im(z)| \ge 
%    %\frac{|\Im(z)|}{\left(|z| + \sup_{q \in \cH(C)}|\eta(q)|\right)^2}.
%    %\end{equation}
%    %This gives a sufficient bound.
%    %We write now
%    %\begin{align}
%    %    \Im(\bF_z) = \frac1{2i}\left(\bF_z - \bF_z^*\right) = \bF_z^* \left( \frac{\bF_z^{* -1} - \bF_z^{-1}}{2i}\right)\bF_z = \bF_z^* \Im(\bF_z^{*-1}) \bF_z = \bF_z^* \left(\Im(\overline{z}\bI)- \Im(\boldeta^*)\right) \bF_z.
%    %\end{align}
%    %Now  since $-\Im(\boldeta^*) = \Im(\boldeta) \preceq \bzero$ by assumption, we have the bound
%First, we bound the lowest singular value of $\bF_z$ using Lemma~\ref{lemma:re_im_properties}:
%    \begin{align}
%    \norm{\bF_z^{*-1}}_\op \le \norm{\E[\bG(\bS)] - z\bI}_\op
%\le
%    \norm{\Im(\bS)^{-1}}_\op  + |z|
%    \le  \eps + |z|.
%    \end{align}
%Consequently, 
%by Eq.~\eqref{eq:imagine_part_F} once again, we have
%    \begin{equation}
%        \Im(\alpha^{-1}\bF_z) \succeq  \frac{\Im(z)}{\alpha}\bF_z\bF_z^*
%\succeq\frac{\Im(z)}{\alpha(|z| + r(z))^2} \bI_k = 2 r(z) \bI \succ r(z) \bI,
%\end{equation}
%which gives the desired conclusion.
%\end{proof}
%

%\begin{theorem}[Theorem 2.1, \notate{Cite}]
%Assume $\boldeta$ is holomorphic and satisfies $\Im(\boldeta(\bZ))\preceq\bzero$ for $\bZ\in\bbH_k^+$. 
%Then 
%\begin{equation}
%\bQ = \frac{1}{\alpha} \bF_z(\bQ)    
%\end{equation}
%has a unique solution in $\bbH^-_k$.
%\end{theorem}



%{Cite: Speicher; Operator-valued semicircular elements: solving a quadrtic matrix} for proving uniqueness of the fixed point equation for the operator valued semi-circular law.
%Given $\boldeta:\R^{k\times k} \mapsto \R^{k\times k}$ and $z \in \C$, define
%    \begin{equation}
%        \bF_z(\bQ) := \left( z\bI_{k} - \boldeta(\bQ)\right)^{-1}.
%    \end{equation}
%The goal is to show that $\bQ \mapsto \bF_z(\bQ)$ has exactly one solution in $\bQ \in \bbH_k^-$, for any $z \in \bbH^+$.

%\begin{proof}[Proof of Proposition~\ref{prop:existence_unqiuness}] 
%Fix any $z \in \bbH_+$. Then for any $R > R(z), r<r(z)$, $\alpha^{-1}\bF_z$ has a unique fixed point in $\cH(R,r)$ be Theorem~\notate{ref}. Uniqueness of this solution on $\bbH_+^k$ follows from observing that $\cH(R_p,r_p) \uparrow \bbH_+^k$ as $p\to\infty$ for $R_p := p R(z), r_p := r(z)/p$.
%\end{proof}


    

%\paragraph{Verification of the properties of $\eta$}
%
%For a symmetric matrix $\bSec\in\R^{k\times k}$, let $\bSec^{1/2}$ denote its square root in $\C^{k\times k}$.
%Define
%\begin{equation}
%    \boldeta(\bQ) := \E_\bSec[(\bI -\bSec\bQ)^{-1}\bSec]
%\end{equation}
%where the expectation is over $\bSec$ supported on real symmetric matrices.
%We show the following.
%\begin{lemma}
%The map $\boldeta$ is well-defined and holomorphic on $\bbH_k^-$. Furthermore, $\Im(\boldeta(\bQ))\preceq\bzero$ for $\bQ\in\bbH_k^{-}$.
%\end{lemma}
%\begin{proof}
%Fix $\bQ \in\bbH_k^-$.
%Let us first show that $(\bI-\bSec\bQ)$ is invertible for any fixed $\bSec$ real and symmetric.
%Namely, we'll show that for all non-zero $\bv \in\C^k$, $\bv^*(\bI - \bSec\bQ) \neq \bzero^*$.
%First note that if non-zero $\bv$ satisfies $\bSec\bv =\bzero$, then this is clear.
%Meanwhile, to see this for the case of $\bv$ with $\bu := \bSec\bv \neq \bzero,$ note that $\bSec,\Re(\bQ)$ and $\Im(\bQ)$ are all self-adjoint and hence
%\begin{equation}
%    \bv^* (\bI - \bSec \bQ)\bu =  \bv^*\bSec \bv - \bv^* \bSec \Re(\bQ) \bSec\bv  -i\bv^* \bSec \Im(\bQ) \bSec\bv = \bv^*\bSec \bv - \bu^* \Re(\bQ) \bu - i \bu^* \Im(\bQ) \bu
%\end{equation}
%is of the form $ a + ib$ for $a,b\in\R$ and $b = \bu^* \Im(\bQ) \bu  < 0$ by assumption on $\bQ$. Hence $\bv^*(\bI - \bSec\bQ) \neq \bzero^*$.
%
%This now implies that $\boldeta(\bQ)$ is holomorphic and well defined on $\bbH_k^-$. What remains is to show is that $\Im(\bQ) \preceq \bzero$ if $\bQ\in\bbH_k^-$ .
%
%%Let $\cG = \{\, \bSec\,\, \textrm{invertible} \,\}$.
%%On $\cG$,  we have $(\bI -\bSec\bQ)^{-1}\bSec = (\bSec^{-1} - \bQ)^{-1}$.
%%Since $\Im((\bSec^{-1} - \bQ)) = -\Im(\bQ) \succ\bzero$, we conclude by Lemma~\ref{lemma:re_im_properties} that $(\bSec^{-1} - \bQ)^{-1} \in\bbH_k^-$ on $\cG$.
%
%%Now for $\cG^c$,
%For fixed $\bSec$, let $\sigma_{\min} \equiv \sigma_{\min}(\bSec) \in\R$, and for $\eps \in( 0,1)$ define
%\begin{equation}
%    \bSec_\eps := \bSec + \eps \sigma_{\min} \sign(\sigma_{\min}) \bI_k
%\end{equation}
%The matrix $\bSec_\eps$ is then invertible, real and symmetric and hence 
%\begin{equation}
%\Im((\bI - \bSec_\eps\bQ)^{-1}\bSec_\eps) 
%=\Im((\bSec_\eps^{-1} - \bQ)^{-1})
%\prec \bzero
%\end{equation}
%so that for any non-zero $\bv \in\C^k$, $\eps \mapsto \bv^* \Im(
%(\bI - \bSec_\eps\bQ)^{-1}\bSec_\eps)\bv$ is strictly negative for all $\eps \in(0,1)$.
%Furthermore, it is continuous (see for instance the explicit form in Eq.~\eqref{eq:im_invs}). So taking $\eps \to 0$ shows that $\Im(
%(\bI - \bSec\bQ)^{-1}\bSec) \preceq \bzero$.
%Finally, note that
%    $\Im(\boldeta(\bQ))= \E\left[\Im((\bI - \bSec\bQ)^{-1}\bSec)\right] \preceq \bzero.$
%\end{proof}
%
%\begin{corollary}
%For any $z \in \mathbb{H}_+,$
%\begin{equation}
%   \frac1\alpha \bI - \E_{\widehat\nu}[(\bI + \grad^2 \rho \bQ)^{-1} \grad^2 \rho \bQ] + z \bQ = 0
%\end{equation}
%has a unique solution satisfying $\Im(\bQ(z)) \succ \bzero.$
%\end{corollary}
%



%Let $\bQ \in\bbH^{-1}$.
%Let's first show that $(\bI - \bSec^{1/2}\bQ\bSec^{1/2}) $
%is invertible. 
%Namely, we show that for any $\bv \in \C^k$, $\bv^* (\bI - \bSec^{1/2}\bQ \bSec^{1/2})$
%
%
%
%
%
%Let $\bA$  denote this matrix.
%We'll show that for all non-zero $\bv \in\C^{k}$, $\bv^*\bA \bv\neq 0$.
%
%We have
%\begin{equation}
%    \Re(\bA) = \bI - \bSec^{1/2}\Re(\bQ) \bSec^{1/2},\quad
%    \Im(\bA) 
%    = - \bSec^{1/2} \Im(\bQ) \bSec^{1/2}.
%\end{equation}
%So for any $\bv \in\C^k$, we have
%\begin{equation}
%   \bv^*\bA \bv = \bv^*\Re(\bA)\bv + i \bv^*\Im(\bA)\bv= 
%   \begin{cases}
%    \norm{\bv}_2^2,  & \bSec^{1/2}\bv  = 0\\
%    a + i b & \bSec^{1/2}\bv \neq 0
%   \end{cases}
%\end{equation}
%where $a,b\in\R$ with $b>0$ since $\bu := \bSec^{1/2}\bv \neq 0$ implies
%    $\bv^* \Im(\bA) \bv = -\bu \Im(\bQ) \bu  >0$
%by assumption on $\bQ$.
%
%
%





%Once again, let's start with $k=1$. Then we have
%\begin{equation}
%\Im(\eta(q)) = \Im\left(\E\left[\frac{d}{1 - dq}\right]\right)
%= \E\left[\frac{ d^2 \Im(q)}{|1 -dq|^2}\right] < 0.
%\end{equation}
%
%Consider the imaginary part of the term inside the expectation. We have
%\begin{align}
%    \Im((\bI - \bSec \bQ )^{-1}\bSec) &= 
%    \frac1{2i}\left((\bI - \bSec \bQ )^{-1}\bSec)-
%    (\bI - \bSec \bQ^* )^{-1}\bSec)
%    \right)\\
%   & = \frac1{2i} \left((\bI-\bSec\bQ)^{-1} \bSec(\bQ - \bQ^*)(\bI-\bSec\bQ^*)^{-1} \bSec\right)\\
%   &= \left((\bI-\bSec\bQ)^{-1}\bSec\right) \Im(\bQ) \left((\bI-\bSec\bQ)^{-1} \bSec\right)^* \\
%   &\preceq \bzero.
%\end{align}
%\bns{More work to be done here. Need assumption guaranteeing that $\P(\bSec \succ \bzero)\neq 0$ and need to bound the "modulus" properly.}

%\subsection{Convergence of $\mathbf{S}_n$ to $\mathbf{S}_\star$}
%The goal of this section is to prove the following Lemma.
%\begin{claim}
%For $z \in \mathbb{H}_+$, let $\bS_\star$ be the unique solution to
%\begin{equation}
%   \frac1\alpha_n \bI - \E_\nu[(\bI + \grad^2 \rho \bS)^{-1} \grad^2 \rho \bS] + z \bS = 0.
%\end{equation}
%Then \begin{equation}
%    \norm{\bS_n - \bS_\star}_{F} \le \dots
%\end{equation}
%with probability $\dots$.
%\end{claim}
%Let us introduce the notation
%\begin{equation}
%        \bG(\bS, \bSec) := (\bI + \bSec\bS)^{-1}\bSec.
%\end{equation}
%Throughout this section, let 
%\begin{equation}
%\bA := \Re(\bS_\star),\quad\bB := \Im(\bS_\star),
%\quad\bA_n :=\Re(\bS_n),\quad\bB_n := \Im(\bS_n).
%\end{equation}
%Note that $\bB,\bB_n \succ\bzero$.
%Consider the tensor $\bT_n:\C^{k\times k}\mapsto \C^{k \times k}$
%defined by
%\begin{equation}
%    \bT(\bM) := 
%     \alpha \bF_z(\bS_\star) \E[\bG(\bS_\star,\bSec) \bM  \bG(\bS_n,\bSec) ] \bF_z(\bS_n).
%\end{equation}
%Our goal is to write 
%\begin{equation}
%    \bS_n - \bS_\star  \approx \bT(\bS_n - \bS_\star)
%\end{equation}
%then bound the operator norm $\norm{\bT}_\op := \norm{\bT}_{F\to F}$ to obtain a bound on $\norm{\bS_n - \bS_\star}_F$.
%The following observation will simplify some computations: the matrix $\bS_\star$, and hence $\bG(\bS_\star,\bSec)$ are symmetric. To see this, it's sufficient to note that  for $\Im(z)>0$, if $\Im(\bS_*)\succ \bzero$ and solves $\bF_z(\bS) = \bS$, then $\Im(\bS_*^\sT)\succ\bzero$ and also solves the fixed point equation. Uniqueness (Lemma~\notate{ref}) then proves $\bS_\star = \bS_\star^\sT$.
%
%
%Our first lemma gives a deterministic bound on $\Tr(|\bT|^p)$ for all positive integer $p$.
%\begin{lemma}
%For any $\bM \in\C^{k\times k}$ and any integer $p>0$, we have
%\begin{align}
%    \norm{\bT^p(\bM)}_F^2&\le 
%    \alpha^{2p} 
%    \norm{\bM}_F^2 
%\left(\norm{\E\left[\bB^{-1/2} \bF \bG \bB\bG^* \bF^*\bB^{-1/2}\right]}_\op^{2p-1}
%\norm{\bB}_\op
%\norm{\bB^{-1/2}\bF^* \bF \bB^{-1/2}}_\op
%\norm{
%\E[ \bG^*
%\bB\bG]}_\op
%\norm{\bB_n^{-1}}_\op\right)^{1/2}
%\nonumber\\
%&
%    \quad\left(\norm{\E\left[\bB_n^{-1/2} \bF_n \bG_n \bB_n\bG_n^* \bF_n^*\bB_n^{-1/2}\right]}_\op^{2p-1}
%\norm{\bB_n}_\op
%\norm{\bB_n^{-1/2}\bF_n^* \bF_n \bB_n^{-1/2}}_\op
%\norm{
%\E[ \bG_n^*
%\bB_n\bG_n]}_\op
%\norm{\bB^{-1}}_\op\right)^{1/2}
%\end{align}
%where 
%$\bT^p(\bA) := \bT(\bT^{p-1}(\bA))$
%and
%\begin{equation}
%\bF_n := \bF_z(\bS_n),\quad \bF:=\bF_z(\bS_\star),\quad \bG_n \equiv\bG_n(\bSec) :=  \bG(\bS_n, \bSec),\quad \bG \equiv \bG(\bSec) := \bG(\bS_\star,\bSec).
%\end{equation}
%\end{lemma}
%
%\begin{proof}
%In what follows, 
%let $\bSec_1,\dots,\bSec_p, \widetilde \bSec_1,\dots,\widetilde\bSec_p$ be i.i.d. copies of $\bSec$.
%We use the shorthand $\bG_{n,i} \equiv \bG(\bQ_n;\bSec_i)$ and 
%$\widetilde\bG_{n,i} \equiv \bG(\bQ_n;\widetilde\bSec_i)$.
%Now we write
%\bns{Fix $\bF$ to $\bF_n$. Fix $\bB$ to $\bB^{-1}$.}
%\begin{align}
%\frac1{\alpha^{2p}}\frac1k\Tr\left( |
%\bT^p(\bM)|^2\right)
%&= \frac1k \Tr\left(
%\E\left[
%\prod_{i=p}^1 (\bF \bG_i)\bM \prod_{i=1}^p(\bG_{n,i} \bF)
%\prod_{i=p}^1 ( \bF^*\widetilde \bG_{n,i}^*)
%\bM^*
%\prod_{i=1}^p ( \widetilde \bG_{i}^*\bF^*)
%\right]
%\right)
%\\
%&= 
%\frac1k 
%\E\left[
%\Tr\left(
%\bB^{1/2}\bM \prod_{i=1}^p(\bG_{n,i} \bF)
%\prod_{i=p}^1 ( \bF^*\widetilde \bG_{n,i}^* )
%\bB_n^{-1/2}\bB_n^{1/2}
%\bM^*
%\prod_{i=1}^p ( \widetilde \bG_{i}^*\bF^*)
%\prod_{i=p}^1 (\bF \bG_i)\bB^{-1/2}
%\right)
%\right]
%\\
%&\le 
%\frac1k 
%\E\left[
%\Tr\left(
%\left|
%\bB^{1/2}
%\bM \prod_{i=1}^p(\bG_{n,i} \bF)
%\prod_{i=p}^1 ( \bF^*\widetilde \bG_{n,i}^* )
%\bB_n^{-1/2}
%\right|^2\right) \right]^{1/2}\\
%&\hspace{10mm}
%\E\left[\Tr\left(
%\left|
%\bB_n^{1/2}
%\bM^*
%\prod_{i=1}^p ( \widetilde \bG_{i}^*\bF^*)
%\prod_{i=p}^1 (\bF \bG_i)
%\bB^{-1/2}
%\right|^2
%\right)
%\right]^{1/2}
%\end{align}
%where the inequality follows by H\"older for $L_p(S^p)$ norms.
%We bound the second expectation as 
%\begin{align}
%&\E\left[\Tr\left(
%\left|
%\bB_n^{1/2}
%\bM^*
%\prod_{i=1}^p ( \widetilde \bG_{i}^*\bF^*)
%\prod_{i=p}^1 (\bF \bG_i)
%\bB^{-1/2}
%\right|^2
%\right)
%\right]\\
%&=
%\E\left[\Tr\left(
%\bB_n^{1/2}
%\bM^*
%\prod_{i=1}^p ( \widetilde \bG_{i}^*\bF^*)
%\prod_{i=p}^1 (\bF \bG_i)
%\bB^{-1}
%\prod_{i=1}^p (\bG_i^*\bF^* )
%\prod_{i=p}^1 (\bF \widetilde \bG_{i})
%\bM
%\bB_n^{1/2}
%\right)
%\right]\\
%&\le
%\E\left[\Tr\left(
%\bB_n^{1/2}
%\bM^*
%\prod_{i=1}^p ( \widetilde \bG_{i}^*\bF^*)
%\bB^{-1}
%\prod_{i=p}^1 (\bF\widetilde \bG_{i})
%\bM
%\bB_n^{1/2}
%\right)
%\right]
%\norm{\E\left[\bB^{1/2} \bF \bG \bB^{-1}\bG^* \bF^*\bB^{1/2}\right]}_\op^p\\
%&\le
%\E\left[\Tr\left(
%\bB_n^{1/2}
%\bM^*
%\widetilde \bG_1^*
%\prod_{i=2}^p (\bF^* \widetilde \bG_{i}^*)
%\bB^{-1}
%\prod_{i=p}^2 (\widetilde \bG_{i}\bF) \widetilde \bG_1
%\bM
%\bB_n^{1/2}
%\right)
%\right]
%\norm{\E\left[\bB^{1/2} \bF \bG \bB^{-1}\bG^* \bF^*\bB^{1/2}\right]}_\op^p
%\norm{\bB^{-1}}_\op
%\norm{\bB^{1/2}\bF^* \bF \bB^{1/2}}_\op
%\\
%&\le 
%\E\left[\Tr\left(
%\bB_n^{1/2}
%\bM^*
%\widetilde \bG_1^*
%\bB^{-1}
%\widetilde\bG_1
%\bM
%\bB_n^{1/2}
%\right)
%\right]
%\norm{\E\left[\bB^{1/2} \bF \bG \bB^{-1}\bG^* \bF^*\bB^{1/2}\right]}_\op^{2p-1}
%\norm{\bB^{-1}}_\op
%\norm{\bB^{1/2}\bF^* \bF \bB^{1/2}}_\op\\
%&\le 
%\E\left[\Tr\left(
%\bM^*
%\bM
%\right)
%\right]
%\norm{\E\left[\bB^{1/2} \bF \bG \bB^{-1}\bG^* \bF^*\bB^{1/2}\right]}_\op^{2p-1}
%\norm{\bB^{-1}}_\op
%\norm{\bB^{1/2}\bF^* \bF \bB^{1/2}}_\op
%\norm{
%\E[ \bG^*
%\bB^{-1}\bG]}_\op
%\norm{\bB_n}_\op.
%\end{align}
%A similar computation gives the bound on the first expectation 
%\begin{align}
%    &\E\left[
%\Tr\left(
%\left|
%\bB^{1/2}
%\bM \prod_{i=1}^p(\bG_{n,i} \bF)
%\prod_{i=p}^1 ( \bF^*\widetilde \bG_{n,i}^* )
%\bB_n^{-1/2}
%\right|^2\right) \right]\\
%&\le
%\E\left[\Tr\left(
%\bM^*
%\bM
%\right)
%\right]
%\norm{\E\left[\bB_n^{1/2} \bF_n \bG_n \bB_n^{-1}\bG_n^* \bF_n^*\bB_n^{1/2}\right]}_\op^{2p-1}
%\norm{\bB_n^{-1}}_\op
%\norm{\bB_n^{1/2}\bF_n^* \bF_n \bB_n^{1/2}}_\op
%\norm{
%\E[ \bG_n^*
%\bB_n^{-1}\bG_n]}_\op
%\norm{\bB}_\op
%\end{align}
%where we used that all matrices involved are symmetric.
%\end{proof}
%
%
%\begin{lemma}
%    We have the bounds
%    \begin{equation}
%\norm{\E\left[\bB^{-1/2} \bF \bG \bB\bG^* \bF^*\bB^{-1/2}\right]}_\op \le 
%1 -\frac{\alpha \Im(z)}{2} \lambda_{\min}(\bB^{-1/2} \bS_\star \bS_\star^* \bB^{-1/2}),
%\quad
%\norm{\E[ \bG^*
%\bB\bG]}_\op \le
%    \norm{\bS_\star^{-1}\bB\bS_\star^{*-1}}_\op.
%    \end{equation}
%and 
%    \begin{equation}
%\frac1\alpha\norm{\E\left[\bB_n^{-1/2} \bF_n \bG_n \bB_n\bG_n^* \bF_n^*\bB_n^{-1/2}\right]}_\op 
%    \le \bI -  \frac{\Im(z)}{\alpha}\lambda_{\min} \left(\bB_n^{-1/2} \bF_n\bF_n^* \bB_n^{-1/2}\right) 
%,\quad
%\norm{\E[ \bG_n^*
%\bB_n\bG_n]}_\op \le \norm{\Im(\bF_n^{-1})}_\op
%    \end{equation}
%with probability
%\begin{equation}
%\dots
%\end{equation}
%for $n> n_0(z)$ where ..
%\end{lemma}
%
%\begin{proof}
%By Lemma~\ref{lemma:re_im_properties}, we have
%$\Im(\bG_\star) = -\bG_\star\bB \bG_\star^*$, and $\Im(\bS_\star^{-1}) = - \bS_\star^{-1} \bB\bS_{\star}^{*-1}.$
%So rewriting the fixed point for $\bS_\star$ as
%    $z\bI = \E[\bG_\star] - \frac1\alpha \bS_\star$ and taking the imaginary parts
%gives
%\begin{equation}
%\bzero\prec \Im(z) \bI = - \E[\bG_\star \bB \bG_\star^*] + \frac1\alpha\bS_\star^{-1}\bB\bS_\star^{*-1}.
%\end{equation}
%This implies
%\begin{equation}
%    \alpha \bB^{-1/2} \bS_\star\E[\bG_\star \bB\bG_\star^*] \bS_\star^* \bB^{-1/2} \prec 
%    \bI - \frac{\alpha\Im(z)}{2} \bB^{-1/2}\bS_\star\bS^*_\star \bB^{-1/2},\quad
%   \alpha \E[\bG_\star\bB\bG_\star^*]  \preceq \bS_\star^{-1}\bB\bS_\star - \Im(z)\bI,
%\end{equation}
%giving the bounds desired.
%%\begin{equation}
%%    \norm{
%%\alpha \bB^{-1/2} \bS_\star\E[\bG_\star \bB\bG_\star^*] \bS_\star^* \bB^{-1/2} }_\op < 1 -\frac{\alpha \Im(z)}{2} \lambda_{\min}(\bB^{-1/2} \bS_\star \bS_\star^* \bB^{-1/2}),
%%\end{equation}
%
%For the remaining two bounds, once again let us write, by definition of $\bF_n$,
%   $z\bI = \E[\bG_n] - \bF_n^{-1},$
%which then gives, after multiplying to the left by $\bF_n$ and to the right by $\bF_n^*$,
%\begin{equation}
%    z \bF_n \bF_n^* =  \E[\bF_n \bG_n \bF_n^*] - \bF_n^*.
%\end{equation}
%Taking the imaginary part using Lemma~\ref{lemma:re_im_properties} then gives
%\begin{equation}
%    \Im(z) \bF_n \bF_n^* = - \E[\bF_n \bG_n\bB_n \bG_n^* \bF_n^*]  + \Im ( \bF_n)
%\end{equation}
%so that
%\begin{equation}
%    \E[\bF_n \bG_n \bB_n \bG_n^* \bF_n^*] = \Im(\bF_n - \alpha\bQ_n)  + \alpha\bB_n - \Im(z) \bF_n\bF_n^*.
%\end{equation}
%Letting $\bE_n := (\alpha^{-1}\bI  -\bF_n^{-1} \bQ_n)$,
%\begin{align}
%    \norm{\Im(  \bF_n^{-1} (\bF_n - \alpha\bQ_n) \bF_{n}^{*-1})}_\op
%    &=
%   \alpha \norm{\Im(   \bE_n \bF_{n}^{*-1})}_\op
%   \le  
%    \alpha\norm{\bE_n}_\op\norm{\bF_n^{-1}}_\op
%    \le \norm{\bE_n}_\op \norm{\bQ_n^{-1}}_\op \norm{\bI - \bE_n}_\op\\
%    &\le \frac2{\Im(z)} \left(\frac{1}{n^2}\norm{\bH}_\op^2  + |z|^2\right) \omega_{\textrm{FP}}(z;n,k) (1 + \omega_{\textrm{FP}}(z; n, k))
%\end{align}
%and taking $n \le n_0(z)$ so that 
%\bns{Fix this}
%\begin{equation}
%     \frac{10(\sfK^2 + |z|)}{\alpha\Im(z)^2}\omega_{\textrm{FP}}(z;n,k)(1 + \omega_{\textrm{FP}}(z;n,k))  \le  1,
%\end{equation}
%we have on the event $\cG_0$ of Lemma~\notate{ref} that
%\begin{equation}
%    \frac1{\alpha}\norm{\E[\bB_n^{-1/2}\bF_n \bG_n \bB_n \bG_n^* \bF_n^* \bB_n^{-1/2}]  }_\op
%    \le \bI -  \frac{\Im(z)}{\alpha}\lambda_{\min} \left(\bB_n^{-1/2} \bF_n\bF_n^* \bB_n^{-1/2}\right),
%\end{equation}
%and similarly,
%\begin{equation}
%\norm{\E[\bG_n \bB_n\bG_n^*]}_\op \le \norm{\Im(\bF_n^{-1})}_\op.
%\end{equation}
%\end{proof}


\subsection{Uniqueness of \texorpdfstring{$\mathbf{S}_\star$}{S*} and convergence of \texorpdfstring{$\mathbf{S}_n$}{Sn} to \texorpdfstring{$\mathbf{S}_\star$}{S*}}
\label{sec:UniquenessSstar}
The goal of this section is to show that the asymptotic Stieltjes transform defined 
in Section \ref{sec:AsymptoticST} is the unique solution of the fixed point equation of Eq.~\eqref{eq:fp_eq}, and to derive a bound on the difference of this quantity and the empirical Stieltjes transform.
%\begin{claim}
%For $z \in \mathbb{H}_+$, there exists a unique solution $\bS_\star$
%\begin{equation}
%   \frac1\alpha_n \bI - \E_\nu[(\bI + \grad^2 \rho \bS)^{-1} \grad^2 \rho \bS] + z \bS = 0,
%\end{equation}
%and
%\begin{equation}
%    \norm{\bS_n - \bS_\star}_{F} \le \dots
%\end{equation}
%with probability $\dots$.
%\end{claim}
Our approach is to study a certain linear operator whose invertibility implies the uniqueness of the solution of~\eqref{eq:fp_eq}.
To define this operator, first introduce the notation
\begin{equation}
        \boldeta(\bS, \bW) := (\bI + \bW\bS)^{-1}\bW.
\end{equation}
%Throughout this section, let 
%\begin{equation}
%\bA_\star := \Re(\bS_\star),\quad\bB := \Im(\bS_\star),
%\quad\bA_n :=\Re(\bS_n),\quad\bB_n := \Im(\bS_n).
%\end{equation}
For a given $\bS\in\C^{k\times k}$ with $\Im(\bS)\succeq \bzero$, $z\in\bbH_+$ $\alpha >1, \nu\in\cuP(\R^{k+k_0+1})$,
define $\bT_\bS(\;\cdot\;; z,\alpha, \nu):\C^{k\times k}\to \C^{k \times k}$ 
\begin{equation}
\label{eq:def_T}
    \bT_\bS(\bDelta;
    z,\alpha,\nu)
    := \bF_z(\bS_\star;\nu) \E[\boldeta(\bS_\star,\bW) \bDelta  \boldeta(\bS,\bW) ] \bF_z(\bS;\nu),
\end{equation}
where
$\bS_\star = \bS_\star(z; \alpha,\nu)$,
%\begin{equation}
%\bS_\star :=\begin{cases}
%     \bS_\star(z; \alpha,\nu) & \Im(z) > 0\\
%     \lim_{\eps \to0}
%         \bS_\star(z + i\eps; \alpha,\nu) & \Im(z) = 0
%    \end{cases}.
%\end{equation}
%whenever
%the right-hand-side is finite (notice that it will be finite for any $z\in\bbH_+$ and $\Im(\bS) \succ\bzero$, which is the setting of this section. We state it more generally for its utility later on in the analysis.)
Now, the significance of this operator is highlighted by the following relation: 
letting $\bS_\star$ be as defined above and suppressing the dependence on $z,\alpha,\nu$,
we have for any $\bS \in \bbH_+^k$
\begin{align*}
   \bF_z(\bS_\star) - \bF_z(\bS) &=
        \left(\E[\bfeta(\bS_\star,\bW)] - z\bI\right)^{-1}
        -
        \left(\E[\bfeta(\bS, \bW)] - z\bI\right)^{-1}\\
        &=
        -\left(\E[\bfeta(\bS_\star,\bW)] - z\bI\right)^{-1}
      \E[\bfeta(\bS_\star, \bW) - 
      \bfeta(\bS,\bW)] 
        \left(\E[\bfeta(\bS, \bW)] - z\bI\right)^{-1}\\
        &=
        \bF_z(\bS_\star) 
      \E[\bfeta(\bS_\star, \bW)(\bS_\star - \bS) 
      \bfeta(\bS,\bW))] 
     \bF_z(\bS) \\
   &=\bT_{\bS}(\bS_\star - \bS).
\end{align*}
Summarizing, we have
\begin{align}
\label{eq:relation_F_T}
   \bF_z(\bS_\star) - \bF_z(\bS) =\bT_{\bS}(\bS_\star - \bS).
\end{align}

To prove uniqueness of $\bS_\star$, we'll consider $\bS_0$ to be any solution of $\alpha_n \bS = \bF_z(\bS)$ in $\bbH_+^k$ and show that $\bT_0 := \bT_{\bS_0}$ has a convergent Neumann series. 
Similarly we will derive the rate of convergence of $\bS_n$ to $\bS_\star$ by bounding the norm of $(\id - \bT_{\bS_n})^{-1}$.

Our first lemma of this section gives a deterministic bound on $\norm{\bT_\bS^p}_{\op\to\op}$ for all positive integer $p$, which will later allow us to assert the convergence of the Neumann series of $\bT_\bS$.
Here, $\bT^p$ is the $p$-fold composition of $\bT$,
namely $\bT^p(\bA) = \bT(\bT^{p-1}(\bA))$.
\begin{lemma}
\label{lemma:op_norm_bound_power_T}
Fix $\bS\in\C^{k\times k}$ with $\Im(\bS)\succeq \bzero,$ $\Im(z) \ge 0,\alpha >1$, and $\nu\in\cuP(\R^{k+k_0+1})$ such that $\bT_\bS(\;\cdot\;, z,\alpha,\nu)$ of Eq.~\eqref{eq:def_T} is defined.
We have for any
$\bB,\bB_\star \succ \bzero$,
and integer $p>0$, we have
\begin{align}
    \norm{\bT_\bS^p}_{\op \to\op}&\le 
    \left(
 \norm{\E[\bB^{-1/2} \bF\bfeta \bB\bfeta^* \bF^* \bB^{-1/2}]}_\op^p
\norm{\bB^{-1}}_\op
\norm{\bB}_\op
\right)^{1/2}
\nonumber
\\
&\hspace{5cm}\left(
 \norm{\E[\bB_\star^{-1/2} \bF_\star\bfeta_\star \bB_\star\bfeta_\star^* \bF_\star^* \bB_\star^{-1/2}]}_\op^{p}
\norm{\bB_\star^{-1}}_\op
\norm{\bB_\star}_\op
\right)^{1/2},
\nonumber
\end{align}
where $\bS_\star$ is as defined in Eq.~\eqref{eq:def_T} and 
\begin{equation}
\bF_\star := \bF_z(\bS_\star;\nu),\quad \bF:=\bF_z(\bS;\nu),\quad \bfeta_\star  :=\bfeta(\bS_\star, \bW),\quad \bfeta := \bfeta(\bS,\bW).
%,\quad\bB := \Im(\bS),\quad\bB_\star:=\Im(\bS_\star).
\end{equation}

\end{lemma}

\begin{proof}
In what follows, 
let $\bW_1,\dots,\bW_p, \widetilde \bW_1,\dots,\widetilde\bW_p$ be i.i.d. copies of $\bW$.
We use the shorthand $\bfeta_{\star,i} \equiv \bfeta(\bS_\star;\bG_i)$ and 
$\widetilde\bfeta_{\star,i} \equiv \bfeta(\bS_\star;\widetilde\bG_i)$. Similarly define $\bfeta_i, \widetilde\bfeta_i$ for $\bS$ replacing $\bS_\star$.
Fix any $\bv,\bu\in \C^{k}$ and $\bDelta \in \C^{k\times k}$ and write
\begin{align*}
\left|\bv^* \bT^p(\bDelta) \bu\right|^2
&=  \bu^* \bT^p(\bDelta)^* \bv \bv^*  \bT^p(\bDelta) \bu\\
&=  \Tr\left(
\bu^*\E\left[
\prod_{i=p}^1 (\bF \bfeta_i)\bDelta \prod_{i=1}^p(\bfeta_{\star,i} \bF_\star)
\bv\bv^*
\prod_{i=p}^1 ( \bF_\star^*\widetilde \bfeta_{\star,i}^*)
\bDelta^*
\prod_{i=1}^p ( \widetilde \bfeta_{i}^*\bF^*)
\right]\bu
\right)
\\
&= 
\E\left[
\Tr\left(
\bDelta \prod_{i=1}^p(\bfeta_{\star,i} \bF_\star)
\bv\bv^*
\prod_{i=p}^1 ( \bF_\star^*\widetilde \bfeta_{\star,i}^* )
\bDelta^*
\prod_{i=1}^p ( \widetilde \bfeta_{i}^*\bF^*)
\bu\bu^*
\prod_{i=p}^1 (\bF \bfeta_i)
\right)
\right]
\\
&\le 
\E\left[
\Tr\left(
\left|
\bDelta \prod_{i=1}^p(\bfeta_{\star,i} \bF_\star)
\bv\bv^*
\prod_{i=p}^1 ( \bF_\star^*\widetilde \bfeta_{\star,i}^* )
\right|^2\right) \right]^{1/2}\\
&\hspace{10mm}
\E\left[\Tr\left(
\left|
\bDelta^*
\prod_{i=1}^p ( \widetilde \bfeta_{i}^*\bF^*)\bu\bu^*
\prod_{i=p}^1 (\bF \bfeta_i)
\right|^2
\right)
\right]^{1/2}
\end{align*}
where the inequality follows by Cauchy-Schwarz for (random) 
matrices.
We bound the second expectation as 
\begin{align*}
&\E\left[\Tr\left(
\left|
\bDelta^*
\prod_{i=1}^p ( \widetilde \bfeta_{i}^*\bF^*)
\bu\bu^*
\prod_{i=p}^1 (\bF \bfeta_i)
\right|^2
\right)
\right]\\
&=
\E\left[\Tr\left(
\bDelta^*
\prod_{i=1}^p ( \widetilde \bfeta_{i}^*\bF^*)
\bu\bu^*
\prod_{i=p}^1 (\bF \bfeta_i)
\prod_{i=1}^p (\bfeta_i^*\bF^* )
\bu\bu^*
\prod_{i=p}^1 (\bF \widetilde \bfeta_{i})
\bDelta
\right)
\right]\\
&=
\E\left[
\bu^*
\prod_{i=p}^1 (\bF \widetilde \bfeta_{i})
\bDelta
\bDelta^*
\prod_{i=1}^p ( \widetilde \bfeta_{i}^*\bF^*)
\bu
\right]
\E\left[
\bu^*
\prod_{i=p}^1 (\bF \bfeta_i)
\prod_{i=1}^p (\bfeta_i^*\bF^* )
\bu
\right]
\\
&\le
\E\left[
\bu^*
\prod_{i=p}^1 (\bF \widetilde \bfeta_{i})
\bDelta
\bDelta^*
\prod_{i=1}^p ( \widetilde \bfeta_{i}^*\bF^*)
\bu
\right]
\norm{\bB^{-1}}_\op \norm{\E[\bB^{-1/2} \bF\bfeta \bB\bfeta^* \bF^* \bB^{-1/2}]}_\op^p
\norm{\bB}_\op \norm{\bu}_2^2\\
&\le
\norm{\bDelta}_\op^2
 \norm{\E[\bB^{-1/2} \bF\bfeta \bB\bfeta^* \bF^* \bB^{-1/2}]}_\op^{2p}
\norm{\bB^{-1}}_\op^2
\norm{\bB}_\op^2 \norm{\bu}_2^4,
\end{align*}
where the last two inequalities can be proven by induction over $p$.
A similar computation gives the bound on the first expectation 
\begin{align*}
    \E\left[
\Tr\left(
\left|
\bDelta \prod_{i=1}^p(\bfeta_{\star,i} \bF_\star)
\bv\bv^*
\prod_{i=p}^1 ( \bF_\star^*\widetilde \bfeta_{\star,i}^* )
\right|^2\right) \right]
\le
\norm{\bDelta}_\op^2
 \norm{\E[\bB_\star^{-1/2} \bF_\star\bfeta_\star \bB_\star\bfeta_\star^* \bF_\star^* \bB_\star^{-1/2}]}_\op^{2p}
\norm{\bB_\star^{-1}}_\op^2
\norm{\bB_\star}_\op^2 \norm{\bv}_2^4.
\end{align*}
Taking supremum over $\bv,\bu$ of unit norm gives that
\begin{align*}
    \frac{\norm{\bT^p(\bDelta)}_\op}{\norm{\bDelta}_\op} &\le
    \Bigg(
 \norm{\E[\bB_\star^{-1/2} \bF_\star\bfeta_\star \bB_\star\bfeta_\star^* \bF_\star^* \bB_\star^{-1/2}]}_\op^p
\norm{\bB_\star^{-1}}_\op
\norm{\bB_\star}_\op\\
&\hspace{60mm}\cdots\norm{\E[\bB^{-1/2} \bF\bfeta \bB\bfeta^* \bF^* \bB^{-1/2}]}_\op^{p}
\norm{\bB^{-1}}_\op
\norm{\bB}_\op
\Bigg)^{1/2}
\end{align*}
for all $\bDelta$. Taking supremum over $\norm{\bDelta}_\op = 1$ gives the result.
\end{proof}

\begin{lemma}
\label{lemma:op_norm_bound_for_sols_fp}
Fix $z \in\bbH_+$, $\hnu \in \cuP_n(\R^{k+k_0+1})$.
Let $\bS_0$ be any solution of $\alpha_n \bS = \bF_z(\bS;\hnu)$ in    $\bbH_+^k$, and let $\bS_n(z;\hnu)$ be the quantity defined in Eq.~\eqref{eq:Sn_def}.
Use the notation
\begin{equation}
\bF_0 := \bF_z(\bS_0;\hnu),\quad \bF_n:=\bF_z(\bS_n;\hnu),\quad \bfeta_0  :=\bfeta(\bS_0, \bW),\quad \bfeta_n := \bfeta(\bS_n,\bW),\quad\bB_n := \Im(\bS_n),\quad\bB_0:=\Im(\bS_0),
\end{equation}
where $\bW\sim \grad^2\ell_{\# \hnu}$.
Then we have the bound
    \begin{equation}
    \nonumber
\frac1\alpha_n\norm{\E\left[\bB_0^{-1/2} \bF_0 \bfeta_0 \bB_0\bfeta_0^* \bF_0^*\bB_0^{-1/2}\right]}_\op \le 
1 -\frac{\alpha_n \Im(z)}{2} \lambda_{\min}(\bB_0^{-1/2} \bS_0 \bS_0^* \bB_0^{-1/2}).
    \end{equation}
    
Further, if 
\begin{equation}
\label{eq:n_n(z)}
     \frac{10(\sfK^2 + |z|^2)}{\Im(z)^2}\Err_{\FP}(z;n,k)(1 + \alpha_n\Err_{\FP}(z;n,k))  \le  \frac12
 \end{equation}
then the following holds,
for any $L\ge 1$,
on the event $\Omega_0\cap\Omega_1(L)$  of 
Lemmas \ref{lemma:standard_norm_bounds},
\ref{lemma:concentration_loo_quad_form}
    \begin{equation}
\frac1\alpha_n\norm{\E\left[\bB_n^{-1/2} \bF_n \bfeta_n \bB_n\bfeta_n^* \bF_n^*\bB_n^{-1/2}\right]}_\op 
    \le 1 -  \frac{\Im(z)}{2\alpha_n}\lambda_{\min} \left(\bB_n^{-1/2} \bF_n\bF_n^* \bB_n^{-1/2}\right)\, .
    \label{eq:SecondBoundFp}
    \end{equation}
    \end{lemma}

\begin{proof}
By Lemma~\ref{lemma:re_im_properties}, we have
$\Im(\bfeta_0) = -\bfeta_0\bB_0 \bfeta_0^*$, and $\Im(\bS_0^{-1}) = - \bS_0^{-1} \bB_0\bS_{0}^{*-1}.$
So rewriting the fixed point for $\bS_0$ as
    $z\bI = \E[\bfeta_0] - \alpha_n^{-1} \bS_0^{-1}$ and taking the imaginary parts
gives
\begin{equation}
\nonumber
\bzero\prec \Im(z) \bI = - \E[\bfeta_0 \bB_0 \bfeta_0^*] + \frac1\alpha_n\bS_0^{-1}\bB_0\bS_0^{*-1}.
\end{equation}
This implies
\begin{equation}
\nonumber
    \alpha_n \bB_0^{-1/2} \bS_0\E[\bfeta_0 \bB_0\bfeta_0^*] \bS_0^* \bB^{-1/2} \prec 
    \bI - \frac{\alpha_n\Im(z)}{2} \bB_0^{-1/2}\bS_0\bS^*_0 \bB_0^{-1/2}
\end{equation}
giving the first bound after substituiting $\bF_0 = \alpha_n 
\bS_0$.
%\begin{equation}
%    \norm{
%\alpha \bB^{-1/2} \bS_\star\E[\bG_\star \bB\bG_\star^*] \bS_\star^* \bB^{-1/2} }_\op < 1 -\frac{\alpha \Im(z)}{2} \lambda_{\min}(\bB^{-1/2} \bS_\star \bS_\star^* \bB^{-1/2}),
%\end{equation}

For the bound \eqref{eq:SecondBoundFp}, once again let us write by definition of $\bF_n$,
   $z\bI = \E[\bfeta_n] - \bF_n^{-1}$
which gives after multiplying to the left by $\bF_n$ and to the right by $\bF_n^*$
\begin{equation}
\nonumber
    z \bF_n \bF_n^* =  \E[\bF_n \bfeta_n \bF_n^*] - \bF_n^*.
\end{equation}
Taking the imaginary part using Lemma~\ref{lemma:re_im_properties} then gives
\begin{equation}
\nonumber
    \Im(z) \bF_n \bF_n^* = - \E[\bF_n \bfeta_n\bB_n \bfeta_n^* \bF_n^*]  + \Im ( \bF_n)
\end{equation}
so that
\begin{equation}\label{eq:Fn-relation}
    \E[\bF_n \bfeta_n \bB_n \bfeta_n^* \bF_n^*] = \Im(\bF_n - \alpha_n\bS_n)  + \alpha_n\bB_n - \Im(z) \bF_n\bF_n^*.
\end{equation}
Letting $\bE_n := (\alpha_n^{-1}\bI  -\bF_n^{-1} \bS_n)$,
\begin{align*}
    \norm{\Im(  \bF_n^{-1} (\bF_n - \alpha_n\bS_n) \bF_{n}^{*-1})}_\op
    &=
   \alpha_n \norm{\Im(   \bE_n \bF_{n}^{*-1})}_\op
   \le  
    \alpha_n\norm{\bE_n}_\op\norm{\bF_n^{-1}}_\op\\
    &
    \le \norm{\bE_n}_\op \norm{\bS_n^{-1}}_\op \norm{\bI - \alpha_n\bE_n}_\op\\
    &\stackrel{(a)}\le \frac1{\Im(z)} \left(\frac{1}{n^2}\norm{\bH}_\op^2  + |z|^2\right) \Err_{\FP}(z;n,k) (1 + \alpha_n\Err_{\FP}(z; n, k))\\
    &\stackrel{(b)}\le \frac{10}{\Im(z)} \left(\sfK^2  + |z|^2\right) \Err_{\FP}(z;n,k) (1 + \alpha_n\Err_{\FP}(z; n, k))\\
    &\stackrel{(c)}\le \frac{\Im(z)}{2}
\end{align*}
on $\Omega_0 \cap\Omega_1(L)$,
where $(a)$ follows from Lemma~\ref{lemma:fix_point_rate}
and Lemma \ref{lemma:as_norm_bounds}, $(b)$ follows from Lemma~\ref{lemma:standard_norm_bounds}, and $(c)$ follows from the assumption in Eq.~\eqref{eq:n_n(z)}.
We conclude that
\begin{equation}
\nonumber
    \Im(\bF_n - \alpha_n\bS_n) - \frac{\Im(z)}{2}\bF_n\bF_n^* \preceq \bzero\, ,
\end{equation}
and therefore, using Eq.~\eqref{eq:Fn-relation},
and the fact that $\Im(z)>0$,
\begin{equation}
\nonumber
    \frac1{\alpha_n}\norm{\E[\bB_n^{-1/2}\bF_n \bfeta_n \bB_n \bfeta_n^* \bF_n^* \bB_n^{-1/2}]  }_\op
    \le 1 -  \frac{\Im(z)}{2\alpha_n}\sigma_{\min} \left(\bB_n^{-1/2} \bF_n\bF_n^* \bB_n^{-1/2}\right)
\end{equation}
as desired.
\end{proof}


\subsubsection{Uniqueness of \texorpdfstring{$\mathbf{S_\star}$}{S*}}
\label{app:sec:UniquenessSstar}
We are now ready to prove uniqueness of the solution $\bS_\star$.
\begin{lemma}
\label{lemma:uniqueness_ST}
    For any $z\in\bbH_+$, $\alpha_0 >1$, $\nu\in\cuP(\R^{k+k_0+1})$, the solution $\bS_\star(z;\alpha_0,\nu)$ of Eq.~\eqref{eq:def_S_star} is the unique solution to $\alpha_0 \bS = \bF_z(\bS;\nu)$ on $\bbH_+^k$.
\end{lemma}
\begin{proof}


Let $\bS_0 \in\bbH_+^k$ be any solution to this fixed point equation.  Then by Eq.~\eqref{eq:relation_F_T}
    \begin{equation}
\label{eq:diff_two_sols}
0 = \bS_0 - \bS_\star - \frac1{\alpha_0} (\bF_z(\bS_0) -\bF_z(\bS_\star)) = \left(\id - \frac1{\alpha_0} \bT_{\bS_0}\right)\left(\bS_0 -\bS_\star\right).
    \end{equation}
So to conclude uniqueness, it's sufficient to show that $\left(\id - \alpha_0^{-1} \bT_{\bS_0}\right)$ is invertible.
Using Lemmas~\ref{lemma:op_norm_bound_power_T} and~\ref{lemma:op_norm_bound_for_sols_fp}, let
\begin{equation}
\nonumber
    \delta_0 := \frac{\alpha_0\Im(z)}{2} \lambda_{\min}(\bB_0^{-1/2}\bS_0 \bS_0^* \bB_0^{-1/2}),
    \quad
    \delta_\star := \frac{\alpha_0\Im(z)}{2} \lambda_{\min}(\bB_\star^{-1/2}\bS_\star \bS_\star^* \bB_\star^{-1/2}).
\end{equation}
Since $\bS_0,\bS_\star \in\bbH_+^k$, we have
by Lemma~\ref{lemma:re_im_properties} that $\delta_0,\delta_\star > 0$.
Hence by Lemmas~\ref{lemma:op_norm_bound_power_T} and~\ref{lemma:op_norm_bound_for_sols_fp} 
\begin{align*}
    \norm{\sum_{p}\alpha_0^{-p} \bT_{\bS_0}^p}_{\op\to\op}  &\le \sum_{p} (1-\delta_0)^{p/2}(1-\delta_\star)^{p/2} \left(\norm{\bB_0}_\op \norm{\bB_0^{-1}}_\op  \norm{\bB_\star}_\op \norm{\bB_\star^{-1}}_\op\right)^{1/2} \\&\le 
    \left( \frac1{\delta_0 \delta_\star}\norm{\bB_0}_\op \norm{\bB_0^{-1}}_\op  \norm{\bB_\star}_\op \norm{\bB_\star^{-1}}_\op\right)^{1/2}  
\end{align*}
%The latter quantity is bounded by Lemma~\ref{lemma:re_im_properties} and since $\bS_0,\bS_\star \in\bbH_+^k$. 
implying convergence of the Neumann series, and in turn, the desired invertibility.
\end{proof}

\subsubsection{Rate of convergence}

\begin{lemma}
\label{lemma:rate_matrix_ST}
Whenever
\begin{equation}
\label{eq:n_n_0_2}
     \frac{10(\sfK^2 + |z|^2)}{\Im(z)^2}\Err_{\FP}(z;n,k)(1 + \alpha_n\Err_{\FP}(z;n,k))  \le  \frac1{2\alpha_n},
 \end{equation}
we have 
\begin{equation}
\nonumber
    \sup_{\hnu \in\cuP_n(\R^{k+k_0+1})}\norm{\bS_\star(z;\alpha_n, \hnu) -\bS_n(z;\hnu)}_\op \le C(\sfK)
    %(\alpha_n+\alpha_n^{-1})^2
    \frac{1 + |z|^4}{\Im(z)^5} 
\Err_{\FP}(z; n, k)
\end{equation}
on the event $\Omega_0 \cap\Omega_1(L)$ of Lemmas \ref{lemma:standard_norm_bounds},
\ref{lemma:concentration_loo_quad_form}, for $L\ge 1$.
\end{lemma}
\begin{proof}
Recalling (for $\bF_{\star} = \bF_z(\bS_{\star})$,
$\bF_{n} = \bF_z(\bS_{n})$) that $\alpha_n^{-1}\bF_\star = \bS_\star$ and 
$\alpha_n^{-1}\bF_n = \bS_n +\bF_n\bE_n$ where $\bE_n := \alpha_n^{-1}\bI - \bF_n^{-1}\bS_n$,
we have by Eq.~\eqref{eq:relation_F_T},
\begin{equation}
    \bS_\star - \bS_n =
    \alpha_n^{-1}(\bF_\star - \bF_n)  + \bF_n\bE_n =
    \frac1\alpha_n \bT_{\bS_n}(\bS_\star -\bS_n) +  \bF_n\bE_n.
\end{equation}
Letting
\begin{equation}
\nonumber
    \delta_\star := \frac{\alpha_n\Im(z)}{2} \lambda_{\min}(\bB_\star^{-1/2} \bS_\star\bS_\star^* \bB_\star^{-1/2}),\quad
    \delta_n := \frac{\Im(z)}{2\alpha_n} \lambda_{\min}(\bB_n^{-1/2} \bF_n\bF_n^* \bB_n^{-1/2}),
\end{equation}
an argument similar to that of Lemma~\ref{lemma:uniqueness_ST} (making use of Lemmas~\ref{lemma:op_norm_bound_power_T} and~\ref{lemma:op_norm_bound_for_sols_fp}) 
implies that $(\id- \alpha_n^{-1}\bT_{\bS_n})$ is invertible and so
\begin{align}
\nonumber
    \norm{\bS_* - \bS_n}_\op 
    &= \norm{\left(\bI - \alpha_n^{-1}\bT_{\bS_n}\right)^{-1} \bF_n \bE_n}\\
\nonumber
    %&\le\norm{\sum_{p=0}^\infty \frac1{\alpha^p}\bT^p  }_\op \norm{\bF_n}_\op\norm{\bE_n}_\op\\
    &\le \sum_{p=0}^\infty \alpha_n^{-p}\norm{\bT_{\bS_n}^p}_{\op\to\op} \norm{\bF_n\bE_n}_\op\\
\nonumber
    &\le \sum_{p=0}^\infty (1-\delta)^{p/2}(1-\delta_n)^{p/2} 
    \left(\norm{\bB_n^{-1}}_\op \norm{\bB_n}_\op 
    \norm{\bB}_\op \norm{\bB^{-1}}_\op\right)^{1/2} \norm{\bF_n}_\op\norm{\bE_n}_\op \\
    &\le \left(
    \frac{1}{\delta}
    \frac{1}{\delta_n}
    \norm{\bB_n^{-1}}_\op \norm{\bB_n}_\op 
    \norm{\bB}_\op \norm{\bB^{-1}}_\op
    \right)^{1/2}
    \norm{\bF_n}_\op \norm{\bE_n}_\op.
    \label{eq:S_Sn_rate_expansion}
\end{align}

Now we collect the bounds appearing on the right-hand side of this equation. On the event $\Omega_0$ of Lemma~\ref{lemma:standard_norm_bounds}, we have Lemma~\ref{lemma:as_norm_bounds} and  Corollary~\ref{cor:S_star_min_singular_value_bound}
(recall that $\bB_n=\Im(\bS_n)$, $\bB_{\star}=\Im(\bS_{\star})$): 
\begin{equation}
\nonumber
    \norm{\bB_n^{-1}}_\op \le  \frac{C_1}{\Im(z)} \left( \sfK^2 + |z|^2 \right),\quad\textrm{and}\quad
    \norm{\bB_\star^{-1}}_\op \le \frac{C_2}{\Im(z)}\left(\sfK^2 + |z|^2\right),
\end{equation}
respectively.
Furthermore, by Lemma~\ref{lemma:re_im_properties}, then Lemma~\ref{lemma:as_norm_bounds} and Corollary~\ref{cor:S_star_min_singular_value_bound} respectively, we have the bounds
\begin{equation}
\nonumber
   \norm{\bB_n}_\op \le \frac1{\Im(z)},\quad  \norm{\bB_\star}_\op \le\frac1{\Im(z)}.
\end{equation}
Meanwhile, to bound the norm of $\bF_n$, we  can observe that 
    $\bF_n = \alpha_n (\bF_n \bE_n + \bS_n)$.
Since the assumption guarantees that
\begin{equation}
\label{eq:bound_En}
\norm{\bE_n}_\op \equiv \Err_{\FP} \le \frac1{2\alpha_n}  \frac1{1+\alpha_n \Err_{\FP}} \frac{|z|^2}{10(\sfK^2 + |z|^2)} \le \frac1{2\alpha_n},
\end{equation}
we conclude that 
\begin{equation}
\nonumber
    \norm{\bF_n}_\op \le 2\alpha_n \norm{\bS_n}_\op \stackrel{(a)}{\le} \frac{2\alpha_n}{\alpha_n \Im(z)} = \frac{2}{\Im(z)},
\end{equation}
where $(a)$ follows from Lemma~\ref{lemma:as_norm_bounds}.
Further, we have
\begin{align}
\label{eq:lb_delta_star}
\delta_{\star} &= \frac{\alpha_n \Im(z)}{2} \lambda_{\min}\left(  
(\bB_\star^{-1/2}\bA_\star\bB_\star^{-1/2} + i \bI)\bB_\star (\bB_\star^{-1/2}\bA_\star\bB_\star^{-1/2} - i\bI)
\right) \\
&\stackrel{(a)}{\ge} \frac{\alpha_n\Im(z)}{2} \lambda_{\min}\left(\bB_\star\right) \ge 
 C_3\frac{\alpha_n\Im(z)^2}{\sfK^2 + |z|^2} 
 \nonumber
\end{align}
where $(a)$ holds since the spectrum of $\bB_{\star}^{-1/2}\bA_{\star}\bB_{\star}^{-1/2}$ is real.
Finally, we lower bound $\delta_n$ by writing
%\delta_{n} \ge \frac{\Im(z)}{2 \alpha} \sigma_{\min}(\bB_n^{-1}) \sigma_{\min}(\bF_n)^2 \ge C  
%\frac{\Im(z)}{\alpha(1 + \alpha \omega_n)} \sigma_{\min}(\bB_n^{-1}) \sigma_{\min}(\bB_n)^2
%\ge 
%C  
%\frac{\Im(z)^4}{\alpha(1 + \alpha \omega_n)} \sigma_{\min}(\bB_n)^2
\begin{align*}
    \delta_n  &= \frac{\Im(z)}{2\alpha_n} \lambda_{\min}(\bB_n^{-1/2} \bS_n \bS_n^{-1}\bF_n \bF_n^*\bS_n^{*-1}\bS_n^*\bB_n^{-1/2})
    \ge \frac{\Im(z)}{2\alpha_n} \sigma_{\min}(\bS_n^{-1} \bF_n) \lambda_{\min}\left( \bB_n^{-1/2} \bS_n \bS_n^* \bB_n^{-1/2}\right).
\end{align*}
Noting that
as a consequence of Eq.~\eqref{eq:bound_En} we have
    $\norm{\bS_n \bF_n^{-1}} = \norm{\alpha_n^{-1}+\bE_n}_\op \le  3/(2\alpha_n)$ gives us the lower bound on $\sigma_{\min}(\bS_n^{-1}\bF_n) \ge 2\alpha_n/3$.
This along with the decomposition of Eq.~\eqref{eq:lb_delta_star} applied to the display above gives 
\begin{equation}
\nonumber
    \delta_n \ge \frac{\Im(z)}{3}\lambda_{\min}(\bB_n^{-1/2} \bS_n\bS_n^* \bB_n^{-1/2}) \ge \frac{\Im(z)}{3} \lambda_{\min}(\bB_n) \ge
C_4\frac{\Im(z)^2}{\sfK^2 + |z|^2}.
\end{equation}
Using these bounds in Eq.~\eqref{eq:S_Sn_rate_expansion} above gives the claim. 
\end{proof}


\subsection{Uniform convergence under test functions: proof of Proposition~\ref{prop:uniform_convergence_lipschitz_test_functions}
}
%Given a function $f :\R\to\R$ and $B\ge 0 $, define the restricted Lipschitz norm
%\begin{equation}
%\norm{f}_{\Lip(B)} := \norm{f\cdot\one_{[-B, B]} + f(B)\one_{(B,\infty)} + f(-B)\one_{(-\infty,-B)}}_\Lip.
%\end{equation}
First, note that Eq.~\eqref{eq:ST_convergence_in_P_seq_measures} of Proposition~\ref{prop:uniform_convergence_lipschitz_test_functions}
can be deduced directly from Lemma~\ref{lemma:rate_matrix_ST}.

The following lemma allows us to deduce convergence of the expectation of a bounded Lipschitz function from convergence of the Stieltjes transform. This result, and its proof are fairly standard. The proof is included in Section~\ref{sec:proof_lemma_f_bound_st} for the sake of completion.
\begin{lemma}
\label{lemma:f_bound_st}
Let $f:\R\to\R$ be continuous. Let $\mu_1,\mu_2$ be two probability measures on $\R$ with support in $[-A,A]$, let $s_1,s_2$ denote their Stieltjes transforms, respectively.
Then for any $\gamma \in (0,1)$, we have
    \begin{align*}
         \bigg|\int f(x_0)\de\mu_1(x_0) - \int f(x_0)\de\mu_2(x_0)\bigg| &\le
        \frac1\pi \norm{f}_{\infty,A} \int_{-2A}^{2A} \Big|s_1(x+i\gamma)-s_2(x+i\gamma)\Big|\de x\\
&+
\gamma\left(
2\norm{f}_{\Lip,A} \log(16 A^2+ 1)
        + \frac{2 \norm{f}_{\infty,A}}{A}\right),
     \end{align*}
where $\norm{f}_{\Lip,A}$ and $\norm{f}_{\infty,A}$ are the Lipschitz constant and $\ell_\infty$ norm, respectively, of the function 
$$x \mapsto f(-A)\one_{\{x < -A\}} + f(x)\one_{\{x\in[-A,A]\}} +  f(A)\one_{\{x > A\}}.$$
\end{lemma}
We'll apply this lemma to our setting. 
For $z\in\bbH_+$, $\hnu\in\cuP(\R^{k+k_0+1)}$, let
\begin{equation}
    s_n(z;\hnu) := \frac1k \Tr\Big( \bS_n(z; \hnu)\Big),\quad
    s_\star(z ;\hnu, \alpha_n) := \frac1k \Tr\Big( \bS_\star(z; \hnu, \alpha_n\Big),
\end{equation}
%
where we recall that $\bS_n$ is 
defined in Eq.~\eqref{eq:Sn_def} and
$\bS_{\star}$ is 
defined by Eq.~\eqref{eq:def_S_star}. 
Recall the definition of $\mu_{\MP} := \mu_{\MP}(\hnu,\alpha_n)$ whose Stieltjes transform is $s_\star$, and let $\mu_n = \mu_n(\widehat \nu_{\bV,\bU,w}, \alpha_n)$ be the ESD of $\bH/n$. Note that by definition, $s_n$ is the Stieltjes transform of $\mu_n$.
%\begin{lemma}
%\label{lemma:supp_bound_mu_star}
%There exist a constant $A_0(\alpha_n,\sfK) > 0$ depending only on $\alpha_n$ and $\sfK$ such that
%\begin{equation}
%    \supp\left(\mu_\star(\nu, \alpha_n)\right) \subseteq [-A_0,A_0].
%\end{equation}
%\end{lemma}
%\begin{proof}
%\end{proof}
Since $\hmu_{\sqrt{d}\bTheta}$ is supported on $[-\sfA_\bR,\sfA_\bR]$ for $\bTheta$ in the range of interest, and $\rho_0''$ is continuous and hence bounded on this range,
to deduce the first statement of Proposition~\ref{prop:uniform_convergence_lipschitz_test_functions}, 
it's sufficient to prove the following lemma.
\begin{lemma}
\label{lemma:LP_bound}
For any Lipschitz function $f:\R\to\R$, we have
\begin{equation}
\nonumber
    \limsup_{n\to\infty} \sup_{\hnu\in\cuP_n(\bR^{k+k_0+1})} 
    \left|
    \frac1{dk} \E\left[\Tr\;f \left(\frac1n \bH(\hnu)\right)\right] - \int f(\lambda) \mu_{\MP}(\hnu,\alpha_n)(\de\lambda)\right| = 0.
\end{equation}
\end{lemma}
\begin{proof}
We apply Lemma~\ref{lemma:f_bound_st}.
By definition of $\mu_\star$,  for $\alpha_n >1$, there exists a constant $A_0(\sfK) > 0$ such that
\begin{equation}
\nonumber
    \supp\left(\mu_\star(\nu, \alpha_n)\right) \subseteq [-A_0(\sfK),A_0(\sfK)].
\end{equation}
Furthermore, on the event $\Omega_0$ of Lemma~\ref{lemma:standard_norm_bounds}, we have the bound
(for a similarly bounded constant $A_1$)
\begin{equation}
\nonumber
    \frac1n\norm{\bH}_\op 
    %\le C \sfK  \left( 1 + \alpha_n^{-1}\right) 
    \le A_1(\sfK).
    \end{equation}
%for some universal $C>0.$
%
Taking  $A := A_1(\sfK) \vee A_0(\sfK)$,
we have by Lemma~\ref{lemma:f_bound_st} that 
for any $\gamma \in (0,1)$,
denoting
\begin{equation}
\nonumber
    \hat I_n(\hnu) := 
    \frac1{dk} \Tr\;f \left(\frac1n \bH(\hnu)\right), \quad
    I_\star(\hnu) :=  \int f(\lambda) \mu_{\MP}(\hnu,\alpha_n)(\de\lambda),
\end{equation}
\begin{align*}
\left|
    \hat I_n(\hnu)-  I_\star(\hnu)\right|
&\le C_4(\sfK)
\bigg(
\norm{f}_{\infty,A(\sfK)}
 \sup_{x \in [-2A,2A]}\left|s_n(x + i \gamma;\hnu) - s_\star(x+ i\gamma;\hnu,\alpha_n)\right|+\gamma \left(\norm{f}_{\Lip}  + \norm{f}_{\infty,A(\sfK)}  \right)  \bigg).
\end{align*}
Meanwhile, on $\Omega_0\cap\Omega_1(1)$ (choosing $L=1$ in the definition of $\Omega_1$), we have by Lemma~\ref{lemma:rate_matrix_ST},
\begin{align*}
  \sup_{x\in[-2A,2A]} |s_n(x + i \gamma) - s_\star(x+i\gamma)| &\le 
  \sup_{x\in [-A,A]}\norm{\bS_n(x+i \gamma) - \bS_\star(x + i\gamma)}_\op\\
  &\le
   C_0(\sfK) \frac{1 + |A|^4+  |\gamma|^4}{\gamma^5} 
   \Err_{\FP}( 2A + i\gamma;n,k)
\end{align*}
whenever Eq.~\eqref{eq:n_n_0_2} is satisfied. 
So choosing $\gamma := \gamma_n \to 0$ slow enough so that Eq.~\eqref{eq:n_n_0_2} is satisfied uniformly for all $z$ with $\Im(z) \in [-2A,A]$, and $\Err_{\FP}(2A + i\gamma_n; n, k) \to 0$   as $n\to\infty$
shows that 
\begin{equation}
\label{eq:hatI_diff_Istar}
\lim_{n\to\infty}\sup_{\hnu\in\cuP_n(\R^{k+k_0+1})}\left|
    \hat I_n(\hnu)-  I_\star(\hnu)\right| = 0
\end{equation}
on $\Omega_0 \cap \Omega_1(1)$. 
So
\begin{align*}
   \left|\E[\hat I_n(\hnu)] -  I_\star(\hnu)\right| \le 
 \E\left[\left|\hat I_n(\hnu) -  I_\star(\hnu)\right| \one_{\Omega_0 \cap \Omega_1(1)}\right] 
 +
 2\|f\|_{\infty,A}  \left(\P\left(\Omega_0^c \right) + \P\left(
 \Omega_1^c\right)\right).
\end{align*}
Taking supremum over $\nu$ then sending $n\to\infty$ and using 
Lemmas~\ref{lemma:standard_norm_bounds} and~\ref{lemma:concentration_loo_quad_form} to bound the probability along with \eqref{eq:hatI_diff_Istar} gives the result.
\end{proof}



%\begin{lemma}
%\label{lemma:LP_bound}
%Let $\mu_\star = \mu_\star(\widehat\nu_{\bV,\bU,\bw},\alpha_n)$
%and 
%$\mu_n = \mu_n(\widehat\nu_{\bV,\bU,\bw},\alpha_n)$.
%   For a Lipschitz function $f:\R\to\R$, 
% define 
% \begin{equation}
% \omega_{\textrm{BL}}(n,d,k, f;\gamma) := 
%C(\sfK)
%\norm{f}_{\infty,A(\sfK)}
%\frac{k}{\gamma^9}\left(L\sqrt{\frac{k_+(d)}{n}} + \frac{1}{n \gamma}\right)
%+ 
% k\gamma( \norm{f}_{\Lip} + \norm{f}_{\infty,A(\sfK)} )
% \end{equation}
% where $A(\sfK)>0$ is a constant depending only on $\sfK$.
%For any $\gamma\in(0,1)$, if
%$\omega_{\textrm{BL}}(n,d,k; f,\gamma) < \alpha_n^{-1}$, 
%  we have on the event $\Omega_0\cap\Omega_1(L)$ defined in
%   Lemmas \ref{lemma:standard_norm_bounds},
%\ref{lemma:concentration_loo_quad_form},
%   \begin{equation}
% k \left|
%    \int f(\lambda) \de \mu_n(\lambda) - \int f(\lambda) \de \mu_\star(\lambda)
%\right|
%\le 
%\omega_{\textrm{BL}}(n,d,k; f,\gamma).
%   \end{equation}
%Consequently, 
%there exists universal constant $c>0$ such that if $d > c$, we have for any $\eps > 0$,
%\begin{align}
% k \left|
% \E\left[\int \log(\lambda \vee \eps) \mu_n(\de \lambda)\right]- \int \log(\lambda \vee \eps) \mu_\star(\de\lambda)
%\right| \le  
%\omega_{\LP}(n,d,k; \gamma, \eps)
%\end{align}
%as long as $\omega_{\LP} < \alpha_n^{-1}$, 
%where
%\begin{equation}
%\omega_{\LP}(n,d,k; \eps):=
%\inf_{\gamma\in(0,1)}
%C(\sfK) \left(\frac{k}{\gamma^9} \left(L \sqrt{\frac{k_+(d)}{n}} + \frac1{n\gamma}\right) + \frac{k\gamma }{\eps}  \right) +  \frac{k}{d}.
%\end{equation}
%\end{lemma}
%\begin{proof}
%On $\Omega_0\cap\Omega_1(L)$, we have by Lemma~\ref{lemma:rate_matrix_ST}, for $z \in\bbH_+$,
%\begin{equation}
%   |s_n(z) - s_\star(z)| \le \norm{\bS_n(z) - \bS_\star(z)}_\op \le
%   C_0(\sfK) \frac{\left(1 + |z|\right)^4}{\Im(z)^5} 
%   \omega_{\textrm{FP}}(z,n,k,\alpha_n)
%\end{equation}
%whenever 
%\begin{equation}
%     \frac{10(\sfK^2 + |z|^2)}{\Im(z)^2}
%     \omega_{\textrm{FP}}(z,n,k,\alpha_n)(1 + \alpha_n\omega_{\textrm{FP}}(z,n,k,\alpha_n))  \le  \frac1{2\alpha_n}.
%\end{equation}
%Now for $z = x+ i\gamma$, $\gamma\in(0,1)$, $|x| \le 2A$, we have,
%\begin{align}
%\omega_{\textrm{FP}}(z,n,k,\alpha_n)&=C_1(\sfK)
%     \left( \frac{1 + |z|^4}{\Im(z)^4} \right)
%\left(  L\sqrt{\frac{k_+(d)}{n}} 
%   +  \frac{1}{n \Im(z)}\right)\\
%   &\le  C_2(\sfK) \frac{|A|^4}{\gamma^4}\left(L \sqrt{\frac{k_+(d)}{n}} + \frac1{n\gamma}\right)
%\end{align}
%as long as $A \ge 1$.
% %will denote some constant that is bounded uniformly over $\alpha_n$ bounded away
%%from $0$ and $\infty$, and $A,\sfK$ bounded.
%So for some $C_3(\sfK) >0$ sufficiently large, when
%%\begin{equation}
%%     \frac{C_3(\sfK) |A|^4}{\gamma^5}\left(L\sqrt{\frac{k_+(d)}{n}} + \frac1{n\gamma}\right)  \le  \frac1{\alpha_n}, 
%% \end{equation}
%for any $A > 1$ and $\gamma  > 0$, we have the bound
%\begin{equation}
%\sup_{x \in [-2A,2A]}\left|s_n(x + i \gamma) - s_\star(x+ i\gamma)\right| \le 
% \frac{C_3(\sfK)  |A|^8}{\gamma^9}\left( L\sqrt{\frac{k_+(d)}{n}} + \frac1{n\gamma}\right)
%\end{equation}
%whenever the quantity on the right is bounded by $\alpha_n^{-1}$.
%Now by definition of $\mu_\star$,  for $\alpha_n >1$, there exists a constant $A_0(\sfK) > 0$ such that
%\begin{equation}
%    \supp\left(\mu_\star(\nu, \alpha_n)\right) \subseteq [-A_0(\sfK),A_0(\sfK)].
%\end{equation}
%Furthermore, on the event $\Omega_0$ of Lemma~\ref{lemma:standard_norm_bounds}, we have the bound
%(for a similarly bounded constant $A_1$)
%\begin{equation}
%    \frac1n\norm{\bH}_\op 
%    %\le C \sfK  \left( 1 + \alpha_n^{-1}\right) 
%    \le A_1(\sfK).
%    \end{equation}
%%for some universal $C>0.$
%%
%Taking  $A(\sfK) := A_1(\sfK) \vee A_0(\sfK)$,
%we have by Lemma~\ref{lemma:f_bound_st} that 
%for any $\gamma \in (0,1)$,
%%\am{The formula below does not match Corollary~\ref{cor:f_bound_st}. RHS is nonlinear in $f$!}
%\begin{align}
%\left|
%    \int f(\lambda) \de \mu_n(\lambda) - \int f(\lambda) \de \mu_\star(\lambda)
%\right|
%&\le C_4(\sfK)
%\bigg(
%\norm{f}_{\infty,A(\sfK)}
% \sup_{x \in [-2A(\sfK),2A(\sfK)]}\left|s_n(x + i \gamma) - s_\star(x+ i\gamma)\right|
% \\
%&\quad\quad\quad+\gamma \left(\norm{f}_{\Lip}  + \norm{f}_{\infty,A(\sfK)}  \right)  \bigg)
%\\
%&\le C_5(\sfK)
%\norm{f}_{\infty,A(\sfK)}
%\frac{1}{\gamma^9}\left(L\sqrt{\frac{k_+(d)}{n}} + \frac{1}{n \gamma}\right)
%+ 
%\gamma( \norm{f}_{\Lip} + \norm{f}_{\infty,A(\sfK)} )
%\end{align}
%whenever this quantity is bounded by $\alpha_n^{-1}$.
%
%Now note that $\lambda \mapsto \log(\lambda \vee \eps)$ has a Lipschitz constant equal to $\eps^{-1}$, and recalling the bounds on $\P(\Omega_0^c)$ and $\P(\Omega_1(L)^c)$ for $L\ge1$, we obtain
%\begin{align}
%&\left|\E\left[\int \log(\lambda\vee \eps) \mu_n(\de \lambda)\right] - k \int \log(\lambda\vee \eps)  \mu_\star(\widehat\nu_{\tilde\bV}) (\de \lambda) \right|
%\le \E\left[\left|\int \log(\lambda\vee \eps) \mu_n(\de \lambda) - k \int \log(\lambda\vee \eps)  \mu_\star(\widehat\nu_{\tilde\bV}) (\de \lambda) \right|\right]\\
%&\quad\quad\le 
%C(\sfK)
%\|{\log^\up{\eps}}\|_{\infty,A(\sfK)}
%\frac{k}{\gamma^9}\left(L\sqrt{\frac{k_+(d)}{n}} + \frac{1}{n \gamma}\right)
%+ 
% k\gamma( \|\log^\up{\eps}\|_{\Lip} + \|{\log^\up{\eps}}\|_{\infty,A(\sfK)} ) 
% + k \log(A(\sfK)) (\P\left(\Omega_0^c\right) + \P(\Omega_1^c(L)))\\
% &\quad\quad\le C_1(\sfK)\left( \frac{k}{\gamma^9} \left(L \sqrt{\frac{k_+(d)}{n}} + \frac1{n\gamma}\right) + \frac{k\gamma }{\eps}  
% + k \left( e^{-c_1d } + e^{- c_2 L k} \vee d^{-c_2 L }\right)\right).
%\end{align}
%%Observe that whenever $k/(\gamma^{9} \sqrt{n}) <1$, we have $k/(n\gamma^{10}) <1$. 
%With the choice 
%\begin{equation}
%    L \equiv L(d) :=\begin{cases}
%        \frac1{c_2} \vee 1& k \ge \log(d)\\ 
%        \frac1{c_2}\log(d)  \vee 1& k < \log(d),
%    \end{cases}
%\end{equation}
%we have for $d > 1/{c_1}$,
%\begin{equation}
%    k \left( e^{-c_1d } + e^{- c_2 L k} \vee d^{-c_2 L }\right) \le \frac{k}{d}.
%\end{equation}
%This concludes the bound.
%
%\end{proof}

\subsection{Proofs of technical results of this section.}
\label{section:RMT_appendix_technical_results}

\subsubsection{Proof of Lemma~\ref{lemma:re_im_properties}}
\label{sec:proof_lemma_re_im_properties}
For $\bZ\in\bbH^+_k$, let $\bA = \Re(\bZ)$ and $\bB =\Im(\bZ)$. Since $\bB \succ \bzero$ we can write
\begin{equation}
\label{eq:bA_decomp}
   \bZ = (\bA + i \bB)  =  \bB^{1/2} \left(\bB^{-1/2}\bA \bB^{-1/2} + i \bI\right) \bB^{1/2}.
\end{equation}
The spectrum of $\bB^{-1/2}\bA\bB^{-1/2}$ is real since it's self-adjoint, and hence its perturbation by $i\bI$ does not contain $0$, 
proving invertibility of $\bZ$.
Now by Eq.~\eqref{eq:bA_decomp},
   \begin{align*}
       \Im(\bZ^{-1}) 
       &=
       \Im\left(
       \bB^{-1/2}(\bB^{-1/2}\bA\bB^{-1/2} + i\bI)^{-1}\bB^{-1/2}
       \right)\\
       &=
       \Im\left(
       \bB^{-1/2}(\bB^{-1/2}\bA\bB^{-1/2} + i\bI)^{-1}
       (\bB^{-1/2}\bA\bB^{-1/2} - i\bI)^{-1}
       (\bB^{-1/2}\bA\bB^{-1/2} - i\bI)\bB^{-1/2}
       \right)\\
       &=
       -\bB^{-1/2}(\bB^{-1/2}\bA\bB^{-1/2} + i\bI)^{-1}
       (\bB^{-1/2}\bA\bB^{-1/2} - i\bI)^{-1}
       \bB^{-1/2}\\
       &= 
       -(\bA + i\bB)^{-1}
       \bB
   (\bA - i\bB)^{-1}\\
&= - \bZ^{-1} \Im(\bZ) \bZ^{*-1}\\
&\prec \bzero,
   \end{align*}
where in the last line we used that $\Im(\bZ)\succ\bzero.$
To prove the bounds in Item~\textit{1},
note that, for any vector $\bx\in \C^k$,
\begin{align}
\label{eq:modulus_is_positive}
\|(\bB^{-1/2}\bA\bB^{-1/2} + i\bI)\bx\|_2 \ge \|\bx\|_2,
\end{align}
whence, for any $\bx\in \C^k$,
$\|(\bB^{-1/2}\bA\bB^{-1/2} + i\bI)^{-1}\bx\|_2 \le \|\bx\|_2$.
Therefore, taking inverses of both sides of Eq.~\eqref{eq:bA_decomp}
we conclude that $\|\bZ^{-1}\bx\|_2 \le \|\bB^{-1}\bx\|_2 
=\|\Im(\bZ)^{-1}\bx\|_2$ as desired for the bound on $\norm{\bZ^{-1}}_\op$. Using Eq.~\eqref{eq:modulus_is_positive} once again we conclude that 
$\norm{\bZ\bx}_2 \ge \norm{\bB\bx}_2$ giving the desired bound on $\Im(\bZ).$

To prove Item~\textit{2}, first consider the case where $\bW$ is invertible. In this case, we can write
   $(\bI + \bW\bZ)^{-1} \bW = (\bW^{-1} + \bZ)^{-1}.$
Noting that $\Im(\bW^{-1} + \bZ) = \Im(\bZ) \succ \bzero$, we see that an application of Item \textit{1} gives both claims. 
For non-invertible $\bW$, let $s_{\min} := \lambda_{*}(\bW)$
be the non-zero eigenvalue of $\bW$ with the smallest absolute value,
and define $\bW_\eps := \bW + \eps |s_{\min}|$ for $\eps \in (0,1)$.
We have by the previous argument that the statement holds for $\bW$ replaced with $\bW_\eps.$ Taking $\eps\to 0$ proves it in the general non-invertible case.
\qed

\subsubsection{Proof of Lemma~\ref{lemma:tensor_trace_properties}}
\label{sec:proof_lemma_tensor_trace_properties}
Let $\bM_{i,j}\in\C^{d\times d}$ be the blocks of $\bM$ for $i,j \in[k]$.
We obtain the first bound in \textit{1} by writing
\begin{align}
\nonumber
    \norm{(\bI_k \otimes \Tr)\bM}_F^2
    &= \sum_{i,j \in[k]} \Tr(\bM_{ij})^2
    \le d \sum_{i,j\in[k]} \norm{\bM_{ij}}_F^2
    = d \norm{\bM}_F^2.
\end{align}
Now let $\bx\in\R^{d}$ be a random variable distributed uniformly on the sphere of radius $\sqrt{d}$.
For any $\bv,\bu \in \C^{k}$ we have
\begin{equation}
\label{eq:tensor_tr_to_E_sphere}
    \bu^* \left(\left(\bI_k \otimes \Tr\right)\bM \right)\bv 
   = \sum_{i,j} \overline{u_i} v_j \Tr(\bM_{i,j}) 
  =  \sum_{i,j} \overline{u_i} v_j \E[\bx^\sT\bM_{i,j} \bx] = \E[(\bu\otimes \bx)^* \bM (\bv\otimes \bx)].
\end{equation}
Optimizing over $\bv,\bu$ of unit norm gives
\begin{align}
\nonumber
   \norm{(\bI_k \otimes \Tr)\bM}_\op  
%   &= \max_{\norm{\bv}=\norm{\bu} = 1} \sum_{i,j} u_i v_j \E[\bx^\sT\bM_{i,j}\bx]\\
   %&\le \max_{\norm{\bv} = \norm{\bu} = 1} \E[(\bu \otimes \bx)^\sT \bM (\bv \otimes \bx)]
    \le 
\max_{\norm{\bv}_2 = \norm{\bu}_2 = 1}\norm{\bM}_\op  \E[\norm{\bx}_2^2] \norm{\bu}\norm{\bv}
 = d \norm{\bM}_\op
\end{align}
giving the second bound in Item \textit{1}.
%
For the claim in \textit{2},
take $\bv = \bu$ in Eq.~\eqref{eq:tensor_tr_to_E_sphere} to conclude the (strict) positivity of $(\bI_k\otimes \Tr)\bM$ from that of $\bM$.
For \textit{3}, once consider Eq.~\eqref{eq:tensor_tr_to_E_sphere} and minimize over $\bv = \bu$ with unit norm and use \textit{2} to write
\begin{align}
\nonumber
   \lambda_{\min}\left((\bI_k \otimes \Tr)\bM\right)
   %&= \min_{\norm{\bv}=1}\left| \sum_{i,j} v_i v_j \E[\bx^\sT\bM_{i,j}\bx ] \right|
   = \min_{\norm{\bv} = 1} \E[(\bv \otimes \bx)^* \bM (\bv \otimes \bx)]
   \ge \min_{\norm{\bv} = 1} \lambda_{\min}(\bM) \norm{\bv \otimes \bx}_2^2
   %&= \min_{\norm{\bv} = 1} \sigma_{\min}(\bA) \norm{\bv \otimes \bx}_2^2\\
    = d \lambda_{\min}(\bM),
\end{align}
giving the claim.
Finally, \textit{4} follows by linearity of the involved operators.
%For \textit{3.}, defining $(\overline \Tr)(\bA) := \overline{\Tr(\bA)} = \Tr(\bA^*),$ we have
%\begin{align} 
%\Im\left( (\bI_k \otimes \Tr) \bA\right) 
%&= \frac1{2i}\left((\bI_k \otimes \Tr)(\bA) - (\bI_k \otimes \overline\Tr )(\bA)\right)\\
%&= (\bI_k \otimes \Tr)\left(\frac1{2i} \left(\bA - \bA^*\right) \right)\\
%&= (\bI_k \otimes \Tr)\Im(\bA).
%\end{align}
\qed
\subsubsection{
Proof of Lemma~\ref{lemma:algebra_lemma}}
\label{sec:proof_lemma_algebra_lemma}
Suppress the argument $z$ in what follows.
By Woodbury, we have for each $i\in[n],$
\begin{equation}
\nonumber
\bR  = \bR_i - \bR_i\bxi_i
\left(\bI_k + \bW_i\bxi_i^\sT \bR_i \bxi_i\right)^{-1} \bW_i\bxi_i^\sT \bR_i.
\end{equation}
So
\begin{align*}
 \bxi_i \bW_i\bxi_i^\sT \bR 
&= 
 \bxi_i \bW_i\bxi_i^\sT  \bR_i -
\bxi_i \bW_i\bxi_i^\sT
\bR_i\bxi_i
\left(\bI_k + \bW_i\bxi_i^\sT \bR_i \bxi_i\right)^{-1} \bW_i\bxi_i^\sT \bR_i\\
&= 
 \bxi_i \bW_i\bxi_i^\sT  \bR_i -
\bxi_i
\left(
\bI_k - 
\left(\bI_k +\bW_i \bxi_i^\sT \bR_i \bxi_i\right)^{-1}
\right)
\bW_i\bxi_i^\sT \bR_i\\
&=
\bxi_i
\left(\bI_k + \bW_i\bxi_i^\sT \bR_i \bxi_i\right)^{-1}
\bW_i\bxi_i^\sT \bR_i.
\end{align*}

To prove the identity in~Eq.~\eqref{eq:alg_id1},
for any $\bA \in\R^{k\times k}$, note that 
we have
\begin{align*}
\left(\bI_k \otimes \Tr\right) \bxi_i \bA  \bW_i\bxi_i^\sT \bR_i
   &=\left(\bI_k \otimes \Tr\right) (\bI_k \otimes \bx_i)
   \bA  
\bW_i
   (\bI_k \otimes \bx_i)^\sT \bR_i\\
   &= \left(\bI_k \otimes \Tr\right) \left(\sum_{a\in[k]} 
   (\bA \bW_i)_{j,a}
   \bx_i\bx_i^\sT  (\bR_i)_{a,l}
   \right)_{j,l \in[k]}
   \\
   &= \left( \sum_{a \in [k]}
   (\bA \bW_i)_{j,a}
   \Tr\left(\bx_i\bx_i^\sT  \left(\bR_i\right)_{a,l}
   \right)\right)_{j,l \in [k]}\\
   &= \left( \sum_{a \in [k]}
   (\bA \bW_i)_{j,a}
   \bx_i^\sT  \left(\bR\right)_{a,l} \bx_i
   \right)_{j,l \in [k]}\\
   &= 
   (\bA \bW_i)
   \bxi_i^\sT \bR_i\bxi_i.
\end{align*}
Using this for 
    $\bA :=\left( \bI_k + \bW_i\bxi_i^\sT \bR_i \bxi_i\right)^{-1}$
gives the result.
To prove the identity of Eq.~\eqref{eq:alg_id2}, we write
\begin{align*}
   \left(\bI_k \otimes \Tr \right)(\bR_i - \bR) &= \left(\bI_k \otimes \Tr \right)\bR_i \bxi_i \bW_i\bxi_i^\sT \bR\\
   &= \left(\Tr\left( \sum_{b,c \in [k]} (\bR_i)_{a,b} \bx_i(\bW_i)_{b,c}\bx_i^\sT \bR_{c,d}\right)  \right)_{a,d \in [k]}\\
   &= \left( \sum_{b,c \in [k]}  \bx_i^\sT  (\bR_i)_{a,b}^\sT (\bW_i)_{b,c}\bR_{c,d}^\sT \bx_i  \right)_{a,c \in [k]}\\
   &=  \bxi_i^\sT \bR_i^\sT (\bW_i \otimes \bI_d) \bR^\sT \bxi_i.
\end{align*}
Symmetry of the matrices $\bR_i$ and $\bR$ gives the conclusion.
%Now for the identity in Eq.~\eqref{eq:alg_id2}, by~Eq.~\eqref{eq:woodbury_cor}, we have
%\begin{align}
%    \bM_i^{-1}  \bz_i (\bI_k + \widetilde \bz_i^\sT \bM_i^{-1} \bz_i)^{-1} \widetilde \bz_i^\sT \bM_i^{-1}
%    &= \bM_i^{-1} \bz_i \widetilde \bz_i^\sT \bM^{-1}\\
%    &= \bM_i^{-1} \left(
%\sum_{a\in[k]} (\grad^2 \rho_{i})_{j,a} \bx_i \bx_i^\sT (\bM^{-1})_{a,l}
%    \right)_{j,l \in[k]}\\
%    &= \left(\sum_{j,a \in[k]} 
%    (\bM_i^{-1})_{b,j} 
%  (\grad^2 \rho_{i})_{j,a} \bx_i \bx_i^\sT (\bM^{-1})_{a,l}    
%    \right)_{b,l \in[k]}.
%\end{align}
%Taking $(\bI \otimes \Tr)$ and noting that $\bM_i,\grad^2 \rho_i$ and $\bM$ are symmetric, we have
%\begin{align}
%(\bI \otimes \Tr)
%    \left(\bM_i^{-1}  \bz_i (\bI_k + \widetilde \bz_i^\sT \bM_i^{-1} \bz_i)^{-1} \widetilde \bz_i^\sT \bM_i^{-1}\right)
%&= \left(\sum_{j,a \in[k]} 
%    \bx_i^\sT (\bM^{-1})_{l,a}   
%   (\grad^2 \rho_{i})_{a,j}
%(\bM_i^{-1})_{j,b}
%\bx_i
%    \right)_{b,l \in[k]}\\
%    &= 
%   \left((\bI_k \otimes \bx_i)^\sT 
%   \bM^{-1}(\grad^2\rho_i \otimes  \bI_k) \bM_{i}^{-1} (\bI_k \otimes \bx_i)\right)^{\sT}
%\end{align}
%which gives the result.
\qed

\subsubsection{Proof of Lemma~\ref{lemma:as_norm_bounds}}
\label{sec:proof_lemma_as_norm_bounds}
Since the eigenvalues $\left\{\lambda_j\right\}_{j\in[dk]}$ of $\bH_i$ are  real, we have
for the first two bounds of Eq.~\eqref{eq:det_norm_bound_lemma_eq123}
\begin{align}
\nonumber
   \norm{\bR_i}_{F}^2 
   &= \sum_{j=1}^{dk} \frac{1}{|\lambda_j - z n|^2 }
\le \frac{dk}{n^2} \frac1{\Im(z)^2}\quad\textrm{and}\quad
   \norm{\bR_i}_{\op}
   = \max_{i \in [dk]} \frac{1}{|\lambda_i - z n| }
\le \frac{1}{n} \frac1{\Im(z)}.
\end{align}
The bounds on $\norm{\bR}_{F}^2$, $\norm{\bR}_{\op}$
follow similarly.

For the third bound of Eq.~\eqref{eq:det_norm_bound_lemma_eq123}, given a vector $\ba\in\R^{nk}\simeq \R^{n}\otimes \R^k$, 
we write its entries as $\ba = (a_{i,l}: \; i\in[n],
l\in [k])$.
Then write
\begin{align}
\nonumber
    %\norm{(\Diag(\partial_{l,j}\ell_i)_{i\in [n]})_{l,j\in[k]}}_\op
    %\norm{\bW}_\op
    %&\le 
    \<\ba,\bSec\ba\> 
= \sum_{i=1}^n \< \ba_{i,\cdot}, \nabla^2_{\bv}\ell(\bv_i,\bu_i,\eps_i) \ba_{i,\cdot}\> 
    =  \sum_{i=1}^n \< \ba_{i,\cdot}, \bW_i \ba_{i,\cdot}\>  
    \le\sfK  \sum_{i=1}^n \norm{\ba_{i,\cdot}}_2^2\, .
\end{align}
%
Optimizing over $\norm{\ba}_2 = 1$ gives $\norm{\bW}_\op \le \sfK$ as desired.

The inequality $\norm{\bH}_\op \le\sfK \norm{\bX}_\op^2$ follows directly fro the previous one.

For the bound in~\eqref{eq:det_norm_bound_lemma_eq4}
recall that $\bH$ is self-adjoint, and hence for $z\in\bbH_+$, 
$\Im(\bH/n - z\bI)^{-1} \succ \bzero$. So by Lemma~\ref{lemma:tensor_trace_properties} and Lemma~\ref{lemma:re_im_properties}, we can bound
\begin{align*}
   \lambda_{\min}(\Im((\bI\otimes\Tr)\bR)) &= \lambda_{\min}\left( 
    \frac1n \left( \bI_k \otimes \Tr\right) \Im\left(\left(\bH/n - z \bI_{nk}\right)^{-1}\right)\right)
    &\ge  \lambda_{\min}\left(\Im((\bH/n - z\bI)^{-1})\right)\\
&=  \Im(z)\lambda_{\min}\left(
(\bH/n -z\bI)^{-1}(\bH/n -z^*\bI)^{-1}
\right)
&\ge  \Im(z)(\norm{\bH/n}_\op  + |z|)^{-2}.
\end{align*}
The conclusion now readily follows.

Finally, for the bound in~\eqref{eq:det_norm_bound_lemma_eq5} note that
\begin{equation}
    \lambda_{\min}(\Im(\bxi_i^\sT\bR_i \bxi_i)) = 
    \lambda_{\min}(\bxi_i^\sT\Im(\bR_i) \bxi_i) = \frac{\sigma_{\min}(\bxi_i^\sT\bxi_i)}{n} \lambda_{\min}
    \big(\Im( (\bH_i/n - z\bI_{n,k})^{-1})\big)
\end{equation}
and that  $\lambda_{\min}(\bxi_i^\sT\bxi_i) = \lambda_{\min}(\bI_k \otimes \bx_i^\sT\bx_i)= \norm{\bx_i}_2^2$ to derive the conclusion.
\qed



\subsubsection{Proof of Lemma~\ref{lemma:f_bound_st}}
\label{sec:proof_lemma_f_bound_st}
Lemma~\ref{lemma:f_bound_st} is a direct corollary of the two lemmas of this section.
Let
\begin{equation}
    \rho(x; x_0, \gamma) := \frac1\pi \Im\left( \frac{1}{x - (x_0 + i\gamma)} \right)
\end{equation}
be the density of a Cauchy distribution with location $x_0\in\R$ and scale $\gamma>0$. Recall that we have, for any continuous bounded function $f$,
$$\lim_{\gamma \to 0}\int f(x) \rho(x; x_0, \gamma) \, \de x = f(x_0).$$
The next lemma gives a quantitative version of this
fact.

\begin{lemma}
\label{lemma:quant_dirac_integral}
Fix positive reals $B > A > 0$. Define for $x_0 \in\R$,
\begin{equation}
    \Delta_{f,B,\gamma}(x_0):= f(x_0) - \int_{-B}^B  f(x)\rho(x;x_0,\gamma)\, \de x.
\end{equation}
We have the bounds
\begin{equation}
\nonumber
        \sup_{x_0 \in [-A,A]}\left|\Delta_{ f,B,\gamma}(x_0)\right|\leq \norm{f}_{\Lip} \gamma\log(4B^2+\gamma^2)
        +\norm{f}_\infty \frac{\gamma}{2} \left(\frac1{B-A} + \frac1{B+A}\right)
\end{equation}
and
\begin{equation}
\nonumber
    \sup_{x_0 \in \R \setminus [-A,A]}\left|\Delta_{ f,B,\gamma}(x_0)\right|
    \leq 
    2\norm{f}_\infty.
\end{equation}
\end{lemma}



\begin{proof}
The second bound is immediate since $\rho$ is a density. 
To show the first, fix $x_0\in[-B,B]$. We have
    \begin{align}
    \nonumber
        \Delta_{ f,B,\gamma}(x_0) =&  f(x_0)\int_{-\infty}^\infty\rho(x;x_0,\gamma)\de x
        -\int_{-B}^B  f(x) \rho(x;x_0,\gamma)\de x\\
        \nonumber
        =&\int_{-B}^B \left(f(x_0)-f(x) \right) \rho(x;x_0,\gamma)\de x +  f(x_0)\left(1-\int_{-B}^{B} \rho(x;x_0,\gamma)\de x\right)\\
        \leq & \norm{f}_{\Lip}\left(\int_{-B}^B |x_0-x|\rho(x;x_0,\gamma)\de x\right)
        +\norm{f}_\infty\left(1-\int_{-B}^B \rho(x;x_0,\gamma)\de x_0\right).
        \label{eq:last_eq_in_DeltafB_bound}
    \end{align}
    By a change of variable, the first integral above can be bounded as
    \begin{align*}
        \int_{-B}^B |x_0-x|\rho(x;x_0,\gamma)\de x = 
        \int_{-B-x_0}^{B-x_0} |x|\rho(x;0,\gamma)\de x
        \leq& \frac{2}{\pi}\int_{0}^{2B} x \frac{\gamma}{x^2+\gamma^2}\de x
        = \frac{\gamma}{\pi}\log\left(\frac{4B^2}{\gamma^2} + 1\right)
    \end{align*}
where we used the even symmetry of the integrand and that $|x_0 | \le B$ to deduce the inequality.
%\bns{This can be strengthened if necessary by bounding by $A$ instead but probably not needed.}
Meanwhile, the second integral in Eq.~\eqref{eq:last_eq_in_DeltafB_bound} is bounded as
    \begin{align*}
        1-\int_{-B}^B \rho(x;x_0,v)\de x
        %\le  1 - \int_{0}^{2B} \rho(x;0,\gamma) \de x
        &= 1- \frac{1}{\pi }\left[\arctan \left( \frac{B-x_0}{\gamma}\right) - \arctan \left( \frac{-B-x_0}{\gamma}\right)\right]\\
        &\le  1- \frac{1}{\pi }\left[\arctan \left( \frac{B-A}{\gamma}\right) + \arctan \left( \frac{B+A}{\gamma}\right)\right]\\
        &\le
        \frac12 \left( \frac{\gamma}{B-A}  + \frac{\gamma}{B+A}\right),
    \end{align*}
   where in the last line we used that $1 -2\pi^{-1}\arctan(t) \le t^{-1}$. This concludes the proof.
\end{proof}
\begin{lemma}
Let $f:\R\to\R$ be continuous. Let $\mu_1,\mu_2$ be two probability measures on $\R$ and let $s_1,s_2$ denote their corresponding Stieltjes transforms, respectively.
Then for any positive reals $B > A \ge 0$ and $\gamma  > 0$, we have
    \begin{align*}
         \bigg|\int f(x_0)\de\mu_1(x_0) - \int f(x_0)\de\mu_2(x_0)\bigg| &\le
        \frac1\pi \norm{f}_\infty \int_{-B}^B \Big|s_1(x+i\gamma)-s_2(x+i\gamma)\Big|\de x\\
&+2\norm{f}_{\Lip} \gamma\log(4B^2+\gamma^2)
        +\norm{f}_\infty \gamma \left(\frac1{B-A} + \frac1{B+A}\right)\\
&+2 \norm{f}_\infty \big(\mu_1\left(\R \setminus [-A,A]\right) + \mu_2\left(\R \setminus [-A,A]\right)\big).
     \end{align*} 
\end{lemma}


\begin{proof}
Rewriting $f$ in terms of the quantity $\Delta_{f,B,\gamma}$ defined in Lemma~\ref{lemma:quant_dirac_integral}, we have
\begin{align}
\nonumber
    \int f(x_0) \left(\de \mu_1(x_0) - \de \mu_2(x_0)\right) 
    &= \int\left( \int_{-B}^B f(x) \rho(x;x_0, \gamma) \de x  + \Delta_{f,B,\gamma}(x_0) \right) 
     \left(\de \mu_1(x_0) - \de \mu_2(x_0)\right) \\
     \nonumber
    &= 
     \int_{-B}^B f(x)\left(\int  \rho(x;x_0, \gamma) 
     \left(\de \mu_1(x_0) - \de \mu_2(x_0)\right) \right)\de x\\
     &\quad\quad+
\int \Delta_{f,B,\gamma}(x_0)
    \left(\de \mu_1(x_0) - \de \mu_2(x_0)\right)
     \label{eq:decomp_Ex_diff}
\end{align}
where the change of order of integration is justified by integrability of the continuous $f$ over $[-B,B]$.
Noting that
for $j\in\{1,2\}$,
\begin{equation}
\nonumber
     \int \rho(x;x_0,\gamma) \de \mu_j(x_0)=
     \frac1{\pi}\Im(s_j(x+i\gamma)),
\end{equation}
%\am{I actually get (using the Wikipedia convention for ST):
%\begin{equation}
%     \int \rho(x;x_0,\gamma) \de \mu_j(x_0)=
%     \frac1{\pi}\Im(s_j(x+i\gamma)),
%\end{equation}
%Please double check and propagate below
%}
the first term in Eq.~\eqref{eq:decomp_Ex_diff} is bounded as
\begin{align}
\nonumber
     \int_{-B}^B f(x)\left(\int  \rho(x;x_0, \gamma) 
     \left(\de \mu_1(x_0) - \de \mu_2(x_0)\right) \right)\de x 
     &=
     \frac1\pi\int_{-B}^B f(x) \left(
    \Im\left(s_1(x + i \gamma) - s_2(x+ i\gamma)\right)
     \right)
     \de x\\
     &\le \frac1\pi \norm{f}_\infty \int_{-B}^B \left|s_1(x + i \gamma) - s_2(x+ i\gamma)\right| \de x.
     \label{eq:decom_Ex_diff_bound_1}
\end{align}

To bound the second term in Eq.~\eqref{eq:decomp_Ex_diff}, 
for each $j\in\{1,2\}$
we have
\begin{align*}
   \int \Delta_{f,B,\gamma}(x_0) \de \mu_j(x_0)  
   &\le  \int_{-A}^A \left|\Delta_{f,B,\gamma}(x_0)\right| \de \mu_j(x_0)+ \int_{\R\setminus[-A,A]} \left|\Delta_{f,B,\gamma}(x_0) \right|\de \mu_j(x_0)\\
   &\le  
\sup_{x_0 \in [-A,A]} \left|\Delta_{f,B,\gamma}(x_0)\right|
        +
        \sup_{x_0 \in \R\setminus [-A,A]} \left|\Delta_{f,B,\gamma}(x_0)\right| \cdot
        \mu_j\left(\R \setminus [-A,A]\right).
\end{align*} 
Applying Lemma~\ref{lemma:quant_dirac_integral} and combining with Eq.~\eqref{eq:decomp_Ex_diff} and Eq.~\eqref{eq:decom_Ex_diff_bound_1} gives the desired bound.


\end{proof}




\section{Proof of Lemmas for Theorem~\ref{thm:general}}
\label{section:kac_rice}

\subsection{Derving the Kac-Rice equation on the parameter manifold: Proof of Lemma~\ref{prop:kac_rice_manifold}}
\label{sec:proof_of_kac_rice_on_manifold}
The goal of this section is to verify the generalization of the Kac-Rice integral to our setting and derive Proposition~\ref{prop:kac_rice_manifold} of Section~\ref{sec:pf_thm1_kr_integral}.
We continue suppressing the arguments in the definitions whenever it doesn't not cause confusion. For instance, we will often write $\bL$ for $\bL(\bbV;\bw)$ and $\cM$ for $\cM(\cuA,\cuB,\sPi).$


\subsubsection{Some properties of the parameter manifold and the gradient process}

We begin with the following lemma establishing basic regularity conditions of the manifold $\cM$.
\label{sec:properties_manifold}
\begin{lemma}[Regularity of the parameter manifold]
\label{lemma:manifold_dim}
Assume that $\ell,\rho$ and $(\cuA,\cuB)$ satisfy Assumptions~\ref{ass:loss},~\ref{ass:regularizer} and~\ref{ass:sets}, respectively. 
Then for $\sfsigma_\bR,\sfsigma_\bV,\sfsigma_\bL,\sfsigma_\bG \ge 0$,
$\cM(\cuA,\cuB,\sPi)$ is a differentiable manifold of dimension ${dk + (n-k)(k+k_0)}$, and in particular is bounded and orientable. 
\end{lemma}
\begin{proof}
Let 
\begin{align}
    \cV_0 :=\Big\{
   (\bTheta,\bbV) &: \hmu\in\cuA,\;
   \hnu\in\cuB,\; 
   \sfA_\bR\succ \bR(\hmu_{\bTheta,\bTheta_0}) \succ\sfsigma_\bR,\;\\
   &\quad 
\sfA_\bV \succ \E_{\hnu}[\bv\bv^\sT] \succ \sfsigma_{\bV}, \;
\E_{\hnu}[\grad\ell \grad\ell^\sT] \succ \sfsigma_{\bL},\;
\sigma_{\min}\left( \bJ_{(\bbV,\bTheta)} \bG\right) > n\,\sfsigma_{\bG}
    \Big\}.
\end{align}
We first prove that $\cV_0$ is a bounded open set. To do so, given that $\cuA$ and $\cuB$ are open under the weak convergence topology, it suffices to show that 
    $\sigma_{\min}(\bJ_{\bbV,\bTheta} \bG(\bbV,\bTheta)) > n \sfsigma_{\bG}$
forms an open set. This follows directly from the regularity properties given in the assumptions: since the second order partial derivatives of the form $\partial_{i,j}\ell$, $i\in[k+k_0],j\in[k]$ are, by Assumption~\ref{ass:loss}, continuous, 
and the regularizer $\rho_0$ is second order continuously differentiable by Assumption~\ref{ass:regularizer}, $\bG$ is continuously differentiable, and hence the constraint defined by 
$\sigma_{\min}(\bJ_{\bbV,\bTheta} \bG(\bbV,\bTheta)) > n \sfsigma_{\bG}$ for any $\sfsigma_\bG \ge 0$ indeed defines an open set. Checking the remaining constraints defining $\cV_0$ to define an open set is straightforward.
Now note that $\bzero$ is a regular value of $\bG(\bbV,\bTheta)$ when restricted to $\cV_0$ for any $\sfsigma_\bG \ge 0$, which implies $\cM = \{(\bTheta,\bbV):\; (\bTheta,\bbV)\in\cV_0, \; \bG(\bTheta,\bbV) = \bzero\}$ is a bounded regular level set, and hence is a bounded oriented differentiable manifold. The dimension can then be found to be $dk + (n-k)(k+k_0)$ by dimension counting.
\end{proof}

Next, we move on to studying the null space of the covariance of $\bzeta$ and showing that its degeneracy is constant in $(\bTheta,\bbV) \in\cM$. 
First, by straightforward computations, we obtain the following for the mean $\bmu(\bTheta,\bbV)$ and covariance $\bSigma(\bTheta,\bbV)$ of $\bzeta(\bTheta,\bbV)$:

\begin{equation}
 \bmu := [\brho_1^\sT,\dots,\brho_k^\sT,-\bv_1^\sT,\dots,-\bv_k^\sT, -\bv_{0,1}^\sT,\dots,-\bv_{0,k_0}^\sT]^\sT,
\quad 
\bSigma:= 
   \begin{bmatrix}
       \bL^\sT \bL \otimes \bI_{d}  & \bM & \bM_0\\
       \bM^\sT  & \bTheta^\sT \bTheta \otimes \bI_n & \bTheta^\sT\bTheta_0 \otimes \bI_n\\
       \bM_0^{\sT} & \bTheta_0^\sT\bTheta  \otimes \bI_n& \bTheta_0^{\sT}\bTheta_0 \otimes \bI_n
   \end{bmatrix} ,
\end{equation}
where
\begin{equation}
    \bM :=  \begin{bmatrix}
        \btheta_1\bell_1^\sT & \dots  & \btheta_k \bell_1^\sT\\
        \vdots  &   & \vdots \\
        \btheta_1\bell_k^\sT & \dots  & \btheta_k \bell_k^\sT\\
    \end{bmatrix}
    \in \R^{d k\times n k},\quad
    \bM_0 :=
    \begin{bmatrix}
        \btheta_{0,1}\bell_1^\sT & \dots  & \btheta_{0,k_0} \bell_1^\sT\\
        \vdots  &   & \vdots \\
        \btheta_{0,1}\bell_k^\sT & \dots  & \btheta_{0,k_0} \bell_k^\sT\\
        \end{bmatrix} \in \R^{d{k} \times n k_0 }.
\end{equation}
The following lemma characterizes the nullspace of $\bSigma$.
\begin{lemma}[Eigenvectors of the nullspace of $\bSigma$]
\label{lemma:eig_vecs_NS_Sigma}
Let
% \begin{equation}
% \ba_{i,j} := \ba_{i,j}(\bTheta,\bbV) := \left(
% \underbrace{\bzero^\sT, \dots, \bzero^\sT}_{j -1}, \btheta_i^\sT, \underbrace{\bzero^\sT,\dots,\bzero^\sT}_{k-j}, 
% \underbrace{\bzero^\sT, \dots, \bzero^\sT}_{i -1}, -\bell_j^{\sT}, \underbrace{\bzero^\sT,\dots,\bzero^\sT}_{k-i}, 
% \underbrace{\bzero^\sT, \dots, \bzero^\sT}_{k_0}
% \right), \quad i,j \in [k].
% \end{equation}
\begin{equation}
    \ba_{i,j}(\bTheta,\bbV):=
    \big(\be_{k,j}^\sT\otimes\btheta_i^\sT,
    -\be_{k,i}^\sT\otimes\bell_j(\bbV)^\sT,
    \underbrace{\bzero^\sT, \dots, \bzero^\sT}_{k_0}
    \big)^\sT,\qquad i,j\in[k],
\end{equation}
% \begin{equation}
% \ba_{0,i,j} \equiv \ba_{0,i,j}(\bbV) := \left(
% \underbrace{\bzero^\sT, \dots, \bzero^\sT}_{j -1}, \btheta_{0,i}^{\sT}, \underbrace{\bzero^\sT,\dots,\bzero^\sT}_{k-j}, 
% \underbrace{\bzero^\sT, \dots, \bzero^\sT}_{k},
% \underbrace{\bzero^\sT, \dots, \bzero^\sT}_{i -1}, -\bell_j^{\sT}, \underbrace{\bzero^\sT,\dots,\bzero^\sT}_{k_{0}-i}
% \right),\quad i\in[k_0], j\in[k].
% \end{equation}
\begin{equation}
    \ba_{0,i,j}(\bbV):= \big(\be_{k,j}^\sT\otimes\btheta_{0,i}^\sT,
    \underbrace{\bzero^\sT, \dots, \bzero^\sT}_{k},
    -\be_{k_0,i}^\sT\otimes\bell_j(\bbV)^\sT
    \big)^\sT,\qquad i\in[k_0],j\in[k],
\end{equation}
in which $(\be_{k,j})_{j\in[k]}$ and $(\be_{k_0,i})_{i\in[k_0]}$ are the canonical basis vectors for $\R^k$ and $\R^{k_0}$, respectively. 
Then for any $(\bTheta,\bbV) \in \cM(\cuA,\cuB,\sPi)$:
\begin{enumerate}
    \item We have
  \begin{equation}
 \Nul(\bSigma(\bTheta,\bbV)) =   \mathrm{span}\left(\left\{\ba_{i,j}(\bTheta,\bbV): i,j \in [k]\right\}\cup \left\{\ba_{0,i,j}(\bbV): i\in[k_0], j \in[k]  \right\} \right);
  \end{equation}

  \item the collection of basis vectors
$\left\{\ba_{i,j}(\bTheta,\bbV): i,j \in [k]\right\}\cup \left\{\ba_{0,i,j}(\bbV): i\in[k_0], j \in[k]  \right\}$ are linearly independent, and hence, 
\begin{equation}
\dim(\Nul(\bSigma(\bTheta,\bbV))) = k^2 + kk_0,\;\; \rank(\bSigma(\bTheta,\bbV)) 
=dk + (n-k)(k+k_0),
%\quad\textrm{for}\quad (\bTheta,\bbV)\in\cM(\cuA,\cuB,\sPi);
\end{equation}
\item the mean $\bmu(\bTheta,\bbV)$ of $\bzeta(\bTheta,\bbV)$ is orthogonal to $\Nul(\bSigma(\bTheta,\bbV))$.
\end{enumerate}
\end{lemma}


\begin{proof}
Fix $(\bTheta,\bbV)$ throughout and suppress these in the notation.
It is easy to check that $\bSigma \ba_{i,j} = \bzero$ and $\bSigma \ba_{0,i,j} = \bzero$ and hence the span of these vectors is contained in $\Nul(\bSigma)$.
We show the converse. Let $\bc \in\R^{dk}$, $\bd \in \R^{nk + nk_0}$ such that $\bSigma [\bc^\sT, \bd^\sT]^\sT= \bzero$. We then have
\begin{align*}
    \left(\bL^\sT\bL \otimes \bI_d\right) \bc + [\bM, \bM_0] \bd &= \bzero\\
    \left[] \bM, \bM_0\right]^\sT\bc + \left(\bR \otimes \bI_n\right) \bd &= \bzero.
\end{align*}
Let us define the sets
\begin{align*}
    \cS_1:= &\{
    \be_{k,j}\otimes\btheta_i:\; i,j\in[k]
    \}\cup \{\be_{k,j}\otimes \btheta_{0,i} 
    \; : i\in[k_0],j\in[k]\}\subset \R^{dk},\\
    \cS_2:=&\{
    (\be_{k,i}^\sT\otimes\bell_j^\sT,
    \bzero^\sT)^\sT:\; i,j\in[k]\} \cup 
    \{(\bzero^\sT,
    \be_{k_0,i}^\sT\otimes\bell_j^\sT, )^\sT :\;i\in[k_0],j\in[k]\}
    \subset\R^{nk+nk_0}.
\end{align*}
Note that 
% $(\bM,\bM_0) \bd  \in \vspan\left\{ \btheta_1, \dots,\btheta_k, \btheta_{0,1},\dots,\btheta_{0,k_0}\right\}$
%and 
%$\left( \bM, \bM_0\right)^\sT\bc \in \vspan\left\{\bell_1,\dots,\bell_k\right\}$,
$[\bM,\bM_0]\bd\in\vspan(\cS_1)$,
$[\bM,\bM_0]^\sT\bc\in \vspan(\cS_2)$, and $\bL^\sT\bL$ and $\bR$ are invertible. This implies that  
%$\bc \in \vspan\left\{ \btheta_1, \dots,\btheta_k, \btheta_{0,1},\dots,\btheta_{0,k_0}\right\}$
$\bc\in\vspan(\cS_1)$
and 
%$\bd \in \vspan\left\{\bell_1,\dots,\bell_k\right\}$.
$\bd\in\vspan(\cS_2)$
%
Hence, any $\ba \in \Nul(\bSigma)$ satisfies
\begin{equation}
    \ba \in
    \mathrm{span}\left(\left\{\ba_{i,j}: i,j \in [k]\right\}\cup \left\{\ba_{0,i,j}: i\in[k_0], j \in[k]  \right\} \cup
    \left\{ \tilde \ba_{i,j} : i,j \in [k] \right\}
    \cup
\left\{ \tilde \ba_{0,i,j} : i\in [k_0],j\in[k] \right\}
 \right)
\end{equation}
where 
% \begin{equation}
% \tilde\ba_{i,j} := \left(
% \underbrace{\bzero^\sT, \dots, \bzero^\sT}_{j -1}, \btheta_i^\sT, \underbrace{\bzero^\sT,\dots,\bzero^\sT}_{k-j}, 
% \underbrace{\bzero^\sT, \dots, \bzero^\sT}_{k+k_0}
% \right)
% \end{equation}
\begin{equation*}
\tilde\ba_{i,j} := \big(
\be_{k,j}^\sT\otimes\btheta_i^\sT,
\underbrace{\bzero^\sT, \dots, \bzero^\sT}_{k+k_0}
\big)^\sT\qquad\text{and}\qquad
\tilde\ba_{0,i,j} := \big(
\be_{k,j}^\sT\otimes\btheta_{0,i}^\sT,
\underbrace{\bzero^\sT, \dots, \bzero^\sT}_{k+k_0}
\big)^\sT.
\end{equation*}
%and
% \begin{equation}
% \tilde\ba_{0,i,j} := \left(
% \underbrace{\bzero^\sT, \dots, \bzero^\sT}_{j -1}, \btheta_{0,i}^\sT, \underbrace{\bzero^\sT,\dots,\bzero^\sT}_{k-j}, 
% \underbrace{\bzero^\sT, \dots, \bzero^\sT}_{k+k_0}
% \right).
% \end{equation}



We'll show that the collection $\{\bSigma \tilde \ba_{i,j}\} \cap \{\bSigma \tilde \ba_{0,i,j}\}$ is linearly independent.
This will then imply the desired inclusion
 $\Nul(\bSigma) \subseteq    \mathrm{span}\left(\left\{\ba_{i,j}: i,j \in [k]\right\}\cup \left\{\ba_{0,i,j}: i\in[k_0], j \in[k]  \right\} \right)$.
One can compute 
\begin{equation*}
    \bSigma \tilde \ba_{i,j} =
     \begin{bmatrix}(\bL^\sT\bL \otimes \bI_d)(\be_j\otimes \btheta_i)\\
   (\bR(\bTheta,\bTheta_0) \otimes \bI_n) (\be_j \otimes \bell_j)
   \end{bmatrix}
    ,
    \quad
    \bSigma \tilde \ba_{0,i,j} =
     \begin{bmatrix}(\bL^\sT\bL \otimes \bI_d)(\be_j\otimes \btheta_{i,0})\\
   (\bR(\bTheta,\bTheta_0) \otimes \bI_n) (\be_j \otimes \bell_j)
   \end{bmatrix}.
\end{equation*}
Once again, the linear independence of the columns of $\bL$ and $(\bTheta,\bTheta_0)$ now implies the desired linear independence.

Finally, the statement about the mean follows by using the constraint
$\bG(\bbV,\bTheta) = \bzero$ for $(\bTheta,\bbV) \in \cM(\cuA,\cuB,\sPi)$ and carrying out the computation.
\end{proof}


As a corollary, we have the following.
\begin{corollary}
\label{cor:proj}
Under Assumptions~\ref{ass:loss} and~\ref{ass:regularizer},
there exists 
$\bB_{\bSigma}:\R^m\rightarrow\R^{m_n \times (m_n-r_k)}$, 
whose columns
   %$\left\{\boldb_1(\bTheta,\bbV),\dots, \boldb_{m-k(k+k_0)}(\bTheta,\bbV)\right\}$ 
   are twice continuously differentiable,
   %in $(\bTheta,\bbV)$, 
   and are, for each $(\bTheta,\bbV)\in \cM(\cuA,\cuB,\sPi)$, an orthonormal basis of 
$\Col(\bSigma(\bTheta,\bbV))$.
\end{corollary}








\subsubsection{Concluding the proof of Proposition~\ref{prop:kac_rice_manifold}}

Let us cite the following lemma which will be useful in checking the non-degeneracy of the process required for the Kac-Rice formula to hold.
\begin{lemma}[Proposition 6.5,~\cite{azais2009level}]
\label{lemma:proc_grad_as_0}
    Let $\bh:\cU \to\R^m$ be a $C^2$ random process over $\cU \subseteq \R^m$. 
Let $\mathcal{K} \subseteq \cU$ be a compact set. 
Fix $\bu \in\R^m$.
Assume that for some $\delta >0$,
\begin{equation}
    \sup_{\bt \in \mathcal{K}} \sup_{ \bs  \in B_\delta^m(\bu) } p_{\bh(\bt)}(\bs)  < \infty.
\end{equation}
Then 
    \begin{equation}
        \P\left( \exists \bt \in \mathcal{K}:  \bh(\bt) = \bu,\; \det(\bJ_\bt \bh(\bt)) = 0\right) = 0.
    \end{equation}
\end{lemma}
%\begin{remark}
%    Clearly, we don't need uniformity in $\bTheta$ since we'll likely be restricting our attention to $\bTheta$ in a bounded subset of $\R^{d\times k}$, but for now I'll leave it as so. The modification is straightforward.
%\end{remark}

We are now ready to prove the proposiiton.
\begin{proof}[Proof of Proposition~\ref{lemma:kac_rice_manifold}]
Throughout, fix $\bw$ and use $\E$ and $\P$ for the conditional expectation and probability.
Consider a local chart $(\cU, \bpsi)$ of $\cM(\cuA,\cuB,\sPi)$.
%, where $\cU \subseteq \cM(\cuA,\cuB,\sPi)$ is open in the topology of the manifold and $\bpsi:\cU\rightarrow\R^{m-r_k}$. 
We will prove the claim when the parameter space is restricted to $\cU$. Since $\cM(\cuA,\cuB,\sPi)$ is an oriented bounded manifold, extension to $\cM(\cuA,\cuB,\sPi)$ can then be done via partitions of unity.
Namely, letting 
\begin{align}
    N_0 :=\left|\left\{
   (\bTheta,\bbV)\in \cU :
   \bzeta = \bzero,  
   \bH \succ n \sfsigma_\bH
    \right\}\right|,
\end{align}
%
%
%\begin{align}
%    N_0 :=\left|\left\{
%   (\bTheta,\bbV)\in \cU :
%   \hmu\in \cuA,\, 
%   \hmu\in \cuA,\, 
%   \bzeta = \bzero,  
%   \bL^\sT \bbV = \bzero, 
% \bR \succ\bzero,\;
% \bH \succ \bzero
%    \right\}\right|,
%\end{align}
we show that
\begin{align}
\E[N_0] &=\int_{(\bTheta,\bbV)\in \cU}  \E\left[\left| \det (\de\bz(\bTheta,\bbV))\right|
    \one_{\bH \succ n\sfsigma_\bH}
    \big| \bz(\bTheta,\bbV)= \bzero\right] p_{(\bTheta
    ,\bbV)}(\bzero)  \de_\cM V.
\end{align}

Define $\cO := \left\{ \bH  \succ n \sfsigma_\bH\right\}$, and for any $\bs\in\bpsi(\cU)$, define
\begin{equation}
\bg(\bs) := \bz(\bpsi^{-1}(\bs)) = \bB_{\bSigma}(\bpsi^{-1}(\bs))^\sT \bzeta(\bpsi^{-1}(\bs)),\quad
\bh(\bs) := (\bH/n )\circ\bpsi^{-1}(\bs),
\end{equation}
where $\bg: \bpsi(\cU) \mapsto \R^{m_0} $ and $\bh :\bpsi(\cU) \mapsto \R^M$ for
$m_0 = nk + nk_0 + dk - k^2- kk_0$ and $M = d^2k^2$,
and note that in this notation,
\begin{equation}
    N_0 = \left|\left\{\bs \in \bpsi(\cU) :  \bz(\bs) = \bzero,\; \bh(\bs) \in\cO\right\}\right|.
\end{equation}

We check the conditions of Theorem~\ref{thm:kac_rice}:
Clearly, $\bpsi(\cU)$ is an open subset of $\R^m$ by definition, and
condition \emph{(1.)} is by definition of the process. Condition \emph{(2.)} holds by Assumptions~\ref{ass:loss} and~\ref{ass:regularizer}. Meanwhile, condition \emph{(3.)} is guaranteed by definition of $\bB_{\bSigma}(\bt)$ and Lemma~\ref{lemma:eig_vecs_NS_Sigma}. 

We move on to condition \emph{(4.)}. % is the statement of Lemma~\ref{lemma:kr_condition4}. 
Let
\begin{equation}
    \cE_0(\sfsigma_\bH) :=\left\{ \exists (\bTheta,\bbV) \in \cM(\cuA,\cuB,\sPi)  : \bz(\bTheta,\bbV) =0,\; 
        \lambda_{\min}\left(\bH(\bTheta,\bbV) \right) =  n\sfsigma_\bH\right\}.
\end{equation}
We will show that
    \begin{equation}
    \P\left(\cE_0(\sfsigma_\bH)\right)= 0
    \end{equation}
by applying Lemma~\ref{lemma:proc_grad_as_0}.
Observe that for any $(\bTheta,\bbV)$ such that $\bz(\bTheta,\bbV) = \bzero$, we have by orthogonality of the mean of $\bzeta$ to the null space of $\bSigma$ that $\grad_\bTheta \hat R_n (\bTheta) = \bzero.$
Furthermore, at any such point, $\bH(\bTheta,\bbV)/n = \grad^2_\bTheta \hat R_n(\bTheta) = \bJ_{\bTheta} \tilde\bzeta$ where we defined the (non-Gaussian) process
 $\tilde \bzeta(\bTheta) := 
\grad_\bTheta \hat R_n(\bTheta)$.
Hence, we can immediately see that
$\cE_0 \subseteq  \tilde\cE_0$ for
\begin{align*}
\tilde\cE_0 :&=
\left\{
\exists \bTheta : \; 
\sfA_\bR \succeq \bR \succeq \sfsigma_\bR,\;
\tilde \bzeta(\bTheta) = \bzero,\;
\left|\det\left(\bJ_\bTheta \tilde \bzeta(\bTheta) - \sfsigma_\bH \bI\right)\right| = 0
\right\}\\
&=
\left\{
\exists \bTheta : \; 
\sfA_\bR \succeq \bR \succeq \sfsigma_\bR,\;
\tilde \bzeta(\bTheta) - \sfsigma_\bH \btheta = -\sfsigma_\bH \btheta,\;
\left|\det\left(\bJ_\bTheta (\tilde \bzeta(\bTheta) - \sfsigma_\bH \btheta) \right)\right| = 0
\right\};
\end{align*}
here, the vector $\btheta\in\R^{dk}$ denotes the concatenation of the columns of $\bTheta.$

We apply Lemma~\ref{lemma:proc_grad_as_0} to this event to show that it has probability $0$.
Notice that the density of the process $\bTheta\mapsto \tilde \bzeta(\bTheta) - \sfsigma_\bH \btheta$ at $\bu = -\sfsigma_\bH\btheta$ is the density of $\tilde\bzeta(\bTheta)$ at zero. In turn, under Assumption~\ref{ass:density}, the random variable $\tilde \bzeta(\bTheta)$ has a bounded density in a neighborhood of $0$ uniformly in $\bTheta$ (with constant depending on $\sfA_\bR$).
Meanwhile, Assumption~\ref{ass:loss} guarantees that $\bTheta \mapsto \tilde \bzeta(\bTheta)$ is $C^2$. So Lemma~\ref{lemma:proc_grad_as_0} applies to the compact set $\cK := \{\bTheta: \sfA_\bR \succeq \bR \succeq \sfsigma_\bR\}$, allowing us to deduce
$\P(\cE_0) \le \P(\tilde\cE_0) = 0$, verifying condition~\textit{(4.)} of Theorem~\ref{thm:kac_rice}.

Finally, condition \emph{(5.)} results from Lemma~\ref{lemma:proc_grad_as_0} noting that $\overline{\bpsi(\cU)}$ is compact. 
%Finally, we check condition \emph{(5.)}.

% We cannot apply Lemma~\ref{lemma:proc_grad_as_0} directly prove this condition is that the parameter set $\psi(\cU)$ may be unbounded. However, it is bound by a high probability. 
% Namely, for $C>0$, define the event $\cG_C := \left\{\norm{\bX}_\op \le C \sqrt{n}\right\}$.
% %and let $ C_0:= \norm{\bTheta_0}_2$.
% On $\cG_C$, for any $\bt = (\bTheta,\bbV)$ such that $\bz(\bt) = \bzero$, we have $\bzeta(\bt) = \bzero$ by Lemma~\ref{lemma:mean_cov}, implying that
% \begin{equation}
% \norm{\bbV}_\op = \norm{\bX(\bTheta,\bTheta_0)}_\op \le C \sfA_\bR \sqrt{n}.
%     %\norm{\bv_i}_2 = \norm{\bX \btheta_i}_2 \le C \sfA_\bR \sqrt{n} =: C_1(n),\quad
%     %\norm{\bu_{i_0}}_2 = \norm{\bX \btheta_{0,i_0}}_2 \le C C_0\sqrt{n} =:C_0(n).
% \end{equation}
% %
% So denoting
% \begin{equation}
%     \cB_C :=\{(\bTheta,\bbV)\in\cM : \|\bbV\|_\op \le C \sfA_\bR \sqrt{n}\},
% \end{equation}
% we have that \kas{perhaps mention that $\bTheta$ is by default bounded}
% \begin{align}
% \label{eq:prob_in_cond_5}
%   \P\left( \left\{\exists \bs \in\psi(\cU) : \bg(\bs) = \bzero,\;\det( \bJ_{\bs} \bg(\bs) ) = 0\right\} \cap \cG_C\right)
%   &=
%   \P\left( \left\{\exists \bs \in {\psi(\cU\cap \cB_C)} : \bg(\bs) = \bzero,\;\det( \bJ_\bs \bg(\bs) ) = 0\right\} \cap \cG_C\right)\\
%   &\le
%   \P\left( \left\{\exists \bs \in \overline{\psi(\cU\cap \cB_C)} : \bg(\bs) = \bzero,\;\det( \bJ_\bs \bg(\bs) ) = 0\right\} \right),
% \end{align}
% which puts us in a position to apply Lemma~\ref{lemma:proc_grad_as_0}: by Corollary~\ref{cor:proj} and Assumptions~\ref{ass:loss} and~\ref{ass:regularizer}, $\bs\mapsto \bz(\bs)$ is $C^2$. 
% Furthermore, since $\bz(\bs)$ is a non-degenerate Gaussian for all $\bs$, we have for its density $p_{\bz(\bs)}(\bu)$
% computed in Lemma~\ref{lemma:density},
% \begin{equation}
%     \sup_{\bs \in\psi(\cU\cap\cB_C)} \sup_{\bu \in B_2^k(\delta)} p_{\bz(\bs)}(\bu)
%     \le \sup_{\bs \in\psi(\cU\cap\cB_C)} p_{\bz(\bs)}(\E[\bz(\bs)]) = 
%     \sup_{\bs \in\psi(\cU\cap\cB_C)} \frac{\det^*(\bSigma(\psi^{-1}(\bs))^{-1/2})}{(2\pi)^{m_0/2}}< C_0(\sfs_\bR,\sfs_\bL,n,d)
% \end{equation}
% for some $C_0 \in(0,\infty)$ depending depending on $(\sfs_\bR,\sfs_\bL)$ and $n,d$. 
% So an application Lemma~\ref{lemma:proc_grad_as_0} allows conclude that the probability in Eq.~\eqref{eq:prob_in_cond_5} is equal to $0$.
% Therefore,
% \begin{align}
% \P\left( \left\{\exists \bs \in\psi(\cU) : \bz(\bs) = \bzero,\;\det( \bJ_\bs \bz(\bs) ) = 0\right\}\right)
% %&\le
% %  \P\left( \left\{\exists \bs \in\overline{\psi(\cU\cap \cB_C)} : \bz(\bs) = \bzero,\;\det( \bJ_\bs \bz(\bs) ) = 0\right\} \cap \cG_C\right) + \P(\cG_C^c)\nonumber\\
%   &\le  \P(\cG_C^c)
% \end{align}
% for all $C>0$. Taking $C\to\infty$ so that $\P(\cG_C^c) \to0$ shows that condition \emph{(5.)} of Theorem~\ref{thm:kac_rice} holds. 

So we conclude 
the integral formula
\begin{equation}
\label{eq:N0_kac_rice_app}
\E[N_0] =  \int_{\bs\in\bpsi(\cU)} \E\left[\left|\det \bJ_\bs\bz(\bs) \right| \one_{\bh(\bs) \in \cO} \big| \bg(\bs) = \bzero \right]  p_{\bz(\bs)}(\bzero)\de \bs.
\end{equation}
For $\bs\in\bpsi(\cU)$ we have
\begin{align*}
    \E\left[|\det \bJ_\bs \bz(\bs)| \one_{\bh(\bs)\in\cO} \big| \bg(\bs) = \bzero \right]=
    \left|\det\left( 
\de\bpsi^{-1}(\bs)\right)\right|
    \E\Big[&
    \left| \det \left( \de \bz(\bTheta,\bbV) \big|_{(\bTheta,\bbV) = \psi^{-1}(\bs)}\right)
    \right| \\
    &\one_{(\bH/n)\circ\bpsi^{-1}(\bs)\succ \bzero } 
\big| 
 \bz(\bpsi^{-1}(\bs)) = \bzero 
    \Big].
\end{align*}
Changing variables via $\bpsi(\bTheta,\bbV) = \bs$ and noting once again that $p_{\bz(\bs)}(\bzero) = p_{(\bTheta,\bbV)}(\bzero)$ defined in Lemma~\ref{lemma:density}, we have via Eq.~\eqref{eq:N0_kac_rice_app}


\begin{align*}
\E[N_0] &= \int_{\bs\in\bpsi(\cU)} 
    \E\left[
\left|\det \left( \de \bz(\bTheta,\bbV) \big|_{(\bTheta,\bbV) = \psi^{-1}(\bs)}\right)\right|
    \one_{\bh(\bs) \in \cO}
\big| 
 \bz(\psi^{-1}(\bs)) = \bzero 
    \right] p_{\bpsi^{-1}(\bs)}(\bzero) 
   \left|\det \left(\de\bpsi^{-1}(\bs)\right)\right|
    \de \bs\\
    &=\int_{(\bTheta,\bbV) \in \cU}  \E\left[\left| \det (\de\bz(\bTheta,\bbV))\right|
    \one_{\bH/n \succ \sfsigma_\bH}
    \big| \bz (\bTheta,\bbV) = \bzero\right] p_{(\bTheta,\bbV)}(\bzero)  \de_\cM V
\end{align*}
as desired.
\end{proof}

%\subsection{Reduction to the desired constraints}
%\label{sec:reduction_desired_constraints}
%\bns{This section needs rewriting.}
%%\begin{assumption}
%%    
%%For any $\bw \in \supp(\P_\bw)^n$, assume that one of these 
%%\begin{enumerate}
%%    \item The manifolds
%%\begin{equation}
%%   \left\{(\bV,\bU):
%%    (\bV,\bU)^\sT(\bV,\bU) \succ \bzero,
%%    \quad\det\left( \grad_{\bV,\bU} \left(\bL^\sT (\bV,\bU)\right)^\sT\grad_{\bV,\bU} \left(\bL^\sT (\bV,\bU)\right) \right) = 0\; 
%%   \right\}
%%\end{equation}
%%and
%%\begin{equation}
%%   \left\{(\bV,\bU):
%%    (\bV,\bU)^\sT(\bV,\bU) \succ \bzero,
%%    \quad
%%    \det\left(\bL^\sT\bL\right) = 0,\;  
%%   \right\}    
%%\end{equation}
%%each have dimension at most \bns{$n-d +k+k_0-1$}; or
%%\item For each $i\in[k]$ and $i_0\in[k_0]$ each of the manifolds
%%\begin{equation}
%%    \left\{\bv_i \in\R^n: \exists (\bV_{-i},\bU) \;\textrm{s.t.}\;
%%\det\left(\grad_{\bV,\bU} \left(\bL(\bV,\bU)^\sT (\bV,\bU)\right)\right) = 0,\; 
%%    \det\left(\bL(\bV,\bU)^\sT\bL(\bV,\bU)\right) = 0,\; 
%%    (\bV,\bU)^\sT(\bV,\bU) \succ \bzero
%%    \right\}
%%\end{equation}
%%and
%%\begin{equation}
%%    \left\{\bu_{i_0} \in\R^n: \exists (\bV,\bU_{-{i_0}}) \;\textrm{s.t.}\;
%%\det\left(\grad_{\bV,\bU} \left(\bL(\bV,\bU)^\sT (\bV,\bU)\right)\right) = 0,\; 
%%    \det\left(\bL(\bV,\bU)^\sT\bL(\bV,\bU)\right) = 0,\; 
%%    (\bV,\bU)^\sT(\bV,\bU) \succ \bzero
%%    \right\}
%%\end{equation}
%%has dimension at most $n-d - 1$.
%%\end{enumerate}
%%
%%\end{assumption}
%%
%\begin{assumption}
%\label{ass:curvature}
%For any $\bw \in \supp(\P_\bw)^n$, assume that
%%\begin{enumerate}
%%    \item The manifolds
%%\begin{equation}
%%   \left\{(\bV,\bU):
%%    (\bV,\bU)^\sT(\bV,\bU) \succ \bzero,
%%    \quad\det\left( \grad_{\bV,\bU} \left(\bL^\sT (\bV,\bU)\right)^\sT\grad_{\bV,\bU} \left(\bL^\sT (\bV,\bU)\right) \right) = 0\; 
%%   \right\}
%%\end{equation}
%%and
%%\begin{equation}
%%   \left\{(\bV,\bU):
%%    (\bV,\bU)^\sT(\bV,\bU) \succ \bzero,
%%    \quad
%%    \det\left(\bL^\sT\bL\right) = 0,\;  
%%   \right\}    
%%\end{equation}
%%each have dimension at most $(n-d)+k+k_0-1$; or
%%\item For each $i\in[k]$ and $i_0\in[k_0]$ each of the manifolds
%%\begin{equation}
%%    \left\{\bv_i \in\R^n: \exists (\bV_{-i},\bU) \;\textrm{s.t.}\;
%%    \det\left(\bL(\bV,\bU)^\{\sT\bL(\bV,\bU)\right) = 0,\; 
%%    (\bV,\bU)^\sT(\bV,\bU) \succ \bzero
%%    \right\}
%%\end{equation}
%%and
%%\begin{equation}
%%    \left\{\bu_{i_0} \in\R^n: \exists (\bV,\bU_{-{i_0}}) \;\textrm{s.t.}\;
%%\det\left(\grad_{\bV,\bU} \left(\bL(\bV,\bU)^\sT (\bV,\bU)\right)\right) = 0,\; 
%%    \det\left(\bL(\bV,\bU)^\sT\bL(\bV,\bU)\right) = 0,\; 
%%    (\bV,\bU)^\sT(\bV,\bU) \succ \bzero
%%    \right\}
%%\end{equation}
%%has dimension at most $n-d - 1$.
%%\end{enumerate}
%for each $i\in[k]$ and $i_0\in[k_0]$ each of the manifolds
%\am{ Also, would be clearer to define these as projections.} 
%\bns{I will rewrite this later. It doesn't seem clear to me how to ensure that $\eps_\bR>0 \Rightarrow \exists \eps_\bV >0$, but perhaps this can be absorbed in the definition of $\cB$.}
%\begin{equation}
%    \left\{\bv_i \in\R^n: \exists (\bV_{-i},\bU) \;\textrm{s.t.}\;
%    %(\bV,\bU)^\sT(\bV,\bU) \succ \bzero,\quad
%\det\left(\left|\bD_{\bV,\bU} \left(\bL(\bV,\bU)^\sT 
%(\bV,\bU)\right)\right|\right) = 0,\; \det(|\bV,\bU|) > 0
%    \right\},\quad
%\end{equation}
%\begin{equation}
%    \left\{\bv_i \in\R^n: \exists (\bV_{-i},\bU) \;\textrm{s.t.}\;
%    %(\bV,\bU)^\sT(\bV,\bU) \succ \bzero,\quad
%    \det\left(|\bL(\bV,\bU)|\right) =0,\; 
%    \det|(\bV,\bU)|  >\bzero
%    \right\},
%\end{equation}
%\begin{equation}
%    \left\{\bu_{i_0} \in\R^n: \exists (\bV,\bU_{-{i_0}}) \;\textrm{s.t.}\;
%    %(\bV,\bU)^\sT(\bV,\bU) \succ \bzero,\quad
%\det\left(\left|\bD_{\bV,\bU} \left(\bL(\bV,\bU)^\sT (\bV,\bU)\right)\right|\right) = 0,\; 
%\det\left(\left|(\bV,\bU)\right|\right) >0
%    \right\},
%\end{equation}
%and
%\begin{equation}
%    \quad
%    \left\{\bu_{i_0} \in\R^n: \exists (\bV,\bU_{-{i_0}}) \;\textrm{s.t.}\;
%%(\bV,\bU)^\sT(\bV,\bU) \succ \bzero,\quad
%    \det\left(\left|\bL(\bV,\bU)\right|\right) = 0,\; 
%\det(|(\bV,\bU) |) >0
%    \right\}
%\end{equation}
%each have dimension at most $n-d-1$.
%
%\end{assumption}
%
%%\begin{remark}
%%   Note that the first part of this assumption implies the second (but could be easier to check). This follows from the requirement that $(\bV,\bU)^\sT(\bV,\bU) \succ \bzero$.  \am{What is the first part and what the second part? Would be better to give names.}
%%\end{remark}
%%
%%
%\begin{lemma}
%\label{lemma:as_no_zeros_degenerate}
%    Under Assumption~\ref{ass:curvature}, we have for all $\bw \in \supp(\P_\bw)^n$,
%    \begin{equation}
%        \P\left(\forall(\bTheta,\bbV): \bzeta =\bzero, \bR\succ \bzero\; \Rightarrow
%        \det \bD\bL^\sT (\bbV) \neq 0,\; \bL^\sT\bL \succ\bzero,\;\bbV^\sT\bbV\succ\bzero
%        | \bw \right) = 1.
%    \end{equation}
%\end{lemma}
%\begin{proof}
%Since $n>d$, almost surely, $\bX$ will have full column rank so that
%\begin{equation}
%   \P\left(\bzeta = 0,\; \bR\succ\bzero \Rightarrow \bbV^\sT\bbV \succ \bzero \right)= 1.
%\end{equation}
%Now the result follows since under Assumption~\ref{ass:curvature},  the subspace
%$\{\bX\btheta_i:\btheta_i\in\reals^d\}=\{\bX\btheta_{0,i_0}:\btheta_{0,i_0}\in\reals^d\}$ 
%will almost surely be outside of the sets defined in the statement of the assumption.
%%So we show that
%%\begin{equation}
%%    \P\left(\exists (\bTheta,\bV,\bU) \;\textrm{s.t.}\; 
%%    \bzeta = 0, \bR\succ \bzero, (\bV,\bU)^\sT(\bV,\bU) \succ\bzero,\;
%%\det \grad \bL^\sT (\bV,\bU) \neq 0,\; \bL^\sT\bL \succ\bzero
%%    \right)= 0.
%%\end{equation}
%%Letting $\cS_i$ be the sets in Assumption~\ref{ass:curvature}, we have
%
%\end{proof}
%
%\begin{lemma}
%    Under Assumptions~\ref{ass:differntiability},\ref{ass:density},~\ref{ass:sets} and~\ref{ass:curvature}, we have for any open bounded $\cA\subseteq \R^{d\times k}$,
%    \begin{equation}
%        \E[Z_n(\cA,\cB)|\bw] = 
%        \int_{\bt \in \cM(\cA,\cB)}  \E\left[\left| \det (\de \bz(\bt) )\right|
%    \one_{\bH(\bt) \succ \bzero}
%    \big| \bzeta (\bt) = \bzero, \bw\right] p_\bt(\bzero)  \de S.
%    \end{equation}
%\end{lemma}
%
%\begin{proof}
%Once again, write $\E,\P$ for the conditional quantities given $\bw$.
%Recall the definition
%\begin{equation}
%    Z_n(\cA,\cB) = \left| \left\{
%   (\bTheta,\bbV) : \hmu_{\bTheta,\bTheta_0}\in\cA,\;\hnu_{\bbV}\in\cB, \;\bzeta = 0,  
%   \bL^\sT \bbV = \bzero, \;
% \bR(\hat \mu_{\bTheta,\bTheta_0}) \succ\eps_\bR,\;
%  \bH \succ \eps_\bH
%    \right\}\right|.
%\end{equation}
%By Lemma~\ref{lemma:as_no_zeros_degenerate},
%\begin{equation}
%    \E[Z_n(\cA) ] = \E[\tilde Z_n(\cA)]
%\end{equation}
%
%where
%\begin{equation}
%   \tilde Z_n(\cA) = \left| \left\{
%   (\bTheta,\bbV) : \bTheta\in\cA, \;\bzeta = 0,  
%   \bL^\sT \bbV = \bzero, 
% \bR \succ\eps_\bR,\;
%  \bH \succ \eps_\bH,\;
%|\bD_\bbV (\bL(\bbV)^\sT\bbV)|^2 \succ \bzero,\;
%\bL^\sT\bL \succ \bzero\;
%\bbV^\sT\bbV\succ \bzero
%    \right\}\right|.
%\end{equation}
%
%Since the mean of $\bzeta$ is in the columns space of its covariance by Lemma~\ref{lemma:mean_cov},
%the latter quantity is equal to $\E[N]$ defined in Lemma~\ref{lemma:kac_rice_manifold}.
%\end{proof}
%
%
%%\textcolor{blue}{
%%\begin{question}
%%I don't see how to get a non-asymptotic statement like we want in the main Theorem from this.
%%To illustrate the problem I'm facing, 
%%for simplicity, take $k = 1$ and $k_0 = 0$ and let $\bt \in\R^N$.
%% Suppose that after applying the formula above, we obtain something of the form
%% \begin{equation}
%%\E[Z_n] =    \int_{\bt \in\cM} e^{n\varphi(\bt)} h(\bt) \de V= 
%% \lim_{\eps \to 0}
%%        \frac1{2 \eps}
%%        \int_{\bt \in\R^N} e^{n\varphi(\bt)} h(\bt) \one_{\{|g(\bt)| \le \eps\}}  \norm{\grad g(\bt)}_2\de V .
%% \end{equation}
%% for some density $h(\bt)$, and let $\cM = \{g(\bt) = 0\}$  for some $g :\R^{N} \to \R$.
%%Then this implies (under a proper application of our asymptotics) for some $G$
%%\begin{equation}
%%    \frac1n \log(\E[Z_n]) \le \lim_{\eps\to 0} \left\{\frac1{n}\log\frac1{2\eps} + 
%%    \sup_{\nu : |\E_\nu{G(T)}| \le \eps} \left\{\E_\nu[\varphi(T)] - \KL(\nu \| \mu_h) \right\} + o(1)
%%    \right\}.
%%\end{equation}
%%\end{question}
%%}
%%\bns{Actually maybe i know how to do this by instead bounding the term on the LHS in the first display above by the integral + some term depending on $\eps$, that holds for all $\eps$ sufficiently small. Such a bound should exist by smoothness of the integrand }
%
%

\subsection{Integration over the manifold: Proof of Lemma~\ref{lemma:manifold_integral}}
\label{section:manifold_integration}
In this section, we upper bound the integral on the manifold appearing in Lemma~\ref{lemma:kac_rice_manifold} by an integral over a \emph{$\beta$-blowup} of the manifold $\cM$ as stated in Lemma~\ref{lemma:manifold_integral}.

%Recalling the error term $\Err_0(\beta,n,k)$ in the statement of Lemma~\ref{lemma:manifold_integral}, we will here obtain 
%the explicit quantity
%\begin{equation}
%    \Err_0(\beta,n,k) := 
%    \left(\frac1{1 - \beta\;r_k^2 C }\right)^{(m-r_k)/2}
%        \left(\frac{r_k^{5/2} C}{\beta(\sfsigma_{\bG} - \beta C r_k^2)}\right)^{r_k}.
%\end{equation}



%defined by
%\begin{equation}
%    \cM^\up{\beta} := \{\bu\in\R^{nk+nk_0 +dk}: \exists\;\bu_0\in\cM,\quad \norm{\bu-\bu_0}_2\le \beta\}.
%\end{equation}
%Recall the quantity 
%$\sfsigma_{\bG} := 1 \wedge \inf_{\bu \in\cM} \sigma_{\min}(\bJ_{\bu} \bg(\bu)^\sT).
%$
%The following lemma is the main result of this section. 

%\begin{lemma}[Manifold integral lemma]
%\label{lemma:manifold_integral} 
%Let $r_k := k(k+k_0)$, $m:= nk + nk_0 + dk$.
%Let $f: \R^{m}\rightarrow \R$ be a nonnegative continuous function. 
%There exists a constant $C = C(\sfA_{\bV},\sfA_{\bR})>0$ that depends only on $(\sfA_{\bV},\sfA_{\bR})$, such that for positive
%\begin{equation}
%   \beta \le \frac{C(\sfA_{\bV},\sfA_{\bR}) \sfsigma_{\bG}^3}{r_k^6} \wedge 1,
%\end{equation}
%we have
%\begin{equation}
%        \int_{(\bTheta,\bbV)\in \cM} f(\bTheta,\bbV) \de_\cM V
%        \le
%        E_f(\beta,n,k) 
%        \int_{(\bTheta,\bbV)\in \cM^{(\beta)}} f(\bTheta,\bbV)\de(\bTheta,\bbV),
%\end{equation}
%where the multiplicative error is given by
%\begin{equation}
%    E_f(\beta,n,k) := \left(\frac1{1 - \beta\;r_k^2 C }\right)^{(m-r_k)/2}
%        \left(\frac{r_k^{5/2} C}{\beta(\sfsigma_{\bG} - \beta C r_k^2)}\right)^{r_k}
%\exp\{\beta\;\norm{\log f}_{\Lip,\cM^{(1)}}\}.
%\end{equation}
%\end{lemma}

\subsubsection{Preliminaries}
Throughout this section, we'll use $m_n := nk +nk_0 + dk$ and $r_k := k(k+k_0)$.
To avoid working with tensors when differentiating, we will vectorize $\bG,\bbV$ and $\bTheta$. We will define the function $\bg :\R^{m_n}\to\R^{r_k}$ as the ``vectorization'' of $\bG :\R^{d\times k} \times \R^{n \times (k+k_0)} \to \R^{k\times(k+k_0)}.$
Using this notation, we'll define the quantities for $\tau >0$,
\begin{equation}
%\sfsigma_{\bG} := 1 \wedge \inf_{\bu \in\cM} \sigma_{\min}(\bJ_{\bu} \bg(\bu)^\sT),\quad
    \sfA_{\bG,1}^\up{\tau} := 1 \vee  
    \sup_{ \bu \in \cM^\up{\tau} }\|\bJ_\bu\bg(\bu)\|_\op,
\quad\quad
    \sfA_{\bG,2}^\up{\tau} := 1 \vee  
    \sup_{ \bu \in \cM^\up{\tau} }\|(\grad^2_\bu g_j(\bu))_{j\in[r_k]}\|_\op.    %\inf_{ \bu \in \cM }\sum_{j=1}^{r_k}\| \grad^2_\bu \bg_j(\bu)\|_\op
\end{equation}
Recall additionally
$\sfsigma_{\bG}$ satisfying $\sfsigma_\bG \le \inf_{\bu \in\cM} \sigma_{\min}(\bJ_{\bu} \bg(\bu)^\sT)$ of Section~\ref{sec:definitions}.
We assume without loss of generality that $\sfsigma_\bG \le 1$.
For $t>0$ define a normal tubular neighborhood $\cT(t)$ around the manifold $\cM$ by
\begin{equation}
\label{eq:tube_def}
    \cT(t):=
    \left\{(\bu,\bx)\in \cM\times \R^{r_k}: \norm{\bx}_2 < t \right\}.
\end{equation}
Define the function
$\bvarphi: \R^{m}\times \R^{r_k} \rightarrow \R^{m}$ by 
    \begin{equation}\label{eq:noraml-tube-chart}
        \bvarphi(\bu,\bx) := \bu + \bJ_{\bu}\bg(\bu)^\sT \bx.
    \end{equation}
This function will serve as our coordinate change. We will first show that this function is a local diffeomorphism and that the smallest singular value of its jacobian is lower bounded uniformly on $\cT(t)$ for the correct choice of $t$. 

\subsubsection{Showing that \texorpdfstring{$\varphi$}{phi} is a Local Diffeomorphism}
   %Let $\eps>0$ such that
   %    $\eps\le  
   %     \inf_{\bu\in\cM} \frac{\eps_D^2}{4\norm{\bD^2_{\bu}\bg(\bu)}_\op}.
   %     $
\begin{lemma}[Local diffeomorphism]\label{lem:local-diffeomorphis}

Assume 
$\tau \; \sfA_{\bG,2} \le (\sfsigma_\bG \wedge 1/2)$. 
Then, $\bvarphi$ restricted to the domain $\cT(\tau)$ is a local diffeomorphism 
and for $(\bu,\bx) \in\cT(\tau)$, we have the bound
\begin{equation}
    |\det(\de \bvarphi(\bu,\bx))| \ge (1 - \tau\;\sfA_{\bG,2})^{(m-r_k)/2} ( \sfsigma_{\bG} - \tau \sfA_{\bG,2})^{r_k}.
\end{equation}
\end{lemma}
\begin{proof} 
    Note that $\cT(\tau)$ is an open subset of the smooth manifold $\cM\times \R^{r_k}$, and consequently is a smooth manifold embedded in the ambient space of dimension $m+r_k$ with 
    $ \dim\left(\cT(\tau)\right) = \dim(\cM)+r_k = m.$
To lower bound the determinant, we will show that $\de\bvarphi(\bu,\bx) : T_{(\bu,\bx)}\cT(\tau)\rightarrow \R^m$ is a low-rank perturbation of a matrix that is approximately identity.

First, we'll compute the determinant of $\de \bvarphi$ for $(\bu,\bx)\in \cT(\tau)$. 
The (euclidean) Jacobian of $\bvarphi$ is given by
  \begin{equation}
      \bJ_{\bu,\bx}\bvarphi(\bu,\bx)= 
      \begin{bmatrix}
          \bI_{m} + 
          \bJ_\bu \circ \{ \bJ_{\bu}\bg(\bu)^\sT\bx\} \;\;,
          & 
          \bJ_\bu\bg(\bu)^\sT
      \end{bmatrix}
      \in \R^{m\times(m+r_k)}.
  \end{equation}
Letting $\bB(\bu)\in\R^{m\times(m-r_k)}$ be an orthonormal basis of $T_\bu\cM$,
%, and let $\bB^\perp_\bu \in\R^{m\times r_k}$ be a basis of the orthogonal complement. 
we define
      \begin{equation}\bB_0(\bu,\bx) := 
        \begin{bmatrix}
            \bB({\bu})& \bzero_{m\times r_k}\\
            \bzero_{r_k\times(m-r_k)}& \bI_{r_k} 
        \end{bmatrix}\in\R^{(m+r_k)\times(m)},
    \end{equation}
and note that since $T_{(\bu,\bx)} \cT(\tau)$ is isomorphic to $T_{(\bu)}\cM\oplus \R^{r_k}$
at $(\bu,\bx)\in \cT(\tau)$, 
we have that 
\begin{equation}
\label{eq:det_dphi_identification}
    \det(\de \bvarphi )  = \det(\bA),\quad \textrm{where}\quad
    \bA :=  \left[\bB
    +
    (\bJ_{\bu}\circ\{\bJ_{\bu} \bg(\bu)^\sT\bx\})\bB
    ,  \bJ_\bu \bg(\bu)^\sT \right]\in \R^{m\times m}.
\end{equation}
To simplify notation, from hereon, let us use the shorthand
    $\bH_0 := (\bJ_{\bu}\circ\{\bJ_{\bu} \bg(\bu)^\sT\bx\})$ and $\bF_0 :=\bJ_\bu \bg(\bu)^\sT.$
Now to lower bound the determinant above, noting that
\begin{equation}
\bA^\sT\bA = \begin{bmatrix}
\bB^\sT \left(\bI_{m} + \bH_0\right)^\sT(\bI_{m} +\bH_0) \bB &   \bB^\sT\bH_0^\sT \bF_0\\
\bF_0^\sT  \bH_0\bB &  \bF_0^\sT\bF_0
\end{bmatrix}
\end{equation}
(using $\bF^\sT\bB = 0$, $\bB^\sT\bB = \bI_{m}$),
we have
\begin{equation}
\label{eq:det_AA_lb}
    \det(\bA^\sT\bA) \ge \det\left(\bB^\sT \left(\bI_{m} + \bH_0\right)^\sT(\bI_{m} +\bH_0) \bB \right) 
\lambda_{\min}(\bA^\sT\bA)^{r_k}.
\end{equation}

The definition in Eq.~\eqref{eq:det_dphi_identification} readily gives a na\"ive bound on the smallest singular value of $\bA$ through
\begin{equation}
\label{eq:sigmamin_A_lb}
\sigma_{\min}(\bA) \ge \sigma_{\min}([\bB, \bF_0]) - \|\bH_0\|_\op 
\ge  \sfsigma_{\bG} -  \|\bH_0\|_\op,
\end{equation}
since $\sfsigma_\bG \le 1.$
Meanwhile, we have
\begin{equation}
\label{eq:det_BHB_lb}
    \det
\left(\bB^\sT \left(\bI_{m} + \bH_0\right)^\sT(\bI_{m} +\bH_0) \bB \right) 
\ge \lambda_{\min}(\bI_{m-r_k} + \bDelta_0)^{m-r_k}
\end{equation}
for some matrix $\bDelta_0 \in \R^{(m-r_k)\times (m-r_k)}$ satisfying
$\|\bDelta_0\|_\op \le (1 + 2\|\bH_0\|_\op)\|\bH_0\|_\op.$
Finally, to bound $\|\bH_0\|_\op$, we expand via its definition 
\begin{equation}
   \|\bH_0\|_\op  =
\Big\|\bJ_{\bu}\circ\Big\{\sum_{j=1}^{r_k} \grad_\bu g_j(\bu) x_j\Big\}\Big\|_\op
=
\Big\|\sum_{j=1}^{r_k} \grad^2_\bu g_j(\bu) x_j\Big\|_\op \le \|\bx\|_2 
\|(\grad^2_\bu g_j(\bu))_{j\in[r_k]}\|_\op  \le \sfA_{\bG,2}\;\tau.
\end{equation}
So for $\tau \; \sfA_{\bG,2} \le 1/2$, we have $\|\bDelta_{0}\|_\op \le 2 \tau \;\sfA_{\bG,2}.$
Combining this with Equations~\eqref{eq:det_AA_lb},~\eqref{eq:sigmamin_A_lb} and~\eqref{eq:det_BHB_lb} concludes the proof.


%for the matrix $\bA\in\R^{m \times m}$  defined by (suppressing the argument of $\bB(\bu)$),
%\begin{equation}
%\bA := \bA_0 + \bA_1 \quad 
%\textrm{where}\quad
%\bA_0 := \left[\bB ,  \bJ_\bu \bg(\bu)^\sT \right],\quad
%\bA_1 := \left[(\bJ_{\bu}\circ\{\bJ_{\bu} \bg(\bu)^\sT\bx\})\bB , \bzero \right].
%\end{equation}

\end{proof}


\subsubsection{
Showing that \texorpdfstring{$\varphi$}{phi} is a 
Global diffeomorphism}
%The next step is to find $\eps_\cM>0$ such that for any $\eps\in(0,\eps_\cM)$, $\bvarphi$ becomes a global diffeomorphism on $\cT(\eps)$.
%
%To establish such a bound, we need to know how the function $\bg$ behaves when we move further away from the manifold within the ambient space. Hence throughout, let us fix a constant $\Delta>0$, and let a $\Delta$-blowup of $\cM$ to be
%\begin{equation}
%    \cM^\Delta := \{\bu\in\R^{m}: \exists\bu_0 \in\cM,\quad \norm{\bu-\bu_0}\le \Delta\}.
%\end{equation}
%To avoid cluttering, let us define 
%
%
%\begin{equation}
%   \sfD^\Delta := \sup_{\bu\in\cM^\Delta} \norm{\bJ_\bu\bg(\bu)}_\op,  \quad 
%   \sfD := \sup_{\bu\in\cM} \norm{\bJ_\bu\bg(\bu)}_\op,
%\end{equation}
%\begin{equation}
%   \sfH^\Delta := \sup_{\bu\in\cM^\Delta} \norm{\bD^2_\bu\bg(\bu)}_\op,  \quad 
%   \sfH := \sup_{\bu\in\cM} \norm{\bD^2_\bu\bg(\bu)}_\op.
%\end{equation}
%The main result of this section is the following:

%\begin{lemma}[Global diffeomorphism] \label{lem:global-radius}
%    Let $\beta\in(0,\eps_\cM)$, where
%    \begin{equation}
%        \eps_\cM :=  \frac1{6\sfD^2}\left(
%        \left(\frac{\eps_D^3}{32(\sfD^\Delta)^2
%        \sfH^\Delta}\right) 
%        \wedge
%        (2\eps_D\Delta)\right).
%    \end{equation}
%    Then, $\bvarphi$ is a global diffeomorphism on the manifold $\cT(\eps)$.
%\end{lemma}

\begin{lemma}[Implicit function theorem: modified Theorem 2.9.10 \cite{hubbard2015vector}] 
\label{lem:implicit-function}
      Let $\boldf:\cU\rightarrow\R^p$ 
    for $\cU \subseteq \R^{m+p}$
      be a smooth function. 
      Fix $(\bx_0,\by_0)\in\cU$ such that 
     $\boldf(\bx_0,\by_0)=\bzero$, $\bJ_\bx \boldf(\bx_0,\by_0)\in\R^{p\times p}$ is invertible, and $\bJ_\by \boldf(\bx_0,\by_0)=\bzero$.
    Choose $\delta_0>0$ so that $\Ball^{p+m}_{\delta_0}((\bx_0,\by_0)) \subseteq \cU$, and
\begin{equation}
    %\sum_{j=1}^p\sup_{(\bx,\by) \in B_{\delta_0}(\bx_0,\by_0)} \|\grad^2 \boldf_j(\bx,\by)\|_\op^2 
    \sup_{(\bx,\by) \in \Ball^{p+m}_{\delta_0}((\bx_0,\by_0))}
\big\|\big(\grad^2 \boldf_{j}(\bx,\by)\big)_{j\in[p]}\big\|_\op 
    \le  \frac1{\delta_0}\sigma_{\min}\left( \bJ_{\bx} \boldf(\bx_0,\by_0)\right).
\end{equation}
      Then, for $r_0:= \delta_0 \sigma_{\min}(\bJ_\bx\boldf(\bx_0,\by_0))/2$, there exists a unique smooth mapping
      $\bpsi:\Ball_{r_0}^p(\by_0)\rightarrow \Ball_{\delta_0}^m(\bx_0)$
      satisfying $\bpsi(\by_0)=\bx_0$, and
      \begin{equation}
          \{(\bx,\by) \in \Ball_{\delta_0}^m(\bx_0)\times\Ball_{r_0}^p(\by_0) 
          :\boldf(\bx,\by) = \bzero \} = 
          \{(\bpsi(\by),\by): \by\in \Ball_{r_0}^p(\by_0)\}.
      \end{equation}
\end{lemma}
\begin{proof}
We invoke Theorem 2.9.10 of \cite{hubbard2015vector} which asserts the existence and uniqueness of such a $\bpsi$ under the conditions of the lemma, if the Jacobian satisfies:
\begin{equation}
\label{eq:jacobian_lipschitz_IFT}
     \|\bJ\boldf(\bx_1,\by_1) - \bJ\boldf(\bx_2,\by_2)\|_\op
     \le 
     \frac{1}{2 r_0 \|\bL^{-1}\|_\op^2} \|(\bx_1, \by_1) - (\bx_2,\by_2)\|_2,
\end{equation}
where 
\begin{equation}
   \bL :=   \begin{bmatrix}
        \bJ_{\bx}  \boldf(\bx_0,\by_0) & 
        \bJ_{\by}  \boldf(\bx_0,\by_0)\\
\bzero & \bI_m
   \end{bmatrix}.
\end{equation}
To de-clutter notation, define
\begin{equation}
    s_{\min}(\bx_0,\by_0) := \sigma_{\min} (\bJ_\bx\boldf(\bx_0,\by_0)^\sT),
\quad 
A_{\delta_0}(\bx_0,\by_0) :=  
\sup_{(\bx,\by) \in \Ball_{\delta_0}((\bx_0,\by_0))}
\big\|\big(\grad^2 \boldf_{j}(\bx,\by)\big)_{j\in[p]}\big\|_\op 
%\sum_{j=1}^p\sup_{(\bx,\by) \in B_{\delta_0}(\bx_0,\by_0)} \|\grad^2 \boldf_j(\bx,\by)\|_\op^2.
\end{equation}
So for any $\bz_1 := (\bx_1,\by_1), \bz_2 :=(\bx_2,\by_2)\in B_{\delta_0}(\bx_0,\by_0)$ we have
\begin{align*}
     \|\bJ\boldf(\bz_1) - \bJ\boldf(\bz_2)\|_\op
&=  
\Big\|\Big(\int_{0}^1 \grad^2 \boldf_{j}(\bz_1  + t(\bz_2 - \bz_1)) (\bz_2 - \bz_1) \de t\Big)_{j\in[p]}\Big\|_\op \\
&=
\Big\|\int_{0}^1 \Big(\grad^2 \boldf_{j}(\bz_1  + t(\bz_2 - \bz_1)) (\bz_2 - \bz_1) \Big)_{j\in[p]} \de t\Big\|_\op \\
&\le 
\sup_{t}\Big\|\Big(\grad^2 \boldf_{j}(\bz_1  + t(\bz_2 - \bz_1)) (\bz_2 - \bz_1) \Big)_{j\in[p]}\Big\|_\op \\
&\le 
%\sup_{\bz \in \cB_{\delta_0}(\bx_0,\by_0)}
%\big\|\big(\grad^2 \boldf_{j}(\bz)\big)_{j\in[p]}\big\|_\op 
A_{\delta_{0}}(\bx_0,\by_0)
\|(\bz_2 - \bz_1) \|_2.
%\\
%    &= \sum_{j=1}^{p} \|\grad \boldf_{j}(\bx_1,\by_1) - \grad \boldf_{j}(\bx_2,\by_2) \|_2^2\\
%    &\le
%  \frac12  A_{\delta_0}(\bx_0,\by_0)  \|(\bx_1,\by_1) - (\bx_2,\by_2)\|_2^2\\
%  &\le  \frac{s_{\min}(\bx_0,\by_0)^2}{\delta_0^2} 
%  \|(\bx_1,\by_1) - (\bx_2,\by_2)\|_2^2.
\end{align*}
So by assumptions of the lemma,
\begin{equation}
     \|\bJ\boldf(\bx_1,\by_1) - \bJ\boldf(\bx_2,\by_2)\|_\op
  \le  \frac{s_{\min}(\bx_0,\by_0)}{\delta_0} 
  \|(\bx_1,\by_1) - (\bx_2,\by_2)\|_2.
\end{equation}
After substituting the definition of $r_0$, and noting that $\bJ_\by \boldf(\bx_0,\by_0) =\bzero$ by assumption so that $\sigma_{\min}(\bL) = s_{\min}(\bx_0,\by_0),$
we conclude that
the inequality of Eq.~\eqref{eq:jacobian_lipschitz_IFT} holds, giving the existence and uniqueness of the $\bpsi$ defined in the statement.

So all the conditions are satisfied, and we can apply \cite{hubbard2015vector} to get the existence of a function $\bpsi:B_R(\by_0)\rightarrow B_{\delta}(\bx_0)$ such that $\boldf(\bpsi(\by),\by) = \bzero$ for all $\by\in B_R(\by_0)$. 
%This, alongside the uniqueness of $\bpsi$ implies the set inequality of the lemma.
    \end{proof}


\begin{lemma}[Tangentional-Decomposition]
\label{lem:tangentional-decomposition}
Fix a point $\bu_0 \in\cM \subseteq \R^{m_n}$. Let $\bU_0:= [\bB_0^\perp, \bB_0]\in\R^{m_n\times m_n}$, where $\bB_0\in\R^{m_n\times (m_n-r_k)},\bB_0^\perp \in\R^{m_n\times r_k}$ are basis matrices for the tangent space and normal space of $\cM$ at $\bu_0$ respectively.  For $\bu \in \R^{m_n}$, let $[\bu_N^\sT,\bu_T^\sT]^\sT := \bU_0^{-1}\bu$, and use the notation $\bJ_{\bu_N} \bg(\bu)
            := \bJ_{\bu_N}\bg(\bU_0
            [\bu_N^\sT,\bu_T^\sT]^\sT
            )\in \R^{r_k\times r_k}$ and $\bJ_{\bu_T} \bg(\bu) := \bJ_{\bu_T}\bg(\bU_0
            [\bu_N^\sT,\bu_T^\sT]^\sT
            )\in \R^{r_k\times (m_n- r_k)}$.
Then 
\begin{enumerate}
    \item we have
\begin{equation}
\label{eq:derivatives_along_tangent_and_normal}
\bJ_{\bu} \bg(\bu) = [\bJ_{\bu_N}\bg(\bu), \bJ_{\bu_T}\bg(\bu)]\bU_0^\sT
,\quad
        \bJ_{\bu_T}
        \bg(\bu_0)=\bzero,\quad
        \sigma_{\min}\left(\bJ_{\bu_N}\bg(\bu_0)\right)\ge \sfsigma_\bG.
\end{equation}
\item
For any $\tau\in(0,1)$,
\begin{equation}
\label{eq:variation_msv}
            \norm{\bu -\bu_0}_2\le \frac{\sfsigma_\bG}{2 \sfA_{\bG,2}^\up{\tau}}\wedge \tau
            \quad\Rightarrow\quad
            \sigma_{\min}\left(
            \bJ_{\bu_N}
            \bg(\bu)\right)
            \geq
        \frac{\sfsigma_{\bG}}{2}.
\end{equation}
\end{enumerate}

\end{lemma}

\begin{proof}
That the Jacobian of $\bg$ can be decomposed as in~\eqref{eq:derivatives_along_tangent_and_normal} follows from the chain rule. Then $\bJ_{\bu_\bT}\bg(\bu_0) =  \bJ_{\bu}\bg(\bu_0) \bB_{0}  = \bzero$ by definition of $\cM$, and $\sigma_{\min}(\bJ_{\bu_\bT}\bg(\bu_0)) =\sigma_{\min}( \bJ_{\bu}\bg(\bu_0)\bB_{0}) \ge \sfsigma_{\bG}$ follow immediately.

Now to prove Eq.~\eqref{eq:variation_msv}, an argument similar to the one in Lemma~\ref{lem:implicit-function} gives
\begin{align}
\nonumber
|\sigma_{\min}\left(\bJ_{\bu_N}  \bg(\bu) \right)-\sigma_{\min}\left(\bJ_{\bu_N}  \bg(\bu_0) \right)|
&\le 
\|\bJ_{\bu_N} \bg(\bu) - \bJ_{\bu_N} \bg(\bu_0)\|_\op \le \sfA_{\bG,2}^\up{\tau} \|\bu - \bu_0\|_2 
\le
\sfA_{\bG,2}^\up{\tau} \left( \frac{\sfsigma_{\bG}}{2 \sfA_{\bG,2}^\up{\tau}} \wedge \tau\right)\\
&\le  \frac{\sfsigma_{\bG}}{2}.
\end{align}
\end{proof}
\begin{corollary}[Implicit chart] \label{cor:implicit-chart}
Fix arbitrary $\bu_0 \in\cM \subseteq \R^m$. 
Use the notation introduced in in Lemma~\ref{lem:tangentional-decomposition}. For $\bu\in\R^m$, denote $[\bu_N^\sT,\bu_T^\sT]^\sT := \bU_0^{-1}\bu$.
Fix $\tau,\tau_N,\tau_T >0$ such that
\begin{equation}
\label{eq:tau_N_tau_T}
\tau_N  < \frac{\sfsigma_{\bG}}{4 \sfA_{\bG,2}^\up{\tau}} 
\wedge \frac{\tau}{2}, \quad\quad  \tau_T <  \frac{\sfsigma_{\bG} \tau_N }{2} \wedge \frac{\tau}{2}.
\end{equation}
Then there exists smooth functions 
$\bpsi_0: \Ball_{\tau_T}^{m-r_k}(\bu_{0,T}) \rightarrow\R^{r_k}$,
$\tilde\bpsi_0 :\Ball_{\tau_T}^{m-r_k}(\bu_{0,T})  \rightarrow \R^m$ such that
the following hold:
\begin{enumerate}
    \item we have
\begin{equation}
  \tbpsi_0(\bz) := \bU[\bpsi_0(\bz)^\sT,\bz^\sT]^\sT, \quad\quad
  \tbpsi_0(\bu_{0,T}) = \bu_0,\quad\quad
     \bJ_{\bz} \psi(\bz)  = -(\bJ_{\bu_N} \bg\circ\tbpsi_0(\bz))^{-1} \bJ_{\bu_T}\bg\circ\tbpsi_0(\bz),\quad\quad
\end{equation}
and
\begin{equation}
    \{\bu: (\bu_N,\bu_\bT) \in\Ball^{r_K}_{\tau_N}(\bu_{0,N}) \times\Ball^{m-r_k}_{\tau_T}(\bu_{0,T}) :\bg(\bu)=\bzero \} = 
    \{
    \tbpsi_0(\bz)
    : \bz \in\Ball_{\tau_T}^{m-r_k}(\bu_{0,T})\}.
\end{equation} 
\item we have
\begin{equation}
\label{eq:hess_psi_ub}
\bJ_{\bz} \bpsi_0(\bu_{0,T}) = \bzero,\quad\quad
\textrm{and}
    \quad\quad
    \sup_{\bz\in\Ball_{\tau_T}^{m-r_k}(\bu_{0,T})}
    \|\big(\grad^2 \bpsi_{0,j}(\bz)\big)_{j\in[r_k]}\|_\op \le  16 \frac{(\sfA_{\bG,1}^\up{\tau})^2 \sfA_{\bG,2}^\up{\tau}}{\sfsigma_{\bG}^3}.
\end{equation}
\end{enumerate}
\end{corollary}
\begin{proof}
The statement is largely a corollary of Lemma~\ref{lem:implicit-function} applied to $(\bu_N,\bu_T) \mapsto \bg(\bU [\bu_N^\sT,\bu_\bT^\sT]^\sT])$.

Indeed, the conditions of Lemma~\ref{lem:implicit-function} are verified by Lemma~\ref{lem:tangentional-decomposition}: we have 
$\bJ_{\bu_T}\bg(\bu_0)=\bzero$,
and by the choice of $\tau_N$,
\begin{equation}
\sup_{\bu\in\Ball_{\tau_N}^m(\bu_0)}  \|(\grad^2_{\bu_N,\bu_T} \bg_j(\bu))_{j\in[r_k]}\|_\op = 
\sup_{\bu\in\Ball_{\tau_N}^m(\bu_0)}  \|(\grad^2_{\bu} \bg_j(\bu))_{j\in[r_k]}\|_\op \le  \sfA_{\bG,2}^{\up{\tau}} \le  \frac{\sfsigma_{\bG}}{2 \tau_N}
\stackrel{(a)}{\le} \frac1{\tau_N} \sigma_{\min}(\bJ_{\bu_N}\bg(\bu_0)),
\end{equation}
where $(a)$ follows by the lower bound in Lemma~\ref{lem:tangentional-decomposition}.
Furthermore, $\bJ_{\bu_N}\bg(\bu_0)$ is non-singular since $\bu_0\in\cM$.

So Lemma~\ref{lem:implicit-function} holds and gives the statement of the lemma (after a change of variables), except for the display~\eqref{eq:hess_psi_ub}.
The equality in this display holds by
\begin{equation}
\bJ_{\bz} \bpsi_0(\bu_{0,T})  = -(\bJ_{\bu_N} \bg(\bu_0))^{-1} \bJ_{\bu_T}\bg(\bu_0) \stackrel{(a)}{=} \bzero
\end{equation}
where $(a)$ is by Lemma~\ref{lem:tangentional-decomposition}.
For the bound in Eq.~\eqref{eq:hess_psi_ub}, we differentiate 
$\bJ_{\bz} \bpsi_0(\bz)  = -(\bJ_{\bu_N} \bg\circ\tbpsi_0(\bz))^{-1} \bJ_{\bu_T}\bg\circ\tbpsi_0(\bz)$ element-wise. 
Namely, for $i \in[r_k], j\in[m-r_k]$, this can be written as
\begin{equation}
            (\grad_{\bu_N}g_i\circ \tbpsi_0(\bz))^\sT
            \left({\partial_{\bz_j}}
            \bpsi(\bz)\right) = 
            -\partial_{\bu_{T,j}}g_i\circ
            \tbpsi_0(\bz),
            \quad\quad\textrm{where}\quad\quad 
            \partial_{\bz_j}\bpsi(\bz) =   
            \left(\partial_{\bz_j}\psi_{l}(\bz) \right)_{l\in r_k} \in\R^{r_k}.
        \end{equation}
        Differentiating both sides now gives 
        %\kas{check transpose}
        %\bns{Check what you needed to check please}
    \begin{equation}    
       \bJ_\bz\left(\grad_{\bu_N}g_i\circ \tbpsi_0(\bz)\right)^\sT
       \left(\partial_{\bz_j}\bpsi(\bz)\right)
        +        
        \bJ_\bz \left(\partial_{\bz_j}
        \bpsi(\bz)\right)^\sT
        (\grad_{\bu_N}g_i\circ \tbpsi_0(\bz))
        =
       - \bJ_\bz \tbpsi_0(\bz)^\sT \grad_{\bu}
       \left( \partial_{\bu_{T,j}}g_i\circ
        \tbpsi_0(\bz)\right).
        \end{equation}
    By re-ordering this equation and letting $\bu = \tbpsi_0(\bz)$, we get
    \begin{align*}
    \bJ_\bz\left(\partial_{\bz_j} \bpsi(\bz)\right)^\sT (\grad_{\bu_N} g_i(\bu)) = -\bJ_\bz \tbpsi_0(\bz)^\sT\grad_\bu \left(\partial_{\bu_{T,j}}g_i(\bu)\right)
    - \bJ_\bz(\tbpsi_0(\bz))^\sT \bJ_\bu\left(\grad_{\bu_N}g_i(\bu)\right)^\sT(\partial_{\bz_j}\bpsi(\bz)).
    \end{align*}
    Next, by concatenating different values of $i$ and $j$,
    % \begin{align}
    %        \bJ_\bz\left(\partial_{\bz_j} \bpsi(\bz)\right)^\sT
    %        \left(\bJ_{\bu_N} \bg(\bu)\right)^\sT = 
    %        -\bJ_\bz \tbpsi_0(\bz)^\sT \bJ(\partial_{\bu_{T,j}} \bg(\bu))^\sT - \bJ_\bz\left(\tbpsi_0(\bz)\right)^\sT
    %        \left(\bJ_\bu(\grad_{\bu_N}g_i(\bu))^\sT
    %        \right)_{i\in[r_k]} 
    %        \left(\bI_{r_k} \otimes \partial_{\bz_j} \bpsi_0(\bz)\right),
    % \end{align}
    %and further by concatenating different values of $j$, 
    \begin{align*}
           &\left(\bJ_\bz\left(\partial_{\bz_j} \bpsi(\bz)\right)\right)_{j\in[m-r_k]}^\sT
           \left(\bI_{m-r_k}\otimes \bJ_{\bu_N} \bg(\bu)\right)^\sT = 
           -\bJ_\bz \tbpsi_0(\bz)^\sT
           \left(\bJ(\partial_{\bu_{T,j}} \bg(\bu))\right)_{j\in[m-r_k]}^\sT \\
           &\hspace{30mm}- \bJ_\bz\left(\tbpsi_0(\bz)\right)^\sT
           \left(\bJ_\bu(\grad_{\bu_N}g_i(\bu))
           \right)_{i\in[r_k]} ^\sT
           \left(\bI_{r_k} \otimes \partial_{\bz_j} \bpsi_0(\bz)^\sT\right)_{j\in[m-r_k]}^\sT.
    \end{align*}
     Each term in the equation above has a bounded operator norm. Namely, recall from Lemma \ref{lem:tangentional-decomposition} that since $\norm{\bu-\bu_0}\le \sqrt{\tau_T^2 +\tau_N^2} \le{\sfsigma_\bG}/{(2\sfA_{\bG,2}^\up{\tau})}\wedge\tau$, we have $\norm{\bJ_{\bu_N}\bg(\bu)^{-1}}_\op\le 2/{\sfsigma_\bG}$, and recall from Lemma \ref{lem:implicit-function} that
    \begin{equation}
          \norm{\bJ\bpsi_0(\bz)}_\op\le \norm{\bJ_{\bu_N}\bg(\bu)^{-1}}_\op\norm{\bJ_{\bu_T}\bg(\bu)}_\op\le \frac{2\sfA_{\bG,1}}{\sfsigma_{\bG,1}^\up{\tau}},\qquad
         \norm{\bJ \tbpsi_0(\bz)}_\op \le 1+ \norm{\bJ\bpsi_0(\bz)}_\op \le 1+  \frac{2\sfA_{\bG,1}^\up{\tau}}{\sfsigma_\bG}.
    \end{equation}
    Further, recall that by definition, on the set $\bu\in \Ball_\tau^m(\bu_0)$,
    \begin{equation}
        \norm{\left(\bJ(\partial_{\bu_{T,j}} \bg(\bu))\right)_{j\in[m-r_k]}}_\op = \norm{\left(\bJ_\bu(\grad_{\bu_N}g_i(\bu))
           \right)_{i\in[r_k]}}_\op 
           \le \sfA_{\bG,2}^\up{\tau}.
    \end{equation}
    Hence overall we can simplify the operator norm of the grand matrix of the second derivatives of $\bpsi$ to
    \begin{equation}
        \norm{\left(\bJ_\bz\left(\partial_{\bz_j} \bpsi_0(\bz)\right)\right)_{j\in[m-r_k]}}_\op = \norm{\big(\grad^2 \bpsi_{0,j}(\bz)\big)_{j\in[r_k]}}_\op 
           \le \frac2{\sfsigma_\bG}
           \frac{2\sfA_{\bG,1}^\up{\tau}\sfA_{\bG,2}^\up{\tau}}{\sfsigma_\bG}\left( 2+ \frac{2\sfA_{\bG,1}^\up{\tau}}{\sfsigma_\bG}\right)
           \le \frac{16(\sfA_{\bG,1}^\up{\tau})^2 \sfA_{\bG,2}^\up{\tau}}{\sfsigma_{\bG}^3}.
    \end{equation}

    
\end{proof}
\begin{lemma}[Global diffeomorphism]
\label{lemma:global_diff}
Fix $\tau >0$.
If $\beta$ satisfies
    \begin{equation}
        \beta\le 
        \frac{\sfsigma_\bG^3}{32(\sfA_{\bG,1}^{(\tau)})^2
        \sfA_{\bG,2}^{(\tau)}}
        \wedge
        \frac{\sfsigma_\bG}{4}\tau,
    \end{equation}
then the mapping $\bvarphi$ defined in Eq.~\eqref{eq:noraml-tube-chart} is a global diffeomorphism on the tube $\cT\left(\beta/(2\sfA_{\bG,1})\right)$. 
    %\kas{note $\sfA_{\bG,1}$ really needs to be only sup on the manifold cause I want this to get canceled later in the integral. Add the definition.}
\end{lemma}
\begin{proof} 
   Recall that by Lemma \ref{lem:local-diffeomorphis}, $\bvarphi$ is a local diffeomorphism on the tube $\cT\left(\beta/(2\sfA_{\bG,1})\right)$. To complete the proof, we need to show that it is also a bijection on this set.
    Let $(\bu_0,\bx_0),(\bu_1,\bx_1)\in \cT\left(\beta/(2\sfA_{\bG,1})\right)$, and assume there exists some $\bw\in\R^{m}$ such that
        \begin{align*}
       \bw &=
       \varphi(\bu_0, \bx_0)  =
       \varphi(\bu_1, \bx_1) \\
       &= 
       \bu_0+ \bJ_{\bu}\bg
       (\bu_0)^\sT\bx_0 =
       \bu_1+ \bJ_{\bu}\bg
       (\bu_1)^\sT\bx_1.
    \end{align*}
Then we have the bounds
    \begin{align}
        &\norm{\bu_1-\bu_0}_2 = \norm{\bJ_\bu \bg(\bu_0)^\sT\bx_0 - \bJ_\bu \bg(\bu_1)^\sT \bx_1}_2 \stackrel{(a)}\le       \sfA_{\bG,1}\left(\norm{\bx_0}_2+\norm{\bx_1}_2\right)\le \beta,
    \end{align}
since $\bu_0,\bu_1 \in\cM$ and by the choice of $\beta$, 
and similarly
\begin{align}
        &\norm{\bw-\bu_0}_2 = \|{\bJ_{\bu} \bg(\bu_0)^\sT\bx_0}\|_2
        \le \sfA_{\bG,1} \norm{\bx_0}_2\le \frac\beta2.
\end{align}
So $\bu_1\in\Ball_{\beta}^{m}(\bu_0)$, $\bw\in \Ball_\beta^{m}(\bu_0)$.
Now for any $\tau_N,\tau_T$ as in Eq.~\eqref{eq:tau_N_tau_T}, 
Corollary~\ref{cor:implicit-chart} furnishes the 
functions $\tbpsi_0$ and $\bpsi_0$, defined therein, that are smooth on the set $\Ball_{\tau_T}^{m-r_k}(\bu_{0,T})$.
Since $\sfA_{\bG,1}^\up{\tau},\sfA_{\bG,2}^\up{\tau} \ge1$ and $\sfsigma_{\bG}\le 1$ by definition, it is easy to check that for $\tau_T = \beta,$ there exists a choice of $\tau_N$  so that both satisfy Eq.~\eqref{eq:tau_N_tau_T}. So the functions $\bpsi_0, \tbpsi_0$ are hence defined and smooth on $\Ball_\beta^{m-r_k}(\bu_{0,T})$, $\tbpsi_0(\bu_{0,T}) = \bu_0$ and there exists $\bz_1 \in \Ball_{\beta}^{m-r_k}(\bu_{0,T}) $ such that $\tbpsi_0(\bz_1) = \bu_1$.
Defining the function
\begin{equation}
F_0(\bz) := \|{\tbpsi_0(\bz) - \bw}\|^2,
\end{equation}
we will show \textbf{(1.)} that both $\bz_1$ and $\bu_{0,T}$ are critical points of $\bF_0$, and \textbf{(2.)} that $\bF_0$ is strictly convex on $\Ball_{\beta}^{m-r_k}(\bu_{0,T})$. 
This will imply that $\bz_1 = \bu_{0,T}$, and hence that $\bu_1 = \tbpsi_0(\bz_1) = \tbpsi_0(\bu_{0,T}) = \bu_0$.

To verify \textbf{(1.)}, we use the identity for $\bJ_{\bz} \bpsi_0(\bz)$ derive in Corollary~\ref{cor:implicit-chart}. Namely,
    \begin{align*}
        \frac12\bJ_\bz F_0(\bz_1) =& 
        \bJ_\bz \tbpsi_0(\bz_1)^\sT(\tbpsi_0(\bz_1)-\bw)\\
        =&  \bJ_\bz(\bU(\bpsi_0(\bz_1),\bz_1))^\sT
        (\bJ_{\bu}
        \bg(\bu_1)^\sT\bx)\\
        =& [
            \bJ_\bz\bpsi_0(\bz_1)^\sT,\bI
        ]\bU^\sT
        (\bJ_{\bu}
        \bg(\bu_1)^\sT\bx)\\
        =&  [
            -(\bJ_{\bu_N}
            \bg(\bu_1)^{-1}
            \bJ_{\bu_T}
            \bg(\bu_1))^\sT,
            \bI]
        \bU^\sT\bU
        [
            \bJ_{\bu_N}
            \bg(\bu_1),
            \bJ_{\bu_T}
            \bg(\bu_1)
        ]^\sT\bx\\
        =& [
            -\bJ_{\bu_T}
            \bg{(\bu_1)^\sT
            (\bJ_{\bu_N}
            \bg(\bu_1)^{-1}})^\sT,
            \bI]
        \begin{bmatrix}
            \bJ_{\bu_N}
            \bg(\bu_1)^\sT\\ 
            \bJ_{\bu_T}
            \bg(\bu_1)^\sT
        \end{bmatrix}\bx=\bzero.
    \end{align*}
A similar calculation then shows that $\bJ_\bz\left(F_0{}\right) = \bzero$.

Now to prove \textbf{(2.)}, we compute the hessian $\grad_\bz^2F_0\in\R^{(m-r_k)\times(m-r_k)}$ as
     \begin{align} 
        \frac12\grad^2_\bz F_0(\bz) &=
         \bJ_\bz\tbpsi(\bz)^\sT\bJ_\bz\tbpsi(\bz)
         + \sum_{i=1}^{m}
        (\tbpsi_i(\bz)-\bw_i)\grad^2 \tbpsi_i(\bz)\\
        &\succeq \sigma_{\min}(\bJ_\bz\tbpsi(\bz))^2 - 
        \big\|(\grad^2 \tbpsi_i(\bz))_{i\in [m]}\big\|_\op\big\|\tbpsi(\bz)-\bw\big\|_2.
\nonumber
     \end{align}
     Both terms in the equation above can be further simplified using the definition of $\tbpsi$: namely, we have
$$\big\|(\grad^2\tbpsi_i(\bz))_{i\in [m]}\big\|_\op = \big\|(\grad^2\bpsi_j(\bz) )_{j\in[m-r_k]}\big\|_\op$$
and
     \begin{align}
             \bJ_\bz \tbpsi(\bz)^\sT\bJ_\bz \tbpsi(\bz) &= 
            \bI_{m-r_k}+
             \bJ_\bz\bpsi(\bz)^\sT\bJ_\bz\bpsi(\bz)^\sT
             \succeq 
             1-\norm{\bJ_\bz\bpsi(\bz)}_\op^2\\
             &\stackrel{(a)}\succeq 1 -
             \beta^2\hspace{-3mm} 
             \sup_{\bz\in \Ball_\beta^{m-r_k}(\bu_{0,T})}
             \norm{(\grad^2\bpsi_j(\bz) )_{j\in[m-r_k]}}_\op^2 ,
             \nonumber
     \end{align}
      where in $(a)$ we used $\bJ_\bz \bpsi(\bu_{0,T})=\bzero$ 
      and bounded the Jacobian at points near $\bu_{0,T}$ via a computation similar to that done in Lemma~\ref{lem:implicit-function}.
Finally, using Corollary \ref{cor:implicit-chart} to bound the second derivative and combining with the displays above gives
    \begin{equation}
        \frac12\grad^2_\bz F(\bz) \succeq 
        1 -\left( 8 \frac{(\sfA_{\bG,1}^\up{\tau})^2 \sfA_{\bG,2}^\up{\tau}}{\sfsigma_{\bG}^3}\beta\right)^2 -  8 \frac{(\sfA_{\bG,1}^\up{\tau})^2 \sfA_{\bG,2}^\up{\tau}}{\sfsigma_{\bG}^3}\beta \succ 0.
    \end{equation}
So $F$ is a strictly convex on the open convex set $\Ball_{\beta}^{m-r_k}(\bu_{0,T}).$

Finally, since $\bu_0 = \bu_1$, we have $\bJ_{\bu}\bg(\bu_0)^\sT \bx_0 =\bJ_{\bu}\bg(\bu_1)^\sT \bx_1.$ Then $\bx_0 = \bx_1$
follows from the fact that $\bJ_{\bu}\bg(\bu)$ is full rank on $\cM$.
\end{proof}


%\subsubsection{Local Diffeomorphism}
%\begin{lemma}[Projection onto the manifold]
%\label{lem:convex-projection}
%Fix arbitrary $\bu_0\in\cM\subset \R^m$ and $\tau\in(0,1)$ and work under 
%the definitions and assumptions of Corollary~\ref{cor:implicit-chart}.
%Let $\beta>0$ satisfy
%    \begin{equation}
%        \beta\le 
%        \frac{\sfsigma_\bG^3}{32(\sfA_{\bG,1}^{(\tau)})^2
%        \sfA_{\bG,2}^{(\tau)}}
%        \wedge
%        \frac{\sfsigma_\bG}{4}\tau.
%    \end{equation}
%For $\bw\in \cB^{m}_{\beta}(\bu_0)$, the function
%        $F_0(\bz):= \norm{\bpsi_0(\bz) -\bw}_2^2$
%    is strictly convex on the set $\cB^{m-r_k}_{\beta}(\bu_{0,T})$.
%\end{lemma}
%\begin{proof}
%For any $\tau_N,\tau_T$ as in Eq.~\eqref{eq:tau_N_tau_T}, 
%Corollary~\ref{cor:implicit-chart} furnishes the 
%functions $\tbpsi$ and $\bpsi$, defined therein, that are smooth on the set $\cB_{\tau_T}^{m-r_k}(\bu_{0,T})$.
%Since $\sfA_{\bG,1}^\up{\tau},\sfA_{\bG,2}^\up{\tau} \ge1$ and $\sfsigma_{\bG}\le 1$ by definition, it is easy to check that there exists a choice of $\tau_N,\tau_T$ satisfying Eq.~\eqref{eq:tau_N_tau_T}, so that $\beta \le \tau_T$. The function $\bpsi, \tbpsi$ are hence smooth on
%$\cB_\beta^{m-r_k}(\bu_{0,T})$.
%
%
%    This allows us to directly compute $\grad_\bz^2F_0\in\R^{(m-r_k)\times(m-r_k)}$ as
%     \begin{align} 
%        \frac12\grad^2_\bz F_0(\bz) =
%         \bJ_\bz\tbpsi(\bz)^\sT\bJ_\bz\tbpsi(\bz)
%         + \sum_{i=1}^{m}
%        (\tbpsi_i(\bz)-\bw_i)\grad^2 \tbpsi_i(\bz)
%        &\succeq \sigma_{\min}(\bJ_\bz\tbpsi(\bz))^2 - 
%        \big\|(\grad^2 \tbpsi_i(\bz))_{i\in [m]}\big\|_\op\big\|\tbpsi(\bz)-\bw\big\|_2.
%     \end{align}
%     Both terms in the equation above can be further simplified using the definition of $\tbpsi$: namely, we have
%$$\big\|(\grad^2\tbpsi_i(\bz))_{i\in [m]}\big\|_\op = \big\|(\grad^2\bpsi_j(\bz) )_{j\in[m-r_k]}\big\|_\op$$
%and
%     \begin{align}
%            & \bJ_\bz \tbpsi(\bz)^\sT\bJ_\bz \tbpsi(\bz) = 
%            \bI_{m-r_k}+
%             \bJ_\bz\bpsi(\bz)^\sT\bJ_\bz\bpsi(\bz)^\sT
%             \succeq 
%             1-\norm{\bJ_\bz\bpsi(\bz)}_\op^2
%             \stackrel{(a)}\succeq 1 -
%             \beta^2\hspace{-3mm} 
%             \sup_{\bz\in \cB_\beta^{m-r_k}(\bu_{0,T})}
%             \norm{(\grad^2\bpsi_j(\bz) )_{j\in[m-r_k]}}_\op^2 ,
%     \end{align}
%      where in (a) we used $\bJ_\bz \bpsi(\bu_{0,T})=\bzero$ 
%      and bounded the Jacobian at points near $\bu_{0,T}$ via a computation similar to that done in Lemma~\ref{lem:implicit-function}.
%Finally, using Corollary \ref{cor:implicit-chart} to bound the second derivative and combining with the displays above gives
%    \begin{equation}
%        \frac12\grad^2_\bz F(\bz) \succeq 
%        1 -\left( 8 \frac{(\sfA_{\bG,1}^\up{\tau})^2 \sfA_{\bG,2}^\up{\tau}}{\sfsigma_{\bG}^3}\beta\right)^2 -  8 \frac{(\sfA_{\bG,1}^\up{\tau})^2 \sfA_{\bG,2}^\up{\tau}}{\sfsigma_{\bG}^3}\beta \succ 0.
%    \end{equation}
%Hence, since the domain of $F$ is an open convex set, $F$ is a strictly convex.    
%\end{proof}
%
%\subsubsection{Global results from Local diffeomorphism}
%
%\begin{lemma}[Global bijection]
%    Consider the setting of lemma \ref{lem:convex-projection}.
%    Fix $\tau\in(0,1)$ and let $\beta>0$ satisfy the upper bound there.
%    The mapping $\bvarphi$ defined in Eq.\eqref{eq:noraml-tube-chart} is a global diffeomorphism on the tube $\cT\left(\beta/(2\sfA_{\bG,1})\right)$. 
%    %\kas{note $\sfA_{\bG,1}$ really needs to be only sup on the manifold cause I want this to get canceled later in the integral. Add the definition.}
%\end{lemma}
%\begin{proof}
%   Recall that by Lemma \ref{lem:local-diffeomorphis}, $\bvarphi$ is a local diffeomorphism on the tube $\cT\left(\beta/(2\sfA_{\bG,1})\right)$. To complete the proof, we need to show that it is also a bijection on this set.
%    Let $(\bu_0,\bx_0),(\bu_1,\bx_1)\in \cT\left(\beta/(2\sfA_{\bG,1})\right)$, and assume there exists some $\bw\in\R^{m}$ such that
%        \begin{align}
%       \bw &=
%       \varphi(\bu_0, \bx_0)  =
%       \varphi(\bu_1, \bx_1) \\
%       &= 
%       \bu_0+ \bJ_{\bu}\bg
%       (\bu_0)^\sT\bx_0 =
%       \bu_1+ \bJ_{\bu}\bg
%       (\bu_1)^\sT\bx_1.
%    \end{align}
%Then we have the bounds
%    \begin{align}
%        &\norm{\bu_1-\bu_0}_2 = \norm{\bJ_\bu \bg(\bu_0)^\sT\bx_0 - \bJ_\bu \bg(\bu_1)^\sT \bx_1}_2 \stackrel{(a)}\le       \sfA_{\bG,1}\left(\norm{\bx_0}_2+\norm{\bx_1}_2\right)\le \beta,
%    \end{align}
%since $\bu_0,\bu_1 \in\cM$ and by the choice of $\beta$, 
%and similarly
%\begin{align}
%        &\norm{\bw-\bu_0}_2 = \|{\bJ_{\bu} \bg(\bu_0)^\sT\bx_0}\|_2
%        \le \sfA_{\bG,1} \norm{\bx_0}_2\le \frac\beta2.
%\end{align}
%So $\bu_1\in\cB_{\beta}^{m}(\bu_0)$, $\bw\in \cB_\beta^{m}(\bu_0)$.
%Applying Corollary \ref{cor:implicit-chart} at $\bu_0$ for 
% 
%    \kas{we can apply implicit theorem since $\beta<\tau_N$ and $\beta\le \tau_T$ at the same time}     
%But by Lemma~\ref{lem:convex-projection}, the function
%$F_0(\bz) = \norm{\tbpsi(\bz) - \bw}^2$ is strictly convex on the set $\cB_\beta^{m-r_k}(\bu_{0,T})$
%
%    Further, for the given $\beta$ and $\bu_0$, by Corollary \ref{cor:implicit-chart}, there exists $\bz_1\in\cB^{m-r_k}_\beta(\bu_{0,T})$ such that $\bu_1 = \tbpsi(\bz_1)$. Hence around the point $\bu_0$ for the given radius $\beta$, Lemma \ref{lem:convex-projection} implies that the function 
%    $F_0(\bz) = \norm{\tbpsi(\bz) - \bw}^2$ is strictly convex on the set $\cB_\beta^{m-r_k}(\bu_{0,T})$.  We show that both $\bu_{0,T}$ and $\bz_1$ are critical points of this function, and therefore should be equal. Write
%    \begin{align}
%        \frac12\bJ_\bz F_0(\bz_1) =& 
%        \bJ_\bz \tbpsi(\bz_1)^\sT(\tbpsi(\bz_1)-\bw)\\
%        =&  \bJ_\bz(\bU(\bpsi(\bz_1),\bz_1))^\sT
%        (\bJ_{\bu}
%        \bg(\bu_1)^\sT\bx)\\
%        =& \begin{bmatrix}
%            \bJ_\bz\bpsi(\bz_1)^\sT,\bI
%        \end{bmatrix} \bU^\sT
%        (\bJ_{\bu}
%        \bg(\bu_1)^\sT\bx)\\
%        =&  \begin{bmatrix}
%            -(\bJ_{\bu_N}
%            \bg(\bu_1)^{-1}
%            \bJ_{\bu_T}
%            \bg(\bu_1))^\sT&
%            \bI
%        \end{bmatrix}\bU^\sT\bU
%        \begin{bmatrix}
%            \bJ_{\bu_N}
%            \bg(\bu_1)& 
%            \bJ_{\bu_T}
%            \bg(\bu_1)
%        \end{bmatrix}^\sT\bx\\
%        =& \begin{bmatrix}
%            -\bJ_{\bu_T}
%            \bg{(\bu_1)^\sT
%            (\bJ_{\bu_N}
%            \bg(\bu_1)^{-1}})^\sT&
%            \bI
%        \end{bmatrix}
%        \begin{bmatrix}
%            \bJ_{\bu_N}
%            \bg(\bu_1)^\sT\\ 
%            \bJ_{\bu_T}
%            \bg(\bu_1)^\sT
%        \end{bmatrix}\bx=\bzero,
%    \end{align}
%    Equivalently, $\bJ_\bz\left(\norm{\tbpsi(\bu_{0,T})-\bw}^2\right) = \bzero$.
%    This implies that $\bu_1=\bu_0$, and since $\bJ_\bu \bg(\bu_0)^\sT$ and $\bJ_\bu \bg(\bu_0)^\sT$ are both full rank we further get $\bx_0=\bx_1$. Hence, $\bvarphi$ is a bijection on the set $\cT\left(\frac{\beta}{2\sfA_{\bG,1}}\right)$.
%    
%\end{proof}
%
%
%\begin{lemma}\label{lemma:local-radius}
%    Consider the assumptions and setting of the Lemma \ref{lem:convex-projection}, and further assume that $\beta < \tau$.
%    Then, the mapping $\bvarphi$ defined in Eq.~\eqref{eq:noraml-tube-chart} is a bijection on the set 
%    \begin{equation}
%        \cN_\beta(\bu_0):=\left\{(\bu,\bx)\in \cT\bigg(\frac{\beta}{2\sfA_{\bG,1}^{(\tau)}}\bigg): \norm{\bu - 
%        \bu_0}_2 < \frac\beta2\right\}.
%    \end{equation}
%\end{lemma}
%\begin{proof}
%    Let $(\bu_1,\bx_1),
%    (\bu_2,\bx_2)\in \cN_\beta(\bu_0)$, and assume there exists some $\bw\in\R^{m}$
%
%    
%    Note for the given $\beta$ and $\bu_0$, by Corollary \ref{cor:implicit-chart}, there exists $\bz_1,\bz_2\in\cB^{m-r_k}_\beta(\bu_{0,T})$ such that $\bu_1 = \tbpsi(\bz_1)$ and $\bu_2 = \tbpsi(\bz_2)$. Hence,
%    \begin{align}
%       \norm{\bu_0-\bw}\le  \norm{\bu_1-\bw} + \norm{\bu_1-\bu_0} &\le
%        \norm{\bJ_{\bu}\bg(\bu_1)^\sT\bx_1} +\frac\beta2
%        \stackrel{(a)}\le 
%        \sfA_{\bG,1}^{(\tau)} \norm{\bx_1}_2 +\frac\beta2\le 
%        \beta,
%    \end{align}
%    where in (a) we used $\norm{\bu_1-\bu_0}_2\le\tau$. Hence for the given $\bu_0$, $\beta$, and $\bw$, Lemma \ref{lem:convex-projection} implies that the function $F_0(\bz) = \|{\tbpsi(\bz) - \bw}\|^2$ is strictly convex on the set $\cB^{m-r_k}_{\beta}(\bu_{0,T})$. We show that both $\bz_1$ and $\bz_2$ are critical points of this function, and therefore should be equal. Write
%    \begin{align}
%        \frac12\bJ_\bz F_0(\bz_1) =& 
%        \bJ_\bz \tbpsi(\bz_1)^\sT(\tbpsi(\bz_1)-\bw)\\
%        =&  \bJ_\bz(\bU(\bpsi(\bz_1),\bz_1))^\sT
%        (\bJ_{\bu}
%        \bg(\bu_1)^\sT\bx)\\
%        =& \begin{bmatrix}
%            \bJ_\bz\bpsi(\bz_1)^\sT,\bI
%        \end{bmatrix} \bU^\sT
%        (\bJ_{\bu}
%        \bg(\bu_1)^\sT\bx)\\
%        =&  \begin{bmatrix}
%            -(\bJ_{\bu_N}
%            \bg(\bu_1)^{-1}
%            \bJ_{\bu_T}
%            \bg(\bu_1))^\sT&
%            \bI
%        \end{bmatrix}\bU^\sT\bU
%        \begin{bmatrix}
%            \bJ_{\bu_N}
%            \bg(\bu_1)& 
%            \bJ_{\bu_T}
%            \bg(\bu_1)
%        \end{bmatrix}^\sT\bx\\
%        =& \begin{bmatrix}
%            -\bJ_{\bu_T}
%            \bg{(\bu_1)^\sT
%            (\bJ_{\bu_N}
%            \bg(\bu_1)^{-1}})^\sT&
%            \bI
%        \end{bmatrix}
%        \begin{bmatrix}
%            \bJ_{\bu_N}
%            \bg(\bu_1)^\sT\\ 
%            \bJ_{\bu_T}
%            \bg(\bu_1)^\sT
%        \end{bmatrix}\bx=\bzero,
%    \end{align}
%    Equivalently, $\bJ_\bz\left(\norm{\tbpsi(\bz_2)-\bw}^2\right) = \bzero$.
%    This implies that $\bu_1=\bu_2$, and hence $\bvarphi$ is a bijection on the set $\bN_\eps(\bu_0)$.
%    
%\end{proof}
%
%Now we have everything to show that $\bvarphi$ is a global diffeomorphism:
%
%
%\begin{proof}[Proof of lemma \ref{lem:global-radius}]
%    By lemma \ref{lem:local-diffeomorphis}, we know that for the given $\beta$, $\bvarphi$ is a local diffeomorphism for $\cT(\beta)$.
%    To complete the proof, we must show that $\bvarphi$ is a bijection on $\cT(\beta)$. By contradiction, assume that there exists 
%    $(\bu_1,\bx_1),(\bu_2,\bx_2)\in \cT(\beta)$ such that
%    \begin{equation}
%        \bu_1
%        +\bJ_{\bu}\bg(\bu_1)^\sT\bx =
%        \bu_2 + \bJ_{\bu}\bg(\bu_2)^\sT\bx.
%    \end{equation}
%    This implies that
%    \begin{equation}
%        \norm{\bu_1-
%        \bu_2} = \norm{\bJ_\bu\bg(\bu_1)^\sT\bx_1-\bJ_{\bu}\bg(\bu_2)^\sT\bx_2}\le \sup_{(\bu)\in\cM}
%        \norm{\bJ_{\bu}\bg(\bu)}_\op (\norm{\bx_1}+\norm{\bx_2})\le 2\sfD \eps_\cM .
%    \end{equation}
%    However, for the given $\eps_\cM$, we can use lemma \ref{lemma:local-radius}, this time around the point $\bu_1$, to conclude that $\bvarphi$ is a bijection on the set 
%    \begin{equation}
%        \{(\bu,\bx)\in \cT(2\eps_\cM): 
%        \norm{\bu_1-\bu}\le 2\sfD \eps_\cM\},
%    \end{equation}
%    which is a contradiction. This completes the proof.   
%\end{proof}
%

\subsubsection{Integration over the manifold }
\begin{lemma}[Manifold integral lemma]
\label{lem:intg-tube} 
Let $f:\cM\rightarrow \R$ be a differentiable nonnegative function. Assume
$\beta$ satisfies
    \begin{equation}
        \beta\le 
        \sup_{t > 0} \left\{
        \frac{\sfsigma_\bG^3}{64(\sfA_{\bG,1}^{(t)})^2
        \sfA_{\bG,2}^{(t)}}
        \wedge
        \frac{\sfsigma_\bG}{8}t\right\}.
    \end{equation}
Then
    \begin{equation}
        \int_{\bu\in \cM} f(\bu) \de_\cM V
        \le  
        \Err_{\textrm{blow-up}}(n,\beta) 
\exp\{\beta\;\norm{\log f}_{\Lip,\cM^{(\beta)}}\}
        \int_{\by\in \cM^{(\beta)}} f(\by)\de\by
    \end{equation}
    where
    \begin{equation}
       \Err_{\textrm{blow-up}}(n,\beta)  :=\left(\frac1{1 - \beta\;\sfA_{\bG,2}/\sfA_{\bG,1}}\right)^{(m_n-r_k)/2}
        \left(\frac{\sqrt{r_k} \sfA_{\bG,1}}{\beta(\sfsigma_{\bG} - \beta \sfA_{\bG,2}/\sfA_{\bG,1})}\right)^{r_k}
    \end{equation}
\end{lemma}
\begin{proof}
Fix $\beta$ to satisfy the condition of the lemma and let $\tbeta := \beta /\sfA_{\bG,1}.$
Recall the definition of Eq.~\eqref{eq:tube_def}. We can upper bound the integral of interest by an integral over $\cT(\tbeta)$ as
    \begin{align}
        \int_{\bu\in\cM} f(\bu)\de_\cM V& \stackrel{(a)}\le 
        \left(\frac{\sqrt{r_k}}{\tbeta}\right)^{r_k}
        \int_{\bu\in\cM} \int_{\bx\in\R^{{r_k}}} 
        f(\bu)\one_{\{\norm{\bx}_2\le \tbeta\}} \de_\cM V\de\bx
         = \left(\frac{\sqrt{r_k}}{\tbeta}\right)^{r_k} 
        \int_{(\bu,\bx)\in \cT(\tbeta)} f(\bu) \de_{\cM\times\R^{r_k}} V,
        \label{eq:manifold_int_lemma_bound_1}
    \end{align}
   where $\de_{\cM\times\R^{r_k}} V$ is the volume element of $\cM\times \R^{r_k},$
  and in $(a)$ we used $\vol(\Ball^{r_k}_{\tbeta}(\bzero))^{-1} \le \left(\sqrt{r_k}/{\tbeta}\right)^{r_k}$.

By Lemma~\ref{lemma:global_diff}, $\varphi$ is a global diffeomorphism on $\cT(\tbeta)$ for the choice of $\tbeta$. Then using the uniform lower bound on
$|\det(\de\bvarphi^{-1}(\bx))| = |\det  \left(\de\bvarphi(\bx)\right)^{-1}|$ of Lemma~\ref{lem:local-diffeomorphis}, we can bound the integral over $\cT(\tbeta)$ in the previous display as
  \begin{align}
  \nonumber
        \int_{(\bu,\bx)\in \cT(\tbeta)} f(\bu) \de_{\cM\times\R^{r_k}} V
        & \stackrel{(b)}{=} 
        \int_{\by\in\bvarphi(\cT(\tbeta))} f(\tbpi_\cM(\by)) \big| 
        \det\left( \de\bvarphi^{-1}(\by) \right)\big|\de\by\\
        &{\le}
        \left(\frac1{1 - \tbeta\;\sfA_{\bG,2}}\right)^{(m-r_k)/2}
        \left(\frac1{\sfsigma_{\bG} - \tbeta \sfA_{\bG,2}}\right)^{r_k}
        \int_{\by\in\bvarphi(\cT(\tbeta))} f(\tbpi_\cM(\by)) \de\by.
        \label{eq:manifold_int_lemma_bound_2}
  \end{align}
Now for any  $\by\in \bvarphi(\cT(\tbeta))$,
since $\bvarphi$ is bijective on this set, there exists a unique $(\bu,\bx) \in \cT(\tbeta)$ such that $\by = \bvarphi(\bu,\bx) = \bu + \bJ_{\bu}\bg(\bu)^\sT\bx$. From this we see that
    \begin{align}
        &\norm{\by - \bu}_2 = 
        \|{\bJ_{\bu}
        \bg(\bu)^\sT\bx}\|_2
        \le \|{\bJ_{\bu}
        \bg(\bu)}\|_\op\norm{\bx}\le
        \sfA_{\bG,1} \tbeta = \beta, \quad\quad\textrm{and similarly}\quad\quad
        \norm{\by - \tbpi_\cM(\by)} 
        \le \beta.
    \end{align}
So
\begin{equation}
 \bvarphi(\cT(\tbeta))\subseteq\cM^{(\beta)}\quad\quad\textrm{and}\quad\quad
   f(\tbpi_{\cM}(\by)) \le \exp \left\{\|\log f\|_{{\Lip},\cM^{(\beta)}}\beta\right\} f(\by).
\end{equation}
These along with the nonnegativity of $f$ now give the bound
\begin{equation}
        \int_{\by\in\bvarphi(\cT(\tbeta))} f(\tbpi_\cM(\by)) \de\by \le
        \exp \left\{\|\log f\|_{{\Lip},\cM^{(\beta)}}\beta\right\}
        \int_{\by\in \cM^{(\beta)}} f(\by)\de\by.
\end{equation}
Combining with the bounds in Eq.~\eqref{eq:manifold_int_lemma_bound_1} and
Eq.~\eqref{eq:manifold_int_lemma_bound_2} yields the lemma.
\end{proof}



\subsubsection{Proof of Lemma~\ref{lemma:manifold_integral}}

By computing derivatives of the component of $g_j$ for $j \in[r_k]$ and using the 
Lipschitz assumptions of Assumption~\ref{ass:loss}
on the partials of $\ell$ and the continuity assumptions of Assumption~\ref{ass:regularizer} on the partials of $\rho$, one can check that there exists a constant $C_0(\sfA_{\bV},\sfA_{\bR})>0$ 
depending only on $\sfA_{\bV},\sfA_{\bR}$, such that
\begin{equation}
    \sup_{\bu \in\cM^\up{1}}\max_{j \in [r_k]}\|\grad g_{j}(\bu)\|_2 
   \vee  
    \sup_{\bu \in\cM^\up{1}}\max_{j \in [r_k]}\|\grad^2 g_{j}(\bu)\|_\op
    \le  C_0(\sfA_{\bV},\sfA_{\bR})\; r_k.
\end{equation}
So by definition of $\sfA_{\bG,1}^\up{t},\sfA_{\bG,2}^\up{t}$, 
\begin{equation}
   1 \le \sfA_{\bG,1}\vee \sfA_{\bG,2} \le \sfA_{\bG,1}^{\up{1}}
 \vee \sfA_{\bG,2}^{\up{1}} \le  r_k^2 \; C_0(\sfA_\bV,\sfA_\bR).
\end{equation}
And so there exists some constant $C_1 = C_1(\sfA_{\bV},\sfA_{\bR})> 0$ such that if 
\begin{equation}
   \beta \le \frac{C_1(\sfA_{\bV},\sfA_{\bR}) \sfsigma_{\bG}^3}{r_k^6},
\end{equation}
then $\beta$ satisfies the upper bound of Lemma~\ref{lemma:manifold_integral}.
\qed

\subsection{Bounding the density of the gradient process: Proof of Lemma~\ref{lemma:density_bounds}}
\label{sec:density_bound}

We begin by computing the density $p_{\bTheta,\bbV}(\bzero)$ appearing in Lemma~\ref{lemma:kac_rice_manifold}.

\begin{lemma}[Density] 
\label{lemma:density}
Let $\bB_{\bSigma}$ be as in Corollary~\ref{cor:proj}.
For $(\bTheta,\bbV)\in \cM$, the density of $\bz(\bTheta,\bbV) = \bB_{\bSigma}(\bTheta,\bV))^\sT\bzeta(\bTheta,\bbV)$ at $\bzero$ is given by 
\begin{equation}
\label{eq:density}
  p_{\bTheta,\bbV}(\bzero) 
    :=
\frac{
    \exp\left\{-\frac{1}2\left(
    n^2\Tr\left(\bRho (\bL^\sT\bL)^{-1}\bRho\right) + \Tr(\bbV \bR^{-1}\bbV) + n\Tr\left(\bRho (\bL^\sT\bL)^{-1}\bL^\sT \bbV \bR^{-1}(\bTheta,\bTheta_0)^\sT\right)
%    \Tr\left(\bbV
%\bR^{-1}(\bTheta)
%    \bbV^\sT\right)
    \right)\right\}}
    {
\det^*(2\pi\bSigma(\bTheta,\bbV))^{1/2}
    }
\end{equation}
%\begin{equation}
%\label{eq:density}
%  p_{\bTheta,\bbV}(\bzero) 
%    :=
%%\frac{
%%\det^*(\bSigma(\bTheta,\bbV))^{-1/2}
%%\det\left( \bR(\bTheta) \right)^{-n/2}\det\left(\bL^\sT\bL\right)^{-d/2}
%%}{(2 \pi)^{(dk + (n-k)(k+k_0))/2}}
%\det^*(2\pi\bSigma(\bTheta,\bbV))^{-1/2}
%    \exp\left\{-\frac12\left(
%\Tr\left(\bRho (\bL^\sT\bL)^{-1}\bRho\right) + \Tr(\bbV \bR^{-1}\bbV) + \Tr\left(\bRho (\bL^\sT\bL)^{-1}\bL^\sT \bbV \bR^{-1}(\bTheta,\bTheta_0)^\sT\right)
%%    \Tr\left(\bbV
%%\bR^{-1}(\bTheta)
%%    \bbV^\sT\right)
%    \right)\right\}
%\end{equation}
where $\det^*$ denotes the product of the non-zero eigenvalues.
\end{lemma}
\begin{proof}
In what follows, let us suppress the argument $(\bTheta,\bbV)$ throughout.
Recall Lemma~\ref{lemma:eig_vecs_NS_Sigma} giving the mean and covariance of $\bzeta$.
What we need to show is that the quantity multiplying the factor $-1/2$ in the exponent of~\eqref{eq:density} is equal to 
$\bmu^\sT\bB (\bB^\sT\bSigma\bB)^{-1}\bB^\sT \bmu =
\bmu^\sT\bSigma^\dagger \bmu$. This fact follows from straightforward  (albeit tedious) algebra after applying the stationary condition $\bG = \bzero$. 

Indeed, to see this, let $\ba = (\ba_1^\sT ,\ba_2^\sT)^\sT$ where $\ba_1 \in\R^{dk}$, $\ba_2 \in\R^{n(k+k_0)}$, such that
$\bSigma^\dagger \bmu = \ba.$ 
Since $\bmu$ is orthogonal to the null space of $\bSigma$, we must have $\bSigma \ba = \bmu$, and hence
\begin{align}
\label{eq:pinv_lin_eq}
    \left(\bL^\sT \bL \otimes \bI_d \right)\ba_1 &+ [\bM, \bM_0] \ba_2 
    =  \overline\brho\\
    [\bM,\bM_0]^\sT \ba_1 &+(\bR \otimes \bI_n) \ba_2   = -\overline \bv
\end{align}
where $\overline \bv \in \R^{n(k+k_0)}$ and $\overline \brho\in\R^{dk}$ denotes the concatenation of the columns of $\bbV$ and $n\bRho$, respectively. 
Solving the for $\ba_1$ in terms of $\ba_2$, and vice-versa for the second equation allows us to conclude that 
\begin{align}
\label{eq:muTa}
   \bmu^\sT\bSigma^\dagger \bmu = \bmu^\sT\ba &=  \overline\brho^\sT(\bL^\sT\bL \otimes \bI_d)^{-1}\overline\brho + \overline\bv^\sT(\bR\otimes \bI_n)^{-1}\overline \bv 
   \underbrace{-\overline \brho^\sT(\bL^\sT\bL \otimes \bI_d)^{-1} [\bM,\bM_0]\ba_2}_{=:\textrm{(I)}}\\
   &\quad+ \underbrace{\overline\bv^\sT(\bR\otimes \bI_n)^{-1}[\bM,\bM_0]^\sT \ba_1}_{=:\textrm{(II)}}.
   \nonumber
\end{align}
Now write
\begin{align*}
  \textrm{(I)}  &\stackrel{(a)}{=}-\overline \btheta^\sT\left(\bI_k \otimes \bRho (\bL^\sT\bL)^{-1}\bL\right)\ba_2\\
  &\stackrel{(b)}=
   \overline\bv^\sT \left(\bI_k \otimes  \bL(\bL^\sT \bL)^{-1}\bL^{\sT}\right)
\ba_2\\ 
&=  \overline\bv^\sT(\bR\otimes \bI_n)^{-1}
\left(\bR \otimes \bL(\bL^\sT \bL)^{-1}\bL^{\sT}\right)
\ba_2\\
&\stackrel{(c)}{=}   \overline\bv^\sT(\bR\otimes \bI_n)^{-1}[\bM,\bM_0]^\sT (\bL^\sT\bL \otimes \bI_d)^{-1}[\bM,\bM_0]\ba_2\\
&\stackrel{(d)}{=} -\textrm{(II)} + \overline\bv^\sT (\bR \otimes \bI_n)^{-1} [\bM,\bM_0]^\sT (\bL^\sT\bL \otimes \bI_d)^{-1} \overline\brho,
\end{align*}
where in $(a)$ we used $\overline\btheta\in\R^{d(k+k_0)}$ to denote the concatenation of the columns of $(\bTheta,\bTheta_0)$, in $(b)$ we used the constraint $\bG(\bbV,\bTheta) =\bzero$, in $(c)$ we used  the identity (easily verifiable directly from the definitions)
\begin{equation}
   [\bM,\bM_0]^\sT(\bL^\sT\bL \otimes \bI_d)^{-1}[\bM,\bM_0] = \bR\otimes \bL(\bL^\sT\bL)^{-1}\bL^\sT,
\end{equation}
and in $(d)$ we used Eq.~\eqref{eq:pinv_lin_eq} to write $\ba_1$ appearing in \textrm{(II)} in terms of $\ba_2.$
Combining with Eq.~\eqref{eq:muTa} we conclude that
\begin{equation}
\bmu^\sT \bSigma^\dagger \bmu  = n^2 \Tr\left(\bRho (\bL^\sT\bL)^{-1}\bRho\right) + \Tr(\bbV \bR^{-1}\bbV) + n\Tr\left(\bRho (\bL^\sT\bL)^{-1}\bL^\sT \bbV \bR^{-1}(\bTheta,\bTheta_0)^\sT\right).
\end{equation}
\end{proof}

Next, the following lemma bound the pseudo determinant term appearing in the expression for the density above.
%This section gives a bound on the density term of Lemma~\ref{lemma:density}.
%First we deal with the covariance.
%Recall the definition of $\bSigma(\bTheta,\bbV)$ in Lemma~\ref{lemma:mean_cov}.
%Asymptotically, one expects that the psuedo-determinant term appearing in the density in Lemma~\ref{lemma:density} asymptotically satisfies $\det^*(\bSigma(\bTheta,\bbV)) \asymp \det(\bR(\bTheta))^{n} \det(\bL(\bbV)^\sT\bL(\bbV))^d$ since it's a rank $r_k$ perturbation away from a block diagonal matrix whose determinant is the latter quantity. The following lemma gives a bound to this effect.
\begin{lemma}[Bounding the determinant of the covariance]
\label{lemma:det_star_bound}
Let $r_k := k^2 + k_0 k$.
Under the assumptions of Section~\ref{sec:assumptions},
for any $(\bTheta,\bbV)  \in \cM(\cuA,\cuB)$, we have
   \begin{equation}
       \det^* \left(\bSigma(\bTheta,\bbV))\right)^{-1/2} \le \det(\bR(\bTheta))^{-n/2} \det(\bL(\bbV)^{\sT}\bL(\bbV))^{-(d-r_k)/2}
   \end{equation}
\end{lemma}
\begin{proof}
We'll suppress the index $(\bTheta,\bbV)\in\cM$ throughout the proof.
   Recall the definition of $\bSigma$.
   Let $\bM_1 := [\bM,\bM_0]\in\R^{dk \times (nk + nk_0)}$ where $\bM,\bM_0$ are the off diagonal blocks of $\bSigma$ defined in that lemma.
For any $\eps >0$, we have
\begin{align*}
   \det\left(\bSigma + \eps\bI\right)  &\ge \det\left(
   \begin{pmatrix}
       \bL^\sT \bL \otimes \bI_{d} + \eps \bI_{kd}  & \bM_1 \\
       \bM_1^\sT  & \bR \otimes \bI_n 
   \end{pmatrix}
   \right)\\
&=\det\left(\bR \otimes \bI_n\right) 
\det\left(
\left(\bL^\sT\bL + \eps \bI_{k}\right)\otimes \bI_d - \bM_1 \left(\bR^{-1}\otimes \bI_n\right) \bM_1^\sT
\right).
\end{align*}
With some algebra, one can show that
\begin{equation}
   \bM_1 \left(\bR^{-1}\otimes \bI_n\right) \bM_1^\sT = \bL^\sT\bL \otimes (\bTheta,\bTheta_0)\bR^{-1}(\bTheta,\bTheta_0)^\sT.
\end{equation}
Denoting the rank $r_k$ orthogonal projector $\bP_\bR :=
(\bTheta,\bTheta_0)\bR^{-1}(\bTheta,\bTheta_0)^\sT\in\R^{d\times d}$ and using $\bP_R^\perp$ for the complementary orthogonal projector, we can compute the second determinant term in the above display as
\begin{align*}
    \det\left(
\left(\bL^\sT\bL + \eps \bI_{k}\right)\otimes \bI_d - \bM_1\left(\bR\otimes \bI_n\right) \bM_1^\sT
\right) &=  \det\left( (\bL^\sT\bL + \eps \bI_{k})\otimes \bP_\bR^\perp +
(\bL^\sT\bL + \eps\bI_{k}) \otimes \bP_\bR - \bL^\sT\bL\otimes \bP_\bR
\right)\\
&=\det\left( (\bL^\sT\bL + \eps \bI_{k})\otimes \bP_\bR^\perp +
\eps\bI_{k} \otimes \bP_\bR
\right)\\
&= \det\left(\bL^\sT\bL+ \eps \bI_{k}\right)^{d-r_k} \eps^{r_k}.
\end{align*}
So we conclude that for any $\eps >0$,
\begin{equation}
   \det(\bSigma +\eps \bI) \ge \det(\bR)^n \det(\bL^\sT\bL + \eps\bI_k)^{d-r_k}  \eps^{r_k}.
\end{equation}
Using that the dimension of the nullspace of $\bSigma$ is $r_k$ by Lemma~\ref{lemma:eig_vecs_NS_Sigma}, we then have
\begin{align*}
    \det^*(\bSigma) &:= \lim_{\eps \to 0} \frac1{\eps^{r_k}} \det(\bSigma + \eps\bI)\\
    &\ge \det(\bR)^{n}  \det(\bL^\sT\bL )^{d-r_k}
\end{align*}
as claimed.
\end{proof}
\begin{proof}[Proof of Lemma~\ref{lemma:density_bounds}]
The proof is a direct corollary of Lemma~\ref{lemma:det_star_bound}.
Indeed, rewriting the expression for $p_{\bTheta,\bbV}(\bzero)$ from Lemma~\ref{lemma:density} in terms of $\hmu,\hnu$, and ignoring exponentially trivial factors for large enough $n$, we reach the statement of the lemma.
\end{proof}

Finally, for future reference, we record the following uniform bound on the density.
\begin{corollary}[Uniform bound on the density]
\label{cor:uniform_density_bound}
In the setting of Lemma~\ref{lemma:density_bounds}, we have the following uniform bound holding for all $(\bTheta,\bbV) \in\cM$:
\begin{equation}
    p_{\bTheta,\bbV}(\bzero) \le \frac{\sfsigma_{\bR}^{-nk/2} \sfsigma_{\bL}^{-(d-r_k)/2}}{ (2\pi)^{(dk + nk + nk_0 -r_k)/2} n^{dk/2}}
\end{equation}
\end{corollary}



%%%%%%%%%%%%%%%%%%%%%%%%%%%%%%%%%%%%%%%%%%%%%%%%%%%%%%%%%%%%%%%%%
%%%%%%%%%%%%%%%%%%%%%%%%%%%%%%%%%%%%%%%%%%%%%%%%%%%%%%%%%%%%%%%%%
%%%%%%%%%%%%%%%%%%%%%%%%%%%%%%%%%%%%%%%%%%%%%%%%%%%%%%%%%%%%%%%%%
%%%%%%%%%%%%%%%%%%%%%%%%%%%%%%%%%%%%%%%%%%%%%%%%%%%%%%%%%%%%%%%%%
%%%%%%%%%%%%%%%%%%%%%%%%%%%%%%%%%%%%%%%%%%%%%%%%%%%%%%%%%%%%%%%%%

\subsection{Analysis of the determinant: Proof of Lemma~\ref{lemma:CE_bound}}
\label{sec:determinant_bound}
\subsubsection{Relating the determinant of the differential to the determinant of the Hessian: Proof of Lemma~\ref{lemma:lb_singular_value_Df}}
%The objective of analysis in this section is the term
%$\E[|\det( \de\bz(\bTheta,\bbV) )| \one_{\bH(\bTheta,\bbV)  \succ n\sfsigma_\bH} | \bzeta =\bzero,\bw ]$, for $(\bTheta,\bbV) \in\cM(\cuA,\cuB)$.
%We will relate the determinant of $\de z(\bTheta,\bbV))$ by the determinant of $\bH(\bTheta,\bbV)$ by a perturbation argument. For this reason, we require a lowerbound on the smallest singular value of $\de \bz(\bTheta,\bbV)$. The following lemma, whose proof is largely algebraic, gives this lower bound.

%\begin{lemma}[Lower bound on the singular values of $\bJ_{(\bTheta,\bbV)} \bzeta $]
%\label{lemma:lb_singular_value_Df}
%%Let $\grad^2 \rho(\bTheta) \in\R^{dk \times dk}$ be the Hessian of $\rho$ viewed as a map $\rho : \R^{dk} \to \R$.
%Under Assumption~\ref{ass:loss} and~\ref{ass:regularizer}, we have the bound
%   \begin{equation}
%\sigma_{\min}(\bJ_{(\bTheta,\bbV)} \bzeta) \ge \;\frac{\sigma_{\min}(\bH(\bTheta,\bbV))}{\Err_{{\sigma}}(\bX^\sT\bX; \sfK,\tilde\sfK)}
%   \end{equation}
%   where the multiplicative error is given by
%\begin{equation}
%\Err_\sigma(\bX^\sT\bX;\sfK,\tilde\sfK) :=
%C(\sfK,\tilde\sfK, \sfA_\bR)
%\left(
%\|(\bX^\sT\bX)^{-1}\|_\op^{3/2} +1
%\right)
%\left(\|\bX^\sT\bX\|_\op^{7/2}+ n^3\right)
%%    E(\bX^\sT\bX;\sfK,\tilde\sfK) := C(\|\bX^\sT\bX\|_\op^{5/2} \|(\bX^\sT\bX)^{-1}\|_\op^{3/2} (\sfK^2 + \tilde{\sfK}^2 + 1 ))
%\end{equation}
%for some universal constant $C>0$ depending only on $\sfK,\tilde\sfK$.
%\end{lemma}
We suppress  the dependence on $(\bTheta,\bbV)$ in what follows.
In analogy with the definition of $\bSec$ in Eq.~\eqref{eq:SecDef},
define $\tilde{\bSec}$ as follows
%
\begin{equation}
   \tilde\bSec := \begin{pmatrix}
\tilde\bSec_{i,j}(\bbV)
   \end{pmatrix}_{i,j \in[k]}
,\quad
    \tilde\bSec_{i,j}:= \Diag\left\{\frac{\partial^2}{\partial {v_i}\partial {u_j}}\ell(\bbV)\right\}.\label{eq:TildeSecDef}
\end{equation}
We'll also introduce the notation
\begin{equation}
\bS := n\grad^2\rho(\bTheta)\quad\quad
    \tbH := (\bI_k\otimes\bX)^\sT \tilde\bSec (\bI_k \otimes \bX),
    \quad\quad
    \widehat\bSigma := (\bI_k\otimes\bX)^\sT(\bI_k\otimes\bX).
\end{equation}
Without loss of generality, we'll assume $\bS$ is invertible throughout (otherwise we can perturb $\bS$ and take the perturbation parameter to $0$).
A direct computation yields
    \begin{equation}
        \bJ \bzeta = \begin{pmatrix}
            \bS& (\bI_k\otimes\bX)^\sT\bSec &
            (\bI_k\otimes \bX)^\sT\tilde\bSec\\
            \bI_k\otimes\bX& -\bI&\bzero\\
            \bzero &\bzero &-\bI
        \end{pmatrix}.
    \end{equation}
We'd like our bounds to be in terms of the matrix $\bH$ which is interpreted as the second derivative at the minimizer, instead of the obtuse quantity $(\bI\otimes \bX)^\sT\bK^2 (\bI\otimes \bX).$ For this reason, we introduce the approximate isometry
%\bns{I will normalize by $\sqrt{n}$ at the end. Perhaps the correct way to do this is to normalize only $\bX$ by $\sqrt{n}$ and not $\bI$, but it turns out the bound is the same order regardless using this approach because we're using a crude bound on the operator norm of the inverse later.}
%\begin{equation}
%    \bN_0 := \begin{pmatrix}
%        \bI & \bzero  &  \bzero\\
%        \bzero & \frac1{\sqrt{n}}(\bI \otimes \bX) & \bzero\\
%        \bzero & \bzero & \frac1{\sqrt{n}} (\bI \otimes \bX).
%    \end{pmatrix}
%\end{equation}
\begin{equation}
    \bN_0 := \begin{pmatrix}
        \bI_{dk} & \bzero  &  \bzero\\
        \bzero & (\bI_k \otimes \bX) & \bzero\\
        \bzero & \bzero & (\bI_k \otimes \bX)
    \end{pmatrix}
\end{equation}
which satisfies $\norm{\bN_0}_\op \le \norm{\bX}_\op$.
Then, to lower bound $\sigma_{\min}(\bJ \bzeta),$ we will use the inequality
\begin{equation}
\label{eq:lb_isometry}
    \sigma_{\min}(\bJ \bzeta)\ge \norm{\bX}_\op^{-1} \sigma_{\min}(\bJ \bzeta\; \bN_0),
\end{equation}
and then bound $\sigma_{\min}(\bD \bzeta \bN_0)$ by 
bounding the operator norm of the inverse of $\bN_1 := (\bD\bzeta \bN_0)^\sT(\bD \bzeta \bN_0)$ and bounding its operator norm block-wise. 
This matrix $\bN_1$ can be straightforwardly computed to be
%\begin{equation}
%   \bN_1 := (\bD \bzeta(\bt) \bN_0)^{\sT}(\bD \bzeta(\bt) \bN_0)
%   := \begin{pmatrix}
%       \widehat \bSigma &  -\frac1{\sqrt{n}}\widehat\bSigma & \bzero\\
%    -\frac1{\sqrt{n}} \widehat\bSigma & \frac1n\bH^2 + \frac1n \widehat\bSigma & \frac1n \bH \tilde\bH^\sT\\
%    \bzero & \frac1n \tilde \bH\bH & \frac1n \tilde\bH \tilde\bH^\sT +\frac1n \widehat \bSigma
%   \end{pmatrix}.
%\end{equation}
\begin{equation}
   \bN_1 
   = \begin{pmatrix}
       \bS^2 + \widehat \bSigma &  \bS \bH -\widehat\bSigma & \bS \tbH\\
    \bH \bS - \widehat\bSigma & \bH^2 +  \widehat\bSigma &  \bH \tbH\\
    \tbH^\sT\bS &  \tbH^\sT\bH &  \tbH^\sT \tbH +\widehat \bSigma
   \end{pmatrix}.
\end{equation}

\noindent So letting
\begin{equation}    
\bA_0 := \bS^2 + \widehat \bSigma,\quad
\bB_0  :=  \begin{pmatrix}
   \bS\bH-  \widehat\bSigma &  \bS\tbH
\end{pmatrix},
\quad
\bC_0 :=\begin{pmatrix}
   \bH^2 + \widehat\bSigma &  \bH \tbH\\
     \tbH^\sT\bH & \tbH^\sT \tbH + \widehat \bSigma
\end{pmatrix},\quad
\bD_0 := \bC_0 - \bB^\sT_0 \bA_0^{-1} \bB_0,
\end{equation}
%\begin{align}
%(\bD_0 - \bC_0 \bA_0^{-1} \bB_0)^{-1} = 
%\begin{pmatrix}
%    \frac1n\bH^2  & \frac1n \bH \tilde\bH^\sT\\
%    \frac1n \tilde \bH\bH & \frac1n \tilde\bH \tilde\bH^\sT +\frac1n \widehat \bSigma
%\end{pmatrix}^{-1}
%=:
%\begin{pmatrix}
%    \bA_1 & \bB_1\\
%    \bC_1 & \bD_1\\
%\end{pmatrix}^{-1}.
%\end{align}
we note that since,
\begin{equation}
    \bN_1^{-1} = \begin{pmatrix}
        \bA_0^{-1} + \bA^{-1} \bB_0 \bD_0^{-1} \bB_0^\sT \bA^{-1} & 
- \bA^{-1} \bB_0 \bD_0^{-1}\\
-\bD_0^{-1} \bB_0^\sT \bA^{-1} &  \bD_0^{-1}
    \end{pmatrix},
\end{equation}
we can bound  
\begin{equation}
\label{eq:M1_norm_bound}
    \norm{\bN_1^{-1}}_\op \le \norm{\bA_0^{-1}}_\op  + \left(1 + \norm{\bA_0^{-1}}_\op\norm{\bB_0}_\op\right)^2\norm{\bD_0^{-1}}_\op.
\end{equation}

Let us first compute $\bD_0$, then bound the norm of its inverse. We'll explicate the computation only for the upper-left and lower-right blocks; for the former we have
\begin{align*}
   &\bH^2 + \widehat\bSigma - (\bH \bS - \widehat\bSigma)(\bS^2 + \widehat\bSigma)^{-1}(\bS \bH - \widehat\bSigma)\\
    &= \bH\left(\bI - (\bI + \bS^{-1}\widehat\bSigma\bS^{-1})^{-1}\right)\bH
    + \widehat\bSigma^{1/2}\left(\bI - (\bI + \widehat\bSigma^{-1/2}\bS^2 \widehat\bSigma^{-1/2})^{-1} \right)\widehat\bSigma^{1/2} + \bH\bS(\bS^2 + \widehat\bSigma)^{-1}\widehat\bSigma\\
 &\quad+\widehat\bSigma(\bS^2 + \widehat\bSigma)^{-1}\bS\bH
    \\
    &= \bH(\bI + \bS^{-1}\widehat\bSigma\bS^{-1})^{-1}\bS^{-1}\widehat\bSigma\bS^{-1}\bH
    +\widehat\bSigma^{1/2}(\bI + \widehat\bSigma^{-1/2}\bS^2 \widehat\bSigma^{-1/2})^{-1}
    \widehat\bSigma^{-1/2}\bS^2+ \bH\bS(\bS^2 + \widehat\bSigma)^{-1}\widehat\bSigma\\
    &\quad+\widehat\bSigma(\bS^2+ \widehat\bSigma)^{-1}\bS\bH\\
    &=  \bH(\bS\widehat\bSigma^{-1}\bS + \bI)^{-1} \bH + \bS(\bI + \bS \widehat\bSigma^{-1}\bS)^{-1}\bS
    +  \bH(\bS\widehat\bSigma^{-1}\bS + \bI)^{-1} \bS + \bS(\bS\widehat\bSigma^{-1}\bS + \bI)^{-1} \bH\\
    &= (\bH+\bS)(\bS\widehat\bSigma^{-1}\bS + \bI)^{-1} (\bH +\bS).
\end{align*}
Meanwhile, for the lower-right block we compute
\begin{align*}
    \tbH^\sT\tbH + \widehat\bSigma - \tbH^\sT \bS(\bS^2 + \widehat\bSigma)^{-1} \bS\tbH 
    &= \tbH^\sT\left(\bS^{-1}\widehat\bSigma \bS^{-1} + \bI\right)^{-1} \tbH + \widehat\bSigma.
\end{align*}
So denoting
\begin{equation}
    \tilde\bS := \bS\widehat\bSigma^{-1}\bS + \bI,
\end{equation}
a similar computation gives the remaining blocks which allows us to write
\begin{align*}
   \bD_0
   &= \begin{pmatrix}
       (\bH+\bS)\tbS^{-1} (\bH +\bS) & (\bH+\bS)\tbS^{-1} \tbH\\
       \tbH^\sT\tbS^{-1}(\bH +\bS) & 
       \tbH^\sT\tbS^{-1}\tbH + \widehat\bSigma
   \end{pmatrix} =: 
   \begin{pmatrix}
      \bA_1 & \bB_1\\
      \bB_1^\sT &\bD_1
   \end{pmatrix}.
\end{align*}
%\begin{align}
%    \widehat\bH^\sT\tilde\bH  + \widehat\bSigma  - \widehat\bH^\sT\bS(\bS^2 + \widehat\bSigma)^{-1}\bS\tilde\bH &= \tilde\bH^\sT(\bS\widehat\bSigma^{-1}\bS + \bI)^{-1}\tilde\bH + \widehat\bSigma.
%\end{align}
Now to invert $\bD_0$, a straightforward computation gives
$(\bD_1 - \bB_1^\sT \bA_1^{-1} \bB_1)^{-1} =    \widehat\bSigma^{-1}$
and so 
\begin{align*}
   \bD_0^{-1}
&= 
\begin{pmatrix}
   (\bH +\bS)^{-1}\tbS (\bH +\bS)^{-1} + (\bH +\bS)^{-1}\tbH \widehat\bSigma^{-1}\tbH(\bH +\bS)^{-1}
   & -(\bH +\bS)^{-1}\tbH \widehat\bSigma^{-1}\\
   -\widehat\bSigma^{-1} \tbH^\sT(\bH +\bS)^{-1} & \widehat\bSigma^{-1}
\end{pmatrix}.
\end{align*}

So we have the bound 
\begin{align}
\label{eq:D0_inv_norm}
   \norm{\bD_0^{-1}}_\op \le \|(\bH+\bS)^{-2}\tbS\|_\op + \left( 1 + \|{\tbH}\|_\op\|{(\bH +\bS)^{-1}}\|_\op \right)^2 \|\widehat\bSigma^{-1}\|_\op.
\end{align}
So using the crude bound $\|\bB_0\|_\op \le \|\bS \bH\|_\op + \|\widehat\bSigma\|_\op + \|\bS\tbH\|_\op$, and the bound of Eq.~\eqref{eq:D0_inv_norm} into Eq.~\eqref{eq:M1_norm_bound} gives
\begin{align*}
    &\|\bN_1^{-1}\| \le \|(\bS^2 + \widehat\bSigma)^{-1}\|_\op\\
&+\left(1 + 
    \|(\bS^2 + \widehat\bSigma)^{-1}\|_\op\cdot(\|\bS \bH\|_\op + \|\widehat\bSigma\|_\op + \|\bS\tbH\|_\op)
    \right)^2\\
    &\hspace{50mm}\cdots\left(
    \|(\bH+\bS)^{-2}\tbS\|_\op + \left( 1 + \|{\tbH}\|_\op\|{(\bH +\bS)^{-1}}\|_\op \right)^2 \|\widehat\bSigma^{-1}\|_\op\right).
\end{align*}
Recalling the definition $\sfK = \sup_{\bv,\bu,w} \norm{\grad^2 \ell(\bv,\bu,w)}_{\op}$, and
$\tilde\sfK = \sup_{\bv,\bu,w} \norm{(\partial_{i}\partial_j \ell(\bv,\bu,w))_{i\in[k_0], j\in[k]}}_{\op}$ from Appendix~\ref{sec:RMT}, we can simplify the bounds as
\begin{align*}
    \left(1 + 
    \|(\bS^2 + \widehat\bSigma)^{-1}\|_\op\cdot(\|\bS \bH\|_\op + \|\widehat\bSigma\|_\op + \|\bS\tbH\|_\op)
    \right)^2
&\le 
    C_0 \left(1 + 
    \|\widehat\bSigma^{-1}\|_\op^2 \|\widehat\bSigma\|_\op^2\cdot(\|\bS\|_\op^2 (\sfK + \tilde\sfK)^2 + 1)
    \right)\\
&\le C_1(\sfK,\tilde\sfK)
    \|\widehat\bSigma^{-1}\|_\op^2 
\|\widehat\bSigma\|_\op^2
    \left( \|\bS\|_\op^2 + 1
    \right)
\end{align*}
and
\begin{align*}
    &\left(
    \|(\bH+\bS)^{-2}\tbS\|_\op + \left( 1 + \|{\tbH}\|_\op\|{(\bH +\bS)^{-1}}\|_\op \right)^2 \|\widehat\bSigma^{-1}\|_\op\right)
    \\
&\hspace{10mm}\le 2\|(\bH+\bS)^{-2}\|_\op
\left(\|\bS\|_\op^2 \|\widehat\bSigma^{-1}\|_\op + 1 + \tilde\sfK^2 \|\widehat\bSigma\|_\op^2\|\widehat\bSigma^{-1}\|_\op    +  
\sfK^2 \|\widehat\bSigma^{-1}\|_\op  \|\widehat\bSigma\|_\op^2 
+  \|\widehat\bSigma^{-1}\|_\op \|\bS\|_\op^2
\right)\\
&
\hspace{10mm}\le 4\|(\bH+\bS)^{-2}\|_\op\|\widehat\bSigma^{-1}\|_\op
\left(\|\bS\|_\op^2 +  \|\widehat\bSigma\|_\op + \tilde\sfK^2 \|\widehat\bSigma\|_\op^2    +  
\sfK^2   \|\widehat\bSigma\|_\op^2 
\right)\\
&\hspace{10mm}\le C_3(\sfK,\tilde \sfK) \|(\bH+\bS)^{-2}\|_\op\|\widehat\bSigma^{-1}\|_\op
\left( \|\bS\|_\op^2 + \|\widehat\bSigma\|_\op^2 +1
\right).
\end{align*}
Then using $\bS^2+ \widehat\bSigma \succeq \bS^2 + \sfK^{-1}\bH  \succeq 
 \sfK^{-1}\left(\bS^2 +\bH \right)$ for $\sfK \ge 1$,
 the bound on $\|\bN_1^{-1}\|_\op$
 becomes
\begin{align*}
\|\bN_1^{-1}\|_\op
&\le  \|(\bS^2 + \widehat\bSigma)^{-1}\|_\op +  
C_4(\sfK,\tilde\sfK)
\|(\bS + \bH)^{-2}\|_\op
\|\widehat \bSigma^{-1}\|_\op^3 \| \widehat\bSigma\|_\op^2 \left(\|\bS\|_\op^2 + \|\widehat\bSigma\|_\op^2 + 1\right)^2\\
&\le C_5(\sfK,\tilde\sfK)
\|(\bS + \bH)^{-2}\|_\op
\left(
\|\widehat\bSigma^{-1}\|_\op^3 +1
\right)
\left(\|\widehat\bSigma\|_\op^2+ \|\bS\|_\op^2  + 1\right)^3
\end{align*}

Using Eq.~\eqref{eq:lb_isometry} then gives the desired lower bound
\begin{align*}
\sigma_{\min}(\bJ\bzeta) &\ge \frac{
\sigma_{\min}(\bH + \bS) 
}
{
\|\widehat\bSigma\|_\op^{1/2}
C_5(\sfK,\tilde\sfK)^{1/2}
\left(
\|\widehat\bSigma^{-1}\|_\op^3 +1
\right)^{1/2}
\left(\|\widehat\bSigma\|_\op^2+ \|\bS\|_\op^2  + 1\right)^{3/2}
}\\
&\ge \frac{
\sigma_{\min}(\bH + \bS) 
}
{
C_6(\sfK,\tilde\sfK)
\left(
\|\widehat\bSigma^{-1}\|_\op^{3/2} +1
\right)
\left(\|\widehat\bSigma\|_\op^{7/2}+ \|\bS\|_\op^3  + 1\right)
}
\end{align*}
allowing us to deduce the lemma after using $\|\bS\|_\op \le n \sup_{\|\bTheta\|^2 \le \sfA_{\bR}} \|\grad^2\rho(\bTheta)\| \le C_7(\sfA_{\bR})$ by continuity in Assumption~\ref{ass:regularizer}.

\qed



%
%\begin{align}
%    &\|\bM_1^{-1}\| \le \|(\bS^2 + \widehat\bSigma)^{-1}\|_\op\\
%&+\left(1 + 
%    \|(\bS^2 + \widehat\bSigma)^{-1}\|_\op\cdot(\|\bS \bH\|_\op + \|\widehat\bSigma\|_\op + \|\bS\tbH\|_\op)
%    \right)^2
%    \left(
%    \|(\bH+\bS)^{-2}\tbS\|_\op + \left( 1 + \|{\tbH}\|_\op\|{(\bH +\bS)^{-1}}\|_\op \right)^2 \|\widehat\bSigma^{-1}\|_\op\right)\\
%    &\le \frac{C_0}{\sigma_{\min}(\bH + \bS)^2 \sigma_{\min}(\bS^2 + \widehat\bSigma)} \left( 1 + \|\bS\|_\op^2 \|\widehat\bSigma^{-1}\|_\op + \|\widehat\bSigma^{-1}\|_\op \|\bH + \bS\|_\op^2
%   + \|\tbH\|_\op^2\|\widehat\bSigma^{-1}\|_\op
%    \right)\cdot\\
%    &\quad\quad\left(\|\bS^2\|_\op +  \|\widehat\bSigma\|_\op + \|(\bS^2 +\widehat\bSigma)^{-1}\|_\op(\|\bS\|_\op^2 \sfK^2 \|\widehat\bSigma\|_\op^2 
%    +\|\bS\|_\op^2 \tilde\sfK^2 \|\widehat\bSigma\|_\op^2
%    +\|\widehat\bSigma\|_\op^2
%    )\right)\\
%    &\le 
%    \frac{C_1 
%    (1+ \|\bS\|_\op^2)}{\sigma_{\min}(\bH + \bS)^2 \sigma_{\min}(\bS^2 + \widehat\bSigma)} 
%    \left( 1 + \|\widehat\bSigma\|_\op^2 \|\widehat\bSigma^{-1}\|_\op (\sfK^2 + \tilde\sfK^2 + 1 )\right)^2.
%\end{align}
%Using Eq.~\eqref{eq:lb_isometry} then gives the desired lower bound
%\begin{align}
%\sigma_{\min}(\bD \bzeta) &\ge \frac{
%\sigma_{\min}(\bH + \bS) \sigma_{\min}(\bS^2 + \widehat\bSigma)^{1/2}
%}{(1 + \|\bS\|)_\op} \frac1{C_2\|\widehat\bSigma\|_\op^{1/2} \left(  \|\widehat\bSigma\|_\op^2 \|\widehat\bSigma^{-1}\|_\op (\sfK^2 + \tilde\sfK^2 + 1 )\right)}\\
%&\ge \sigma_{\min}(\bH +\bS)  \frac1{(1 + \|\bS\|_\op)E(\widehat\bSigma; \sfK,\tilde\sfK)}.
%\end{align}
%where
%\begin{equation}
%    E(\widehat\bSigma;\sfK,\tilde\sfK) := C_2(1 +  \|\widehat\bSigma\|_\op^2 \|\widehat\bSigma^{-1}\|_\op (\sfK^2 + \tilde\sfK^2 + 1 )).
%\end{equation}

This lower bound on the smallest singular of $\bJ\bzeta$ allows us to deduce the following relation between the determinants a priori mentioned on the conditioning event $\bzeta = \bzero$.

\begin{lemma}[Relating $\det(\de \bz)$ to $\det(\bH)$]
\label{lemma:dz_to_detH}
Let $\Err_{\sigma}(\bX)$ be the error term defined in Lemma~\ref{lemma:lb_singular_value_Df} and $r_k = k^2 + k_0k$. 
Under the assumptions of Section~\ref{sec:assumptions}, we have for any $(\bTheta,\bbV) \in\cM$, we have on the event $\{\bzeta(\bTheta,\bbV) = 0\}$,
\begin{equation}
|\det( \de\bz(\bTheta,\bbV) )| \le \ \frac{|\det\left(\bH(\bTheta,\bbV) \right)| }{\sigma_{\min}(\bH(\bTheta,\bbV))^{r_k}}  \Err_\sigma(\bX)^{r_k}.
\end{equation}
\end{lemma}
\begin{proof}
Let $r_k := k^2 + kk_0$. Let $m = nk + nk_0 + dk$ be the dimension of the full space.
We'll suppress the dependence on the indices $(\bTheta,\bbV)$ in what follows.
Let $\bB_{\bSigma},\bB_\bT \in \R^{m \times (m -r_k)}$ be the basis matrices for the column space of the covariance $\bSigma$, and tangent space of $\cM$, respectively. Recall that the co-dimensions of both of these spaces are $r_k$.
Let $\bB_{\bT^c},\bB_{\bSigma^c}$ be bases for the orthocomplements.

Recall that by Lemma~\ref{lemma:eig_vecs_NS_Sigma}, $\bzeta$ is identically equal to zero whenever $\bzeta$ is in the nullspace of $\bSigma$. Hence, $\bB_{\bSigma^c}^\sT \bJ\bzeta\bB_{\bT} = \bzero$ so that 
\begin{equation}
\det(\de\bz)  = \det\left(\bB_{\bSigma}^\sT \bJ \bzeta \bB_\bT\right)  =\frac{ \det\left( \bJ \bzeta\right)}
   {\det( \bB_{{\bSigma}^c}^\sT \bJ \bzeta \bB_{\bT^c})}.
\end{equation}
Using that
\begin{equation}
\label{eq:projected_grad_singular_val_lb}
    \sigma_{\min}(\bB_{{\bSigma}^c}^\sT \bJ \bzeta \bB_{\bT^c}) \ge 
   % \sigma_{\min}(\bJ \bzeta ) \lambda_{\min}(\bB_{\bSigma^c}^\sT\bB_{\bSigma^c})^{1/2} \lambda_{\min}(\bB_{\bT^c}^\sT \bB_{\bT^c})^{1/2} =
    \sigma_{\min}(\bJ \bzeta ),
\end{equation}
we conclude
\begin{align*}
   \left|\det\left(\bB_{\bSigma}^\sT \bJ \bzeta \bB_\bT\right) \right| \stackrel{(a)}{=}  
  \frac{\left|\det\left( \bH \right)\right|}{\prod_{i=1}^{r_k} \sigma_i(\bB_{\bSigma^c}^\sT \bJ \bzeta \bB_{\bT^c})}
  \stackrel{(b)}{\le} \frac{\left( \Err_\sigma(\bX)\right)^{r_k}}
  {\sigma_{\min}(\bH)^{r_k}} 
  \det\left( \bH \right)
\end{align*}
where $(a)$ follows from Eq.~\eqref{eq:det_projection} and the identities $|\det(\bJ\bzeta)| = |\det(\bH)|$ and $|\det(\bA)| = \prod_{i} |\lambda_i(\bA)| = \prod_i \sigma_i(\bA)$ for any square matrix $\bA,$ and $(b)$ follows from Eq.~\eqref{eq:projected_grad_singular_val_lb} and Lemma~\ref{lemma:lb_singular_value_Df}.
\end{proof}



\label{section:determinant_bound}

\subsubsection{Conditioning and concentration}
Using our random matrix theory results of Section~\ref{sec:RMT},
we bound the conditional expectation of the bound obtained from Lemma~\ref{lemma:dz_to_detH}. 
One difficulty we will face is bounding the smallest singular value of $\bX^\sT\bX$ on the event $\bzeta(\bTheta,\bbV) =\bzero. $ Notice on this event, we have $\bX^\sT \bbV + n\bRho =\bzero$ and $\bX[\bTheta,\bTheta_0] = \bbV.$ The next lemma gives bounds on the moments of the inverse singular value in terms of the extreme singular values of $\bbV,[\bTheta,\bTheta_0]$ and $\bL$.
%\am{Perhaps put an indicator of $\sigma_{\min}(\bX^\sT\bX)\ge c$ in the very beginning?}


Let us introduce the following quantities for this section
\begin{equation}
   \sfA_{\bL}  :=  1\vee
   \sup_{\substack{\|\bV\|_\op \le \sqrt{n}\sfA_\bV \\ 
  \|\bw\|_2 \le \sfA_{\bw}\sqrt{n}
   }} \frac1{\sqrt{n}}\|\bL(\bV,\bw)\|_\op,\quad
   \sfA_{\bRho} := 1 \vee
   \sup_{\|\bTheta\|_\op \le \sfA_\bR} \|\bRho(\bTheta)\|_\op.
\end{equation}
Note that by Assumptions~\ref{ass:loss} and~\ref{ass:regularizer}, we have $\sfA_\bL,\sfA_\bRho$ are bounded by some positive constant $C(\sfA_\bV,\sfA_\bw),C(\sfA_\bR)$, depending only on
$(\sfA_\bV,\sfA_\bw)$, $\sfA_\bR$ (and $\sOmega$), respectively.


\begin{lemma}[Lower bounding $\sigma_{\min}(\bX^\sT\bX)$ at the zeros of $\bzeta$] 
\label{lemma:lsv_sigma_conditional}
Under the assumption of Section~\ref{sec:assumptions},
for any $p>1$ and $\|\bw\|_2 \le \sfA_{\bw}\sqrt{n}$, if $n> d+k$,
\begin{equation}
\E\left[\sigma_{\min}(\bX^\sT\bX)^{-p} | \bzeta = \bzero, \bw\right] \le 
\frac{C^p}{d^{p}} \left( 
\frac{\sfA_{\bR}^{5} \sfA_{\bbV}^{4}}{\sfsigma_{\bbV}^{4} \sfsigma_{\bR}^{2}} \frac{\sfA_{\bL}^{4} }{\sfsigma_{\bL}^{4}}\left(\sfA_{\bRho}^{2} + 1\right)
\right)^p \left( \alpha_n - \frac{k}{d} - 1\right)^{-p} 
\end{equation}
for some universal constant $C>0$.
\end{lemma}
\begin{proof}
Note that on $\{\bzeta = \bzero\}$, we have $\bX[\bTheta,\bTheta_0] = \bbV$ and $\bL^\sT \bX = - \bRho^\sT$. Letting $\bP_{\bTheta}, \bP_\bL$ be the projections on the columns spaces of $[\bTheta,\bTheta_0],\bL$ respectively, we can write 
\begin{equation}
    \bX = \bP_\bL^\perp \bX \bP_\bTheta^\perp - \bL(\bL^\sT\bL)^{-1} \bRho^\sT \bP_{\bTheta}^\perp  + \bbV\bR^{-1} \bTheta^\sT.
\end{equation}
Letting $\bB_\bTheta\in\R^{d \times (k+k_0)}$ be a basis matrix for the columnspace of $(\bTheta,\bTheta_0)$ and $\bB_\bTheta^\perp \in\R^{d \times(d - k-k_0)}$ be a basis matrix for its complement, we define
\begin{equation}
    \tilde\bX^\sT\tilde\bX = \begin{pmatrix}
    \bG^\sT\bP_{\bL}^\perp\bG -  \bA^\sT\bA &  (\bG -\bA)^\sT \bC\\
    \bC^\sT(\bG - \bA) & \bC^\sT\bC
    \end{pmatrix} \in \R^{d\times d}
\end{equation}
where $\bG\in\R^{n\times(d - k - k_0)}$ is a matrix of i.i.d. Guassian entries and 
    $\bA := \bL(\bL^\sT\bL)^{-1} \bRho^\sT \bB_{\bTheta}^\perp,  \bC:= \bbV\bR^{-1} (\bTheta,\bTheta_0)^{\sT} \bB_{\bTheta}$.
Then by Gaussian conditioning, we immediately see that $\E[\sigma_{\min}(\bX^\sT\bX)^{-p} | \bzeta = 0] = \E[\bsigma_{\min}(\tilde\bX^\sT\tilde\bX)^{-p}].$
Now we can bound $\sigma_{\min}(\tilde\bX^\sT\tilde\bX)^{-p}$ using the block inversion formula. Namely,
\begin{align}
\nonumber
   &\E\left[\sigma_{\min}(\tilde\bX^\sT\tilde\bX)^{-p}\right] \\
   &\le \E\Big[ \Big(\|{(\bC^\sT\bC)^{-1}}\|_\op + 
   \left(1 + \|{(\bC^\sT\bC)^{-1}}\|_\op (\|\bG\bP_L^\perp\|_\op +\|\bA\|_\op)\|\bC\|_\op\right)^2
   \|((\bG - \bA)^\sT\bP_{\bL}^\perp\bP_{\bC}^\perp\bP_\bL^\perp (\bG - \bA))^{-1}\|_\op\Big)^p\Big]\\
   &\le 2^p \Big( 
   \underbrace{\E\Big[\left(1 + \|{(\bC^\sT\bC)^{-1}}\|_\op (\|\bG\bP_L^\perp\|_\op +\|\bA\|_\op)\|\bC\|_\op\right)^{4p}\Big]^{1/2}}_{\mathrm{(I)}}
   \underbrace{
   \E\Big[\|((\bG - \bA)^\sT\bP_{\bL}^\perp\bP_{\bC}^\perp\bP_\bL^\perp (\bG - \bA))^{-1}\|_\op^{2p}\Big]^{1/2}}_{\mathrm{(II)}}\nonumber \\
   &\hspace{20mm}+ \|{(\bC^\sT\bC)^{-1}}\|_\op^p \Big). \label{eq:lb_sigma_min_XX_decomp}
\end{align}
Let's bound each of the terms $\mathrm{(I)}$ and $\mathrm{(II)}$ in what follows.

\noindent\textbf{The term $\mathrm{(I)}$:}
By recognizing $\|\bG\bP_{\bL}^\perp\|_\op$ as the operator norm of a standard Gaussian matrix in $\R^{(n-k)\times(d- k-k_0)}$
and using the bounds
\begin{equation}
\|(\bC^\sT\bC)^{-1}\|_\op \le  \frac{\|(\bTheta,\bTheta_0)\|^4_\op}{\lambda_{\min}(\bbV^\sT\bbV)},
\quad \|\bC\|_\op \le \frac{\|\bbV\|_\op \|(\bTheta,\bTheta_0)\|_\op}{\lambda_{\min}(\bR)},\quad
\|\bA\|_\op \le  \frac{\|\bL\|_\op \|\bRho\|_\op}{\lambda_{\min}(\bL^\sT\bL)},
\end{equation}
we can bound
\begin{align*}
\mathrm{(I)}^2 &\le 
C_1^{p}
\left( 1 +
 \|{(\bC^\sT\bC)^{-1}}\|_\op^{4p}\|\bC\|_\op^{4p}
\left(
\|\bA\|_\op^{4p}
+  n^{2p}
\right)\right)\\
&\le C_1^p \left(1 +  \frac{\norm{\bR}_\op^{10p} \|\bbV\|_\op^{4p} }{\sigma_{\min}(\bbV)^{8p} \sigma_{\min}(\bR)^{4p} } \left( \frac{\|\bL\|_\op^{4p} \|\bRho\|_\op^{4p}}{\sigma_{\min}(\bL)^{8p}} + n^{2p} \right)\right)\\
&\stackrel{(a)}{\le} C_1^p \left( 1 + \frac{\sfA_{\bR}^{10p} \sfA_{\bbV}^{4p}}{\sfsigma_{\bbV}^{8p} \sfsigma_{\bR}^{4p}}\left( \frac{\sfA_{\bL}^{4p} \sfA_{\bRho}^{4p}}{\sfsigma_{\bL}^{8p}} + 1\right) \right)\\
&\le C_2^p  \frac{\sfA_{\bR}^{10p} \sfA_{\bbV}^{8p}}{\sfsigma_{\bbV}^{8p} \sfsigma_{\bR}^{4p}} \frac{\sfA_{\bL}^{8p} }{\sfsigma_{\bL}^{8p}}\left(\sfA_{\bRho}^{4p} + 1\right).
\end{align*}


\noindent\textbf{The term $\mathrm{(II)}$:}
For the second term, we similarly observe that
\begin{equation}
\E\left[\|((\bG - \bA)^\sT\bP_{\bL}^\perp\bP_{\bC}^\perp\bP_\bL^\perp (\bG - \bA))^{-1}\|_\op^{2p}\right] \le \E\left[\sigma_{\min}(\tilde\bG)^{-4p}\right]
\end{equation}
where $\tilde\bG \in\R^{(n - 2k-k_0) \times (d - k - k_0)}$ is a Gaussian matrix with i.i.d. uncentered entries. 
A standard result then gives
\begin{equation}
    \mathrm{(II)}^{2} \le  C_3^{p}\left( \sqrt{n - 2k - k_0} - \sqrt{d - k-k_0}\right)^{-4p} \le  C_3 \left(\frac{\sqrt{n - k} + \sqrt{d}}{ n - d- k}\right)^{4p} = 
    \frac{C_3^{p}}{d^{2p}} \left( \left(\alpha_n - \frac{k}{d}\right)^{1/2} -1 \right)^{-4p}
\end{equation}
for some universal $C_3>0$, whenever $n > d + k$.
%A standard lower bound on the smallest singular value of $\tilde \bG$ gives
%\begin{equation}
%    \P\left( \sigma_{\min}(\tilde \bG) \le \sqrt{n - 2k - k_0} - \sqrt{d - k-k_0} -t \right) \le e^{-t^2/2}
%\end{equation}
%for any $t>0$.

\noindent\textbf{Combining the bounds:}
Combining the above bounds on $\mathrm{(I)}$ and $\mathrm{(II)}$ with Eq.~\eqref{eq:lb_sigma_min_XX_decomp} we obtain
\begin{align*}
\E\left[\sigma_{\min}(\tilde\bX^\sT\tilde\bX)^{-p}\right]
&\le C_4^p \left( \frac1{d^p}\left( 
\frac{\sfS_{\bR}^{5} \sfS_{\bbV}^{4}}{\sfs_{\bbV}^{4} \sfs_{\bR}^{2}} \frac{\sfS_{\bL}^{4} }{\sfs_{\bL}^{4}}\left(\sfS_{\bRho}^{2} + 1\right)
\right)^p \left( \alpha_n - \frac{k}{d} -1 \right)^{-p}  
+  \frac{1}{n^p}  \left(\frac{\sfS_{\bR}^2}{\sfs_{\bbV}^2}\right)^{p}\right)\\
&\le
\frac{C_4^p}{d^{p}} \left( 
\frac{\sfS_{\bR}^{5} \sfS_{\bbV}^{4}}{\sfs_{\bbV}^{4} \sfs_{\bR}^{2}} \frac{\sfS_{\bL}^{4} }{\sfs_{\bL}^{4}}\left(\sfS_{\bRho}^{2} + 1\right)
\right)^p \left( \alpha_n - \frac{k}{d} - 1\right)^{-p} 
\end{align*}
as desired.


%\begin{align}
%   \E\left[\sigma_{\min}(\tilde\bX^\sT\tilde\bX)^{-p}\right] 
%   &\le 
%   C_3^p \left(
%   \frac{\|(\bTheta,\bTheta_0)\|^5_\op \|\bbV\|_\op^2}{\lambda_{\min}(\bbV^\sT\bbV) \lambda_{\min}(\bR)}
%   \right)^{2p}
%   \left(\left(
%    \frac{\|\bL\|_\op \|\bRho\|_\op}{\lambda_{\min}(\bL^\sT\bL)  }
%   \right)^{2p}+
%    n^{p}
%\right)
%\frac{1}{d^p} \left( \left(\alpha - \frac{k}{d}\right)^{1/2} -1 \right)^{-2p}\\
%%\left(\frac{\sqrt{\alpha}+ 1}{(\alpha - 1)\sqrt{d} - k/\sqrt{d}} + 1\right)^{p}\\
%&\le
%   C_3^p \left(
%   \frac{\|\bR\|^3_\op \|\bbV\|_\op^2 \|\bL\|_\op^2}{\lambda_{\min}(\bL^\sT\bL)\lambda_{\min}(\bbV^\sT\bbV) \lambda_{\min}(\bR)}
%   \right)^{2p}
%   \frac{\left(
%    \|\bRho\|_\op^{2p}+
%    n^{p}
%\right)}{d^p}
%\left( \left(\alpha - \frac{k}{d}\right)^{1/2} -1 \right)^{-2p}
%%\left(\frac{\sqrt{\alpha}+ 1}{(\alpha - 1)\sqrt{d} - k/\sqrt{d}} + 1\right)^{p}\\
%\end{align}
\end{proof}

Before proceeding to the analysis of the conditional expectation, let us give the following lemma regarding concentration of Lipschitz functions of $\bH_0$. The proof of this is standard.


\begin{lemma}[Concentration of Lipschitz functions of the Hessian]
\label{lemma:concentration_lipschitz_func}
Assume $f :\R \to\R$ is Lipschitz.  
Recall $\bH_0 = \left(\bX\otimes\bI_k\right)^\sT \bSec \left(\bX \otimes \bI_k\right)$, where 
$\bK$ was defined in Eq.~\eqref{eq:SecDef}, and let $\bS\in\R^{dk\times dk}$ be any deterministic symmetric matrix.
Then there exist absolute constants $c,C > 0$ such that, 
for any 
\begin{equation}
    t \ge \frac{\sfK\norm{f}_{\Lip}}{\alpha_n^{1/2}} \frac{k}{n^{1/2}},
\end{equation}
we have
\begin{equation}
    \P_\bX\left(\left|\frac1n\Tr(f((\bH_0 + \bS)/n)) - \frac1n\E_\bX\left[ \Tr(f((\bH_0 + \bS)/n))\right]\right| \ge t\right) 
\le
     C \exp \left\{ 
    -c\frac{ t n^{3/2} \alpha^{1/2} }{  k \sfK \norm{f}_{\Lip}}
    \right\}
%C_1 \exp \left\{ 
%    -c_1\frac{ t^2 n^2 \alpha_n }{\norm{f}_{\Lip}^2(\gamma^{1/2} + 1)^2 k^2 \sfK^2}
%    \right\}
%    +
%    C_2\exp\left\{  -c_2 n\gamma\right\}.
\end{equation}
The same bound holds for any matrix $\bSec= (\bSec_{ij})_{i,j\le k}$ with $\bSec_{ij}\in\reals^{n\times n}$
a diagonal matrix with diagonal entries bounded (in absolute value) by $\sfK$.
\end{lemma}

\begin{proof}
First, we bound the variation of the function
\begin{equation}
\label{eq:lip_func_of_X}
   g(\bX) := \Tr f\Big((\bI \otimes \bX)^\sT \bSec (\bI \otimes \bX)/n + \bS/n\Big).
\end{equation}
Let $\bM = (\bI \otimes \bX)^\sT \bSec (\bI \otimes \bX)/n + \bS/n$ and
$\bM' = (\bI \otimes \bX')^\sT \bSec (\bI \otimes \bX')/n + \bS/n$.
Use $\{\lambda_i\}_{i\in[dk]}$ and $\{ \lambda_i'\}_{i \in[dk]}$ to denote their eigenvalues respectively. By Hoffman-Wielandt, we have
    \begin{align*}
        \left| \Tr f(\bM) - \Tr f(\bM') \right|
        &= \left| \sum_{i=1}^{dk} f(\lambda_i) - \sum_{i=1}^{dk}  f(\lambda_i')\right|
         \le \norm{f}_{\Lip} \min_{\sigma } \sum_{i=1}^{dk} \left|\lambda_i - \lambda_{\sigma(i)}' \right| 
         \le \norm{f}_{\Lip} (dk)^{1/2} \norm{\bM - \bM'}_F.
    \end{align*}
We can bound the norm of the difference 
\begin{align*}
    \norm{\bM - \bM'}_F &
    \le\frac1n\norm{(\bI \otimes \bX) - (\bI \otimes \bX')}_F \norm{\bK}_\op \left(\norm{\bI \otimes \bX}_\op + \norm{\bI \otimes \bX'}_\op \right)\\
    &\le \frac{k^{1/2}}{n} \sfK \norm{\bX - \bX'}_F (\norm{\bX}_\op + \norm{\bX'}_\op).
\end{align*}
Now for any $\gamma > 4$, define
\begin{equation}
\cA_\gamma :=     \left\{ \norm{\bX}_\op \le n^{1/2}\gamma^{1/2} \right\}.
\end{equation}
The above bound on the variation of $g$ implies that on $\cA_\gamma$, we have
\begin{equation}
    |g(\bX) - g(\bX') | \le \frac{C_0\,\sfK\, \norm{f}_{\Lip}}{\alpha_n^{1/2}} k\gamma^{1/2}    \norm{\bX - \bX'}_F\, .
\end{equation}
%the function defined in
% Eq.~\eqref{eq:lip_func_of_X} has Lipschitz constant bounded by
%\begin{equation}
%    \norm{g}_{\Lip} = \frac{2 \gamma(n) k \sfK \norm{f}_{\Lip} } { \alpha^{1/2}}
%\end{equation}
We can apply Gaussian concentration on this event. For $t>0$, we have for some universal constant $C_2,c_2,C_1,c_1 >0$,
\begin{align*}
\P\left(\left| \frac1n \Tr f(\bH_0) - \frac1n \E\left[\Tr f(\bH_0)\right] \right| \ge t \right)&\le 
    \P\left(\left\{\left| \frac1n \Tr f(\bH_0) - \frac1n \E\left[\Tr f(\bH_0)\right] \right| \ge t \right\} \cap \cA_\gamma\right)  
    +
    \P\left( \cA_\gamma^c\right) \\
    &\le  C_1 \exp \left\{ 
    -c_1\frac{ t^2 n^2 \alpha }{ \gamma  k^2 \sfK^2 \norm{f}_{\Lip}^2}
    \right\}
    +
     C_2\exp\left\{  -c_2 n\gamma\right\}.
\end{align*}
Choosing $\gamma$ to satisfy
\begin{equation}
\gamma = 4\frac{t n^{1/2}\alpha_n^{1/2}}
{k \sfK \norm{f}_{\Lip}},
\end{equation}
for appropriate universal $c_3>0$ gives the desired bound, as long as 
\begin{equation}
    t \ge \frac{\sfK\norm{f}_{\Lip}}{\alpha_n^{1/2}} \frac{k}{n^{1/2}}.
\end{equation}
\end{proof}

%\begin{lemma}[Interlacing theorem]
%Let $\bH,\bH_k \in \R^{m \times m}$.
%    Suppose $\rank(\bH - \bH_k) \le r_k.$ Then for any monotonically non-decreasing function $f$, we have
%    \begin{equation}
%    \sum_{i = 1}^m f(\lambda_i(\bH_k))  \le  \sum_{i=r_k}^{m } f(\lambda_i(\bH)) +  r_k f( \lambda_{\max}(\bH_k)).
%    \end{equation}
%\end{lemma}
%\begin{proof}
%This is a direct consequence of Weyl's inequality and interlacing of the eigenvalues of $\bH$ and $\bH_k$. Namely, we have for $i \in \{1, \dots , m- r_k\}$
%   $\lambda_i(\bH_k) \le \lambda_{i + r_k}(\bH) $
%and for $i\in\{m - r_k  +1, m\}$,
%  $\lambda_i(\bH_k) \le \lambda_{\max}(\bH_k)$. \am{Here eigenvalues are increasing: Please double check
%  that this matches the notations section}
%\end{proof}


\subsubsection{Proof of Lemma~\ref{lemma:CE_bound}}
\label{sec:pf_lemma_CE_bound}
%\begin{lemma}[Bounding the conditional expectation of the determinant]
% \label{lemma:CE_bound}
%Fix $\tau_0,\tau_1 \in (0,1)$, and $\bw$ satisfying $\|\bw\|_2 \le \sfA_{\bw}\sqrt{n}$.
%Then under the assumptions of Section~\ref{sec:assumptions}, there exist constants $C,c>0$ depending only on $\sOmega$ and $C_0(\tau_0)$ depending only on $\sOmega$ and $\tau_0$ such that for all $n > C_0(\tau_0)$,
%\begin{enumerate}
%\item for any $(\bbV,\bTheta) \in\cM(\cuA,\cuB)$ satisfying $\mu_{\star}(\hnu,\hmu)((-\infty, \tau_0)) < \tau_0,$ we have
%\begin{align}
%&\E[|\det \big(\de \bz(\bt)\big)|\one_{\bH/n + \grad^2\rho \succ \sfsigma_\bH} \big|\bzeta(\bt) = 0 ,\bw ]
%%&
%%\quad{\le}
%%\frac{(1 + \sfS_{\grad^2\rho})^{r_k}\; \sfC_0^{3r_k}\; n^{dk+2r_k}}{(n\sfs_\bH)^{r_k}}
%%\Bigg(\exp\left\{
%%\E\left[\Tr\log^{(\eps)}\left(\frac{\bH+\bS}n\right) \right] + n C(\sfD,k)\eps_n
%%\right\}
%%\left(\frac{\sfC_1}{\eps}\right)^{12r_k}
%%+ \exp\left\{ -n^{3/2} \eps_n\right\}
%%\Bigg)\\
%%&
%\le
%n^{dk}
%\Bigg(\exp\left\{
%\E\left[\Tr\log^{(\tau_1)}\left(\frac{\bH}n +\grad^2\rho\right)\Big| \bw \right] +  \frac{C n^{1-1/4}}{\tau-1}
%\right\}
%+ \exp\left\{ -n^{5/4}\right\}
%\Bigg)
%\end{align}
%
%\item For $(\bbV,\bTheta)\in\cM$ satisfying 
%$\mu_{\star}(\hnu,\hmu)((-\infty, \tau_0)) \ge \tau_0,$
%\begin{align}
%\E[|\det \big(\de \bz(\bt)\big)|\one_{\bH/n + \grad^2\rho \succ \sfsigma_\bH} \big|\bzeta(\bt) = 0,\bw ]
%& \le C \exp \left\{ 
%    -c 
%    \tau_0^2
%    n^{3/2}
%    \right\}.
%\end{align}
%\end{enumerate}
%
%
%%\notate{put it in the old form.
%%\begin{equation}
%%    \omega_{\DC,1}(n,k,\eps) :=  ,
%%    \omega_{\DC,2}(n,k,\eps)
%%\end{equation}
%%\label{lemma:concentration_det}
%%Let $\bDelta_k$ be as in Lemma~\ref{lemma:conditioning}.
%%For any $\eps\in(0,1)$,  we have
%%\begin{align}
%%    \E_\bX\left[|\det\left(\bH + \bE_k\right)| \norm{\bX}_\op^3
%%    \one_{\left(\bH +\bE_k\right)/n\succ \eps_\bH}\right]
%%    \leq
%%    n^{dk}\left(\exp\left\{ \E[\Tr(\log^\up{\eps}(\bH/n))
%%    + n \omega_{\DC,1}(n,k,\eps)\right\} + \omega_{\DC,2}(n,k)\right)\, ,
%%\end{align}
%%where we recall that  $\log^\up{\eps}(t) := \log( \eps \vee t)$,
%% and
%%\begin{align}
%%\omega_{\DC,1}(n,k,\eps)&:= 
%%C(\sfD)\left(
%%     \frac{\log(n)k^2  }{n^{1/2}\eps}
%%     + \frac{r_k}{n}\log\left(\frac{1}{\eps}\right)\right),\\
%%     \omega_{\DC,2}(n, k) &:= C(\sfD) \exp\left\{-c(\sfD) n \log(n) k\right\}.
%%%
%%\end{align}
%%}
%
%\end{lemma}
Throughout the proof, we will use the notation
$\bS = \bS(\bTheta) := n\grad^2\rho(\bTheta).$ 
%We'll also write $E(\;\cdot\;) := E(\;\cdot\;; \sfK,\tilde\sfK)$ for the error term introduced in Lemma~\ref{lemma:lb_singular_value_Df}.

\noindent\textbf{Step 1: Conditioning as a perturbation.}
Let us first recall that the event $\{\bzeta = \bzero\}$, is equivalent
to $\bL^\sT\bX = - n\bRho^\sT$ and $\bX(\bTheta,\bTheta_0) = \bbV$. 
So letting $\bP_{\bTheta}, \bP_\bL$ be the projections onto the columns spaces of $(\bTheta,\bTheta_0),\bL$ respectively, we have on $\{\bzeta = \bzero\}$
\begin{equation}
    \bX = \bP_\bL^\perp \bX \bP_\bTheta^\perp - n \bL(\bL^\sT\bL)^{-1} \bRho^\sT \bP_{\bTheta}^\perp  + \bbV\bR^{-1} \bTheta^\sT.
\end{equation}
Hence for any function $g$,
\begin{equation}
    \E[g(\bX) | \bzeta = \bzero] = \E[g(\bX + \bDelta_{0,k})]
\end{equation}
for some matrix $\bDelta_{0,k} = \bDelta_{0,k}(\bTheta,\bbV)$ 
satisfying
%\begin{align}
%    &\rank(\bDelta_{0,k}) \le r_k,\\
%    &\norm{\bDelta_{0,k}}_\op \le \norm{\bX}_\op.
%\end{align}
\begin{align}
    &\rank(\bDelta_{0,k}) \le 4(k+k_0),\\
    &\norm{\bDelta_{0,k}}_\op \le C_0\sqrt{n}\max\left(\frac{\norm{\bX}_\op}{\sqrt{n}} ,
    \frac{\sfA_{\bbV}}{\sfsigma_{\bR}^{1/2}} + \frac{\sfA_{\bRho}}{ \sfsigma_{\bL}^{1/2}}\right)
\end{align}
for some $C_0 >0.$
%Let $\bP_1, \bP_2$ be the orthogonal projectors onto the column space of $\bL$, $\bTheta,\bTheta_0$ respectively. Then on the event $\bzeta =0$, 
%$\bP_1 \bX =0$ and $\bX\bP_2 = (\bV,\bU) \bR^{-1}(\bTheta,\bTheta_0)$ so on this event
%\begin{equation}
%    \bX = (\bP_1  + \bP_1^\perp)\bX (\bP_2 + \bP_2^\perp) =  
%    \bP_1^\perp (\bV,\bU) \bR^{-1}(\bTheta,\bTheta_0) + \bP_1^\perp \bX \bP_2^\perp.
%\end{equation}
Consequently, letting
\begin{equation}
\label{eq:Delta_1k_def}
   \Delta_{1,k}  :=  (\bI \otimes \bDelta_{0,k})^\sT \bSec (\bI\otimes \bX ) + (\bI \otimes \bX)^\sT \bSec (\bI \otimes \bDelta_{0,k}) + (\bI \otimes \bDelta_{0,k})^\sT \bSec (\bI\otimes \bDelta_{0,k}),
\end{equation}
%and
%\begin{equation}
%   \Delta_{2,k}  :=   \bDelta_{0,k}^\sT \bX  + \bX^\sT\bDelta_{0,k} + \bDelta_{0,k}^\sT \bDelta_{0,k},
%\end{equation}
we have
\begin{align}
&\E[|\det \big(\de \bz(\bt)\big)|\one_{\bH \succ n\eps_\bH} \big|\bzeta(\bt) = 0 ]\\
&\quad\stackrel{(a)}{\le}
\E\left[  \frac{|\det\left(\bH\right)| }{\sigma_{\min}(\bH)^{r_k}}  \Err_\sigma(\bX)^{r_k}
\one_{\{\bH\succ n\eps_\bH \}}
\bigg| \bzeta = \bzero\right]\\
&
\quad\stackrel{(b)}{\le}
\frac{1}{(n\eps_\bH)^{r_k}}
\E\left[  \big|\det\left(\bH + \bDelta_{1,k}\right)\big|   \Err_\sigma(\bX + \bDelta_{0,k})^{r_k} \one_{\{\bH + \bDelta_{1,k} \succ n\eps_\bH\}}\right]
\label{eq:concentration_decomp_0}
\end{align}
where 
$(a)$ follows by Lemma~\ref{lemma:dz_to_detH} and  $(b)$ follows by Eq.~\eqref{eq:conditioning_generic}.
Observe that $\rank(\bI\otimes \bDelta_{0,k}) \le 4r_k.$ 
Letting $r_k' := 12 r_k$, we have from the definition in~Eq.~\eqref{eq:Delta_1k_def}, 
that
\begin{align}
\rank(\bDelta_{1,k}) &\le r'_k\\
\|\bDelta_{1,k}\|_\op &\le  3\sfK \|\bX\|_\op (1 + \|\Delta_{0,k}\|_\op)^2 \le 
3\sfK \|\bX\|_\op \left(1 +
\norm{\bX}_\op+
    \frac{\sigma_{\max}(\bbV)}{\lambda_{\min}(\bR)^{1/2}} + \frac{\sigma_{\max}(\bRho)}{ \lambda_{\min}(\bL^\sT\bL)^{1/2}}
\right)^2.
\end{align}
Then by Cauchy's interlacing theorem, we have for any $i\ge dk-r'_k$,
\begin{equation}
\label{eq:interlacing_2}
    \lambda_{i+r'_k}(\bH+  \bDelta_{1,k})\leq \lambda_{i}(\bH ),\quad \lambda_i(\bH + \bDelta_{1,k})\leq \|\bH +  \bDelta_{1,k}\|_\op.
\end{equation}

\noindent \textbf{Step 2: Needed bounds on some moments.}
Define the quantities
\begin{equation}
   \Delta_3 := \E\left[\Err(\bX + \bDelta_{0,k})^{2r_k}\right]^{1/2},\quad
   \Delta_4(p) := 
 \E\left[ 
\left( \frac{\|\bH +  \bDelta_{1,k}\|_\op}{n}\right)^{p}\right] \quad\textrm{for}\quad p>1.
\end{equation}
These will reappear in several places in the proof, we let us preempt this by giving a bound on these quantities.
%
First we bound $\Delta_3$. Recalling the definition of the error term $\Err_\sigma(\bX)$ introduced in Lemma~\ref{lemma:lb_singular_value_Df}, we write
\begin{align}
       \Delta_3^2 &=  \E\left[E_\sigma(\bX)^{2r_k}|\bzeta = \bzero\right] 
= C_0(\sfK,\tilde\sfK, \sfA_\bR)^{2r_k} \E\left[ (\|\bX^\sT\bX\|_\op^{7r_k} + n^{7r_k}) \left( \sigma_{\min}(\bX^\sT\bX)^{-3r_k} + 1 \right)\right]
\\
    &\le C_1(\sfK,\tilde\sfK, \sfA_\bR)^{r_k} 
    \underbrace{\E\left[(\|\bX\|_\op^{28r_k} + \|\bDelta_{0,k}\|_\op^{28r_k} + n^{14 r_k} ) \right]^{1/2}}_{\mathrm{(I)}}
    \underbrace{
   \E\left[ (\sigma_{\min}(\bX^\sT\bX)^{-6r_k} + 1)| \bzeta = \bzero\right]^{1/2}}_{\mathrm{(II)}}.
\end{align}
The term $\mathrm{(I)}$ can be bounded directly as
\begin{align}
  \mathrm{(I)}
  &\le C_3^{r_k} 
   \left(n^{7 r_k} +   \left( \frac{\|\bbV\|_\op}{\sigma_{\min}(\bR)^{1/2}} + \frac{n\|\bRho\|_\op}{\sigma_{\min}(\bL)} \right)^{14r_k}
  \right)
  \le C_4^{r_k}n^{7r_k} \left(1 +  \left(\frac{\sfA_{\bbV}^2}{\sfsigma_{\bR}} + \frac{\sfA_{\bRho}^2}{\sfsigma_{\bL}^2}\right)\right)^{7r_k}.
\end{align}
For the term $\mathrm{(II)}$, we use~Lemma~\ref{lemma:lsv_sigma_conditional} to bound the smallest singular value of $\bX^\sT\bX$ on $\{\bzeta = 0\}$:
\begin{align}
    \mathrm{(II)} \le 
\frac{C_5^{r_k}}{d^{3r_k}} \left( 
\frac{\sfA_{\bR}^{5} \sfA_{\bbV}^{4}}{\sfsigma_{\bbV}^{4} \sfsigma_{\bR}^{2}} \frac{\sfA_{\bL}^{4} }{\sfsigma_{\bL}^{4}}\left(\sfA_{\bRho}^{2} + 1\right)
\right)^{3r_k} \left( \alpha - \frac{k}{d} - 1\right)^{-3r_k} + 1.
\end{align}
So we obtain
\begin{equation}
    \Delta_3 \le  \sfC_0^{3r_k}n^{4r_k} 
%
%    \left(1 +  \left(\frac{\sfA_{\bbV}^2}{\sfsigma_{\bR}} + \frac{\sfA_{\bRho}^2}{\sfsigma_{\bL}^2}\right)\right)^{5r_k}
%\left( 
%\frac{\sfA_{\bR}^{5} \sfA_{\bbV}^{4}}{\sfsigma_{\bbV}^{4} \sfsigma_{\bR}^{2}} \frac{\sfA_{\bL}^{4} }{\sfsigma_{\bL}^{4}}\left(\sfA_{\bRho}^{2} + 1\right)
%\right)^{3r_k} \left( \alpha - \frac{k}{d} - 1\right)^{-3r_k}
\end{equation}
where we defined
\begin{equation}
   \sfC_0 := 
    C_6(\sfK,\tilde\sfK,\sfA_\bR)\, \alpha
    \left(1 +  \left(\frac{\sfA_{\bbV}^2}{\sfsigma_{\bR}} + \frac{\sfA_{\bRho}^2}{\sfsigma_{\bL}^2}\right)\right)^{2}
\left( 
\frac{\sfA_{\bR}^{5} \sfA_{\bbV}^{4}}{\sfsigma_{\bbV}^{4} \sfsigma_{\bR}^{2}} \frac{\sfA_{\bL}^{4} }{\sfsigma_{\bL}^{4}}\left(\sfA_{\bRho}^{2} + 1\right)
\right) \left(\left( \alpha - \frac{k}{d} - 1\right)^{-1} + 1 \right)
\end{equation}

To bound $\Delta_4(p)$, we have for any $p>1$,
\begin{align}
 %\E\left[ 
%\left( \frac{\|\bH + \bS+ \bDelta_{1,k}\|_\op}{n}\right)^{p}\right] 
\Delta_4(p)
%&\le
%\frac{2^p}{n^p}
%\left( \E\left[ 
%\|\bH\|^p_\op | \bzeta = 0\right] + \|\bS \|_\op^p\right)\\
&\le 
\frac{C_7^p}{n^p}
\left( \E\left[ 
\sfK^p(\|\bX\|^{2p}_\op + \|\bDelta_{0,k}\|_\op^{2p})\right] + \|\bS \|_\op^p\right)\\
&\le\frac{C_8^p}{n^p}
\left( 
\sfK^p n^p + 
\sfK^p\left(\frac{\|\bbV\|_\op}{\sigma_{\min}(\bR)^{1/2}} + n\frac{\|\bRho\|_\op}{ \sigma_{\min}(\bL)}\right)^{2p}
  + \|\bS \|_\op^p\right)\\
&\le \sfC_1^p
%  &\le C_2^p \sfK^p
%\left( 
%1 + 
%\frac{\sfA_{\bbV}^2}{\sfsigma_{\bR}} + \frac{\sfA_{\bRho}^2}{ \sfsigma_{\bL}^2}
%  + \sfA_{\grad^2\rho}\right)^p.\\
\end{align}
where 
\begin{equation}
    \sfC_1 := C_{9} \sfK 
\left( 
1 + 
\frac{\sfA_{\bbV}^2}{\sfsigma_{\bR}} + \frac{\sfA_{\bRho}^2}{ \sfsigma_{\bL}^2}
  + \sfA_{\grad^2\rho}\right).
\end{equation}

\noindent \textbf{Step 3: Proof of item \textit{1.}}
We will use the constraint on the minimum singular value of the Hessian to constrain the  asymptotic spectral measure $\mu_{\star}(\hnu_{\bbV}, \hmu_{\sqrt{d}\bTheta})$. 
Namely, fixing $\tau_0>0$,
we will show that for $\bbV$ satisfying
\begin{equation}
\label{eq:constraint_on_V_asymp}
\mu_{\star}(\hnu,\hmu)((-\infty, -\tau_0)) \ge \tau_0,
\end{equation}
 the value of the expectation in~\eqref{eq:concentration_decomp_0} is small.
%More precisely, We will that, by Lipschitz concentration (Lemma~\ref{lemma:concentration_lipschitz_func}), the expectation in~\eqref{eq:concentration_decomp_0} is small --in an appropriate sense-- for any $\bbV$ such that 
%$\inf \supp(\mu_\star(\widehat\nu_{\bV,\bU,\bw})) < s_0$.
To this ends, define the event
\begin{equation}
    \Omega_{3}:=\left\{
    \big|\left\{ \lambda \in \spec\left(\bH/n \right) : \lambda \leq 0\right\} \big| < r'_k
    \right\},
\end{equation}
Then by Eq.~\eqref{eq:interlacing_2},  $\{\lambda_{\min}(\bH + \bDelta_{1,k})/n> 0\}\subseteq \Omega_{3}$. 
We'll bound the probability of $\Omega_3$ for $\bbV$ satisfying~\eqref{eq:constraint_on_V_asymp}. 
Define the Lipschitz test function $f_{\tau_0}:\R\to\R$ as
\begin{equation}f_{\tau_0}(\lambda) = 
    \begin{cases}
        1 & \text{ if } \lambda\leq -\tau_0,\\
        1- \frac{1}{\tau_0}(\lambda +\tau_0)& \text{ if }-\tau_0<\lambda\leq 0\\
        0& \text{ if } 0<\lambda.
    \end{cases}
\end{equation}
This function has Lipschitz modulus bounded by $\tau_0^{-1}$. Furthermore,  
if
$\bbV$ satisfies Eq.~\eqref{eq:constraint_on_V_asymp}, then
$\tau_0 <
    \E_{\mu_\star}\left[ f_{\tau_0}(\Lambda) \right].$
    %\E_{\mu_\star}\left[ f_{1}(\lambda) \right] 
On the other hand, on the event $\Omega_{3},$ we have
   $\Tr\,f_{\tau_0}(\bH /n)
   =   \sum_{i=1}^{n} f_{\tau_0}(\lambda_i(\bH /n))
   \le r_k'.$
%   \le 
%\frac1{n}  \sum_{i=1}^{n}  \one_{\{\lambda_i((\bH +\bS)/n) \le 0\}}
Hence, we can bound for any such $\bbV$
\begin{align}
    %\P\left( \left\{x_{\min}(\mu_{\widehat\nu_{\bV,\bU}}) < -\delta \right\} \cap \cA_n \right)
\P\left(\Omega_{3} \right)
   % &\le \P\left( \left\{\frac{1}{dk}\Tr\left(f_{\delta}(\bH/n)\right)\leq \frac{3  r_k}{dk}\right\} \cap
   % \left\{\E_{\mu_{\star}} 
   % \left[ f_{\delta}(\Lambda)\right] \geq \delta\right\}
   % \right)\\
    &\le  
    \P\left(\left|\frac1{dk} \Tr(f_{\tau_0}(\bH/n)) - \E_{\mu_{\star}}[f_{\tau_0}(\Lambda)] \right| > \tau_0- \frac{r_k'}{dk}\right)\\
    &\le
    \P\left(\left|\frac1{n} \Tr(f_{\tau_0}(\bH/n)) - 
    \frac1{n} \E\left[\Tr(f_{\tau_0}(\bH + /n))\right]
    \right| > \frac{dk}{n}\tau_0- \frac{r_k'}{n} - \omega_{n}(\tau_0)\right)\\
    &
\le
     C_{10} \exp \left\{ 
    -c\frac{ 
    \tau_0(k\tau_0- r_k'/d - \alpha_n\omega_{n}(\tau_0))
    n^{3/2} }{ \alpha_n^{1/2} k \sfK }
    \right\}
\end{align}
where $\omega_{n}(\tau_0) \to 0$ for any $\delta >0$ as $n\to\infty,$ by Proposition~\ref{prop:uniform_convergence_lipschitz_test_functions}.
So we conclude that
for any $\bbV$ satisfying~\eqref{eq:constraint_on_V_asymp}
\begin{align}
&\E\left[  \big|\det\left(\bH  + \bDelta_{1,k}\right)\big|   \Err_\sigma(\bX + \bDelta_{0,k})^{r_k} \one_{\{\bH  + \bDelta_{1,k} \succ 0\}}\right]
\le 
\E\left[  \big|\det\left(\bH +  \bDelta_{1,k}\right)\big|   \Err_\sigma(\bX + \bDelta_{0,k})^{r_k} \one_{\Omega_3}\right]\\
%&\hspace{20mm}\le 
%\E\left[\big|\det\left(\bH + \bS + \bDelta_{1,k}\right)\big|^4\right]^{1/4}\E\left[E(\bX^\sT\bX + \bDelta_{2,k})^{2r_k}\right]^{1/2} \P\left(\one_{\Omega_3}\right)^{1/4}\\
&\hspace{20mm}\le n^{dk + 4r_k}  
C_{11}
\sfC_0^{3r_k}
\;
\sfC_1^{dk} \exp \left\{ 
    -c_2\frac{ 
    \delta(k\delta- r_k'/d - \alpha_n\omega_{n}(\delta))
    n^{3/2} }{ \alpha_n^{1/2} k \sfK }
    \right\}\\
    &\hspace{20mm}\stackrel{(a)}{\le} \exp\left\{ - c_3 \delta^2 n^{3/2}\right\},
\end{align}
where in $(a)$ we took $n > C_{12}(\tau_0)$ for some $C_{12}$ so that so that
$\max\{\sfC_0,\sfC_1\} \le  e^{\sqrt{n}}$ and $\omega_{n}(\tau) <\tau/3.$


\noindent\textbf{Step 4: 
Proof of item \textit{2.}}
We now deal with the determinant term for $(\bTheta,\bbV)$ not satisfying~\eqref{eq:constraint_on_V_asymp}. 
Namely, we'll show concentration of the determinant.

First, note that since $t\mapsto \log t$ is monotonically increasing for $t>0$, when $\bH +  \bDelta_{1,k} \succ \bzero$, we have for any $\tau_1>0$,
  \begin{align}
\label{eq:log_det_to_log_eps_2}
      \log \left(\det((\bH + \bDelta_{1,k})/n)\right) =& \sum_{i=1}^{dk}\log 
      (\lambda_i(\bH+  \bDelta_{1,k} )/n)\\
      %=& \sum_{i=1}^{dk}\log^{(\eps_\bH)}
      %(\lambda_i(\bH+\bE_k))\\
      \leq& \sum_{i=1}^{dk -r'_k}\log \lambda_i(\bH/n) + r_k'\log\left( \frac{\|\bH +  \bDelta_{1,k}\|_\op}{n}\right)\\
      %\leq & \sum_{i=1}^{dk} \log^{(\eps)} \lambda_i(\bH/n) - 3r_k\log\eps + 3r_k\log(2\norm{\bH}_\op)\\
      \leq& \Tr\left(\log^{(\tau_1)}(\bH /n) \right) + 
r_k'\log\left( \frac{\|\bH+  \bDelta_{1,k}\|_\op}{n \tau_1}\right),
  \end{align}
where we defined $\log^{(\tau_1)}(t) := \log(t \vee \tau_1).$
Combining with the result of Step 1, we conclude that
\begin{align}
\label{eq:step_2_det_conc_result}
&\E[|\det \big(\de \bz\big)|\one_{\bH  \succ \bzero} \big|\bzeta = 0 ]\\
&
\quad{\le}
\frac{n^{dk}}{(n\sfsigma_\bH)^{r_k}}
\E\left[ 
\exp\left\{
\Tr\log^{(\tau_1)}\left(\frac{1}n\bH\right) \right\}  
\left( \frac{\|\bH + \bDelta_{1,k}\|_\op}{n \tau_1}\right)^{12 r_k}
\Err_0(\bX + \bDelta_{0,k})^{r_k} \one_{\{\bH +  \bDelta_{1,k} \succ \bzero\}}\right].
\end{align}
Noting that for any $\eps>0$, 
$t\mapsto\log^\up{\eps}(t)$ has Lipschitz modulus bounded by $\eps^{-1}$, we now apply Lipschitz concentration (Lemma~\ref{lemma:concentration_lipschitz_func}) to $\Tr\log^\up{\tau_1}(\bH /n)$.
To this end, for any $t_n>0$, define the event
%\begin{equation}
%   t >  \frac{\sfK}{\alpha_n^{1/2}} \frac{k}{\eps n^{1/2}},
%\end{equation}
define the event 
\begin{equation}
    \Omega_4 := \left\{
    \left|\frac1{n}\Tr\Big(\log^{(\tau_1)}
    \left(\bH/{n}\right)\Big) - 
    \frac1{n}\E_\bX\left[ \Tr\Big(\log^{(\tau_1)}
    \left(\bH/{n}\right)
    \Big)\right]\right| \le t_n
    \right\}.
\end{equation}
Then the expectation in Eq.~\eqref{eq:step_2_det_conc_result} is bounded as
\begin{align}
&\E\left[ 
\exp\left\{
\Tr\log^{(\tau_1)}\left(\frac{\bH}n\right) \right\}  
\left( \frac{\|\bH + \bDelta_{1,k}\|_\op}{n \tau_1}\right)^{r_k'}
\Err_\sigma(\bX + \bDelta_{0,k})^{r_k} \one_{\{\bH +  \bDelta_{1,k} \succ \bzero\}}\right]\\
&\quad\le 
\exp\left\{
\E\left[\Tr\log^{(\tau_1)}\left(\frac{\bH}n\right) \right] + nt_n
\right\}
\E\left[ 
\left( \frac{\|\bH + \bDelta_{1,k}\|_\op}{n \tau_1}\right)^{r_k'}
\Err_\sigma(\bX + \bDelta_{0,k})^{r_k} \one_{\{\bH +  \bDelta_{1,k} \succ \bzero\}}\right]\\
&\hspace{20mm}+
\E\left[ 
\left( \frac{\|\bH +  \bDelta_{1,k}\|_\op}{n}\right)^{dk}
\Err_\sigma(\bX + \bDelta_{0,k})^{r_k} \one_{\{\bH +  \bDelta_{1,k} \succ \bzero\}} \one_{\Omega_4^c}\right]\\
&\le 
\Delta_3\Bigg(\exp\left\{
\E\left[\Tr\log^{(\tau_1)}\left(\frac{\bH}n\right) \right] + nt_n
\right\}
\left(\frac{1}{\tau_1}\right)^{12r_k}
\Delta_4(24 r_k)^{1/2}+
\P(\Omega_4^c)^{1/4}
\Delta_4(4dk)^{1/4}
\Bigg) ,
\label{eq:concentration_bound_decomp}
\end{align}
%
Choosing $t_n$ in the definition of $\Omega_4$ as 
\begin{equation}
\label{eq:choice_of_t_concentraion}    
%t_n :=  \frac{\sfK k^2\log(n)}{ n^{1/2} \eps \alpha_n^{1/2}}
%\quad 
%\bns{\textrm{change to}}
t_n :=  \frac{ c_1\sfK k}{ \tau_1 \alpha_n^{1/2}} n^{-1/4}
\end{equation}
for appropriate constant $c$, we conclude 
by Lemma~\ref{lemma:concentration_lipschitz_func}
we have
$\P(\Omega_4^c) \le  C_{13}(\sfK) \exp\left\{- n^{5/4}\right\}$.
%\begin{equation}
%    %\P(\Omega_4^c) \le  C_5(\sfK) \exp\left\{- n c_1(\sfK) \log(n) k\right\}.
%   %\bns{\textrm{Change to}} 
%    \P(\Omega_4^c) \le  C_5(\sfK) \exp\left\{- c_1(\sfK)n^{3/2} \eps_n\right\}.
%\end{equation}
Then after combining with Eq.~\eqref{eq:concentration_bound_decomp} along with the bounds on $\Delta_3$ and $\Delta_4$ derived previously we have
\begin{align}
&\E[|\det \big(\de \bz\big)|\one_{\bH  \succ \bzero} \big|\bzeta = \bzero ]\\
&
\quad{\le}
\frac{\; \sfC_0^{3r_k}\; n^{dk+4r_k}}{(n\sfsigma_\bH)^{r_k}}
\Bigg(\exp\left\{
\E\left[\Tr\log^{(\tau_1)}\left(\frac{\bH}n\right) \right] + n\frac{c_1 \sfK k }{\tau_1 \alpha_n^{1/2}} n^{-1/4}
\right\}
\left(\frac{\sfC_1}{\tau_1}\right)^{12r_k}
+C_{14}\sfC_1^{dk} \exp\left\{- n^{5/4}  \right\}
\Bigg)\\
&\quad{\stackrel{(a)}{\le}}
{n^{dk}}
\Bigg(\exp\left\{
\E\left[\Tr\log^{(\tau_1)}\left(\frac{\bH}n\right) \right] + n \frac{C_{15} n^{-1/4}}{\tau_1}
\right\}
+C_{16}\exp\left\{- n^{5/4}\right\}
\Bigg)
\end{align}
where in $(a)$ we took $n > C_{17}$ so that  $\sfC_0,\sfC_1 \le  e^{n^{1/2}}$ and
used that
$\sfsigma_{\bH}^{-1} =  e^{o(n)}$.
\qed








\subsection{Asymptotics of the Kac-Rice integral}
\label{sec:kr_asymptotics}
What remains now is to study the asymptotics of the integral to derive the upper bound of Theorem~\ref{thm:general}. 
Let us begin by proving Lemma~\ref{lemma:asymp_1} in the next section.

\subsubsection{Proof of Lemma~\ref{lemma:asymp_1}}
\label{sec:proof_prop_asymp_1}

Fix $\tau_0,\tau_1 \in(0,1)$ and $\beta$ as in the statement of the proposition.
Once again, we suppress the indices $(\bTheta,\bbV)$ in the arguments.

\noindent\textbf{Step 1: obtaining the hard constraint on the support.}
First, we show that
\begin{align}
\mathrm{(I)} &:= \limsup_{n\to\infty}\frac1n\log\left(
\E_\bw\left[
\int_{\cM}
\E[|\det \big(\de \bz\big)|\one_{\bH \succ n\sfsigma_\bH} \big|\bzeta = 0 ,\bw]
p_{\bTheta,\bbV}(\bzero)
\right)
\one_{\bw\in\cG}
\right]
\one_{\mu_{\star}(\hnu,\hmu)((-\infty,- \tau_0)) \ge \tau_0} \de_{\cM} V\\
&= -\infty.
\end{align}
Directly by item \textit{(2.)} of Lemma~\ref{lemma:CE_bound}, followed by the bound on $p_{\bTheta,\bbV}(\bzero)$ of Corollary~\ref{cor:uniform_density_bound}, we have for some $C_0,c_0$ depending only on $\sOmega$,
\begin{align}
\textrm
{(I)}&\le 
\limsup_{n\to\infty}\frac1n \log\left(
 C_0 e^{-c_0 \tau_0^2 n^{3/2}} 
 \E_\bw\left[
\int_{\cM} 
p_{\bTheta,\bbV}(\bzero)
\de_{\cM} V\;
\one_{\bw \in\cG}
\right]
\right)\\
&\le 
\limsup_{n\to\infty}\frac1n \log\left(
\frac{
 C_0 e^{-c_0 \tau_0^2 n^{3/2}}
\sfsigma_{\bR}^{-nk/2} \sfsigma_{\bL}^{-(d-r_k)/2}}{ (2\pi)^{(dk + nk + nk_0 -r_k)/2} n^{dk/2}}
 \E_\bw[\vol(\cM) \one_{\bw \in \cG}]
\right).
\end{align}
To estimate $\vol(\cM)$, we use Lemma~\ref{lemma:manifold_integral} with $f=1$ and the $\beta$ chosen. Letting $\Err_{\textrm{blow-up}}(\beta,n)$ be the multiplicative error defined therein, we have
\begin{equation}
    \vol(\cM) \le 
    \Err_{\textrm{blow-up}}(\beta,n)
    \vol(\cM^\up{\beta}) \le 
    \Err_{\textrm{blow-up}}(\beta,n)
     \vol\left(\Ball_{(k+k_0)\sfA_{\bbV}}^{n(k+k_0)}(\bzero)\right)
    \vol\left(\Ball_{(k+k_0)\sfA_{\bTheta}}^{dk}(\bzero)\right)
\end{equation}
where we used that $\cM \subseteq
\Ball_{(k+k_0)\sfA_{\bV}}^{n(k+k_0)}(\bzero) \times \Ball_{(k+k_0)\sfA_{\bTheta}}^{dk}(\bzero)$. Evaluating these terms, substituting into the upperbound on (I), then taking $n\to\infty$ shows the claim.  (Recall that $\sfsigma_\bR^{-1},\sfsigma_\bL^{-1},\sfA_{\bbV},\sfA_{\bR} = O(1)$).

\noindent\textbf{Step 2: bounding the asymptotically dominating term.}
Define
\begin{align}
F_{n,\tau_1}(\bbV,\bTheta,\bw)
&:=
 \frac{k}{2\alpha_n}\log(\alpha_n)+
\frac{k}{\alpha_n}\E\left[\frac1{dk}\Tr\log^{(\tau_1)}\left(\frac{\bH}n\right)\Big| \bw \right]  - \frac{1}{2\alpha_n}\log \det\left(\frac{\bL^\sT\bL}{n}\right)\\
&\quad+ \frac{1}{2\alpha_n}\Tr(\bTheta^\sT\bTheta) 
   -\frac{n}{2}
\Tr\left(\bRho (\bL^\sT\bL)^{-1}\bRho^\sT\right) + \frac1{2}\Tr\left(\frac1n\bbV^\sT\bL (\bL^\sT\bL)^{-1}\bL^\sT \bbV \bR^{-1}\right)\\
&+ \frac12 \Tr\left(\frac1n\bbV(\bI - \bR^{-1})\bbV^\sT\right)
-\frac1{2}\log\det(\bR).
\end{align}
We show that
\begin{align}
\mathrm{(II)} &:=
\limsup_{n\to\infty}\frac1n\log\left(
\E_\bw\left[
\int_{\cM}
\E[|\det \big(\de \bz\big)|\one_{\bH \succ n\sfsigma_\bH} \big|\bzeta = 0, \bw ]
\; \one_{\bw \in \cG}
\right]
\right)
\one_{\mu_{\star}(\hnu,\hmu)((-\infty, -\tau_0])< \tau_0} \de_{\cM} V \\
&\le \limsup_{n\to\infty}
   \frac1n\log\left( \E_\bw\left[\int_{\cM}\exp\left\{nF_{n,\tau_1}(\bbV,\bTheta)\right\} p_{1}(\bbV) p_{2}(\bTheta)
   \one_{\mu_{\star}(\hnu,\hmu)((-\infty, -\tau_0]) < \tau_0}
   \de_\cM V
   \; \one_{\bw\in\cG}
   \right]\right)
   \label{eq:bound_step_2_asymptotics_1}
\end{align}
In what follows, we use 
\begin{equation}
    K_{n,\tau_1}(\bbV,\bTheta,\bw) := 
\E\left[\frac1{dk}\Tr\log^{(\tau_1)}\left(\frac{\bH}n\right) \bigg| \bw\right],
\end{equation}
By item \textit{(1.)} of Lemma~\ref{lemma:CE_bound},
there exists $C_1$ depending only on $\sOmega$ such that
for any $(\hnu,\hmu)$ satisfying the 
support condition
$\mu_{\star}(\hnu,\hmu)((-\infty, -\tau_0])< \tau_0$,
\begin{align}
\E[|\det \big(\de \bz\big)|\one_{\bH  \succ n\sfsigma_\bH} \big|\bzeta = \bzero,\bw ]\le
n^{dk}
\Bigg(\exp\left\{
\frac{nk}{\alpha_n} K_{n,\tau_1}(\bbV,\bTheta,\bw) +  \frac{C_1 n^{1-1/4}}{\tau_1}
\right\}
+ \exp\left\{ -n^{5/4} \right\}
\Bigg)\
\end{align}
Note that we have the uniform-bound
\begin{align}
    \exp\left\{- n^{5/4}\right\} \le \exp\left\{\frac{nk}{\alpha_n} \log(\tau_1) + \frac{C_1 n^{1-1/4}}{\tau_1}\right\} &\le 
    \exp\left\{
\frac{nk}{\alpha_n} K_{n,\tau_1}(\bbV,\bTheta,\bw) +  \frac{C_1 n^{1-1/4}}{\tau_1}
\right\},
\end{align}
holding for $n$ large enough, uniformly over all $(\bTheta,\bbV,\bw)$.
Then since the term $n^{1-1/4}C_1/\tau_1$ is exponentially trivial, we conclude that
\begin{equation}
   \mathrm{(II)}  \le  \limsup_{n\to\infty} \frac1n \log \E_\bw\left[ \int_{\cM} 
   \exp\left\{
   \frac{nk}{\alpha} K_{n,\tau_1}(\bbV,\bTheta,\bw)
   \right\}
\one_{\{\mu_{\star}(\hnu,\hmu)((-\infty, -\tau_0])< \tau_0\}}
   p_{\bbV,\bTheta}(\bzero)\de_\cM V \; \one_{\bw\in\cG}\right].
\end{equation}
What remains to
conclude Eq.~\eqref{eq:bound_step_2_asymptotics_1} is to recall the bound on $p_{\bbV,\bTheta}(\bzero)$ in Lemma~\ref{lemma:density_bounds} and ignore the exponentially trivial factors of $\sfA_{\bL}^{r_k^2/2}$, and simplify to obtain a bound in terms of of $F_{n,\tau_1}$.

\noindent \textbf{Step 3: Estimating the integral over the manifold.}
We now rewrite the bound on $\mathrm{(II)}$ as an expectation over the blow-up of the manifold from Lemma~\ref{lemma:manifold_integral}. 
Choose a sequence $\beta_n = c_0 \sfsigma_{\bG,n}^3$ for sufficiently small constant $c_0>0$ so that $\beta_n$ satisfies the condition in Lemma~\ref{lemma:manifold_integral} for all $n$ sufficiently large.
We apply this lemma with this chosen value of $\beta_n$ to the function 
\begin{equation}
    f(\bTheta,\bbV) :=  e^{nF_{n,\tau_1}(\bTheta,\bbV)} p_1(\bbV)p_2(\bTheta).
\end{equation}
Via Wielandt-Hoffman, it's easy to verify that under Assumption~\ref{ass:loss} and~\ref{ass:regularizer}, guaranteeing the Lipschitzness and local Lipschitzness of the derivatives of the loss and the regularizer, respectively, that

\begin{equation}
 \|\log f\|_{\Lip,\cM^\up{1}} 
\le C_3(\tau_1, \sfA_{\bR},\sfA_{\bV},\sfA_{\bw},\sfsigma_{\bR},\sfsigma_{\bV},\alpha_n, r_k) \; n
\end{equation} 
for some $C_3$ that remains bounded for $\alpha_n$ in a compact subset of $(1,\infty)$, so that
\begin{equation}
 \lim_{n\to\infty}\frac1n \left(\beta_n\|\log f\|_{\Lip,\cM^\up{1}}  \right) = 0
\end{equation}
by the choice of $\beta_n$.
Similarly,
recalling $\Err_{\textrm{blow-up}}(\beta,n)$ the multiplicative error term defined in Lemma~\ref{lem:intg-tube}, we see that
\begin{align}
    \limsup_{n\to\infty} \log(
    \Err_{\textrm{blow-up}}(\beta_n,n)) 
    &=
   \limsup_{n\to\infty} \frac1n \log\left(\left(\frac1{1 - \beta_n\;r_k^2 C_4 }\right)^{(m-r_k)/2}
        \left(\frac{r_k^{5/2} C_4}{\beta_n(\sfsigma_{\bG,n} - \beta_n C_4 r_k^2)}\right)^{r_k}\right) =0,
\end{align}
by the choice of $\beta_n$ and Assumption~\ref{ass:params} that $\sfsigma_{\bG,n} = e^{-o(n)}$.
Combining with \textbf{Step 2} we conclude that
\begin{align}
   \mathrm{(II)} &\le \limsup_{n\to\infty}\frac1n\log
   \E\left[\E\left[\exp\left\{nF_{n,\tau_1}(\bbV,\bTheta)\right\}
   \one_{\{\mu_{\star}(\hnu,\hmu)((-\infty,-\tau_0]) < \tau_0\} \cap \cM^{(\beta_n)}} \Big| \bw
   \right] \one_{\bw\in\cG}\right],
%   &\limsup_{n\to\infty}\frac1n\log
%   \E\left[\E\left[\exp\left\{nF_{n,\tau_1}(\bbV,\bTheta)\right\}
%   \one_{\{\mu_{\star}(\hnu,\hmu)((-\infty, \tau_0]) < \tau_0\} \cap \cM^{(\beta_n)}} \Big| \bw
%   \right] \one_{\bw\in\cG}\right]
\end{align}
where the expectation is under $p_1(\bbV),p_2(\bTheta).$


%For instance, note that we have
%\begin{equation}
%    \|\log p_1\|_{\Lip,\cM^\up{1}} \le C_2 (k+k_0)^{1/2} (1 + \sfA_{\bV} ) 
%    \quad\quad\textrm{and}\quad\quad
%    \|\log p_2\|_{\Lip,\cM^\up{1}} \le C_3 k^{1/2} (1 + \sfA_{\bR}) 
%\end{equation}
%for universal constants $C_2,C_3 >0.$
%
%Similarly for $F_{n,\tau_1}(\bbV,\bTheta)$: we have
%\begin{equation}
%    \frac{nk}{dk \alpha_n}\| K_{n,\tau_1}(\bbV,\bTheta)\|_{\Lip,\cM^\up{1}} \le  \|\\|(dk)^{1/2}\|\|
%\end{equation}
%

\noindent\textbf{Step 4: Concluding.} Finally, we write the bound on \textrm{(II)} in terms of empirical measures $\hmu,\hnu$ of $\sqrt{d}[\bTheta,\bTheta_0]$, $[\bbV,\bw]$ respectively.
Set
\begin{equation}
K_{\tau_1}(\hnu,\hmu) := \int \log(\lambda \vee \tau_1) \mu_{\star}(\hnu,\hmu)(\de \lambda)
\end{equation}
then note that
by the uniform bounds of Proposition~\ref{prop:uniform_convergence_lipschitz_test_functions}, we have
\begin{align}
\exp\left\{\frac{nk}{\alpha_n} K_{n,\tau_1}(\bbV,\bTheta) 
\right\}
\le
\exp\left\{\frac{nk}{\alpha_n} K_{\tau_1}(\hnu,\hmu) +   \frac{nk}{\alpha_n} \omega_{\textrm{ST}}(n, \tau_1)
\right\}
\end{align}
for $\omega_{\textrm{ST}}(n, \tau_1) = o(1)$ uniformly over $\hnu,\hmu$ so that 
$n k\omega_{\textrm{ST}}(n, \tau_1)/\alpha_n$ is exponentially trivial.
Furthermore, it's easy to check that for any $\beta \in (0,1)$, 
we have
\begin{equation}
 \cM^\up{\beta_n} \subseteq \{
 (\bTheta,\bbV) : (\hmu,\hnu) \in \cuM^\up{\beta_n}  \}
 \subseteq \{
 (\bTheta,\bbV) : (\hmu,\hnu) \in \cuM^\up{\beta}  \}
\end{equation}
for $n$ sufficiently large,
since $\beta_n \to 0$.
So since the integrand is nonnegative, 
and combining with the bound on \textrm{(II)} from \textbf{Step 3}, and recalling the definition of $\phi_{\tau_1}$ from the statement of the proposition, we obtain
\begin{equation}
   \mathrm{(II)} \le \limsup_{n\to\infty}\frac1n\log
   \E_\bw\left[\E\left[\exp\left\{n\phi_{\tau_1}(\hnu,\hmu)\right\}
   \one_{\{\mu_{\star}(\hnu,\hmu)((-\infty, -\tau_0]) < \tau_0\} \cap \cuM^{(\beta)}}
   | \bw \right] \one_{\bw\in\cG}\right] 
   %+ o_\beta(1;\tau_1),
\end{equation}
Finally, combining this with the bound on $\textrm{(I)}$ from \textbf{Step 1}, and invoking Lemma~\ref{lemma:kac_rice_manifold} gives the result of the lemma.
\qed


\subsubsection{Large deviations and completing the proof of Theorem~\ref{thm:general}}
\label{sec:proof_thm1_large_deviations}

To obtain the asymptotic formula of Theorem~\ref{thm:general} and complete the proof, we study the limit obtained in Lemma~\ref{lemma:asymp_1} and obtain an upper bound via Varadhan's integral lemma. 


%\begin{lemma}
%\label{lem:varadhan}
%     Working under the assumptions of Section~\ref{sec:assumptions} and the notation of Lemma~\ref{lemma:asymp_1} and Theorem~\ref{thm:general},
%   % \begin{equation}
%   %     \cuV:= \overline{\cuM}\cap\{(\mu,\nu):\mu_{\star}(\mu,\nu)((-\infty,0)) =0,\;\; \nu_{w}=\P_w, \;\;\mu_{{\btheta_0}}=\mu_{0}\},
%   % \end{equation}
%   % and for $\delta>0$, let $\cG_\delta$ be any set satisfying
%   % \begin{equation}
%   % \cG_\delta \subseteq \cB_{\sfA_{w}\sqrt{n}}^n (\bzero) \cap \{\bw \in\R^n :  d_{\textrm{LU}}(\hnu_{\bw}, \P_w) < \delta\}.
%   % \end{equation}
%  we have 
%    \begin{equation}
%        \lim_{\delta\to 0}\limsup_{n\to \infty}\frac1n \log\left(
%        \E[Z_n(\cuA,\cuB,\sPi,\bTheta_0)
%         \one_{
%             \cG_\delta}]\right) \le
%        \sup_{(\nu,\mu)\in \cuS}
%        \big\{\phi_0(\nu,\mu) - \KL(\nu_{\bv|\bw}\| \cN(\bzero,\bI_{k+k_0})) 
%        -\frac1\alpha \KL(\mu_{\btheta|{\btheta_0}}\| \cN(\bzero,\bI_k)) \big\}      
%    \end{equation}
%\end{lemma}
%







%To apply Varadhan's Lemma, we must first establish a Large Deviation Principle (LDP) for the empirical measure $(\hmu,\hnu)$. To do so, we find good rate functions for $\hmu$, and $\hnu$  separately and then use the contraction principle to derive a good rate function for $(\hmu,\hnu)$.

%Let us first obtain the LDP for $\hmu$.
% Let $C_b(\R^{k_0}) $ denote the collection of bounded continuous
%    functions mapping $\R^{k_0}$ to $\R$.
%For any $\varrho \in C_b(\R^{k_0})$, 
%   define the logarithmic moment generating function as the limit
%    \begin{equation}
%        \Lambda_0(\varrho):=  \lim_{n\to\infty} \frac1n \log \E\left[
%       \exp\left\{ \sum_{i=1}^d\varrho(\sqrt{d} \btheta_{0,j}) \right\} \right]
%       = \lim_{n\to\infty}  \frac1n
%       d \langle \varrho,\hmu_{\sqrt{d} \bTheta_0}\rangle
%         = 
%         \frac1\alpha\langle \varrho,\mu_{0} \rangle.
%    \end{equation}
%    Noting that $\{\hmu_{\sqrt{d_n}\bTheta_0}\}_{n \in \N}$ is a deterministic sequence of points in the set $\cuP(\R^{k_0})$,
%     indexed by $n$,
%     that converge in the weak topology, the collection is exponentially tight, and therefore 
%by the G\"artner-Ellis theorem alongside 
%the weak convergence implied by Assumption \ref{ass:theta_0} of $\hat\mu_{\sqrt{d}\bTheta_0}$ to $\mu_0,$ 
%     $\hmu_{\sqrt{d}\bTheta_0}$ satisfies a full LDP with good rate function $\Lambda^\star_0$ defined as the Legendre transform of $\Lambda_0$:
%     \begin{equation}
%         \Lambda^\star_0(\tilde\mu_0):= \sup_{\varrho\in C_b(\R^{k_0})} 
%         \left\{ \langle\varrho,\tilde\mu_0\rangle - \langle \varrho,\mu_{0}\rangle\right\},
%     \end{equation}     
%     for $\tilde\mu_0 \in\cuP(\R^{k_0})$.
%Meanwhile, an immediate application of Sanov's Theorem establishes an LDP for $\hmu_{\sqrt{d}\hat\bTheta}$ when $\hat\bTheta$ with rows i.i.d. from $\normal(\bzero,\bI_k/d)$, with good rate function 
%\begin{equation}
%I_1(\tilde\mu_1) := \frac{1}{\alpha}\KL(\tilde\mu_1\|\cN(\bzero,\bI_k))
%\end{equation}
%for $\tilde\mu_1 \in\cuP({\R^{k}}).$
%
%Letting $F:\cuP(\R^{k}) \times \cuP(\R^{k_0}) \mapsto \cuP(\R^{k+k_0})$ defined by
%\begin{equation}
%    \tilde \mu := F(\tilde \mu_1,\tilde \mu_0) \quad\quad 
%    \tilde \mu_{\cdot | \btheta_0} =  \tilde \mu_1
%\end{equation}
%The contraction principle~\notate{Statement 4.2.7 of Dembo} then gives the good rate function for the joint $\mu$ as 
%\begin{equation}
%    I_m(\tilde \mu) := \inf_{(\tilde \mu_1,\tilde \mu_0): \tilde \mu_{\btheta} }
%\end{equation}
%
%
%
%
%and 
%    $ \KL(\nu\|\cN(\bzero,\bI_{k+k_0})\times \P_w) $, respectively.
%    Also by Lemma 6.2.6 from \cite{dembo2009large}, the laws of the sequences $\{\hmu_{\cI_{\btheta_1}|\cI_{\btheta_0}}\}$ and $\{\hnu\}$ are exponentially tight. Therefore, by contraction principle, $(\hmu,\hnu)$ satisfies the LDP with the good rate function
%    \begin{equation}
%       \frac1\alpha\KL(\mu_{\cI_{\btheta_1}|\cI_{\btheta_0}}\|\cN(\bzero,\bI_k))
%        + \KL(\nu\|\cN(\bzero,\bI_{k+k_0})\times \P_w)  + 
%        \Lambda^\star(\mu_{\cI_0}).
%    \end{equation}
%------






First, by Sanov's Theorem applied to $\hmu$, viewing the marginals 
$\{\hmu_{\sqrt{d_n}\bTheta_0}\}_{n}$ as a deterministic sequence of points in the set $\cuP(\R^{k_0})$  converging to $\mu_0$ (and hence exponentially tight), we have for any Borel measurable $\cU_1\subseteq\cuP(\R^{k+k_0})$,
\begin{equation}
 \lim_{n\to\infty} \frac1n\log \P_{\substack{\bTheta\sim \cN(\bI/d) \\ \bTheta_0\sim \hmu_{\sqrt{d}\bTheta_0}} }\left(\hmu\in\cU_1\right)   \le -\inf_{\substack{\mu\in\overline{\cU_1}\\ \mu_{(\btheta_0)} = \mu_0 }} 
 \frac1\alpha\KL(\mu_{\cdot|\btheta_0}\|\cN(\bzero,\bI_k)).
\end{equation}
Similarly, by Sanov's once again, for any measurable $\cU_2\subseteq\cuP(\R^{k+k_0+1}), $ we have
\begin{equation}
 \lim_{n\to\infty} \frac1n\log \P_{\substack{\bbV\sim \cN(\bI) \\ \bw\sim \P_\bw} }\left(\hnu\in\cU_2\right)   \le -\inf_{\substack{\nu\in\overline\cU_2}} 
 \KL(\nu\|\cN(\bzero,\bI_{k+k_0})\times \P_w).
\end{equation}
The contraction principle then gives the LDP for the pair $(\hmu,\hnu):$ 
\begin{equation}
    \lim_{n\to\infty} \frac1n \log \P((\hmu,\hnu) \in   \cU) 
    \le -\inf_{\substack{(\mu,\nu) \in\cU\\ \mu_{(\btheta_0)} = \mu_0}} I(\mu,\nu)
\end{equation}
 for
    \begin{equation}
        I(\mu,\nu) :=      \frac1\alpha\KL(\mu_{\cdot|\btheta_0}\|\cN(\bzero,\bI_k))+ \KL(\nu\|\cN(\bzero,\bI_{k+k_0})\times \P_w)) 
    \end{equation}
    and $\cU$ measurable subset of $\cuP(\R^{k+k_0}) \times \cuP(\R^{k+k_0+1})$.

  
Now we use this LDP above alongside Varadhan's integration lemma to bound the limit in Lemma~\ref{lemma:asymp_1}.
   % \kas{Mistake here: secon moments are not bounded on the blow up. chagne the def of blow up}
Observe first that one can directly show the exponent $\phi_{\tau_1}(\hmu,\hnu)$ of Lemma~\ref{lemma:asymp_1} is uniformly bounded for $(\hmu,\hnu) \in {\cuM^\up{\beta}}$. Indeed, in Section~\ref{sec:RMT} in the proof of Proposition~\ref{prop:uniform_convergence_lipschitz_test_functions}, we showed that $\mu_\star(\hnu,\hmu)$ is compactly supported, with support bounded uniformly in $(\hmu,\hnu)$, and hence its truncated logarithmic potential is finite. Furthermore, the bounds $\sfA_{\bR},\sfA_{\bbV},\sfsigma_{\bR},\sfsigma_{\bL},\sfsigma_{\bV}$ in the definition of $\cuM$ guarantee uniform bounds on the functionals of $\nu,\mu$ appearing in the definition of $\phi_{\tau_1}$, so we have sufficient exponential tightness to apply Lemma 4.3.6 of \cite{dembo2009large} in what follows:
Define for $\beta\ge 0, \tau_0\ge 0, \delta\ge 0$,
    \begin{equation}
        \cuS(\beta,\tau_0,\delta):= 
        \overline{\cuM^{(\beta)}} \cap 
        \{(\mu,\nu):\; \mu_\star(\mu,\nu)((-\infty,-\tau_0))\le \tau_0, \;\;
        \de_{\textrm{BL}}(\nu_{w},\P_w)\le\delta,
        \},
    \end{equation} 
    and
    \begin{equation}
        \cuS_0(\beta,\tau_0,\delta) := \cuS(\beta,\tau_0,\delta) \cap
       \{\mu_{(\btheta_0)}  = \mu_0\}.
    \end{equation}
    %Next, note that since $\cuS_{\beta,\delta_1,\gamma}$ is a closed set, 
    Then we have
    %foll4owing an application of Lemma~\ref{lemma:asymp_1} gives us :
    \begin{align} 
         \limsup_{n\to\infty}&\frac1n \log ( \E[ Z_n\one_{\bw \in \cG_\delta}])
        \le
        \limsup_{n\to\infty}\frac1n\log
      \E\left[\exp\left\{n\phi_{\tau_1}(\hnu,\hmu)\right\}
       \one_{\{(\hmu,\hnu)\in\cuS(\beta,\tau_0,\gamma)\}}
       \right]\\
%       \le&
%        \sup_{(\mu,\nu)\in \cuS_{\beta,\tau_1,\delta}}
%        \Big\{ \phi_{\tau_1}(\mu,\nu) -  \frac1\alpha\KL(\mu_{\cdot|\btheta_0}\|\cN(\bzero,\bI_k))
%         - \KL(\nu\|\cN(\bzero,\bI_{k+k_0})\times \P_w)  - 
%        \Lambda^\star(\mu_{\cI_0}) \Big\}\\
%        \label{eq:varadhan+blowup}
        \le& \sup_{\substack{(\mu,\nu)\in \cuS_0(\beta,\tau_0,\delta)}}
        \big\{ \phi_{\tau_1}(\mu,\nu) - \frac1\alpha\KL(\mu_{\cI_{\btheta}|\cI_{\btheta_0}}\|\cN(\bzero,\bI_k))
         -\KL(\nu\|\cN(\bzero,\bI_{k+k_0})\times \P_w)) 
        \big\},
        \label{eq:varadhan+blowup}
    \end{align}
    where in the first inequality we used Lemma~\ref{lemma:asymp_1}.
%where in the last inequality we used  that $\Lambda^\star(\nu) = \infty$ for any $\nu\neq \mu_{\btheta_0}$, and $\Lambda^\star(\mu_{\btheta_0}) = 0$.
Now what's left is to show that we can the parameters $\tau_0,\tau_1$ and $\beta$ to $0$, and in the second we used the referenced Varadhan's integral upper bound from~\cite{dembo2009large} and that of $\cuS_0(\beta,\tau_0,\delta)$ is closed. (To see that it is indeed closed, note that we have shown in Section~\ref{sec:RMT} that if $(\mu,\nu) \mapsto \mu_\star$ is continuous in the topology of weak convergence, meanwhile, for a weakly converging sequence of random variables $X_n \to X$, we have $\P(X \in \cU) \le \liminf_{n} \P(X_n \in\cU)$ for any open set $\cU$).

Now note that $\cuS_0(\beta,\tau_0,\delta)$ is a compact subset of $\cuP(\R^{k+k_0})\times \cuP(\R^{k+k_0+1})$. This can be verified through Prokhorov's Theorem: if we show $\cuM^\up{\beta}$ is tight, Prokhorov implies that $\overline{\cuM^\up{\beta}}$ is compact which gives the compactness of $\cuS_0(\beta,\tau_0,\delta);$ the closed subset of $\overline{\cuM^\up{\beta}}$. 

To prove $\cuM^\up{\beta}$ is tight, note that for any $(\mu,\nu)\in {\cuM^\up{\beta}}$, 
letting $\bar\btheta^\sT = [\btheta^\sT,\btheta_0^\sT],$
we have for some $C$ depending on $\beta$,
    $(\sfA_\bR + C(\beta))\bI \succ \E_{\mu}[\bar\btheta \bar\btheta^\sT]$. Hence $\|\E_{\mu}[\btheta]\|_2^2\le (k+k_0)(\sfA_{\bR} + C(\beta))$, so an application of Markov's yields
    $\P_{\mu}\left(\|\btheta\|_2 > t\right) \le (k+k_0)(C(\beta) + \sfA_\bR)/t^2 \to 0$ as $t\to 0$.
With a similar inequality applied with $\nu$ instead of $\mu$ then gives tightness of $\cuM^\up{\beta}$ for any $\beta\ge 0$.


Now 
let us first take $\beta\rightarrow 0$
in the bound of
    Eq.~\eqref{eq:varadhan+blowup}.
Take a sequence $\{\beta_i\}_{i\in\N}$ such that $\beta_i\to 0$, and by compactness, take  $\{(\mu_{\beta_i},\nu_{\beta_i})\in \cuS_0(\beta_i,\tau_0,\delta)\}$ so that 
    \begin{equation}
        \limsup_{\beta\to 0}\sup_{(\nu,\mu)\in \cuS_0(\beta,\tau_0,\delta)} \{\phi_{\tau_1}(\mu,\nu) - I(\mu,\nu)  \}=
        \limsup_{i\to \infty} \{\phi_{\tau_1}(\mu_{\beta_i},\nu_{\beta_i}) - I(\mu_{\beta_i},\nu_{\beta_i})\}.
    \end{equation}
Noting that $\{\cuS_0(\beta_i,\tau_0,\delta)\}_{i\in \N}$ is a decreasing sequence of closed sets, every converging subsequence of $\{(\mu_{\beta_i},\nu_{\beta_i})\}$ converges to a point in the set $\bigcap_{i\in \N}\cuS_0(\beta_i,\tau_0,\delta) = \cuS_0(0,\tau_0,\delta)$. Since $I(\mu,\nu)$ is lower semi-continuous and $\phi_{\tau_1}(\mu,\nu)$ 
is continuous on $\cuM^\up{\beta}$ for $\beta$ sufficiently small, we conclude that $\phi_{\tau_1} - I$ is upper semi-continuous so that
    \begin{align}
         \limsup_{\beta\to 0}\sup_{(\nu,\mu)\in \cuS_0(\beta,\tau_0,\delta)} \phi_{\tau_1}(\mu,\nu) - I(\mu,\nu) &\le 
         \sup_{(\nu,\mu)\in \cuS_0(0,\tau_0,\delta)}\phi_{\tau_1}(\mu,\nu) - I(\mu,\nu).
    \end{align}
A similar argument then shows the same for the limit $\tau_0\to 0$
after observing that the sequence of sets indexed by decreasing $\tau_0$ is indeed decreasing, and similarly for the limit $\delta \to 0$. Combining with Eq.~\eqref{eq:varadhan+blowup} and noting that 
$\cuV = \cuS_0(0, 0,0 )$ where $\cuV$ was defined in the statement of Theorem~\ref{thm:general}, we have
    \begin{align} 
    \label{eq:beta-tau1-gamma}
        \limsup_{\delta\to 0}\limsup_{n\to\infty} &\frac1n \log\left( \E[ Z_n(\cuA,\cuB, \sPi,\bTheta_0)] \one_{\cG_\delta}  \right) 
        \le \sup_{(\mu,\nu)\in \cuV }
        \left\{\phi_{\tau_1}(\nu,\mu) - I(\mu,\nu)\right\}.
    \end{align}

Now let us pass to the $\tau_1 \to 0$ limit.
    First, recalling the definition of $\phi_{\tau_1}$ from Lemma~\ref{lemma:asymp_1}, we note that we can write 
    \begin{equation}
       \phi_{\tau_1}(\mu,\nu) - I(\mu,\nu) = F(\mu,\nu) + \int \log(\lambda \vee \tau_1) \mu_{\star}(\mu,\nu)(\de \lambda)
    \end{equation}
    for some $F(\mu,\nu)$ that is uniformly upper bounded on $\cuM$: indeed, the definition of $\cM$ guarantees that all terms (other than the logarithmic potential) are finite, and $-I(\mu,\nu) \le 0$ by definition.

Choosing a sequences
    $\{(\mu_{i},\nu_{i})\in \cuV\}$ 
    and $\{\tau_{i}\}_{i\in\N}$ with $\tau_{i}\to 0$ satisfying
\begin{equation}
 \mathrm{(I)}:= \limsup_{\tau_1\to 0}\sup_{(\nu,\mu)\in \cuV} \{\phi_{\tau_1}(\mu,\nu) - I(\mu,\nu)  \}
        = \limsup_{i\to \infty} \left\{
        F(\mu_i,\nu_i) + \int \log(\lambda \vee \tau_i) \mu_{\star}(\mu_i,\nu_i)(\de \lambda)
        \right\},
\end{equation}
we have by compactness (after passing to a subsequence and relabeling) that
$(\mu_i,\nu_i)$ converge to $(\mu_0,\nu_0) \in \cuV$ so that $\mu_\star(\mu_i,\nu_i) \to \mu_\star(\mu_0,\nu_0)$ weakly.  
Since $\mu_\star$ is compactly supported for any $\mu,\nu$ so that the positive part of the logarithm is uniformly integrable, and $F(\mu,\nu)$ is uniformly upper bounded on $\cM$, we have by Fatou's Lemma that
\begin{equation}
   \mathrm{(I)}  \le F(\mu_0,\nu_0) + \int \log(\lambda) \mu_\star(\mu_0,\nu_0)(\de \lambda) \le \sup_{(\mu,\nu)\in\cuV }  \left\{\phi_0(\mu,\nu) - I(\mu,\nu)\right\}.
\end{equation}
Finally, by noting that
$\KL(\nu\|\cN(\bzero,\bI_{k+k_0})\times\P_w) = \KL(\nu_{\cdot| w}\|\cN(\bzero,\bI_{k+k_0})) $ for $(\mu,\nu) \in\cuV$, we see that $\phi_0(\mu,\nu) - I(\mu,\nu) = -\Phi_\gen(\mu,\nu)$ for $\Phi_\gen$ as in the statement of Theorem~\ref{thm:general}.




%and assume for contradiction that
%\begin{equation}
%    \mathrm{(I)} >  \sup_{(\mu,\nu) \in\cuV} \left\{\phi_{0}(\mu,\nu) -  I(\mu,\nu)\right\}.
%\end{equation}
%
%    
%    \begin{align}
% \mathrm{(I)}:= \limsup_{\tau_1\to 0}\sup_{(\nu,\mu)\in \cuV} \{\phi_{\tau_1}(\mu,\nu) - I(\mu,\nu)  \}
% &> 
%        \limsup_{i\to \infty} \left\{
%        F(\mu_i,\nu_i) + \int \log(\lambda \vee \tau_i) \mu_{\star}(\mu_i,\nu_i)(\de \lambda)
%        \right\}\\
%        \int\limsup_{i\to \infty} \left\{
%        F(\mu_i,\nu_i) + \log(\lambda \vee \tau_i) \mu_{\star}(\mu_i,\nu_i)(\de \lambda)
%        \right\}.
%    \end{align}
%
%\begin{align}
%    \mathrm{(I)}:=\limsup_{\tau_1\to 0} 
%    \sup_{\cuS} \int \log^{(\tau_1)}(\lambda )\mu_\star(\mu,\nu)(\de\lambda)
%    =& \limsup_{\tau_1\to 0} 
%    \int \big(0\wedge\log^{(\tau_1)}(\lambda ) + \log^{(1)}(\lambda)\big)
%    \mu_\star(\mu_{\tau_1},\nu_{\tau_1})(\de\lambda),
%\end{align}
%
% Let $\{(\mu_{\tau_{1,j}},\nu_{\tau_{1,j}})\}_{j\in\cuJ}$ be a converging subsequence of $\{(\mu_{\tau_{1,j}},\nu_{\tau_{1,j}})\}_{i\in\N}$. Since $\cuS$ is compact, the limit point of any converging subsequence of  $\{(\mu_{\tau_{1,j}},\nu_{\tau_{1,j}})\}$ is a point in $\cuS$. Further since for any $(\mu,\nu)\in\cuS$, $\mu_\star(\mu,\nu)((-\infty,0]=0)$,   for any $\lambda>0$, $\limsup_{\tau_1\to 0} 0\wedge\log^{(\tau_1)}(\lambda ) = 0\wedge\log(\lambda)$, and $\log^{(1)}(\cdot)$ is integrable $\mu_\star(\mu,\nu)$, we can use Fatou's lemma alongside the dominated convergence to get
%\begin{equation}\label{eq:tau0}
%       \mathrm{(I)}\le \sup_{(\mu,\nu)\in\cuS} \int \log(\lambda) \mu_\star(\mu,\nu)(\de\lambda).
%\end{equation} 
% Hence combining the Eq.~\eqref{eq:beta-tau1-gamma} and Eq.~\eqref{eq:tau0} gives the desired result.
 

 
     



\section{Technical lemmas for the proof of Theorems~\ref{thm:convexity},~\ref{thm:global_min} and~\ref{thm:simple_critical_point_variational_formula}}

%Since $\rho''(t) = \lambda$ under this assumption, note that for any $(\nu,\mu) \in\cuP(\R^{k+k_0+1})\times \cuP(\R^{k+k_0})$, 
%\begin{equation}
%    \mu_{\star}(\mu,\nu) =
%    \mu_{\MP}(\mu) \boxplus  \delta_{\lambda}.
%\end{equation}
%So we use the notation $\mu_{\star,\lambda}(\nu)$ in this setting. 

%Note that under the setting of Assumption~\ref{ass:convexity}, we have for all $\lambda\ge0$,  and $\nu\in\cuP^(\R^{k+k_0+1})$,
%\begin{equation}
%\supp\left(\mu_{\star,\lambda}(\nu)\right) \subseteq [0,\infty).
%\end{equation}

%Note that under Assumption~\ref{ass:convexity}, for any $\nu\in\cuP(\R^{k+k_0+1}),$  we have
%    \begin{equation}
%        \inf \supp(\mu_\star(\nu)) = -\inf_{\bS \succ \bzero } \frac1k \Tr\left(\frac1\alpha \bS^{-1} - \E_\nu[(\bI_{k} + \grad^2\ell \bS)^{-1} \grad^2 \ell]\right) \ge 0
%    \end{equation}


\subsection{The logarithmic potential: Proof of Lemma~\ref{lemma:variational_log_pot}}
\label{sec:log_pot_proof}
Recall the definition of $K_z$ in Lemma~\ref{lemma:variational_log_pot}. We extend it below to complex $z$:
for any $\nu\in\cuP(\R^{k+k_0+1})$, $z \in \C \setminus \supp(\mu_{\star,0}(\nu))$ 
with $\Im(z) \ge 0$, $\bQ\in\bbH_+^k$,
\begin{equation}
    K_z(\bQ;\nu):= -\alpha z \Tr(\bQ) + \alpha \E_{\nu}[\log\det(\bI + \grad^2 \ell(\bv,\bu, w)\bQ) ]  - \log\det(\bQ) - k (\log(\alpha) + 1)
\end{equation}
where $\log$ denotes the complex logarithm (with a branch on the negative real axis). 

\begin{lemma}
\label{lemma:log_pot_z}
Under Assumptions \ref{ass:regime} to \ref{ass:params} of Section~\ref{sec:assumptions} along with
the additional Assumption~\ref{ass:convexity}, we have
    \begin{equation}
    \label{eq:log_pot}
        k\int\log(\zeta - z) \de \mu_{\star,0}(\zeta)
=  K_z(\bS_\star(z;\nu);\nu),
    \end{equation}  
where 
$\bS_\star(z;\nu)$ is the unique solution of \eqref{eq:fp_eq} 
was defined in Eq.~\eqref{eq:def_S_star} for $z\in\bbH_+$
(see also Eq.~\eqref{eq:def_S_star}), and is extended by analytic continuation to $x\in\R \setminus \supp(\mu_{\star,0}(\nu))$.

%Furthermore, 
%letting 
%\begin{equation}
%    \bS_\star(x;\nu) := \lim_{\eps\to0} \bS_\star(x +  i\eps;\nu), \quad\quad\textrm{for}\quad\quad x\in\R,
%\end{equation}

Consequently, for any $\lambda \ge 0$, and $\nu$ with $\inf\supp(\mu_{\star,0}(\nu))\ge -\lambda$,
\begin{equation}
\label{eq:log_pot_0}
k\int \log(\zeta )\mu_{\star,\lambda}(\nu)(\de\zeta) \le  \limsup_{\delta \to 0+}K_{-(\lambda+\delta)}(\bS_{\star}(-(\lambda+\delta);\nu);\nu).
\end{equation}
%with equality if
%$\inf\supp(\mu_{\star,\lambda}(\nu))> -\lambda$.
\end{lemma}
\begin{proof}
Fix $z\in\bbH_+$.
By taking derivatives, one can easily see that $\bQ = \bS_\star(z;\nu)$ is a critical point of $K_z(\bQ;\nu)$,
whence
\begin{equation}
   \frac{\partial}{\partial z} K_z(\bS_\star(z;\nu);\nu) =  -\alpha \Tr(\bS_\star(z;\nu)) = - k  \int \frac1{\zeta - z} \mu_{\star,0}(\nu)(\de\zeta) =  
k \frac{\partial}{\partial z}\int \log(\zeta )\mu_{\star,0}(\nu)(\de\zeta).
\end{equation}
Equation \eqref{eq:log_pot} now follows by showing
%
\begin{align}
    \lim_{\Re(z)\to-\infty}\left| K_z(\bS_\star(z;\nu);\nu)-k\int \log(\zeta )\mu_{\star,\lambda}(\nu)(\de\zeta)
    \right| = 0\, .
\end{align}
%
Analytic continuation then gives the equality for $z$ on the real line outside of the support.

We next prove Eq.~\eqref{eq:log_pot_0}.


Since $\mu_{\star,\lambda}(\nu)$ is compactly supported
by Corollary~\ref{cor:S_star_min_singular_value_bound},
 we always have 
\begin{equation}
\int_1^{\infty} \log(\zeta )\mu_{\star,\lambda}(\nu)(\de\zeta)  <\infty\, . 
\end{equation}
%
Further, if 
\begin{equation}
\int_{0}^{1} \log(\zeta-z )\mu_{\star,\lambda}(\nu)(\de\zeta)  = -\infty,
\end{equation}
then Eq.~\eqref{eq:log_pot_0} holds trivially.
Therefore, it's sufficient to consider the case where $|\log(\zeta)|$ is integrable with respect to $\mu_{\star,\lambda}$ for any $\lambda\ge 0$. In this case, Eq.~\eqref{eq:log_pot_0} follows directly by domination:
\begin{align}
 \int \log(\zeta )\mu_{\star,\lambda}(\nu)(\de\zeta) 
= 
\lim_{\delta \to 0+}\int \log(\zeta  + \delta)\mu_{\star,\lambda}(\nu)(\de\zeta) 
=
\lim_{\delta \to 0+}
K_{-(\lambda+\delta)}(\bS_{\star}(-(\lambda+\delta);\nu);\nu).
\end{align}
\end{proof}


\begin{lemma}[Local strict convexity of $K$]
\label{lemma:strict_convexity_K}
Fix  $x\in\reals_{\ge 0}$.
Assume that 
\begin{equation}
\label{eq:ass_nondegenerate_as}
    \P_\nu(\grad^2\ell(\bv,\bu,w) = \bzero ) \neq 1.
\end{equation}
Under the 
Assumptions \ref{ass:regime} to \ref{ass:params} 
 of Section~\ref{sec:assumptions} along with Assumption~\ref{ass:convexity}, if $\bS\succ\bzero$ satisfies
\begin{equation}
\label{eq:derivative_K_0}
    \alpha^{-1} \bS^{-1} -\E_\nu[(\bI + \grad^2\ell \bS)^{-1}\grad^2\ell] = x\bI,
\end{equation}
then $\bS \mapsto K_{-x}(\bS; \nu)$ is
strictly convex at $\bS$.
\end{lemma}

\begin{proof}
%We will prove this using the approach of
%Lemma~\ref{lemma:uniqueness_ST} for proving uniqueness of the Stieltjes transform for $\Im(z) > 0$.
%For ease of notation, denote $\bD \equiv  \grad^2\ell$.
%Fix $\bS_1,\bS_2 \succ\bzero$ any two solutions of Eq.~\eqref{eq:derivative_K_0}. In a manner similar to Section~\notate{ref}, we define the operator
%\begin{equation}
%    \bT(\bDelta) := \alpha_n^2 \bS_1 \E[(\bI + \bD \bS_1)^{-1}\bD\bDelta 
%(\bI + \bD \bS_2)^{-1} \bD
%    ]\bS_2.
%\end{equation}
%By the approach of  Lemma~\ref{lemma:uniqueness_ST} and the bounds of Lemma~\ref{lemma:op_norm_bound_power_T}, we note that, to prove uniquness, it's sufficient to show that
%\begin{equation}
%\label{eq:op_norm_bound_for_unique_critical_point_K0}
%    \alpha_n\norm{\E[\bB^{-1/2} \bS_i(\bI + \bD \bS_i)^{-1} \bD \bB (\bI + \bD \bS_i)^{-1} \bD \bS_i \bB^{-1/2}]}_\op < 1
%\end{equation}
%for $i\in\{1,2\}$, and any choice of $\bB \succ\bzero$ of bounded condition number.
%Indeed, this will show that the Neumann series of $\alpha_n\bT$ is convergent implying the invertibility of $(\bI - \alpha_n\bT)^{-1}$, allowing us to carry out an argument analogous to Lemma~\ref{lemma:uniqueness_ST}.
%%
%We will pursue the bound in~\eqref{eq:op_norm_bound_for_unique_critical_point_K0} and leave the remaining details to the reader. Namely, we will prove this bound for $\bB = \bS_i$.
%
%%Suppress the subscript $i$ for $\bS_i$ in what follows, and define for $t>0$,
%%    \begin{equation}
%%        \bF(t)  := \frac1{t\alpha} \bS^{-1}   -\E_\nu[(\bI  + \grad^2 \ell t \bS)^{-1}\grad^2\ell].
%%    \end{equation}
%% 
%%We show next that the expectation in~\eqref{eq:op_norm_bound_for_unique_critical_point_K0} is implied by the relation $\bF'(1) \prec 0$: Indeed, the derivative $\bF'(t)$ can be computed directly as
%%\begin{align}
%%    \bF'(t) &=  
%%   \E[(\bI + t \bD \bS)^{-1}\bD \bS(\bI + t\bD\bS)^{-1}\bD] 
%%   - \frac1{t^{2} \alpha} \bS^{-1}
%%    \\
%%&= 
%%\frac1{t^2}
%%   \bS^{-1/2}\left(t^2\E[\bS^{1/2}(\bI + t \bD \bS)^{-1}\bD \bS(\bI + t\bD\bS)^{-1}\bD\bS^{1/2}] 
%%   - \frac1{\alpha} \right)\bS^{-1/2},
%%   \end{align}
%%from which the claim readily follows.  
%%What remains is to show that $\bF'(1) \prec \bzero$ indeed holds.
%%
%%
%%By symmetry and positive semi-definiteness of $\bD$ and $\bS$, we can rewrite the previous display as
%%\begin{align}
%%\label{eq:F_prime_1_simple_form}
%%\bF'(1) = 
%%   \bS^{-1/2}\left(
%%   \E[(\bS^{1/2}\bD^{1/2}(\bI +  \bD^{1/2} \bS \bD^{1/2})^{-1}\bD^{1/2} \bS^{1/2})^2]
%%   - \frac1{\alpha} \right)\bS^{-1/2}.
%%\end{align}
%%Letting $\bA = \bD^{1/2}\bS^{1/2}$ we note that 
%%\begin{align}
%%    \bI - \bA^\sT(\bI + \bA\bA^\sT)\bA = (\bI + \bA^\sT\bA)^{-1} \succ \bzero,
%%\end{align}
%%so that $\bI \succ \bA^\sT(\bI + \bA\bA^\sT)\bA$, allowing us to conclude 
%%\begin{equation}
%%\bA^\sT(\bI + \bA\bA^\sT)\bA \succeq
%%(\bA^\sT(\bI + \bA\bA^\sT)\bA)^2,
%%\end{equation}
%%with the relation holding \emph{strictly} whenever $\bA$ is invertible. Since $\P_\nu(\bA\, \textrm{invertible})\neq 0$,
%%and noting $\E[(\bA^\sT(\bI + \bA\bA^\sT)\bA)^2]$ corresponds to the expectation term in Eq.~\eqref{eq:F_prime_1_simple_form}, we conclude
%%\begin{align}
%%    \bF'(1)  &\prec
%%   \bS^{-1/2}\left(
%%   \E[(\bS^{1/2}\bD^{1/2}(\bI +  \bD^{1/2} \bS \bD^{1/2})^{-1}\bD^{1/2} \bS^{1/2})]
%%   - \frac1{\alpha} \right)\bS^{-1/2}
%%=\bzero
%%\end{align}
%%by assumption on $\bS$ being a solution to the fixed-point equation. This conclude the proof of the lemma.
%%\end{proof}
%With this choice of $\bB$, we have by symmetry and positive semi-definiteness of $\bD$ and $\bS$, 
%\begin{align}
%\bS^{1/2}(\bI + \bD \bS)^{-1} \bD \bS (\bI + \bD \bS)^{-1} \bD \bS^{1/2}
%   = \left(
%   (\bS^{1/2}\bD^{1/2}(\bI +  \bD^{1/2} \bS \bD^{1/2})^{-1}\bD^{1/2} \bS^{1/2})^2
%   \right)
%\end{align}
%Letting $\bA = \bD^{1/2}\bS^{1/2}$ we note that 
%\begin{align}
%    \bI - \bA^\sT(\bI + \bA\bA^\sT)\bA = (\bI + \bA^\sT\bA)^{-1} \succ \bzero,
%\end{align}
%so that $\bI \succ \bA^\sT(\bI + \bA\bA^\sT)\bA$, allowing us to conclude 
%\begin{equation}
%\bA^\sT(\bI + \bA\bA^\sT)\bA \succeq
%(\bA^\sT(\bI + \bA\bA^\sT)\bA)^2,
%\end{equation}
%with the relation holding \emph{strictly} whenever $\bA$ is invertible. Since $\P_\nu(\bA\, \textrm{invertible})\neq 0$,
%taking expectation proves Eq.~\eqref{eq:op_norm_bound_for_unique_critical_point_K0}, implying $\bS_1 = \bS_2$.
%
%\bns{from here}
For ease of notation, denote $\bD :=  \grad^2\ell$ and suppress its arguments throughout.
%We now show that the function is strictly convex locally around $\bS_0$.
 For any $\bZ\succ\bzero$, let $\bH_\bS(\bZ)$ denote the second derivative tensor of $K_{-x}$ at $\bS$, applied to $\bZ$.
Now let $\bS_0$ be a point satisfying the critical point equation~\eqref{eq:derivative_K_0}.
To save on notation,  we denote
\begin{equation}
   \bM := \bM(\bS_0;\bD):=  \bS_0^{1/2}\bD^{1/2}(\bI+\bD^{1/2} \bS_0\bD^{1/2})^{-1} \bD^{1/2}\bS_0^{1/2},\quad\quad\textrm{and}\quad\quad
   \bA := \bS_0^{-1/2}\bZ\bS_{0}^{-1/2}.
\end{equation}
By Assumption~\ref{ass:convexity}, we have $\bD \succeq \bzero$ almost surely, and hence $\bM$ satisfies  $\bzero \preceq \bM \prec\bI$.
Therefore we can lower bound
\begin{align}
\label{eq:second_derivative_K_0}
  \frac1\alpha\Tr( \bZ \bH_{\bS_0}( \bZ)) &= \frac1\alpha\Tr(\bZ \bS_0^{-1} \bZ \bS_0^{-1}) - 
  \E[
\Tr(\bS_0^{-1/2}\bZ \bS_0^{-1/2}\bM(\bS_0;\bD) \bS_0^{-1/2}\bZ\bS_0^{-1/2}\bM(\bS_0;\bD) )]\\
&= \frac1\alpha \Tr(\bA^2) - \E[\Tr(\bA\bM\bA\bM)]\\
&\stackrel{(a)}{\ge} \frac1\alpha \Tr(\bA^2) - \E[\Tr(\bA\bM\bA)]\\
  %\Tr(\bZ \bD^{1/2}(\bI+\bD^{1/2} \bS\bD^{1/2})^{-1} \bD^{1/2} \bZ \bD^{1/2}(\bI  +\bD^{1/2}\bS\bD^{1/2})^{-1} \bD^{1/2})].
 &\stackrel{(b)}{\ge} x  \Tr(\bZ\bS_0^{-1} \bZ) \ge 0,
\end{align}
for all $x\ge0$,
where $(a)$ follows from $\bM\prec\bI$, and in $(b)$ we used that
%where $(a)$ follows from $\bM\prec\bI$, holding by definition of $\bM$.
for $\bS_0$ satisfying Eq.~\eqref{eq:derivative_K_0}, we have 
$\E[\bM(\bS_0;\bD)] = \alpha^{-1}\bI   - x\bS_0$, giving 
%$\Tr( \bZ \bH(\bS_0) \bZ) \ge 0$ for any $x$ nonnegative,
convexity at $\bS_0$. To show strict convexity, we'll show that the inequality $(a)$ cannot hold with equality under the assumptions of the lemma.

Let $\{\lambda_i\}$ be the eigenvalues of $\bA^{1/2}\bM\bA^{1/2}$, and $\{\tilde\lambda_i\}$ be the eigenvalues of $\bA$. 
Write
\begin{align}
 \Tr(\bA \bM\bA)- \Tr(\bA \bM \bA \bM)
 %&=
%\Tr(\bA^{1/2}\bM\bA^{1/2}\bA)- \Tr(\bA^{1/2}\bM\bA^{1/2} \bA^{1/2}\bM\bA^{1/2})\\
 &= \langle \bA^{1/2}\bM \bA^{1/2}, \bA\rangle -\| \bA^{1/2}\bM \bA^{1/2}\|_F^2 
=  \sum_{i=1}^k \lambda_i \tilde\lambda_i -\sum_{i=1}^k \lambda_i^2.
\end{align}
Since $\bM \prec \bI$, we have $\lambda_i < \tilde\lambda_i$ for all $i$. So if $\bM \neq \bzero$, we have $\lambda_i >0$ for some $i$ so that
    $\sum_{i} \lambda_i \tilde\lambda_i > \sum_i \lambda_i^2.$
So
\begin{equation}
  \Tr(\bA \bM \bA \bM) < \Tr(\bA \bM\bA)
\end{equation}
for all $\bM \neq \bzero$. Since the assumption
in Eq.~\eqref{eq:ass_nondegenerate_as} 
implies that $\P(\bM = \bzero) \neq 1$,  we conclude that $(a)$ holds strictly as desired.
%, where the second relation holds strictly when $\bD\succ\bzero.$ So by the assumption that $\bD\succ\bzero$ on a set of a measure non-zero, we conclude
%\begin{align}
%  \E[
%\Tr(\bS_0^{-1/2}\bZ \bS_0^{-1/2}\bM(\bS_0;\bD) \bS_0^{-1/2}\bZ\bS_0^{-1/2}\bM(\bS_0;\bD) )] &<
%  \E[
%\Tr(\bS_0^{-1/2}\bZ \bS_0^{-1/2}\bM(\bS_0;\bD) \bS_0^{-1/2}\bZ\bS_0^{-1/2})]\\
%&\stackrel{(a)}{=} 
%\frac1\alpha \Tr(\bS_0^{-1}\bZ \bS_0^{-1} \bZ)
%- x\Tr(\bZ\bS_0^{-1}\bZ)
%,
%\end{align}
%where in $(a)$ we used that $\E[\bM(\bS_0;\bD)] = \alpha^{-1}\bI   - x\bS$ by Eq.~\eqref{eq:derivative_K_0}.
%%\begin{align}
%  \Tr(\bZ \bD^{1/2}(\bI+\bD^{1/2} \bS_0\bD^{1/2})^{-1} \bD^{1/2} \bZ \bD^{1/2}(\bI  +\bD^{1/2}\bS_0\bD^{1/2})^{-1} \bD)
%  &= 
%  \Tr(\bS_0^{-1/2}\bZ \bS_0^{-1/2}\bM \bS^{-1/2}\bZ\bS^{-1/2}\bM )\\
%  &\ge 
%  \Tr(\bS^{-1/2}\bZ \bS^{-1/2}\bM \bS^{-1/2}\bZ\bS^{-1/2}),
%\end{align} 
%with strict inequality whenever $\bD\succ\bzero$ is invertible. 
%By the assumption that $\bD \succ\bzero$ with non-zero probability, and noting that $\E[\bM(\bS;\bD)] =(\alpha^{-1}-x)\bI$ whenever $\bS$ satisfies Eq.~\eqref{eq:derivative_K_0}, we conclude that
%Now to conclude, we use this upper bound in~\eqref{eq:second_derivative_K_0}, we can now lower bound
%\begin{align}
%  \frac1\alpha\Tr( \bZ \bH(\bS_0) \bZ) > 
%  x \Tr(\bZ\bS_0^{-1}\bZ) \ge 0,
%\end{align}
%for all $x\ge 0$,
%giving the desired strict convexity.
\end{proof}

%\begin{lemma}[Variational principle for the log potential]
%\label{lemma:variational_log_pot}
%   Under Assumption~\ref{ass:convexity} , for any $\nu\in\cuB$, and $\lambda \ge0$,
%   we have
%    \begin{equation}
%    \label{eq:log_pot}
%        k\int\log(\zeta ) \mu_{\star,\zeta}(\nu) (\de\zeta)
%\le \inf_{\bS\succ\bzero} K_{-\lambda}(\bS;\nu).
%    \end{equation}  
%\end{lemma}
We move on to the proof of Lemma~\ref{lemma:variational_log_pot}
\begin{proof}[Proof of Lemma~\ref{lemma:variational_log_pot}]
Without loss of generality, assume that $\log(\zeta)$ is absolutely integrable under $\mu_{\star,\lambda}$
for all $\lambda\ge 0$. Indeed, for $\lambda>0$ this holds because $\mu_{\star,\lambda}$ is compactly supported inside
$[\lambda,\infty)$.  For $\lambda=0$, the positive part of $\log(\zeta)$ is integrable because
$\mu_{\star,0}$ is compactly supported, and  lack of absolute integrability implies that the integral diverges to $-\infty$.

Under Assumption~\ref{ass:convexity}, we have $\supp(\mu_{\star,0}(\nu))\subseteq[0,\infty),$  so for any $\lambda \ge0,$ Eq.~\eqref{eq:log_pot_0} of Lemma~\ref{lemma:log_pot_z} yields
\begin{equation}
k\int \log(\zeta )\mu_{\star,\lambda}(\nu)(\de\zeta) \le  \limsup_{\delta \to 0}K_{-(\lambda+\delta)}(\bS_{\star}(-(\lambda+\delta);\nu);\nu).
\end{equation}

Now observe that the absolute integrability assumption implies that $\P_\nu(\grad^2\ell(\bv,\bu,w) = \bzero ) \neq 1.$ 
Indeed, otherwise, we must have $\mu_{\star,0}(\nu) = \delta_0$
(for instance, this can be seen by noting for any $z \in\C - \supp(\mu_{\star,0}(\nu))$ with $\Im(z) >0$,
$\bS(z) = \alpha^{-1}z^{-1}\bI$, implying a degenerate measure).
So Lemma~\ref{lemma:strict_convexity_K} holds with $x = \delta + \lambda$.
In particular, this gives that at the point $\bS_\star(-(\delta+\lambda);\nu)$ which satisfies Eq.~\eqref{eq:derivative_K_0}, the continuous function $\bS \mapsto K_{-(\delta+\lambda)}(\bS;\nu)$ is strictly convex, implying that $\bS_\star(-(\delta+\lambda);\nu)$ is the unique global minimum. 
Combining this with the above display gives
\begin{align}
k\int \log(\zeta )\mu_{\star,\lambda}(\nu)(\de\zeta) 
&\le \limsup_{\delta\to 0+}  \inf_{\bS\succ\bzero} K_{-(\lambda+\delta)}(\bS;\nu)
\le 
  \inf_{\bS\succ\bzero}
  \limsup_{\delta\to 0+} 
  K_{-(\lambda+\delta)}(\bS;\nu) = 
  \inf_{\bS\succ\bzero}
  K_{-\lambda}(\bS;\nu),
\end{align}
as desired.

\end{proof}





%\begin{lemma}
%Assume that  \bns{These are the assumptions on $\nu$ that we need. Is convexity enough or do we need $\inf\supp\mu_\star(\nu) > 0.$}
%\begin{equation}
%\left|\int \log(\lambda) \mu_\star(\nu)\de \lambda \right| < \infty,\quad
%\bS_0(\nu) := \lim_{\eps \to 0 }\bS_\star(i \eps; \nu) \succ \bzero,\quad\textrm{and}\quad
%    \P_\nu(\grad^2\ell(\bv,\bu,w) \succ \bzero ) \neq 0.
%\end{equation}
%Then under Assumption~\ref{ass:convexity}, we have
%
%%for any $\nu$ with $\inf\supp(\mu_\star(\nu)) > 0$,
%    \begin{equation}
%    \label{eq:log_pot}
%        k\int\log(\lambda ) \mu_\star(\nu) (\de\lambda)
%= \inf_{\bS\succ\bzero} K_0(\bS;\nu).
%    \end{equation}  
%\end{lemma}
%\begin{proof}
%Recall that for any $\eps>0$, 
%by Lemma~\ref{lemma:log_pot_z},
%\begin{equation}
%\label{eq:recalling_log_pot_z}
%    k\int\log(\lambda - i\eps) \mu_\star(\nu)(\de\lambda)= - i\alpha_n \eps \Tr(\bS_\star(i\eps;\nu)) + K_0(\bS_\star(i\eps;\nu);\nu)
%\end{equation}
%%Since we assume the existence of the log potential of $\mu_\star$, we have by domination
%%\begin{equation}
%%    \lim_{\eps\to 0}k\int\log(\lambda - i\eps) \mu_\star(\nu)(\de\lambda) = 
%%    k\int\log(\lambda ) \mu_\star(\nu)(\de\lambda).
%%\end{equation} 
%%Letting
%%\begin{equation}
%%\bS_0(\nu) := \lim_{\eps \to 0}\bS_{\star}(i\eps;\nu),
%%\end{equation}
%Taking $\eps\to 0$ in Eq.~\eqref{eq:recalling_log_pot_z}, we see that it's sufficient for 
%\begin{equation}
%    \inf_{\bS\succ\bzero} K_0(\bS;\nu) = K_0(\bS_0(\nu);\nu),
%    %,\quad\quad\textrm{and}\quad\quad
% %\bS_0(\nu)  \succ\bzero.
%\end{equation}
%to hold, but
%this follows from the previous lemma.
%\end{proof}

%\begin{lemma}
%Under Assumption~\ref{ass:convexity}, for any $\nu\in\cuP(\R^{k+k_0+1}),$  we have
%    \begin{equation}
%        \inf \supp(\mu_\star(\nu)) = -\inf_{\bS \succ \bzero } \frac1k \Tr\left(\frac1\alpha \bS^{-1} - \E_\nu[(\bI_{k} + \grad^2\ell \bS)^{-1} \grad^2 \ell]\right) \ge 0
%    \end{equation}
%\end{lemma}


%For $\nu$ a measure on $\R^{k+k_0 + 1}$, $\bS\succ\bzero,$
%recall the definition
%\begin{equation}
%    K_{-\lambda}(\bS;\nu):=
%    \lambda \alpha \Tr(\bS)  + 
%    \alpha \E_{\nu}[\log\det(\bI + \grad^2 \ell(\bv,\bu, w)\bS) ]  - \log\det(\bS) - k (\log(\alpha_n) + 1).
%\end{equation}

%\subsection{Variational formula under convexity: Proof of Theorem~\ref{thm:convexity}}


%\begin{align}
%\phi(\nu,\mu,\bS,\bR)
%&:=
% \frac{k}{2\alpha}\log(\alpha)+
%    \lambda \alpha \Tr(\bS)  + 
%    \E_{\nu}[\log\det(\bI + \grad^2 \ell(\bv,\bu, w)\bS) ]  -\frac1\alpha \log\det(\bS) - \frac{k}{\alpha} \log(\alpha_n) -\frac{k}{\alpha}\\
%  &\quad- \frac{1}{2\alpha}\log \det\left( \E_{\nu}[\grad \ell\grad\ell^\sT]\right)
%+ \frac{1}{2\alpha} \Tr\left(\bR_{11}\right) 
%-\frac1{2}\log\det(\bR)
%+ \frac12 \Tr\left((\bI_k - \bR^{-1})\E[\bbv\bbv^\sT]\right)
%\\
%   &\quad-\frac1{2\alpha}
%\Tr\left( (\E_\nu[\grad\ell\grad\ell^\sT])^{-1}\E_\mu[\grad \rho \grad \rho^\sT]\right) + \frac1{2\alpha}\Tr\left(\E\left[\bbv\grad\ell^\sT\right] (\E[\grad\ell\grad\ell^\sT])^{-1}\E[\grad\ell\bbv^\sT] \bR^{-1}\right).
%\end{align}
%
%
%%\begin{theorem}
%%   Consider the setting of Theorem~\ref{thm:general}, and define
%%\begin{align}
%%\phi(\nu,\mu,\bS,\bR)
%%&:=
%% \frac{k}{2\alpha}\log(\alpha)+
%%    \lambda\Tr(\bS)  + 
%%    \E_{\nu}[\log\det(\bI + \grad^2 \ell(\bv,\bu, w)\bS) ]  -\frac1\alpha \log\det(\bS) - \frac{k}{\alpha} \log(\alpha_n) -\frac{k}{\alpha}\\
%%  &\quad- \frac{1}{2\alpha}\log \det\left( \E_{\nu}[\grad \ell\grad\ell^\sT]\right)
%%+ \frac{1}{2\alpha} \Tr\left(\bR_{11}\right) 
%%-\frac1{2}\log\det(\bR)
%%+ \frac12 \Tr\left((\bI_k - \bR^{-1})\E[\bbv\bbv^\sT]\right).
%%% &\quad-\frac{\lambda^2}{2\alpha}
%%%\Tr\left( (\E_\nu[\grad\ell\grad\ell^\sT])^{-1}\bR_{11}\right) + \frac{\lambda^2}{2\alpha}\Tr\left( (\E[\grad\ell\grad\ell^\sT])^{-1} [\bR_{11},\bR_{00}] \bR^{-1}[\bR_{11},\bR_{00}]^\sT\right).
%%\end{align}
%%Then we have
%%\begin{align}
%%   &\limsup_{\delta\to 0 }\limsup_{n\to\infty}\E[Z_n(\cuA,\cuB, \sPi,\bw) \one_{\bw\in\cG_\delta}] \\
%%   &\le 
%%   \sup_{(\mu,\nu)\in\cuM \cap \cuS_0}\inf_{\bS\succ\bzero}\left\{
%%   \phi(\nu,\mu,\bS,\bR(\mu))
%%   -  \KL(\nu_{\cI_{v}|\cI_{w}} \| \cN(\bzero, \bI_{k+k_0})) - \frac1\alpha \left( \mu_{\cI_{\btheta}| \cI_{\btheta_0}}\| \cN(\bzero,\bI_k) \right)
%%   \right\}.
%%\end{align}
%%
%%\end{theorem}
%
%
%\begin{theorem}
%\label{thm:convexity}
%   Consider the setting of Theorem~\ref{thm:general} and Assumption~\ref{ass:convexity}. 
%  For $\bR \succ\bzero,$  define the set
%\begin{align}
%\nonumber
%\cuV(\bR) := \Big\{\nu \in  \cuB &:\;
%\sfA_{\bR} \succ \bR \succ\sfsigma_\bR,\;
%\sfA_\bV \succ \E_{\nu}[\bv\bv^\sT] \succ \sfsigma_{\bV},\;
%\E_{\nu}[\grad\ell \grad\ell^\sT] \succ \sfsigma_{\bL},\;\\
%&\quad\quad\quad\nonumber
%\E_\nu[\grad \ell(\bv,\bv_0,w)(\bv,\bv_0)^\sT] + 
%     \lambda (\bR_{11},\bR_{1,0}) =   \bzero_{k\times (k+k_0)},\;
% \nu_{\cI_{w}} = \P_w,\; \Big\}
%\end{align}
%and
%\begin{equation}
%   \cuT := \{\mu \in\cuA : \mu_{\cI_1} = \mu_{\btheta_0} \}.
%\end{equation}
%Let
%\begin{align}
%\Phi(\nu,\mu,\bS,\bR)
%&:=
%    \lambda\Tr(\bS)  + 
%    \E_{\nu}[\log\det(\bI + \grad^2 \ell(\bv,\bu, w)\bS) ]  -\frac1\alpha \log\det(\bS) - \frac{k}{2\alpha} \log(\alpha) -\frac{k}{\alpha}\\
%  &\quad- \frac{1}{2\alpha}\log \det\left( \E_{\nu}[\grad \ell\grad\ell^\sT]\right)
%+ \frac{1}{2\alpha} \Tr\left(\bR_{11}\right)
%% &\quad-\frac{\lambda^2}{2\alpha}
%%\Tr\left( (\E_\nu[\grad\ell\grad\ell^\sT])^{-1}\bR_{11}\right) + \frac{\lambda^2}{2\alpha}\Tr\left( (\E[\grad\ell\grad\ell^\sT])^{-1} [\bR_{11},\bR_{00}] \bR^{-1}[\bR_{11},\bR_{00}]^\sT\right).
%-  \KL(\nu_{\cI_{v}|\cI_{w}} \| \cN(\bzero, \bR)) - \frac1\alpha \KL ( \mu_{\cI_{\btheta}| \cI_{\btheta_0}}\| \cN(\bzero,\bI_k)).
%\end{align}
%Then we have
%\begin{align}
%   &\limsup_{\delta\to 0 }\limsup_{n\to\infty}\E[Z_n(\cuA,\cuB, \sPi,\bw) \one_{\bw\in\cG_\delta}] 
%   \le 
%\sup_{\mu\in \cuT}
%   \sup_{\nu\in\cuV(\bR(\mu))}\inf_{\bS\succ\bzero}
%   \Phi(\nu,\mu,\bS,\bR(\mu)).
%\end{align}
%\end{theorem}

%\bns{This was moved to main paper. Delete}
%We'll show that under Assumption~\ref{ass:convexity}, the formula for $\phi$ of Theorem~\ref{thm:general} can be simplified to the one in the statement.
%The following identity can be easily verified from the definition of $\KL$:
%\begin{align}
%   -\frac12 \log \det(\bR)  + \frac12 \Tr((\bI_k - \bR)^{-1} \E[\bbv\bbv^\sT]) -
%   \KL(\nu_{\cI_{v}\| \cI_{w}} \| \cN(\bzero, \bI_{k+k_0}))
%= - \KL(\nu_{\cI_{v}\| \cI_{w}} \| \cN(\bzero, \bR)).
%\end{align}
%Meanwhile, for $\rho(t) = \lambda t^2/2$, so that $\grad_{\btheta}\rho(\btheta) = \lambda \btheta,$  one obtains after using the constraint $\E[\grad\ell \bbv^\sT + \grad \rho [\btheta,\btheta_0]^\sT] = \bzero$  that
%\begin{align}
%   &-\frac1{2\alpha}
%\Tr\left( (\E_\nu[\grad\ell\grad\ell^\sT])^{-1}\E_\mu[\grad \rho \grad \rho^\sT]\right) + \frac1{2\alpha}\Tr\left(\E\left[\bbv\grad\ell^\sT\right] (\E[\grad\ell\grad\ell^\sT])^{-1}\E[\grad\ell\bbv^\sT] \bR(\mu)^{-1}\right)\\
%&\hspace{20mm}=
%   -\frac{\lambda^2}{2\alpha}
%\Tr\left( (\E_\nu[\grad\ell\grad\ell^\sT])^{-1}\bR_{11}(\mu)\right) + \frac{\lambda^2}{2\alpha}\Tr\left((\E[\grad\ell\grad\ell^\sT])^{-1}
% [\bR_{11}(\mu),\bR_{10}(\mu)] \bR(\mu)^{-1} [\bR_{11}(\mu),\bR_{10}(\mu)]^\sT
%\right)\\
%&\hspace{20mm}= 0
%\end{align} where the last equality follows from $[\bR_{11},\bR_{10}] \bR^{-1} [\bR_{11},\bR_{10}]^\sT - \bR_{11}$.
%Finally, using the upper bound on the log potential in Lemma~\ref{lemma:variational_log_pot}, one concludes the theorem as a corollary of Theorem~\ref{thm:general}.
%\qed

%\subsection{Variational formula under convexity}
%For $\bTheta \in\R^{d\times k}$, $\bTheta_0\in\R^{d\times k}$, 
%$\hmu_{\bTheta,\bTheta_0}$ to be the empirical distribution of
%rows of $[\bTheta,\bTheta_0]\in \R^{d\times (k+k_0)}$. We further define  
%%
%\begin{align}
%\bR(\hmu_{\bTheta,\bTheta_*}) :=\left(\begin{matrix}
%\bTheta^{\sT}\bTheta & \bTheta^{\sT}\bTheta_0\\
%\bTheta_0^{\sT}\bTheta & \bTheta_0^{\sT}\bTheta_0\\
%\end{matrix}\right)=\int \bt\bt^{\sT} \hmu_{\bTheta,\bTheta_0}(\de\bt)\, .
%\end{align}
%%
%Given block matrix $\bR\in \sS_{k+k_0}$ we define
%%
%\begin{align}
%\bR = \left(\begin{matrix}
%\bR_{11} & \bR_{10}\\
%\bR_{01} & \bR_{00}\\
%\end{matrix}\right)
%\;\;\Rightarrow\;\; \bR/\bR_{00} = \bR_{11}-\bR_{10}\bR_{00}^{-1}
%\bR_{01}\, .
%\end{align}
%%
%Let $\cA\subseteq \cuP(\R^{k+k_0}),\cB \subseteq \cuP(\R^{k+k_0+1})$ and $\eps_\bH,\eps_{\bL},\eps_{\bR}> 0$.
%Define
%%
%\begin{align}
%Z_n(\cA,\cB,\bTheta_0,\eps_\bH,\eps_\bR) := \left|\Big\{\bTheta\in \R^{d\times k}:\; 
%\nabla_{\bTheta} \hR_n(\bTheta)=\bzero,\,
%\nabla^2_{\bTheta} \hR_n(\bTheta)\succeq \eps_\bH,\,
%\;\; \hmu_{\bTheta,\bTheta_0} 
%\in \cA,\,
%\hat\nu_{\bX\bTheta,\bX\bTheta_0} \in\cB,\;
%\bR(\hmu_{\bTheta,\bTheta_0}) \succ\eps_{\bR}
%\Big\}\right|\,.
%\end{align}
%\bns{We will need assumptions guaranteeing a quantity lower bound on $\bL^\sT\bL\succ \eps_\bL$, $\tilde\bV^\sT\tilde\bV\succ\eps_\bV $ from $\bR \succ \eps_\bR$.}
%%
%Further, we denote by 
%\begin{equation}
%\cuL(\bTheta_0) := \left\{
%\mu \in\cuP(\R^{k+k_0})  :  \mu_{\btheta_0} = \widehat\mu_{\bTheta_0},  \bR(\mu) \succ\eps_{\bR}
%\right\},
%\end{equation}
%and
%\begin{align}
%    \cV(\P_w) &:= \Big\{
%    \nu \in\cuP(\R^{k+k_0 + 1}) : \E_\nu[\grad \ell \bv^\sT] = \bzero_{k\times k}, \;
%     \E_\nu[\grad \ell \bu^\sT] = \bzero_{k\times k_0}, 
%     \;
%       \nu_W =  \P_{\bw},\;\\
%&\quad\quad\quad\textcolor{red}{\inf_{\bS \succ\bzero} G(\bS;\nu) < \eps_{\bH}},\;
%\textcolor{red}{\E_\nu[\grad \ell \grad\ell^\sT]\succ\eps_{\bL}},\,
%\textcolor{red}{\E_\nu[(\bv,\bu)(\bv,\bu)^\sT] \succ \eps_{\bV} }
%%    \inf_{\bA \succ \bzero_{k\times k}} 
%    %\frac1k \Tr\left(\bG(\bA; \nu)\right) \le 0
%    \Big\}\, .
%\end{align}
%%
%Further define $\Phi_n:\cuP(\R^{k+k_*})\times\cuP(\R^{k+k_*+1})\times\sS^{k+k_0}\times \sS^k \to \R$
%via
%\begin{align}
%    \Phi_n(\mu,\nu,\bR,\bS)&:=  
%\frac{k}{2 \alpha_n} \log(\alpha_n) -
% \E_{\nu}[\log\det(\bI + \grad^2 \ell(\bv,\bu, w)\bS) ]  +\frac1{\alpha_n} \log\det(\bS) + \frac{k}{\alpha_n}\\
%&\quad+ \frac1{2\alpha_n} \log\det(\E_{\nu_{\tilde\bv,w}}[\grad\ell\grad\ell^\sT]) 
%-\frac{1}{2\alpha_n} \Tr\left(\bR_{11} \right)
%+\KL(\nu_{\tilde\bv | \bw}, \cN( 0 ,\bR))
%+\frac{1}{\alpha_n}\KL(\mu_{\btheta|\btheta_0}, \cN( 0 , \bI_k ))
%\\
% &= \frac{1}{2\alpha_n} \log\det\left(\E_\nu\left[ \grad \ell\grad\ell^\sT\right]\right)
%+ 
%\frac1{\alpha_n}\log\det\bS 
%- \E_{\nu}\left[\log \det \left(\bI_k + \grad^2 \ell^{1/2} \bS \grad^2 \ell^{1/2}\right)\right]
%\nonumber\\
%&\quad+\frac{k}{2\alpha_n} 
%+ \frac{k}{2\alpha_n} \log(\alpha_n)
%-\frac1{2\alpha_n} \log\det\left(
%\bR/\bR_{00}
%\right)  \\
%&\quad + \KL\left(\nu_{\bv,\bu | W}\| \cN(\bzero, \bR)\right)
%+\frac{1}{\alpha_n}\KL\big(\mu_{\btheta|\btheta_0}\| \normal(0,\bR/\bR_{00})\big)
%-\frac1{2\alpha_n} \Tr\left(\E_{\mu_{\btheta|\btheta_0}}[\btheta\btheta^\sT(\bR/\bR_{00})^{-1}]\right)
%.\nonumber
%\end{align}
%Let $G: \sS^k\times \cuP(\R^{k+k_0+1})\mapsto \R $ be defined by
%\begin{equation}
%    G(\bS ;\nu ) := \frac1k \Tr\left(\frac1\alpha_n \bS^{-1} - \E_\nu[(\bI_{k} + \grad^2\ell \bS)^{-1} \grad^2 \ell]\right).
%\end{equation}
%
%%\textcolor{blue}{
%%\begin{align}
%%     \frac{d}{2}\Tr\left( \E_{\mu_{\theta| \theta_0}} \left[\bR_{11}\right]\right) 
%%    -\KL(\mu_{\btheta|\btheta_0}\| \cN(0,\bI)) &= 
%%      H(\mu_{\btheta| \btheta_0}) -  \frac{d}{2}\log(2\pi)\\
%%     &=
%%      H(\mu_{\btheta| \btheta_0})  
%%    + \E_{\mu_{\btheta|\btheta_0}}\left[\log\left( \frac{\det(\bR/ \bR_{00})^{-d/2}}{(2\pi)^{d/2}}\exp\left\{-\frac{1}{2} \btheta^\sT(\bR/\bR_{00})^{-1}\btheta
%%    \right\}\right)\right]\\
%%&\quad+\frac{d}{2}\log\det\left(\bR/\bR_{00}\right) + \frac12 \Tr\left(\E_{\mu_{\btheta|\btheta_0}}\left[\btheta\btheta^\sT\right](\bR/\bR_{00})^{-1}\right).
%%\end{align}
%%}
%
%
%\begin{theorem}[Formula under convexity]
%\label{thm:convexity}
%Under Assumption~\notate{ref, and assumption on $\cA,\cB$ from LDP},
% we have  
%\begin{align}
% \frac{1}{n}\log\E_{\bX}\left[Z(\cA,\cB,\bTheta_0,\eps_\bH,\eps_\bR)\right]
%    =-\inf_{ \mu\in\cA\cap\cuL(\bTheta_0)} 
%    \inf_{\nu \in \cB\cap\cV(\P_w)}\sup_{\bS\succ\bzero}\Phi_n(\mu,\nu, \bR(\mu), \bS) 
%   + \textcolor{red}{\omega(k,n,d,\eps_\bH,\eps_\bL,\eps_{\bV},\eps_{\bR})}
%\end{align}
%%
%where
%\begin{equation}
%\textcolor{red}{\omega(k,n,d,\eps_\bH,\eps_\bL,\eps_{\bV},\eps_{\bR})} = \dots.
%\end{equation}
%\end{theorem}
%%\begin{remark}
%%Clearly, the constraints on $\nu$ \textcolor{red}{in red}
%%\begin{equation}
%% \inf_{\bS \succ\bzero} G(\bS;\nu) < 0 ,\quad \E_{\nu}[\grad \ell\grad\ell^\sT] \succ\bzero
%%\end{equation}
%%and  on $\mu$
%%\begin{equation}
%%    \bR(\mu)\succ\bzero
%%\end{equation}
%%can be removed and still obtain a valid upper bound. Specifically, removing the support constraint can be done before applying the large deviation result, if one first realizes the logarithmic potential as a variational principle at that stage.
%%\end{remark}
%
%


%\subsection{Solving the asymptotic formula}
%
%\begin{remark}
%    Note the conditions on the determinant and trace in 1. above imply that the 
%    symmetric matrix 
%\begin{equation}
%    \E[\grad \ell \grad \ell^\sT]^{1/2}\bS \bSigma^{-1} \bS \E[\grad \ell \grad \ell^\sT]^{1/2}
%\end{equation}
%has eigenvalues all equal to $1/\alpha$ and hence 
%\begin{equation}
%    \E[\grad \ell \grad \ell^\sT]^{1/2}\bS \bSigma^{-1} \bS \E[\grad \ell \grad \ell^\sT]^{1/2} = \frac1\alpha \bI_k.
%\end{equation}
%
%\end{remark}
%
%\bns{The following proof has been moved. Erase it}
%\begin{proof}[Proof of Theorem~\ref{thm:global_min}]
%Let $\bSigma = \bSigma(\bR) := \bR/\bR_{00}$ for ease of notation.
%Define
%\begin{align}
%\Phi_1(\mu)
%&:=\frac{k}{2\alpha}
%- \frac{1}{2\alpha} \Tr\left(\bR_{11}(\mu)\right)
%+ \frac1{2\alpha} \log \det(\bSigma(\bR(\mu)))
% + \frac1\alpha \KL ( \mu_{\btheta| \btheta_0}\| \cN(\bzero,\bI_k)).\\
%    \Phi_2(\nu,\bR,\bS) &:=
%    -\lambda\Tr(\bS)  +
%    \frac{1}{2\alpha} \log\det\left(\E_\nu\left[ \grad \ell\grad\ell^\sT\right]\right)
%+ 
%\frac1{\alpha}\log\det\bS 
%- \E_{\nu}\left[\log \det \left(\bI_k + \grad^2 \ell^{1/2} \bS \grad^2 \ell^{1/2}\right)\right]\\
%&\quad+\frac{k}{2\alpha} 
%+ \frac{k}{2\alpha} \log(\alpha)
%-\frac1{2\alpha} \log\det\left(
%\bSigma(\bR)
%\right)   + \KL\left(\nu_{\bv,\bv_0 | W}\| \cN(\bzero, \bR)\right).
%\end{align}
%We'll lower bound each of these functions separately in what follows.
%
%\noindent\textbf{Lower bounding $\Phi_1$:}
%For any fixed $\bR,$  it's easy to see that the minimizing measure of 
%\begin{align}
%    \inf_{\mu : \bR(\mu) = \bR}  \frac1{\alpha}\KL(\mu_{\btheta| \btheta_0} \| \cN(\bzero,\bI_k)) 
%\end{align}
%will be Gaussian.
%By Gaussian conditioning, one then sees that for $\btheta_0 \in\R^{k_0}$,
%$\mu'_{{\btheta}|{\btheta_0}}(\btheta_0)$ 
%of Definition~\ref{def:opt_FP_conds}
%is the unique Gaussian measure joint measure $\mu'$ satisfies the second moment constraint $\bR = \bR(\mu')$.
%We can then directly compute for the fixed $\bR$,
%\begin{align}
%     \frac1{\alpha}\KL( \mu' \| \cN(\bzero,\bI_k))  
%     &= 
%     \frac1{2\alpha}\E_{\btheta_0\sim\mu_{\btheta_0}}\left[ -\log\det(\bSigma(\bR)) - k 
%     + \Tr(\bR_{11} - \bR_{10} \bR_{00}^{-1} \bR_{10})+
%      \btheta_0^\sT \bR_{00}^{-1}\bR_{01}\bR_{10}\bR_{00}^{-1} \btheta_0\right]\\
%      &=- \frac1{2\alpha}\log\det(\bSigma(\bR)) - \frac{k}{2\alpha} + \frac{1}{2\alpha}\Tr(\bR_{11}).
%\end{align}
%Consequently, for any $\mu$ as in the statement of the theorem, we have
%\begin{equation}
%    \Phi_1(\mu) \ge \inf_{\mu \in\cuP(\R^{k+k_0})} \Phi_1(\mu) = \inf_{\bR\succeq \bzero} \left\{
%\frac{k}{2\alpha}
%- \frac{1}{2\alpha} \Tr\left(\bR_{11}(\mu)\right)
%+ \frac1{2\alpha} \log \det(\bSigma(\bR(\mu))) 
%- \frac1\alpha \KL(\mu' \| \cN(\bzero,\bI_k))
%    \right\} = 0,
%\end{equation}
%with equality if and only if $\mu = \mu'$.
%
%\noindent \textbf{Lower bounding $\Phi_2$:}
%To lower bound $\Phi_2$, we'll rewrite the divergence term as a divergence involving the distribution $\nu^\star$ of the proximal operator in the Definition~\ref{def:opt_FP_conds}.
%In what follows, we use $(\bv,\bv_0,W)$, $\bv\in\R^{k},\bv_0\in\R^{k_0},W\in\R$ to denote random variables whose distribution is $\nu^\star.$
%To that end, let $\bg,\bg_0$ be jointly Gaussian as in Definition~\ref{def:opt_FP_conds}, and $W\sim\P_w$. 
%For any $\bS,\bR \succ\bzero$,
%denoting
%$p_{\bS,\bR}(\bv| \bv_0, W)$ the conditional density of $\Prox(\bg; \bS, \bg_0, W)$ given $W,\bv_0$,
%we find that
%\begin{align}
%    p^\star_{\bS,\bR}(\bv|\bv_0, W) =& \exp\left\{ -\frac12 (\bS\grad\ell(\bv,\bv_0, W )+\bv-\bmu(\bv_0,\bR))^\sT\bSigma(\bR)^{-1}(\bS\grad\ell(\bv,\bv_0, W )+\bv-\bmu(\bv_0,\bR))\right\}\\
%    &\quad\quad(2\pi)^{-k/2}\det(\bSigma(\bR))^{-1/2}
%     \det\left(\bI_k+\grad^2\ell(\bv,\bv_0,W)^{1/2}\bS \grad^2\ell(\bv,\bv_0,W)^{1/2}\right),
%\end{align}
%where $\bmu := \bmu(\bv_0, \bR) := \bR_{10}\bR_{00}^{-1}\bv_0$.
%So the divergence of the conditional measure $\nu_{\bv|\bv_0,W}$ from $\cN(\bmu,\bSigma)$ can be written as
%\begin{align*}
%    \KL\left(\nu_{\bv|\bv_0,W}\|  \cN(\bmu,\bSigma)\right) 
%    &=\KL\left(\nu_{\bv|\bv_0,W}\|  p^\star_{\bS,\bR}(\cdot | \bv_0, W) \right) 
%    +
%     \E_\nu\left[\log\det\left(\bI_k+\grad^2\ell^{1/2}\bS \grad^2\ell^{1/2}\right)\right]
%     -\frac12\E_\nu\left[\grad \ell^\sT \bS \bSigma(\bR)^{-1} \bS\grad \ell\right]\\
%     &-\E_\nu\left[
%    \grad \ell^\sT \bS \bSigma(\bR)^{-1} \left(\bv - \bmu\right)
%     \right].
%\end{align*}
%Recall that $\nu\in\cuV(\bR)$
%implies that $\E[\grad \ell \cdot (\bv^\sT,\bv_0^\sT)] + \lambda (\bR_{00},\bR_{01}) = \bzero,$ whence
%\begin{align}
%\E_\nu\left[
%    \grad \ell^\sT \bS \bSigma(\bR)^{-1} \left(\bv - \bmu\right)
%     \right]  &=  \Tr\left( \bS \bSigma(\bR)^{-1} \E_{\nu}[\bv \grad\ell^\sT - \bR_{10}\bR_{00}^{-1}\bv_0\grad\ell^\sT]\right)  \\
%&=
%-\lambda\Tr\left( \bS 
%(\bR_{00} - \bR_{10}\bR_{00}^{-1}\bR_{01})^{-1}
%(\bR_{00} - \bR_{10}\bR_{00}^{-1}\bR_{01})\right) \\
%&= -\lambda \Tr(\bS).
%\end{align}
%This, along with the chain rule for the KL-divergence and the expansion of the conditional KL above gives
%\begin{align}
%\nonumber
%  \KL\left(\nu_{\bv, \bv_0|W}\|  \cN(\bzero,\bR)\right) &= 
%\KL\left(\nu_{\bv|\bv_0,W}\|  p^\star(\cdot | \bv_0, W) \right) 
%    +
%     \E_\nu\left[\log\det\left(\bI_k+\grad^2\ell^{1/2}\bS \grad^2\ell^{1/2}\right)\right]
%     -\frac12\E_\nu\left[\grad \ell^\sT \bS \bSigma(\bR)^{-1} \bS\grad \ell\right]\\
%  &\quad\quad+\KL\left(\nu_{\bv_0|W}\|  \cN(\bzero,\bR_{00})\right) + \lambda \Tr(\bS).
%\end{align}
%By substituting this equality for the KL term into $\Phi_2$ and carrying out the appropriate cancellations, 
%this shows that for any $\mu,\nu$ as in the statement, 
%\begin{align}
%   \sup_{\bS\succ\bzero}\Phi_2(\nu,\bR(\mu),\bS) 
%     &=
%    \sup_{\bS\succ\bzero} \bigg\{\frac1{2\alpha} \log\det \left(
%    \E_\nu[\grad \ell \grad \ell^\sT ] \bS^2 \bSigma(\bR(\mu))^{-1}
%    \right)-\frac12\E_\nu[\grad\ell^\sT \bS \bSigma(\bR(\mu))^{-1}\bS \grad\ell]\\
%    &\hspace{15mm}+\frac k{2\alpha}\log(\alpha e) +\KL(\nu_{\bv|\bv_0,W}\|p^\star_{\bS,\bR(\mu)})+\KL(\nu_{\bv_0|W}\|\cN(\bzero,\bR(\mu)_{00}))\bigg\}\\
%&\stackrel{(a)}{\ge}
%    \sup_{\bS\succ\bzero} \left\{M(\bS;\nu, \bR(\mu)) \right\}
%\end{align}
%where
%\begin{equation}
%    M(\bS;\nu, \bR) = \frac1{2\alpha} \log\det \left(
%    \E_{\nu}[\grad \ell \grad \ell^\sT ] \bS^2 \bSigma(\bR)^{-1}
%    \right)-\frac12\E_{\nu}[\grad\ell^\sT \bS \bSigma(\bR)^{-1}\bS \grad\ell]+\frac k{2\alpha}\log(\alpha e).
%\end{equation}
%The inequality in $(a)$ follows from non-negativity of the KL-divergence and holds with equality if and only if $\nu_{\bv,\bv_0| W} = \nu_{\bv,\bv_0|W}^\star$, the measure induced by the density $p^\star_{\bS,\bR}$ defined above.
%Since $\E_\nu[\grad\ell\grad\ell^\sT]\succ \bzero,  \bR(\mu) \succ\bzero$ for such measures, one can check that $M(\bS;\nu,\bR)$ is strictly concave in $\bS$ and is uniquely maximized at 
%\begin{equation}
%    \bS= \bS^\star(\nu,\bR) =\frac1{\sqrt\alpha}\bSigma(\bR)^{1/2}\left(\bSigma(\bR)^{-1/2}\E_{\nu}[\grad\bell\grad\bell^\sT]^{-1} \bSigma(\bR)^{-1/2}\right)^{1/2}\bSigma(\bR)^{1/2},
%\end{equation}
%with $M(\bS^\star(\nu,\bR);\nu, \bR) = 0$.
%\newline
%
%\noindent\textbf{Concluding:}
%Using the lower bounds above, for any $\mu,\nu$ as in the statement, we have by design
%\begin{align}
%   \sup_{\bS\succ\bzero}  
%   \Phi(\mu,\nu,\bR(\mu),\bS) &=
%    \Phi_1(\mu)  
%    +\sup_{\bS\succ\bzero}\Phi_2(\nu,\bR(\mu),\bS)
%    = \inf_{\mu_0 : \bR(\mu_0) = \bR(\mu)}
%  \left\{
%    \Phi_{1}(\mu_0) + \sup_{\bS\succ\bzero} \Phi_2(\nu',\bR(\mu), \bS) 
%    \right\}\\
%    &= \inf_{\mu_0 : \bR(\mu_0) = \bR(\mu)}
%    \Phi_{1}(\mu_0) + \sup_{\bS\succ\bzero} \Phi_2(\nu',\bR(\mu), \bS) 
%    \stackrel{(a)}{\ge} 0 + \sup_{\bS\succ\bzero} \Phi_2(\nu',\bR(\mu), \bS)\\
%    &\stackrel{(b)}{\ge} 0 + 
%\sup_{\bS\succ\bzero}  M(S; \nu,\bR(\mu)) = 0.
%\end{align}
%where in $(a)$ we used that for any $\bR \succ\bzero$, the Gaussian measure $\mu' = \mu'(\bR)$ chosen previously satisfies $\Phi_1(\mu') = 0$.
%By the previous steps, $(a)$ and $(b)$ hold with equality if and only $(\mu,\nu)$ are as given in Definition~\ref{def:opt_FP_conds}.
%
%%Combining with Eq.~\eqref{eq:phi_1_LB} shows that $\sup_{\bS}\Phi \ge0$, with equality if and only if the inequality in $(a)$ and the inequality in Eq.~\eqref{eq:phi_1_LB} hold with equality, i.e., if and only if $\nu$ and $\mu$ satisfy the equations in Definition~\ref{def:opt_FP_conds}.
%
%\textbf{Proof of \textit{1.} and \textit{2.}:}
%To prove \textit{1.}. let $\Omega_0 := \{\widehat\bTheta_n \in \cE(\bTheta_0)\}$, and $\Omega_1 := \{ n C_0(\alpha) \succ \bX^\sT\bX\succ nc_0(\alpha)\}$. 
%Since Assumption~\ref{ass:loss} guarantees that $\|\grad^2\hat R_n(\bTheta)\|_\op =O(1)$ on $\Omega_1$, by Lemma~\ref{lemma:jacobian_lb}, $\sigma_{\min}(\bJ_{(\bbV,\bTheta)} \bG) = e^{-o(n)}$ on this event, so that on $\Omega_0 \cap\Omega_1$, the point $\hat\bTheta_n \in \cZ_n$ of Eq.~\eqref{eq:set_of_zeros_main} for some choice of $\sPi$ satisfying Assumption~\ref{ass:params}.
%
%For $\eps >0$, consider the set 
%\begin{equation}
%    \cuA_\eps:=  \{\mu : d_{\BL}(\mu,\mu_\opt) \le\eps \},\quad
%    \quad\quad
%    \cuB_\eps := 
%     \{\nu : d_{\BL}(\nu,\nu_\opt) \le\eps \}.
%\end{equation}
%Then
%there exists some $c_0(\eps) >0$ such that
%for any $\mu \in\cuT(\cuA^c_\eps),\nu \in\cuV(\bR(\mu),\cuB^c_\eps)$, $\sup_{\bS\succ\bzero}\Phi(\mu,\nu,\bR(\mu)) > c(\eps)$ uniformly. 
%So
%using the shorthand 
%$\hmu:=\hat\mu_{\sqrt{d}[\hat\bTheta_n,\bTheta]}$ and 
%$\hnu := \hat\nu_{[\bX\hat\bTheta_n,\bX\bTheta]}$,
% we can bound for any $\delta>0$,
%\begin{align}
%    \P\left( \left\{ d_{\LU}(\hmu, \mu_\opt) > \eps\right\}
%\cup
%\left\{ d_{\LU}(\hnu, \nu_\opt) > \eps\right\}
%    \right)
%    &\le  \P\left(\{(\hmu,\hnu) \in\cuA_\eps^c \times \cuB_\eps^c\} \cap \{\bw \in\cG_\delta\}\cap \Omega_0\cap\Omega_1\right) + \P(\Omega_0^c)
%+ \P(\Omega_1^c)
%+ \P(\cG_\delta^c)\\
%&\le \P\left(\one_{\hat\bTheta_n \in \cZ_n(\cuA_\eps^c,\cuB_\eps^c, \sPi)} \one_{\bw \in\cG_\delta}\right)
%+ \P(\Omega_0^c)
%+ \P(\Omega_1^c)
%+ \P(\cG_\delta^c).
%\end{align}
%Taking $n\to\infty$ and noting that
%\begin{equation}
%    \lim_{n\to\infty }
%( \P(\Omega_0^c)
%+ \P(\Omega_1^c)
%+ \P(\cG_\delta^c)) = 0
%\end{equation}
%by the assumption on $\hat\bTheta_n$ and Assumption~\ref{ass:noise},
%we conclude by Theorem~\ref{thm:convexity} that for all $\eps>0$
%\begin{equation}
%    \lim_{n\to\infty}\P\left( \left\{ d_{\LU}(\hmu, \mu_\opt) > \eps\right\}
%\cup
%\left\{ d_{\LU}(\hnu, \nu_\opt) > \eps\right\}
%    \right)
%    \le 
%\lim_{n\to\infty}
%\E[\cZ_{n}(\cuA_\eps^c, \cuB_\eps^c, \sPi)\one_{\{\bw\in\cG_\delta\}} ] \le \lim_{n\to\infty} e^{- n c(\eps)} = 0
%\end{equation}
%giving the statement of \textit{1.} of the Theorem. 
%Finally, for \textit{2.}, by Lemma~\ref{lemma:rate_matrix_ST} along with an argument similar to that of Lemma~\ref{lemma:asymp_ST}, we can deduce that for any $\hnu \Rightarrow \nu$ in probability,
%\begin{equation}
%     \frac1{dk} (\bI_k \otimes \Tr) \left(\bH(\hnu_n) - z\bI_{dk}\right)^{-1} \to \alpha \bS_\star(\nu,z)
%\end{equation}
%in probability. The claim now follows by from \textit{1.} after recalling the definition of $\bH$.
%
%
%
%\end{proof}
%
%
%
%
%\newpage
%
%\subsection{Solving the asymptotic formula}
%To solve the formula, let us define the multivariate proximal operator in what follows. For $\bz \in\R^k, \bu \in\R^{k_\star}, \bS\in\R^{k\times k}, w\in\R,$ let
%\begin{equation}
%    \Prox(\bz;\bS, \bu, w)=\arg\min_{\bx\in\R^k}\left( \frac12(\bx-\bz)^\bT\bS^{-1}(\bx-\bz) + \ell(\bx,\bu,w)\right)\in\R^k.
%\end{equation}
%Observe that for $\ell$ convex in $\bv$ at fixed $\bu,w$,  the map $\bz \mapsto \Prox(\bz; \bS, \bu, w)$  is invertible  for any $\bS \succeq \bzero_{k\times k}$, with inverse given by
%
%\begin{equation}
%\Prox^{-1}(\bv; \bS, \bu, w) = \bS\grad\ell(\bv,\bu,w)+\bv,
%\end{equation}
%which can be derived from the first order conditions
%\begin{align}
%    %\rho_{\bQ,\bu,w}(\bv)=&\min_{\bx\in\R^k}\left( \frac12 (\bx-\bv)^\bT\bQ^{-1}(\bx-\bv)+\rho(\bx,\bu,w)\right)\in\R\\
%     \bS\grad\ell(\Prox(\bz;\bS, \bu, w),\bu,w)=\bz- \Prox_{\bu,w}(\bz; \bS, \bu, w)
%\end{align}
%where $\grad \ell \in\R^{k}$  is the gradient of $\ell$ with respect to the first $k$ variables.
%The following defines the optimality conditions for a given $(\mu,\nu,\bR,\bS)$.
%
%\begin{definition}
%\label{def:opt_FP_conds}
%   We say that the pair $(\mu',\nu')\in \cuL(\bTheta_0)\times \cV(\P_w)$ satisfy the \emph{fixed point conditions} if the following holds:
%   With the definitions $\bR' = \bR(\mu')$, $(\bg^\sT,\bg_0^\sT)^\sT \sim \cN\left( \bzero_{k+k_0},\bR'\right),\;W\sim\P_w,$
%%\begin{align}
%%\Tr\left(\bS (
%%\bR_{11} - \bR_{12}\bR^{-1}_{22}\bR_{21})^{-1} \bS 
%%\E_{\nu}[\grad \rho \grad \rho^\sT]\right) &= \Tr\left(\frac1\alpha \bI_k\right)\\
%%\det\left( \bS (
%%\bR_{11} - \bR_{12}\bR^{-1}_{22}\bR_{21})^{-1} \bS 
%%\E_{\nu}[\grad \rho \grad \rho^\sT]\right) &= \det\left(\frac1\alpha \bI_k\right),
%%\end{align}
%and
%\begin{equation}
%    \bS'\equiv \bS'(\nu', \mu') =\frac1{\sqrt\alpha_n}(\bR'/\bR'_{00})^{1/2}\left((\bR'/\bR'_{00})^{-1/2}\E_{\nu'}[\grad\bell\grad\bell^\sT]^{-1} (\bR'/\bR'_{00})^{-1/2}\right)^{1/2}(\bR'/\bR'_{00})^{1/2},
%\end{equation}
%the following fixed point equations are satisfied
%   \begin{align}
%&\nu' \equiv \nu'_{\bv,\bu,W} = \mathrm{Law}(
%\Prox( \bg; \bS', \bg_0, W),\bg_0,W),\\
%&\mu_{\btheta_0}' = \mathrm{Law}(\hat\mu_{\bTheta_0}),\quad  \mu'_{\btheta|\btheta_0} = \cN(\bzero,\bR'/\bR'_{00}),\\
%    &\E_{\nu'}\left[ \grad \ell(\bv,\bu,W) (\bv,\bu)^\sT\right] = \bzero_{k\times (k+k_0)}.
%\end{align}
%\end{definition}
%\begin{remark}
%    Note the conditions on the determinant and trace in 1. above imply that the 
%    symmetric matrix 
%\begin{equation}
%    \E[\grad \ell \grad \ell^\sT]^{1/2}\bS \bSigma^{-1} \bS \E[\grad \ell \grad \ell^\sT]^{1/2}
%\end{equation}
%has eigenvalues all equal to $1/\alpha$ and hence 
%\begin{equation}
%    \E[\grad \ell \grad \ell^\sT]^{1/2}\bS \bSigma^{-1} \bS \E[\grad \ell \grad \ell^\sT]^{1/2} = \frac1\alpha \bI_k.
%\end{equation}
%
%\end{remark}
%
%\begin{theorem}
%Fix $\cA,\cB$ as in Theorem~\notate{ref}. For any $\mu\in\cuL(\bTheta_0)\cap \cA, \nu \in\cV(\P_w) \cap \cB$,
%\begin{equation}
%    \sup_{\bS\succ\bzero}\Phi(\mu,\nu,\bR(\mu),\bS) \ge 0,
%\end{equation}
%with equality if and only if 
% $(\mu,\nu)$ satisfy the fixed point conditions of Definition~\ref{def:opt_FP_conds}.
%%For any $\bR \succeq \bzero_{(k+k^\star)\times (k+k^\star)}$, $\nu \in\cQ(\P_w)$, we have
%\end{theorem}
%\begin{proof}
%Let $\bSigma \equiv \bSigma(\bR) = \bR/\bR_{00}$ for ease of notation.
%%Let $(\mu',\nu',\bS')$ satisfy the conditions in Definition~\ref{def:opt_FP_conds}.
%%We'll show 
%%\begin{equation}
%%    \Phi(\mu,\nu,\bR(\mu),\bS) \ge \Phi(\mu',\nu',\bR(\mu'),\bS) = 0
%%\end{equation}
%%with equali
%Define
%\begin{align}
%\Phi_1(\mu) &:= \frac{1}{\alpha_n}\KL\big(\mu_{\btheta|\btheta_0}\| \normal(\bzero,\bR/\bR_{00})\big)
%-\frac1{2\alpha_n} \Tr\left(\E_{\mu_{\btheta|\btheta_0}}[\btheta\btheta^\sT(\bR/\bR_{00})^{-1}]\right).\\
%    \Phi_2(\nu,\bR,\bS) &:=
%    \frac{1}{2\alpha_n} \log\det\left(\E_\nu\left[ \grad \ell\grad\ell^\sT\right]\right)
%+ 
%\frac1{\alpha_n}\log\det\bS 
%- \E_{\nu}\left[\log \det \left(\bI_k + \grad^2 \ell^{1/2} \bS \grad^2 \ell^{1/2}\right)\right]\\
%&\quad+\frac{k}{2\alpha_n} 
%+ \frac{k}{2\alpha_n} \log(\alpha_n)
%-\frac1{2\alpha_n} \log\det\left(
%\bSigma(\bR)
%\right)   + \KL\left(\nu_{\bv,\bu | W}\| \cN(\bzero, \bR)\right).
%\end{align}
%Note that $\Phi(\mu,\nu,\bR(\mu),\bS) =  \Phi_1(\mu)+
%\Phi_2(\nu,\bR(\mu),\bS).$
%
%First observe that for any $\mu \in\cA \cap\cuL(\bTheta_0)$,
%\begin{equation}
%\label{eq:phi_1_LB}
%\Phi_1(\mu) \ge \inf_{\mu\in\cuP(\R^{k+k_0})} \Phi_1(\mu) = 0
%\end{equation}
%with the minimizer achieved uniquely at $\mu =\mu'$ of Definition~\ref{def:opt_FP_conds}.
%
%To deal with $\Phi_2$, we'll rewrite the divergence term as a divergence involving the distribution of the proximal operator in the definition~\ref{def:opt_FP_conds}.
%To that end, let $\bg,\bg_0$ be jointly Gaussian as in definition~\ref{def:opt_FP_conds}, and $W\sim\P_w$. Denoting
%$p(\bv|\bg_\star,W)\equiv p_{\bS,\bR}(\bv| \bg_\star, W)$ the conditional density of $\Prox(\bg; \bS, \bg_\star, W)$ given $W,\bg_\star$,
%we find that
%\begin{align}
%    \pi_{\bS,\bR}(\bv|\bu, W) =& \exp\left\{ -\frac12 (\bS\grad\ell(\bv,\bu, W )+\bv-\bmu(\bu,\bR))^\sT\bSigma(\bR)^{-1}(\bS\grad\ell(\bv,\bu, W )+\bv-\bmu(\bu,\bR))\right\}\\
%    &\quad\quad(2\pi)^{-k/2}\det(\bSigma(\bR))^{-1/2}
%     \det\left(\bI_k+\grad^2\ell(\bv,\bu,W)^{1/2}\bS \grad^2\ell(\bv,\bu,W)^{1/2}\right),
%\end{align}
%where $\bmu\equiv \bmu(\bu, \bR)=\bR_{12}\bR_{22}^{-1}\bu$.
%So the divergence of the conditional measure $\nu_{\bv|\bu,W}$ from $\cN(\bmu,\bSigma)$ can be written as
%\begin{align*}
%    \KL\left(\nu_{\bv|\bu,W}\|  \cN(\bmu,\bSigma)\right) 
%    &=\KL\left(\nu_{\bv|\bu,W}\|  p(\cdot | \bu, W) \right) 
%    +
%     \E_\nu\left[\log\det\left(\bI_k+\grad^2\ell^{1/2}\bS \grad^2\ell^{1/2}\right)\right]
%     -\frac12\E_\nu\left[\grad \ell^\sT \bS \bSigma(\bR)^{-1} \bS\grad \ell\right]\\
%     &-\E_\nu\left[
%    \grad \ell^\sT \bS \bSigma(\bR)^{-1} \left(\bv - \bmu\right)
%     \right].
%\end{align*}
%Noting that $\nu\in\cV(\P_w)$ implies that
%$\E_\nu\left[
%    \grad \ell^\sT \bS \bSigma(\bR)^{-1} \left(\bv - \bmu\right)
%     \right] =0$, we conclude by an application of the chain rule for the KL-divergence that
%\begin{align}
%\nonumber
%  \KL\left(\nu_{\bv, \bu|W}\|  \cN(\bzero,\bR)\right) &= 
%\KL\left(\nu_{\bv|\bu,W}\|  p(\cdot | \bu, W) \right) 
%    +
%     \E_\nu\left[\log\det\left(\bI_k+\grad^2\ell^{1/2}\bS \grad^2\ell^{1/2}\right)\right]
%     -\frac12\E_\nu\left[\grad \ell^\sT \bS \bSigma(\bR)^{-1} \bS\grad \ell\right]\\
%  &\quad\quad+\KL\left(\nu_{\bu|W}\|  \cN(\bzero,\bR_{00})\right).
%\end{align}
%This shows that for any $\mu,\nu$ as in the statement,
%\begin{align}
%   \sup_{\bS\succ\bzero}\Phi_2(\nu,\bR(\mu),\bS) 
%     &=
%    \sup_{\bS\succ\bzero} \bigg\{\frac1{2\alpha} \log\det \left(
%    \E_\nu[\grad \ell \grad \ell^\sT ] \bS^2 \bSigma(\bR(\mu))^{-1}
%    \right)-\frac12\E_\nu[\grad\ell^\sT \bS \bSigma(\bR(\mu))^{-1}\bS \grad\ell]\\
%    &\quad+\frac k{2\alpha}\log(\alpha e) +\KL(\nu_{\bv|\bu,W}\|p_{\bS,\bR(\mu)})+\KL(\nu_{\bu|W}\|\cN(\bzero,\bR_{00}(\mu)))\bigg\}\\
%&\stackrel{(a)}{\ge}
%    \sup_{\bS\succ\bzero} \left\{\frac1{2\alpha} \log\det \left(
%    \E_\nu[\grad \ell \grad \ell^\sT ] \bS^2 \bSigma(\bR(\mu))^{-1}
%    \right)-\frac12\E_\nu[\grad\ell^\sT \bS \bSigma(\bR(\mu))^{-1}\bS \grad\ell]+\frac k{2\alpha}\log(\alpha e) \right\}\\
%    &=0
%\end{align}
%with the maximum achieved uniquely at $\bS = \bS'(\nu,\mu)$ by strict concavity of the term above in $\bS$. Combining with Eq.~\eqref{eq:phi_1_LB} shows that $\sup_{\bS}\Phi \ge0$, with equality if and only if the inequality in $(a)$ and the inequality in Eq.~\eqref{eq:phi_1_LB} hold with equality, i.e., if and only if $\nu$ and $\mu$ satisfy the equations in Definition~\ref{def:opt_FP_conds}.
%
%
%\end{proof}
%
%
%
%\begin{proof}
%Let $\bg,\bg_\star$  be jointly Gaussian with mean $\bzero$ and covariance $\bR$, and let
%$\pi_{\bS,\bR}(\bv| \bg_\star, W)$ be the conditional density of $\Prox(\bg; \bS, \bg_\star, W)$ given $W,\bg_\star$.
%Then we can write
%\begin{align}
%    \pi_{\bS,\bR}(\bv|\bu, W) =& \exp\left\{ -\frac12 (\bS\grad\ell(\bv,\bu, W )+\bv-\bmu(\bu,\bR))^\sT\bSigma(\bR)^{-1}(\bS\grad\ell(\bv,\bu, W )+\bv-\bmu(\bu,\bR))\right\}\\
%    &(2\pi)^{-k/2}\det(\bSigma(\bR))^{-1/2}
%     \det\left(\bI_k+\grad^2\ell(\bv,\bu,W)^{1/2}\bS \grad^2\ell(\bv,\bu,W)^{1/2}\right),
%\end{align}
%where $\bmu(\bu, \bR)=\bR_{12}\bR_{22}^{-1}\bu$ and $\bSigma(\bR)=\bR_{11}-\bR_{12}\bR_{22}^{-1}\bR_{21}$.
%Noting that
%\begin{align}
%  \KL\left(\nu_{\bv, \bu|W}\|  \cN(\bzero,\bR)\right) &= 
%  \KL\left(\nu_{\bv|\bu,W}\|  \cN(\bmu,\bSigma)\right) + 
%  \KL\left(\nu_{\bu|W}\|  \cN(\bzero,\bR_{2,2})\right),
%\end{align}
%and by using the density above,\bns{Is the $+$ below a $-$?}
%\begin{align}
%    \KL\left(\nu_{\bv|\bu,W}\|  \cN(\bm(\bR),\bSigma(\bR))\right) 
%    &=\KL\left(\nu_{\bv|\bu,W}\|  \pi_{\bS,\bR}(\cdot | \bu, W) \right) 
%    +
%     \E_\nu\left[\log\left(\det\left(\bI_k+\grad^2\ell^{1/2}\bS \grad^2\ell^{1/2}\right)\right)\right]\\
%     &-\frac12\E_\nu\left[\grad \ell^\sT \bS \bSigma(\bR)^{-1} \bS\grad \ell\right]
%     {\color{red}+}\E_\nu\left[
%    \grad \ell^\sT \bS \bSigma(\bR)^{-1} \left(\bv - \bmu(\bu;\bR)\right)
%     \right].
%\end{align}
%Recalling the condition $\E_\nu[\grad\ell\bv^\sT] = \bzero, \E_\nu[\grad\ell \bu^\sT] = \bzero,$ and the definition of $\bmu$, 
%we conclude 
%\begin{align}
%\E_\nu\left[
%    \grad \ell^\sT \bS \bSigma(\bR)^{-1} \left(\bv - \bmu(\bu;\bR)\right)
%     \right]
%     &=
%\Tr\left(\E_\nu\left[
%     \bS \bSigma(\bR)^{-1} \left(\bv - \bmu(\bu;\bR)\grad \ell^\sT\right)
%     \right]\right) = 0.
%\end{align}
%
%Combining, we can rewrite $\Phi$ (recalling its definition) as
%\begin{align}
%\Phi(\bR,\nu,\bS) &:= \frac{1}{2\alpha} \log\det\left(\E_\nu\left[ \grad \ell\grad\ell^\sT\right]\right) \\
%&\quad+ 
%\frac1{\alpha}\log\det\bS 
%- \E\left[\log \det \left(\bI_k + \grad^2 \ell^{1/2} \bS \grad^2 \ell^{1/2}\right)\right]
%\\
%&\quad+\frac{k}{2\alpha} 
%+ \frac{k}{2\alpha} \log(\alpha)
%-\frac1{2\alpha} \log\det\left(
%\bSigma(\bR)
%\right)  + \KL\left(\nu_{\bv,\bu | W}\| \cN(\bzero, \bR)\right)\\
%     &=\frac1{2\alpha} \log\det \left(
%    \E_\nu[\grad \ell \grad \ell^\sT ] \bS^2 \bSigma(\bR)^{-1}
%    \right)-\frac12\E[\grad\ell^\sT \bS \bSigma(\bR)^{-1}\bS \grad\ell]\\
%    &\quad+\frac k{2\alpha}\log(\alpha e) +\KL(\nu_{\bv|\bu,W}\|\pi_{\bv|\bu,W})+\KL(\nu_{\bu|W}\|\cN(\bzero,\bR_{22})).
%\end{align}
%The claim now follows.
%\end{proof}





































%\section{Large deviations}
%\bns{There are many issues here that are not addressed in what is below: 
%\begin{enumerate}
%    \item We need to take into account the conditioning on $\bw$. So at some point one must bound 
%        $d(\widehat\nu_\bw,\P_w)$ for appropriate $d$. 
%    \item We need to deal with the integral constraint on the manifold: the problematic constraint is of the the form $\bL(\bV)^\sT\bV = 0$. This is a manifold of zero measure under the Gaussian distribution on $\bV$. I think this will require more work: namely, I believe we cannot get around quantifying the error term in the approximation of this integral to one where the constraint is replaced with $\norm{bL(\bV)^\sT\bV}\le \eps$,
%    and doing a sensitivity analysis on the constraint $|\E[\ell(V)V]|\le \eps$ that shows up in the asumptotic formula.
%    \item In the non-convex case, we have to deal with functions of the form 
%    \begin{equation}
%        \nu \mapsto F(\E_\nu[G(\bX,\bS(\nu)])),
%    \end{equation}
%    for $\bS$  that isn't a ``nice" function of $\nu$ (see Lemma~\ref{lemma:log_pot_z}).
%%    \item Even in the convex case, it seems to me that we require that the constraint on $\nu$
%%    \begin{equation}
%%        \inf\supp(\mu_\star(\nu)) >0
%%    \end{equation}
%%to appear in the asymptotic formula to get the tight bound. Though, we may be able to deal with this by 
%\end{enumerate}
%}
%
%
%
%%\section{Question about the left edge of support of $\mu_\star$}
%%  @Andrea: Is it clear to you what the correct formulation of the following is for the non convex case:
%%  Let 
%%  \begin{equation}
%%  \bG(\bS;\nu) :=  \alpha^{-1} \bS -\E_{\nu}[(\bI + \bD\bS)^{-1}\bD ],
%%  \end{equation}
%%  where the expectation is over $\bD\sim\nu$ for $\bD$ symmetric.
%%  If $\bD$ is positive semi-definite almost surely, then 
%%  \begin{equation}
%%      -\inf_{\bS\succ\bzero} \frac1k \Tr(\bG(\bS;\nu)) = \inf \supp(\mu_\star(\nu))
%%  \end{equation}
%%  where $\mu_\star(\nu)$ is the measure whose Stieltjes transform is characterized by $\bG(\bS_\star(z);\nu) = -z\bI$. In the general case where $\bD$ is not assume PSD, we need some constraint on the set in the $\inf$ constraint. It's not clear to me what the correct one is in the general $k$ case. As a reminder for the $k=1$, I copied over your old notes below. 
%%
%%\paragraph{Limit of the log determinant}
%%
%%Will assume $\nu$ to have bounded support with $x_{\min} = \min(x:x\in\supp(\nu))$,
%%$x_{\max} = \max(x:x\in\supp(\nu))$.
%%
%%Let $u_M=\infty$ if $x_{\min}\ge 0$ and $u_M:=-1/x_{\min}$ otherwise. For $z\in\bbC$ and $u\in (0,u_{M})$,
%%define (note that the logarithm arguments are positive in this domain):
%%%
%%\begin{align}
%%K(u;z) = -\delta zu +\delta\int \log(1+ux)\, \nu(\de x)-\log u-\log\delta-1\, ,
%%\end{align}
%%%
%%For $u\in\bbH:=\{w\in\bbC:\Im(w)>0\}$
%%the upper half plane, we define $K(u;z)$ to be the unique analytic function with 
%%whose value for  $u\in (0,u_{0})$ is given above. We denote by $\bbU:= \bbH\cap (0,u_0)$,
%%
%%Let $\bW\in\reals^{n\times d}$ be a matrix with i.i.d. entries $W_{ij}\sim\normal(0,1)$,
%%and $D$ a diagonal matrix whose entries empirical distribution converges to $\nu$
%%(with max/min entry converging to $x_{\max}$, $x_{\min}$). For $\Re(z)\le \lambda_{\min}(\bH)$, we define 
%%%
%%\begin{align}
%%\kappa_n(z) := \frac{1}{d} \log\det (\bH-z\id_d)\, ,\;\;\; \bH:= \frac{1}{n}\bW^{\sT}\bD\bW\, .
%%\end{align}
%%%
%%we consider $n,d\to\infty$ with $n/d\to\delta\in(1,\infty)$.
%%We also define the Stieltjis transform
%%%
%%\begin{align}
%%s_n(z) := \frac{1}{d} \Tr\big\{ (\bH-z\id_d)^{-1}\big\}\, ,\, .
%%\end{align}
%%%
%%
%%We also define
%%%
%%\begin{align}
%%G(u) := \frac{1}{\delta u} -\int \frac{x}{1+xu}\nu(\de x)\, .
%%\end{align}
%%
%%\begin{lemma}\label{lemma:Min_G}
%%The function $G$ has a unique local (hence global) minimum $u_{0}$ in $(0,u_M)$, and
%%\begin{align}
%%    \lim_{n,d\to\infty}\lambda_{\min}(\bH) = -G(u_0)\, .\label{eq:LimLambda_min}
%%\end{align}
%%\end{lemma}
%%%
%%\begin{proof}
%%To prove existence of the minimum we proceed sligthly differently depending whether 
%%$x_{\min}\ge 0$ or $x_{\min}<0$:
%%\begin{itemize}
%%\item If $x_{\min}\ge 0$, then $G(u) = 1/(\delta u)+O(1)$ as $u\to 0$ and
%%$G(u) = -(1-\delta^{-1})u^{-1}+O(u^{-2})$ as $u\to\infty$. Hence there is necessarily a local minimum
%%in $(0,\infty)$.
%%\item If   $x_{\min}\ge 0$, then again $G(u) = 1/(\delta u)+O(1)$ as $u\to 0$ and 
%%%
%%\begin{align}
%%    u_M^2G'(u_M) = -\frac{1}{\delta} +\int_{x_{\min}}^{\infty}\left(\frac{x}{x-x_{\min}}\right)^2\nu(\de x) >0\,. 
%%\end{align}
%%\end{itemize}
%%%
%%To prove uniqueness, note that
%%%
%%\begin{align}
%%    u^2G'(u) = -\frac{1}{\delta} +\int\left(\frac{xu}{1+xu}\right)^2\nu(\de x)\, ,
%%\end{align}
%%%
%%and this  is strictly increasing on $(0,u_M)$.
%%
%%The limit in Eq.~\eqref{eq:LimLambda_min} follows from \am{Find reference}
%%\end{proof}
%%%
%%\begin{lemma}\label{lemma:ustar}
%%Let $u_0$ be defined as in the statement of Lemma \ref{lemma:Min_G}.
%%For $z\in (-\infty,-G(u_0)]$ there is a unique solution $u_*=u_*(z)$ to 
%%%
%%\begin{align}
%%G(u) = -z\, ,\;\;\;\; u\in (0,u_0]\, ,
%%\end{align}
%%%
%%Further:
%%%
%%\begin{enumerate}
%%\item $u_*(z)$ is analytic, with $u_*(z) = -1/(\delta z)+o(1/z)$ as $z\to -\infty$.
%%\item  For $z\in (-\infty,-G(u_0))$  $u_*$ is the unique solution to 
%%%
%%\begin{align}
%%G(u) = -z\, ,\;\;\;\; u\in (0,u_M), G'(u)<0\, .
%%\end{align}
%%%
%%\item We have
%%%
%%\begin{align}
%%\lim_{n,d\to\infty}s_n(z) = \delta \, u_*(z)\, .
%%\end{align}
%%\end{enumerate}
%%\end{lemma}
%%%
%
%
%
%
%\newpage
%
%%The constraint becomes:
%%\begin{align}
%%    %\inf_{\{\bS : \inf_{\bD \in\supp_{\nu} \lambda_{\min}((\bI + \bD\bS)^{-1}\bD) \ge 0 }\}} \bQ(\bS;\nu) 
%%    \sup_{\xi>0} \inf_{\bS \succ\bzero}\sup_{\bD \in\supp(\nu)} \left\{
%%    \bQ(\bS;\nu)  - \xi\lambda_{\min}((\bI +\bD \bS)^{-1} \bD)
%%    \right\} < -\delta
%%\end{align}
%
%
%
%%\begin{lemma}
%%Consider the standard probability simplex $\Delta^m$ \emph{as a subset of } $\R^{m}$. For any $m>1$, there exists a cover for $\Delta^m$ with balls $\cB_2^m(m^{-1/2})$ of size at most \bns{add}.
%%\end{lemma}
%%\begin{proof}
%%We will bound the covering number by the packing number which we will crudely bound via a volume argument.
%%Define 
%%\begin{align}
%%   \Delta_2^m(\eps):= \{\bx\in\R^{m+1} : \bx^\sT\one = 1, \inf_{\by \in\Delta^m}\norm{\bx - \by}_2 \le \eps \}.
%%   %\{\bx\in\R^{m+1} : \bx^\sT\one = 1, \inf_{\by \in\Delta^m}\norm{\bx - \by}_1 \le \eps\sqrt{m} \} =: \Delta_1^m(\eps \sqrt{m}).
%%\end{align}
%%Note that 
%%\begin{align}
%%   \area\left(\Delta_2^m(\eps)\right)  &=
%%   \area\left(\{\bx \in\R^{m+1} : \bx = \by + \bdelta + \eps \one,\, \by\in\Delta^m,\,
%%   \norm{\bdelta}_2 \le \eps,\, \bdelta^\sT\one = 0 
%%   \}\right)\\
%%   &\le 
%%   \area\left(\{\bx \in\R^{m+1} : \bx^\sT\one = 1 + (m+1)\eps,\, \bx \succeq \bzero
%%   \}\right),
%%\end{align}
%%the latter inequality follows by inclusion.
%%Letting $\tilde\Delta_2^m(\eps),\tilde\Delta^m\subseteq\R^m$ be an embedding of $\Delta_2^m(\eps),\Delta^m$ in $\R^m$ (fix position and orientation for both arbitrarily), we have
%%\begin{equation}
%%\vol(\tilde \Delta^m + \cB_2^m(\eps)) =\vol(\tilde \Delta_2^m(\eps))  
%%   =\area(\Delta^m_2(\eps)) \le \area(\Delta_1^m(\eps \sqrt{m}).
%%\end{equation}
%%
%%Now observe that the 
%%
%%\begin{equation}
%%    \area(\Delta_1^m(\eps\sqrt{m}) = \area\left(
%%    \{\bx \in\R^{m+1}: \bx\succ\bzero, \bx^\sT \one = \eps \sqrt{m}\}
%%    \right).
%%\end{equation}
%%
%%\newpage
%%For $\eps >0$,
%%let $N(\eps)$ be the covering number of the simplex $\Delta^m$ with balls of radius $\cB_2^m(2\eps)$. 
%%Recall that a standard packing argument shows that
%%\begin{equation}
%%\label{eq:packing_bound}
%%    N(\eps) \le \frac{\vol(\Delta^m + \cB_2^m(\eps))}{\vol(\cB_2^m(\eps))}.
%%\end{equation}
%%
%%For $a >0$, let 
%%$\Delta^m(a)$
%%%\begin{equation}
%%%\Delta^m(a) := \Delta^m + \cB_1^m(a)
%%%\end{equation}
%%be a scaling of the standard simplex where the edges are scaled by $a$; more explicitly, it is given by
%%\begin{equation}
%%   \mathrm{convhull}\left(\{a \, \be_i : i \in[m]\}\right) \equiv \{\bx \in\R^{m+1}: \bx\succ\bzero, \bx^\sT \one = a\}.
%%\end{equation}
%%where $\be_i \in\R^{m+1}$ are the canonical basis elements. (Note that $\Delta^m(1) = \Delta^m).$
%%One can show~\bns{reference or show calculation} that 
%%\begin{equation}
%%\label{eq:vol_simplex_scaled}
%%    \vol(\Delta^m(a)) = \frac{a^{m} \sqrt{m+1}}{m!}.
%%\end{equation}
%%
%%To compute the numerator in~\eqref{eq:packing_bound}, we note that
%%\begin{equation}
%%    |\Delta^m + \cB^m_2(\eps) |\le  |\Delta^m(\eps \sqrt{m})|.
%%\end{equation}
%%Indeed, for any $\bv$ in the set on left, we have $\bv = \tilde \bv + \bu$
%%for $\bv \in \Delta^m, \bu \in\cB^m_2(\eps)$ and hence
%%\begin{equation}
%%    \bv^\sT \one = \norm{\bv}_1 \le \norm{ \tilde \bv}_1 + \sqrt{m} \norm{\bu}_2  \le 1 +  \sqrt{m} \eps.
%%\end{equation}
%%and use Eq.~\eqref{eq:vol_simplex_scaled} to compute the desired quantity.
%%To see Eq.~\eqref{eq:vol_simplex_scaled}, consider 
%%\end{proof}
%%
%
%
%\subsection{Quantitative Varadhan Lemma- Sanov Approach}
%
%
%
%The goal of this section is to drive a non-asymptotic variant of Varadhan's lemma over the space of the probability measures. Namely, let $\Sigma\subseteq\R^k$ be a Polish space and let $X_1,\dots,X_n$ be a sequanece of $\Sigma$-valued random variables, identically distributed according the the measure $\mu\in \cM_1(\Sigma)$. The empirical distribution of $\bX:=(X_1,\dots,X_n)$ is given by
%\begin{equation}
%    \widehat\nu_\bX:= \frac1n\sum_{i=1}^n \delta_{X_i}.
%\end{equation}
%To set up the framework, let us equip $\cM_1(\Sigma)$ with the Lipschitz Bounded distance $d_{BL}$. Note that the topology induced on $\cM_1(\Sigma)$ corresponds to the weak convergence of probability measures. 
%
%Throughout this section, let $\Delta^m$ be the standard $m$-simplex given by
%\begin{equation}
%    \Delta^m:=\{(x_0,\dots,x_m): \sum_{i=0}^m x_i=1,x_i\geq 0\}.
%\end{equation}
%Further, let $C_b(\Sigma)$ denote the collection of bounded continuous functions $\phi:\Sigma\rightarrow \R$, equipped with the supremum norm.
%
%As before, let $N^d(A,\eps)$ be the $\eps$-covering number of the set $A$ equipped with the distance function $d$.
%\begin{lemma}[Covering number of the simplex]
%    Let $\eps>0$. Then,
%    \begin{equation}
%        \log N^{\norm{.}_2}(\Delta^m,\eps)\leq \frac1\eps\log m.
%    \end{equation}
%\end{lemma}
%\begin{proof}
%    
%\end{proof}
%
%
%
%Next lemma drives a non-asymptotic rate function for the random variable $\widehat\nu_\bX$.
%\begin{lemma} [Non-asymptotic Sanov on convex sets]
%    For any $n>0$ and any compact convex set $B\subseteq\cM_1(\Sigma)$, we have the following statement:
%    \begin{equation}
%        \frac1n\log\P_\mu\left[ \widehat \nu_\bX\in B\right] \leq -\inf_{\nu\in B} \KL(\nu\|\mu).
%    \end{equation}
%\end{lemma}
%
%\begin{proof} 
%Observe that for any $\phi\in C_b(\Sigma)$, 
%\begin{align}
%    \P_\mu\left[ \widehat \nu_\bX\in B\right]\leq & 
%    \E\left[ \exp\left\{ n\langle\phi,\widehat\nu_\bX \rangle
%    -n\inf_{\nu\in B} \langle \phi,\nu\rangle \right\} 
%    \one_{B}(\widehat\nu_\bX)\right]\\
%    \leq& \exp\left\{ -n\inf_{\nu\in B}\langle
%    \phi,\nu\rangle \right\}
%    \E\left[\exp\left\{
%    n\langle \phi, \frac1n\sum_{i=1}^n \delta_{X_i} \rangle\right\}\right]\\
%    =& \exp\left\{ -n\inf_{\nu\in B} \left[ \langle \phi,\nu\rangle - \log\int_\Sigma e^\phi\de\mu \right] \right\}\,
%\end{align}
%where the last equality is the result of the independence of the variables $\delta_{X_i}$.
%
%The next step is to optimize the parameter $\phi$. Note that $C_b(\Sigma)$ is a convex vector space, and
%$\log\int_\Sigma e^\phi\de\mu $ is a convex function of $\phi$, and hence
%\begin{equation}
%    \langle \phi,\nu\rangle - \log\int_\Sigma e^\phi\de\mu
%\end{equation}
%is concave and upper semi-continuous in the argument $\phi\in C_b(\Sigma)$ and linear and lower semi-continuous in $\nu\in B$. \kas{ I think it's continuous in both?}The conclusion follows by using the Sion's min-max theorem:
%\begin{align}
%    \frac1n\log\P_\mu\left[ \widehat\nu_\bX\in B\right]&
%    \leq -\sup_{\phi\in C_b(\Sigma) }\inf_{\nu\in B} \left[\langle \phi,\nu\rangle - \log\int_\Sigma e^\phi\de\mu\right]\\
%    =& - \inf_{\nu\in B} \sup_{\phi\in C_b(\Sigma) }\left[ \langle \phi,\nu\rangle - \log\int_\Sigma e^\phi\de\mu\right]\\
%    =& -\inf_{\nu\in B} \KL(\nu\|\mu),
%\end{align}
%    where the last inequality is the result of \ref{dembo: 6.2.13}[dembo: 6.2.13].
%
%
%
%\end{proof}
%
%
%
%
%
%
%
%
%\begin{lemma}[Quantitative Varadhan's Lemma]
%    Let $k\equiv k(n)>0$, $\Sigma\subseteq \R^k$ be a compact polish space with diameter $R$, and $F:\cM_1(\Sigma)\rightarrow \R$ be a Borel-measurable function. Define
%    \begin{equation}
%        \Phi(n,\mu):= \frac1n\log\E_\mu\exp\left\{
%        nF(\widehat\nu_\bX)\right\},
%    \end{equation}
%
%    and assume that the following conditions hold: \kas{todo: the tail condition for the non-compact case}
%    \begin{enumerate}
%        \item $F$ is Lipschitz with the Lipschitz constant $L$.
%    \end{enumerate}
%\end{lemma}
%Then,
%\begin{equation}
%    \Phi(n,\mu)\leq \sup_{\nu\in\cM_1(\Sigma)}\left\{
%    F(\nu) - \KL(\nu\|\mu)\right\} + ?.
%\end{equation}
%
%
%\begin{proof}
%
%
%    Let $m>0$ and $\eps>0$. We will chose exact values of these parameters later. By lemma \ref{} for some constant $C$,
%    \begin{equation}
%        N^{\norm{.}_2}\left(\Sigma, R\left(\frac 1{Cm}\right)^{1/k}
%        \right)\leq m.
%    \end{equation}
%    Let us define, for any $\nu\in\cM_1(\Sigma)$, its discretized version $\nu^{(m)}\in\Delta^m$ via
%    \begin{equation}
%        \nu^{(m)}_i\equiv \nu^{(m)}(a_i) := \nu(A_i),\text{ for } i=0,\dots,m
%    \end{equation}
%    where $A_1,\dots,A_m$ are Borel-measurable sets that form an $R\left(1/{Cm}\right)^{1/k}$-partition of $\Sigma$, and $a_i\in A_i$ are arbitrary fixed points.
%    \kas{Needs details about the partitioning: measurable, radius, number of them}.
%        
%    
%    We start the proof by covering the simplex $\Delta^m$; namely,
%
%    \kas{Andrea, you can read the following 6 equations plus the definitions we used in lemma 44. Note that in this section we used $\cM_1$ to represent the probabilty space, and it is not equal to the manifold $\cM$.}
%    \kas{The following four equations are for the case where we are on the manifold $\cM=\{\bV\in \Sigma^{ n}: \E_{\widehat\nu_\bV}[G]=0\}$}
%    \begin{align}
%    &\frac1n\int_{\bV\in\cM} \exp\{n F(\widehat\nu_\bV)\} p_1(\bV) \de V \\  
%    &\leq \frac1n\log(N(\Delta^m,\eps)) + 
%    \sup_{\nu\in \Delta^m} \frac1n\log 
%    \int_{\bV\in\cM} \exp\{n F(\widehat\nu_\bV)\} p_1(\bV) \one_{\{\|\widehat\nu_\bV^{(m)} - \nu\|\leq \eps\}}\de_\cM V \\
%    &  \leq \frac1n\log(N(\Delta^m,\eps)) + 
%    \sup_{\nu\in \Delta^m} \left\{
%    \sup_{\bar\nu\in \cP(\Sigma),\|\bar\nu^{(m)}-\nu\|\leq \eps, \E_{\bar\nu}(\bG)=0} F(\bar \nu) + 
%    \frac1n\log 
%    \int_{\bV\in\cM} p_1(\bV) \one_{\{\|\widehat\nu^{(m)}_\bV - \nu\|\leq \eps\}}\de_\cM V\right\}\\
%     &  \leq \frac1n\log(N(\Delta^m,\eps)) + 
%    \sup_{\nu\in \Delta^m} \left\{
%    \sup_{\bar\nu\in \cP(\Sigma),\|\bar\nu^{(m)}-\nu\|\leq \eps, \E_{\bar\nu}(\bG)=0} F(\bar \nu) + 
%    \frac1n\log 
%    \int_{\bV} p_1(\bV) \one_{\{\|\widehat\nu^{(m)}_\bV - \nu\|\leq \eps\}} \one_{\{ \E_{\widehat\nu_\bV}[\bG]\leq \eps_2 
%    \}} \de \bV +\omega_{\cM}(\eps_2) \right\}\\
%     &  \leq \frac1n\log(N(\Delta^m,\eps)) + 
%    \sup_{\nu\in \Delta^m} \left\{
%    \sup_{\bar\nu\in \cP(\Sigma),\|\bar\nu^{(m)}-\nu\|\leq \eps, \E_{\bar\nu}(\bG)=0} F(\bar \nu) + 
%    \frac1n\log 
%    \P\left[\{\norm{\hat\nu^{(m)}_\bV - \nu}\leq \eps\} \cap \{\E_{\widehat\nu_\bV} [\bG]\leq \eps_2\}\right]
%    +\omega_{\cM}(\eps_2) \right\}\\
%     &\leq \frac1n\log(N(\Delta^m,\eps)) + 
%    \sup_{\nu\in \Delta^m} \left\{
%    \sup_{\bar\nu\in \cP(\Sigma),\|\bar\nu^{(m)}-\nu\|\leq \eps, \E_{\bar\nu}(\bG)=0} F(\bar \nu)  
%    -\inf_{\bar\nu:\norm{\bar\nu^{(m)} - \nu}\leq \eps  ,\E_{\bar\nu} [\bG]\leq \eps_2} \KL(\bar\nu\| \cN(\bzero, \bR(\bTheta)))   
%    +\omega_{\cM}(\eps_2) \right\}
%    \end{align}
%
%\kas{The rest is the equivalence of the last equations, but over the ambient space, ignoring the manifold.}
%
%
%
%    
%    \begin{align}
%        \Phi(n,\mu) =& \frac1n\log\E\left[\exp\left\{
%        nF(\widehat\nu_\bX)\right \}\right]\\
%        \leq &
%        \frac1n\log\left( N^{\norm{.}_2}(\Delta^m,\eps)\right)
%        + \sup_{\nu\in\Delta^m} \frac1n\log\E_\mu\left[
%        \exp\{nF(\widehat\nu_\bX)\}
%        \one(\norm{{\widehat\nu_\bX}^{(m)}-\nu}_2\leq \eps)
%        \right]\\
%        \leq& \frac1n\log\left( N^{\norm{.}_2}(\Delta^m,\eps)\right) + 
%        \sup_{\nu\in\Delta^m} \left\{ \sup_{\substack{\bar \nu\in \cM_1(\Sigma):\\
%        \norm{\bar\nu^{(m)}-\nu}_2\leq \eps}} F(\bar \nu) 
%        +\frac1n\log\P\left[
%        \norm{\widehat\nu_\bX^{(m)}-\nu}_2\leq \eps\right]
%        \right\}.
%    \end{align}
%For the ease of notation let us define, for any $\nu\in \Delta^m$,
%\begin{equation}
%    \Omega(\nu,\eps;m):= \left\{\bar\nu\in\cM_1(\Sigma): \norm{\bar\nu^{(m)}-\nu}_2\leq \eps \right\}.
%\end{equation}
%Note that $\Omega(\nu,\eps;m)$ is a convex compact subset of $\cM_1(\Sigma)$. To see this, consider any two probability measures $\nu_1,\nu_2\in\Omega(\nu,\eps;m)$ and $i\in[m]$:
%\begin{align}
%    (\alpha \nu_1^{(m)} + (1-\alpha)\nu_2^{(m)})(a_i) = &
%    \alpha \nu_1(A_i) + (1-\alpha)\nu_2(A_i)\\
%    =& \alpha \nu_1^{(m)}(a_i) + (1-\alpha)\nu_2^{(m)}(a_i),
%\end{align}
%and hence,
%\begin{align}
%    \norm{(\alpha \nu_1^{(m)} + (1-\alpha)\nu_2^{(m)})-\nu}_2 \leq& \alpha\norm{\nu_1^{(m)}-\nu}_2+ (1-\alpha)\norm{\nu_2^{(m)}-\nu}_2\leq \eps,
%\end{align}
%which proves the convexity of $\Omega(\nu,\eps;m)$. \kas{Is the compactness obvious?} 
%
%Further, note that
%\begin{align}
%    d_{BL}(\nu_1^{(m)},\nu_1) \leq& 
%    \sup_{f\in \cF_{LU}}\left\{ \sum_{i=0}^m 
%    \sup_{a\in A_i}|f(a)-f(a_i)|\nu_1(A_i)
%    \right\}\\
%    \leq & R\left(\frac1{Cm}\right)^{1/k},
%\end{align}
%and
%\begin{align}
%    d_{BL}(\nu_1^{(m)},\nu)\leq&
%    \sup_{f\in \cF_{LU}}\left\{ \sum_{i=0}^m 
%    f(a_i)(\nu^{(m)}(a_i) - \nu(a_i))\right\}\\
%    \leq& \left(\sum_{i=1}^m f(a_i)^2\right)^{1/2} \norm{\nu_1^{m} - \nu}_2\\
%    \leq& \sqrt{m}\eps. 
%\end{align}
%Hence overall we get
%\begin{align}
%    d_{BL}(\nu_1,\nu_2)\leq& 
%    d_{BL}(\nu_1,\nu^{(m)}_1)+ d_{BL}(\nu^{(m)}_1,\nu)
%    + d_{BL}(\nu_2,\nu^{(m)}_2)+ d_{BL}(\nu^{(m)}_2,\nu)\\
%    \leq & C_1\left(\sqrt{m}\eps + R\left(\frac1{Cm}\right)^{1/k}\right)
%\end{align}
%for some constant $C_1$.
%
%This allows us to apply lemma \ref{}[Sanov] alongside the lipcshitzness assumption of the function $F$ to get
%\begin{align}
%        \Phi(n,\mu)
%        \leq& \frac1n\log\left( N^{\norm{.}_2}(\Delta^m,\eps)\right)+ 
%        \sup_{\nu\in\Delta^m} \left\{ \sup_{\bar\nu\in \Omega(\nu,\eps;m)} F(\bar \nu) 
%        - \inf_{\bar\nu\in\Omega(\nu,\eps;m)}\KL(\bar\nu\|\mu)
%        \right\}\\
%        \leq&  
%        \frac1n\log\left( N^{\norm{.}_2}(\Delta^m,\eps)\right) + \sup_{\nu\in\Delta^m}
%        \left\{ \sup_{\bar\nu\in \Omega(\nu,\eps;m)} F(\bar\nu) - \KL(\bar\nu\|\mu) + 
%        LC_1\left(\sqrt{m}\eps + R\left(\frac1{Cm}\right)^{1/k}\right) \right\}\\
%        \leq& \frac1n\log\left( N^{\norm{.}_2}(\Delta^m,\eps)\right) +
%        LC_1\left(\sqrt{m}\eps + R\left(\frac1{Cm}\right)^{1/k}\right)+
%        \sup_{\nu\in\cM_1(\Sigma)}\left\{
%        F(\nu)-\KL(\nu\|\mu) 
%        \right\}.
%\end{align}
%Next, let 
%\begin{equation}
%    \eps = , \quad m = .
%\end{equation}
%Consequently, we can use lemma \ref{}[simplex] to get
%\begin{equation}
%    \Phi(n,\mu) \leq \sup_{\nu\in\cM_1(\Sigma)}
%    \left\{ F(\nu) + \KL(\nu\|\mu)
%    \right\} + 
%\end{equation}
%\end{proof}
%
%
%\subsection{Indicator function}
%\kas{tackeling $x_{\min}(\mu^*)$}
%In this section, we 
%
%
%
%
%\newpage
%\subsection{Quantitative Varadhan Lemma- Cramer approach}\kas{Specialized to our choice of function $F$. TODO: the minmax needs lower semi cont condition to be checked. instead of $\R^{k_1}$, use generic $\bSigma$.}
%Throughout this section, $\cM_1(\Sigma)$ denotes the space of the (Borel) probability measures on the (Polish) space $\Sigma$. The following theorem presents a variant of Varadhan's lemma applicable in non-asymptotic settings, providing explicit quantitative error bounds.
%\begin{theorem}
%     Let $k_1\equiv k_1(n)$, $k_2\equiv k_2(n)$, and let $G:\R^{k_1}\rightarrow \R^{k_2}$ and $F:\R^{k_2}\rightarrow \R$ be two continuous and Borel-measurable fucntions. \kas{only Borel or cont. which one?}Define
%    \begin{equation}
%        \Phi(n,\mu) := \frac1n\log \E_\mu \exp\left\{ n F(\widehat S_n)\right\},
%    \end{equation}
%    where $\mu\in\cM_1(\R^{k_1})$, $(X_1,\dots,X_n)\sim \mu^{\otimes n}$ and $\widehat S_n = \frac1n\sum_{i=1}^n G(X_i)$. 
%      
%
%
%    
%    Assume that the following conditions hold:
%    \begin{enumerate}
%        %\item  $\Psi_*:=\sup_{\nu\in\cM_1(\R^{k_1})} \Psi(\nu;\mu)$.
%        
%        %\item   $\cH_*:=\sup_{\bs\in\R^{k_2}}\cH(\bs)$ is achieved at a point $\bs_*\in\R^{k_2}$.\kas{Can we get rid of the unique assumption?}
%        %and there exists a neighborhood of $\bs_*$ such as $\cB(\bs_*, r_*)$ where $\bz$ is the unique maximizer of the function $\cH$ at $\cB(\bs_*,r_*)$.%
%        
%        % \item The Hessian of the function $F$ has is Lipschitz in a neighborhood of $\bs_*$, i.e. there exists $L_n$ such that for any $\bs_1,\bs_2\in\cB(\bs_*,r_*)$:
%        % \begin{equation}
%        %      \| \nabla^2 F(\bs_1) - \nabla^2 F(\bs_2)\| \leq L_n 
%        %     \|\bs_1-\bs_2\|_2.
%        % \end{equation}
%        
%        \item \kas{Generalizing the Lipschitz assumption:} $F$ is Pseudo-Lipschitz; i.e. there exists constants $L\geq0$ and $p\geq 0$ such that for any $\bz_1,\bz_2\in\R^{k_2}$:
%        \begin{equation}
%            |F(\bz_1)-F(\bz_2)|\leq L(1+\|\bz_1\|^p+\|\bz_2\|^p)\norm{\bz_1-\bz_2}.
%        \end{equation}
%        
%        \item For some constant $\alpha>0$, there exists $L_0,L_1$ and $K_0,K_1$ such that for all $\bz\in\R^{k_2}$:
%        \begin{align}
%            &F(\bz)\leq L_0+L_1 \|\bz\|^\alpha,\\
%            &\P_\mu\left[\norm{\widehat S_n}\geq t\right]\leq \exp\left\{K_0 n-K_1 n t^\alpha\right\},
%        \end{align}
%        and $K_1>L_1$.
%    \end{enumerate}
%
%
%    
%    Then,\kas{add explicit terms}
%    \begin{equation}
%         \Phi(n,\mu)\leq \sup_{\nu\in\cM_1(\Sigma)} 
%         \left\{F\left(\int_{\Sigma} G(\bx)\de\nu(\bx)\right) - \KL(\nu\|\mu) \right\}
%        + O\left(\frac{k_2}{n}\log\left(\frac{n}{k_2}\right)\right).  
%    \end{equation}   
%\kas{This is not the correct rate when the assumptions' parameters are dependent on $k_2$; For the correct rate in this case we should use the last equation in this section for the explicit dependency.}
%\kas{Postponed to the time when we have an exact formula for F and G to avoid cluttering.}
%\end{theorem}
%
%
%For the ease of notation from now on, Let $\Lambda$ be the log-Laplace transform of the random variable  $G(X)$ where $X \sim\mu$, and let $\Lambda^\star$ be the corresponding Legendre-Frenchel transform, i.e.
%     \begin{align}
%         &\Lambda(\blambda) = \log \E_\mu \exp\{\langle\blambda,G(X) \rangle\},\\
%         & \Lambda^\star(\bz) = \sup_{\blambda\in\R^{k_2}}
%         \{\langle \blambda, \bz\rangle - \Lambda(\blambda)\}.
%     \end{align}
%   
%    Further, define
%    \begin{align}
%        &\Psi(\nu;\mu):=F\left(\int_{\R^{k_1}} G(\bx)\de\nu(\bx)\right) -
%        \KL(\nu\|\mu),\\
%        &\cH(\bz):= F(\bz)-\Lambda^*(\bz).
%    \end{align}
%Before proceeding to the proof, let us break down the function $\Phi$ into two parts:
%    \begin{align}
%        \Phi_1(n;\mu)&:=\frac1n \log\E_\mu \exp\{nF(\widehat S_n)\}\one_{B(\bzero,R)}(\widehat S_n),\\
%        \Phi_2(n;\mu)&:=\frac1n \log\E_\mu \exp\{nF(\widehat S_n)\}\one_{B(\bzero,R)^c}(\widehat S_n) .\\      
%    \end{align}
%where $R$ is a positive parameter and its exact value will be chosen later.
%
%From now on, let $N_d(B,\epsilon)$ be the minimum number of $\epsilon$-balls covering the set $B\subseteq \R^d$.
%\begin{lemma}\kas{add refrence}
%    Let $\epsilon<1$, $\bx\in\R^d$ and $r>0$. Then,
%    \begin{equation}
%       \log N_d(B(\bx,r),\epsilon)\leq C d\log\left(\frac r{\epsilon}\right)
%    \end{equation}
%    for some absolute constant $C$.
%\end{lemma}
%
%
%The first step of the proof is to find a non-asymptotic rate function for the random variable $\widehat S_n$.
%\begin{lemma}[Cramer's Upper Bound on open sets] For any $n>0$ and any open measurable set $B\in \R^{k_2}$, we have the following statement:
%\begin{equation}
%    \frac1n\log \P_\mu\left[ \widehat S_n\in B\right] \leq
%    -\inf_{\bz\in B} \Lambda^\star(\bz).
%\end{equation}   
%\end{lemma}
%\begin{proof}
%    Observe that for any $\blambda\in\R^{k_2}$,
%    \begin{align}
%        \P_{\mu}\left[\widehat S_n\in B \right]  \leq&
%        \E\left[\exp\left\{ n\langle \blambda, \widehat S_n\rangle  -n\inf_{\bz\in B} 
%        \langle \blambda, \bz\rangle \right\} \
%        \one_{B}(\widehat S_n)\right]\\
%        \leq & \exp\left\{ -n\inf_{\bz\in B} 
%        \langle \blambda, \bz \rangle\right\}
%        \E \exp\left\{ n\langle \blambda,\widehat S_n \rangle\right\}\\
%        =& \exp\left\{-n\inf_{\bz\in B}[\langle  \blambda, \bs
%        \rangle - \Lambda(\blambda)]\right\}.
%    \end{align}
%    Note that the last equality is the result of the independence of the variables $G(X_i)$.
%    
%    The next step is to optimize the parameter $\blambda$. Note that the Laplace transform $\Lambda(.)$ is a convex function \kas{Dembo 2.2.31}, and hence 
%    $$\langle \blambda, \bz\rangle - \Lambda(\blambda)$$
%    is concave in the argument $\blambda$ and linear in $\bs$. The conclusion follows by using the min-max theorem: \kas{I think we need some finiteness over $\Lambda$ here?}
%    \begin{align}
%        \frac1n\log\P_\mu\left[\widehat S_n\in B\right]\leq&
%        -\sup_{\blambda\in\R^{k_2}}\inf_{ \bz\in\cB }[\langle  \blambda, \bz
%        \rangle - \Lambda(\blambda)]\\
%        =&-\inf_{ \bz\in B}\sup_{\blambda\in\R^{k_2}}[\langle  \blambda, \bz
%        \rangle - \Lambda(\blambda)]\\
%        =&- \inf_{\bz\in B} \Lambda^\star( \bz).
%    \end{align}  
%\end{proof}
%
%
%
%\begin{lemma}
%    Assume the condition \ref{}1 of Theorem \ref{}3 holds and let $\epsilon>0$. Then,
%    \begin{equation}
%        \Phi_1(n;\mu)\leq
%        \frac{CK_2}{n}\log\left(\frac{R}{\epsilon}\right) + \cH_* +L\epsilon(1+2(R+\epsilon)^p).
%    \end{equation}
%\end{lemma}
%\begin{proof}
%We start the proof by covering the set $B(\bzero, R)$ with $\epsilon$-balls. Hence,
%    \begin{align}
%        \Phi_1(n;\mu)=&\frac1n \log\E_\mu \exp\{nF(\widehat S_n)\}\one_{B(\bzero,R)}(\widehat S_n)\\
%        \leq &\frac1n
%        \log\left(N_{k_2} \left(B(\bzero,R),\epsilon\right)\right) + 
%        \sup_{\bz\in B(\bzero,R)} \frac1n\log\E\left[ 
%        \exp\{nF(\widehat S_n)\}
%        \one_{B(\bz,\epsilon)}(\widehat S_n)\right] \\
%        \leq & \frac{Ck_2}{n}\log\left(\frac{R}{\epsilon}\right) + 
%        \sup_{\bz\in B(\bzero,R)}
%        \left\{\sup_{\bar\bz\in B(\bz,\epsilon)}F(\bar\bz)+
%        \frac1n \log\P_\mu\left[\widehat S_n\in B(\bz,\epsilon)\right]\right\}\\        
%        \leq & \frac{Ck_2}{n}\log\left(\frac{R}{\epsilon}\right) + 
%        \sup_{\bz\in B(\bzero,R)}
%        \left\{\sup_{\bar\bz\in B(\bz,\epsilon)}F(\bar\bz)
%        -\inf_{\bar\bz \in B(\bz,\epsilon)}\Lambda^\star(\bar\bz)\right\},
%    \end{align}
%
%where the last inequality is the result of the lemma \ref{}36.
%
%\kas{Is the following assumption required for what follows(existence of a converging sequence)?
%$$B(\bzero,R+\epsilon)\subseteq D_{\Lambda^\star}:=\{\bz: \Lambda^\star(\bz)<\infty\}.$$}
%
%Now note that for any $\delta>0$, there exists $\bz_\delta\in B(\bz,\epsilon)$ and $\bar\bz_\delta\in B(\bz,\epsilon)$ such that
%\begin{equation}
%    \Lambda^\star(\bar\bz_\delta)\leq \inf_{\bar\bz\in B(\bz,\epsilon)}\Lambda^\star(\bar\bz) + \delta,\qquad 
%    F(\bz_\delta) \geq  \sup_{\bar\bz\in B(\bz,\epsilon)} F(\bar\bz) -\delta.
%\end{equation}
%
%    
%Further, assumption 1 of Thķorem 1 implies that
%\begin{align}
%    F(\bz_\delta)\leq &F(\bar\bz_\delta) + L(1+\|\bz_\delta\|^p+\|\bar\bz_\delta\|^p)\|\bar\bz_\delta-\bz_\delta\|\\
%    \leq & F(\bar\bz_\delta)+ L\epsilon(1+2(R+\epsilon)^p).
%\end{align}
%
%
%By incorporating these facts,
%\begin{align}
%    \Phi_1(n;\mu)\leq& \frac{CK_2}{n}\log\left(\frac{R}{\epsilon}\right) + 
%    \sup_{\bz\in B(\bzero,R)} \left\{
%    \lim_{\delta\rightarrow 0}
%     F(\bz_\delta) -\Lambda^\star(\bar \bz_\delta) + 2\delta
%    \right\}\\
%    \leq& \frac{CK_2}{n}\log\left(\frac{R}{\epsilon}\right) + 
%    \sup_{\bz\in B(\bzero,R)} \left\{
%    \sup_{\bar\bz\in B(\bz,\epsilon)}
%     F(\bar\bz) -\Lambda^\star(\bar \bz) + L\epsilon(1+2(R+\epsilon)^p)
%    \right\}\\
%     \leq& \frac{CK_2}{n}\log\left(\frac{R}{\epsilon}\right)+
%     L\epsilon(1+2(R+\epsilon)^p) + 
%    \sup_{\bz\in B(\bzero,R+\epsilon)} \left\{
%     F(\bz) -\Lambda^\star( \bz) 
%    \right\}\\
%    \leq &\frac{CK_2}{n}\log\left(\frac{R}{\epsilon}\right)  +L\epsilon(1+2(R+\epsilon)^p)+ \cH_*,
%\end{align}
%which completes the proof.
%\end{proof}
%
%\begin{lemma}
%    Assume condition 2\ref{} of Theorem 3 \ref{} holds. Then,
%    \begin{equation}
%    \Phi_2(n,\mu)  \le L_0  + K_0  - (K_1 - L_1)R^\alpha + \frac1n \log\left(\frac{K_1}{K_1 - L_1}\right).
%    \end{equation}
%\end{lemma}
%\begin{proof}
%
%In what follows, define
%\begin{equation}
%    Y_n := \norm{\widehat S_n }.
%\end{equation}
%Using assumption 2,
%    \begin{align}
%        \Phi_2(n,\mu) =&  \frac1n \log\E_\mu \left[ \exp
%        \{nF(\widehat S_n)\}\one_{B(\bzero,R)^c}(\widehat S_n)\right] \\       
%        \leq&\frac1n \log\E_\mu\left[
%        \exp\left\{n(L_0+L_1\norm{\widehat S_n}^\alpha )\right\}
%        \one_{B(\bzero,R)^c}(\widehat S_n)\right]\\
%        =&L_0 + \frac1n \log\E_\mu\left[
%        \exp\left\{nL_1 Y_n^\alpha \right\}
%        \one_{Y_n \geq R}\right]\\
%        =& L_0 + \frac1n \log 
%        \int_{0}^{\infty}
%        \P\mu\left[\exp\left\{nL_1 Y_n^\alpha \right\}
%        \one_{Y_n > R} \ge t\right]\de t.
%\end{align}
%
%Now denoting $\varphi(R) := \exp\{n L_1 R^\alpha\},$ we have
%\begin{equation}
%    \P\mu\left[
%    \exp\left\{nL_1 Y_n^\alpha \right\} \one_{Y_n \geq R} \ge t
%    \right]  = 
%    \begin{cases}
%    \P_\mu\left[
%    Y_n > R
%    \right]  & t< \varphi(R)\\ 
%    \P_\mu\left[
%    Y_n >  \varphi^{-1}(t)
%    \right]  & t\ge   \varphi(R)
%    \end{cases},
%\end{equation}
%so that, applying the tail-bound of Assumption 2 \ref{}, and using that $K_1 > L_1$
%we have
%\begin{align}
%\int_{0}^{\infty}
%        \P\left(\exp\left\{nL_1 Y_n^\alpha \right\}
%        \one_{Y_n > R} \ge t\right)\de t
%        &= 
%        \varphi(R) \P(Y_n > R) + \int_{\varphi(R)}^\infty \P(Y_n \ge \varphi^{-1}(t)) \de t\\
%        &\le \varphi(R) e^{K_0 n - K_1 n R^\alpha}
%        + e^{K_0 n} \int_{\varphi(R)}^\infty \left(\frac{1}{t}\right)^{K_1/L_1} \de t\\
%        &= e^{n( K_0 -(K_1- L_1) r^\alpha)}  + 
%        \frac{L_1}{K_1- L_1}e^{K_0 n} e^{ - n r^\alpha(K_1- L_1)}.
%\end{align}
%Combining with the equation ~\notate{ref} gives the bound 
%\begin{equation}
%    \Phi_3(n,\mu)  \le L_0  + K_0  - (K_1 - L_1)R^\alpha + \frac1n \log\left(\frac{K_1}{K_1 - L_1}\right).
%\end{equation}
%
%\end{proof}
%
%
%
%\begin{lemma}
%    Consider the setting of the theorem 3. Then,
%    \begin{equation}
%        \cH_*=\Psi_*.
%    \end{equation}
%\end{lemma}
%\begin{remark}
%    This equality can be justified for any fixed dimensions \( k_2 \) and \( k_1 \) by applying Varadhan's lemma twice: once using Cramér's theorem for the random variable \(\widehat{S}_n\), and once using Sanov's theorem for the empirical measure \(\frac{1}{n} \sum_{i=1}^n \delta_{X_i}\). For completeness, however, we provide an independent proof of this statement.
%\end{remark}
%
%
%
%\begin{proof}\kas{Is there any assumptions needed on $D_\Lambda$?}
%Note that following the definition of $\Psi$,
%\begin{align}
%    \sup_{\nu\in\cM_1(\R^{k_1})}\Psi(\nu;\mu) = & 
%    \sup_{\nu\in\cM_1(\R^{k_1})} F\left(\int_{\R^{k_1}} G(\bx)\de \nu(\bx)\right) -\KL(\nu\|\mu)\\
%    =&\sup_{\bz\in\R^{k_2}}\sup_{\substack{\nu \in \cM_1(\R^{k_1}),\\ \int_{\R^{k_1} }G(\bx)\de\nu(\bx)=\bz}} F(\bz) - \KL(\nu\|\mu)\\
%    =&\sup_{\bz\in\R^{k_2}}\sup_{\nu\in\cM_1(\R^{k_1})} \inf_{\blambda\in\R^{k_2}}
%    F(\bz) +\langle\lambda,\int_{\R^{k_1}} G(\bx)\de\nu(\bx) -\bz\rangle
%    -\KL(\nu\|\mu),
%\end{align}
%where we used the Lagrange multiplier theorem in the last equality. 
%
%Next, note that $\cM_1(\R^{k_1})$ is a convex set and $\KL(.\|\mu)$ is a convex function in the space of all probability measures $\cM_1(\R^{k_1})$.  Hence, Von Neumann minimax Theorem is applicable in our setting:
%
%\begin{align}
%    \sup_{\nu\in\cM_1(\R^{k_1})}\Psi(\nu;\mu) = &\sup_{\bz\in\R^{k_2}} \inf_{\blambda\in\R^{k_2}}  \sup_{\nu\in\cM_1{\R^{k_1}}}  
%    F(\bz) +\langle\lambda,\int_{\R^{k_1}} G(\bx)\de\nu(\bx) -\bz\rangle
%    -\KL(\nu\|\mu)\\
%    =&\sup_{\bz\in\R^{k_2}} \inf{\blambda\in\R^{k_2}} F(\bz)-\langle \blambda,\bz\rangle 
%    +\sup_{\nu\in\cM_1(\R^{k_1})} \int_{\R^{k_1}} \left(
%    \langle \blambda,G(\bx)\rangle
%    -\frac{\de\nu(\bx)}{\de\mu(x)}\right)\de\nu(\bx).
%\end{align}
%By optimizing over $\nu$, we get the desired result:
%\begin{align}
%    \Psi_* = & \sup_{\bz\in\R^{k_2}} \inf_{\blambda\in\R^{k_2}} F(\bz)-\langle \blambda,\bz\rangle   + \bLambda(\blambda)=\cH_*. 
%\end{align}
%
%
%    
%\end{proof}
%
%
%
%
%Now, we can move on to the proof of Theorem 3.
%\begin{proof}[Proof of Theorem 3]
%Let
%\begin{equation}
%    R=\left(\frac{L_0+K_0 - \cH_*}{K_1-L_1}\right)^{1/\alpha}.
%\end{equation}
%Hence, lemma \ref{} implies that
%\begin{align}
%    \Phi_2(n;\mu)& \leq L_0  + K_0  - (K_1 - L_1)R^\alpha + \frac1n \log\left(\frac{K_1}{K_1 - L_1}\right)\\
%    &=   \cH_* +\frac1n\log\left(\frac{K_1}{K_1-L_1}\right).
%\end{align}
%Further, let 
%\begin{equation}
%    \epsilon =\min\{ \frac{k_2}{n},R\}.
%\end{equation}
%Using the result of lemma \ref{ },
%
%\begin{align}
%    \Phi_1 \leq& \cH_* +
%    \frac{Ck_2}{n}\log\left(\frac{R}{\epsilon}\right) 
%    +L\epsilon(1+2(R+\epsilon)^p)\\
%    =& \cH_* + \frac{k_2}{n}\left[C\log\left(\frac{n}{k_2}\right) + \frac{C}{\alpha}
%    \log\left(\frac{L_0+K_0 - \cH_*}{K_1-L_1}\right) +
%    L+2L\left(\left(\frac{L_0+K_0 - \cH_* }{K_1-L_1}\right)^{1/\alpha}+\frac{k_2}{n}\right)^p\right]
%\end{align}
%Hence, incorporating the result of Lemma \ref{} gives us the final result:
%
%\begin{align}
%    \Phi(n;\mu) \leq&
%    \max\{\Phi_1(n;\mu),\Phi_2(n;\mu)\} + \frac{\log 3}{n}\\
%     \leq & \Psi_* + \max\bigg\{
%     \frac{k_2}{n}\left[C\log\left(\frac{n}{k_2}\right) + \frac{C}{\alpha}
%    \log\left(\frac{L_0+K_0 - \cH_*}{K_1-L_1}\right) +
%    L+2L\left(\left(\frac{L_0+K_0 - \cH_* }{K_1-L_1}\right)^{1/\alpha}+\frac{k_2}{n}\right)^p\right]\\
%    &\qquad\qquad ,\frac1n  \log\frac{K_1}{K_1-L_1}
%    \bigg\}\\
%    \leq& \Psi_* + O\left(\frac{k_2}{n}\log\left(\frac{n}{k_2}\right)\right).
%\end{align}
%\end{proof}
%
%
%
%
%
%
%
%
%
%\newpage
%\subsection{Adding the Stieltjse Transform constraint}
%Throughout this section, let $\widehat\nu_\bX = \frac1n \sum_{i=1}^n \delta(\bX_i)$ be the empirical distribution of columns of the matrix $\bX$. Recall that $S^\star(\nu)\in\sS^{k\times k}$ satisfies the equation 
%\begin{equation}
%   \bG(S;\nu):= \frac1\alpha S^{-1} - \E_\nu[(\bI_k+\nabla^2\bell S)^{-1}\nabla^2\bell]=0.
%\end{equation}
%We are interested in bounding
%\begin{equation}
%    \frac1n\log\E_{\mu} \left[ \exp\left\{nF\left(
%    \E_{\widehat\nu_X}\left[ H(V,S^\star(\widehat\nu_X) \right]
%    \right) \right\} \right],
%\end{equation}
%where $X = (X_1,\dots,X_n)\sim \mu^{\otimes n}$ and $V|X\sim \widehat \nu_X$.
%\paragraph{Step 1: Covering $\sS^{k\times k}$}.
%Let $R>0$. Let us first consider the term
%\begin{equation}
%    \Phi_1:=\frac1n\E_{\mu} \left[ \exp\left\{nF\left(
%    \E_{\widehat\nu_X}\left[ H(V,S^\star(\widehat\nu_\bX) \right]
%    \right) \right\} \one_{B(\bzero,R)}S^\star(\widehat\nu_X) \right].
%\end{equation}
%Consider an $\eps$ cover of $B(\bzero,R)\subset \sS^{k\times k}$. Using the log-sum inequality, we get
%\begin{align}
%    \Phi_1& \leq \frac{k^2}{n} \log\left(\frac{R}{\eps}\right)
%    +\sup_{\bS\in B(\bzero,R)} \frac1n\E_{\mu} \left[ \exp\left\{nF\left(
%    \E_{\widehat\nu_X}\left[ H(V,S^\star(\widehat\nu_X) \right]
%    \right) \right\} \one_{B(\bS,\eps)}(S^\star(\widehat\nu_X)) \right]\\
%    &\leq \frac{k^2}{n} \log\left(\frac{R}{\eps}\right)
%    +\sup_{\bS\in B(\bzero,R)} \frac1n\E_{\mu} \left[ \exp\left\{nF\left(
%    \E_{\widehat\nu_X}\left[ H(V,\bS) \right]
%    \right) \right\} \one_{B(\bS,\eps)}(S^\star(\widehat\nu_X)) \right] + \omega_F(\eps)\\
%    &\leq \frac{k^2}{n} \log\left(\frac{R}{\eps}\right)
%    +\sup_{\bS\in B(\bzero,R)} \frac1n\E_{\mu} \left[ \exp\left\{nF\left(
%    \E_{\widehat\nu_X}\left[ H(V,\bS) \right]
%    \right) \right\} 
%    \one(\bG(\bS,\widehat\nu_X)\leq \eps_2) \right] + \omega_F(\eps)
%\end{align}
%\paragraph{Step 2: Indicator function}
%Next, we need to show that for any $\bS\in\sS^{k\times k}$,
%\begin{equation}
%    \frac1n\E_{\mu} \left[ \exp\left\{nF\left(
%    \E_{\widehat\nu_X}\left[ H(V,S) \right]
%    \right) \right\} \one(\bG(S,\widehat\nu_X)\leq \eps_2) \right]
%    \leq \sup_{\substack{\nu,\\ \bG(\bS,\nu)\leq \eps_2}} F(\E_\nu[H(V,\bS)]) - \KL(\nu\|\mu) + \omega_\bG(\eps_2),
%\end{equation}
%where $\bG$ is a linear function of $\nu$.
%\paragraph{Step 3: } 
%Lastly, we need to show 
%\begin{equation}
%    \sup_{\bS} \sup_{\nu:\bG(\bS,\nu)\leq \eps_2}
%    F(\E_\nu[H(V,\bS)]) - \KL(\nu\|\mu) + \omega_F(\eps) + \omega_\bG(\eps_2) \leq \sup_{\nu} F(\E_{\nu}[H(V,S^\star(\nu))]) - \KL(\nu\|\mu) + \omega(\eps) 
%\end{equation}
%
%
%
%
%
%
%
%
%
%
%
%
%\newpage
%\subsection{Quantitative Varadhan Lemma on the Manifold}
%
%we are interested in applying the Large deviation results from the last section to an object of the form
%\begin{equation}
%    \int_{\bt\in \cM} \exp\{nF(\widehat\nu_\bt)\} P(\bt) \de V,
%\end{equation}
%where $P(\bt_i)\sim\cN(\bzero, \bR(\bTheta)$ is the Gaussian density, as defined in \ref{}.
%
%To be able to apply Lemme \ref{}[Varadhan], we write the integration over the manifold $\cM$ as integration over the whole space using the Dirac delta function:
%
%\begin{lemma}
%    \begin{equation}
%        \int_{\bt \in \cM}\exp\{nF(\widehat\nu_\bt)\} P(\bt) \de V =  
%        \lim_{\eps\rightarrow 0 }\int_{\R^{nk+nk_0}} \exp\{nF(\widehat\nu_\bt)\}P(\bt)
%        \one
%        \left(\bL(\bt)^\sT(\bt)\leq \eps\right)  \de \bt
%    \end{equation}
%\end{lemma}
%
%
%\begin{lemma}
%    let $L:\R^{k_1}\rightarrow\R^{r_k}$ and $F:\R^{k_1}\rightarrow\R^{k_2}$ be two borel-measurable functions and let $X$ be  a random variable with distribution $\nu\in\cM_1(\R^{k_1})$. Then,
%    \begin{equation}
%        \E[F(X)|L(X)=0] = \frac{\E[F(X)\cdot\delta(L(X))]}
%        {\E[\delta(L(X))]},
%    \end{equation}
%    where $\delta$ is the Dirac Delta function.
%\end{lemma}
%\begin{lemma}
%    Let $L:\R^{k_1}\rightarrow\R^{r_k}$. Then,
%    \begin{equation}
%        \E\left[
%        \delta(\sum_{i=1}^n L(X_i)) \leq
%        \right]
%    \end{equation}
%\end{lemma}
%\begin{proof}
%    First, observe that by definition,
%    \begin{align}
%        \E[\delta(\sum_{i=1}^n L(X_i))] 
%        =& \lim_{\eps\rightarrow 0 }
%        \frac1{2\eps}\P\left(|\sum_{i=1}^n L(X_i)|\leq \eps\right)\\
%        \leq & \lim \frac{1}{2\eps} \exp\{-n\inf_{\bz\in B(\bzero,n\eps)}\Lambda^\star_L(\bz)\}.
%    \end{align}
%    
%\end{proof}
%
%
%\begin{lemma}
%    Let $k_1\equiv k_1(n)$, $k_2\equiv k_2(n)$, and let $G:R^{k_1}\rightarrow \R^{k_2}$. Let $\cM\subset\R^{nk_1}$ be a smooth manifold defined by
%    \begin{equation}
%        \cM\equiv\cM(n):=\{\bx=(\bx_1^\sT,\dots,\bx_n^\sT)\in\R^{nk_1}: \sum_{i=1}^{n}L(\bx_i)=0\}
%    \end{equation}
%    , where $L:\R^{k_1}\rightarrow\R^{r_k}$ is smooth and differentiable. Further
%    Define
%    \begin{equation}
%        \Phi(n,\mu):= \frac1n\log
%        \int_{\bx\in\cM}
%        \exp\left\{
%        n F(\widehat S_n(\bx))\de \mu^{\otimes n }(\bx)
%        \right\}\de V.
%    \end{equation}
%    Then, under the assumptions of the Theorem \ref{},
%    \begin{equation}
%        \Phi(n,\mu)\leq \sup_{\substack{\nu\in \cM_1(\R^{k_1}),\\ \E_\nu(L(X))=0}}
%    \end{equation}
%    
%\end{lemma}
%
%\begin{proof}
%    
%\end{proof}


\subsection{Simplifying the constraint set in Theorem~\ref{thm:global_min}}
\label{sec:simplifying_constraint_set}
We state and prove the two lemmas referenced in the proof of Theorem~\ref{thm:global_min} that allow us to simplify the set of critical points on which the rate function bound is applicable. 
\begin{lemma}[Lower bounding the smallest singular value of the Jacobian]
\label{lemma:jacobian_lb}
Assume $\rho(t) = \lambda \; t^2/2$ for $\lambda \ge0$.
For any critical point $\bTheta$ of $\widehat R_n(\bTheta)$, 
we have under Assumption~\ref{ass:loss},
\begin{equation}
    \sigma_{\min}\left( \bJ_{(\bTheta,\bbV)} \bG^\sT\right) \ge 
     \frac{\sigma_{\min}(\grad^2 \hat R_n(\bTheta))
     \sigma_{\min}([\bTheta,\bTheta_0])
     }{(1 + \|\bX\|_\op )}.
\end{equation}
\end{lemma}

\begin{proof}
Recall that $\bJ_{(\bTheta,\bbV)}\bG\in\reals^{k(k+k_0)\times (dk + n (k+k_0))}$ denotes the Euclidean Jacobian of the function $\bg : \R^{dk + n (k+k_0)} \to \R^{k(k+k_0)}$ obtained from vectorizing $\bG$ and its arguments.
Namely,
\begin{equation}
    \bg(\bTheta,\bbV) = (g_{i,j}(\bTheta,\bbV)_{i\in[k],j\in[k+k_0]},\quad
    g_{i,j}(\bTheta,\bbV) = \begin{cases}
       \frac1n\bell_i^\sT\bv_j  + \lambda \btheta_i^\sT\btheta_j & j \le k\\
       \frac1n\bell_i^\sT\bv_{0,j}  + \lambda \btheta_i^\sT\btheta_{0,j - k} & 
       j  > k
    \end{cases},
    i \in[k].
\end{equation}
%So the first $k$ columns of $\bJ \bG^\sT$ are given by
%\begin{equation}
%\frac1n
%     \begin{bmatrix}
%         \bSec (\bI_k \otimes \bv_1) \\
%         \vdots\\
%         \bSec (\bI_k \otimes \bv_k) \\
%    \end{bmatrix}  + 
%        (\bI_k \otimes \bell)
% \quad \textrm{where} \quad 
%\bar\bell = \begin{bmatrix}
%    \bell_1\\
%    \vdots\\
%    \bell_k
%\end{bmatrix} \in\R^{nk},
%\end{equation}
%and the last $k$ columns are given by
%with the definitions of Eq.~\eqref{eq:SecDef}
%\begin{equation}
%    \bJ\bG^\sT = \begin{bmatrix}
%    \frac1n
%     \begin{bmatrix}
%         \bSec (\bI_k \otimes \bv_1) \\
%         \vdots\\
%         \bSec (\bI_k \otimes \bv_k) \\
%    \end{bmatrix} 
%    & 
%    \frac1n
%   \begin{bmatrix}
%         \tilde\bSec (\bI_k \otimes \bv_1) \\
%         \vdots\\
%         \tilde\bSec (\bI_k \otimes \bv_k) \\
%   \end{bmatrix} 
%   & 
%   \lambda\begin{bmatrix}
%          \bI_k\otimes \btheta_1\\
%         \vdots\\
%         \bI_k \otimes \btheta_k \\
%   \end{bmatrix}\\
%   %%%%%%%%%%%%%%
%   \frac1n
%     \begin{bmatrix}
%         \bSec (\bI_k \otimes \bv_{0,1}) \\
%         \vdots\\
%         \bSec (\bI_k \otimes \bv_{0,k_0}) \\
%    \end{bmatrix}
%    & 
%    \frac1n
%   \begin{bmatrix}
%         \tilde\bSec (\bI_k \otimes \bv_{0,1}) \\
%         \vdots\\
%         \tilde\bSec (\bI_k \otimes \bv_{0,k_0}) \\
%   \end{bmatrix} 
%   & 
%   \lambda\begin{bmatrix}
%          \bI_k\otimes \btheta_{0,1}\\
%         \vdots\\
%         \bI_k \otimes \btheta_{0,k_0} \\
%   \end{bmatrix}\\
%    \end{bmatrix}
%    +
%    \begin{bmatrix}
%        (\bI_k \otimes \bar\bell) & 
%        \bzero_{nk^2 \times k_0} &  \lambda (\bI_k \otimes \bar\btheta)\\
%        \bzero_{nkk_0 \times k} & 
%        (\bI_{k_0} \otimes \bar\bell)
%         & \bzero_{dk}
%    \end{bmatrix}
%\end{equation}
Recalling the definitions in Eq.~\eqref{eq:SecDef} and Eq.~\eqref{eq:TildeSecDef} of $\bSec$ and $\tilde\bSec$, with sufficient diligence, the desired Jacobian can be computed to be 
\begin{align}
    \bJ\bG^\sT &= \begin{bmatrix}
    \frac1n
     \begin{bmatrix}
         \bSec (\bI_k \otimes \bv_1),& \dots ,&
         \bSec (\bI_k \otimes \bv_k) 
    \end{bmatrix} 
&
   \frac1n
     \begin{bmatrix}
         \bSec (\bI_k \otimes \bv_{0,1}), &
         \dots,&
         \bSec (\bI_k \otimes \bv_{0,k_0}) \\
    \end{bmatrix}\\
    \frac1n
   \begin{bmatrix}
         \tilde\bSec (\bI_k \otimes \bv_1), &
         \dots,&
         \tilde\bSec (\bI_k \otimes \bv_k) \\
   \end{bmatrix} 
   &
    \frac1n
   \begin{bmatrix}
         \tilde\bSec (\bI_k \otimes \bv_{0,1}), &
         \dots,&
         \tilde\bSec (\bI_k \otimes \bv_{0,k_0}) 
   \end{bmatrix} \\
   \lambda\begin{bmatrix}
          \bI_k\otimes \btheta_1, &
         \dots,&
         \bI_k \otimes \btheta_k 
   \end{bmatrix}
   &
   \lambda\begin{bmatrix}
          \bI_k\otimes \btheta_{0,1},&
         \dots,&
         \bI_k \otimes \btheta_{0,k_0} 
   \end{bmatrix} 
    \end{bmatrix}\\
    &\hspace{80mm}+
     \begin{bmatrix}
       \frac1n(\bI_{k + k_0} \otimes \bL) \\
       [\lambda(\bI_k \otimes \bTheta), \bzero_{dk \times kk_0}]
    \end{bmatrix} \in\R^{(nk + nk_0 + dk) \times k(k+k_0)}.
\end{align}
Define
\begin{equation}
    \bB := \begin{bmatrix}
       (\bI_k \otimes \bX^\sT)  &
        \bzero_{dk\times n k_0} &
        \bI_{dk}
    \end{bmatrix} \in\R^{dk \times (nk + nk_0 + dk)}.
\end{equation}
Recalling the definition of $\bH_0$ in Eq.~\eqref{eq:bH_def} and
noting that at any critical point of $\hat R_n(\bTheta)$, we have
\begin{equation}
  \bX^\sT\bK[\bv_i,\bv_{0,j}] = \bH_0 [\btheta_i , \btheta_{0,j}]
  \quad
  \textrm{and}
  \quad
  \bX^\sT\bell_{i}  = - \lambda \btheta_i \quad \textrm{for}\quad i\in[k],j\in[k_0],
\end{equation}
we can compute at a critical point
\begin{equation}
    \bB \bJ\bG^\sT = 
       \left(\frac1n \bH_0 + \lambda\bI_{dk}\right)\bA,
       \quad\textrm{where}
       \quad
       \bA := \left[(\bI_k \otimes \btheta_1),\dots,(\bI_k\otimes\btheta_k),(\bI_k \otimes \btheta_{0,1},\dots,(\bI_k\otimes\btheta_{0,k_0})\right].
\end{equation}
By direct computation, one finds that for some permutation matrix $\bP \in \R^{dk \times dk}$, we have
\begin{equation}
    \bA^\sT\bA = \bP (\bI_k \otimes [\bTheta,\bTheta_0]^\sT[\bTheta,\bTheta_0]) \bP^\sT,
\end{equation}
so that $\sigma_{\min}(\bA) =  \sigma_{\min}([\bTheta,\bTheta_0]),$  and hence
\begin{equation}
    \|\bB\|_\op \sigma_{\min}(\bJ \bG^\sT) \ge 
    \sigma_{\min}(\grad^2 \hat R_n(\bTheta)) \sigma_{\min}([\bTheta,\bTheta_0]).
\end{equation}
Using $\|\bB\|_\op \le \|\bX\|_\op + 1$ gives the claim.
\end{proof}

%\begin{lemma}[Lower bounding the smallest singular value of the Jacobian]
%\label{lemma:jacobian_lb}
%Assume $\rho(t) = \lambda \; t^2/2$ for $\lambda \ge0$.
%For any critical point $\bTheta$ of $\widehat R_n(\bTheta)$, 
%we have under Assumptions~\ref{ass:loss} on $\ell$
%\begin{equation}
%    \sigma_{\min}\left( \bJ_{(\bbV,\bTheta)} \bG^\sT\right) \ge 
%    C \frac{\sigma_{\min}(\grad^2 \hat R_n(\bTheta))}{(1 + \lambda \|\grad^2\hat R_n(\bTheta)\|_\op) } \frac{\sigma_{\min}([\bTheta,\bTheta_0])}{ \|\bX\|_\op \vee 1},
%\end{equation}
%for some universal constant $C>0$.
%\end{lemma}
%
%\begin{proof}
%\bns{fix this}
%%\subsubsection{Proof of Lemma~\ref{lemma:jacobian_lb}}
%\label{sec:proof_of_lemma_jacobian_lb}
%%For the case of $\lambda = 0$, 
%%consider the block matrix
%%\begin{equation}
%%    \bB_0 := \begin{bmatrix}
%%       \bI_k \otimes  \bX^\sT & \bzero & \bzero \\ 
%%        \bzero & \bzero_{kd \times kn} &\bzero\\
%%        \bzero & \bzero & \bzero_{kd \times kd}
%%    \end{bmatrix}.
%%\end{equation}
%%By explicitly computing $\bJ_{(\bbV,\bTheta)} \bG$, and with some algebra, one can see that
%%the singular values of 
%% $\bB_0 \bJ_{(\bbV,\bTheta)} \bG^\sT$ are equal to those of $\grad^2\hat R_n(\bTheta) (\bI_k \otimes  [\bTheta,\bTheta_0])).$ This gives the bound
%% \begin{equation}
%%     \sigma_{\min}(\bR) \sigma_{\min}(\bH)  \le \|\bB_0\|_\op \sigma_{\min}(\bJ_{(\bbV,\bTheta)} \bG).
%% \end{equation}
%%Noting that $\|\bB_0\|_\op \le \|\bX\|_\op $  gives the result.
%%Now for $\lambda >0$, let
%Let
%\begin{equation}
%    \bB_1 := \begin{bmatrix}
%       \bI_k \otimes \bX^\sT & \bzero & \bzero \\ 
%        \bzero & \bzero_{kd\times kn} &\bzero\\
%        \bzero & \bzero &\bI_d
%    \end{bmatrix}
%\end{equation}
%and denote the hessian of the loss by $\bH_0$, namely, $\bH_0 = \grad^2\hat R_n(\bTheta)- \lambda \bI$.
%By explicit computation, for $(\bTheta,\bbV)$ satisfying $\bG(\bTheta,\bbV) = \bzero$, the singular values of 
%$\bB_1 \bJ_{(\Theta,\bbV)} \bG^\sT$ can be shown to be equal to those of 
%\begin{equation}
%\bA:=
%\bA_0
%\begin{bmatrix}
%   \bI_{k}\otimes [\bTheta,\bTheta_0] & \bzero \\
%   \bzero & \bI_k\otimes [\bTheta,\bTheta_0]
%\end{bmatrix}
%\begin{bmatrix}
%   \bI_{k + k_0} \\
%   \bP
%\end{bmatrix},
%\quad\quad\textrm{where}\quad\quad
%\bA_0:=
%\begin{bmatrix}
%   \bH_0 & -\lambda\bI_{dk} \\
%   \lambda \bI_{dk} & \lambda \bI_{dk}
%\end{bmatrix}
%\end{equation}
%where $\bP \in\R^{(k+k_0)\times (k+k_0)}$ is some matrix obtained from a  permutation matrix by setting $k_0$ of the ones to $0$.
%Then
%\begin{equation}
%   \bA_0^\sT\bA_0  =   
%\begin{bmatrix}
%   \bH_0^2 +  \lambda^2 \bI& -\lambda \bH_0 + \lambda^2\bI_{dk} \\
%   -\lambda\bH_0 + \lambda^2 \bI_{dk} & 2 \lambda^2 \bI_{dk}.
%\end{bmatrix}
%\end{equation}
%So by the block inverse formula
%\begin{align}
%   \sigma_{\min}(\bA_0)^2 &\ge  \left( \|(\bH_0^2 + \lambda^2\bI)^{-1} \|_\op +\big(1 + \|(\bH_0^2 + \lambda^2\bI)^{-1} \|_\op \|-\lambda \bH_0 + \lambda^2 \bI\|_\op \big)^2  \|2(\bH_0 + \lambda\bI)^{-2}\|_\op  \right)^{-1}\\
%   &\ge  
%   \frac{\sigma_{\min}(\bH_0^2 + \lambda^2 \bI)}{1 + 2 \big(1 + \lambda \|\bH_0 - \lambda \bI\|_\op \big)^2}
%\ge   
%   \frac{\sigma_{\min}( \grad^2\hat R_n(\bTheta))^2}{2 + 4 \big(1 + \lambda \|\grad^2\hat R_n(\bTheta))\|_\op \big)^2}.
%\end{align}
%So \begin{equation}
%    \|\bB_1\|_\op\sigma_{\min}(\bJ_{(\bbV,\bTheta)} \bG^\sT) \ge 
%    \sigma_{\min}(\bB_1\bJ_{(\bbV,\bTheta)} \bG^\sT) \ge  \sigma_{\min}(\bA_0)  \sigma_{\min}([\bTheta,\bTheta_0]) \sigma_{\min}(\bI + \bP^\sT\bP).
%\end{equation}
%Now by definition of $\bP$, the matrix $\bP^\sT\bP$ is a diagonal matrix with only $1$'s and $0$'s along its main diagonal.  Meanwhile, $\|\bB_1\|_\op \le \|\bX\|_\op \vee 1$. Combining these gives the bound of the lemma.
%\end{proof}
%
\begin{lemma}\label{lemma:VolumeBound}
Assume $\sigma_{\min}(\bTheta_0) \succ r\bI$ for some $r>0$.
Define
   \begin{equation}
       \cS_{\delta,R} := \{ \bTheta \in\R^{d\times k} : \|\bTheta\|_F \le R,\quad \sigma_{\min}([\bTheta,\bTheta_0]) \le \delta \}
   \end{equation}
   for $\delta < r/2$.
Then 
%
\begin{equation}
    \vol(\cS_{\delta,R}) \le
     k (C(R,r))^{dk} (\sqrt{k} \delta)^{d - k - k_0+1}.
\end{equation}
for constant $C(R,r)>0$ depending only on $r$ and $R.$
\end{lemma}
\begin{proof}
If $\bTheta \in\cS_{\delta,R}$, then there exists some $(\bbeta^\sT,\bbeta_0^\sT)^\sT\in\R^{k+k_0}$ with norm 1 such that $\bTheta\bbeta+\bTheta_0\bbeta_0 = \bu$ for some $\bu$ with $\|\bu\|_2 \le \delta.$
Let $j = \argmax_{i\in[k]} |\beta_i|$, where $\beta_i$ is the $i$-th coordinate of $\bbeta$, and denote by $\bP^\perp_{j}$ the projection onto the orthocomplement of 
\begin{equation}
    \textrm{span}\left(\{\btheta_{0,i}\}_{i\in[k_0]} \cap \{\btheta_{i}\}_{i\in[k], i\neq j} \right).
\end{equation}
Since
$\bu = \sum_{i=1}^k \beta_i \btheta_i + \sum_{i=1}^{k_0} \beta_{0,i} \btheta_{0,i},$
we have $\delta \ge |\beta_j|\|\bP_{j}^\perp \btheta_j\|_2$.
Now note that $\sigma_{\min}(\bTheta_0) > 2\delta$
and $\|\bTheta\|_F \le R$ implies that there must exist some constant $c_0(R,c)$ depending only on $R$ and $r$ and such that $\|\bbeta\|_2 \ge c_0(R,c).$ 
Indeed, we have
\begin{equation}
    r \|\bbeta_0\|_2 \le \|\bTheta_0 \bbeta_0\| = \|\bu - \bTheta\bbeta\| \le \delta + R \|\bbeta\|_2.
\end{equation}
Using that $\|\bbeta\|_2^2 + \|\bbeta_0\|_2^2 = 1$ and $\delta < r/2$, this then gives
\begin{equation}
   \|\bbeta\|_2^2 \ge \frac12\frac{r^2}{r^2 + 2 R^2}.
\end{equation}
Hence, for $j$ being the index of maximum mass as above, we have $|\beta_j| \ge c_0 k^{-1/2}$. This allows us to conclude that
$\cS_{\delta,R} \subseteq \bigcup_{j=1}^k  \cV_{j}(\delta, R)$
where
\begin{equation}
    \cV_{j}(\delta, R) := \left\{\|\bTheta\|_F \le R ,\quad   \|\bP_{-j} \btheta_j\|_2 \le \frac{\sqrt{k}}{c_0} \delta\right\}.
\end{equation}
Meanwhile, for any $j \in[k]$,
\begin{equation}
    \vol( \cV_j(\delta,R)) \le  (C R)^{dk} \left(\frac{\sqrt{k} \delta}{c_0}\right)^{d - k - k_0+1}.
\end{equation}
Bounding the volume of the union by the sum of the volumes gives the claim.

%$\delta \ge \|\bP_{-j} \bTheta \bbeta\|_2  = |\beta_j| \|\bP_{-j}\btheta_j\|_2,$ 
%where
%$j := \argmax_{i \in[k+k_0]} |\beta_i|$ and $\bP_{-j}$ is the projection onto the span of $\{\btheta_i\}_{i\neq j}$. 
%Since $|\beta_j| \le k^{-1/2},$ this implies that
%    $\cS_{\delta,R} \subseteq \bigcup_{j=1}^k  \cV_{j}(\delta, R)$
%where
%\begin{equation}
%    \cV_{j}(\delta, R) := \{\|\bTheta\|_F \le R ,\quad   \|\bP_{-j} \btheta_j\|_2 \le \sqrt{k} \delta\}.
%\end{equation}
\end{proof}


\begin{lemma}
\label{lemma:min_sv_Theta}
Let $\bL(\bV,\bV_0,\bw)\in\R^{n\times k}$ be as defined in~\eqref{eq:def_bL_bRho}.
Assume $\sigma_{\min}(\bTheta_0) \succ c_0\bI$ for some $c_0 >0$.
Then under Assumption~\ref{ass:loss} on the loss, with the ridge regularizer $\rho(t) = \lambda t^2/2$,
%\begin{equation}
%    \bQ(\bTheta) := \frac1n \sum_{i=1}^n (\grad \ell \grad \ell^\sT)(\bTheta\bx_i,\bTheta_{0}\bx_i, \bw_i).
%\end{equation}
for any fixed $C,c>0$ and $\lambda \ge 0$, there exists $\delta >0$ sufficiently small such that
\begin{equation}
\lim_{n\to\infty}\P\left( \exists \bTheta :\sigma_{\min}([\bTheta,\bTheta_0]) < \delta,\;  \sigma_{\min}(\bL(\bX\bTheta,\bX\bTheta_0,\bw)) \ge  n\, c,\; \|\bTheta\|_F \le C,\;
\grad \hat R_n(\bTheta)  = \bzero
\right)   = 0.
\end{equation}
\end{lemma}
\begin{proof}
Let
\begin{equation}
    \bF(\bTheta) := \frac1{\sqrt{n}} \bX^\sT \bL(\bX\bTheta,\bX\bTheta_0,\bw) + \sqrt{n}\lambda \bTheta\, ,
\end{equation}
be the scaled gradient of the empirical risk.
Conditional on $\bX[\bTheta,\bTheta_0] = \bbV$ and $\bw,$ the random variable $\bP_{[\bTheta,\bTheta_0]}^\perp\bF(\bTheta)$ is  distributed as $\bU_{\bTheta}\bZ_0$ where  $\bU_{\bTheta}\in\reals^{n\times (d-k-k_0)}$ is a basis of the orthogonal complement of $[\bTheta,\bTheta_0]$
and
\begin{equation}
    \bZ_0 \sim \cN(\bzero, \bI_{d- k - k_0} \otimes \bL^\sT\bL/n)\, .
\end{equation}
Therefore we can bound for any fixed $\bTheta$,
\begin{align}
\label{eq:small_ball_prob}
\P\left( \|\bF(\bTheta)\|_F \le \eps \sqrt{nk} , \bL^\sT\bL \succ c n  \right)
&\le\E\left[\P\left( \|\bF(\bTheta)\|_F \le \eps \sqrt{nk} , \bL^\sT\bL \succ c n  \Big| \bX[\bTheta,\bTheta_0]=\bbV, \bw \right)\right]\\
&\le (C_0 \eps)^{dk - k - k_0}\, ,\nonumber
\end{align}
for some $C_0>0$ depending only on $c$.
For fixed $\delta,\eps >0$, consider now the sets
\begin{equation}
    \cA_0(\delta) := \{\bTheta : \sigma_{\min}(\bTheta,\bTheta_0) \le \delta\},\quad\quad
    \cA(\delta,\eps) := \{\bTheta \in\cA_0(\delta) : \|\bF(\bTheta)\|_F \le \eps\}.
\end{equation}
Since Assumption~\ref{ass:loss} guarantees that for any $\bTheta_1,\bTheta_2$,
\begin{equation}
    \|\bF(\bTheta_1) - \bF(\bTheta_2)\|_F^2 \le C_1 \,k \frac1n \|\bX\|^4_\op  \|\bTheta_1 -\bTheta_2\|_F^2,
\end{equation}
We have on the high probability event $\Omega_0 := \{\|\bX\|_\op \le 3 \sqrt{n}\},$
for any $\tilde\bTheta \in \cA_0(\delta)$ with $\bF(\tilde\bTheta) = \bzero$, 
\begin{equation}
\|\bTheta - \tilde \bTheta\|_F \le \eps/2 \;\;\Rightarrow\;\;
    \|\bF(\bTheta)\|_F  \le C_2 \sqrt{k n} \eps,\quad
\sigma_{\min}([\bTheta,\bTheta_0]) \le \delta + \eps\, .
\end{equation}
Namely, letting $\Ball_{\eps/2}^{d\times k}(\bzero)$ be the Euclidean ball in $\R^{d\times k}$ of radius $\eps/2$, this 
shows that
\begin{equation}
\cA_0(\delta) + \Ball_{\eps/2}^{d\times k}(\bzero)  \subseteq   \cA(\delta + \eps , C_2 \sqrt{k n} \eps).
\end{equation}
And since $\cA(\delta + \eps , C_2 \sqrt{k n} \eps) \subseteq \cA_0(\delta +\eps)$, standard bounds on the covering number $\cN_{\eps}( \cA_0(\delta)\cap\Ball_{C}^{d\times k}(\bzero))$ of $\cA_0(\delta)\cap\Ball_{C}^{d\times k}(\bzero)$ with Euclidean balls of radius $\eps$ give
\begin{equation}
   \cN_\eps(
   \cA_0(\delta)\cap\Ball_C^{d\times k}(\bzero))\;
   \vol(\Ball_{\eps/2}^{d\times k}(\bzero))  \le  \vol\big((\cA_0(\delta)\cap\Ball_C^{d\times k}(\bzero)) + \Ball_{\eps/2}^{d\times k}(\bzero)\big) 
   \le   \vol(\cA_0(\delta +\eps)\cap\Ball_C^{d\times k}(\bzero)),
\end{equation}
where $``+"$ denotes the Minkowski sum of sets.
Therefore by~\eqref{eq:small_ball_prob}, 
%
\begin{align*}
   &\P\left(\exists \bTheta \in \cA_0(\delta) : \bF(\bTheta) = \bzero, \bL^\sT\bL \succ n c,\; \|\bTheta\|_F \le C\right)   \\
  &\le 
  \P\left(
  \exists \bTheta \in \cN_\eps(\cA_0(\delta)) : \|\bF(\bTheta)\|_F \le 2 C_2 \sqrt{kn} \eps, \;
  \bL^\sT \bL \succ c_0 n,\; 
  \|\bTheta\|_F \le C
  \right)\\
  &\le \cN_\eps(\cA_0(\delta) \cap\Ball_C^{d\times k}(\bzero)) (2 C_2 C_0 \eps)^{dk - k - k_0}\\
  &\le (C_3 \eps)^{dk - k- k_0} \left(\frac{1}{C_4 \eps}\right)^{dk}   \vol(\cA_0(\delta +\eps) \cap \Ball_{C}^{d\times k}(\bzero))\\
  &\stackrel{(a)}{\le} (C_3 \eps)^{dk - k- k_0} \left(\frac{1}{C_4 \eps}\right)^{dk}  k (C_5)^{dk}  (\sqrt {k} (\delta + \eps) )^{d -k -k_0 -1}\\
  &\le k C_6^{dk} \frac{(\sqrt{k}(\delta+\eps))^{d-k-k_0-1}}{\eps^{k+k_0}}\, ,
\end{align*}
where in step $(a)$ we used Lemma \ref{lemma:VolumeBound}. 
Now choose $\delta,\eps$ sufficiently small so that the latter quantity converges to $0$ as $n\to\infty.$



\end{proof}


%

\subsection{Proof of Theorem~\ref{thm:simple_critical_point_variational_formula}}
To see that $F$ is convex, define $F:   L^2\times\sfS^k_{\ge}\times \R^{k\times k_0}\to \R$  via
\begin{align}
\cuF(\bu,\bK,\bM):= \E[\ell(\bu + \bK \bz_1 + \bM \bz_0, \bR_{00}^{1/2}\bz_0, w)]  + \lambda(\bK^2 + \bM\bM^\sT)\, .
\end{align}
Since $\cuF$ is jointly convex, the convexity of $F$ will follow if we conclude that the set
\begin{equation}
    \cA := \{(\bK,\bu) \in \sfS^k \times L_2 : \E[\bu\bu^\sT]\preceq \alpha^{-1}\bK^2\}
\end{equation}
is jointly convex. But this is clear once we write
\begin{equation}
\label{eq:constraint_set_A}
    \cA = \bigcap_{\substack{V\in L^2\\\bv \in \R^k}} \cA_0(V,\bv),
    \quad\quad 
    \cA_0(V,\bv) := \{(\bK,\bu) : \bv^\sT\bK \bv  -2\E[V\, \bv^\sT\bu] + \E[V^2] \ge 0\},
\end{equation}
since $\cA_0(V,\bv)$ is convex. This shows the claim in~\textit{1.} of the theorem.

To prove the claim in~\textit{2.}, 
let $\bg = \bK\bz_1 + \bM \bz_0, \bg_0  = \bR_{00}^{1/2} \bz_0$,
and write
\begin{align}
    F(\bK,\bM)
&=
\inf_{\bu \in \cS(\bK)} 
\left\{
\E\left[\ell(\bu+ \bg,\bg_0,w) 
\right]
+
\frac{\lambda}{2}\bR_{11}
\right\}\\
&=
\inf_{\bu \in L^2} 
\sup_{\bQ\succ \bzero} 
\left\{
\E\left[\ell(\bu+ \bg,\bg_0,w) 
\right]
+
\frac12\Tr\left(
(\E[
\bu\bu^\sT]
- \alpha^{-1} \bK^2
)\bQ\right)+
\frac{\lambda}{2}\bR_{11}
\right\}\\
&= 
\sup_{\bQ\succ \bzero} 
\inf_{\bu \in L^2} 
\left\{
\E\left[\ell(\bu+ \bg,\bg_0,w) 
\right]
+
\frac12\Tr\left(
(\E[
\bu\bu^\sT]
- \alpha^{-1} \bK^2
)\bQ\right)+
\frac{\lambda}{2}\bR_{11}
\right\}\\
&= 
\sup_{\bS\succ \bzero} 
\inf_{\bx \in L^2} 
\left\{
\E\left[\ell(\bx,\bg_0,w) 
+\frac12 (\bg - \bx)^\sT\bS^{-1}(\bg-\bx)
\right]
-\frac1{2\alpha} \Tr\left(\bS^{-1}\bK^2\right) +
\frac{\lambda}{2}\bR_{11}
\right\}\\
&=\sup_{\bS\succ \bzero}  G(\bK,\bM,\bS).
\end{align}
where $G(\bK,\bM, \bS)$ is the objective in Eq.~\eqref{eq:min_max_critical_pts} after the reparametrization 
\begin{equation}
(\bR/\bR_{00},\bR_{10}\bR_{00}^{-1/2}) = (\bK^2,\bM), 
\end{equation}
i.e.,
\begin{equation}
    G(\bK,\bM,\bS) :=
       \E\left[\More_{\ell(\cdot, \bg_0,w)}(\bg;\bS)\right] - \frac1{2\alpha}\Tr(\bS^{-1}\bK^2) 
       + \frac{\lambda}{2}\Tr(\bK^2 + \bM\bM^\sT)
\end{equation}
%
with the Moreau envelope defined in Eq.~\eqref{eq:moreau_def}.
%We'll first show that 
%\begin{equation}
%\label{eq:F_and_G_relation}
%    \sup_{\bS\succ\bzero} G(\bK,\bM,\bS) =  F(\bK,\bM).
%\end{equation}
%By optimizing over $\bx\in L^2$, we can write
%%where
%%\begin{equation}
%%    \cA := \{\bu \in L^2 : \alpha\E[\bu\bu^\sT]\preceq \bSigma\}.
%%\end{equation}
%This proves Eq.~\eqref{eq:F_and_G_relation}.
%, for $\bSigma\in\sfS^k$ define
%\begin{equation}
%    \cA(\bSigma) :=
%    \left\{(\bu, \bB)  \in L_2 \times  \sfS^k : 
%    \E[\bu\bu^\sT]  \preceq \bB,\;
%    \bB \preceq \alpha^{-1} \bSigma
%    \right\}\,
%\end{equation}
 %note that we can write
%\begin{equation}
    %F(\bSigma,\bM) = \inf_{(\bu,\bB)\in\bS(\bSigma)}
   %\E[\ell(\bu + \bSigma \bz_1 + \bM \bz_0, \bR_{00}^{1/2}\bz_0, w)]  + \lambda(\bSigma^2 + \bM\bM^\sT).
%\end{equation}
Now, by straightforward differentiation of $G(\bK,\bM,\bS)$ with respect to each of $\bK,\bM,\bS$, one can show that the critical points of $G(\bK,\bM,\bS)$ are given by $(\bK,\bM, \bS) = (\bR^\opt/\bR_{00},\bR_{10}^\opt \bR_{00}^{-1},\bS^\opt)$ by checking that the stationarity conditions corresponds to Eq.~\eqref{eq:opt_fp_eqs}.
Furthermore, by definition, $\bS^\opt(\bR)$, the solution of~\eqref{eq:opt_fp_eqs} is unique for each $\bR$ (as the limit of the Stieltjes Transform $\bS_\star$ or as the minimizer of a strongly convex program as seen in the proof of Theorem~\ref{thm:global_min}).
Then by differentiation of $\bG$ with respect to $\bS$, one can show that $G(\bK,\bM,\bS)$ is locally concave at $\bS = \bS^\opt$ for fixed $\bR.$ This shows that indeed $\bS^\opt(\bR)$ is the maximizer of $G(\bK,\bM,\bS).$
%To prove the characterization in Eq.~\eqref{eq:min_max_critical_pts}, what remains is to show that the critical points $\bR^\opt$ of $\sup_{\bS \succ\bzero}G(\bR,\bS)$ are local minima. 
Combined with the convexity of $F(\bK,\bM)$ proves the claim.

Finally, the claim in~\textit{3.} follows from strict convexity of  
$(\bK,\bM,\bu) \mapsto \E\left[\ell(\bu+ \bg,\bg_0,w)
\right]
+
\frac{\lambda}{2}\bR_{11}$ 
under condition~\textit{(a)} and~\textit{(b)}, and the convexity of the constraint set $\cA$ of Eq.~\eqref{eq:constraint_set_A}.




\section{Proofs for applications}
\subsection{Analysis of multinomial regression: Proof of Proposition~\ref{prop:multinomial}}
   Consider the regularized multinomial regression problem with empirical risk
   \begin{equation}
       \hat R_{n,\lambda}(\bTheta) := \frac1n\sum_{i=1}^n \left\{ \log\left(1 + \sum_{j=1}^k e^{\bx_i^\sT\btheta_j}\right) - \by_i^\sT\bTheta^{\sT}\bx_i \right\}  + \frac{\lambda}{2} \|\bTheta\|_F^2
   \end{equation}
   for $\lambda \ge 0$. We will prove Proposition~\ref{prop:multinomial}
   for $\lambda =0$, but it will be convenient to study the above problem for $\lambda$ near $0$ for additional regularity.

For future reference, we  introduce the notation
%
\begin{align}
\ell_{i}(\bv):=
   \log\Big( \sum_{j=1}^k e^{v_j} + 1\Big)  
   -  \by_{i}^\sT \bv\, .\label{eq:ell_i}
\end{align}
%
We will use the following lemma that is 
an adaptation of Lemmas S6.1 and S6.4 of~\cite{tan2024multinomial}.
Although the parametrization in these lemmas is slightly different,
the proof required here can be derived from the proofs of those lemmas. For the convenience of the reader, we provide it here.
\begin{lemma}\label{lemma:LocalStrongMultinomial}
Under the assumptions of Proposition~\ref{prop:multinomial}, for any constant
$\rho>0$, there exists $c = c(\rho;\bR_{00},\alpha,\rho)>0$ such that
%
\begin{align}
&   \lim_{n\to\infty}\P\Big(\frac1n\sum_{i=1}^n \grad\ell_{i}(\bTheta^\sT\bx_i) 
    \grad\ell_{i}(\bTheta^\sT\bx_i)^\sT \succeq c\bI_k
    \;\;\forall \bTheta:\|\bTheta\|_F^2\le \rho\Big) = 1\,,
    \label{eq:outer_prod_singular_value_lb}\\
&\lim_{n\to\infty}\P\Big(\nabla^2 \hR_{n,0}(\bTheta)\succeq c\bI_{dk}\;\;\forall \bTheta:\|\bTheta\|_F^2\le \rho\Big) = 1\,.
\label{eq:hessian_singular_value_lb_multinomial}
\end{align}
\end{lemma}
\begin{proof}[Proof (Due to~\cite{tan2024multinomial}).]

Define the event
\begin{equation}
\Omega_{1,n} := \left\{ \sigma_{\max}(\bX) \le C_0 \sqrt{n},\quad \sigma_{\min}(\bX)\ge c_0\sqrt{n}\right\}
\end{equation}
for $C_0,c_0$ chosen so that $\Omega_{1,n}$ are high probability sets.

\noindent\textbf{Proof of Eq.~\eqref{eq:hessian_singular_value_lb_multinomial}.}
For a subset $\cI\subseteq [n]$, let $\bX_\cI = (\bx_i^\sT)_{i\in\cI} \in\R^{|\cI| \times d}$.
Now since $n/d_n \rightarrow \alpha$, we can choose a $\beta \in(\alpha^{-1},1)$ so that for some constant $c_0>0$ independent of $n$, we have $\lim_{n\to\infty}\P(\Omega_{3,n}) =1$ for the event
\begin{equation}
  \Omega_{1,n} := \left\{\sigma_{\min}(\bX_{\cI}) > c_0\sqrt{n} \;\; \textrm{for all}\;\; \cI \subseteq [n] \;\;\textrm{with}\;\; |\cI| \ge \beta n \right\}.
\end{equation}
(See for instance Lemma S6.3 of~\cite{tan2024multinomial}. The argument is a standard union bound using lower bound on the singular values of Gaussian matrices.)
For any $D  > 1$, $\|\bTheta\|_F^2 \le \rho$, we have on the event $\Omega_{1,n}$
\begin{equation}
   D \cdot|\{i\in[n] : \|\bTheta^\sT\bx_i\|_F^2 \ge D\}| \le  \sum_{i=1}^n \|\bTheta^\sT\bx_i\|_F^2 \le \|\bX\|_\op^2 \|\bTheta\|_F^2 \le C_0^2 \rho n =: \rho_0 n,
\end{equation}
so that, choosing $D =D_\beta := \rho_0 (1-\beta)^{-1}$,  we have for
\begin{equation}
      |\cI_\beta| \ge \beta n  \quad\textrm{for}\quad \cI_\beta := \{i \in[n] : \|\bTheta^\sT\bx_i\|_F^2 \le D_\beta \}\, .
\end{equation}
Furthermore, we have for each $i\in\cI_\beta$,
\begin{equation}
\label{eq:entry_wise_bound_p_hessian}
    \bp(\bTheta^\sT\bx_i) \in  [c_1,1-c_1]^k
\end{equation}
for some $c_1\in(0,1)$ independent of $n$.

Now we can compute for $\bv\in\R^k$ (note that, with the definition \eqref{eq:ell_i}, $\grad^2 \ell_i(\bv)$ is independent of $\by_i$, so we omit the subscript $i$):
\begin{equation}
    \grad^2 \ell(\bv) = \Diag(\bp(\bv)) - \bp(\bv)\bp(\bv)^\sT,
\end{equation}
Hence, for any $\bu$, letting $\bu^2 = \bu\odot \bu$,
\begin{equation}
  \bu^\sT \grad^2 \ell(\bv)\bu   = \bu^2 \odot \bp(\bv) - (\bu^\sT\bp(\bv))^2
  \stackrel{(a)}{\ge}
 \bu^2 \odot \bp(\bv) - (\bu^2 \odot \bp(\bv))\|\bp(\bv)\|_1  \stackrel{(b)}{=} 
 (\bu^2 \odot \bp(\bv)) (p_0(\bv)),
\end{equation}
where $(a)$ is an application of Cauchy Schwarz, and $(b)$ follows from the identity  $\sum_{j\in[k]}p_j(\bv) + p_0(\bv) = 1.$
This, along with Eq.~\eqref{eq:entry_wise_bound_p_hessian} then implies
that for $i\in \cI_\beta$, 
\begin{equation}
    \lambda_{\min}(\grad^2 \ell(\bTheta^\sT\bx_i))  > c_3
\end{equation}
for some $c_3$ independent of $n$.
Now noting that
\begin{equation}
\grad^2 \hat R_{n,0}(\bTheta) = \frac1n\sum_{i=1}^n \grad^2\ell(\bTheta^\sT\bx_i) \otimes (\bx_i\bx_i^\sT),
\end{equation}
we can bound 
on the high probability event $\Omega_{1,n}\cap\Omega_{2,n}$,
\begin{align}
    \lambda_{\min}(n\grad^2 \hat R_{n,0}(\bTheta)) 
    \ge
    \lambda_{\min}\left(\sum_{i\in\cI_\beta} \grad^2\ell(\bTheta^\sT\bx_i) \otimes (\bx_i\bx_i^\sT)\right)
    \ge 
    c_3 \lambda_{\min}\left(\sum_{i\in\cI_\beta} \bI_k \otimes (\bx_i\bx_i^\sT)\right)
   % &\ge c_3 \lambda_{\min}(\bX_{\cI_\star}^\sT\bX_{\cI_\star})\\
    \ge c_3 c_0^2 n,
\end{align}
showing the claim.


\noindent
\textbf{Proof of Eq.~\eqref{eq:outer_prod_singular_value_lb}.}
By the upper bound on $\|\bTheta_0\|_F$, we can find some $\gamma\in(0,1)$ constant in $n$ so that the event
\begin{align}
\Omega_{3,n} &:= \left\{\sum_{i=1}^n \one_{\{\by_i = \be_{j}\}} \ge \gamma n  \quad\textrm{for all}\quad j \in\{0,\dots,k\} \right\}
\end{align} 
holds with high probability.
Without loss of generality, for what follows, assume that $\gamma n$ is an integer.

We next construct $\gamma n$ subsets of $[n]$ so that, in each subset, there is exactly one index corresponding to a label of each class.  Namely, find disjoint index sets $\{\cI_l\}_{l\in [\gamma n]}$, $\cI_l\subseteq [n]$, so that for each $l\in [\gamma n]$, we have
\begin{equation}
|\cI_l| = k+1, \quad\quad\textrm{and,  for each}\; j\in\{0,\dots,k\},\;\sum_{i\in \cI_l} \one_{\by_i = \be_j} =1.
\end{equation}
%
We claim that,  on the event $\Omega_{1,n}$, for any $\bTheta$ with  $\|\bTheta\|_F^2\le \rho$,
there must exist a $\cL \subseteq [\gamma n]$ such that
\begin{equation}
\label{eq:cL_index_set_bound}
    |\cL| \ge \frac{\gamma n}{2} \quad\textrm{s.t. for all $l\in\cL$}.\quad \sum_{i \in\cI_l} \|\bTheta^\sT\bx_i\|_F^2 \le \frac{ 2\rho_0}{\gamma}\, .
\end{equation}
Indeed, this follows from
\begin{equation}
|\cL^c| \min_{l\in\cL^c} \sum_{i\in\cI_l} \|\bTheta^\sT\bx_i\|_F^2 \le 
\sum_{l\in[\gamma n]} \sum_{i\in\cI_k} \|\bTheta^\sT\bx_i\|_F^2  \le 
\|\bX\|_\op^2 \|\bTheta\|_F^2
\le C_0^2 \rho n.
\end{equation}
%\begin{equation}
%\sum_{l\in\cL} \sum_{i\in\cI_l}\|\bTheta^\sT\bx_i\|_F^2 \le  \sum_{l \in [\gamma n]}  \sum_{i\in\cI_l} \|\bTheta^\sT \bx_i\|_F^2 \le  \|\bX\|_\op^2 \|\bTheta\|_F^2 \le   C_0^2 \rho n =:  \rho_0 n.
%\end{equation}
%%Letting $\cI_\star = \cup_{l\in\cL} \cI_l$ be the union of these indices, we have for each $i\in\cI_\star$ 
%
Therefore, by definition of $\bp$, we have for each $i\in \cup_{l\in\cL} \cI_l$,
\begin{equation}
\label{eq:entry_wise_bounds_p}
    \bp(\bTheta^\sT\bx_i)\in [c_4,1-c_4]^k
\end{equation}
for some $c_4 \in (0,1)$  independent of $n$.

By straightforward computation, we have for each $l\in[\gamma n]$
\begin{equation}
    \sum_{i \in\cI_l} \grad \ell_i(\bTheta^\sT\bx_i) \grad \ell_i(\bTheta^\sT\bx_i)^\sT = \bS^\sT\bB^\sT(\bI_{k+1} - \bP_l)^\sT(\bI_{k+1} - \bP_l)\bB \bS
\end{equation}
where $\bS \in\R^{k\times k}$ is a permutation matrix,
\begin{equation}
    \bP_l = (p_j(\bTheta^\sT\bx_i))_{i\in [\cI_l], j\in\{0,\dots,k\}} \quad\textrm{and}\quad
    \bB = \begin{bmatrix} \bI_k \\
    \bzero^\sT
    \end{bmatrix} \in\R^{(k+1) \times k}.
\end{equation}
The bound in Eq.~\eqref{eq:entry_wise_bounds_p} then implies 
that for each $l\in\cL$, $\bP_l\in\R^{(k+1)\times (k+1)}$, which is a stochastic matrix, has entries that are in $[c_5,1-c_5]$ for some $c_5>0$. Hence, it is a stochastic matrix of an irreducible and aperiodic Markov chain, so that $(\bI_{k+1} - \bP_l)\bu = \bzero$ if and only if $\bu =a\one_{k+1}$ for $a\in\R$. 
Letting $\cP$ be the space of these stochastic matrices whose entries are in $[c_5,1-c_5]$, we have by compactness of this space that 
for some $\bP_\star \in \cP$, 
\begin{align}
    \min_{l\in\cL}\min_{\substack{\bv \in\R^{k}
    \\
    \|\bv\|_2 = 1
    }}\bv^\sT\bB^\sT(\bI_{k+1} - \bP_l)^\sT(\bI_{k+1} - \bP_l)\bB\bv
    &\ge  
    \min_{\substack{\bu\in\R^{k+1}\\ \|\bu\|_2 =1\\
    u_{k+1} = 0
    }}\bu^\sT(\bI_{k+1} -\bP_\star)^\sT(\bI_{k+1} -\bP_\star )\bu \ge c_5
\end{align}
for some $c_6>0$ independent of $n$ (but dependent on $c_1$).
So we can bound
\begin{align}
    \lambda_{\min}\left( \frac1n\sum_{i=1}^n \grad \ell_i(\bTheta^\sT\bx_i)
    \grad \ell_i(\bTheta^\sT\bx_i)^\sT
    \right) 
    \ge 
    \lambda_{\min}\left(\frac1n \sum_{l\in |\cL|} 
    \bB^\sT (\bI_{k+1} - \bP_l)^\sT (\bI_{k+1} - \bP_l) \bB
    \right)
    \ge \frac{|\cL|}n  c_6.
\end{align}
The bound on $|\cL|$ now gives the desired result of Eq.~\eqref{eq:outer_prod_singular_value_lb}.
\end{proof}


%\begin{proof}
%We'll use $\be_{p,j}$ in what follows to denote the $j$th canonical basis vector in $\R^{p}$.
%Define the set
%\begin{equation}
%A_y := \{ \bY \in\R^{n \times k}: \sum_{i=1}^n \one_{\{\by_i = \be_{k,j}\}} \ge \gamma n  \quad\textrm{for all}\quad j \in\{0,\dots,k\} \}.
%\end{equation}
%
%Fix any $\bQ \in\R^{(k+1)\times k}$  satisfying
%\begin{equation}
%    \bQ^\sT\bQ = \bI_k ,\quad\quad \bQ\bQ^\sT = \bI - \frac{1}{k+1}\one\one^\sT.
%\end{equation}
%For $i\in[n]$, define $\tilde\by_i\in\R^{k+1}$ by 
%\begin{equation}
%   \tilde \by_{i}  := \begin{cases}
%   \be_{k+1,j} & \by_i =\be_{k,j},\quad j\in[k]\\
%   \be_{k+1,k+1} & \by_i = \be_0,
%   \end{cases}
%\end{equation}
%i.e., $\tilde\by_i$ is a standard one-hot encoding.
%Define for $\bv \in\R^{k}$,
%\begin{equation}
%    l_i(\bv) :=  \log\left(\sum_{j=1}^{k+1} \exp\{\be_{j}^\sT\bQ \bv\}\right) - \tilde \by_i^\sT \bQ \bv.
%\end{equation}
%
%Lemma S6.1 of~\cite{tan2024multinomial} proves that for all $\by\in A_y$, and any $\bM$ satisfying $\|\bM \bQ^\sT\|_F^2 \le n c_0$, there exsits some constant $c$ depending on $\gamma$ and $c_0$ so that
%\begin{equation}
%   \frac1n \sum_{i=1}^n \bQ \grad l_i(\bM^\sT\be_i) \grad l_i(\bM^\sT\be_i)^\sT  \succeq
%    c(\gamma,c_0) \bI_k.
%\end{equation}
%\end{proof}





\begin{lemma}[Bounded empirical risk minimizer for multinomial regression]
\label{lemma:equivalence_multinomial}
Under the assumptions of Proposition~\ref{prop:multinomial}, 
%Let $\hat R_{n,\lambda}(\bTheta)$ be the risk for multinomial regression with regularization parameter $\lambda \ge 0$.
the following are equivalent:
\begin{enumerate}
 \item There exists $C>0$ independent of $n$  such that, for all $\lambda>0$ 
    \begin{equation}
     \lim_{n\to\infty} \P\left(\|\hat\bTheta_\lambda\|_F < C \right) = 1.
    \end{equation}
    \item There exists $C>0$ independent of $n$ and $\lambda$, such that, 
    \begin{equation}
        \lim_{\lambda \to 0+} \lim_{n\to\infty} \P\left(\|\hat\bTheta_\lambda\|_F < C \right) = 1.
    \end{equation}
\item There exists $C>0$ such that
    \begin{equation}
         \lim_{n\to\infty} \P\left( \hat\bTheta \;\mbox{\rm exists}\;,\|\hbTheta\|_F 
 < C\right) = 1.
    \end{equation}
\end{enumerate}
%
If any of the above holds, then, for any $\delta>0$, we have
 \begin{equation}
        \lim_{\lambda \to 0+} \lim_{n\to\infty} \P\left(\|\hbTheta-\hbTheta_\lambda\|_F < \delta \right) = 1.\label{eq:LambdaPerturbation}
    \end{equation}
On the other hand, if for all $C>0$, we have
    \begin{equation}
    \label{eq:diverging_in_lambda}
        \lim_{\lambda \to 0} \lim_{n\to\infty} \P\left(\|\hat\bTheta_\lambda\|_F > C\right) = 1,
    \end{equation}
    then for all $C>0,$
    \begin{equation}
    \label{eq:any_seq_minimzers_diverges}
         \lim_{n\to\infty} \P\left( \hat\bTheta \;\mbox{\rm exists}\;,\|\hbTheta\|_F 
 < C\right) = 0.
    \end{equation}
\end{lemma}
\begin{proof}
We will show $\textit{1}\Rightarrow \textit{2}\Rightarrow \textit{3} \Rightarrow \textit{1}$. 

\vspace{0.15cm}

\noindent $\textit{1}\Rightarrow \textit{2}$. This is obvious.

\vspace{0.15cm}

\noindent $\textit{3}\Rightarrow \textit{1}$. Define $r_n(\lambda) := \inf_{\bTheta}\hR_{n,\lambda}(\bTheta)$.
This is a non-decreasing non-negative concave function of $\lambda$. By the envelope theorem, for any
$0\le \lambda_1<\lambda_2$, we have 
%
\begin{align}
\frac{1}{2}\|\hbTheta_{\lambda_1}\|^2_F \ge \frac{r(\lambda_2)-r(\lambda_1)}{\lambda_2-\lambda_1} \ge \frac{1}{2}\|\hbTheta_{\lambda_2}\|^2_F\, ,
\end{align}
%
where, for $\lambda=0$, $\|\hbTheta_{0}\|_F$ is the norm of any minimizer when this exists.
It follows in particular that $\|\hbTheta_{\lambda}\|_F\le \|\hbTheta\|_F$ for any $\lambda>0$, and therefore the claim follows. 

\vspace{0.15cm}

\noindent $\textit{2}\Rightarrow \textit{3}$. Fix $C$ as in point \textit{2}, $\delta_0>0$ and we chose $\lambda_0>0$
such that
$\lim_{n\to\infty}\P(\|\hbTheta_{\lambda}\|_F<C)\ge 1-\delta_0$ for all $\lambda\in(0,\lambda_0)$. Let $c_0 = c_0(2C)>0$
be given as per Lemma \ref{lemma:LocalStrongMultinomial}. Hence, with probability $1-\delta_0-o_n(1)$,
%
\begin{align}
\|\bTheta\|_F\le 2C \;\;\Rightarrow\;\;
\hR_{n,0}(\bTheta)&\ge \hR_{n,0}(\hbTheta_{\lambda}) +\< \hR_{n,0}(\hbTheta_{\lambda}),\bTheta-\hbTheta_{\lambda}\>
+\frac{c_0}{2}\|\bTheta-\hbTheta_{\lambda}\|_F^2\\
&\ge \hR_{n,0}(\hbTheta_{\lambda}) -\lambda\< \hbTheta_{\lambda},\bTheta-\hbTheta_{\lambda}\>
+\frac{c_0}{2}\|\bTheta-\hbTheta_{\lambda}\|_F^2\, .
\end{align}
%
Recalling that $\|\hbTheta_{\lambda}\|_F<C$, this implies
%
\begin{align}
\frac{2\lambda C}{c_0}<C \;\;\Rightarrow \;\; \|\bTheta-\hbTheta_{\lambda}\|_F\le \frac{2\lambda}{c_0}\|\hbTheta_{\lambda}\|_F\le \frac{2\lambda C}{c_0}\, .
\end{align}
%
The first condition can be satisfied by eventually decreasing $\lambda_0$. We thus conclude that, for each 
$\delta_0>0$, there exists $\lambda_0>0$ such that 
%
\begin{align}
\lim_{n\to\infty}\P\Big(\|\hbTheta\|_F\le 2C, \|\hbTheta_{\lambda}-\hbTheta\|_F\le \frac{2\lambda C}{c_0}\Big)\ge 1-\delta_0\, .
\end{align}
%
The claim \textit{3} follows by dropping the second inequality
 $\|\hbTheta_{\lambda}-\hbTheta\|_F\le 2\lambda C/c_0$ and noting that $\delta_0$ is arbitrary.

Equation \eqref{eq:LambdaPerturbation} follows by dropping the 
inequality $\|\hbTheta\|_F\le 2C$ in the last display.
 
\vspace{0.15cm}

Finally Eq.~\eqref{eq:diverging_in_lambda} implies Eq.~\eqref{eq:any_seq_minimzers_diverges} ~ by
the monotonicity of $\|\hbTheta_{\lambda}\|_F$ in $\lambda$ proven above.
\end{proof}
%
%We will show $\textit{1}\Rightarrow \textit{2}\Rightarrow \textit{3} \Rightarrow \textit{1}$. 
%
%\am{Modified the proof of first implication}
%
%Let $\Omega_{n,1}(\lambda,C)$ be the event $\{\|\hat\bTheta_\lambda\|_F < C\}.$ 
%The stationarity conditions of $\bTheta \mapsto \hat R_{n,\lambda}(\bTheta) $ at $\hat\bTheta_\lambda$
%are given by $\grad \hat R_{n,0}(\hat\bTheta_\lambda) = - \lambda \hat\bTheta_\lambda,$
%so on $\Omega_{n,1}$,
%%$\|\grad \hat R_{n,0}(\hat\bTheta_\lambda)\|_2 \le \lambda C$,
%so we have for any $\eps >0$ and any  $C>0$,
%\begin{equation}
%\|\grad \hat R_{n,0}(\hat\bTheta_\lambda)\|_2 \le \lambda C\, .
%\end{equation}

%Define the event
%
%\begin{align}
%
%\Omega_{n,0}(C):= \Big\{\grad^2\hat R_{n,0}(\bTheta) \succeq c_0\bI,\;\;\;
%\forall\bTheta:\; \|\bTheta\|_F\le C \Big\}\, .
%
%\end{align}
%
%By an argument similar to Lemma S6.2 of~\cite{tan2024multinomial}, for
%any $C>0$ there exists some $c_0>0$ independent of $n$
%such that $\P(\Omega_{n,0})\to 1$ as $n\to\infty$.

%Letting $C$ be such that $\lim_{\lambda\to 0}\lim_{n\to\infty}\P(\Omega_{n,1}(\lambda,C))=1$. On  $\Omega_{n,0}%(2C)\cap\Omega_{n,1}(\lambda,C)$. So on the intersection of these events, for any $\|\tilde\bTheta\|_F\le 2C$
%\begin{equation} 
%   \hat R_{n,0}(\tilde\bTheta) \ge
%   \hat R_{n,0}(\hat\bTheta_\lambda)  + 
%   \langle \grad \hat R_{n,0}(\hat\bTheta_{\lambda}),(\tilde\bTheta - \hat\bTheta_\lambda) \rangle  + \frac{c_0}{4} %\|\tilde\bTheta - \hat\bTheta_\lambda\|_F\cdot\Big(\|\tilde\bTheta - \hat\bTheta_\lambda\|_F\wedge C\Big),
%\end{equation}
%so that minimizing over $\tilde\bTheta$ on both sides gives
%\begin{align*}
%\|\grad \hat R_{n,0}(\hat\bTheta_\lambda)\|_F\le \frac{c_0C}{2} \;\;\Rightarrow\;\;
%   \min_{\tilde\bTheta}\hat R_{n,0}(\tilde\bTheta) &\ge 
%  \hat R_{n,0}(\hat\bTheta_\lambda)   - \frac{1}{c_0}\|\grad \hat R_{n,0}(\hat\bTheta_\lambda)\|_F^2\\
%  &\ge \hat R_{n,0}(\hat\bTheta_\lambda)-\frac{C^2}{c_0}\lambda^2.
%\end{align*}
%
%Since $\hat R_{n,0}(\hat\bTheta_\lambda) >c $ for some $c$ on $\Omega_{n,1}(\lambda,C)$, \am{Why???} taking  $n\to\infty$ followed by $\lambda \to0$ gives the claim of $\textit{2},$
%proving $\textit{1}\Rightarrow\textit{2}$.
%
%
%To show the implication $\textit{2}\Rightarrow \textit{3},$ note that when $\hat\bTheta$ diverges, $\hat R_{n,0}%(\hat\bTheta) \to 0$ giving the contrapositive of the desired statement. \am{The opposite of the conclusion is not that 
%$\hbTheta$ `diverges'.}
%
%Finally, to prove $\textit{3}\Rightarrow \textit{1}$, extend the above definition by letting
%$\Omega_{n,1}(0,C):=\{\hbTheta \mbox{ exists } \|\hbTheta\|_F<C\}$. 
%On $\Omega_{n,0}(2C)\cap \Omega_{n,1}(0,C)$, we have
%\begin{equation}
%\hat R_{n,0}(\hat\bTheta_{\lambda})  \ge  \hat R_{n,0}(\hat\bTheta)
%+\frac{c_0}{4} \|\tilde\bTheta - \hat\bTheta_\lambda\|_F\cdot\Big(\|\tilde\bTheta - \hat\bTheta_\lambda\|_F\wedge C\Big)
%\, .
%\end{equation}
%On the other hand
%\begin{align}
%\hat R_{n,0}(\hat\bTheta_{\lambda})&\le \hat R_{n,0}(\hat\bTheta) +
%\frac{\lambda}{2}\Big(\|\hbTheta\|_F^2-\|\hbTheta_{\lambda}\|_F^2\Big)\\
%&\le \hat R_{n,0}(\hat\bTheta) + \frac{\lambda C^2}{2}\, .
%\end{align}
%
%Using together the last two displays, we get 
%\begin{equation}
%\frac{c_0}{4} \|\tilde\bTheta - \hat\bTheta_\lambda\|_F\cdot\Big(\|\tilde\bTheta - \hat\bTheta_\lambda\|_F\wedge C\Big)\le \frac{\lambda C^2}{2}
%\end{equation}
%
%Once again by Lemma S6.2 of~\cite{tan2024multinomial}, there exists a high probability event $\Omega_{n,4}$ so that $\grad^2 \hat R_{n,0}(\hat\bTheta) \succ c_1 \bI$ for $c_1$ independent of $n$ (and $\lambda)$ on $\Omega_{3,n}\cap\Omega_{4,n}$, and so on the intersection of these events along with the intersection of the high probability event $\Omega_{5,n}$ where $\bTheta \mapsto \hat R_{n,0}(\bTheta)$ is Lipschitz with constant $C_1 >0$ independent of $n$ (and $\lambda$), we have
%\begin{equation}
%    \|\hat\bTheta_{\lambda} - \hat\bTheta\|_F^2 \le \frac{2}{c_1} \left(\hat R_{n,0}(\hat\bTheta_{\lambda})  - \hat R_{n,0}(\hat\bTheta)\right) \le
%     \frac{2 C_1}{c_1}\|\hat\bTheta_\lambda - \hat\bTheta\|_F.
%\end{equation}
%So if $\hat\bTheta$ is bounded by a constant independent of $n$ and $\lambda$, then so is $\hat\bTheta_\lambda$, proving the final implication.

%Now, by inspection of the argument for the implication \textit{3.}$\Rightarrow$\textit{1.}, we see that if instead of assuming that $\Omega_{3,n}$ is a high probability event, we assume that $\P(\Omega_{3,n}) > 0$, then we conclude that there exists $C>0$ so that  $\lim_{\lambda \to 0}\lim_{n\to\infty} \P(\|\hat\bTheta_{\lambda}\|_F < C) > 0$. 
%Taking the contrapositive of this statement proves that Eq.~\eqref{eq:diverging_in_lambda} implies ~Eq.~\eqref{eq:any_seq_minimzers_diverges}.

%\end{proof}

\begin{lemma}
\label{lemma:multinomial_regularized}
Consider the setting of Proposition~\ref{prop:multinomial}.
For any $\lambda >0$, the equations 
\begin{align}
\label{eq:multinomial_FP_regularized}
   \alpha \E\left[(\bp(\bv) - \by)(\bp(\bv) - \by)^\sT\right]  &= \bS^{-1} (\bR/\bR_{00}) \bS^{-1}\\
  \E[(\bp(\bv) - \by) (\bv^\sT,\bg_0^\sT)]  &= \lambda (\bR_{11},\bR_{10})
\end{align}
have a unique solution $(\bR_{\opt}(\lambda),\bS_{\opt}(\lambda))$.

Furthermore, letting $\hat\bTheta_\lambda$ be the unique minimizer of $\hat R_{n,\lambda}$, and $(\mu^\opt(\lambda),\nu^\opt(\lambda))$ by defined in terms of $\bR^\opt(\lambda),\bS^\opt(\lambda)$ via Definition~\ref{def:opt_FP_conds},
then points $\textit{1.-3.}$ of Theorem~\ref{thm:global_min} hold.
\end{lemma}
\begin{proof}
Uniqueness of the solution follows from Theorem \ref{prop:simple_critical_point_variational_formula}
(point \textit{3.(a)} holds because $\lambda>0$).

In order to prove that the conclusions of Theorem~\ref{thm:global_min} holds, we
will apply that theorem to a modification of the problem under consideration.
Namely, we will perform a smoothing of the labels $\by_i$, and check that the assumption holds.

For $\eps>0$, let $\by_{i,\eps}\in\R^k$ be a smoothing of $\by_i$ so that it has entries in $[0,1]$, 
$\by_{i,\eps} := \boldf_{\eps}(\bTheta_0^{\sT}\bx_i, w_i)$ for some $\boldf_\eps$ that 
is a $C^2$, Lipschitz function of $\bTheta_0^{\sT}\bx_i$ for all fixed $\eps>0$, where $w_i$ is a uniform random variables; and satisfies 
$\P\left(\by_i \neq \by_{i,\eps}| \bTheta_0^\sT\bx_i\right) \le \eps$ (such a smoothing can be constructed, for instance, by writing $\by_i$ as a function of a uniformly distributed random variable and indicators, then smoothing the indicator appropriately).
Define the smoothed regularized MLE, for $\lambda>0$,
\begin{equation}
     \hat\bTheta_{n,\eps,\lambda} := 
     \argmin_{\bTheta\in\R^{d\times k}}  \hat R_{n,\eps,\lambda}(\bTheta),
    \quad
\hat R_{n,\eps,\lambda}(\bTheta) :=
     \frac1n \sum_{i=1}^n
    \ell_{i,\eps}(\bTheta^\sT\bx_i) + \frac{\lambda}{2} \|\bTheta\|_F^2,
\end{equation}
where
\begin{equation}
    \ell_{i,\eps}(\bTheta^\sT\bx_i):=
   \log\Big( \sum_{j=1}^k e^{\bx_i^\sT\btheta_j} + 1\Big)  
   -  \by_{i,\eps}^\sT \bTheta \bx_i
  \, .
\end{equation}
%
Note that $\ell_{i,\eps}(\bTheta^\sT\bx_i) = 
L(\bTheta^\sT\bx_i,\boldf_{\eps}(\bTheta_0^\sT\bx_i,w_i))$  depends on $\bTheta_0^\sT\bx_i,w_i$ through the labeling function $\boldf_{\eps}$. 
To avoid clutter, we use the notation $\ell_{i,\eps}(\bTheta^\sT\bx_i)$ instead of $\ell(\bTheta^\sT\bx_i,\bTheta_0^\sT\bx_i,w_i)$ which is used in other sections.


We check the conditions of Theorem~\ref{thm:global_min} for this setting.
Assumptions~\ref{ass:regime} and~\ref{ass:theta_0} are given.
It's easy to check that Assumptions~\ref{ass:loss},~\ref{ass:Data},~\ref{ass:convexity} and~\ref{ass:noise} hold.

We next show that the minimizer $\hat\bTheta_{n,\eps,\lambda}$ is, with high probability, in the set of critical points $\cE(\bTheta_0,\bw)$ defined in Theorem~\ref{thm:global_min} 
for which our theory applies.
For this, we will need to show that $\sigma_{\max}(\hat\bTheta_{\eps,\lambda}, \bTheta_0) \le C,$  for some $C$ independent of $n$,
and that the Hessian  along with the gradient outer product are 
lower bounded
at $\hat\bTheta_{\eps,\lambda}$ by some $c>0$ independent of $n$.
In what follows, let 
\begin{equation}
    \Omega_{1,n} := \{   C_0\sqrt{n} \ge \sigma_{\max}(\bX) \ge \sigma_{\min}(\bX)\ge \sqrt{n} c_0 \}
\end{equation}
for some $C_0,c_0 >0$ chosen so that $\Omega_{1,n}$ is a high probability event.

\paragraph{Upper bound on $\bR(\hmu_{\sqrt{d}[\hat\bTheta_{\eps,\lambda},\bTheta_0]})$.}
For any $\lambda >0$, it is easy to see $\|\hat\bTheta_{\eps,\lambda}\|_F \le C_0/\lambda$ for some $C_0$ independent of $n,\eps$, since the multinomial loss is lower bounded by zero.
This along with the assumption that $\bR_{00}=\lim_{n\to\infty}\bTheta_0^{\sT}\bTheta_0$
is finite  implies  $
\bR(\hmu_{\sqrt{d}[\hat\bTheta_{\eps,\lambda},\bTheta_0]})\prec C\bI$ for all fixed $\lambda >0$.

\paragraph{Lower bound on the Hessian $\grad^2_{\bTheta}\hat R_{n,\eps,\lambda}(\bTheta)$.} Clearly, since 
$\bTheta\mapsto \ell_{i,\eps}(\bTheta)$ is convex, we have for any $\lambda >0$, $\grad^2_{\bTheta}\hat R_{n,\eps,\lambda}(\bTheta) \succeq \lambda/2\bI.$ 

\paragraph{Lower bound on the gradient outer product $\E_{\hnu}[\nabla\ell\nabla\ell^{\sT}]$.}

Let
\begin{equation}
    \bA_i := 
    \grad\ell_{i}(\hat\bTheta_\lambda^\sT\bx_i)
    \grad\ell_{i}(\hat\bTheta_\lambda^\sT\bx_i)^\sT,
    \quad\quad
    \bA_{i,\eps} := 
    \grad\ell_{i,\eps}(\hat\bTheta_{\eps,\lambda}^\sT\bx_i)
    \grad\ell_{i,\eps}(\hat\bTheta_{\eps,\lambda}^\sT\bx_i)^\sT
\end{equation}

By Lemma \ref{lemma:LocalStrongMultinomial}, using the definition of $\ell_i$ therein,
 for any fixed $\lambda >0$ there exists $c_1(\lambda)$ independent of $n$
 such that, with high probability
\begin{equation}\label{eq:nablaell_1}
    \frac1n\sum_{i=1}^n \bA_i \succ c_1(\lambda) \bI\, .
\end{equation}
We'll show that the smallest singular value of $n^{-1}\sum_i \bA_i$ is sufficiently close to that of $n^{-1}\sum_{i}\bA_{i,\eps}$.

First, on $\Omega_{1,n}$, we have by strong convexity for $\lambda >0$, 
\begin{equation}
\label{eq:min_lipschitz_in_y}
    \|\hat\bTheta_{\eps,\lambda} - \hat\bTheta_{\lambda}\|_F \le \frac{C}{ \lambda \sqrt{n}} \|\bY_{\eps} - \bY\|_F
\end{equation}
where $\bY,\bY_{\eps} \in\R^{n\times k}$ are the matrices whose rows are the labels $\by_i^\sT,\by_{i,\eps}^\sT$, respectively, for $i\in[n]$.


Meanwhile $\|\grad\ell_{i}(\hat\bTheta_{\lambda}^\sT\bx_i)\|_2,\|\grad\ell_{i,\eps}(\hat\bTheta_{\eps,\lambda}^\sT\bx_i)\|_2   \le C$ for some $C > 0$ independent of $n$.
Further using the  Lipschitz continuity of the minimizer in the labels, we have, on $\Omega_{1,n}$,
\begin{align}
\label{eq:grad_diff_to_y_diff}
 \|\grad\ell_{i}(\hat\bTheta_{\lambda}^\sT\bx_i) - \grad\ell_{i,\eps}(\hat\bTheta_{\eps,\lambda}^\sT\bx_i)\|_2  
 &\le \|\bp(\hat\bTheta_{\eps,\lambda}^\sT \bx_i) - \bp(\hat\bTheta_{\lambda}^\sT\bx_i)\|_2 + 
 \|\by_{i} - \by_{\eps,i}\|_2\\
 &\le C 
 \|(\hat\bTheta_{\eps,\lambda}- \hat\bTheta_{\lambda})^\sT\bx_i\|_2 +  
 \|\by_{i} - \by_{\eps,i}\|_2.
 %&\le \left(\frac{C}{\lambda} + 1\right) \|\by_{i} - \by_{\eps,i}\|_2\, .
\end{align}


Now note that we have for all $i,j\in[n]$,
\begin{align}
\nonumber
\Tr\left(\bA_i(\bA_j - \bA_{j,\eps}))\right)
&= 
\left(\grad\ell_{j}(\hat\bTheta_\lambda^\sT\bx_i)^\sT
\grad\ell_{i}(\hat\bTheta_\lambda^\sT\bx_i)
\right)^2 -
\left(\grad\ell_{i}(\hat\bTheta_{\eps,\lambda}^\sT\bx_i)^\sT
\grad\ell_{j}(\hat\bTheta_\lambda^\sT\bx_j)
\right)^2\\
\label{eq:outer_prod_to_grad_diff}
&\quad\le
C\|\grad\ell_{i}(\hat\bTheta_\lambda^\sT\bx_i) 
-\grad\ell_{i,\eps}(\hat\bTheta_{\eps,\lambda}^\sT\bx_i)
\|_2
\end{align}
where in the inequality we used Cauchy Schwarz and that $\|\grad \ell_{i,\eps}\|_2,\|\grad \ell_{i}\|_2$ are uniformly bounded by a constant $C>0$ independent of $n$.
A similar bound clearly holds for $\Tr(\bA_{i,\eps}(\bA_j - \bA_{j,\eps})).$
So we can bound
\begin{align}
\label{eq:grad_outer_product_diff}
    \left\|\frac1n\sum_{i=1}^n  
    \bA_i -
    \frac1n\sum_{i=1}^n  \bA_{i,\eps} \right\|_F^2
    &=
    \frac1{n^2} \sum_{i,j\in[n]} \Tr\left((\bA_i-\bA_{i,\eps})(\bA_j-\bA_{j,\eps})^\sT \right)\\
    &\le
    \frac1{n^2} \sum_{i,j\in[n]}\left\{ |\Tr\left(\bA_i(\bA_j - \bA_{j,\eps}))\right)| + |\Tr\left(\bA_{i,\eps}(\bA_j - \bA_{j,\eps})\right)|
    \right\}\\
    &\stackrel{(a)}\le 
    \frac{C}{n}  \sum_{i=1}^n
    \|\grad\ell_{i}(\hat\bTheta_\lambda^\sT\bx_i) 
-\grad\ell_{i,\eps}(\hat\bTheta_{\eps,\lambda}^\sT\bx_i)
\|_2\\
&\stackrel{(b)}{\le}
\frac{C}{n}\sum_{i=1}^n \| (\hat\bTheta_{\eps,\lambda} - \hat\bTheta_{\lambda})^\sT\bx_i \|_2 +\|\by_i - \by_{\eps,i}\|_2\\
&\stackrel{(c)}{\le} \frac{C}{\sqrt{n}} \left( \| \bX (\hat\bTheta_{\eps,\lambda} - \hat\bTheta_{\lambda})\|_F + \| \bY - \bY_{\eps}\|_F\right)\\
& \stackrel{(d)}{\le} \frac{C}{\sqrt{n}}\left(\frac{\|\bX\|_\op}{\sqrt{n}\lambda} + 1\right) \|\bY - \bY_{\eps}\|_F,
\end{align}
where in $(a)$ we used Eq.~\eqref{eq:outer_prod_to_grad_diff}, in $(b)$ we used
Eq.~\eqref{eq:grad_diff_to_y_diff}, in $(c)$ we used $\|\bv\|_1 \le \sqrt{n}\|\bv\|_2$ for $\bv\in\R^{n}$, and in $(d)$ we used Eq.~\eqref{eq:min_lipschitz_in_y}.
So on the high probability events $\Omega_{1,n}\cap\Omega_{2,n}$,
\begin{equation}\label{eq:nablaell_2}
    \frac1n\sum_{i=1}^n \grad\ell_{i,\eps}(\hat\bTheta_{\eps,\lambda}^\sT\bx_i) 
    \grad\ell_{i,\eps}(\hat\bTheta_{\eps,\lambda}^\sT\bx_i)^\sT  \succeq
    \frac1n\sum_{i=1}^n \grad\ell_{i}(\hat\bTheta_{\lambda}^\sT\bx_i)
    \grad\ell_{i}(\hat\bTheta_{\lambda}^\sT\bx_i)^\sT -  \left(\frac{C}{\sqrt{n}}(\lambda^{-1} + 1) \| \bY - \bY_{\eps}\|_F\right)^{1/2} \bI_k.
\end{equation}
%
By construction of the smoothing, we have
    $\E[\|\by_i - \by_{i,\eps}\|_2^2 | \bTheta_0^\sT\bx_i] \le C \eps$
and $\|\by_i - \by_{i,\eps}\|_2^2 < C$ almost surely, for some $C >0$ independent of $n$. So by Hoeffding's inequality, for any $\delta> 0$ we can choose $\eps = \eps(\delta,\lambda)>0$
such that
\begin{equation}
    \lim_{n\to\infty }\P\left(\frac1{\sqrt{n}} \|\bY - \bY_\eps\|_F > \delta\right) =0\, .
\end{equation}
%
By choosing $\delta$ sufficiently small and using Eqs.~\eqref{eq:nablaell_1} 
\eqref{eq:nablaell_2}, we conclude that, with high probability,
\begin{equation}
    \frac1n\sum_{i=1}^n \grad\ell_{i,\eps}(\hat\bTheta_{\eps,\lambda}^\sT\bx_i) 
    \grad\ell_{i,\eps}(\hat\bTheta_{\eps,\lambda}^\sT\bx_i)^\sT
    \succeq \frac{c_1(\lambda)}{2}\bI\, .
%    \frac1n\sum_{i=1}^n \grad\ell_{i} 
%    \grad\ell_{i}^\sT -  \left(\frac{4 C_1^3 k^3}{\sqrt{n}} \| \bY - \bY_{\eps}\|_F\right)^{1/2} \bI_k.
\end{equation}
This shows that for all $\lambda,\eps >0$, with high probability  the minimizer $\hat\bTheta_{\eps,\lambda} \in \cE(\bTheta_0)$ for all $\eps >0$, i.e.,
%, when
%$\bw$ is in the high probability set
\begin{equation}
\lim_{n\to\infty}\P\big(\hat\bTheta_{\eps,\lambda} \in \cE(\bTheta_0) \big) = 1\,.
\end{equation}
Hence we have shown that the conditions of Theorem~\ref{thm:global_min} are satisfied.
This allows us to conclude the statement of the lemma when $\by$ is replaced with $\by_{\eps}$ in equations~\eqref{eq:multinomial_FP_regularized}, and $\hat\bTheta_{\lambda}$ is replaced by $\hat\bTheta_{\eps,\lambda}$ for $\eps >0$ sufficiently small.
Continuity of these equations in $\eps$ along with the consequence of strong convexity~\eqref{eq:min_lipschitz_in_y} allows us to then pass to the limit $\eps\to 0$ giving the desired claim.
\end{proof}

\begin{proof}[Proof of Proposition~\ref{prop:multinomial}]
\noindent{\emph{Claim \textit{1(a)}.}}
Since the system~\eqref{eq:FP_multinomial} has a (finite) solution $(\bR^\opt,\bS^\opt)$,
by Theorem~\ref{thm:simple_critical_point_variational_formula}, this solution is unique implying the claim. 

\noindent{\emph{Claim \textit{1(b)}.}}
Let $(\bR^\opt(\lambda),\bS^\opt(\lambda))$ the unique solution for $\lambda>0$,
which corresponds to the unique minimizer of $F(\bK,\bM)$ defined in 
Eq.~\eqref{eq:FKM_Def}, by Theorem \ref{prop:simple_critical_point_variational_formula}. Since $F(\,\cdot\, )$ depends continuously on $\lambda$
and has a unique minimizer for $\lambda=0$ (by the previous point), it follows that $(\bR^\opt(\lambda),\bS^\opt(\lambda)) \rightarrow (\bR^\opt,\bS^\opt)$ as $\lambda\to 0$.
In particular, there exists $C>0$ independent of $n$ and $\lambda$, such that,
for all $\lambda>0$ smalle enough
    \begin{equation}
         \lim_{n\to\infty} \P\left(\|\hat\bTheta_\lambda\|_F < C \right) = 1.\label{eq:LastBDD}
    \end{equation}
The equivalence of Lemma~\ref{lemma:equivalence_multinomial} then 
implies claim~\textit{(b)}.

\noindent{\emph{Claims \textit{1(c)}, \textit{1(d)}.}}
Since Eq.~\eqref{eq:LastBDD} implies Eq.~\eqref{eq:LambdaPerturbation}, 
and $\lim_{\lambda\to 0}(\bR^\opt(\lambda),\bS^\opt(\lambda)) = (\bR^\opt,\bS^\opt)$,
statements $(c)$ and $(d)$ follow from Lemma \ref{lemma:multinomial_regularized}.


\noindent{\emph{Claims \textit{2}.}}
If  the system~\eqref{eq:FP_multinomial} does not have a (finite) solution,
then we claim that  $\lim_{\lambda\to 0}\Tr(\bR^\opt(\lambda))=\infty$.
Indeed, if it by contradiction $\lim\inf_{\lambda\to 0}\Tr(\bR^\opt(\lambda))<\infty$,
then we can find a sequence $\bR^{\opt}(\lambda_i)$, $i\in \N$ 
with $\lambda_i\to 0$ and $\bR^{\opt}(\lambda_i)$ converging to a finite
limit  $\bR^{\opt}$ (recall that $\Tr(\bR)\le C$ is a compact subset of $\bR\succeq \bzero$), and this would be a solution of the system \eqref{eq:FP_multinomial}
with $\lambda=0$, leading to a contradiction.

Since  $\Tr(\bR^\opt(\lambda))$  is unbounded as $\lambda\to 0$, by Lemma~\ref{lemma:multinomial_regularized} there exists a sequence $\lambda_i\to 0$ 
such that for any $C>0$,
\begin{equation}
   \lim_{i\to \infty} \lim_{n\to\infty} \P(\|\hat\bTheta_{\lambda_i}\|_F > C)  = 1.
\end{equation}
Applying Lemma~\ref{lemma:equivalence_multinomial}, 
we thus conclude that Eq.~\eqref{eq:any_seq_minimzers_diverges}.
\end{proof}



%\begin{proof}[Proof of Proposition~\ref{prop:multinomial}]
%\textbf{Bounded MLE exists:}
%First assume that there exists some $C>0$ such that
%\begin{equation}
%    \lim_{n\to\infty}\P\left(\hat\bTheta \textrm{ exists},\|\hat\bTheta\| < C\right) = 1.
%\end{equation}
%Then by Lemma S6.2 of~\cite{tan2024multinomial}, w.h.p., the hessian at the MLE is lower bounded by a constant independent of $n$, implying that for any $\eps>0$,
%\begin{equation}
%    \lim_{\lambda \to 0} \limsup_{n}\P\left(\|\hat\bTheta_{n,\lambda} - \hat\bTheta_n\|>\eps \right)= 0.
%\end{equation}
%By continuity of the fixed point equations in $\lambda$, we conclude that there must exist some $\bR_\opt$ given as the $\lambda \to 0$ limit of $\bR_{\opt}(\lambda)$ that solves these equations at $\lambda = 0$.
%By Lemma~\notate{ref}, this solution must then be unique by strict convexity
%
%\textbf{No bounded MLE:}
%Conversely, assume that for any $C>0$,
%\begin{equation}
%    \lim_{n\to\infty}\P\left(\|\hat\bTheta\| < C\right) < 1.
%\end{equation}
%Then by Lemma~\notate{ref}, we have that for any $C  > 0$,
%\begin{equation}
%    \lim_{\lambda \to 0} \lim_{n\to\infty} \P\left( \|\hat\bTheta_{\lambda}\|_F < C \right) < 1.
%\end{equation}
%So for any $C >0,$ by Lemma~\ref{lemma:multinomial_regularized},
%\begin{align}
%   \one_{\{\|\bR_\opt(\lambda)\| < C/2 \}}&\le \lim_{n\to\infty} \P\left( \|\bR_\opt(\lambda)\| < C/2 ,\; \|\bR_\opt(\lambda) - \bR(\hat \bTheta_\lambda)\| < C/2\right)\\
%   &\le \lim_{n\to\infty}  \P\left( \|\bR(\hat\bTheta_\lambda)\| < C\right)\\
%   &\le \lim_{n\to\infty}  \P\left( \|\hat\bTheta_\lambda\| < C\right).
%\end{align}
%Taking $\lambda \to 0$ shows that
%$\lim_{\lambda \to 0} \bR_\opt(\lambda) = \infty$, implying that fixedpoint equations are not satisfied for any finite $\bR$ at $\lambda =0$.

%\end{proof}


%\subsubsection{AMP}
%Define the GAMP recursion
%\begin{align}
%    \hat\bTheta^{t+1} =& 
%    \alpha\bX^\sT\left(\bY - 
%    \bp(\Prox(\bU^{t} + \bY\bS_t; \bS_t))\right) \bS_{t+1}
%    + \hat\bTheta^t \bS_t^{-1}\bS_{t+1}\in\R^{d\times k},\\
%    \bU^{t+1} = & \bX \hat \bTheta^{t+1}
%    -\left( \bY - \bp(\Prox(\bU^{t} + \bY\bS_t;\bS_t))\right)\bS_{t+1}\in \R^{n\times k},
%\end{align}
%where the $\bp$ and $\Prox$ are applied row-wise, with the state evaluation iteration given by \kas{$St+1 -> St s_ts_t+1^{-1} =$}
%\begin{align}
%    \alpha \; 
%    \E[\left(\bp(\bv_t)- \by \right)
%    \left(\bp(\bv_t) - \by\right)^\sT ]   &=  \bS_t^{-1}(\bR_{t+1}/ \bR_{00})\bS_{t+1}^{-1},\\
%   \bR_{01,t} - \alpha\bS_{t+1}\E[   (\bp(\bv_t)- \by )\bg_0^\sT  ] &= \bR_{01,t+1}  ,\\
%    \bS_t(\bI_k - \E[(\bI_k + \bS_t \bJ \bp(\bv_k))^{-1}])^{-1}  &= \alpha\bS_{t+1},
%\end{align}
%for $\bv_t = \Prox(\bg_t + \bS_t\by)$, where 
%\begin{equation}
%(\bg_t^\sT,\bg_0^\sT)\sim \cN(\bzero, \bR^t),
%\qquad\bR^t = 
%    \begin{bmatrix}
%        \bR_{11,t}&\bR_{01,t}\\
%        \bR_{01,t}^\sT& \bR_{00}
%    \end{bmatrix}.
%\end{equation}


%\input{Appendix_E_EXPERIMENTS}

%\input{KR_integral_analysis}
%\input{Large_deviations}
%\input{Convex_case_analysis}



\bibliographystyle{amsalpha}
\bibliography{all-bibliography}

\end{document}  