\section{Related Work}
\label{sec:related}

Natural Language Interfaces (NLIs) are essential for facilitating access to KGs, enabling users to interact with these complex data structures through conversational natural language ____. By translating unstructured input into formal query languages such as SPARQL, NLIs bridge the gap
between user-friendly interaction and the highly structured nature of semantic data. The development of NLIs addresses a significant challenge
in making RDF-based knowledge graphs more accessible to non-technical
users, thereby expanding their applicability across diverse fields such as healthcare, e-commerce, and education.

LLMs have recently shown great potential in the
translation of natural language into SPARQL queries. By leveraging their
ability to process and generate complex text, LLMs offer a robust framework for automating query generation, reducing the need for manual intervention, and making knowledge graphs accessible to non-expert users ____.
Early systems like SGPT leverage transformer-based architectures to generate SPARQL queries, addressing structural and semantic requirements____.
Another significant approach involves fine-tuning the OpenLLaMA model for SPARQL query generation.
By adapting this LLM to specific datasets containing question-to-SPARQL pairs, the model effectively translates natural language questions into SPARQL
queries over life science knowledge graphs. Despite its success, the model's performance is contingent on the quality and representativeness of the training data,
which may limit its generalizability to other domains____. BART, another transformer-based model, has also been applied to SPARQL query generation.
Its encoder-decoder structure is well-suited for handling long contexts and generating queries that require nested reasoning.
For instance, the NLQxform system employs a fine-tuned BART model to translate natural language questions into SPARQL,
integrating steps like entity linking and template-based corrections to improve accuracy. While BART performs well in structured domains,
it faces challenges when dealing with noisy or incomplete inputs____. 


Template-based approaches complement these methods by providing deterministic frameworks for query generation. Systems like CatSQL and BERT-based methods enhance
SQL generation by combining predefined templates with semantic corrections and contextual embeddings.

Although these approaches ensure interpretability and efficiency, their reliance on predefined patterns limits flexibility in handling complex or ambiguous queries____. 
Despite advances in data integration, ETL, and query generation, challenges remain in scaling these systems for diverse domains and ensuring seamless
interaction with KGs. This work aims to integrate semantic technologies with AI-driven methods for efficient and user-friendly data querying and visualization.