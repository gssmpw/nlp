%%%%%%%%%%%%%%%%%%%%%%%%%%%%%%%%%%%%%%%%%%%%%%%%%
\section{Experimental Results} \label{sec:simulation}

\subsection{Simulation Setup}
The simulation environment is based on site-specific maps of the University of Southern California (USC) campus, covering an area of $512 \times 512$ [m$^2$]. This environment reflects dense urban features, including buildings and terrain, providing a realistic scenario to evaluate AutoBS.

Simulations are executed on a high-performance machine with a Tesla T4 GPU, 12GiB of RAM, and an Intel Xeon CPU @ 2.20GHz. Developed in Python with PyTorch for model training, the framework leverages PMNet for pathloss predictions. 
% , significantly accelerating the reward computation of coverage and capacity over the site-specific map.

Table~\ref{table:paramter} summarizes the key simulation parameters, including network configurations, environmental factors, and training setup.

\begin{table}[!h]
\centering
\resizebox{.85\columnwidth}{!}{\begin{minipage}[t]{.9\columnwidth}
    \centering
    \begin{tabular} {l l}
\toprule[1pt]
% \textbf{\textit{Type (or Parameter)}} & \textbf{\textit{Value}} \\
% \midrule[.8pt]

\textbf{Network Config.} \\
\cmidrule(lr){1-1} \cmidrule(lr){2-2}
Area & $512 \times 512$ [m$^2$]  \\ 
Carrier frequency & $2.5$ [GHz] \\
Effective bandwidth & $1$ [MHz] \\
Antenna & Isotropic (vertical) \\
Input power & $0$ [dBm] \\
Noise figure & $3$ [dB] \\
Coverage threshold & $-90.015$ [dBm]  \\

\cmidrule(lr){1-2}
\textbf{AutoBS Train Config.} \\
\cmidrule(lr){1-1} \cmidrule(lr){2-2}

DT generator & PMNet$_{\mathrm{usc}}$ \cite{lee2024scalable} \\ 
DRL algorithm & PPO \cite{schulman2017proximal} \\ 
Reward function & Pathgain + Coverage ($\nu_2=0$) \\ 

Learning rate & $1.0 \times 10^{-5}$ \\
Gamma & $0.1$ \\
% Gradient Clipping & $40.0$ \\
% Exploration type Size & StochasticSampling \\
SGD minibatch size & $256$ \\

\# of samples for train (test) & $1100$ ($50$)\\
\# of episode & $5$ 
\\


\bottomrule[1pt]
\end{tabular}




% \textbf{One-layer dielectric} & \\
% Permitivity & $4.5$ \\
% Conductivity & $0.05$ $S/m$ \\
% Thickness & $0.3$ $m$ \\

% \textbf{Dielectric half-space (Dry Earth)} & \\
% Permitivity & $4$ \\
% Conductivity & $0.001$ $S/m$ \\
% Thickness & $0$ $m$ \\

% \textbf{Sinusoid} & \\

% Phase & $0^{\circ}$ \\

% \textbf{Half-wave dipole} & \\
% Waveform & Sinusoid \\
% Polarization & Vertical \\
% Receiver Threshold & $-250$ $dBm$ \\
% Transmission line loss & $0$ $dB$ \\
% VSWR & $1$ \\

% \textbf{Isotropic vertical} & \\
% Waveform & Sinusoid \\
% Polarization & Vertical \\
% Receiver Threshold & $-250$ $dBm$ \\
% Transmission line loss & $0$ $dB$ \\
% VSWR & $1$ \\

% \textbf{Transmitter Properties} & \\
% Alignment & Fixed Global \\
% Coordinate & Spherical \\
% Phi & $0^{\circ}$ \\
% Theta & $90^{\circ}$ \\
% Roll & $0^{\circ}$ \\

% \textbf{Receiver Properties} & \\
% Antenna & Half-wave dipole \\
% Alignment & Fixed Global \\
% Coordinate & Spherical \\
% Phi & $0^{\circ}$ \\
% Theta & $90^{\circ}$ \\
% Roll & $0^{\circ}$ \\

% \textbf{Simulation} & \\
% Propagation Model & Full 3-D \\
% Ray Tracing & $0.25^{\circ}$ \\
% Number of Reflections & $10$ \\
% Number of transmissions & $0$ \\
% Number of diffractions & $1$ \\
% Raytracing methods & SBR \\

% \cmidrule(lr){1-2}
% \emph{\textbf{AutoBS}} \\
% \cmidrule(lr){1-1} \cmidrule(lr){2-2}
% \tpurp{RX Spacing} & $5$ $m$ \\ 
% \tpurp{TX size in TX map ($\rho$)} & $5 \times 5$ $px$ \\
% Bilinear Interpolation & $2$ nearest neighbour \\
% Power Range & \numrange{-255}{0} $dBm$ \\
% Dimensions & $512\times512$ $m^2$ \\ 

% \cmidrule(lr){1-2}
% \emph{\textbf{Pmnet}} \\
% \cmidrule(lr){1-1} \cmidrule(lr){2-2}
% Batch Size & $16$ \\ 
% Learning rate & $1e-4$ \\
% Learning rate decay & $0.45$ \\
% Step & 10 \\
    \caption{Simulation Parameters.}
    \label{table:paramter}
    % \vspace{-.5em}
\end{minipage}}
\end{table}


%%%%%%%%%%%%%%%%%%%%%%%%%%%%
\BfPara{Benchmark}
To evaluate the performance of the proposed AutoBS framework, we compare it against two baseline methods: a heuristic-based deployment and an exhaustive search strategy. These baselines serve as the lower and upper bounds for performance comparison.

\begin{itemize}
    \item \textbf{Heuristic}:    
    The heuristic deployment method randomly places BSs within the deployable area (\ie $\mathcal{B}$), without incorporating learning or optimization. This simple approach serves as a performance lower bound, as it overlooks the impact of the site-specific information on the channel. It is also the assumption underlying stochastic-geometry system evaluations \cite{haenggi2012stochastic}.

    \item \textbf{Exhaustive}:    
    The exhaustive search provides an upper bound for performance by systematically evaluating all possible BS placement configurations across the site-specific map. It guarantees a global optimal solution in terms of coverage or capacity based on the configuration criteria. However, the computational cost is exceedingly high, particularly for multi-BS deployments, as the complexity grows exponentially with the number of BSs and map size. Due to this limitation, the exhaustive search is restricted to $50$ samples in our baseline experiments. Additionally, in multi-BS deployments, the exhaustive method operates synchronously, unlike the asynchronous deployment strategy used by AutoBS. While an asynchronous exhaustive method could be defined to reduce computation time somewhat, it would still be computationally intense and would yield suboptimal results. Our focus remains on synchronous exhaustive deployment as it provides a true upper bound for comparison with AutoBS.

    \item \textbf{AutoBS (Proposed)}:    
    The AutoBS leverages the PPO agent to make near-optimal BS deployment decisions in real time. It is trained using a fully connected neural network with four layers, each consisting of 128 nodes, to optimize both coverage and capacity. AutoBS achieves rapid deployment decisions at millisecond timescales. 
    %, aided by the PMNet framework for fast and accurate pathloss predictions. With its focus on balancing performance and computational efficiency, AutoBS offers a practical, scalable solution suitable for large-scale network deployments.
\end{itemize}

%%%%%%%%%%%%%%%%%%%%%%%%%%%%
\begin{figure*}[!ht]
    \centering  
    \begin{subfigure}[t]{.20\linewidth}
        \centering  
        \includegraphics[width=\linewidth]{figure/building_map.png}
        \caption{Building Map} \label{fig:bld map}
    \end{subfigure}\hspace*{.015\textwidth}%
    \begin{subfigure}[t]{.20\linewidth}
        \centering  
        \includegraphics[width=\linewidth]{figure/Fig3/random.png}
        \caption{Heuristic}
    \end{subfigure}\hspace*{.015\textwidth}%
    \begin{subfigure}[t]{.20\linewidth}
        \centering
        \includegraphics[width=\linewidth]{figure/Fig3/exhaustive.png}
        \caption{Exhaustive}
    \end{subfigure}
    \begin{subfigure}[t]{.20\linewidth}
        \centering  
        \includegraphics[width=\linewidth]{figure/Fig3/autobs.png}
        \caption{AutoBS}
    \end{subfigure}\hspace*{.015\textwidth}%
\caption{Comparison of Single (Static) BS deployment in terms of coverage using \emph{SionnaRT}. Light green areas indicate higher received signal strength. Fig. \ref{fig:bld map} shows a building map $\mathcal{S}$ used in each state $s_t$.}
\label{fig:comparison_coverage_singlebs}
% \vspace{-1.em}
\end{figure*}


%%%%%%%%%%%%%%%%%%%%%%%%%%%%
\subsection{Performance Analysis}

%%%%%%%%%%%%%%%%%%%%%%%%%%%%
\begin{table}[!h]
\centering
\resizebox{.8\linewidth}{!}{\begin{minipage}{.85\linewidth}
\small
\centering
\begin{tabularx}{1\linewidth}{l c c}
\toprule[1pt]
\textbf{\textit{Scheme}} & \textbf{\textit{Coverage} ($V$)} & \textbf{\textit{Capacity} ($C$)}  \\
\midrule[.8pt]

Heuristic & $43.37 \%$ & $1.892 $ \\
AutoBS & $59.44 \%$ & $2.893 $ \\
\cmidrule(lr){1-3}
Exhaustive for $V$ & $64.23 \%$ & $2.981$ \\
Exhaustive for $C$ & $63.08 \%$ & $3.036$  \\

\bottomrule[1pt]
\end{tabularx}





\end{minipage}}
\caption{Comparison results for Single (Static) BS deployment between exhaustive, random, and AutoBS deployment.}
\label{table:comparison_singlebs}
% % \vspace{-.5em}
\end{table}

\BfPara{Comparison}
Table \ref{table:comparison_singlebs} presents the performance of Heuristic, Exhaustive, and AutoBS deployments. AutoBS achieves significant improvements in both coverage and capacity compared to Heuristic deployment. Its performance closely approaches that of the Exhaustive method, which represents the global optimal solution, highlighting the effectiveness of policy-guided BS placement.

AutoBS focuses on coverage optimization through a reward function based on pathloss predictions from PMNet. Although capacity metrics are not explicitly included in the reward function (\ie $\nu_2=0$), the observed capacity difference between AutoBS and Exhaustive search remains minimal—which will be further discussed in Sec. \ref{sec:appendix}. 
%By balancing computational efficiency and network performance, AutoBS offers a scalable and practical solution for real-world, large-scale deployments.

%%%%%%%%%%%%%%%%%%%%%%%%%%%%
\BfPara{Coverage}
Fig. \ref{fig:comparison_coverage_singlebs} presents the coverage results for single BS deployment. The building map, representing the input state, includes key geographical features such as buildings and obstacles, which significantly affect BS placement and signal propagation.

In the \emph{Heuristic} method, BSs are placed randomly across the deployable area, leading to inefficient coverage with noticeable gaps. This highlights the limitations of non-optimized deployments. The \emph{Exhaustive} search method evaluates every possible BS placement option, achieving the best coverage by systematically considering all configurations, but at the price of much higher complexity. 
%While this guarantees optimal results, it is computationally expensive and impractical for larger maps or multi-BS scenarios.

The \emph{AutoBS} method, trained with the PPO algorithm and PMNet for fast pathloss predictions, achieves coverage performance close to the exhaustive search but at a fraction of the computational cost. 
%Although AutoBS doesn’t guarantee the absolute optimal solution, its learned policy effectively balances performance and efficiency.

It is important to note that AutoBS uses PMNet during training for fast and accurate pathloss map generation. \footnote{The plots are generated with SionnaRT purely because of its nicer graphical representation (the underlying numerical results are identical to those obtained with PMNet).} 
%, while SionnaRT, which is more detailed but computationally demanding, is used only for visualization purposes and not for training \footnote{}.


%%%%%%%%%%%%%%%%%%%%%%%%%%%%
\BfPara{Convergence}
Fig. \ref{fig:convergence_singlebs} presents the convergence behavior for single BS deployment, comparing the AutoBS with Heuristic and Exhaustive methods (note that Heuristic and Exhaustive methods do not involve iterative learning). The AutoBS stabilizes between $50$ to $100$ training episodes, approaching the performance of the global optimal solution (\ie Exhaustive), demonstrating its efficient learning and stable convergence.

%It is important to note that the Heuristic and Exhaustive methods do not involve iterative learning. The Heuristic deployment serves as a baseline, reflecting non-optimized lower-bound performance, while the Exhaustive deployment represents the global optimum but at the cost of high computational complexity. AutoBS offers a practical balance between optimality and computational efficiency. 
% Further quantitative comparisons of coverage and capacity can be found in Table~\ref{table:comparison_singlebs}.
\begin{figure}[!h]
\centering
\includegraphics[width=.8\linewidth]{figure/convergence/convergence_singlebs_reward_5.pdf}
\caption{Convergence behavior for Single (Static) BS deployment for Heuristic, Exhaustive, and AutoBS deployments over $200$ steps.}
\label{fig:convergence_singlebs}
% \vspace{-.5em}
\end{figure}




%%%%%%%%%%%%%%%%%%%%%%%%%%%%
\begin{figure*}[h!]
    \centering  
    \begin{subfigure}[t]{.20\linewidth}
        \centering  
        \includegraphics[width=\linewidth]{figure/building_map.png}
        \caption{Building map} \label{fig:bld map2}
    \end{subfigure}\hspace*{.015\textwidth}%
    \begin{subfigure}[t]{.20\linewidth}
        \centering  
        \includegraphics[width=\linewidth]{figure/Fig5/random.png}
        \caption{Heuristic}
    \end{subfigure}\hspace*{.015\textwidth}%
    \begin{subfigure}[t]{.20\linewidth}
        \centering
        \includegraphics[width=\linewidth]{figure/Fig5/exhaustive.png}
        \caption{Exhaustive}
    \end{subfigure}
    \begin{subfigure}[t]{.20\linewidth}
        \centering  
        \includegraphics[width=\linewidth]{figure/Fig5/autobs.png}
        \caption{AutoBS}
    \end{subfigure}\hspace*{.015\textwidth}%
\caption{Comparison results for Multi (Asynchronous) BS deployment in terms of coverage using \emph{SionnaRT}. Light green areas indicate higher received signal strength. Fig. \ref{fig:bld map2} shows a building map $\mathcal{S}$ used in each state $s_t$.}
\label{fig:comparison_coverage_multibs}
% \vspace{-1.em}
\end{figure*}


%%%%%%%%%%%%%%%%%%%%%%%%%%%%
\BfPara{Multi-BS Deployment (Asynchronous)}
Table \ref{table:comparison_multibs} and Fig. \ref{fig:comparison_coverage_multibs} present the results for asynchronous multi-BS deployment. 
The reward function is computed based only on the specific metrics (e.g., coverage, capacity, and/or pathgain) of each newly deployed BS. For example, when deploying the second BS, only the performance metrics for that BS are considered in the reward calculation.

For the multi-BS case, AutoBS performs an asynchronous deployment, \ie adds one BS at a time,  in a "greedy" manner. This can be a practical requirement, \eg when adding BSs to an existing infrastructure. In a greenfield deployment (like investigated here), the sequential approach will give worse results than a synchronous approach. Potential iterations to fine-tune already-placed BSs will be subject of our future research.  

\begin{table}[!h]
\centering
\resizebox{.8\linewidth}{!}{\begin{minipage}{.85\linewidth}
\small
\centering
\begin{tabularx}{1\linewidth}{l c c}
\toprule[1pt]
\textbf{\textit{Scheme}} & \textbf{\textit{Coverage} ($V$)} & \textbf{\textit{Capacity} ($C$)}  \\
\midrule[.8pt]

Heuristic & $63.95 \%$ & $2.966$ \\
AutoBS & $71.65 \%$ & $3.637$ \\
\cmidrule(lr){1-3}
Exhaustive for $V$ & $81.06 \%$ & $3.995$ \\
Exhaustive for $C$ & $75.36\%$ & $4.023$ \\

\bottomrule[1pt]
\end{tabularx}




\end{minipage}}
\caption{Comparison results for Multi (Asynchronous) BS deployment between exhaustive, random, and AutoBS deployment.}
\label{table:comparison_multibs}
\vspace{-.5em}
\end{table}

While the overall findings align with those from the single (static) BS deployment, some notable insights emerge. One key observation is that deploying two BSs using the Heuristic method performs similarly or even worse than deploying a single BS with AutoBS or Exhaustive search in terms of coverage and capacity. This emphasizes the critical role of optimized deployment strategies. As shown in Table \ref{table:comparison_multibs} for Heuristic and Table \ref{table:comparison_singlebs} for AutoBS and Exhaustive search, non-optimized deployments fall short even when more BSs are deployed.

These findings highlight two key aspects. First, AutoBS, despite using an asynchronous deployment strategy, achieves $90$\% of the capacity of the global optimum (\ie provided by the exhaustive method). Second, AutoBS’s ability to learn site-specific characteristics enables it to efficiently deploy BSs. 
% The results demonstrate that AutoBS offers a scalable, practical alternative to exhaustive search, achieving competitive performance without the prohibitive computational costs—which will be elaborated in the following.
% Another key insight is the minimal convergence gap between AutoBS and Exhaustive search, especially in the multi-BS scenario, as illustrated in Fig. \ref{fig:convergence_multibs}. 



% \begin{figure}[!h]
% \centering
% \includegraphics[width=1.\linewidth]{figure/convergence/v35.png}
% \caption{Convergence behavior for Multi (Asynchronous) BS deployment for Heuristic, Exhaustive, and AutoBS deployments over $200$ steps.}
% \label{fig:convergence_multibs}
% % \vspace{-.5em}
% \end{figure}


%%%%%%%%%%%%%%%%%%%%%%%%%%%%
\begin{table*}[h!]
\centering
\resizebox{.65\linewidth}{!}{\begin{minipage}{.68\linewidth}
\small
\begin{table}[t]
    \centering
    \caption{Number of real multiplications and sum rate for different methods (``Ericsson'' site, $N_{\sf{T}} = 64$, $N_{\sf{U}} = 4$, SNR = 27dB).}
    \resizebox{\columnwidth}{!}{
    \begin{tabular}{ccc}
        \toprule
        \textbf{Methods} & \textbf{\# of Multiplications} & \textbf{Sum-Rate (b/s/Hz)}\\
        \midrule
        WMMSE \cite{shi2011iteratively} ($I=12.5$)    & 36.1\,M & 37.29
        \\ 
        MAML-CNN (zero-shot) & 3.77\,M & 37.01 \\
        \textbf{PaPP} (zero-shot) & \textbf{1.05\,M} & \textbf{38.10}\\
        Zero Forcing \cite{nayebi2017precoding} & 8.4\,K & 32.14\\
        \bottomrule
    \end{tabular} }
    \label{tab:complexity}
\end{table}
Table \ref{tab:complexity} shows an example of the computational complexity of different precoding methods, measured by the number of real multiplications required for processing when the deployment site is ``Ericsson''. This analysis offers insights into the trade-offs between computational demands and sum-rate performance.

The \gls{WMMSE} algorithm is renowned for achieving good weighted sum rate in mMIMO systems. However, this performance comes at the cost of high computational complexity. Specifically, with stopping criteria of $10^{-3}$ and an average of 12.5 iterations for this setup, the WMMSE method requires approximately 36 million multiplications. This complexity can limit real-time applications or systems constrained by computational resources. The total number of real multiplications for the WMMSE method is given by
\[
 4I \bigg(\frac{2}{3} N_{\sf{T}}^3N_{\sf{U}} + N_{\sf{T}}^2N_{\sf{U}} + 2N_{\sf{T}}(2N_{\sf{U}}^2+N_{\sf{U}})+ N_{\sf{U}}^2+\frac{14}{3}N_{\sf{U}}\bigg) \, ,
\]
where $I$ is the total number of iterations. 

As another baseline, we consider the method proposed in \cite{lyu2023downlink}, which combines a multilayer perceptron (MLP) model with the WMMSE algorithm. However, we replace the original MLP model with a CNN model similar to the approach in \cite{hojatian2021unsupervised}.
This modification was made to better capture spatial features in the \gls{CSI}, enabling the model to handle the increased complexity introduced by a larger number of antennas and users. By leveraging the representational power of CNNs, MAML-CNN achieves more competitive results in our experimental settings compared to the original MAML-MLP. Since it combines a \gls{DNN} with an additional matrix inversion step, it exhibits significantly higher complexity than \gls{ZF} but remains less demanding than WMMSE. 
The number of real multiplications required by MAML-CNN is given by
\begin{align*}
 & \
C_{\text{out}}N_{\sf{T}}N_{\sf{U}}C_{\text{in}}k^2 + C_{\text{out}}N_{\sf{T}}N_{\sf{U}}(3N_{\sf{U}}+1) \\
  & + 8\bigg(\frac{4}{3} N_{\sf{T}}^3 + N_{\sf{T}}^2 (3N_{\sf{U}}+2) + N_{\sf{T}}(2N_{\sf{U}}+3)\bigg) \, ,
\end{align*}
where $C_{\text{in}}$ and  $C_{\text{out}}$ are the input and output channels of the CNN layer, and $k$  is the kernel size. 

The proposed PaPP method leverages a DNN to provide a zero-shot precoding solution. While it exhibits greater computational complexity compared to traditional methods like ZF due to its convolutional and fully connected layers, this approach offers significant performance benefits, including improved interference mitigation and adaptability to unseen sites. The total number of real multiplications required for the PaPP method can be expressed as
\begin{align*}
 & \ C_{\text{out}}N_{\sf{T}}N_{\sf{U}}C_{\text{in}}k^2 + C_{\text{out}}N_{\sf{T}}N_{\sf{U}}D_{\text{FC1}} \\
& +D_{\text{FC1}}D_{\text{FC2}} + D_{\text{FC2}}D_{\text{FC3}} + D_{\text{FC3}}D_{\text{FC4r}} + D_{\text{FC3}}D_{\text{FC4i}}\, ,
\end{align*}
where $D_{\text{FCi}}$ ($i=1,2,3,4$) represent the sizes of the \glspl{FCL}. 

The \gls{ZF} \cite{nayebi2017precoding} precoding method offers significantly lower computational complexity, requiring approximately 8.4 thousand multiplications. This efficiency stems from its reliance on simpler linear algebra operations, specifically the inversion of smaller matrices when $N_{\sf{T}}>N_{\sf{U}}$. Despite its computational efficiency, ZF is susceptible to performance degradation in high-interference scenarios or under adverse channel conditions. The total number of real multiplications required by ZF is
\[
8 N_{\sf{U}}^2 N_{\sf{T}} + \frac{8}{3} N_{\sf{U}}^3 \, .
\]
\end{minipage}}
\caption{Time complexity analysis.}
\label{table:complexity}
\vspace{-1.em}
\end{table*}


\subsection{Complexity Analysis}
Table \ref{table:complexity} highlights the time complexity analysis, emphasizing a key advantage of AutoBS: its ability to drastically reduce inference time, achieving millisecond-level decisions even in multi-BS deployments. 
%AutoBS efficiently determines near-optimal deployment locations with minimal computational overhead, presenting high performance in both single and multi-BS scenarios.

The efficiency gains are particularly notable in multi-BS deployments, where exhaustive search faces exponential complexity. AutoBS’s asynchronous deployment strategy enables it to scale seamlessly with growing network size, making it well-suited for large-scale networks and real-time optimization. 
% By balancing computational efficiency with near-optimal performance, AutoBS offers a practical, scalable solution for network operators to manage complex deployment challenges in dynamic environments.


