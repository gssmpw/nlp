%%%%%%%%%%%%%%%%%%%%%%%%%%%%%%%%%%%%%%%%%%%%%%%%%
\section{Introduction} \label{sec:intro}
The rollout of 6G networks demands higher base station (BS) density due to the use of higher frequencies like millimeter-wave (mmWave), which offers enhanced bandwidth and low latency. However, these frequencies suffer from severe signal attenuation and limited propagation range, particularly in complex urban environments. 
As a result, dense BS deployment becomes essential to maintain reliable coverage and capacity. Furthermore, cell-free massive MIMO, where a significant number of access points with one of a few antennas each are distributed over an area \cite{interdonato2019ubiquitous}
is anticipated to be widely used in 6G.  

However, optimizing BS placement in such environments presents significant challenges due to highly site-specific conditions.
Traditional BS deployment methods often rely on manual planning, based on real-world measurement of the propagation conditions \cite{molisch2023wireless}, and/or computationally intensive ray-tracing simulations using tools like \emph{WirelessInsite} \cite{WirelessInsite} or SionnaRT \cite{sionna}. 
The fact that these approaches are time- and labor- consuming makes them less suited for the dense deployment in 6G. Furthermore, they are not suited for real-time adaptation of network topology, e.g., dynamic addition and placement of (mobile) BSs in reaction to change of user density, e.g., at special events. 

%but also struggle to efficiently adapt to site-specific dynamics and large-scale 6G deployments. The complexity of modeling site-specific propagation characteristics—such as reflection, diffraction, and blockage caused by buildings—demands more advanced, automated solutions.

Such network planning can be also seen as a special case of Digital Twin (DT) networks, which create virtual replicas of physical environments, enabling real-time simulation and optimization of network performance under site-specific conditions. Again, manually optimizing BS placement within DT frameworks remains computationally expensive.

Therefore, large-scale and real-time network optimization requires the integration of machine learning (ML)-based approaches to automate and expedite the process.

To address these challenges, we propose AutoBS, an autonomous BS deployment framework that uses Deep Reinforcement Learning (DRL) and a DT generator to optimize BS placement in a site-specific manner. AutoBS integrates PMNet \cite{lee2024scalable, lee2023robust}, a fast and accurate ML-based path gain predictor, to calculate reward based on both coverage and capacity metrics. By autonomously learning optimal placement strategies, AutoBS significantly reduces the computational overhead and deployment time compared to conventional methods, making it well-suited for real-time and large-scale 6G network optimization.


%%%%%%%%%%%%%%%%%%%%%%%%%%%%
\BfPara{Contributions}
This paper presents AutoBS, an autonomous BS deployment framework utilizing DRL and a DT generator (PMNet). Our key contributions are summarized as follows:
\begin{itemize}
\item \textbf{AutoBS Framework Design:} We introduce a novel DRL-based framework for BS deployment that incorporates PMNet for real-time, site-specific channel predictions\footnote{Selected results and codes are available at \href{https://github.com/abman23/autobs}{github.com/abman23/autobs}.} 
PMNet enables fast, precise reward calculations, allowing AutoBS to effectively optimize coverage and capacity through an adaptive reward function (see \textbf{Fig. \ref{fig:training}} in Sec. \ref{sec:autobs}). Details on reward design are provided in \textbf{Table \ref{table:reward}} in Sec. \ref{sec:appendix}.
    
\item \textbf{Support for Multi-BS Deploy.:} AutoBS handles both static single BS and asynchronous multi-BS deployments, providing flexibility across diverse deployment scenarios (see \textbf{Fig. \ref{fig:comparison_coverage_multibs}} in Sec. \ref{sec:simulation}).

\item \textbf{Fast and Near-Optimal Deployment:} Simulations demonstrate that AutoBS reduces inference time from hours to milliseconds compared to exhaustive methods, particularly in multi-BS deployments, making it practical for large-scale, real-time optimization (see \textbf{Table \ref{table:complexity}} in Sec. \ref{sec:simulation}).
\end{itemize}


