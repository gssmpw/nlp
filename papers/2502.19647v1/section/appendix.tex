\section{Appendix: Reward Design} \label{sec:appendix}
\vspace{-.5em}
\begin{table}[h!]
\centering
\resizebox{.7\linewidth}{!}{\begin{minipage}{.85\linewidth}
\centering
\small
\begin{tabularx}{1\linewidth}{l c c}
\toprule[1pt]
\textbf{\textit{Reward}} & \textbf{\textit{Coverage} ($V$)} & \textbf{\textit{Capacity} ($C$)}  \\
\midrule[.8pt]

% Exponential-Reward & $45.125 \%$ \; \tikz{
% \draw[gray,line width=.3pt] (0,0) -- (1.1,0);
% \draw[white, line width=0.01pt] (0,-2pt) -- (0,2pt);
% \draw[black,line width=1pt] (0.14,0) -- (0.3,0);
% \draw[black,line width=1pt] (0.14,-2pt) -- (0.14,2pt);
% \draw[black,line width=1pt] (0.3,-2pt) -- (0.3,2pt);} 
% & $2.214$  \; \tikz{
% \draw[gray,line width=.3pt] (0,0) -- (1.1,0);
% \draw[white, line width=0.01pt] (0,-2pt) -- (0,2pt);
% \draw[black,line width=1pt] (0.05,0) -- (0.15,0);
% \draw[black,line width=1pt] (0.05,-2pt) -- (0.05,2pt);
% \draw[black,line width=1pt] (0.15,-2pt) -- (0.15,2pt);}
% \\

% Relu-Reward  & $43.707 \%$ \; \tikz{
% \draw[gray,line width=.3pt] (0,0) -- (1.1,0);
% \draw[white, line width=0.01pt] (0,-2pt) -- (0,2pt);
% \draw[black,line width=1pt] (0.05,0) -- (0.15,0);
% \draw[black,line width=1pt] (0.05,-2pt) -- (0.05,2pt);
% \draw[black,line width=1pt] (0.15,-2pt) -- (0.15,2pt);}
% & $2.094$  \; \tikz{
% \draw[gray,line width=.3pt] (0,0) -- (1.1,0);
% \draw[white, line width=0.01pt] (0,-2pt) -- (0,2pt);
% \draw[black,line width=1pt] (0.05,0) -- (0.15,0);
% \draw[black,line width=1pt] (0.05,-2pt) -- (0.05,2pt);
% \draw[black,line width=1pt] (0.15,-2pt) -- (0.15,2pt);}
% \\



Capacity Only  & $57.642 \%$ 
% \; \tikz{
% \draw[gray,line width=.3pt] (0,0) -- (1.1,0);
% \draw[white, line width=0.01pt] (0,-2pt) -- (0,2pt);
% \draw[black,line width=1pt] (0.05,0) -- (0.15,0);
% \draw[black,line width=1pt] (0.05,-2pt) -- (0.05,2pt);
% \draw[black,line width=1pt] (0.15,-2pt) -- (0.15,2pt);}
& $2.862$  
% \; \tikz{
% \draw[gray,line width=.3pt] (0,0) -- (1.1,0);
% \draw[white, line width=0.01pt] (0,-2pt) -- (0,2pt);
% \draw[black,line width=1pt] (0.05,0) -- (0.15,0);
% \draw[black,line width=1pt] (0.05,-2pt) -- (0.05,2pt);
% \draw[black,line width=1pt] (0.15,-2pt) -- (0.15,2pt);}
\\

Coverage + Capacity  & $57.833 \%$ 
% \; \tikz{
% \draw[gray,line width=.3pt] (0,0) -- (1.1,0);
% \draw[white, line width=0.01pt] (0,-2pt) -- (0,2pt);
% \draw[black,line width=1pt] (0.05,0) -- (0.15,0);
% \draw[black,line width=1pt] (0.05,-2pt) -- (0.05,2pt);
% \draw[black,line width=1pt] (0.15,-2pt) -- (0.15,2pt);}
& $2.831$  
% \; \tikz{
% \draw[gray,line width=.3pt] (0,0) -- (1.1,0);
% \draw[white, line width=0.01pt] (0,-2pt) -- (0,2pt);
% \draw[black,line width=1pt] (0.05,0) -- (0.15,0);
% \draw[black,line width=1pt] (0.05,-2pt) -- (0.05,2pt);
% \draw[black,line width=1pt] (0.15,-2pt) -- (0.15,2pt);}
\\

Pathgain + Capacity  & $58.747 \%$ 
% \; \tikz{
% \draw[gray,line width=.3pt] (0,0) -- (1.1,0);
% \draw[white, line width=0.01pt] (0,-2pt) -- (0,2pt);
% \draw[black,line width=1pt] (0.05,0) -- (0.15,0);
% \draw[black,line width=1pt] (0.05,-2pt) -- (0.05,2pt);
% \draw[black,line width=1pt] (0.15,-2pt) -- (0.15,2pt);}
& $2.883$  
% \; \tikz{
% \draw[gray,line width=.3pt] (0,0) -- (1.1,0);
% \draw[white, line width=0.01pt] (0,-2pt) -- (0,2pt);
% \draw[black,line width=1pt] (0.05,0) -- (0.15,0);
% \draw[black,line width=1pt] (0.05,-2pt) -- (0.05,2pt);
% \draw[black,line width=1pt] (0.15,-2pt) -- (0.15,2pt);}
\\

Coverage Only & $\textbf{59.537} \% $ 
% \; \tikz{
% \draw[gray,line width=.3pt] (0,0) -- (1.1,0);
% \draw[white, line width=0.01pt] (0,-2pt) -- (0,2pt);
% \draw[black,line width=1pt] (0.14,0) -- (0.3,0);
% \draw[black,line width=1pt] (0.14,-2pt) -- (0.14,2pt);
% \draw[black,line width=1pt] (0.3,-2pt) -- (0.3,2pt);} 
& $2.884$  
% \; \tikz{
% \draw[gray,line width=.3pt] (0,0) -- (1.1,0);
% \draw[white, line width=0.01pt] (0,-2pt) -- (0,2pt);
% \draw[black,line width=1pt] (0.05,0) -- (0.15,0);
% \draw[black,line width=1pt] (0.05,-2pt) -- (0.05,2pt);
% \draw[black,line width=1pt] (0.15,-2pt) -- (0.15,2pt);}
\\

Pathgain + Coverage & $59.438 \%$ 
% \; \tikz{
% \draw[gray,line width=.3pt] (0,0) -- (1.1,0);
% \draw[white, line width=0.01pt] (0,-2pt) -- (0,2pt);
% \draw[black,line width=1pt] (0.05,0) -- (0.15,0);
% \draw[black,line width=1pt] (0.05,-2pt) -- (0.05,2pt);
% \draw[black,line width=1pt] (0.15,-2pt) -- (0.15,2pt);}
& $\textbf{2.893}$  
% \; \tikz{
% \draw[gray,line width=.3pt] (0,0) -- (1.1,0);
% \draw[white, line width=0.01pt] (0,-2pt) -- (0,2pt);
% \draw[black,line width=1pt] (0.05,0) -- (0.15,0);
% \draw[black,line width=1pt] (0.05,-2pt) -- (0.05,2pt);
% \draw[black,line width=1pt] (0.15,-2pt) -- (0.15,2pt);}
\\



\bottomrule[1pt]
\end{tabularx}




% \begin{tabularx}{1\linewidth}{l c }
% \toprule[1pt]
% {Scheme} & Collision Rate $C$  \\
% \cmidrule(lr){1-1} \cmidrule(lr){2-2} 


% Heuristic ($K=10$) & $0.0341$ (1.00x) \; \tikz{
% \draw[gray,line width=.3pt] (0,0) -- (1.1,0);
% \draw[white, line width=0.01pt] (0,-2pt) -- (0,2pt);
% \draw[black,line width=1pt] (0.14,0) -- (0.3,0);
% \draw[black,line width=1pt] (0.14,-2pt) -- (0.14,2pt);
% \draw[black,line width=1pt] (0.3,-2pt) -- (0.3,2pt);}
% % & $0.4941$\; \tikz{
% % \draw[gray,line width=.3pt] (0,0) -- (1.1,0);
% % \draw[white, line width=0.01pt] (0,-2pt) -- (0,2pt);
% % \draw[black,line width=1pt] (0.35,0) -- (0.55,0);
% % \draw[black,line width=1pt] (0.35,-2pt) -- (0.35,2pt);
% % \draw[black,line width=1pt] (0.55,-2pt) -- (0.55,2pt);}
% \\ 

% \textbf{PPO ($K=10$)}  & $0.0079$ (4.31x) \; \tikz{
% \draw[gray,line width=.3pt] (0,0) -- (1.1,0);
% \draw[white, line width=0.01pt] (0,-2pt) -- (0,2pt);
% \draw[black,line width=1pt] (0.05,0) -- (0.15,0);
% \draw[black,line width=1pt] (0.05,-2pt) -- (0.05,2pt);
% \draw[black,line width=1pt] (0.15,-2pt) -- (0.15,2pt);} 
% % & $0.9821$\; \tikz{
% % \draw[gray,line width=.3pt] (0,0) -- (1.1,0);
% % \draw[white, line width=0.01pt] (0,-2pt) -- (0,2pt);
% % \draw[black,line width=1pt] (0.9,0) -- (1.0,0);
% % \draw[black,line width=1pt] (0.9,-2pt) -- (0.9,2pt);
% % \draw[black,line width=1pt] (1.0,-2pt) -- (1.0,2pt);} 
% \\
% \midrule
% {} & Collision Rate $C$ \\
% \cmidrule(lr){1-1} \cmidrule(lr){2-2}


% Heuristic ($K=100$) & $0.2781$ (1.00x) \; \tikz{
% \draw[gray,line width=.3pt] (0,0) -- (1.1,0);
% \draw[white, line width=0.01pt] (0,-2pt) -- (0,2pt);
% \draw[black,line width=1pt] (0.34,0) -- (0.5,0);
% \draw[black,line width=1pt] (0.34,-2pt) -- (0.34,2pt);
% \draw[black,line width=1pt] (0.5,-2pt) -- (0.5,2pt);}
% % & $0.7219$\; \tikz{
% % \draw[gray,line width=.3pt] (0,0) -- (1.1,0);
% % \draw[white, line width=0.01pt] (0,-2pt) -- (0,2pt);
% % \draw[black,line width=1pt] (0.55,0) -- (0.70,0);
% % \draw[black,line width=1pt] (0.55,-2pt) -- (0.55,2pt);
% % \draw[black,line width=1pt] (0.70,-2pt) -- (0.70,2pt);}
% \\ 

% \textbf{PPO ($K=100$)}  & $0.1379$ (2.01x) \; \tikz{
% \draw[gray,line width=.3pt] (0,0) -- (1.1,0);
% \draw[white, line width=0.01pt] (0,-2pt) -- (0,2pt);
% \draw[black,line width=1pt] (0.25,0) -- (0.35,0);
% \draw[black,line width=1pt] (0.25,-2pt) -- (0.25,2pt);
% \draw[black,line width=1pt] (0.35,-2pt) -- (0.35,2pt);} 
% % & $0.8621$\; \tikz{
% % \draw[gray,line width=.3pt] (0,0) -- (1.1,0);
% % \draw[white, line width=0.01pt] (0,-2pt) -- (0,2pt);
% % \draw[black,line width=1pt] (0.8,0) -- (0.9,0);
% % \draw[black,line width=1pt] (0.8,-2pt) -- (0.8,2pt);
% % \draw[black,line width=1pt] (.9,-2pt) -- (0.9,2pt);} 
% \\


% \bottomrule[1pt]
% \end{tabularx}

\end{minipage}}
\caption{Reward design variations for single (static) BS deployment.}
\label{table:reward}
\vspace{-.5em}
\end{table}

This section presents empirical findings from experiments with different reward designs in the AutoBS framework. The pathgain is defined as $P = \sum_{{i,j} \in \mathcal{R}}{P_{i,j}}$, and Table \ref{table:reward} summarizes coverage and capacity outcomes for each reward configuration. Although other transformations, such as logarithmic scaling (\eg $\log{V}$), were tested, they are omitted here for brevity. These findings are empirical and sensitive to training hyperparameters (\eg learning rate) and reward design parameters (\eg $\nu$).

The results provide valuable insights. As expected, the Coverage Only reward achieves the highest coverage, aligning with its direct optimization objective (\eg $\nu_2, \nu_3 = 0$). Interestingly, in terms of capacity, the combination of Pathgain and Coverage (\ie $\nu_2=0$) outperforms the Capacity Only reward. Notably, Coverage Only also surpasses Capacity Only in capacity, highlighting complex dynamics within the DRL training process. These findings suggest that site-specific channel characteristics are implicitly embedded in rewards derived from PMNet’s pathloss maps. While Capacity Only smooths variations through logarithmic scaling, Pathgain captures these fluctuations on a linear scale, providing more granular feedback for effective learning.

Throughout this work, the Pathgain + Coverage reward configuration is primarily used.





