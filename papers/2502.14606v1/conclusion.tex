Automated testing of interactive software, such as 3D games, is challenging and has predominantly remained a manual task. In this paper, we have presented \approach, a curiosity driven \marlacronym approach for automated testing of 3D games. \approach adopts fully cooperative \marlacronym to explore the game world and achieve predefined coverage criteria. The experimental results show that \approach is effective and efficient compared to a baseline approach employing a single agent \rlacronym.

The work presented in this paper is an attempt to get insight into the use of multiple agents in the contest of testing 3D games. As such, it did not address all possible aspects related to the application of \marlacronym for testing 3D games. However, the results presented here serve as a good starting point for further exploring different dimensions. Hence, we envision a number of directions in which this work could be extended in future work. 

While in this work we adopted a cooperative \marlacronym scheme, other  agent schemes are worth exploring, such as a competitive scheme where agents compete against each other to achieve better performance. Moreover, fully cooperative agents where, unlike in this work, all agents are \emph{active} is also interesting to explore. In these cases, however, the environment could become non-stationary, which is not a trivial issue to manage. The number of agents deployed, beyond two, is another aspect that requires a comprehensive empirical exploration.

It is also worth investigating whether using deep \rlacronym could have an advantage over the tabular Q-learning algorithm used in this work, especially in cases where the size and complexity of the testing problem is higher. 

The current work mainly focused on achieving full coverage of the game level under test based on some predefined notions of coverage. It would be interesting to explore the possibility of identifying erratic behaviors that could result in a bug.

