

\begin{figure*}[!htb]
\includegraphics[width=1\textwidth]{figures/boxplot_all.png}
\caption{Entity coverage, entity connection coverage and time boxplots}
 \label{fig:Coverage-and-Time}
\end{figure*}


\begin{table*}[!htb]
{%\scriptsize
\begin{center}
\begin{tabular}{|ll|lll|lll|lll|}
\hline
 &    & \multicolumn{3}{|c|}{Entity coverage} & \multicolumn{3}{|c|}{Entity  connection coverage} & \multicolumn{3}{|c|}{Time} \tabularnewline
 Size & Id & P-value & Effect size & Magnitude & P-value & Effect size & Magnitude & P-value & Effect size & Magnitude \tabularnewline
\hline
\hline
Small&S1&\textless 0.0001&$1.000$&large&\textless 0.0001&$1.000$&large&0.00018&$1.000$&large\tabularnewline
Small&S2&\textless 0.0001&$1.000$&large&\textless 0.0001&$1.000$&large&0.00028&$0.000$&large\tabularnewline
Small&S3&\textless 0.0001&$1.000$&large&\textless 0.0001&$1.000$&large&0.00018&$0.000$&large\tabularnewline
Small&S4&0.37&NA&NA&NA&NA&NA&0.00018&$0.000$&large\tabularnewline
Small&S5&NA&NA&NA&NA&NA&NA&0.076&NA&NA\tabularnewline \hline

Medium&M1&\textless 0.0001&$1.000$&large&0.00017&$1.000$&large&\textless 0.0001&$0.000$&large\tabularnewline
Medium&M2&0.00011&$1.000$&large&0.00018&$1.000$&large&0.054&NA&NA\tabularnewline
Medium&M3&\textless 0.0001&$1.000$&large&0.00015&$1.000$&large&\textless 0.0001&$0.135$&large\tabularnewline \hline

Large&L1&0.0012&$0.900$&large&NA&NA&NA&0.00018&$0.000$&large\tabularnewline
Large&L2&0.00026&$0.985$&large&0.033&$0.775$&large&0.00018&$0.000$&large\tabularnewline
Large&L3&0.0014&$1.000$&large&0.0017&$0.920$&large&0.36&NA&NA\tabularnewline
Large&L4&0.91&NA&NA&0.00018&$1.000$&large&0.00025&$0.010$&large\tabularnewline
\hline
\end{tabular}\end{center}
}
\caption{Summary statistics}
\label{tab:Statistics}
\end{table*}


This section presents the results achieved by \approach and the baseline on each of the subjects of the experiment according to the metrics defined in the previous section. For effectiveness (\rqa) we report the coverage values achieved by the approaches for \emph{entity} and \emph{entity connection} coverage metrics. For \emph{spatial} coverage instead, we present only two heatmaps, because of space limitations. However, we make available an online appendix~\cite{online-appendix} with all data. For efficiency (\rqb) we report the time spent for each approach on the different levels of the game under test.

Figure~\ref{fig:Coverage-and-Time} shows the entity coverage for all the \lr levels in the experiment and the total time spent. Please note that one level for \med and two levels for \ext did not finish the executions, hence the plots reported in the figure are missing these three data points. Table~\ref{tab:Statistics} presents the results of the Wilcoxon test of statistical significance, both for coverage and time, as well as the effect size computed using the Vargha-Delaney statistic ($\hat{A}$)~\cite{DBLP:journals/stvr/ArcuriB14}.

%% Small size
\subsection{\med size levels}
As can be seen from Figure~\ref{fig:Coverage-and-Time} (top row), both entity and entity connection coverage values tend to be quite high for all \med size levels, except for level S1 where the baseline performs poorly. For levels S2 and S3, the baseline achieves high coverage but not as high as \approach. Looking at Table~\ref{tab:Statistics}, we can see that for the three levels (S1, S2, S3) the differences between the coverage achieved by \approach and the baseline are statistically significant with a large effect size. In the other two levels (S4, S5) there is no statistically significant difference.

Figure~\ref{fig:MediumHeatmap} shows the spatial coverage of one of the \med size levels (level S1 in Figure~\ref{fig:Coverage-and-Time}) where the stark contrast between the \approach and the baseline could be observed. We do not report all figures here due to lack of space, however, we make all data available in our online appendix~\cite{online-appendix}.

\begin{figure*}[!htb]
	\hspace*{\fill}%
	\subcaptionbox{\approach}{\includegraphics[width=0.48\textwidth]{figures/HeatMap-Multi-levelS1.png}}\hfill%
	\subcaptionbox{baseline}{\includegraphics[width=0.48\textwidth]{figures/HeatMap-Single-levelS1.png}}%
	\hspace*{\fill}%
	\caption{Spatial coverage for level S1 of \med size. The darker the color the less explored the area.}
	\label{fig:MediumHeatmap}
\end{figure*}
Looking at the time spent by both approaches on all of the levels (Figure~\ref{fig:Coverage-and-Time}; top row, last column), we can see that except level S1, where the baseline takes less time than \approach, in all the other cases \approach is in general faster than the baseline. The differences are also statistically significant (see Table~\ref{tab:Statistics}, 'Time' column) with the exception of S5 where the two approaches spend a comparable amount of time.

%% Medium size
\subsection{\lrg size levels}
Differently from the \med size levels, the coverage achieved on the \lrg levels is generally lower (Figure~\ref{fig:Coverage-and-Time}, middle row). In particular, the baseline achieves low entity (below 60\%) and entity connection (below 50\%) coverage, while \approach achieves higher coverage (above 70\% entity and 50\% entity connection) but it is still low compared to the coverage achieved on the \med size levels.

Figure~\ref{fig:LargeHeatmap} shows the spatial coverage of one of the \lrg size levels (level M3 in Figure~\ref{fig:Coverage-and-Time}) where we observe the marked difference between the \approach and the baseline.

The total time spent for the three levels of \lrg size (Figure~\ref{fig:Coverage-and-Time}; middle row, last column) shows a similar trend as that of \med size levels where \approach is faster than the baseline, and in two cases (levels M1 and M3) the difference is statistically significant with large effect size (see Table~\ref{tab:Statistics}; middle row, 'Time' column). 

\begin{figure*}[!htb]
	\hspace*{\fill}%
	\subcaptionbox{\approach}{\includegraphics[width=0.48\textwidth]{figures/HeatMap-Multi-levelM3.png}}\hfill%
	\subcaptionbox{baseline}{\includegraphics[width=0.48\textwidth]{figures/HeatMap-Single-levelM3.png}}%
	\hspace*{\fill}%
	\caption{Spatial coverage for level M3 of \lrg size. The darker the color the less explored the area.}
	\label{fig:LargeHeatmap}
\end{figure*}

%% Large size
\subsection{\ext size levels}
As expected, the coverage achieved for levels of \ext level is lower than those of \med. As can be seen from Figure~\ref{fig:Coverage-and-Time} (bottom row), the maximum coverage achieved is around 55\%. Consistent with \med and \lrg levels, the coverage achieved by \approach is higher than that of the baseline. It is important to note that we have increased the budget allocated (number of episodes and number of actions per episode, see Table~\ref{tab:parameters}) since the levels are quite large. However the achieved coverage is low regardless, and hence future experimentation with larger budgets could be useful.

Figure~\ref{fig:ExtremeHeatmap} shows the spatial coverage for one of the \ext size levels (level L3 in Figure~\ref{fig:Coverage-and-Time}). As can be seen from the heatmap, the level is quite large and the agent (player) would need to cover a lot of ground to achieve full coverage.

\begin{figure*}[!htb]
	\hspace*{\fill}%
	\subcaptionbox{\approach}{\includegraphics[width=0.48\textwidth]{figures/HeatMap-Multi-levelL3.png}}\hfill%
	\subcaptionbox{baseline}{\includegraphics[width=0.48\textwidth]{figures/HeatMap-Single-levelL3.png}}%
	\hspace*{\fill}%
	\caption{Spatial coverage for level L3 of \ext size. The darker the color the less explored the area.}
	\label{fig:ExtremeHeatmap}
\end{figure*}
Looking at the time spent for the four levels of \ext size (see Table~\ref{tab:Statistics}; bottom row, `Time' column), we observe a similar trend as the \med and \lrg size levels where in three of the levels (L1, L2, and L4) \approach is statistically significantly faster than the baseline where as for L3 there is no statistically significant difference in the time spent by the two approaches. 

\subsection{Discussion}
Overall the results show that \approach performs better than the single-agent variant, both in terms of coverage and time spent. Given the complexity of 3D games and the limitation due to partial observability, a single agent would need a lot of effort to effectively explore the game world. On the other hand, deploying multiple agents drastically improves performance since the active agent gets more accurate and immediate feedback on its actions. This tends to reduce the chances of the agent being `stuck' in certain areas of the game (see Figure~\ref{fig:MediumHeatmap}). Based on these results we can answer positively the first research question:
the \approach is \emph{more effective} than the single-agent variant.

Regarding the efficiency, i.e., time spent, the results are somehow unexpected in that \approach tends to consume less time compared to the baseline. Given the fact that two agents are running and exchanging information on actions performed, we had expected the time spent by \approach to be higher compared to the baseline. However, the results show the opposite. For the simpler levels (e.g., those of \med size) where \approach is able to achieve full coverage, the time spent is clearly less than the maximum allowed (since testing stops when full coverage is reached). On the other hand when full coverage is not achieved, it appears that in \approach the agent is less likely to get `stuck' and so less likely to waste more time trying to get out (see for example Figure~\ref{fig:MediumHeatmap}). Based on the results, for what concerns the second research question as well, we can answer positively that \approach is also \emph{more efficient} than the baseline.
