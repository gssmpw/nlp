

Recent years have seen remarkable progress in \rlacronym in solving immensely challenging multi-player real strategy video games \cite{vinyals2019grandmasterStarCraft,zhou2021hierarchical,OpenAIHideandSeekbaker2019,silver2017masteringGo}. The aim here is for an agent to achieve phenomenal skills to finish these games either by collaborating or competing with other agents. 
The promising result of \rlacronym in game play opens the possibility of using such solutions for automated testing of games. A few research work is found using single agent \rlacronym for game play testing and coverage~\cite{bergdahl2020augmenting, sestini2022automated, zheng2019wuji, holmgaard2018automated, ariyurek2019automated, tufano2022using, borovikov2019winning}.

Despite the growing success of \marlacronym, considering the inherent challenges of \marlacronym compared to its single agent counterpart, there is little work towards using \marlacronym architecture in automated game testing and coverage. 
One research work is found~\cite{gordillo2021improving}  
where the idea is to use multiple agents for maximizing game coverage. The solution uses multiple \rlacronym agents learning in a distributed fashion. Each agent is motivated by curiosity during learning. Achieving convergence in a distributed \marlacronym architecture is notoriously challenging, thus the authors do not focus on techniques for optimizing the training of agents in a distributed structure, rather they focus on collecting and combining the exploration data to measure the coverage. 
The use of \marlacronym in the field of automated game testing and coverage requires more research and our work goes in this direction where we show that employing two fully cooperating agents tends to achieve better coverage. Different agent configurations are the subject of future work.