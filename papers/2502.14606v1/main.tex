
\documentclass[conference]{IEEEtran}

%\usepackage{cite}
\usepackage{amsmath,amssymb,amsfonts}
\usepackage{amssymb,mathtools}
\usepackage{algorithm,algpseudocode}
\usepackage{enumitem}
\usepackage{graphicx}
\usepackage{textcomp}
\usepackage{xcolor}
\usepackage{xspace}
\usepackage{todonotes}
\usepackage{enumitem}
\usepackage{url}
\usepackage{hyperref}
\usepackage{subcaption}
% \usepackage{subfigure}
\usepackage[export]{adjustbox}
\usepackage{flushend}
\usepackage[T1]{fontenc}
%\def\BibTeX{{\rm B\kern-.05em{\sc i\kern-.025em b}\kern-.08em
%    T\kern-.1667em\lower.7ex\hbox{E}\kern-.125emX}}

% \usepackage{color}
%\renewcommand\UrlFont{\color{blue}\rmfamily}
%

\newcommand{\aagent}{\texttt{active agent}\xspace}
\newcommand{\pagent}{\texttt{passive agent}\xspace}
\newcommand{\approach}{\texttt{cMarlTest}\xspace}
\newcommand{\tool}{\texttt{RLbT}\xspace}
\newcommand{\curiosityrl}{\texttt{curiosity-based RL}\xspace}
\newcommand{\sparserl}{\texttt{sparse-reward RL}\xspace}
\newcommand{\encov}{\texttt{Entity Coverage}\xspace}
\newcommand{\enconcov}{\texttt{Entity Connection Coverage}\xspace}
\newcommand{\fourrooms}{\texttt{4-room}\xspace}
\newcommand{\eightrooms}{\texttt{8-room}\xspace}
\newcommand{\randomlarge}{\texttt{LargeMaze\_1}\xspace}
\newcommand{\randommedium}{\texttt{LargeMaze\_2}\xspace}
\newcommand{\rl}{reinforcement learning\xspace}
\newcommand{\RL}{Reinforcement Learning\xspace}
\newcommand{\MARL}{Multi-Agent Reinforcement Learning\xspace}
\newcommand{\marlacronym}{MARL\xspace}
\newcommand{\rlacronym}{RL\xspace}
\newcommand{\SARL}{\texttt{Single-Agent Reinforcement Learning}\xspace}
\newcommand{\sut}{\texttt{Lab Recruits}\xspace}
\newcommand{\lr}{\texttt{Lab Recruits}\xspace}

\newcommand{\buttonsdoors}{\texttt{L1}\xspace}
\newcommand{\med}{\texttt{Small}\xspace}
\newcommand{\lrg}{\texttt{Medium}\xspace}
\newcommand{\ext}{\texttt{Large}\xspace}

\newcommand{\rqa}{\textbf{RQ1}\xspace}
\newcommand{\rqaa}{\textbf{RQ1.1}\xspace}
\newcommand{\rqb}{\textbf{RQ2}\xspace}
\newcommand{\rqbb}{\textbf{RQ2.1}\xspace}


\newcommand*{\MyIndent}{\hspace*{0.5cm}}%
\newcommand\FK[1]{\textcolor{blue}{#1}}
\newcommand\RF[1]{\textcolor{blue}{#1}}
\newcommand\EXAMPLE[1]{\textcolor{brown}{#1}}


%%
%% end of the preamble, start of the body of the document source.
\setlength {\marginparwidth }{2cm}
\begin{document}

%%
%% The "title" command has an optional parameter,
%% allowing the author to define a "short title" to be used in page headers.
\title{Curiosity Driven Multi-agent Reinforcement Learning for 3D Game Testing}


\author{\IEEEauthorblockN{Raihana Ferdous}
\IEEEauthorblockA{\textit{Consiglio Nazionale delle Ricerche} \\
Pisa, Italy \\
raihana.ferdous@isti.cnr.it}
\and
\IEEEauthorblockN{Fitsum Kifetew}
\IEEEauthorblockA{\textit{Fondazione Bruno Kessler} \\
Trento, Italy \\
kifetew@fbk.eu}
\and
\IEEEauthorblockN{Davide Prandi}
\IEEEauthorblockA{\textit{Fondazione Bruno Kessler} \\
Trento, Italy \\
prandi@fbk.eu}
\and
\IEEEauthorblockN{Angelo Susi}
\IEEEauthorblockA{\textit{Fondazione Bruno Kessler} \\
Trento, Italy \\
susi@fbk.eu}
}


%%
%% This command processes the author and affiliation and title
%% information and builds the first part of the formatted document.
\maketitle



\begin{abstract} 
Recently testing of games via autonomous agents has shown great promise in tackling challenges faced by the game industry, which mainly relied on either manual testing or record/replay. In particular \RL (\rlacronym) solutions have shown potential by learning directly from playing the game without the need for human intervention.

In this paper, we present \approach, an approach for testing 3D games through curiosity driven \MARL (\marlacronym). \approach deploys multiple agents that work collaboratively to achieve the testing objective.
%and exercise the various entities/functionalities of the game. 
The use of multiple agents helps resolve issues faced by a single agent approach.

We carried out experiments on different levels of a 3D game comparing the performance of \approach with a single agent \rlacronym variant. Results are promising where, considering three different types of coverage criteria, \approach achieved higher coverage. \approach was also more efficient in terms of the time taken, with respect to the single agent based variant.

\end{abstract}


\begin{IEEEkeywords}
Curiosity driven Reinforcement learning, game testing, coverage based testing
\end{IEEEkeywords}



\section{Introduction} \label{sec:introduction}
\section{Introduction}
\label{sec:introduction}
The business processes of organizations are experiencing ever-increasing complexity due to the large amount of data, high number of users, and high-tech devices involved \cite{martin2021pmopportunitieschallenges, beerepoot2023biggestbpmproblems}. This complexity may cause business processes to deviate from normal control flow due to unforeseen and disruptive anomalies \cite{adams2023proceddsriftdetection}. These control-flow anomalies manifest as unknown, skipped, and wrongly-ordered activities in the traces of event logs monitored from the execution of business processes \cite{ko2023adsystematicreview}. For the sake of clarity, let us consider an illustrative example of such anomalies. Figure \ref{FP_ANOMALIES} shows a so-called event log footprint, which captures the control flow relations of four activities of a hypothetical event log. In particular, this footprint captures the control-flow relations between activities \texttt{a}, \texttt{b}, \texttt{c} and \texttt{d}. These are the causal ($\rightarrow$) relation, concurrent ($\parallel$) relation, and other ($\#$) relations such as exclusivity or non-local dependency \cite{aalst2022pmhandbook}. In addition, on the right are six traces, of which five exhibit skipped, wrongly-ordered and unknown control-flow anomalies. For example, $\langle$\texttt{a b d}$\rangle$ has a skipped activity, which is \texttt{c}. Because of this skipped activity, the control-flow relation \texttt{b}$\,\#\,$\texttt{d} is violated, since \texttt{d} directly follows \texttt{b} in the anomalous trace.
\begin{figure}[!t]
\centering
\includegraphics[width=0.9\columnwidth]{images/FP_ANOMALIES.png}
\caption{An example event log footprint with six traces, of which five exhibit control-flow anomalies.}
\label{FP_ANOMALIES}
\end{figure}

\subsection{Control-flow anomaly detection}
Control-flow anomaly detection techniques aim to characterize the normal control flow from event logs and verify whether these deviations occur in new event logs \cite{ko2023adsystematicreview}. To develop control-flow anomaly detection techniques, \revision{process mining} has seen widespread adoption owing to process discovery and \revision{conformance checking}. On the one hand, process discovery is a set of algorithms that encode control-flow relations as a set of model elements and constraints according to a given modeling formalism \cite{aalst2022pmhandbook}; hereafter, we refer to the Petri net, a widespread modeling formalism. On the other hand, \revision{conformance checking} is an explainable set of algorithms that allows linking any deviations with the reference Petri net and providing the fitness measure, namely a measure of how much the Petri net fits the new event log \cite{aalst2022pmhandbook}. Many control-flow anomaly detection techniques based on \revision{conformance checking} (hereafter, \revision{conformance checking}-based techniques) use the fitness measure to determine whether an event log is anomalous \cite{bezerra2009pmad, bezerra2013adlogspais, myers2018icsadpm, pecchia2020applicationfailuresanalysispm}. 

The scientific literature also includes many \revision{conformance checking}-independent techniques for control-flow anomaly detection that combine specific types of trace encodings with machine/deep learning \cite{ko2023adsystematicreview, tavares2023pmtraceencoding}. Whereas these techniques are very effective, their explainability is challenging due to both the type of trace encoding employed and the machine/deep learning model used \cite{rawal2022trustworthyaiadvances,li2023explainablead}. Hence, in the following, we focus on the shortcomings of \revision{conformance checking}-based techniques to investigate whether it is possible to support the development of competitive control-flow anomaly detection techniques while maintaining the explainable nature of \revision{conformance checking}.
\begin{figure}[!t]
\centering
\includegraphics[width=\columnwidth]{images/HIGH_LEVEL_VIEW.png}
\caption{A high-level view of the proposed framework for combining \revision{process mining}-based feature extraction with dimensionality reduction for control-flow anomaly detection.}
\label{HIGH_LEVEL_VIEW}
\end{figure}

\subsection{Shortcomings of \revision{conformance checking}-based techniques}
Unfortunately, the detection effectiveness of \revision{conformance checking}-based techniques is affected by noisy data and low-quality Petri nets, which may be due to human errors in the modeling process or representational bias of process discovery algorithms \cite{bezerra2013adlogspais, pecchia2020applicationfailuresanalysispm, aalst2016pm}. Specifically, on the one hand, noisy data may introduce infrequent and deceptive control-flow relations that may result in inconsistent fitness measures, whereas, on the other hand, checking event logs against a low-quality Petri net could lead to an unreliable distribution of fitness measures. Nonetheless, such Petri nets can still be used as references to obtain insightful information for \revision{process mining}-based feature extraction, supporting the development of competitive and explainable \revision{conformance checking}-based techniques for control-flow anomaly detection despite the problems above. For example, a few works outline that token-based \revision{conformance checking} can be used for \revision{process mining}-based feature extraction to build tabular data and develop effective \revision{conformance checking}-based techniques for control-flow anomaly detection \cite{singh2022lapmsh, debenedictis2023dtadiiot}. However, to the best of our knowledge, the scientific literature lacks a structured proposal for \revision{process mining}-based feature extraction using the state-of-the-art \revision{conformance checking} variant, namely alignment-based \revision{conformance checking}.

\subsection{Contributions}
We propose a novel \revision{process mining}-based feature extraction approach with alignment-based \revision{conformance checking}. This variant aligns the deviating control flow with a reference Petri net; the resulting alignment can be inspected to extract additional statistics such as the number of times a given activity caused mismatches \cite{aalst2022pmhandbook}. We integrate this approach into a flexible and explainable framework for developing techniques for control-flow anomaly detection. The framework combines \revision{process mining}-based feature extraction and dimensionality reduction to handle high-dimensional feature sets, achieve detection effectiveness, and support explainability. Notably, in addition to our proposed \revision{process mining}-based feature extraction approach, the framework allows employing other approaches, enabling a fair comparison of multiple \revision{conformance checking}-based and \revision{conformance checking}-independent techniques for control-flow anomaly detection. Figure \ref{HIGH_LEVEL_VIEW} shows a high-level view of the framework. Business processes are monitored, and event logs obtained from the database of information systems. Subsequently, \revision{process mining}-based feature extraction is applied to these event logs and tabular data input to dimensionality reduction to identify control-flow anomalies. We apply several \revision{conformance checking}-based and \revision{conformance checking}-independent framework techniques to publicly available datasets, simulated data of a case study from railways, and real-world data of a case study from healthcare. We show that the framework techniques implementing our approach outperform the baseline \revision{conformance checking}-based techniques while maintaining the explainable nature of \revision{conformance checking}.

In summary, the contributions of this paper are as follows.
\begin{itemize}
    \item{
        A novel \revision{process mining}-based feature extraction approach to support the development of competitive and explainable \revision{conformance checking}-based techniques for control-flow anomaly detection.
    }
    \item{
        A flexible and explainable framework for developing techniques for control-flow anomaly detection using \revision{process mining}-based feature extraction and dimensionality reduction.
    }
    \item{
        Application to synthetic and real-world datasets of several \revision{conformance checking}-based and \revision{conformance checking}-independent framework techniques, evaluating their detection effectiveness and explainability.
    }
\end{itemize}

The rest of the paper is organized as follows.
\begin{itemize}
    \item Section \ref{sec:related_work} reviews the existing techniques for control-flow anomaly detection, categorizing them into \revision{conformance checking}-based and \revision{conformance checking}-independent techniques.
    \item Section \ref{sec:abccfe} provides the preliminaries of \revision{process mining} to establish the notation used throughout the paper, and delves into the details of the proposed \revision{process mining}-based feature extraction approach with alignment-based \revision{conformance checking}.
    \item Section \ref{sec:framework} describes the framework for developing \revision{conformance checking}-based and \revision{conformance checking}-independent techniques for control-flow anomaly detection that combine \revision{process mining}-based feature extraction and dimensionality reduction.
    \item Section \ref{sec:evaluation} presents the experiments conducted with multiple framework and baseline techniques using data from publicly available datasets and case studies.
    \item Section \ref{sec:conclusions} draws the conclusions and presents future work.
\end{itemize}

\section{Running Example} \label{sec:runningexample}

This section introduces an example taken from \sut, an open source 3D game\footnote{\url{https://github.com/iv4xr-project/labrecruits}} where the player navigates a maze-like environment composed of rooms and corridors connected by doors. The doors are connected to buttons that open/close one or more doors. 
The game level may also contain other entities, such as 
%There could be other entities in a game level such as 
desks, chairs, fire hazards, extinguishers, etc. \sut allows the user to design and load a game level. Our example, illustrated in Figure~\ref{fig:runningexample}, is a simple level in \sut composed of three doors (door1, door2, door3), four buttons (bttn1, bttn2, bttn3, bttn4), and other non interactable entities such as desks. \sut could be played by either human players or test agents~\cite{prasetya2020aplib} by controlling the avatar (labelled as `agent' in Figure~\ref{fig:runningexample}). Our approach uses the test agents for interacting with the game.


\begin{figure}[tb]
    \begin{center}
    %\includegraphics[width=0.48\textwidth]{figures/button_doors1_edited.png}
    \includegraphics[width=0.48\textwidth]{figures/button_doors_1_variant.png}
    %\includegraphics[width=8.3cm,height=4cm]{gameInDev.png}
    \end{center}
    % \vspace{-3mm}
    \caption{Level \buttonsdoors~in \sut.} 
    \label{fig:runningexample}
\end{figure}

\section{\approach: Curiosity Driven \marlacronym for Game Testing} \label{sec:approach}
\section{Temporal Representation Alignment}
\label{sec:approach}

When training a series of short-horizon goal-reaching and instruction-following tasks, our goal is to learn a representation space such that our policy can generalize to a new (long-horizon) task that can be viewed as a sequence of known subtasks.
We propose to structure this representation space by aligning the representations of states, goals, and language in a way that is more amenable to compositional generalization.

\paragraph{Notation.}
We take the setting of a goal- and language-conditioned MDP $\cM$ with state space $\cS$, continuous action space $\cA \subseteq (0,1)^{d_{\cA}}$, initial state distribution $p_0$, dynamics $\p(s'\mid s,a)$, discount factor $\gamma$, and language task distribution $p_{\ell}$.
A policy $\pi(a\mid s)$ maps states to a distribution over actions. We inductively define the $k$-step (action-conditioned) policy visitation distribution as:
\begin{align*}
    p^{\pi}_{1}(s_{1} \mid s_1, a_{1})
    &\triangleq p(s_1 \mid s_1, a_1),\\
    p^{\pi}_{k+1}(s_{k+1} \mid s_1, a_1)
    &\triangleq \nonumber\\*
      &\mspace{-120mu} \int_{\cA}\int_{\cS} p(s_{k+1} \mid s,a) \dd p^{\pi}_{k}(s \mid s_{1},a_1) \dd
        \pi(a \mid s)\\
    p^{\pi}_{k+t}(s_{k+t} \mid s_t,a_t)
    &\triangleq p^{\pi}(s_{k} \mid s_1, a_1) . \eqmark
        \label{eq:successor_distribution}
\end{align*}
Then, the discounted state visitation distribution can be defined as the distribution over $s^{+}$\llap, the state reached after $K\sim \operatorname{Geom}(1-\gamma)$ steps:
\begin{equation}
    p^{\pi}_{\gamma}(s^{+}  \mid  s,a) \triangleq \sum_{k=0}^{\infty} \gamma^{k} p^{\pi}_{k}(s^{+} \mid s,a).
    \label{eq:discounted_state_visitation}
\end{equation}

We assume access to a dataset of expert demonstrations $\cD = \{\tau_{i},\ell_i\}_{i=1}^{K}$, where each trajectory
\begin{equation}
    \tau_{i} = \{s_{t,i},a_{t,i}\}_{t=1}^{H} \in \cS \times \cA
    \label{eq:trajectory}
\end{equation}
is gathered by an expert policy $\expert$, and is then annotated with $p_{\ell}(\ell_{i} \mid s_{1,i}, s_{H,i})$.
Our aim is to learn a policy $\pi$ that can select actions conditioned on a new language instruction $\ell$.
As in prior work~\citep{walke2023bridgedata}, we handle the continuous action space by representing both our policy and the expert policy as an isotropic Gaussian with fixed variance; we will equivalently write $\pi(a\mid s, \varphi)$ or denote the mode as $\hat{a} = \pi(s,\varphi)$ for a task $\varphi$.

\begin{rebuttal}
    \subsection{Representations for Reaching Distant Goals}
    \label{sec:reaching_goals}

    We learn a goal-conditioned policy $\pi(a\mid s,g)$ that selects actions to reach a goal $g$ from expert demonstrations with behavioral cloning.
    Suppose we directly selected actions to imitate the expert on two trajectories in $\cD$:
    
    \begin{equation}
        \mspace{-100mu}\begin{tikzcd}[remember picture,sep=small]
            s_1 \rar & s_2 \rar  & \ldots \rar & s_{H} \rar & w      \quad \\
            w \rar   & s_1' \rar & \ldots \rar & s_{H}' \rar & g\quad
        \end{tikzcd}
        \begin{tikzpicture}[remember picture,overlay] \coordinate (a) at (\tikzcdmatrixname-1-5.north east);
            \coordinate (b) at (\tikzcdmatrixname-2-5.south east);
            \coordinate (c) at (a|-b);
            \draw[decorate,line width=1.5pt,decoration={brace,raise=3pt,amplitude=5pt}]
        (a) -- node[right=1.5em] {$\tau_{i}\in \cD$} (c); \end{tikzpicture}
        \label{eq:trajectory_diagram}
    \end{equation}
    When conditioned with the composed goal $g$, we would be unable to imitate effectively
        as the composed state-goal $(s,g)$ is jointly out of the training distribution.

    What \emph{would} work for reaching $g$ is to first condition the policy on the intermediate waypoint $w$, then upon reaching $w$, condition on the goal $g$, as the state-goal pairs $(s_{i},w)$, $(w,g)$, and $(s_{i}',g)$ are all in the training distribution.
    If we condition the policy on some intermediate waypoint distribution $p(w)$ (or sufficient statistics thereof) that captures all of these cases, we can stitch together the expert behaviors to reach the goal $g$.

    Our approach is to learn a representation space that captures this ability, so that a GCBC objective used in this space can effectively imitate the expert on the composed task.
     We begin with the goal-conditioned behavioral cloning~\citep{kaelbling1993learning}
        loss $\cL_{\textsc{bc}}^{\phi,\psi,\xi}$ conditioned with waypoints $w$.
    \begin{equation}
        \cL_{\textsc{bc}}\bigl(\{s_{i},a_{i},s_{i}^{+},g_{i}\}_{i=1}^{K}\bigr) = \sum_{{i=1}}^{K} \log \pi\bigl(a_{i} \mid s_{i},\psi(g_{i})\bigr).
        \label{eq:goal_conditioned_bc}
    \end{equation}
    Enforcing the invariance needed to stitch \cref{eq:trajectory_diagram} then reduces to aligning \mbox{$\psi(g) \leftrightarrow \psi(w).$}
    The temporal alignment objective $\phi(s)\leftrightarrow \phi(s^{+})$ accomplishes this indirectly by aligning both $\psi(w)$ and $\psi(g)$ to the shared waypoint representation $\phi(w)$:

    \csuse{color indices}
    \begin{align}
        &\cL_{\textsc{nce}}\bigl(\{s_{i},s_{i}^{+}\}_{i=1}^{K};\phi,\psi\bigr) =
        \log \biggl( {\frac{e^{\phi(s^+_{\i})^{T}\psi(s_{\i})}}{\sum_{{\j=1}}^{K}
                e^{\phi(s^+_{\i})^{T}\psi(s_{\j})}}} \biggr)  \nonumber\\*
                &\mspace{100mu} +
        \sum_{{\j=1}}^{K} \log \biggl( {\frac{e^{\phi(s^+_{\i})^{T}\psi(s_{\i})}}{\sum_{{\i=1}}^{K}
                e^{\phi(s^+_{\i})^{T}\psi(s_{\j})}}} \biggr)
        \label{eq:goal_alignment}
    \end{align}

        
\end{rebuttal}
\subsection{Interfacing with Language Instructions}
\label{sec:language_instructions}

To extend the representations from \cref{sec:reaching_goals} to compositional instruction following with language tasks, we need some way to ground language into the $\psi$ (future state)
representation space.
We use a similar approach to GRIF~\citep{myers2023goal}, which uses an additional CLIP-style \citep{radford2021learning} contrastive alignment loss with an additional pretrained language encoder $\xi$:
\csuse{no color indices}
\begin{align}
    &\cL_{\textsc{nce}}\bigl(\{g_{i},\ell_{i}\}_{i=1}^{K};\psi,\xi\bigr)
    = \sum_{{i=1}}^{K} \log \biggl( {\frac{e^{\psi(g_{\i})^{T}\xi(\ell_{\i})}}{\sum_{{\j=1}}^{K}
            e^{\psi(g_{\i})^{T}\xi(\ell_{\j})}}} \biggr)  \nonumber\\*
            &\mspace{100mu} +
    \sum_{{\j=1}}^{K} \log \biggl( {\frac{e^{\psi(g_{\i})^{T}\xi(\ell_{\i})}}{\sum_{{\i=1}}^{K}
            e^{\psi(g_{\i})^{T}\xi(\ell_{\j})}}} \biggr)
    \label{eq:task_alignment}
\end{align}

\subsection{Temporal Alignment}
\label{sec:temporal_alignment}

Putting together the objectives from \cref{sec:reaching_goals,sec:language_instructions} yields the Temporal Representation Alignment (\Method) approach.
\Method{} structures the representation space of goals and language instructions to better enable compositional generalization.
We learn encoders $\phi, \psi ,$ and $\xi$ to map states, goals, and language instructions to a shared representation space.

\csuse{color indices}
\begin{align}
    \cL_{\textsc{nce}} \label{eq:NCE}
    &(\{x_{i}, y_{i}\}_{i=1}^{K};f,h) =
        \sum_{{\i=1}}^{K} \log \biggl( {\frac{e^{f(y_{\i})^{T}h(x_{\i})}}{\sum_{{\j=1}}^{K}
        e^{f(y_{\i})^{T}h(x_{\j})}}} \biggr) \nonumber\\*
      &\mspace{100mu} +
        \sum_{{\j=1}}^{K} \log \biggl( {\frac{e^{f(y_{\i})^{T}h(x_{\i})}}{\sum_{{\i=1}}^{K}
        e^{f(y_{\i})^{T}h(x_{\j})}}} \biggr) \\
    \cL_{\textsc{bc}} \label{eq:BC}
    &\bigl(\{s_{i},a_{i},s^{+}_{i},\ell_{i}\}_{i=1}^{K};\pi,\psi,\xi\bigr) = \nonumber\\*
      &\mspace{-10mu} \sum_{{i=1}}^{K} \log
        \pi\bigl(a_{i} \mid s_{i},\xi(\ell_{i})\bigr) + \log \pi\bigl(a_{i} \mid
        s_{i},\psi(s^{+}_{i})\bigr) \\
    \cL_{\textsc{tra}}
    &\label{eq:TRA} \bigl( \{s_{i},a_{i},s_{i}^{+},g_{i},\ell_{i}\}_{i=1}^{K}; \pi,\phi,\psi,\xi\bigr)
        \\
    &= \underbrace{\cL_{\textsc{bc}}\bigl(\{s_{i},a_{i},s_{i}^{+},\ell_{i}\}_{i=1}^{K};\pi,\psi,\xi\bigr)}_{\text{behavioral
    cloning}} \nonumber\\*
    &+
        \underbrace{\cL_{\textsc{nce}}\bigl(\{s_{i},s_{i}^{+}\}_{i=1}^{K};\phi,\psi\bigr)}_{\text{temporal alignment}}
        + \underbrace{\cL_{\textsc{nce}}\bigl(\{g_{i},\ell_{i}\}_{i=1}^{K};\psi,\xi\bigr)}_{\text{task alignment}} \nonumber
\end{align}Note that the NCE alignment loss uses a CLIP-style symmetric contrastive objective~\citep{radford2021learning,eysenbach2024inference} \-- we highlight the indices in the NCE alignment loss~\eqref{eq:NCE} for clarity.

Our overall objective is to minimize \cref{eq:TRA} across states, actions, future states, goals, and language tasks within the training data:
\begin{align}
    &\min_{\pi,\phi,\psi,\xi} \mathbb{E}_{\substack{(s_{1,i},a_{1,i},\ldots,s_{H,i},a_{H,i},\ell) \sim \mathcal{D} \\
    i\sim\operatorname{Unif}(1\ldots H) \\
    k\sim\operatorname{Geom}(1-\gamma)}} \\*
    &\mspace{10mu}
    \Bigl[\cL_{\text{TRA}}\bigl(\{s_{t,i},a_{t,i},s_{\min(t+k,H),i},s_{H,i},\ell\}_{i=1}^{K};\pi,\phi,\psi,\xi\bigr)\Bigr].
    \label{eq:overall_objective}
\end{align}

\begin{algorithm}
    \caption{Temporal Representation Alignment}
    \label{alg:tra}
    \begin{algorithmic}[1]
        \State \textbf{input:} dataset $\mathcal{D} = (\{s_{t,i},a_{t,i}\}_{t=1}^{H},\ell_i)_{i=1}^N$
        \State initialize networks $\Theta \triangleq (\pi,\phi,\psi,\xi)$
        \While{training}
        \State sample batch $\bigl\{(s_{t,i},a_{t,i},s_{t+k,i},\ell_i)\bigr\}_{i=1}^K\sim\mathcal{D}$ \\
        \hspace*{2ex} for $k\sim\operatorname{Geom}(1-\gamma)$
        \State $\Theta \gets \Theta - \alpha \nabla_{\Theta} \cL_{\text{TRA}}\bigl(\{s_{t,i},a_{t,i},s_{t+k,i},\ell_i\}_{i=1}^K; \Theta\bigr)$
        \EndWhile
        \smallskip
        \State \textbf{output:} \parbox[t]{\linewidth}{language-conditioned policy $\pi(a_{t} | s_{t}, \xi(\ell))$ \\
            goal-conditioned policy $\pi(a_{t} | s_{t}, \psi(g))$
        }
    \end{algorithmic}
\end{algorithm}

\subsection{Implementation}
\label{sec:implementation}

A summary of our approach is shown in \cref{alg:tra}.
In essence, TRA learns three encoders: $\phi$, which encodes states, $\psi$ which encodes future goals, and $\xi$ which encodes language instructions.
Contrastive losses are used to align state representations $\phi(s_{t})$ with future goal representations $\psi(s_{t+k})$, which are in turn aligned with equivalent language task specifications $\xi(\ell)$ when available.
We then learn a behavior cloning policy $\pi$ that can be conditioned on either the goal or language instruction through the representation $\psi(g)$ or $\xi(\ell)$, respectively.

\begin{rebuttal}
    \subsection{Temporal Alignment and Compositionality}
    \label{sec:compositionality}

    We will formalize the intuition from \cref{sec:reaching_goals} that \Method{} enables compositional generalization by considering the error on a ``compositional'' version of $\cD,$ denoted $\cD^{*}$.
    Using the notation from \cref{eq:trajectory}, we can say $\cD$ is distributed according to:
    \begin{align}
        &\cD \triangleq \cD^{H} \sim \prod_{i=1}^{K} p_0(s_{1,i}) p_{\ell}(\ell_{i} \mid s_{1,i}, s_{H,i})
            \nonumber\\*
          &\mspace{60mu} \prod_{t=1}^{H} \expert(a_{t,i} \mid s_{t,i}) \p(s_{t+1,i} \mid s_{t,i}, a_{t,i}) ,
            \label{eq:dataset_distribution}
    \end{align}
    or equivalently
    \begin{align}
            &\cD^{H} \sim \prod_{i=1}^{K} p_0(s_{1,i}) p_{\ell}(\ell_{i} \mid s_{1,i}, s_{H,i}) \nonumber\\*
            &\mspace{60mu} \prod_{t=1}^{H}
            e^{\sigma^2\|\expert(s_{t,i}) - a_{t,i}\|^2}\p(s_{t+1,i} \mid s_{t,i}, a_{t,i}) ,
            \label{eq:dataset_distribution_2}
    \end{align}
    by the isotropic Gaussian assumption.
    We will define $\cD^{*} \triangleq \cD^{H'}$ to be a longer-horizon version of $\cD$ extending the behaviors gathered under $\expert$ across a horizon $\alpha H \ge H' \ge H$ that additionally satisfies a ``time-isotropy'' property: the marginal distribution of the states is uniform across the horizon, i.e., $p_0(s_{1,i}) = p_0(s_{t,i})$ for all $t \in \{1\ldots H'\}$.

    We will relate the in-distribution imitation error $\textsc{Err}(\bullet; \cD)$ to the compositional out-of-distribution imitation error $\textsc{Err}(\bullet;\cD^{*})$.
    We define
    \begin{align}
        \textsc{Err}(\hat{\pi}; \tilde{\cD})
        &= \E_{\tilde{\cD}}\Bigl[\frac{1}{H}\sum_{t=1}^{H} \mathbb{E}_{\hat{\pi}}\left[\|\tilde{a}_{t,i} -
        \hat{\pi}(\tilde{s}_{t,i}, \tilde{s}_{H, i})\|^{2}/d_{\cA}\right]\Bigr] \nonumber\\
        &\quad \text{for} \quad \{\tilde{s}_{t,i},\tilde{a}_{t,i},\tilde{\ell}_{i}\}_{t=1}^{H} \sim
            \tilde{\cD}.
            \label{eq:imitation_error}
    \end{align}
    On the training dataset this is equivalent to the expected behavioral cloning loss from \cref{eq:BC}.

                            
    \begin{assumption}
        \label{asm:policy_factorization}
        The policy factorizes through inferred waypoints as:
\begin{align}
    &\textrm{goals: }\pi(a \mid s, g)
        = \nonumber\\*
      &\mspace{50mu} \int \pi(a\mid s, w) \p(s_{t}=w \mid s_{t+k}=g) \dd{w}
        \label{eq:goal-conditioned} \\
    &\textrm{language: } \pi(a \mid s, \ell)
        = \int \pi(a\mid s, w) \nonumber\\*
      &\mspace{20mu} \p(s_{t}=w \mid s_{t+k}=g) \p(s_{t+k}=g \mid \ell) \dd{w} \dd{g} ,
        \label{eq:language-conditioned}
        \end{align}
        where denote by $\pi(s,g)$ the MLE estimate of the action $a$.

    \end{assumption}

    \makerestatable
    \begin{theorem}
        \label{thm:compositionality}
        Suppose $\cD$ is distributed according to \cref{eq:dataset_distribution} and $\cD^{*}$ is distributed according to \cref{eq:dataset_distribution}.
        When $\gamma > 1-1/H$ and $\alpha > 1$, for optimal features $\phi$ and $\psi$ under \cref{eq:overall_objective}, we have
        \begin{gather}
            \textsc{Err}(\pi; \cD^{*}) \le \textsc{Err}(\pi; \cD) +  \frac{\alpha -1}{2 \alpha }+\Bigl(\frac{ \alpha - 2 }{2\alpha}\Bigr) \1 \{\alpha >2\}  .
            \label{eq:compositionality}
        \end{gather}
    \end{theorem}

    We can also define a notion of the language-conditioned compositional generalization error:
    \begin{equation*}
        \errl(\pi; \cD^{*}) \triangleq \E_{\cD^{*}}\Bigl[\frac{1}{H}\sum_{t=1}^{H}
            \mathbb{E}_{\pi}\bigl[\|\tilde{a}_{t,i} - \pi(\tilde{s}_{t,i}, \tilde{\ell}_{i})\|^{2}\bigr]\Bigr].
            \label{eq:language_error}
    \end{equation*}

    \makerestatable
    \begin{corollary}
        \label{thm:language}
        Under the same conditions as \cref{thm:compositionality},
        \begin{equation*}
            \errl(\pi; \cD^{*}) \le \errl(\pi; \cD) +  \frac{\alpha -1}{2 \alpha }+\Bigl(\frac{ \alpha - 2 }{2\alpha}\Bigr) \1 \{\alpha >2\}  .
            \label{eq:compositionality_language}
        \end{equation*}

    \end{corollary}

    The proofs as well as a visualization of the bound are in \cref{app:compositionality}. Policy implementation details can be found in \cref{app:tra_impl}

    
                
        
    \end{rebuttal}


\section{Evaluation Setup} \label{sec:experiment}
\section{Experiments}
\label{sec:exp}
Following the settings in Section \ref{sec:existing}, we evaluate \textit{NovelSum}'s correlation with the fine-tuned model performance across 53 IT datasets and compare it with previous diversity metrics. Additionally, we conduct a correlation analysis using Qwen-2.5-7B \cite{yang2024qwen2} as the backbone model, alongside previous LLaMA-3-8B experiments, to further demonstrate the metric's effectiveness across different scenarios. Qwen is used for both instruction tuning and deriving semantic embeddings. Due to resource constraints, we run each strategy on Qwen for two rounds, resulting in 25 datasets. 

\subsection{Main Results}

\begin{table*}[!t]
    \centering
    \resizebox{\linewidth}{!}{
    \begin{tabular}{lcccccccccc}
    \toprule
    \multirow{3}*{\textbf{Diversity Metrics}} & \multicolumn{10}{c}{\textbf{Data Selection Strategies}} \\
    \cmidrule(lr){2-11}
    & \multirow{2}*{\textbf{K-means}} & \multirow{2}*{\vtop{\hbox{\textbf{K-Center}}\vspace{1mm}\hbox{\textbf{-Greedy}}}}  & \multirow{2}*{\textbf{QDIT}} & \multirow{2}*{\vtop{\hbox{\textbf{Repr}}\vspace{1mm}\hbox{\textbf{Filter}}}} & \multicolumn{5}{c}{\textbf{Random}} & \multirow{2}{*}{\textbf{Duplicate}} \\ 
    \cmidrule(lr){6-10}
    & & & & & \textbf{$\mathcal{X}^{all}$} & ShareGPT & WizardLM & Alpaca & Dolly &  \\
    \midrule
    \rowcolor{gray!15} \multicolumn{11}{c}{\textit{LLaMA-3-8B}} \\
    Facility Loc. $_{\times10^5}$ & \cellcolor{BLUE!40} 2.99 & \cellcolor{ORANGE!10} 2.73 & \cellcolor{BLUE!40} 2.99 & \cellcolor{BLUE!20} 2.86 & \cellcolor{BLUE!40} 2.99 & \cellcolor{BLUE!0} 2.83 & \cellcolor{BLUE!30} 2.88 & \cellcolor{BLUE!0} 2.83 & \cellcolor{ORANGE!20} 2.59 & \cellcolor{ORANGE!30} 2.52 \\    
    DistSum$_{cosine}$  & \cellcolor{BLUE!30} 0.648 & \cellcolor{BLUE!60} 0.746 & \cellcolor{BLUE!0} 0.629 & \cellcolor{BLUE!50} 0.703 & \cellcolor{BLUE!10} 0.634 & \cellcolor{BLUE!40} 0.656 & \cellcolor{ORANGE!30} 0.578 & \cellcolor{ORANGE!10} 0.605 & \cellcolor{ORANGE!20} 0.603 & \cellcolor{BLUE!10} 0.634 \\
    Vendi Score $_{\times10^7}$ & \cellcolor{BLUE!30} 1.70 & \cellcolor{BLUE!60} 2.53 & \cellcolor{BLUE!10} 1.59 & \cellcolor{BLUE!50} 2.23 & \cellcolor{BLUE!20} 1.61 & \cellcolor{BLUE!30} 1.70 & \cellcolor{ORANGE!10} 1.44 & \cellcolor{ORANGE!20} 1.32 & \cellcolor{ORANGE!10} 1.44 & \cellcolor{ORANGE!30} 0.05 \\
    \textbf{NovelSum (Ours)} & \cellcolor{BLUE!60} 0.693 & \cellcolor{BLUE!50} 0.687 & \cellcolor{BLUE!30} 0.673 & \cellcolor{BLUE!20} 0.671 & \cellcolor{BLUE!40} 0.675 & \cellcolor{BLUE!10} 0.628 & \cellcolor{BLUE!0} 0.591 & \cellcolor{ORANGE!10} 0.572 & \cellcolor{ORANGE!20} 0.50 & \cellcolor{ORANGE!30} 0.461 \\
    \midrule    
    \textbf{Model Performance} & \cellcolor{BLUE!60}1.32 & \cellcolor{BLUE!50}1.31 & \cellcolor{BLUE!40}1.25 & \cellcolor{BLUE!30}1.05 & \cellcolor{BLUE!20}1.20 & \cellcolor{BLUE!10}0.83 & \cellcolor{BLUE!0}0.72 & \cellcolor{ORANGE!10}0.07 & \cellcolor{ORANGE!20}-0.14 & \cellcolor{ORANGE!30}-1.35 \\
    \midrule
    \midrule
    \rowcolor{gray!15} \multicolumn{11}{c}{\textit{Qwen-2.5-7B}} \\
    Facility Loc. $_{\times10^5}$ & \cellcolor{BLUE!40} 3.54 & \cellcolor{ORANGE!30} 3.42 & \cellcolor{BLUE!40} 3.54 & \cellcolor{ORANGE!20} 3.46 & \cellcolor{BLUE!40} 3.54 & \cellcolor{BLUE!30} 3.51 & \cellcolor{BLUE!10} 3.50 & \cellcolor{BLUE!10} 3.50 & \cellcolor{ORANGE!20} 3.46 & \cellcolor{BLUE!0} 3.48 \\ 
    DistSum$_{cosine}$ & \cellcolor{BLUE!30} 0.260 & \cellcolor{BLUE!60} 0.440 & \cellcolor{BLUE!0} 0.223 & \cellcolor{BLUE!50} 0.421 & \cellcolor{BLUE!10} 0.230 & \cellcolor{BLUE!40} 0.285 & \cellcolor{ORANGE!20} 0.211 & \cellcolor{ORANGE!30} 0.189 & \cellcolor{ORANGE!10} 0.221 & \cellcolor{BLUE!20} 0.243 \\
    Vendi Score $_{\times10^6}$ & \cellcolor{ORANGE!10} 1.60 & \cellcolor{BLUE!40} 3.09 & \cellcolor{BLUE!10} 2.60 & \cellcolor{BLUE!60} 7.15 & \cellcolor{ORANGE!20} 1.41 & \cellcolor{BLUE!50} 3.36 & \cellcolor{BLUE!20} 2.65 & \cellcolor{BLUE!0} 1.89 & \cellcolor{BLUE!30} 3.04 & \cellcolor{ORANGE!30} 0.20 \\
    \textbf{NovelSum (Ours)}  & \cellcolor{BLUE!40} 0.440 & \cellcolor{BLUE!60} 0.505 & \cellcolor{BLUE!20} 0.403 & \cellcolor{BLUE!50} 0.495 & \cellcolor{BLUE!30} 0.408 & \cellcolor{BLUE!10} 0.392 & \cellcolor{BLUE!0} 0.349 & \cellcolor{ORANGE!10} 0.336 & \cellcolor{ORANGE!20} 0.320 & \cellcolor{ORANGE!30} 0.309 \\
    \midrule
    \textbf{Model Performance} & \cellcolor{BLUE!30} 1.06 & \cellcolor{BLUE!60} 1.45 & \cellcolor{BLUE!40} 1.23 & \cellcolor{BLUE!50} 1.35 & \cellcolor{BLUE!20} 0.87 & \cellcolor{BLUE!10} 0.07 & \cellcolor{BLUE!0} -0.08 & \cellcolor{ORANGE!10} -0.38 & \cellcolor{ORANGE!30} -0.49 & \cellcolor{ORANGE!20} -0.43 \\
    \bottomrule
    \end{tabular}
    }
    \caption{Measuring the diversity of datasets selected by different strategies using \textit{NovelSum} and baseline metrics. Fine-tuned model performances (Eq. \ref{eq:perf}), based on MT-bench and AlpacaEval, are also included for cross reference. Darker \colorbox{BLUE!60}{blue} shades indicate higher values for each metric, while darker \colorbox{ORANGE!30}{orange} shades indicate lower values. While data selection strategies vary in performance on LLaMA-3-8B and Qwen-2.5-7B, \textit{NovelSum} consistently shows a stronger correlation with model performance than other metrics. More results are provided in Appendix \ref{app:results}.}
    \label{tbl:main}
    \vspace{-4mm}
\end{table*}


\begin{table}[t!]
\centering
\resizebox{\linewidth}{!}{
\begin{tabular}{lcccc}
\toprule
\multirow{2}*{\textbf{Diversity Metrics}} & \multicolumn{3}{c}{\textbf{LLaMA}} & \textbf{Qwen}\\
\cmidrule(lr){2-4} \cmidrule(lr){5-5} 
& \textbf{Pearson} & \textbf{Spearman} & \textbf{Avg.} & \textbf{Avg.} \\
\midrule
TTR & -0.38 & -0.16 & -0.27 & -0.30 \\
vocd-D & -0.43 & -0.17 & -0.30 & -0.31 \\
\midrule
Facility Loc. & 0.86 & 0.69 & 0.77 & 0.08 \\
Entropy & 0.93 & 0.80 & 0.86 & 0.63 \\
\midrule
LDD & 0.61 & 0.75 & 0.68 & 0.60 \\
KNN Distance & 0.59 & 0.80 & 0.70 & 0.67 \\
DistSum$_{cosine}$ & 0.85 & 0.67 & 0.76 & 0.51 \\
Vendi Score & 0.70 & 0.85 & 0.78 & 0.60 \\
DistSum$_{L2}$ & 0.86 & 0.76 & 0.81 & 0.51 \\
Cluster Inertia & 0.81 & 0.85 & 0.83 & 0.76 \\
Radius & 0.87 & 0.81 & 0.84 & 0.48 \\
\midrule
NovelSum & \textbf{0.98} & \textbf{0.95} & \textbf{0.97} & \textbf{0.90} \\
\bottomrule
\end{tabular}
}
\caption{Correlations between different metrics and model performance on LLaMA-3-8B and Qwen-2.5-7B.  “Avg.” denotes the average correlation (Eq. \ref{eq:cor}).}
\label{tbl:correlations}
\vspace{-2mm}
\end{table}

\paragraph{\textit{NovelSum} consistently achieves state-of-the-art correlation with model performance across various data selection strategies, backbone LLMs, and correlation measures.}
Table \ref{tbl:main} presents diversity measurement results on datasets constructed by mainstream data selection methods (based on $\mathcal{X}^{all}$), random selection from various sources, and duplicated samples (with only $m=100$ unique samples). 
Results from multiple runs are averaged for each strategy.
Although these strategies yield varying performance rankings across base models, \textit{NovelSum} consistently tracks changes in IT performance by accurately measuring dataset diversity. For instance, K-means achieves the best performance on LLaMA with the highest NovelSum score, while K-Center-Greedy excels on Qwen, also correlating with the highest NovelSum. Table \ref{tbl:correlations} shows the correlation coefficients between various metrics and model performance for both LLaMA and Qwen experiments, where \textit{NovelSum} achieves state-of-the-art correlation across different models and measures.

\paragraph{\textit{NovelSum} can provide valuable guidance for data engineering practices.}
As a reliable indicator of data diversity, \textit{NovelSum} can assess diversity at both the dataset and sample levels, directly guiding data selection and construction decisions. For example, Table \ref{tbl:main} shows that the combined data source $\mathcal{X}^{all}$ is a better choice for sampling diverse IT data than other sources. Moreover, \textit{NovelSum} can offer insights through comparative analyses, such as: (1) ShareGPT, which collects data from real internet users, exhibits greater diversity than Dolly, which relies on company employees, suggesting that IT samples from diverse sources enhance dataset diversity \cite{wang2024diversity-logD}; (2) In LLaMA experiments, random selection can outperform some mainstream strategies, aligning with prior work \cite{xia2024rethinking,diddee2024chasing}, highlighting gaps in current data selection methods for optimizing diversity.



\subsection{Ablation Study}


\textit{NovelSum} involves several flexible hyperparameters and variations. In our main experiments, \textit{NovelSum} uses cosine distance to compute $d(x_i, x_j)$ in Eq. \ref{eq:dad}. We set $\alpha = 1$, $\beta = 0.5$, and $K = 10$ nearest neighbors in Eq. \ref{eq:pws} and \ref{eq:dad}. Here, we conduct an ablation study to investigate the impact of these settings based on LLaMA-3-8B.

\begin{table}[ht!]
\centering
\resizebox{\linewidth}{!}{
\begin{tabular}{lccc}
\toprule
\textbf{Variants} & \textbf{Pearson} & \textbf{Spearman} & \textbf{Avg.} \\
\midrule
NovelSum & 0.98 & 0.96 & 0.97 \\
\midrule
\hspace{0.10cm} - Use $L2$ distance & 0.97 & 0.83 & 0.90\textsubscript{↓ 0.08} \\
\hspace{0.10cm} - $K=20$ & 0.98 & 0.96 & 0.97\textsubscript{↓ 0.00} \\
\hspace{0.10cm} - $\alpha=0$ (w/o proximity) & 0.79 & 0.31 & 0.55\textsubscript{↓ 0.42} \\
\hspace{0.10cm} - $\alpha=2$ & 0.73 & 0.88 & 0.81\textsubscript{↓ 0.16} \\
\hspace{0.10cm} - $\beta=0$ (w/o density) & 0.92 & 0.89 & 0.91\textsubscript{↓ 0.07} \\
\hspace{0.10cm} - $\beta=1$ & 0.90 & 0.62 & 0.76\textsubscript{↓ 0.21} \\
\bottomrule
\end{tabular}
}
\caption{Ablation Study for \textit{NovelSum}.}
\label{tbl:ablation}
\vspace{-2mm}
\end{table}

In Table \ref{tbl:ablation}, $\alpha=0$ removes the proximity weights, and $\beta=0$ eliminates the density multiplier. We observe that both $\alpha=0$ and $\beta=0$ significantly weaken the correlation, validating the benefits of the proximity-weighted sum and density-aware distance. Additionally, improper values for $\alpha$ and $\beta$ greatly reduce the metric's reliability, highlighting that \textit{NovelSum} strikes a delicate balance between distances and distribution. Replacing cosine distance with Euclidean distance and using more neighbors for density approximation have minimal impact, particularly on Pearson's correlation, demonstrating \textit{NovelSum}'s robustness to different distance measures.







\section{Results and Discussion} \label{sec:results}

\begin{table*}[t]
\centering
\fontsize{11pt}{11pt}\selectfont
\begin{tabular}{lllllllllllll}
\toprule
\multicolumn{1}{c}{\textbf{task}} & \multicolumn{2}{c}{\textbf{Mir}} & \multicolumn{2}{c}{\textbf{Lai}} & \multicolumn{2}{c}{\textbf{Ziegen.}} & \multicolumn{2}{c}{\textbf{Cao}} & \multicolumn{2}{c}{\textbf{Alva-Man.}} & \multicolumn{1}{c}{\textbf{avg.}} & \textbf{\begin{tabular}[c]{@{}l@{}}avg.\\ rank\end{tabular}} \\
\multicolumn{1}{c}{\textbf{metrics}} & \multicolumn{1}{c}{\textbf{cor.}} & \multicolumn{1}{c}{\textbf{p-v.}} & \multicolumn{1}{c}{\textbf{cor.}} & \multicolumn{1}{c}{\textbf{p-v.}} & \multicolumn{1}{c}{\textbf{cor.}} & \multicolumn{1}{c}{\textbf{p-v.}} & \multicolumn{1}{c}{\textbf{cor.}} & \multicolumn{1}{c}{\textbf{p-v.}} & \multicolumn{1}{c}{\textbf{cor.}} & \multicolumn{1}{c}{\textbf{p-v.}} &  &  \\ \midrule
\textbf{S-Bleu} & 0.50 & 0.0 & 0.47 & 0.0 & 0.59 & 0.0 & 0.58 & 0.0 & 0.68 & 0.0 & 0.57 & 5.8 \\
\textbf{R-Bleu} & -- & -- & 0.27 & 0.0 & 0.30 & 0.0 & -- & -- & -- & -- & - &  \\
\textbf{S-Meteor} & 0.49 & 0.0 & 0.48 & 0.0 & 0.61 & 0.0 & 0.57 & 0.0 & 0.64 & 0.0 & 0.56 & 6.1 \\
\textbf{R-Meteor} & -- & -- & 0.34 & 0.0 & 0.26 & 0.0 & -- & -- & -- & -- & - &  \\
\textbf{S-Bertscore} & \textbf{0.53} & 0.0 & {\ul 0.80} & 0.0 & \textbf{0.70} & 0.0 & {\ul 0.66} & 0.0 & {\ul0.78} & 0.0 & \textbf{0.69} & \textbf{1.7} \\
\textbf{R-Bertscore} & -- & -- & 0.51 & 0.0 & 0.38 & 0.0 & -- & -- & -- & -- & - &  \\
\textbf{S-Bleurt} & {\ul 0.52} & 0.0 & {\ul 0.80} & 0.0 & 0.60 & 0.0 & \textbf{0.70} & 0.0 & \textbf{0.80} & 0.0 & {\ul 0.68} & {\ul 2.3} \\
\textbf{R-Bleurt} & -- & -- & 0.59 & 0.0 & -0.05 & 0.13 & -- & -- & -- & -- & - &  \\
\textbf{S-Cosine} & 0.51 & 0.0 & 0.69 & 0.0 & {\ul 0.62} & 0.0 & 0.61 & 0.0 & 0.65 & 0.0 & 0.62 & 4.4 \\
\textbf{R-Cosine} & -- & -- & 0.40 & 0.0 & 0.29 & 0.0 & -- & -- & -- & -- & - & \\ \midrule
\textbf{QuestEval} & 0.23 & 0.0 & 0.25 & 0.0 & 0.49 & 0.0 & 0.47 & 0.0 & 0.62 & 0.0 & 0.41 & 9.0 \\
\textbf{LLaMa3} & 0.36 & 0.0 & \textbf{0.84} & 0.0 & {\ul{0.62}} & 0.0 & 0.61 & 0.0 &  0.76 & 0.0 & 0.64 & 3.6 \\
\textbf{our (3b)} & 0.49 & 0.0 & 0.73 & 0.0 & 0.54 & 0.0 & 0.53 & 0.0 & 0.7 & 0.0 & 0.60 & 5.8 \\
\textbf{our (8b)} & 0.48 & 0.0 & 0.73 & 0.0 & 0.52 & 0.0 & 0.53 & 0.0 & 0.7 & 0.0 & 0.59 & 6.3 \\  \bottomrule
\end{tabular}
\caption{Pearson correlation on human evaluation on system output. `R-': reference-based. `S-': source-based.}
\label{tab:sys}
\end{table*}



\begin{table}%[]
\centering
\fontsize{11pt}{11pt}\selectfont
\begin{tabular}{llllll}
\toprule
\multicolumn{1}{c}{\textbf{task}} & \multicolumn{1}{c}{\textbf{Lai}} & \multicolumn{1}{c}{\textbf{Zei.}} & \multicolumn{1}{c}{\textbf{Scia.}} & \textbf{} & \textbf{} \\ 
\multicolumn{1}{c}{\textbf{metrics}} & \multicolumn{1}{c}{\textbf{cor.}} & \multicolumn{1}{c}{\textbf{cor.}} & \multicolumn{1}{c}{\textbf{cor.}} & \textbf{avg.} & \textbf{\begin{tabular}[c]{@{}l@{}}avg.\\ rank\end{tabular}} \\ \midrule
\textbf{S-Bleu} & 0.40 & 0.40 & 0.19* & 0.33 & 7.67 \\
\textbf{S-Meteor} & 0.41 & 0.42 & 0.16* & 0.33 & 7.33 \\
\textbf{S-BertS.} & {\ul0.58} & 0.47 & 0.31 & 0.45 & 3.67 \\
\textbf{S-Bleurt} & 0.45 & {\ul 0.54} & {\ul 0.37} & 0.45 & {\ul 3.33} \\
\textbf{S-Cosine} & 0.56 & 0.52 & 0.3 & {\ul 0.46} & {\ul 3.33} \\ \midrule
\textbf{QuestE.} & 0.27 & 0.35 & 0.06* & 0.23 & 9.00 \\
\textbf{LlaMA3} & \textbf{0.6} & \textbf{0.67} & \textbf{0.51} & \textbf{0.59} & \textbf{1.0} \\
\textbf{Our (3b)} & 0.51 & 0.49 & 0.23* & 0.39 & 4.83 \\
\textbf{Our (8b)} & 0.52 & 0.49 & 0.22* & 0.43 & 4.83 \\ \bottomrule
\end{tabular}
\caption{Pearson correlation on human ratings on reference output. *not significant; we cannot reject the null hypothesis of zero correlation}
\label{tab:ref}
\end{table}


\begin{table*}%[]
\centering
\fontsize{11pt}{11pt}\selectfont
\begin{tabular}{lllllllll}
\toprule
\textbf{task} & \multicolumn{1}{c}{\textbf{ALL}} & \multicolumn{1}{c}{\textbf{sentiment}} & \multicolumn{1}{c}{\textbf{detoxify}} & \multicolumn{1}{c}{\textbf{catchy}} & \multicolumn{1}{c}{\textbf{polite}} & \multicolumn{1}{c}{\textbf{persuasive}} & \multicolumn{1}{c}{\textbf{formal}} & \textbf{\begin{tabular}[c]{@{}l@{}}avg. \\ rank\end{tabular}} \\
\textbf{metrics} & \multicolumn{1}{c}{\textbf{cor.}} & \multicolumn{1}{c}{\textbf{cor.}} & \multicolumn{1}{c}{\textbf{cor.}} & \multicolumn{1}{c}{\textbf{cor.}} & \multicolumn{1}{c}{\textbf{cor.}} & \multicolumn{1}{c}{\textbf{cor.}} & \multicolumn{1}{c}{\textbf{cor.}} &  \\ \midrule
\textbf{S-Bleu} & -0.17 & -0.82 & -0.45 & -0.12* & -0.1* & -0.05 & -0.21 & 8.42 \\
\textbf{R-Bleu} & - & -0.5 & -0.45 &  &  &  &  &  \\
\textbf{S-Meteor} & -0.07* & -0.55 & -0.4 & -0.01* & 0.1* & -0.16 & -0.04* & 7.67 \\
\textbf{R-Meteor} & - & -0.17* & -0.39 & - & - & - & - & - \\
\textbf{S-BertScore} & 0.11 & -0.38 & -0.07* & -0.17* & 0.28 & 0.12 & 0.25 & 6.0 \\
\textbf{R-BertScore} & - & -0.02* & -0.21* & - & - & - & - & - \\
\textbf{S-Bleurt} & 0.29 & 0.05* & 0.45 & 0.06* & 0.29 & 0.23 & 0.46 & 4.2 \\
\textbf{R-Bleurt} & - &  0.21 & 0.38 & - & - & - & - & - \\
\textbf{S-Cosine} & 0.01* & -0.5 & -0.13* & -0.19* & 0.05* & -0.05* & 0.15* & 7.42 \\
\textbf{R-Cosine} & - & -0.11* & -0.16* & - & - & - & - & - \\ \midrule
\textbf{QuestEval} & 0.21 & {\ul{0.29}} & 0.23 & 0.37 & 0.19* & 0.35 & 0.14* & 4.67 \\
\textbf{LlaMA3} & \textbf{0.82} & \textbf{0.80} & \textbf{0.72} & \textbf{0.84} & \textbf{0.84} & \textbf{0.90} & \textbf{0.88} & \textbf{1.00} \\
\textbf{Our (3b)} & 0.47 & -0.11* & 0.37 & 0.61 & 0.53 & 0.54 & 0.66 & 3.5 \\
\textbf{Our (8b)} & {\ul{0.57}} & 0.09* & {\ul 0.49} & {\ul 0.72} & {\ul 0.64} & {\ul 0.62} & {\ul 0.67} & {\ul 2.17} \\ \bottomrule
\end{tabular}
\caption{Pearson correlation on human ratings on our constructed test set. 'R-': reference-based. 'S-': source-based. *not significant; we cannot reject the null hypothesis of zero correlation}
\label{tab:con}
\end{table*}

\section{Results}
We benchmark the different metrics on the different datasets using correlation to human judgement. For content preservation, we show results split on data with system output, reference output and our constructed test set: we show that the data source for evaluation leads to different conclusions on the metrics. In addition, we examine whether the metrics can rank style transfer systems similar to humans. On style strength, we likewise show correlations between human judgment and zero-shot evaluation approaches. When applicable, we summarize results by reporting the average correlation. And the average ranking of the metric per dataset (by ranking which metric obtains the highest correlation to human judgement per dataset). 

\subsection{Content preservation}
\paragraph{How do data sources affect the conclusion on best metric?}
The conclusions about the metrics' performance change radically depending on whether we use system output data, reference output, or our constructed test set. Ideally, a good metric correlates highly with humans on any data source. Ideally, for meta-evaluation, a metric should correlate consistently across all data sources, but the following shows that the correlations indicate different things, and the conclusion on the best metric should be drawn carefully.

Looking at the metrics correlations with humans on the data source with system output (Table~\ref{tab:sys}), we see a relatively high correlation for many of the metrics on many tasks. The overall best metrics are S-BertScore and S-BLEURT (avg+avg rank). We see no notable difference in our method of using the 3B or 8B model as the backbone.

Examining the average correlations based on data with reference output (Table~\ref{tab:ref}), now the zero-shoot prompting with LlaMA3 70B is the best-performing approach ($0.59$ avg). Tied for second place are source-based cosine embedding ($0.46$ avg), BLEURT ($0.45$ avg) and BertScore ($0.45$ avg). Our method follows on a 5. place: here, the 8b version (($0.43$ avg)) shows a bit stronger results than 3b ($0.39$ avg). The fact that the conclusions change, whether looking at reference or system output, confirms the observations made by \citet{scialom-etal-2021-questeval} on simplicity transfer.   

Now consider the results on our test set (Table~\ref{tab:con}): Several metrics show low or no correlation; we even see a significantly negative correlation for some metrics on ALL (BLEU) and for specific subparts of our test set for BLEU, Meteor, BertScore, Cosine. On the other end, LlaMA3 70B is again performing best, showing strong results ($0.82$ in ALL). The runner-up is now our 8B method, with a gap to the 3B version ($0.57$ vs $0.47$ in ALL). Note our method still shows zero correlation for the sentiment task. After, ranks BLEURT ($0.29$), QuestEval ($0.21$), BertScore ($0.11$), Cosine ($0.01$).  

On our test set, we find that some metrics that correlate relatively well on the other datasets, now exhibit low correlation. Hence, with our test set, we can now support the logical reasoning with data evidence: Evaluation of content preservation for style transfer needs to take the style shift into account. This conclusion could not be drawn using the existing data sources: We hypothesise that for the data with system-based output, successful output happens to be very similar to the source sentence and vice versa, and reference-based output might not contain server mistakes as they are gold references. Thus, none of the existing data sources tests the limits of the metrics.  


\paragraph{How do reference-based metrics compare to source-based ones?} Reference-based metrics show a lower correlation than the source-based counterpart for all metrics on both datasets with ratings on references (Table~\ref{tab:sys}). As discussed previously, reference-based metrics for style transfer have the drawback that many different good solutions on a rewrite might exist and not only one similar to a reference.


\paragraph{How well can the metrics rank the performance of style transfer methods?}
We compare the metrics' ability to judge the best style transfer methods w.r.t. the human annotations: Several of the data sources contain samples from different style transfer systems. In order to use metrics to assess the quality of the style transfer system, metrics should correctly find the best-performing system. Hence, we evaluate whether the metrics for content preservation provide the same system ranking as human evaluators. We take the mean of the score for every output on each system and the mean of the human annotations; we compare the systems using the Kendall's Tau correlation. 

We find only the evaluation using the dataset Mir, Lai, and Ziegen to result in significant correlations, probably because of sparsity in a number of system tests (App.~\ref{app:dataset}). Our method (8b) is the only metric providing a perfect ranking of the style transfer system on the Lai data, and Llama3 70B the only one on the Ziegen data. Results in App.~\ref{app:results}. 


\subsection{Style strength results}
%Evaluating style strengths is a challenging task. 
Llama3 70B shows better overall results than our method. However, our method scores higher than Llama3 70B on 2 out of 6 datasets, but it also exhibits zero correlation on one task (Table~\ref{tab:styleresults}).%More work i s needed on evaluating style strengths. 
 
\begin{table}%[]
\fontsize{11pt}{11pt}\selectfont
\begin{tabular}{lccc}
\toprule
\multicolumn{1}{c}{\textbf{}} & \textbf{LlaMA3} & \textbf{Our (3b)} & \textbf{Our (8b)} \\ \midrule
\textbf{Mir} & 0.46 & 0.54 & \textbf{0.57} \\
\textbf{Lai} & \textbf{0.57} & 0.18 & 0.19 \\
\textbf{Ziegen.} & 0.25 & 0.27 & \textbf{0.32} \\
\textbf{Alva-M.} & \textbf{0.59} & 0.03* & 0.02* \\
\textbf{Scialom} & \textbf{0.62} & 0.45 & 0.44 \\
\textbf{\begin{tabular}[c]{@{}l@{}}Our Test\end{tabular}} & \textbf{0.63} & 0.46 & 0.48 \\ \bottomrule
\end{tabular}
\caption{Style strength: Pearson correlation to human ratings. *not significant; we cannot reject the null hypothesis of zero corelation}
\label{tab:styleresults}
\end{table}

\subsection{Ablation}
We conduct several runs of the methods using LLMs with variations in instructions/prompts (App.~\ref{app:method}). We observe that the lower the correlation on a task, the higher the variation between the different runs. For our method, we only observe low variance between the runs.
None of the variations leads to different conclusions of the meta-evaluation. Results in App.~\ref{app:results}.

\section{Related Works} \label{sec:relatedworks}
\section{Related Work}
\label{sec:related}
FL has emerged as a crucial learning scheme for distributed training that aims to preserve user data privacy. 
However, research has uncovered various privacy vulnerabilities in FL, particularly in the form of MIA and LDIA.
This section discusses relevant works that highlight these threats in both FL and FD settings, and contextualize our research within this landscape.

\BfPara{MIA and LDIA} 
Shokri \etal~\cite{shokri2017membership} pioneered MIA research by demonstrating how model output confidence scores could reveal training data membership. 
Nasr \etal~\cite{nasr2019comprehensive} extended this to FL, showing how both passive and active adversaries could exploit gradients and model updates.
LDIA represents another significant privacy threat in FL.
Gu \etal~\cite{gu2023ldia} introduced LDIA as a new attack vector where adversaries infer label distributions from model updates. Wainakh \etal~\cite{wainakh2021user} further explored user-level label leakage through gradient-based attacks in FL.
Recent works have exposed the vulnerability of FD to inference attacks.
Yang \etal~\cite{yang2022fd} proposed FD-Leaks for performing MIA in FD settings through logit analysis. Liu \etal~\cite{liu2023mia} and Wang \etal~\cite{wang2024graddiff} enhanced MIA using shadow models via respective approaches MIA-FedDL and GradDiff, though their assumptions were limited to homogeneous environments.

\iffalse
The study of MIA in ML models was pioneered by Shokri \etal~\cite{shokri2017membership}.
They demonstrated how adversaries could exploit model output confidence scores to infer whether a specific data point was used in training.
This seminal work laid the foundation for subsequent research into privacy vulnerabilities in various ML paradigms, including FL.
Building on this, Nasr \etal~\cite{nasr2019comprehensive} extended the concept to FL environments, introducing white-box MIAs.
Their privacy analysis demonstrated how both passive and active adversaries could exploit gradients and model updates to infer private information.
\BfPara{LDIA in FL} 
LDIA represents another significant privacy threat in FL.
Gu \etal~\cite{gu2023ldia} introduced LDIA as a new attack vector, where adversaries seek to infer the distribution of labels in clients' training data by analyzing model updates.
This work showed that even when individual data points are protected, the overall data distribution might still be exposed, leading to privacy concerns.
Wainakh \etal~\cite{wainakh2021user} further explored user-level label leakage by performing gradient-based attacks in FL.

\BfPara{LDIA and MIA in FD} 
% FD has been proposed as a lightweight alternative to traditional FL.
% It reduces communication overhead by exchanging logits or softmax values instead of model parameters.
Yang \etal~\cite{yang2022fd} proposed FD-Leaks, an attack designed to perform MIA in FD settings.
By analyzing logits, adversaries can infer membership information, potentially revealing sensitive training data.
Similarly, Liu \etal~\cite{liu2023mia} presented MIA-FedDL, which enhances MIA by using shadow models to infer membership information with higher accuracy in FD settings.
Despite their contributions, they made assumptions that are infeasible in heterogeneous environments.
\fi
% Despite their contributions, these works make several assumptions that may limit the feasibility of the proposed attacks in practice.
% For instance, both FD-Leaks and MIA-FedDL assume that the adversaries have access to all logits or can easily build shadow models that mimic the behavior of target models.
% Such assumptions are often unrealistic in heterogeneous environments.

\BfPara{Defenses and Countermeasures}
DPSGD~\cite{abadi2016deep} can be employed during the training phase to mitigate against privacy attacks to the client model. 
Additionally, specialized MIA defense methods such as SELENA~\cite{tang2022mitigating}, HAMP~\cite{chen2023overconfidence} and DMP\cite{shejwalkar2021membership} can be integrated into the training process.
Several studies have proposed enhanced FD frameworks with improved privacy protection mechanisms to reduce client privacy leakage.
%Several studies have proposed defenses to mitigate MIA and LDIA.
%Wang \etal~\cite{wang2024graddiff} proposed GradDiff, a gradient-based defense mechanism that employs differential comparison to detect and mitigate MIA in FD settings.
Zhu \etal~\cite{zhu2021data} investigated data-free knowledge distillation for heterogeneous federated learning.
They presented an approach that reduces the need for public datasets.
Chen \etal~\cite{chen2023best} proposed FedHKD, where clients share hyper-knowledge based on data representations from local datasets for federated distillation without requiring public datasets or models.

% Our work builds upon these foundations, specifically investigating the privacy vulnerabilities in PDA-FD frameworks. 
%Unlike previous studies that focused on traditional FL or specific attack scenarios in FD, our work aim to provide a comprehensive analysis of LDIA and MIA across multiple PDA-FD frameworks.



\section{Conclusion} \label{sec:conclusion}
\section{Conclusion}
In this work, we propose a simple yet effective approach, called SMILE, for graph few-shot learning with fewer tasks. Specifically, we introduce a novel dual-level mixup strategy, including within-task and across-task mixup, for enriching the diversity of nodes within each task and the diversity of tasks. Also, we incorporate the degree-based prior information to learn expressive node embeddings. Theoretically, we prove that SMILE effectively enhances the model's generalization performance. Empirically, we conduct extensive experiments on multiple benchmarks and the results suggest that SMILE significantly outperforms other baselines, including both in-domain and cross-domain few-shot settings.

\section*{Acknowledgment}
We acknowledge the support  of the PNRR project FAIR -  Future AI Research (PE00000013),  under the NRRP MUR program funded by the NextGenerationEU.

\bibliographystyle{IEEEtran}
\bibliography{IEEEabrv,reference}



\end{document}
