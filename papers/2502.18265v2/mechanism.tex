\def\Brem{B_{\text{rem}}}
\def\Nrem{\mathcal{N}_{\text{rem}}}
\def\tinit{t_{\text{init}}}
\def\tmin{t_{\min}}
\def\tmax{t_{\max}}
\def\val{\text{val}}
\def\Event{\mathcal{E}}
\def\Bin{\mathcal{B}}

%In the second period we partition the interval $[v_{max}, n \cdot v_{max}]$ into a set $\mathcal{T}$, of $T=|\mathcal{T}|$ intervals, each increasing in size according to the power towering described earlier. 
%\begin{equation}
%\begin{split}
 %   \mathcal{T} &= \left\{[t_1, t_2], (t_2, t_3], \dots, (t_{T}, t_{T+1}]\right\}
%\end{split}   
%\end{equation}

\begin{figure}[t]
\centering\includegraphics[width=0.6\textwidth]{ProcurementAuctions}
    \caption{The input sequence is partitioned into four periods. Periods 2 to 4 are further partitioned into phases, with each phase consisting of multiple rounds.}
    \label{fig:phases}
\end{figure}


\begin{algorithm}[t]
\caption{LM-Mechanism}\label{alg:LMMECH}
\begin{algorithmic}[1]
\STATE \textbf{Input}: Set $\N$ of agents arriving in random order, budget $B$
%, at each timestep $i \in [n]$ an agent $b(i)$ in secretary order.
%\STATE \textbf{Parameters}: $m,a,OPT_{min},OPT_{max},$ rounds $\{n_j\}$
\STATE \textbf{Initialization:} Solution $S \gets \emptyset$ and $\val \gets 0$, budget available $\Brem \gets B$, agents available $\Nrem \gets \N$
\STATE $\vmax, \Nrem \gets \text{LearningMaxValue}(\Nrem)$
\STATE $\tmin, \tmax, (S, \val, \Nrem, \Brem) \gets \text{PowerTowerSearch}((S, \val, \Nrem, \Brem), B, [\vmax, n\cdot\vmax])$
%\STATE[The constant in the definition of $m$ and $a$ will be determined later on]{Let $m = \Theta(\log\log\log(t_{\max} / t_{\min}))$ and $1/a = \Theta(m \log\log(t_{\max} / t_{\min}))$}
% Set $a$, $m$ to the proper values.
\STATE $\tinit, (S, \val, \Nrem, \Brem) \gets \text{BinarySearch}((S, \val, \Nrem, \Brem), B, [\tmin, \tmax])$
\STATE $(S, \val) \gets \text{Exploitation}((S, \val, \Nrem, \Brem), B, \tinit, [\tmin, \tmax])$
\RETURN $(S, \val)$
\end{algorithmic}
\end{algorithm}

%Below we present the proof of the competitiveness of \hyperref[alg:LMMECH]{LM-Mechanism}.




\section{Posted-Price Mechanism: Building Blocks and Outline of the Analysis}
\label{sec:mech}

We next present the main steps of our posted-price \hyperref[alg:LMMECH]{LM-Mechanism} and the key ideas of its analysis. \hyperref[alg:LMMECH]{LM-Mechanism} assumes that $\opt / \vmax$ is sufficient large (LM stands for \emph{Large Market}). E.g., it suffices that $\opt / \vmax \geq 10^7$. In Section~\ref{sec:MainTheorem}, we use Dynkin's algorithm \cite{Dynkin63} and an adaptation of the posted-price mechanism in \cite[Section~4]{Bada2012} to deal with the case where $\opt / \vmax < 10^7$. 

As explained in Section~\ref{sec:approach}, \hyperref[alg:LMMECH]{LM-Mechanism} consists of $4$ periods. Each of the first three periods ``consumes'' a constant fraction of the agent sequence (and at most a constant fraction of the budget $B$) and provides the next period with a more refined estimation of $\opt$. More specifically, \hyperref[alg:LMMECH]{LM-Mechanism} proceeds along the following periods: 
%
\begin{description}
\item [Period 1 -- LearningMaxValue:] The first period calls Mechanism~\ref{alg:Vmax}, usually referred to as LearningMaxValue, which 
examines about $1/3$ of the agent sequence and returns the maximum value of an agent in this part. We let $\Event_1$ denote the event that the value returned by \hyperref[alg:Vmax]{LearningMaxValue} is indeed $\vmax$, which occurs with probability $1/3$. Assuming $\Event_1$, we have a first rough estimation of $\opt$, since $\vmax \leq \opt \leq n\cdot\vmax$. 

\item [Period 2 -- PowerTowerSearch:] The second period calls Mechanism~\ref{alg:PowerTower}, usually referred to as PowerTowerSearch,
which decides on two consecutive intervals $A = [a_{\min}, a_{\max}]$ and $B = [b_{\min}, b_{\max}]$ such that $a_{\max} = 2^{\frac{a_{\min}}{\vmax}}a_{\min}$ and $b_{\max} = 2^{\frac{b_{\min}}{\vmax}}b_{\min}$. PowerTowerSearch then selects one of the two intervals uniformly at random and returns its endpoints as $t_{\min}$ and $t_{\max}$. We let $\Event_2$ denote the event that $\opt \in A \cup B$ and $\Event_5$ denote the event that conditional on $\Event_2$, the right interval of $A$ and $B$ is chosen. 
%Te of them $\frac{}{}$ which  returns $\tmin$ and $\tmax$ such that $\tmax = 2^{\tmin/\vmax}\cdot\tmin$. We let $\Event_2$ denote the event that $\opt \in [\tmin, \tmax]$. %, for the values $\tmin$ and $\tmax$ returned by \hyperref[alg:powersearch]{PowerTowerSearch}. 
In Section~\ref{sec:SecondPeriod}, we discuss the implementation of \hyperref[alg:PowerTower]{PowerTowerSearch} and prove that $\Event_2$ occurs with probability at least $0.9$. 

\item [Period 3 -- BinarySearch:] The third period calls Mechanism~\ref{alg:binarysearch}, usually referred to as BinarySearch, which performs binary search in the interval $[\tmin, \tmax]$ and returns $\tinit$. Assuming a proper execution of BinarySearch, which is implied by event $\Event_3$ formally defined later on, we have that $\opt / 4 \leq \tinit \leq O(\log\log(\tmax/\tmin))\opt$. The analysis of \hyperref[alg:binarysearch]{BinarySearch} is presented in Section~\ref{sec:ThirdPeriod}.

\item [Period 4 -- Exploitation:] The fourth period calls Mechanism~\ref{alg:exploitation}, usually referred to as Exploitation, which performs an adaptive search, starting with $\tinit$, on possible values of the mechanism's linear-price threshold that are powers of $2$ (see also \cite[Section~3]{Bada2012}). Assuming its proper execution, which is also implied by event $\Event_3$, Exploitation collects $\Omega(\opt)$ value from its part of the agent sequence. The analysis of \hyperref[alg:exploitation]{Exploitation} is presented in Section~\ref{sec:FourthPeriod}.
\end{description}

Different periods, as they run, update \hyperref[alg:LMMECH]{LM-Mechanism}'s state with consists of its current solution $S$, the total value collected so far $\val$, the budget available $\Brem$ and the sequence of agents not yet considered $\Nrem$. 
%
Each of the last three periods examine a (relatively small) constant fraction of the agent sequence. We let $\Event_4$ denote the event that the total number of agents examined by the $4$ periods of \hyperref[alg:LMMECH]{LM-Mechanism} is at most $n$ (which along with events $\Event_2$ and $\Event_3$ also imply that the total budget expended is at most $B$). The correctness of \hyperref[alg:LMMECH]{LM-Mechanism} is implied by event $\Event = \Event_1 \cap \Event_2 \cap \Event_3 \cap \Event_4 \cap \Event_5$. The correctness probability of \hyperref[alg:LMMECH]{LM-Mechanism} is analyzed in Section~\ref{sec:successProb}.

%In Lemma~\ref{lemma:GoodEvent}, we show that $\Event$ occurs with probability at least $1/10$. 

The two key technical ingredients in the analysis of \hyperref[alg:LMMECH]{LM-Mechanism} is a partitioning of the last $3$ periods into phases and rounds (as depicted Fig.~\ref{fig:phases}) and \hyperref[alg:TestTHRESHOLD]{TestEstimate}, which is applied to each round and tests if the current threshold is too large or too small. 
%
%Moreover, if \hyperref[alg:TestTHRESHOLD]{TestEstimate} is applied with a threshold $\hat{t} = \Theta(\opt)$, it collects a fair amount of value. Before we outline 
%
We next discuss their main properties. 

%(i) our partitioning into rounds and phases and of (i) \hyperref[alg:TestTHRESHOLD]{TestEstimate}. 

\subsection{Partitioning into Rounds and Phases}
\label{sec:rounds}

To partition the agent sequence into rounds, we employ a standard technique \cite{IndependentTrick}, where round lengths are drawn from a binomial distribution to ensure independence among agents and that the resulting prefix of the sequence is a \emph{random subset} of $\N$, where each agent participates with a given probability (often referred to as the round's \emph{participation probability} and denoted by $q$). 

Formally, for any suffix $\N' \subseteq \N$ of the agent sequence and any $a \in (0, 1)$, an \emph{$a$-round} (or simply a \emph{round}, if $a$ is clear from the context) is a prefix of $\N'$ with length drawn from a binomial distribution $\Bin(|\N'|, a)$ with parameters $|\N'|$ and $a$. We usually refer to $a$ as the the \emph{length parameter} (or the \emph{length probability}) of the round. 
%
Along with the random arrival order of the agents, selecting the length of a round as a binomially distributed random variable ensures that every two agents in $\N'$ are included in the given round independently with probability $a$ (which allows us to analyze the properties of each round using standard concentration inequalities). 

Given the agent sequence $\N$, we define a sequence of $\kappa \geq 1$ rounds with length parameters $a_1, \ldots, a_\kappa$ as follows: we select the length $x_1$ of the $1$st round from the binomial distribution $\Bin(n, a_1)$, let $n_1 = n-x_1$ be the number of agents remaining, select the length $x_2$ of the $2$nd round from $\Bin(n_1, a_2)$, $\ldots$, let $n_i= n_{i-1}-x_i$ be the number of agents remaining after the $i$-th round, select the length $x_{i+1}$ of the $(i+1)$-th round from $\Bin(n_i, a_{i+1})$, $\ldots$, select the length $x_{\kappa}$ of the $\kappa$-th round from $\Bin(n_{\kappa-1}, a_{\kappa})$. The following analyzes the distribution of the resulting random subsets of agents: 

\begin{lemma}\label{lemma:RoundPro}
For any $\kappa \geq 1$, we consider a sequence of round lengths distributed binomially as described above with length parameters $a_1, \ldots, a_\kappa$. Then, each agent $b \in \N$ is included in round $\kappa$ independently of the other agents with participation probability
%
\begin{equation}\label{eq:participation}
                q_{\kappa} = a_{\kappa}\prod_{i=1}^{\kappa-1}(1-a_{i})
\end{equation}
\end{lemma}

\begin{proof}[Proofsketch.]
The proof consists of two parts. First we observe that due to secretary agent arrivals, the probability that an agent belongs to any fixed interval of length $n_i$ of an agent sequence of length $n$ is $n_i / n$. Applying standard properties of the binomial distribution repeatedly, we derive the desired participation probability. Next, we use secretary agent arrivals and the binomial distribution of round lengths and show that the probability that any two agents belong to round $\kappa$ is $q_\kappa^2$. 
%
%This allows us to express the probability of two agents being in the same round as the number of permutations where both agents occupy positions within the round, divided by the total number of permutations. Finally, we repeatedly apply combinatorial identities to obtain the desired result. 
%
A detailed proof is given in Appendix~\ref{A1}. 
%Lemma \ref{lemma:RoundPro} indicates that adjusting the length parameter of a round directly influences its participation probability.
\end{proof}

In our mechanism, we draw multiple rounds with significantly varying lengths, which complicates the calculation of each round's participation probability using \eqref{eq:participation}. The following lower bound on the participation probability allows us to argue about the properties of each round using its length parameter $a$. The proof requires the precise definition of the length parameters of all rounds considered by our mechanism and is deferred to Appendix~\ref{sec:PART}. For the proof, we use that the length parameter of every round is $o(1)$ and that their sum over all rounds is significantly less than $1$ (which also allows us to show that event $\Event_4$ occurs with probability at least $0.97$, see Lemma~\ref{lemma:Fit} in Appendix~\ref{A3}).  

\begin{lemma}\label{lemma:PART}
The participation probability of any round with length parameter $a$ is at least $6a/(10e)$. 
\end{lemma}


%To prove the aforementioned lemma, we express the participation probability as in Lemma \ref{lemma:RoundPro}. Then, we show that the cumulative product of terms one minus the length parameter for each previous Round is bounded by $6/10e$. To achieve this, we analyze the product of each Period's parameters separately and, after some calculations, derive the desired bound. The full proof of this lemma is provided in Appendix \ref{proof:PART}, as it requires a complete definition of all round lengths. %From now on, when drawing a round with length parameter $a$, we will assume its participation probability to be $\frac{a}{3e}$.

%To prove the aforementioned lemma we express the participation probability as in Lemma 3.2.. Then, we prove that the product of one minus the length parameters of previous rounds is bounded by $1/3e$. To do that we consider the product of each period's parameters separately and after some calculations we end up with the desired bound. The proof of this lemma is provided in the Appendix \ref{proof:PART}, as it requires to completely define all of the round lengths first. From now on, when drawing a round with length parameter $a$ we will consider its participation probability to be $\frac{a}{3e}$.

\textit{Good, Dense Rounds and Phases.}
%
%For each round $j$, we let $\N_j$ denote its set of agents. 
%
Our goal is to use TestThreshold on rounds in order to assess wether a threshold $t$ underestimates the true optimum. To this end, we will call an $\left(a,t\right)$-round, a round with length parameter $a$ on which we test a threshold $t$. For our test to be correct consistently we need the expected number of agents included in both the optimal solution and our $\left(a,t\right)$-round to be at least a fixed constant, i.e. $a\ge \frac{\vmax}{\opt}$. Since we don't know the value of $\opt$ and $t$ is our estimate of it, we say that an $\left(a,t\right)$-round is `good' if $a\ge 81\, e\, \vmax/t$. 

We allocate a budget $B_j = 3 \, C \cdot a_j \cdot B$ to every round $j$ with length parameter $a_j$, where $C = 1/(8e)$ is a fixed constant used in the following sections. For each round $j$ with set of agents $\N_j$ and length parameter $a_j$, we let $S^\ast_j \subseteq \N_j$ be the set of agents in round $j$ with maximum value $f(S^\ast_j)$ subject to the budget constraint $\sum_{ i \in S_j^\ast} c_i \leq B_j$, and let $\opt_j = f(S^\ast_j)$ be the optimal value of round $j$. 

We say that an $\left(a_j,t_j\right))$-round $j$ is \emph{dense} if $\opt_j \geq C \cdot a_j \cdot \opt$. Intuitively, a round is dense if it gets a fair share (with respect to its length parameter) of the optimal value. The following shows that `good' rounds are dense with good probability. 

\begin{lemma}\label{lemma:roundProb}
A  `good' $\left(a_j,t_j\right)$-round $j$, such that $t_j\le \opt$, is dense with probability at least $0.9$.
\end{lemma}

The high level intuition behind the proof of Lemma~\ref{lemma:roundProb} is that a good round is a random set that includes, in expectation, sufficiently many agents of the optimal solution. The requirement on the round being `good' is mostly technical and is used in the proof of Lemma~\ref{lemma:roundProb}. Lemma~\ref{lemma:roundProb} is key to the  analysis of \hyperref[alg:LMMECH]{LM-Mechanism} and can be found in Section~\ref{proof4.5}.


Applying \hyperref[alg:TestTHRESHOLD]{TestEstimate} to a dense round with threshold $t \leq \opt / 4$ should verify that $t$ is not too large and can be increased. To amplify the confidence of our test, we apply \hyperref[alg:TestTHRESHOLD]{TestEstimate} to a sequence of rounds. A \emph{$\left(\delta,a,t\right)$-phase} is a sequence of $\delta$ consecutive rounds all with the same length parameter $a$ and all testing the same threshold $t$ (see also Fig.~\ref{fig:phases}). 
%
We note that \hyperref[alg:TestTHRESHOLD]{TestEstimate} is always applied with the same threshold $t$ to every round of a given phase.
%
A $\left(\delta,a,t\right)$-phase is \emph{dense} if at least $\delta / 2$ of its rounds are dense. In Appendix~\ref{negdep}, we show that: 

\begin{lemma}\label{lemma:PhaseProbability}
A $\left(\delta,a,t\right)$-phase, consisting of `good' rounds and with $t\le \opt$, is dense with probability at least $1-\exp(\frac{-4\cdot\delta}{45})$.
%    \begin{equation}
 %       1 - \exp\lp\{ - \frac{4\cdot \delta}{45} \rp\} 
%    \end{equation}
\end{lemma}

For the proof of Lemma~\ref{lemma:PhaseProbability}, we observe that the events that different $a$-rounds are dense are negatively associated, in the sense that conditional on a round not being dense, the event that another round is dense becomes more likely. Then, Lemma~\ref{lemma:PhaseProbability} by an application of standard concentration bounds. 

\begin{remark}We should highlight that the randomness in the agent arrivals and the internal randomness of \hyperref[alg:LMMECH]{LM-Mechanism} are only used (i) to partition the agent sequence into rounds consisting of random subsets of agents; and (ii) to select the right interval between $A$ and $B$ in Period $2$. Everything else in \hyperref[alg:LMMECH]{LM-Mechanism} is deterministic. In fact, once the rounds are formed, \hyperref[alg:TestTHRESHOLD]{TestEstimate} and the other parts of  \hyperref[alg:LMMECH]{LM-Mechanism} operate without assuming anything about the order in which the agents arrive. \qed
 \end{remark}

\subsection{Estimating the Probability of Success}
\label{sec:successProb}

In the following, we always assume that \hyperref[alg:LMMECH]{LM-Mechanism} gathers positive value only in realizations where event $\Event$ holds (and always condition on $\Event$, even without mentioning that). As discussed after the outline of \hyperref[alg:LMMECH]{LM-Mechanism}, $\Event$ is the conjunction of events $\Event_1, \ldots, \Event_5$ formally defined below.

%\begin{definition}\label{goodevent}
%    $\Event$ is the event where all the following hold:
    \begin{enumerate}
        \item [$\Event_1$:] An agent with the maximum utility $\vmax$ is included in Period 1.
        
        \item [$\Event_2$:] In Period 2, the interval $[\vmax, n \cdot \vmax]$ is partitioned into sub-intervals. \hyperref[alg:PowerTower]{PowerTowerSearch} identifies two consecutive intervals $A$ and $B$. Then, $\Event_2$ denotes that $\opt / 8 \in A \cup B$.
        
        \item [$\Event_3$:] Every phase during periods $3$ and $4$ is dense with respect to the threshold $\hat{t}$ used by \hyperref[alg:TestTHRESHOLD]{TestEstimate} for that phase.  

        
        \item [$\Event_4$:] The sum of the realized round lengths drawn by \hyperref[alg:LMMECH]{LM-Mechanism} is less than $n$.
        
        \item [$\Event_5$:] Given $\Event_2$, the interval $I$ of $A$ and $B$ chosen uniformly at random satisfies $\opt / 8 \in I$. 
    \end{enumerate}
%\end{definition}


%We define event $\Event$ through a number of events, formally stated below, that must hold and then we claim that event $\mathcal{E}$ holds with constant probability. 
The following shows that event $\Event$ occurs with constant probability/

\begin{lemma}\label{lemma:GoodEvent}
    Event $\Event$ happens with probability at least $1/20$.
\end{lemma}
\begin{proof}
%As we haven't yet presented all the details of the mechanism, we cannot present the complete proofs of the bounds here. The full proofs can be found 
In Appendix~\ref{A3}, we show that $\Prob{\Event_2} \ge 0.9$ in Lemma~\ref{lemma:E2}, $\Prob{\Event_3} \ge 0.9$ in Lemma~\ref{lemma:BinaryProb} and $\Prob{\Event_4} \geq 0.97$ in Lemma~\ref{lemma:Fit}.
%The interested reader may refer to the Appendix \ref{A3}, more specifically $\Prob{\mathcal{E}_2}$ is bounded in Lemma~\ref{lemma:E2}, $\Prob{\mathcal{E}_3}$ is bounded in Lemma~\ref{lemma:BinaryProb} and $\Prob{\mathcal{E}_4}$ is bounded in Lemma~\ref{lemma:Fit} .
%    For the events above the following bounds hold:
%   \begin{equation}
%        \begin{split}
%            \Prob{\mathcal{E}_1} = \frac{1}{3}, \Prob{\mathcal{E}_2} \ge 0.9, \Prob{\mathcal{E}_3} \ge 0.9 ,\Prob{\mathcal{E}_4} \ge 0.97,
%            \Prob{\mathcal{E}_5 | \mathcal{E}_2} = 0.5
%        \end{split}
%    \end{equation}
We also have that $\Prob{\Event_1} \geq 1/3$ due to random agent arrivals and the properties of the binomial distribution. Finally, we argue that $\Prob{\Event_5\,|\,\Event_2} = 1/2$. Given that the set of possible realizations is restricted to a subset of those where event $\Event_2$ occurs, $\Event_5$ occurs with probability $1/2$, independently of any other events in the realizations considered in the conditioning on $\Event_2$. I.e., we can think that once we condition that the event $\Event_2$ occurs, the mechanism flips a fair coin (which is completely independent from $\Event_2$ and also from $\Event_1 \cap \Event_3 \cap \Event_4$): with probability $1/2$, the mechanism selects the right interval (and proceeds collecting value), and with probability $1/2$, the mechanism selects the wrong interval (and does not collect anything). Then, 
%
\[
\Prob{\lnot \Event_1 \cup \lnot \Event_2 \cup \lnot \Event_3 \cup \lnot \Event_4} \leq 2/3 + 0.1 +0.1 + 0.03 \leq 0.9 \]
%
Putting everything together:
%
$\Prob{\mathcal{E}} = \Prob{\Event_1 \cap \Event_2 \cap \Event_3 \cap \Event_4} \cdot \Prob{\Event_5 \,|\,\Event_2} \geq \frac{1}{10} \cdot \frac{1}{2} = 1/20$.
%%Putting everything together:
%
 %   \begin{equation*}
 %       \begin{split}
 %           \Prob{\mathcal{E}} = \Prob{\lnot (\lnot \mathcal{E}_1 \cup \lnot \mathcal{E}_2 \cup \lnot \mathcal{E}_3 \cup \lnot \mathcal{E}_4 )} \cdot \Prob{\mathcal{E}_5 | \mathcal{E}_2}  \ge (1-\frac{2}{3} - 0.1 - 0.1-0.03) \cdot \frac{1}{2} \ge \frac{1}{20}
 %       \end{split}
 %   \end{equation*}
\end{proof}



\subsection{Estimation Testing}
\label{sec:estimate}

The key building block of our \hyperref[alg:LMMECH]{LM-Mechanism} is an online test, usually referred to as \hyperref[alg:TestTHRESHOLD]{TestThreshold}, which is applied to a $(\delta,a,t)$-phase. \hyperref[alg:TestTHRESHOLD]{TestThreshold}, whose pseudocode can be found in Mechanism~\ref{alg:TestTHRESHOLD}, determines whether:

\begin{quote}
    \textit{it is possible to collect value at least $C\cdot a\cdot \hat{t}$, where $C = 1/(7e)$ by applying linear-pricing with threshold $\hat{t}$ to a round with length parameter $a$.}
\end{quote}

\hyperref[alg:TestTHRESHOLD]{TestThreshold} determines the set $\N_{\text{round}}$ of agents of each round by drawing their sizes from a binomial distribution $\Bin(|\Nrem|, a)$. Then, it offers every agent $b \in \N_{\text{round}}$ the corresponding linear price $p_b = f_S(b) B /\hat{t}$ and collects value $f_S(b)$, if the price $p_b$ is accepted. We write that \hyperref[alg:TestTHRESHOLD]{TestThreshold} succeeds in a round, if it collects value at least  $C\cdot a\cdot \hat{t}$ from the present round, and fails, otherwise. We write that \hyperref[alg:TestTHRESHOLD]{TestThreshold} succeeds in a phase (or just succeeds) when at least half of the rounds are successful.

\hyperref[alg:TestTHRESHOLD]{TestThreshold} is the only point where \hyperref[alg:LMMECH]{LM-Mechanism} makes offers to the agents, collects value and expends budget. Periods $2$ and $3$ determine the length parameters and the number of phases to which \hyperref[alg:TestTHRESHOLD]{TestThreshold} is applied, and use their success / fail responses to guide our adaptive search for a good estimate of $\opt$. Then, Period $4$ applies \hyperref[alg:TestTHRESHOLD]{TestThreshold} to phases with appropriate length parameters in order to collect a total value of $\Omega(\opt)$. 
If during the execution of \hyperref[alg:LMMECH]{LM-Mechanism}, either the budget or the agent sequence is exhausted, or an agent with value larger than $\vmax$ is found, event $\Event$ is violated and \hyperref[alg:TestTHRESHOLD]{TestThreshold} (and \hyperref[alg:LMMECH]{LM-Mechanism}) abort (without  any value). 


%We present a our function TestThreshold in pseudocode:
\def\valest{\mathrm{val}\mbox{-}\mathrm{test}}
\begin{algorithm}[t]
  \caption{TestThreshold}\label{alg:TestTHRESHOLD}
 \begin{algorithmic}[1]
     \STATE \textbf{Input:} Current state $(S, \val, \Nrem, \Brem)$, budget $B$, phase size $\delta$, length parameter $a$, threshold $\hat{t}$.
     \STATE \textbf{Initialization:} $C \leftarrow \frac{1}{7e}$, successes$\gets 0$
    %\STATE
    \FOR{$j=1$ to $\delta$}
    \STATE Draw $\tau \sim \Bin(|\Nrem|,a)$, $\mathcal{N}_{\text{round}} \leftarrow \Nrem[1 : \tau]$
    \STATE $\Nrem \leftarrow \Nrem \setminus \mathcal{N}_{\text{round}},i \leftarrow 1, \valest \leftarrow 0$
    \STATE \textbf{if} $\Nrem = \emptyset$ \textbf{then} abort \textbf{end if}
     \WHILE{$i \leq |\mathcal{N}_{\text{round}}| $}
     \STATE \textbf{if} $f(\mathcal{N}_{\text{round}}[i]) > v_{max}$ \textbf{then} abort \textbf{end if}
    \STATE \textbf{else} 
    %\bindent
    \begin{ALC@g}
    \STATE $p_i \leftarrow f_S(\mathcal{N}_{\text{round}}[i]) \cdot B\,/\, \hat{t} $\label{step:price}
    \IF{$ p_i \geq c_i$}
    \STATE $S \gets S\cup \{\mathcal{N}_{\text{round}}[i]\}$
    \STATE $\val \gets \val + f_S(\mathcal{N}_{\text{round}}[i])$\ \  and\ \ $\valest \leftarrow \valest + f_S(\mathcal{N}_{\text{round}}[i])$
    \STATE $\Brem \leftarrow \Brem - p_i$ 
    \STATE \textbf{if} $\Brem \leq 0$ \textbf{then} abort \textbf{end if}
    \ENDIF
    \end{ALC@g}
    %\eindent
    \STATE \textbf{end if}
    \STATE \textbf{if} $\valest \ge C \cdot a\cdot \hat{t}$ \textbf{then} $\text{successes} \gets \text{successes} +1$ \textbf{end if}\label{step:test}
    \STATE $i \gets i + 1$
    \ENDWHILE
    \ENDFOR
    \STATE \textbf{if} $\text{successes} \ge \delta /2$ \textbf{then} $(1, S, \val, \Nrem, \Brem)$ \textbf{else }$(0, S, \val, \Nrem, \Brem)$
    %\STATE \textbf{return} $(0, S, \val, \Nrem, \Brem)$
 \end{algorithmic}
\end{algorithm}

%We next condition on \hyperref[alg:TestTHRESHOLD]{TestThreshold}'s application with threshold $\hat{t}$ to a dense round with respect to $\hat{t}$ and establish its main properties (for dense rounds). We highlight that every time \hyperref[alg:TestTHRESHOLD]{TestThreshold} is applied in periods $2$ to $4$, the length parameter $a$ of each round is good with respect to chosen $\hat{t}$ (see Lemma~\ref{lemma:good1} in Section~\ref{sec:SecondPeriod}, Lemma~\ref{lemma:good2} in Section~\ref{sec:ThirdPeriod} and Lemma~\ref{lemma:good3} in Section~\ref{sec:FourthPeriod}). Therefore, due to Lemma~\ref{lemma:roundProb}, every round to which \hyperref[alg:TestTHRESHOLD]{TestThreshold} is applied is dense with probability at least $0.9$.

We next consider applications of \hyperref[alg:TestTHRESHOLD]{TestThreshold} to `good' rounds/phases and establish its main properties. We highlight that every time \hyperref[alg:TestTHRESHOLD]{TestThreshold} is applied in periods $2$, $3$ and $4$, the length parameter $a$ of each round respects the goodness property with respect to the chosen $t$ (see Proposition~\ref{prop:good1} in Section~\ref{sec:SecondPeriod}, Proposition~\ref{prop:good2} in Section~\ref{sec:ThirdPeriod} and Proposition~\ref{prop:good3} in Section~\ref{sec:FourthPeriod}). 
%
In fact, this is the key objective behind how the four periods of \hyperref[alg:LMMECH]{LM-Mechanism} are structured: 
\hyperref[alg:PowerTower]{PowerTowerSearch}, \hyperref[alg:binarysearch]{BinarySearch} and 
\hyperref[alg:exploitation]{Exploitation} are carefully designed in order to ensure that  \hyperref[alg:TestTHRESHOLD]{TestThreshold} is applied to `good' rounds.
%
Therefore, due to Lemma~\ref{lemma:roundProb} (resp. Lemma~\ref{lemma:PhaseProbability}), every round (resp. phase) to which \hyperref[alg:TestTHRESHOLD]{TestThreshold} is applied is dense with probability at least $0.9$ (resp. close enough to $1$).

Due to the linear pricing scheme in step~\ref{step:price} and the success criterion in  step~\ref{step:test}, every application of \hyperref[alg:TestTHRESHOLD]{TestThreshold} to a `good' $(a,\hat{t})$-round (regardless of its success or failure) expends a budget of at most 
%
\[ \frac{C \cdot a \cdot \hat{t} + \vmax}{\hat{t}} \cdot B \leq \left( C \cdot a + \frac{\vmax}{\hat{t}}\right) B \leq 2 \cdot C \cdot a \cdot B\,,
\]
%
where the last inequality holds because the round is good. Thus:

\begin{property}\label{property:a}
When \hyperref[alg:TestTHRESHOLD]{TestThreshold} is applied to a `good' $\left(a,\hat{t}\right)$-round the budget expended does not exceed the round's budget $B_j = 3\cdot C\cdot a\cdot B$. 
\end{property}

Using Property~\ref{property:a}, that \hyperref[alg:TestTHRESHOLD]{TestThreshold} and that the sum of the length parameters of all rounds is less than $1$, we prove the following in Appendix~\ref{budgetproofs}:
%which upper bounds the total budget expended by \hyperref[alg:LMMECH]{LM-Mechanism}:

\begin{lemma}\label{BUDGET}
The total budget expended by \hyperref[alg:LMMECH]{LM-Mechanism} is at most $B/10$.
\end{lemma}

%We will prove that TestThreshold function correctly identifies if the current estimate is below $\opt$ when the round is dense. This is formally presented in the following lemma:

%\begin{lemma}
%\label{lemma:Justify}
\begin{property}\label{property:b}
\hyperref[alg:TestTHRESHOLD]{TestThreshold} succeeds (i.e., it collects value at least $C\cdot a \cdot \hat{t}$) when applied to an $\left(a_j,\hat{t}_j\right)$-round that is dense and $\hat{t}_j \le \opt / 4$.
\end{property}
%\end{lemma}
\begin{proof}
Let $S_j$ be the part of the solution obtained by applying \hyperref[alg:TestTHRESHOLD]{TestThreshold} with threshold $\hat{t}$ to an $a_j$-round $j$ that is dense with respect to $\hat{t}$. We assume that the value of each agent $b$ is $f(b) \leq\vmax$, as otherwise \hyperref[alg:TestTHRESHOLD]{TestThreshold} aborts. Let also $S^\ast_j$ denote the optimal solution for round $j$.

Suppose some agents $ b \in S_j^{*} \setminus S_j $ were included in the optimal solution but not in $S_j$. This can happen for one of two reasons: either their private costs exceeded the mechanism's offer, i.e., $ c_b > f_{S_j}(b)\,B\,/\,\hat{t}$, or $\Brem$ was insufficient to make an appropriate offer to $ b $. However, by design, only the first case can occur. Due to our linear pricing scheme, no agent is ever offered a price larger than $\vmax\,B\,/\,\hat{t} \le \frac{B}{1024}$, where the inequality holds because we always use thresholds $\hat{t} \geq 1024\,\vmax$. Combining this with Lemma~\ref{BUDGET}, we conclude that all agents $ b \in S_j^{*} \setminus S_j $ declined our offer.

We can rewrite the sum of the values of these agents as:
\begin{equation}\label{eq:basic}
            \sum_{b\in S_j^{*} \setminus S_j} f_{S_j}(b) = \sum_{b\in S_j^{*} \setminus S_j} \frac{f_{S_j}(b)}{c_b} \cdot c_b < \sum_{b\in S_j^{*} \setminus S_j}\frac{\hat{t}}{B}\cdot c_b \le \hat{t}\cdot \frac{B_j}{B} = 3 \cdot C\cdot a_j \cdot \hat{t}
\end{equation}
%
The second to last inequality follows from the hypothesis $ c_b > f_{S_j}(b)\,B\,/\,\hat{t}$ and the last inequality follows from the fact that $S_j^{*}$ is budget-feasible with respect to $B_j = 3 \cdot C\cdot a_j\cdot B$.

Using monotonicity and submodularity of the valuation function we have:
%
\begin{equation}\label{eq:basic2}
    f(S_j^{*})-f(S_j) \le f(S_j^{*}\cup S_j) -f(S_j) \le \sum_{b\in S_j^{*}\setminus S_j} f_{S_j}(b)
\end{equation}

Applying \eqref{eq:basic} to \eqref{eq:basic2}, we get that $f(S_j) \geq f(S_j^{*}) - 3 \cdot C\cdot a_j \cdot \hat{t}$. Finally, using the hypothesis that round $j$ is dense, and hence $f(S^\ast_j) \geq C \cdot a_j \cdot \opt$, we obtain that:
\[
        f(S_j) \ge C \cdot a_j \cdot \opt -  3\cdot C\cdot a_j \cdot \hat{t}
\]
%
Since $\hat{t} \le \frac{\opt}{4}$, 
%
$f(S_j)\ge 4\cdot C \cdot a_j \cdot \hat{t} -3 \cdot C\cdot a_j\cdot\hat{t} = C\cdot a_j \cdot \hat{t}$.
%
Thus, \hyperref[alg:TestTHRESHOLD]{TestThreshold} it collects value at least $C\cdot a_j \cdot \hat{t}$ and succeeds. %This concludes the proof of the lemma.
\end{proof}
 The Lemma above extends to phases the following way:
 \begin{lemma}\label{lemma:phasesucc}
     \hyperref[alg:TestTHRESHOLD]{TestThreshold} succeeds (i.e., it collects value at least $C\cdot a \cdot \hat{t}$ and returns $1$) when applied to a $\left(\delta_j,a_j,\hat{t}_j\right)$-phase that is dense and $\hat{t}_j \le \opt / 4$.
 \end{lemma}

The next four sections discuss how periods $1$ to $4$ are structured so that \hyperref[alg:TestTHRESHOLD]{TestThreshold} is applied to phases so that (conditional on event $\Event$) a total value of $\Omega(\opt)$ is collected. 

 %Throughout the rest of our analysis we are going to assume that event $\mathcal{E}$ occurred.

%Another important property of our testing procedure is that we expend very little budget in each round compared to $B$. We prove that our mechanism never runs out of Budget during the rounds that will described later, but we defer the proof to the Main Theorem section, as we need to first define all the round lengths we are going to use.

%We have now established the main properties of our criterion and we are ready to showcase how to use it to find the correct interval from $D$ in which $OPT$ is contained.


