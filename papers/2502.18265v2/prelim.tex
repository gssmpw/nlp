\section{Model, Definitions and Preliminaries}
\label{sec:prelim}

%\paragraph{Preliminaries.}
%
A set function $f : 2^{\N} \to \reals_{\geq 0}$ is \emph{non-decreasing} (often referred to as \emph{monotone}) if for all $S \subseteq T \subseteq \N$, $f(S) \leq f(T)$. A function $f$ is \emph{submodular} if $f$ has non-increasing marginal values, i.e., for all $S \subseteq T \subseteq \N$ and all $i \not\in T$, $f(T \cup \{i\}) - f(T) \leq f(S \cup \{i\}) - f(S)$. We let $f_S(i) = f(S \cup \{ i \}) - f(S)$ denote the marginal value of $i$ with respect to $S$. In this work, we consider valuation functions $f : 2^{\N} \to \reals_{\geq 0}$ that are normalized, i.e. have $f(\emptyset) = 0$, monotone and submodular. 

A valuation function $f$ is accessed by \emph{value queries}, which for any given set $S \subseteq \N$, return $f(S)$. We let $\vmax = \max_{i \in \N} \{ f(\{ i \})\}$ denote the maximum value of any agent. In the following, all logarithms are base-$2$ unless stated otherwise. 

%\paragraph{Budget-Feasible Procurement Auctions and Mechanisms.}
\vskip2pt\textit{Budget-Feasible Procurement Auctions and Mechanisms.}
%
We consider procurement auctions with a set $\N = \{1, \ldots, n\}$ of agents. Each agent $i$ has a private cost $c_i \in \reals_{\geq 0}$ for participating in the solution. The buyer has a budget $B \in \reals_{\geq 0}$ and aims to maximize a monotone submodular valuation $f : 2^{\N} \to \reals_{\geq 0}$, subject to the constraint that the sum of payments to the agents must be at most $B$. 

A (deterministic) \emph{direct revelation mechanism} $\mathcal{M} = (A, p)$ is a pair consisting of an allocation function and a payment function. The allocation function $A: \mathbb{R}_{\geq 0}^{n} \to 2^{\N}$ maps a bid vector $\vec{b} = (b_1, \ldots, b_n)$ submitted by the agents to a solution $A(\vec{b}) = S \subseteq \N$. The payment function $p: \mathbb{R}_{\geq 0}^{n} \to \mathbb{R}_{\geq 0}^{n}$ computes the payments $p(\vec{b})$ allocated to the agents for the solution $A(\vec{b})$. A randomized mechanism is a probability distribution over deterministic mechanisms. 

Given a mechanism $\mathcal{M} = (A, p)$, each agent $i$ aims to maximize their \emph{utility} $u_i(\vec{b})$ through their bid. It is $u_i(\vec{b}) = p_i(\vec{b}) - c_i$, if $i$ is included in the solution $A(\vec{b})$, and $u_i(\vec{b}) =  p_i(\vec{b})$, otherwise. 

A mechanism is \emph{budget-feasible} if for every bid vector $\vec{b}$, $\sum_{i \in \N} p_i(\vec{b}) \leq B$. A mechanism is \emph{individually rational}, if for every bid vector $\vec{b}$ and all agents $i \in \N$, $u_i(\vec{b}) \geq 0$. Hence,  $p_i(\vec{b}) \geq 0$ for all agents $i \in \N$, and $p_i(\vec{b}) \geq c_i$ for all agents $i$ included in the solution $A(\vec{b})$. Due to individual rationality and the budget constraint, we always let $p_i(\vec{b}) = 0$ for all agents $i \not\in A(\vec{b})$.

A mechanism is \emph{truthful}, if reporting their cost $c_i$ is a dominant strategy for every agent $i$, i.e., for all $i\in \N$, all $b \in \reals_{\geq 0}$ and every bidding vector $\vec{b}_{-i}$ of the other agents, $u_i(\vec{b}_{-i}, c_i) \geq u_i(\vec{b}_{-i}, b)$. 

For randomized mechanisms, that we consider in this work, we require that the mechanism is budget feasible and individually rational \emph{with certainty} and \emph{universally truthful}, i.e., the mechanism is a probability distribution over deterministic truthful mechanisms. 

%\paragraph{Online Budget-Feasible Procurement Auctions.}
\vskip2pt\textit{Online Budget-Feasible Procurement Auctions.}
%
In the online setting, agents arrive sequentially and the decision about whether the present agent $i$ is included in the solution and $i$'s payment is \emph{irrevocable} and is taken upon $i$'s arrival, using only information about the agents having arrived before $i$ and without any knowledge about future agent arrivals whatsoever. 

In this work, we consider an online procurement auctions with \emph{secretary arrivals}, where the set $\N$ of agents is chosen adversarially and the agents arrive in random order (formally, the agents' arrival order is drawn uniformly at random from all permutations of $\N$). 

%\paragraph{Posted-Price Mechanisms.}
\vskip2pt\textit{Posted-Price Mechanisms.}
%
We restrict our attention to (online) sequential \emph{posted-price} mechanisms. A posted-price mechanism decides on the price $p_i$ of each agent $i$ upon $i$'s arrival and makes $i$ a take-it-or-leave-it offer. Agent $i$ accepts or rejects the offer, depending on whether $p_i \geq c_i$ or not, without revealing any other information about their private cost $c_i$ whatsoever. If agent $i$ accepts the offer, $i$ is included in the mechanism's solution and the budget currently available decreases by $p_i$. Otherwise, we let $i$'s price be $p_i = 0$ and agent $i$ is discarded. 

Randomized posted-price mechanisms are individually rational and universally (and obviously) truthful, because among the two options available (accept or reject the mechanism's offer), the agents select the option that maximizes their utility $\max\{ 0, c_i - p_i \}$. Moreover, posted-price mechanisms are budget feasible, assuming that each price offered doed not exceed the budget currently available. 

We further restrict our attention to \emph{linear-price} mechanisms, which maintain a threshold $\hat{t}$ and offer a price $p_i = f_S(i) B / \hat{t}$ to every agent $i$ (assuming that $p_i$ does not exceed the budget currently available). The threshold $\hat{t}$ is an estimation of the optimal value and is maintained based on the marginal values of previous agents and their accept / reject decisions. 

We say that a linear-price mechanism is \emph{adaptive}, if it may use  a different threshold for determining the linear price offered to each agent, and \emph{non-adaptive}, if the mechanism decides on a fixed threshold before its first offer to an agent and uses it for all remaining agents.

%\paragraph{Competitive Ratio.}
\vskip2pt\textit{Competitive Ratio.}
%
We evaluate the performance of our online mechanisms using the competitive ratio \cite{BoroYan1998}. We compare the performance our our mechanism against the offline optimum, denoted $\opt$, which is the value of an optimal solution $S^\ast \subseteq \N$ that maximized $f(S^\ast)$ subject to $\sum_{i \in S^\ast} c_i \leq B$. We note that $\opt = f(S^\ast)$ is defined with respect to a computationally unrestricted algorithm with full advance knowledge of the set $\N$ of agents and their costs $c_1, \ldots, c_n$. 

In our setting, the competitive ratio $cr$ of an algorithm is the worst-case ratio, over all possible sets of agents $\N$, of $\opt$ divided by the algorithm's expected cost $\mathbb{E}[\alg]$ (where the expectation is taken over both the agents' arrival order and the algorithm's random choices). 



\section{A Logarithmic Lower Bound for Non-Adaptive Linear-Price Mechanisms}
\label{sec:lower_bound}

To motivate the use of non-adaptive linear-price mechanisms, we first prove the following: 

\begin{theorem}\label{thm:non-adaptive-lower-bound} Every randomized non-adaptive linear-price mechanism is at least $\frac{\log n}{2}$-competitive.
\end{theorem}

\begin{proof}[Proofsketch.]
We consider a family of instances $\mathcal{I} = \{ I_0, I_1, ..., I_{\log n} \}$. Every instance consists of $n$ value-private cost pairs $(v,c)$, where for instance $I_i$, $v = 1$ and $c = B/2^i$, for every $i \in \{0,1,...,\log n\}$. %(i.e., $I_i$ consists of $n$ value-cost pairs $(1, B/2^i)$).

We construct a probability distribution $\mathcal{D}_{\mathcal{I}}$ over the family $\mathcal{I}$ of instances. For every $i = 0, 1, 2, \ldots, \log(n)-1$, we let $I_{i}$ occur with probability $p_{i} = 1/2^{i+1}$ in $\mathcal{D}_{\mathcal{I}}$. We let $I_{\log n}$ occur with probability $p_{\log n} = 1/n$ in $\mathcal{D}_{\mathcal{I}}$, so that the sum of $p_i$s is equal to $1$. To conclude the proof, we show that the expected value of any deterministic non-adaptive linear-price mechanism on $\mathcal{D}_{\mathcal{I}}$ is at most $1$, while the expected value of the optimal solution on $\mathcal{D}_{\mathcal{I}}$ is $\frac{\log n}{2}+1$. Then, the theorem follows from Yao's principle %, which extends the lower bound to any randomized mechanism
(\cite[Chapter~8.4]{BoroYan1998} and \cite{Yao1977}). We provide a detailed proof in Section~\ref{sec:app:lower_bound}.
\end{proof}

The key idea behind Theorem~\ref{thm:non-adaptive-lower-bound} is that a non-adaptive linear-price mechanism has to decide on a fixed threshold $\hat{t}$ before getting any information about the agents' private costs (which for posted-price mechanisms is obtained only by offering prices to the agents and collecting their accept / reject responses). Hence, essentially the best approach of non-adaptive linear-price mechanisms is to learn $v_{\max}$ and select a threshold $\hat{t}$ from $\{ 1, 2, 4, \ldots, 2^{\log n}\}$ uniformly at random (which is what the posted-price mechanism of \cite[Section~4]{Bada2012} actually does). Hence, to improve on the logarithmic competitive ratio of \cite[Section~4]{Bada2012}, we next present an adaptive linear-price mechanism. 