\def\N{\mathcal{N}}
\def\reals{\mathbb{R}}
\def\nats{\mathbb{N}}
\def\eps{\varepsilon}

\section{Introduction}
\label{s:intro}

We consider a model of procurement auctions, introduced by Singer \cite{Singer10}, where a set $\N$ of $n$ strategic agents are willing to offer their services to a buyer. Each agent $i \in \N$ has a private cost $c_i \in \reals_{\geq 0}$ for offering their service. The buyer aims to maximize a public monotone submodular value function $f : 2^{\N} \to \reals_{\geq 0}$ for the subset of agents hired, subject to a hard budget constraint that the total amount of payments to the agents should not exceed the buyer's budget $B$. 

In the algorithmic problem, the buyer is aware of the agent costs $c_1, \ldots, c_n$ and aims to compute a subset $S \subseteq \N$ of agents (a.k.a. a solution) that maximizes $f(S)$ subject to $\sum_{i \in S} c_i \leq B$. Maximizing a monotone submodular function subject to a budget (a.k.a.  knapsack) constraint is a classical NP-hard optimization problem whose approximability is well understood (see e.g., \cite{NemhauserWF78,KhullerMN99,Sviri04,BadaDO19}). 

In the mechanism design setting of \cite{Singer10,Singer13}, we aim to design \emph{individually rational} and \emph{truthful} (a.k.a. \emph{incentive compatible}) mechanisms that compensate each agent $i$ included in the solution with a carefully chosen payment $p_i$. Individual rationality requires that $p_i \geq c_i$, i.e., that the payment allocated to each agent in the solution should be no less than their cost. Truthfulness requires that the agent selection and the corresponding payments should ensure that no agent has an incentive to falsely report a higher cost to the mechanism, in an attempt to extract a larger payment. 
%
The mechanism must be budget-feasible with respect to the payments $p_i$ to the agents included in the solution, i.e., it must be $\sum_{i\in S} p_i \leq B$, and in the ideal case, the mechanism should also be computationally efficient.   
%
Conceptually, individual rationality and truthfulness require that the payments are large enough to cover the agent costs and incentivize truthful reporting, while budget feasibility requires that the payments are kept low, so that the value of the solution is maximized. 

%i.e. the decision about procuring the service of an agent or not and the corresponding payment must endure that reporting their true cost to the mechanism is a dominant strategy for all agents. 
%the price $p_i$ paid to each agent $i \in \N$ for her service 
%whose service is procured must satisfy \emph{individual rationality}, which requires that $c_i \leq p_i$, i.e., the agent's cost does not exceed the price offered by the buyer, and \emph{budget feasibility}, which requires that the sum of payments is at most $B$. Moreover, in addition to computational efficiency, the mechanism must be \emph{truthful} (or \emph{incentive-compatible}), i.e. the decision about procuring the service of an agent or not and the corresponding payment must endure that reporting their true cost to the mechanism is a dominant strategy for all agents. 

%The main challenge in the design of truthful mechanisms for procurement auctions is that the budget constraint restricts the total amount of payments used to compensate high-value agents so that they find it in their best interest to report their true cost to the mechanism (and not to declare a cost higher than their true one, thus aiming to extract a higher payment from the mechanism). 

Compensating strategic agents for their effort in a way that is incentive-compatible, value-efficient and budget-feasible is an intriguing algorithmic problem with significant practical applications, which include crowdsourcing \cite{SingerM13,SinglaK13,Liu2015}, crowdsensing \cite{Zheng2022}, participatory sensing \cite{Restuccia2016} (see also the references in the survey of Liu et al. \cite{LCLW2024_survey}).

Since its introduction by Singer \cite{Singer10} and due to its practical and theoretical significance, the problem of designing truthful mechanisms for budget-feasible procurement auctions has received significant attention. A long line of research has shown that budget-feasible procurement can be approximated within small constant factors by computationally efficient truthful mechanisms for additive \cite{GravinJLZ2020}, monotone submodular \cite{JalalyT2021,BalkanskiGGST2022} and non-monotone submodular valuations \cite{AmanatidisKS22,BalkanskiGGST2022} (see Section~\ref{sec:related} for a more detailed discussion), in the \emph{bidding} (a.k.a. \emph{sealed-bid}) setting, where the mechanism may ask the agents to report their costs. 

Previous work also considered \emph{online} budget-feasible procurement auctions with \emph{secretary arrivals}, where the agents are chosen adversarially and arrive sequentially in random order, and the mechanism has to decide irrevocably about selecting an agent in the solution and the corresponding payment, upon the agent's arrival and without any knowledge about future agent arrivals. With agent bidding,  online procurement auctions admit constant factor approximations for additive \cite{SingerM13}, monotone submodular \cite{Bada2012} and non-monotone submodular \cite{AmanatidisKS22} valuations. 

However, in practical applications, such as crowdsourcing, in addition to dealing with the agents online, it is desirable (and often necessary) that the mechanism interacts with the agents through \emph{sequential posted pricing}. 
%
A posted-price mechanism decides on the payment $p_i$ offered to an agent $i$ upon $i$'s arrival. Then, the agent accepts or rejects the offer, depending on whether $p_i \geq c_i$ or not, without revealing any other information about their private cost $c_i$ whatsoever. 
%
Posted-price mechanisms are used extensively in practice, due to their simplicity, transparency, obvious truthfulness \cite{Li2017} and agent privacy regarding their costs. E.g., posted pricing is the standard form of pricing in crowdsourcing applications, such as Mechanical Turk (see also the discussion in \cite{BalkHart16,ChawlaHMS2010} for the motivation and the advantages of sequential posted-price mechanisms). 

%However, in practical applications, such as crowdsourcing, in addition to having to deal with the agents online, it is desirable (or even necessary) that the mechanism interacts with the agents through \emph{posted prices}. 
%
%A posted-price mechanism decides on the price $p_i$ for each agent $i$ upon $i$'s arrival and makes $i$ a take-it-or-leave-it offer. Then, the agent $i$ accepts or rejects the offer, depending on whether $p_i \geq c_i$ or not, without revealing any other information about their private cost $c_i$ whatsoever. 
%
%Posted-price mechanisms are used extensively in practice, due to their simplicity, obvious truthfulness \cite{obvious} and agent privacy regarding their costs. In fact, posted pricing is the standard form of pricing in crowdsourcing applications, such as Mechanical Turk (see e.g., the discussion in \cite{BalkHart2018}). 

In the context of online budget-feasible procurement, Badanidiyuru, Kleinberg and Singer \cite{Bada2012} asked whether \emph{posted-price mechanisms are as powerful, in terms of the asymptotic behavior of their competitive ratio, as standard bidding mechanisms}. Towards resolving this question, they presented a randomized constant-competitive posted-price mechanism for symmetric submodular valuations%
%
\footnote{A function $f : 2^{\N} \to \reals_{\geq 0}$ is \emph{symmetric submodular} if there is a nondecreasing concave function $g:\nats \to \reals_{\geq 0}$ such that $f(S) = g(|S|)$ for all $S \subseteq \N$.}
%
and agents sampled independently from an unknown distribution. For monotone submodular valuations and secretary arrivals, Badanidiyuru et al. \cite{Bada2012} presented a randomized constant-competitive  online mechanism with agent bidding, and a randomized $O(\log n)$-competitive posted-price mechanism. Their main open question concerned the existence of a constant-competitive posted-price mechanism for monotone submodular valuations and secretary agent arrivals. 

Subsequently, Balkanski and Hartline \cite{BalkHart16} considered posted-price mechanisms in the Bayesian setting, where the agents arrive as independent samples from a known distribution. They presented mechanisms that are constant-competitive against the Bayesian optimal mechanism for additive, symmetric submodular and monotone submodular value functions. 

\subsection{Contribution and General Approach}
\label{sec:approach}

In this work, we present a randomized constant-competitive posted-price mechanism for online procurement auctions with monotone submodular valuations, thus resolving the main open question of Badanidiyuru et al. \cite{Bada2012}. More formally, we show the following:

\begin{theorem}\label{thm:main_informal}
There is a universally truthful $O(1)$-competitive randomized posted-price mechanism for budget-feasible online procurement auctions with secretary agent arrivals and  monotone submodular buyer's valuations.
\end{theorem}

The mechanism of Theorem~\ref{thm:main_informal} satisfies budget-feasibility with certainty, and is individually rational and universally (and obviously) truthful%
%
\footnote{A randomized mechanism is universally truthful if it can be expressed as a probability distribution over truthful deterministic mechanisms. A mechanism is \emph{obviously truthful} \cite{Li2017} if it has an equilibrium in obviously dominant strategies. A strategy is \emph{obviously dominant} if the best outcome under any possible deviation from it is no better than the worst outcome under the given strategy.}, 
%
with the latter properties following directly from the definition of posted-price mechanisms, where the agents to act in their best interest when selecting whether to accept the payment offered or not. 

Posted-price mechanisms for online procurement typically operate by learning an estimation of the optimal value $\opt$ (usually referred to as a \emph{threshold}, for brevity) $\hat{t}$ and using it to determine the payments (a.k.a. \emph{prices}) offered to the agents. We restrict our attention to \emph{linear-price} mechanisms, where the price offered to agent $i$ is $p_i = f_S(i) B / \hat{t}$, where $B$ is the buyer's budget, $\hat{t}$ is the mechanism's current threshold, $S$ is the subset of agents selected in the solution up to $i$'s arrival, and $f_S(i) = f(S \cup \{ i \}) - f(S)$ is the marginal value due to $i$'s inclusion in the solution.

One can show (see e.g., the proof of Lemma~\ref{lem:threshold} or \cite[Lemma~4.4]{Bada2012}) that linear-price mechanisms with thresholds $\hat{t}$ that remain within a constant fraction of $\opt$ for a constant fraction of the agent sequence are constant-competitive. Hence, the main challenge in the design of efficient linear-price mechanisms (and the main difference between them) is how to maintain a threshold $\hat{t}$ that mostly remains within a constant fraction of $\opt$. 

The posted-price mechanism of Badanidiyuru et al. \cite[Section~4]{Bada2012} uses linear prices and a fixed threshold $\hat{t}$ determined before the first offer to an agent. In Theorem~\ref{thm:non-adaptive-lower-bound}, we show that the competitive ratio of any linear-price mechanism that selects a fixed threshold without any knowledge of the agent costs is $\Omega(\log n)$. Hence, we consider \emph{adaptive linear-price} mechanisms based on a threshold that evolves over time. 

Our approach is inspired by the constant-competitive adaptive linear-price mechanism of \cite[Section~3]{Bada2012} for symmetric submodular valuations and agents sampled independently from an unknown distribution.
%
As in \cite[Section~3]{Bada2012}, our approach is based on an online test (cf. Mechanism~\ref{alg:TestTHRESHOLD}, usually referred to as \emph{TestThreshold}) that efficiently decides if the current threshold $\hat{t}$ is less than $\opt / 4$. TestThreshold comes with an one-sided guarantee: with constant probability, TestThreshold succeeds (i.e., it responds that the current threshold $\hat{t} \leq \opt/4$), if it is applied with any threshold $\hat{t} \leq \opt / 4$ to a random subset of agents of size $\Omega( \vmax n / \opt)$, where $\vmax = \max_{i \in \N} \{ f(\{i\}) \}$ is the maximum value of an agent (Property~\ref{property:b}). If TestThreshold is applied with a threshold $\hat{t} > \opt / 4$, there is no guarantee on its response. If TestThreshold fails (i.e., it responds that the current threshold $\hat{t} > \opt / 4$), we become aware that $\hat{t}$ is rather too large and should be decreased; otherwise, TestThreshold contributes $\Omega(\vmax \hat{t} / OPT)$ to the total value of the mechanism.  

The high level idea of our approach is to first obtain a rough estimation of $\opt$ and then apply binary search, guided by TestThreshold, to the possible range of $\opt$ defined by our initial estimation. At the conceptual level, our approach consists of three periods:
%
\begin{description}
\item[Learning $\vmax$:] 
As in standard secretary algorithms, we start with a \emph{learning period} that considers a constant fraction of the agents, where we do not make any offers and aim to estimate $\vmax$ with constant probability. We note that  $\vmax \leq \opt \leq n \cdot \vmax$.

\item[Binary Search:] 
Using TestThreshold, we apply binary search to the set $\{ \vmax, 2\vmax, 4\vmax, \ldots$, $2^{\log n} \vmax\}$ of different estimations of $\opt$ expressed as a power of $2$ times $\vmax$. This period aims to compute a threshold $\hat{t}$ within a constant fraction of $\opt$. Our application of binary search requires $\Theta(\log\log n)$ \emph{phases}, each consisting of $O(\log\log n)$ negatively-dependent \emph{rounds} where we apply TestThreshold (Appendix~
\ref{negdep}). Hence, we can prove that using $O((\log\log n)^2)$ applications of TestThreshold to the next constant fraction of agents, we end up with a threshold $\hat{t} \geq \opt/8$ with constant probability (Lemma~\ref{lemma:dist}). 

\item[Exploitation:]
Applying TestThreshold to the last constant fraction of agents, where every time the test succeeds $\hat{t}$ is doubled, and every time the test fails $\hat{t}$ is halved, we collect an expected total value that is within a constant fraction of $\opt$ (Lemma~\ref{lemma:compratio}). 
\end{description}

The three-period approach above comprises a constant-competitive posted-price mechanism for online budget-feasible procurement, but only under a \emph{large market assumption}. Specifically, in order to show a constant competitive ratio, we need to assume that $\opt / \vmax = \Omega((\log\log n)^2)$, i.e., that the optimal solution consists of sufficiently many agents.
%
Such large market is necessary in order to guarantee that each of the $O((\log\log n)^2)$ applications of TestThreshold considers a random subset of agents of size $\Omega( \vmax n / \opt)$. This is necessary, so that we can prove that the entire period of Binary Search succeeds with constant probability. 

We can deal with the case where $\opt / \vmax = o((\log\log n)^2)$ by applying Dynkin's secretary algorithm \cite{Dynkin63} with constant probability, which results in competitive ratio of $O((\log\log n)^2)$ (which can be improved to $O(\log\log n)$ with a careful analysis). 

To remove the large market assumption and obtain a constant-competitive mechanism, we assume that $\opt \geq \beta \vmax$, where $\beta$ is a sufficiently large constant. We define a power tower sequence as $\psi_1 = \beta$ and $\psi_{j+1} = 2^{\psi_j}\cdot \psi_j$ and apply TestThreshold with thresholds $\psi_j \vmax$ in order to determine a pair of consecutive intervals $[\psi_j \vmax, \psi_{j+1} \vmax)$ and $[\psi_{j+1} \vmax, \psi_{j+2}\vmax)$ such that $\opt$ is included in their union. We show how to compute such a pair of intervals with constant probability by $O(\log^\ast n)$ applications of TestThreshold to a constant fraction of the agent sequence in total (Lemma~\ref{lemma:Fit}). 
%
Selecting one of these intervals $[\psi_j \vmax, \psi_{j+1} \vmax)$ and $[\psi_{j+1} \vmax, \psi_{j+2}\vmax)$ at random and applying Binary Search and Exploitation, as described above, to the chosen interval (instead of the interval $[\beta \vmax, n \vmax)]$ initially considered) satisfies the required large market property and results in a constant-competitive posted-price mechanism, are required by %thus concluding the proof of 
Theorem~\ref{thm:main_informal}. 

\subsection{Related Work}
\label{sec:related}

From an algorithmic viewpoint, the greedy algorithm of Nemhauser et al. \cite{NemhauserWF78} gives an approximation ratio of $e/(e-1)$ for the algorithmic problem of maximizing a monotone submodular function subject to a knapsack constraint \cite{Sviri04}, and this approximation guarantee is best possible in polynomial-time  \cite{KhullerMN99}, under standard complexity assumptions. More recently, Badanidiyuru et al. \cite{BadaDO19} presented a $(9/8+\eps)$-approximation with  polynomially many demand queries.

Singer \cite{Singer10,Singer13} was the first to study the approximability of budget-feasible procurement auctions by truthful mechanisms and presented a randomized $O(1)$-approximation for monotone submodular valuations. Subsequently, Chen et al. \cite{ChenGL2011} significantly improved the approximation ratio to $7.91$ (resp. $3$) for randomized and $8.34$ (resp. $2+\sqrt{2}$) for deterministic mechanisms when the buyer's valuation is monotone submodular (resp. additive), using a greedy (resp. knapsack) based strategy. Chen et al. also proved an unconditional lower bound of $1+\sqrt{2}$ (resp. $2$) on the approximability of budget-feasible auctions with additive values by deterministic truthful (resp. randomized universally truthful) mechanisms. For additive values, Gravin et al. \cite{GravinJLZ2020} matched the best possible approximation ratio of $2$ for randomized mechanisms and presented a $3$-approximate deterministic mechanism. Anari et al. \cite{AnariGN2014} gave best possible $e/(e-1)$-approximation mechanisms for large markets. 

For monotone submodular valuations, Jalaly and Tardos \cite{JalalyT2021} presented a randomized polynomial-time $5$-approximation mechanism, thus improving on \cite{ChenGL2011}, 
and a deterministic (resp. randomized) $4.56$ (resp. $4$) approximation mechanism, which assumes access to an exact algorithm for value maximization.
%a polynomial-time deterministic $(1+e)$-approximation for large markets. 
The approximation ratio for large markets was improved to $2$ for deterministic (possibly exponential-time) mechanisms and to $3$ for randomized polynomial-time mechanisms by Anari et al. \cite{AnariGN2014}. Recently, Balkanksi et al. \cite{BalkanskiGGST2022} presented a deterministic polynomial-time clock auction that is $4.75$-approximate for monotone submodular valuations. The best known approximation ratio for monotone submodular valuations is $4.45$ for deterministic and $4.3$ for randomized mechanisms, achieved by the polynomial-time clock auction of Han et al. \cite{HanWHC23}. %The clock auctions of Balkanksi et al. \cite{BalkanskiGGST2022} and Han et al. \cite{HanWHC23} achieve small constant approximation guarantees without resorting to the bidding setting, where the mechanism may ask the agents for their costs, but apply to the offline setting, where the agents may be contacted multiple times by the mechanism. 

Polynomial-time constant-factor approximations are also known for non-monotone submodular valuations \cite{AmanatidisKS22,HuangHC023}, where the best known ratio is $64$ for deterministic  \cite{BalkanskiGGST2022} and $12$ for randomized mechanisms \cite{HanWHC23} (both achieved by clock auctions). Moreover, constant-factor approximation mechanisms (albeit non-polynomial time ones) are known for XOS valuations \cite{BeiCGL2012,AmanatidisBM17}, while for subadditive valuations the best known approximation guarantee is $O(\frac{\log n}{\log\log n})$ due to Balkanski et al. \cite{BalkanskiGGST2022}, which is achieved by a clock auction, matches the randomized approximation of \cite{BeiCGL2012} and improves on the $O(\log^3 n)$ deterministic approximation of \cite{DobzinskiPS11}. We refer an interested reader to the survey of Liu et al. \cite{LCLW2024_survey} for a detailed discussion of previous work on budget-feasible procurement. 

The online version of budget-feasible auctions, where the agents arrive sequentially in random order and the decision about their acceptance in the solution is online and irrevocable, was introduced by Singer and Mittal \cite{SingerM13} and Badanidiyuru et al. \cite{Bada2012}. %\cite{Bada2012} focused on the posted-price setting that we consider in this work.
Amanatidis et al. \cite{AmanatidisKS22} presented $O(1)$-competitive randomized online universally truthful mechanisms for monotone and non-monotone submodular valuations in the bidding setting. Online budget-feasible procurement is closely related to the \emph{submodular knapsack secretary} problem \cite{BateniHZ13,KesselheimT17}, where budget feasibility is with respect to the agent costs (which are revealed to the algorithm upon each agent's arrival). Feldman et al. \cite{FeldmanNS11} presented a randomized $20e$-approximation for submodular knapsack secretaries.



The clock auctions of Balkanksi et al. \cite{BalkanskiGGST2022} and Han et al. \cite{HanWHC23} achieve small constant approximation guarantees without resorting to  bidding (i.e., the agents never report their costs to the mechanism), but
they are not online, because the agents receive multiple offers by the mechanism. To the best of our knowledge, the online $O(\log n)$-competitive mechanism of \cite[Section~4]{Bada2012} and the constant-competitive mechanisms in the Bayesian setting of 
\cite{BalkHart16} are the only known posted-price mechanisms for online budget-feasible procurement auctions with monotone submodular valuations. 

A quite standard approach to the design of efficient budget-feasible mechanisms (see e.g., \cite[Section~5]{Bada2012} and in 
\cite[Section~4]{AmanatidisKS22}) and of online algorithms for submodular knapsack secretaries (see e.g., \cite{FeldmanNS11}) is to first learn the costs of a random subset of agents. Then, based on these costs, the mechanism approximates the optimal solution of the resulting random instance and uses its value as a threshold to post linear prices to the remaining agents. We should highlight that this approach does not fit in the framework of online posted-price mechanisms, since it requires knowledge of the costs of a significant fraction of agents (which is obtained through bidding). 