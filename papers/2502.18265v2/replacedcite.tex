\section{Related Work}
\label{sec:related}

From an algorithmic viewpoint, the greedy algorithm of Nemhauser et al. ____ gives an approximation ratio of $e/(e-1)$ for the algorithmic problem of maximizing a monotone submodular function subject to a knapsack constraint ____, and this approximation guarantee is best possible in polynomial-time  ____, under standard complexity assumptions. More recently, Badanidiyuru et al. ____ presented a $(9/8+\eps)$-approximation with  polynomially many demand queries.

Singer ____ was the first to study the approximability of budget-feasible procurement auctions by truthful mechanisms and presented a randomized $O(1)$-approximation for monotone submodular valuations. Subsequently, Chen et al. ____ significantly improved the approximation ratio to $7.91$ (resp. $3$) for randomized and $8.34$ (resp. $2+\sqrt{2}$) for deterministic mechanisms when the buyer's valuation is monotone submodular (resp. additive), using a greedy (resp. knapsack) based strategy. Chen et al. also proved an unconditional lower bound of $1+\sqrt{2}$ (resp. $2$) on the approximability of budget-feasible auctions with additive values by deterministic truthful (resp. randomized universally truthful) mechanisms. For additive values, Gravin et al. ____ matched the best possible approximation ratio of $2$ for randomized mechanisms and presented a $3$-approximate deterministic mechanism. Anari et al. ____ gave best possible $e/(e-1)$-approximation mechanisms for large markets. 

For monotone submodular valuations, Jalaly and Tardos ____ presented a randomized polynomial-time $5$-approximation mechanism, thus improving on ____, 
and a deterministic (resp. randomized) $4.56$ (resp. $4$) approximation mechanism, which assumes access to an exact algorithm for value maximization.
%a polynomial-time deterministic $(1+e)$-approximation for large markets. 
The approximation ratio for large markets was improved to $2$ for deterministic (possibly exponential-time) mechanisms and to $3$ for randomized polynomial-time mechanisms by Anari et al. ____. Recently, Balkanksi et al. ____ presented a deterministic polynomial-time clock auction that is $4.75$-approximate for monotone submodular valuations. The best known approximation ratio for monotone submodular valuations is $4.45$ for deterministic and $4.3$ for randomized mechanisms, achieved by the polynomial-time clock auction of Han et al. ____. %The clock auctions of Balkanksi et al. ____ and Han et al. ____ achieve small constant approximation guarantees without resorting to the bidding setting, where the mechanism may ask the agents for their costs, but apply to the offline setting, where the agents may be contacted multiple times by the mechanism. 

Polynomial-time constant-factor approximations are also known for non-monotone submodular valuations ____, where the best known ratio is $64$ for deterministic  ____ and $12$ for randomized mechanisms ____ (both achieved by clock auctions). Moreover, constant-factor approximation mechanisms (albeit non-polynomial time ones) are known for XOS valuations ____, while for subadditive valuations the best known approximation guarantee is $O(\frac{\log n}{\log\log n})$ due to Balkanski et al. ____, which is achieved by a clock auction, matches the randomized approximation of ____ and improves on the $O(\log^3 n)$ deterministic approximation of ____. We refer an interested reader to the survey of Liu et al. ____ for a detailed discussion of previous work on budget-feasible procurement. 

The online version of budget-feasible auctions, where the agents arrive sequentially in random order and the decision about their acceptance in the solution is online and irrevocable, was introduced by Singer and Mittal ____ and Badanidiyuru et al. ____. %____ focused on the posted-price setting that we consider in this work.
Amanatidis et al. ____ presented $O(1)$-competitive randomized online universally truthful mechanisms for monotone and non-monotone submodular valuations in the bidding setting. Online budget-feasible procurement is closely related to the \emph{submodular knapsack secretary} problem ____, where budget feasibility is with respect to the agent costs (which are revealed to the algorithm upon each agent's arrival). Feldman et al. ____ presented a randomized $20e$-approximation for submodular knapsack secretaries.



The clock auctions of Balkanksi et al. ____ and Han et al. ____ achieve small constant approximation guarantees without resorting to  bidding (i.e., the agents never report their costs to the mechanism), but
they are not online, because the agents receive multiple offers by the mechanism. To the best of our knowledge, the online $O(\log n)$-competitive mechanism of \cite[Section~4]{Bada2012} and the constant-competitive mechanisms in the Bayesian setting of 
____ are the only known posted-price mechanisms for online budget-feasible procurement auctions with monotone submodular valuations. 

A quite standard approach to the design of efficient budget-feasible mechanisms (see e.g., \cite[Section~5]{Bada2012} and in 
\cite[Section~4]{AmanatidisKS22}) and of online algorithms for submodular knapsack secretaries (see e.g., ____) is to first learn the costs of a random subset of agents. Then, based on these costs, the mechanism approximates the optimal solution of the resulting random instance and uses its value as a threshold to post linear prices to the remaining agents. We should highlight that this approach does not fit in the framework of online posted-price mechanisms, since it requires knowledge of the costs of a significant fraction of agents (which is obtained through bidding).