%\documentclass{article}
\documentclass[11pt]{llncs}
\usepackage{graphicx,graphics,hyperref,color}
\usepackage{alphabeta}
\usepackage{float}
%\usepackage{bbold}
\setlength{\parindent}{0pt}
\usepackage{amsmath}
\usepackage[ruled,vlined,linesnumbered,longend]{algorithm2e}
\usepackage{algorithmic}
%\usepackage{cleveref}

\newlength\myindent
\setlength\myindent{2em}
\newcommand\bindent{%
    \begingroup
    \setlength{\itemindent}{\myindent}
    \addtolength{\algorithmicindent}{\myindent}
}
\newcommand\eindent{\endgroup}
\usepackage{eqparbox}
\renewcommand{\algorithmiccomment}[1]{\hfill\eqparbox{COMMENT}{\# #1}}



\newtheorem{notation}{Notation}
\newtheorem{assumption}{Assumption}
\newtheorem{law}{Law}
\newcommand{\set}[1]{\{#1\}}
\newcommand{\To}{\Longrightarrow}
\newcommand{\xml}{\en{XML}}

%\documentclass[10pt, runningheads,a4paper]{llncs}

\usepackage[margin=1in]{geometry}
\usepackage{thmtools}
\usepackage{preamble}
\usepackage{maths}

\newcommand{\vmax}{v_{\text{max}}}
\def\Ball{\mathrm{Ball}}
\def\ALG{\mathrm{ALG}}
\def\alg{\mathrm{ALG}}
\def\OPT{\mathrm{OPT}}
\def\opt{\mathrm{OPT}}
\def\Exp{\mathbb{E}}

\allowdisplaybreaks %to allow page break in equations


\title{A Competitive Posted-Price Mechanism for Online Budget-Feasible Auctions}
%andcharalamp@gmail.com
%fotakis@cs.ntua.gr
%patsilinak@mail.ntua.gr
%
\institute{School of ECE, National Technical University of Athens, Greece \and Université Paris-Dauphine, Université PSL, CNRS \and
 Archimedes, Athena Research Center, Greece\\\email{andcharalamp@gmail.com, fotakis@cs.ntua.gr, thanostolias@mail.ntua.gr, panagiotis.patsilinakos@dauphine.psl.eu}}
\author{Andreas Charalampopoulos\inst{1,3} \and Dimitris Fotakis\inst{1,3} \and Panagiotis Patsilinakos\inst{2} \and Thanos Tolias\inst{1,3} } 

\date{19/12/2024}

\begin{document}

\maketitle

\begin{abstract}
We consider online procurement auctions, where the agents arrive sequentially, in random order, and have private costs for their services. The buyer aims to maximize a monotone submodular value function for the subset of agents whose services are procured, subject to a budget constraint on their payments. We consider a posted-price setting where upon each agent's arrival, the buyer decides on a payment offered to them. The agent accepts or rejects the offer, depending on whether the payment exceeds their cost, without revealing any other information about their private costs whatsoever. We present a randomized online posted-price mechanism with constant competitive ratio, thus resolving the main open question of (Badanidiyuru, Kleinberg and Singer, EC 2012). Posted-price mechanisms for online procurement typically operate by learning an estimation of the optimal value, denoted as $\opt$, and using it to determine the payments offered to the agents. The main challenge is to learn $\opt$ within a constant factor from the agents' accept / reject responses to the payments offered. Our approach is based on an online test of whether our estimation is too low compared against $\opt$ and a carefully designed adaptive search that gradually refines our estimation. 
\end{abstract}

\section{Introduction}
\label{sec:intro}

\begin{figure*}[tb]
    \centering
    \includegraphics[width=0.848\linewidth]{figs/circuitnn.pdf} 
    \caption{Illustration of differentiable CircuitNN. CircuitNN is designed based on differentiable NAND gates. After DAS is guided by PI and PO pairs of the truth table, CircuitNN can get the precise circuit architecture logic equivalent to the truth table.}
    \label{fig:circuitnn}
\end{figure*}

% 1. Describe the importance of logic synthesis
% 2. Existing Problems
% (a) Neural Architecture Search: Unstable, Predefined Setting, etc.
% (b) Circuit Generation: Probabilistic Model, Logic Equivalence

With the rapid advancement of technology, the scale of integrated circuits (ICs) has expanded exponentially. 
This expansion has introduced significant challenges in chip manufacturing, particularly concerning power and area metrics.
A primary objective in IC design is achieving the same circuit function with fewer transistors, thereby reducing power usage and area occupancy.

Logic synthesis~\cite{hachtel2005logicsynth}, a critical step in electronic design automation (EDA), transforms behavioral-level circuit designs into optimized gate-level circuits, ultimately yielding the final IC layout. 
The primary goal of logic synthesis is to identify the physical implementation with the fewest gates for a given circuit function. 
This task constitutes a challenging NP-hard combinatorial optimization problem. 
Current logic synthesis tools~\cite{brayton2010abc, wolf2013yosys} rely on human-designed heuristics, often leading to sub-optimal outcomes.

Differentiable architecture search (DAS) techniques~\cite{liu2018darts, chu2020darts} offer novel perspectives on addressing challenges in this problem.
Circuit functions can be represented through truth tables, which map binary inputs to their corresponding outputs. 
Truth tables provide a precise representation of input-output relationships, ensuring the design of functionally equivalent circuits.
Inspired by this, researchers~\cite{deepmind2024ai4sys, wang2024tnet} have begun exploring the application of DAS to synthesize circuits directly from truth tables.
Specifically, \citet{deepmind2024ai4sys} proposed CircuitNN, a framework that learns differentiable connection structures with logic gates, enabling the automatic generation of logic circuits from truth tables.
This approach significantly reduces the complexity of traditional circuit generation. 
Building on this, \citet{wang2024tnet} introduced T-Net, a triangle-shaped variant of CircuitNN, incorporating regularization techniques to enhance the efficiency of DAS.

Despite these advancements, several challenges remain. 
The computational complexity of DAS grows quadratically with the number of gates, posing scalability issues.
Although triangle-shaped architecture~\cite{wang2024tnet} partially mitigates this problem, redundancy persists. 
%Additionally, DAS is susceptible to converging to local optima, limiting the ability to search architectures that satisfy the given truth tables~\cite{liu2018darts}. 
%Furthermore, hyperparameters (network depth and layer width) require extensive searches, introducing complexity and prolonging the synthesis process. 
Additionally, DAS is susceptible to converging to local optima~\cite{liu2018darts} and hyperparameters (network depth and layer width) require extensive searches. 
The challenges arise from the vast search space in DAS. 
% Even with predefined settings for CircuitNN, finding a configuration that meets the truth table requires extensive trial and error during the DAS process. 
Intuitively, limiting the search space through predefined parameters (network depth, gates per layer, and connection probabilities) can significantly reduce the complexity.

Recent advances~\cite{openai2023gpt4, abramson2024alphafold3, esser2024sd3, li2024mar} in conditional generative models have demonstrated remarkable performance across language, vision, and graph generation tasks. 
Motivated by these developments, we propose a novel approach to circuit generation that generates preliminary circuit structures to guide DAS in generating refined circuits matching specified truth tables. 
Firstly, we introduce CircuitVQ, a tokenizer with a discrete codebook for circuit tokenization. 
Built upon our Circuit AutoEncoder framework~\cite{hou2022graphmae,li2023maskgae,wu2025mgvga}, CircuitVQ is trained through a circuit reconstruction task. 
Specifically, the CircuitVQ encoder encodes input circuits into discrete tokens using a learnable codebook, while the decoder reconstructs the circuit adjacency matrix based on these tokens.
Subsequently, the CircuitVQ encoder serves as a circuit tokenizer for CircuitAR pretraining, which employs a masked autoregressive modeling paradigm~\cite{chang2022maskgit, li2023mage}. 
In this process, the discrete codes function as supervision signals. 
After training, CircuitAR can generate discrete tokens progressively, which can be decoded into initial circuit structures by the decoder of the CircuitVQ. 
These prior insights can guide DAS in producing refined circuits that match the target truth tables precisely.

Our key contributions can be summarized as follows:
\begin{itemize}
\item We introduce CircuitVQ, a circuit tokenizer that facilitates graph autoregressive modeling for circuit generation, based on our Circuit AutoEncoder framework;
\item Develop CircuitAR, a model trained using masked autoregressive modeling, which generates initial circuit structures conditioned on given truth tables;
\item Propose a refinement framework that integrates differentiable architecture search to produce functionally equivalent circuits guided by target truth tables;
\item Comprehensive experiments demonstrating the scalability and capability emergence of our CircuitAR and the superior performance of the proposed circuit generation approach.
\end{itemize}

% Motivation
% (a) Diffusion (Vision, Graph), Autoregressive (Language, Vision)
% (b) Circuit Generation for Predefined Setting
% (c) Neural Architecture Search for Strict Logic Equivalence

% Contribution
% (a) Circuit Tokenizer (new transformer arch, training strategy)
% (b) CircuitAR (train and gen strategies, post-ar strategy)
% (c) Extensive Evaluation including BitD (Bit Distance) for Scalability

\section{Preliminary}

\paragraph{Notation} Consider a sentence of $T$ tokens $\vx=\{\vx_1,\ldots, \vx_T\}\in\gX$, and let $P$ be the unknown target language distribution, $\tilde P(\vx)$ be the empirical distribution of the training data (which is an approximation of $P$), and $Q$ be the distribution of our model at hand. Since our paper is also closely related to RLHF, we will also use $\pi$ to represent the distributions. In particular, we sometimes write $\pi_\theta$ for a distribution that is parameterized by $\theta$, where $\theta$ is usually the set of trainable parameters of the LLM; we write $\pr$ for a reference distribution that should be clear given the context. The next token prediction loss is minimizing the forward-KL between $P$ and $Q$. 




\def\Brem{B_{\text{rem}}}
\def\Nrem{\mathcal{N}_{\text{rem}}}
\def\tinit{t_{\text{init}}}
\def\tmin{t_{\min}}
\def\tmax{t_{\max}}
\def\val{\text{val}}
\def\Event{\mathcal{E}}
\def\Bin{\mathcal{B}}

%In the second period we partition the interval $[v_{max}, n \cdot v_{max}]$ into a set $\mathcal{T}$, of $T=|\mathcal{T}|$ intervals, each increasing in size according to the power towering described earlier. 
%\begin{equation}
%\begin{split}
 %   \mathcal{T} &= \left\{[t_1, t_2], (t_2, t_3], \dots, (t_{T}, t_{T+1}]\right\}
%\end{split}   
%\end{equation}

\begin{figure}[t]
\centering\includegraphics[width=0.6\textwidth]{ProcurementAuctions}
    \caption{The input sequence is partitioned into four periods. Periods 2 to 4 are further partitioned into phases, with each phase consisting of multiple rounds.}
    \label{fig:phases}
\end{figure}


\begin{algorithm}[t]
\caption{LM-Mechanism}\label{alg:LMMECH}
\begin{algorithmic}[1]
\STATE \textbf{Input}: Set $\N$ of agents arriving in random order, budget $B$
%, at each timestep $i \in [n]$ an agent $b(i)$ in secretary order.
%\STATE \textbf{Parameters}: $m,a,OPT_{min},OPT_{max},$ rounds $\{n_j\}$
\STATE \textbf{Initialization:} Solution $S \gets \emptyset$ and $\val \gets 0$, budget available $\Brem \gets B$, agents available $\Nrem \gets \N$
\STATE $\vmax, \Nrem \gets \text{LearningMaxValue}(\Nrem)$
\STATE $\tmin, \tmax, (S, \val, \Nrem, \Brem) \gets \text{PowerTowerSearch}((S, \val, \Nrem, \Brem), B, [\vmax, n\cdot\vmax])$
%\STATE[The constant in the definition of $m$ and $a$ will be determined later on]{Let $m = \Theta(\log\log\log(t_{\max} / t_{\min}))$ and $1/a = \Theta(m \log\log(t_{\max} / t_{\min}))$}
% Set $a$, $m$ to the proper values.
\STATE $\tinit, (S, \val, \Nrem, \Brem) \gets \text{BinarySearch}((S, \val, \Nrem, \Brem), B, [\tmin, \tmax])$
\STATE $(S, \val) \gets \text{Exploitation}((S, \val, \Nrem, \Brem), B, \tinit, [\tmin, \tmax])$
\RETURN $(S, \val)$
\end{algorithmic}
\end{algorithm}

%Below we present the proof of the competitiveness of \hyperref[alg:LMMECH]{LM-Mechanism}.




\section{Posted-Price Mechanism: Building Blocks and Outline of the Analysis}
\label{sec:mech}

We next present the main steps of our posted-price \hyperref[alg:LMMECH]{LM-Mechanism} and the key ideas of its analysis. \hyperref[alg:LMMECH]{LM-Mechanism} assumes that $\opt / \vmax$ is sufficient large (LM stands for \emph{Large Market}). E.g., it suffices that $\opt / \vmax \geq 10^7$. In Section~\ref{sec:MainTheorem}, we use Dynkin's algorithm \cite{Dynkin63} and an adaptation of the posted-price mechanism in \cite[Section~4]{Bada2012} to deal with the case where $\opt / \vmax < 10^7$. 

As explained in Section~\ref{sec:approach}, \hyperref[alg:LMMECH]{LM-Mechanism} consists of $4$ periods. Each of the first three periods ``consumes'' a constant fraction of the agent sequence (and at most a constant fraction of the budget $B$) and provides the next period with a more refined estimation of $\opt$. More specifically, \hyperref[alg:LMMECH]{LM-Mechanism} proceeds along the following periods: 
%
\begin{description}
\item [Period 1 -- LearningMaxValue:] The first period calls Mechanism~\ref{alg:Vmax}, usually referred to as LearningMaxValue, which 
examines about $1/3$ of the agent sequence and returns the maximum value of an agent in this part. We let $\Event_1$ denote the event that the value returned by \hyperref[alg:Vmax]{LearningMaxValue} is indeed $\vmax$, which occurs with probability $1/3$. Assuming $\Event_1$, we have a first rough estimation of $\opt$, since $\vmax \leq \opt \leq n\cdot\vmax$. 

\item [Period 2 -- PowerTowerSearch:] The second period calls Mechanism~\ref{alg:PowerTower}, usually referred to as PowerTowerSearch,
which decides on two consecutive intervals $A = [a_{\min}, a_{\max}]$ and $B = [b_{\min}, b_{\max}]$ such that $a_{\max} = 2^{\frac{a_{\min}}{\vmax}}a_{\min}$ and $b_{\max} = 2^{\frac{b_{\min}}{\vmax}}b_{\min}$. PowerTowerSearch then selects one of the two intervals uniformly at random and returns its endpoints as $t_{\min}$ and $t_{\max}$. We let $\Event_2$ denote the event that $\opt \in A \cup B$ and $\Event_5$ denote the event that conditional on $\Event_2$, the right interval of $A$ and $B$ is chosen. 
%Te of them $\frac{}{}$ which  returns $\tmin$ and $\tmax$ such that $\tmax = 2^{\tmin/\vmax}\cdot\tmin$. We let $\Event_2$ denote the event that $\opt \in [\tmin, \tmax]$. %, for the values $\tmin$ and $\tmax$ returned by \hyperref[alg:powersearch]{PowerTowerSearch}. 
In Section~\ref{sec:SecondPeriod}, we discuss the implementation of \hyperref[alg:PowerTower]{PowerTowerSearch} and prove that $\Event_2$ occurs with probability at least $0.9$. 

\item [Period 3 -- BinarySearch:] The third period calls Mechanism~\ref{alg:binarysearch}, usually referred to as BinarySearch, which performs binary search in the interval $[\tmin, \tmax]$ and returns $\tinit$. Assuming a proper execution of BinarySearch, which is implied by event $\Event_3$ formally defined later on, we have that $\opt / 4 \leq \tinit \leq O(\log\log(\tmax/\tmin))\opt$. The analysis of \hyperref[alg:binarysearch]{BinarySearch} is presented in Section~\ref{sec:ThirdPeriod}.

\item [Period 4 -- Exploitation:] The fourth period calls Mechanism~\ref{alg:exploitation}, usually referred to as Exploitation, which performs an adaptive search, starting with $\tinit$, on possible values of the mechanism's linear-price threshold that are powers of $2$ (see also \cite[Section~3]{Bada2012}). Assuming its proper execution, which is also implied by event $\Event_3$, Exploitation collects $\Omega(\opt)$ value from its part of the agent sequence. The analysis of \hyperref[alg:exploitation]{Exploitation} is presented in Section~\ref{sec:FourthPeriod}.
\end{description}

Different periods, as they run, update \hyperref[alg:LMMECH]{LM-Mechanism}'s state with consists of its current solution $S$, the total value collected so far $\val$, the budget available $\Brem$ and the sequence of agents not yet considered $\Nrem$. 
%
Each of the last three periods examine a (relatively small) constant fraction of the agent sequence. We let $\Event_4$ denote the event that the total number of agents examined by the $4$ periods of \hyperref[alg:LMMECH]{LM-Mechanism} is at most $n$ (which along with events $\Event_2$ and $\Event_3$ also imply that the total budget expended is at most $B$). The correctness of \hyperref[alg:LMMECH]{LM-Mechanism} is implied by event $\Event = \Event_1 \cap \Event_2 \cap \Event_3 \cap \Event_4 \cap \Event_5$. The correctness probability of \hyperref[alg:LMMECH]{LM-Mechanism} is analyzed in Section~\ref{sec:successProb}.

%In Lemma~\ref{lemma:GoodEvent}, we show that $\Event$ occurs with probability at least $1/10$. 

The two key technical ingredients in the analysis of \hyperref[alg:LMMECH]{LM-Mechanism} is a partitioning of the last $3$ periods into phases and rounds (as depicted Fig.~\ref{fig:phases}) and \hyperref[alg:TestTHRESHOLD]{TestEstimate}, which is applied to each round and tests if the current threshold is too large or too small. 
%
%Moreover, if \hyperref[alg:TestTHRESHOLD]{TestEstimate} is applied with a threshold $\hat{t} = \Theta(\opt)$, it collects a fair amount of value. Before we outline 
%
We next discuss their main properties. 

%(i) our partitioning into rounds and phases and of (i) \hyperref[alg:TestTHRESHOLD]{TestEstimate}. 

\subsection{Partitioning into Rounds and Phases}
\label{sec:rounds}

To partition the agent sequence into rounds, we employ a standard technique \cite{IndependentTrick}, where round lengths are drawn from a binomial distribution to ensure independence among agents and that the resulting prefix of the sequence is a \emph{random subset} of $\N$, where each agent participates with a given probability (often referred to as the round's \emph{participation probability} and denoted by $q$). 

Formally, for any suffix $\N' \subseteq \N$ of the agent sequence and any $a \in (0, 1)$, an \emph{$a$-round} (or simply a \emph{round}, if $a$ is clear from the context) is a prefix of $\N'$ with length drawn from a binomial distribution $\Bin(|\N'|, a)$ with parameters $|\N'|$ and $a$. We usually refer to $a$ as the the \emph{length parameter} (or the \emph{length probability}) of the round. 
%
Along with the random arrival order of the agents, selecting the length of a round as a binomially distributed random variable ensures that every two agents in $\N'$ are included in the given round independently with probability $a$ (which allows us to analyze the properties of each round using standard concentration inequalities). 

Given the agent sequence $\N$, we define a sequence of $\kappa \geq 1$ rounds with length parameters $a_1, \ldots, a_\kappa$ as follows: we select the length $x_1$ of the $1$st round from the binomial distribution $\Bin(n, a_1)$, let $n_1 = n-x_1$ be the number of agents remaining, select the length $x_2$ of the $2$nd round from $\Bin(n_1, a_2)$, $\ldots$, let $n_i= n_{i-1}-x_i$ be the number of agents remaining after the $i$-th round, select the length $x_{i+1}$ of the $(i+1)$-th round from $\Bin(n_i, a_{i+1})$, $\ldots$, select the length $x_{\kappa}$ of the $\kappa$-th round from $\Bin(n_{\kappa-1}, a_{\kappa})$. The following analyzes the distribution of the resulting random subsets of agents: 

\begin{lemma}\label{lemma:RoundPro}
For any $\kappa \geq 1$, we consider a sequence of round lengths distributed binomially as described above with length parameters $a_1, \ldots, a_\kappa$. Then, each agent $b \in \N$ is included in round $\kappa$ independently of the other agents with participation probability
%
\begin{equation}\label{eq:participation}
                q_{\kappa} = a_{\kappa}\prod_{i=1}^{\kappa-1}(1-a_{i})
\end{equation}
\end{lemma}

\begin{proof}[Proofsketch.]
The proof consists of two parts. First we observe that due to secretary agent arrivals, the probability that an agent belongs to any fixed interval of length $n_i$ of an agent sequence of length $n$ is $n_i / n$. Applying standard properties of the binomial distribution repeatedly, we derive the desired participation probability. Next, we use secretary agent arrivals and the binomial distribution of round lengths and show that the probability that any two agents belong to round $\kappa$ is $q_\kappa^2$. 
%
%This allows us to express the probability of two agents being in the same round as the number of permutations where both agents occupy positions within the round, divided by the total number of permutations. Finally, we repeatedly apply combinatorial identities to obtain the desired result. 
%
A detailed proof is given in Appendix~\ref{A1}. 
%Lemma \ref{lemma:RoundPro} indicates that adjusting the length parameter of a round directly influences its participation probability.
\end{proof}

In our mechanism, we draw multiple rounds with significantly varying lengths, which complicates the calculation of each round's participation probability using \eqref{eq:participation}. The following lower bound on the participation probability allows us to argue about the properties of each round using its length parameter $a$. The proof requires the precise definition of the length parameters of all rounds considered by our mechanism and is deferred to Appendix~\ref{sec:PART}. For the proof, we use that the length parameter of every round is $o(1)$ and that their sum over all rounds is significantly less than $1$ (which also allows us to show that event $\Event_4$ occurs with probability at least $0.97$, see Lemma~\ref{lemma:Fit} in Appendix~\ref{A3}).  

\begin{lemma}\label{lemma:PART}
The participation probability of any round with length parameter $a$ is at least $6a/(10e)$. 
\end{lemma}


%To prove the aforementioned lemma, we express the participation probability as in Lemma \ref{lemma:RoundPro}. Then, we show that the cumulative product of terms one minus the length parameter for each previous Round is bounded by $6/10e$. To achieve this, we analyze the product of each Period's parameters separately and, after some calculations, derive the desired bound. The full proof of this lemma is provided in Appendix \ref{proof:PART}, as it requires a complete definition of all round lengths. %From now on, when drawing a round with length parameter $a$, we will assume its participation probability to be $\frac{a}{3e}$.

%To prove the aforementioned lemma we express the participation probability as in Lemma 3.2.. Then, we prove that the product of one minus the length parameters of previous rounds is bounded by $1/3e$. To do that we consider the product of each period's parameters separately and after some calculations we end up with the desired bound. The proof of this lemma is provided in the Appendix \ref{proof:PART}, as it requires to completely define all of the round lengths first. From now on, when drawing a round with length parameter $a$ we will consider its participation probability to be $\frac{a}{3e}$.

\textit{Good, Dense Rounds and Phases.}
%
%For each round $j$, we let $\N_j$ denote its set of agents. 
%
Our goal is to use TestThreshold on rounds in order to assess wether a threshold $t$ underestimates the true optimum. To this end, we will call an $\left(a,t\right)$-round, a round with length parameter $a$ on which we test a threshold $t$. For our test to be correct consistently we need the expected number of agents included in both the optimal solution and our $\left(a,t\right)$-round to be at least a fixed constant, i.e. $a\ge \frac{\vmax}{\opt}$. Since we don't know the value of $\opt$ and $t$ is our estimate of it, we say that an $\left(a,t\right)$-round is `good' if $a\ge 81\, e\, \vmax/t$. 

We allocate a budget $B_j = 3 \, C \cdot a_j \cdot B$ to every round $j$ with length parameter $a_j$, where $C = 1/(8e)$ is a fixed constant used in the following sections. For each round $j$ with set of agents $\N_j$ and length parameter $a_j$, we let $S^\ast_j \subseteq \N_j$ be the set of agents in round $j$ with maximum value $f(S^\ast_j)$ subject to the budget constraint $\sum_{ i \in S_j^\ast} c_i \leq B_j$, and let $\opt_j = f(S^\ast_j)$ be the optimal value of round $j$. 

We say that an $\left(a_j,t_j\right))$-round $j$ is \emph{dense} if $\opt_j \geq C \cdot a_j \cdot \opt$. Intuitively, a round is dense if it gets a fair share (with respect to its length parameter) of the optimal value. The following shows that `good' rounds are dense with good probability. 

\begin{lemma}\label{lemma:roundProb}
A  `good' $\left(a_j,t_j\right)$-round $j$, such that $t_j\le \opt$, is dense with probability at least $0.9$.
\end{lemma}

The high level intuition behind the proof of Lemma~\ref{lemma:roundProb} is that a good round is a random set that includes, in expectation, sufficiently many agents of the optimal solution. The requirement on the round being `good' is mostly technical and is used in the proof of Lemma~\ref{lemma:roundProb}. Lemma~\ref{lemma:roundProb} is key to the  analysis of \hyperref[alg:LMMECH]{LM-Mechanism} and can be found in Section~\ref{proof4.5}.


Applying \hyperref[alg:TestTHRESHOLD]{TestEstimate} to a dense round with threshold $t \leq \opt / 4$ should verify that $t$ is not too large and can be increased. To amplify the confidence of our test, we apply \hyperref[alg:TestTHRESHOLD]{TestEstimate} to a sequence of rounds. A \emph{$\left(\delta,a,t\right)$-phase} is a sequence of $\delta$ consecutive rounds all with the same length parameter $a$ and all testing the same threshold $t$ (see also Fig.~\ref{fig:phases}). 
%
We note that \hyperref[alg:TestTHRESHOLD]{TestEstimate} is always applied with the same threshold $t$ to every round of a given phase.
%
A $\left(\delta,a,t\right)$-phase is \emph{dense} if at least $\delta / 2$ of its rounds are dense. In Appendix~\ref{negdep}, we show that: 

\begin{lemma}\label{lemma:PhaseProbability}
A $\left(\delta,a,t\right)$-phase, consisting of `good' rounds and with $t\le \opt$, is dense with probability at least $1-\exp(\frac{-4\cdot\delta}{45})$.
%    \begin{equation}
 %       1 - \exp\lp\{ - \frac{4\cdot \delta}{45} \rp\} 
%    \end{equation}
\end{lemma}

For the proof of Lemma~\ref{lemma:PhaseProbability}, we observe that the events that different $a$-rounds are dense are negatively associated, in the sense that conditional on a round not being dense, the event that another round is dense becomes more likely. Then, Lemma~\ref{lemma:PhaseProbability} by an application of standard concentration bounds. 

\begin{remark}We should highlight that the randomness in the agent arrivals and the internal randomness of \hyperref[alg:LMMECH]{LM-Mechanism} are only used (i) to partition the agent sequence into rounds consisting of random subsets of agents; and (ii) to select the right interval between $A$ and $B$ in Period $2$. Everything else in \hyperref[alg:LMMECH]{LM-Mechanism} is deterministic. In fact, once the rounds are formed, \hyperref[alg:TestTHRESHOLD]{TestEstimate} and the other parts of  \hyperref[alg:LMMECH]{LM-Mechanism} operate without assuming anything about the order in which the agents arrive. \qed
 \end{remark}

\subsection{Estimating the Probability of Success}
\label{sec:successProb}

In the following, we always assume that \hyperref[alg:LMMECH]{LM-Mechanism} gathers positive value only in realizations where event $\Event$ holds (and always condition on $\Event$, even without mentioning that). As discussed after the outline of \hyperref[alg:LMMECH]{LM-Mechanism}, $\Event$ is the conjunction of events $\Event_1, \ldots, \Event_5$ formally defined below.

%\begin{definition}\label{goodevent}
%    $\Event$ is the event where all the following hold:
    \begin{enumerate}
        \item [$\Event_1$:] An agent with the maximum utility $\vmax$ is included in Period 1.
        
        \item [$\Event_2$:] In Period 2, the interval $[\vmax, n \cdot \vmax]$ is partitioned into sub-intervals. \hyperref[alg:PowerTower]{PowerTowerSearch} identifies two consecutive intervals $A$ and $B$. Then, $\Event_2$ denotes that $\opt / 8 \in A \cup B$.
        
        \item [$\Event_3$:] Every phase during periods $3$ and $4$ is dense with respect to the threshold $\hat{t}$ used by \hyperref[alg:TestTHRESHOLD]{TestEstimate} for that phase.  

        
        \item [$\Event_4$:] The sum of the realized round lengths drawn by \hyperref[alg:LMMECH]{LM-Mechanism} is less than $n$.
        
        \item [$\Event_5$:] Given $\Event_2$, the interval $I$ of $A$ and $B$ chosen uniformly at random satisfies $\opt / 8 \in I$. 
    \end{enumerate}
%\end{definition}


%We define event $\Event$ through a number of events, formally stated below, that must hold and then we claim that event $\mathcal{E}$ holds with constant probability. 
The following shows that event $\Event$ occurs with constant probability/

\begin{lemma}\label{lemma:GoodEvent}
    Event $\Event$ happens with probability at least $1/20$.
\end{lemma}
\begin{proof}
%As we haven't yet presented all the details of the mechanism, we cannot present the complete proofs of the bounds here. The full proofs can be found 
In Appendix~\ref{A3}, we show that $\Prob{\Event_2} \ge 0.9$ in Lemma~\ref{lemma:E2}, $\Prob{\Event_3} \ge 0.9$ in Lemma~\ref{lemma:BinaryProb} and $\Prob{\Event_4} \geq 0.97$ in Lemma~\ref{lemma:Fit}.
%The interested reader may refer to the Appendix \ref{A3}, more specifically $\Prob{\mathcal{E}_2}$ is bounded in Lemma~\ref{lemma:E2}, $\Prob{\mathcal{E}_3}$ is bounded in Lemma~\ref{lemma:BinaryProb} and $\Prob{\mathcal{E}_4}$ is bounded in Lemma~\ref{lemma:Fit} .
%    For the events above the following bounds hold:
%   \begin{equation}
%        \begin{split}
%            \Prob{\mathcal{E}_1} = \frac{1}{3}, \Prob{\mathcal{E}_2} \ge 0.9, \Prob{\mathcal{E}_3} \ge 0.9 ,\Prob{\mathcal{E}_4} \ge 0.97,
%            \Prob{\mathcal{E}_5 | \mathcal{E}_2} = 0.5
%        \end{split}
%    \end{equation}
We also have that $\Prob{\Event_1} \geq 1/3$ due to random agent arrivals and the properties of the binomial distribution. Finally, we argue that $\Prob{\Event_5\,|\,\Event_2} = 1/2$. Given that the set of possible realizations is restricted to a subset of those where event $\Event_2$ occurs, $\Event_5$ occurs with probability $1/2$, independently of any other events in the realizations considered in the conditioning on $\Event_2$. I.e., we can think that once we condition that the event $\Event_2$ occurs, the mechanism flips a fair coin (which is completely independent from $\Event_2$ and also from $\Event_1 \cap \Event_3 \cap \Event_4$): with probability $1/2$, the mechanism selects the right interval (and proceeds collecting value), and with probability $1/2$, the mechanism selects the wrong interval (and does not collect anything). Then, 
%
\[
\Prob{\lnot \Event_1 \cup \lnot \Event_2 \cup \lnot \Event_3 \cup \lnot \Event_4} \leq 2/3 + 0.1 +0.1 + 0.03 \leq 0.9 \]
%
Putting everything together:
%
$\Prob{\mathcal{E}} = \Prob{\Event_1 \cap \Event_2 \cap \Event_3 \cap \Event_4} \cdot \Prob{\Event_5 \,|\,\Event_2} \geq \frac{1}{10} \cdot \frac{1}{2} = 1/20$.
%%Putting everything together:
%
 %   \begin{equation*}
 %       \begin{split}
 %           \Prob{\mathcal{E}} = \Prob{\lnot (\lnot \mathcal{E}_1 \cup \lnot \mathcal{E}_2 \cup \lnot \mathcal{E}_3 \cup \lnot \mathcal{E}_4 )} \cdot \Prob{\mathcal{E}_5 | \mathcal{E}_2}  \ge (1-\frac{2}{3} - 0.1 - 0.1-0.03) \cdot \frac{1}{2} \ge \frac{1}{20}
 %       \end{split}
 %   \end{equation*}
\end{proof}



\subsection{Estimation Testing}
\label{sec:estimate}

The key building block of our \hyperref[alg:LMMECH]{LM-Mechanism} is an online test, usually referred to as \hyperref[alg:TestTHRESHOLD]{TestThreshold}, which is applied to a $(\delta,a,t)$-phase. \hyperref[alg:TestTHRESHOLD]{TestThreshold}, whose pseudocode can be found in Mechanism~\ref{alg:TestTHRESHOLD}, determines whether:

\begin{quote}
    \textit{it is possible to collect value at least $C\cdot a\cdot \hat{t}$, where $C = 1/(7e)$ by applying linear-pricing with threshold $\hat{t}$ to a round with length parameter $a$.}
\end{quote}

\hyperref[alg:TestTHRESHOLD]{TestThreshold} determines the set $\N_{\text{round}}$ of agents of each round by drawing their sizes from a binomial distribution $\Bin(|\Nrem|, a)$. Then, it offers every agent $b \in \N_{\text{round}}$ the corresponding linear price $p_b = f_S(b) B /\hat{t}$ and collects value $f_S(b)$, if the price $p_b$ is accepted. We write that \hyperref[alg:TestTHRESHOLD]{TestThreshold} succeeds in a round, if it collects value at least  $C\cdot a\cdot \hat{t}$ from the present round, and fails, otherwise. We write that \hyperref[alg:TestTHRESHOLD]{TestThreshold} succeeds in a phase (or just succeeds) when at least half of the rounds are successful.

\hyperref[alg:TestTHRESHOLD]{TestThreshold} is the only point where \hyperref[alg:LMMECH]{LM-Mechanism} makes offers to the agents, collects value and expends budget. Periods $2$ and $3$ determine the length parameters and the number of phases to which \hyperref[alg:TestTHRESHOLD]{TestThreshold} is applied, and use their success / fail responses to guide our adaptive search for a good estimate of $\opt$. Then, Period $4$ applies \hyperref[alg:TestTHRESHOLD]{TestThreshold} to phases with appropriate length parameters in order to collect a total value of $\Omega(\opt)$. 
If during the execution of \hyperref[alg:LMMECH]{LM-Mechanism}, either the budget or the agent sequence is exhausted, or an agent with value larger than $\vmax$ is found, event $\Event$ is violated and \hyperref[alg:TestTHRESHOLD]{TestThreshold} (and \hyperref[alg:LMMECH]{LM-Mechanism}) abort (without  any value). 


%We present a our function TestThreshold in pseudocode:
\def\valest{\mathrm{val}\mbox{-}\mathrm{test}}
\begin{algorithm}[t]
  \caption{TestThreshold}\label{alg:TestTHRESHOLD}
 \begin{algorithmic}[1]
     \STATE \textbf{Input:} Current state $(S, \val, \Nrem, \Brem)$, budget $B$, phase size $\delta$, length parameter $a$, threshold $\hat{t}$.
     \STATE \textbf{Initialization:} $C \leftarrow \frac{1}{7e}$, successes$\gets 0$
    %\STATE
    \FOR{$j=1$ to $\delta$}
    \STATE Draw $\tau \sim \Bin(|\Nrem|,a)$, $\mathcal{N}_{\text{round}} \leftarrow \Nrem[1 : \tau]$
    \STATE $\Nrem \leftarrow \Nrem \setminus \mathcal{N}_{\text{round}},i \leftarrow 1, \valest \leftarrow 0$
    \STATE \textbf{if} $\Nrem = \emptyset$ \textbf{then} abort \textbf{end if}
     \WHILE{$i \leq |\mathcal{N}_{\text{round}}| $}
     \STATE \textbf{if} $f(\mathcal{N}_{\text{round}}[i]) > v_{max}$ \textbf{then} abort \textbf{end if}
    \STATE \textbf{else} 
    %\bindent
    \begin{ALC@g}
    \STATE $p_i \leftarrow f_S(\mathcal{N}_{\text{round}}[i]) \cdot B\,/\, \hat{t} $\label{step:price}
    \IF{$ p_i \geq c_i$}
    \STATE $S \gets S\cup \{\mathcal{N}_{\text{round}}[i]\}$
    \STATE $\val \gets \val + f_S(\mathcal{N}_{\text{round}}[i])$\ \  and\ \ $\valest \leftarrow \valest + f_S(\mathcal{N}_{\text{round}}[i])$
    \STATE $\Brem \leftarrow \Brem - p_i$ 
    \STATE \textbf{if} $\Brem \leq 0$ \textbf{then} abort \textbf{end if}
    \ENDIF
    \end{ALC@g}
    %\eindent
    \STATE \textbf{end if}
    \STATE \textbf{if} $\valest \ge C \cdot a\cdot \hat{t}$ \textbf{then} $\text{successes} \gets \text{successes} +1$ \textbf{end if}\label{step:test}
    \STATE $i \gets i + 1$
    \ENDWHILE
    \ENDFOR
    \STATE \textbf{if} $\text{successes} \ge \delta /2$ \textbf{then} $(1, S, \val, \Nrem, \Brem)$ \textbf{else }$(0, S, \val, \Nrem, \Brem)$
    %\STATE \textbf{return} $(0, S, \val, \Nrem, \Brem)$
 \end{algorithmic}
\end{algorithm}

%We next condition on \hyperref[alg:TestTHRESHOLD]{TestThreshold}'s application with threshold $\hat{t}$ to a dense round with respect to $\hat{t}$ and establish its main properties (for dense rounds). We highlight that every time \hyperref[alg:TestTHRESHOLD]{TestThreshold} is applied in periods $2$ to $4$, the length parameter $a$ of each round is good with respect to chosen $\hat{t}$ (see Lemma~\ref{lemma:good1} in Section~\ref{sec:SecondPeriod}, Lemma~\ref{lemma:good2} in Section~\ref{sec:ThirdPeriod} and Lemma~\ref{lemma:good3} in Section~\ref{sec:FourthPeriod}). Therefore, due to Lemma~\ref{lemma:roundProb}, every round to which \hyperref[alg:TestTHRESHOLD]{TestThreshold} is applied is dense with probability at least $0.9$.

We next consider applications of \hyperref[alg:TestTHRESHOLD]{TestThreshold} to `good' rounds/phases and establish its main properties. We highlight that every time \hyperref[alg:TestTHRESHOLD]{TestThreshold} is applied in periods $2$, $3$ and $4$, the length parameter $a$ of each round respects the goodness property with respect to the chosen $t$ (see Proposition~\ref{prop:good1} in Section~\ref{sec:SecondPeriod}, Proposition~\ref{prop:good2} in Section~\ref{sec:ThirdPeriod} and Proposition~\ref{prop:good3} in Section~\ref{sec:FourthPeriod}). 
%
In fact, this is the key objective behind how the four periods of \hyperref[alg:LMMECH]{LM-Mechanism} are structured: 
\hyperref[alg:PowerTower]{PowerTowerSearch}, \hyperref[alg:binarysearch]{BinarySearch} and 
\hyperref[alg:exploitation]{Exploitation} are carefully designed in order to ensure that  \hyperref[alg:TestTHRESHOLD]{TestThreshold} is applied to `good' rounds.
%
Therefore, due to Lemma~\ref{lemma:roundProb} (resp. Lemma~\ref{lemma:PhaseProbability}), every round (resp. phase) to which \hyperref[alg:TestTHRESHOLD]{TestThreshold} is applied is dense with probability at least $0.9$ (resp. close enough to $1$).

Due to the linear pricing scheme in step~\ref{step:price} and the success criterion in  step~\ref{step:test}, every application of \hyperref[alg:TestTHRESHOLD]{TestThreshold} to a `good' $(a,\hat{t})$-round (regardless of its success or failure) expends a budget of at most 
%
\[ \frac{C \cdot a \cdot \hat{t} + \vmax}{\hat{t}} \cdot B \leq \left( C \cdot a + \frac{\vmax}{\hat{t}}\right) B \leq 2 \cdot C \cdot a \cdot B\,,
\]
%
where the last inequality holds because the round is good. Thus:

\begin{property}\label{property:a}
When \hyperref[alg:TestTHRESHOLD]{TestThreshold} is applied to a `good' $\left(a,\hat{t}\right)$-round the budget expended does not exceed the round's budget $B_j = 3\cdot C\cdot a\cdot B$. 
\end{property}

Using Property~\ref{property:a}, that \hyperref[alg:TestTHRESHOLD]{TestThreshold} and that the sum of the length parameters of all rounds is less than $1$, we prove the following in Appendix~\ref{budgetproofs}:
%which upper bounds the total budget expended by \hyperref[alg:LMMECH]{LM-Mechanism}:

\begin{lemma}\label{BUDGET}
The total budget expended by \hyperref[alg:LMMECH]{LM-Mechanism} is at most $B/10$.
\end{lemma}

%We will prove that TestThreshold function correctly identifies if the current estimate is below $\opt$ when the round is dense. This is formally presented in the following lemma:

%\begin{lemma}
%\label{lemma:Justify}
\begin{property}\label{property:b}
\hyperref[alg:TestTHRESHOLD]{TestThreshold} succeeds (i.e., it collects value at least $C\cdot a \cdot \hat{t}$) when applied to an $\left(a_j,\hat{t}_j\right)$-round that is dense and $\hat{t}_j \le \opt / 4$.
\end{property}
%\end{lemma}
\begin{proof}
Let $S_j$ be the part of the solution obtained by applying \hyperref[alg:TestTHRESHOLD]{TestThreshold} with threshold $\hat{t}$ to an $a_j$-round $j$ that is dense with respect to $\hat{t}$. We assume that the value of each agent $b$ is $f(b) \leq\vmax$, as otherwise \hyperref[alg:TestTHRESHOLD]{TestThreshold} aborts. Let also $S^\ast_j$ denote the optimal solution for round $j$.

Suppose some agents $ b \in S_j^{*} \setminus S_j $ were included in the optimal solution but not in $S_j$. This can happen for one of two reasons: either their private costs exceeded the mechanism's offer, i.e., $ c_b > f_{S_j}(b)\,B\,/\,\hat{t}$, or $\Brem$ was insufficient to make an appropriate offer to $ b $. However, by design, only the first case can occur. Due to our linear pricing scheme, no agent is ever offered a price larger than $\vmax\,B\,/\,\hat{t} \le \frac{B}{1024}$, where the inequality holds because we always use thresholds $\hat{t} \geq 1024\,\vmax$. Combining this with Lemma~\ref{BUDGET}, we conclude that all agents $ b \in S_j^{*} \setminus S_j $ declined our offer.

We can rewrite the sum of the values of these agents as:
\begin{equation}\label{eq:basic}
            \sum_{b\in S_j^{*} \setminus S_j} f_{S_j}(b) = \sum_{b\in S_j^{*} \setminus S_j} \frac{f_{S_j}(b)}{c_b} \cdot c_b < \sum_{b\in S_j^{*} \setminus S_j}\frac{\hat{t}}{B}\cdot c_b \le \hat{t}\cdot \frac{B_j}{B} = 3 \cdot C\cdot a_j \cdot \hat{t}
\end{equation}
%
The second to last inequality follows from the hypothesis $ c_b > f_{S_j}(b)\,B\,/\,\hat{t}$ and the last inequality follows from the fact that $S_j^{*}$ is budget-feasible with respect to $B_j = 3 \cdot C\cdot a_j\cdot B$.

Using monotonicity and submodularity of the valuation function we have:
%
\begin{equation}\label{eq:basic2}
    f(S_j^{*})-f(S_j) \le f(S_j^{*}\cup S_j) -f(S_j) \le \sum_{b\in S_j^{*}\setminus S_j} f_{S_j}(b)
\end{equation}

Applying \eqref{eq:basic} to \eqref{eq:basic2}, we get that $f(S_j) \geq f(S_j^{*}) - 3 \cdot C\cdot a_j \cdot \hat{t}$. Finally, using the hypothesis that round $j$ is dense, and hence $f(S^\ast_j) \geq C \cdot a_j \cdot \opt$, we obtain that:
\[
        f(S_j) \ge C \cdot a_j \cdot \opt -  3\cdot C\cdot a_j \cdot \hat{t}
\]
%
Since $\hat{t} \le \frac{\opt}{4}$, 
%
$f(S_j)\ge 4\cdot C \cdot a_j \cdot \hat{t} -3 \cdot C\cdot a_j\cdot\hat{t} = C\cdot a_j \cdot \hat{t}$.
%
Thus, \hyperref[alg:TestTHRESHOLD]{TestThreshold} it collects value at least $C\cdot a_j \cdot \hat{t}$ and succeeds. %This concludes the proof of the lemma.
\end{proof}
 The Lemma above extends to phases the following way:
 \begin{lemma}\label{lemma:phasesucc}
     \hyperref[alg:TestTHRESHOLD]{TestThreshold} succeeds (i.e., it collects value at least $C\cdot a \cdot \hat{t}$ and returns $1$) when applied to a $\left(\delta_j,a_j,\hat{t}_j\right)$-phase that is dense and $\hat{t}_j \le \opt / 4$.
 \end{lemma}

The next four sections discuss how periods $1$ to $4$ are structured so that \hyperref[alg:TestTHRESHOLD]{TestThreshold} is applied to phases so that (conditional on event $\Event$) a total value of $\Omega(\opt)$ is collected. 

 %Throughout the rest of our analysis we are going to assume that event $\mathcal{E}$ occurred.

%Another important property of our testing procedure is that we expend very little budget in each round compared to $B$. We prove that our mechanism never runs out of Budget during the rounds that will described later, but we defer the proof to the Main Theorem section, as we need to first define all the round lengths we are going to use.

%We have now established the main properties of our criterion and we are ready to showcase how to use it to find the correct interval from $D$ in which $OPT$ is contained.



 
\section{First Period -- Learning the Maximum Value}
\label{sec:FirstPeriod}

To learn $\vmax$, we sample the first $\tau$ agents, where $\tau \sim \Bin(n,1/3)$. Conditional on event $\Event$, the agent with the maximum value $\vmax$ is among the first $\tau$ agents. In this case, we can use the highest observed value to construct the interval $[\vmax, n \cdot \vmax]$, which is guaranteed to contain $\opt$, due to the submodularity of $f$. The pseudocode for the first period is given in Mechanism~\ref{alg:Vmax}.

\begin{algorithm}[t]
  \caption{LearningMaxValue}\label{alg:Vmax}
 \begin{algorithmic}[1]
     \STATE \textbf{Input:} Current set $\Nrem$ of agents available 
     \STATE Sample $\tau \sim \Bin(|\Nrem|, 1/3)$, 
     $\mathcal{N}_{\text{period}} \leftarrow \Nrem[1 : \tau]$
    \STATE Offer price $p=0$ to the agents in $\mathcal{N}_{\text{period}}$
    \STATE Let $\vmax \leftarrow \max_{b \in \mathcal{N}_{\text{period}}}f(b)$ and $\Nrem \leftarrow \Nrem \setminus \mathcal{N}_{\text{period}}$
    \STATE \textbf{return} $\vmax, \Nrem$
 \end{algorithmic}
\end{algorithm}


\section{Second Period -- Power Tower Search}
\label{sec:SecondPeriod}

Next, we further refine our estimation of $\opt$, narrowing it down to two intervals of power tower length. To this end, we partition the interval $[\vmax,n\cdot \vmax]$ into an interval sequence  $\mathcal{T}$ as follows:
%
\begin{equation}
    \mathcal{T} = \left\{[t_1,t_2],(t_2,t_3],\dots,(t_T,t_{T+1}]\right\},
\end{equation} 
%
where $t_1 = \vmax, t_2 = 10^7\,\vmax$, $t_i = 2^{\frac{t_{i-1}}{\vmax}}\cdot t_{i-1}$ for $2<i<T+1$, and $t_{T+1}=n\cdot \vmax$, with $T=|\mathcal{T}|$. 
%
Using \hyperref[alg:TestTHRESHOLD]{TestThreshold}, we test all intervals $(t_i, t_{i+1}]$, for $i = 2, \ldots, T-1$, so that we find the correct interval to pick. We exclude $[t_1, t_2]$ and $(t_T, t_{T+1}]$, because the correct interval can be inferred without directly testing these values. To test each interval $(t_i, t_{i+1}]$, we apply \hyperref[alg:TestTHRESHOLD]{TestThreshold} with threshold $\hat{t} = t_i$ to a $(\gamma_i, a_i,t_i)$-phase, where the length parameter is $a_i = \frac{81\cdot e\cdot \vmax}{t_i}$ and the number of rounds is $\gamma_i =\frac{3}{2}\cdot\log\left(\frac{t_i}{\vmax}\right)$. The pseudocode of \hyperref[alg:PowerTower]{PowerTowerSearch} is given in Mechanism~\ref{alg:PowerTower}.

\begin{algorithm}[t]
  \caption{PowerTowerSearch}\label{alg:PowerTower}
 \begin{algorithmic}[1]
     \STATE \textbf{Input:} Current state $(S, \val, \Nrem, \Brem)$, budget $B$, interval $[\vmax, n \cdot \vmax]$.
    \STATE \textbf{Construct:} sequence $\mathcal{T}$ of interval s.t. $t_1 \gets \vmax, t_2 \gets 10^7\cdot \vmax$, $t_i \gets 2^{\frac{t_{i-1}}{\vmax}}\cdot t_{i-1}$ and $t_{T+1} \gets n\cdot \vmax$.
    \STATE \textbf{Initialization:} Length parameters $a_i = \frac{81\cdot e\cdot \vmax}{t_i}$ and phase sizes $\gamma_i =\frac{3}{2}\cdot\log\left(\frac{t_i}{\vmax}\right) $, $\text{index} \gets 1$.
    %\STATE
    \FOR{$i=2$ to $|\mathcal{T}|$}
    \STATE $(\text{hit}, S, \val, \Nrem, \Brem) \leftarrow \text{TestThreshold}((S, \val, \Nrem, \Brem), B,\gamma_i, a_i, t_i )$
    \STATE \textbf{if} {$\text{hit} = 1$} \textbf{then} $\text{index} \gets i$ \textbf{end if}
    \ENDFOR
    \STATE With probability $1/2$: $t_{\min} \gets \mathcal{T}[\text{index}-1]$ and $t_{\max} \gets \mathcal{T}[\text{index}]$.
    \STATE Otherwise $t_{\min} \gets \mathcal{T}[\text{index}]$ and $t_{\max} \gets \mathcal{T}[\text{index}+1]$.
    \STATE \textbf{return} $t_{\min}, t_{\max}, (S, \val, \Nrem, \Brem)$.
 \end{algorithmic}
\end{algorithm}

A $(a_i, \gamma_i,t_i)$-phase is considered \emph{successful}, if \hyperref[alg:TestTHRESHOLD]{TestThreshold} applied with threshold $\hat{t} = t_i$ returns success in at least $\gamma_i/2$ rounds. Each length parameter $a_i$ is chosen such that every $(a_i,t_i)$-round is `good'.
 
\begin{proposition}\label{prop:good1}
    During PowerTowerSearch, all phases consist of `good' $(a_i,t_i)$-rounds.
\end{proposition}

The phase lengths $\gamma_i$ are carefully chosen so that (i) the total number of agents examined during this period is at most a constant fraction of $n$; and (ii) we ensure that a phase with threshold much larger than $\opt$ cannot be successful. Formally:

%Part of the rationale for testing each estimate in a Phase instead of just a round is to prevent extreme overestimations of \( OPT \) from succeeding. More formally:
\begin{lemma}\label{lemma:PERIOD1}
    Let $\frac{\opt}{8}\in (t_{i-1}, t_{i}]$. Then,   for every $j \geq i+1$, the $(\gamma_j, a_j,t_j)$-phase tested with threshold $\hat{t} = t_j$ is not successful. 
\end{lemma}

\begin{proof}
    Without loss of generality, let $t_{i} = \frac{\opt}{8}$. We prove that the $(\gamma_{i+1}, a_{i+1},t_{i+1})$-phase tested with threshold $\hat{t} = t_{i+1}$ cannot be successful (if proven for $j=i+1$, this must also apply to every $j > i+1$). 
    
    If the $(\gamma_{i+1}, a_{i+1},t_{i+1})$-phase tested with threshold $\hat{t} = t_{i+1}$ was successful, the total value that we would have collected during that phase would be:
\[
            \text{Value} \ge \frac{\gamma_{i+1}}{2} \cdot C \cdot a_{i+1} \cdot t_{i+1}  
\]
    Using the definitions of $a_i$ and $\gamma_i$, we get that:
\[
         \text{Value}\ge \frac{\frac{3}{2}\cdot \log\left(\frac{t_{i+1}}{\vmax}\right)}{2} \cdot C \cdot \frac{81\cdot e \cdot \vmax}{t_{i+1}} \cdot t_{i+1} 
\]
%
    Since $t_{i+1} = 2^{t_{i}/\vmax}\cdot t_i$ and using that $C = 1/(7e)$, we obtain that:
    \begin{equation}
         \text{Value}> 8\cdot\log\left(2^{\frac{t_i}{\vmax}}\cdot \frac{t_{i}}{\vmax}\right)\cdot \vmax
    \end{equation}
    Since $2^{\frac{t_i}{\vmax}}\cdot \frac{t_{i}}{\vmax} \ge 2^{\frac{t_i}{\vmax}}$, we get 
    %
    $\text{Value} > 8 \cdot t_i =\opt$, a contradiction to the definition of $\opt$.
\end{proof}

After testing all $(\gamma_i, a_i,t_{i})$-phases corresponding to intervals $(t_i, t_{i+1}]$, each with threshold $t_i$, we select the interval corresponding to the last successful phase and the interval preceding it. I.e., if $(t_i, t_{i+1}]$ corresponds to the last successful phase, we select the intervals $(t_{i-1}, t_i]$ and $(t_i, t_{i+1}]$. In the definition of event $\Event_2$, we require that either $\frac{\opt}{8} \in (t_{i-1}, t_{i}]$ or $\frac{\opt}{8} \in (t_{i}, t_{i+1}]$, where $(t_{i-1}, t_i]$ and $(t_i, t_{i+1}]$ are the intervals selected by PowerTowerSearch. Let $i^\ast$ be such that $\frac{\opt}{8} \in (t_{i^{*}}, t_{i^{*}+1}]$. Then, for the event $\Event_2$ to occur, we require that either the $(\gamma_{i^\ast}, a_{i^\ast}, t_{i^\ast})$-phase or the $(\gamma_{i^\ast+1}, a_{i^\ast+1}, t_{i^\ast+1})$-phase is the last successful phase. 

We next show that the event $\Event_2$ is implied by the event that the $(\gamma_{i^\ast}, a_{i^\ast}, t_{i^\ast})$-phase is dense. So, let us assume that the $(\gamma_{i^\ast}, a_{i^\ast}, t_{i^\ast})$-phase is dense, which occurs with probability at least $0.9$, by Lemma~\ref{lemma:E2} (which, in turn, follows from Lemma~\ref{lemma:PhaseProbability}). Hence, with probability at least $0.9$, the $(\gamma_{i^\ast}, a_{i^\ast}, t_{i^\ast})$-phase is successful. If the $(\gamma_{i^\ast}, a_{i^\ast}, t_{i^\ast})$-phase is the last successful phase, then we select the intervals $(t_{i^\ast-1}, t_{i^\ast}]$ and $(t_{i^\ast}, t_{i\ast+1}]$ and the event $\Event_2$ occurs. Otherwise, Lemma \ref{lemma:PERIOD1} implies that the $(\gamma_{i^\ast+1}, a_{i^\ast+1}, t_{i^\ast+1})$-phase must be the last successful phase. Then, we select the intervals $(t_{i^\ast}, t_{i^\ast+1}]$ and $(t_{i^\ast+1}, t_{i\ast+2}]$ and the event $\Event_2$ occurs again. 



\section{Third Period -- Binary Search}
\label{sec:ThirdPeriod}
   % If the optimal value lies in the interval $[t_1, t_2]$, then the algorithm fails to gather sufficient value. However, due to its combination with Dynkin's algorithm, we obtain an $O(1)$ approximation ratio. In the following, we assume that the second period returned a different interval.
  
  We next show how to apply \emph{binary search} to the interval returned by Period $2$. To apply binary search to the given interval $(t_{\min},t_{\max}]$, it is essential that the feedback on each estimation is correct with high probability. To achieve this, we test each estimation $\hat{t}$ of the optimum on $(m,a,\hat{t})$-phases, where $m=\left\lceil 8\cdot \log\log\left(\frac{t_{\max}}{t_{\min}}\right)\right\rceil $ rounds and $a=\frac{1}{6\,\left\lceil\log \log\left(\frac{t_{\max}}{t_{\min}}\right)\right\rceil \cdot m}$. The value of $m$ is set so that we can lower bound the probability of the event $\Event_3$ by union bound on the number of different phases. The length parameter $a$ is set so that the total number of phases used in Period $3$ times the number $m$ of rounds in each phase times $a$ is at most $1/3$ (which implies that Period $3$ ``consumes'' about $1/3$ of the agent sequence). We next show that this choice of $a$ makes our rounds good:
  \begin{proposition}\label{prop:good2}
    $(a,t_{\min})$-rounds (and thus all rounds) during BinarySearch are good.
  \end{proposition}
  
  \begin{proof}
  We need to verify that $\frac{t_{\min}}{\vmax} \ge \frac{81\,e}{a}$. Using our choice of $a$, we obtain that: 
      \begin{equation}
          \begin{split}
      \frac{t_{\min}}{\vmax} \ge 486\cdot e \cdot \left\lceil\log\log\left(\frac{t_{\max}}{t_{\min}}\right)\right\rceil\cdot \left\lceil 8\cdot  \log\log\left(\frac{t_{\max}}{t_{\min}}\right)\right\rceil 
          \end{split}
      \end{equation}
      Using the fact that $t_{\max} = 2^{\frac{t_{\min}}{\vmax}}\cdot t_{\min}$, we get:
      \begin{equation}
          \frac{t_{\min}}{\vmax}\ge 486\cdot e\cdot \left\lceil\log\left(\frac{t_{\min}}{\vmax}\right)\right\rceil\cdot \left\lceil 8\cdot  \log\left(\frac{t_{\min}}{\vmax}\right)\right\rceil\,, 
      \end{equation}
      which is true for $\frac{t_{\min}}{\vmax}\ge  10^7$.
  \end{proof}
The number of rounds needed to conduct BinarySearch is $\ell = \frac{1}{6\cdot a}$. The pseudocode for BinarySearch is presented in Mechanism~\ref{alg:binarysearch}.

\begin{algorithm}[t]
\caption{BinarySearch}\label{alg:binarysearch}
\begin{algorithmic}[1]
\STATE \textbf{Input}: Current state $(S, \val, \Nrem, \Brem)$, budget $B$, search interval $[t_{\min},t_{\max}]$
\STATE \textbf{Initialization:} Phase size $m \gets \left \lceil 8\cdot\log\log\left(\frac{t_{\max}}{t_{\min}}\right)\right\rceil$\,, length parameter $a \gets \frac{1}{6\,\left\lceil\log\log\left(\frac{t_{\max}}{t_{\min}}\right)\right\rceil\cdot m}$, \\
%
\hspace*{2.1cm}$\text{low} \gets 0$, 
$\text{high} \gets \left\lceil \log\left(\frac{t_{\max}}{t_{\min}}\right) \right\rceil$, $\text{mid} \gets \left\lceil (\text{high} + \text{low})/2 \right\rceil$
%\STATE
    \WHILE{$\text{low} \leq \text{high}$}
    \STATE $\text{hit} \gets 0$
    \STATE $(\text{hit}, S, \val, \Nrem, \Brem) \leftarrow \text{TestThreshold}((S, \val, \Nrem, \Brem), B,m, a, 2^{\text{mid}}\cdot t_{\min})$
    \STATE \textbf{if} $\text{hit} = 1$ \textbf{then} $\text{low} \gets \text{mid}$, $\text{mid} \gets \lceil (\text{high} + \text{low})/2 \rceil$.
    \STATE \textbf{else} $\text{high} \gets \text{mid}$, $\text{mid} \gets \lfloor \text{high} + \text{low})/2 \rfloor$ \textbf{end if}
    \ENDWHILE
    \STATE \textbf{return} $(2^{\text{mid}}\cdot t_{\min}, (S, \val, \Nrem, \Brem))$.
\end{algorithmic}
\end{algorithm}
 

By Lemma~\ref{lemma:phasesucc}, if all thresholds not greater than $\frac{\opt}{4}$ are tested on dense phases, we cannot end up with a substantial underestimation of $\opt$ after conducting BinarySearch. Below we prove an even sharper bound on the possible estimates that BinarySearch may return.
\begin{lemma}\label{lemma:dist}
    Let $\hat{t}$ be the estimate \hyperref[alg:binarysearch]{BinarySearch} returns, under event $\mathcal{E}$. Then:
    \begin{equation}
        \frac{\opt}{8} \le \hat{t} \le 84\cdot e\cdot \left\lceil\log\log\left(\frac{t_{\max}}{t_{\min}}\right)\right\rceil\cdot \opt
    \end{equation}
\end{lemma}
\begin{proof}
    Suppose $\hat{t}<\frac{\opt}{8}$. This would imply that an intermediate estimate $\hat{t}' \in \left[\frac{\opt}{8},\frac{\opt}{4}\right]$ failed in a dense phase. However, this cannot happen under event $\mathcal{E}$, due to Lemma  \ref{lemma:phasesucc}.
    Now consider the case where $\hat{t}> 84 \, e\, \left\lceil\log\log\left(\frac{t_{\max}}{t_{\min}}\right)\right\rceil\cdot \opt$. This would mean that a dense phase succeeded using an overestimated value  $\hat{t}' >84\, e\, \left\lceil\log\log\left(\frac{t_{\max}}{t_{\min}}\right)\right\rceil\cdot \opt$. Specifically, in that phase, the accumulated value would be:
    \begin{equation}
        \begin{split}
            \text{Value}\ge \frac{m}{2}\cdot C \cdot a \cdot \hat{t}' 
        \end{split}
    \end{equation}
    By using the assumption of the overestimation $\hat{t}'$, along with the definitions of $C$ and $a$ we get:
    \begin{equation}
        \begin{split}
            \text{Value}> \frac{m}{2}\cdot \frac{1}{7e} \cdot \frac{1}{6\left\lceil\log \log\left(\frac{t_{\max}}{t_{\min}}\right)\right\rceil \cdot m} \cdot 84\, e\, \left\lceil\log\log\left(\frac{t_{\max}}{t_{\min}}\right)\right\rceil\cdot \opt=\opt
        \end{split}
    \end{equation}    
    which clearly contradicts the definition of $\opt$.

   Thus, we conclude that $\hat{t} \in \left[\frac{\opt}{8},\ \ 84 \, e\,\log\log\left(\frac{t_{\max}}{t_{\min}}\right)\cdot \opt\right]$
\end{proof}

\section{Fourth Period -- Exploitation}\label{sec:FourthPeriod}


Lemma \ref{lemma:dist} indicates that, after running BinarySearch, our estimation is at most $84\cdot e \cdot \left\lceil\log\log\left(\frac{t_{\max}}{t_{\min}}\right)\right\rceil$ times $\opt$. In the final stage of our mechanism, we aim to converge to a threshold $\hat{t}$ close to $\opt$ and apply $\hat{t}$ to a constant fraction of agents, thereby collecting a total value of $\Omega(\opt)$. 

\begin{algorithm}[t]
\caption{Exploitation}\label{alg:exploitation}
\begin{algorithmic}[1]
\STATE \textbf{Input}: Current state $(S, \val, \Nrem, \Brem)$, budget $B$, initial threshold $\tinit$, search interval $[t_{\min}, t_{\max}]$
%
\STATE \textbf{Initialization:} $\hat{t} \gets \tinit$, Phase size $m \gets \left \lceil 8\cdot\log\log\left(\frac{t_{\max}}{t_{\min}}\right)\right\rceil$\,, length parameter $a \gets \frac{1}{6\,\left\lceil\log\log\left(\frac{t_{\max}}{t_{\min}}\right)\right\rceil\cdot m}$
%\STATE
    \FOR{$i = 1$ to $\frac{1}{6 \cdot m \cdot a} = \left\lceil\log\log\left(\frac{t_{\max}}{t_{\min}}\right)\right\rceil$}
    \STATE $(\text{hit}, S, \val, \Nrem, \Brem) \leftarrow \text{TestThreshold}((S, \val, \Nrem, \Brem), B, a, \hat{t})$
    \STATE \textbf{if} $\text{hit} =1$ \textbf{then} $\hat{t} \gets 2 \cdot \hat{t}$ \textbf{else} $\hat{t} \gets \hat{t}/2$ \textbf{end if}
    \ENDFOR
    \STATE \textbf{return} $(S, \val)$.
\end{algorithmic}
\end{algorithm}

\hyperref[alg:exploitation]{Exploitation}, whose pseudocode is presented in Mechanism~\ref{alg:exploitation}, is applied across  a total number of $\ell/m$ $(m,a,\hat{t})$-phases, thus using $\ell = 1/(6a)$ rounds in total. Under the event $\Event$, it holds that $\tmin \le \frac{\opt}{8}$. Thus, we can leverage Proposition~\ref{prop:good2} to claim the following:
\begin{proposition}\label{prop:good3}
    Conditioning on the event $\Event$, all rounds during Exploitation are good. 
\end{proposition}
The mechanism operates by doubling the estimation after every successful phase and halving it after every failed phase. Under the event $\Event$, the threshold $\hat{t}$ never falls below $\opt / 8$, by Lemma \ref{lemma:phasesucc}. Combining this with Lemma~\ref{lemma:dist}, we conclude that we can have no more than \[ \frac{\frac{\ell}{m} + \log\log\log\left(\frac{t_{\max}}{t_{\min}} \right)+ \log(84e)+2}{2} \] failed phases; exceeding this would cause our estimation to drop below \( \frac{\opt}{8} \). Consequently, we are guaranteed at least \( \frac{\frac{\ell}{m} -\log\log\log\left(\frac{t_{\max}}{t_{\min}} \right)- \log(84e)-2}{2}\) successful phases, which ensures a constant competitive ratio. Formally,

\begin{lemma} \label{lemma:compratio}
    Conditional on the event $\Event$, Exploitation collects a total value of at least $\frac{\opt}{4032e}$.
\end{lemma}
\begin{proof}
    As we have argued already, we are guaranteed to have at least $\frac{\frac{\ell}{m} -\log\log\log\left(\frac{t_{\max}}{t_{\min}} \right)- \log(84e)-2}{2}$ successful Phases, which in turn means that the value gathered is greater than:
\[
        \text{Value}\ge \underbrace{\frac{\frac{\ell}{m} -\log\log\log\left(\frac{t_{\max}}{t_{\min}} \right)- \log(84e)-2}{2}}_{\text{successful Phases}} \cdot \underbrace{\frac{m}{2} \cdot C \cdot a\cdot \frac{\opt}{8}}_{\text{Value of each Phase}} 
\]
    Now using the definitions of $a$, $\ell$, $C$, and $t_{\max}$, we get that
\[
        \text{Value} \ge\left(\frac{1}{6} - \frac{\log\log\log\left(2^{\frac{t_{\min}}{\vmax}}\right) + \log(84e)+2}{6\cdot\log\log\left(2^{\frac{t_{\min}}{\vmax}}\right)}\right)\cdot \frac{\opt}{224e}
\]
    Finally, by the monotonocity of $g(x) = \frac{-\log\log\log(x) - \log(84e)-2}{6\cdot\log\log(x)}$, along with the fact that $\frac{t_{\min}}{\vmax} \ge  10^7,$ we conclude that $\text{Value}  \ge \frac{1}{4032e}\cdot \opt$.
\end{proof}


\section{Putting Everything Together and Removing the Large Market Assumption}\label{sec:MainTheorem}

%To extend our result for dense phases, we must account for the dependencies between the outcomes of rounds. Conditional on a previous round having failed, subsequent rounds are more likely to succeed. This dependency arises from the constraint that the utility of all rounds must sum to the total utility of all agents. 
%We present our \hyperref[alg:LMMECH]{LM-Mechanism} written in Pseudocode:

%\begin{algorithm}[H]
%\caption{LM-Mechanism}\label{alg:LMMECH}
%\begin{algorithmic}[1]
%\STATE \textbf{Input}: Set of agents $\mathcal{N}$, budget $B$%, at each timestep $i \in [n]$ an agent $b(i)$ in secretary order.
%\STATE \textbf{Parameters}: $m,a,OPT_{min},OPT_{max},$ rounds $\{n_j\}$
%\STATE \textbf{Initialize:} Timestep $i=0$, solution $S =\emptyset$
%\STATE $\vmax, i \gets \text{Period 1}(\mathcal{N}).$
%\STATE $t_{min},t_{max} , i \gets \text{TestIntervals}(\mathcal{N}[i:n],B,[\vmax,n\cdot\vmax])$.
%\STATE Set $a,m$ to the proper values.
%\STATE $t, i \gets \text{BinarySearch}(\mathcal{N}[i:n],B, [t_{min},t_{max}],a,m)$.
%\STATE Exploitation$(\mathcal{N}[i:n],B,t,a,m)$.
%\end{algorithmic}
%\end{algorithm}

%We are now ready to 

Below we present the proof of the competitiveness of \hyperref[alg:LMMECH]{LM-Mechanism}.

\begin{theorem}\label{theorem:lm}
    Assuming that $\opt > 10^{7}\cdot \vmax$, \hyperref[alg:LMMECH]{LM-Mechanism} is $O(1)$-competitive.
\end{theorem}

\begin{proof}
In the four periods of \hyperref[alg:LMMECH]{LM-Mechanism}, we assume that the event $\Event$ occurs. Lemma~\ref{lemma:GoodEvent} proves that $\Event$ happens with probability at least $1/20$. Conditional on the event $\Event$, the \hyperref[alg:LMMECH]{LM-Mechanism} achieves a competitive ratio of $4032e$ due to Lemma~\ref{lemma:compratio}. Overall, the expected total value of \hyperref[alg:LMMECH]{LM-Mechanism} is  at least $\frac{1}{20} \cdot \frac{1}{4032e}\cdot \opt$.
\end{proof}

Finally, we need to remove the Large Market Assumption that $\opt > 10^{7}\cdot \vmax$. To this end, we present the mechanism PostedPrices, whose pseudocode can be found in Mechanism~\ref{alg:PPMECH}.
\hyperref[alg:PPMECH]{PostedPrices} invokes three different mechanisms, with a constant probability each, which deal with instances of different market size (i.e., magnitude of $\frac{\opt}{\vmax}$). We are now ready to prove our main result:


\begin{algorithm}[t]
\caption{PostedPrices}\label{alg:PPMECH}
\begin{algorithmic}[1]
\STATE \textbf{Input}: Set of agents $\mathcal{N}$, budget $B$%, at each timestep $i \in [n]$ an agent $b(i)$ in secretary order.
%\STATE \textbf{Parameters}: $m,a,OPT_{min},OPT_{max},$ rounds $\{n_j\}$
\STATE With probability $0.1$, execute Dynkin's algorithm on $\N$.
\STATE With probability $0.1$, execute \hyperref[alg:MMMECH]{MediumMarket} on the set of agents $\N$ with budget $B$.
\STATE With probability $0.8$, execute \hyperref[alg:LMMECH]{LM-Mechanism} on the set of agents $\N$ with budget $B$.
\end{algorithmic}
\end{algorithm} 

\begin{algorithm}[t]
\caption{MediumMarket}\label{alg:MMMECH}
\begin{algorithmic}[1]
\STATE \textbf{Input}: Set of agents $\mathcal{N}$, budget $B$%, at each timestep $i \in [n]$ an agent $b(i)$ in secretary order.
%\STATE \textbf{Parameters}: $m,a,OPT_{min},OPT_{max},$ rounds $\{n_j\}$
\STATE Learn $\vmax$ using \hyperref[alg:Vmax]{LearningMaxValue}

\STATE Pick uniformly at random a threshold $t \in \left\{2^6 \cdot \vmax,2^8 \cdot \vmax,\dots,2^{23} \cdot \vmax\right\}$

\STATE Apply linear pricing with threshold $t$ and budget $B$ to the rest of the agent sequence.


\end{algorithmic}
\end{algorithm} 



\begin{theorem}
    \hyperref[alg:PPMECH]{PostedPrices} is a universally truthful $O(1)$-competitive posted-price mechanism for online budget-feasible procurement auctions with secretary agent arrivals and  monotone submodular buyer's valuations.
\end{theorem}
\begin{proof}
    Universal truthfulness follows from the fact that 
    \hyperref[alg:PPMECH]{PostedPrices} is a probability distribution over three universally truthful (and posted-price) mechanisms. 
    
    Regarding the competitive ratio of \hyperref[alg:PPMECH]{PostedPrices}, we consider the following cases:
    \begin{enumerate}
        \item $\opt<1024\cdot \vmax$: Then Dynkin's algorithm is executed with probability $0.1$ and collects an expected value of at least $\frac{1}{10}\cdot\frac{\opt}{1024 \cdot e}$\,.
        \item $1024\cdot\vmax \le\opt<10^7\cdot \vmax$: Then \hyperref[alg:LMMECH]{MedianMarket} is executed with probability $0.1$. With probability at least $0.9$, the optimal value for the set of agents not used for the calculation of $\vmax$ by \hyperref[alg:Vmax]{LearningMaxValue}, in step~2, is at least $\opt / 4$. With probability at least $1/18$, the threshold $t$ selected is such that $\frac{\opt}{16}\le t \le \frac{\opt}{8}$\,. Thus, by Lemma~\ref{lem:threshold}, the chosen threshold $t$ results in at least $(\frac{1}{4}-\frac{\vmax}{\opt})\cdot \frac{\opt}{4}>\frac{\opt}{18}$ value. Overall the expected total value is at least:
        \begin{equation}
            \frac{1}{10}\cdot \frac{9}{10}\cdot \frac{1}{18} \cdot\frac{\opt}{18} = \frac{\opt}{3600}
        \end{equation}
        \item $10^7\cdot \vmax\le\opt$: Then, \hyperref[alg:LMMECH]{LM-Mechanism} is executed with probability $0.8$, which by Theorem \ref{theorem:lm} results in an expected total value of at least
        a $0.8\cdot\frac{1}{20}\cdot \frac{\opt}{4032e}$\,.
    \end{enumerate}
%
 Putting the three cases together, we conclude that the competitive ratio of \hyperref[alg:PPMECH]{PostedPrices} is $O(1)$. We note that at many places, we prioritized simplicity over trying to optimize the resulting constant.
\end{proof}


\section{Conclusions}

In this work, we have introduced a randomized, constant-competitive posted-price mechanism for online procurement auctions, thus resolving the main open question of Badanidiyuru, Kleinberg and Singer \cite{Bada2012}. Our findings demonstrate that in online procurement auctions, sequential posted-price mechanisms can be as powerful, in terms of the asymptotics of their competitive ratio, as seal-bid mechanisms, whose performance has been extensively studied. We note that despite our mechanism is elaborate to design and analyze, its interface with the agents is simple and transparent. 

In addition to improving the constants in our analysis, an interesting direction for further research is whether our refined adaptive search can be generalized and applied to other posted-price mechanisms, such as combinatorial auctions with submodular bidders \cite{Assadi019,AssadiS20,DuttingKL24}, towards  improved competitive ratios. 

\section{Acknowledgement}
This work has been partially supported by project MIS 5154714 of the National Recovery and Resilience Plan Greece 2.0 funded by the European Union under the NextGenerationEU Program.

\bibliographystyle{IEEEtran} % We choose the "plain" reference style
\bibliography{references} % Entries are in the refs.bib file

\appendix

% Appendix
\clearpage\appendix
\section*{\Large Appendix}

\subsection{Lower bounds on sample complexity}\label{sec:sample_compexity}
We establish a lower bound for generalized linear measurements using standard information-theoretic arguments based on Fano's inequality. While the upper bound in Theorem~\ref{thm:alg_general} is derived for the maximum probability of error over all  $k$-sparse vectors, the lower bound applies even in the weaker setting of the average probability of error, where 
$\bx$ is chosen uniformly at random.
\begin{theorem}[Lower bound for GLMs]\label{thm: lower_bdglm} Consider any  sensing matrix $\vecA$.
For a uniformly chosen $k$-sparse vector $\bx$, an algorithm $\phi$ satisfies $$\bbP\inp{\phi(\vecA, \by) \neq \bx}\leq \delta$$   only if the number of measurements $$m\geq \frac{k\log\inp{\frac{n}{k}}}{I}\inp{1 - \frac{h_2(\delta) + \delta k\log{n}}{k\log{n/k}}}$$ for some $I$ such that $I\geq {I(y_i; \bx|\vecA)}, \, i\in [m]$. In particular, when $y\in \inb{-1, 1}$, we have $\bbE\insq{\inp{g(\vecA_i^T\bx)}^2} \geq I(y_i, \bx|\vecA)$ where the expectation is over the randomness of $\vecA$ and $\bx$.
\end{theorem}
The lower bound can be interpreted in terms of a communication problem, where the input message $\bx$ is encoded to $\vecA\bx$. The decoding function takes in as input the encoding map $\vecA$ and the output vector $\by$ in order to recover $\bx$ with high probability. For optimal recovery, one needs at least $\frac{\text{message entropy}}{\text{capacity}}$ number of measurements (follows from noisy channel coding theorem~\cite{thomas2006elements}). In Theorem~\ref{thm: lower_bdglm}, the entropy of the message set $\log{n \choose k}\approx k\log{n/k}$ and the proxy for capacity is the upper bound on mutual information $I$. We provide a detailed proof of the theorem in  Section~\ref{sec:proofs}.


We first present lower bounds for \bcs\  and \logreg. The lower bound for \bcs\ is given for any sensing matrix $\vecA$ which satisfies the power constraint given by \eqref{eq:power_constraint}, whereas the one for \logreg\ is only for the special case when each entry of the sensing matrix is iid $\cN(0,1)$. Recall that \eqref{eq:power_constraint} holds in this case.  For \bcs\ (and \logreg\ respectively), we can use the upper bound of $\bbE\insq{\inp{g(\vecA_i^T\bx)}^2}$ on the mutual information term. The dependence of $\sigma^2$ (and $1/\beta^2$ respectively) requires careful bounding of this term, which is done in the formal proofs in Appendix~\ref{proof:sec:lower_bd}.


As mentioned earlier, we need at least $k\log\inp{n/k}$ measurements for \bcs and \logreg. This is because the entropy of a randomly chosen $k$-sparse vector is approximately $k\log\inp{n/k}$ and we learn at most one bit with each measurement. However, due to corruption with noise, we learn less than a bit of information about the unknown signal with each measurement. The information gain gets worse as the noise level increases. 
Our lower bounds make this reasoning explicit.  
\begin{corollary}[\bcs\ lower bound]\label{thm: lower_bd_bcs} Suppose, each row $\vecA_i, \, i\in [1:m]$ of the sensing matrix $\vecA$ satisfies the power constraint~\eqref{eq:power_constraint}.
For a uniformly chosen $k$-sparse vector $\bx$, an algorithm $\phi$ satisfies $$\bbP\inp{\phi(\vecA, {\by}) \neq \bx}\leq \delta$$ for the problem of $\bcs$ only if the number of measurements $$m\geq \frac{k+\sigma^2}{2}\log\inp{\frac{n}{k}}\inp{1 - \frac{h_2(\delta) + \delta k\log{n}}{k\log{n/k}}}.$$ 
\end{corollary}

\begin{corollary}[\logreg\ lower bound]\label{thm: lower_bd_log_reg} Consider a Gaussian  sensing matrix $\vecA$ where each entry is chosen iid $N(0,1)$.
For a uniformly chosen $k$-sparse vector $\bx$, an algorithm $\phi$ satisfies $$\bbP\inp{\phi(\vecA, \bw) \neq \bx}\leq \delta$$ for the problem of $\logreg$ only if the number of measurements $$m\geq \frac{1}{2}\inp{k+\frac{1}{\beta^2}}\log\inp{\frac{n}{k}}\inp{1 - \frac{h_2(\delta) + \delta k\log{n}}{k\log{n/k}}}.$$ 
\end{corollary}



Theorem~\ref{thm: lower_bdglm} also implies an information theoretic lower bound for \spl, which is presented below and proved in Appendix~\ref{proof:sec:lower_bd}. Note that the denominator term in the bound $\frac{1}{2}\log\inp{1+\frac{k}{\sigma^2}}$ is the capacity of a Gaussian channel with power constraint $k$ and noise variance $\sigma^2$. 
\begin{corollary}[\spl\ lower bound]\label{thm: spl_lower_bd_1}
Under the average power constraint \eqref{eq:power_constraint} on  $\vecA$, for a uniformly chosen $k$-sparse vector $\bx$, an algorithm $\phi$ satisfies $$\bbP\inp{\phi(\vecA, {\by}) \neq \bx}\leq \delta$$ only if the number of measurements
$$m\geq \frac{k\log\inp{\frac{n}{k}}-\inp{h_2(\delta) + \delta k\log{n}}}{\frac{1}{2}\log\inp{1+\frac{k}{\sigma^2}}}.$$
\end{corollary} 

\subsection{Tighter upper and lower bounds for \spl}\label{sec:tighter_bounds_spl}
We present information theoretic upper and lower bounds for \spl\ in this section. Similar to Section~\ref{sec:alg}, our upper bound is for the maximum probability of error, while the lower bounds hold even for the weaker criterion of average probability of error.

We first present an upper bound based on the maximum likelihood estimator (MLE) where  we  decode to $\hat{\bx}$ if, on output $\by$, 
\begin{align*}
\hat{\bx} = \argmax_{\stackrel{\bx\in \inb{0,1}^n}{\wh{\bx} = k}}\,\, p(\by|{\bx})
\end{align*} where $p(\by|{\bx})$ denotes the probability density function of $\by$ on input $\bx$.
\begin{theorem}[MLE upper bound for \spl]\label{thm:upper_bd_mle} Suppose  entries of the measurement matrix $\vecA$ are i.i.d. $\cN(0,1).$
The MLE  is correct with high probability if 
\begin{align}m\geq \max_{l\in[1:k]}  \frac{nN(l)}{\frac{1}{2}\log\inp{\frac{ l}{2\sigma^2}+1}}\label{eq:upper_bd_mle}
\end{align}where  $N(l):=  \frac{k}{n} h_2\inp{\frac{l}{k}} + (1-\frac{k}{n})h_2\inp{\frac{l}{n-k}}$. 
\end{theorem}
We prove the theorem in Appendix~\ref{proof:MLE}. The main proof idea involves analysing the probability that the output of the MLE is $2l$ Hamming distance away from the unknown signal $\bx$ for different values of $l\in [1:k]$ (assuming $k\leq n/2$). This depends on the number of such vectors (approximately $2^{nN(l)}$) and the probability that the MLE outputs a vector which is $2l$ Hamming distance away from $\bx$. 

Note that when $l = k\inp{1-\frac{k}{n}}$, $nN(l) = nh_2(k/n)\approx k\log{\frac{n}{k}}$ and $\log\inp{\frac{k\inp{1-k/n}}{2\sigma^2}+1}\leq \log\inp{\frac{k}{2\sigma^2}+1}$.
Thus, $m$ is at least $\frac{2k\log{n/k}}{\log\inp{\frac{k}{2\sigma^2}+1}}$ (see the bound for Corollary~\ref{thm: spl_lower_bd_1}). It is not immediately clear if this value of $l= k\inp{1-\frac{k}{n}}$ is the optimizer. However, for large $n$, this appears to be the case numerically as shown in Plot~\ref{plot:1}.

\begin{figure}[t]
\includegraphics[width=7cm]{Unknown2.png}
\centering
\caption{The figure shows the plot of the MLE upper bound \eqref{eq:upper_bd_mle} (given by m1) for different values of $k$. This is displayed in blue color. A plot of $\frac{2nN(l)}{\log\inp{\frac{ l}{2\sigma^2}+1}}$ is also presented for $l = k\inp{1-\frac{k}{n}}$ in orange color, given by m2. A part of the plot is zoomed in to emphasize the closeness between the lines. In these plots,  $\sigma^2$ is set to 1,  $n$ is 50000 and $k$ ranges from 1000 to 25000 $(n/2)$. }\label{plot:1}
\end{figure}


Inspired by the MLE analysis, we derive a lower bound with the same structure as \eqref{eq:upper_bd_mle}. We generate the unknown signal $\bx$ using the following distribution: A vector $\tilde{\bx}$ is chosen uniformly at random from the set of all $k$-sparse vectors. Given $\tilde{\bx}$, the unknown input signal $\bx$ is chosen uniformly from the set of all $k$-sparse vector which are at a Hamming distance $2l$ from $\bx$. 
The lower bound is then obtained by computing upper and lower bounds on $I(\vecA, \by;\bx|\tilde{\bx})$.
We show this lower bound only for random matrices where each entry is chosen iid $\cN(0,1)$.
\begin{theorem}[\spl\ lower bound]\label{thm:lower_bd_spl}
If each entry of $\vecA$ is chosen iid $\cN(0,1)$, then for a uniformly chosen $k$-sparse vector $\bx$, an algorithm $\phi$ satisfies 
\begin{align}
    \bbP\inp{\phi(\vecA, {\by}) \neq \bx}\leq \delta\label{eq:spl_lower_bd_l}
\end{align}  only if the number of measurements $$m\geq \max_l\frac{nN(l) - 2\log{n}- h_2(\delta) - \delta k\log{n}}{\frac{1}{2}\log\inp{1+\frac{l}{\sigma^2}\inp{2-\frac{l}{k}}}} .$$
\end{theorem} The proof of Theorem~\ref{thm:lower_bd_spl} is given in Appendix~\ref{proof:MLE}.

If we choose $l = k\inp{1-\frac{k}{n}}$ in Theorem~\ref{thm:lower_bd_spl}, we recover corollary~\ref{thm: spl_lower_bd_1} for the special case of Gaussian design.
% \begin{corollary}\label{corollary2:lower_bd_spl}
% If  each entry of $\vecA$ is chosen iid $\cN(0,1)$, then for a uniformly chosen $k$-sparse vector $\bx$, an algorithm $\phi$ satisfies 
% $$\bbP\inp{\phi(\vecA, {\by}) \neq \bx}\leq \delta$$
% only if the number of measurements 
% $$m\geq \frac{k\log\inp{\frac{n}{k}} - 2\log{n}- h_2(\delta) - \delta k\log{n}}{\log\inp{1+\frac{k}{\sigma^2}}} .$$
% \end{corollary}

% Corollary~\ref{corollary2:lower_bd_spl} can also be proved directly for any sensing matrix $\vecA$ which satisfies \eqref{eq:power_constraint} (non-necessarily a Gaussian design). 


% \begin{figure}[t]
% \includegraphics[width=8cm]{plot.png}
% \centering
% \caption{The figure shows the plot of the MLE upper bound \eqref{eq:upper_bd_mle} (given by m1) for different values of $n$. This is displayed in blue color. A plot of $\frac{2nN(l)}{\log\inp{\frac{ l}{2\sigma^2}+1}}$ is also presented for $l = k\inp{1-\frac{k}{n}}$ in orange color, given by m2. In these plots,  $\sigma^2$ is set to 1 and $k$ is $0.2n$. }\label{plot:1}
% \end{figure}



\section{Proofs Missing from Section~\ref{sec:rounds}} %Rounds and Phases}
\subsection{Partitioning into Rounds}\label{A1}
Suppose we wish to construct $\kappa$ rounds with length parameters $\{a_1,a_2,\dots,a_{\kappa}\}$. We will analytically compute the participation probability of the $\kappa$-th round.

\begin{proof}[Proof of Lemma \ref{lemma:RoundPro}:]
    Consider an agent $b$ and let $R_\kappa$ denote the set of agents in the $\kappa$-th round, then:
    \begin{equation}
        \Prob{b \in R_\kappa} = \sum_{x_{1} = 0}^{n_1}\Prob{x_{1}}\sum_{x_{2} = 0}^{n_{2}}\dots\sum_{x_{\kappa-1} = 0}^{n_{\kappa-1}}\Prob{x_{\kappa-1}}\sum_{x_{\kappa} = 1}^{n_{\kappa}}\Prob{x_{\kappa}}\cdot \Prob{b\in R_\kappa| x_1,\dots x_{\kappa}}.
    \end{equation}
    Where $n_1=n$ and $n_i = n-\sum_{j=1}^{i-1}x_i$. By using the fact that the order of the agents follows a uniform distribution  we get that:
    \begin{equation}
        \Prob{b\in R_\kappa} =\sum_{x_{1} = 0}^{n_1}\Prob{x_{1}}\sum_{x_{2} = 0}^{n_{2}}\dots\sum_{x_{\kappa-1} = 0}^{n_{\kappa-1}}\Prob{x_{\kappa-1}}\sum_{x_{\kappa} = 1}^{n_{\kappa}}\Prob{x_{\kappa}}\cdot \frac{x_{\kappa}}{n}.
    \end{equation}
     By definition, $\sum_{x_{\kappa} = 1}^{n_{\kappa}}\Prob{x_{\kappa}}\cdot x_{\kappa} = \E{}{\kappa\text{-th round length}} = a_{\kappa}\cdot n_{\kappa}$, thus:
    \begin{equation}
        \Prob{b\in R_\kappa} =\sum_{x_{1} = 0}^{n_1}\Prob{x_{1}}\sum_{x_{2} = 0}^{n_{2}}\dots\sum_{x_{\kappa-1} = 0}^{n_{\kappa-1}}\Prob{x_{\kappa-1}}\frac{n_{\kappa}}{n}\cdot a_{\kappa}.
    \end{equation}
     Observing that $n_{\kappa} = n_{\kappa-1} - x_{\kappa-1}$ and applying the same argument as before repeatedly, we finally get:
        \begin{equation}
        \Prob{b\in R_\kappa} =(1-a_1)\cdot(1-a_2)\cdots (1-a_{\kappa-1})\cdot a_{\kappa}.
    \end{equation}

    Now it remains to prove the independency among agents.

    Suppose 2 agents $b_1,b_2$, then:
    \begin{equation}
        \Prob{\{b_1,b_2\}\in R_\kappa} =\sum_{x_{1} = 0}^{n_1-2}\Prob{x_{1}}\sum_{x_{2} = 0}^{n_{2}-2}\dots\sum_{x_{\kappa-1} = 0}^{n_{\kappa-1}-2}\Prob{x_{\kappa-1}}\sum_{x_{\kappa} = 2}^{n_{\kappa}}\Prob{x_{\kappa}}\cdot \Prob{\{b_1,b_2\}\in R_\kappa|x_1,\dots,x_{\kappa}}.
    \end{equation}
    Since agent order follows a uniform distribution and the round lengths are drawn from a binomial distribution, we get:
    \begin{equation}
        \Prob{\{b_1,b_2\}\in R_\kappa} =\sum_{x_{1} = 0}^{n_1-2}\Prob{x_{1}}\sum_{x_{2} = 0}^{n_{2}-2}\dots\sum_{x_{\kappa-1} = 0}^{n_{\kappa-1}-2}\Prob{x_{\kappa-1}}\sum_{x_{\kappa} = 2}^{n_{\kappa}}\begin{pmatrix}
            n_\kappa\\
            x_\kappa
        \end{pmatrix}\cdot a_\kappa ^{x_\kappa}\cdot (1-a_\kappa)^{n_\kappa - x_\kappa}\cdot  \frac{\begin{pmatrix}
            x_\kappa\\
            2
        \end{pmatrix}}{\begin{pmatrix}
            n\\
            2
        \end{pmatrix}}.
    \end{equation}
    Using the identity: 
    \begin{equation}
        \begin{pmatrix}
            n_\kappa\\
            x_\kappa
        \end{pmatrix} \cdot \begin{pmatrix}
            x_\kappa\\
            2
        \end{pmatrix} = \begin{pmatrix}
            n_\kappa\\
            2
        \end{pmatrix}\cdot \begin{pmatrix}
            n_\kappa-2\\
            x_\kappa-2
        \end{pmatrix},
    \end{equation} we obtain:
    \begin{equation}
        \Prob{\{b_1,b_2\}\in R_\kappa} =\sum_{x_{1} = 0}^{n_1-2}\Prob{x_{1}}\sum_{x_{2} = 0}^{n_{2}-2}\dots\sum_{x_{1} = 0}^{n_{\kappa-1}-2}\Prob{x_{\kappa-1}}\cdot \frac{\begin{pmatrix}
            n_\kappa\\
            2
        \end{pmatrix}}{\begin{pmatrix}
            n\\
            2
        \end{pmatrix}}\sum_{x_{\kappa} = 2}^{n_{\kappa}}\begin{pmatrix}
            n_\kappa-2\\
            x_\kappa-2
        \end{pmatrix}\cdot a_\kappa ^{x_\kappa}\cdot (1-a_\kappa)^{n_\kappa - x_\kappa}.
    \end{equation}
    Noting that:
    \begin{equation}
        \sum_{x_{\kappa} = 2}^{n_{\kappa}}\begin{pmatrix}
            n_\kappa-2\\
            x_\kappa-2
        \end{pmatrix}\cdot a_\kappa ^{x_\kappa}\cdot (1-a_\kappa)^{n_\kappa - x_\kappa} = a_\kappa ^2 \cdot \sum_{x_{\kappa} = 0}^{n_{\kappa}-2}\begin{pmatrix}
            n_\kappa-2\\
            x_\kappa
        \end{pmatrix}\cdot a_\kappa ^{x_\kappa}\cdot (1-a_\kappa)^{n_\kappa-2 - x_\kappa} =a_\kappa ^2.
    \end{equation}
    and thus:
    \begin{equation}
        \Prob{\{b_1,b_2\}\in R_\kappa} =\sum_{x_{1} = 0}^{n_1-2}\Prob{x_{1}}\sum_{x_{2} = 0}^{n_{2}-2}\dots\sum_{x_{\kappa-1} = 0}^{n_{\kappa-1}-2}\Prob{x_{\kappa-1}}\cdot \frac{\begin{pmatrix}
            n_\kappa\\
            2
        \end{pmatrix}}{\begin{pmatrix}
            n\\
            2
        \end{pmatrix}}\cdot a_{\kappa}^2.
    \end{equation}
    Now we focus on the last sum,
    \begin{equation}\label{eq:2}
        \sum_{x_{\kappa-1} = 0}^{n_{\kappa-1}-2}\Prob{x_{\kappa-1}}\cdot \frac{\begin{pmatrix}
            n_{\kappa}\\
            2
        \end{pmatrix}}{\begin{pmatrix}
            n\\
            2
        \end{pmatrix}}\cdot a_{\kappa}^2 =a_{\kappa}^2\sum_{x_{\kappa-1} = 0}^{n_{\kappa-1}-2} \begin{pmatrix}
            n_{\kappa-1}\\
            x_\kappa
        \end{pmatrix}\cdot a_{\kappa-1} ^{x_{\kappa-1}}\cdot (1-a_{\kappa-1})^{n_{\kappa-1} - x_{\kappa-1}}\cdot \frac{\begin{pmatrix}
            n_\kappa\\
            2
        \end{pmatrix}}{\begin{pmatrix}
            n\\
            2
        \end{pmatrix}}.
    \end{equation}
    We notice that:
    \begin{equation}\label{eq:1}
        \begin{pmatrix}
            n_{\kappa-1}\\
            x_\kappa
        \end{pmatrix}\cdot \frac{\begin{pmatrix}
            n_\kappa\\
            2
        \end{pmatrix}}{\begin{pmatrix}
            n\\
            2
        \end{pmatrix}} = \begin{pmatrix}
            n_{\kappa-1}-2\\
            x_{\kappa-1}
        \end{pmatrix}\cdot \frac{\begin{pmatrix}
            n_{\kappa-1}\\
            2
        \end{pmatrix}}{\begin{pmatrix}
            n\\
            2
        \end{pmatrix}}, 
    \end{equation}
    and by applying  (\ref{eq:1}) to  (\ref{eq:2}) we get:
    \begin{equation}\label{eq:4}
        \sum_{x_{\kappa-1} = 0}^{n_{\kappa-1}-2}\Prob{x_{\kappa-1}}\cdot \frac{\begin{pmatrix}
            n_\kappa\\
            2
        \end{pmatrix}}{\begin{pmatrix}
            n\\
            2
        \end{pmatrix}}\cdot a_{\kappa}^2 =a_{\kappa}^2\frac{\begin{pmatrix}
            n_{\kappa-1}\\
            2
        \end{pmatrix}}{\begin{pmatrix}
            n\\
            2
        \end{pmatrix}}\cdot \sum_{x_{\kappa-1} = 0}^{n_{\kappa-1}-2}\begin{pmatrix}
            n_{\kappa-1}-2\\
            x_{\kappa-1}
        \end{pmatrix}\cdot a_{\kappa-1} ^{x_{\kappa-1}}\cdot (1-a_{\kappa-1})^{n_{\kappa-1} - x_\kappa}.
    \end{equation}
    Next we notice that:
    \begin{equation}\label{eq:3}
        \sum_{x_{\kappa-1} = 0}^{n_{\kappa-1}-2}\begin{pmatrix}
            n_{\kappa-1}-2\\
            x_{\kappa-1}
        \end{pmatrix}\cdot a_{\kappa-1} ^{x_{\kappa-1}}\cdot (1-a_{\kappa-1})^{n_{\kappa-1} - x_\kappa} = (1-a_{\kappa -1})^{2},
    \end{equation} and combining (\ref{eq:4}), (\ref{eq:3}) we get that:
    \begin{equation}
        \sum_{x_{\kappa-1} = 0}^{n_{\kappa-1}-2}\Prob{x_{\kappa-1}}\cdot \frac{\begin{pmatrix}
            n_\kappa\\
            2
        \end{pmatrix}}{\begin{pmatrix}
            n\\
            2
        \end{pmatrix}}\cdot a_{\kappa}^2 =\frac{\begin{pmatrix}
            n_{\kappa-1}\\
            2
        \end{pmatrix}}{\begin{pmatrix}
            n\\
            2
        \end{pmatrix}}\cdot (1-a_{\kappa-1})^2\cdot a_{\kappa}^2
    \end{equation}
    It is now evident that by following the same procedure for the rest of the sums we get that:
    \begin{equation}
        \Prob{\{b_1,b_2\}\in R_\kappa} = a_{\kappa}^2\cdot\prod_{i=1}^{\kappa-1}(1-a_i)^2 = \Prob{b_1\in R_\kappa}\cdot\Prob{b_2\in R_\kappa} 
    \end{equation}
    This confirms that agent selections are independent.
\end{proof}

Lemma \ref{lemma:RoundPro} also proves that rounds are indeed equivalent with a random set of the whole input where each agent is included with the same probability.\\

An important property of the lengths drawn in Period 2 that will turn our very useful in the rest of our analysis is the following:
\begin{lemma}\label{lemma:SeqBound}
    For every $i\in \{2,\dots, T\}$ (where $T$ is the number of intervals in the partitioning of $[\vmax, n\cdot\vmax]$ used in Section~\ref{sec:SecondPeriod}), it holds that:
    \[a_i\cdot \gamma_i \le \frac{1}{i^{10}}\]
\end{lemma}
\begin{proof}
    We first notice the following:
    \begin{equation}
        \begin{split}
            a_i \cdot \gamma_i = 81e\cdot \frac{\vmax}{t_i}\cdot \frac{3}{2}\cdot\log\left(\frac{t_i}{\vmax}\right)
        \end{split}
    \end{equation}
    Applying the fact that $\log\left(\frac{t_i}{\vmax}\right) < \frac{t_{i-1}}{\vmax}$, we get:
    \begin{equation}
         a_i \cdot \gamma_i < 122e\cdot \frac{t_{i-1}}{t_i}
    \end{equation}
    Now consider the following sequences of real numbers:
    \begin{equation}
        \begin{split}
        h_{1,i} &= 122e\cdot \frac{t_{i-1}}{t_i}\\
        h_{2,i} &= \frac{1}{i^{10}}
        \end{split}
    \end{equation}
    we have that:
    \begin{equation}
      h_{1,2} = \frac{122e}{10^7} < \frac{1}{1024} = h_{2,2}  
    \end{equation}
    Taking into account that $h_{1,i} = 122e\cdot \frac{t_{i-1}}{t_i} = 122e\cdot 2^{-\frac{t_{i-1}}{\vmax}}$ it becomes evident that $h_1$ drops to $0$ faster than $h_2$ and thus the desired inequality is proved.
\end{proof}

\subsection{The Proof of Lemma~\ref{lemma:PART}}
\label{sec:PART}

We are now ready to prove Lemma~\ref{lemma:PART}
\begin{proof}[Proof of Lemma \ref{lemma:PART}:]\label{proof:PART}
    First let $R$ be the set of all the rounds that \hyperref[alg:LMMECH]{LM-Mechanism} uses.
    Using lemma \ref{lemma:RoundPro} we get the following trivial bound for the participation probability $q_j$ of the $j-$th round:
    \begin{equation}
        q_j \ge a_j\cdot \prod_{i\in R}(1-a_i) 
    \end{equation}
    All that remains is to bound $\prod_{i\in R}(1-a_i)$.
    We break this product into 3 parts:
    \begin{enumerate}
        \item The first term of the product is $\frac{2}{3}$ and it corresponds to $\tau$.
        \item Next comes the following product of terms from the tests of the second period:
        \begin{equation}
        \begin{split}
        \prod_{i=2}^{T}(1-a_i)^{\gamma_i} \ge \prod_{i=2}^{T} (1-a_i\cdot \gamma_i), 
        \end{split}
        \end{equation}
        where the last inequality comes from taking the first order Taylor approximation of the convex function $g(x) = (1-x)^{\log(\frac{c}{x})}$.
        Using Lemma \ref{lemma:SeqBound}, we get:
        \begin{equation}
           \prod_{i=2}^{T}(1-a_i)^{\gamma_i} \ge \prod_{i=2}^{T} \left(1-\frac{1}{i^{10}}\right) \ge \frac{9}{10}.
        \end{equation} 
        \item Lastly, we have the terms from the BinarySearch and Exploitation:
        \begin{equation}
            \prod_{i=1}^{2\cdot\ell} (1-a) = (1-a)^{2\cdot\ell}\ge \frac{1}{e}.
        \end{equation}
    \end{enumerate}
    Combining all of the above together:
    \begin{equation}
        q_j \ge a_j\cdot \prod_{i\in R}(1-a_i) \ge \frac{6}{10e}\cdot a_j
    \end{equation}
\end{proof}
Before we delve deeper into the properties of our rounds we first need to establish a basic property of Threshold-Based mechanisms.

\subsection{The Proof of Lemma \ref{lemma:roundProb}}\label{proof4.5}

\begin{lemma}[\cite{Bada2012}]\label{lem:threshold}
Let $k = \frac{\opt}{v_{\max}}$. For any $ \hat{t} = h \cdot \opt$, such that $h\in (0,1)$, the following holds:
    \begin{equation}
            \text{PostedPrices}(\hat{t},\mathcal{N},B) \ge \min\left\{(1-h)\cdot \opt,\ \  \left(h-\frac{1}{k}\right)\cdot \opt\right\}\,,
        \end{equation} where $\text{PostedPrices}(\hat{t},\mathcal{N},B)$ denotes the expected value gathered by applying linear pricing with threshold $\hat{t}$ to the set $\N$ of agents with budget $B$.
\end{lemma}
\begin{proof}
    Suppose that we use \(\hat{t}=h \cdot \opt\) for the entire input. Let \( S \) be the allocation obtained using \(\hat{t}\).
    We consider the following cases:
    \begin{enumerate}
        \item There is an agent $b$ such that $b \in S^{*}\setminus S$ cause of insufficient Budget. In this case:
        \begin{equation}
            \begin{split}
                &\sum_{b_{j}\in S}p_{j} = \sum_{b_{j}\in S} \frac{B}{\hat{t}}(f_{S_{j}}(b_{j})) = \frac{B\cdot f(S)}{\hat{t}} = B\cdot \frac{f(S)}{\opt \cdot h} \text{ and } \\
            &c(b) \le \frac{B}{\hat{t}} \cdot f(b) \le B\cdot \frac{1}{\opt\cdot h} \cdot \frac{\opt}{k} = B\cdot \frac{1}{h\cdot k}
            \end{split}
        \end{equation}
        and thus
        \begin{equation}\begin{split}
            &B < c(b) + \sum_{b_{j}\in S}p_{j} \le B\cdot \frac{f(S)}{\opt \cdot h} + B\cdot \frac{1}{h\cdot k} \Longrightarrow \\
            &\opt\cdot \left(h-\frac{1}{k}\right) < f(S)
        \end{split}
        \end{equation}
        \item Otherwise: $c(b)> \frac{B}{\hat{t}}\cdot f_{S}(b)$ for every $b\in S^{*}\setminus S$ and thus
        \begin{equation}
            \sum_{b\in S^{*}\setminus S} f_S(b) = \sum_{b\in S^{*}\setminus S} \frac{f_{S}(b)}{c(b)} \cdot c(b) \le \sum_{b\in S^{*}\setminus S}\frac{\hat{t}}{B}\cdot c(b)\le \hat{t}  
        \end{equation}
        and it follows that \begin{equation}
        \begin{split}
                        &\opt-f(S) \le f(S^{*}\cup S) -f(S) \le \sum_{b\in S^{*}\setminus S} f_S(b)\le \hat{t} \Longrightarrow \\
            &f(S) \ge (1-h)\cdot \opt
        \end{split}
        \end{equation}
    \end{enumerate}
    
    Finally, we conclude that $f(S)\ge \min\{(1-h),(h-\frac{1}{k})\}\cdot \opt$. 
\end{proof}

We notice that the last bound is maximized for $\hat{h} = \frac{k+1}{2k}$ and thus we consider it to be exactly that for the rest of our analysis.

Let $\hat{t}=\hat{h}\cdot \opt = \frac{k+1}{2k}\cdot \opt$ and $S$ be the allocation we get by applying the threshold $\hat{t}$ to the whole input, for some permutation of the agents. Denote $\{s_i\}_{i=1}^{|S|}$ the elements of $S$ in order of appearance and let $S_i = \bigcup_{j=1}^{i}\{s_j\}$. Now, letting $w_i = f_{S_{i-1}}(s_i)$, we can bound the cost of all subsets $X\subseteq S$ the following way:
 
\begin{lemma}\label{lem4.6}
     For every subset $X\subseteq S$ it is true that:
   \begin{equation}
       \begin{split}
           c(X)\le \frac{\sum_{i: s_i \in X}w_i}{\hat{h}\cdot \opt}\cdot B
       \end{split}
   \end{equation}
\end{lemma}
\begin{proof}
    The cost of each agent in $S$ is at most $\frac{f_{S_i}(s_i)}{\hat{t}}\cdot B = \frac{w_i}{\hat{t}}\cdot B.$ Consequently, for each subset $X$ of $S$ we get: \begin{equation}
        \begin{split}
            c(X) = \sum_{j:b_j \in X}c(b_j) 
            \le \sum_{i:s_i \in X}\frac{w_i}{\hat{t}}\cdot B . 
        \end{split}
    \end{equation}
\end{proof}
The following lemma showcases a way to lower bound the value of subsets $X \subseteq S$.
\begin{lemma}\label{lem4.7}
    Let $I$ be the set of indices of elements $s_i\in S$ sampled in a round. Then, the following holds:
    \begin{equation}
        \begin{split}
            f\lp(\bigcup_{j\in I}\{s_j\}\rp) \ge \sum_{j\in I} w_j
        \end{split}
    \end{equation}
\end{lemma}
\begin{proof}
   Let $J = \bigcup_{j\in I}\{s_j\}$. The value of the set $J$ is the same regardless of the order of appearance of elements $s_j$. Thus we can suppose that they appear in the same order as in $S$ and thus:
   \begin{equation}
       \begin{split}
           f(J) = \sum_{j\in I} f_{S_{j-1}\cap J}(s_j) \ge \sum_{j\in I} f_{S_{j-1}}(s_j) = \sum_{j\in I} w_j  
       \end{split}
   \end{equation}
\end{proof}

Combining the lemmas \ref{lem4.6} and \ref{lem4.7} we get the following bound for the cost of all subsets $X\subseteq S$.

\begin{corollary}\label{cor4.2}
    For every $X\subseteq S$ it holds that:
    \begin{equation}
        \begin{split}
            c(X) \le \frac{2k}{k+1}\cdot \frac{f(X)}{OPT}\cdot B
        \end{split}
    \end{equation}
\end{corollary}
Let, once again, $R$ be the set of rounds. Suppose that $X$ is the subset of agents of $S$ that is included in a round $j$ with length parameter $a_j$, such that $a_j \cdot \frac{\opt}{\vmax}\ge81\, e$. Now
consider the following random variables for $j\in R, i \in [|S|]$:
\begin{equation}
    \begin{split}
            X_i^{j} &= \begin{cases} 
      w_i & \text{ w.p.  } \frac{6}{10e}\cdot a_j\\
      0 & \text{ otherwise  } 
   \end{cases}\\
   X^{j}&= \sum_{i=1}^{|S|}X_{i}^{j}
    \end{split}
\end{equation}

 By lemmas ~\ref{lemma:RoundPro},~\ref{lem4.7} it is evident that $X^{j}$ is a lower bound of the value $f(X)$ contained in round $j$. Furthermore, we can prove the following:

\begin{lemma}\label{lemmaTool}
    Let $\opt_j$ be the value of an optimal solution for an $a_j$-round $j$, when restricted to budget $B_j = 3\cdot C\cdot a_j\cdot B$ budget, where $a_j$ is good with respect to $\opt$. If $C\cdot a_j\cdot \opt\le X^j$ then 
    \begin{equation}
        \begin{split}
            \opt_j \ge C\cdot a_j\cdot \opt
        \end{split}
    \end{equation}
\end{lemma}
\begin{proof}
Let $X$ be the set of agents included in round $j$. Define $S_{j} = X\cap S$ and denote by $s_{j,i}$ the $i-$th element of $S_j$ in the order of appearance within the round. We can select the first $\mu$ agents $ \{s_{j,i}\}_{i=1}^{\mu}$ (for some $\mu \in \mathbb{N}$) such that their total value $f( \{s_{j,i}\}_{i=1}^{\mu})$ does not exceed $C\cdot a_j\cdot \opt$ by more than $\vmax.$ More specifically, there exists $\mu \in \mathbb{N}$ such that:
\begin{equation}
    \begin{split}
        C\cdot a_j\cdot \opt &\le f\left(\bigcup_{i=1}^{\mu}\{s_{j,i}\}\right ) \\
        &\le C\cdot a_j\cdot \opt+ \vmax\ \\
        &= \left(C\cdot a_j+ \frac{\vmax}{\opt}\right)\cdot \opt 
    \end{split}
\end{equation}
Which, by Corollary \ref{cor4.2}, implies:
\begin{equation}
            c\lp(\bigcup_{i=1}^{\mu}\{s_{j,i}\}\rp) \le 2\cdot\left (C\cdot a_j \cdot + \frac{\vmax}{\opt}\right)\cdot B\le 3\cdot C\cdot a_j\cdot B,
\end{equation}
where the last inequality is due to $a_j$ being good with respect to $\opt$, i.e., $a_j \geq 81\,e\,\vmax\,/\,\opt$.
\end{proof}

Lemma~\ref{lemmaTool} enables us to analyze the value of $X^j$ to determine whether a round is dense, rather than relying on $\opt_j$. Now we can use concentration bounds on the random variable $X_{j}$ to bound the probability that the $j$-th round is dense.

We now ready to conclude this section with presenting the proof of Lemma~\ref{lemma:roundProb}.

\begin{proof}[Proof of lemma \ref{lemma:roundProb}:]

     By Lemma \ref{lemmaTool}:
    \begin{equation}
        \begin{split}
            \Prob{\text{round $j$ is not dense}} \le \Prob{X^{j}<C\cdot a_j \cdot \opt}
        \end{split}
    \end{equation}
    We notice that $\frac{1}{2}\cdot \frac{6}{10e} \cdot \frac{k-1}{2k} \ge C$ (for $k\ge 10^7$) and thus:
    \begin{equation}
        \Prob{\text{round $j$ is not dense}} \le \Prob{X^{j}<\frac{1}{2}\cdot \frac{6}{10e}\cdot a_j \cdot \frac{k-1}{2k}\cdot \opt}
    \end{equation}
    Now using the fact that $\E{}{X^j} = q_j\cdot f(S)$, along with Lemma \ref{lemma:RoundPro}:
        \begin{equation}
        \Prob{\text{round $j$ is not dense}} \le\Prob{X^{j}\le \frac{1}{2}\cdot \E{}{X^{j}}} 
    \end{equation}
    We can now apply Chernoff bound (setting $\delta = \frac{1}{2}$) on random variable $X^{j}$ to get:
    \begin{equation}
        \begin{split}
           \Prob{\text{round $j$ is not dense}} &\le \Prob{X^{j}\le (1-\delta)\E{}{X^{j}}}\\
            &\le \exp\lp\{\frac{-\delta^2 \E{}{X^{j}}}{2\cdot \max{X_{i}^{j}}}\rp\}
        \end{split}
    \end{equation}
    Using the fact that $\E{}{X^{j}} =  \frac{6}{10e}\cdot a_j\cdot f(S) \ge \frac{6}{10e}\cdot a_j\cdot \frac{k-1}{2k}\cdot \opt$:
    \begin{equation}
        \Prob{\text{round $j$ is not dense}} \le \exp\lp\{\frac{-\delta^2 \cdot \frac{6}{10e}\cdot a_j\cdot\frac{k-1}{2k}\cdot \opt}{2\vmax}\rp\} 
    \end{equation}
    After calculations we reach the following:
    \begin{equation}
        \Prob{\text{round $j$ is not dense}} \le\exp\lp\{\frac{- a_j \cdot  (k-1)}{27e}\rp\} \le 0.1
    \end{equation}
    The last inequality follows from the fact that $a_j$ is good with respect to $\opt$, i.e., $a_j \geq 81\,e\,\vmax\,/\,\opt$, and from the hypothesis that $k\ge 10^7$.
\end{proof}

\subsection{The Proof of Lemma~\ref{lemma:PhaseProbability}}\label{negdep}

To extend our analysis to dense phases we should first pay attention to the dependency between the probability of success of consecutive rounds. Indeed, we should notice that conditional to the previous round having failed the next round is more likely to succeed. Such a dependency bares a significant resemblance to the Balls and Bins problem \cite{johnson1977urn}. Below we present how to utilize results from \cite{negativedepend} to prove that concentration of mass arguments still hold in this setting.

In the following we will present a complete proof of the validity of Chernoff bound on the following random variables:

\begin{equation}
    \begin{split}
    Q_i &= \begin{cases} 
      1 ,& \text{ if round } $i$ \text{ is dense}\\
      0, & \text{ otherwise  } 
   \end{cases}\\
    \end{split}
\end{equation}

Firstly, we need to present some important notions from \cite{negativedepend}.

\begin{definition}[Negative Association]
   Let $\mathbf{X} := (X_1, \dots, X_n)$ be a vector of random variables.

\begin{enumerate}
    \item [$(-A)$] The random variables, $\mathbf{X}$ are \emph{negatively associated} if for any two disjoint index sets, $I, J \subseteq [n]$,
\[
\mathbb{E}[f(X_i, i \in I)g(X_j, j \in J)] \leq \mathbb{E}[f(X_i, i \in I)]\mathbb{E}[g(X_j, j \in J)]
\]
for all functions $f : \mathbb{R}^{|I|} \to \mathbb{R}$ and $g : \mathbb{R}^{|J|} \to \mathbb{R}$ that are both non-decreasing or both non-increasing.
\end{enumerate}
\end{definition}

\begin{lemma}[Zero-One Lemma for ($-A$)] \label{ZeroOne}
    If $X_1, \dots, X_n$ are zero-one random variables such that $\sum_i X_i = 1$, then $X_1, \dots, X_n$ satisfy ($-A$).
\end{lemma}

\begin{proposition}\label{proptrans}
    \begin{enumerate}
    \item If $\mathbf{X}$ and $\mathbf{Y}$ satisfy ($-A$) and are mutually independent, then the augmented vector $(\mathbf{X}, \mathbf{Y}) = (X_1, \dots, X_n, Y_1, \dots, Y_m)$ satisfies ($-A$).
    
    \item Let $\mathbf{X} := (X_1, \dots, X_n)$ satisfy ($-A$). Let $I_1, \dots, I_k \subseteq [n]$ be disjoint index sets, for some positive integer $k$. For $j \in [k]$, let $h_j : \mathbb{R}^{|I_j|} \to \mathbb{R}$ be functions that are all non-decreasing or all non-increasing, and define $Y_j := h_j(X_i, i \in I_j)$. Then the vector $\mathbf{Y} := (Y_1, \dots, Y_k)$ also satisfies ($-A$). That is, non-decreasing (or non-increasing) functions of disjoint subsets of negatively associated variables are also negatively associated.
\end{enumerate}
\end{proposition}

\begin{proposition}\label{lemChern}
    The Chernoff--Hoeffding bounds are applicable to sums of variables that satisfy the negative association condition ($-A$).
\end{proposition}

 We are now ready to complete our proof:
 \begin{lemma}
     We can apply Chernoff bound on $\{Q_i\}_{i\in[\ell]}$
 \end{lemma}
 \begin{proof}
     Consider the following random variables:
     \begin{equation}
         \begin{split}
              Y_i^{j} &= \begin{cases} 
      1 & \text{ if agent $b_i$ is in round $j$}\\
      0 & \text{ otherwise  } 
   \end{cases}\\
    Q_j &= \begin{cases} 
      1 & \text{ if  } \sum_{i=1}^{n}w_i \cdot Y_{i}^{j} \ge  C\cdot a_j \cdot  OPT\\
      0 & \text{ otherwise  } 
   \end{cases}
         \end{split}
     \end{equation}

     By lemma \ref{ZeroOne} it is immediate that $(Y_i^{1},Y_{i}^2,\dots,Y_{i}^{\ell})$ are negatively associated. By Proposition \ref{proptrans}.1, along with the independency among agents, we can conclude that the whole set $(Y_{i}^{j})_{i\in [n],j\in[\ell]}$ is negatively associated. Finally, we use proposition \ref{proptrans}.2, setting 
     \begin{equation}
         \begin{split}
             h_j = \begin{cases} 
      1 & \text{ if  } \sum_{i=1}^{n}w_i\cdot Y_{i}^{j} \ge C\cdot a\cdot OPT\\
      0 & \text{ otherwise  } 
   \end{cases}
         \end{split}
     \end{equation} which is an increasing function of $Y_i^{j}$, to get that
     $(Q_1,Q_2,\dots,Q_\ell)$ are negatively associated. We now get the desired result directly from Proposition \ref{lemChern}
 \end{proof}

The following lemma effectively leverages this dependency to demonstrate that all phases, whose rounds are good with respect to their thresholds, are dense with significant probability.

\begin{proof}[Proof of Lemma \ref{lemma:PhaseProbability}]
Proposition \ref{lemma:roundProb} suggests that every round succeeds with probability at least $0.9$. Now let $\{Q_i\}_{i=1}^{\delta}$ be random variables under the following distribution:
\begin{equation}
    \begin{split}
    Q_i &= \begin{cases} 
      1 ,& \text{ if round } $i$ \text{ is dense}\\
      0, & \text{ otherwise  } 
   \end{cases}\\
    \end{split}
\end{equation}
 By leveraging the previous results, we can argue that the variables $ Q_i $ exhibit negative dependence, which permits the use of concentration inequalities.

Let $Q = \sum_{i=1}^m  Q_i$. The probability a Phase is not dense is, by applying Chernoff bound on $Q$, at most:
\begin{equation}
    \begin{split}
       \Prob{ Q < \frac{\delta}{2} } &\le \Prob{ Q < \frac{1}{1.8} \cdot \E{}{Q} } \\
       &\le \exp\lp\{ - \frac{(\frac{0.8}{1.8})^2\cdot 0.9 \cdot \delta}{2} \rp\} \\
       & = \exp\lp\{ - \frac{4 \cdot \delta}{45} \rp\}
    \end{split}
\end{equation}
\end{proof}


 \subsection{Lemmas Used in Estimating the Correctness Probability in Section~\ref{sec:estimate}}
 \label{A3}

\begin{lemma}\label{lemma:E2}
    The event $\mathcal{E}_2$ happens with probability at least $0.9$.
\end{lemma}
\begin{proof}
    To successfully identify the correct intervals, we need the last, or the second to last, successful Phase to include $\frac{\opt}{8}$. Let us denote by $P_i$ the Phase that tests threshold $t_i$. Let $\frac{\opt}{8}\in (t_j,t_{j+1}]$, then, if $P_j$ is dense, $P_j$ will be successful and all $P_i$, with $i>j+1$ will fail, by Lemma \ref{lemma:PERIOD1}. Thus, the set of intervals we end up with will include $(t_j,t_{j+1}].$ Thus, we need to investigate the probability that Phase $P_j$ is dense. By Lemma \ref{lemma:PhaseProbability} we have that:
\begin{equation}
    \begin{split}
            \Prob{P_j \text{ is dense}} &\ge 1-\exp\left\{-\frac{4\cdot \gamma_{j}}{45}\right\}\\
        &\ge 1-\exp\left\{-\frac{4\cdot \gamma_{2}}{45}\right\}  =1-\exp\left\{-3\right\}\ge 0.9 
    \end{split}
\end{equation}
\end{proof}

 \begin{lemma}\label{lemma:BinaryProb}
    All phases tested with a threshold $t\le\frac{\opt}{8}$, during BinarySearch and Exploitation, are dense with probability at least $0.9$ (i.e., the event $\Event_3$ occurs with probability at least $0.9$). 
\end{lemma}

\begin{proof}
In the worst case we might need all of the $\frac{2\ell}{m}$ Phases to be dense.
Lemma \ref{lemma:roundProb} suggests that every round succeeds with probability at least $0.9$. Now let $\{Q_i\}_{i=1}^{m}$ be random variables under the following distribution:
\begin{equation}
    \begin{split}
    Q_i &= \begin{cases} 
      1 ,& \text{ if round } $i$ \text{ is dense}\\
      0, & \text{ otherwise  } 
   \end{cases}\\
    \end{split}
\end{equation}

Let $Q = \sum_{i=1}^m  Q_i$.
As argued before (\ref{negdep}), we can apply concentration of mass arguments to $Q$. The probability a Phase is not dense is, by applying Chernoff bound on $Q$, at most:
\begin{equation}
           \Prob{ Q < \frac{m}{2} } \le \Prob{ Q < \frac{1}{1.8} \cdot \E{}{Q} } 
       \le \exp\lp\{ - \frac{(\frac{0.8}{1.8})^2\cdot 0.9 \cdot m}{2} \rp\}
\end{equation}
Using the definitions of $m$ and $t_{\max}$, we get:
\begin{equation}
    \Prob{ Q < \frac{m}{2} } \le\frac{1}{\log\left(\frac{t_{\max}}{t_{\min}}\right)}= \frac{1}{t_{\min}}
\end{equation}
Leveraging our hypothesis that $t_{\min}\ge 10^7$:
\begin{equation}\label{eq:77}
    \Prob{ Q < \frac{m}{2} } \le \frac{1}{20\left\lceil\log\left(t_{\min}\right)\right\rceil}
\end{equation}
To get the desired probability bound, we will now use Union Bound on the event where the first $\frac{2\cdot\ell}{m}$ Phases are dense. Namely:
\begin{equation}
    \begin{split}
         \Prob{\text{First }\frac{2\cdot\ell}{m} \text{ Phases are dense} } &\ge 1 - 2\cdot \left\lceil\log\log\left(\frac{t_{\max}}{t_{\min}}\right)\right\rceil \cdot \Prob{\text{A Phase is not dense}} 
    \end{split}
\end{equation}
Now using the definition of $t_{\max} = 2^{\frac{t_{\min}}{\vmax}}\cdot t_{\min}$, along with inequality~\ref{eq:77}:
\begin{equation}
    \Prob{\text{First }\frac{2\cdot\ell}{m} \text{ Phases are dense} } \ge 1 - \frac{\left\lceil\log\left(t_{\min}\right)\right\rceil}{10\cdot \left\lceil\log\left(t_{\min}\right)\right\rceil} = 0.9
\end{equation}
\end{proof}

Sampling round lengths from a distribution introduces the risk that we run out of agents during the execution of our mechanism. Consequently, our analysis relies on the assumption that the total round lengths remain within the input size. Fortunately, the following lemma guarantees that this condition holds with constant probability:  

\begin{lemma}\label{lemma:Fit}
    We will not run out of agents during the execution of \hyperref[alg:LMMECH]{LM-Mechanism} with probability at least $0.97$ (i.e., the event $\Event_4$ occurs with probability at least $0.97$).
\end{lemma}
\begin{proof}
    We will investigate the round lengths drawn as follows:
\begin{enumerate}
    \item Applying Chernoff bound onto the random length $\tau$ drawn for the estimation of $\vmax$ we get that:
    \begin{equation}
        \begin{split}
            \Prob{\tau \ge \frac{2}{5}\cdot n} \le \exp\left\{-\frac{n}{225}\right\}\le 0.01,
        \end{split}
    \end{equation}
    where the last inequality holds for $n\ge10^7$, which must be true for our assumption $\opt\ge10^7\vmax$ to hold.
    %We will denote the event where $\tau \ge \frac{2}{5}n$ as $E_\tau$
    Thus:
    \begin{equation}
        \Prob{\tau < \frac{2}{5}\cdot n}\ge 0.99
    \end{equation}
    \item We now consider the random lengths $n_{i,j}$ drawn during Period 2. Consider $m_{i,j}\sim B(n,a_{i})$. If we were to use $m_i$ as our round lengths then our risk of running out of agents would be greater than with $n_i$. We observe that: 
    \begin{equation}
        \E{}{\sum_{i=2}^{T}\sum_{j=1}^{\gamma_i} n_{i,j}}\le\E{}{\sum_{i=2}^{T}\sum_{j=1}^{\gamma_i} m_{i,j}} = n\cdot \sum_{i=2}^{T}\gamma_i \cdot a_i
    \end{equation}
    By applying Markov's inequality on the sum of these random variables, and noticing that  $\E{}{m_{i,j}} = n\cdot a_{i}$, we get:
    \begin{equation}
        \Prob{\sum_{i=2}^{T}\sum_{j=1}^{\gamma_i} n_{i,j} \ge 100\cdot n \cdot\sum_{i=2}^{T} a_{i} \cdot \gamma_i}\le\Prob{\sum_{i=2}^{T}\sum_{j=1}^{\gamma_i} m_{i,j} \ge 100\cdot n \cdot\sum_{i=2}^{T} a_{i} \cdot \gamma_i}\le 0.01
    \end{equation}
    Thus:
    \begin{equation}
        \begin{split}
            \Prob{\sum_{i=2}^{T}\sum_{j=1}^{\gamma_i} n_{i,j} \ge 100\cdot n \cdot\sum_{i=2}^{T} a_{i} \cdot\gamma_i}\le 0.01 
        \end{split}
    \end{equation}
    By lemma \ref{lemma:SeqBound} we get  $\sum_{i=2}^{T}a_i \cdot \gamma_i < \frac{1}{500}$, and thus:
    \begin{equation}
        \begin{split}
            100\cdot n \cdot\sum_{i=2}^{T} a_{i} \cdot \gamma_i < \frac{n}{5}
        \end{split}
    \end{equation}
    Overall:
    \begin{equation}
        \Prob{\sum_{i=2}^{T}\sum_{j=1}^{\gamma_i} n_{i,j} < \frac{n}{5}}\ge 0.99 
    \end{equation}
    \item Finally, we draw $2\cdot\ell$ lengths $n_i\sim B(N_i,a)$ for the rounds of BinarySearch and Exploitation with the same length parameter $a$. Suppose, similarly with before, the random variables $m_i \sim B(n,a)$ for $i\in[2\,\ell]$. It is evident that 
    \begin{equation}
        \E{}{\sum_{i=1}^{2\,\ell}n_i}\le\E{}{\sum_{i=1}^{2\,\ell}m_i} =  n\cdot 2\,\ell \cdot a = \frac{n}{3}.
    \end{equation} 
    By applying now Chernoff bound on $m_i$ we get that:
        \begin{equation}
        \begin{split}
                        \Prob{\sum_{i=1}^{2\,\ell} n_{i} \ge \frac{2}{5}\cdot n}\le\Prob{\sum_{i=1}^{2\,\ell} m_{i} \ge \frac{2}{5}\cdot n}\le 0.01. 
        \end{split}
    \end{equation}
    Overall we get:
            \begin{equation}
        \begin{split}
\Prob{\sum_{i=1}^{2\,\ell} n_{i} < \frac{2}{5}\cdot n}\ge 0.99
        \end{split}
    \end{equation}
\end{enumerate}
All in all, by Union Bound on the events described before, we get that we will not run out of agents during our Mechanism with probability at least $1 - 0.01-0.01-0.01 = 0.97$ 
\end{proof}

%We are now ready to present a complete proof for lemma \ref{lemma:GoodEvent}
%\begin{proof}[Proof of lemma \ref{lemma:GoodEvent}]
%    The bound on event $\mathcal{E}_2$ and $\mathcal{E}_3$ are an immidiate consequence of lemmas \ref{lemma:E2} and \ref{lemma:BinaryProb}. The bound on the event $\mathcal{E}_4$ is obtained from lemma \ref{lemma:Fit}. We then conclude the proof as presented in the main text.
%\end{proof}



\section{The Proof of Lemma~\ref{BUDGET}}\label{budgetproofs}

Below we prove that all needed rounds can be conducted without fear of running out of budget.
\begin{proof}[Proof of Lemma \ref{BUDGET}:]
According to Property \ref{property:a} during Period 2 in worst case the overall budget expended is the following:
\begin{equation}
    \begin{split}
        \sum_{i=2}^{T}  \gamma_i\cdot3\cdot C\cdot a_i\cdot B &\le\sum_{i=2}^{T}\frac{3\cdot C}{i^{10}}\cdot B < 0.04\cdot B ,
    \end{split}
\end{equation}
where the second to last inequality follows from Lemma~\ref{lemma:SeqBound}.

According to Property \ref{property:a}, during Period 3 and 4, in worst case the overall budget expended is the following:
\begin{equation}
    \begin{split}
        \sum_{i=1}^{2\cdot\ell} 3\cdot C\cdot a\cdot B         &\le 6\cdot C\cdot \ell \cdot a\cdot B \\
        &= C\cdot B\\
        &\le 0.06\cdot B
    \end{split}
\end{equation}    
Overall we have expended a budget of at most $B/10$.
\end{proof}

\end{document}
