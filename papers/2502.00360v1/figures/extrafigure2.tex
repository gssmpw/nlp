\begin{figure*}
    \centering
    \begin{minipage}[!t]{0.44\textwidth}
        \includegraphics[width=\textwidth]{figures/experiments/abl_sdf.pdf}
        % \setlength{\abovecaptionskip}{2pt} 
        \caption{\textbf{Ablation Study of Reconstruction Losses.} To make the geometric differences more obvious, the viewing angles of meshes are shifted slightly upwards compared to the reference image.}
        \label{fig:abl_sdf}
    \end{minipage}
    \hfill
    \begin{minipage}[!t]{0.44\textwidth}
        \includegraphics[width=\textwidth]{figures/lim.pdf}
        % \setlength{\abovecaptionskip}{2pt} 
        \caption{\textbf{Limitations of Our Results.} Under specific semantics and views, while our generated results can achieve semantic alignment through visual misalignment, they lack geometric consistency.}
        \label{fig:limitation}
    \end{minipage}
\end{figure*}

\begin{figure*}[t] 
  \begin{minipage}[!t]{0.49\textwidth}
    \centering
    \includegraphics[width=0.9\textwidth]{figures/fab_results.pdf}
    \caption{\textbf{Fabrication Results.} Results show that the manufactured outcomes are nearly identical to the simulations, delivering eye-catching visual effects. 
    % For the three-semantics generation results, we additionally print white models to show the geometric features.
    }
    \label{fig:fab}
  \end{minipage}
  \hfill
  \begin{minipage}[!t]{0.49\textwidth}
    \centering
    \includegraphics[width=\textwidth]{figures/WireArt.pdf}
    \caption{\textbf{Comparison with Wire Art \cite{Fabricable3DWireArt}.} We use the same semantics for comparison. The top-row result highlights the limitations of Wire Art, which arise from its dependence on projections to convey information. All results demonstrate that our model can capture perceptual 3D characteristics while delivering high levels of creativity, visual appeal.}
    \label{fig:wireArt}
    % \includegraphics[width=\textwidth]{figures/experiments/presentation.pdf}
    % \caption{\textbf{The generation results of different representations.} The input semantics is ``a flower''\&``a butterfly''. We compare with Dreamfusion~\cite{stable-dreamfusion} using NeRF and Fantasia3D~\cite{chen2023fantasia3d} using \textsc{DMTet}~\cite{shen2021dmtet}, adding multiple-view SDS guidance to their frameworks. Results show that our 3DGS geometry is more stable and detailed.}
    %  \label{fig:presentation}
  \end{minipage}
 
  % \begin{minipage}[!t]{0.49\textwidth}
  % %   \centering
  %    % \raggedleft
  %   \centering
  %   \includegraphics[width=0.45\textwidth]{figures/creat.pdf}
  %   % \setlength{\abovecaptionskip}{2pt} 
  %   \caption{\textbf{Todo.}}
  %   \label{fig:creativity}
  % \end{minipage}
  % \hfill
  \raggedleft
  \begin{minipage}[!t]{0.49\textwidth}
    \vspace{0.5cm} 
    \includegraphics[width=\textwidth]{figures/ShadowArt.pdf}
    \caption{\textbf{Comparison with Similar Works.} We compare with Shadow Art \cite{MP09}, Wire Art \cite{Fabricable3DWireArt}. Considering that Shadow Art only accepts binary images as input, the inputs for Wire Art during comparison are RGB images restored in \ref{3_2}, while the inputs for Shadow Art are their masks. The input views are [0,0], [0,90], [-90,0]. The results illustrate that our models effectively integrate multiple semantic elements, presenting the information in a manner that is more readily perceivable to observers. }
    \label{fig:ShadowArt}
  \end{minipage}
\end{figure*}
