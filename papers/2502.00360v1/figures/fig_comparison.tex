\begin{figure}[ht]
    % \raggedleft
    \centering
    \includegraphics[width=0.45\textwidth]{figures/ShadowArt.pdf}
    % \setlength{\abovecaptionskip}{2pt} 
    \caption{\textbf{Comparison with different multi-view inverse design problems.} We compare with shadow art \cite{MP09}, wire art \cite{Fabricable3DWireArt}. Considering that the input for shadow art is a given image, the inputs for shadow art and wire art during comparison are respectively the masks and RGB images derived from the main-view restored in \ref{3_2}. The input views are [0,0], [0,90], [-90,0]. The results illustrate that our models effectively integrate multiple semantic elements, presenting the information in a manner that is more readily perceivable to observers. }
    \label{fig:ShadowArt}
\end{figure}

\begin{figure}[ht]
    % \raggedleft
    \centering
    \includegraphics[width=0.45\textwidth]{figures/WireArt.pdf}
    % \setlength{\abovecaptionskip}{2pt} 
    \caption{\textbf{Comparison with wire art \cite{Fabricable3DWireArt}.} We input semantics for comparison. The first-line result highlights the angular limitations of Wire Art, which arise from its dependence on shadows to convey information. All results demonstrate that our model not only captures perceptual 3D characteristics but also offers high levels of creativity, visual appeal, and engagement.}
    \label{fig:wireArt}
\end{figure}