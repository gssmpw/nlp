%%%%%%%% ICML 2025 EXAMPLE LATEX SUBMISSION FILE %%%%%%%%%%%%%%%%%

\documentclass{article}

% Recommended, but optional, packages for figures and better typesetting:
\usepackage{microtype}
\usepackage{graphicx}
\usepackage{subfigure}
\usepackage{booktabs} % for professional tables

% hyperref makes hyperlinks in the resulting PDF.
% If your build breaks (sometimes temporarily if a hyperlink spans a page)
% please comment out the following usepackage line and replace
% \usepackage{icml2025} with \usepackage[nohyperref]{icml2025} above.
\usepackage{hyperref}


% Attempt to make hyperref and algorithmic work together better:
\newcommand{\theHalgorithm}{\arabic{algorithm}}

% Use the following line for the initial blind version submitted for review:
% \usepackage{icml2025}

% If accepted, instead use the following line for the camera-ready submission:
\usepackage[preprint]{icml2025}

% For theorems and such
\usepackage{amsmath}
\usepackage{amssymb}
\usepackage{mathtools}
\usepackage{amsthm}

% if you use cleveref..
\usepackage[capitalize,noabbrev]{cleveref}

%%%%%%%%%%%%%%%%%%%%%%%%%%%%%%%%
% THEOREMS
%%%%%%%%%%%%%%%%%%%%%%%%%%%%%%%%
\theoremstyle{plain}
\newtheorem{theorem}{Theorem}[section]
\newtheorem{proposition}[theorem]{Proposition}
\newtheorem{lemma}[theorem]{Lemma}
\newtheorem{corollary}[theorem]{Corollary}
\theoremstyle{definition}
\newtheorem{definition}[theorem]{Definition}
\newtheorem{assumption}[theorem]{Assumption}
\theoremstyle{remark}
\newtheorem{remark}[theorem]{Remark}

% Todonotes is useful during development; simply uncomment the next line
%    and comment out the line below the next line to turn off comments
%\usepackage[disable,textsize=tiny]{todonotes}
\usepackage[textsize=tiny]{todonotes}


% --- disable by uncommenting  
% \renewcommand{\TODO}[1]{}
% \renewcommand{\todo}[1]{#1}
%
%
%
%
%
%%%%%%%%%%%%%%%%%%%%%%%% CUSTOM IMPORTS START %%%%%%%%%%%%%%%%%%%%%%%%
% \usepackage{amsmath} % For \mathcal
% \usepackage{amssymb} % For \triangleq
\usepackage{multirow}
\usepackage{makecell} % For multirowcell
% \usepackage[pdftex]{graphicx} % For \includegraphics
% \usepackage{subcaption} % For subfigure inside figure
\usepackage{bm} % for bold variable names
\usepackage{stmaryrd} % for integer division // (\sslash)
\usepackage{dsfont} % for \mathds{1}
\usepackage{pifont} % for checkmark/crossmark
\usepackage{ifthen} % for ifthenelse in \expect
\usepackage{booktabs} % professional-quality tables
% \usepackage{dblfloatfix} % to put the teaser figure in the bottom
\usepackage{soul}
\usepackage{wrapfig} % For half-page sized figures/tables 
\usepackage{colortbl} % To color table borders
\usepackage{xspace}
% <= For algorithms start =>
% \usepackage[ruled,vlined]{algorithm2e}
% \newcommand\mycommfont[1]{\footnotesize\textcolor{blue}{#1}}
% \SetCommentSty{mycommfont}
% <= For algorithms end =>
% \usepackage[accsupp]{axessibility} % Improves PDF readability for those with disabilities.

% To input the code
\usepackage{listings}
\usepackage{xcolor}

\definecolor{codegreen}{rgb}{0,0.6,0}
\definecolor{codegray}{rgb}{0.5,0.5,0.5}
\definecolor{codepurple}{rgb}{0.58,0,0.82}
\definecolor{backcolour}{rgb}{0.95,0.95,0.92}

\lstdefinestyle{mystyle}{
    backgroundcolor=\color{backcolour},   
    commentstyle=\color{codegreen},
    keywordstyle=\color{magenta},
    numberstyle=\tiny\color{codegray},
    stringstyle=\color{codepurple},
    basicstyle=\ttfamily\footnotesize,
    breakatwhitespace=false,         
    breaklines=true,                 
    captionpos=b,                    
    keepspaces=true,                 
    numbers=left,                    
    numbersep=5pt,                  
    showspaces=false,                
    showstringspaces=false,
    showtabs=false,                  
    tabsize=2
}
\lstset{style=mystyle}


%%%%%%%%%%%%%%%%%%%%%%%% CUSTOM IMPORTS END %%%%%%%%%%%%%%%%%%%%%%%%
%
%
%
%%%%%%%%%%%%%%%%%%%%%%%% CUSTOM COMMANDS START %%%%%%%%%%%%%%%%%%%%%%%%
\newcommand{\red}[1]{{\color{red}#1}}
% \newcommand{\todo}[1]{{\color{red}#1}}

\definecolor{mediumtealblue}{rgb}{0.0, 0.33, 0.71}
\definecolor{darkpastelgreen}{rgb}{0.01, 0.75, 0.24}
\definecolor{azure}{rgb}{0.0, 0.5, 1.0}

\definecolor{crimsonred}{rgb}{0.86, 0.08, 0.24}
\definecolor{firebrick}{rgb}{0.7, 0.13, 0.13}
\definecolor{carmine}{rgb}{0.59, 0.0, 0.09}
\definecolor{rosewood}{rgb}{0.4, 0.0, 0.04}
\definecolor{deepcherry}{rgb}{0.6, 0.13, 0.13}


% Let's define shortcuts for our G, D and F models
\newcommand{\A}{\mathsf{A}}
\newcommand{\C}{\mathsf{C}}
\newcommand{\D}{\mathsf{D}}
\newcommand{\E}{\mathsf{E}}
\newcommand{\F}{\mathsf{F}}
\newcommand{\G}{\mathsf{G}}
\newcommand{\M}{\mathsf{M}}
\newcommand{\Ss}{\mathsf{S}}
\newcommand{\T}{\mathsf{T}}

\newcommand{\W}{\mathcal{W}}
\newcommand{\Z}{\mathcal{Z}}

\newcommand{\R}{\mathbb{R}}
\newcommand{\N}{\mathbb{N}}

\newcommand{\x}{\bm{x}}
\newcommand{\z}{\bm{z}}
\newcommand{\ctx}{\text{ctx}}
\newcommand{\act}{\bm{a}}
\newcommand{\coords}{\bm{c}}
\newcommand{\coordsbase}{\coords_\text{base}}
\newcommand{\scale}{s}
\newcommand{\scalef}{s_f}
\newcommand{\scaleh}{s_h}
\newcommand{\scalew}{s_w}
\newcommand{\scales}{\bm{s}}
\newcommand{\offsets}{\bm{\delta}}
\newcommand{\offsetf}{\delta_f}
\newcommand{\offseth}{\delta_h}
\newcommand{\offsetw}{\delta_w}
\newcommand{\pe}{\text{PE}}
\newcommand{\fuse}{\texttt{fuse}}
\newcommand{\fvdmini}{FVD@512}
\newcommand{\megapatch}{\bm{P}}
\newcommand{\megasigma}{\bm{\sigma}}

\newcommand{\Normal}{\mathcal{N}}

\newcommand{\ours}{\textcolor{azure}{(ours)}}
\newcommand{\apref}[1]{\ref{#1}}
\newcommand{\params}{\bm{\theta}}

\newcommand{\fix}{\marginpar{FIX}}
\newcommand{\new}{\marginpar{NEW}}
% \newcommand{\todo}[1]{\textcolor{red}{#1}}
\newcommand{\modelfullname}{Hierarchical Patch Diffusion Model}
\newcommand{\modelname}{HPDM}
\newcommand{\modelnameS}{\modelname-S}
\newcommand{\modelnameM}{\modelname-M}
\newcommand{\modelnameL}{\modelname-L}
\newcommand{\modelnameTTV}{\modelname-T2V}
\newcommand{\modelnameTTVK}{\modelname-T2V-1K}
\newcommand{\SnapVideo}{SnapVideo}
\newcommand{\parallelwork}{(concurrent work)}
\newcommand{\best}[1]{\textcolor{azure}{#1}}
\newcommand{\secondbest}[1]{\textcolor{azure}{#1}}
\newcommand{\gridsample}{\texttt{grid\_sample}_\text{3D}}
\newcommand{\concat}{\texttt{concat}_\text{3D}}
\newcommand{\noise}{\bm{\varepsilon}}
\newcommand{\patch}{\bm{p}}
\newcommand{\patchparams}{\bm{\phi}_{\coords}}
\newcommand{\patchrec}{\hat{\bm{x}}_p}

\newcommand{\cmark}{\ding{51}}
\newcommand{\xmark}{\ding{55}}

\newcommand{\figref}[1]{Fig.~\ref{#1}}
\newcommand{\secref}[1]{\S\ref{#1}}
\newcommand{\tabref}[1]{Tab.~\ref{#1}}
\newcommand{\inlinesection}[1]{\noindent{\textbf{#1}}}

\newcommand{\green}[1]{\textcolor{green}{{#1}}}
\newcommand{\sergey}[1]{{\color{blue}{S:#1}}}

% \newcommand{\codeurl}{https://example.com}
\newcommand{\projecturl}{https://snap-research.github.io/hpdm}
\newcommand{\projecthref}{\href{\projecturl}{\projecturl}}
% \newcommand{\projecturlextra}{https://u2wjb9xxz9q.github.io/additional-results}
% \newcommand{\projecthrefextra}{\href{\projecturlextra}{\projecturlextra}}
% \newtheorem*{simplestatement*}{A trivial but serviceable statement}

\newcommand{\checkmarknew}{\ding{51}} 
\newcommand{\crossmark}{\ding{55}}

% \newcommand{\regname}{Scale Equivariance\xspace}
\newcommand{\Regname}{Scale Equivariance\xspace}
\newcommand{\regname}{scale equivariance\xspace}
\newcommand{\regshortname}{SE\xspace}

\newcommand{\regchfname}{Chopping High Frequencies\xspace}
\newcommand{\regchfshortname}{CHF\xspace}

\newcommand{\rdag}{$^{\textcolor{red}{\dag}}$}
\newcommand{\Diffusability}{\textit{Diffusability}\xspace}
\newcommand{\diffusability}{\textit{diffusability}\xspace}

\newcommand{\expect}[2][]{
\ifthenelse{\equal{#1}{}}{
\mathbb{E}\left[#2\right]
}{
\underset{#1}{\mathbb{E}}\left[#2\right]
}}
\newcommand{\updated}[1]{{\color{red}{#1}}}
\newcommand{\newtable}[0]{
\arrayrulecolor{blue}
\color{blue}
}

\DeclareMathOperator{\zigzag}{zigzag}
\DeclareMathOperator{\Enc}{Enc}
\DeclareMathOperator{\Dec}{Dec}


\newcommand{\cellbest}{\cellcolor{azure!35}}
\newcommand{\cellsecond}{\cellcolor{azure!10}}
% \newcommand{\cellthird}{\cellcolor{Yellow!25}}

% \newcommand{\fid}{$\text{FID}_\text{50K}$\xspace}
\newcommand{\fid}{FID\xspace}
\newcommand{\fvd}{$\text{FVD}_\text{10K}$\xspace}
\newcommand{\fvdfull}{$\text{FVD}_\text{50K}$\xspace}
\newcommand{\fidtenk}{$\text{FID}_\text{10K}$\xspace}
\newcommand{\psnrsmall}{$\text{PSNR}_\text{512}$\xspace}
% \newcommand{\dinofid}{$\text{DiNOFID}_\text{50K}$\xspace}
% \newcommand{\dinofid}{$\text{FD}_\text{DINOv2}$\xspace}
% \newcommand{\dinofidfivek}{$\text{FD}_\text{DINOv2}^\text{5k}$\xspace}
\newcommand{\dinofid}{$\text{FDD}$\xspace}
\newcommand{\dinofidfivek}{$\text{FDD}_\text{5K}$\xspace}
% \newcommand{\inceptionscore}{$\text{IS}_\text{50K}$\xspace}
\newcommand{\inceptionscore}{IS\xspace}
% \newcommand{\cmsae}{$\text{CMS-AE}_{8 \times 16\times 16}$\xspace}
% \newcommand{\cmsaei}{$\text{CMS-AE}_{16\times 16}$\xspace}
\newcommand{\cmsae}{$\text{CMS-AE}_{V}$\xspace}
\newcommand{\cmsaei}{$\text{CMS-AE}_{I}$\xspace}
\newcommand{\cvae}{CV-AE\xspace}
\newcommand{\cvaefull}{CogVideoX-AE\xspace}
\newcommand{\dcae}{DC-AE\xspace}
\newcommand{\fluxae}{FluxAE\xspace}
\newcommand{\ltxae}{LTX-AE\xspace}

\newcommand{\ditstwo}{DiT-S$/$2}
\newcommand{\ditbtwo}{DiT-B$/$2}
\newcommand{\ditbone}{DiT-B$/$2}
\newcommand{\ditltwo}{DiT-L$/$2}
\newcommand{\ditxltwo}{DiT-XL$/$2}

\newcommand{\loss}{\mathcal{L}}

% Some links
\newcommand{\fluxaediffuserslink}{\url{https://huggingface.co/black-forest-labs/FLUX.1-dev/tree/main/vae}\xspace}

% People handles
\definecolor{DarkGreen}{RGB}{1,80,52}
\definecolor{DarkRed}{RGB}{150,30,30}
\newcommand{\rameen}[1]{\textbf{\textcolor{DarkGreen}{@Rameen:#1}}}
\newcommand{\willi}[1]{\textbf{\textcolor{red}{@Willi:#1}}}
\newcommand{\yanyu}[1]{\textbf{\textcolor{blue}{@Yanyu:#1}}}
\newcommand{\alex}[1]{\textbf{\textcolor{magenta}{@Alex:#1}}}
\newcommand{\sharath}[1]{\textbf{\textcolor{purple}{@Sharath:#1}}}
\newcommand{\ivan}[1]{\textbf{\textcolor{DarkRed}{@Ivan:#1}}}

% To avoid splitting footnote onto two pages
% \interfootnotelinepenalty=10000
%%%%%%%%%%%%%%%%%%%%%%%% CUSTOM COMMANDS END %%%%%%%%%%%%%%%%%%%%%%%%

% \newcommand{\papertitle}{Making Autoencoders Diffusable Again}
\newcommand{\papertitle}{Improving the Diffusability of Autoencoders}
% \newcommand{\papertitle}{Improving Autoencoders' Diffusability}
% \newcommand{\papertitle}{Towards Diffusable Autoencoders}

% The \icmltitle you define below is probably too long as a header.
% Therefore, a short form for the running title is supplied here:
\icmltitlerunning{\papertitle}

\begin{document}

\twocolumn[

% \icmltitle{Analyzing and Improving Video Autoencoders for Latent Diffusion Models}
% \icmltitle{Better Regularizations for Video Autoencoders}
% \icmltitle{On Video AutoEncoders for Latent Diffusion Models}
% \icmltitle{Principled Regularization for Video AutoEncoders}
% \icmltitle{Diffusable Video AutoEncoders}
% \icmltitle{Video AutoEncoders with Diffusable Latents}
\icmltitle{\papertitle}

% It is OKAY to include author information, even for blind
% submissions: the style file will automatically remove it for you
% unless you've provided the [accepted] option to the icml2025
% package.

% List of affiliations: The first argument should be a (short)
% identifier you will use later to specify author affiliations
% Academic affiliations should list Department, University, City, Region, Country
% Industry affiliations should list Company, City, Region, Country

% You can specify symbols, otherwise they are numbered in order.
% Ideally, you should not use this facility. Affiliations will be numbered
% in order of appearance and this is the preferred way.
\icmlsetsymbol{equal}{*}

\begin{icmlauthorlist}
\icmlauthor{Ivan Skorokhodov}{snap}
\icmlauthor{Sharath Girish}{snap}
\icmlauthor{Benran Hu}{snap,cmu}
\icmlauthor{Willi Menapace}{snap}
\icmlauthor{Yanyu Li}{snap}
\icmlauthor{Rameen Abdal}{snap}
\icmlauthor{Sergey Tulyakov}{snap}
\icmlauthor{Aliaksandr Siarohin}{snap}
%\icmlauthor{}{sch}
%\icmlauthor{}{sch}
\end{icmlauthorlist}

\icmlaffiliation{snap}{Snap Inc.}
\icmlaffiliation{cmu}{Carnegie Mellon University}
% \icmlaffiliation{sch}{School of ZZZ, Institute of WWW, Location, Country}

\icmlcorrespondingauthor{Ivan Skorokhodov}{iskorokhodov@gmail.com}
\icmlcorrespondingauthor{Aliaksandr Siarohin}{aliaksandr.siarohin@gmail.com}

% You may provide any keywords that you
% find helpful for describing your paper; these are used to populate
% the "keywords" metadata in the PDF but will not be shown in the document
\icmlkeywords{Diffusion Models, Latent Diffusion, Video Generation, Image Generation, AutoEncoders, VAE, Variational AutoEncoders, ICML}

\vskip 0.3in
]

% this must go after the closing bracket ] following \twocolumn[ ...

% This command actually creates the footnote in the first column
% listing the affiliations and the copyright notice.
% The command takes one argument, which is text to display at the start of the footnote.
% The \icmlEqualContribution command is standard text for equal contribution.
% Remove it (just {}) if you do not need this facility.

\printAffiliationsAndNotice{}  % leave blank if no need to mention equal contribution
% \printAffiliationsAndNotice{\icmlEqualContribution} % otherwise use the standard text.

Large language model (LLM)-based agents have shown promise in tackling complex tasks by interacting dynamically with the environment. 
Existing work primarily focuses on behavior cloning from expert demonstrations and preference learning through exploratory trajectory sampling. However, these methods often struggle in long-horizon tasks, where suboptimal actions accumulate step by step, causing agents to deviate from correct task trajectories.
To address this, we highlight the importance of \textit{timely calibration} and the need to automatically construct calibration trajectories for training agents. We propose \textbf{S}tep-Level \textbf{T}raj\textbf{e}ctory \textbf{Ca}libration (\textbf{\model}), a novel framework for LLM agent learning. 
Specifically, \model identifies suboptimal actions through a step-level reward comparison during exploration. It constructs calibrated trajectories using LLM-driven reflection, enabling agents to learn from improved decision-making processes. These calibrated trajectories, together with successful trajectory data, are utilized for reinforced training.
Extensive experiments demonstrate that \model significantly outperforms existing methods. Further analysis highlights that step-level calibration enables agents to complete tasks with greater robustness. 
Our code and data are available at \url{https://github.com/WangHanLinHenry/STeCa}.
\section{Introduction}

Despite the remarkable capabilities of large language models (LLMs)~\cite{DBLP:conf/emnlp/QinZ0CYY23,DBLP:journals/corr/abs-2307-09288}, they often inevitably exhibit hallucinations due to incorrect or outdated knowledge embedded in their parameters~\cite{DBLP:journals/corr/abs-2309-01219, DBLP:journals/corr/abs-2302-12813, DBLP:journals/csur/JiLFYSXIBMF23}.
Given the significant time and expense required to retrain LLMs, there has been growing interest in \emph{model editing} (a.k.a., \emph{knowledge editing})~\cite{DBLP:conf/iclr/SinitsinPPPB20, DBLP:journals/corr/abs-2012-00363, DBLP:conf/acl/DaiDHSCW22, DBLP:conf/icml/MitchellLBMF22, DBLP:conf/nips/MengBAB22, DBLP:conf/iclr/MengSABB23, DBLP:conf/emnlp/YaoWT0LDC023, DBLP:conf/emnlp/ZhongWMPC23, DBLP:conf/icml/MaL0G24, DBLP:journals/corr/abs-2401-04700}, 
which aims to update the knowledge of LLMs cost-effectively.
Some existing methods of model editing achieve this by modifying model parameters, which can be generally divided into two categories~\cite{DBLP:journals/corr/abs-2308-07269, DBLP:conf/emnlp/YaoWT0LDC023}.
Specifically, one type is based on \emph{Meta-Learning}~\cite{DBLP:conf/emnlp/CaoAT21, DBLP:conf/acl/DaiDHSCW22}, while the other is based on \emph{Locate-then-Edit}~\cite{DBLP:conf/acl/DaiDHSCW22, DBLP:conf/nips/MengBAB22, DBLP:conf/iclr/MengSABB23}. This paper primarily focuses on the latter.

\begin{figure}[t]
  \centering
  \includegraphics[width=0.48\textwidth]{figures/demonstration.pdf}
  \vspace{-4mm}
  \caption{(a) Comparison of regular model editing and EAC. EAC compresses the editing information into the dimensions where the editing anchors are located. Here, we utilize the gradients generated during training and the magnitude of the updated knowledge vector to identify anchors. (b) Comparison of general downstream task performance before editing, after regular editing, and after constrained editing by EAC.}
  \vspace{-3mm}
  \label{demo}
\end{figure}

\emph{Sequential} model editing~\cite{DBLP:conf/emnlp/YaoWT0LDC023} can expedite the continual learning of LLMs where a series of consecutive edits are conducted.
This is very important in real-world scenarios because new knowledge continually appears, requiring the model to retain previous knowledge while conducting new edits. 
Some studies have experimentally revealed that in sequential editing, existing methods lead to a decrease in the general abilities of the model across downstream tasks~\cite{DBLP:journals/corr/abs-2401-04700, DBLP:conf/acl/GuptaRA24, DBLP:conf/acl/Yang0MLYC24, DBLP:conf/acl/HuC00024}. 
Besides, \citet{ma2024perturbation} have performed a theoretical analysis to elucidate the bottleneck of the general abilities during sequential editing.
However, previous work has not introduced an effective method that maintains editing performance while preserving general abilities in sequential editing.
This impacts model scalability and presents major challenges for continuous learning in LLMs.

In this paper, a statistical analysis is first conducted to help understand how the model is affected during sequential editing using two popular editing methods, including ROME~\cite{DBLP:conf/nips/MengBAB22} and MEMIT~\cite{DBLP:conf/iclr/MengSABB23}.
Matrix norms, particularly the L1 norm, have been shown to be effective indicators of matrix properties such as sparsity, stability, and conditioning, as evidenced by several theoretical works~\cite{kahan2013tutorial}. In our analysis of matrix norms, we observe significant deviations in the parameter matrix after sequential editing.
Besides, the semantic differences between the facts before and after editing are also visualized, and we find that the differences become larger as the deviation of the parameter matrix after editing increases.
Therefore, we assume that each edit during sequential editing not only updates the editing fact as expected but also unintentionally introduces non-trivial noise that can cause the edited model to deviate from its original semantics space.
Furthermore, the accumulation of non-trivial noise can amplify the negative impact on the general abilities of LLMs.

Inspired by these findings, a framework termed \textbf{E}diting \textbf{A}nchor \textbf{C}ompression (EAC) is proposed to constrain the deviation of the parameter matrix during sequential editing by reducing the norm of the update matrix at each step. 
As shown in Figure~\ref{demo}, EAC first selects a subset of dimension with a high product of gradient and magnitude values, namely editing anchors, that are considered crucial for encoding the new relation through a weighted gradient saliency map.
Retraining is then performed on the dimensions where these important editing anchors are located, effectively compressing the editing information.
By compressing information only in certain dimensions and leaving other dimensions unmodified, the deviation of the parameter matrix after editing is constrained. 
To further regulate changes in the L1 norm of the edited matrix to constrain the deviation, we incorporate a scored elastic net ~\cite{zou2005regularization} into the retraining process, optimizing the previously selected editing anchors.

To validate the effectiveness of the proposed EAC, experiments of applying EAC to \textbf{two popular editing methods} including ROME and MEMIT are conducted.
In addition, \textbf{three LLMs of varying sizes} including GPT2-XL~\cite{radford2019language}, LLaMA-3 (8B)~\cite{llama3} and LLaMA-2 (13B)~\cite{DBLP:journals/corr/abs-2307-09288} and \textbf{four representative tasks} including 
natural language inference~\cite{DBLP:conf/mlcw/DaganGM05}, 
summarization~\cite{gliwa-etal-2019-samsum},
open-domain question-answering~\cite{DBLP:journals/tacl/KwiatkowskiPRCP19},  
and sentiment analysis~\cite{DBLP:conf/emnlp/SocherPWCMNP13} are selected to extensively demonstrate the impact of model editing on the general abilities of LLMs. 
Experimental results demonstrate that in sequential editing, EAC can effectively preserve over 70\% of the general abilities of the model across downstream tasks and better retain the edited knowledge.

In summary, our contributions to this paper are three-fold:
(1) This paper statistically elucidates how deviations in the parameter matrix after editing are responsible for the decreased general abilities of the model across downstream tasks after sequential editing.
(2) A framework termed EAC is proposed, which ultimately aims to constrain the deviation of the parameter matrix after editing by compressing the editing information into editing anchors. 
(3) It is discovered that on models like GPT2-XL and LLaMA-3 (8B), EAC significantly preserves over 70\% of the general abilities across downstream tasks and retains the edited knowledge better.


\section{Related Work}
Our work draws on and contributes to research in mobility aids and the built environment, online image-based survey for urban assessment, personalized routing applications and accessibility maps.

\subsection{Mobility Aids and the Built Environment}
People who use mobility aids (\textit{e.g.,} canes, walkers, mobility scooters, manual wheelchairs and motorized wheelchairs) face significant challenges navigating their communities.
Studies have repeatedly found that sidewalk conditions can significantly impede mobility among these users~\cite{bigonnesse_role_2018,fomiatti_experience_2014,f_bromley_city_2007,rosenberg_outdoor_2013, harris_physical_2015,korotchenko_power_2014}. 
In a review of the physical environment's role in mobility, \citet{bigonnesse_role_2018} summarized factors affecting mobility aid users, including uneven or narrow sidewalks (\textit{e.g.,}~\cite{fomiatti_experience_2014,f_bromley_city_2007}), rough pavements (\textit{e.g.,}~\cite{fomiatti_experience_2014,f_bromley_city_2007}), absent or poorly designed curb ramps (\textit{e.g.,}~\cite{rosenberg_outdoor_2013, f_bromley_city_2007, korotchenko_power_2014}), lack of crosswalks (\textit{e.g.,}~\cite{harris_physical_2015}), and various temporary obstacles (\textit{e.g.,}~\cite{harris_physical_2015}).

Though most research on mobility disability and the built environment has focused on wheelchair users~\cite{bigonnesse_role_2018}, mobility challenges are not experienced uniformly across different user populations~\cite{prescott_factors_2020, bigonnesse_role_2018}. 
For example, crutch users could overcome a specific physical barrier (such as two stairs down to a street), whereas motorized wheelchair users could not (without a ramp)~\cite{bigonnesse_role_2018}. 
Such variability demonstrates how person-environment interaction can differ based on mobility aids and environmental factors~\cite{sakakibara_rasch_2018,smith_review_2016}.
Further, mobility aids such as canes, crutches, or walkers are more commonly used than wheelchairs in the U.S.~\cite{taylor_americans_2014, firestine_travel_2024}: in 2022, approximately 4.7 million adults used a cane, crutches, or a walker, compared to 1.7 million who used a wheelchair~\cite{firestine_travel_2024}.
This underscores the importance of considering a diverse range of mobility aid users in urban accessibility research.
For example, \citet{prescott_factors_2020} explored the daily path areas of users of manual wheelchairs, motorized wheelchairs, scooters, walkers, canes, and crutches and found that the type of mobility device had a strong association with users' daily path area size.
Our study aims to further advance knowledge of how different mobility aid users perceive sidewalk barriers, with a more inclusive understanding of urban accessibility.

\begin{figure*}
    \centering
    \includegraphics[width=1\linewidth]{figures/figure-tutorial.png}
    \caption{Survey Part 2.1 showed all 52 images and asked participants to rate their passability based on their lived experience and use of their mobility aid. Above is the interactive tutorial we showed at the beginning of this part.}
    \Description{This figure shows a screenshot from the online survey. In survey part 2.1, participants were presented with 52 images and were asked to rate their passibility based on their lived experience and use of their mobility aid. The screenshot shows the interactive tutorial shown before this section.}
    \label{fig:survey-part2-instructions}
\end{figure*}

\subsection{Online Image-Based Survey for Urban Assessment}
Sidewalk barriers hinder individuals with mobility impairments not just by preventing particular travel paths but also by reducing confidence in self-navigating and decreasing one's willingness to travel to areas that might be physically challenging or unsafe~\cite{vasudevan_exploration_2016,clarke_mobility_2008}.
Prior work in this area traditionally uses three main study methods: in-person interviews (\textit{e.g}.~\cite{rosenberg_outdoor_2013,castrodale_mobilizing_2018}), GPS-based activity studies (\textit{e.g.,}~\cite{prescott_exploration_2021, prescott_factors_2020,rosenberg_outdoor_2013}), and online-questionnaires (\textit{e.g.,}~\cite{carlson_wheelchair_2002}). 
In-person interviews, while providing detailed and nuanced information, are limited by small sample sizes~\cite{rosenberg_outdoor_2013}. GPS-based activity studies involve tracking mobility aids user activity over a period of time, offering insights into movement patterns and activity space; however, these studies are constrained by geographical location~\cite{prescott_exploration_2021}. In contrast, online questionnaires can reach much larger populations and cover broader geographical regions, but they often yield high-level information that lacks the depth and nuance of the other approaches~\cite{carlson_wheelchair_2002}.
Our study aims to strike a balance between these approaches, capturing nuanced perspectives of mobility aid users about the built environment while maintaining a sufficiently large enough sample size for robust statistical analysis. 
Building on~\citet{bigonnesse_role_2018}'s work, we explore not only the types of factors considered to be barriers, but the \textit{intensity} of these barriers and their differential impacts.

Visual assessment of environmental features has long been employed by researchers across diverse fields, including human well-being~\cite{humpel_environmental_2002}, ecosystem sustainability~\cite{gobster_shared_2007}, and public policy~\cite{dobbie_public_2013}. 
These studies examine the relationship between images and the reactions they provoke in respondents or compare differences in reactions between groups.
Over the past decade, online visual preference surveys have gained popularity (\textit{e.g.,}~\cite{evans-cowley_streetseen_2014, salesses_collaborative_2013, goodspeed_research_2017}), where respondents are asked to make pairwise comparisons between randomly selected images.
Using this approach has two advantages: it adheres to the law of comparative judgment~\cite{thurstone_law_2017} by allowing respondents to make direct comparisons, and it prevents inter-rater inconsistency possible with scale ratings~\cite{goodspeed_research_2017}.
Additionally, online surveys generally offer advantages of increased sample sizes, reduced costs, and greater flexibility~\cite{wherrett_issues_1999}.
For people with disabilities, online surveys can be particularly beneficial. They help reach hidden or difficult-to-access populations~\cite{cook_challenges_2007,wright_researching_2005} and are believed to encourage more honest answers to sensitive questions~\cite{eckhardt_research_2007} by providing a higher level of anonymity and confidentiality~\cite{cook_challenges_2007, wright_researching_2005}.

\begin{figure*}
    \centering
    \includegraphics[width=1\linewidth]{figures/figure-comaprison-screenshot.png}
    \caption{In survey Part 2.2, participants were asked to perform a series of pairwise comparisons based on their 2.1 responses.}
    \Description{This figure shows a screenshot from the online survey. In Survey Part 2.2, participants were asked to perform a series of pairwise comparisons based on their 2.1 responses.}
    \label{fig:survey-part2b-pairwise}
\end{figure*}

\subsection{Personalized Routing Applications and Accessibility Maps}
Navigation challenges faced by mobility aid users can be mitigated through the provision of routes and directions that guide them to destinations safely, accurately, and efficiently~\cite{kasemsuppakorn_understanding_2015}. However, current commercial routing applications (\textit{e.g.}, \textit{Google Maps}) do not provide sufficient guidance for mobility aid users.
To address this gap, significant research has focused on routing systems for this population over the past two decades~\cite{barczyszyn_collaborative_2018, karimanzira_application_2006, matthews_modelling_2003, kasemsuppakorn_understanding_2015, volkel_routecheckr_2008, holone_people_2008, wheeler_personalized_2020, gharebaghi_user-specific_2021, ding_design_2007}.
One early, well-known prototype system is \textit{MAGUS}~\cite{matthews_modelling_2003}, which computes optimal routes for wheelchair users based on shortest distance, minimum barriers, fewest slopes, and limits on road crossings and challenging surfaces.
\textit{U-Access}~\cite{sobek_u-access_2006} provides the shortest route for people with three accessibility levels: unaided mobility, aided mobility (using crutch, cane, or walker), and wheelchair users.
However, U-Access only considers distance and ignores other
important factors for mobility aid users~\cite{barczyszyn_collaborative_2018}.
A series of projects by Kasemsuppakorn \textit{et al}.~\cite{kasemsuppakorn_personalised_2009, kasemsuppakorn_understanding_2015} attempted to create personalized routes for wheelchair users using fuzzy logic and \textit{Analytic Hierarchy Process} (AHP).

While influential, many personalized routing prototypes face limited adoption due to a scarcity of accessibility data for the built environment. 
Geo-crowdsourcing~\cite{karimi_personalized_2014}, a.k.a. volunteered geographic information (VGI)~\cite{goodchild_citizens_2007}, has emerged as an effective solution~\cite{karimi_personalized_2014, wheeler_personalized_2020}.
In this approach, users annotate maps with specific criteria or share personal experiences of locations, typically using web applications based on Google Maps or \textit{OpenStreetMap} (OSM)~\cite{karimi_personalized_2014}.
Examples include \textit{Wheelmap}~\cite{mobasheri_wheelmap_2017}, \textit{CAP4Access}~\cite{cap4access_cap4access_2014}, \textit{AXS Map}~\cite{axs_map_axs_2012}, and \textit{Project Sidewalk}~\cite{saha_project_2019}.
Recent research demonstrated the potential of using crowdsourced geodata for personalized routing~\cite{goldberg_interactive_2016, bolten_accessmap_2019,menkens_easywheel_2011, neis_measuring_2015}.
For example, \textit{EasyWheel}~\cite{menkens_easywheel_2011}, a mobile social navigation system based on OSM, provides wheelchair users with optimized routing, accessibility information for points of interest, and a social community for reporting barriers. 
\textit{AccessMap}~\cite{bolten_accessmap_2019} offers routing information tailored to users of canes, manual wheelchairs, or powered wheelchairs, calculating routes based on OSM data that includes slope, curbs, stairs and landmarks. 
Our work builds on the above by gathering perceptions of sidewalk obstacles from different mobility aid users to create generalizable profiles based on mobility aid type. We envision that these profiles can provide starting points in tools like Google Maps for personalized routing but can be further customized by the end user to specify additional needs (\textit{e.g.}, ability to navigate hills, \textit{etc.})

Beyond routing applications, our study data can contribute to modeling and visualizing higher-level abstractions of accessibility. 
Similar to \textit{AccessScore}~\cite{li_interactively_2018}, data from our survey can provide personalizable and interactive visual analytics of city-wide accessibility. By identifying both differences between mobility groups and common barriers within groups, we can develop analytical tools to prioritize barriers and assess the impact of their mitigation or removal, potentially benefiting the broadest range of mobility group users. Incorporating perceptions of passibility into urban planning processes provides a new dimension for urban planners' toolkits, which are often narrowly focused on compliance with ADA standards.




\vspace{-5pt}
\section{Method}
\label{sec:method}
\section{Overview}

\revision{In this section, we first explain the foundational concept of Hausdorff distance-based penetration depth algorithms, which are essential for understanding our method (Sec.~\ref{sec:preliminary}).
We then provide a brief overview of our proposed RT-based penetration depth algorithm (Sec.~\ref{subsec:algo_overview}).}



\section{Preliminaries }
\label{sec:Preliminaries}

% Before we introduce our method, we first overview the important basics of 3D dynamic human modeling with Gaussian splatting. Then, we discuss the diffusion-based 3d generation techniques, and how they can be applied to human modeling.
% \ZY{I stopp here. TBC.}
% \subsection{Dynamic human modeling with Gaussian splatting}
\subsection{3D Gaussian Splatting}
3D Gaussian splatting~\cite{kerbl3Dgaussians} is an explicit scene representation that allows high-quality real-time rendering. The given scene is represented by a set of static 3D Gaussians, which are parameterized as follows: Gaussian center $x\in {\mathbb{R}^3}$, color $c\in {\mathbb{R}^3}$, opacity $\alpha\in {\mathbb{R}}$, spatial rotation in the form of quaternion $q\in {\mathbb{R}^4}$, and scaling factor $s\in {\mathbb{R}^3}$. Given these properties, the rendering process is represented as:
\begin{equation}
  I = Splatting(x, c, s, \alpha, q, r),
  \label{eq:splattingGA}
\end{equation}
where $I$ is the rendered image, $r$ is a set of query rays crossing the scene, and $Splatting(\cdot)$ is a differentiable rendering process. We refer readers to Kerbl et al.'s paper~\cite{kerbl3Dgaussians} for the details of Gaussian splatting. 



% \ZY{I would suggest move this part to the method part.}
% GaissianAvatar is a dynamic human generation model based on Gaussian splitting. Given a sequence of RGB images, this method utilizes fitted SMPLs and sampled points on its surface to obtain a pose-dependent feature map by a pose encoder. The pose-dependent features and a geometry feature are fed in a Gaussian decoder, which is employed to establish a functional mapping from the underlying geometry of the human form to diverse attributes of 3D Gaussians on the canonical surfaces. The parameter prediction process is articulated as follows:
% \begin{equation}
%   (\Delta x,c,s)=G_{\theta}(S+P),
%   \label{eq:gaussiandecoder}
% \end{equation}
%  where $G_{\theta}$ represents the Gaussian decoder, and $(S+P)$ is the multiplication of geometry feature S and pose feature P. Instead of optimizing all attributes of Gaussian, this decoder predicts 3D positional offset $\Delta{x} \in {\mathbb{R}^3}$, color $c\in\mathbb{R}^3$, and 3D scaling factor $ s\in\mathbb{R}^3$. To enhance geometry reconstruction accuracy, the opacity $\alpha$ and 3D rotation $q$ are set to fixed values of $1$ and $(1,0,0,0)$ respectively.
 
%  To render the canonical avatar in observation space, we seamlessly combine the Linear Blend Skinning function with the Gaussian Splatting~\cite{kerbl3Dgaussians} rendering process: 
% \begin{equation}
%   I_{\theta}=Splatting(x_o,Q,d),
%   \label{eq:splatting}
% \end{equation}
% \begin{equation}
%   x_o = T_{lbs}(x_c,p,w),
%   \label{eq:LBS}
% \end{equation}
% where $I_{\theta}$ represents the final rendered image, and the canonical Gaussian position $x_c$ is the sum of the initial position $x$ and the predicted offset $\Delta x$. The LBS function $T_{lbs}$ applies the SMPL skeleton pose $p$ and blending weights $w$ to deform $x_c$ into observation space as $x_o$. $Q$ denotes the remaining attributes of the Gaussians. With the rendering process, they can now reposition these canonical 3D Gaussians into the observation space.



\subsection{Score Distillation Sampling}
Score Distillation Sampling (SDS)~\cite{poole2022dreamfusion} builds a bridge between diffusion models and 3D representations. In SDS, the noised input is denoised in one time-step, and the difference between added noise and predicted noise is considered SDS loss, expressed as:

% \begin{equation}
%   \mathcal{L}_{SDS}(I_{\Phi}) \triangleq E_{t,\epsilon}[w(t)(\epsilon_{\phi}(z_t,y,t)-\epsilon)\frac{\partial I_{\Phi}}{\partial\Phi}],
%   \label{eq:SDSObserv}
% \end{equation}
\begin{equation}
    \mathcal{L}_{\text{SDS}}(I_{\Phi}) \triangleq \mathbb{E}_{t,\epsilon} \left[ w(t) \left( \epsilon_{\phi}(z_t, y, t) - \epsilon \right) \frac{\partial I_{\Phi}}{\partial \Phi} \right],
  \label{eq:SDSObservGA}
\end{equation}
where the input $I_{\Phi}$ represents a rendered image from a 3D representation, such as 3D Gaussians, with optimizable parameters $\Phi$. $\epsilon_{\phi}$ corresponds to the predicted noise of diffusion networks, which is produced by incorporating the noise image $z_t$ as input and conditioning it with a text or image $y$ at timestep $t$. The noise image $z_t$ is derived by introducing noise $\epsilon$ into $I_{\Phi}$ at timestep $t$. The loss is weighted by the diffusion scheduler $w(t)$. 
% \vspace{-3mm}

\subsection{Overview of the RTPD Algorithm}\label{subsec:algo_overview}
Fig.~\ref{fig:Overview} presents an overview of our RTPD algorithm.
It is grounded in the Hausdorff distance-based penetration depth calculation method (Sec.~\ref{sec:preliminary}).
%, similar to that of Tang et al.~\shortcite{SIG09HIST}.
The process consists of two primary phases: penetration surface extraction and Hausdorff distance calculation.
We leverage the RTX platform's capabilities to accelerate both of these steps.

\begin{figure*}[t]
    \centering
    \includegraphics[width=0.8\textwidth]{Image/overview.pdf}
    \caption{The overview of RT-based penetration depth calculation algorithm overview}
    \label{fig:Overview}
\end{figure*}

The penetration surface extraction phase focuses on identifying the overlapped region between two objects.
\revision{The penetration surface is defined as a set of polygons from one object, where at least one of its vertices lies within the other object. 
Note that in our work, we focus on triangles rather than general polygons, as they are processed most efficiently on the RTX platform.}
To facilitate this extraction, we introduce a ray-tracing-based \revision{Point-in-Polyhedron} test (RT-PIP), significantly accelerated through the use of RT cores (Sec.~\ref{sec:RT-PIP}).
This test capitalizes on the ray-surface intersection capabilities of the RTX platform.
%
Initially, a Geometry Acceleration Structure (GAS) is generated for each object, as required by the RTX platform.
The RT-PIP module takes the GAS of one object (e.g., $GAS_{A}$) and the point set of the other object (e.g., $P_{B}$).
It outputs a set of points (e.g., $P_{\partial B}$) representing the penetration region, indicating their location inside the opposing object.
Subsequently, a penetration surface (e.g., $\partial B$) is constructed using this point set (e.g., $P_{\partial B}$) (Sec.~\ref{subsec:surfaceGen}).
%
The generated penetration surfaces (e.g., $\partial A$ and $\partial B$) are then forwarded to the next step. 

The Hausdorff distance calculation phase utilizes the ray-surface intersection test of the RTX platform (Sec.~\ref{sec:RT-Hausdorff}) to compute the Hausdorff distance between two objects.
We introduce a novel Ray-Tracing-based Hausdorff DISTance algorithm, RT-HDIST.
It begins by generating GAS for the two penetration surfaces, $P_{\partial A}$ and $P_{\partial B}$, derived from the preceding step.
RT-HDIST processes the GAS of a penetration surface (e.g., $GAS_{\partial A}$) alongside the point set of the other penetration surface (e.g., $P_{\partial B}$) to compute the penetration depth between them.
The algorithm operates bidirectionally, considering both directions ($\partial A \to \partial B$ and $\partial B \to \partial A$).
The final penetration depth between the two objects, A and B, is determined by selecting the larger value from these two directional computations.

%In the Hausdorff distance calculation step, we compute the Hausdorff distance between given two objects using a ray-surface-intersection test. (Sec.~\ref{sec:RT-Hausdorff}) Initially, we construct the GAS for both $\partial A$ and $\partial B$ to utilize the RT-core effectively. The RT-based Hausdorff distance algorithms then determine the Hausdorff distance by processing the GAS of one object (e.g. $GAS_{\partial A}$) and set of the vertices of the other (e.g. $P_{\partial B}$). Following the Hausdorff distance definition (Eq.~\ref{equation:hausdorff_definition}), we compute the Hausdorff distance to both directions ($\partial A \to \partial B$) and ($\partial B \to \partial A$). As a result, the bigger one is the final Hausdorff distance, and also it is the penetration depth between input object $A$ and $B$.


%the proposed RT-based penetration depth calculation pipeline.
%Our proposed methods adopt Tang's Hausdorff-based penetration depth methods~\cite{SIG09HIST}. The pipeline is divided into the penetration surface extraction step and the Hausdorff distance calculation between the penetration surface steps. However, since Tang's approach is not suitable for the RT platform in detail, we modified and applied it with appropriate methods.

%The penetration surface extraction step is extracting overlapped surfaces on other objects. To utilize the RT core, we use the ray-intersection-based PIP(Point-In-Polygon) algorithms instead of collision detection between two objects which Tang et al.~\cite{SIG09HIST} used. (Sec.~\ref{sec:RT-PIP})
%RT core-based PIP test uses a ray-surface intersection test. For purpose this, we generate the GAS(Geometry Acceleration Structure) for each object. RT core-based PIP test takes the GAS of one object (e.g. $GAS_{A}$) and a set of vertex of another one (e.g. $P_{B}$). Then this computes the penetrated vertex set of another one (e.g. $P_{\partial B}$). To calculate the Hausdorff distance, these vertex sets change to objects constructed by penetrated surface (e.g. $\partial B$). Finally, the two generated overlapped surface objects $\partial A$ and $\partial B$ are used in the Hausdorff distance calculation step.

Our goal is to increase the robustness of T2I models, particularly with rare or unseen concepts, which they struggle to generate. To do so, we investigate a retrieval-augmented generation approach, through which we dynamically select images that can provide the model with missing visual cues. Importantly, we focus on models that were not trained for RAG, and show that existing image conditioning tools can be leveraged to support RAG post-hoc.
As depicted in \cref{fig:overview}, given a text prompt and a T2I generative model, we start by generating an image with the given prompt. Then, we query a VLM with the image, and ask it to decide if the image matches the prompt. If it does not, we aim to retrieve images representing the concepts that are missing from the image, and provide them as additional context to the model to guide it toward better alignment with the prompt.
In the following sections, we describe our method by answering key questions:
(1) How do we know which images to retrieve? 
(2) How can we retrieve the required images? 
and (3) How can we use the retrieved images for unknown concept generation?
By answering these questions, we achieve our goal of generating new concepts that the model struggles to generate on its own.

\vspace{-3pt}
\subsection{Which images to retrieve?}
The amount of images we can pass to a model is limited, hence we need to decide which images to pass as references to guide the generation of a base model. As T2I models are already capable of generating many concepts successfully, an efficient strategy would be passing only concepts they struggle to generate as references, and not all the concepts in a prompt.
To find the challenging concepts,
we utilize a VLM and apply a step-by-step method, as depicted in the bottom part of \cref{fig:overview}. First, we generate an initial image with a T2I model. Then, we provide the VLM with the initial prompt and image, and ask it if they match. If not, we ask the VLM to identify missing concepts and
focus on content and style, since these are easy to convey through visual cues.
As demonstrated in \cref{tab:ablations}, empirical experiments show that image retrieval from detailed image captions yields better results than retrieval from brief, generic concept descriptions.
Therefore, after identifying the missing concepts, we ask the VLM to suggest detailed image captions for images that describe each of the concepts. 

\vspace{-4pt}
\subsubsection{Error Handling}
\label{subsec:err_hand}

The VLM may sometimes fail to identify the missing concepts in an image, and will respond that it is ``unable to respond''. In these rare cases, we allow up to 3 query repetitions, while increasing the query temperature in each repetition. Increasing the temperature allows for more diverse responses by encouraging the model to sample less probable words.
In most cases, using our suggested step-by-step method yields better results than retrieving images directly from the given prompt (see 
\cref{subsec:ablations}).
However, if the VLM still fails to identify the missing concepts after multiple attempts, we fall back to retrieving images directly from the prompt, as it usually means the VLM does not know what is the meaning of the prompt.

The used prompts can be found in \cref{app:prompts}.
Next, we turn to retrieve images based on the acquired image captions.

\vspace{-3pt}
\subsection{How to retrieve the required images?}

Given $n$ image captions, our goal is to retrieve the images that are most similar to these captions from a dataset. 
To retrieve images matching a given image caption, we compare the caption to all the images in the dataset using a text-image similarity metric and retrieve the top $k$ most similar images.
Text-to-image retrieval is an active research field~\cite{radford2021learning, zhai2023sigmoid, ray2024cola, vendrowinquire}, where no single method is perfect.
Retrieval is especially hard when the dataset does not contain an exact match to the query \cite{biswas2024efficient} or when the task is fine-grained retrieval, that depends on subtle details~\cite{wei2022fine}.
Hence, a common retrieval workflow is to first retrieve image candidates using pre-computed embeddings, and then re-rank the retrieved candidates using a different, often more expensive but accurate, method \cite{vendrowinquire}.
Following this workflow, we experimented with cosine similarity over different embeddings, and with multiple re-ranking methods of reference candidates.
Although re-ranking sometimes yields better results compared to simply using cosine similarity between CLIP~\cite{radford2021learning} embeddings, the difference was not significant in most of our experiments. Therefore, for simplicity, we use cosine similarity between CLIP embeddings as our similarity metric (see \cref{tab:sim_metrics}, \cref{subsec:ablations} for more details about our experiments with different similarity metrics).

\vspace{-3pt}
\subsection{How to use the retrieved images?}
Putting it all together, after retrieving relevant images, all that is left to do is to use them as context so they are beneficial for the model.
We experimented with two types of models; models that are trained to receive images as input in addition to text and have ICL capabilities (e.g., OmniGen~\cite{xiao2024omnigen}), and T2I models augmented with an image encoder in post-training (e.g., SDXL~\cite{podellsdxl} with IP-adapter~\cite{ye2023ip}).
As the first model type has ICL capabilities, we can supply the retrieved images as examples that it can learn from, by adjusting the original prompt.
Although the second model type lacks true ICL capabilities, it offers image-based control functionalities, which we can leverage for applying RAG over it with our method.
Hence, for both model types, we augment the input prompt to contain a reference of the retrieved images as examples.
Formally, given a prompt $p$, $n$ concepts, and $k$ compatible images for each concept, we use the following template to create a new prompt:
``According to these examples of 
$\mathord{<}c_1\mathord{>:<}img_{1,1}\mathord{>}, ... , \mathord{<}img_{1,k}\mathord{>}, ... , \mathord{<}c_n\mathord{>:<}img_{n,1}\mathord{>}, ... , $
$\mathord{<}img_{n,k}\mathord{>}$,
generate $\mathord{<}p\mathord{>}$'', 
where $c_i$ for $i\in{[1,n]}$ is a compatible image caption of the image $\mathord{<}img_{i,j}\mathord{>},  j\in{[1,k]}$. 

This prompt allows models to learn missing concepts from the images, guiding them to generate the required result. 

\textbf{Personalized Generation}: 
For models that support multiple input images, we can apply our method for personalized generation as well, to generate rare concept combinations with personal concepts. In this case, we use one image for personal content, and 1+ other reference images for missing concepts. For example, given an image of a specific cat, we can generate diverse images of it, ranging from a mug featuring the cat to a lego of it or atypical situations like the cat writing code or teaching a classroom of dogs (\cref{fig:personalization}).
\vspace{-2pt}
\begin{figure}[htp]
  \centering
   \includegraphics[width=\linewidth]{Assets/personalization.pdf}
   \caption{\textbf{Personalized generation example.}
   \emph{ImageRAG} can work in parallel with personalization methods and enhance their capabilities. For example, although OmniGen can generate images of a subject based on an image, it struggles to generate some concepts. Using references retrieved by our method, it can generate the required result.
}
   \label{fig:personalization}\vspace{-10pt}
\end{figure}



\section{Experiments}
\seclabel{experiments}
Our experiments are designed to test a) the extent to which open loop execution is an issue for precise mobile manipulation tasks, b) how effective are blind proprioceptive correction techniques, c) do object detectors and point trackers perform reliably enough in wrist camera images for reliable control, d) is occlusion by the end-effector an issue and how effectively can it be mitigated through the use of video in-painting models, and e) how does our proposed \name methodology compare to large-scale imitation learning? 


\subsection{Tasks and Experimental Setup}
We work with the Stretch RE2 robot. Stretch RE2 is a commodity mobile manipulator with a 5DOF arm mounted on top of a non-holomonic base. We upgrade the robot to use the Dex Wrist 3, which has an eye-in-hand RGB-D camera (Intel D405). 
We consider 3 task families for a total
of 6 different tasks: a) holding a knob to pull open a cabinet or drawer, b) holding a
handle to pull open a cabinet, and c) pushing on objects (light buttons, books
in a book shelf, and light switches). Our focus is on generalization. {\it
Therefore, we exclusively test on previously unseen instances, not used during
development in any way.} 
\figref{tasks} shows the instances that we test on. 

All tasks involve some precise manipulation, followed by execution of a motion
primitive. {\bf For the pushing tasks}, the precise motion is to get the
end-effector exactly at the indicated point and the motion primitive is to push
in the direction perpendicular to the surface and retract the end-effector 
upon contact. The robot is positioned such
that the target position is within the field of view of the wrist camera. A user
selects the point of pushing via a mouse click on the wrist camera image. The
goal is to push at the indicated location. Success is determined by whether the
push results in the desired outcome (light turns on / off or book gets pushed in). 
The original rubber gripper bends upon contact, we use a rigid known tool
that sticks out a bit. We take the geometry of the tool into account while servoing.

{\bf For the opening articulated object tasks}, the precise manipulation is grasping the
knob / handle, while the motion primitive is the whole-body motion that opens
the cupboard. Computing and executing this full body motion is difficult. We
adopt the modular approach to opening articulated objects (MOSART) from Gupta \etal~\cite{gupta2024opening} and invoke it
after the gripper has been placed around the knob / handle. The whole tasks 
starts out with the robot about 1.5m way from the target object, with the 
target object in view
from robot's head mounted camera. We use MOSART to compute articulation
parameters and convey the robot to a pre-grasp
location with the target handle in view of the wrist camera. At this point,
\name (or baseline) is used to center the gripper around the knob / handle, 
before resuming MOSART: extending the gripper till contact, close the gripper, and play rest of the predicted motion plan. Success is 
determined by whether the cabinet opens by more than $60^\circ$
or the drawer is pulled out by more than $24cm$, similar to the criteria used in \cite{gupta2024opening}.


For the precise manipulation part, all baselines consume the current and
previous RGB-D images from the wrist camera and output full body motor
commands.

% % Please add the following required packages to your document preamble:
% % \usepackage{graphicx}
% \begin{table*}[!ht]
% \centering
% \caption{}
% \label{tab:my-table}
% \resizebox{\textwidth}{!}{%
% \begin{tabular}{lcccccc}
% \toprule
%  & \multicolumn{2}{c}{ours} & \multicolumn{2}{c}{Gurobi} & \multicolumn{2}{c}{MOSEK} \\
%  & \multicolumn{1}{l}{time (s)} & \multicolumn{1}{l}{optimality gap (\%)} & \multicolumn{1}{l}{time (s)} & \multicolumn{1}{l}{optimality gap (\%)} & \multicolumn{1}{l}{time (s)} & \multicolumn{1}{l}{optimality gap (\%)} \\ \hline
% \begin{tabular}[c]{@{}l@{}}Linear Regression\\ Synthetic \\ (n=16000, p=16000)\end{tabular} & 57 & 0.0 & 3351 & - & 2148 & - \\ \hline
% \begin{tabular}[c]{@{}l@{}}Linear Regression\\ Cancer Drug Response\\ (n=822, p=2300)\end{tabular} & 47 & 0.0 & 1800 & 0.31 & 212 & 0.0 \\ \hline
% \begin{tabular}[c]{@{}l@{}}Logistic Regression\\ Synthetic\\ (n=16000, p=16000)\end{tabular} & 271 & 0.0 & N/A & N/A & 1800 & - \\ \hline
% \begin{tabular}[c]{@{}l@{}}Logistic Regression\\ Dorothea\\ (n=1150, p=91598)\end{tabular} & 62 & 0.0 & N/A & N/A & 600 & 0.0 \\
% \bottomrule
% \end{tabular}%
% }
% \end{table*}

% Please add the following required packages to your document preamble:
% \usepackage{multirow}
% \usepackage{graphicx}
\begin{table*}[]
\centering
\caption{Certifying optimality on large-scale and real-world datasets.}
\vspace{2mm}
\label{tab:my-table}
\resizebox{\textwidth}{!}{%
\begin{tabular}{llcccccc}
\toprule
 &  & \multicolumn{2}{c}{ours} & \multicolumn{2}{c}{Gurobi} & \multicolumn{2}{c}{MOSEK} \\
 &  & time (s) & opt. gap (\%) & time (s) & opt. gap (\%) & time (s) & opt. gap (\%) \\ \hline
\multirow{2}{*}{Linear Regression} & \begin{tabular}[c]{@{}l@{}}synthetic ($k=10, M=2$)\\ (n=16k, p=16k, seed=0)\end{tabular} & 79 & 0.0 & 1800 & - & 1915 & - \\ \cline{2-8}
 & \begin{tabular}[c]{@{}l@{}}Cancer Drug Response ($k=5, M=5$)\\ (n=822, p=2300)\end{tabular} & 41 & 0.0 & 1800 & 0.89 & 188 & 0.0 \\ \hline
\multirow{2}{*}{Logistic Regression} & \begin{tabular}[c]{@{}l@{}}Synthetic ($k=10, M=2$)\\ (n=16k, p=16k, seed=0)\end{tabular} & 626 & 0.0 & N/A & N/A & 2446 & - \\ \cline{2-8}
 & \begin{tabular}[c]{@{}l@{}}DOROTHEA ($k=15, M=2$)\\ (n=1150, p=91598)\end{tabular} & 91 & 0.0 & N/A & N/A & 634 & 0.0 \\
 \bottomrule
\end{tabular}%
}
% \vspace{-3mm}
\end{table*}

\begin{figure*}
\insertW{1.0}{figures/figure_6_cropped_brighten.pdf}
\caption{{\bf Comparison of \name with the open loop (eye-in-hand) baseline} for opening a cabinet with a knob. Slight errors in getting to the target cause the end-effector to slip off, leading to failure for the baseline, where as our method is able to successfully complete the task.}
\figlabel{rollout}
\end{figure*}

\begin{table}
\setlength{\tabcolsep}{8pt}
  \centering
  \resizebox{\linewidth}{!}{
  \begin{tabular}{lcccg}
  \toprule
                              & \multicolumn{2}{c}{\bf Knobs} & \bf Handle & \bf \multirow{2}{*}{\bf Total} \\
                              \cmidrule(lr){2-3} \cmidrule(lr){4-4}
                              & \bf Cabinets & \bf Drawer & \bf Cabinets & \\
  \midrule
  RUM~\cite{etukuru2024robot}  & 0/3    & 1/4         & 1/3         & 2/10 \\
  \name (Ours) & 2/3    & 2/4         & 3/3     &  7/10 \\
  \bottomrule
  \end{tabular}}
  \caption{Comparison of \name \vs RUM~\cite{etukuru2024robot}, a recent large-scale end-to-end imitation learning method trained on 1200 demos for opening cabinets and 525 demos for opening drawers across 40 different environments. Our evaluation spans objects from three environments across two buildings.}
  \tablelabel{rum}
\end{table}

\subsection{Baselines}
We compare against three other methods for the precise manipulation part of
these tasks. 
\subsubsection{Open Loop (Eye-in-Hand)} To assess the precision requirements of
the tasks and to set it in context with the manipulation capabilities of the
robot platform, this baseline uses open loop execution starting from estimates
for the 3D target position from the first wrist camera image.
\subsubsection{MOSART~\cite{gupta2024opening}}
The recent modular system for opening cabinets and drawers~\cite{gupta2024opening}
reports impressive performance with open-loop control (using the head camera from 1.5m away), combined with proprioception-based feedback to 
compensate for errors in perception and control when interacting with handles. 
We test if such correction is also sufficient for interacting with knobs. Note 
that such correction is not possible for the smaller buttons and pliable books.

\subsubsection{\name (no inpainting)} To understand how much of an issue
occlusion due to the end-effector is during manipulation, we ablate the use of
inpainting. %

\subsubsection{Robot Utility Models (RUM)~\cite{etukuru2024robot}}
For the opening articulated object tasks, we also compare to Robot Utility Models (RUM), 
a closed-loop imitation learning method recently proposed by Etukuru et al. \cite{etukuru2024robot}.
RUM is trained on a substantial dataset comprising expert demonstrations, including 
1,200 instances of cabinet opening and 525 of drawer opening, gathered from roughly 
40 different environments.
This dataset stands as the most extensive imitation 
learning dataset for articulated object manipulation to date, establishing RUM as a 
strong baseline for our evaluation.

Similar to our method, we use MOSART to compute articulation
parameters and convey the robot to a pre-grasp location
with the target handle in view of the wrist camera.
One of the assumptions of RUM is a good view of the handle.
To benefit RUM, we try out three different heights of the wrist camera,
and \textit{report the best result for RUM.}

\begin{figure*}
\insertW{1.0}{figures/figure_9_cropped_brighten.pdf}
\caption{{\bf \name \vs open loop (eye-in-hand) baseline for pushing on user-clicked points}. Slight errors in getting to the target cause failure, where as \name successfully turns the lights off. Note the quality of CoTracker's track ({\color{blue} blue dot}).}
\figlabel{rollout_v2}
\end{figure*}

\begin{figure*}
\insertW{1.0}{figures/figure_5_v2_cropped_brighten.pdf}
\caption{{\bf Comparison of \name with and without inpainting}. Erroneous detection without inpainting causes execution to fail, where as with inpainting the target is correctly detected leading to a successful grasp and a successful execution.}
\figlabel{rollouts2}
\end{figure*}


\subsection{Results}
\tableref{results} presents results from our experiments. 
Our training-free approach \name successfully 
solves over 85\% of task instances that we test on.
As noted, all these
tests were conducted on unseen object instances in unseen
environments that were not used for development in any way. We discuss our key
experimental findings below.

\subsubsection{Closing the loop is necessary for these precise tasks} 
While the proprioception-based strategies proposed in MOSART~\cite{gupta2024opening}
work out for handles, they are inadequate for targets like knobs and just
don't work for tasks like pushing buttons. Using estimates from the wrist
camera is better, but open loop execution still fails for knobs and pushing
buttons. 

\subsubsection{Vision models work reasonably well even on wrist camera images}
Inpainting works well on wrist camera images (see \figref{occlusion} and \figref{inpainting}).
Closing the loop using feedback from vision detectors and point trackers on
wrist camera images also work well, particularly when we use in-painted images.
See some examples detections and point tracks in \figref{rollout} and \figref{rollout_v2}. 
Detic~\cite{zhou2022detecting} was able to reliably detect the knobs and
handles and CoTracker~\cite{karaev2023cotracker} was able to successfully track
the point of interaction letting us solve 24/28 task instances.

\subsubsection{Erroneous detections without inpainting hamper performance on 
handles and our end-effector out-painting strategy effectively mitigates it} 
As shown in \figref{rollouts2}, presence of the end-effector caused the object
detector to miss fire leading to failed execution. Our out painting approach
mitigates this issue leading to a higher success rate than the 
approach without out-painting. Interestingly, CoTracker~\cite{karaev2023cotracker} is quite robust
to occlusion (possibly because it tracks multiple points) and doesn't benefit
from in-painting. 


\subsubsection{Closed-loop imitation learning struggles on novel objects}
As presented in \tableref{rum}, \name significantly outperforms RUM in a paired evaluation on unseen objects across three novel environments. A common failure mode of RUM is its inability to grasp the object's handle, even when it approaches it closely.
Another failure mode we observe is RUM misidentifying keyholes or cabinet edges as handles, also resulting in failed grasp attempts.
These result demonstrate that a modular approach that leverages the broad generalization capabilities of vision foundation models is able to generalize much better than an end-to-end imitation learning approach trained on 1000+ demonstrations, which must learn all aspects of the task from scratch.



\section{Conclusion}

In this paper, we propose a sample weight averaging strategy to address variance inflation of previous independence-based sample reweighting algorithms. 
We prove its validity and benefits with theoretical analyses. 
Extensive experiments across synthetic and multiple real-world datasets demonstrate its superiority in mitigating variance inflation and improving covariate-shift generalization.  

\section*{Impact Statement}

This work focuses on improving the representations in autoencoders that serve as a backbone for latent diffusion training, ultimately enhancing generative performance.
Our improvements can facilitate beneficial applications such as boosting creativity, supporting educational content creation, and reducing computational overhead in generative workflows.
Beyond these considerations, we do not identify additional ethical or societal implications beyond those already known to accompany large-scale generative modeling.


%%%%%%%%%%%%%%%%%%%%%%%%%%%%%%%%%%%%%%%%%%%%%%%%%%%%%%%%%%%%%%%%%%%%%%%%%%%%%%%
%%%%%%%%%%%%%%%%%%%%%%%%%%%%%%%%%%%%%%%%%%%%%%%%%%%%%%%%%%%%%%%%%%%%%%%%%%%%%%%
% Bibliography
% In the unusual situation where you want a paper to appear in the
% references without citing it in the main text, use \nocite
% \nocite{StyleGAN}
%%%%%%%%%%%%%%%%%%%%%%%%%%%%%%%%%%%%%%%%%%%%%%%%%%%%%%%%%%%%%%%%%%%%%%%%%%%%%%%
%%%%%%%%%%%%%%%%%%%%%%%%%%%%%%%%%%%%%%%%%%%%%%%%%%%%%%%%%%%%%%%%%%%%%%%%%%%%%%%
\bibliography{main}
\bibliographystyle{icml2025}

%%%%%%%%%%%%%%%%%%%%%%%%%%%%%%%%%%%%%%%%%%%%%%%%%%%%%%%%%%%%%%%%%%%%%%%%%%%%%%%
%%%%%%%%%%%%%%%%%%%%%%%%%%%%%%%%%%%%%%%%%%%%%%%%%%%%%%%%%%%%%%%%%%%%%%%%%%%%%%%
% APPENDIX
%%%%%%%%%%%%%%%%%%%%%%%%%%%%%%%%%%%%%%%%%%%%%%%%%%%%%%%%%%%%%%%%%%%%%%%%%%%%%%%
%%%%%%%%%%%%%%%%%%%%%%%%%%%%%%%%%%%%%%%%%%%%%%%%%%%%%%%%%%%%%%%%%%%%%%%%%%%%%%%
\newpage
\appendix
\onecolumn

\section{Implementation Details}
\label{ap:details}

\inlinesection{DiT model details}.
To strengthen the baseline DiT performance, we integrated into it the latest advancements from the diffusion model literature.
Namely, we used self conditioning~\cite{RIN} and RoPE~\cite{RoPE} positional embeddings.
Besides, we switched to the rectified flow diffusion parametrization~\cite{NormFlowsWithStochInterp, RecFlow, LSGM}, which was recently shown to have better scalability with a fewer amount of inference steps~\cite{SD3}.

\inlinesection{DiT training details}.
All the DiT models are trained for 400,000 steps with 10,000 warmup steps of the learning rate from 0 to 0.0003 and then its gradual decay towards 0.00001.
We used weight decay of 0.01 and AdamW~\cite{AdamW} optimizer with beta coefficients of 0.9 and 0.99.
We used posterior sampling from the encoder distribution for VAE-based autoencoders.
In contrast to the original work, we found it helpful to do learning rate decay to 0.00001 using the cosine learning rate schedule.
We used the same model sizes for DiT-S (small), DiT-B (base), DiT-L (large) and DiT-XL (extra large), as the original work~\cite{DiT}:
\begin{itemize}
    \item DiT-S: hidden dimensionality of 384, 12 transformer blocks, and 6 attention heads in the multi-head attention.
    \item DiT-B: hidden dimensionality of 768, 12 transformer blocks, and 12 attention heads in the multi-head attention.
    \item DiT-L: hidden dimensionality of 1024, 24 transformer blocks, and 16 attention heads in the multi-head attention.
    \item DiT-XL: hidden dimensionality of 1152, 28 transformer blocks, and 16 attention heads in the multi-head attention.
\end{itemize}
We used gradient clipping with the norm of 16 for all the DiT models.
Our models were trained in the FSDP~\cite{FSDP} framework with the full sharding strategy on a single node of $8 \times$ NVidia A100 80GB GPUs or $8 \times$ NVidia H100 80GB GPUs (depending on their availability in our computational cluster).

For \cvae, since it is considerably slower than other autoencoders, we trained LDMs on pre-extracted latents.
For this, we pre-extracted them on random 17-frames clips.
In essence, this reduces

\inlinesection{Autoencoders training details}.
Since none of the autoencoders had their training pipelines released, we had to develop the training recipes for each of the autoencoder baselines individually which would not be detrimental to neither their reconstruction capability nor downstream diffusion performance.
To do this, we ablated multiple hyperparameters (the most important ones being learning rate and KL regularization strength) to arrive to a proper setup.
We chose the KL weight in such a way that the KL penalty maintains approximately the same magnitude as the pre-trained checkpoint.

Each autoencoder is trained with AdamW~\cite{AdamW} optimizer, with betas of 0.9 and 0.99, and weight decay of 0.01.
The learning rate was grid-searched individually for each autoencoder and is provided in \cref{tab:hyperparameters}.
In all the cases, we used mixed precision training with BFloat16.

During training, we maintained an exponential moving average of the weights~\cite{EDMv2}, initialized from the same parameters as the starting model, and having a half life of 5,000 steps.
% For each autoencoder, we were searching for the KL regularization coefficient in a way that the KL magnitude remains the same 

We emphasize that, when applying our regularization strategy on top of an autoencoder baseilne, we do not alter other hyperparameters (like learning rate), except for KL regularization which we disable for \regshortname-regularized models (even though we found it helpful in some of our explorations).

For each autoencoder, we freeze the last output layers of the decoder.
The motivation is the following: they were fine-tuned with the adversarial loss, which we want to exclude from the equation without hurting the ability of an autoencoder to model textural details which \fid would be sensitive to~\citep{LDM} and which do not influence the latent space properties.
Namely, we freeze the last normalization and output convolution layers.
In each case, the amount of frozen parameters constitute a negligible amount of total parameters.

Other hyperparameters for autoencoders training are provided in \cref{tab:hyperparameters}.

\section{Hyperparameter Search}\label{app:hype}
\normalsize
We exclusively conduct hyperparameter search on fold 0. 
For \textbf{GraFITi}~\citep{Yalavarthi2024.GraFITi} the hyperparameters for the search are as follows:
\begin{itemize}
    \item The number of layers, with possible values [1, 2, 3, 4].
    \item The number of attention heads, with possible values [1, 2, 4].
    \item The latent dimension, with possible values [16, 32, 64, 128, 256].
\end{itemize}

For the \textbf{LinODEnet} model~\citep{Scholz2022.Latenta} we search the hyperparameters from:
\begin{itemize}
    \item The hidden dimension, with possible values [16, 32, 64, 128].
    \item The latent dimension, with possible values [64, 128, 192, 256].
\end{itemize}

For \textbf{GRU-ODE-Bayes}~\citep{DeBrouwer2019.GRUODEBayesd} we tune the hidden size from [16, 32, 64, 128, 256]

For \textbf{Neural Flows}~\citep{Bilos2021.Neurald} we define the hyperparameter spaces for the search are as follows:
\begin{itemize}
    \item The number of flow layers, with possible values [1, 2, 4].
    \item The hidden dimension, with possible values [16, 32, 64, 128, 256].
    \item The flow model type, with possible values [GRU, ResNet].
\end{itemize}

For the \textbf{CRU}~\citep{Schirmer2022.Modelingb} the hyperparameter space is as follows:
\begin{itemize}
    \item The latent state dimension, with possible values [10, 20, 30].
    \item The number of basis functions, with possible values [10, 20].
    \item The bandwidth with possible values [3, 10].
\end{itemize}


\section{Additional Exploration}
\label{ap:freq-reg}

In \cref{sec:method}, we outlined the base \regname strategy to regularize the spectrum of an autoencoder which has a strong advantage of being very easy to implement by a practitioner.
However, it could be beneficial to possess more advanced tools for a finer-grained control over the latent space spectral properties.
This section outlines them and provides the corresponding ablation.

\subsection{Explicitly Chopping off High-Frequency Components}

Rather than applying downsampling to produce latents and RGB targets for regualrization, it is possible to replace some ratio of high-frequency components with zeros. To do so, DCT is applied to the latents and RGB targets where a chosen set of frequency components are masked out. The modified components are then translated back to the spatial domain by inverse DCT to form the training latents and reconstruction targets.

\begin{equation}\label{eq:hard-hf-penalty}
\loss_\text{CHF}(x) = d(x, \Dec(z)) + d( {D^{-1}}(D({x}) * \mathbf{M}), \Dec({D^{-1}}(D({z}) * \mathbf{M}) ) + \loss_\text{reg},
\end{equation}
where $D$ and $D^{-1}$ represent DCT and its inverse, respectively. $\mathbf{M}$ is a $B \times B$ binary mask indicating which frequencies to zero out defined as follows:
% \willi{Are we applying this in B * AEupsamplefactor DCT space for x?}
\begin{equation}\label{eq:zigzag-mask}
\mathbf{M}(u,v) =
\begin{cases}
    1, & \text{if } \zigzag (u,v) < B^2 - N,\\
    0, & \text{otherwise}.
\end{cases}
\end{equation}
% \willi{The table has some lines that suggest M may not be cut of in zigzag order as the equation implies}
$N$ controls the frequency cutoff. We provide the ablation for this strategy in \cref{table:cuthf-ablation}.

\begin{table}[h]
\caption{Ablations for explicit high-frequency chop off for DiT-S/2 trained for 200,000 iterations on top of Flux AE with such a regularization. While it can achieve better results for some of the baselines than naive downsampling, we opt out for the latter strategy due to its simplicity. For the non-zigzag order ablation, we cut across each $x$ and $y$ axes independently}
\label{table:cuthf-ablation}
\centering
% \resizebox{1.0\linewidth}{!}{
\begin{tabular}{llcc}
\toprule
Stage II & Stage I  & \dinofidfivek \\
\midrule
DiT-S/2 & FluxAE + chop off 90\% (non-zigzag order) & 912.4 \\
DiT-S/2 & FluxAE + chop off 70\% (non-zigzag order) & 915.6 \\
DiT-S/2 & FluxAE + chop off 30\% (non-zigzag order) & 929.7 \\
DiT-S/2 & FluxAE + chop off 10\% (non-zigzag order) & 916.5 \\
DiT-S/2 & FluxAE + chop off 90\% (zigzag order) & 935.5 \\
DiT-S/2 & FluxAE + chop off 70\% (zigzag order) & 932.8 \\
DiT-S/2 & FluxAE + chop off 30\% (zigzag order) & 962.9 \\
DiT-S/2 & FluxAE + chop off 10\% (zigzag order) & 930.1 \\
\midrule
DiT-S/2 & FluxAE (vanilla) & 992.0 \\
DiT-S/2 & FluxAE with optimal (out of 8) KL $\beta$ & 929.6 \\
% \midrule
% DiT-B/2 (orig) & \multirow{4}{*}{SD-VAE-ft-MSE} & 43.47 & $-$ \\
% DiT-L/2 (orig) & & 23.33 & $-$ \\
% DiT-XL/2 (orig) & & 19.47 & $-$ \\
% ~+ 6.6M steps (orig) & & 12.03 & n/a & 121.50 \\
% ~+ 6.6M steps (orig) & & 12.03 & $-$ \\
\bottomrule
\end{tabular}
% }
\end{table}

In \cref{fig:progressive-dct-cut}, we provided the visualizations for a FluxAE resiliense with and without such chopping high-frequency regularization for $50\%$ HF dropout rate.
In \cref{fig:progressive-dct-cut-downreg}, we provide an equivalent visualization for \regshortname-fine-tuned FluxAE: while it is less resilient to frequency dropout than \regchfshortname, but is still noticeably better than the vanilla model.

\begin{figure}[t]
\centering
% \includegraphics[width=0.95\linewidth]{assets/progressive-dct-cut-qualitative.pdf}
% \includegraphics[width=\linewidth]{assets/progressive-dct-cut-qualitative-new.png}
\begin{minipage}{0.7\linewidth}
\setlength{\unitlength}{\linewidth}
\centering
\vspace{-2.21cm}
\begin{picture}(1,1)
% \put(0.05, 0){\includegraphics[width=0.95\linewidth]{assets/progressive-dct-cut-qualitative-new.png}}
\put(0.05, 0){\includegraphics[width=0.95\linewidth]{assets/grid-004-dreg.png}}

% Column captions - adjust Y slightly below the image (e.g., -0.03)
\put(0.175, -0.01){\makebox(0,0)[t]{\small 0\%}}
\put(0.415, -0.01){\makebox(0,0)[t]{\small 25\%}}
\put(0.65, -0.01){\makebox(0,0)[t]{\small 50\%}}
\put(0.89, -0.01){\makebox(0,0)[t]{\small 75\%}}

% Row captions - adjust X slightly left of the image (e.g., -0.08)
% \put(0.035, 0.61){\makebox(0,0)[r]{\rotatebox{90}{\small FluxAE}}}
\put(0.035, 0.365){\makebox(0,0)[r]{\rotatebox{90}{\small FluxAE}}}
\put(0.035, 0.12){\makebox(0,0)[r]{\rotatebox{90}{\small FluxAE + \regshortname}}}
\end{picture}
\end{minipage}
\caption{
RGB and FluxAE reconstruction with/without scale equivariance regularization for different percentages of chopped-off high frequency components.
}
\label{fig:progressive-dct-cut-downreg}
\end{figure}



\subsection{Soft Penalty for High-Frequency Components}

Instead of directly removing some of the components, which might become a too strict regularization signal, one can consider penalizing the amplitudes of high-frequency components in a soft manner.
Concretely, given a $B \times B$ block, we construct the following weight penalty matrix:
% amplitude = x_dct.abs() # [..., b, b]
% x_coords = torch.linspace(0, 1, steps=block_size, device=x.device, dtype=x.dtype).unsqueeze(0).expand(block_size, -1) # [b, b]
% y_coords = torch.linspace(0, 1, steps=block_size, device=x.device, dtype=x.dtype).unsqueeze(1).expand(-1, block_size) # [b, b]
% weight = (x_coords + y_coords).float().pow(power) / 4.0 # [b, b]
% weight = misc.unsqueeze_left(weight, amplitude) # [..., b, b]
% loss = (amplitude * weight).reshape(amplitude.shape[0], -1).mean(dim=1) # [batch_size]

\begin{equation}\label{eq:soft-hf-penalty-matrix}
\mathcal{W}_{uv} = (u + v)^p / B^p.
\end{equation}

Next, the soft regularization loss itself is computed as:
\begin{equation}\label{eq:soft-hf-penalty}
\loss_\text{softreg} = \sum_{u,v} D_{uv}(z) \cdot \mathcal{W}_{uv}.
\end{equation}

During training, when enabled, we add it to the main loss with the weigh $\gamma$.
We found it beneficial in some of our experiments when it is added with a small coefficient (e.g., 0.01).
While it is possible to achieve higher results with more fine-grained regularization, we opt to use the simpler version since we believe it would be easier to employ by the community.

To ablate its importance, we trained DiT-B/1 model on top of FluxAE models, fine-tuned with a different strength $\gamma$.
The results are presented in \cref{table:softregweight-abl}.

\begin{table}[ht]
\caption{Ablating the regularization strength $\alpha$ of our proposed \regname regularization.}
\label{table:softregweight-abl}
\centering
% \resizebox{1.0\linewidth}{!}{
\begin{tabular}{llcc}
\toprule
Stage II & Stage I  & \fid$_{5k}$ & \dinofid$_{5k}$ \\
\midrule
\multirow{6}{*}{DiT-B/1}
& FluxAE + FT-\regshortname $\gamma = 0.001$ & 26.43 & 497.14 \\
& FluxAE + FT-\regshortname $\gamma = 0.025$ & 25.46 & 477.61 \\
& FluxAE + FT-\regshortname $\gamma = 0.01$ & 26.72 & 487.06 \\
& FluxAE + FT-\regshortname $\gamma = 0.05$ & \cellbest{24.28} & \cellbest{458.11} \\
& FluxAE + FT-\regshortname $\gamma = 0.1$ & \cellsecond{25.84} & \cellsecond{461.97} \\
\bottomrule
\end{tabular}
% }
\end{table}


\begin{figure}
\centering
\includegraphics[width=0.4\linewidth]{assets/zigzag.pdf}
\caption{Illustration of the zigzag indexing order of DCT.}
\label{fig:zigzag}
\end{figure}

\subsection{ImageNet $512^2$ experiments}

We trained our DiT-L/2 for class-conditional $512^2$ ImageNet-1K generation for 400K steps for FluxAE~\cite{Flux}, the results are presented in \cref{table:imagenet-512}.

\begin{table}[ht]
\caption{Class-conditional generation results on ImageNet-1K $512^2$ without guidance. The original DiT paper reports the results after 3M training steps, while we use 400K steps for our models.}
\label{table:imagenet-512}
\centering
% \resizebox{1.0\linewidth}{!}{
\begin{tabular}{llcc}
\toprule
Stage II & Stage I  & \fid & \dinofid \\
\midrule
% 3278: "fid50k": 13.130795660857324, "dinofid50k": 249.46464787074206,
% 3279: "fid50k": 13.697963826137052, "dinofid50k": 267.7557197224488,
% 3280: "fid50k": 11.631835931219992, "dinofid50k": 203.55857327388986,
\multirow{3}{*}{DiT-L/2} & \fluxae (vanilla) & \cellsecond{13.13} & \cellsecond{249.4} \\
& \fluxae + FT & 13.69 & 267.7 \\
& \fluxae + FT-\regshortname \ours & \cellbest{11.63} & \cellbest{203.5} \\
\midrule
DiT-XL/2 (orig) + 3M steps & SD-VAE-ft-MSE & 12.03 & $-$ \\
\bottomrule
\end{tabular}
% }
\end{table}


\subsection{Ablating regularization strength $\alpha$}

To ablate the importance of the regularization strength $\alpha$, we train FluxAE for 10,000 steps with a varying strength.
The results are presented in \cref{table:regweight-abl}.

\begin{table}[ht]
\caption{Ablating the regularization strength $\alpha$ of our proposed \regname regularization.}
\label{table:regweight-abl}
\centering
% \resizebox{1.0\linewidth}{!}{
\begin{tabular}{llcc}
\toprule
Stage II & Stage I  & \fid$_{5k}$ & \dinofid$_{5k}$ \\
\midrule
% 3278: "fid50k": 13.130795660857324, "dinofid50k": 249.46464787074206,
% 3279: "fid50k": 13.697963826137052, "dinofid50k": 267.7557197224488,
% 3280: "fid50k": 11.631835931219992, "dinofid50k": 203.55857327388986,
% & \fluxae + FT & 13.69 & 267.7 \\
% & \fluxae + FT-\regshortname \ours & \cellbest{11.63} & \cellbest{203.5} \\
% \midrule
% DiT-XL/2 (orig) + 3M steps & SD-VAE-ft-MSE & 12.03 & $-$ \\
\multirow{6}{*}{DiT-B/2}
& FluxAE + FT-\regshortname $\alpha = 0.01$ & 33.99 & 641.95 \\
& FluxAE + FT-\regshortname $\alpha = 0.05$ & 33.86 & 645.94 \\
& FluxAE + FT-\regshortname $\alpha = 0.1$ & \cellsecond{28.62} & \cellsecond{586.91} \\
& FluxAE + FT-\regshortname $\alpha = 0.25$ & \cellbest{26.84} & \cellbest{558.36} \\
& FluxAE + FT-\regshortname $\alpha = 0.5$ & 29.63 & 569.92 \\
& FluxAE + FT-\regshortname $\alpha = 1$ & 33.22 & 612.45 \\
\bottomrule
\end{tabular}
% }
\end{table}


\section{Additional visualizations}
\label{ap:visualizations}

This section provides additional visualizations for the LDM experiments.

\begin{figure*}[h]
\centering
\includegraphics[width=\linewidth]{assets/extra-sample-grids-cfg3.0/ldm2840-ditxl-flux-init-01.jpg}
\includegraphics[width=\linewidth]{assets/extra-sample-grids-cfg3.0/ldm2903-ditxl-flux-ft-01.jpg}
\includegraphics[width=\linewidth]{assets/extra-sample-grids-cfg3.0/ldm2839-ditxl-flux-reg-01.jpg}
\caption{Uncurated samples from DiT-XL/2 for FluxAE (top), FluxAE + FT (middle) and FluxAE + \regshortname (bottom) on class-conditional ImageNet $256 \times 256$ for random classes. During inference, we used 256 steps with the guidance scale of 3.0.}
\label{fig:ap:extra-samples-ditxl-flux}
\end{figure*}

\begin{figure*}[h]
\centering
\includegraphics[width=\linewidth]{assets/extra-samples/cfg3.0--ldm3164-ditxl-flux-reg-1M-00.jpg}
\includegraphics[width=\linewidth]{assets/extra-samples/cfg3.0--ldm3164-ditxl-flux-reg-1M-01.jpg}
\caption{Uncurated samples from DiT-XL/2 trained for 1M steps on top FluxAE + \regshortname (bottom) on class-conditional ImageNet $256 \times 256$ for random classes. During inference, we used 256 steps with the guidance scale of 3.0.}
\label{fig:ap:extra-samples-ditxl-flux-1m}
\end{figure*}

\begin{figure*}[h]
\centering
\includegraphics[width=\linewidth]{assets/extra-samples/cfg3.0--class88-ldm3164-ditxl-flux-reg-1M-00.jpg}
\caption{Uncurated samples from DiT-XL/2 trained for 1M steps on top FluxAE + \regshortname (bottom) on class-conditional ImageNet $256 \times 256$ for class 88. During inference, we used 256 steps with the guidance scale of 3.0.}
\label{fig:ap:extra-samples-ditxl-flux-1m-class-88}
\end{figure*}

\begin{figure*}[h]
\centering
\includegraphics[width=\linewidth]{assets/extra-samples/cfg3.0--class130-ldm3164-ditxl-flux-reg-1M-00.jpg}
\caption{Uncurated samples from DiT-XL/2 trained for 1M steps on top FluxAE + \regshortname (bottom) on class-conditional ImageNet $256 \times 256$ for class 130. During inference, we used 256 steps with the guidance scale of 3.0.}
\label{fig:ap:extra-samples-ditxl-flux-1m-class-130}
\end{figure*}

\begin{figure*}[h]
\centering
\includegraphics[width=\linewidth]{assets/extra-samples/cfg3.0--class279-ldm3164-ditxl-flux-reg-1M-00.jpg}
\caption{Uncurated samples from DiT-XL/2 trained for 1M steps on top FluxAE + \regshortname (bottom) on class-conditional ImageNet $256 \times 256$ for class 279. During inference, we used 256 steps with the guidance scale of 3.0.}
\label{fig:ap:extra-samples-ditxl-flux-1m-class-279}
\end{figure*}

\begin{figure*}[h]
\centering
\includegraphics[width=\linewidth]{assets/extra-samples/cfg3.0--class555-ldm3164-ditxl-flux-reg-1M-00.jpg}
\caption{Uncurated samples from DiT-XL/2 trained for 1M steps on top FluxAE + \regshortname (bottom) on class-conditional ImageNet $256 \times 256$ for class 555. During inference, we used 256 steps with the guidance scale of 3.0.}
\label{fig:ap:extra-samples-ditxl-flux-1m-class-555}
\end{figure*}

\begin{figure*}[h]
\centering
\includegraphics[width=\linewidth]{assets/extra-samples/cfg3.0--ldm3278-512x512-ditl2-flux-orig-00.jpg}
\includegraphics[width=\linewidth]{assets/extra-samples/cfg3.0--ldm3279-512x512-ditl2-flux-ft-00.jpg}
\includegraphics[width=\linewidth]{assets/extra-samples/cfg3.0--ldm3280-512x512-ditl2-flux-reg-00.jpg}
\caption{Uncurated samples from DiT-XL/2 trained for 400K steps on top FluxAE + \regshortname (bottom) on class-conditional ImageNet $512 \times 512$ for random classes. During inference, we used 256 steps with the guidance scale of 3.0.}
\label{fig:ap:extra-samples-ditxl-flux-r512}
\end{figure*}

\begin{figure*}[h]
\centering
\includegraphics[width=\linewidth]{assets/extra-samples/cfg1.5--ldm3271-ditb1-cmsaei-vanilla-01.jpg}
\includegraphics[width=\linewidth]{assets/extra-samples/cfg1.5--ldm3308-ditb1-cmsaei-ft-01.jpg}
\includegraphics[width=\linewidth]{assets/extra-samples/cfg1.5--ldm3310-ditb1-cmsaei-reg-01.jpg}
\caption{Uncurated samples from DiT-B/1 for \cmsaei (top), \cmsaei + FT (middle) and \cmsaei + \regshortname (bottom) on class-conditional ImageNet $256 \times 256$. During inference, we used 256 steps with the guidance scale of 1.5.}
\label{fig:ap:extra-samples-ditb-cmsaei}
\end{figure*}

\begin{figure*}[h]
\centering
\includegraphics[width=\linewidth]{assets/extra-samples/cfg3.0--ldm3045-ditxl2-cvae-vanilla-00.jpg}
\includegraphics[width=\linewidth]{assets/extra-samples/cfg3.0--ldm3046-ditxl2-cvae-ft-00.jpg}
\includegraphics[width=\linewidth]{assets/extra-samples/cfg3.0--ldm3047-ditxl2-cvae-ft-reg-00.jpg}
\caption{Uncurated samples from DiT-XL/2 for \cvaefull (top), \cvaefull + FT (middle) and \cvaefull + \regshortname (bottom) on class-conditional Kinetics $17 \times 256 \times 256$. During inference, we used 256 steps with the guidance scale of 3.0.}
\label{fig:ap:extra-samples-ditxl-cvae}
\end{figure*}

\begin{figure*}[h]
\centering
\includegraphics[width=\linewidth]{assets/extra-sample-grids-cfg3.0/ldm2942-ditb2-cvae-init-00.jpg}
\includegraphics[width=\linewidth]{assets/extra-sample-grids-cfg3.0/ldm2907-ditb2-cvae-ft-00.jpg}
\includegraphics[width=\linewidth]{assets/extra-sample-grids-cfg3.0/ldm2908-ditb2-cvae-reg-00.jpg}
\caption{Uncurated samples from DiT-B/2 for \cvaefull (top), \cvaefull + FT (middle) and \cvaefull + \regshortname (bottom) on class-conditional Kinetics $17 \times 256 \times 256$. During inference, we used 256 steps with the guidance scale of 3.0.}
\label{fig:ap:extra-samples-ditb-cvae}
\end{figure*}


\begin{figure*}[h]
\centering
\includegraphics[width=\linewidth]{assets/extra-sample-grids-cfg3.0/ldm2943-ditb1-ltx-init-00.jpg}
\includegraphics[width=\linewidth]{assets/extra-sample-grids-cfg3.0/ldm2940-ditb1-ltx-ft-00.jpg}
\includegraphics[width=\linewidth]{assets/extra-sample-grids-cfg3.0/ldm2936-ditb1-ltx-reg-kl-00.jpg}
\caption{Uncurated samples from DiT-B/1 for \ltxae (top), \ltxae + FT (middle) and \ltxae + \regshortname (bottom) on class-conditional Kinetics $17 \times 256 \times 256$. During inference, we used 256 steps with the guidance scale of 3.0.}
\label{fig:ap:extra-samples-ditb-ltxae}
\end{figure*}

\section{Limitations.}

We identify the following limitations of our work and the proposed regularization:
\begin{enumerate}
    \item While we did our best to verify that our framework works in the most general setup possible, testing 4 different autoencoders across 2 different domains (image and videos), our study would be more complete when verified across other diffusion parametrizations~\cite{EDM, DDPM, VDM++} or architectures~\cite{EDMv2}.
    \item We observed that our regularization still affects the reconstruction slightly: for example, \cref{table:ae-rec} shows that FluxAE \fid increased from 0.183 to 0.55 (though for some AEs, like \cvaefull, it improves). We are convinced that this \fid increase could be mitigated by training with adversarial losses, which we omitted in this work for simplicity.
    \item There is a mild sensitivity to hyperparameters: for example, we found that varying the SHF regularization weight might improve the results (see \cref{table:regweight-abl}), or adding a small KL regularization (which we disabled in the end for our regularization for simplicity).
    % \item For some setups, autoencoder training should be longer: for example, we observed that DiT-B/1 training on top of \cmsaei leads to better results after 200,000 fine-tuning steps rather than 10,000 steps as was used in the current work.
    \item None of the explored autoencoders released their training pipelines, and it is non-trivial to fine-tune them even without any extra regularization. For example, we observed that any fine-tuning of \dcae~\cite{DC-AE} was leading to divergent reconstructions in our training pipeline (we explored dozens of different hyperparameter setups). 
\end{enumerate}

We leave the exploration of these limitations for future work.

% \section{Failed experiments}
\label{ap:failed-experiments}


We tried two theory-inspired regularization strategies:
\begin{itemize}
    \item MiniLDM-regulzation training. This was inspired by LSGM~\cite{LSGM} where the authors showed that it's the correct objective for LDMs. We spent a month trying to make it work, but in all the cases it . At the same time, \cite{CausRegTok} were able to make it work for discrete autoregressive models.
    \item Lipszhitz regularization. This was inspired by the lower bound which \cite{LFM} provided:
    \begin{equation}
        W(p, p') \leq ...
    \end{equation}
    This more or less worked, but it was tough to train. Also, it was leading to unstable training in the LDM itself. Something to revisit
    \item 
\end{itemize}


\end{document}
