%%%%%%%% ICML 2025 EXAMPLE LATEX SUBMISSION FILE %%%%%%%%%%%%%%%%%

\documentclass{article}

% Recommended, but optional, packages for figures and better typesetting:
\usepackage{microtype}
\usepackage{graphicx}
\usepackage{subfigure}
\usepackage{booktabs} % for professional tables

% hyperref makes hyperlinks in the resulting PDF.
% If your build breaks (sometimes temporarily if a hyperlink spans a page)
% please comment out the following usepackage line and replace
% \usepackage{icml2025} with \usepackage[nohyperref]{icml2025} above.
\usepackage{hyperref}


% Attempt to make hyperref and algorithmic work together better:
\newcommand{\theHalgorithm}{\arabic{algorithm}}

% Use the following line for the initial blind version submitted for review:
% \usepackage{icml2025}

% If accepted, instead use the following line for the camera-ready submission:
\usepackage[preprint]{icml2025}

% For theorems and such
\usepackage{amsmath}
\usepackage{amssymb}
\usepackage{mathtools}
\usepackage{amsthm}

% if you use cleveref..
\usepackage[capitalize,noabbrev]{cleveref}

%%%%%%%%%%%%%%%%%%%%%%%%%%%%%%%%
% THEOREMS
%%%%%%%%%%%%%%%%%%%%%%%%%%%%%%%%
\theoremstyle{plain}
\newtheorem{theorem}{Theorem}[section]
\newtheorem{proposition}[theorem]{Proposition}
\newtheorem{lemma}[theorem]{Lemma}
\newtheorem{corollary}[theorem]{Corollary}
\theoremstyle{definition}
\newtheorem{definition}[theorem]{Definition}
\newtheorem{assumption}[theorem]{Assumption}
\theoremstyle{remark}
\newtheorem{remark}[theorem]{Remark}

% Todonotes is useful during development; simply uncomment the next line
%    and comment out the line below the next line to turn off comments
%\usepackage[disable,textsize=tiny]{todonotes}
\usepackage[textsize=tiny]{todonotes}


% --- disable by uncommenting  
% \renewcommand{\TODO}[1]{}
% \renewcommand{\todo}[1]{#1}
%
%
%
%
%
%%%%%%%%%%%%%%%%%%%%%%%% CUSTOM IMPORTS START %%%%%%%%%%%%%%%%%%%%%%%%
% \usepackage{amsmath} % For \mathcal
% \usepackage{amssymb} % For \triangleq
\usepackage{multirow}
\usepackage{makecell} % For multirowcell
% \usepackage[pdftex]{graphicx} % For \includegraphics
% \usepackage{subcaption} % For subfigure inside figure
\usepackage{bm} % for bold variable names
\usepackage{stmaryrd} % for integer division // (\sslash)
\usepackage{dsfont} % for \mathds{1}
\usepackage{pifont} % for checkmark/crossmark
\usepackage{ifthen} % for ifthenelse in \expect
\usepackage{booktabs} % professional-quality tables
% \usepackage{dblfloatfix} % to put the teaser figure in the bottom
\usepackage{soul}
\usepackage{wrapfig} % For half-page sized figures/tables 
\usepackage{colortbl} % To color table borders
\usepackage{xspace}
% <= For algorithms start =>
% \usepackage[ruled,vlined]{algorithm2e}
% \newcommand\mycommfont[1]{\footnotesize\textcolor{blue}{#1}}
% \SetCommentSty{mycommfont}
% <= For algorithms end =>
% \usepackage[accsupp]{axessibility} % Improves PDF readability for those with disabilities.

% To input the code
\usepackage{listings}
\usepackage{xcolor}

\definecolor{codegreen}{rgb}{0,0.6,0}
\definecolor{codegray}{rgb}{0.5,0.5,0.5}
\definecolor{codepurple}{rgb}{0.58,0,0.82}
\definecolor{backcolour}{rgb}{0.95,0.95,0.92}

\lstdefinestyle{mystyle}{
    backgroundcolor=\color{backcolour},   
    commentstyle=\color{codegreen},
    keywordstyle=\color{magenta},
    numberstyle=\tiny\color{codegray},
    stringstyle=\color{codepurple},
    basicstyle=\ttfamily\footnotesize,
    breakatwhitespace=false,         
    breaklines=true,                 
    captionpos=b,                    
    keepspaces=true,                 
    numbers=left,                    
    numbersep=5pt,                  
    showspaces=false,                
    showstringspaces=false,
    showtabs=false,                  
    tabsize=2
}
\lstset{style=mystyle}


%%%%%%%%%%%%%%%%%%%%%%%% CUSTOM IMPORTS END %%%%%%%%%%%%%%%%%%%%%%%%
%
%
%
%%%%%%%%%%%%%%%%%%%%%%%% CUSTOM COMMANDS START %%%%%%%%%%%%%%%%%%%%%%%%
\newcommand{\red}[1]{{\color{red}#1}}
% \newcommand{\todo}[1]{{\color{red}#1}}

\definecolor{mediumtealblue}{rgb}{0.0, 0.33, 0.71}
\definecolor{darkpastelgreen}{rgb}{0.01, 0.75, 0.24}
\definecolor{azure}{rgb}{0.0, 0.5, 1.0}

\definecolor{crimsonred}{rgb}{0.86, 0.08, 0.24}
\definecolor{firebrick}{rgb}{0.7, 0.13, 0.13}
\definecolor{carmine}{rgb}{0.59, 0.0, 0.09}
\definecolor{rosewood}{rgb}{0.4, 0.0, 0.04}
\definecolor{deepcherry}{rgb}{0.6, 0.13, 0.13}


% Let's define shortcuts for our G, D and F models
\newcommand{\A}{\mathsf{A}}
\newcommand{\C}{\mathsf{C}}
\newcommand{\D}{\mathsf{D}}
\newcommand{\E}{\mathsf{E}}
\newcommand{\F}{\mathsf{F}}
\newcommand{\G}{\mathsf{G}}
\newcommand{\M}{\mathsf{M}}
\newcommand{\Ss}{\mathsf{S}}
\newcommand{\T}{\mathsf{T}}

\newcommand{\W}{\mathcal{W}}
\newcommand{\Z}{\mathcal{Z}}

\newcommand{\R}{\mathbb{R}}
\newcommand{\N}{\mathbb{N}}

\newcommand{\x}{\bm{x}}
\newcommand{\z}{\bm{z}}
\newcommand{\ctx}{\text{ctx}}
\newcommand{\act}{\bm{a}}
\newcommand{\coords}{\bm{c}}
\newcommand{\coordsbase}{\coords_\text{base}}
\newcommand{\scale}{s}
\newcommand{\scalef}{s_f}
\newcommand{\scaleh}{s_h}
\newcommand{\scalew}{s_w}
\newcommand{\scales}{\bm{s}}
\newcommand{\offsets}{\bm{\delta}}
\newcommand{\offsetf}{\delta_f}
\newcommand{\offseth}{\delta_h}
\newcommand{\offsetw}{\delta_w}
\newcommand{\pe}{\text{PE}}
\newcommand{\fuse}{\texttt{fuse}}
\newcommand{\fvdmini}{FVD@512}
\newcommand{\megapatch}{\bm{P}}
\newcommand{\megasigma}{\bm{\sigma}}

\newcommand{\Normal}{\mathcal{N}}

\newcommand{\ours}{\textcolor{azure}{(ours)}}
\newcommand{\apref}[1]{\ref{#1}}
\newcommand{\params}{\bm{\theta}}

\newcommand{\fix}{\marginpar{FIX}}
\newcommand{\new}{\marginpar{NEW}}
% \newcommand{\todo}[1]{\textcolor{red}{#1}}
\newcommand{\modelfullname}{Hierarchical Patch Diffusion Model}
\newcommand{\modelname}{HPDM}
\newcommand{\modelnameS}{\modelname-S}
\newcommand{\modelnameM}{\modelname-M}
\newcommand{\modelnameL}{\modelname-L}
\newcommand{\modelnameTTV}{\modelname-T2V}
\newcommand{\modelnameTTVK}{\modelname-T2V-1K}
\newcommand{\SnapVideo}{SnapVideo}
\newcommand{\parallelwork}{(concurrent work)}
\newcommand{\best}[1]{\textcolor{azure}{#1}}
\newcommand{\secondbest}[1]{\textcolor{azure}{#1}}
\newcommand{\gridsample}{\texttt{grid\_sample}_\text{3D}}
\newcommand{\concat}{\texttt{concat}_\text{3D}}
\newcommand{\noise}{\bm{\varepsilon}}
\newcommand{\patch}{\bm{p}}
\newcommand{\patchparams}{\bm{\phi}_{\coords}}
\newcommand{\patchrec}{\hat{\bm{x}}_p}

\newcommand{\cmark}{\ding{51}}
\newcommand{\xmark}{\ding{55}}

\newcommand{\figref}[1]{Fig.~\ref{#1}}
\newcommand{\secref}[1]{\S\ref{#1}}
\newcommand{\tabref}[1]{Tab.~\ref{#1}}
\newcommand{\inlinesection}[1]{\noindent{\textbf{#1}}}

\newcommand{\green}[1]{\textcolor{green}{{#1}}}
\newcommand{\sergey}[1]{{\color{blue}{S:#1}}}

% \newcommand{\codeurl}{https://example.com}
\newcommand{\projecturl}{https://snap-research.github.io/hpdm}
\newcommand{\projecthref}{\href{\projecturl}{\projecturl}}
% \newcommand{\projecturlextra}{https://u2wjb9xxz9q.github.io/additional-results}
% \newcommand{\projecthrefextra}{\href{\projecturlextra}{\projecturlextra}}
% \newtheorem*{simplestatement*}{A trivial but serviceable statement}

\newcommand{\checkmarknew}{\ding{51}} 
\newcommand{\crossmark}{\ding{55}}

% \newcommand{\regname}{Scale Equivariance\xspace}
\newcommand{\Regname}{Scale Equivariance\xspace}
\newcommand{\regname}{scale equivariance\xspace}
\newcommand{\regshortname}{SE\xspace}

\newcommand{\regchfname}{Chopping High Frequencies\xspace}
\newcommand{\regchfshortname}{CHF\xspace}

\newcommand{\rdag}{$^{\textcolor{red}{\dag}}$}
\newcommand{\Diffusability}{\textit{Diffusability}\xspace}
\newcommand{\diffusability}{\textit{diffusability}\xspace}

\newcommand{\expect}[2][]{
\ifthenelse{\equal{#1}{}}{
\mathbb{E}\left[#2\right]
}{
\underset{#1}{\mathbb{E}}\left[#2\right]
}}
\newcommand{\updated}[1]{{\color{red}{#1}}}
\newcommand{\newtable}[0]{
\arrayrulecolor{blue}
\color{blue}
}

\DeclareMathOperator{\zigzag}{zigzag}
\DeclareMathOperator{\Enc}{Enc}
\DeclareMathOperator{\Dec}{Dec}


\newcommand{\cellbest}{\cellcolor{azure!35}}
\newcommand{\cellsecond}{\cellcolor{azure!10}}
% \newcommand{\cellthird}{\cellcolor{Yellow!25}}

% \newcommand{\fid}{$\text{FID}_\text{50K}$\xspace}
\newcommand{\fid}{FID\xspace}
\newcommand{\fvd}{$\text{FVD}_\text{10K}$\xspace}
\newcommand{\fvdfull}{$\text{FVD}_\text{50K}$\xspace}
\newcommand{\fidtenk}{$\text{FID}_\text{10K}$\xspace}
\newcommand{\psnrsmall}{$\text{PSNR}_\text{512}$\xspace}
% \newcommand{\dinofid}{$\text{DiNOFID}_\text{50K}$\xspace}
% \newcommand{\dinofid}{$\text{FD}_\text{DINOv2}$\xspace}
% \newcommand{\dinofidfivek}{$\text{FD}_\text{DINOv2}^\text{5k}$\xspace}
\newcommand{\dinofid}{$\text{FDD}$\xspace}
\newcommand{\dinofidfivek}{$\text{FDD}_\text{5K}$\xspace}
% \newcommand{\inceptionscore}{$\text{IS}_\text{50K}$\xspace}
\newcommand{\inceptionscore}{IS\xspace}
% \newcommand{\cmsae}{$\text{CMS-AE}_{8 \times 16\times 16}$\xspace}
% \newcommand{\cmsaei}{$\text{CMS-AE}_{16\times 16}$\xspace}
\newcommand{\cmsae}{$\text{CMS-AE}_{V}$\xspace}
\newcommand{\cmsaei}{$\text{CMS-AE}_{I}$\xspace}
\newcommand{\cvae}{CV-AE\xspace}
\newcommand{\cvaefull}{CogVideoX-AE\xspace}
\newcommand{\dcae}{DC-AE\xspace}
\newcommand{\fluxae}{FluxAE\xspace}
\newcommand{\ltxae}{LTX-AE\xspace}

\newcommand{\ditstwo}{DiT-S$/$2}
\newcommand{\ditbtwo}{DiT-B$/$2}
\newcommand{\ditbone}{DiT-B$/$2}
\newcommand{\ditltwo}{DiT-L$/$2}
\newcommand{\ditxltwo}{DiT-XL$/$2}

\newcommand{\loss}{\mathcal{L}}

% Some links
\newcommand{\fluxaediffuserslink}{\url{https://huggingface.co/black-forest-labs/FLUX.1-dev/tree/main/vae}\xspace}

% People handles
\definecolor{DarkGreen}{RGB}{1,80,52}
\definecolor{DarkRed}{RGB}{150,30,30}
\newcommand{\rameen}[1]{\textbf{\textcolor{DarkGreen}{@Rameen:#1}}}
\newcommand{\willi}[1]{\textbf{\textcolor{red}{@Willi:#1}}}
\newcommand{\yanyu}[1]{\textbf{\textcolor{blue}{@Yanyu:#1}}}
\newcommand{\alex}[1]{\textbf{\textcolor{magenta}{@Alex:#1}}}
\newcommand{\sharath}[1]{\textbf{\textcolor{purple}{@Sharath:#1}}}
\newcommand{\ivan}[1]{\textbf{\textcolor{DarkRed}{@Ivan:#1}}}

% To avoid splitting footnote onto two pages
% \interfootnotelinepenalty=10000
%%%%%%%%%%%%%%%%%%%%%%%% CUSTOM COMMANDS END %%%%%%%%%%%%%%%%%%%%%%%%

% \newcommand{\papertitle}{Making Autoencoders Diffusable Again}
\newcommand{\papertitle}{Improving the Diffusability of Autoencoders}
% \newcommand{\papertitle}{Improving Autoencoders' Diffusability}
% \newcommand{\papertitle}{Towards Diffusable Autoencoders}

% The \icmltitle you define below is probably too long as a header.
% Therefore, a short form for the running title is supplied here:
\icmltitlerunning{\papertitle}

\begin{document}

\twocolumn[

% \icmltitle{Analyzing and Improving Video Autoencoders for Latent Diffusion Models}
% \icmltitle{Better Regularizations for Video Autoencoders}
% \icmltitle{On Video AutoEncoders for Latent Diffusion Models}
% \icmltitle{Principled Regularization for Video AutoEncoders}
% \icmltitle{Diffusable Video AutoEncoders}
% \icmltitle{Video AutoEncoders with Diffusable Latents}
\icmltitle{\papertitle}

% It is OKAY to include author information, even for blind
% submissions: the style file will automatically remove it for you
% unless you've provided the [accepted] option to the icml2025
% package.

% List of affiliations: The first argument should be a (short)
% identifier you will use later to specify author affiliations
% Academic affiliations should list Department, University, City, Region, Country
% Industry affiliations should list Company, City, Region, Country

% You can specify symbols, otherwise they are numbered in order.
% Ideally, you should not use this facility. Affiliations will be numbered
% in order of appearance and this is the preferred way.
\icmlsetsymbol{equal}{*}

\begin{icmlauthorlist}
\icmlauthor{Ivan Skorokhodov}{snap}
\icmlauthor{Sharath Girish}{snap}
\icmlauthor{Benran Hu}{snap,cmu}
\icmlauthor{Willi Menapace}{snap}
\icmlauthor{Yanyu Li}{snap}
\icmlauthor{Rameen Abdal}{snap}
\icmlauthor{Sergey Tulyakov}{snap}
\icmlauthor{Aliaksandr Siarohin}{snap}
%\icmlauthor{}{sch}
%\icmlauthor{}{sch}
\end{icmlauthorlist}

\icmlaffiliation{snap}{Snap Inc.}
\icmlaffiliation{cmu}{Carnegie Mellon University}
% \icmlaffiliation{sch}{School of ZZZ, Institute of WWW, Location, Country}

\icmlcorrespondingauthor{Ivan Skorokhodov}{iskorokhodov@gmail.com}
\icmlcorrespondingauthor{Aliaksandr Siarohin}{aliaksandr.siarohin@gmail.com}

% You may provide any keywords that you
% find helpful for describing your paper; these are used to populate
% the "keywords" metadata in the PDF but will not be shown in the document
\icmlkeywords{Diffusion Models, Latent Diffusion, Video Generation, Image Generation, AutoEncoders, VAE, Variational AutoEncoders, ICML}

\vskip 0.3in
]

% this must go after the closing bracket ] following \twocolumn[ ...

% This command actually creates the footnote in the first column
% listing the affiliations and the copyright notice.
% The command takes one argument, which is text to display at the start of the footnote.
% The \icmlEqualContribution command is standard text for equal contribution.
% Remove it (just {}) if you do not need this facility.

\printAffiliationsAndNotice{}  % leave blank if no need to mention equal contribution
% \printAffiliationsAndNotice{\icmlEqualContribution} % otherwise use the standard text.

\begin{abstract}
Out-of-distribution (OOD) detection and OOD generalization are widely studied in Deep Neural Networks (DNNs), yet their relationship remains poorly understood. We empirically show that the degree of Neural Collapse (NC) in a network layer is inversely related with these objectives: stronger NC improves OOD detection but degrades generalization, while weaker NC enhances generalization at the cost of detection. This trade-off suggests that a single feature space cannot simultaneously achieve both tasks. To address this, we develop a theoretical framework linking NC to OOD detection and generalization. We show that entropy regularization mitigates NC to improve generalization, while a fixed Simplex Equiangular Tight Frame (ETF) projector enforces NC for better detection. Based on these insights, we propose a method to control NC at different DNN layers. In experiments, our method excels at both tasks across OOD datasets and DNN architectures. 

\begin{comment}   

Out-of-distribution (OOD) detection and OOD generalization are critical for deploying machine learning models in real-world scenarios. While substantial progress has been made in addressing these problems independently, few works have attempted to tackle them jointly. However, existing methods often rely on auxiliary OOD training data and primarily focus on covariate-shifted OOD data that share labels with in-distribution (ID) data. In contrast, we tackle the more realistic and challenging task of jointly detecting and generalizing to semantic OOD data with disjoint labels from the ID data, without auxiliary OOD training data.
Achieving both objectives simultaneously is inherently difficult due to a fundamental conflict — OOD generalization requires enhanced transferability, while OOD detection necessitates the inhibition of transfer.
To address this, we leverage insights from neural collapse (NC) — a phenomenon in deep networks where top-layer representations suppress feature variability and adopt a Simplex Equiangular Tight Frame (ETF) structure, impairing transferability. By controlling NC, we unify OOD detection and generalization: preventing NC enhances OOD transfer while inducing NC improves OOD detection.
Our proposed method excels at both tasks across various OOD datasets and architectures. 

\end{comment}


\end{abstract}
\section{Introduction}

% State of the world (robots for creative activites)
The term ``robot,'' originally signifying `forced labor,' has long been associated with labor and work. Robots have demonstrated their utility in various automated productive and social contexts, where the primary goals are improving productivity, safety, and fostering social interactions with humans~\cite{simoes2022designing, weidemann2021role, honig2018understanding}. However, an increasing number of cases feature using of robots in creative settings. Unlike productive contexts, where the focus is on efficiency and task completion~\cite{arents2022smart}, or social contexts, where communication and trust are prioritized~\cite{nam2020trust, saunderson2019robots}, creative environments prioritize artistic innovation and expression~\cite{hsueh2024counts}. This shift fundamentally alters the dynamics of human-robot interaction, redefining the roles and expectations for both humans and robots.

For instance, robots’ social behaviors are leveraged to support the generation and expression of creative ideas~\cite{hu2021exploring, sandoval2022human, alves2020creativity}, and programmable robotic movements and trajectories are employed to inspire artistic activities such as sketching~\cite{lin2020your}. These studies often engage participants from creative fields who possess limited prior experience with robotics, and are typically conducted in short-term, experimental settings. Consequently, the findings from these studies remain constrained since much can be learned from professional practitioners' experiences to inform system design such as digital fabrication~\cite{hirsch2023nothing}. There is a notable gap in research examining the long-term, active, and practical experience of integrating robotic systems into the creative processes. As a result, the deeper insights into how robots facilitate and shape creative processes, beyond simply augmenting human creativity, remain underexplored. In this study, we aim to better understand the impacts of robots on creative processes and outcomes.

As early as Leonardo da Vinci's 16th century ``Automaton,'' artists have explored the creative affordances of robotic systems~\cite{shanken2002cybernetics, pagliarini2009development, jeon2017robotic}. The artistic creation process typically encompasses various stages, including the exploration of materials and techniques, ongoing experimentation and iteration, and the continual refinement of the artists' insights into their creative subjects~\cite{lewis2023art, sturdee2022state}. Therefore, investigating the artistic process involving robots offers an opportunity to gain deeper insights into robots' creative potential. Robotic art, in particular, provides a compelling case for this exploration.

We define robotic art as artworks that utilize robotic or automated machines to create artistic experiences and tangible artifacts. One example is robotic installation art, in which robots are programmed to follow specific rules that embody the artist’s expression (\autoref{fig:teaser} (a)). Another example is responsive art, in which robots react to their environment, with behaviors that change over time or in response to spectators (\autoref{fig:teaser} (b)). Additionally, there are robotic creators, which possess a degree of agency, allowing them to collaborate with human artists and produce works that extend beyond mere replication of human-created art (\autoref{fig:teaser} (c) and (d)). As such, robotic art becomes a rich case for exploring human-machine interactions in creative contexts. Gaining a deeper understanding of how robots facilitate artistic expression can provide insights for designing computing systems to support creative activities~\cite{gomez2021robot}.

% Therefore, we did...
We draw on semi-structured, in-depth interviews with renowned professional robotic artists to investigate the use of robots in artistic practice. Specifically, our goal is to understand how artistic exploration of robotic systems challenges conventional assumptions about the functions of robots, such as their roles in automating repetitive tasks or serving human needs. We also explore the implications of robots in the artistic process and examine how creativity may emerge within robotic art. To address these interrelated inquiries, our study focuses on the practice of robotic art, posing the research question: \textit{How do robotic artists utilize robots in their artistic practice?} We approach this inquiry through the perspectives and experiences of robotic artists, who creatively design, modify, and repurpose robotic systems for artistic expression and exploration.

% The key findings are...
Our findings highlight the social, material, and temporal dimensions of artists' practices that shape their creativity and artistic outcomes. The creation of robotic art is largely a social process, as artists receive both explicit and implicit feedback through the audience's reactions and reception of their work. Simultaneously, the embodiment and malfunctions inherent to robotic systems drive artistic experimentation. The temporal processes of creation and exhibition, beyond just the final product, further enhance the creative value. Our empirical analysis presents how creativity emerges through the interplay of social, material, and temporal interactions among artists, robots, audiences, and the environment.

% The contributions of this work are...
We make two main contributions to HCI in this study. 
First, we elucidate the interactive mechanisms among key actors---human creators, machines, audiences, and environments---within the practice of robotic art, a topic that remains underexplored in HCI. Our findings reveal the significance of sociality (e.g., interactions between artists and audiences), materiality (e.g., the embodiment and malfunctions of robots), and temporality (e.g., the processes of creation and exhibition) in shaping creative values. We propose that these three facets are central to the creative process and facilitate the emergence of creativity in robotic art.
Second, drawing from the findings, we offer implications for \textit{socially informed}, \textit{material-attentive}, and \textit{process-oriented} creation with computing systems. We suggest leveraging these three aspects to enhance creativity and the creative experience. Specifically, we discuss the value of incorporating implicit audience feedback, designing with technical malfunctions, and focusing on the post-creation process to foster alternative creative experiences with machines~\cite{alter2010designing, juarez2022glitch}.



%\subsection{End-User Programming}

%\kenneth{The way I like to think about Related Work is that this section should (sometimes subtly, not explicitly, but effectively!) answer some underlying questions that reviewers might want to ask. So, here we go:}\steven{sounds good!}

\subsection{Ways of Optimizing Prompts for LLMs}
%\subsection{Prompt Engineering and How Good Humans Are at It}
Prompts are the primary means by which users interact with, utilize, and instruct LLMs. 
Since the emergence of these models, researchers and developers have invested significant effort into understanding how to craft better prompts for more effective use. 

\paragraph{Automatic Prompt Optimization.}
Much of the prior work has focused on automatically optimizing prompts. 
A common theme across these studies is the use of gold-standard labels to guide the optimization process.
For example, \citet{pryzant2023automatic} introduced an automatic prompt optimization method inspired by gradient descent; 
\citet{manas2024improving} presented an approach that begins with a user prompt and iteratively generates revised prompts to maximize consistency between the generated image and prompt, without human intervention; 
\citet{wan2024teach} explored two types of prompt optimization, instruction and exemplar, and suggested that combining both can yield optimal results; 
\citet{sun2023autohint} combined zero-shot and few-shot learning to optimize prompts automatically; %eliminating the need for manual prompt engineering; 
and \citet{levi2024intent} improved prompt optimization through synthetic data generation and iterative refinement, focusing on aligning prompts with user intent by creating challenging boundary cases for iterative prompt refinement.
While these studies were interesting and relevant, they generally assumed the availability of gold-standard labels and did not address situations where labels are absent or where standards are constantly evolving.

\paragraph{User-Driven Prompt Optimization.}
In addition to automatic prompt optimization, some research has focused on human capabilities in optimizing prompts. 
\citet{zhou2023revisiting} found that manual prompting often outperforms automated methods in various scenarios; 
\citet{10.1145/3544548.3581388} discovered that people tend to design prompts opportunistically rather than systematically, which often leads to lower success rates. 
To the best of our knowledge, the most relevant prior work is by \citet{wang2024end}, who developed an iterative refinement system that enables users to prompt LLMs to build a personalized classifier for social media content. 
Their study explored three user strategies for improving prompts and measured their effectiveness. 
While conceptually related to our work, their focus was not on how users evolve their understanding and expectations when interacting with LLMs. 
Instead, participants labeled ground truth at the beginning of the study, prior to using the system.



%--------------------- dead kitten --------------
\begin{comment}
 





The most relevant prior work is by \citet{wang2024end}, who developed an iterative refinement system allowing users to prompt LLMs to build a personalized classifier for social media content.
While their work is closely related to ours in concept, their study did not focus on how users evolve their understanding and expectations while working with LLMs. 
Instead, participants labeled ground truth at the outset before using the system.


\kenneth{The key question for our paper is this: Did prior work try to measure users' prompt engineering performance *over multiple iterations*? (What do we know about human performance in prompt engineering?) I think you can maybe find some papers, especially papers for automatic prompt optimization like DSPy, measuring users' individual prompt's output accuracy (or MSE) or performance (e.g., BLEU in generation task), but it might be hard to find papers capture and measure *multiple iterations* from the same user for the same prompt.--This is the main argument for our paper: we did something that was hard and thus has not been done.}

\kenneth{Take a look at this survey paper:~\cite{chen2023unleashing}}



\steven{iterative tool involve human}
PromptIDE is an interactive tool that helps the experts to iteratively refine tools by providing various prompts, visualizing their performance on small validation datasets, and iterative optimizing them based on quantitative feedback~\cite{strobelt2022interactive}. \steven{gold label exists}

PromptAID is a visual analytics system that helps non-experts iteratively improve prompts through exploration, perturbation, testing, and refinement. It supports prompts through keyword adjustment, paraphrasing, and adding few-shot examples. \steven{has test dataset, it is a complex system}

\steven{automate prompting}
\citet{pryzant2023automatic} introduces an automatic prompt optimization prompt inspired by gradient descent. \steven{this fell into software designing, involve gold labels}

The study starts from a user prompt and iteratively generates revised prompts with the goal of maximizing a consistency score between the generated image and prompt without a human in the loop\cite{manas2024improving}\steven{without human involvement in the loop, gold labels}

\citet{zhou2023revisiting} found that manual prompting often performed better than automated methods in various steps. 

\cite{wan2024teach} explores the distinction between two types of prompt optimization: instruction optimizer and exemplar. This study suggested combining both approaches could lead to optimal results.

\cite{sun2023autohint} combines zero-shot and few-shot learning to optimize prompts automatically, without manual efforts in prompt engineering.

\cite{levi2024intent} improve prompt engineering optimization by synthetic data generation and iterative refinement, focusing on aligning prompts with user intent by generating challenging boundary cases and using these to refine the prompt iteratively.





\paragraph{Prompt Engineering Tools.}
\kenneth{After making the first point, we can have a follow-up paragraph to say that many tools were created to help people do prompt engineering (list a few and name their focuses), but again, they did not focus on measuring how good humans are in prompt engineering--- Of course, there could be an argument that suggests: no matter how good you are, you will always need some tool. It is true---for example, ChainForge basically create a easy-to-use UI that make things easier, which is not really about accuracy---But for annotation tasks, performance is still critical and it is always good to know how well human did, almost like many AI leaderboard has various "human" performance for comparison.}
PromptMaker, a platform for rapidly prototyping new ML models using prompt-based programming, was difficult to evaluate their prompts systematically~\cite{10.1145/3491101.3503564}.

\cite{arawjo2024chainforge}  is an Open-source visual toolkit for prompt engineering and on-demand hypothesis testing of text-generation LLMs.

 promptfoo is test-driven LLM development, not trial-and-error, producing matrix views that let you quickly evaluate outputs across many prompts~\cite{webster2023promptfoo}.

\cite{madaan2024self} introduces a method that LLM iterative improve their output by using their own feedback, without external supervision. 

\saniya{austin etal points:
1. used only COPRO, evaluation criteria utilized a custom LLM-as-a-judge metric. The paper showed that their automated prompt optimizer worked better tha DSPy }
   
\end{comment}


\subsection{Tools for Prompt Engineering}
With the advances in LLMs, numerous tools have been developed to assist with prompt engineering. 
Most of these tools follow a software-engineering paradigm, where testing (such as unit tests or integration tests) is a central concept, and thus often assume the existence of gold-standard labels.
For example, PromptIDE is an interactive tool that helps experts iteratively refine prompts by providing various inputs, visualizing their performance on small validation datasets, and optimizing them based on quantitative feedback~\cite{strobelt2022interactive}; 
PromptAid is a visual analytics system for interactively creating, refining, testing, and iterating prompts while tracking accuracy changes~\cite{mishra2023promptaid};
%It allows users to adjust prompts through keyword modifications, paraphrasing, and adding few-shot examples; 
ChainForge is an open-source visual toolkit for prompt engineering and on-demand hypothesis testing of text-generation LLMs~\cite{arawjo2024chainforge};
and, promptfoo applies a test-driven approach to LLM development, producing matrix views that enable quick evaluation of outputs across multiple prompts~\cite{webster2023promptfoo}.
While these tools are inspiring and valuable, the scenarios we focus on do not rely on the constant availability of gold labels.

%\cite{mishra2023promptaid}


\begin{comment}






\kenneth{In here, we want to answer this questions: Why do we need to built \system? Can't we just use some existing tools??? The underlying answer could be: all the tools, including the one we mentioned in previous subsection, were not really aiming for ``general users'' and only thing general users can reliably use is probably chat interface come with ChatGPT etc.}

\citet{10.1145/3544548.3581388} mentioned that people tended to design prompts opportunistically, not systematically, which resulted in less success. \system provides a systematic process for composing and refining prompts, allowing non-expert users to adapt to the prompt creation process effortlessly.

\saniya{Amy Zhang points:
\newline 1. Accuracy didnot improve; reported improvements in recall
\newline 2. Observed that humans are pretty bad at being consistent
\newline 3. Quoted  Miles Turpin, Julian Michael, Ethan Perez, and Samuel Bowman. 2024. Language models don't always say what they think: unfaithful explanations
in chain-of-thought prompting. Advances in Neural Information Processing Systems 36 (2024).
Han Wang, Ming Shan Hee, Md Rabiul Awal, Kenny Tsu Wei Choo, and Roy Ka-Wei Lee. 2023. Evaluating GPT-3 Generated Explanations for
Hateful Content Moderation. arXiv:2305.17680 [cs.CL] for not using LLM prompt explanations
\newline 4. They had a bigger training set of around 700 examples: paper excerpt: "This process resulted in a balanced dataset of 800 comments. We randomly divided our dataset into a training dataset and a test dataset of 100 examples for each participant. The training dataset was used to help participants create their classiiers, whereas the test dataset was labeled by participants and used to evaluate their created classiiers."
}
    
\end{comment}

\subsection{Human-LLM Collaborative Data Annotation}
%Another relevant area of research involves using LLMs for data annotation. 
Beyond simply treating LLMs as automatic labelers---common in countless NLP projects~\cite{tan2024large}---a growing body of work explores how to combine human and LLM efforts to achieve better annotation outcomes, such as improved accuracy or speed.
Even as LLMs outperform humans in many labeling tasks, human-AI collaboration often produces better results than either alone~\cite{vaccaro2024combinations}.
For example, \citet{kim2024meganno+} introduced a human-LLM collaborative annotation system where LLMs handle bulk annotation tasks, while humans selectively verify and refine the annotations. 
%\steven{However, this system was limited to deployment within Jupyter Notebook, lacking an end-to-end solution. This design imposed significant barriers, as it required users to possess technical expertise for system setup before using the tool, limiting accessibility and scalability in non-technical domains.}
\citet{goel2023llms} proposed an approach that combines LLMs with human expertise to efficiently generate ground truth labels for medical text annotation.
Additionally, \citet{10.1145/3613904.3642834} demonstrated how aggregating crowd workers' labels with GPT-4's output can achieve higher labeling accuracy than either source alone.
These studies generally aim to split the workflow of data labeling between humans and LLMs in a smart way, making the task more effective or efficient. 

In contrast, our work does not focus on dividing or combining the workload, but on how humans can teach LLMs---through prompt refinement---to better label the specific type of data.
Few prior studies have emphasized iterative prompt refinement in human-LLM collaborative data annotation.
For example, \citet{liu2024harnessing} developed a workflow for video content analysis, refining prompts to improve LLM-generated annotations and align them with human judgment.
Additionally, \citet{zhang2023llmaaa} proposed LLMAAA, which uses LLMs as annotators in a feedback loop to label data efficiently.
Their study shows that poorly designed prompts result in subpar performance, especially in complex tasks. %while incorporating demonstrations and aligning label descriptions with natural language significantly enhances accuracy and reliability.
Our work advances this relatively understudied area of human-LLM collaborative annotation research.

%----------------------------- dead kitten --------------------------------

\begin{comment}








\steven{\citet{vaccaro2024combinations} emphaized that designing innovative processes for integrating humans and AI is as critical as developing advanced AI technologies. This aligns with the need for LLM-powered systems that iteratively guide AI outputs to meet user-specific standards, prioritizing effective collaboration between users and AI systems.}

\steven{\citet{liyanage2024gpt} found that GPT-4, using few-shot, zero-shot, and Chain-of-Thoughts (CoT) prompting techniques, could not outperform models fine-tuned on human-labeled data. Among these, the few-shot approach exhibited the highest degree of similarity to human annotations. However, in scenarios where gold labels are unavailable, fine-tuning is not applicable, and alternative methods must be explored.}

\steven{\citet{liu2024harnessing} developed a workflow for video content analysis, iteratively crafting prompts to enhance LLMs' ability to generate structured annotations and comprehensive explanations that aligned with human judgment. }

\steven{\citet{zamfirescu2023herding} found that while prompts can effectively address most UX goals, they struggle with nuanced, edge-case, or spontaneous interactions. The study highlights that the effectiveness of each instruction in the prompt is highly sensitive to its phrasing and location. Additionally, highly prescriptive prompts, though reliable, limited the spontaneity and flexibility of GPT responses.
In our system, users are only required to provide task information—such as task descriptions, rules, and examples—to construct instructions, allowing for greater flexibility in accommodating diverse task requirements..}

\steven{\citet{guyre2024prompt} illustrates how prompt engineering can empower non-experts to design tailored conversational agents by iteratively refining prompts and infusing domain-specific knowledge. Their study emphasizes democratizing chatbot development, allowing users to align AI behavior with their specific goals and values.}

\steven{\citet{zhang2023llmaaa} proposes LLMAAA that leverages LLMs as Active Annotators in a feedback loop to efficiently annotate data. The study highlights that poorly designed prompts lead to suboptimal performance by LLM annotators, particularly in complex or domain-specific tasks. However, incorporating demonstrations and aligning label descriptions with natural language significantly enhances annotation accuracy and reliability.}

%\kenneth{Here, we then answer this question: Did people create ANYTHING to support LLM-powered data annotation? There are two parts of the answer to this: 1) Many or even most papers, including our CHI paper last year, focus on the labeling performance of LLMs, for example, as compared to crowdsourcing. They did not focus on the UI aspect of it. 2) Some prompt chaining tools, like ChainForge, can support workflow like this, but (a) hey do not focus on data annotation in particular so some functions are missing, like data resampling, and (b) more importantly, they do not aim to support general users. Most of them expect you to know some programming, e.g., ChainForge clearly say it's a visual programming tool. They're not really aiming for generic users.}


\cite{kim2024meganno+} introduced a human-LLM collaborative annotation system that allows LLM to handle bulk annotation tasks while humans verify selectively to refine annotation. 

\cite{goel2023llms} introduced an approach that combines LLM wth human expertise to create an efficient method for generating
ground truth labels for medical text annotation.


\cite{shankar2024validates} introduced a tool, EvalGen, to address the challenge of validating LLM. 
EvalGen helps users design evaluation criteria for LLM outputs and align that evaluation with human preferences through a mixed-initiative system.
A key finding is the concept of criteria drift, where users modify their evaluation standards while grading outputs. 


\cite{brade2023promptify} Promptify utilizes an LLM-powered suggestion engine to help users quickly explore and craft diverse prompts for text-to-image generation tasks.

    
\end{comment}


%\subsection{Survey Study in Data Annotation}
%\steven{
We conducted a survey study to investigate how individuals interact with LLMs and utilize gold-standard labels in the data annotation process. 
The participants primarily represent roles in research, machine learning engineering, and software development. \\
\textbf{Workflows: }Participants described diverse workflows for integrating LLMs into data annotation process, highlighting a common iterative and human-in-the-loop approach. \textbf{Most workflows begin with manual annotation of a small subset of data to establish a baseline.} Participants then employ prompt engineering, iteratively refining LLM prompts by evaluating their performance against the manually annotated subset. \\
Once refined, the prompts are used to label larger datasets, with participants using tools or manual checks to review the LLM's annotations and identify any invalid labels. The process is typically concluded with a thorough manual verification of the dataset. \\
One participant mentioned they manually tabulate data points along with their descriptions. \\
\textbf{Initialize Prompting: }Most participants use their pre-defined prompts to initialized the annotation on their known tasks. 
For new tasks, one participant mentioned that they initialize the annotation process with LLMs by starting with a clear problem definition and iteratively refining a classification-based approach. For less familiar tasks, some participants may seek suggestions from the LLM to guide the initial setup.
\textbf{Revising Prompt: } Participants use a small dataset to finetune the prompt. They address issues by adding rules or context examples to tackle failure cases. When inconsistencies or error arise, they revisit and recheck the manually tagged dataset to improve performance. Some participants also engage the LLM by asking questions about data points and their descriptions, retraining to against inconsistencies to minimize hallucinations and enhance annotation reliability.
}

\subsection{Gold-Standard Labels in Annotation Tasks}\label{sec:related-work-gold-label}
Decades of research have shown that gold-standard labels play a critical role in quality control for data annotation pipelines~\cite{han2020crowd,gadiraju2015training,le2010ensuring,doroudi2016toward,hettiachchi2021challenge}.
Embedding items with known labels into the data annotation process allows requesters to reliably capture quality signals, 
such as workers' level of expertise~\cite{abraham2016many, abassi2019worker, yang2018improving} %\kenneth{TODO: Add refs about using gold labels to decide workers' expertise level}\steven{added}
or attentiveness to tasks~\cite{hettiachchi2021challenge, oleson2011programmatic}. %\kenneth{TODO: Add refs about using gold labels to do attention checks for workers}\steven{added}
These insights enable requesters to take appropriate actions, such as 
retraining annotators~\cite{le2010ensuring, doroudi2016toward,hettiachchi2021challenge}, %\kenneth{TODO: Add refs about retraining workers}\steven{added}
removing low-performing workers~\cite{10.1145/3613904.3642834, snow2008cheap,downs2010your,le2010ensuring}, %\kenneth{TODO: Add refs about removing or blocking low-performing workers}\steven{added}
or identifying potential issues in the annotation interfaces~\cite{toomim2011utility,10.1145/3613904.3642834, rahmanian2014user, komarov2013crowdsourcing}. %\kenneth{TODO: Add refs for crowd worker interfaces. At least cite: Toomim, M., Kriplean, T., Pörtner, C., \& Landay, J. (2011, May). Utility of human-computer interactions: Toward a science of preference measurement. In Proceedings of the SIGCHI Conference on Human Factors in Computing Systems (pp. 2275-2284).}\steven{added}
Gold labels are also beneficial for requesters during post-annotation data processing. 
They can be used to weight labels from different workers in label aggregation~\cite{abassi2017gold,abassi2019worker}, %\kenneth{TODO: Add label aggregation methods that use gold labels particularly to weight different workers}\steven{added}
improve label aggregation strategies~\cite{khattak2011quality, snow2008cheap},  %\kenneth{TODO: Add label aggregation methods that learn whatever from gold labels}\steven{added}
or 
exclude unreliable workers' outputs entirely~\cite{abassi2019worker}. %\kenneth{TODO: Cite ref using gold labels to remove workers from label aggregation}\steven{added}
Beyond requesters, gold labels are also beneficial for data labelers like crowd workers. 
Gold labels can be used to train workers~\cite{doroudi2016toward, le2010ensuring, gadiraju2015training,han2020crowd}, %\kenneth{TODO: Cite ref that uses gold labels for worker training}\steven{added}
provide real-time feedback to help them recalibrate their understanding of the task~\cite{le2010ensuring,hettiachchi2021challenge}, %\kenneth{TODO: Cite the visible gold paper from Amazon}\steven{added}
or remind them to pay more attention~\cite{ hettiachchi2021challenge,oleson2011programmatic}. %\kenneth{TODO: Cite attention check papers}\steven{amazon paper also warn workers in real time}

While gold labels are useful for quality control, as stated in the Introduction (Section~\ref{sec:intro}), %\kenneth{TODO: Update references}\steven{done}
they are not always available in real-world scenarios due to constraints such as data privacy or the cost of gathering gold labels~\cite{liu2019deep, yang2019evaluating, oikarinen2021detecting, slote2024unlocking}.
To address these challenges, researchers have developed methods to generate (approximations of) quality signals without gold labels. 
In the realm of LLM-powered data annotation, for instance, CoPrompter evaluates how well an LLM's output aligns with user-specified requirements as a feedback mechanism~\cite{joshi2024coprompter}. %\kenneth{TODO: Cite: Joshi, I., Shahid, S., Venneti, S., Vasu, M., Zheng, Y., Li, Y., ... \& Chan, G. Y. Y. (2024). CoPrompter: User-Centric Evaluation of LLM Instruction Alignment for Improved Prompt Engineering. arXiv preprint arXiv:2411.06099.}\steven{added}
Other studies also leverage the stability~\cite{chiang2023can} %\kenneth{TODO: Add ref}\steven{added}
%chiang2023can found LLM evaluation are stable over different formatting
or confidence~\cite{gligoric2024can} %\kenneth{TODO: Add ref}\steven{added}
%gligoric2024can introduce CONFIDENCEDRIVEN INFERENCE: a method that combines LLM annotations and LLM confidence indicators to strategically select which human annotations should be collected
of LLM outputs to infer quality signals.
%Our research investigates how effectively humans can iteratively refine prompts to guide LLMs in labeling data when gold-standard labels are unavailable, providing alternative quality signals.
Our research examines how effectively humans can refine prompts to guide LLMs in labeling data without gold-standard labels, providing insights into human prompting capabilities in the absence of reliable guidance signals.










%------------- dead kitten -------------
\begin{comment}




\kenneth{------------------------KENNETH IS WORKING HERE----------------------}



Gold-standard labels are widely used for quality control and crowd worker training~\cite{doroudi2016toward, gadiraju2015training,le2010ensuring,hettiachchi2021challenge}. For example, \citet{hettiachchi2021challenge} demonstrated that incorporating visible gold questions -- where annotators receive periodic feedback based on pre-labeled gold-standard examples -- could improve their work quality. 
Similarly, \citet{doroudi2016toward} found that providing expert examples was the most effective method of training for crowd workers and can help workers avoid specific types of incorrect responses. 
Additionally, \citet{le2010ensuring} employed dynamic learning systems that leveraged gold-standard labels to deliver real-time feedback and improve worker outcomes.
These studies, however, predominantly address the annotators' perspective -- workers who adhere to predefined guidelines and follow established standards.
While annotators are crucial components of the task pipeline, our study shifts focus to the requesters' perspective, those responsible for task design and pipeline management.
For requesters, gold-standard labels serve as signals to assess worker performance and refine training processes, thereby improving the overall quality of the entire pipeline.
Critically, the aforementioned studies assume the availability of gold-standard labels, typically under controlled experimental settings. 
In real-world scenarios, this assumption often does not hold due to constraints such as data privacy, security concerns, or the absence of labeled data~\cite{liu2019deep, yang2019evaluating, oikarinen2021detecting, slote2024unlocking}. 
To address this gap, our research explores settings where predefined gold-standard labels are unavailable. 
We designed a novel framework for requesters to iteratively develop and evolve their labeling standards through interactions with LLMs. 
By bridging the divide between controlled experiments and real-world challenges, our work highlights the potential of adaptive, LLM-driven approaches for dynamic task management without reliance on predefined gold-standard labels.

\steven{\citet{hettiachchi2021challenge} demonstrated that incorporating visible gold questions -- where annotators receive periodic feedback based on pre-labeled gold-standard examples -- could improve their work quality. 
Their study leveraged gold-standard labels to train crowd workers to align with pre-defined standards, effectively guiding annotators thorugh examples and feedback. 
While this approach focues on improving labeling quality at the annotator level, our work shifts the focus to requester and researcher perspective. Instead solely training labelers to meet pre-existing standards, we emphasize the broader implications of designing system in the entire labeling process, particularly in context involving dynamic or subjective tasks. \citet{gadiraju2015training} showed that training workers with gold labels can enhance accuracy and decrease response times. \citet{han2020crowd} used gold standard labels to guide crowd workers in revising incorrect judgments to align with predefined standards. 
}

\steven{
\citet{doroudi2016toward} found that providing expert examples was the most effective method of training for crowd workers. In our study, however, each participant was treated as an individual researcher rather than a crowd worker. While this finding underscores the value of providing gold labels to improve language model performance, it does not directly highlight their significance for researchers. Furthermore, \citet{doroudi2016toward} observed that gold standard labels help workers avoid specific types of incorrect responses. 
In contrast, our task is subjective, with participants’ standards potentially shifting across iterations. Introducing pre-set gold standard labels could enforce a uniform standard across each participant, which might not align with the iterative and subjective nature of our study
}

\steven{\citet{gadiraju2015training} showed that training workers with gold labels can enhance accuracy and decrease response times. [They were still focusing on crowd worker level.] }

\steven{\citet{han2020crowd} used gold standard labels for quality control and to guide crowd workers in revising incorrect judgments to align with predefined standards.}

\steven{\citet{le2010ensuring} employed gold standard labels within a dynamic learning environment that provided real-time feedback to train workers. However, the selection of specific examples for training could influence worker responses, potentially introducing bias in their judgments. [This is why we implemented a random sample in our system]}


\steven{\citet{liu2019deep} developed a HITL system that kept model upgrading with progressively collected data without having a pre-labeled data. [\textbf{they used 30 samples per iteration.} -add to justification for 10 and 50 instances.]}

\steven{\citet{wall2019using} found that end-users could build models without using expert patterns that have comparable performance to those who built by expert. This approach was required more effort and more mental demand than those who received guidance.}

\kenneth{TODO: Add references to every part of this paragraph.}
Decades of research have established that gold-standard labels are highly effective for quality control in data annotation~\cite{han2020crowd,gadiraju2015training,le2010ensuring,doroudi2016toward,hettiachchi2021challenge}. 
Embedding items with known labels into the annotation process enables requesters to monitor annotator or data quality and take actions such as retraining annotators, removing them from the pipeline, or reducing their weight in label aggregation. 
Beyond requesters, gold labels also allow for real-time feedback to workers, helping them recalibrate their understanding of the task or focus more carefully.
While gold labels are widely recognized as useful for quality control, most research assumes their availability.
However, as discussed in our Introduction (Section X), this assumption does not necessarily hold in real-world scenarios due to constraints such as data privacy or the cost of gathering gold labels~\cite{liu2019deep, yang2019evaluating, oikarinen2021detecting, slote2024unlocking}. 
To address these challenges, researchers have developed systems to provide proxy quality signals without gold labels. 
For instance, CoPrompter evaluates how well an LLM's output aligns with user-specified requirements as a feedback mechanism. 
Other studies leverage the stability or confidence of LLM outputs to infer quality signals.
Our research investigates how effectively humans can refine prompts to guide LLMs when gold-standard labels are unavailable.
    
\end{comment}

%\subsection{Explanations in AI-Assisted Tools}


%\subsection{Variables in System}
%There are lots of variables in a system could impact user's performance. 
\citet{kulesza2012tell} suggested that the more users understand the underlying system, the more effectively they can control it. 

\steven{\citet{lee2024clarify} introduces a system that allows non-expert users to train and correct models by directly interact with model using natural languages. In each iteration, the system will use similarity score between user description and image and display images above a threshold. The system will also provide 0-1 score indicating how well description separates the error cases from the correct prediction. Basically using metrics to guide user.
It does not mentioned about the sample size selection.}

\steven{[Data Instance:] In active learning, the goal is to minimize the amount of interaction required by users by querying the most important information~\cite{bernard2018vial}. [This can be used to justify why we increase to 50, to ensure the diversity. We cannot deploy algorithms to find most representative data sample because of the technical limitation of Google App Script]}

\steven{[Data Instance:] \citet{vermetten2022analyzing} investigated how the number of sample size affects the reliability of algorithm comparisons in iterative optimization. The study found that small sample sizes lead to high variability in performance estimates and larger sample sizes could decrease the impact of outliers. The performance could loss due to small samples and increasing sample size consistently improves reliability. }

\steven{\citet{purohit2018ranking} suggested capping the maximum number of annotation tasks assigned per unit of time to manage workload effectively to mitigate annotator burnout.}

\steven{\citet{pandey2022modeling} mentioned annotator can develop a mental representation of a concept by seeing a sufficient number of examples.}

\steven{\citet{wang2016human} limited users to verify the top-50 in each round, where users did binary classification on whether image was match or not.}

\steven{[explanation]\citet{kulesza2015principles} presents a system that explains the reason behind each prediction for users to better understand the system's logic to tailor the system toward their needs. In the system, users will modify feature weights within the model. n our LLM-powered system, users need to use natural language to guide the system. However, this can be more challenging because large models are less responsive to prompt variations compared to smaller models~\cite{zhuo2024prosa}.}

\steven{The more users understand the underlying system, the more effectively they can control it~\cite{kulesza2012tell}.}

\steven{\citet{teso2023leveraging} discusses a general framework for incorporating explanations into interactive machine learning. Users can get a better understanding of the machine's logic by observing the machine's explanations. [In LLM system, the explanation is the supporting argument for selecting a label.] Once understanding the bugs and limitation, users could modify the algorithm to correct flaws~\cite{kulesza2015principles}. [In our case, user cannot directly modify LLMs but only provide natural language to guide them. Also, subjective tasks does not have universal correct answers, where users need to provide their own standards to steer LLMs. ] }

\steven{[Task Difficulty:] 
A task being too difficult can frustrate users~\cite{zheng2022virtual}, particularly when exceeding their skill level, and a task being to easy can lead to boredom~\cite{zhang2021personalized}.
  These study focused on the impacts of difficulty on users' performance on a pre-defined task. However, in our study, our work prioritizes the dynamics of human-LLM interaction, emphasizing how effective humans could guide LLMs to align with their standard. In this context, the difficulty level of the task itself is less critical, as our primary objective is to assess the effectiveness of human guidance, regardless of the inherent complexity of the task.}


\steven{[task type:]\citet{cayir2016study} found the complexity and definition of a task significantly influence user performance. }

\steven{[task type:] \citet{hettiachchi2022survey} discusses different task assignment methods, including the modeling of worker performance and the impact of task heterogeneity on assignment strategies.
\citet{zhen2021crowdsourcing} provides a detailed exploration of task assignment challenges, task types, and their effects on worker performance and task outcomes. 
}
% \steven{ending of related word}We wanted to design a system to bridge the gap of xxxx: a graphical interface implemented on Google Sheet add-on, generalizing to single-class data annotation tasks, without requiring extensive knowledge of programming and system configuration. By combining the widespread familiarity and advanced features of Google Sheets with large-scale data annotation and iteration tracking, we aimed to make it easier for people to experiment with and benefit from LLMs.
\vspace{-5pt}
\section{Method}
\label{sec:method}
\begin{figure*}[t]
\begin{center}
\includegraphics[width=.85\linewidth]{fig_overview_v3.pdf}
\end{center}
\caption{
FastAtlas Overview: In each frame, we compute charts spanning fully or partially visible triangles (a), determine texture space bounding boxes for the visible portions of the view-space projections of each chart, and tightly pack these boxes into atlases (b, here $2K \times 2K$). We simultaneously bijectively parameterize and shade the charts into their atlas boxes, obtaining high quality texture space shading (c), and use this shading to render the shaded frames (d).}
\label{fig:overview}
\label{fig:alg_overview}
\end{figure*}

\section{Overview}
\label{sec:overview}
Our work has two core contributions: a real-time, GPU-based algorithm for tight packing of general parameterized charts into compact atlases; and a real-time TSS method that
utilizes this packing.  

\paragraph*{FastAtlas Packing.}
FastAtlas runs entirely on the GPU as a series of compute shaders. It takes the bounding boxes of parameterized charts as input, and packs them into an atlas (Fig~\ref{fig:overview}b, Sec.~\ref{sec:pack}). As such, the only input it requires are the dimensions of the bounding boxes.
Its outputs are deterministic; identical input charts are packed into identical atlases. This is critical for TSS and similar applications, as it ensures that consecutive frames taken from the same camera view have the same shading. Even minute shading differences across such frames can cause sampling jitter, leading to undesirable flicker \cite{baker2012rock}. 
While prior methods such as \cite{mueller2018shading,hladky2019tessellated,hladky2021snakebinning,Neff2022MSA} cap the dimensions of the charts that can be packed as-is for a given atlas size, and scale down all charts that exceed these dimensions, we scale all charts by the same factor, and do so only when strictly necessary to achieve packing success (Figs~\ref{fig:atlas},~\ref{fig:sas_issues}). 

\paragraph*{TSS using FastAtlas.}
Our end-to-end TSS atlas generation method combines the packing method above with a novel approach for computing seamless per-frame charts. 
We define our charts as the connected components of the visible surfaces in each frame (Fig.~\ref{fig:overview}a), and efficiently compute them using a parallel union-find algorithm (Sec.~\ref{sec:visible}). Since the boundaries of these charts coincide with the contours of the rendered surface, they are {\em invisible} to the viewer. This approach 
eliminates the artifacts caused by shading discontinuities along visible seams (Fig.~\ref{fig:seams}). 

\begin{parWithWrapFigure}
\begin{wrapfigure}{l}{.27\columnwidth}%
\includegraphics[width=\linewidth]{fig_inset_view_plane.pdf}%
\end{wrapfigure}
We bijectively parametrize the {\em visible portions} of our charts by projecting them to view space (inset). This maps a constant number of texels to each pixel in the final rendered output, evenly distributing residual undersampling error across all image pixels. While conceptually straightforward, efficiently parameterizing charts containing partially visible triangles using viewspace projection is non-trivial, as the visible portions may no longer be triangular (e.g. green triangle in the inset); applying naive projection to triangles with vertices behind the camera may produce ill-posed results. Clipping triangles before projection is both computationally expensive and significantly complicates downstream operations. We avoid explicit clipping by observing that all that is required for atlas packing is the dimensions of, potentially conservative, bounding boxes of these projected visible portions. We compute such bounding boxes without explicit chart clipping by adapting a conservative screen coverage estimator \shortcite{Blinn:CalculatingScreenCoverage} (Sec.~\ref{sec:box}). We then pack the computed boxes using FastAtlas. 
\end{parWithWrapFigure}

Finally, we shade the visible portion of each chart into its corresponding atlas bounding box (Fig~\ref{fig:overview}c). 
The resulting texture is then used during rasterization as a standard texture map (Fig. ~\ref{fig:overview}d). 
Our framework is compatible with all existing approaches for texture space shading, including forward shading, raytraced illumination, or deferred shading in texture space \cite{baker:2016}. In the examples shown, we use the standard forward shading based rendering pipeline included in the G3D Innovation Engine \cite{G3D17}, a commercial grade renderer.


Our goal is to increase the robustness of T2I models, particularly with rare or unseen concepts, which they struggle to generate. To do so, we investigate a retrieval-augmented generation approach, through which we dynamically select images that can provide the model with missing visual cues. Importantly, we focus on models that were not trained for RAG, and show that existing image conditioning tools can be leveraged to support RAG post-hoc.
As depicted in \cref{fig:overview}, given a text prompt and a T2I generative model, we start by generating an image with the given prompt. Then, we query a VLM with the image, and ask it to decide if the image matches the prompt. If it does not, we aim to retrieve images representing the concepts that are missing from the image, and provide them as additional context to the model to guide it toward better alignment with the prompt.
In the following sections, we describe our method by answering key questions:
(1) How do we know which images to retrieve? 
(2) How can we retrieve the required images? 
and (3) How can we use the retrieved images for unknown concept generation?
By answering these questions, we achieve our goal of generating new concepts that the model struggles to generate on its own.

\vspace{-3pt}
\subsection{Which images to retrieve?}
The amount of images we can pass to a model is limited, hence we need to decide which images to pass as references to guide the generation of a base model. As T2I models are already capable of generating many concepts successfully, an efficient strategy would be passing only concepts they struggle to generate as references, and not all the concepts in a prompt.
To find the challenging concepts,
we utilize a VLM and apply a step-by-step method, as depicted in the bottom part of \cref{fig:overview}. First, we generate an initial image with a T2I model. Then, we provide the VLM with the initial prompt and image, and ask it if they match. If not, we ask the VLM to identify missing concepts and
focus on content and style, since these are easy to convey through visual cues.
As demonstrated in \cref{tab:ablations}, empirical experiments show that image retrieval from detailed image captions yields better results than retrieval from brief, generic concept descriptions.
Therefore, after identifying the missing concepts, we ask the VLM to suggest detailed image captions for images that describe each of the concepts. 

\vspace{-4pt}
\subsubsection{Error Handling}
\label{subsec:err_hand}

The VLM may sometimes fail to identify the missing concepts in an image, and will respond that it is ``unable to respond''. In these rare cases, we allow up to 3 query repetitions, while increasing the query temperature in each repetition. Increasing the temperature allows for more diverse responses by encouraging the model to sample less probable words.
In most cases, using our suggested step-by-step method yields better results than retrieving images directly from the given prompt (see 
\cref{subsec:ablations}).
However, if the VLM still fails to identify the missing concepts after multiple attempts, we fall back to retrieving images directly from the prompt, as it usually means the VLM does not know what is the meaning of the prompt.

The used prompts can be found in \cref{app:prompts}.
Next, we turn to retrieve images based on the acquired image captions.

\vspace{-3pt}
\subsection{How to retrieve the required images?}

Given $n$ image captions, our goal is to retrieve the images that are most similar to these captions from a dataset. 
To retrieve images matching a given image caption, we compare the caption to all the images in the dataset using a text-image similarity metric and retrieve the top $k$ most similar images.
Text-to-image retrieval is an active research field~\cite{radford2021learning, zhai2023sigmoid, ray2024cola, vendrowinquire}, where no single method is perfect.
Retrieval is especially hard when the dataset does not contain an exact match to the query \cite{biswas2024efficient} or when the task is fine-grained retrieval, that depends on subtle details~\cite{wei2022fine}.
Hence, a common retrieval workflow is to first retrieve image candidates using pre-computed embeddings, and then re-rank the retrieved candidates using a different, often more expensive but accurate, method \cite{vendrowinquire}.
Following this workflow, we experimented with cosine similarity over different embeddings, and with multiple re-ranking methods of reference candidates.
Although re-ranking sometimes yields better results compared to simply using cosine similarity between CLIP~\cite{radford2021learning} embeddings, the difference was not significant in most of our experiments. Therefore, for simplicity, we use cosine similarity between CLIP embeddings as our similarity metric (see \cref{tab:sim_metrics}, \cref{subsec:ablations} for more details about our experiments with different similarity metrics).

\vspace{-3pt}
\subsection{How to use the retrieved images?}
Putting it all together, after retrieving relevant images, all that is left to do is to use them as context so they are beneficial for the model.
We experimented with two types of models; models that are trained to receive images as input in addition to text and have ICL capabilities (e.g., OmniGen~\cite{xiao2024omnigen}), and T2I models augmented with an image encoder in post-training (e.g., SDXL~\cite{podellsdxl} with IP-adapter~\cite{ye2023ip}).
As the first model type has ICL capabilities, we can supply the retrieved images as examples that it can learn from, by adjusting the original prompt.
Although the second model type lacks true ICL capabilities, it offers image-based control functionalities, which we can leverage for applying RAG over it with our method.
Hence, for both model types, we augment the input prompt to contain a reference of the retrieved images as examples.
Formally, given a prompt $p$, $n$ concepts, and $k$ compatible images for each concept, we use the following template to create a new prompt:
``According to these examples of 
$\mathord{<}c_1\mathord{>:<}img_{1,1}\mathord{>}, ... , \mathord{<}img_{1,k}\mathord{>}, ... , \mathord{<}c_n\mathord{>:<}img_{n,1}\mathord{>}, ... , $
$\mathord{<}img_{n,k}\mathord{>}$,
generate $\mathord{<}p\mathord{>}$'', 
where $c_i$ for $i\in{[1,n]}$ is a compatible image caption of the image $\mathord{<}img_{i,j}\mathord{>},  j\in{[1,k]}$. 

This prompt allows models to learn missing concepts from the images, guiding them to generate the required result. 

\textbf{Personalized Generation}: 
For models that support multiple input images, we can apply our method for personalized generation as well, to generate rare concept combinations with personal concepts. In this case, we use one image for personal content, and 1+ other reference images for missing concepts. For example, given an image of a specific cat, we can generate diverse images of it, ranging from a mug featuring the cat to a lego of it or atypical situations like the cat writing code or teaching a classroom of dogs (\cref{fig:personalization}).
\vspace{-2pt}
\begin{figure}[htp]
  \centering
   \includegraphics[width=\linewidth]{Assets/personalization.pdf}
   \caption{\textbf{Personalized generation example.}
   \emph{ImageRAG} can work in parallel with personalization methods and enhance their capabilities. For example, although OmniGen can generate images of a subject based on an image, it struggles to generate some concepts. Using references retrieved by our method, it can generate the required result.
}
   \label{fig:personalization}\vspace{-10pt}
\end{figure}
\section{Empirical Evaluation}
\begin{table*}[!ht]
    \centering
    \resizebox{0.88\textwidth}{!}{    
    \begin{tabular}{r|cccccc|cccccc}
        \toprule 
        & \multicolumn{6}{c}{\textbf{LLaVA-1.5-7B}} & \multicolumn{6}{c}{\textbf{LLaVA-1.5-13B}} \\ 
        \cmidrule(lr){2-7}\cmidrule(lr){8-13}
        & \multicolumn{3}{c}{\textbf{MM-SafetyBench}} & \multicolumn{3}{c|}{\textbf{MOSSBench}} & \multicolumn{3}{c}{\textbf{MM-SafetyBench}} & \multicolumn{3}{c}{\textbf{MOSSBench}} \\
        \textbf{Method} & \textbf{DSR}$\uparrow$ & \textbf{RR}$\uparrow$ & \textbf{Avg}$\uparrow$ & \textbf{DSR}$\uparrow$ & \textbf{RR}$\uparrow$ & \textbf{Avg}$\uparrow$ & \textbf{DSR}$\uparrow$ & \textbf{RR}$\uparrow$ & \textbf{Avg}$\uparrow$ & \textbf{DSR}$\uparrow$ & \textbf{RR}$\uparrow$ & \textbf{Avg}$\uparrow$\\
        \midrule
        w/o Defense          & 0.06  & 0.98  & 0.52  & 0.14  & 0.97  & 0.55  & 0.10  & 0.97  & 0.53  & 0.30  & 0.96  & 0.63  \\
        \midrule
        \multicolumn{13}{c}{Baseline} \\
        \midrule
        Responsible          & 0.12  & 0.96  & 0.54  & 0.32  & 0.96  & 0.64  & 0.18  & 0.96  & 0.57  & 0.47  & 0.92  & 0.70  \\
        Policy               & 0.08  & 0.96  & 0.52  & 0.18  & 0.98  & 0.58  & 0.12  & 0.97  & 0.55  & 0.34  & 0.97  & 0.65  \\
        Demonstration        & 0.15  & 0.97  & 0.56  & 0.37  & 0.95  & 0.66  & 0.25  & 0.96  & 0.60  & 0.52  & 0.92  & \textbf{0.72}  \\
        SFT                  & 0.20  & 0.95  & 0.58  & 0.50  & 0.88  & 0.69  & 0.13  & 0.98  & 0.55  & 0.49  & 0.88  & 0.68 \\
        SafeDecoding         & 0.08  & 0.97  & 0.53  & 0.31  & 0.94  & 0.62  & 0.12  & 0.96  & 0.54  & 0.42  & 0.93  & 0.68  \\
        Caption              & 0.09  & 0.98  & 0.53  & 0.21  & 0.98  & 0.60  & 0.12  & 0.97  & 0.55  & 0.27  & 0.94  & 0.60  \\
        Caption (w/o image)  & 0.16  & 0.95  & 0.55  & 0.34  & 0.94  & 0.64  & 0.22  & 0.93  & 0.57  & 0.45  & 0.89  & 0.67 \\
        Intention            & 0.07  & 0.98  & 0.53  & 0.20  & 0.99  & 0.59  & 0.11  & 0.96  & 0.54  & 0.26  & 0.97  & 0.61  \\
        \midrule
        % \multicolumn{13}{c}{} \\
        % \midrule
        \midrule
        \multicolumn{13}{c}{SR++} \\
        \midrule        
        Responsible-Demonstration & 0.18 & 0.95 & 0.57 & 0.40 & 0.94 & 0.67 & 0.29 & 0.96 & 0.62 & 0.58 & 0.85 & \textbf{0.72} \\
        Responsible-Policy & 0.12 & 0.96 & 0.54 & 0.27 & 0.97 & 0.62 & 0.18 & 0.96 & 0.57 & 0.46 & 0.94 & 0.70 \\
        Policy-Demonstration & 0.13 & 0.96 & 0.55 & 0.37 & 0.97 & 0.67 & 0.20 & 0.96 & 0.58 &0.51 & 0.93 & \textbf{0.72}\\
        Responsible-Policy-Demonstration & 0.15 & 0.96 & 0.55 & 0.38 & 0.95 & 0.66 & 0.25 & 0.97 & 0.61 & 0.53 & 0.88 & 0.70\\
        \midrule
        \multicolumn{13}{c}{SR+MO} \\
        \midrule     
        Responsible-SFT & 0.56 & 0.93 & \textbf{0.75} & 0.61 & 0.72 & 0.67 & 0.35 & 0.96 & 0.65 & 0.74 & 0.62 & 0.68 \\
        Responsible-SafeDecoding & 0.30 & 0.96 & 0.63 & 0.54 & 0.87 & \underline{0.70} & 0.23 & 0.96 & 0.59 & 0.63 & 0.79 & 0.71\\
        Demonstration-SFT & 0.60 & 0.90 & \textbf{0.75} & 0.65 & 0.77 & \textbf{0.71} & 0.56 & 0.92 & \textbf{0.74} & 0.67 & 0.70 & 0.68\\
        Demonstration-SafeDecoding & 0.38 & 0.96 & \underline{0.67} & 0.55 & 0.87 & \textbf{0.71} & 0.40 & 0.96 & \underline{0.68} & 0.62 & 0.78 & 0.70\\
        \midrule
        \multicolumn{13}{c}{QR++} \\
        \midrule   
        Caption-Intention & 0.09 & 0.97 & 0.53 & 0.20 & 0.98 & 0.59 & 0.14 & 0.95 & 0.55 & 0.26 & 0.96 & 0.61\\
        % Caption-Intention (w/o image) & 0.18 & 0.96 & 0.57 & 0.32 & 0.95 & 0.64 & 0.25 & 0.92 & 0.59 & 0.45 & 0.92 & 0.68\\
        \midrule
        % \multicolumn{13}{c}{} \\
        % \midrule
        \midrule
        \multicolumn{13}{c}{QR\textbar{}SR} \\
        \midrule   
        Caption-Responsible & 0.34 & 0.96 & 0.65 & 0.53 & 0.79 & 0.66 & 0.33 & 0.96 & 0.65 & 0.50 & 0.82 & 0.66\\
        Intention-Responsible & 0.36 & 0.97 & \underline{0.67} & 0.51 & 0.86 & 0.68 & 0.27 & 0.96 & 0.61 & 0.49 & 0.90 & 0.70\\
        Caption-Responsible (w/o image) & 0.96 & 0.25 & 0.60 & 0.93 & 0.16 & 0.55 & 0.60 & 0.80 & \underline{0.70} & 0.72 & 0.72 & \textbf{0.72}\\
        % Responsible-Intention (w/o image) & 0.99 & 0.06 & 0.52 & 0.95 & 0.17 & 0.56 & 0.61 & 0.81 & 0.71 & 0.68 & 0.77 & 0.72\\
        \midrule
        \multicolumn{13}{c}{QR\textbar{}MO} \\
        \midrule
        Caption-SafeDecoding & 0.20 & 0.96 & 0.58 & 0.39 & 0.88 & 0.64 & 0.33 & 0.94 & 0.63 & 0.40 & 0.90 & 0.65 \\
        Intention-SFT & 0.28 & 0.97 & 0.62 & 0.43 & 0.78 & 0.61 & 0.25 & 0.96 & 0.60 & 0.50 & 0.88 & 0.69\\
        Caption-SafeDecoding (w/o image) & 0.24 & 0.95 & 0.60 & 0.41 & 0.89 & 0.65 & 0.36 & 0.85 & 0.61 & 0.56 & 0.84 & 0.70\\
        \bottomrule
    \end{tabular}}
    \caption{Comparison results of ensemble strategies with the corresponding individual defenses. \textbf{Bold} indicates the best overall performance, while \underline{underlined} highlights the top three methods.} % and the full score is 100\%
    \label{tab:en_inter_results}
\end{table*}


\subsection{Experimental Setup}
We empirically evaluate various defense methods and their ensemble strategies on LLaVA-1.5-7B and LLaVA-1.5-13B~\cite{liu2024visual} to validate their effectiveness in standard settings. Using MM-SafetyBench and MOSSBench datasets, we assess safety and helpfulness by measuring defense success rate (DSR) on harmful queries and response rate (RR) on benign queries. We evaluate 28 defense methods, including system reminders, optimization techniques, query refactoring, and noise injection, as well as inter- and intra-mechanism ensembles. Detailed descriptions of defense methods and experimental setups are provided in Appendix~\ref{sec:defense strategies} and~\ref{sec:experiment_detail}. 
For a broader evaluation, we add more experiments in Appendix~\ref{sec:utility}, ~\ref{sec:diverse_attacks} and~\ref{sec:time}, including evaluation with the MM-Vet dataset for testing the quality of model's response on general queries, tests on JailbreakV-28K for more diverse and complex attack scenarios, and a comparison of inference time for different defense methods.

\subsection{Individual Defense Results}

Table~\ref{tab:indi_results} shows results of individual defense methods across four categories. Most methods, except for noise injection, effectively improve model safety across different models and datasets, as evidenced by increased defense success rates. This aligns with our analysis in Figure~\ref{fig:analysis results} where system reminder, model optimization and query refactoring lead to an overall increase in refusal probabilities. 

\paragraph{Safety shift defenses compromise helpfulness.} System reminder and model optimization methods generally reduce response rates on the benign subset while increasing defense success rates on the harmful subset. This confirms that safety shift tend to compromise helpfulness. This is more pronounced in MOSSBench than MM-SafetyBench due to the more apparent harmfulness and concealed harmlessness in MOSSBench queries.

\paragraph{Harmfulness discrimination defenses mitigate over-defense.} Query refactoring methods, except for Caption (w/o image), generally achieve the highest response rates on the benign subset, particularly for MOSSBench with misleadingly benign queries. This validates that harmfulness discrimination improves the model's ability to distinguish between truly harmful and benign queries. Notably, the removal of images in the Caption (w/o image) significantly reduces response rates for both harmful and benign queries, highlighting the crucial role images play in jailbreaking LVLMs.
% \paragraph{Image matters.} The removal of images in the Caption (w/o image) and Intention (w/o image) defenses leads to significant improvements in DSR compared to their image-included counterparts, underscoring the crucial role that images play in jailbreaking LVLMs.

\paragraph{Multimodal defense is challenging.}
However, all individual defense methods still exhibit limited defense success rates. While larger-scale LVLMs (i.e., LLaVA-1.5-13B) tend to achieve slightly higher success rates, they are also more susceptible to over-defense. This underscores the inherent challenges of jailbreak defense for LVLMs, especially when relying on individual defense methods. 

\subsection{Ensemble Defense Results}
Table~\ref{tab:en_inter_results} provides the empirical evaluation of both inter-mechanism and intra-mechanism ensemble strategies, leading to the following insights:

\paragraph{Ensembles improve safety.} Compared to individual methods, most ensemble strategies effectively enhance safety across both datasets and model sizes, showing increased defense success rates, especially in \textit{SR+MO} and \textit{QR\textbar{}SR} methods.

\paragraph{Inter-mechanism ensembles amplify.} Our evaluation shows most \textit{SR++} and \textit{SR+MO} ensembles improve defense success rates while reducing responses rates, whereas the \textit{QR++} ensemble better maintain responses rates. This confirms that inter-mechanism ensembles can amplify a single defense mechanism. Specifically, safety shift ensembles would further enhance model safety at the expense of helpfulness, while harmfulness discrimination ensemble better preserves helpfulness. Among inter-mechanism ensembles, those combining different types of specific methods (e.g., SR+MO) show a more pronounced amplification effect than those combining the same type (e.g., SR++). 
Notably, the Demonstration-SFT method excels in defense strength, utility, and response rate. Its success comes from combining two strong safety shift defenses, Demonstration and SFT, which complement each other and boost overall performance.

\paragraph{Intra-mechanism ensembles complement.} Compared to inter-mechanism ensembles, most \textit{QR\textbar{}SR} and \textit{QR\textbar{}MO} methods—except those without input images—can simultaneously maintain decent defense success rates and stable response rates,
compared to the undefended model and individual defense methods. This demonstrates that intra-mechanism ensemble can complement each other to achieve a more balanced trade-off. Additionally, the removal of input images offering a most conservative ensemble for multimodal defense while still maintaining certain helpfulness.
% In contrast, the defenses in intra-mechanism ensemble complement each other, strengthening safety while maintaining a stable level of helpfulness.
% In contrast, intra-mechanism ensembles combine the strengths of both mechanisms to achieve a more balanced trade-off. Specifically, \textit{QR\textbar{}SR} and \textit{QR\textbar{}MO} increase the refusal probability for harmful queries, while maintaining or even decreasing the refusal probability for benign queries, thereby improving the model's ability to distinguish between benign and harmful queries. This makes them a better choice for general scenarios where balancing safety and helpfulness is essential. 


\subsection{How Do Fine-tuning Affect Model Safety?}
We examine how different fine-tuning methods impact the safety of LVLMs by training LLaVA-1.5-7B using DPO and SFT with two datasets: SPA-VL~\cite{zhang2024spa} and VLGuard~\cite{zong2024safety}. SPA-VL focuses on safety discussions, while VLGuard emphasizes query rejection. We also test the effect of adding 5000 general instruction-following data from LLaVA.  

Table~\ref{tab:training_dataset_results} shows that DPO with SPA-VL and LLaVA provides a slight safety boost without significantly changing response behavior. In contrast, SFT has a stronger impact, but its effectiveness depends on the dataset. SPA-VL improves safety while maintaining helpfulness, though it may miss some harmful cases. VLGuard, however, makes the model overly defensive, rejecting too many queries. Adding LLaVA data helps balance safety and helpfulness, reducing excessive refusals.  


\begin{table}[ht]
    \centering
    \resizebox{0.49\textwidth}{!}{
    \begin{tabular}{r|cccccc}
        \toprule 
        & \multicolumn{3}{c}{\textbf{MM-SafetyBench}} & \multicolumn{3}{c}{\textbf{MOSSBench}} \\
        \textbf{Method} & \textbf{DSR}$\uparrow$ & \textbf{RR}$\uparrow$ & \textbf{Avg}$\uparrow$ & \textbf{DSR}$\uparrow$ & \textbf{RR}$\uparrow$ & \textbf{Avg}$\uparrow$ \\
        \midrule
        w/o Defense          & 0.06  & 0.98  & 0.52  & 0.14  & 0.97  & 0.55 \\
        \midrule
        \multicolumn{7}{c}{DPO} \\
        \midrule
        \multicolumn{1}{l|}{SPA-VL + LLaVA}          & 0.06  & 0.97  & 0.52  & 0.28  & 0.97  & 0.63  \\
        \midrule
        \multicolumn{7}{c}{SFT} \\
        \midrule
        \multicolumn{1}{l|}{SPA-VL}          & 0.24  & 0.96  & 0.60  & 0.58  & 0.78  & 0.68  \\
        + LLaVA     & 0.20  & 0.95  & 0.58  & 0.50  & 0.88  & 0.69  \\
        \midrule
        \multicolumn{1}{l|}{VLGuard}          & 1.00  & 0.09  & 0.55  & 0.90  & 0.21  & 0.55  \\
        + LLaVA     & 0.97  & 0.43  & 0.70  & 0.76  & 0.58  & 0.67  \\
        \bottomrule
    \end{tabular}}
    \caption{Comparison of varying fine-tuning settings.} % and the full score is 100\%
    \label{tab:training_dataset_results}
\end{table}

% In this work, we propose WildLong, a novel framework for synthesizing diverse, scalable, and realistic instruction-response datasets designed for long-context tasks. Our approach addresses key challenges in dataset creation by leveraging meta-information extraction from real-world user queries, graph-based modeling of co-occurrence relationships, and adaptive instruction-response generation.
% WildLong is built on the principles of diversity, scalability, and realism, enabling it to support complex reasoning tasks such as cross-document comparison, and aggregation, which are essential for real-world applications. By integrating meta-information into the data generation process and systematically exploring new combinations through graph-based modeling, WildLong generates diverse datasets that reflect the complexity of extended contexts.
% Experimental results demonstrate that WildLong significantly improves long-context task performance, surpassing other open-source long-context-optimized models across multiple benchmarks. Importantly, this improvement is achieved without requiring supplementary short-context instruction tuning, highlighting the robustness and generalizability of our approach.
% The success of WildLong highlights the potential of structured, meta-information-driven data synthesis to enhance the capabilities of LLMs for complex, real-world tasks. By addressing the critical gaps in long-context dataset diversity and quality, WildLong sets a new standard for long-context instruction tuning and paves the way for further advancements in equipping LLMs to tackle the challenges of extended-context reasoning.
% We propose WildLong, a framework for synthesizing diverse, scalable, and realistic instruction-response datasets for long-context tasks. By leveraging meta-information extraction, graph-based modeling, and adaptive instruction generation, WildLong generates long-context instruction-tuning data with real-world complexity.
% Experiments show improved long-context task performance while retaining short-context performance without additional short-context fine-tuning, demonstrating its robustness and generalizability. We hope WildLong provides insights into generalizing instruction tuning and inspires further advancements in long-context reasoning for LLMs.
We propose WildLong, a framework for synthesizing diverse, scalable, and realistic instruction-response datasets for long-context tasks. 
It integrates meta-information extraction to ensure realistic complexity, graph-based modeling for systematic instruction expansion, and adaptive instruction generation for enhanced contextual relevance.
Our fine-tuned models consistently outperform baselines and maintain short-context performance without mixing short-context data. Notably, our finetuned Llama-3.1-8B model surpasses most open-source long-context models on Longbench-Chat and demonstrates competitive performances with even larger models across benchmarks.
WildLong enables the synthesis of instruction-tuning data that produces robust models capable of handling diverse long-context tasks. Extending beyond synthetic QA and summarization, it bridges the gap to more complex, realistic challenges, advancing the effectiveness of long-context LLMs.
We hope WildLong provides insights into generalizing synthetic data and inspires further progress in long-context reasoning for LLMs.
\section*{Impact Statement}

This work focuses on improving the representations in autoencoders that serve as a backbone for latent diffusion training, ultimately enhancing generative performance.
Our improvements can facilitate beneficial applications such as boosting creativity, supporting educational content creation, and reducing computational overhead in generative workflows.
Beyond these considerations, we do not identify additional ethical or societal implications beyond those already known to accompany large-scale generative modeling.


%%%%%%%%%%%%%%%%%%%%%%%%%%%%%%%%%%%%%%%%%%%%%%%%%%%%%%%%%%%%%%%%%%%%%%%%%%%%%%%
%%%%%%%%%%%%%%%%%%%%%%%%%%%%%%%%%%%%%%%%%%%%%%%%%%%%%%%%%%%%%%%%%%%%%%%%%%%%%%%
% Bibliography
% In the unusual situation where you want a paper to appear in the
% references without citing it in the main text, use \nocite
% \nocite{StyleGAN}
%%%%%%%%%%%%%%%%%%%%%%%%%%%%%%%%%%%%%%%%%%%%%%%%%%%%%%%%%%%%%%%%%%%%%%%%%%%%%%%
%%%%%%%%%%%%%%%%%%%%%%%%%%%%%%%%%%%%%%%%%%%%%%%%%%%%%%%%%%%%%%%%%%%%%%%%%%%%%%%
\bibliography{main}
\bibliographystyle{icml2025}

%%%%%%%%%%%%%%%%%%%%%%%%%%%%%%%%%%%%%%%%%%%%%%%%%%%%%%%%%%%%%%%%%%%%%%%%%%%%%%%
%%%%%%%%%%%%%%%%%%%%%%%%%%%%%%%%%%%%%%%%%%%%%%%%%%%%%%%%%%%%%%%%%%%%%%%%%%%%%%%
% APPENDIX
%%%%%%%%%%%%%%%%%%%%%%%%%%%%%%%%%%%%%%%%%%%%%%%%%%%%%%%%%%%%%%%%%%%%%%%%%%%%%%%
%%%%%%%%%%%%%%%%%%%%%%%%%%%%%%%%%%%%%%%%%%%%%%%%%%%%%%%%%%%%%%%%%%%%%%%%%%%%%%%
\newpage
\appendix
\onecolumn

\section{Implementation Details}
\label{ap:details}

\inlinesection{DiT model details}.
To strengthen the baseline DiT performance, we integrated into it the latest advancements from the diffusion model literature.
Namely, we used self conditioning~\cite{RIN} and RoPE~\cite{RoPE} positional embeddings.
Besides, we switched to the rectified flow diffusion parametrization~\cite{NormFlowsWithStochInterp, RecFlow, LSGM}, which was recently shown to have better scalability with a fewer amount of inference steps~\cite{SD3}.

\inlinesection{DiT training details}.
All the DiT models are trained for 400,000 steps with 10,000 warmup steps of the learning rate from 0 to 0.0003 and then its gradual decay towards 0.00001.
We used weight decay of 0.01 and AdamW~\cite{AdamW} optimizer with beta coefficients of 0.9 and 0.99.
We used posterior sampling from the encoder distribution for VAE-based autoencoders.
In contrast to the original work, we found it helpful to do learning rate decay to 0.00001 using the cosine learning rate schedule.
We used the same model sizes for DiT-S (small), DiT-B (base), DiT-L (large) and DiT-XL (extra large), as the original work~\cite{DiT}:
\begin{itemize}
    \item DiT-S: hidden dimensionality of 384, 12 transformer blocks, and 6 attention heads in the multi-head attention.
    \item DiT-B: hidden dimensionality of 768, 12 transformer blocks, and 12 attention heads in the multi-head attention.
    \item DiT-L: hidden dimensionality of 1024, 24 transformer blocks, and 16 attention heads in the multi-head attention.
    \item DiT-XL: hidden dimensionality of 1152, 28 transformer blocks, and 16 attention heads in the multi-head attention.
\end{itemize}
We used gradient clipping with the norm of 16 for all the DiT models.
Our models were trained in the FSDP~\cite{FSDP} framework with the full sharding strategy on a single node of $8 \times$ NVidia A100 80GB GPUs or $8 \times$ NVidia H100 80GB GPUs (depending on their availability in our computational cluster).

For \cvae, since it is considerably slower than other autoencoders, we trained LDMs on pre-extracted latents.
For this, we pre-extracted them on random 17-frames clips.
In essence, this reduces

\inlinesection{Autoencoders training details}.
Since none of the autoencoders had their training pipelines released, we had to develop the training recipes for each of the autoencoder baselines individually which would not be detrimental to neither their reconstruction capability nor downstream diffusion performance.
To do this, we ablated multiple hyperparameters (the most important ones being learning rate and KL regularization strength) to arrive to a proper setup.
We chose the KL weight in such a way that the KL penalty maintains approximately the same magnitude as the pre-trained checkpoint.

Each autoencoder is trained with AdamW~\cite{AdamW} optimizer, with betas of 0.9 and 0.99, and weight decay of 0.01.
The learning rate was grid-searched individually for each autoencoder and is provided in \cref{tab:hyperparameters}.
In all the cases, we used mixed precision training with BFloat16.

During training, we maintained an exponential moving average of the weights~\cite{EDMv2}, initialized from the same parameters as the starting model, and having a half life of 5,000 steps.
% For each autoencoder, we were searching for the KL regularization coefficient in a way that the KL magnitude remains the same 

We emphasize that, when applying our regularization strategy on top of an autoencoder baseilne, we do not alter other hyperparameters (like learning rate), except for KL regularization which we disable for \regshortname-regularized models (even though we found it helpful in some of our explorations).

For each autoencoder, we freeze the last output layers of the decoder.
The motivation is the following: they were fine-tuned with the adversarial loss, which we want to exclude from the equation without hurting the ability of an autoencoder to model textural details which \fid would be sensitive to~\citep{LDM} and which do not influence the latent space properties.
Namely, we freeze the last normalization and output convolution layers.
In each case, the amount of frozen parameters constitute a negligible amount of total parameters.

Other hyperparameters for autoencoders training are provided in \cref{tab:hyperparameters}.

\begin{table}[h!]
    \caption{Hyperparameters}
    \label{tab:TrainingParams}
    \centering
    \begin{tabular}{l|l}
        \textbf{Parameters} & \textbf{Tuning} \\% & \textbf{Estimation}  \\
        \hline
        Sampling time                   & $0.05$  \\   %& $0.05$     \\
        Reward discount factor $\gamma$ & $0.99$  \\   %& $0.99$     \\
        Learning rate for actor         & $10^{-3}$ \\% & $10^{-3}$  \\
        Learning rate for critic        & $10^{-3}$ \\% & $10^{-3}$  \\
        $L_2$ Regularization factor     & $10^{-5}$ \\% & $10^{-4}$  \\
        Optimizer parameter $\epsilon$  & $10^{-8}$ \\% & $10^{-8}$  \\
        Minimum batch size              & $1024$  \\%   & $64$     \\
        Experience buffer length        & $10^{6}$  \\% & $10^{6}$   \\
    \end{tabular}
\end{table}

\section{Additional Exploration}
\label{ap:freq-reg}

In \cref{sec:method}, we outlined the base \regname strategy to regularize the spectrum of an autoencoder which has a strong advantage of being very easy to implement by a practitioner.
However, it could be beneficial to possess more advanced tools for a finer-grained control over the latent space spectral properties.
This section outlines them and provides the corresponding ablation.

\subsection{Explicitly Chopping off High-Frequency Components}

Rather than applying downsampling to produce latents and RGB targets for regualrization, it is possible to replace some ratio of high-frequency components with zeros. To do so, DCT is applied to the latents and RGB targets where a chosen set of frequency components are masked out. The modified components are then translated back to the spatial domain by inverse DCT to form the training latents and reconstruction targets.

\begin{equation}\label{eq:hard-hf-penalty}
\loss_\text{CHF}(x) = d(x, \Dec(z)) + d( {D^{-1}}(D({x}) * \mathbf{M}), \Dec({D^{-1}}(D({z}) * \mathbf{M}) ) + \loss_\text{reg},
\end{equation}
where $D$ and $D^{-1}$ represent DCT and its inverse, respectively. $\mathbf{M}$ is a $B \times B$ binary mask indicating which frequencies to zero out defined as follows:
% \willi{Are we applying this in B * AEupsamplefactor DCT space for x?}
\begin{equation}\label{eq:zigzag-mask}
\mathbf{M}(u,v) =
\begin{cases}
    1, & \text{if } \zigzag (u,v) < B^2 - N,\\
    0, & \text{otherwise}.
\end{cases}
\end{equation}
% \willi{The table has some lines that suggest M may not be cut of in zigzag order as the equation implies}
$N$ controls the frequency cutoff. We provide the ablation for this strategy in \cref{table:cuthf-ablation}.

\begin{table}[h]
\caption{Ablations for explicit high-frequency chop off for DiT-S/2 trained for 200,000 iterations on top of Flux AE with such a regularization. While it can achieve better results for some of the baselines than naive downsampling, we opt out for the latter strategy due to its simplicity. For the non-zigzag order ablation, we cut across each $x$ and $y$ axes independently}
\label{table:cuthf-ablation}
\centering
% \resizebox{1.0\linewidth}{!}{
\begin{tabular}{llcc}
\toprule
Stage II & Stage I  & \dinofidfivek \\
\midrule
DiT-S/2 & FluxAE + chop off 90\% (non-zigzag order) & 912.4 \\
DiT-S/2 & FluxAE + chop off 70\% (non-zigzag order) & 915.6 \\
DiT-S/2 & FluxAE + chop off 30\% (non-zigzag order) & 929.7 \\
DiT-S/2 & FluxAE + chop off 10\% (non-zigzag order) & 916.5 \\
DiT-S/2 & FluxAE + chop off 90\% (zigzag order) & 935.5 \\
DiT-S/2 & FluxAE + chop off 70\% (zigzag order) & 932.8 \\
DiT-S/2 & FluxAE + chop off 30\% (zigzag order) & 962.9 \\
DiT-S/2 & FluxAE + chop off 10\% (zigzag order) & 930.1 \\
\midrule
DiT-S/2 & FluxAE (vanilla) & 992.0 \\
DiT-S/2 & FluxAE with optimal (out of 8) KL $\beta$ & 929.6 \\
% \midrule
% DiT-B/2 (orig) & \multirow{4}{*}{SD-VAE-ft-MSE} & 43.47 & $-$ \\
% DiT-L/2 (orig) & & 23.33 & $-$ \\
% DiT-XL/2 (orig) & & 19.47 & $-$ \\
% ~+ 6.6M steps (orig) & & 12.03 & n/a & 121.50 \\
% ~+ 6.6M steps (orig) & & 12.03 & $-$ \\
\bottomrule
\end{tabular}
% }
\end{table}

In \cref{fig:progressive-dct-cut}, we provided the visualizations for a FluxAE resiliense with and without such chopping high-frequency regularization for $50\%$ HF dropout rate.
In \cref{fig:progressive-dct-cut-downreg}, we provide an equivalent visualization for \regshortname-fine-tuned FluxAE: while it is less resilient to frequency dropout than \regchfshortname, but is still noticeably better than the vanilla model.

\begin{figure}[t]
\centering
% \includegraphics[width=0.95\linewidth]{assets/progressive-dct-cut-qualitative.pdf}
% \includegraphics[width=\linewidth]{assets/progressive-dct-cut-qualitative-new.png}
\begin{minipage}{0.7\linewidth}
\setlength{\unitlength}{\linewidth}
\centering
\vspace{-2.21cm}
\begin{picture}(1,1)
% \put(0.05, 0){\includegraphics[width=0.95\linewidth]{assets/progressive-dct-cut-qualitative-new.png}}
\put(0.05, 0){\includegraphics[width=0.95\linewidth]{assets/grid-004-dreg.png}}

% Column captions - adjust Y slightly below the image (e.g., -0.03)
\put(0.175, -0.01){\makebox(0,0)[t]{\small 0\%}}
\put(0.415, -0.01){\makebox(0,0)[t]{\small 25\%}}
\put(0.65, -0.01){\makebox(0,0)[t]{\small 50\%}}
\put(0.89, -0.01){\makebox(0,0)[t]{\small 75\%}}

% Row captions - adjust X slightly left of the image (e.g., -0.08)
% \put(0.035, 0.61){\makebox(0,0)[r]{\rotatebox{90}{\small FluxAE}}}
\put(0.035, 0.365){\makebox(0,0)[r]{\rotatebox{90}{\small FluxAE}}}
\put(0.035, 0.12){\makebox(0,0)[r]{\rotatebox{90}{\small FluxAE + \regshortname}}}
\end{picture}
\end{minipage}
\caption{
RGB and FluxAE reconstruction with/without scale equivariance regularization for different percentages of chopped-off high frequency components.
}
\label{fig:progressive-dct-cut-downreg}
\end{figure}



\subsection{Soft Penalty for High-Frequency Components}

Instead of directly removing some of the components, which might become a too strict regularization signal, one can consider penalizing the amplitudes of high-frequency components in a soft manner.
Concretely, given a $B \times B$ block, we construct the following weight penalty matrix:
% amplitude = x_dct.abs() # [..., b, b]
% x_coords = torch.linspace(0, 1, steps=block_size, device=x.device, dtype=x.dtype).unsqueeze(0).expand(block_size, -1) # [b, b]
% y_coords = torch.linspace(0, 1, steps=block_size, device=x.device, dtype=x.dtype).unsqueeze(1).expand(-1, block_size) # [b, b]
% weight = (x_coords + y_coords).float().pow(power) / 4.0 # [b, b]
% weight = misc.unsqueeze_left(weight, amplitude) # [..., b, b]
% loss = (amplitude * weight).reshape(amplitude.shape[0], -1).mean(dim=1) # [batch_size]

\begin{equation}\label{eq:soft-hf-penalty-matrix}
\mathcal{W}_{uv} = (u + v)^p / B^p.
\end{equation}

Next, the soft regularization loss itself is computed as:
\begin{equation}\label{eq:soft-hf-penalty}
\loss_\text{softreg} = \sum_{u,v} D_{uv}(z) \cdot \mathcal{W}_{uv}.
\end{equation}

During training, when enabled, we add it to the main loss with the weigh $\gamma$.
We found it beneficial in some of our experiments when it is added with a small coefficient (e.g., 0.01).
While it is possible to achieve higher results with more fine-grained regularization, we opt to use the simpler version since we believe it would be easier to employ by the community.

To ablate its importance, we trained DiT-B/1 model on top of FluxAE models, fine-tuned with a different strength $\gamma$.
The results are presented in \cref{table:softregweight-abl}.

\begin{table}[ht]
\caption{Ablating the regularization strength $\alpha$ of our proposed \regname regularization.}
\label{table:softregweight-abl}
\centering
% \resizebox{1.0\linewidth}{!}{
\begin{tabular}{llcc}
\toprule
Stage II & Stage I  & \fid$_{5k}$ & \dinofid$_{5k}$ \\
\midrule
\multirow{6}{*}{DiT-B/1}
& FluxAE + FT-\regshortname $\gamma = 0.001$ & 26.43 & 497.14 \\
& FluxAE + FT-\regshortname $\gamma = 0.025$ & 25.46 & 477.61 \\
& FluxAE + FT-\regshortname $\gamma = 0.01$ & 26.72 & 487.06 \\
& FluxAE + FT-\regshortname $\gamma = 0.05$ & \cellbest{24.28} & \cellbest{458.11} \\
& FluxAE + FT-\regshortname $\gamma = 0.1$ & \cellsecond{25.84} & \cellsecond{461.97} \\
\bottomrule
\end{tabular}
% }
\end{table}


\begin{figure}
\centering
\includegraphics[width=0.4\linewidth]{assets/zigzag.pdf}
\caption{Illustration of the zigzag indexing order of DCT.}
\label{fig:zigzag}
\end{figure}

\subsection{ImageNet $512^2$ experiments}

We trained our DiT-L/2 for class-conditional $512^2$ ImageNet-1K generation for 400K steps for FluxAE~\cite{Flux}, the results are presented in \cref{table:imagenet-512}.

\begin{table}[ht]
\caption{Class-conditional generation results on ImageNet-1K $512^2$ without guidance. The original DiT paper reports the results after 3M training steps, while we use 400K steps for our models.}
\label{table:imagenet-512}
\centering
% \resizebox{1.0\linewidth}{!}{
\begin{tabular}{llcc}
\toprule
Stage II & Stage I  & \fid & \dinofid \\
\midrule
% 3278: "fid50k": 13.130795660857324, "dinofid50k": 249.46464787074206,
% 3279: "fid50k": 13.697963826137052, "dinofid50k": 267.7557197224488,
% 3280: "fid50k": 11.631835931219992, "dinofid50k": 203.55857327388986,
\multirow{3}{*}{DiT-L/2} & \fluxae (vanilla) & \cellsecond{13.13} & \cellsecond{249.4} \\
& \fluxae + FT & 13.69 & 267.7 \\
& \fluxae + FT-\regshortname \ours & \cellbest{11.63} & \cellbest{203.5} \\
\midrule
DiT-XL/2 (orig) + 3M steps & SD-VAE-ft-MSE & 12.03 & $-$ \\
\bottomrule
\end{tabular}
% }
\end{table}


\subsection{Ablating regularization strength $\alpha$}

To ablate the importance of the regularization strength $\alpha$, we train FluxAE for 10,000 steps with a varying strength.
The results are presented in \cref{table:regweight-abl}.

\begin{table}[ht]
\caption{Ablating the regularization strength $\alpha$ of our proposed \regname regularization.}
\label{table:regweight-abl}
\centering
% \resizebox{1.0\linewidth}{!}{
\begin{tabular}{llcc}
\toprule
Stage II & Stage I  & \fid$_{5k}$ & \dinofid$_{5k}$ \\
\midrule
% 3278: "fid50k": 13.130795660857324, "dinofid50k": 249.46464787074206,
% 3279: "fid50k": 13.697963826137052, "dinofid50k": 267.7557197224488,
% 3280: "fid50k": 11.631835931219992, "dinofid50k": 203.55857327388986,
% & \fluxae + FT & 13.69 & 267.7 \\
% & \fluxae + FT-\regshortname \ours & \cellbest{11.63} & \cellbest{203.5} \\
% \midrule
% DiT-XL/2 (orig) + 3M steps & SD-VAE-ft-MSE & 12.03 & $-$ \\
\multirow{6}{*}{DiT-B/2}
& FluxAE + FT-\regshortname $\alpha = 0.01$ & 33.99 & 641.95 \\
& FluxAE + FT-\regshortname $\alpha = 0.05$ & 33.86 & 645.94 \\
& FluxAE + FT-\regshortname $\alpha = 0.1$ & \cellsecond{28.62} & \cellsecond{586.91} \\
& FluxAE + FT-\regshortname $\alpha = 0.25$ & \cellbest{26.84} & \cellbest{558.36} \\
& FluxAE + FT-\regshortname $\alpha = 0.5$ & 29.63 & 569.92 \\
& FluxAE + FT-\regshortname $\alpha = 1$ & 33.22 & 612.45 \\
\bottomrule
\end{tabular}
% }
\end{table}


\section{Additional visualizations}
\label{ap:visualizations}

This section provides additional visualizations for the LDM experiments.

\begin{figure*}[h]
\centering
\includegraphics[width=\linewidth]{assets/extra-sample-grids-cfg3.0/ldm2840-ditxl-flux-init-01.jpg}
\includegraphics[width=\linewidth]{assets/extra-sample-grids-cfg3.0/ldm2903-ditxl-flux-ft-01.jpg}
\includegraphics[width=\linewidth]{assets/extra-sample-grids-cfg3.0/ldm2839-ditxl-flux-reg-01.jpg}
\caption{Uncurated samples from DiT-XL/2 for FluxAE (top), FluxAE + FT (middle) and FluxAE + \regshortname (bottom) on class-conditional ImageNet $256 \times 256$ for random classes. During inference, we used 256 steps with the guidance scale of 3.0.}
\label{fig:ap:extra-samples-ditxl-flux}
\end{figure*}

\begin{figure*}[h]
\centering
\includegraphics[width=\linewidth]{assets/extra-samples/cfg3.0--ldm3164-ditxl-flux-reg-1M-00.jpg}
\includegraphics[width=\linewidth]{assets/extra-samples/cfg3.0--ldm3164-ditxl-flux-reg-1M-01.jpg}
\caption{Uncurated samples from DiT-XL/2 trained for 1M steps on top FluxAE + \regshortname (bottom) on class-conditional ImageNet $256 \times 256$ for random classes. During inference, we used 256 steps with the guidance scale of 3.0.}
\label{fig:ap:extra-samples-ditxl-flux-1m}
\end{figure*}

\begin{figure*}[h]
\centering
\includegraphics[width=\linewidth]{assets/extra-samples/cfg3.0--class88-ldm3164-ditxl-flux-reg-1M-00.jpg}
\caption{Uncurated samples from DiT-XL/2 trained for 1M steps on top FluxAE + \regshortname (bottom) on class-conditional ImageNet $256 \times 256$ for class 88. During inference, we used 256 steps with the guidance scale of 3.0.}
\label{fig:ap:extra-samples-ditxl-flux-1m-class-88}
\end{figure*}

\begin{figure*}[h]
\centering
\includegraphics[width=\linewidth]{assets/extra-samples/cfg3.0--class130-ldm3164-ditxl-flux-reg-1M-00.jpg}
\caption{Uncurated samples from DiT-XL/2 trained for 1M steps on top FluxAE + \regshortname (bottom) on class-conditional ImageNet $256 \times 256$ for class 130. During inference, we used 256 steps with the guidance scale of 3.0.}
\label{fig:ap:extra-samples-ditxl-flux-1m-class-130}
\end{figure*}

\begin{figure*}[h]
\centering
\includegraphics[width=\linewidth]{assets/extra-samples/cfg3.0--class279-ldm3164-ditxl-flux-reg-1M-00.jpg}
\caption{Uncurated samples from DiT-XL/2 trained for 1M steps on top FluxAE + \regshortname (bottom) on class-conditional ImageNet $256 \times 256$ for class 279. During inference, we used 256 steps with the guidance scale of 3.0.}
\label{fig:ap:extra-samples-ditxl-flux-1m-class-279}
\end{figure*}

\begin{figure*}[h]
\centering
\includegraphics[width=\linewidth]{assets/extra-samples/cfg3.0--class555-ldm3164-ditxl-flux-reg-1M-00.jpg}
\caption{Uncurated samples from DiT-XL/2 trained for 1M steps on top FluxAE + \regshortname (bottom) on class-conditional ImageNet $256 \times 256$ for class 555. During inference, we used 256 steps with the guidance scale of 3.0.}
\label{fig:ap:extra-samples-ditxl-flux-1m-class-555}
\end{figure*}

\begin{figure*}[h]
\centering
\includegraphics[width=\linewidth]{assets/extra-samples/cfg3.0--ldm3278-512x512-ditl2-flux-orig-00.jpg}
\includegraphics[width=\linewidth]{assets/extra-samples/cfg3.0--ldm3279-512x512-ditl2-flux-ft-00.jpg}
\includegraphics[width=\linewidth]{assets/extra-samples/cfg3.0--ldm3280-512x512-ditl2-flux-reg-00.jpg}
\caption{Uncurated samples from DiT-XL/2 trained for 400K steps on top FluxAE + \regshortname (bottom) on class-conditional ImageNet $512 \times 512$ for random classes. During inference, we used 256 steps with the guidance scale of 3.0.}
\label{fig:ap:extra-samples-ditxl-flux-r512}
\end{figure*}

\begin{figure*}[h]
\centering
\includegraphics[width=\linewidth]{assets/extra-samples/cfg1.5--ldm3271-ditb1-cmsaei-vanilla-01.jpg}
\includegraphics[width=\linewidth]{assets/extra-samples/cfg1.5--ldm3308-ditb1-cmsaei-ft-01.jpg}
\includegraphics[width=\linewidth]{assets/extra-samples/cfg1.5--ldm3310-ditb1-cmsaei-reg-01.jpg}
\caption{Uncurated samples from DiT-B/1 for \cmsaei (top), \cmsaei + FT (middle) and \cmsaei + \regshortname (bottom) on class-conditional ImageNet $256 \times 256$. During inference, we used 256 steps with the guidance scale of 1.5.}
\label{fig:ap:extra-samples-ditb-cmsaei}
\end{figure*}

\begin{figure*}[h]
\centering
\includegraphics[width=\linewidth]{assets/extra-samples/cfg3.0--ldm3045-ditxl2-cvae-vanilla-00.jpg}
\includegraphics[width=\linewidth]{assets/extra-samples/cfg3.0--ldm3046-ditxl2-cvae-ft-00.jpg}
\includegraphics[width=\linewidth]{assets/extra-samples/cfg3.0--ldm3047-ditxl2-cvae-ft-reg-00.jpg}
\caption{Uncurated samples from DiT-XL/2 for \cvaefull (top), \cvaefull + FT (middle) and \cvaefull + \regshortname (bottom) on class-conditional Kinetics $17 \times 256 \times 256$. During inference, we used 256 steps with the guidance scale of 3.0.}
\label{fig:ap:extra-samples-ditxl-cvae}
\end{figure*}

\begin{figure*}[h]
\centering
\includegraphics[width=\linewidth]{assets/extra-sample-grids-cfg3.0/ldm2942-ditb2-cvae-init-00.jpg}
\includegraphics[width=\linewidth]{assets/extra-sample-grids-cfg3.0/ldm2907-ditb2-cvae-ft-00.jpg}
\includegraphics[width=\linewidth]{assets/extra-sample-grids-cfg3.0/ldm2908-ditb2-cvae-reg-00.jpg}
\caption{Uncurated samples from DiT-B/2 for \cvaefull (top), \cvaefull + FT (middle) and \cvaefull + \regshortname (bottom) on class-conditional Kinetics $17 \times 256 \times 256$. During inference, we used 256 steps with the guidance scale of 3.0.}
\label{fig:ap:extra-samples-ditb-cvae}
\end{figure*}


\begin{figure*}[h]
\centering
\includegraphics[width=\linewidth]{assets/extra-sample-grids-cfg3.0/ldm2943-ditb1-ltx-init-00.jpg}
\includegraphics[width=\linewidth]{assets/extra-sample-grids-cfg3.0/ldm2940-ditb1-ltx-ft-00.jpg}
\includegraphics[width=\linewidth]{assets/extra-sample-grids-cfg3.0/ldm2936-ditb1-ltx-reg-kl-00.jpg}
\caption{Uncurated samples from DiT-B/1 for \ltxae (top), \ltxae + FT (middle) and \ltxae + \regshortname (bottom) on class-conditional Kinetics $17 \times 256 \times 256$. During inference, we used 256 steps with the guidance scale of 3.0.}
\label{fig:ap:extra-samples-ditb-ltxae}
\end{figure*}

\section{Limitations.}

We identify the following limitations of our work and the proposed regularization:
\begin{enumerate}
    \item While we did our best to verify that our framework works in the most general setup possible, testing 4 different autoencoders across 2 different domains (image and videos), our study would be more complete when verified across other diffusion parametrizations~\cite{EDM, DDPM, VDM++} or architectures~\cite{EDMv2}.
    \item We observed that our regularization still affects the reconstruction slightly: for example, \cref{table:ae-rec} shows that FluxAE \fid increased from 0.183 to 0.55 (though for some AEs, like \cvaefull, it improves). We are convinced that this \fid increase could be mitigated by training with adversarial losses, which we omitted in this work for simplicity.
    \item There is a mild sensitivity to hyperparameters: for example, we found that varying the SHF regularization weight might improve the results (see \cref{table:regweight-abl}), or adding a small KL regularization (which we disabled in the end for our regularization for simplicity).
    % \item For some setups, autoencoder training should be longer: for example, we observed that DiT-B/1 training on top of \cmsaei leads to better results after 200,000 fine-tuning steps rather than 10,000 steps as was used in the current work.
    \item None of the explored autoencoders released their training pipelines, and it is non-trivial to fine-tune them even without any extra regularization. For example, we observed that any fine-tuning of \dcae~\cite{DC-AE} was leading to divergent reconstructions in our training pipeline (we explored dozens of different hyperparameter setups). 
\end{enumerate}

We leave the exploration of these limitations for future work.

% \section{Failed experiments}
\label{ap:failed-experiments}


We tried two theory-inspired regularization strategies:
\begin{itemize}
    \item MiniLDM-regulzation training. This was inspired by LSGM~\cite{LSGM} where the authors showed that it's the correct objective for LDMs. We spent a month trying to make it work, but in all the cases it . At the same time, \cite{CausRegTok} were able to make it work for discrete autoregressive models.
    \item Lipszhitz regularization. This was inspired by the lower bound which \cite{LFM} provided:
    \begin{equation}
        W(p, p') \leq ...
    \end{equation}
    This more or less worked, but it was tough to train. Also, it was leading to unstable training in the LDM itself. Something to revisit
    \item 
\end{itemize}


\end{document}
