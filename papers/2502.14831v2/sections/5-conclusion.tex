\section{Conclusion}
\label{sec:conclusion}

% In the future, we plan to explore the following topics:
% \begin{itemize}
%     \item More regularization strategies. There are some theory-motivated regularizations as we discussed in \cref{ap:failed-experiments}, which can guide us what to explore.
%     \item LDMs depend on the amount of computation used, just like LLMs. When pursuing high-compression AEs, we lose high compute, so we need to compensate for it somehow. Maybe we can boost at some diffusion steps
%     \item Variable-copmression AEs, so that we can encode simple videos in less latents, and difficult videos in more latents. This would be another way to eliminate redundancy in representations.
% \end{itemize}

% \paragraph{Conclusion}
We have shown that modern Latent Diffusion Models (LDMs) rely just as critically on their autoencoders as on the more frequently investigated diffusion architectures.
While prior work has largely focused on improving reconstruction quality and compression rates for the autoencoders, our study focuses on \diffusability, revealing how latent spectra with excessive high-frequency components can hamper the downstream diffusion process.
Through a systematic analysis of several autoencoders, we uncovered stark discrepancies between latent and RGB spectral properties and demonstrated that they lead to worse LDM synthesis quality.
Building on this insight, we developed a regularization strategy that aligns the latent and RGB spaces across different frequencies.
Our approach maintains reconstruction fidelity and improves diffusion training by suppressing spurious high-frequency details in the latent code.
% Experiments on both image and video autoencoders, including Flux~\cite{Flux}, CosmosTokenizer~\cite{CosmosTokenizer}, CogVideoX-AE~\cite{CogVideo}, and LTX-AE~\cite{LTX-video}, confirm that improving \diffusability yields clearer, more accurate samples.
Potential future directions include exploring more advanced frequency-based regularizations, adaptive compression methods and scale equivariance regularization in the temporal axis for video autoencoders to further optimize the trade-off between reconstruction quality, compression rate, and diffusability.

% \paragraph{Contributions and Future Directions}
% In summary, we (1) provided a detailed analysis of how latent spectra affect \diffusability, (2) introduced a practical solution for regularizing latent frequencies, and (3) demonstrated consistent improvements across a range of autoencoders.
