\begin{figure}[t]
\centering
% \includegraphics[width=0.95\linewidth]{assets/progressive-dct-cut-qualitative.pdf}
% \includegraphics[width=\linewidth]{assets/progressive-dct-cut-qualitative-new.png}
\begin{minipage}{\linewidth}
\setlength{\unitlength}{\linewidth}
\centering
\vspace{-2.21cm}
\begin{picture}(1,1)
\put(0.05, 0){\includegraphics[width=0.95\linewidth]{assets/grid-004-cuthf.png}}

% Column captions - adjust Y slightly below the image (e.g., -0.03)
\put(0.175, -0.01){\makebox(0,0)[t]{\small 0\%}}
\put(0.415, -0.01){\makebox(0,0)[t]{\small 25\%}}
\put(0.65, -0.01){\makebox(0,0)[t]{\small 50\%}}
\put(0.89, -0.01){\makebox(0,0)[t]{\small 75\%}}

% Row captions - adjust X slightly left of the image (e.g., -0.08)
\put(0.035, 0.61){\makebox(0,0)[r]{\rotatebox{90}{\small RGB}}}
\put(0.035, 0.365){\makebox(0,0)[r]{\rotatebox{90}{\small FluxAE}}}
\put(0.035, 0.12){\makebox(0,0)[r]{\rotatebox{90}{\small FluxAE + CHF}}}
\end{picture}
\end{minipage}
\caption{
RGB and autoencoder reconstructions with progressively erased DCT high-frequency components.
RGB faces minimal degradation (top), as a higher percentage of the latent DCT spectrum is removed, but the Flux AE reconstructions (middle) quickly degrade when the high-frequency components from the latents are being removed.
A high-frequency cutoff regularization forces the autoencoder to rely more on the low frequency region of the latents and leads to better compression and resilience to high-frequency error accumulation in diffusion models.
}
\label{fig:progressive-dct-cut}
\end{figure}
