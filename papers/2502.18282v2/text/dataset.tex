In political science, researchers often estimate individuals' or groups' political preferences and ideological positions by analyzing observable behaviors, such as voting patterns and survey responses \cite{martin2002dynamic,ho2008measuring}. For example, \citet{doi:10.1073/pnas.2120284119} created SCOPE to gather respondents' views on Supreme Court decisions. By comparing collected survey responses with the Court's voting record, they demonstrated that the Court has adopted a more conservative stance than the general U.S. public. In this study, we use \textsc{SCOPE} \cite{doi:10.1073/pnas.2120284119} to prompt various LLMs to assess their political leanings and subsequently compare their alignment with surveyed human opinions and political leanings in their training data.

The SCOPE dataset comprises of 32 cases, each represented by a binary-choice question asking respondents to express their views on the Court's ruling as either supportive (\textit{pro}) or opposing (\textit{opp}). \autoref{fig:case_example} provides an example of a survey question. \autoref{tab:keywords} in \autoref{app:keywords} lists all 32 cases along with their corresponding legal topics in the SCOPE dataset. For each case, between 1,500 and 2,158 respondents indicate whether they are \textit{pro} or \textit{opp} regarding the Court's decision. Additionally, SCOPE captures each respondent's self-identified political ideology, enabling the categorization of participants into self-identified Democrats or Republicans. \autoref{tab:keywords} in \autoref{app:keywords} showcases the distribution of choices \{\textit{pro}, \textit{opp}\} among the overall surveyed respondents, as well as within the self-identified Democratic and Republican respondents. Further descriptive statistics on respondents' backgrounds are available in the original study \cite{doi:10.1073/pnas.2120284119}.









% \subsection{Estimating political leanings with SCOTUS cases}

% In political science, it is often \ye{sth verb is missing here} to estimate the political preferences or ideological positions of individuals or groups based on their observed behavior, such as voting or survey response. [cite Martin Quinn, Dan ho et al. ]. For example, Jesse 2022 (cite) conducted surveys to gauge US citizens' opinions on politically significant cases of the Supreme Court of the United States (SCOTUS).

% The survey (hereafter Jesse 2022) dataset contains 32 cases hand-picked by political experts. For each case, respondents were provided with background information on the case, then asked to select one of two response options: either in support of or in opposition to the Court’s decision.  An example case from the survey is shown in \autoref{fig:case_example}. \autoref{tab:keywords} contains full list of cases, for more details of the dataset, please consult the original work\cite{doi:10.1073/pnas.2120284119}.
% % \ye{worth listing the cases in the appendix}

% Leveraging the Jesse 2022 survey data, we prompt various LLMs to gauge their political leanings on the contentious issues in the U.S. Supreme Court cases. Furthermore, political experts annotate each case with carefully chosen keywords, which enable us to efficiently retrieve the relevant documents of each case from the training corpora (see section \ref{sec:pd_corpora} ).

% \ye{would be helpful to provide some statistics of the data. besides how many cases, how many questions were asked about each (on average, in total), how many respondents, their political leanings, etc.}

% % We focus on the Jesse 2002 dataset because the SCOTUS cases address contentious issues such as abortion, gun control, and voting rights, which often a good testbed to evaluate the political preference of LLMs and subsequently to study the alignment with human opinions. Moreover, Given that the pretraining data of LLMs are scraped from internet, which includedes both the public opinions (forum, blog post etc.) and the courts’ opinion (official legal documents). The special design of the Jesse2022 allows to investigate the alignment of LLMs to not only the US public but also the US supreme court.





