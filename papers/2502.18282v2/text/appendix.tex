\newpage
\appendix
\renewcommand{\arraystretch}{1}






\section{Implementation Details}
\label{app:implementation}
We downloaded the OLMo-SFT, OLMo-Instruct, LLama3-7b models, BLOOMZ and T0 from HuggingFace Hub \cite{wolf2020huggingfacestransformersstateoftheartnatural} and ran the downloaded LLMs on an A100 GPU. We accessed the other models through the DeepInfra API.  We use default generation parameters from the `transformers' library, except for temperature. We set temperature to 1 to when probing LLMs views on SCOPE cases. When using LLama3-70 to detect the stance score of training documents, we set temperature to 0 to reduce variation to a minimum . We collected all GPT responses in November 2024.


\begin{figure*}[h]
\centering
\resizebox{\linewidth}{!}{
\includegraphics[width=\linewidth]{fig/llm_prompt.png}
}
\caption{LLM prompt}
\label{fig:llm_prompt}
\end{figure*}


\section{LLM Response Collection}
\label{app:prompts}

\autoref{fig:llm_prompt} demonstrates how we prompt the LLMs for responses. 
Prior research has shown that LLMs are sensitive to the prompt format and the sequence of answer options \cite{webson-pavlick-2022-prompt}, and they may display inconsistencies in their responses \cite{elazar-etal-2021-measuring}. To mitigate these issues, we implemented three variations of prompts, following \citet{NEURIPS2023_a2cf225b}. We also randomize the order of the answer choices within each format, producing six unique prompt forms.
\autoref{fig:prompts} demonstrates the prompts we used to query the LLMs' political preference.


\begin{figure*}[htbp]
\centering

\caption{Prompts Used to Query Political Preference}

% \begin{subfigure}[b]{0.45\textwidth}
%     \centering\includegraphics[width=0.99\textwidth]{fig/prompt_human.pdf}
%     \caption{Human}
%     \label{fig:prompt_human}   
% \end{subfigure}

\begin{subfigure}[b]{0.5\textwidth}
    \centering\includegraphics[width=\textwidth]{fig/prompt_ab.pdf}
    \caption{Question Template: AB}
    \label{fig:prompt_a}   
\end{subfigure}


  \begin{subfigure}[b]{0.5\textwidth} 
    \centering\includegraphics[width=\textwidth]{fig/prompt_repeat.pdf}
     \caption{Question Template: Repeat} \label{fig:prompt_b}    
  \end{subfigure}
  
 \begin{subfigure}[b]{0.5\textwidth}
\centering\includegraphics[width=\textwidth]{fig/prompt_compare.pdf}
  \caption{Question Template: Compare} \label{fig:prompt_c}
  \end{subfigure}

  
  \label{fig:prompts}
  \end{figure*}



\paragraph{Mapping LLM Response to Preferences}
To map LLM generated sequences of tokens to actions (i.e., opinion preference), we use an iterative, rule-based matching pipeline in the following order:\\ 1. Check for exact matches (i.e., check for exact overlaps with the desired answer, such as "A" or "Yes")\\ 2. Check for normalized matches (e.g. "A)" or "YES"). For the few unmatched sequences, we manually coded the actions.


% \ye{do we have results on how much variance we get between the different prompts/variations? this would be interesting to report}\sx{TO ADD in app.}


% \begin{figure}[htbp]
% \centering

% \caption{Prompts Used to Query Political Preference}


% % \begin{subfigure}[b]{0.45\textwidth}
% %     \centering\includegraphics[width=0.99\textwidth]{fig/prompt_human.pdf}
% %     \caption{Human}
% %     \label{fig:prompt_human}   
% % \end{subfigure}

% \begin{subfigure}[b]{0.5\textwidth}
%     \centering\includegraphics[width=\textwidth]{fig/prompt_ab.pdf}
%     \caption{Question Template: AB}
%     \label{fig:prompt_a}   
% \end{subfigure}


%   \begin{subfigure}[b]{0.5\textwidth} 
%     \centering\includegraphics[width=\textwidth]{fig/prompt_repeat.pdf}
%      \caption{Question Template: Repeat} \label{fig:prompt_b}    
%   \end{subfigure}
  
%  \begin{subfigure}[b]{0.5\textwidth}
% \centering\includegraphics[width=\textwidth]{fig/prompt_compare.pdf}
%   \caption{Question Template: Compare} \label{fig:prompt_c}
%   \end{subfigure}

  
%   \label{fig:prompts}
%   \end{figure}


  





% \paragraph{No significant difference found in instance-level alignments}
% As shown in formular X, the instance-level Evaluation of a LLM m’s alignment, commonly takes the form of calculating A\_ins(M, G1) , which is the the average alignments /(1- distance) between m and human group G1,  and contrasting it with that of another group G2,  A\_ins(M, G2). However, the average alignment A\_ins(M, G1) >  A\_ins(M, G2) , does not necessarily mean that the difference between the two distances is significant. Here we introduce a statistical test to be performed in this case.
% Given a LLM m, we calculate its instance-level alignment A\_ins(m, G) with  different groups g isin G separately. Our research question here is: Is there a significant difference between the A\_ins(m, G) in terms of the mean?
% The null and alternative hypotheses thus result in:





%     \item The threshold can be can be either chosen in advance (0.05, 0.01), or by large-sample approximations, or by data-resampling (bootstrap) techniques
% \end{itemize}

% \begin{table}[]
    \resizebox{0.99\linewidth}{!}{
\begin{tabular}{|l|r|r|r|r|}
\hline
              & \multicolumn{1}{l|}{dem} & \multicolumn{1}{l|}{pub} & \multicolumn{1}{l|}{rep} & \multicolumn{1}{l|}{court} \\ \hline
gpt-4o        & \textbf{0.82}            & 0.8                      & 0.74                     & 0.76                       \\ \hline
llama3-70b    & \textbf{0.7}             & 0.66                     & 0.61                     & 0.61                       \\ \hline
llama3-8b     & \textbf{0.82}            & 0.8                      & 0.74                     & 0.72                       \\ \hline
gemma01       & 0.83                     & \textbf{0.84}            & 0.79                     & 0.76                       \\ \hline
olmo-instruct & 0.84                     & \textbf{0.85}            & 0.78                     & 0.78                       \\ \hline
olmo-sft      & 0.86                     & \textbf{0.87}            & 0.79                     & 0.79                       \\ \hline
gemma1        & \textbf{0.78}            & \textbf{0.78}            & 0.74                     & 0.7                        \\ \hline
T0            & 0.77                     & 0.79                     & \textbf{0.8}             & 0.69                       \\ \hline
bloomz-0      & 0.84                     & \textbf{0.87}            & 0.84                     & 0.83                       \\ \hline

\end{tabular}
    }
    \caption{Cell (i, j) represents the A\_{instance} of  llm i   to a human demographic group j. For each row of llm, \textbf{Bold} indicates best aligned human group }
\end{table}


\newpage
\begin{table*}[]

\begin{keywords}
	Mixture approximations,
	distributed time delays,
	delay differential equations,
	linear chain trick.
\end{keywords}

\caption{The distribution of choices  among the respondents, together with the Keywords used to retrieve relevant documents from the pretraining data}

% }
\label{tab:keywords}
\end{table*}


% \section{Distance-Based Alignment}
% \label{app:ins-level}
% % \subsection{Instance-level Alignment}




\begin{table*}[t]
\resizebox{0.98\linewidth}{!}{
\begin{tabular}{|l|r|r|r|r|r|r|r|r|r|}
\hline
          & gpt-4o        & llama3-70b    & llama3-8b     & olmo-instruct       & olmo-sft            & gemma1        & T0                 & bloomz-0            & random\_1           \\ \hline
\rowcolor[HTML]{C0C0C0} 
court     & 0.76          & 0.61          & 0.72          & 0.78                & 0.79                & 0.7           & 0.69               & 0.83                & 0.76                \\ \hline
\rowcolor[HTML]{C0C0C0} 
dem       & {\ul 0.82}    & {\ul 0.7}     & {\ul 0.82}    & 0.84                & 0.86                & {\ul 0.78}    & 0.77               & 0.84                & 0.77                \\ \hline
\rowcolor[HTML]{C0C0C0} 
pub       & 0.8           & 0.66          & 0.8           & {\ul \textbf{0.85}} & {\ul \textbf{0.87}} & {\ul 0.78}    & 0.79               & {\ul \textbf{0.87}} & {\ul \textbf{0.81}} \\ \hline
\rowcolor[HTML]{C0C0C0} 
rep       & 0.74          & 0.61          & 0.74          & 0.78                & 0.79                & 0.74          & {\ul \textbf{0.8}} & 0.84                & 0.78                \\ \hline
dolma17   & \textbf{0.84} & \textbf{0.72} & \textbf{0.84} & \textbf{0.85}       & 0.85                & \textbf{0.82} & 0.75               & 0.81                & 0.72                \\ \hline
redpajama & \textbf{0.84} & \textbf{0.72} & 0.82          & \textbf{0.85}       & 0.84                & 0.81          & 0.73               & 0.79                & 0.71                \\ \hline
oscar     & 0.83          & 0.71          & 0.81          & \textbf{0.85}       & 0.84                & \textbf{0.82} & 0.73               & 0.79                & 0.71                \\ \hline
c4        & 0.82          & 0.69          & 0.82          & \textbf{0.85}       & 0.85                & 0.79          & 0.79               & 0.82                & 0.73                \\ \hline
pile      & 0.81          & 0.67          & 0.77          & 0.83                & 0.81                & 0.77          & 0.76               & 0.8                 & 0.71                \\ \hline
\end{tabular}

}
\caption{Instance-level alignment. Cell (i, j) represents the $A\_{\mathrm{INS}}$ of  llm i   to a human demographic group j. For each column of llm, {\ul underlined} indicated best aligned among the human groups, \textbf{Bold} indicates best aligned overall respondent groups (human+corpora)}
\label{tab:jsd}
\end{table*}

\label{app:instance_level}


\begin{table*}[t]
\resizebox{0.99\linewidth}{!}{
\begin{tabular}{|r|r|r|r|r|r|r|r|r|r|r|}
\hline
\multicolumn{1}{|l|}{} & \textbf{gpt-4o} & \textbf{llama3-70b} & \textbf{llama3-8b} & \textbf{gemma01} & \textbf{olmo-instruct} & \textbf{olmo-sft} & \textbf{gemma1} & \textbf{T0} & \textbf{bloomz-0} & \textbf{random\_1} \\ \hline
p                      & 0.011           & 0.002               & 0.001              & 0.341            & 0.013                  & 0.026             & 0.002           & 0.006       & 0.058             & 0.003              \\ \hline
p*                     & 0.066           & 0.022               & 0.012              & 0.682            & 0.066                  & 0.104             & 0.022           & 0.048       & 0.174             & 0.027    \\ \hline
\end{tabular}
}
\caption{Anova p values. p*: Bonferroni correction}
\label{tab:instance_sig}
\end{table*}

% \begin{table}[]
    \resizebox{0.99\linewidth}{!}{
\begin{tabular}{|l|l|l|l|l|l|}
\hline
           & dem     & pub     & rep    & court\_vote & sum      \\ \hline
count      & 1955    & 6990    & 499    & 556         & 10000    \\ \hline
percentage & 19.55\% & 69.90\% & 4.99\% & 5.56\%      & 100.00\% \\ \hline
\end{tabular}
    }
    \caption{Randomness testing}
    \label{tab:simulation}
\end{table}



\paragraph{Alignment Metrics}
Many previous work employ distance-based metrics \cite{santurkar2023whose} such as Jensen--Shannon divergence (JSD), Wasserstein distance (WD); see e.g. \cite{manning1999foundations}), which define the (mis)alignment as the distance between the opinion distributions of two groups on each individual instance. The aggregated score is then used to evaluate the alignment of two groups’ opinion distributions on the questions of a certain collection. Given a LLM $D_{1}$ and a human group $D_{2}$, following \cite{santurkar2023whose}, we define the instance alignment as:
\begin{equation}\label{egAINS}
A_\mathrm{dis}\left(D_1, D_2\right)= 1- \frac{1}{|Q|} \sum_{q \in Q}\delta\left(G_1, G_2;q\right)\, ,
\end{equation}
where
\begin{equation}
  \delta\left(D_1, D_2;q\right) = \operatorname{JSD}\left(D_1(q), D_2(q)\right)\, ,
\end{equation}
is the Jensen--Shannon divergence (i.e. the symmetrised Kullback–Leibler divergence).




\paragraph{Better aligned significantly or by chance?}

The upper part of \autoref{tab:jsd} (grey background) displays the Human-LLM alignment in the SCOPE dataset. The results show that LLMs are more aligned with the general survey respondents and the self-identified democratic respondents, which confirms the midliberal leaning in previous research. 


However, observing the inequality in the average distance $A_\mathrm{dis}(M, D_1)>A_\mathrm{dis}(M, D_2)$, does not necessarily mean that the difference between the two distances is significant, as it could be due to chance.
% However, we notice that significance tests are generally not used when comparing the LLM-opinion alignment.  Thus it is possible that the greater similarity to a certain demographic group (such libertarian) over another group (such as conservative) is attributable to chance rather than a systematic improvement. 
\paragraph{Significance Test with ANOVA}
One method for hypothesis testing the significance of differences between the means of multiple entities is the ANOVA (Analysis of Variance) test. We choose to use repeated measures ANOVA in this work because the preference distributions of varaious entities are derived from the same data set, which means that they are statistically \textit{dependent}. First, we conduct repeated measures ANOVA to test if there is a significance of differences between the means of multiple entities. Should there be a significant difference between entities ( ANOVA p-value < 0.05), we perform a post hoc pairwise one-tailed T-test to test which pairs of $A_\mathrm{dis}$ have means that are significantly different from each other. 
% Our simulation shows that without significance test, it is possible that the greater similarity to a certain demographic group (such liberal) over another group (such as conservative) are attributable to chance rather than a systematic significantly better alignment: First, we generate a randomized preference distribution \footnote{populate it with random samples from a uniform distribution over $[0, 1)$}, denoted as $D_r$, to simulate a 'pseudo LLM.' Next, we evaluate the instance-level alignment of $D_r$ with various human groups using the (JSD) metric. We then identify the human group to which the pseudo LLM is most aligned. This process is repeated across 10,000 random trials, each using a different random seed. Finally, we count the number of times each human group is identified as the most aligned with the pseudo LLM. The results in \autoref{tab:simulation} show that around $70\%$ of the time the randomized preference distribution is most aligned with the overall human respondents and $20\%$ with the democrats. We have also tried bootstrapping test and found similar results. 
% See Appendix for the details. \qv{This is appendix, maybe de}



% Given a LLM m, we calculate its instance-level alignment A\_ins(m, G) with  different groups g \isin G separately. In order to determine whether there is a significant difference between the A\_ins(m, G) in terms of the mean, a statistics test is needed:


% \begin{itemize}
%     \item Null hypothesis H\_0: there are no significant differences between the alignments of M and different human groups.
%     \item Alternative hypothesis H\_1: there is a significant difference between the alignments of M and different human groups.
% \end{itemize}


% if p < threshold, corr(Y1, X) shows how a certain LLM Y1 is correlated with a certain human group (X) (T2)
% But, a corr(Y1, X) > corr(Y2, X) can’t prove Y1 is more strongly related to Y2.
% As Y1, Y2 are dependent, measured on the same set of questions. So we use Repeated Measures ANOVA The ANOVA (Analysis of Variance) test is used to determine whether there are statistically significant differences between the means of three or more groups. 
% To control for Type I errors (false positives), Once the ANOVA shows a significant effect, we perform multiple pairwise post hoc comparisons by running  one-tail Paired t-tests with Bonferroni Correction. 

% \paragraph{Significance Test Results}
Table \ref{tab:instance_sig} shows the p-values of the ANOVA test. In general, most LLMs show no significant differences in alignment with different entities, except LLaMA 3-70B, LLaMA 3-B, and Gemma. A paired t-test conducted on these three models revealed significant differences in only two specific cases: (1) the alignment of Gemma with DOLMA compared to its alignment with the Court and (2) the alignment of Gemma with OSCAR compared to the Court. These findings suggest that, in general, LLMs do not show substantial variation alignment between different entities when evaluated in a distance-based measure. 






% \subsection{System-level Alignment}





\section{Keyword List} 
\label{app:keywords}
We define the keywords for each case as [keyword 1, keyword 2, plaintiff, defendant], with the two keywords derived from Jesse's original dataset. We manually adjusted some keywords as necessary to refine the search scope. Including the names of the parties enhances the precision of document retrieval, because in the U.S., cases are typically cited using the names of the parties involved in the format ``plaintiff v. defendant". When acronyms or abbreviations are commonly used, we manually edit the party names for better retrieval result; for example, we use NCAA instead of the full name ``National Collegiate Athletic Association". The complete list of keywords of all cases are available in \autoref{tab:keywords}. An example of a retrieved document is provided in \autoref{app:example_doc_text}.

\section{Relevant Documents Retrieval} We used the WIMBD API \cite{elazar2024whats} to retrieve documents based on defined keywords. Due to the API and token limitations of LLama3, we retrieved only documents with word counts below this threshold. \autoref{fig:doc_len_distribution} displays the distribution of document lengths, showing that most contain fewer than 4,000 words. 
% \autoref{fig:fetched_doc} illustrates . 
\autoref{tab:stance_stats} provides additional statistics such as the number of documents matching the keywords in the Dolma dataset (\textit{documents matched}) and the subset we fetched (those with fewer than 4,000 words, \textit{documents fetched})

\begin{figure*}[]
\centering
\includegraphics[width=0.75\linewidth]{fig/prompt_stance.pdf}
\caption{Prompt used to evaluate the stance scores of the retrieved documents.}
\label{fig:prompt_stance}
\end{figure*}

\section{Quality Assessment of Stance Detection}
\label{app:stance_quality}
To evaluate the quality of LLaMA3-70B’s stance detection, we conducted a two-round quality assessment.
In the first round, we randomly sampled 20 documents from the retrieved relevant documents. Two annotators independently labeled the documents: Annotator 1, a research assistant who is a native English speaker and a U.S. citizen, and Annotator 2, the first author of this paper. The annotation process followed the exact template used to prompt LLaMA3-70B, as shown in \autoref{fig:prompt_stance}. The inter-annotator agreement, measured by Spearman’s $\rho$, was 0.76. The Spearman’s $\rho$ between Annotator 1 and LLaMA3-70B’s labels was 0.7.
In the second round, Annotator 1 labeled an additional 40 documents. The overall Spearman’s $\rho$ between Annotator 1 and LLaMA3-70B’s labels across all 60 documents was 0.68. Based on this, we consider the alignment between LLaMA3-70B’s outputs and human annotations to be strong.

\begin{figure*}[]
\centering
\includegraphics[width=0.9\linewidth]{fig/ci.png}
\caption{Bootstrapped sample means and their 95\% confidence intervals for each docket. Each bar represents the average stance score for a given case docket, while the error bars denote the 5th and 95th percentiles of the bootstrap distribution (based on repeatedly sampling 80\% of the data).
}
\label{fig:ci}
\end{figure*}


\paragraph{Bootstrap Resampling} 
We applied a bootstrap resampling procedure to assess the robustness of political stance score estimation. For each of the 32 cases in SCOPE, we generated 100 bootstrap samples by randomly subsampling 80\% of its retrieved documents' stance scores. The mean score was computed for each subsample, creating a bootstrap distribution of means. We derived 95\% confidence intervals (CIs) using the percentile method, with bounds defined by this distribution's 5th and 95th percentiles. The sample mean (calculated on the full dataset) and its CI bounds were recorded for all dockets. As shown in \autoref{fig:ci} , all sample means lie within their respective CIs, confirming the reliability of our estimates and quantifying their variability.

% \end{landscape}



\begin{table*}[h]
\rotatebox{90}{
    \centering
    \begin{tabular}{l|r|r|r|r|r|r}
\hline
docket & Survey wave & \# doc fetched & avg. length doc fetched & avg. stance score & \# doc matched & avg. length doc matched \\ \hline
19-123    & 2021 &    59 &  1,346 &  3.40 &   113 &   11,111 \\
19-1257A  & 2021 &   124 &  1,369 &  2.27 &   165 &    5,414 \\
19-1257B  & 2021 &   384 &  1,148 &  4.42 &   441 &    4,900 \\
19-251    & 2021 &   338 &    909 &  4.13 &   396 &    2,784 \\
20-255    & 2021 &   520 &    978 &  4.23 &   577 &    1,739 \\
18-1259   & 2021 &   150 &  2,256 &  4.01 & 2,139 &   49,361 \\
19-783    & 2021 &   257 &  1,573 &  4.27 & 1,205 &   24,803 \\
20-512    & 2021 &   602 &  1,292 &  1.46 &   683 &    3,830 \\
20A87     & 2021 &    96 &  1,848 &  3.57 &   206 &   12,722 \\
20-107    & 2021 &   282 &    993 &  1.47 &   300 &    1,592 \\
19-422    & 2021 &   277 &  1,862 &  3.19 &   825 &   22,763 \\
20-18     & 2021 &    57 &  1,323 &  4.76 &   488 &  122,176 \\
17-1618   & 2020 &    87 &  1,513 &  4.75 &   102 &    4,025 \\
18-107    & 2020 &   405 &  1,246 &  4.87 &   562 &    5,803 \\
18-587    & 2020 & 2,005 &  1,047 &  4.31 & 2,216 &    2,311 \\
18-1195   & 2020 &   224 &  1,318 &  4.02 &   292 &    4,835 \\
19-431    & 2020 &   699 &  1,204 &  3.99 &   908 &    6,311 \\
18-1323   & 2020 &   185 &  1,712 &  3.60 &   253 &    7,368 \\
19-635    & 2020 &   878 &  1,181 &  1.56 & 1,214 &    6,384 \\
19-715    & 2020 &   456 &  1,216 &  1.51 &   693 &   12,316 \\
19-7      & 2020 & 1,047 &  1,159 &  4.19 & 1,205 &    2,920 \\
19-465    & 2020 &   814 &  1,040 &  3.63 &   895 &    2,015 \\
08-1224   & 2010 &   316 &  1,389 &  2.21 &   565 &   11,954 \\
08-1521   & 2010 & 1,570 &  2,380 &  4.47 & 5,903 &   32,346 \\
08-472    & 2010 &   108 &  1,165 &  2.62 &   172 &   13,837 \\
07-1428   & 2010 &    72 &  1,884 &  3.12 &   141 &   10,650 \\
07-21     & 2010 &   727 &  1,366 &  3.08 &   945 &    5,139 \\
08-205    & 2010 &   187 &  2,613 &  2.89 &   731 &   20,631 \\
07-5439  & 2010 &   927 &  1,458 &  2.25 & 1,360 &    5,799 \\
05-908    & 2010 &   471 &  1,802 &  2.73 &   974 &   15,752 \\
05-380    & 2010 &   402 &  2,012 &  3.53 &   983 &   12,542 \\
05-184    & 2010 &    56 &  2,419 &  3.17 &   176 &   20,370 \\ \hline
\end{tabular}

    }
    \caption{Descriptive statistics of the documents retrieved from the Dolma dataset.}
    \label{tab:stance_stats}

\end{table*}






\begin{figure*}[]
\centering
\includegraphics[width=0.8\linewidth]{fig/doc_len_distribution.pdf}
\caption{Distribution of length of all the matched documents.}
\label{fig:doc_len_distribution}
\end{figure*}


\begin{figure*}[]
\centering
\includegraphics[width=\linewidth]{fig/corpora_heatmap.png}
\caption{Pearson Alignment. Cell $(i, j)$ represents the Pearson correlation $\rho$ of LLM $i$ to entity $j$. $*$ shows *p* value < 0.05, $**$ shows p-value < 0.001. }
\label{fig:corpora_heatmap}
\end{figure*}

\section{Corpora-Human Alignment}
\label{app:corpora_human}
\autoref{fig:corpora_heatmap} presents the alignments between different training corpora and surveyed human opinions. The political leanings of these pretraining corpora appear to be quite similar; however, they differ from those of the human respondents surveyed. Further, among the 5 corpora, DOLMa, RedPajama and OSCAR high correlation to each other. They are less correlated to C4 and the Pile, which might be due to the different curation process of the dataset.


\section{Post-training}
\label{app:post-training}

Previous research report that LLMs that have undergone human-alignment procedures tend to have stronger political views\cite{santurkar2023whose,perez-etal-2023-discovering}. Therefore, we also investigated the correlation between OLMO’s political leanings and the stance scores from the instruction-tuning dataset TULU , as well as the RLHF dataset UltraFeedback. However, no significant correlation was observed. This could be attributed to the small size of the documents, and only limited number of relevant documents retrieved from these datasets—only 15 out of the 32 cases had relevant documents in TULU, and just 10 cases had relevant documents in UltraFeedback. Prior research \cite{feng-etal-2023-pretraining} also suggests that the shift introduced by post-training is relatively small. We also explored the correlation between LLMs' political leanings and that in their post-training data, but did not observe any significant correlation. 

The key difference between OLMo-SFT (Supervised Fine-Tuning) and OLMo-Instruct lies in their fine-tuning objectives and intended uses.
OLMo-SFT is fine-tuned for general language tasks using labeled data, using the TULU dataset \cite{ivison2023camels}. It is optimized for structured responses but isn’t specifically trained to follow user instructions. OLMo-instruct is further fine-tuned to follow human instructions, using the Ultrafeedback dataset \cite{cui2023ultrafeedback}. It is optimized for handling detailed user instructions and conversational prompts, ideal for interactive and task-oriented use.


\section{Williams Test Results}
\label{app:williams}
\autoref{fig:williams_app} includes a comprehensive overview of the Williams Test results of all LLMs.


\begin{figure*}[]
\centering
\includegraphics[width=0.9\linewidth]{fig/williams9.png}
\caption{Bootstrapped sample means and their 95\% confidence intervals for each docket. Each bar represents the average stance score for a given case docket, while the error bars denote the 5th and 95th percentiles of the bootstrap distribution (based on repeatedly sampling 80\% of the data).
}
\label{fig:williams_app}
\end{figure*}

\section{Example of a Retrieved Document}
\label{app:example_doc_text}
\autoref{fig:doc_len_distribution} demonstrates the full text of a relevant document we retrieved from the pretraining dataset Dolma. The document is on case \textit{McDonald v. Chicago} about the topic of gun control:\\

\onecolumn

\begin{tcolorbox}[width=\linewidth,title={Example of a Retrieved Document}]
\footnotesize
   % \begin{itemize}
   %     \item We also evaluated instance-level alignment of LLMs with different groups by using distance-based metrics such as Jensen–Shannon divergence (cite) and Wasserstein distance (cite). Though at the first glace, the result seems to confirm the prior studies: LLMs display a mid-liberal leanings.
   %     \item However, extra significance test shows generally there are no significant difference in the means of LLMs' alignment with different groups.
   %     \item Further randomized simulation test also show that the mid-liberal leaning can be attributed to chance because of the difference of different groups' opinion distribution.
   % \end{itemize}

   In the run-up to the 2008 presidential election, many gun owners worried about the consequences of victory for Democrat candidate Barack Obama. Given Obama’s record as an Illinois state senator, where he stated his support for an all-out ban on handguns, among other gun control stances, pro-gun advocates were concerned that gun rights might suffer under an Obama presidential administration.\\
After Obama’s election, gun sales reached a record pace as gun owners snatched up guns, particularly those that had been branded assault weapons under the defunct 1994 assault weapons ban, out of an apparent fear that Obama would crack down on gun ownership. The Obama presidency, however, had limited impact gun rights.\\
When Obama was running for the Illinois state senate in 1996, the Independent Voters of Illinois, a Chicago-based non-profit, issued a questionnaire asking if candidates supported legislation to ``ban the manufacture, sale, and possession of handguns,” to ''ban assault weapons” and to instate “mandatory waiting periods and background checks” for gun purchases. Obama answered yes on all three accounts.\\
Obama also cosponsored legislation to limit handgun purchases to one per month. He also voted against letting people violate local weapons bans in cases of self-defense and stated his support for the District of Columbia’s handgun ban that was overturned by the U.S. Supreme Court in 2008. He also called it a “scandal” that President George W. Bush did not authorize a renewal of the Assault Weapons Ban.\\
Just weeks after Obama’s inauguration in January 2009, attorney general Eric Holder announced at a press conference that the Obama administration would be seeking a renewal of the expired ban on assault weapons.\\
``As President Obama indicated during the campaign, there are just a few gun-related changes that we would like to make, and among them would be to reinstitute the ban on the sale of assault weapons,” Holder said.\\
U.S. Rep. Carolyn McCarthy, D-New York, introduced legislation to renew the ban. However, the legislation did not receive an endorsement from Obama.\\
In the aftermath of a mass shooting in Tucson, Ariz., that wounded U.S. Rep. Gabrielle Giffords, Obama renewed his push for “common sense” measures to tighten gun regulations and close the so-called gun show loophole.\\
While not specifically calling for new gun control measures, Obama recommended strengthening the National Instant Background Check system in place for gun purchases and rewarding states supplying the best data that would keep guns out of the hands of those the system is meant to weed out. Later, Obama directed the Department of Justice to begin talks about gun control, involving ``all stakeholders” in the issue. The National Rifle Association declined an invitation to join the talks, with LaPierre saying there is little use in sitting down with people who have ``dedicated their lives” to reducing gun rights. As the summer of 2011 ended, however, those talks had not led to recommendations by the Obama administration for new or tougher gun laws.\\
One of the Obama administration’s few actions on the subject of guns has been to strengthen a 1975 law that requires gun dealers to report the sale of multiple handguns to the same buyer. The heightened regulation, which took effect in August 2011, requires gun dealers in the border states of California, Arizona, New Mexico and Texas to report the sale of multiple assault-style rifles, such as AK-47s and AR-15s.\\
The story through much of his first term in office was a neutral one. Congress did not take up serious consideration of new gun control laws, nor did Obama ask them to. When Republicans regained control of the House of Representatives in the 2010 midterm, chances of far-reaching gun control laws being enacted were essentially squashed. Instead, Obama urged local, state, and federal authorities to stringently enforce existing gun control laws. In fact, the only two major gun-related laws enacted during the Obama administration’s first term actually expand the rights of gun owners.\\
The first of these laws, which took effect in February 2012, allows people to openly carry legally owned guns in national parks. The law replaced a Ronald Reagan era policy that required guns to remain locked in glove compartments or trunks of private vehicles that enter national parks. The other law allows Amtrak passengers to carry guns in checked baggage; a reversal of a measure put in place by President George W. Bush in response to the terrorist attacks of Sept. 11, 2001.''
Obama’s two nominations to the U.S. Supreme Court, Sonia Sotomayor, and Elena Kagan were considered likely to rule against gun owners on issues involving the Second Amendment. However, the appointees did not shift the balance of power on the court. The new justices replaced David H. Souter and John Paul Stevens, two justices who had consistently voted against an expansion of gun rights, including the monumental Heller decision in 2008 and McDonald decision in 2010.\\
Earlier in his first term, Obama had expressed his express support for the Second Amendment. ``If you’ve got a rifle, you’ve got a shotgun, you’ve got a gun in your house, I’m not taking it away. Alright?” he said. However, the legislation to overhaul gun control failed on April 17, 2013, when the Republican-controlled Senate rejected a measure banning assault-style weapons and expanding gun-buyer background checks. \\
In January 2016, President Obama began his final year in office by going around the gridlocked Congress by issuing a set of executive orders intended to reduce gun violence. According to a White House Fact Sheet, the measures aimed to improve background checks on gun buyers, increase community safety, provide additional federal funding for mental health treatment, and advance the development of ``smart gun” technology. \\
During his eight years in office, President Barack Obama had to deal with more mass shootings than any of his predecessors, speaking to the nation on the subject of gun violence at least 14 times. In each address, Obama offered sympathy for the loved ones of the deceased victims and repeated his frustration with the Republican-controlled Congress to pass stronger gun control legislation. After each address, gun sales soared.\\
In the end, however, Obama made little progress in advancing his ``common-sense gun laws” at the federal government level — a fact he would later call one of the biggest regrets of his time as president.\\
In 2015, Obama told the BBC that his inability to pass gun laws had been``the one area where I feel that I've been most frustrated and most stymied.
\label{fig:example_doc_text}
\end{tcolorbox}


