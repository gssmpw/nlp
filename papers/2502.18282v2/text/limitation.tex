\paragraph{Multi-choice Format} 
Our work probes LLMs' political views using questions from a public opinion survey, requiring LLMs to answer in a binary-choice format. However, the methodology laid out in this article does not rely on the binary format. Correlation coefficients, the Williams test and the Jensen--Shannon divergence immediately generalize to more refined analysis of political biases, such as continuous distributions, multiple-choice formats or clusterings. Recent research \cite{rottger-etal-2024-political} indeed suggests that such constrained formats may not accurately reflect real-world LLM usage, where users tend to talk in open-ended discussions on controversial topics \cite{ouyang-etal-2023-shifted}. They also found in unconstrained settings, LLMs may respond differently than when restricted to a fixed set of options. We leave this question to future analysis. Furthermore, we point out that in certain real-world applications, such as voting assistants \cite{chalkidis-2024-investigating}, often necessitate LLMs to function within a binary or multiple-choice framework. 


\paragraph{Partisan Aggregation in Political Alignment Analysis}
Our analysis compares LLMs’ political leanings to human survey responses aggregated by partisan groups, such as Democrats and Republicans. However, this approach has inherent limitations. Political opinions on controversial issues can resist strict partisan categorization, as individuals within the same party do not always align neatly with partisan divisions, as individuals within the same party may hold diverse or even opposing views. Recent research has highlighted the pluralism of human opinions and proposed incorporating fine-grained human values into AI systems \cite{plank-2022-problem, xu-etal-2024-lens,pmlr-v235-sorensen24a}. Future research could explore LLMs’ response uncertainty—using metrics such as entropy or confidence scores across multiple generations—to assess whether these models capture the ambiguity of opinions on contentious topics. We call for more work to contribute to aligning LLMs with pluralistic human values.

\paragraph{U.S.-Centric Perspectives}  
While the expert-chosen cases within SCOPE address contentious issues and serve as strong indicators of political orientation, the framework is not without its limitations. Notably, akin to other political surveys employed in recent LLM evaluation studies (e.g.  ANES in \citealt{Bisbee_Clinton_Dorff_Kenkel_Larson_2024}), SCOPE is based on U.S. centric public opinion data and focuses on the American partisan political ideology. This emphasis constrains its applicability when assessing LLMs that have been trained on multilingual or globally diverse datasets, as showed in our experiment results on the BLOOMZ model. 
Despite these limitations, we propose a method that enables comparisons between the alignment of LLMs with the surveyed human opinions and their pretraining corpora, thus enabling flexibility across various ideological frameworks or questionnaires. We encourage future research to adopt our approach on alternative ideological theories and political surveys. This will contribute to a more comprehensive understanding of LLMs' political positioning.


