\section{Related Work}
\noindent  \textbf{SoC for HPC.} The landscape of HPC has evolved from traditional architectures that rely on separate CPUs and GPUs to heterogeneous SoC designs that integrate multiple processing elements____.  This integration facilitates zero-copy memory access, significantly reducing data transfer overheads and enabling more efficient computation____. Previous studies have demonstrated that SoC-based architectures can lead to substantial performance gains in various HPC workloads. For example, optimized zero-copy matrix computations have been shown to reduce execution costs by 34\%____, while specialized SoC implementations have improved plasma detection accuracy and processing speed____. \\

\noindent  \textbf{Apple Silicon for HPC Applications.} The introduction of the Apple M1 marked a milestone in ARM-based HPC, influencing other major chip manufacturers to accelerate their ARM SoC development____. Although ARM architectures do not inherently provide power efficiency advantages over x86 alternatives____, the M1 chips have demonstrated competitive performance with low power consumption, making them viable for scientific and HPC applications____. Recent studies have explored using Apple Silicon’s unified memory architecture and integrated accelerators for diverse computational workloads. For instance, investigations into machine learning applications have shown performance benefits in training various classifiers____ and image classification tasks____. n cryptography, Apple Silicon’s specialized instructions have been utilized to improve execution efficiency and memory footprint in secure computation algorithms____.  In addition, previous work in scientific computing -- ranging from real-time rendering____ and physics simulations ____ to chemical engineering applications____ -- indicates promising Apple Silicon SoC energy-efficient computation capabilities.