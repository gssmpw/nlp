\section{Background and related work}
\subsection{Amazon's growing market dominance}
\label{section:background:amazon}

By 2018 more than 60\% of US consumers had bought an item through Amazon____ and by 2024 more than 50\% (180 million) were Amazon Prime (subscription service) members____. Here we highlight substantial investments Amazon made to acquire its significant market share, and hence shopper data, which we revisit in analyses.

Amazon began by selling books, strategically acquired other book sellers, and by 1998 referred to itself as "Earth's Biggest Bookstore"____.
In 2009 Amazon acquired the leading online footwear company, Zappos____, and in 2017 Amazon acquired Whole Foods Market Inc. for \$13.7 billion, where reporters attributed this large spend as an opportunity to acquire data____. 
Beyond acquisitions, Amazon has invested in creating a platform where third-party sellers offer a broader array of products than any one company could manufacture or source____, providing an even broader array of purchases data. This was highlighted in 2020 by Amazon's CEO:  "Third-party sales now account for approximately 60\% of physical product sales on Amazon, [...] growing faster than Amazon’s own retail sales"____.

Amazon has also made acquisitions in the healthcare space, acquiring the online pharmacy PillPack in 2019____ and One Medical, a chain of primary care clinics, in 2022.  
Again, reporters described this as a data opportunity____.  Although HIPAA rules____ protect health records from the rest of the Amazon business, we ask if there are risks due to the lack of protection in the other direction. Could purchases data be used to help determine how medical services are targeted? In this work we show how Amazon purchases can reveal health information, such as whether someone has diabetes.


\subsection{Demographics and inference}

Related work shows the utility of demographic data in predicting purchases and informing business models____, including studies specific to online retail____. Given the utility of consumer demographics, inferring demographics from purchases is also an active area of research, where the goal is to provide businesses with more data on their customers, to improve their market strategy____. These previous studies have used data provided by retailers____ and vendor loyalty program providers____.
Our analyses also infer consumer demographics from purchases data. 
Yet in contrast, our goal is to measure related privacy risks using data crowdsourced from consumers, rather than optimizing model utility or accuracy.

There is also related work analyzing the privacy risks of purchases data, where the focus is on re-identification risks____.
For example, De Montjoye et al. used 3 months of credit card records to show how easily users could be re-identified in a de-identified dataset, and showed risks were higher for women versus men____. Further research suggests re-identification risks are understated in marketing datasets due to their longitudinal nature____.

Other work shows how different types of users' digital traces, such as social media Likes____ or web browsing histories____, can be used to infer sensitive sociodemographics. In an influential 2013 paper, Kosinski et al. used a dataset of 58k Facebook users' Likes and demographic profiles to show how well a model could infer users' personal attributes from the Likes, reporting AUC scores of 0.93 for predicting gender, and 0.73 for cigarette use____. Our paper adds to this body of work by highlighting similar risks with purchases.