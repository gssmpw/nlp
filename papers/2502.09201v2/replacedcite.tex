\section{Related Work}
In a classical environment, it has been proved that OT implies public-key encryption____. 
This means that the security of these protocols is based on quantum weak or unproven assumptions. 
Quantum cryptography offers the possibility to build OT protocols whose security is based solely on symmetric key primitives ____.
Specifically, unconditionally binding commitment schemes are essential ____ for a practical version of the most relevant quantum oblivious transfer protocol known in the literature ____, even if simulation-based security is not a must ____.

Despite the strong security assumption behind qOT, execution is very slow, also due to the huge amount of data to be exchanged during the post-processing phase required to deal with imperfect quantum channels ____.
In our experiments, the commitment phase appeared as one substantial bottleneck in qOT execution.
Indeed, a way to commit to many bits (states) and open the commitment of only some of them is needed, ideally with an unconditionally binding commitment scheme whose security is based only on one-way functions.