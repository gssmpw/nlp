\section{Related Work}
\label{sec:related-work}

____ proposed a method for generating samples from low-density regions of diffusion models. However, due to the intractability of likelihood in diffusion models, their approach relies on approximations. Subsequent work by ____ demonstrated that likelihood is, in fact, tractable even in stochastic diffusion models, challenging the need for such approximations. Building on this, our proposed methods provide explicit likelihood control for both deterministic sampling—via Prior Guidance (\autoref{sec:scaling}) and Density Guidance (\autoref{sec:dgs})—and stochastic sampling through Stochastic Density Guidance (\autoref{sec:sold}).

____ observed that scaling the latent code alters the amount of detail in deterministically generated images.
While this phenomenon has been widely acknowledged, we provide a rigorous analysis (\autoref{sec:scaling}) and prove that it is a direct consequence of Score Alignment \autoref{eq:vfa}, which guarantees that scaling leads to a monotonic change in the likelihood of the generated image, $\log p_0(\vx_0)$.
Furthermore, we introduce tractable numerical tools (\autoref{fig:vfa-verification}) that can verify whether any given CNF model (not necessarily score-based) exhibits this behavior.

____ proposed auto-guidance as a method for improving sample quality by targeting high-density regions. However, ____ found that the highest-density regions in diffusion models contain cartoon-like or blurry images, which raises concerns about the effectiveness of purely maximizing likelihood. In contrast, we introduce multiple cost-free methods for explicitly controlling the likelihood of generated samples. Additionally, while ____ observed that scaling the score function leads to oversimplified images, we demonstrate that DGS \autoref{eq:dgs} enables effective control over image detail—both increasing and decreasing it—when the scaling is adapted both temporally and spatially \autoref{eq:quantile-score-scaling}.

____ introduced Riemannian Langevin Dynamics, an SDE with a non-diagonal diffusion matrix, similar in structure to our Stochastic Density Guidance (\autoref{sec:sold}). However, a key distinction is that our diffusion matrix is a projection onto the orthogonal complement of the subspace spanned by the score function. As a result, it is not positive definite and cannot serve as a Riemannian metric tensor, making our approach fundamentally different in its mathematical formulation and behavior.

Recently, ____ proposed a method for measuring local intrinsic dimension, which, in the case of images, corresponds to the amount of detail present. However, we show that negative $\log p$ is a more effective measure of image detail and provide empirical comparisons in \autoref{fig:logp-vs-flipd}. Moreover, while ____ focus on measuring image detail, our methods enable direct manipulation of it, allowing for finer control over generative model outputs.