\section{Related Work}
For related works, the paper focuses on contemporary solutions incorporating multi-modal input methodologies to supply information to a LLMs and RAG solutions.

While there are several research studies on enhancing human-computer interaction (HCI) using classical AI-aided solutions\cite{10257282, 10181189}, in recent years, there has been a surge of approaches focusing on advanced multimodal systems and voice-assisted technologies. A notable contribution in this domain is GazePointAR \cite{10.1145/3613904.3642230}, which presents a context-aware voice assistant aimed specifically at wearable augmented reality (AR) devices. This system addresses the common issue of pronoun disambiguation faced by current voice assistants, such as Amazon Alexa and Google Assistant, which often struggle with contextual understanding. By integrating real-time computer vision capabilities to analyze the user's gaze, action and behaviour understanding \cite{BOUR2019289, 6977392}, pointing gestures, 3D representations \cite{4587762, ARGYRIOU20091}, and conversation history, GazePointAR constructs coherent queries that improve the accuracy of responses. This advancement underscores a shift toward utilizing multimodal interactions to create more natural dialogues and efficient HCI experiences.

Continuing the exploration of immersive interfaces, PANDALens \cite{10.1145/3613904.3642320} introduces an AI-assisted narrative documentation system for Optical See-Through Head Mounted Displays (OHMD). Designed to capture life moments in real-time and minimize user distraction, this tool emphasizes proactive engagement through a mixed-initiative interaction model. By analyzing real-time contextual cues, PANDALens provides automatic suggestions and question prompts, thus significantly enhancing the quality of narrative documentation. Nevertheless, the authors acknowledge challenges related to the accurate interpretation of user intent, particularly in dynamic environments, pointing to the unresolved complexities within multimodal AI interactions.

On another front, there is a growing focus on improving explainability and generalizability within autonomous driving systems. The RAG-Driver framework proposed by Yuan et al. \cite{yuan2024ragdrivergeneralisabledrivingexplanations} exemplifies this trend by introducing a Multi-Modal Large Language Model (MLLM) that offers natural language explanations alongside control signals, thus enhancing user understanding of autonomous decision-making processes. This methodology contrasts with previous opaque end-to-end systems, as RAG-Driver facilitates a two-way interaction that bolsters user trust. The results illustrate significant advancements in explainability and performance metrics, laying the groundwork for improved interpretability in autonomous driving contexts.

In the realm of conversational systems, several studies investigate the role of RAG in enhancing user interaction. A suggestion question generator outlined in one paper \cite{10.1145/3589335.3651905} addresses common difficulties users face when formulating queries, proposing a framework to generate relevant suggestion questions based on initial user inputs. This dynamic contextual approach effectively improves conversational flow, revealing a need for systems that facilitate informed user interactions. Complementarily, the work on AI-safe Autocompletion \cite{10.1145/3627673.3679078} seeks to enhance the complexity of autocomplete suggestions in e-commerce search systems by integrating RAG with safety measures and relevance ranking, showcasing practical implications for conversational AI.

Another pivotal addition to the RAG landscape is REAPER (Reasoning-based Retrieval Planner) \cite{10.1145/3627673.3680087}, designed to improve latency and scalability in conversational shopping assistants. By employing a solitary, smaller language model to generate retrieval plans, REAPER enhances the efficiency of multi-step query handling, reducing response generation times dramatically. Through comparative experiments, REAPER's effectiveness is showcased, potentially setting a new benchmark for future applications in complex query contexts.

A broader synthesis of these advancements can be found in the survey on RAG technologies \cite{10.1145/3637528.3671470}, which provides an extensive overview of RAG techniques concerning integration with LLMs. This comprehensive review categorizes existing research into architectures, training strategies, and applications, while also identifying persistent challenges and proposing future research directions aimed at enhancing the reliability and multilingual capabilities of RA-LLMs. Furthermore, RAG-Ex \cite{10.1145/3626772.3657660} introduces a model-agnostic framework that enhances the explainability of RAG systems, addressing the crucial need for transparency in AI outputs to foster user trust. Collectively, these studies reflect a growing emphasis on usability, explainability, and user engagement across various AI-driven environments.