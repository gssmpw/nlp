\documentclass[journal]{IEEEtran}

% correct a bug for the bibliography
\def\endthebibliography{%
  \def\@noitemerr{\@latex@warning{Empty `thebibliography' environment}}%
  \endlist
}

% *** CITATION PACKAGES ***
%
\usepackage{cite}

% *** GRAPHICS RELATED PACKAGES ***
%
\usepackage[pdftex]{graphicx}
% declare the path(s) where your graphic files are
\graphicspath{{./imgs/pdf/}{./imgs/jpeg/}{./imgs/png/}{./imgs/eps/}}
% and their extensions so you won't have to specify these with
% every instance of \includegraphics
\DeclareGraphicsExtensions{.pdf,.jpeg,.png, eps}
\usepackage[colorlinks=true, linkcolor=blue, citecolor=blue, urlcolor=blue]{hyperref}

% *** MATH PACKAGES ***
%
\usepackage{amsmath}
\usepackage{amssymb}

% Grassetto per minuscole greche
\usepackage{bm}
\usepackage{booktabs}

% *** SPECIALIZED LIST PACKAGES ***
%
\usepackage{algorithmic}
% algorithmic.sty was written by Peter Williams and Rogerio Brito.

% *** ALIGNMENT PACKAGES ***
%
\usepackage{array}

% *** SUBFIGURE PACKAGES ***
% Honestly, I prefer something else
 \usepackage[caption=false,font=footnotesize]{subfig}
\usepackage{svg}
% \svgsetup{inkscapelatex=false}
% \usepackage{svg-extract}
% *** FLOAT PACKAGES ***
%

\usepackage{stfloats}

% *** PDF, URL AND HYPERLINK PACKAGES ***
%
\usepackage{url}

% *** OTHER PACKAGES ***
%
\usepackage{lipsum}
\usepackage{siunitx}
\usepackage{amsthm}
\usepackage[nolist,nohyperlinks]{acronym}
\newacronym{rl}{RL}{Reinforcement Learning}
\newacronym{drl}{DRL}{Deep Reinforcement Learning}
\newacronym{mdp}{MDP}{Markov Decision Process}
\newacronym{ppo}{PPO}{Proximal Policy Optimization}
\newacronym{sac}{SAC}{Soft Actor-Critic}
\newacronym{epvf}{EPVF}{Explicit Policy-conditioned Value Function}
\newacronym{unf}{UNF}{Universal Neural Functional}

% correct bad hyphenation here
\hyphenation{op-tical net-works semi-conduc-tor}

% % baseline
% \renewcommand{\baselinestretch}{0.965}

% Color for the draft
\usepackage{xcolor}

% Theorem
\newtheorem*{remark}{Remark}
\renewcommand\qedsymbol{$\blacksquare$}

%% New Commands %%
\newcommand{\fourier}[1]{\mathcal{F}\left\{#1\right\}}
\newcommand{\rect}[1]{\textnormal{rect}\left(#1\right)}
\newcommand{\sinc}[1]{\textnormal{sinc}\left(#1\right)}
\newcommand{\abs}[1]{\left\lvert#1\right\rvert}
\newcommand{\norm}[1]{\left\lVert#1\right\rVert}

\begin{document}
\title{SoFFT: Spatial Fourier Transform for Modeling Continuum Soft Robots}
%
%
% author names and IEEE memberships
% note positions of commas and nonbreaking spaces ( ~ ) LaTeX will not break
% a structure at a ~ so this keeps an author's name from being broken across
% two lines.
% use \thanks{} to gain access to the first footnote area
% a separate \thanks must be used for each paragraph as LaTeX2e's \thanks
% was not built to handle multiple paragraphs
%

%% Uncomment for the final version %%
\author{Daniele~Caradonna$^{1, 2}$, \and Diego~Bianchi$^{1, 2}$, Franco~Angelini$^{3,4}$, \and Egidio~Falotico$^{1, 2}$
\thanks{$^{1}$ The BioRobotics Institute, Scuola Superiore Sant'Anna, Pisa, Italy.
$^{2}$ Department of Excellence in Robotics and AI, Scuola Superiore Sant'Anna, Pisa, Italy.
$^{3}$Centro di Ricerca ``En\-ri\-co Pi\-ag\-gio'', U\-ni\-ver\-si\-t\`{a} di Pisa, Largo Lucio Lazzarino 1, 56126 Pisa, Italy. 
%}\thanks{
$^{4}$Dipartimento di Ingegneria dell'Informazione,  U\-ni\-ver\-si\-t\`{a} di Pisa, Largo Lucio Lazzarino 1, 56126 Pisa, Italy. 
{\footnotesize \tt daniele.caradonna@santannapisa.it}}
}

% note the % following the last \IEEEmembership and also \thanks - 
% these prevent an unwanted space from occurring between the last author name
% and the end of the author line. i.e., if you had this:
% 
% \author{....lastname \thanks{...} \thanks{...} }
%                     ^------------^------------^----Do not want these spaces!
%
% a space would be appended to the last name and could cause every name on that
% line to be shifted left slightly. This is one of those "LaTeX things". For
% instance, "\textbf{A} \textbf{B}" will typeset as "A B" not "AB". To get
% "AB" then you have to do: "\textbf{A}\textbf{B}"
% \thanks is no different in this regard, so shield the last } of each \thanks
% that ends a line with a % and do not let a space in before the next \thanks.
% Spaces after \IEEEmembership other than the last one are OK (and needed) as
% you are supposed to have spaces between the names. For what it is worth,
% this is a minor point as most people would not even notice if the said evil
% space somehow managed to creep in.

% % The paper headers
% \markboth{Journal of \LaTeX\ Class Files,~Vol.~14, No.~8, October~2023}%
% {Shell \MakeLowercase{\textit{et al.}}: Bare Demo of IEEEtran.cls for IEEE Journals}
% % The only time the second header will appear is for the odd numbered pages
% % after the title page when using the twoside option.
% 
% *** Note that you probably will NOT want to include the author's ***
% *** name in the headers of peer review papers.                   ***
% You can use \ifCLASSOPTIONpeerreview for conditional compilation here if
% you desire.


% If you want to put a publisher's ID mark on the page you can do it like
% this:
%\IEEEpubid{0000--0000/00\$00.00~\copyright~2015 IEEE}
% Remember, if you use this you must call \IEEEpubidadjcol in the second
% column for its text to clear the IEEEpubid mark.

% use for special paper notices
%\IEEEspecialpapernotice{(Invited Paper)}

% make the title area
\maketitle

% As a general rule, do not put math, special symbols or citations
% in the abstract or keywords.
\begin{abstract}
    \begin{abstract}  
Test time scaling is currently one of the most active research areas that shows promise after training time scaling has reached its limits.
Deep-thinking (DT) models are a class of recurrent models that can perform easy-to-hard generalization by assigning more compute to harder test samples.
However, due to their inability to determine the complexity of a test sample, DT models have to use a large amount of computation for both easy and hard test samples.
Excessive test time computation is wasteful and can cause the ``overthinking'' problem where more test time computation leads to worse results.
In this paper, we introduce a test time training method for determining the optimal amount of computation needed for each sample during test time.
We also propose Conv-LiGRU, a novel recurrent architecture for efficient and robust visual reasoning. 
Extensive experiments demonstrate that Conv-LiGRU is more stable than DT, effectively mitigates the ``overthinking'' phenomenon, and achieves superior accuracy.
\end{abstract}  
\end{abstract}

% Note that keywords are not normally used for peerreview papers.
\begin{IEEEkeywords}
Modeling, Control, and Learning for Soft Robots
\end{IEEEkeywords}

%%%%%%%%%% Sections %%%%%%%%%%%%%
\section{Introduction} \label{sec:introduction}
    \section{Introduction}
\label{sec:introduction}
The business processes of organizations are experiencing ever-increasing complexity due to the large amount of data, high number of users, and high-tech devices involved \cite{martin2021pmopportunitieschallenges, beerepoot2023biggestbpmproblems}. This complexity may cause business processes to deviate from normal control flow due to unforeseen and disruptive anomalies \cite{adams2023proceddsriftdetection}. These control-flow anomalies manifest as unknown, skipped, and wrongly-ordered activities in the traces of event logs monitored from the execution of business processes \cite{ko2023adsystematicreview}. For the sake of clarity, let us consider an illustrative example of such anomalies. Figure \ref{FP_ANOMALIES} shows a so-called event log footprint, which captures the control flow relations of four activities of a hypothetical event log. In particular, this footprint captures the control-flow relations between activities \texttt{a}, \texttt{b}, \texttt{c} and \texttt{d}. These are the causal ($\rightarrow$) relation, concurrent ($\parallel$) relation, and other ($\#$) relations such as exclusivity or non-local dependency \cite{aalst2022pmhandbook}. In addition, on the right are six traces, of which five exhibit skipped, wrongly-ordered and unknown control-flow anomalies. For example, $\langle$\texttt{a b d}$\rangle$ has a skipped activity, which is \texttt{c}. Because of this skipped activity, the control-flow relation \texttt{b}$\,\#\,$\texttt{d} is violated, since \texttt{d} directly follows \texttt{b} in the anomalous trace.
\begin{figure}[!t]
\centering
\includegraphics[width=0.9\columnwidth]{images/FP_ANOMALIES.png}
\caption{An example event log footprint with six traces, of which five exhibit control-flow anomalies.}
\label{FP_ANOMALIES}
\end{figure}

\subsection{Control-flow anomaly detection}
Control-flow anomaly detection techniques aim to characterize the normal control flow from event logs and verify whether these deviations occur in new event logs \cite{ko2023adsystematicreview}. To develop control-flow anomaly detection techniques, \revision{process mining} has seen widespread adoption owing to process discovery and \revision{conformance checking}. On the one hand, process discovery is a set of algorithms that encode control-flow relations as a set of model elements and constraints according to a given modeling formalism \cite{aalst2022pmhandbook}; hereafter, we refer to the Petri net, a widespread modeling formalism. On the other hand, \revision{conformance checking} is an explainable set of algorithms that allows linking any deviations with the reference Petri net and providing the fitness measure, namely a measure of how much the Petri net fits the new event log \cite{aalst2022pmhandbook}. Many control-flow anomaly detection techniques based on \revision{conformance checking} (hereafter, \revision{conformance checking}-based techniques) use the fitness measure to determine whether an event log is anomalous \cite{bezerra2009pmad, bezerra2013adlogspais, myers2018icsadpm, pecchia2020applicationfailuresanalysispm}. 

The scientific literature also includes many \revision{conformance checking}-independent techniques for control-flow anomaly detection that combine specific types of trace encodings with machine/deep learning \cite{ko2023adsystematicreview, tavares2023pmtraceencoding}. Whereas these techniques are very effective, their explainability is challenging due to both the type of trace encoding employed and the machine/deep learning model used \cite{rawal2022trustworthyaiadvances,li2023explainablead}. Hence, in the following, we focus on the shortcomings of \revision{conformance checking}-based techniques to investigate whether it is possible to support the development of competitive control-flow anomaly detection techniques while maintaining the explainable nature of \revision{conformance checking}.
\begin{figure}[!t]
\centering
\includegraphics[width=\columnwidth]{images/HIGH_LEVEL_VIEW.png}
\caption{A high-level view of the proposed framework for combining \revision{process mining}-based feature extraction with dimensionality reduction for control-flow anomaly detection.}
\label{HIGH_LEVEL_VIEW}
\end{figure}

\subsection{Shortcomings of \revision{conformance checking}-based techniques}
Unfortunately, the detection effectiveness of \revision{conformance checking}-based techniques is affected by noisy data and low-quality Petri nets, which may be due to human errors in the modeling process or representational bias of process discovery algorithms \cite{bezerra2013adlogspais, pecchia2020applicationfailuresanalysispm, aalst2016pm}. Specifically, on the one hand, noisy data may introduce infrequent and deceptive control-flow relations that may result in inconsistent fitness measures, whereas, on the other hand, checking event logs against a low-quality Petri net could lead to an unreliable distribution of fitness measures. Nonetheless, such Petri nets can still be used as references to obtain insightful information for \revision{process mining}-based feature extraction, supporting the development of competitive and explainable \revision{conformance checking}-based techniques for control-flow anomaly detection despite the problems above. For example, a few works outline that token-based \revision{conformance checking} can be used for \revision{process mining}-based feature extraction to build tabular data and develop effective \revision{conformance checking}-based techniques for control-flow anomaly detection \cite{singh2022lapmsh, debenedictis2023dtadiiot}. However, to the best of our knowledge, the scientific literature lacks a structured proposal for \revision{process mining}-based feature extraction using the state-of-the-art \revision{conformance checking} variant, namely alignment-based \revision{conformance checking}.

\subsection{Contributions}
We propose a novel \revision{process mining}-based feature extraction approach with alignment-based \revision{conformance checking}. This variant aligns the deviating control flow with a reference Petri net; the resulting alignment can be inspected to extract additional statistics such as the number of times a given activity caused mismatches \cite{aalst2022pmhandbook}. We integrate this approach into a flexible and explainable framework for developing techniques for control-flow anomaly detection. The framework combines \revision{process mining}-based feature extraction and dimensionality reduction to handle high-dimensional feature sets, achieve detection effectiveness, and support explainability. Notably, in addition to our proposed \revision{process mining}-based feature extraction approach, the framework allows employing other approaches, enabling a fair comparison of multiple \revision{conformance checking}-based and \revision{conformance checking}-independent techniques for control-flow anomaly detection. Figure \ref{HIGH_LEVEL_VIEW} shows a high-level view of the framework. Business processes are monitored, and event logs obtained from the database of information systems. Subsequently, \revision{process mining}-based feature extraction is applied to these event logs and tabular data input to dimensionality reduction to identify control-flow anomalies. We apply several \revision{conformance checking}-based and \revision{conformance checking}-independent framework techniques to publicly available datasets, simulated data of a case study from railways, and real-world data of a case study from healthcare. We show that the framework techniques implementing our approach outperform the baseline \revision{conformance checking}-based techniques while maintaining the explainable nature of \revision{conformance checking}.

In summary, the contributions of this paper are as follows.
\begin{itemize}
    \item{
        A novel \revision{process mining}-based feature extraction approach to support the development of competitive and explainable \revision{conformance checking}-based techniques for control-flow anomaly detection.
    }
    \item{
        A flexible and explainable framework for developing techniques for control-flow anomaly detection using \revision{process mining}-based feature extraction and dimensionality reduction.
    }
    \item{
        Application to synthetic and real-world datasets of several \revision{conformance checking}-based and \revision{conformance checking}-independent framework techniques, evaluating their detection effectiveness and explainability.
    }
\end{itemize}

The rest of the paper is organized as follows.
\begin{itemize}
    \item Section \ref{sec:related_work} reviews the existing techniques for control-flow anomaly detection, categorizing them into \revision{conformance checking}-based and \revision{conformance checking}-independent techniques.
    \item Section \ref{sec:abccfe} provides the preliminaries of \revision{process mining} to establish the notation used throughout the paper, and delves into the details of the proposed \revision{process mining}-based feature extraction approach with alignment-based \revision{conformance checking}.
    \item Section \ref{sec:framework} describes the framework for developing \revision{conformance checking}-based and \revision{conformance checking}-independent techniques for control-flow anomaly detection that combine \revision{process mining}-based feature extraction and dimensionality reduction.
    \item Section \ref{sec:evaluation} presents the experiments conducted with multiple framework and baseline techniques using data from publicly available datasets and case studies.
    \item Section \ref{sec:conclusions} draws the conclusions and presents future work.
\end{itemize}
\section{Background} \label{sec:background}
    %% CRT
    \subsection{Cosserat Rod Theory} \label{background:cosserat_rod_theory}
        The \ac{CRT} associates to the elastic rod of length $L$ an oriented backbone, parameterized by the material curvilinear abscissa $s \in [0, \, L]$. The oriented curve is described by its pose $\bm{g}(s, t) \in SE(3)$ for each cross-section $s$, defined as
\begin{equation} \label{eq:se3_definition}
    \bm{g}(s, t) = \begin{bmatrix}
                \bm{R}(s, t) & \bm{r}(s, t) \\
                \bm{0}^{\top} & 1 
        \end{bmatrix} \, ,
\end{equation}
where $\bm{R}(s, t) \in SO(3)$ is the rotation matrix and $\bm{r}(s, t) \in \mathbb{R}^{3}$ is the position.
% Strain
The space-evolution of $\bm{g}$ can be uniquely described by the strain field $\bm{\xi}(s, t) = \begin{bmatrix} \bm{\kappa}^{\top} & \bm{\sigma}^{\top} \end{bmatrix}^{\top} \in \mathbb{R}^6$, which is defined as
\begin{equation} \label{eq:strain_field}
    \bm{\xi}(s, t) = \left(\bm{g}^{-1}(s, t) \, \bm{g}'(s, t)\right)^{\vee} \, ,
\end{equation}
where ${\left(\cdot\right)}^{\vee}$ denotes the vee operator \cite[Ch.\ 3]{murray}, $\bm{\kappa} = \begin{bmatrix}
    \kappa_x & \kappa_y & \kappa_z
\end{bmatrix}^{\top} \in \mathbb{R}^{3}$ contains the angular strain modes, such as twisting ($\kappa_x$), and bending ($\kappa_y , \, \kappa_z$), and $\bm{\sigma} = \begin{bmatrix}
    \sigma_x & \sigma_y & \sigma_z
\end{bmatrix}^{\top} \in \mathbb{R}^{3}$ represents the linear strain modes, such as stretching ($\sigma_x$), and shear ($\sigma_y, \, \sigma_z$).

% eta
The time evolution of $\bm{g}(s, t)$ can be described with the velocity field $\bm{\eta}(s, t) \in \mathbb{R}^6$, defined as
\begin{equation} \label{eq:velocity_field}
    \bm{\eta}(s, t) = \left(\bm{g}^{-1}(s, t) \, \dot{\bm{g}}(s, t)\right)^{\vee} \, .
\end{equation}
Moreover, the mixed derivative equality allows us to write a relationship between the two fields, i.e.,
\begin{equation} \label{eq:mixed_derivative}
    \bm{\eta}'(s, t) = \dot{\bm{\xi}}(s, t) - \textnormal{ad}_{\bm{\xi}} \, \bm{\eta}(s, t) \, . 
\end{equation}

To model the statics and dynamics of a Cosserat rod, both internal and external wrenches must be defined. The internal wrench consists of the passive internal wrench $\bm{\mathcal{F}}_i(s, t) \in \mathbb{R}^6$, which accounts for the mechanical impedance of the rod (i.e., its elasticity and damping), and the active internal wrench $\bm{\mathcal{F}}_a(s, t) \in \mathbb{R}^6$, which represents the distributed actuation applied to the \acp{CSR}.
\begin{equation} \label{eq:internal_passive}
    \bm{\mathcal{F}}_i (s, t) = \bm{\Sigma}(s) \left(\bm{\xi} - \bm{\xi}^*\right) + \bm{\Psi}(s) \, \dot{\bm{\xi}} \, ,
\end{equation}
\begin{equation} \label{eq:internal_active}
    \bm{\mathcal{F}}_a(s, t) = \bm{B}_{\bm{\tau}}(\bm{\xi}, s) \, \bm{\tau}(t) \, .
\end{equation}
%
In \eqref{eq:internal_passive}, $\bm{\Sigma}(s) = \textnormal{diag} \left(G J_x, E J_y, E J_z, E A, G A, G A \right) \in \mathbb{R}^{6 \times 6}$ is the stiffness matrix, $\bm{\Psi} = \beta  \, \textnormal{diag} \left(J_x, 3 J_y, 3 J_z, 3 A, A, A \right) \in \mathbb{R}^{6 \times 6}$ the damping matrix, where $A(s)$ and $J_i(s)$ are the cross-section's area and the second moment of area, respectively. 
Furthermore, $E(s)$ is the Young modulus, $G(s)$ is the shear modulus, $\beta(s)$ is the damping coefficient, and $\bm{\xi}^{*}(s) \in \mathbb{R}^{6}$ is the stress-free strain field.

Concerning \eqref{eq:internal_active}, $\bm{B}_{\bm{\tau}} \in \mathbb{R}^{6 \times n_a}$ is the actuation matrix \cite{renda2017screw}, $\bm{\tau} \in \mathbb{R}^{n_a}$ is the actuators' magnitude vector, and $n_a$ is the number of the actuators.
The external wrench $\bm{\mathcal{F}}_e(s, t) \in \mathbb{R}^6$ collects the distributed external forces applied to the robot, such as gravity or contact forces; the contribution of gravity is detailed in the following
\begin{equation} \label{eq:ext_gravity}
    \bm{\mathcal{F}}_{e, \, g}(s, t) = \bm{\mathcal{M}}(s) \, \textnormal{Ad}^{-1}_{\bm{g}} \, \bm{\mathcal{G}} \, ,
\end{equation}
where $\bm{\mathcal{G}} \in \mathbb{R}^{6}$ is the gravity acceleration twist, $\bm{\mathcal{M}} = \rho \, \textnormal{diag}\left(J_x, J_y, J_z, A, A, A\right)$ is the cross-section's inertia matrix, and $\rho(s)$ is the cross-section's density.

The statics of a Cosserat Rod can be written as
\begin{equation} \label{eq:crt_statics}
    \left(\bm{\mathcal{F}}_i - \bm{\mathcal{F}}_a\right)' + \textnormal{ad}_{\bm{\xi}}^{*} \left(\bm{\mathcal{F}}_i - \bm{\mathcal{F}}_a\right) + \bm{\mathcal{F}}_e = \bm{0} \, ,
\end{equation}
where, in \eqref{eq:internal_passive}, the damping contribution is neglected due to the static regime.

Finally, the dynamics of the Cosserat rod can be derived using the Poincaré principle \cite{renda2018discrete}, yielding 
\begin{equation} \label{eq:crt_dynamics}
    \bm{\mathcal{M}} \dot{\bm{\eta}} + \textnormal{ad}^*_{\bm{\eta}} \left(\bm{\mathcal{M}} \bm{\eta}\right) = \left(\bm{\mathcal{F}}_i - \bm{\mathcal{F}}_a \right)' + \textnormal{ad}^*_{\bm{\xi}} \left(\bm{\mathcal{F}}_i - \bm{\mathcal{F}}_a \right) + \bm{\mathcal{F}}_{e} .
\end{equation}
    %% Notation %%
    \begin{table}[t]
\vspace{-2mm}
  \centering
  \caption{Nomenclature used in prompt construction.}
  \label{tab:nomenclature}%
  % \vskip 0.15in
  \begin{small}
  % \begin{sc}
  \begin{adjustbox}{max width=\columnwidth}
    \begin{tabular}{cl}
    \toprule
    \multicolumn{1}{c}{\textbf{Symbol}} & \multicolumn{1}{l}{\textbf{Description}} \\
    \midrule
     $P$ & prompt\\
     $F$ & number of features\\
     $T$ & number of targets\\
     $N$ & number of training examples or shots\\
     $M$ & number of text examples\\
     ${X_1, X_2, \dots, X_F}$ & text heading for feature columns\\
     ${Y_1, Y_2, \dots, Y_T}$ & text heading for target columns\\
     $x_{i,j}$ &  $j$th feature value of the $i$th training example\\
     $y_{i,j}$ &  $j$th target value of the $i$th training example\\
     $x^*_{i,j}$ &  $j$th feature value of the $i$th test example\\
     $y^*_{i,j}$ &  $j$th target value of the $i$th test example\\
     $\langle prefix \rangle$ & text string with side information\\
     $\langle d \rangle$ & separates $X_j$ and $x_{i,j}$ or $Y_j$ and $y_{i,j}$\\
     $\langle s \rangle$ & separates $X_j\langle d \rangle x_{i,j}$ and $X_{j+1}\langle d \rangle x_{i,j+1}$\\
      & or $Y_j\langle d \rangle y_{i,j}$ and $Y_{j+1}\langle d \rangle y_{i,j+1}$\\
     $\langle t \rangle$ & separates examples\\
    \bottomrule
    \end{tabular}%
  \end{adjustbox}
  % \end{sc}
  \end{small}
\vspace{-3mm}
\end{table}%
 
    %% SoA
    \subsection{State of the Art} \label{background:soa}
        The dynamics of a Cosserat rod \eqref{eq:crt_dynamics} is a set of \acp{PDE}, whose solution lies in a functional space, i.e., an infinite-dimension space. However, many approaches have been proposed in the literature to efficiently describe the robot with a finite number of \ac{DoFs}, trying to find the best trade-off between accuracy and computational efficiency.

% PCC - PCS
One of the most common modeling approaches is the \ac{PCC} \cite{webster2010design}. The core idea is to divide the backbone into segments and assume them as circumference arcs. This approximation fits very well for the slender shape of the \acp{CSR}. Furthermore, it provides the dynamics in the classical Lagrangian form \cite{siciliano}, allowing the transfer of many controllers from the rigid robot literature \cite{della2018dynamic}. In addition, the \ac{PCC} approach accounts for bending and elongation modes, describing effectively the deformations excited by tendon-driven and fluidic actuation sources.

% Grazioso
To include the twisting mode, \cite{grazioso2019geometrically} proposed to fit the pieces with a helix (i.e., constant curvature and torsion), enabling a general curve in the 3D space.

% Renda
However, the approximation of unshearable rods is unsuitable for interaction tasks where shear deformation is significant, particularly in cases involving contact with irregular objects or the environment. To expand the previous models, \cite{renda2018discrete} proposed the \ac{PCS} approach, which consists of considering the strain field constant along the length of the single piece. 

% PAC, PAS
Differently, \cite{della2020soft} considers the curvature an affine function instead of constant for each single segment.
%Instead of constant curvature, another approach explored is to consider it an affine function for a single segment \cite{della2020soft}.
With only two \ac{DoFs}, this model exhibits nonlinear phenomena such as snap effect \cite{armanini2017elastica, caradonna2024model}. A further work \cite{stella2023piecewise} proposed the \ac{PAC}, in which each piece is described by an affine function in $s$.

Following this methodology, \cite{li2023piecewise} presented the \ac{PLS}, in which the strain field is computed by linear interpolation of samples. In addition, \ac{PLS} is applied in the case of interactions, exploiting excellent performances w.r.t. the other modeling methodologies \cite{xun2024cosserat}.
In general, we can call \ac{PAS} all the approaches that use the linear approximation of each segment of the strain field. 

% GVS
Differently from the piecewise methods, in \cite{renda2020geometric, boyer2020dynamics} the \ac{GVS} is proposed, in which the strain field is assumed as generated by a truncated functional basis of space-dependent vectors \cite{armanini2023soft}. It is worth highlighting that, with a polynomial basis function, the \ac{GVS} approach coincides with the affine strain and curvature approximation. However, the \ac{GVS} approach allows using other basis functions, such as trigonometric and Gaussian.

Furthermore, utilizing the Magnus expansion \cite[Chap. IV.7]{hairergeometric} and Zanna's collocation method \cite{zanna1999collocation}, the Authors provided an efficient and recursive algorithm to compute the Jacobian of the \ac{CSR}. 
The primary advantage of this approach is that, regardless of the functional basis chosen, the classic Lagrangian form \cite{siciliano} of the dynamics can be found.
This significantly eases the transfer of classical controllers from rigid to soft robotics.

 Following this approach, the Authors proposed a strain-dependent functional basis called \ac{ISP} \cite{renda2024dynamics}. In this approach, the basis functions are derived from the robot's statics, achieving a minimal number of \ac{DoFs} corresponding to the strain modes excited by the actuators’ routing and external forces. However, under dynamic regimes, further strain modes might be excited, requiring a user-specified extended basis.

In \cite{mathew2024reduced}, the Authors present a comprehensive explanation of the \ac{GVS} framework, covering every possible type of functional basis. 
Additionally, it has been incorporated into \ac{SoRoSim} \cite{mathew2022sorosim}, which allows for rapid simulation of soft robots, even when they involve numerous amounts of \ac{DoFs}.

Since the choice of basis is critical, the research community has explored methods to optimize the number of bases and reduce the system's order.
For instance, in \cite{pustina2024nonlinear}, the Authors introduced eigenmanifolds to perform modal analysis for \acp{CSR}. This approach facilitates the evaluation of \ac{PCC} models with an increasing number of segments by employing similarity metrics compared to high-fidelity models such as the Finite Element Method.

% Alkayas
Finally, \cite{alkayas2024soft} presents a data-driven reduction method based on \ac{POD}. The key concept involves applying \ac{SVD} to the strain data to identify the least significant singular values, enabling truncation without significant loss of accuracy. 
First, a \ac{GVS} digital twin with a high number of \ac{DoFs} is derived by fitting the experimental data. Then, after simulating the fitted \ac{GVS} model, a reduction in the number of \ac{DoFs} is performed using the \ac{POD} method.

% Our Contributions
In this context, we propose a novel approach that treats the backbone of the continuum robot as a space-time varying signal and analyzes it using the \ac{FT}. This framework unifies existing methods by interpreting them as reconstructors based on discrete strain samples. Moreover, this perspective provides theoretical justification for various heuristic methods, such as the minimum number of segments required in piecewise approaches.

Additionally, by applying the \ac{FT} to experimental data, we establish a data-driven methodology with a strong mathematical foundation for identifying the optimal functional bases from a predefined signal dictionary. This approach relies solely on the geometric properties of the robot and its actuators. 
The method detects the functional bases in the dynamic situation and extends the static strain analysis of the \ac{ISP} method.
\section{Spatial Fourier Transform} \label{sec:spatial_ft}
    As seen in Sec. \ref{background:cosserat_rod_theory}, the backbone of a \ac{CSR} can be completely described by the strain field $\bm{\xi}: [0, \, L] \times [0, \, +\infty) \rightarrow \mathbb{R}^6$. The main idea of this work is to consider it as a signal in space and time, applying the \ac{FT} to analyze the robot’s deformation.

% Domain Adaption
To apply it, we must adapt the strain field's domain. Hence, let us assume to extend the domain of $s$ extends to $\mathbb{R}$ and that $\bm{\xi}: \mathbb{R} \times [0, \, +\infty) \rightarrow \mathbb{R}^{6}$, where the strain field is null for all $s \notin [0, \, L]$. This assumption is equivalent to applying a spatial window over the interval $[0, \, L]$.

To formalize this, let $\bm{\Phi}(s): \mathbb{R} \rightarrow \mathbb{R}^6$ be a function such that $\bm{\xi}(s) = \bm{\Phi}(s) \, \forall s \in [0, \, L]$. 
The previous assumption is equivalent to stating that $\bm{\xi}(s, t) = \bm{\Phi}(s) \cdot \Pi_L(s)$, where
\begin{equation} \label{eq:length_space_window}
    \Pi_L(s) = \rect{\frac{s - L/2}{L}} \, ,
\end{equation}
is the window function in the range $[0, \, L]$ and $\rect{\cdot}$ is the rectangular function \cite[Ch. 4]{bracewell2007fourier}.
% % Curvature Example
% \begin{figure}
%     \centering
%     \includesvg[width=0.9\linewidth]{imgs/space_shift_example.svg}
%     \caption{Impact of the phase in a planar rod with a sinusoidal curvature function. Varying the phase, the deformation is distributed differently along the rod.}
%     \label{fig:space_shift_example}
% \end{figure}
% Curvature Example
\begin{figure}
    \centering
    \includegraphics[width=0.9\linewidth]{imgs/space_shift_example.pdf}
    \caption{Impact of the phase in a planar rod with a sinusoidal curvature function. Varying the phase, the deformation is distributed differently along the rod.}
    \label{fig:space_shift_example}
\end{figure}
%
\subsection{Continous \ac{SFT}} \label{spatial_ft:csft}
The Continous \ac{SFT} of the strain field $\bm{\xi}$ is defined as
\begin{equation} \label{eq:continous_spatial_fourier_transform}
    \bm{\Xi}\left(jk, \, t\right) = \fourier{\bm{\xi}(s, t)} = \int_{0}^{L} \bm{\xi}(s, \, t) \, e^{-jks} \, \textnormal{d} s \, ,
\end{equation}
where $k = 2 \pi \nu \in \mathbb{R}$ is the angular wavenumber, $\nu \in \mathbb{R}$ is the wavenumber (i.e., spatial frequency), and $j = \sqrt{-1}$ the imaginary unity. In this case, the signal $\bm{\xi}(s, t)$ is a signal limited in space ($s \in [0, L]$) and aperiodic. Hence, the spatial spectrum 
$\bm{\Xi}(j k, t)$ will be infinite, composed of all the spatial frequencies, as shown in Fig. \ref{fig:concept}(a). 
This consideration is coherent with the infinite \ac{DoFs} of a \ac{CSR}, considering the strain field generated by an infinite-dimension trigonometric functional basis.

% Magnitude and Phase of the Strain Field
Moreover, the \ac{SFT} of the strain field $\bm{\Xi}(jk) \in \mathbb{C}^{6}$ can be analyzed by examining its magnitude and phase. 
The magnitude $\abs{\bm{\Xi}(jk)}$ quantifies the extent of deformation of the slender body, whereas the phase $\angle\bm{\Xi}(jk)$ indicates how the deformation is distributed along the rod, impacting with the rod's shape.
Consider, for instance, a planar backbone characterized by a sinusoidal curvature $\kappa = \sin\left(\pi s + \phi \right) \in \mathbb{R}$.
As illustrated in Fig. \ref{fig:space_shift_example}, this sinusoidal curvature generates a backbone shape that changes as the phase $\phi$ varies. Although the magnitude of the curvature remains constant, the deformation is differently distributed along the rod, leading to variations in the overall shape of the backbone.

\subsection{Statics of a Cosserat Rod in the Spatial Frequency Domain}
We can exploit the properties of \ac{FT} \eqref{eq:fourier_definition}-\eqref{eq:fourier_conv} in the case of the Spatial Spectrum $\bm{\Xi}(jk)$, allowing us to rewrite \eqref{eq:crt_statics} in the spatial frequency domain, such as
\begin{equation} \label{eq:static_ft}
    jk \, \left(\bm{F}_i(jk) - \bm{F}_a(jk)\right) + \textnormal{ad}_{\bm{\Xi}}^{*} \ast \left(\bm{F}_i(jk) - \bm{F}_a(jk)\right) = \bm{0} \, ,
\end{equation}
where $\bm{F}_i(jk) = \fourier{\bm{\mathcal{F}}_i(s)}$, $\bm{F}_a(jk) = \fourier{\bm{\mathcal{F}}_a(s)}$, and $\ast$ is the convolution in the spatial frequency domain.

In the case of internal passive wrench $\bm{\mathcal{F}}_i(s)$, the \ac{FT} can be expressed as
\begin{equation} \label{eq:internal_passive_static_ft}
    \bm{F}_i(j k) = \bm{\Sigma}(jk) \ast \left(\bm{\Xi}(jk) - \bm{\Xi}^{*}(jk) \right) \, ,
\end{equation}
where $\bm{\Sigma}(jk) = \fourier{\bm{\Sigma}(s)}$ is the \ac{SFT} of the stiffness matrix. It is worth highlighting that the stiffness matrix $\bm{\Sigma}(s)$ acts as a convolutional filter applied to the strain field in the spatial frequency domain.

In the case of internal active wrench $\bm{\mathcal{F}}_a(s)$, the \ac{SFT} can be expressed as
\begin{equation}
    \bm{F}_a(j k) = \bm{B}_{\bm{\tau}}(\bm{\Xi}, jk) \, \bar{\bm{\tau}} \, ,
\end{equation}
where $\bm{B}_{\bm{\tau}}(\bm{\Xi}, jk) = \fourier{\bm{B}_{\bm{\tau}}(\bm{\xi}, s)}$ is the \ac{SFT} of the actuation matrix, and $\bar{\bm{\tau}} \in \mathbb{R}^{n_a}$ a generic constant input.

\subsection{Discrete \ac{SFT} and Sampled Strain Field}
For each time instant, let us assume to have $N$ samples of the strain field equally spaced along the rod $\bm{\xi}(n \lambda_s, t)$, where $n \in [0, N - 1]$, and $\lambda_s = L / N$ is the sampling wavelength. These samples can be directly measured using shape sensors \cite{floris2021fiber}, motion capture systems \cite{field2009motion}, or estimated using shape estimation algorithms (e.g., \cite{lilge2022continuum}).
%
In the discrete case, we can define Discrete \ac{SFT} of the strain field as
\begin{equation} \label{eq:discrete_spatial_fourier_transform}
    \bm{\Xi}(jk, t) = \sum_{n = 0}^{N - 1} \bm{\xi}(n \lambda_s, t) \, e^{-jk n \lambda_s} \, . 
\end{equation}
%
Moreover, the sampled signal $\bm{\xi}(n \lambda_s, t)$ in the spatial spectrum can be expressed as
\begin{equation} \label{eq:repetition_spectrum}
    \bm{\Xi}_{\textnormal{d}}(jk, t) = \frac{1}{\lambda_s} \sum_{n = - \infty}^{+ \infty} \, \bm{\Xi}\left(jk - j n k_s, \, t\right) \, ,
\end{equation}
where $k_s = \frac{2 \pi}{\lambda_s}$ denotes the sampling angular wavenumber. Coherently with time-varying signals, the spatial spectrum of the discretized strain field consists of an infinite series of replicas of the \ac{SFT}, each shifted by a multiple of $k_s$. Since the spatial spectrum of $\bm{\xi}$ is unlimited, the aliasing effect is inevitable, as illustrated in Fig. \ref{fig:concept}(b). 
 To address this, we can select the last most significant component $k_{\textnormal{max}} \in \mathbb{R}$ and apply the Nyquist-Shannon theorem \cite{shannon1949communication}, resulting in $k_s > 2 \, k_{\textnormal{max}}$.

In the context of modeling \acp{CSR}, it can be more meaningful to express the previous relation in terms of the wavelength, such as
\begin{equation} \label{eq:nyquist_wavelength}
    \lambda_{s} < \frac{\lambda_{\textnormal{max}}}{2} \rightarrow N > \frac{2}{\lambda_{\textnormal{max}}} L \, ,
\end{equation}
where $\lambda_{\textnormal{max}} = \frac{2 \pi}{k_{\textnormal{max}}}$. Eq. \eqref{eq:nyquist_wavelength} provides useful information about the minimum number of pieces a \ac{CSR}, assuming that most of the signal's energy is concentrated up to $k_{\textnormal{max}}$.
This result supports the intuitive and heuristic understanding that more segments are needed for longer rods, as $\lambda_s = L / N$.

\subsection{Energy-based Criterion for Truncation} \label{spatial_ft:truncation}
To determine $k_{\textnormal{max}}$, it is possible to compute the quantity of the signal energy contained up the component $k_{\textnormal{max}}$ and compare it to the total energy of the signal.
Recalling \eqref{eq:discrete_spatial_fourier_transform} and the Parseval identity \eqref{eq:fourier_parseval} in the discrete domain, the truncation index $E_{\textnormal{tr}, \xi_i}$ can be defined as
\begin{equation} \label{eq:discrete_truncation_criterion}
    E_{\textnormal{tr}, \xi_i}\left(k_{\textnormal{max}}\right) = \frac{N}{N_{\textnormal{max}}} \left(\frac{\sum_{n = 0}^{N_{\textnormal{max}} - 1} |\Xi_{i}(n)|^{2}}{\sum_{n = 0}^{N - 1} |\Xi_{i}(n)|^{2}}\right) \, ,
\end{equation}
where $N_{\textnormal{max}} < N$ is the number of the sample associated with the angular wavenumber $k_{\textnormal{max}}$. 
The truncation index is the ratio of the energy accumulated up to the spectral component $k_\textnormal{max}$ to the total energy of the signal, giving a measure of the truncation accuracy.
% It is worth highlighting that this ratio corresponds to the \ac{SNR}, considering the components $k > k_{\textnormal{max}}$ as noise.

Moreover, the truncation index can assume an interesting physical interpretation. By rewriting \eqref{eq:discrete_truncation_criterion} in terms of internal passive wrench \eqref{eq:internal_passive_static_ft}, the truncation index represents the ratio between the deformation energy stored up to $k_{\textnormal{max}}$ and the total deformation energy of the \ac{CSR}. 
Furthermore, the stiffness matrix differentiates between strain modes, leveraging the robot's geometric and physical characteristics.

\subsection{Interpretation of Spatial Discretization Techniques in the Spatial Frequency Domain}
%  % Spatial Discretization Comparison Image
% \begin{figure*}
%     \centering
%     \includesvg[width=1.0\linewidth]{imgs/spectra.svg}
%     \caption{Comparison of the different spatial discretization methodologies. By treating the strain field as a signal, existing modeling approaches can be interpreted as reconstructors.}
%     \label{fig:spectra}
% \end{figure*}
% %
 % Spatial Discretization Comparison Image
\begin{figure*}
    \centering
    \includegraphics[width=1.0\linewidth]{imgs/spectra.pdf}
    \caption{Comparison of the different spatial discretization methodologies. By treating the strain field as a signal, existing modeling approaches can be interpreted as reconstructors.}
    \label{fig:spectra}
\end{figure*}
%
From the samples $\bm{\xi}(n \lambda_s)$, the spatial discretization techniques exposed in the Sec. \ref{background:soa} can be seen as signal reconstructors.

%% PCS %%
The \ac{PCS} method reconstructs the sampled strain, assuming constant strain along the single piece. Assuming $n_p$ pieces with the same length $\lambda_p = L / n_p$, the strain field can be written as
\begin{equation} \label{eq:pcs_strain}
    \xi_i(s, t) = \sum_{h = 0}^{n_{p} - 1} q_h(t) \, \Pi_h\left(s\right) \, ,
\end{equation}
where $\xi_i \in \mathbb{R}$ is the $i$-th element of the strain field,
$\Pi_h\left(s\right) =  \rect{\frac{s - \left(h + \frac{1}{2}\right) \lambda_p}{\lambda_p}}$, and $q_h \in \mathbb{R}$ is the constant value assumed in the $h$-th piece by $\xi_i$.

We can operate the \ac{SFT} of \eqref{eq:pcs_strain}, resulting in 
\begin{equation}
    \Xi_{i, \textnormal{PCS}}(jk, t) = \sum_{h = 0}^{n_p - 1} q_i(t) \, \Pi_h\left(jk\right) \, ,
\end{equation}
where $\Pi_h\left(jk\right) = \lambda_p \, \sinc{k \frac{\lambda_p}{2}} e^{-j k \left(h + \frac{1}{2}\right)\lambda_p}$, and $\sinc{\cdot}$ is the cardinal sine.
If the coefficients $q_h$ correspond to the $\xi_i(h \lambda_p)$, the \ac{PCS} approach can be interpreted as a spatial \ac{ZOH} of the discrete signal $\bm{\xi}$. The reconstructed signal through \ac{PCS} can be expressed as $\bm{\Xi}_{\textnormal{PCS}}(jk) = \bm{H}_0 (jk)  \, \bm{\Xi}_d(jk)$, where
\begin{equation}
    \bm{H}_0(jk) =  \lambda_p \left(\sinc{k \frac{\lambda_p}{2}} e^{-j k \frac{\lambda_p}{2}}\right) \bm{I}_{6} \, .
\end{equation}

Similar to the time \ac{ZOH}, the \ac{PCS} approach provides a shift in space of $\lambda_p / 2$, as evident in Fig. \ref{fig:spectra}. Furthermore, for any multiple of $2 \pi / \lambda_p$, the magnitude of $\bm{H}_0(jk)$ falls to 0. Note that these considerations are valid also for the \ac{PCC} case, due to the generality of \ac{PCS} method.

%% PAS %%
Concerning the \ac{PAS} method, we can write the reconstructed strain as
\begin{equation} \label{eq:pas_strain}
    \xi_i(s, t) = \sum_{h = 0}^{n_p - 1} \left(q_{h, 0}(t) + q_{h, 1}(t) s \right) \Pi_h\left(s\right) ,
\end{equation}
where $q_{h, \, 0}, \, q_{h, \, 1} \in \mathbb{R}$ are the time-varying coefficients of the $i$-th element of the strain field.
In the case of \ac{PLS}, it coincides with the \ac{FOH} reconstructor in space. The reconstructed spectrum can be written as $\bm{\Xi}_{\textnormal{PLS}}(jk) = \bm{H}_1(jk) \, \bm{\Xi}_d(jk)$, where
\begin{equation}
    \bm{H}_1(jk) =  \lambda_p \left(\sinc{k \frac{\lambda_p}{2}}\right)^2 \left(1 + jk \lambda_p\right) e^{-j k \frac{\lambda_p}{2}} \bm{I}_{6} \, .
\end{equation}

Differently from the piecewise methods, the \ac{GVS} can be seen as a fitting of the samples $\bm{\xi}(n \lambda_s)$ with specific functional bases, such as
\begin{equation} \label{eq:gvs_strain}
    \bm{\xi}(s, t) = \bm{B}_{\bm{q}}(s) \, \bm{q}(t) + \bm{\xi}^{*}(s) \, ,
\end{equation}
where $\bm{B}_{\bm{q}}(s) \in \mathbb{R}^{6 \times n_q}$ is the functional basis matrix and $\bm{q} \in \mathbb{R}^{n_q}$ is the time-dependent coefficients, considered as a vector of virtual joint variables.

The \ac{SFT} of the \ac{GVS} can be expressed as
\begin{equation} \label{eq:gvs_sft}
    \bm{\Xi}_{\textnormal{GVS}}(jk, t) = \left(\bm{B}_{\bm{q}}\left(jk\right) \, \bm{q}(t) + \bm{\Xi}^{*} \right) \ast \Pi_{L}\left(jk\right)\, , 
\end{equation}
where $\bm{B}_{\bm{q}}\left(jk\right) = \fourier{\bm{B}_{\bm{q}}(s)}$, and $\Pi_{L}\left(jk\right) = L \, \sinc{k \frac{L}{2}} e^{-j k \pi L}$ is the \ac{SFT} of \eqref{eq:length_space_window}. The convolution with the \ac{SFT} of the window function is necessary due to the domain adaption (Sec. \ref{sec:spatial_ft}). 
Eq. \eqref{eq:gvs_sft} states also in the \ac{ISP} case, requiring a numerical solution as in the space-domain.
\section{Space-Time Fourier Transform} \label{sec:space_time_ft}
    The \ac{STFT} of the strain field $\bm{\xi}(s, t)$ is defined as
\begin{equation} \label{eq:space_time_ft}
    \bm{\Xi}\left(jk, \, j \omega\right) = \int_{0}^{+\infty} \int_{0}^{L} {\bm{\xi}}(s, \, t) \, e^{-j\left(k s + \omega t\right)} \, \textnormal{d} s \, \textnormal{d} t \, ,
\end{equation}
where $\omega = 2 \pi f \in \mathbb{R}$, and $f  \in \mathbb{R}$ is the time-frequency. 

For the discrete case, let us assume to have $M \times N$ samples of the strain field $\bm{\xi}(n \lambda_s, \, m T_s)$, where $m \in [0, \, M - 1]$ and $T_s$ is the sampling period. 
The Discrete \ac{STFT} can be written as
\begin{equation} \label{eq:discrete_stft}
    \bm{\Xi}\left(jk, j \omega\right) = \sum_{m = 0}^{M - 1} \sum_{n = 0}^{N - 1} \bm{\xi}\left(n \lambda_s, m T_s \right) e^{-j \left( k n \lambda_s + \omega m T_s\right)} \, .
\end{equation}
% Discussion on utility
This analysis provides useful information about the dynamic response of the system and highlights certain components in the spatial spectrum that arise during the transient regime.

% \Xi(\cdot, j \omega)
By examining the curves $\bm{\Xi}(\cdot, \ j \omega)$, it is possible to observe the time evolution of each spatial harmonic, providing a useful tool to identify resonance or anti-resonance peaks, as well as to assess the relevance of specific spatial harmonics in the time-frequency domain. This information can be leveraged to discard irrelevant harmonics.

% \Xi(j k, \cdot)
Conversely, the curves $\bm{\Xi}(jk, \ \cdot)$ represent the spatial harmonics excited by an input with a specific time-frequency. These curves are useful for understanding the strain profile in the spatial-frequency, offering insights into how the shape of the \ac{CSR} changes in response to an input of a specific time-frequency.

Moreover, if the \ac{CSR} is integrated within a control framework operating at a specific frequency, analyzing $\bm{\Xi}(jk, \, \cdot)$ allows the user to select an optimal set of basis functions.

\subsection{Dynamics of a Cosserat Rod in the Space-Time Frequency Domain} \label{space_time_ft:dynamics_stft}
From the properties of the 2D \ac{FT} \eqref{eq:fourier_definition}-\eqref{eq:fourier2d_product}, the dynamics of an elastic rod \eqref{eq:crt_dynamics} can be described with a set of algebraic equations in the space-time frequency domain, such as
\begin{equation} \label{eq:dynamics_ft}
    \begin{split}
        &\bm{M} \ast \left(j \omega \, \bm{\mathcal{H}}\right) + \textnormal{ad}_{\bm{\mathcal{H}}}^{*} \ast \left(\bm{M} \ast \bm{\mathcal{H}}\right) = \\
        &jk \left(\bm{F}_i - \bm{F}_a\right) + \textnormal{ad}_{\bm{\Xi}}^{*} \ast \left(\bm{F}_i - \bm{F}_a\right) + \bm{F}_e
    \end{split} \, ,
\end{equation}
where $\bm{M}(jk) = \fourier{\bm{\mathcal{M}}(s)}$, and $\bm{\mathcal{H}}(jk, \, j \omega) = \fourier{\bm{\eta}(s, t)}$.
Moreover, the mixed derivative equality \eqref{eq:mixed_derivative} holds also in the frequency domain, such as 
\begin{equation} \label{eq:mixed_derivative_ft}
    jk \, \bm{\mathcal{H}} = j \omega \, \bm{\Xi} - \left(\textnormal{ad}_{\bm{\Xi}} \ast \bm{\mathcal{H}}\right) \, .
\end{equation}
In \eqref{eq:dynamics_ft}, the \ac{STFT} of the internal wrenches \eqref{eq:internal_passive}, \eqref{eq:internal_active} can be expressed as
\begin{equation} \label{eq:internal_passive_ft}
    \bm{F}_i(j k , \, j \omega) = \bm{\Sigma}(jk) \ast \left(\bm{\Xi} - \bm{\Xi}^{*}\right) + \bm{\Psi}(jk) \ast \left(j \omega \bm{\Xi}\right) \, ,
\end{equation}
\begin{equation} \label{eq:internal_active_ft}
    \bm{F}_a\left(jk, \, j \omega\right) = \bm{B}_{\bm{\tau}}\left(\bm{\Xi}, jk\right) \ast \bm{T}(j \omega) \, ,
\end{equation}
where $\bm{\Psi}(jk) = \fourier{\bm{\Psi}(s)}$, and $\bm{T}(j \omega) = \fourier{\bm{\tau}(t)}$.

Finally, in the case of \ac{GVS} parameterization, we can exploit the separation property of the 2D \ac{FT} \eqref{eq:fourier2d_product} in \eqref{eq:gvs_strain}, resulting in
\begin{equation} \label{eq:gvs_stft}
    \bm{\Xi}(jk, \, j\omega) = \left(\bm{B}_{\bm{q}}\left(jk\right) \, \bm{Q}(j \omega) + \bm{\Xi}^{*}(jk)\right) \ast \Pi_{L}\left(    jk \right) \, ,
\end{equation}
where $\bm{Q}(j \omega) = \fourier{\bm{q}(t)}$. 
With this relation, it is possible to study separately the space and time spectra.
\section{Data-driven Spectrum Extraction} \label{sec:spectrum_extraction}
    % % Procedure (with STFT)
% \begin{figure*}
%     \centering
%     % \includegraphics[width=1.0\linewidth]{imgs/sofft_procedure.png}
%     \includesvg[width=1.0\linewidth]{imgs/sofft_procedure2.svg}
%     \caption{Illustration of the proposed data-driven methodology. The robot is subjected to the standard signals and the samples of the strain field are measured by the sensors. Through \ac{FFT}, the space-time spectrum can be computed.}
%     \label{fig:procedure}
% \end{figure*}
% Procedure (with STFT)
\begin{figure*}
    \centering
    \includegraphics[width=1.0\linewidth]{imgs/sofft_procedure2.pdf}
    \caption{Illustration of the proposed data-driven methodology. The robot is subjected to the standard signals and the samples of the strain field are measured by the sensors. Through \ac{FFT}, the space-time spectrum can be computed.}
    \label{fig:procedure}
\end{figure*}
%
As outlined in Sec. \ref{sec:spatial_ft} and \ref{sec:space_time_ft}, meaningful insights into the strain field can be derived from both the spatial \eqref{eq:continous_spatial_fourier_transform} and space-time \eqref{eq:space_time_ft} spectra.
This section presents a data-driven approach for extracting the \ac{SFT} and \ac{STFT}, directly from the real-world robots. 
This method only requires knowledge of the robot's length and is independent of its physical parameters.
The proposed procedure consists of the steps exposed below and summarized in Fig. \ref{fig:procedure}.
\begin{enumerate}
    \item \textit{Sensorization}: The first step is to sensorize the \ac{CSR}, in order to measure the samples $\bm{\xi}(n \lambda_s)$. The choice of $\lambda_s$ is crucial as it determines the maximum spatial frequency of the sampled spectrum, i.e., $\nu_s / 2$.
    
    \item \textit{Motor Babbling}: Standard signals are applied to the actuators $\bm{\tau}(t)$. Common signals in identification literature include step, chirp, and white noise, thanks to their properties in frequency.
    As a result, samples $\bm{\xi}(n \lambda_s, \, m T_s)$ are obtained, where $T_s$ represents the sampling period of the sensing framework.

    \item \textit{Compute Spectra}: From the samples $\bm{\xi}(n \lambda_s, \, m T_s)$, we compute a time-series of the \ac{SFT} and \ac{STFT} using the \acf{FFT} algorithm. Before this, preprocessing techniques like zero-padding can be applied to improve the resolution of the \ac{FFT}.

    \item \textit{Spectrum Analysis}: From $\bm{\Xi}(jk, \, j \omega)$ we can identify $k_{\textnormal{max}}$ and the shape of the spatial spectrum, which aids in selecting the optimal basis functions. 
    Furthermore, we can derive key dynamic characteristics of the system, such as resonance or anti-resonance phenomena.
    
    \item \textit{Modeling}: 
    Based on the information gathered, the user can determine the minimum number of sections into which the robot can be divided, thereby reducing the need for extensive sensorization ($k_{\textnormal{max}} < \frac{k_s}{2}$). 
    Additionally, optimal basis functions can be extracted using well-established signal processing algorithms such as \ac{MP} \cite{mallat1993matching} or \ac{BPD} \cite{chen2001atomic}.
\end{enumerate}

Viewing strain as a signal offers new insights into the frequency domain of motor babbling \cite{george2020first}. Since standard signals (e.g., white noise) can sample all frequencies in the spectrum, motor babbling can be interpreted as a method for exploring the robot's space-time spectrum. This aligns with the classic definition of motor babbling, where random actuations are applied to the robot to sample its workspace \cite{george2020first}.

% BPDN
% BPDN for Strain Fitting
\subsection{\ac{BPD} for Strain Fitting} \label{spectrum_extraction:bpd}
The Basis Pursuit Denoising (BPD) problem enables the identification of an optimal basis that preserves the signal's sparsity while mitigating the influence of noise. 
This method is particularly well-suited for applications in soft robotics, where a trade-off between accuracy and reduction in \ac{DoFs} is essential. Specifically, \ac{BPD} selects the best combination of basis functions from a predefined signal dictionary to achieve an optimal reconstruction of the input data.

To adapt the \ac{BPD} problem to the \ac{GVS} approach, the optimization problem can be formulated as
\begin{equation} \label{eq:bpdn_opt_problem}
    \bm{q} = \underset{\bm{q}}{\arg \min} \left\{\frac{1}{2}\norm{\bm{\xi} - \bm{B}_{\bm{q}} \, \bm{q}}^{2}_{2} + \norm{\bm{\gamma} \odot \bm{q}}_1\right\} \, ,
\end{equation}
where $\odot$ is element-wise multiplication, $\norm{\cdot}_i$ is the $\mathcal{\ell}^i$-norm, the basis function matrix $\bm{B}_{\bm{q}}$ represents the signal dictionary and $\bm{\xi}$ the noisy data. Finally, the parameter $\bm{\gamma} \in \mathbb{R}^{n_q}$ controls the trade-off between sparsity and reconstruction accuracy, allowing the user to prioritize either a more compact representation or a more precise fit.

Moreover, by exploiting the orthogonality of trigonometric functions, the user can extract Fourier coefficients using the \ac{SALSA} algorithm \cite{afonso2010fast} and employ a trigonometric basis to denoise the noisy strain field samples.

To evaluate the relevance of each basis, it is possible to adapt the truncation index \eqref{eq:discrete_truncation_criterion} in the continuous case.
The energy associated with the $i$-th basis $b_i(s)$ can be computed as $E_i(q_i) = q_i^2 \, \frac{1}{2 \pi}\int_{-\infty}^{+\infty} |b_i(jk)|^{2} \, \textnormal{d}k$, where $b_i(jk) = \fourier{b_i(s)}$.
Therefore, the energy fraction relative to the total reconstructed strain field energy is
\begin{equation}
    E_{\textnormal{tr}, b_i}(q_i) = \frac{E_i(q_i)}{\sum\limits_{j = 0}^{n - 1} E_j(q_j)} \, .
\end{equation}

It is worth highlighting that, thanks to the continuous \ac{SFT}, the energy evaluation can be performed for all the wavenumbers. However, if the integral is difficult to compute, the user can still use the discrete Parseval identity with an arbitrary range and resolution frequency.
Finally, the user can consider truncating one or more bases if their energy contribution is less than a specified threshold.

% \subsection{Error Propagation in $SE(3)$} \label{spectrum_extraction:error_propagation}
% After the \ac{BPD} and eventually a truncation, there will be a fitting error $\bm{\epsilon}(s) \in \mathbb{R}^{6}$, such that $\bm{\xi} = \bm{B}_{\bm{q}}(s) \, \bm{q} + \bm{\xi}^{*} + \bm{\epsilon}(s)$. This error propagates in the Forward Kinematics of the \ac{CSR}, affecting especially the tip's pose.
% Let be $\check{\bm{\xi}} = \bm{B}_{\bm{q}}(s) \, \bm{q} + \bm{\xi}^{*} \in \mathbb{R}^{6}$ the reconstructed strain. Recalling \eqref{eq:strain_field}, the backbone space-evolution can be described by
% \begin{equation} \label{eq:diff_err_backbone}
%     \bm{g}'(s) = \bm{g}(s) \left( \check{\bm{\xi}}(s) + \bm{\epsilon}(s)\right)^{\wedge} \, .
% \end{equation}
% Using Magnus expansion and Zanna collocation method as in the \ac{GVS} method, the \eqref{eq:diff_err_backbone} can be solved as
% \begin{equation} \label{eq:error_fk}
%     \bm{g}((n+1) \lambda_s) = \bm{g}(n \lambda_s) \exp_{SE(3)} \left( \hat{\bm{\Omega}}(\lambda_s) + \hat{\bm{\Omega}}_{\bm{\epsilon}}(\lambda_s) \right) \, ,
% \end{equation}
% where $\bm{\Omega} \in \mathbb{R}^{6}$  and $\bm{\Omega}_{\bm{\epsilon}} \in \mathbb{R}^{6}$ are the Magnus expansion of the reconstructed strain and the fitting error, respectively. 

% % Identification
% %%% Identification %%%
\subsection{Parametric Identification \textcolor{red}{Doesn't Work :(}}
To identify the physical parameters of the robot, we can expand the regressor-driven method proposed in \cite{stella2022experimental}, in the case of \ac{GVS} model. From the experimental data $\bm{\xi}(n \lambda_s, m T_s)$ and fixed a functional basis, we can fit the data, obtaining the experimental configurations $\bm{q}_{\textnormal{exp}}$. Assuming known the density $\rho$, and $E$, $\beta$ constant along the rod's length, we can reorganize the dynamics such as
\begin{equation} \label{eq:regressor}
    \bm{\delta}_m\left(\bm{q}, \dot{\bm{q}}, \ddot{\bm{q}}, \bm{\tau}\right) = \bm{Y}_m\left(\bm{q}, \dot{\bm{q}}\right) \bm{\pi} \quad \forall \, m \in [0, \, M - 1] \, ,
\end{equation}
where $\bm{Y}_m\left(\bm{q}, \dot{\bm{q}}\right) = - \begin{bmatrix}\bar{\bm{K}}\bm{q} & \bar{\bm{D}}\dot{\bm{q}}\end{bmatrix}$, $\bar{\bm{K}} , \, \bar{\bm{D}}$ are the stiffness and damping matrix with unitary $E$ and $\beta$, $\bm{\delta}_m\left(\bm{q}, \dot{\bm{q}}, \ddot{\bm{q}}, \bm{\tau}\right) = \bm{M}\left(\bm{q}\right) \ddot{\bm{q}} + \bm{C}\left(\bm{q}, \dot{\bm{q}}\right) \dot{\bm{q}} + \bm{G}\left(\bm{q}\right) - \bm{B}\left(\bm{q}\right) \bm{\tau}$, the subscript $\left(\cdot\right)_{m}$ indicates the time sample $m T_s$, and $\bm{\pi} = \begin{bmatrix}
    E & \beta
\end{bmatrix}^{\top}$.

The best set of parameters can be found by solving a \ac{WLS} problem, such as
\begin{equation} \label{eq:wls_functional}
    \hat{\bm{\pi}} = \underset{\bm{\pi}}{\arg \, \min} \left\{\frac{1}{2}\left(\bm{\delta} - \bm{Y}\bm{\pi}\right)^{\top} \bm{W} \left(\bm{\delta} - \bm{Y}\bm{\pi}\right)\right\} \, ,
\end{equation}
where $\bm{Y} = \begin{bmatrix} \bm{Y}^{\top}_0 & \bm{Y}^{\top}_{1} & \dots & \bm{Y}^{\top}_{M - 1} \end{bmatrix}^{\top}$, $\bm{\delta} = \begin{bmatrix} \bm{\delta}^{\top}_0 & \bm{\delta}^{\top}_{1} & \dots & \bm{\delta}^{\top}_{M - 1} \end{bmatrix}^{\top}$, and $\bm{W} \in \mathbb{R}^{\left(n \cdot M\right) \times \left(n \cdot M\right)}$. The weight matrix $\bm{W}$ allows us to give more importance to different strain modes or different time samples.

The solution of \eqref{eq:wls_functional} can be expressed as
\begin{equation} \label{eq:wsl_identification}
    \hat{\bm{\pi}} = \bm{Y}^{\dagger}_{\bm{W}} \, \bm{\delta} \, ,
\end{equation}
where $\bm{Y}^{\dagger}_{\bm{W}} = \left(\bm{Y}^{\top} \bm{W} \bm{Y}\right)^{-1} \bm{Y}^{\top} \bm{W}$ is the weighted Moore-Penrose left pseudoinverse.
\section{Numerical Validation} \label{sec:numerical_validation}
    The data-driven procedure is validated through numerical examples. We applied this method to two simulated robots.
% H-Support
The former is H-Support (Fig. \ref{fig:h_support}), a \ac{CSR} with a cylindrical cross-section, actuated by three longitudinal and four helicoidal actuators. This robot is a modified version of the existing I-Support \cite{manti2016soft}.
Table \ref{tab:sim_parameters} lists the geometrical and physical parameters.
% Conical H-Support
With the same actuation and physical characteristics, the latter robot is the conical variant of the H-Support (Fig. \ref{fig:conical_hsupport_sketch}).
% % H-Support
% \begin{figure}
%     \centering
%     \includesvg[width=1.0\linewidth]{imgs/h_support_sketch.svg}
%     \caption{Sketch of the H-Support robot, a cylindrical \ac{CSR} with 3 longitudinal and 4 helicoidal actuators. The geometrical and physical parameters are listed in Tab. \ref{tab:sim_parameters}.}
%     \label{fig:h_support}
% \end{figure}
% H-Support
\begin{figure}
    \centering
    \includegraphics[width=1.0\linewidth]{imgs/h_support_sketch.pdf}
    \caption{Sketch of the H-Support robot, a cylindrical \ac{CSR} with 3 longitudinal and 4 helicoidal actuators. The geometrical and physical parameters are listed in Tab. \ref{tab:sim_parameters}.}
    \label{fig:h_support}
\end{figure}
% % Conical Robot
% \begin{figure}
%     \centering
%     \includesvg[width=1.0\linewidth]{imgs/conical_hsupport_sketch.svg}
%     \caption{A sketch of the Conical H-Support. The conical shape is described by a $s$-varying cross-section's radius of $R_{\textnormal{cs}}(s) = \bar{R}_{\textnormal{cs}} \left(1 -0.9 s\right)$. The geometrical and physical parameters are listed in Tab. \ref{tab:sim_parameters}.}
%     \label{fig:conical_hsupport_sketch}
% \end{figure}
% Conical Robot
\begin{figure}
    \centering
    \includegraphics[width=1.0\linewidth]{imgs/conical_hsupport_sketch.pdf}
    \caption{A sketch of the Conical H-Support. The conical shape is described by a $s$-varying cross-section's radius of $R_{\textnormal{cs}}(s) = \bar{R}_{\textnormal{cs}} \left(1 -0.9 s\right)$. The geometrical and physical parameters are listed in Tab. \ref{tab:sim_parameters}.}
    \label{fig:conical_hsupport_sketch}
\end{figure}
% Parameters in Simulation
\begin{table}
\centering
\caption{Geometrical and physical parameters of the Simulated H-Support}
\label{tab:sim_parameters}
    \begin{tabular}{lll}
    \toprule
    Name                    &   Symbol          &   Value                                                               \\
    \midrule
    Length                  &   $L$               &   \SI{1.0}{\meter}                                                            \\
    Cross-Section Radius               &   $\bar{R}_{\textnormal{cs}}$               &   \SI{0.1}{\meter}                                                            \\
    Density                 &   $\bar{\rho}$    &   \SI{1000.0}{\kilogram/\meter^3}                                     \\     
    Young's Modulus         &   $E$             &   \SI{1.0}{M\pascal}                                             \\
    Poisson Ratio           &   $\nu$           &   0.5                                                                     \\
    Damping Coefficient                &   $\beta$         &   \SI{0.01}{M\pascal \cdot \second}         \\
    Stress-Free Strain      & $\bm{\xi}^{*}$    &   $[0, \, 0, \, 0, \, 1, \, 0, \, 0]^{\top}$    \\
    \bottomrule
    \end{tabular}
\end{table}
The simulations were developed using the \ac{SoRoSim} \cite{mathew2022sorosim} framework.
The robots are simulated with three different functional bases: (i) polynomial \eqref{eq:polynomial_basis}, (ii) trigonometrical \eqref{eq:fourier_basis}, and (iii) Gaussian \eqref{eq:gaussian_basis}. In all cases, we employed a second-order truncation.

For each basis, we followed the procedure outlined in Sec. \ref{sec:spectrum_extraction}.
The strain field was sampled with a sampling wavenumber of $\nu_s = \SI{100}{\meter^{-1}}$, corresponding a sampling wavelength of $\lambda_s = \SI{0.01}{\meter}$. The \acp{CSR} were tested using the standard signals for a simulation time of $t_{\textnormal{f}} = \SI{10}{\second}$. Finally, \eqref{eq:crt_dynamics} was solved with the Runge-Kutta 4th order algorithm, with a sampling frequency of $f_s = \SI{5}{k \hertz}$, obtaining the samples $\bm{\xi}(n \lambda_s, \, m T_s)$.

\subsection{H-Support} \label{numerical_validation:hsupport}
    In Fig. \ref{fig:stft_time} is shown the \acp{STFT} of the three simulations, organized in functional basis (row) and strain modes (columns). The results are discussed in the following subsections.
    % %
    % \begin{figure*}
    %     \centering
    %     \includesvg[width=1.0\linewidth]{imgs/simulations/stft_time.svg}
    %     \caption{The space-time spectra of the H-Support numerical example discussed in Sec. \ref{numerical_validation:hsupport}. The \acp{STFT} are organized in functional basis (rows) and strain modes (columns). The \acp{STFT} show the time-frequency response varying the components of the spatial spectrum. The values are normalized to $|\Xi_i(j0, j0)|$.}
    %     \label{fig:stft_time}
    % \end{figure*}
    %
    \begin{figure*}
        \centering
        \includegraphics[width=1.0\linewidth]{imgs/simulations/stft_time.pdf}
        \caption{The space-time spectra of the H-Support numerical example discussed in Sec. \ref{numerical_validation:hsupport}. The \acp{STFT} are organized in functional basis (rows) and strain modes (columns). The \acp{STFT} show the time-frequency response varying the components of the spatial spectrum. The values are normalized to $|\Xi_i(j0, j0)|$.}
        \label{fig:stft_time}
    \end{figure*}
    %
    % Discussion
    \subsubsection{Spatial Spectrum Components}
    % % Strain Analysis
    % \begin{figure}
    %     \centering
    %     \includesvg[width=1.0\linewidth]{imgs/simulations/strain_analysis.svg}
    %     \caption{Strain Analysis of the H-Support robot with only an active helicoidal actuator with a magnitude of $\SI{1}{\newton}$. For bending and shear, the numerical analysis reveals a trigonometric pattern; for twisting and shear, a constant pattern.}
    %     \label{fig:strain_analysis}
    % \end{figure}
    % Strain Analysis
    \begin{figure}
        \centering
        \includegraphics[width=1.0\linewidth]{imgs/simulations/strain_analysis.pdf}
        \caption{Strain Analysis of the H-Support robot with only an active helicoidal actuator with a magnitude of $\SI{1}{\newton}$. For bending and shear, the numerical analysis reveals a trigonometric pattern; for twisting and shear, a constant pattern.}
        \label{fig:strain_analysis}
    \end{figure}
    %
    From Fig. \ref{fig:stft_time}, we can infer the spatial spectrum composition for each strain mode.

    Regarding the twisting mode ($\kappa_x$), the most prominent component is the constant term across each functional basis.
    In contrast, the bending and shear modes exhibit distinct spectra. Counterintuitively, the harmonic at $\nu = 0 , \textnormal{m}^{-1}$ is significantly less prominent than the subsequent harmonics. This behavior can be attributed to the helicoidal actuators, which excite the bending and shear modes with a sinusoidal profile along the rod.
    
    To validate this observation, we performed strain analysis using the \ac{ISP} method \cite{renda2020geometric, renda2024dynamics}. Specifically, the strain field can be numerically computed by solving
    \begin{equation} \label{eq:implicit_strain} 
        \bm{\xi}(s) = \bm{\Sigma}^{-1}(s) , \bm{B}_{\bm{\tau}}\left(\bm{\xi}, s\right) \bm{\tau} + \bm{\xi}^{*}(s) \, . 
    \end{equation}
    
    The solution of \eqref{eq:implicit_strain} corresponds to the strain modes excited by the actuators in the static regime \cite{renda2020geometric}.
    Fig. \ref{fig:strain_analysis} shows the results, highlighting a constant stretch and twist. Moreover, the analysis reveals that the bending and shear strains exhibit a sinusoidal profile. Notably, the bending and shear strains concerning the same axis (i.e., $\kappa_y, \sigma_y$ and $\kappa_z, \sigma_z$) are in phase opposition. This indicates that the two strain modes exhibit destructive interference in space. Furthermore, this finding reveals a coupling between the bending and shear modes in static conditions, which arises from the specific geometry of the actuator.

    \subsubsection{Resonance and Anti-resonance Peaks}
    Regardless of the chosen functional basis, the system displays a sequence of resonance and anti-resonance peaks in time, which is characteristic of systems with a high number of \ac{DoFs} \cite{Ewins1999}. The frequency range of this sequence slightly varies with the strain mode.

    % Torsion
    Concerning the torsion $\kappa_x$, the sequence starts with an anti-resonance peak at $\SI{0.6}{\hertz}$ and ends at $\SI{5}{\hertz}$ with the lowest anti-resonance peak. The highest resonance peak can be found at $\SI{1.75}{\hertz}$ and is invariant to the choice of the functional basis.

    In Fig. \ref{fig:torsion_bode}, the Bode diagram of the polynomial $\kappa_x$ is presented. The phase diagram shows the distinct behavior of the constant component ($\nu = \SI{0}{\meter^{-1}}$) compared to the other harmonics.

    At the first anti-resonance peak, the phase of the constant component decreases smoothly, while the other harmonics exhibit a rapid phase shift. Near the highest resonance peak, the phase of all harmonics increases rapidly. Subsequently, the phase decreases again at the last anti-resonance peak.
    % % Torsion Bode
    % \begin{figure}
    %     \centering
    %     \includesvg[width=1.0\linewidth]{imgs/simulations/torsion_bode.svg}
    %     \caption{Bode diagram of the twisting mode $\kappa_x$. The magnitude is normalized w.r.t. $|\kappa_x(j0, \, j 0)|$.}
    %     \label{fig:torsion_bode}
    % \end{figure}
    % Torsion Bode
    \begin{figure}
        \centering
        \includegraphics[width=1.0\linewidth]{imgs/simulations/torsion_bode.pdf}
        \caption{Bode diagram of the twisting mode $\kappa_x$. The magnitude is normalized w.r.t. $|\kappa_x(j0, \, j 0)|$.}
        \label{fig:torsion_bode}
    \end{figure}
    
    % Curvature and Shear
    As previewed in the strain analysis (Fig. \ref{fig:strain_analysis}), the bending and shear modes are coupled, exhibiting similar sequences of resonance and anti-resonance peaks.
    Since resonance and anti-resonance phenomena are associated with constructive and destructive interference, it is valuable to examine the phase.
    Fig. \ref{fig:hsupport_bode} presents the Bode diagram for the couples $\kappa_y, \sigma_y$ and $\kappa_z, \sigma_y$, where the phase changes rapidly at the resonance and anti-resonance peaks.
    
    For the first pair, the phase plot shows that the two phases decrease continuously with an offset of $\pi$. Conversely, for the second pair, the phases vary in sync. This behavior reflects the interference patterns observed in the static case during the strain analysis.
    The shear $\sigma_y(\nu = \SI{10}{\meter^{-1}})$ exhibits an intense anti-resonance peak at $\SI{4.8}{\hertz}$ which results in a phase increase of $\pi$.
    As a consequence, the phase difference in both of the pairs converges toward destructive interference.

    % Peaks along the basis
    Another observation from Fig. \ref{fig:stft_time} is that the peak frequencies exhibit slight variations when using the Gaussian basis. While the first spatial harmonics of the Gaussian basis behave similarly to those of the other bases, the higher harmonics display peaks at neighboring frequencies.
    For instance, in the torsion mode at $\nu = \SI{50}{\meter^{-1}}$, the first anti-resonance peak occurs at $\SI{0.4}{\hertz}$ instead of $\SI{0.6}{\hertz}$. Similarly, for $\kappa_z$, the spatial harmonic $\nu = \SI{5}{\meter^{-1}}$ exhibits an anti-resonance peak at $\SI{12.5}{\hertz}$, which is the highest peak in frequency across all strain modes and bases.
    % % Bending & Shear
    % \begin{figure}
    %     \centering
    %     \includesvg[width=1.0\linewidth]{imgs/simulations/hsupport_bode.svg}
    %     \caption{Comparison between the Bode diagrams of $\kappa_y$, $\sigma_y$, and $\kappa_z$. In the last row, the difference in phase between the modes is reported.}
    %     \label{fig:hsupport_bode}
    % \end{figure}
    % Bending & Shear
    \begin{figure}
        \centering
        \includegraphics[width=1.0\linewidth]{imgs/simulations/hsupport_bode.pdf}
        \caption{Comparison between the Bode diagrams of $\kappa_y$, $\sigma_y$, and $\kappa_z$. In the last row, the difference in phase between the modes is reported.}
        \label{fig:hsupport_bode}
    \end{figure}
    
    \subsubsection{Frequency-varying Behavior}
    As discussed in Sec. \ref{sec:space_time_ft}, the \ac{STFT} illustrates a well-known characteristic of \acp{CSR}: their deformations vary depending on the input's time-frequency. To show that, we reported in Fig. \ref{fig:stft_space} the spatial spectrum varying the time-frequency. For each time-frequency, the spatial spectrum's pattern corresponds to the theoretical one, imposed by the \ac{SoRoSim} framework. Consequently, the shape of the spatial spectrum remains invariant w.r.t. time-frequency, with variations occurring only in magnitude.
    % % STFT Space
    % \begin{figure}
    %     \centering
    %     \includesvg[width=1.0\linewidth]{imgs/simulations/stft_space.svg}
    %     \caption{Spatial Spectrum varying the time-frequencies.    
    %     The spatial harmonics $\nu > \SI{2}{\meter^{-1}}$ are not taken into account in the case of trigonometric basis since the analysis is limited to the second order.}
    %     \label{fig:stft_space}
    % \end{figure}
    % STFT Space
    \begin{figure}
        \centering
        \includegraphics[width=1.0\linewidth]{imgs/simulations/stft_space.pdf}
        \caption{Spatial Spectrum varying the time-frequencies.    
        The spatial harmonics $\nu > \SI{2}{\meter^{-1}}$ are not taken into account in the case of trigonometric basis since the analysis is limited to the second order.}
        \label{fig:stft_space}
    \end{figure}

    \subsection{Conical H-Support} \label{numerical_validation:conical_hsupport}
    % Intro & Description of the prototype
    The numerical validation is also performed on a conical H-Support variant (Fig. \ref{fig:conical_hsupport_sketch}), where the physical and geometrical parameters are kept the same and listed in Table \ref{tab:sim_parameters}.
    The conical profile of the \ac{CSR} is realized by the linear function of the cross-section's radius, such as $R_{\textnormal{cs}}(s) = \bar{R}_{\textnormal{cs}} \left(1 -0.9 s\right)$.
    As a consequence, the cross-section's area and second moment of area will be $A(s) \propto s^2$, and $J_i(s) \propto s^4$, respectively.
    Therefore, in contrast to the cylindrical case, the dynamic matrices (i.e., inertia, stiffness, and damping) are $s$-dependent, playing an active role on the \ac{STFT} spectra.   
    % STFT of Conical H-Support 
    Fig. \ref{fig:conical_hsupport_stft} shows the bode diagram for the polynomial basis case.
    % Torsion
    Differently from the cylindrical case, the torsion constant component ($\nu = \SI{0}{\meter^{-1}}$) shows a spectrum similar to the other spatial harmonics.
    Unlike the H-Support case, the higher harmonics show no anti-resonance peak at $\SI{0.6}{\hertz}$ and the difference in magnitude between $\kappa_x(\nu = 0)$ and the other components varies more smoothly. The polynomial dependency of the actuation matrix and stiffness, which affects the excited basis of the torsion mode, justifies this observation.
    % Main resonance peak
    Notably, the highest resonance peak remains the same frequency ($\SI{1.75}{\hertz}$) in the conical case across all the strain modes.

    Regarding the bending and shear modes, the bode diagrams exhibit a coupling between $\kappa_y, \sigma_z$ and $\kappa_z, \sigma_y$. The former pair exhibits no evident anti-resonance peaks since the magnitude and the phase vary smoothly. In contrast, the latter pair shows a pronounced anti-resonance peak with a rapid variation in phase. Differently from the cylindrical case, all the harmonics (except the $\nu = \SI{0}{\meter^{-1}}$), display this peak at $\SI{4.8}{\hertz}$.
   % % STFT Conical
   %  \begin{figure*}
   %      \centering
   %      \includesvg[width=1.0\linewidth]{imgs/simulations/conical_hsupport_stft.svg}
   %      \caption{The space-time spectra of the Conical H-Support numerical example discussed in Sec. \ref{numerical_validation:conical_hsupport}. The \acp{STFT} show the time-frequency response varying the components of the spatial spectrum. The magnitude values are normalized to $|\Xi_i(j0, j0)|$.}
   %      \label{fig:conical_hsupport_stft}
   %  \end{figure*}
  % STFT Conical
    \begin{figure*}
        \centering
        \includegraphics[width=1.0\linewidth]{imgs/simulations/conical_hsupport_stft.pdf}
        \caption{The space-time spectra of the Conical H-Support numerical example discussed in Sec. \ref{numerical_validation:conical_hsupport}. The \acp{STFT} show the time-frequency response varying the components of the spatial spectrum. The magnitude values are normalized to $|\Xi_i(j0, j0)|$.}
        \label{fig:conical_hsupport_stft}
    \end{figure*}
\section{Experimental Validation} \label{sec:experimental_validation}
    In this section, we describe the protocol, models used, and results of two experiments intended to validate that our distributed proprioceptive sensor is able to detect and reject external disturbances. In the first experiment, we applied static forces to the fingertip using calibration weights and examined the fingertip position error based on linear regression trained during the experiments of section \ref{sec:experimental_calibration}, and we compared this to a baseline single-sensor finger consisting of a single POF sensor integrated using the same 3D printing method as the multi-sensor fingers. In the second experiment, we trained a data-driven model using the data acquired during the experiments of section \ref{sec:experimental_calibration} to detect between the finger and a light switch based on deviations of the finger sensor readings from those acquired during actuation in the absence of contact. 
\subsection{Protocol}
\subsubsection{External Disturbance Experiment}
To test the accuracy of our multi-sensor approach in the presence of external disturbances, we applied the same testing protocol as in \ref{subsec:protocol} to each finger with calibration weights tied to the fingertip with nylon fishing line (Fig. \ref{fig:disturbances}a). We ran one test with a \qty{20}{g} weight and one test with a \qty{50}{g} weight for each finger. We then compared the results with those coming from a finger of the same geometry with only one fiber optic running along its length and a single pair of FPCs, one emitter and one receiver. We first tested this version of the finger without weights to determine the baseline calibration for pose estimation purposes and then with the \qty{20}{g} and \qty{50}{g} weights, following the same protocol as before. 

\subsubsection{Contact Detection Experiment}
To demonstrate how the finger's multi-sensor architecture can be leveraged to detect contact with an object in the environment without any knowledge of the actuation state, we selected one finger, placed it underneath a light switch, and actuated it until it flipped the switch while collecting data from the sensors at \qty{8}{Hz}. We actuated the finger in steps of \qty{9}{\degree} with \qty{0.25}{\s} pauses to facilitate webcam image acquisition.

\begin{figure}
\centering
\includegraphics[width=0.9\columnwidth]{fingertip_error_bar_chart.pdf}
\caption{(a) Diagram of experimental setup for applying static loads to fingertip. (b) Pose estimations of the highly integrated finger and single-sensor finger shown with marker feedback. The optical intensity measured by the single-sensor finger increases due to the applied force, resulting in a pose estimation pointing the opposite direction, inconsistent with the operating range of the sensor. (c) Average fingertip position error recorded during each trial.}
\label{fig:disturbances}
\vspace{-1em}
\end{figure}

\begin{figure*}
    \centering
    \includegraphics[width=0.8\textwidth]{demo_figure.pdf}
    \caption{Contact detection experiment. The $Q$-statistic from the PCA model of the sensors is used against a threshold value shown in red in the chart to detect external contacts.}
    \label{fig:contact_detection_figure}
    \vspace{-1em}
\end{figure*}

\subsection{Pose Estimation and Contact Detection Models}
For pose estimation, we modeled the phalanges as a kinematic chain in two dimensional space with four rigid links of length \qty{28}{\mm} connected by joints with angles estimated by the sensors. For the multi-sensor fingers, joint angles were estimated by inverting the models obtained in section \ref{subsec:sensor_model}. For the single-sensor finger, we employed the same linear regression approach as the multi-sensor finger to train a model predicting each of the joint angles using the data from the single sensor acquired during the test with no weight.

For the contact detection model, we applied principle component analysis (PCA) to the sensor readings acquired from the quasi-static tests utilizing a technique commonly used for fault detection in industrial processes \cite{Yin2014}. We trained an $m$-dimensional PCA model with $m=3$ on the three-dimensional sensor data from the finger. The first component of the PCA model explains 91\% of the data variability, which is consistent with the one-DOF actuation of the finger and allowed us to conclude that the principal subspace of the PCA model has dimensionality $\beta=1$. The model detects contact by projecting online sensor observations onto the residual subspace of the PCA model and calculating the $Q$-statistic:
\begin{equation}\label{eq:SPE}
    Q=z^\top P_\textrm{res} P_\textrm{res}^\top z
\end{equation}
where $z$ is the $3\times 1$ vector of online sensor observations and $P_\textrm{res}$ is the matrix of residual PCA coefficients. We used the threshold for anomaly detection from \cite{Yin2014} determined with confidence level $\alpha=\text{90\%}$:
\begin{equation}\label{eq:J_th}
    J_{{\rm th,} Q} = \vartheta_{1}\left(\frac{c_{\alpha}\sqrt{2\vartheta_{2}h_{0}^{2}}}{\vartheta_{1}} + 1 + \frac{\vartheta_{2}h_{0}(h_{0} - 1)}{ \vartheta_{1}^{2}}\right)^{1/ h_{0}}
\end{equation}
where $c_\alpha=1.282$ is the normal deviate for the upper $1-\alpha$ percentile and $h_0$ and $\vartheta_i$ are parameters calculated from the variances of the last $m-\beta$ principal components $\lambda_i$:
$$h_{0} = 1 - \frac{2\vartheta_{1}\vartheta_{3}}{3\vartheta_{2}^{2}} \qquad \vartheta_{i} = \sum_{j = \beta + 1}^{m}(\lambda_{j})^{i}; \quad i = 1, 2, 3.$$

\subsection{Results}
The results from the external disturbance experiment (Fig. \ref{fig:disturbances}c) show that the multi-sensor finger estimates the fingertip position with lower error than the single-sensor finger. We expected based on the underactuated training data that the single-sensor prediction would produce approximately equal estimates for each joint angle and that the primary source of pose estimation error would be due to poses with unequal angles. However, we also see in Fig. \ref{fig:disturbances}b that the single-sensor pose prediction points in the wrong direction. This is due to the fact that the optical intensity reading of the sensor increases in the presence of external forces to the point that the sensors observe optical intensity gain rather than loss during the experiment. Because the sensor model is based on a linear regression that assumes the sensor will only experience optical intensity loss, if gain is observed, the sensor will predict a negative angle outside of the designed operational range. While this effect is also present in the multi-sensor fingers, it is mitigated by the redundancy afforded by the multi-sensor architecture. We also note that the magnitude of the weight did not directly correlate to the prediction error across the fingers, at least for the range of weights that was tested. 

The results from the contact detection experiment are displayed in Fig. \ref{fig:contact_detection_figure} and the supporting video to the paper. First contact was observed in the trial video at $t=\qty{0.7}{s}$, and the model detected contact at $t=\qty{4.2}{s}$. At $t=\qty{12.2}{s}$, the finger came out of contact with the light switch, and the corresponding $Q$ dropped below $J_{{\rm th,} Q}$ shortly afterwards, at the next sensor reading update. We note that the finger is able to detect coming out of contact online at \qty{8}{Hz}, while the webcam acquisition misses a \qty{0.25}{\s} frame due to motion blur then takes another \qty{0.25}{\s} frame to detect the new pose of the finger, further demonstrating the merits of a low processing-power distributed device. 
\section{Conclusion} \label{sec:conclusions}
    \section{Conclusion}
In this work, we propose a simple yet effective approach, called SMILE, for graph few-shot learning with fewer tasks. Specifically, we introduce a novel dual-level mixup strategy, including within-task and across-task mixup, for enriching the diversity of nodes within each task and the diversity of tasks. Also, we incorporate the degree-based prior information to learn expressive node embeddings. Theoretically, we prove that SMILE effectively enhances the model's generalization performance. Empirically, we conduct extensive experiments on multiple benchmarks and the results suggest that SMILE significantly outperforms other baselines, including both in-domain and cross-domain few-shot settings.

%%%%%%%%%%%%%%%%%% APPENDIX %%%%%%%%%%%%%%%%%%
% if have a single appendix:
%\appendix[Proof of the Zonklar Equations]
% or
%\appendix  % for no appendix heading
% do not use \section anymore after \appendix, only \section*
% is possibly needed

% use appendices with more than one appendix
% then use \section to start each appendix
% you must declare a \section before using any
% \subsection or using \label (\appendices by itself
% starts a section numbered zero.)
\subsection{Lloyd-Max Algorithm}
\label{subsec:Lloyd-Max}
For a given quantization bitwidth $B$ and an operand $\bm{X}$, the Lloyd-Max algorithm finds $2^B$ quantization levels $\{\hat{x}_i\}_{i=1}^{2^B}$ such that quantizing $\bm{X}$ by rounding each scalar in $\bm{X}$ to the nearest quantization level minimizes the quantization MSE. 

The algorithm starts with an initial guess of quantization levels and then iteratively computes quantization thresholds $\{\tau_i\}_{i=1}^{2^B-1}$ and updates quantization levels $\{\hat{x}_i\}_{i=1}^{2^B}$. Specifically, at iteration $n$, thresholds are set to the midpoints of the previous iteration's levels:
\begin{align*}
    \tau_i^{(n)}=\frac{\hat{x}_i^{(n-1)}+\hat{x}_{i+1}^{(n-1)}}2 \text{ for } i=1\ldots 2^B-1
\end{align*}
Subsequently, the quantization levels are re-computed as conditional means of the data regions defined by the new thresholds:
\begin{align*}
    \hat{x}_i^{(n)}=\mathbb{E}\left[ \bm{X} \big| \bm{X}\in [\tau_{i-1}^{(n)},\tau_i^{(n)}] \right] \text{ for } i=1\ldots 2^B
\end{align*}
where to satisfy boundary conditions we have $\tau_0=-\infty$ and $\tau_{2^B}=\infty$. The algorithm iterates the above steps until convergence.

Figure \ref{fig:lm_quant} compares the quantization levels of a $7$-bit floating point (E3M3) quantizer (left) to a $7$-bit Lloyd-Max quantizer (right) when quantizing a layer of weights from the GPT3-126M model at a per-tensor granularity. As shown, the Lloyd-Max quantizer achieves substantially lower quantization MSE. Further, Table \ref{tab:FP7_vs_LM7} shows the superior perplexity achieved by Lloyd-Max quantizers for bitwidths of $7$, $6$ and $5$. The difference between the quantizers is clear at 5 bits, where per-tensor FP quantization incurs a drastic and unacceptable increase in perplexity, while Lloyd-Max quantization incurs a much smaller increase. Nevertheless, we note that even the optimal Lloyd-Max quantizer incurs a notable ($\sim 1.5$) increase in perplexity due to the coarse granularity of quantization. 

\begin{figure}[h]
  \centering
  \includegraphics[width=0.7\linewidth]{sections/figures/LM7_FP7.pdf}
  \caption{\small Quantization levels and the corresponding quantization MSE of Floating Point (left) vs Lloyd-Max (right) Quantizers for a layer of weights in the GPT3-126M model.}
  \label{fig:lm_quant}
\end{figure}

\begin{table}[h]\scriptsize
\begin{center}
\caption{\label{tab:FP7_vs_LM7} \small Comparing perplexity (lower is better) achieved by floating point quantizers and Lloyd-Max quantizers on a GPT3-126M model for the Wikitext-103 dataset.}
\begin{tabular}{c|cc|c}
\hline
 \multirow{2}{*}{\textbf{Bitwidth}} & \multicolumn{2}{|c|}{\textbf{Floating-Point Quantizer}} & \textbf{Lloyd-Max Quantizer} \\
 & Best Format & Wikitext-103 Perplexity & Wikitext-103 Perplexity \\
\hline
7 & E3M3 & 18.32 & 18.27 \\
6 & E3M2 & 19.07 & 18.51 \\
5 & E4M0 & 43.89 & 19.71 \\
\hline
\end{tabular}
\end{center}
\end{table}

\subsection{Proof of Local Optimality of LO-BCQ}
\label{subsec:lobcq_opt_proof}
For a given block $\bm{b}_j$, the quantization MSE during LO-BCQ can be empirically evaluated as $\frac{1}{L_b}\lVert \bm{b}_j- \bm{\hat{b}}_j\rVert^2_2$ where $\bm{\hat{b}}_j$ is computed from equation (\ref{eq:clustered_quantization_definition}) as $C_{f(\bm{b}_j)}(\bm{b}_j)$. Further, for a given block cluster $\mathcal{B}_i$, we compute the quantization MSE as $\frac{1}{|\mathcal{B}_{i}|}\sum_{\bm{b} \in \mathcal{B}_{i}} \frac{1}{L_b}\lVert \bm{b}- C_i^{(n)}(\bm{b})\rVert^2_2$. Therefore, at the end of iteration $n$, we evaluate the overall quantization MSE $J^{(n)}$ for a given operand $\bm{X}$ composed of $N_c$ block clusters as:
\begin{align*}
    \label{eq:mse_iter_n}
    J^{(n)} = \frac{1}{N_c} \sum_{i=1}^{N_c} \frac{1}{|\mathcal{B}_{i}^{(n)}|}\sum_{\bm{v} \in \mathcal{B}_{i}^{(n)}} \frac{1}{L_b}\lVert \bm{b}- B_i^{(n)}(\bm{b})\rVert^2_2
\end{align*}

At the end of iteration $n$, the codebooks are updated from $\mathcal{C}^{(n-1)}$ to $\mathcal{C}^{(n)}$. However, the mapping of a given vector $\bm{b}_j$ to quantizers $\mathcal{C}^{(n)}$ remains as  $f^{(n)}(\bm{b}_j)$. At the next iteration, during the vector clustering step, $f^{(n+1)}(\bm{b}_j)$ finds new mapping of $\bm{b}_j$ to updated codebooks $\mathcal{C}^{(n)}$ such that the quantization MSE over the candidate codebooks is minimized. Therefore, we obtain the following result for $\bm{b}_j$:
\begin{align*}
\frac{1}{L_b}\lVert \bm{b}_j - C_{f^{(n+1)}(\bm{b}_j)}^{(n)}(\bm{b}_j)\rVert^2_2 \le \frac{1}{L_b}\lVert \bm{b}_j - C_{f^{(n)}(\bm{b}_j)}^{(n)}(\bm{b}_j)\rVert^2_2
\end{align*}

That is, quantizing $\bm{b}_j$ at the end of the block clustering step of iteration $n+1$ results in lower quantization MSE compared to quantizing at the end of iteration $n$. Since this is true for all $\bm{b} \in \bm{X}$, we assert the following:
\begin{equation}
\begin{split}
\label{eq:mse_ineq_1}
    \tilde{J}^{(n+1)} &= \frac{1}{N_c} \sum_{i=1}^{N_c} \frac{1}{|\mathcal{B}_{i}^{(n+1)}|}\sum_{\bm{b} \in \mathcal{B}_{i}^{(n+1)}} \frac{1}{L_b}\lVert \bm{b} - C_i^{(n)}(b)\rVert^2_2 \le J^{(n)}
\end{split}
\end{equation}
where $\tilde{J}^{(n+1)}$ is the the quantization MSE after the vector clustering step at iteration $n+1$.

Next, during the codebook update step (\ref{eq:quantizers_update}) at iteration $n+1$, the per-cluster codebooks $\mathcal{C}^{(n)}$ are updated to $\mathcal{C}^{(n+1)}$ by invoking the Lloyd-Max algorithm \citep{Lloyd}. We know that for any given value distribution, the Lloyd-Max algorithm minimizes the quantization MSE. Therefore, for a given vector cluster $\mathcal{B}_i$ we obtain the following result:

\begin{equation}
    \frac{1}{|\mathcal{B}_{i}^{(n+1)}|}\sum_{\bm{b} \in \mathcal{B}_{i}^{(n+1)}} \frac{1}{L_b}\lVert \bm{b}- C_i^{(n+1)}(\bm{b})\rVert^2_2 \le \frac{1}{|\mathcal{B}_{i}^{(n+1)}|}\sum_{\bm{b} \in \mathcal{B}_{i}^{(n+1)}} \frac{1}{L_b}\lVert \bm{b}- C_i^{(n)}(\bm{b})\rVert^2_2
\end{equation}

The above equation states that quantizing the given block cluster $\mathcal{B}_i$ after updating the associated codebook from $C_i^{(n)}$ to $C_i^{(n+1)}$ results in lower quantization MSE. Since this is true for all the block clusters, we derive the following result: 
\begin{equation}
\begin{split}
\label{eq:mse_ineq_2}
     J^{(n+1)} &= \frac{1}{N_c} \sum_{i=1}^{N_c} \frac{1}{|\mathcal{B}_{i}^{(n+1)}|}\sum_{\bm{b} \in \mathcal{B}_{i}^{(n+1)}} \frac{1}{L_b}\lVert \bm{b}- C_i^{(n+1)}(\bm{b})\rVert^2_2  \le \tilde{J}^{(n+1)}   
\end{split}
\end{equation}

Following (\ref{eq:mse_ineq_1}) and (\ref{eq:mse_ineq_2}), we find that the quantization MSE is non-increasing for each iteration, that is, $J^{(1)} \ge J^{(2)} \ge J^{(3)} \ge \ldots \ge J^{(M)}$ where $M$ is the maximum number of iterations. 
%Therefore, we can say that if the algorithm converges, then it must be that it has converged to a local minimum. 
\hfill $\blacksquare$


\begin{figure}
    \begin{center}
    \includegraphics[width=0.5\textwidth]{sections//figures/mse_vs_iter.pdf}
    \end{center}
    \caption{\small NMSE vs iterations during LO-BCQ compared to other block quantization proposals}
    \label{fig:nmse_vs_iter}
\end{figure}

Figure \ref{fig:nmse_vs_iter} shows the empirical convergence of LO-BCQ across several block lengths and number of codebooks. Also, the MSE achieved by LO-BCQ is compared to baselines such as MXFP and VSQ. As shown, LO-BCQ converges to a lower MSE than the baselines. Further, we achieve better convergence for larger number of codebooks ($N_c$) and for a smaller block length ($L_b$), both of which increase the bitwidth of BCQ (see Eq \ref{eq:bitwidth_bcq}).


\subsection{Additional Accuracy Results}
%Table \ref{tab:lobcq_config} lists the various LOBCQ configurations and their corresponding bitwidths.
\begin{table}
\setlength{\tabcolsep}{4.75pt}
\begin{center}
\caption{\label{tab:lobcq_config} Various LO-BCQ configurations and their bitwidths.}
\begin{tabular}{|c||c|c|c|c||c|c||c|} 
\hline
 & \multicolumn{4}{|c||}{$L_b=8$} & \multicolumn{2}{|c||}{$L_b=4$} & $L_b=2$ \\
 \hline
 \backslashbox{$L_A$\kern-1em}{\kern-1em$N_c$} & 2 & 4 & 8 & 16 & 2 & 4 & 2 \\
 \hline
 64 & 4.25 & 4.375 & 4.5 & 4.625 & 4.375 & 4.625 & 4.625\\
 \hline
 32 & 4.375 & 4.5 & 4.625& 4.75 & 4.5 & 4.75 & 4.75 \\
 \hline
 16 & 4.625 & 4.75& 4.875 & 5 & 4.75 & 5 & 5 \\
 \hline
\end{tabular}
\end{center}
\end{table}

%\subsection{Perplexity achieved by various LO-BCQ configurations on Wikitext-103 dataset}

\begin{table} \centering
\begin{tabular}{|c||c|c|c|c||c|c||c|} 
\hline
 $L_b \rightarrow$& \multicolumn{4}{c||}{8} & \multicolumn{2}{c||}{4} & 2\\
 \hline
 \backslashbox{$L_A$\kern-1em}{\kern-1em$N_c$} & 2 & 4 & 8 & 16 & 2 & 4 & 2  \\
 %$N_c \rightarrow$ & 2 & 4 & 8 & 16 & 2 & 4 & 2 \\
 \hline
 \hline
 \multicolumn{8}{c}{GPT3-1.3B (FP32 PPL = 9.98)} \\ 
 \hline
 \hline
 64 & 10.40 & 10.23 & 10.17 & 10.15 &  10.28 & 10.18 & 10.19 \\
 \hline
 32 & 10.25 & 10.20 & 10.15 & 10.12 &  10.23 & 10.17 & 10.17 \\
 \hline
 16 & 10.22 & 10.16 & 10.10 & 10.09 &  10.21 & 10.14 & 10.16 \\
 \hline
  \hline
 \multicolumn{8}{c}{GPT3-8B (FP32 PPL = 7.38)} \\ 
 \hline
 \hline
 64 & 7.61 & 7.52 & 7.48 &  7.47 &  7.55 &  7.49 & 7.50 \\
 \hline
 32 & 7.52 & 7.50 & 7.46 &  7.45 &  7.52 &  7.48 & 7.48  \\
 \hline
 16 & 7.51 & 7.48 & 7.44 &  7.44 &  7.51 &  7.49 & 7.47  \\
 \hline
\end{tabular}
\caption{\label{tab:ppl_gpt3_abalation} Wikitext-103 perplexity across GPT3-1.3B and 8B models.}
\end{table}

\begin{table} \centering
\begin{tabular}{|c||c|c|c|c||} 
\hline
 $L_b \rightarrow$& \multicolumn{4}{c||}{8}\\
 \hline
 \backslashbox{$L_A$\kern-1em}{\kern-1em$N_c$} & 2 & 4 & 8 & 16 \\
 %$N_c \rightarrow$ & 2 & 4 & 8 & 16 & 2 & 4 & 2 \\
 \hline
 \hline
 \multicolumn{5}{|c|}{Llama2-7B (FP32 PPL = 5.06)} \\ 
 \hline
 \hline
 64 & 5.31 & 5.26 & 5.19 & 5.18  \\
 \hline
 32 & 5.23 & 5.25 & 5.18 & 5.15  \\
 \hline
 16 & 5.23 & 5.19 & 5.16 & 5.14  \\
 \hline
 \multicolumn{5}{|c|}{Nemotron4-15B (FP32 PPL = 5.87)} \\ 
 \hline
 \hline
 64  & 6.3 & 6.20 & 6.13 & 6.08  \\
 \hline
 32  & 6.24 & 6.12 & 6.07 & 6.03  \\
 \hline
 16  & 6.12 & 6.14 & 6.04 & 6.02  \\
 \hline
 \multicolumn{5}{|c|}{Nemotron4-340B (FP32 PPL = 3.48)} \\ 
 \hline
 \hline
 64 & 3.67 & 3.62 & 3.60 & 3.59 \\
 \hline
 32 & 3.63 & 3.61 & 3.59 & 3.56 \\
 \hline
 16 & 3.61 & 3.58 & 3.57 & 3.55 \\
 \hline
\end{tabular}
\caption{\label{tab:ppl_llama7B_nemo15B} Wikitext-103 perplexity compared to FP32 baseline in Llama2-7B and Nemotron4-15B, 340B models}
\end{table}

%\subsection{Perplexity achieved by various LO-BCQ configurations on MMLU dataset}


\begin{table} \centering
\begin{tabular}{|c||c|c|c|c||c|c|c|c|} 
\hline
 $L_b \rightarrow$& \multicolumn{4}{c||}{8} & \multicolumn{4}{c||}{8}\\
 \hline
 \backslashbox{$L_A$\kern-1em}{\kern-1em$N_c$} & 2 & 4 & 8 & 16 & 2 & 4 & 8 & 16  \\
 %$N_c \rightarrow$ & 2 & 4 & 8 & 16 & 2 & 4 & 2 \\
 \hline
 \hline
 \multicolumn{5}{|c|}{Llama2-7B (FP32 Accuracy = 45.8\%)} & \multicolumn{4}{|c|}{Llama2-70B (FP32 Accuracy = 69.12\%)} \\ 
 \hline
 \hline
 64 & 43.9 & 43.4 & 43.9 & 44.9 & 68.07 & 68.27 & 68.17 & 68.75 \\
 \hline
 32 & 44.5 & 43.8 & 44.9 & 44.5 & 68.37 & 68.51 & 68.35 & 68.27  \\
 \hline
 16 & 43.9 & 42.7 & 44.9 & 45 & 68.12 & 68.77 & 68.31 & 68.59  \\
 \hline
 \hline
 \multicolumn{5}{|c|}{GPT3-22B (FP32 Accuracy = 38.75\%)} & \multicolumn{4}{|c|}{Nemotron4-15B (FP32 Accuracy = 64.3\%)} \\ 
 \hline
 \hline
 64 & 36.71 & 38.85 & 38.13 & 38.92 & 63.17 & 62.36 & 63.72 & 64.09 \\
 \hline
 32 & 37.95 & 38.69 & 39.45 & 38.34 & 64.05 & 62.30 & 63.8 & 64.33  \\
 \hline
 16 & 38.88 & 38.80 & 38.31 & 38.92 & 63.22 & 63.51 & 63.93 & 64.43  \\
 \hline
\end{tabular}
\caption{\label{tab:mmlu_abalation} Accuracy on MMLU dataset across GPT3-22B, Llama2-7B, 70B and Nemotron4-15B models.}
\end{table}


%\subsection{Perplexity achieved by various LO-BCQ configurations on LM evaluation harness}

\begin{table} \centering
\begin{tabular}{|c||c|c|c|c||c|c|c|c|} 
\hline
 $L_b \rightarrow$& \multicolumn{4}{c||}{8} & \multicolumn{4}{c||}{8}\\
 \hline
 \backslashbox{$L_A$\kern-1em}{\kern-1em$N_c$} & 2 & 4 & 8 & 16 & 2 & 4 & 8 & 16  \\
 %$N_c \rightarrow$ & 2 & 4 & 8 & 16 & 2 & 4 & 2 \\
 \hline
 \hline
 \multicolumn{5}{|c|}{Race (FP32 Accuracy = 37.51\%)} & \multicolumn{4}{|c|}{Boolq (FP32 Accuracy = 64.62\%)} \\ 
 \hline
 \hline
 64 & 36.94 & 37.13 & 36.27 & 37.13 & 63.73 & 62.26 & 63.49 & 63.36 \\
 \hline
 32 & 37.03 & 36.36 & 36.08 & 37.03 & 62.54 & 63.51 & 63.49 & 63.55  \\
 \hline
 16 & 37.03 & 37.03 & 36.46 & 37.03 & 61.1 & 63.79 & 63.58 & 63.33  \\
 \hline
 \hline
 \multicolumn{5}{|c|}{Winogrande (FP32 Accuracy = 58.01\%)} & \multicolumn{4}{|c|}{Piqa (FP32 Accuracy = 74.21\%)} \\ 
 \hline
 \hline
 64 & 58.17 & 57.22 & 57.85 & 58.33 & 73.01 & 73.07 & 73.07 & 72.80 \\
 \hline
 32 & 59.12 & 58.09 & 57.85 & 58.41 & 73.01 & 73.94 & 72.74 & 73.18  \\
 \hline
 16 & 57.93 & 58.88 & 57.93 & 58.56 & 73.94 & 72.80 & 73.01 & 73.94  \\
 \hline
\end{tabular}
\caption{\label{tab:mmlu_abalation} Accuracy on LM evaluation harness tasks on GPT3-1.3B model.}
\end{table}

\begin{table} \centering
\begin{tabular}{|c||c|c|c|c||c|c|c|c|} 
\hline
 $L_b \rightarrow$& \multicolumn{4}{c||}{8} & \multicolumn{4}{c||}{8}\\
 \hline
 \backslashbox{$L_A$\kern-1em}{\kern-1em$N_c$} & 2 & 4 & 8 & 16 & 2 & 4 & 8 & 16  \\
 %$N_c \rightarrow$ & 2 & 4 & 8 & 16 & 2 & 4 & 2 \\
 \hline
 \hline
 \multicolumn{5}{|c|}{Race (FP32 Accuracy = 41.34\%)} & \multicolumn{4}{|c|}{Boolq (FP32 Accuracy = 68.32\%)} \\ 
 \hline
 \hline
 64 & 40.48 & 40.10 & 39.43 & 39.90 & 69.20 & 68.41 & 69.45 & 68.56 \\
 \hline
 32 & 39.52 & 39.52 & 40.77 & 39.62 & 68.32 & 67.43 & 68.17 & 69.30  \\
 \hline
 16 & 39.81 & 39.71 & 39.90 & 40.38 & 68.10 & 66.33 & 69.51 & 69.42  \\
 \hline
 \hline
 \multicolumn{5}{|c|}{Winogrande (FP32 Accuracy = 67.88\%)} & \multicolumn{4}{|c|}{Piqa (FP32 Accuracy = 78.78\%)} \\ 
 \hline
 \hline
 64 & 66.85 & 66.61 & 67.72 & 67.88 & 77.31 & 77.42 & 77.75 & 77.64 \\
 \hline
 32 & 67.25 & 67.72 & 67.72 & 67.00 & 77.31 & 77.04 & 77.80 & 77.37  \\
 \hline
 16 & 68.11 & 68.90 & 67.88 & 67.48 & 77.37 & 78.13 & 78.13 & 77.69  \\
 \hline
\end{tabular}
\caption{\label{tab:mmlu_abalation} Accuracy on LM evaluation harness tasks on GPT3-8B model.}
\end{table}

\begin{table} \centering
\begin{tabular}{|c||c|c|c|c||c|c|c|c|} 
\hline
 $L_b \rightarrow$& \multicolumn{4}{c||}{8} & \multicolumn{4}{c||}{8}\\
 \hline
 \backslashbox{$L_A$\kern-1em}{\kern-1em$N_c$} & 2 & 4 & 8 & 16 & 2 & 4 & 8 & 16  \\
 %$N_c \rightarrow$ & 2 & 4 & 8 & 16 & 2 & 4 & 2 \\
 \hline
 \hline
 \multicolumn{5}{|c|}{Race (FP32 Accuracy = 40.67\%)} & \multicolumn{4}{|c|}{Boolq (FP32 Accuracy = 76.54\%)} \\ 
 \hline
 \hline
 64 & 40.48 & 40.10 & 39.43 & 39.90 & 75.41 & 75.11 & 77.09 & 75.66 \\
 \hline
 32 & 39.52 & 39.52 & 40.77 & 39.62 & 76.02 & 76.02 & 75.96 & 75.35  \\
 \hline
 16 & 39.81 & 39.71 & 39.90 & 40.38 & 75.05 & 73.82 & 75.72 & 76.09  \\
 \hline
 \hline
 \multicolumn{5}{|c|}{Winogrande (FP32 Accuracy = 70.64\%)} & \multicolumn{4}{|c|}{Piqa (FP32 Accuracy = 79.16\%)} \\ 
 \hline
 \hline
 64 & 69.14 & 70.17 & 70.17 & 70.56 & 78.24 & 79.00 & 78.62 & 78.73 \\
 \hline
 32 & 70.96 & 69.69 & 71.27 & 69.30 & 78.56 & 79.49 & 79.16 & 78.89  \\
 \hline
 16 & 71.03 & 69.53 & 69.69 & 70.40 & 78.13 & 79.16 & 79.00 & 79.00  \\
 \hline
\end{tabular}
\caption{\label{tab:mmlu_abalation} Accuracy on LM evaluation harness tasks on GPT3-22B model.}
\end{table}

\begin{table} \centering
\begin{tabular}{|c||c|c|c|c||c|c|c|c|} 
\hline
 $L_b \rightarrow$& \multicolumn{4}{c||}{8} & \multicolumn{4}{c||}{8}\\
 \hline
 \backslashbox{$L_A$\kern-1em}{\kern-1em$N_c$} & 2 & 4 & 8 & 16 & 2 & 4 & 8 & 16  \\
 %$N_c \rightarrow$ & 2 & 4 & 8 & 16 & 2 & 4 & 2 \\
 \hline
 \hline
 \multicolumn{5}{|c|}{Race (FP32 Accuracy = 44.4\%)} & \multicolumn{4}{|c|}{Boolq (FP32 Accuracy = 79.29\%)} \\ 
 \hline
 \hline
 64 & 42.49 & 42.51 & 42.58 & 43.45 & 77.58 & 77.37 & 77.43 & 78.1 \\
 \hline
 32 & 43.35 & 42.49 & 43.64 & 43.73 & 77.86 & 75.32 & 77.28 & 77.86  \\
 \hline
 16 & 44.21 & 44.21 & 43.64 & 42.97 & 78.65 & 77 & 76.94 & 77.98  \\
 \hline
 \hline
 \multicolumn{5}{|c|}{Winogrande (FP32 Accuracy = 69.38\%)} & \multicolumn{4}{|c|}{Piqa (FP32 Accuracy = 78.07\%)} \\ 
 \hline
 \hline
 64 & 68.9 & 68.43 & 69.77 & 68.19 & 77.09 & 76.82 & 77.09 & 77.86 \\
 \hline
 32 & 69.38 & 68.51 & 68.82 & 68.90 & 78.07 & 76.71 & 78.07 & 77.86  \\
 \hline
 16 & 69.53 & 67.09 & 69.38 & 68.90 & 77.37 & 77.8 & 77.91 & 77.69  \\
 \hline
\end{tabular}
\caption{\label{tab:mmlu_abalation} Accuracy on LM evaluation harness tasks on Llama2-7B model.}
\end{table}

\begin{table} \centering
\begin{tabular}{|c||c|c|c|c||c|c|c|c|} 
\hline
 $L_b \rightarrow$& \multicolumn{4}{c||}{8} & \multicolumn{4}{c||}{8}\\
 \hline
 \backslashbox{$L_A$\kern-1em}{\kern-1em$N_c$} & 2 & 4 & 8 & 16 & 2 & 4 & 8 & 16  \\
 %$N_c \rightarrow$ & 2 & 4 & 8 & 16 & 2 & 4 & 2 \\
 \hline
 \hline
 \multicolumn{5}{|c|}{Race (FP32 Accuracy = 48.8\%)} & \multicolumn{4}{|c|}{Boolq (FP32 Accuracy = 85.23\%)} \\ 
 \hline
 \hline
 64 & 49.00 & 49.00 & 49.28 & 48.71 & 82.82 & 84.28 & 84.03 & 84.25 \\
 \hline
 32 & 49.57 & 48.52 & 48.33 & 49.28 & 83.85 & 84.46 & 84.31 & 84.93  \\
 \hline
 16 & 49.85 & 49.09 & 49.28 & 48.99 & 85.11 & 84.46 & 84.61 & 83.94  \\
 \hline
 \hline
 \multicolumn{5}{|c|}{Winogrande (FP32 Accuracy = 79.95\%)} & \multicolumn{4}{|c|}{Piqa (FP32 Accuracy = 81.56\%)} \\ 
 \hline
 \hline
 64 & 78.77 & 78.45 & 78.37 & 79.16 & 81.45 & 80.69 & 81.45 & 81.5 \\
 \hline
 32 & 78.45 & 79.01 & 78.69 & 80.66 & 81.56 & 80.58 & 81.18 & 81.34  \\
 \hline
 16 & 79.95 & 79.56 & 79.79 & 79.72 & 81.28 & 81.66 & 81.28 & 80.96  \\
 \hline
\end{tabular}
\caption{\label{tab:mmlu_abalation} Accuracy on LM evaluation harness tasks on Llama2-70B model.}
\end{table}

%\section{MSE Studies}
%\textcolor{red}{TODO}


\subsection{Number Formats and Quantization Method}
\label{subsec:numFormats_quantMethod}
\subsubsection{Integer Format}
An $n$-bit signed integer (INT) is typically represented with a 2s-complement format \citep{yao2022zeroquant,xiao2023smoothquant,dai2021vsq}, where the most significant bit denotes the sign.

\subsubsection{Floating Point Format}
An $n$-bit signed floating point (FP) number $x$ comprises of a 1-bit sign ($x_{\mathrm{sign}}$), $B_m$-bit mantissa ($x_{\mathrm{mant}}$) and $B_e$-bit exponent ($x_{\mathrm{exp}}$) such that $B_m+B_e=n-1$. The associated constant exponent bias ($E_{\mathrm{bias}}$) is computed as $(2^{{B_e}-1}-1)$. We denote this format as $E_{B_e}M_{B_m}$.  

\subsubsection{Quantization Scheme}
\label{subsec:quant_method}
A quantization scheme dictates how a given unquantized tensor is converted to its quantized representation. We consider FP formats for the purpose of illustration. Given an unquantized tensor $\bm{X}$ and an FP format $E_{B_e}M_{B_m}$, we first, we compute the quantization scale factor $s_X$ that maps the maximum absolute value of $\bm{X}$ to the maximum quantization level of the $E_{B_e}M_{B_m}$ format as follows:
\begin{align}
\label{eq:sf}
    s_X = \frac{\mathrm{max}(|\bm{X}|)}{\mathrm{max}(E_{B_e}M_{B_m})}
\end{align}
In the above equation, $|\cdot|$ denotes the absolute value function.

Next, we scale $\bm{X}$ by $s_X$ and quantize it to $\hat{\bm{X}}$ by rounding it to the nearest quantization level of $E_{B_e}M_{B_m}$ as:

\begin{align}
\label{eq:tensor_quant}
    \hat{\bm{X}} = \text{round-to-nearest}\left(\frac{\bm{X}}{s_X}, E_{B_e}M_{B_m}\right)
\end{align}

We perform dynamic max-scaled quantization \citep{wu2020integer}, where the scale factor $s$ for activations is dynamically computed during runtime.

\subsection{Vector Scaled Quantization}
\begin{wrapfigure}{r}{0.35\linewidth}
  \centering
  \includegraphics[width=\linewidth]{sections/figures/vsquant.jpg}
  \caption{\small Vectorwise decomposition for per-vector scaled quantization (VSQ \citep{dai2021vsq}).}
  \label{fig:vsquant}
\end{wrapfigure}
During VSQ \citep{dai2021vsq}, the operand tensors are decomposed into 1D vectors in a hardware friendly manner as shown in Figure \ref{fig:vsquant}. Since the decomposed tensors are used as operands in matrix multiplications during inference, it is beneficial to perform this decomposition along the reduction dimension of the multiplication. The vectorwise quantization is performed similar to tensorwise quantization described in Equations \ref{eq:sf} and \ref{eq:tensor_quant}, where a scale factor $s_v$ is required for each vector $\bm{v}$ that maps the maximum absolute value of that vector to the maximum quantization level. While smaller vector lengths can lead to larger accuracy gains, the associated memory and computational overheads due to the per-vector scale factors increases. To alleviate these overheads, VSQ \citep{dai2021vsq} proposed a second level quantization of the per-vector scale factors to unsigned integers, while MX \citep{rouhani2023shared} quantizes them to integer powers of 2 (denoted as $2^{INT}$).

\subsubsection{MX Format}
The MX format proposed in \citep{rouhani2023microscaling} introduces the concept of sub-block shifting. For every two scalar elements of $b$-bits each, there is a shared exponent bit. The value of this exponent bit is determined through an empirical analysis that targets minimizing quantization MSE. We note that the FP format $E_{1}M_{b}$ is strictly better than MX from an accuracy perspective since it allocates a dedicated exponent bit to each scalar as opposed to sharing it across two scalars. Therefore, we conservatively bound the accuracy of a $b+2$-bit signed MX format with that of a $E_{1}M_{b}$ format in our comparisons. For instance, we use E1M2 format as a proxy for MX4.

\begin{figure}
    \centering
    \includegraphics[width=1\linewidth]{sections//figures/BlockFormats.pdf}
    \caption{\small Comparing LO-BCQ to MX format.}
    \label{fig:block_formats}
\end{figure}

Figure \ref{fig:block_formats} compares our $4$-bit LO-BCQ block format to MX \citep{rouhani2023microscaling}. As shown, both LO-BCQ and MX decompose a given operand tensor into block arrays and each block array into blocks. Similar to MX, we find that per-block quantization ($L_b < L_A$) leads to better accuracy due to increased flexibility. While MX achieves this through per-block $1$-bit micro-scales, we associate a dedicated codebook to each block through a per-block codebook selector. Further, MX quantizes the per-block array scale-factor to E8M0 format without per-tensor scaling. In contrast during LO-BCQ, we find that per-tensor scaling combined with quantization of per-block array scale-factor to E4M3 format results in superior inference accuracy across models. 


% % use section* for acknowledgment
% \section*{Acknowledgment}
% The authors would like to thank...


% Can use something like this to put references on a page
% by themselves when using endfloat and the captionsoff option.



% trigger a \newpage just before the given reference
% number - used to balance the columns on the last page
% adjust value as needed - may need to be readjusted if
% the document is modified later
%\IEEEtriggeratref{8}
% The "triggered" command can be changed if desired:
%\IEEEtriggercmd{\enlargethispage{-5in}}

%%%%%%%%% References %%%%%%%%%%%
% can use a bibliography generated by BibTeX as a .bbl file
% BibTeX documentation can be easily obtained at:
% http://mirror.ctan.org/biblio/bibtex/contrib/doc/
% The IEEEtran BibTeX style support page is at:
% http://www.michaelshell.org/tex/ieeetran/bibtex/
\bibliographystyle{IEEEtran}
\bibliography{biblio}

%%% Biography %%%
%% Uncomment for the final version
\vskip -2.5\baselineskip plus -1fil
\begin{IEEEbiographynophoto}{Krithika Iyer} is a computing PhD candidate at the Kahlert School of Computing and 
the Scientific Computing and Imaging Institute at the University of Utah in the Image Analysis track. She received her B.E. in Electronics \& Telecommunication in 2015. Her research interests include medical image analysis, probabilistic modeling, deep learning, and statistical shape modeling.\end{IEEEbiographynophoto}
\vskip -2.5\baselineskip plus -1fil
\begin{IEEEbiographynophoto}{Mokshagna Sai Teja Karanam} is a computing PhD student at the Kahlert School of Computing and the Scientific Computing and Imaging Institute at the University of Utah in the Image Analysis track. He received his B.E. in Computer Science in 2020 and his Masters in 2024. His research interests include deep learning, computer vision and statistical shape modeling.\end{IEEEbiographynophoto}
\vskip -2.5\baselineskip plus -1fil
\begin{IEEEbiographynophoto}{Shireen Elhabian} is a faculty member at the Kahlert School of Computing and the Scientific Computing and Imaging Institute at the University of Utah. Her research focuses on enhancing diagnostic accuracy through deep learning, probabilistic modeling, and advanced computer vision algorithms. She has published over 100 peer-reviewed publications in prestigious journals and conferences, including IEEE-TMI, MedIA, ICLR, CVPR, ICCV, MICCAI, IPMI, AAAI. \end{IEEEbiographynophoto}
\vskip -3\baselineskip plus -1fil

% that's all folks
\end{document}