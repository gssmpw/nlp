The data-driven procedure is validated through numerical examples. We applied this method to two simulated robots.
% H-Support
The former is H-Support (Fig. \ref{fig:h_support}), a \ac{CSR} with a cylindrical cross-section, actuated by three longitudinal and four helicoidal actuators. This robot is a modified version of the existing I-Support \cite{manti2016soft}.
Table \ref{tab:sim_parameters} lists the geometrical and physical parameters.
% Conical H-Support
With the same actuation and physical characteristics, the latter robot is the conical variant of the H-Support (Fig. \ref{fig:conical_hsupport_sketch}).
% % H-Support
% \begin{figure}
%     \centering
%     \includesvg[width=1.0\linewidth]{imgs/h_support_sketch.svg}
%     \caption{Sketch of the H-Support robot, a cylindrical \ac{CSR} with 3 longitudinal and 4 helicoidal actuators. The geometrical and physical parameters are listed in Tab. \ref{tab:sim_parameters}.}
%     \label{fig:h_support}
% \end{figure}
% H-Support
\begin{figure}
    \centering
    \includegraphics[width=1.0\linewidth]{imgs/h_support_sketch.pdf}
    \caption{Sketch of the H-Support robot, a cylindrical \ac{CSR} with 3 longitudinal and 4 helicoidal actuators. The geometrical and physical parameters are listed in Tab. \ref{tab:sim_parameters}.}
    \label{fig:h_support}
\end{figure}
% % Conical Robot
% \begin{figure}
%     \centering
%     \includesvg[width=1.0\linewidth]{imgs/conical_hsupport_sketch.svg}
%     \caption{A sketch of the Conical H-Support. The conical shape is described by a $s$-varying cross-section's radius of $R_{\textnormal{cs}}(s) = \bar{R}_{\textnormal{cs}} \left(1 -0.9 s\right)$. The geometrical and physical parameters are listed in Tab. \ref{tab:sim_parameters}.}
%     \label{fig:conical_hsupport_sketch}
% \end{figure}
% Conical Robot
\begin{figure}
    \centering
    \includegraphics[width=1.0\linewidth]{imgs/conical_hsupport_sketch.pdf}
    \caption{A sketch of the Conical H-Support. The conical shape is described by a $s$-varying cross-section's radius of $R_{\textnormal{cs}}(s) = \bar{R}_{\textnormal{cs}} \left(1 -0.9 s\right)$. The geometrical and physical parameters are listed in Tab. \ref{tab:sim_parameters}.}
    \label{fig:conical_hsupport_sketch}
\end{figure}
% Parameters in Simulation
\begin{table}
\centering
\caption{Geometrical and physical parameters of the Simulated H-Support}
\label{tab:sim_parameters}
    \begin{tabular}{lll}
    \toprule
    Name                    &   Symbol          &   Value                                                               \\
    \midrule
    Length                  &   $L$               &   \SI{1.0}{\meter}                                                            \\
    Cross-Section Radius               &   $\bar{R}_{\textnormal{cs}}$               &   \SI{0.1}{\meter}                                                            \\
    Density                 &   $\bar{\rho}$    &   \SI{1000.0}{\kilogram/\meter^3}                                     \\     
    Young's Modulus         &   $E$             &   \SI{1.0}{M\pascal}                                             \\
    Poisson Ratio           &   $\nu$           &   0.5                                                                     \\
    Damping Coefficient                &   $\beta$         &   \SI{0.01}{M\pascal \cdot \second}         \\
    Stress-Free Strain      & $\bm{\xi}^{*}$    &   $[0, \, 0, \, 0, \, 1, \, 0, \, 0]^{\top}$    \\
    \bottomrule
    \end{tabular}
\end{table}
The simulations were developed using the \ac{SoRoSim} \cite{mathew2022sorosim} framework.
The robots are simulated with three different functional bases: (i) polynomial \eqref{eq:polynomial_basis}, (ii) trigonometrical \eqref{eq:fourier_basis}, and (iii) Gaussian \eqref{eq:gaussian_basis}. In all cases, we employed a second-order truncation.

For each basis, we followed the procedure outlined in Sec. \ref{sec:spectrum_extraction}.
The strain field was sampled with a sampling wavenumber of $\nu_s = \SI{100}{\meter^{-1}}$, corresponding a sampling wavelength of $\lambda_s = \SI{0.01}{\meter}$. The \acp{CSR} were tested using the standard signals for a simulation time of $t_{\textnormal{f}} = \SI{10}{\second}$. Finally, \eqref{eq:crt_dynamics} was solved with the Runge-Kutta 4th order algorithm, with a sampling frequency of $f_s = \SI{5}{k \hertz}$, obtaining the samples $\bm{\xi}(n \lambda_s, \, m T_s)$.

\subsection{H-Support} \label{numerical_validation:hsupport}
    In Fig. \ref{fig:stft_time} is shown the \acp{STFT} of the three simulations, organized in functional basis (row) and strain modes (columns). The results are discussed in the following subsections.
    % %
    % \begin{figure*}
    %     \centering
    %     \includesvg[width=1.0\linewidth]{imgs/simulations/stft_time.svg}
    %     \caption{The space-time spectra of the H-Support numerical example discussed in Sec. \ref{numerical_validation:hsupport}. The \acp{STFT} are organized in functional basis (rows) and strain modes (columns). The \acp{STFT} show the time-frequency response varying the components of the spatial spectrum. The values are normalized to $|\Xi_i(j0, j0)|$.}
    %     \label{fig:stft_time}
    % \end{figure*}
    %
    \begin{figure*}
        \centering
        \includegraphics[width=1.0\linewidth]{imgs/simulations/stft_time.pdf}
        \caption{The space-time spectra of the H-Support numerical example discussed in Sec. \ref{numerical_validation:hsupport}. The \acp{STFT} are organized in functional basis (rows) and strain modes (columns). The \acp{STFT} show the time-frequency response varying the components of the spatial spectrum. The values are normalized to $|\Xi_i(j0, j0)|$.}
        \label{fig:stft_time}
    \end{figure*}
    %
    % Discussion
    \subsubsection{Spatial Spectrum Components}
    % % Strain Analysis
    % \begin{figure}
    %     \centering
    %     \includesvg[width=1.0\linewidth]{imgs/simulations/strain_analysis.svg}
    %     \caption{Strain Analysis of the H-Support robot with only an active helicoidal actuator with a magnitude of $\SI{1}{\newton}$. For bending and shear, the numerical analysis reveals a trigonometric pattern; for twisting and shear, a constant pattern.}
    %     \label{fig:strain_analysis}
    % \end{figure}
    % Strain Analysis
    \begin{figure}
        \centering
        \includegraphics[width=1.0\linewidth]{imgs/simulations/strain_analysis.pdf}
        \caption{Strain Analysis of the H-Support robot with only an active helicoidal actuator with a magnitude of $\SI{1}{\newton}$. For bending and shear, the numerical analysis reveals a trigonometric pattern; for twisting and shear, a constant pattern.}
        \label{fig:strain_analysis}
    \end{figure}
    %
    From Fig. \ref{fig:stft_time}, we can infer the spatial spectrum composition for each strain mode.

    Regarding the twisting mode ($\kappa_x$), the most prominent component is the constant term across each functional basis.
    In contrast, the bending and shear modes exhibit distinct spectra. Counterintuitively, the harmonic at $\nu = 0 , \textnormal{m}^{-1}$ is significantly less prominent than the subsequent harmonics. This behavior can be attributed to the helicoidal actuators, which excite the bending and shear modes with a sinusoidal profile along the rod.
    
    To validate this observation, we performed strain analysis using the \ac{ISP} method \cite{renda2020geometric, renda2024dynamics}. Specifically, the strain field can be numerically computed by solving
    \begin{equation} \label{eq:implicit_strain} 
        \bm{\xi}(s) = \bm{\Sigma}^{-1}(s) , \bm{B}_{\bm{\tau}}\left(\bm{\xi}, s\right) \bm{\tau} + \bm{\xi}^{*}(s) \, . 
    \end{equation}
    
    The solution of \eqref{eq:implicit_strain} corresponds to the strain modes excited by the actuators in the static regime \cite{renda2020geometric}.
    Fig. \ref{fig:strain_analysis} shows the results, highlighting a constant stretch and twist. Moreover, the analysis reveals that the bending and shear strains exhibit a sinusoidal profile. Notably, the bending and shear strains concerning the same axis (i.e., $\kappa_y, \sigma_y$ and $\kappa_z, \sigma_z$) are in phase opposition. This indicates that the two strain modes exhibit destructive interference in space. Furthermore, this finding reveals a coupling between the bending and shear modes in static conditions, which arises from the specific geometry of the actuator.

    \subsubsection{Resonance and Anti-resonance Peaks}
    Regardless of the chosen functional basis, the system displays a sequence of resonance and anti-resonance peaks in time, which is characteristic of systems with a high number of \ac{DoFs} \cite{Ewins1999}. The frequency range of this sequence slightly varies with the strain mode.

    % Torsion
    Concerning the torsion $\kappa_x$, the sequence starts with an anti-resonance peak at $\SI{0.6}{\hertz}$ and ends at $\SI{5}{\hertz}$ with the lowest anti-resonance peak. The highest resonance peak can be found at $\SI{1.75}{\hertz}$ and is invariant to the choice of the functional basis.

    In Fig. \ref{fig:torsion_bode}, the Bode diagram of the polynomial $\kappa_x$ is presented. The phase diagram shows the distinct behavior of the constant component ($\nu = \SI{0}{\meter^{-1}}$) compared to the other harmonics.

    At the first anti-resonance peak, the phase of the constant component decreases smoothly, while the other harmonics exhibit a rapid phase shift. Near the highest resonance peak, the phase of all harmonics increases rapidly. Subsequently, the phase decreases again at the last anti-resonance peak.
    % % Torsion Bode
    % \begin{figure}
    %     \centering
    %     \includesvg[width=1.0\linewidth]{imgs/simulations/torsion_bode.svg}
    %     \caption{Bode diagram of the twisting mode $\kappa_x$. The magnitude is normalized w.r.t. $|\kappa_x(j0, \, j 0)|$.}
    %     \label{fig:torsion_bode}
    % \end{figure}
    % Torsion Bode
    \begin{figure}
        \centering
        \includegraphics[width=1.0\linewidth]{imgs/simulations/torsion_bode.pdf}
        \caption{Bode diagram of the twisting mode $\kappa_x$. The magnitude is normalized w.r.t. $|\kappa_x(j0, \, j 0)|$.}
        \label{fig:torsion_bode}
    \end{figure}
    
    % Curvature and Shear
    As previewed in the strain analysis (Fig. \ref{fig:strain_analysis}), the bending and shear modes are coupled, exhibiting similar sequences of resonance and anti-resonance peaks.
    Since resonance and anti-resonance phenomena are associated with constructive and destructive interference, it is valuable to examine the phase.
    Fig. \ref{fig:hsupport_bode} presents the Bode diagram for the couples $\kappa_y, \sigma_y$ and $\kappa_z, \sigma_y$, where the phase changes rapidly at the resonance and anti-resonance peaks.
    
    For the first pair, the phase plot shows that the two phases decrease continuously with an offset of $\pi$. Conversely, for the second pair, the phases vary in sync. This behavior reflects the interference patterns observed in the static case during the strain analysis.
    The shear $\sigma_y(\nu = \SI{10}{\meter^{-1}})$ exhibits an intense anti-resonance peak at $\SI{4.8}{\hertz}$ which results in a phase increase of $\pi$.
    As a consequence, the phase difference in both of the pairs converges toward destructive interference.

    % Peaks along the basis
    Another observation from Fig. \ref{fig:stft_time} is that the peak frequencies exhibit slight variations when using the Gaussian basis. While the first spatial harmonics of the Gaussian basis behave similarly to those of the other bases, the higher harmonics display peaks at neighboring frequencies.
    For instance, in the torsion mode at $\nu = \SI{50}{\meter^{-1}}$, the first anti-resonance peak occurs at $\SI{0.4}{\hertz}$ instead of $\SI{0.6}{\hertz}$. Similarly, for $\kappa_z$, the spatial harmonic $\nu = \SI{5}{\meter^{-1}}$ exhibits an anti-resonance peak at $\SI{12.5}{\hertz}$, which is the highest peak in frequency across all strain modes and bases.
    % % Bending & Shear
    % \begin{figure}
    %     \centering
    %     \includesvg[width=1.0\linewidth]{imgs/simulations/hsupport_bode.svg}
    %     \caption{Comparison between the Bode diagrams of $\kappa_y$, $\sigma_y$, and $\kappa_z$. In the last row, the difference in phase between the modes is reported.}
    %     \label{fig:hsupport_bode}
    % \end{figure}
    % Bending & Shear
    \begin{figure}
        \centering
        \includegraphics[width=1.0\linewidth]{imgs/simulations/hsupport_bode.pdf}
        \caption{Comparison between the Bode diagrams of $\kappa_y$, $\sigma_y$, and $\kappa_z$. In the last row, the difference in phase between the modes is reported.}
        \label{fig:hsupport_bode}
    \end{figure}
    
    \subsubsection{Frequency-varying Behavior}
    As discussed in Sec. \ref{sec:space_time_ft}, the \ac{STFT} illustrates a well-known characteristic of \acp{CSR}: their deformations vary depending on the input's time-frequency. To show that, we reported in Fig. \ref{fig:stft_space} the spatial spectrum varying the time-frequency. For each time-frequency, the spatial spectrum's pattern corresponds to the theoretical one, imposed by the \ac{SoRoSim} framework. Consequently, the shape of the spatial spectrum remains invariant w.r.t. time-frequency, with variations occurring only in magnitude.
    % % STFT Space
    % \begin{figure}
    %     \centering
    %     \includesvg[width=1.0\linewidth]{imgs/simulations/stft_space.svg}
    %     \caption{Spatial Spectrum varying the time-frequencies.    
    %     The spatial harmonics $\nu > \SI{2}{\meter^{-1}}$ are not taken into account in the case of trigonometric basis since the analysis is limited to the second order.}
    %     \label{fig:stft_space}
    % \end{figure}
    % STFT Space
    \begin{figure}
        \centering
        \includegraphics[width=1.0\linewidth]{imgs/simulations/stft_space.pdf}
        \caption{Spatial Spectrum varying the time-frequencies.    
        The spatial harmonics $\nu > \SI{2}{\meter^{-1}}$ are not taken into account in the case of trigonometric basis since the analysis is limited to the second order.}
        \label{fig:stft_space}
    \end{figure}

    \subsection{Conical H-Support} \label{numerical_validation:conical_hsupport}
    % Intro & Description of the prototype
    The numerical validation is also performed on a conical H-Support variant (Fig. \ref{fig:conical_hsupport_sketch}), where the physical and geometrical parameters are kept the same and listed in Table \ref{tab:sim_parameters}.
    The conical profile of the \ac{CSR} is realized by the linear function of the cross-section's radius, such as $R_{\textnormal{cs}}(s) = \bar{R}_{\textnormal{cs}} \left(1 -0.9 s\right)$.
    As a consequence, the cross-section's area and second moment of area will be $A(s) \propto s^2$, and $J_i(s) \propto s^4$, respectively.
    Therefore, in contrast to the cylindrical case, the dynamic matrices (i.e., inertia, stiffness, and damping) are $s$-dependent, playing an active role on the \ac{STFT} spectra.   
    % STFT of Conical H-Support 
    Fig. \ref{fig:conical_hsupport_stft} shows the bode diagram for the polynomial basis case.
    % Torsion
    Differently from the cylindrical case, the torsion constant component ($\nu = \SI{0}{\meter^{-1}}$) shows a spectrum similar to the other spatial harmonics.
    Unlike the H-Support case, the higher harmonics show no anti-resonance peak at $\SI{0.6}{\hertz}$ and the difference in magnitude between $\kappa_x(\nu = 0)$ and the other components varies more smoothly. The polynomial dependency of the actuation matrix and stiffness, which affects the excited basis of the torsion mode, justifies this observation.
    % Main resonance peak
    Notably, the highest resonance peak remains the same frequency ($\SI{1.75}{\hertz}$) in the conical case across all the strain modes.

    Regarding the bending and shear modes, the bode diagrams exhibit a coupling between $\kappa_y, \sigma_z$ and $\kappa_z, \sigma_y$. The former pair exhibits no evident anti-resonance peaks since the magnitude and the phase vary smoothly. In contrast, the latter pair shows a pronounced anti-resonance peak with a rapid variation in phase. Differently from the cylindrical case, all the harmonics (except the $\nu = \SI{0}{\meter^{-1}}$), display this peak at $\SI{4.8}{\hertz}$.
   % % STFT Conical
   %  \begin{figure*}
   %      \centering
   %      \includesvg[width=1.0\linewidth]{imgs/simulations/conical_hsupport_stft.svg}
   %      \caption{The space-time spectra of the Conical H-Support numerical example discussed in Sec. \ref{numerical_validation:conical_hsupport}. The \acp{STFT} show the time-frequency response varying the components of the spatial spectrum. The magnitude values are normalized to $|\Xi_i(j0, j0)|$.}
   %      \label{fig:conical_hsupport_stft}
   %  \end{figure*}
  % STFT Conical
    \begin{figure*}
        \centering
        \includegraphics[width=1.0\linewidth]{imgs/simulations/conical_hsupport_stft.pdf}
        \caption{The space-time spectra of the Conical H-Support numerical example discussed in Sec. \ref{numerical_validation:conical_hsupport}. The \acp{STFT} show the time-frequency response varying the components of the spatial spectrum. The magnitude values are normalized to $|\Xi_i(j0, j0)|$.}
        \label{fig:conical_hsupport_stft}
    \end{figure*}