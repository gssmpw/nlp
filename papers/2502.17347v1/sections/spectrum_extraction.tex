% % Procedure (with STFT)
% \begin{figure*}
%     \centering
%     % \includegraphics[width=1.0\linewidth]{imgs/sofft_procedure.png}
%     \includesvg[width=1.0\linewidth]{imgs/sofft_procedure2.svg}
%     \caption{Illustration of the proposed data-driven methodology. The robot is subjected to the standard signals and the samples of the strain field are measured by the sensors. Through \ac{FFT}, the space-time spectrum can be computed.}
%     \label{fig:procedure}
% \end{figure*}
% Procedure (with STFT)
\begin{figure*}
    \centering
    \includegraphics[width=1.0\linewidth]{imgs/sofft_procedure2.pdf}
    \caption{Illustration of the proposed data-driven methodology. The robot is subjected to the standard signals and the samples of the strain field are measured by the sensors. Through \ac{FFT}, the space-time spectrum can be computed.}
    \label{fig:procedure}
\end{figure*}
%
As outlined in Sec. \ref{sec:spatial_ft} and \ref{sec:space_time_ft}, meaningful insights into the strain field can be derived from both the spatial \eqref{eq:continous_spatial_fourier_transform} and space-time \eqref{eq:space_time_ft} spectra.
This section presents a data-driven approach for extracting the \ac{SFT} and \ac{STFT}, directly from the real-world robots. 
This method only requires knowledge of the robot's length and is independent of its physical parameters.
The proposed procedure consists of the steps exposed below and summarized in Fig. \ref{fig:procedure}.
\begin{enumerate}
    \item \textit{Sensorization}: The first step is to sensorize the \ac{CSR}, in order to measure the samples $\bm{\xi}(n \lambda_s)$. The choice of $\lambda_s$ is crucial as it determines the maximum spatial frequency of the sampled spectrum, i.e., $\nu_s / 2$.
    
    \item \textit{Motor Babbling}: Standard signals are applied to the actuators $\bm{\tau}(t)$. Common signals in identification literature include step, chirp, and white noise, thanks to their properties in frequency.
    As a result, samples $\bm{\xi}(n \lambda_s, \, m T_s)$ are obtained, where $T_s$ represents the sampling period of the sensing framework.

    \item \textit{Compute Spectra}: From the samples $\bm{\xi}(n \lambda_s, \, m T_s)$, we compute a time-series of the \ac{SFT} and \ac{STFT} using the \acf{FFT} algorithm. Before this, preprocessing techniques like zero-padding can be applied to improve the resolution of the \ac{FFT}.

    \item \textit{Spectrum Analysis}: From $\bm{\Xi}(jk, \, j \omega)$ we can identify $k_{\textnormal{max}}$ and the shape of the spatial spectrum, which aids in selecting the optimal basis functions. 
    Furthermore, we can derive key dynamic characteristics of the system, such as resonance or anti-resonance phenomena.
    
    \item \textit{Modeling}: 
    Based on the information gathered, the user can determine the minimum number of sections into which the robot can be divided, thereby reducing the need for extensive sensorization ($k_{\textnormal{max}} < \frac{k_s}{2}$). 
    Additionally, optimal basis functions can be extracted using well-established signal processing algorithms such as \ac{MP} \cite{mallat1993matching} or \ac{BPD} \cite{chen2001atomic}.
\end{enumerate}

Viewing strain as a signal offers new insights into the frequency domain of motor babbling \cite{george2020first}. Since standard signals (e.g., white noise) can sample all frequencies in the spectrum, motor babbling can be interpreted as a method for exploring the robot's space-time spectrum. This aligns with the classic definition of motor babbling, where random actuations are applied to the robot to sample its workspace \cite{george2020first}.

% BPDN
% BPDN for Strain Fitting
\subsection{\ac{BPD} for Strain Fitting} \label{spectrum_extraction:bpd}
The Basis Pursuit Denoising (BPD) problem enables the identification of an optimal basis that preserves the signal's sparsity while mitigating the influence of noise. 
This method is particularly well-suited for applications in soft robotics, where a trade-off between accuracy and reduction in \ac{DoFs} is essential. Specifically, \ac{BPD} selects the best combination of basis functions from a predefined signal dictionary to achieve an optimal reconstruction of the input data.

To adapt the \ac{BPD} problem to the \ac{GVS} approach, the optimization problem can be formulated as
\begin{equation} \label{eq:bpdn_opt_problem}
    \bm{q} = \underset{\bm{q}}{\arg \min} \left\{\frac{1}{2}\norm{\bm{\xi} - \bm{B}_{\bm{q}} \, \bm{q}}^{2}_{2} + \norm{\bm{\gamma} \odot \bm{q}}_1\right\} \, ,
\end{equation}
where $\odot$ is element-wise multiplication, $\norm{\cdot}_i$ is the $\mathcal{\ell}^i$-norm, the basis function matrix $\bm{B}_{\bm{q}}$ represents the signal dictionary and $\bm{\xi}$ the noisy data. Finally, the parameter $\bm{\gamma} \in \mathbb{R}^{n_q}$ controls the trade-off between sparsity and reconstruction accuracy, allowing the user to prioritize either a more compact representation or a more precise fit.

Moreover, by exploiting the orthogonality of trigonometric functions, the user can extract Fourier coefficients using the \ac{SALSA} algorithm \cite{afonso2010fast} and employ a trigonometric basis to denoise the noisy strain field samples.

To evaluate the relevance of each basis, it is possible to adapt the truncation index \eqref{eq:discrete_truncation_criterion} in the continuous case.
The energy associated with the $i$-th basis $b_i(s)$ can be computed as $E_i(q_i) = q_i^2 \, \frac{1}{2 \pi}\int_{-\infty}^{+\infty} |b_i(jk)|^{2} \, \textnormal{d}k$, where $b_i(jk) = \fourier{b_i(s)}$.
Therefore, the energy fraction relative to the total reconstructed strain field energy is
\begin{equation}
    E_{\textnormal{tr}, b_i}(q_i) = \frac{E_i(q_i)}{\sum\limits_{j = 0}^{n - 1} E_j(q_j)} \, .
\end{equation}

It is worth highlighting that, thanks to the continuous \ac{SFT}, the energy evaluation can be performed for all the wavenumbers. However, if the integral is difficult to compute, the user can still use the discrete Parseval identity with an arbitrary range and resolution frequency.
Finally, the user can consider truncating one or more bases if their energy contribution is less than a specified threshold.

% \subsection{Error Propagation in $SE(3)$} \label{spectrum_extraction:error_propagation}
% After the \ac{BPD} and eventually a truncation, there will be a fitting error $\bm{\epsilon}(s) \in \mathbb{R}^{6}$, such that $\bm{\xi} = \bm{B}_{\bm{q}}(s) \, \bm{q} + \bm{\xi}^{*} + \bm{\epsilon}(s)$. This error propagates in the Forward Kinematics of the \ac{CSR}, affecting especially the tip's pose.
% Let be $\check{\bm{\xi}} = \bm{B}_{\bm{q}}(s) \, \bm{q} + \bm{\xi}^{*} \in \mathbb{R}^{6}$ the reconstructed strain. Recalling \eqref{eq:strain_field}, the backbone space-evolution can be described by
% \begin{equation} \label{eq:diff_err_backbone}
%     \bm{g}'(s) = \bm{g}(s) \left( \check{\bm{\xi}}(s) + \bm{\epsilon}(s)\right)^{\wedge} \, .
% \end{equation}
% Using Magnus expansion and Zanna collocation method as in the \ac{GVS} method, the \eqref{eq:diff_err_backbone} can be solved as
% \begin{equation} \label{eq:error_fk}
%     \bm{g}((n+1) \lambda_s) = \bm{g}(n \lambda_s) \exp_{SE(3)} \left( \hat{\bm{\Omega}}(\lambda_s) + \hat{\bm{\Omega}}_{\bm{\epsilon}}(\lambda_s) \right) \, ,
% \end{equation}
% where $\bm{\Omega} \in \mathbb{R}^{6}$  and $\bm{\Omega}_{\bm{\epsilon}} \in \mathbb{R}^{6}$ are the Magnus expansion of the reconstructed strain and the fitting error, respectively. 

% % Identification
% %%% Identification %%%
\subsection{Parametric Identification \textcolor{red}{Doesn't Work :(}}
To identify the physical parameters of the robot, we can expand the regressor-driven method proposed in \cite{stella2022experimental}, in the case of \ac{GVS} model. From the experimental data $\bm{\xi}(n \lambda_s, m T_s)$ and fixed a functional basis, we can fit the data, obtaining the experimental configurations $\bm{q}_{\textnormal{exp}}$. Assuming known the density $\rho$, and $E$, $\beta$ constant along the rod's length, we can reorganize the dynamics such as
\begin{equation} \label{eq:regressor}
    \bm{\delta}_m\left(\bm{q}, \dot{\bm{q}}, \ddot{\bm{q}}, \bm{\tau}\right) = \bm{Y}_m\left(\bm{q}, \dot{\bm{q}}\right) \bm{\pi} \quad \forall \, m \in [0, \, M - 1] \, ,
\end{equation}
where $\bm{Y}_m\left(\bm{q}, \dot{\bm{q}}\right) = - \begin{bmatrix}\bar{\bm{K}}\bm{q} & \bar{\bm{D}}\dot{\bm{q}}\end{bmatrix}$, $\bar{\bm{K}} , \, \bar{\bm{D}}$ are the stiffness and damping matrix with unitary $E$ and $\beta$, $\bm{\delta}_m\left(\bm{q}, \dot{\bm{q}}, \ddot{\bm{q}}, \bm{\tau}\right) = \bm{M}\left(\bm{q}\right) \ddot{\bm{q}} + \bm{C}\left(\bm{q}, \dot{\bm{q}}\right) \dot{\bm{q}} + \bm{G}\left(\bm{q}\right) - \bm{B}\left(\bm{q}\right) \bm{\tau}$, the subscript $\left(\cdot\right)_{m}$ indicates the time sample $m T_s$, and $\bm{\pi} = \begin{bmatrix}
    E & \beta
\end{bmatrix}^{\top}$.

The best set of parameters can be found by solving a \ac{WLS} problem, such as
\begin{equation} \label{eq:wls_functional}
    \hat{\bm{\pi}} = \underset{\bm{\pi}}{\arg \, \min} \left\{\frac{1}{2}\left(\bm{\delta} - \bm{Y}\bm{\pi}\right)^{\top} \bm{W} \left(\bm{\delta} - \bm{Y}\bm{\pi}\right)\right\} \, ,
\end{equation}
where $\bm{Y} = \begin{bmatrix} \bm{Y}^{\top}_0 & \bm{Y}^{\top}_{1} & \dots & \bm{Y}^{\top}_{M - 1} \end{bmatrix}^{\top}$, $\bm{\delta} = \begin{bmatrix} \bm{\delta}^{\top}_0 & \bm{\delta}^{\top}_{1} & \dots & \bm{\delta}^{\top}_{M - 1} \end{bmatrix}^{\top}$, and $\bm{W} \in \mathbb{R}^{\left(n \cdot M\right) \times \left(n \cdot M\right)}$. The weight matrix $\bm{W}$ allows us to give more importance to different strain modes or different time samples.

The solution of \eqref{eq:wls_functional} can be expressed as
\begin{equation} \label{eq:wsl_identification}
    \hat{\bm{\pi}} = \bm{Y}^{\dagger}_{\bm{W}} \, \bm{\delta} \, ,
\end{equation}
where $\bm{Y}^{\dagger}_{\bm{W}} = \left(\bm{Y}^{\top} \bm{W} \bm{Y}\right)^{-1} \bm{Y}^{\top} \bm{W}$ is the weighted Moore-Penrose left pseudoinverse.