% BPDN for Strain Fitting
\subsection{\ac{BPD} for Strain Fitting} \label{spectrum_extraction:bpd}
The Basis Pursuit Denoising (BPD) problem enables the identification of an optimal basis that preserves the signal's sparsity while mitigating the influence of noise. 
This method is particularly well-suited for applications in soft robotics, where a trade-off between accuracy and reduction in \ac{DoFs} is essential. Specifically, \ac{BPD} selects the best combination of basis functions from a predefined signal dictionary to achieve an optimal reconstruction of the input data.

To adapt the \ac{BPD} problem to the \ac{GVS} approach, the optimization problem can be formulated as
\begin{equation} \label{eq:bpdn_opt_problem}
    \bm{q} = \underset{\bm{q}}{\arg \min} \left\{\frac{1}{2}\norm{\bm{\xi} - \bm{B}_{\bm{q}} \, \bm{q}}^{2}_{2} + \norm{\bm{\gamma} \odot \bm{q}}_1\right\} \, ,
\end{equation}
where $\odot$ is element-wise multiplication, $\norm{\cdot}_i$ is the $\mathcal{\ell}^i$-norm, the basis function matrix $\bm{B}_{\bm{q}}$ represents the signal dictionary and $\bm{\xi}$ the noisy data. Finally, the parameter $\bm{\gamma} \in \mathbb{R}^{n_q}$ controls the trade-off between sparsity and reconstruction accuracy, allowing the user to prioritize either a more compact representation or a more precise fit.

Moreover, by exploiting the orthogonality of trigonometric functions, the user can extract Fourier coefficients using the \ac{SALSA} algorithm \cite{afonso2010fast} and employ a trigonometric basis to denoise the noisy strain field samples.

To evaluate the relevance of each basis, it is possible to adapt the truncation index \eqref{eq:discrete_truncation_criterion} in the continuous case.
The energy associated with the $i$-th basis $b_i(s)$ can be computed as $E_i(q_i) = q_i^2 \, \frac{1}{2 \pi}\int_{-\infty}^{+\infty} |b_i(jk)|^{2} \, \textnormal{d}k$, where $b_i(jk) = \fourier{b_i(s)}$.
Therefore, the energy fraction relative to the total reconstructed strain field energy is
\begin{equation}
    E_{\textnormal{tr}, b_i}(q_i) = \frac{E_i(q_i)}{\sum\limits_{j = 0}^{n - 1} E_j(q_j)} \, .
\end{equation}

It is worth highlighting that, thanks to the continuous \ac{SFT}, the energy evaluation can be performed for all the wavenumbers. However, if the integral is difficult to compute, the user can still use the discrete Parseval identity with an arbitrary range and resolution frequency.
Finally, the user can consider truncating one or more bases if their energy contribution is less than a specified threshold.

% \subsection{Error Propagation in $SE(3)$} \label{spectrum_extraction:error_propagation}
% After the \ac{BPD} and eventually a truncation, there will be a fitting error $\bm{\epsilon}(s) \in \mathbb{R}^{6}$, such that $\bm{\xi} = \bm{B}_{\bm{q}}(s) \, \bm{q} + \bm{\xi}^{*} + \bm{\epsilon}(s)$. This error propagates in the Forward Kinematics of the \ac{CSR}, affecting especially the tip's pose.
% Let be $\check{\bm{\xi}} = \bm{B}_{\bm{q}}(s) \, \bm{q} + \bm{\xi}^{*} \in \mathbb{R}^{6}$ the reconstructed strain. Recalling \eqref{eq:strain_field}, the backbone space-evolution can be described by
% \begin{equation} \label{eq:diff_err_backbone}
%     \bm{g}'(s) = \bm{g}(s) \left( \check{\bm{\xi}}(s) + \bm{\epsilon}(s)\right)^{\wedge} \, .
% \end{equation}
% Using Magnus expansion and Zanna collocation method as in the \ac{GVS} method, the \eqref{eq:diff_err_backbone} can be solved as
% \begin{equation} \label{eq:error_fk}
%     \bm{g}((n+1) \lambda_s) = \bm{g}(n \lambda_s) \exp_{SE(3)} \left( \hat{\bm{\Omega}}(\lambda_s) + \hat{\bm{\Omega}}_{\bm{\epsilon}}(\lambda_s) \right) \, ,
% \end{equation}
% where $\bm{\Omega} \in \mathbb{R}^{6}$  and $\bm{\Omega}_{\bm{\epsilon}} \in \mathbb{R}^{6}$ are the Magnus expansion of the reconstructed strain and the fitting error, respectively. 