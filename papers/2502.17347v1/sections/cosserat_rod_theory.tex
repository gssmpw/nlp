The \ac{CRT} associates to the elastic rod of length $L$ an oriented backbone, parameterized by the material curvilinear abscissa $s \in [0, \, L]$. The oriented curve is described by its pose $\bm{g}(s, t) \in SE(3)$ for each cross-section $s$, defined as
\begin{equation} \label{eq:se3_definition}
    \bm{g}(s, t) = \begin{bmatrix}
                \bm{R}(s, t) & \bm{r}(s, t) \\
                \bm{0}^{\top} & 1 
        \end{bmatrix} \, ,
\end{equation}
where $\bm{R}(s, t) \in SO(3)$ is the rotation matrix and $\bm{r}(s, t) \in \mathbb{R}^{3}$ is the position.
% Strain
The space-evolution of $\bm{g}$ can be uniquely described by the strain field $\bm{\xi}(s, t) = \begin{bmatrix} \bm{\kappa}^{\top} & \bm{\sigma}^{\top} \end{bmatrix}^{\top} \in \mathbb{R}^6$, which is defined as
\begin{equation} \label{eq:strain_field}
    \bm{\xi}(s, t) = \left(\bm{g}^{-1}(s, t) \, \bm{g}'(s, t)\right)^{\vee} \, ,
\end{equation}
where ${\left(\cdot\right)}^{\vee}$ denotes the vee operator \cite[Ch.\ 3]{murray}, $\bm{\kappa} = \begin{bmatrix}
    \kappa_x & \kappa_y & \kappa_z
\end{bmatrix}^{\top} \in \mathbb{R}^{3}$ contains the angular strain modes, such as twisting ($\kappa_x$), and bending ($\kappa_y , \, \kappa_z$), and $\bm{\sigma} = \begin{bmatrix}
    \sigma_x & \sigma_y & \sigma_z
\end{bmatrix}^{\top} \in \mathbb{R}^{3}$ represents the linear strain modes, such as stretching ($\sigma_x$), and shear ($\sigma_y, \, \sigma_z$).

% eta
The time evolution of $\bm{g}(s, t)$ can be described with the velocity field $\bm{\eta}(s, t) \in \mathbb{R}^6$, defined as
\begin{equation} \label{eq:velocity_field}
    \bm{\eta}(s, t) = \left(\bm{g}^{-1}(s, t) \, \dot{\bm{g}}(s, t)\right)^{\vee} \, .
\end{equation}
Moreover, the mixed derivative equality allows us to write a relationship between the two fields, i.e.,
\begin{equation} \label{eq:mixed_derivative}
    \bm{\eta}'(s, t) = \dot{\bm{\xi}}(s, t) - \textnormal{ad}_{\bm{\xi}} \, \bm{\eta}(s, t) \, . 
\end{equation}

To model the statics and dynamics of a Cosserat rod, both internal and external wrenches must be defined. The internal wrench consists of the passive internal wrench $\bm{\mathcal{F}}_i(s, t) \in \mathbb{R}^6$, which accounts for the mechanical impedance of the rod (i.e., its elasticity and damping), and the active internal wrench $\bm{\mathcal{F}}_a(s, t) \in \mathbb{R}^6$, which represents the distributed actuation applied to the \acp{CSR}.
\begin{equation} \label{eq:internal_passive}
    \bm{\mathcal{F}}_i (s, t) = \bm{\Sigma}(s) \left(\bm{\xi} - \bm{\xi}^*\right) + \bm{\Psi}(s) \, \dot{\bm{\xi}} \, ,
\end{equation}
\begin{equation} \label{eq:internal_active}
    \bm{\mathcal{F}}_a(s, t) = \bm{B}_{\bm{\tau}}(\bm{\xi}, s) \, \bm{\tau}(t) \, .
\end{equation}
%
In \eqref{eq:internal_passive}, $\bm{\Sigma}(s) = \textnormal{diag} \left(G J_x, E J_y, E J_z, E A, G A, G A \right) \in \mathbb{R}^{6 \times 6}$ is the stiffness matrix, $\bm{\Psi} = \beta  \, \textnormal{diag} \left(J_x, 3 J_y, 3 J_z, 3 A, A, A \right) \in \mathbb{R}^{6 \times 6}$ the damping matrix, where $A(s)$ and $J_i(s)$ are the cross-section's area and the second moment of area, respectively. 
Furthermore, $E(s)$ is the Young modulus, $G(s)$ is the shear modulus, $\beta(s)$ is the damping coefficient, and $\bm{\xi}^{*}(s) \in \mathbb{R}^{6}$ is the stress-free strain field.

Concerning \eqref{eq:internal_active}, $\bm{B}_{\bm{\tau}} \in \mathbb{R}^{6 \times n_a}$ is the actuation matrix \cite{renda2017screw}, $\bm{\tau} \in \mathbb{R}^{n_a}$ is the actuators' magnitude vector, and $n_a$ is the number of the actuators.
The external wrench $\bm{\mathcal{F}}_e(s, t) \in \mathbb{R}^6$ collects the distributed external forces applied to the robot, such as gravity or contact forces; the contribution of gravity is detailed in the following
\begin{equation} \label{eq:ext_gravity}
    \bm{\mathcal{F}}_{e, \, g}(s, t) = \bm{\mathcal{M}}(s) \, \textnormal{Ad}^{-1}_{\bm{g}} \, \bm{\mathcal{G}} \, ,
\end{equation}
where $\bm{\mathcal{G}} \in \mathbb{R}^{6}$ is the gravity acceleration twist, $\bm{\mathcal{M}} = \rho \, \textnormal{diag}\left(J_x, J_y, J_z, A, A, A\right)$ is the cross-section's inertia matrix, and $\rho(s)$ is the cross-section's density.

The statics of a Cosserat Rod can be written as
\begin{equation} \label{eq:crt_statics}
    \left(\bm{\mathcal{F}}_i - \bm{\mathcal{F}}_a\right)' + \textnormal{ad}_{\bm{\xi}}^{*} \left(\bm{\mathcal{F}}_i - \bm{\mathcal{F}}_a\right) + \bm{\mathcal{F}}_e = \bm{0} \, ,
\end{equation}
where, in \eqref{eq:internal_passive}, the damping contribution is neglected due to the static regime.

Finally, the dynamics of the Cosserat rod can be derived using the Poincaré principle \cite{renda2018discrete}, yielding 
\begin{equation} \label{eq:crt_dynamics}
    \bm{\mathcal{M}} \dot{\bm{\eta}} + \textnormal{ad}^*_{\bm{\eta}} \left(\bm{\mathcal{M}} \bm{\eta}\right) = \left(\bm{\mathcal{F}}_i - \bm{\mathcal{F}}_a \right)' + \textnormal{ad}^*_{\bm{\xi}} \left(\bm{\mathcal{F}}_i - \bm{\mathcal{F}}_a \right) + \bm{\mathcal{F}}_{e} .
\end{equation}