%%%%%%%%%%%%%%%%%%%%%%%%%%%%ABSTRACT%%%%%%%%%%%%%%%%%%%%%%%%%%%%%%%
% 123 words / 200 words
Continuum soft robots, composed of flexible materials, exhibit theoretically infinite degrees of freedom, enabling notable adaptability in unstructured environments. Cosserat Rod Theory has emerged as a prominent framework for modeling these robots efficiently, representing continuum soft robots as time-varying curves, known as backbones. In this work, we propose viewing the robot's backbone as a signal in space and time, applying the Fourier transform to describe its deformation compactly. This approach unifies existing modeling strategies within the Cosserat Rod Theory framework, offering insights into commonly used heuristic methods. Moreover, the Fourier transform enables the development of a data-driven methodology to experimentally capture the robot's deformation. 
The proposed approach is validated through numerical simulations and experiments on a real-world prototype, demonstrating a reduction in the degrees of freedom while preserving the accuracy of the deformation representation.