The \ac{STFT} of the strain field $\bm{\xi}(s, t)$ is defined as
\begin{equation} \label{eq:space_time_ft}
    \bm{\Xi}\left(jk, \, j \omega\right) = \int_{0}^{+\infty} \int_{0}^{L} {\bm{\xi}}(s, \, t) \, e^{-j\left(k s + \omega t\right)} \, \textnormal{d} s \, \textnormal{d} t \, ,
\end{equation}
where $\omega = 2 \pi f \in \mathbb{R}$, and $f  \in \mathbb{R}$ is the time-frequency. 

For the discrete case, let us assume to have $M \times N$ samples of the strain field $\bm{\xi}(n \lambda_s, \, m T_s)$, where $m \in [0, \, M - 1]$ and $T_s$ is the sampling period. 
The Discrete \ac{STFT} can be written as
\begin{equation} \label{eq:discrete_stft}
    \bm{\Xi}\left(jk, j \omega\right) = \sum_{m = 0}^{M - 1} \sum_{n = 0}^{N - 1} \bm{\xi}\left(n \lambda_s, m T_s \right) e^{-j \left( k n \lambda_s + \omega m T_s\right)} \, .
\end{equation}
% Discussion on utility
This analysis provides useful information about the dynamic response of the system and highlights certain components in the spatial spectrum that arise during the transient regime.

% \Xi(\cdot, j \omega)
By examining the curves $\bm{\Xi}(\cdot, \ j \omega)$, it is possible to observe the time evolution of each spatial harmonic, providing a useful tool to identify resonance or anti-resonance peaks, as well as to assess the relevance of specific spatial harmonics in the time-frequency domain. This information can be leveraged to discard irrelevant harmonics.

% \Xi(j k, \cdot)
Conversely, the curves $\bm{\Xi}(jk, \ \cdot)$ represent the spatial harmonics excited by an input with a specific time-frequency. These curves are useful for understanding the strain profile in the spatial-frequency, offering insights into how the shape of the \ac{CSR} changes in response to an input of a specific time-frequency.

Moreover, if the \ac{CSR} is integrated within a control framework operating at a specific frequency, analyzing $\bm{\Xi}(jk, \, \cdot)$ allows the user to select an optimal set of basis functions.

\subsection{Dynamics of a Cosserat Rod in the Space-Time Frequency Domain} \label{space_time_ft:dynamics_stft}
From the properties of the 2D \ac{FT} \eqref{eq:fourier_definition}-\eqref{eq:fourier2d_product}, the dynamics of an elastic rod \eqref{eq:crt_dynamics} can be described with a set of algebraic equations in the space-time frequency domain, such as
\begin{equation} \label{eq:dynamics_ft}
    \begin{split}
        &\bm{M} \ast \left(j \omega \, \bm{\mathcal{H}}\right) + \textnormal{ad}_{\bm{\mathcal{H}}}^{*} \ast \left(\bm{M} \ast \bm{\mathcal{H}}\right) = \\
        &jk \left(\bm{F}_i - \bm{F}_a\right) + \textnormal{ad}_{\bm{\Xi}}^{*} \ast \left(\bm{F}_i - \bm{F}_a\right) + \bm{F}_e
    \end{split} \, ,
\end{equation}
where $\bm{M}(jk) = \fourier{\bm{\mathcal{M}}(s)}$, and $\bm{\mathcal{H}}(jk, \, j \omega) = \fourier{\bm{\eta}(s, t)}$.
Moreover, the mixed derivative equality \eqref{eq:mixed_derivative} holds also in the frequency domain, such as 
\begin{equation} \label{eq:mixed_derivative_ft}
    jk \, \bm{\mathcal{H}} = j \omega \, \bm{\Xi} - \left(\textnormal{ad}_{\bm{\Xi}} \ast \bm{\mathcal{H}}\right) \, .
\end{equation}
In \eqref{eq:dynamics_ft}, the \ac{STFT} of the internal wrenches \eqref{eq:internal_passive}, \eqref{eq:internal_active} can be expressed as
\begin{equation} \label{eq:internal_passive_ft}
    \bm{F}_i(j k , \, j \omega) = \bm{\Sigma}(jk) \ast \left(\bm{\Xi} - \bm{\Xi}^{*}\right) + \bm{\Psi}(jk) \ast \left(j \omega \bm{\Xi}\right) \, ,
\end{equation}
\begin{equation} \label{eq:internal_active_ft}
    \bm{F}_a\left(jk, \, j \omega\right) = \bm{B}_{\bm{\tau}}\left(\bm{\Xi}, jk\right) \ast \bm{T}(j \omega) \, ,
\end{equation}
where $\bm{\Psi}(jk) = \fourier{\bm{\Psi}(s)}$, and $\bm{T}(j \omega) = \fourier{\bm{\tau}(t)}$.

Finally, in the case of \ac{GVS} parameterization, we can exploit the separation property of the 2D \ac{FT} \eqref{eq:fourier2d_product} in \eqref{eq:gvs_strain}, resulting in
\begin{equation} \label{eq:gvs_stft}
    \bm{\Xi}(jk, \, j\omega) = \left(\bm{B}_{\bm{q}}\left(jk\right) \, \bm{Q}(j \omega) + \bm{\Xi}^{*}(jk)\right) \ast \Pi_{L}\left(    jk \right) \, ,
\end{equation}
where $\bm{Q}(j \omega) = \fourier{\bm{q}(t)}$. 
With this relation, it is possible to study separately the space and time spectra.