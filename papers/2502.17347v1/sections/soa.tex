The dynamics of a Cosserat rod \eqref{eq:crt_dynamics} is a set of \acp{PDE}, whose solution lies in a functional space, i.e., an infinite-dimension space. However, many approaches have been proposed in the literature to efficiently describe the robot with a finite number of \ac{DoFs}, trying to find the best trade-off between accuracy and computational efficiency.

% PCC - PCS
One of the most common modeling approaches is the \ac{PCC} \cite{webster2010design}. The core idea is to divide the backbone into segments and assume them as circumference arcs. This approximation fits very well for the slender shape of the \acp{CSR}. Furthermore, it provides the dynamics in the classical Lagrangian form \cite{siciliano}, allowing the transfer of many controllers from the rigid robot literature \cite{della2018dynamic}. In addition, the \ac{PCC} approach accounts for bending and elongation modes, describing effectively the deformations excited by tendon-driven and fluidic actuation sources.

% Grazioso
To include the twisting mode, \cite{grazioso2019geometrically} proposed to fit the pieces with a helix (i.e., constant curvature and torsion), enabling a general curve in the 3D space.

% Renda
However, the approximation of unshearable rods is unsuitable for interaction tasks where shear deformation is significant, particularly in cases involving contact with irregular objects or the environment. To expand the previous models, \cite{renda2018discrete} proposed the \ac{PCS} approach, which consists of considering the strain field constant along the length of the single piece. 

% PAC, PAS
Differently, \cite{della2020soft} considers the curvature an affine function instead of constant for each single segment.
%Instead of constant curvature, another approach explored is to consider it an affine function for a single segment \cite{della2020soft}.
With only two \ac{DoFs}, this model exhibits nonlinear phenomena such as snap effect \cite{armanini2017elastica, caradonna2024model}. A further work \cite{stella2023piecewise} proposed the \ac{PAC}, in which each piece is described by an affine function in $s$.

Following this methodology, \cite{li2023piecewise} presented the \ac{PLS}, in which the strain field is computed by linear interpolation of samples. In addition, \ac{PLS} is applied in the case of interactions, exploiting excellent performances w.r.t. the other modeling methodologies \cite{xun2024cosserat}.
In general, we can call \ac{PAS} all the approaches that use the linear approximation of each segment of the strain field. 

% GVS
Differently from the piecewise methods, in \cite{renda2020geometric, boyer2020dynamics} the \ac{GVS} is proposed, in which the strain field is assumed as generated by a truncated functional basis of space-dependent vectors \cite{armanini2023soft}. It is worth highlighting that, with a polynomial basis function, the \ac{GVS} approach coincides with the affine strain and curvature approximation. However, the \ac{GVS} approach allows using other basis functions, such as trigonometric and Gaussian.

Furthermore, utilizing the Magnus expansion \cite[Chap. IV.7]{hairergeometric} and Zanna's collocation method \cite{zanna1999collocation}, the Authors provided an efficient and recursive algorithm to compute the Jacobian of the \ac{CSR}. 
The primary advantage of this approach is that, regardless of the functional basis chosen, the classic Lagrangian form \cite{siciliano} of the dynamics can be found.
This significantly eases the transfer of classical controllers from rigid to soft robotics.

 Following this approach, the Authors proposed a strain-dependent functional basis called \ac{ISP} \cite{renda2024dynamics}. In this approach, the basis functions are derived from the robot's statics, achieving a minimal number of \ac{DoFs} corresponding to the strain modes excited by the actuators’ routing and external forces. However, under dynamic regimes, further strain modes might be excited, requiring a user-specified extended basis.

In \cite{mathew2024reduced}, the Authors present a comprehensive explanation of the \ac{GVS} framework, covering every possible type of functional basis. 
Additionally, it has been incorporated into \ac{SoRoSim} \cite{mathew2022sorosim}, which allows for rapid simulation of soft robots, even when they involve numerous amounts of \ac{DoFs}.

Since the choice of basis is critical, the research community has explored methods to optimize the number of bases and reduce the system's order.
For instance, in \cite{pustina2024nonlinear}, the Authors introduced eigenmanifolds to perform modal analysis for \acp{CSR}. This approach facilitates the evaluation of \ac{PCC} models with an increasing number of segments by employing similarity metrics compared to high-fidelity models such as the Finite Element Method.

% Alkayas
Finally, \cite{alkayas2024soft} presents a data-driven reduction method based on \ac{POD}. The key concept involves applying \ac{SVD} to the strain data to identify the least significant singular values, enabling truncation without significant loss of accuracy. 
First, a \ac{GVS} digital twin with a high number of \ac{DoFs} is derived by fitting the experimental data. Then, after simulating the fitted \ac{GVS} model, a reduction in the number of \ac{DoFs} is performed using the \ac{POD} method.

% Our Contributions
In this context, we propose a novel approach that treats the backbone of the continuum robot as a space-time varying signal and analyzes it using the \ac{FT}. This framework unifies existing methods by interpreting them as reconstructors based on discrete strain samples. Moreover, this perspective provides theoretical justification for various heuristic methods, such as the minimum number of segments required in piecewise approaches.

Additionally, by applying the \ac{FT} to experimental data, we establish a data-driven methodology with a strong mathematical foundation for identifying the optimal functional bases from a predefined signal dictionary. This approach relies solely on the geometric properties of the robot and its actuators. 
The method detects the functional bases in the dynamic situation and extends the static strain analysis of the \ac{ISP} method.