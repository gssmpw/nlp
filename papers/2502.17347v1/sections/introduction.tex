% Intro to the field
\IEEEPARstart{S}{\lowercase{oft}} robotics is a promising area of robotics in which robots are made from soft materials, such as silicones or rubbers  \cite{rus2015design}. 
The continuum nature of this type of robot exhibits theoretically infinite \ac{DoFs}, enabling adaptability in unstructured or cluttered environments \cite{del2024growing}. 
However, the infinite \ac{DoFs} of \acp{CSR} make controllers' design arduous. 

% Why modeling is important
The soft robotics community proposed numerous controllers, marking two main approach types: model-based \cite{della2023model} and learning-based \cite{george2018control} controllers. The former exploits an accurate robot model to accomplish the task. The latter utilizes experimental or simulated data to train learning algorithms. In both cases, the a priori knowledge provided by the model significantly improves the controllers' performances \cite{falotico2024learning}. In this context, modeling \acp{CSR} became crucial, and researchers tried to find the best trade-off between accuracy and computational efficiency \cite{armanini2023soft}.

% Cosserat
In the last decade, the \ac{CRT} \cite{renda2014dynamic, gazzola2018forward} has emerged as one of the most effective compromises, accounting for all possible strain modes, including stretching, shearing, bending, and twisting. Moreover, numerous implementations exist with proven computational efficiency \cite{mathew2022sorosim, naughton2021elastica}.
The main idea of \ac{CRT} for \acp{CSR} is to describe them as time-varying oriented curves, referred to as the backbone. The evolution of the backbone is described by a set of \acp{PDE}, reflecting the infinite \ac{DoFs} inherent in the continuum structures.

% Spatial Discretization
To reduce the number of \ac{DoFs} for simulation and control, various spatial discretization techniques have been proposed \cite{armanini2023soft}. The principal approaches in the literature typically divide the backbone into segments, assuming each segment forms a specific curve (e.g., circumference \cite{webster2010design}, helix \cite{grazioso2019geometrically}). 
Mathematically, this is equivalent to reconstructing a signal in space from a finite set of samples by fitting with a specific set of functional bases \cite{renda2020geometric, boyer2020dynamics}, such as polynomials \cite{della2019control}.

% Fitting as signal theory and interpolation theory
In this work, we take this concept further by treating the backbone as a spatial signal and applying the \ac{FT} \cite{bracewell2007fourier}. This approach provides a compact representation of the robot's deformation and unifies existing modeling strategies within the \ac{CRT} framework. 
Additionally, it offers a theoretical foundation for heuristic techniques, such as determining the minimum required number of segments.
Furthermore, the \ac{FT} offers a new experimental methodology for selecting optimal modeling methods based on data, balancing accuracy, and computational efficiency.
% Validation
Finally, we validated the proposed data-driven approach through numerical simulations and real-world experiments.
% % Initial Image
% \begin{figure}
%     \centering
%     \includesvg[width=1.0\linewidth]{imgs/sofft_basic_concept.svg}
%     \caption{Representative scheme for applying the Fourier transform to \acp{CSR}. (a) The strain field can be extracted from the backbone and analyzed in the spatial frequency domain. (b) The spatial spectrum of a sampled strain field. Due to the unbounded nature of the strain field spectrum, aliasing is inevitable.}
%     \label{fig:concept}
% \end{figure}
\begin{figure}
    \centering
    \includegraphics[width=1.0\linewidth]{imgs/sofft_basic_concept.pdf}
    \caption{Representative scheme for applying the Fourier transform to \acp{CSR}. (a) The strain field can be extracted from the backbone and analyzed in the spatial frequency domain. (b) The spatial spectrum of a sampled strain field. Due to the unbounded nature of the strain field spectrum, aliasing is inevitable.}
    \label{fig:concept}
\end{figure}


% %% Paper Outline
% \subsection{Paper Outline}
The paper is organized as follows. In Sec. \ref{sec:background}, we discuss the preliminaries, including a brief description of \ac{CRT} (Sec. \ref{background:cosserat_rod_theory}) and the current state of the art in modeling \acp{CSR} (Sec. \ref{background:soa}). In Secs. \ref{sec:spatial_ft} and \ref{sec:space_time_ft}, we apply the \ac{FT} to the \acp{CSR}. Furthermore, we propose a data-driven methodology in Sec. \ref{sec:spectrum_extraction}. Sections \ref{sec:numerical_validation} and \ref{sec:experimental_validation} validate the approach using both simulated and real-world \acp{CSR}. Finally, Sec. \ref{sec:conclusions} summarizes the contributions and results of this work.

% Appendix and Notation
For the sake of readability, the main variables and symbols used are listed in Table \ref{tab:nomenclature}. Additionally, Appendix \ref{sec:appendix} provides the definitions of the Lie Algebra operators and the properties of the \ac{FT}.