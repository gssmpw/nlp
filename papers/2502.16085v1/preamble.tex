\RequirePackage[l2tabu, orthodox]{nag} % 古いコマンドやパッケージの利用を警告する
\usepackage[all, warning]{onlyamsmath} % amsmath が提供しない数式環境を使用した場合に警告
% for not arxiv
% \usepackage[dvipdfmx]{color}
% \usepackage[dvipdfmx,bookmarks=true,bookmarksnumbered=true,bookmarkstype=toc,%
%      colorlinks=true,%
%      linkcolor=black, %%追加20090108
%      citecolor=black, %%追加20090108
%      urlcolor=black, %%追加20090108
%      bookmarksopen=true,%しおりツリーをデフォルトで開く
%      ]{hyperref}
% \usepackage[dvipdfmx]{graphicx}
% for arxiv
\usepackage{hyperref}
\usepackage{color}
\usepackage{graphicx}

\usepackage{url}

%%%% カスタマイズしたい場合は以下に書く

\newcommand{\TODO}[1]{\fbox{{\textbf TODO:}#1}}
\usepackage{layout}
\usepackage{wrapfig}

% hypcap: figref, tabrefしたときにジャンプ先がキャプション部ではなく図の上になる
\usepackage[figure,figure*,table,table*]{hypcap}
\usepackage{algorithm}
\usepackage{algcompatible}
\usepackage{algpseudocode}
\usepackage{amsmath}

\algnewcommand{\algorithmicand}{\textbf{ and }}
\algnewcommand{\algorithmicor}{\textbf{ or }}
\algnewcommand{\OR}{\algorithmicor}
\algnewcommand{\AND}{\algorithmicand}

\newcommand{\secref}[1]{Section \ref{#1}}
\newcommand{\tabref}[1]{{Table \ref{#1}}}
\newcommand{\figref}[1]{{Fig. \ref{#1}}}
\newcommand{\equref}[1]{{Eq. \ref{#1}}}
\newcommand{\algoref}[1]{{Alg. \ref{#1}}}
\newcommand{\enumref}[1]{{\ref{#1}}}

\newcommand{\switchlanguage}[2]{%
  \ifx\paperlanguage\empty%
  #1%
  \else%
  #2%
  \fi%
}
