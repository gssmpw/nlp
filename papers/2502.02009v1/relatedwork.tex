\section{Related Work}
\subsection{Configuration Analysis and Management}
Recent research has made significant progress in understanding and addressing container security challenges through various methodological approaches. Static analysis techniques have been a commonly used approach, with studies by Smith et al.\cite{6987569} and Johnson et al.\cite{4534479} developing tools for analysing container configurations without execution. These approaches primarily focus on identifying potential security issues through pattern matching and semantic analysis, providing early detection of vulnerabilities in configuration specifications.

Complementing static analysis, dynamic analysis methods have been developed to provide runtime security monitoring and enforcement. Sultan et al.\cite{sultan2019container} proposed a comprehensive framework for detecting and preventing security violations during container execution, enabling real-time protection against emerging threats. Their work demonstrates the importance of runtime monitoring in maintaining container security, particularly in dynamic deployment environments.

The application of Machine Learning (ML) techniques to configuration security represents a more recent development in the field. Studies by Bandari \cite{bandari2021comprehensive} and Farkouh et al.\cite{farkouh2023intelligent} have demonstrated the potential of ML approaches in identifying patterns within secure and vulnerable configurations. These data-driven approaches offer promising capabilities for automating security analysis, though they often require substantial training data and careful feature engineering.

KGSecConfig\cite{haque2022kgsecconfig} introduced a structured approach to configuration security, which leverages knowledge graphs and ontologies to represent and reason about security configurations. This semantic approach enables a more sophisticated analysis of configuration relationships and dependencies, though it requires careful knowledge of engineering and maintenance.

While these approaches have advanced our understanding of container security analysis, they primarily focus on detection rather than automated repair. Our work extends beyond detection by introducing an end-to-end framework that not only identifies misconfigurations but also automatically generates and validates fixes, addressing a critical gap in existing container security management solutions.

\subsection{Source Code Vulnerability Detection and Fixing}
The domain of automated vulnerability detection and repair in source code provides valuable insights for configuration security management. Traditional approaches have relied heavily on SATs and pattern matching. For example, Fabian et al.\cite{yamaguchi2014modeling} introduced a code property graph for vulnerability detection, combining abstract syntax trees, control flow graphs, and program dependence graphs to identify complex vulnerability patterns.
Building on this foundation, there have been major advances in AI (Machine Learning, Deep Learning, and recently LLMs) for detecting and addressing vulnerabilities in source code~\cite{Harer2018AutomatedSV,le2021deepcva,chakraborty2021deep,le2022survey,le2022towards,fu2022linevul,le2022use,fu2022vulrepair,nguyen2024automated,le2024software}. These prior studies have established key principles about context utilisation and fix validation that parallel our approach to configuration repair.

While code and configuration vulnerabilities present distinct challenges, the underlying principles of combining traditional analysis tools with AI capabilities remain valuable. Our work adapts these lessons to the specific requirements of container configurations, addressing unique challenges such as security parameter interdependencies and the need to maintain operational stability while enhancing security posture.