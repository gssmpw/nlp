\section{Literature Review}
\label{sec:rel}

Recent advancements in blockchain research have utilized various modeling techniques to analyze the performance and behavior of blockchain deployment architectures. Melo et al.~\citep{melo_computing2022,melo_supercomp2021} employed Continuous Time Markov Chains (CTMC), Reliability Block Diagrams (RBD), and Stochastic Petri Nets (SPN) to assess the availability and resource provisioning for Hyperledger Fabric and Ethereum platforms, aiding in strategic blockchain application planning. Studies by Jiang et al.~\citep{jiang_springer_p2pna2020}, Wu et al.~\citep{wu_acm_ease2022}, and Ke and Park~\citep{ke_springer_cc2022} have concentrated on Hyperledger Fabric's performance metrics, with models focusing on throughput, discard rates, queue lengths, and mean response times. Additional research by Xu et al.~\citep{xu_ipm2021}, Yuan et al.~\citep{yuan2020performance}, and Sukhwani et al.~\citep{sukhwani2018performance} explored throughput and latency effects through analytical models and Generalized Stochastic Petri Nets (GSPN), and Stochastic Reward Networks (SRN), respectively, to analyze transaction steps.

Our contribution is an SPN model that evaluates Hyperledger Fabric's performance metrics, including \textit{maximum block size rate} and \textit{timeout block rate}, alongside a calibration approach for these parameters. Our work, summarized in Table~\ref{tab:related}, provides a detailed examination of timeout and block size effects, especially dissecting the ordering phase to highlight block formation issues related to timeouts, offering a more nuanced understanding compared to previous studies.

\begin{table}[!htp]
\centering
\footnotesize
\setlength{\tabcolsep}{3pt}  
\begin{center}
\caption{Related Works}
\begin{tabular}{@{}lcccccc@{}}
\toprule
\rowcolor[HTML]{C0C0C0} 
\multicolumn{1}{c}{\textbf{Publication}} & \multicolumn{1}{c}{\textbf{MRT}} & \multicolumn{1}{c}{\textbf{Throughput}} & \multicolumn{1}{c}{\textbf{RU}} & \multicolumn{1}{c}{\textbf{MSBR}} & \multicolumn{1}{c}{\textbf{TBR}} & \multicolumn{1}{c}{\textbf{DTP}} \\ \midrule
\citep{sukhwani2018performance} & \cmark & \cmark & \cmark & \cmark & \cmark & \xmark \\
\citep{jiang_springer_p2pna2020} & \cmark & \cmark & \xmark & \xmark & \xmark & \cmark \\
\citep{yuan2020performance} & \cmark & \cmark & \xmark & \cmark & \cmark & \xmark \\
\citep{xu_ipm2021} & \cmark & \xmark & \xmark& \cmark  & \xmark & \xmark \\
\citep{melo_supercomp2021} & \xmark & \xmark& \cmark& \xmark  & \xmark & \xmark \\
\citep{ke_springer_cc2022} & \cmark & \xmark& \xmark& \xmark  & \xmark & \xmark \\
\citep{wu_acm_ease2022} & \cmark & \xmark& \xmark& \xmark  & \xmark & \xmark \\
\citep{melo_computing2022} & \xmark & \xmark& \cmark& \xmark  & \xmark & \xmark \\ \bottomrule
\end{tabular}
\label{tab:related}
\end{center}
\begin{tablenotes}[flushleft]\footnotesize
\item[]Metrics abbreviations: Mean Response Time (MRT), Resource Utilization (RU), Maximum Size Block Rate (MSBR), Timeout Block Rate (TBR), Discarding transactions probability (DTP)
 \par
\end{tablenotes}
\end{table}