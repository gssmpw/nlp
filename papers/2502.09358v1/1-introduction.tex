\section{Introduction}
\label{sec:introduction}

Graph realization is a fundamental combinatorial problem in the field of Graph Theory, and its studies have fostered interest in and understanding of the discrete structure of graphs. Nowadays, graph realization is a topic commonly covered in introductory Graph Theory courses, providing those new to the world of graphs with many insights into their combinatorial properties.
The Handshaking Lemma, for example, is usually one of the first statements that beginners come across when they begin studying graphs. Although simple to understand, it is the gateway to a world of more intriguing graph-related questions. By observing that the sum of the degrees of all vertices is equal to twice the number of edges in the graph, it follows that not every sequence of $n$ natural numbers can be the degree sequence of some graph with $n$ vertices, and it becomes natural to ask when a sequence of $n$ natural numbers admits a graph with $n$ vertices whose degree sequence corresponds to the given sequence; the topic related to such a question is called \emph{graph realization}.

Given a sequence $\texttt{d}$ of $n$ natural numbers that satisfy the Handshaking Lemma (i.e., the sum of its values is even), it is a simple exercise to verify that it is possible to construct a multigraph (parallel edges and loops are allowed) whose degree sequence corresponds to $\texttt{d}$. However, this question becomes more intriguing when the goal is to realize a simple graph where parallel edges and loops are not allowed. A non-decreasing sequence $\texttt{d}=(d_1,\ldots,d_n)$ of natural numbers is said to be \emph{graphic} if it is realizable by a simple graph, that is, if there exists a labeled simple graph $G$ with $n$ vertices such that \texttt{d} is its degree sequence.
Formally, the classical {\sc Graph Realization} problem is stated as follows:

\defproblema{Graph Realization}
{A non-decreasing sequence $\texttt{d} = (d_1, \dots, d_n)$ of natural numbers.}
{Is $\texttt{d}$ a graphic sequence?}

In 1960, Erd\H{o}s and Gallai provided necessary and sufficient conditions for a sequence of non-negative integers to be graphic, proving the following theorem.

\begin{theorem}[Erdős and Gallai~\cite{erdos60}]\label{erdosgallaithm}
    A non-decreasing sequence $\texttt{d} = (d_1, \dots, d_n)$ of natural numbers is graphic if and only if \\
    \begin{enumerate*}
        \item $\sum\limits_{i=1}^n d_i$ is even, and %
        \item $\sum\limits_{i=1}^k d_i \le k(k - 1) + \sum\limits_{i=k+1}^n \min\{d_i, k\}$, for every $1 \le k \le n$.
    \end{enumerate*}
\end{theorem}

It is not difficult to see that the conditions presented by Erd\H{o}s and Gallai are necessary. However, the sufficiency proof provided by Erd\H{o}s and Gallai is quite elaborated. Several alternative proofs for this sufficiency condition have been shown since then until recently, such as Harary~\cite{harary1969graph} in 1969, Berge~\cite{berge1973graphs} in 1973,  Choudum~\cite{choudum1986simple} in 1986, Aigner and Triesch~\cite{aigner1994realizability} in 1994, Tripathi and Tyagi~\cite{tripathi2008simple} in 2008, and Tripathi, Venugopalan, and West~\cite{tripathi2010short} in 2010. 
%
From Theorem~\ref{erdosgallaithm}, it follows that \GRfull~ can solved in polynomial time. Additionally, Tripathi, Venugopalan, and West~\cite{tripathi2010short} presented a simple constructive proof of Theorem~\ref{erdosgallaithm} that allows us to obtain the graph to be realized in $\mathcal{O}(n\cdot \sum_{i=1}^n d_i)$ time.
Furthermore, algorithms like the one provided by Havel and Hakimi~\cite{Ha55,Ha62}, which iteratively reduce the degree sequence while maintaining its realizability, also give a constructive approach to finding such graphs if one exists. Havel and Hakimi's algorithms run in
$\mathcal{O}(\sum_{i=1}^n d_i)$ time, which is optimal.

Variants of the \GRfull{} problem requiring that the realizing graph belongs to a particular graph class have also been studied in the literature. Examples of already studied classes included trees~\cite{gupta2007graphic}, 
Split graphs~\mbox{\cite{hammer1981splittance,chat2014recognition}}, Chordal, interval, and perfect graphs~\cite{chernyak1987forcibly}. Besides that, sequence pairs representing the degree sequences of a bipartition in a realizing bipartite graph were also studied in~\cite{burstein2017sufficient}. Surprisingly, the question regarding \GRfull{} for the class of bipartite graphs appears to remain open for over 40 years~\mbox{\cite{bar2022realizing,rao2006survey}}. 
In addition, the problem of determining whether a given sequence defines a unique realizing simple graph was studied in~\cite{aigner1994realizability,KOREN1976235,pak2013constructing}, and Bar-Noy, Peleg, and Rawitz~\cite{bar2020vertex} introduced the vertex-weighted variant of \GRfull{} where we are given a sequence $\texttt{d}=(d_1,d_2,\ldots,d_n)$ representing a ``weighted degree'' sequence, and a vector $\texttt{w}=(w_1,w_2,\ldots,w_n)$ representing vertex weights, and asked whether there is a graph with vertex set $V=\{v_1,v_2,\ldots,v_n\}$ such that for each $v_i$ the sum of the weights of its neighbors is equal to $d_i$. 

%Some examples of graph realization problem
In today's interconnected world, many fields face the challenge of structuring systems with specific connectivity requirements.
%
For instance, in social network analysis, building a network where individuals (vertices) have a fixed number of connections (degree) is crucial for analyzing influence, community structures, and information diffusion. 
%
Similarly, urban planners confront similar issues when designing road networks, where intersections must be connected with a specific number of roads to optimize traffic flow. 
%
These examples highlight the significance of addressing connectivity challenges in various fields where specific connectivity patterns must be achieved. 
One of the most studied problems in this context is the realization problem that deals with degree sequences.  
%
According to Bar-Noy, Böhnlein, Peleg, and Rawitz~\cite{bar2022vertex}, \GRfull{} and its variants have interesting applications in network design, randomized algorithms, analysis of social networks, and chemical networks.

In this paper, in the same flavor as Bar-Noy, Peleg, and Rawitz~\cite{bar2020vertex}, we introduce another natural variant of the \GRfull{}, which we propose calling \textsc{Graph Realization with Cut Constraints}. 
%
First, we start with some definitions. For a vertex set $V = \srange{v_1}{v_n}$, a cut list is defined as a list of pairs $\call = \{(S_1, \ell_1), \dots, (S_m, \ell_m)\}$, where each pair $(S_j, \ell_j) \in \mathcal{L}$ consists of a nonempty, proper subset $S_j \subset V$ and a natural number $\ell_j$.
%
Given a cut list $\call$ for a set $V$ and a graph $G$ with vertex set $V$, we say that $G$ \emph{realizes} $\mathcal{L}$ if, for every pair $(S_j, \ell_j) \in \mathcal{L}$, the edge cut $\partial(S_j)$ has size $\ell_j$. For a cut list $\mathcal{L}$, we denote by $w(\call) = \max\limits_j\ \size{S_j}$ the largest size among the subsets $S_j$ in $\mathcal{L}$. 
%
Now, we define our problem:

\defproblema{Graph Realization with Cut Constraints (\GRC{})}
{A cut list $\call$ for a set of vertices $V=\srange{v_1}{v_n}$, and a non-decreasing sequence $\texttt{d} = (d_1, \dots, d_n)$ of natural numbers.}
{Does there exist a simple graph with vertex set $V$ such that, for every $i$, $v_i$ has degree $d_i$ and $G$ realizes $\call$?}

Recall that the degree of a vertex $v_i$ of a graph $G$ is the size of the trivial edge cut $\partial(\{v_i\})$ in $G$. Therefore, one can see \GRfull{} as given a cut list \texttt{d} of all trivial edge cut sizes $(\{v_i\},d_i)$, decide whether there is a graph $G$ realizing \texttt{d}.
%
In \GRC{}, we assume that  $\call$ is a list of some nontrivial edge cut sizes for the realizing graph, i.e., each $S_j$ has a size of at least two and at most $n-2$.
If~$\call=\emptyset$, then the problem becomes the original \GRfull~problem. So, through this work, we always consider $\call\neq \emptyset$ and $w(\call)\geq 2$.

% Without loss of generality, we assume that every $d_i$ is positive as we can remove any $v_i$ of zero degree and update \texttt{d} accordingly while not modifying the instance solution.
%
Although our problem, as we have defined, has never been explored before, it was motivated by an active research topic with several recent results \cite{Ap22,Li24} that aims to learn an unknown graph $G$ or properties of $G$ via cut-queries.
%
In this context, given a graph $G = (V, E)$ with a known vertex set but an unknown edge set, the objective is to reconstruct $G$ or compute some property of $G$ with a minimal number of queries. A cut-query receives $S\subseteq V$ as input and returns the size of the edge cut $\partial(S)$.
% 
One of the main driving interests in this model is its connection with submodular function minimization \cite{Aa05}. Furthermore, these active learning questions have applications in fields like computational biology \cite{Vlad98} and relate to data summarization, where queries reveal ``relevant information'' about the graph. More generally, this type of question can be viewed as a means of determining a property of an unknown object via indirect queries about it \cite{Ai88,Du00}.

Within this framework, our problem can be viewed as a validity check to test whether the cut queries are consistent and if there is some graph that satisfies them. Also, it can be viewed as a variant, where we cannot choose the queries, but rather, we are given cut constraints and want to find one satisfying candidate graph. 
% 
Concerning cut-queries, knowing $\partial(\set{u}), \partial(\set{v}),$ and $\partial(\set{u,v})$, we find out whether or not there is an edge between $u$ and $v$ in $G$. Thus, with at most $\binom{n}{2}+n$ queries, one can always obtain the edge set of  $G$. Similarly, regarding \GRC{}, if $\call$ contains all possible sets of size two, then the problem is trivial. Therefore, in this paper, we are mainly interested in the case where $\call$ has polynomial size with respect to $n$ and does not contain all sets of size two.      


\paragraph{Our Contribution.} 
%
In this work, we study the \GRC{} problem and provide a comprehensive characterization of its computational complexity, focusing on the size of the cut sets involved. 
%
% small cuts
We show that it is polynomial-time solvable for instances where $w(\call)\le3$. Specifically, when $w(\call)=2$, the problem reduces to the classic $f$-factor problem,  which can be solved in polynomial time. 
%
Additionally, we show that cuts of size three, surprisingly, do not increase complexity. Instances involving such cuts can be transformed into equivalent ones where size-three constraints are replaced by cut constraints involving only sets of size at most two, all while preserving the same realizability. 
This shows that even with cut constraints using sets of size three, the problem remains polynomial-time solvable. 
%
% large cuts
On the other hand, we also prove that when cut sets of size four or larger are allowed, the problem cannot be solved in polynomial time unless P~$=$~NP. 
This provides a complete dichotomy regarding the computational complexity of the problem and the size of the cut sets.
In addition, we also prove the NP-completeness for $w(\call)= 6$ when the cut constraints restrict the possibility graph, formally defined in Section~\ref{sec:preliminaries}, to be subcubic and bipartite. 
%
In contrast, when the cut constraints restrict the possibility graph to be a tree, the problem is solvable in polynomial time. 
%
% We also discuss why the reduction technique used for $w(\call)= 3$ does not extend to larger cut sizes.

\paragraph{Related work.}
Several other generalizations and related problems exist in the study of degree sequences and graph realizability. %\cite{Ai94,erdos2018,erdos17,Iv12}.
Aigner and Triesch~\cite{aigner1994realizability} explored the realizability and uniqueness of graphs based on two types of invariants (degree sequences and induced subgraph sizes), focusing on both directed and undirected graphs and their computational complexity.
% 
Similarly, Erdős and Miklós~\cite{erdos2018} discussed the complexity of degree sequence problems, focusing on the second-order degree sequence problem, which is shown to be strongly \classNPC{}. 
%
Erdős et al.~\cite{erdos17} presented a skeleton graph structure for a more general restricted degree sequence problem, studying two cases with specific edge restrictions and examining the connectivity of the realization space.
%
Iványi~\cite{Iv12} explored conditions and algorithms for determining if a sequence is the degree sequence of an $(a, b, n)$-graph,
% (directed and undirected) with edges between vertices constrained by a minimum $a$ and maximum $b$ degree.
which is a (directed or undirected) graph whose vertices degrees are in the $[a, b]$ range.

Another field in Graph Theory that is closely related to the \GRfull{}~problem is the study of graph factors and factorizations.
A \textit{factor} of a graph $G$ is simply a spanning subgraph of $G$.
There have been several studies on graph factors under different constraints, such as conditions on their degrees or restrictions on the classes that they must belong to.
Here, we are particularly interested in graph factors described by their degrees, which we call degree factors.
In this context, given an integer $k$, a \emph{$k$-factor} of a graph $G$ is a $k$-regular spanning subgraph of $G$.
This generalizes many problems, for instance a $1$-factor is the same as a perfect matching, and studies in this area date back to the $19$th century when Petersen \cite{petersen1900} gave one of the first sufficient conditions for a $1$-factor.

The concept of $k$-factors has been generalized to consider other values of degrees rather than a fixed number.
Given two functions $g, f \colon V \to \N$ such that $g \leq f$, a spanning subgraph $H$ of the graph $G = (V, E)$ is a \emph{$(g, f)$-factor} if for every vertex $v$, it holds that $g(v) \leq d_H(v) \leq f(v)$.
If $g = f$, then it is simply called an \emph{$f$-factor}.
The problem of determining if a graph admits a $(g, f)$-factor is known to be solvable in polynomial time \cite{Ans85}.
As will be shown later, the \GRC{} problem generalizes the $f$-factor problem.
Classical results in this field include 
Tutte's theorem on $f$-factors \cite{tutte1952} 
and Lovasz's characterization of~\mbox{$(g,f)$-factors}~\cite{Lovasz70}.
For a detailed treatment of this topic, we refer the reader to the surveys of Akiyama and Kano \cite{akiyama1985} and Plummer~\cite{plummer2007}.

\paragraph{Organization of the text.}
%organization
The remainder of this paper is organized as follows. 
%
In \cref{sec:preliminaries}, we introduce key definitions and notations related to the \GRC{} problem, including some conditions for the realizability of a \GRC{} instance. 
% We also discuss the concepts of \textit{fixed} and \textit{forbidden} edges, highlight their role in simplifying the problem, and define the \textit{possibility graph}.
%
In \cref{sec:small_cuts}, we investigate the \GRC{} problem with cut sizes restricted to three,
%
while \cref{sec:large_cuts} focuses on instances with cuts of size at least four.
%
Finally, in \cref{sec:final_remarks}, we summarize our results and discuss potential extensions of this work. %, and conclude the paper with suggestions for future research.
%
Due to space constraints, some proofs have been omitted.
