\section{Final Remarks}
\label{sec:final_remarks}
%resumo do que fizemos
We introduced the \textsc{Graph Realization with Cut Constraints} problem in this work. %, a generalization of the \GRfull{} problem.
%
This problem is interesting because it combines different graph theory concepts, including degree sequence, cut constraints, $f$-factors, and graph realization.
%
We provide a detailed characterization of its computational complexity based on the size of the cuts.
%
Our results show that the problem can be solved in polynomial time when the cuts are small enough (size at most three). However, the complexity significantly increases when the cuts are larger, and we proved that it becomes \classNPH{}.
%
An interesting direction for future work is identifying other graph classes where the possibility graph $\calg$ of a \GRC{} instance ensures polynomial-time solvability. For example, the idea of \cref{prep:tree_graph} might extend to cactus or, more generally, to graphs with bounded degeneracy or treewidth. The case of a planar possibility graph also deserves further investigation. 
%
We also ask about the complexity of {1-in-3 SAT}$_{(2,2)}$, the variant of {1-in-3 SAT} where each variable occurs exactly four times, twice positive and twice negative. 
