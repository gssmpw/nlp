\section{Preliminaries}
\label{sec:preliminaries}

Let $S$ be a set of vertices of a graph $G=(V, E)$.
We denote the total degree of vertices in $S$ by $d(S) = \sum_{u \in S} d_u$, where $d_u$ represents the degree of vertex $u$.
A simple observation is that the size of the edge cut $\partial(S)$ is determined by the degree of the vertices in $S$ and the edges between vertices of $S$.
If there are $k$ edges between vertices of $S$, then $\size{\partial(S)}=d(S)- 2k$.
%
Since the number of edges in $S$ may vary from $0$ to $\binom{\size{S}}{2}$, a necessary condition for the realizability of an \GRC{} instance is as follows.

\begin{remark}
\label{thm:feasible_cut_sizes}
    A \GRC{} instance $(\texttt{d}, \call)$ is realizable only if, for each cut $(S, \ell) \in \call$, we have $\ell \in \set{ d(S) - 2k : 0 \leq k \leq \binom{\size{S}}{2} }$.
\end{remark}

Since this condition is easily verifiable, we assume henceforth that it holds for any \GRC{} instance. In particular, for cuts of size two, this observation implies that only two feasible values are possible, determining whether an edge must exist between the corresponding vertices, as detailed below.

\begin{remark}
\label{thm:fixed_forbidden_edges}
    Given an instance $I = (\texttt{d}, \call)$ of \GRC{}, in any realization $G$ of $I$, if $(\set{u, v}, d_u + d_v - 2) \in \call$, then $uv \in E(G)$, and if $(\set{u, v}, d_u + d_v) \in \call$, then $uv \notin E(G)$.
\end{remark}

Based on this, we say that an edge $uv$ is \textit{fixed} if $(\set{u, v}, d_u + d_v - 2) \in \call$ and is \textit{forbidden} if $(\set{u, v}, d_u + d_v) \in \call$.
We apply similar terminology when constructing an instance of \GRC{}.
Given an instance $(\texttt{d}, \call)$ of \GRC{}, to \textit{fix} or \textit{forbid} an edge $uv$ means adding the cut $(\set{u, v}, d_u + d_v - 2)$ or $(\set{u, v}, d_u + d_v)$ to $\call$, respectively.

\cref{thm:fixed_forbidden_edges} implies that the \GRC{} problem, when limited to cuts of size two, is equivalent to the \GR{} problem with added constraints: a subset of edges is fixed, and another disjoint one is forbidden. Moreover, we can simplify the problem by focusing only on forbidden edges by reducing the degree of vertices incident to fixed edges and then marking those edges as forbidden.
%
Formally, given an instance $(\texttt{d}, \call)$ and a cut $(\set{u, v}, d_u + d_v - 2) \in \call$, in which case the edge $uv$ is fixed, we can produce an equivalent instance $(\texttt{d}', \call')$ as follows.  For all $i \notin \{u, v\}$  set $d'_i = d_i$. Reduce $d'_u = d_u - 1$ and   $d'_v = d_v - 1$;  
%\begin{align*}
    % \begin{cases}
    %     d'_i = d_i, &\text{ if  $i \notin \{u, v\}$ }\\
    %     d'_u = d_u - 1 \\
    %     d'_v = d_v - 1 
    % \end{cases}
    %
    % d'_i &= d_i, &\mbox{ if  }i \notin \{u, v\};\\
    %d'_i &= d_i, &\forall\ i \notin \{u, v\};\\
    %d'_u &= d_u - 1 ;\\
    %d'_v &= d_v - 1 ;
%\end{align*}
and $\call'$ is obtained from $\call$ by replacing $(\set{u, v}, d_u + d_v - 2)$ with $(\set{u, v}, d_u + d_v)$.


The resulting instance $(\texttt{d}', \call')$ has a realization if and only if $(\texttt{d}, \call)$ has a realization. If $G = (V, E)$ is a realization of $(\texttt{d}, \call)$, then, as discussed above, we must have $uv \in E$, and $G - uv$ is a realization of $\call'$. Conversely, if $G' = (V, E')$ is a realization of $(\texttt{d}', \call')$, then necessarily $uv \notin E'$ due to the cut $(\set{u, v}, d_u + d_v)$, and $G' + uv$ is a realization of $(\texttt{d}, \call)$.

Thus, cut restrictions involving sets of size two can be simply reinterpreted as forbidding edges.
%
Let $F$ be the set of all forbidden edges that cannot appear in any realization of instance $(\texttt{d}, \call)$. Then $\calg = K_n - F$ is what we call the \emph{possibility graph}, which must be a supergraph of any valid realization of~$(\texttt{d}, \call)$.
