%%%%%%%% ICML 2025 LATEX SUBMISSION FILE %%%%%%%%%%%%%%%%%

\documentclass{article}

% Recommended, but optional, packages for figures and better typesetting:
\usepackage{microtype}
\usepackage{graphicx}
\usepackage{subfigure}
\usepackage{booktabs} % for professional tables
\usepackage{multirow, makecell}
\usepackage{xcolor}
\usepackage{enumitem}
\usepackage{algorithm}
% \usepackage{algcompatible}
\usepackage{algpseudocode}
\usepackage{listings}
\usepackage{soul}


% hyperref makes hyperlinks in the resulting PDF.
% If your build breaks (sometimes temporarily if a hyperlink spans a page)
% please comment out the following usepackage line and replace
% \usepackage{icml2025} with \usepackage[nohyperref]{icml2025} above.
\usepackage{hyperref}


% Attempt to make hyperref and algorithmic work together better:
\newcommand{\theHalgorithm}{\arabic{algorithm}}
\makeatletter
\newenvironment{breakablealgorithm}
{
    \begin{center}
        \refstepcounter{algorithm}
        \renewcommand{\caption}[1]
        {
            \addcontentsline{loa}{algorithm}{\protect\numberline{\thealgorithm}##1}
            \parbox{\textwidth}
            % Makes a unbreakable block and can also be done by `minipage`.
            {
                \hrule height.8pt depth0pt \kern2pt
                {\raggedright\textbf{\fname@algorithm~\thealgorithm} ##1\par}
                \kern2pt\hrule\kern2pt
            }
        }
}
{
        \kern2pt\hrule\relax
    \end{center}
}
\makeatother

% Define an improved style with comment coloring
\lstdefinestyle{llmstyle}{
    backgroundcolor=\color{gray!10},   % Light gray background
    frame=single,                      % Boxed frame
    rulecolor=\color{gray!50},         % Lighter frame color
    basicstyle=\ttfamily\footnotesize, % Monospace, smaller font
    breaklines=true,                   % Wrap long lines
    breakatwhitespace=true,            % Ensure proper breaking
    linewidth=\textwidth,              % Fit within page width
    tabsize=2,                         % Set tab size
    captionpos=b,                      % Caption at bottom
    numbers=left,                      % Line numbers on the left
    numberstyle=\tiny\color{gray},     % Small gray line numbers
    keywordstyle=\bfseries\color{blue}, % Bold blue for keywords
    commentstyle=\itshape\color{green!60!black}, % Italic green for comments
    stringstyle=\color{teal},          % Teal for strings
    morecomment=[l]{//},               % Define // as a comment delimiter
    morecomment=[s]{/*}{*/}            % Define /* ... */ as a comment block
    % escapeinside={(*@}{@*)}            % Allows LaTeX commands inside lstlisting
}
\lstset{style=llmstyle}
\sethlcolor{yellow} % Set highlight color

% Use the following line for the initial blind version submitted for review:
% \usepackage{icml2025}

% If accepted, instead use the following line for the camera-ready submission:
\usepackage[accepted]{icml2025}

% For theorems and such
\usepackage{amsmath}
\usepackage{amssymb}
\usepackage{mathtools}
\usepackage{amsthm}

% if you use cleveref..
\usepackage[capitalize,noabbrev]{cleveref}

%%%%%%%%%%%%%%%%%%%%%%%%%%%%%%%%
% THEOREMS
%%%%%%%%%%%%%%%%%%%%%%%%%%%%%%%%
\theoremstyle{plain}
\newtheorem{theorem}{Theorem}[section]
\newtheorem{proposition}[theorem]{Proposition}
\newtheorem{lemma}[theorem]{Lemma}
\newtheorem{corollary}[theorem]{Corollary}
\theoremstyle{definition}
\newtheorem{definition}[theorem]{Definition}
\newtheorem{assumption}[theorem]{Assumption}
\theoremstyle{remark}
\newtheorem{remark}[theorem]{Remark}
% \renewcommand{\algorithmicrequire}{\textbf{Input:}}
% \renewcommand{\algorithmicensure}{\textbf{Output:}}

% Todonotes is useful during development; simply uncomment the next line
%    and comment out the line below the next line to turn off comments
%\usepackage[disable,textsize=tiny]{todonotes}
\usepackage[textsize=tiny]{todonotes}


% The \icmltitle you define below is probably too long as a header.
% Therefore, a short form for the running title is supplied here:
\icmltitlerunning{NARCE: A Neural Algorithmic Reasoner Framework for Online Complex Event Detection}

% Custom command for NARCE
\newcommand{\narce}{\texttt{NARCE}}

\begin{document}

\twocolumn[
\icmltitle{\narce{}: A Mamba-Based Neural Algorithmic Reasoner Framework\\for Online Complex Event Detection}

% It is OKAY to include author information, even for blind
% submissions: the style file will automatically remove it for you
% unless you've provided the [accepted] option to the icml2025
% package.

% List of affiliations: The first argument should be a (short)
% identifier you will use later to specify author affiliations
% Academic affiliations should list Department, University, City, Region, Country
% Industry affiliations should list Company, City, Region, Country

% You can specify symbols, otherwise they are numbered in order.
% Ideally, you should not use this facility. Affiliations will be numbered
% in order of appearance and this is the preferred way.
\icmlsetsymbol{equal}{*}
\icmlsetsymbol{note1}{\dag}
\icmlsetsymbol{note2}{\ddag}
% \newcommand{\icmlEqualContribution}{\textsuperscript{*}Equal contribution }

\begin{icmlauthorlist}
% \icmlauthor{Firstname1 Lastname1}{equal,yyy}
% \icmlauthor{Firstname2 Lastname2}{equal,yyy,comp}
% \icmlauthor{Firstname3 Lastname3}{comp}
% \icmlauthor{Firstname4 Lastname4}{sch}
% \icmlauthor{Firstname5 Lastname5}{yyy}
% \icmlauthor{Firstname6 Lastname6}{sch,yyy,comp}
% \icmlauthor{Firstname7 Lastname7}{comp}
% \icmlauthor{}{sch}
% \icmlauthor{Firstname8 Lastname8}{sch}
% \icmlauthor{Firstname8 Lastname8}{yyy,comp}
% %\icmlauthor{}{sch}
% %\icmlauthor{}{sch}
\icmlauthor{Liying Han}{ucla}
\icmlauthor{Gaofeng Dong}{ucla}
\icmlauthor{Xiaomin Ouyang}{note1,ucla,hkust}
\icmlauthor{Lance Kaplan}{arl}
\icmlauthor{Federico Cerutti}{ub}
\icmlauthor{Mani Srivastava}{note2,ucla,amazon}
\end{icmlauthorlist}

% \icmlaffiliation{yyy}{Department of XXX, University of YYY, Location, Country}
% \icmlaffiliation{comp}{Company Name, Location, Country}
% \icmlaffiliation{sch}{School of ZZZ, Institute of WWW, Location, Country}
\icmlaffiliation{ucla}{Department of Electrical and Computer Engineering, University of California, Los Angeles, US}
\icmlaffiliation{ub}{Department of Information Engineering, University of Brescia, Italy}
\icmlaffiliation{hkust}{Department of Computer Science and Engineering, Hong Kong University of Science and Technology, China}
\icmlaffiliation{arl}{US Army DEVCOM Army Research Laboratory}
\icmlaffiliation{amazon}{Amazon. 

\textsuperscript{\dag}This work was done while the author was at UCLA \textsuperscript{\ddag}The author holds concurrent appointments as an Amazon Scholar and a Professor at UCLA, but the work in this paper is unrelated to Amazon}
% \icmlaffiliation{note1}{\textsuperscript{\dag}This work was done while Xiaomin Ouyang was at UCLA}
% \icmlaffiliation{note2}{\textsuperscript{\ddag}Mani Srivastava holds concurrent appointments as an Amazon Scholar and a Professor at UCLA, but the work in this paper is unrelated to Amazon}

% \icmlcorrespondingauthor{Firstname1 Lastname1}{first1.last1@xxx.edu}
% \icmlcorrespondingauthor{Firstname2 Lastname2}{first2.last2@www.uk}

\icmlcorrespondingauthor{Liying Han}{liying98@ucla.edu}
\icmlcorrespondingauthor{Mani Srivastava}{mbs@ucla.edu}


% You may provide any keywords that you
% find helpful for describing your paper; these are used to populate
% the "keywords" metadata in the PDF but will not be shown in the document
% \icmlkeywords{Machine Learning, ICML}

\vskip 0.3in]

% this must go after the closing bracket ] following \twocolumn[ ...

% This command actually creates the footnote in the first column
% listing the affiliations and the copyright notice.
% The command takes one argument, which is text to display at the start of the footnote.
% The \icmlEqualContribution command is standard text for equal contribution.
% Remove it (just {}) if you do not need this facility.

\printAffiliationsAndNotice{}

 % leave blank if no need to mention equal contribution
% \printAffiliationsAndNotice{\icmlEqualContribution} % otherwise use the standard text.


\begin{abstract}


The choice of representation for geographic location significantly impacts the accuracy of models for a broad range of geospatial tasks, including fine-grained species classification, population density estimation, and biome classification. Recent works like SatCLIP and GeoCLIP learn such representations by contrastively aligning geolocation with co-located images. While these methods work exceptionally well, in this paper, we posit that the current training strategies fail to fully capture the important visual features. We provide an information theoretic perspective on why the resulting embeddings from these methods discard crucial visual information that is important for many downstream tasks. To solve this problem, we propose a novel retrieval-augmented strategy called RANGE. We build our method on the intuition that the visual features of a location can be estimated by combining the visual features from multiple similar-looking locations. We evaluate our method across a wide variety of tasks. Our results show that RANGE outperforms the existing state-of-the-art models with significant margins in most tasks. We show gains of up to 13.1\% on classification tasks and 0.145 $R^2$ on regression tasks. All our code and models will be made available at: \href{https://github.com/mvrl/RANGE}{https://github.com/mvrl/RANGE}.

\end{abstract}


\section{Introduction}

Video generation has garnered significant attention owing to its transformative potential across a wide range of applications, such media content creation~\citep{polyak2024movie}, advertising~\citep{zhang2024virbo,bacher2021advert}, video games~\citep{yang2024playable,valevski2024diffusion, oasis2024}, and world model simulators~\citep{ha2018world, videoworldsimulators2024, agarwal2025cosmos}. Benefiting from advanced generative algorithms~\citep{goodfellow2014generative, ho2020denoising, liu2023flow, lipman2023flow}, scalable model architectures~\citep{vaswani2017attention, peebles2023scalable}, vast amounts of internet-sourced data~\citep{chen2024panda, nan2024openvid, ju2024miradata}, and ongoing expansion of computing capabilities~\citep{nvidia2022h100, nvidia2023dgxgh200, nvidia2024h200nvl}, remarkable advancements have been achieved in the field of video generation~\citep{ho2022video, ho2022imagen, singer2023makeavideo, blattmann2023align, videoworldsimulators2024, kuaishou2024klingai, yang2024cogvideox, jin2024pyramidal, polyak2024movie, kong2024hunyuanvideo, ji2024prompt}.


In this work, we present \textbf{\ours}, a family of rectified flow~\citep{lipman2023flow, liu2023flow} transformer models designed for joint image and video generation, establishing a pathway toward industry-grade performance. This report centers on four key components: data curation, model architecture design, flow formulation, and training infrastructure optimization—each rigorously refined to meet the demands of high-quality, large-scale video generation.


\begin{figure}[ht]
    \centering
    \begin{subfigure}[b]{0.82\linewidth}
        \centering
        \includegraphics[width=\linewidth]{figures/t2i_1024.pdf}
        \caption{Text-to-Image Samples}\label{fig:main-demo-t2i}
    \end{subfigure}
    \vfill
    \begin{subfigure}[b]{0.82\linewidth}
        \centering
        \includegraphics[width=\linewidth]{figures/t2v_samples.pdf}
        \caption{Text-to-Video Samples}\label{fig:main-demo-t2v}
    \end{subfigure}
\caption{\textbf{Generated samples from \ours.} Key components are highlighted in \textcolor{red}{\textbf{RED}}.}\label{fig:main-demo}
\end{figure}


First, we present a comprehensive data processing pipeline designed to construct large-scale, high-quality image and video-text datasets. The pipeline integrates multiple advanced techniques, including video and image filtering based on aesthetic scores, OCR-driven content analysis, and subjective evaluations, to ensure exceptional visual and contextual quality. Furthermore, we employ multimodal large language models~(MLLMs)~\citep{yuan2025tarsier2} to generate dense and contextually aligned captions, which are subsequently refined using an additional large language model~(LLM)~\citep{yang2024qwen2} to enhance their accuracy, fluency, and descriptive richness. As a result, we have curated a robust training dataset comprising approximately 36M video-text pairs and 160M image-text pairs, which are proven sufficient for training industry-level generative models.

Secondly, we take a pioneering step by applying rectified flow formulation~\citep{lipman2023flow} for joint image and video generation, implemented through the \ours model family, which comprises Transformer architectures with 2B and 8B parameters. At its core, the \ours framework employs a 3D joint image-video variational autoencoder (VAE) to compress image and video inputs into a shared latent space, facilitating unified representation. This shared latent space is coupled with a full-attention~\citep{vaswani2017attention} mechanism, enabling seamless joint training of image and video. This architecture delivers high-quality, coherent outputs across both images and videos, establishing a unified framework for visual generation tasks.


Furthermore, to support the training of \ours at scale, we have developed a robust infrastructure tailored for large-scale model training. Our approach incorporates advanced parallelism strategies~\citep{jacobs2023deepspeed, pytorch_fsdp} to manage memory efficiently during long-context training. Additionally, we employ ByteCheckpoint~\citep{wan2024bytecheckpoint} for high-performance checkpointing and integrate fault-tolerant mechanisms from MegaScale~\citep{jiang2024megascale} to ensure stability and scalability across large GPU clusters. These optimizations enable \ours to handle the computational and data challenges of generative modeling with exceptional efficiency and reliability.


We evaluate \ours on both text-to-image and text-to-video benchmarks to highlight its competitive advantages. For text-to-image generation, \ours-T2I demonstrates strong performance across multiple benchmarks, including T2I-CompBench~\citep{huang2023t2i-compbench}, GenEval~\citep{ghosh2024geneval}, and DPG-Bench~\citep{hu2024ella_dbgbench}, excelling in both visual quality and text-image alignment. In text-to-video benchmarks, \ours-T2V achieves state-of-the-art performance on the UCF-101~\citep{ucf101} zero-shot generation task. Additionally, \ours-T2V attains an impressive score of \textbf{84.85} on VBench~\citep{huang2024vbench}, securing the top position on the leaderboard (as of 2025-01-25) and surpassing several leading commercial text-to-video models. Qualitative results, illustrated in \Cref{fig:main-demo}, further demonstrate the superior quality of the generated media samples. These findings underscore \ours's effectiveness in multi-modal generation and its potential as a high-performing solution for both research and commercial applications.
\section{Related Work}
\label{sec:relatedwork}

\subsection{Current AI Tools for Social Service}
\label{subsec:relatedtools}
% the title I feel is quite broad

Harnessing technology for social good has always been a grand challenge in social service \cite{berzin_practice_2015}. As early as the 90s, artificial neural networks and predictive models have been employed as tools for risk assessments, decision-making, and workload management in sectors like child protective services and mental health treatment \cite{fluke_artificial_1989, patterson_application_1999}. The recent rise of generative AI is poised to further advance social service practice, facilitating the automation of administrative tasks, streamlining of paperwork and documentation, optimisation of resource allocation, data analysis, and enhancing client support and interventions \cite{fernando_integration_2023, perron_generative_2023}.

Today, AI solutions are increasingly being deployed in both policy and practice \cite{goldkind_social_2021, hodgson_problematising_2022}. In clinical social work, AI has been used for risk assessments, crisis management, public health initiatives, and education and training for practitioners \cite{asakura_call_2020, gillingham2019can, jacobi_functions_2023, liedgren_use_2016, molala_social_2023, rice_piloting_2018, tambe_artificial_2018}. AI has also been employed for mental health support and therapeutic interventions, with conversational agents serving as on-demand virtual counsellors to provide clinical care and support \cite{lisetti_i_2013, reamer_artificial_2023}.
% commercial solutions include Woebot, which simulates therapeutic conversation, and Wysa, an “emotionally intelligent” AI coach, powered by evidenced-based clinical techniques \cite{reamer_artificial_2023}. 
% Non-clinical AI agents like Replika and companion robots can also provide social support and reduce loneliness amongst individuals \cite{ahmed_humanrobot_2024, chaturvedi_social_2023, pani_can_2024, ta_user_2020}.

Present research largely focuses on \textit{\textbf{AI-based decision support tools}} in social service \cite{james_algorithmic_2023, kawakami2022improving}, especially predictive risk models (PRMs) used to predict social service risks and outcomes \cite{gillingham2019can, van2017predicting}, like the Allegheny Family Screening Tool (AFST), which assesses child abuse risk using data from US public systems \cite{chouldechova_case_2018, vaithianathan2017developing}. Elsewhere, researchers have also piloted PRMs to predict social service needs for the homeless using Medicaid data\cite{erickson_automatic_2018, pourat_easy_2023}, and AI-powered algorithms to promote health interventions for at-risk populations, such as HIV testing among Californian homeless \cite{rice_piloting_2018, yadav_maximizing_2017}.

\subsection{Generative AI and Human-AI Collaboration}
\label{subsec:relatedworkhaicollaboration}
Beyond decision-making algorithms and PRMs, advancements in generative AI, such as large language models (LLMs), open new possibilities for human-AI (HAI) collaboration in social services. 
LLMs have been called "revolutionary" \cite{fui2023generative} and a "seismic shift" \cite{cooper2023examining}, offering "content support" \cite{memmert2023towards} by generating realistic and coherent responses to user inputs \cite{cascella2023evaluating}. Their vastly improved capabilities and ubiquity \cite{cooper2023examining} makes them poised to revolutionise work patterns \cite{fui2023generative}. Generative AI is already used in fields like design, writing, music, \cite{han2024teams, suh2021ai, verheijden2023collaborative, dhillon2024shaping, gero2023social} healthcare, and clinical settings \cite{zhang2023generative, yu2023leveraging, biswas2024intelligent}, with promising results. However, the social service sector has been slower in adopting AI \cite{diez2023artificial, kawakami2023training}.

% Yet, the social service sector is one that could perhaps stand to gain the most from AI technologies. As Goldkind \cite{goldkind_social_2021} writes, social service, as a "values-centred profession with a robust code of ethics" (p. 372), is uniquely placed to inform the development of thoughtful algorithmic policy and practice. 
Social service, however, stands to benefit immensely from generative AI. SSPs work in time-poor environments \cite{tiah_can_2024}, often overwhelmed with tedious administrative work \cite{meilvang_working_2023} and large amounts of paperwork and data processing \cite{singer_ai_2023, tiah_can_2024}. 
% As such, workers often work in time-poor environments and are burdened with information overload and administrative tasks \cite{tiah_can_2024, meilvang_working_2023}. 
Generative AI is well-placed to streamline and automate tasks like formatting case notes, formulating treatment plans and writing progress reports, which can free up valuable time for more meaningful work like client engagement and enhance service quality \cite{fernando_integration_2023, perron_generative_2023, tiah_can_2024, thesocialworkaimentor_ai_nodate}. 

Given the immense potential, there has been emerging research interest in HAI collaboration and teamwork in the Human-Computer Interaction and Computer Supported Cooperative Work space \cite{wang_human-human_2020}. HAI collaboration and interaction has been postulated by researchers to contribute to new forms of HAI symbiosis and augmented intelligence, where algorithmic and human agents work in tandem with one another to perform tasks better than they could accomplish alone by augmenting each other's strengths and capabilities  \cite{dave_augmented_2023, jarrahi_artificial_2018}.

However, compared to the focus on AI decision-making and PRM tools, there is scant research on generative AI and HAI collaboration in the social service sector \cite{wykman_artificial_2023}. This study therefore seeks to fill this critical gap by exploring how SSPs use and interact with a novel generative AI tool, helping to expand our understanding of the new opportunities that HAI collaboration can bring to the social service sector.

\subsection{Challenges in AI Use in Social Service}
\label{subsec:relatedworkaiuse}

% Despite the immense potential of AI systems to augment social work practice, there are multiple challenges with integrating such systems into real-life practice. 
Despite its evident benefits, multiple challenges plague the integration of AI and its vast potential into real-life social service practice.
% Numerous studies have investigated the use of PRMs to help practitioners decide on a course of action for their clients. 
When employing algorithmic decision-making systems, practitioners often experience tension in weighing AI suggestions against their own judgement \cite{kawakami2022improving, saxena2021framework}, being uncertain of how far they should rely on the machine. 
% Despite often being instructed to use the tool as part of evaluating a client, 
Workers are often reluctant to fully embrace AI assessments due to its inability to adequately account for the full context of a case \cite{kawakami2022improving, gambrill2001need}, and lack of clarity and transparency on AI systems and limitations \cite{kawakami2022improving}. Brown et al. \cite{brown2019toward} conducted workshops using hypothetical algorithmic tools 
% to understand service providers' comfort levels with using such tools in their work,
and found similar issues with mistrust and perceived unreliability. Furthermore, introducing AI tools can  create new problems of its own, causing confusion and distrust amongst workers \cite{kawakami2022improving}. Such factors are critical barriers to the acceptance and effective use of AI in the sector.

\citeauthor{meilvang_working_2023} (2023) cites the concept of \textit{boundary work}, which explores the delineation between "monotonous" administrative labour and "professional", "knowledge based" work drawing on core competencies of SSPs. While computers have long been used for bureaucratic tasks like client registration, the introduction of decision support systems like PRMs stirred debate over AI "threatening professional discretion and, as such, the profession itself" \cite{meilvang_working_2023}. Such latent concerns arguably drive the resistance to technology adoption described above. Generative AI is only set to further push this boundary, 
% these concerns are only set to grow in tandem with the vast capabilities of generative and other modern AI systems. Compared to the relatively primitive AI systems in past years, perceived as statistical algorithms \cite{brown2019toward} turning preset inputs like client age and behavioural symptoms \cite{vaithianathan2017developing} into simple numerical outputs indicating various risk scores, modern AI systems are vastly more capable: LLMs 
with its ability to formulate detailed reports and assessments that encroach upon the "core" work of SSPs.
% accept unrestricted and unstructured inputs and return a range of verbose and detailed evaluations according to the user's instructions. 
Introducing these systems exacerbate previously-raised issues such as understanding the limitations and possibilities of AI systems \cite{kawakami2022improving} and risk of overreliance on AI \cite{van2023chatgpt}, and requires a re-examination of where users fall on the algorithmic aversion-bias scale \cite{brown2019toward} and how they detect and react to algorithmic failings \cite{de2020case}. We address these critical issues through an empirical, on-the-ground study that to our knowledge is the first of its kind since the new wave of generative AI.

% W 

% Yet, to date, we have limited knowledge on the real-world impacts and implications of human-AI collaboration, and few studies have investigated practitioners’ experiences working with and using such AI systems in practice, especially within the social work context \cite{kawakami2022improving}. A small number of studies have explored practitioner perspectives on the use of AI in social work, including Kawakami et al. \cite{kawakami2022improving}, who interviewed social workers on their experiences using the AFST; Stapleton et al. \cite{stapleton_imagining_2022}, who conducted design workshops with caseworkers on the use of PRMs in child welfare; and Wassal et al. \cite{wassal_reimagining_2024}, who interviewed UK social work professionals on the use of AI. A common thread from all these studies was a general disregard for the context and users, with many practitioners criticising the failure of past AI tools arising from the lack of participation and involvement of social workers and actual users of such systems in the design and development of algorithmic systems \cite{wassal_reimagining_2024}. Similarly, in a scoping review done on decision-support algorithms in social work, Jacobi \& Christensen \cite{jacobi_functions_2023} reported that the majority of studies reveal limited bottom-up involvement and interaction between social workers, researchers and developers, and that algorithms were rarely developed with consideration of the perspective of social workers.
% so the \cite{yang_unremarkable_2019} and \cite{holten_moller_shifting_2020} are not real-world impacts? real-world means to hear practitioner's voice? I feel this is quite important but i didnt get this point in intro!

% why mentioning 'which have largely focused on existing ADS tools (e.g., AFST)'? i can see our strength is more localized, but without basic knowledge of social work i didnt get what's the 'departure' here orz
% the paragraph is great! do we need to also add one in line 20 21?

\subsection{Designing AI for Social Service through Participatory Design}
\label{subsec:relatedworkpd}
% i think it's important! but maybe not a whole subsection? but i feel the strong connection with practitioners is indeed one of our novelties and need to highlight it, also in intro maybe
% Participatory design (PD) has long been used extensively in HCI \cite{muller1993participatory}, to both design effective solutions for a specific community and gain a deep understanding of that community. Of particular interest here is the rich body of literature on PD in the field of healthcare \cite{donetto2015experience}, which in this regard shares many similarities and concerns with social work. PD has created effective health improvement apps \cite{ryu2017impact}, 

% PD offers researchers the chance to gather detailed user requirements \cite{ryu2017impact}...

Participatory design (PD) is a staple of HCI research \cite{muller1993participatory}, facilitating the design of effective solutions for a specific community while gaining a deep understanding of its stakeholders. The focus in PD of valuing the opinions and perspectives of users as experts \cite{schuler_participatory_1993} 
% In recent years, the tech and social work sectors have awakened to the importance of involving real users in designing and implementing digital technologies, developing human-centred design processes to iteratively design products or technologies through user feedback 
has gained importance in recent years \cite{storer2023reimagining}. Responding to criticisms and failures of past AI tools that have been implemented without adequate involvement and input from actual users, HCI scholars have adopted PD approaches to design predictive tools to better support human decision-making \cite{lehtiniemi_contextual_2023}.
% ; accordingly, in social service, a line of research has begun studying and designing for human-AI collaboration with real-world users (e.g. \cite{holten_moller_shifting_2020, kawakami2022improving, yang_unremarkable_2019}).
Section \ref{subsec:relatedworkaiuse} shows a clear need to better understand SSP perspectives when designing and implementing AI tools in the social sector. 
Yet, PD research in this area has been limited. \citeauthor{yang2019unremarkable} (2019), through field evaluation with clinicians, investigated reasons behind the failure of previous AI-powered decision support tools, allowing them to design a new-and-improved AI decision-support tool that was better aligned with healthcare workers’ workflows. Similarly, \citeauthor{holten_moller_shifting_2020} (2020) ran PD workshops with caseworkers, data scientists and developers in public service systems to identify the expectations and needs that different stakeholders had in using ADS tools.

% Indeed, it is as Wise \cite{wise_intelligent_1998} noted so many years ago on the rise of intelligent agents: “it is perhaps when technologies are new, when their (and our) movements, habits and attitudes seem most awkward and therefore still at the forefront of our thoughts that they are easiest to analyse” (p. 411). 
Building upon this existing body of work, we thus conduct a study to co-design an AI tool \textit{for} and \textit{with} SSPs through participatory workshops and focus group discussions. In the process, we revisit many of the issues mentioned in Section \ref{subsec:relatedworkaiuse}, but in the context of novel generative AI systems, which are fundamentally different from most historical examples of automation technologies \cite{noy2023experimental}. This valuable empirical inquiry occurs at an opportune time when varied expectations about this nascent technology abound \cite{lehtiniemi_contextual_2023}, allowing us to understand how SSPs incorporate AI into their practice, and what AI can (or cannot) do for them. In doing so, we aim to uncover new theoretical and practical insights on what AI can bring to the social service sector, and formulate design implications for developing AI technologies that SSPs find truly meaningful and useful.
% , and drive future technological innovations to transform the social service sector not just within [our country], but also on a global scale.

 % with an on-the-ground study using a real prototype system that reflects the state of AI in current society. With the presumption that AI will continue to be used in social work given the great benefits it brings, we address the pressing need to investigate these issues to ensure that any potential AI systems are designed and implemented in a responsible and effective manner.

% Building upon these works, this study therefore seeks to adopt a participatory design methodology to investigate social workers’ perspectives and attitudes on AI and human-AI collaboration in their social work practice, thus contributing to the nascent body of practitioner-centred HCI research on the use of AI in social work. Yet, in a departure from prior work, which have largely focused on existing ADS tools (e.g., AFST) and were situated in a Western context, our paper also aims to expand the scope by piloting a novel generative AI tool that was designed and developed by the researchers in partnership with a social service agency based in Singapore, with aims of generating more insights on wider use cases of AI beyond what has been previously studied.

% i may think 'While the current lacunae of research on applications of AI in social work may appear to be a limitation, it simultaneously presents an exciting opportunity for further research and exploration \cite{dey_unleashing_2023},' this point is already convincing enough, not sure if we need to quote here
% I like this end! it's a good transition to our study design, do we need to mention the localization in intro as well? like we target at singapore

% Given the increasing prominence and acceptance of AI in modern society, 

% These increased capabilities vastly exacerbate the issues already present with a simpler tool like the AFST: the boundaries and limitations of an LLM system are significantly more difficult to understand and its possible use cases are exponentially greater in scope. 

% Put this in discussion section instead?
% Kawakami et al's work "highlights the importance of studying how collaborative decision-making... impacts how people rely upon and make sense of AI models," They conclude by recommending designing tools that "support workers in understanding the boundaries of [an AI system's] capabilities", and implementing design procedures that "support open cultures for critical discussion around AI decision making". The authors outline critical challenges of implementing AI systems, elucidating factors that may hinder their effectiveness and even negatively affect operations within the organisation.


% Is this needed?:
% talk about the strengths of PD in eliciting user viewpoints and knowledge, in particular when it is a field that is novel or where a certain system has not been used or developed or tested before
\section{Online Complex Event Detection}
\subsection{Complex Event Definitions}

% \textit{\textbf{Atomic Events (AEs)} are low-level, short-time, and the smallest building blocks of complex events.} Usually, they are instantaneous events that most popular deep learning models can easily recognize. For example, an image classification, a 2-second audio clip classification, human activity recognition or an object detection.

% \textit{\textbf{Complex Events (CEs)} are higher-level events defined by atomic events following some temporal patterns.} Suppose we have a set of atomic events $A$ with $n$ elements $A = \{a_1, a_2, \ldots, a_n\}$. Additionally, we have a set of complex events of interest with $k$ elements $E = \{e_1, e_2, \ldots, e_k\}$. Each complex event $e_i$ is defined by a subset of atomic events $A_i$, following some specific rules:
% \begin{align}
%     &e_i = R_i(A_i) = R_i(a_i^1, a_i^2, \ldots, a_i^{n_i}),\nonumber\\ &\textrm{where }a_i^j \in A_i, \quad 1 \leq j \leq n_i, \quad 1 \leq i \leq k.
% \end{align}
% Here, $R_i$ represents the temporal pattern function that maps the relevant atomic events into a complex event $e_i$, $A_i \subseteq A$, and $n_i$ is the number of atomic events in the subset $A_i$.
\begin{definition}
\label{def:AE}
\emph{Atomic events} (\emph{AE}s) are low-level, short-duration, and fundamental building blocks of complex events. They are typically instantaneous or span a small time window and are directly detectable by models such as image classification, object detection, or activity recognition models.
\end{definition}

\begin{definition}
\label{def:CE}
\emph{Complex events} (\emph{CE}s) are high-level events that are defined as sequences or patterns of atomic events (\emph{AE}s) occurring in specific temporal or logical relationships.
\end{definition}
Let $A = \{a_1, a_2, \ldots, a_n\}$ denote the set of all atomic events, where $n$ is the total number of atomic events. Each $a_i$ is associated with a start time and an end time. Similarly, $E = \{e_1, e_2, \ldots, e_k\}$ denote the set of all complex events of interest, where $k$ is the total number of complex events.

Each complex event $e_i \in E$ is defined as:
\[
e_i = R_i(A_i) = R_i(a_i^1, a_i^2, \ldots, a_i^{n_i}),
\]
where $A_i \subseteq A$ is the subset of atomic events relevant to $e_i$, $R_i$ is a \textit{pattern function} that defines the temporal or logical relationship among the atomic events in $A_i$, and $n_i = |A_i|$ is the number of atomic events involved in defining $e_i$. Each $e_i$ is associated with a time $t_{e_i}$, which is the specific time (or time interval) at which the pattern $R_i$ is satisfied, i.e., when the complex event $e_i$ occurs.

\textbf{Temporal Pattern Function ($R_i$):}
The function $R_i$ maps a subset of atomic events $A_i$ to a complex event $e_i$ by defining specific patterns among the atomic events. In this work, we considered four main categories of patterns: \emph{Sequential Patterns}, \emph{Temporal Patterns}, \emph{Repetition Patterns}, and \emph{Combination Patterns}. Some groups have one or more subcategories; their definitions and examples are provided in Table~\ref{tab:ce_patterns}. Importantly, all patterns considered in this work are {\emph{bounded to finite states}, enabling them to be represented by finite state machines (FSMs).

\begin{table*}[t]
\centering
\caption{Category of Complex Event Patterns.}
{\scriptsize
\begin{tabular}{@{}p{2.5cm}p{7.5cm}p{6.5cm}@{}}
\toprule
\textbf{\emph{CE} Category} & \textbf{Features} & \textbf{Examples} \\ \midrule

\textbf{Sequential Patterns} \\ - \emph{Relaxed} & 
Key \emph{AE}s must be in order, may contain unrelated \emph{AE}s in between &
$A \rightarrow u^* \rightarrow B \rightarrow u^* \rightarrow C$, \newline where $u$ represents user-deinfed unrelated \emph{AE}s.\dag \\ \midrule

\textbf{Temporal Patterns} \\ - \emph{Duration Based} & 
Count the time for specific \emph{AE}(s) & ``Wash hands continuously for at least 20 seconds.'' \newline ``Inadequate brushing teeth that lasts less than 2 minutes, allowing a 10-second grace period in case brushing stops temporarily.'' \\

- \emph{Timing Relationship} & 
Relative timing between different \emph{AE}s, such as \emph{min}, \emph{max} timing constraints & 
``After washing hands, eat within 2 minutes.'' \\ \midrule

\textbf{Repetition Patterns} \\- \emph{Frequency Based} & 
Count the occurrences of specific \emph{AE}(s) over time constraints. & 
``Click the mouse 5 times within 10 seconds.'' \\ 

- \emph{Contextual Count} & 
Count the occurrences of specific \emph{AE}(s) over timing related to other \emph{AE}(s). & 
``After eating, wait for at least 10 minutes to work..'' \\ \midrule

\textbf{Combination \newline Patterns} & 
Sequential + Temporal Patterns & 
``Use Restroom $\rightarrow$ Wash (20s) $\rightarrow$ Work'', \newline (After using the restroom, ensure hands are washed for at least 20 seconds consecutively before returning to work.) \\ \bottomrule
\end{tabular}
}
\vspace{-0.5em}% reduce some space between the table and the footnote

\parbox{0.95\linewidth}{%
\raggedright % Left-align the notes
\scriptsize
\textbf{Notes:} \dag for example, the unrelated \emph{AE} $u$ can be any \emph{AE} other than the \emph{key AE}s $=\{A,B,C\}$. $u^*$ means we allow for zero or more unrelated \emph{AE}, $u$, in sequence. 
}
\label{tab:ce_patterns}
\vspace{-1em}
\end{table*}

% here limited to activities are bounded by vocabulary, use whole set \setmunius {A,B,C} to express
% fix the definition here: before c at least a B, before B no C and at least one A
% mention that all of the current CEs are expressable in FSMs
% e.g.,express the sequential pattern as state transition (state-level trace)
% neurosymbolic we map to a window-level trace
\subsection{Online Detection Task Formalization}\label{sec:CED-task}

\begin{figure}[t]
    \centering
        \setlength{\abovecaptionskip}{0.cm}
    \setlength{\belowcaptionskip}{0.cm}
\centerline{\includegraphics[width=1\columnwidth]{figs/ced-illustration.png}}
\caption{An illustration of the Online Complex Event Detection task. The example on the right shows that ``\textit{Using Restroom}" and ``\textit{Eating}" without ``\textit{Washing hands}" triggers the complex event detection, but only at the last action ``\textit{Washing hands}" we attach the corresponding \textit{CE} label ``1".}
\label{fig:ced-illustration}
\vspace{-1em}
\end{figure}

Without loss of generality, let's assume a system receives a raw data stream $\mathbf{X}$ from a single sensor with some modalities $M$. The sensor operates at a sampling rate $r$. The system processes the data stream using a non-overlapping sliding window with a fixed length $\Delta t$. At the $t$th sliding window, the system extracts a data segment:
\begin{equation}
    \mathbf{D}_t=\mathbf{X}(t),
\end{equation}
where $\mathbf{X}(t) \in \mathbb{R}^{(r \times \Delta t) \times m}$, with $m$ being the feature dimension of sensors data from modality $M$.

At each sliding window $t$, there is a corresponding ground-truth \emph{CE} label $y_t$, which represents the complex event occurring at that time. As shown in Fig.~\ref{fig:ced-illustration}, $y_t$ relies on \emph{AE}s that happened in the previous $t-1$ windows and the current window $t$. The example on the right illustrates the online \emph{CE} labeling approach we adopt. Suppose the full pattern of a complex event spans from $t_1$ to $t_2$, i.e., $t_2$ is the exact time when the complex event is observed to occur. In this case, only $y_{t_2}$ is the corresponding \emph{CE} label. All \textit{CE} labels from $y_{t_1}$ to $y_{t_2 - 1}$ are ``0"s, indicating that no complex event is detected before $t_2$. 

For data streams with up to $T$ sliding window clips, the objective of the system with a real-time CED system $f$ is to accurately predict the complex event label at each sliding window $t$, i.e., to minimize the difference between the predicted \emph{CE} label $\hat{y_t}$ and the ground-truth \emph{CE} label ${y_t}$:
\begin{equation}\label{eq:1}
    \min |\hat{y_t} - y_t|, \quad \textrm{ where }\hat{y_t} = f\left(\mathbf{D}_t\right), \quad 1 \leq t \leq T,
\end{equation}
This constitutes a \textit{multi-label multi-class classification} problem. Let $\mathbf{y} = {y_1, y_2, \ldots, y_T}$ represent the ground-truth complex event label sequence. Equation~\ref{eq:1} can be expressed in a vector form as
\begin{equation}
    \min ||f(\mathbf{D}) - \mathbf{y}||, \quad \textrm{ where } \mathbf{D} = \left\{\mathbf{X}(1),\ldots, \mathbf{X}(T)\right\}.
\end{equation}

In other words, for data-driven methods, the supervision only comes from the \textbf{\emph{high-level, coarse CE labels}}. The fine-grained ground-truth \emph{AE} labels will not be provided during training. However, the model must interpret the semantics of \emph{AEs} at each window while simultaneously learning the \emph{CE} rules. This makes the task a \textbf{unique} and \textbf{challenging} combination of \emph{distant supervision}—where labels are only provided at the event level—and \emph{weak supervision}, where the provided labels are high-level and sparse, offering limited direct guidance for learning the lower-level semantics.







\section{\system{}}
\label{sec:respark}

This section describes the implementation of \system{}, including pre-processing, analysis, and organization. 
\system{} utilizes the GPTs from Azure to incorporate the LLM-driven functionalities. 
Specifically, we employ the ``gpt-4-vision-preview'' model as our input involves chart images. 
The prompts are provided in the supplementary materials. 

\subsection{Pre-processing Stage}

Before proceeding with analysis and organization, we need to pre-process the user dataset and reports to (1) acquire necessary data features, (2) recommend the most relevant reports to the data, and (3) extract the analytical objective and their dependencies of the selected report for the subsequent processes. 

\subsubsection{Data Pre-processing}
\label{subsubsec:data_pre_processing}

Based on the findings in the preliminary study (~\autoref{subsec:preliminary_study}), most adjustments entail considerations of data features such as context, scope, fields, and formats. 
Presenting the entire dataset to LLMs is currently impractical due to token limitations, and it does not facilitate a comprehensive understanding of these data features. 
Therefore, we utilize a similar data summary method to LIDA~\cite{dibia2023lida}. 
This method first extracts scope, data type, and unique value count information and samples some values for each data field. 
Subsequently, it employs LLMs to provide brief semantic descriptions for the dataset and each data field. 
The description of the dataset can also help recommend relevant reports(~\autoref{subsubsec:report_retrieval}). 
These pieces of information are then integrated to form a comprehensive data summary.

\subsubsection{Report Pre-processing}
\label{subsubsec:report_pre_processing}

\begin{figure*}[!htb] 
  \centering
  \includegraphics[width=\linewidth]{figs/system.png}
  \caption{
  The interface of \system{}. 
  \system{} consists of four views: data view (b-c), dependency view (d-e), content view (f-g), and generation view (h-k). 
  The data view displays the overall description and data field information. 
  The dependency view displays the extracted interdependent report segments. 
  The content view shows the analytical objective and content of the selected segment. 
  The generation view demonstrates the generated results in real-time. 
  }
  \label{fig:interface}
\end{figure*}

Based on the analysis workflow formulation in~\autoref{subsec:problem_formulation}, our goal is to deduce the analysis segments and their dependencies from the original report for subsequent execution. 
Our preliminary study showed that most reports present analytical content in distinct segments, each focused on a single objective, with related text and visuals grouped together. 
Therefore, ideally, we can find a segmentation that aligns each report section with a specific analysis segment. 
In this light, \system{} should segment the report accordingly, extract the analytical objectives for each segment, and deduce their dependencies.

To achieve this goal, an appropriate report segmentation criteria is very important, as it directly determines the entire analysis workflow (consisting of a sequence of analysis segments). 
The accuracy of segmentation also affects the quality of the extracted analytical objectives and dependencies.

We were initially inspired by prior work in automatic storytelling and insight-mining, which formalizes data insights and their relationships~\cite{ma2023insightpilot, wang2019datashot}. 
For example, Calliope~\cite{shi2020calliope} defines a data story as a series of interrelated insights, with each insight describing data patterns in specific data fields and subsets. 
For instance, ``The average worldwide gross for action movies is increasing over time'' describes an increasing trend measuring ``average (worldwide gross)'' over the breakdown ``year'' within the subset ``genre = action''. 
Based on this definition, we can potentially segment the report by identifying the insight type, measure, breakdown, subset, etc., and combine the insights into segments based on these labels. 
However, we found this definition challenging for segmenting practical data reports, as it doesn't accommodate the flexibility of analyses that involve deeper data transformations, such as creating new fields and describing patterns in derived variables.

Therefore, we need to define new segmentation criteria that accommodate the flexible analysis in the data report. 
Instead of formally defining a ``segment'' or an ``analytical objective'' with a strict data model, we provide a loose description of how segments can be divided and use LLMs to perform segmentation. 
Our preliminary study indicated that most report segments consist of continuous text and possibly a chart.
Based on the study, we execute report segmentation, extract analytical objectives, and deduce the dependencies between segments through the following approach: 
\begin{itemize}
    \item \textbf{Match. } 
    First, for each chart, we match the related paragraph text to form a segment. 
    Based on our preliminary study, we assume that (1) each paragraph corresponds to the nearest preceding or following chart, or none at all, and (2) all text associated with a single chart is continuous. 
    Starting with the first paragraph, LLMs determine whether it matches the nearest preceding or following chart (e.g., describing insights from the chart) or if it doesn't relate to any chart.
    
    \item \textbf{Categorize. } 
    For text that doesn't match a chart, LLMs categorize it to determine if it involves data analysis or serves another purpose, such as providing background information. 
    For continuous text segments that involve data analysis, we further assess whether they belong to the same segment (describe insights derived from the same transformed data). 
    
    \item \textbf{Summarize. } 
    After matching and categorizing, LLMs summarize the analytical objective of each segment and deduce its dependencies with previous segments. 
    We use the six logical relations defined in Calliope to outline dependencies among report segments. 
    LLMs determine whether new content is logically connected to an existing segment or originates directly from the data.
\end{itemize}


\subsubsection{Relevant Report Retrieval}
\label{subsubsec:report_retrieval}


With the pre-processed data and reports, users can select a reference report to analyze the target dataset. 
Based on findings from our preliminary study, various aspects of existing data reports, such as analytical objectives and report content, can serve as helpful reference material. 
However, since the reference report's data may differ from the target dataset, adjustments are necessary to align with the new dataset. 
The closer the reference report's data is to the target dataset, the more aspects can be reused without modification, making the report more suitable for use as a reference.


To facilitate the retrieval of suitable reports, we aim to identify the reports with data similar to the target dataset. 
The core idea is to convert both the dataset and report information into vector embeddings, compute their cosine similarities, and rank the reports from highest to lowest score.
The key question is: which specific information from the dataset and reports should be embedded?
We propose two mechanisms for extracting the embedding of data and report information: topic relevance and field similarity.
\begin{itemize}
    \item \textbf{Topic relevance} refers to the alignment between the topic of the dataset and the report. 
    Typically, datasets and reports within the same domain (e.g., health, economy) exhibit higher topic relevance. 
    For example, a sales dataset is highly topic-relevant to a report analyzing market sales trends. 
    To compute topic relevance, we extract the embedding of the dataset’s name and description, along with the headings and pre-processed analytical objectives of the report. 
    We hypothesize that these elements are semantically related to the corpus's overall topic, and the cosine similarity of their embeddings can reflect their topic relevance.
    \item \textbf{Field similarity} pertains to the alignment of the data fields described in the report with those contained in the dataset. 
    For example, a report on voting intentions across different gender and age groups would exhibit higher field similarity with a dataset containing gender and age information. 
    To compute field similarity, we embed the names and descriptions of the dataset's fields. 
    For the reports, we use LLMs to infer the data fields discussed in the report and embed these inferred fields along with their descriptions. 
    We hypothesize that the cosine similarity between these reflects the degree of field similarity.
\end{itemize}
Finally, we sum the scores from both mechanisms for each report, ranking them from highest to lowest, allowing users to select the most appropriate reference reports.

\subsection{Analysis Stage}
\label{subsec:analysis_stage}

Through the pre-processing stage, we obtain the summary of the new dataset and the segments of the existing report. 
Each segment corresponds to an analytical objective, a dependency on the previous segment or the data, and pieces of report content, including text and charts. 
The next stage is to reproduce the analysis workflow by re-executing each segment on the new data, encompassing reusing and reconstructing the analytical objective, analysis operations, and report contents. 

\subsubsection{Analytical objective correction and insertion}

The analysis workflow is driven by a series of posed analysis objectives. 
Through the pre-processing stage, we obtain the analysis workflow of the existing report by dividing it into segments and extracting each segment's analytical objectives and dependencies. 
To adapt the workflow to new data, \system{} is required to (1) correct the extracted analytical objectives and (2) support the insertion of new objectives according to the data features and segment dependencies. 

\textbf{Analytical objective correction. }
The preliminary study indicates that while existing analytical objectives often remain applicable, alterations or removals may be necessary due to data fields, dependencies, or the data context and scope. 

First, the original objective might involve data fields absent in the new data. 
Given the pre-processed data summary, we employ LLMs to evaluate if the new dataset's fields sufficiently fulfill the objective, considering semantic similarities despite word-to-word differences in field names. 
For example, an objective mentioning ``earn money'' can be related to the data field ``gross.'' 
LLMs are tasked with explaining their decisions to enhance their reasoning~\cite{mialon2023augmented}. 
If the available fields suffice, LLMs should describe the required fields and analysis operations. 
Otherwise, LLMs must explain what external fields are needed to satisfy the objective. 
In such cases, we correct the objective by replacing missing fields with available alternatives. 
For instance, if a movie dataset lacks geographic data, the objective of locating the highest-grossing movies might shift focus to their directors.

Second, the original objective may derive from insights in a prior segment. 
Adjustments might stem from two scenarios. 
If the insight's nature changes (e.g., from an increasing to a decreasing trend), a corresponding shift is needed in the latter related objective (e.g., from identifying causes of growth to exploring reasons behind the downturn). 
Therefore, we provide the model with the newly generated results of the dependent segment and require it to identify and adapt to such variations. 
Additionally, the dependency may call for a context or scope that the data cannot satisfy, such as from local to national or from a 5-year trend to a 20-year trend. 
LLMs must infer whether changes in scope or context affect the objective's applicability, which could lead to its potential removal if the new data does not support similar adjustments. 

Third, minor adjustments are often required for data context and scope adjustments. 
For example, an objective focusing on a 5-year trend needs adjustment to fit a dataset covering only the past three years. 
LLMs should make these modifications based on the context and scope of the provided data.

\textbf{Analytical objective insertion. }
The uniqueness of new datasets and user-driven queries may necessitate adding fresh analytical objectives based on previous insights and dependencies. 
\system{} enables users to embed new objectives at chosen positions, rooted in the data or reliant on preceding analysis segments. 
Users can define the focus data fields and dependencies of these new objectives, and LLMs can suggest potential objectives based on user input.

\subsubsection{Analysis Operation Generation}

% \TODO{code structure, generate a table and a chart}. 
Once the analytical objective has been refined, \system{} generates the requisite analysis operations to fulfill this objective. 
Since these operations are not explicitly detailed in the report, we utilize the code-generation capabilities of LLMs for this phase. 

LLMs are prompted to generate analysis code that aligns with the clarified objective, provided with the data summary and original report content as guides. 
The model is instructed to refer to the original report to deduce the necessary data transformations. 
We also remind the model to generate the code that accommodates the new data, as the reference report content is from a different dataset and only serves as a reference for expected output. 
The model is required first to plan step by step~\cite{kojima2022large} and then generate the Python code that results in transformed data and a chart using matplotlib~\cite{Hunter2007matplotlib} or Seaborn~\cite{Waskom2021seaborn}. 

Upon code generation, we execute it to procure the transformed data and the accompanying chart. 
The execution may also raise errors. 
We relay any execution results, including the transformed data, chart, and potential errors, back to the LLMs. 
The model then assesses whether the code execution is successful and whether the results accurately address the analytical objective and are adequate for generating report content. 
Should the model deduce that revisions are necessary, the cycle of code generation and execution is repeated until satisfactory results are obtained, paving the way for report content creation.

\subsubsection{Report Content Production}

With the execution results in hand, we proceed to generate new report content. 
Given that the code already produces the chart, the model's task in this step is to generate the accompanying textual narrative. 
We instruct the model to produce a narrative that imitates the writing style of the reference report yet is tailored to fit the new data context and the insights derived from the executed analysis. 
We also enable user modification to the report content. 

\subsection{Organization Stage}

After reproducing the analysis workflow and obtaining the new data insights, the next step is to structure the new report. 
As we generate the sequence of segments based on the order of dependencies, the implicit logical structure is adopted naturally. 
Additionally, we inherit the explicit structural elements (such as titles and sections) from the original report. 
Newly inserted analytical objectives are incorporated along with their dependent segments. 
The report and its sections' titles are re-crafted based on the original ones, incorporating new data insights to guide the title generation process. 
User interventions are also supported, allowing for the reorganization of segments into new sections, thereby tailoring the report structure to meet user needs or preferences better.



\section{NovelSpecies Dataset}
\label{sec:novel_dataset}

Proprietary LMMs like GPT4o~\cite{hurst2024gpt4o} and Gemini~\cite{team2023gemini} are trained on vast online text-image data and proprietary data, both non-public and impossible to inspect. Some open-source and open-data LMMs such as LLaVA~\cite{liu2024improved, liu2024visual} are trained on publicly available image-text datasets. However, the text encoders used by such models are often not open-data, for example LLaVA-1.6 34B uses the closed-data Yi-34B model as its language backbone. Even in the rare cases where both image-text training data and text encoder training data are publicly available, it is still difficult to ascertain whether concepts in your benchmark were seen by your LMM through indirect data leakage (i.e. partial / paraphrased mentions). Due to the above issues, it is difficult to evaluate true novel concept recognition ability with existing datasets. 
% \footnote{Knowledge cutoff date: Dec 2023}

One way to bypass this problem with 100\% guaranteed success is to use a dataset that only contains concepts created / discovered after the LMM's knowledge cutoff, i.e. the latest knowledge cutoff date among all of its textual / visual sub-components. Based on this idea, we curate \textbf{NovelSpecies}, a dataset of novel animal species discovered in each recent year, starting with 2023 and 2024. We provide detailed information for each species, including time of discovery, latin name, common name, family category, textual description, and more. Data will be released upon publication.
% Details are described in Sec.\ref{subsec:NovelSpecies_details}.

To create \textbf{NovelSpecies}, we start by collecting the list of species first described in each year by Wikidata~\cite{wikidata}. Then, to make sure we can curate a visual benchmark of novel species, we manually annotate and filter out extinct species and species with too few publicly available images. After filtering, we end up with a dataset of 64 new species, each consisting of 35 human-verified image instances, thus a total of 2240 images. The images are split into training, validation, and test sets. For each specie, there are 5 training images, 15 validation images, and 15 test images. This data split is consistent with our goal of creating a benchmark dataset for novel concept recognition, where the maximum number of training instances for a completely unseen concept can range from 1 to 5.







% and 2170 images in total, which consist of train, validation, and test sets of equal proportion for all species. Finally, all the images are 















% \section{Datasets}
% \label{sec:dataset}


% \subsection{Confusing Pair Extraction}
% Our focus on confusing pairs arises from the need to strengthen the model's performance in distinguishing between visually similar species—a challenge where LLaVA currently shows limitations. Confusing pairs represent instances where the model's classification often fails, typically due to subtle visual cues or shared features among species within similar taxonomic groups. We designed strategy to extract confusing pair for each dataset.

% \paragraph{INaturalist and Novel Species Dataset} We extract confusing pairs with three-steps as following: 

% \begin{enumerate}
%     \item \textbf{Iterative Subset Selection:} We select a random subset of species in each iteration, sampling across different supercategories. This strategy allows us to identify confusing pairs without overloading the system, progressively building a collection of challenging cases from each subset.
%     \item \textbf{Evaluate Classification Patters:} For each species within a subset, we create prompts in a multiple-choice format, incorporating the image and a randomized list of options from all the species in the subset. Based on the response from LLaVA, we are able to highlight specific species that are commonly mistaken for one another, guiding us in selecting pairs for further analysis. The process is repeated across new subsets, incrementally building an ample dataset of confusing pairs.
%     \item \textbf{Identification of confusing pairs: } We choose a threshhold of 0.2. If class A is misclassified into class B with frequency more than 0.2 in the above multiple-choice setting, we consider the pair to be confusing. 
% \end{enumerate}

% \paragraph{SUN Dataset} We adapted the above methodology for scene classification with minor modification on the subset selection process. Instead of taxonomic groupings, we created subsets by selecting a target scene and the nine most similar scenes based on shared object occurrence. The subsequent steps—classification pattern analysis and confusing pair definition—remained consistent with the species datasets.









% \subsection{Curated INaturalist Dataset}
% In this study, we utilize a random sample of 15 classes from the "Mammals" supercategory of the iNaturalist dataset. Below, we outline the reasoning behind our dataset selection and sampling approach.
% \paragraph{iNaturalist Dataset}
% The iNaturalist dataset is known for its complexity and has proven to be a challenging benchmark for many vision-language models. Due to the extensive diversity and fine-grained nature of the categories, most models do not achieve perfect performance on this dataset, leaving ample room for further improvements.
% \paragraph{Sampling Strategy}
% Given the scale of the iNaturalist dataset, which contains approximately 10,000 classes with 50 images per class, it is necessary to reduce its size for practical purposes. Additionally, current models, such as LLaVA, have limitations in handling an excessive number of options. Therefore, we have opted to sample the dataset to manage the number of classes and reduce the computational load.
% \paragraph{Random Sampling Justification}
% Initially, we considered sampling all species from a single order, family, or genus. However, this approach resulted in classes that were too similar, making the classification task more challenging than our models could handle. By employing random sampling, the selected classes that are likely sufficiently distinct from each other, with only a few potentially confusing cases.

% Random sampling also reduces the risk of introducing human bias into the selection process, making it a more defensible approach compared to sampling based on performance metrics. 

% \subparagraph{Data Filtering}
% iNaturalist dataset contains a large number of noisy or low quality images. To ensure the quality of the dataset, we implemented an automatic filtering process to eliminate low-quality images. This step is crucial to prevent noise from negatively impacting model performance. Common issues in low-quality images include:

% 1. \textbf{Blurriness}: Images where the main subject is not in focus.
% 2. \textbf{Species Not Present}: Instances where the species is not visible (e.g., only showing its nest or footprint).
% 3. \textbf{Incomplete Specimen}: Images depicting only parts of deceased animals or broken bodies.
% 4. \textbf{Obstructions}: Cases where the species is almost entirely blocked by objects, making identification impossible.

% To improve image quality, we use CLIP score to select the images with top scores. Scores are calculated by evaluating similarity score with [
%         "a photo of an animal",
%         f"a photo of a \{common\_name\}"
%     ]. We rank the images according to this score and selected top 100 images. We randomly split the images to obtain 50 images for training, 20 images for validation and 30 images for testing. 



\subsection{Baseline Evaluation}  

\textbf{Experimental Setup.}  
We train all baseline models using the AdamW optimizer with Focal Loss. TCN-based models, including \emph{Neural + TCN}, have a receptive field of 8 minutes, sufficient to capture \emph{CE} patterns in the 5-minute training data. Early stopping is applied based on validation loss, and results are averaged over 10 random seeds. Additional training details are provided in Appendix~\ref{sec:baseline-training}. The Pretrained Feature Encoder and the \emph{Neural AE} classifier used in the experiment are described in Appendix~\ref{sec:pretrained-encoder}.


\textbf{Metrics.}  
We evaluate performance using the $F1$ score for each \emph{CE} class $e_i$ and report two aggregated scores:  
\begin{enumerate}[leftmargin=1.5em,nosep]
    \item \emph{Macro $F1$} ($F1\_all$): The unweighted average $F1$ across all classes ($e_0$ to $e_{10}$).
    \item \emph{Positive $F1$} ($F1\_pos$): The average $F1$ over positive event classes ($e_1$ to $e_{10}$), excluding the less important ``negative'' label $e_0$. This serves as our key metric.
\end{enumerate}  
A higher $F1$ score indicates better precision-recall balance, reflecting both correctness and completeness.  



\textbf{Results.}  
We evaluate model performance across different training set sizes, as shown in Fig.~\ref{fig:ce_different_trainingsizes_boxplot}. The results indicate that \emph{\textbf{Mamba achieves the best performance}}, followed by LSTM. The \emph{Neural + X} models underperform compared to end-to-end models, likely due to errors and noise introduced by the \emph{Neural AE} classifier. This also explains why the \emph{Neural AE} + FSM model, despite incorporating correct human-written complex event rules, performs worse. Detailed $F1$ scores for each \emph{CE}, including per-class \emph{F1} scores, are provided in Table~\ref{tab:baseline-results}.


Additionally, we test model generalization on {out-of-distribution (OOD) complex events lasting 15 and 30 minutes, following the same \emph{CE} rules but with extended temporal spans. As shown in Table~\ref{tab:baseline_different_temporal_span}, Mamba generalizes better than LSTM to unseen test data. \emph{\textbf{Training with more labeled sensor data improves performance on 5-minute test data and enhances generalization to longer unseen traces.}} However, we still observe a performance drop as temporal span increases. Moreover, the data-hungry nature of these neural network baselines imposes significant real-world data collection and labeling costs.



% The AE + FSM model incorporates correct complex event rules; its performance declines greatly with longer traces due to cumulative errors from imperfect atomic event inference. While the Mamba model showed the best generalization on the OOD test sets, we still noted a performance drop as the temporal span increased. Additionally, the data required to train these neural network baselines incurs significant data collection and labeling costs in the real world.

% (Also add a figure of Mamba generalization ability w.r.t training size.) however, it's costly to infinitely increase the training data amount




% \begin{figure}[t]
%     \centering
% \includegraphics[width=0.95\columnwidth]{figs/ce_train_results.png}
%     \caption{Average F1 scores of models on complex events with different temporal spans.}
%     \label{fig:ce_train_results}
% \end{figure}

\begin{figure}[t]
    \centering
        \setlength{\abovecaptionskip}{0.cm}
    \setlength{\belowcaptionskip}{0.cm}
\includegraphics[width=0.95\columnwidth]{figs/ce_different_trainingsizes_boxplot.png}
    \caption{Positive $F1$ scores of models on complex events with different training data.}
    \label{fig:ce_different_trainingsizes_boxplot}
    \vspace{-1em}
\end{figure}

% \begin{figure}[t]
%     \centering
% \includegraphics[width=0.95\columnwidth]{figs/ce_different_trainingsizes_barplot.png}
%     \caption{(barplot) Average F1 scores of models on complex events with different training data.}
%     \label{fig:ce_different_trainingsizes_barplot}
% \end{figure}

\begin{table}[t]
    \centering
    \small
    \setlength{\tabcolsep}{4pt}
    % \setlength{\tabcolsep}{4pt} % Reduce space between columns
    \caption{Positive $F1$ scores with a 2-sigma confidence interval for Mamba and LSTM tested on 5-minute and OOD test sets with longer \emph{CE} temporal patterns.}
    \vskip 0.15in
    \begin{tabular}{c c c c c} % Ensure correct column count
        \toprule
        \textbf{Model} &  \textbf{Training} & \multicolumn{3}{c}{\textbf{Positive $F1$}}\\
        \cmidrule(lr){3-5}
         & \textbf{Data Size} & \textbf{5min} & \textbf{15min (OOD)} & \textbf{30min (OOD)} \\
        % \cmidrule(lr){3-5} % Horizontal line under merged column
        % \textbf{data size} & \textbf{data size} & \textbf{5min} & \textbf{15min} & \textbf{30min} \\
        \midrule
        Mamba  & 2000 & .75 $\pm$ .08 & .65 $\pm$ .09 & .51 $\pm$ .11 \\
               & 4000 & .85 $\pm$ .08 & .75 $\pm$ .11 & .66 $\pm$ .16 \\
               & 6000 & .89 $\pm$ .05 & .79 $\pm$ .07 & .69 $\pm$ .14 \\
               & 8000 & .90 $\pm$ .08 & .77 $\pm$ .09 & .70 $\pm$ .12 \\
               & 10000 & .89 $\pm$ .09 & .81 $\pm$ .06 & .73 $\pm$ .06 \\
        \midrule
        LSTM & 10000 & .88 $\pm$ .11 & .74 $\pm$ .17  & .65 $\pm$ .20 \\
        \bottomrule
    \end{tabular}
    \label{tab:baseline_different_temporal_span}
    \vskip -0.1in
\end{table}




\section{\narce{} Framework}
\subsection{Overview}
We propose \narce{}, a framework designed to reduce the need for labeled sensor data by decoupling complex event rule learning from sensor-specific variations, as hypothesized in \emph{Hypothesis II}. Inspired by the Neural Algorithmic Reasoning (NAR) paradigm, which uses Graph Neural Networks (GNNs) to represent and learn algorithms by using symbolic algorithm input-output pairs for training, in \narce{}, we analogously treat each complex event rule as a type of algorithm and leverage Mamba, a state-space model well-suited for long-range dependencies, to learn them.

To achieve this, \narce{} follows a two-stage training process, as shown in Fig.~\ref{fig:narce_overview}. In \textbf{Stage 1}: It learns complex event rules from synthetic concept traces without using sensor data. In \textbf{Stage 2}: It adapts to real sensor data by training a Sensor Adapter that maps sensor embeddings into the latent space of the pretrained \emph{CE} NAR.

\subsection{Stage I: Training \emph{CE} NAR on Concept Traces}
In this stage, we train the \textbf{Mamba-based \emph{CE} NAR} to learn complex event rules independently of sensor data. Instead of using sensor sequences, we generate pseudo \emph{AE} concept traces, sequences of \emph{AE}, governed by the same \emph{CE} rules we aim to detect. These traces are produced by an \textbf{LLM-based Synthesizer}, which simulates human activity sequences in \emph{5-second} windows by: (1) structuring activities into \emph{semantic groups} (e.g., hygiene, work, restroom); (2) defining \emph{probabilistic transitions} between groups and atomic events; (3) assigning \emph{variable durations} for each group and atomic event; and (4) dynamically adjusting \emph{transition probabilities} to ensure the presence of target complex events.

Since LLMs struggle with complex event reasoning, we do not rely on them for online \emph{CE} labeling. Instead, we use FSMs to generate online \emph{CE} labels from concept traces, ensuring reliable supervision. A case study on this limitation is provided in Appendix~\ref{sec:llm_eval}. Detailed prompts used for the LLM synthesizer are provided in Appendix~\ref{sec:llm_synthesizer}.

Once generated, the \emph{AE} concept traces are tokenized and paired with \emph{CE} labels to train the \emph{CE} NAR. Specifically, we:
\begin{enumerate}[leftmargin=1em,nosep]
    \item Tokenize \emph{AE} traces with the \textbf{\emph{AE} Tokenizor}, using a lookup vocabulary table.
    \item Pass tokens through a learnable embedding matrix, the \textbf{Embedding Encoder $f$}, which maps the tokens to a 128-dimensional latent space.
    \item Train the \emph{CE} NAR, a 12-layer Mamba model, identical to the baseline Mamba, using Focal Loss (FL)~[\ref{eq:fl}].
\end{enumerate}

After training, the \emph{CE} NAR is frozen and used as a reasoning module in Stage 2.


\subsection{Stage II: Training the Sensor Adapter}
Once the \emph{CE} NAR is trained on concept traces, we adapt it to real sensor data by training a \textbf{Sensor Adapter $f'$}. The goal of the Sensor Adapter is to map raw sensor embeddings into the latent space of NAR, allowing it to process sensor inputs while preserving the learned event reasoning capabilities.
To achieve this:
\begin{itemize}[leftmargin=1em,nosep]
    \item The Embedding Encoder from Stage 1 is removed, and the Sensor Adapter is introduced.
    \item The \emph{CE} NAR remains frozen, ensuring that the event reasoning remains intact.
    \item The Sensor Adapter is trained using labeled sensor data with online \emph{CE} labels to learn the mapping.
\end{itemize}

We use a 6-layer Mamba block for the Sensor Adapter, though other neural network models can be used. This setup enables online CE detection, allowing the model to infer complex event occurrences from raw sensor streams.


% We bring out \narce{}, as shown in Fig.~\ref{fig:narce_overview}, a framework that aims to reduce the data hunger for labeled sensor data. We hypothesize in \emph{Hypothesis II} that decoupling the learning of complex event rules from noisy sensor data improves data efficiency. \narce{} draws insights from the Neural Algorithmic Reasoning (NAR) paradigm (cite), which uses neural networks, particularly Graph Neural Networks (GNNs), to represent and learn algorithms by using symbolic algorithm input-output pairs for training. Analogously, in \narce{}, we treat each complex event rule as one type of algorithm and use a well-suited backbone model for CED, Mamba, to learn those rules. \narce{} contains a \emph{CE} NAR, which is a Mamba-based NAR, which leverages synthetic concept traces to learn \emph{CE} rules. Concept traces are sequences of atomic events governed by the same \emph{CE} rules we aim to detect generated using large language models (LLMs),  representing abstracted event sequences that encode rule-driven patterns without noisy sensor data. 

% \subsection{Framework Overview}
% Figure~\ref{fig:narce_overview} illustrates the two-stage process of \narce{}. In Stage 1, we pretrain the Mamba-Based \emph{CE} NAR using pseudo AE concept traces—sequences of atomic events generated by an LLM Synthesizer based on predefined complex event rules. These traces are tokenized and paired with corresponding online CE labels to train the embedding encoder and the NAR model, independent of sensor data.

% In Stage 2, we freeze the trained CE NAR and replace its embedding encoder with a Sensor Adapter $f'$. This adapter is trained to map real sensor embeddings into the latent space where the CE NAR operates, allowing the model to reason over real-world sensor inputs and produce online CE labels. This two-stage approach ensures the model first learns robust event reasoning before adapting to sensor-specific variations.


% \subsection{Generating \emph{AE} Concept Trace Dataset}
% To generate pseudo \emph{AE} concept traces for training the \emph{CE} NAR, we use an LLM-based synthesizer that mimics human activity sequences in 5-second windows. The LLM generates simulation codes by (1) structuring activities into semantic groups (e.g., hygiene, work, restroom) with related \emph{AE}s, (2) defining probabilistic transitions between semantic groups and \emph{AE} to maintain natural behavior, (3) assigning variable durations for each group and \emph{AE}, and (4) dynamically adjusting transition probabilities to ensure the presence of target complex events while introducing slight noise for diversity. Users specify a target event scenario and a predefined \emph{AE} set to guide the simulation. Full details, including the exact prompt, are provided in Appendix~\ref{sec:llm_synthesizer}.

% One last thing is the online \emph{CE} labels. We use the prementioned FSMs to generate \emph{CE} labels. The reason why we do not also use LLM is due to the concern that LLMs cannot do CE reasoning reliably. We also did a small case study showing that LLMs are not reliable in our attempt, see Appendix~\ref{sec:llm_eval}.



% \subsection{Training \emph{CE} NAR}
% \textbf{\emph{AE} Tokenizor \& Embedding Encoder.}
% Before feeding AE traces to \emph{CE} NAR, we tokenize the \emph{AE} concept traces into token traces by creating a lookup vocabulary table for each \emph{AE}. The tokens are then passed an Embedding Encoder, which is a learnable embedding matrix that maps the tokens to the high-dimensial latent space. The dimension is 128.

% \textbf{Mamba-based \emph{CE} NAR.}
% We use a 12-layer Mamba,  same as the baseline Mamba models, to serve as the backbone of the NAR. The training loss is also the same Focal Loss (FL)[\ref{eq:fl}].

% \subsection{Training Sensor Adapter}
% We freeze the \emph{CE} NAR after training it with pseudo \emph{AE} concept traces. The sensor adapter, which aims to maps from sensor embedding to the latent reasnignspace of the NAR is trained using labeled sensor data. We use a 6-layer Mamba block, but it can be replaced by any other NN models. 



% We bring out \narce{}, as shown in Fig.~\ref{fig:narce_overview}, a framework that aims to reduce the data hunger for labeled sensor data. We hypothesize in \emph{Hypothesis II} that decoupling the learning of complex event rules from noisy sensor data improves data efficiency. \narce{} draws insights from the Neural Algorithmic Reasoning (NAR) paradigm (cite), which uses neural networks, particularly Graph Neural Networks (GNNs), to represent and learn algorithms by using symbolic algorithm input-output pairs for training. Analogously, in \narce{}, we treat each complex event rule as one type of algorithm and use a well-suited backbone model for CED, Mamba, to learn those rules. \narce{} contains a \emph{CE} NAR, which is a Mamba-based NAR, which leverages synthetic concept traces to learn \emph{CE} rules. Concept traces are sequences of atomic events governed by the same \emph{CE} rules we aim to detect generated using large language models (LLMs),  representing abstracted event sequences that encode rule-driven patterns without noisy sensor data. 



% Decoupling the CE Learning Task
% NARCE simplifies CE detection by decoupling the task into two stages:

% Learning CE Rules from Concept Traces: The reasoning module (Mamba-based NAR) is pretrained using synthetic concept traces to learn CE rules.
% Mapping Sensor Data to CE Rules: An encoder is trained to project sensor data embeddings into the latent space of the pretrained reasoning module.

% Key Claims of the NARCE Framework.
% Concept Samples Are Easier to Obtain and Verify
% Generating and verifying concept (or abstract) traces is significantly more efficient than collecting and annotating large sensor datasets. For example, consider the complex event rule: “The same person carrying a handbag appears nearby three times in an hour.” A corresponding concept trace for a fixed camera sampling rate might look like:

% ["None", "None", "pedestrian A with a handbag", "None", "pedestrian B", "None",  "pedestrian A with a handbag", "None", "None", "pedestrian A with a handbag", ...]
% Concept traces, such as the one above, provide structured and interpretable examples of rule-based behaviors. Using LLMs, we can efficiently generate simulation code for a given complex event rule, producing numerous concept samples for training. These synthetic samples are cheaper to obtain and easier to verify than real-world sensor data, which is often noisy, expensive to annotate, and time-intensive to collect.

% Decoupling the CE Learning Task Simplifies the Problem
% NARCE simplifies the original CE learning task by decoupling it into two stages:

% Stage 1: Learning CE Rules from Concept Traces
% Mamba-based NAR is pretrained on synthetic concept traces to learn latent representations of CE rules. This stage abstracts away the need for raw sensor inputs, focusing solely on understanding the logic and structure of complex events.
% Stage 2: Mapping Sensor Data to NAR Latent Space
% A separate encoder is trained to project sensor data embeddings into the latent space of the pretrained NAR. This two-stage approach allows the reasoning module to generalize across diverse rules while reducing the complexity of sensor data interpretation.
% This decoupled approach enables the framework to leverage the efficiency of pretrained NAR models and ensures that the sensor data processing does not interfere with the core task of rule reasoning. Furthermore, the pretrained NAR module serves as a reusable reasoning core that can generalize across multiple CE detection tasks.
\section{Evaluations}
\label{sec:experiment}

In this section, we demonstrate that \sassha can indeed improve upon existing second-order methods available for standard deep learning tasks.
We also show that \sassha performs competitively to the first-order baseline methods.
Specifically, \sassha is compared to AdaHessian \citep{adahessian}, Sophia-H \citep{sophia}, Shampoo \cite{gupta2018shampoo}, SGD, AdamW \citep{loshchilov2018decoupled}, and SAM \citep{sam} on a diverse set of both vision and language tasks.
We emphasize that we perform an \emph{extensive} hyperparameter search to rigorously tune all optimizers and ensure fair comparisons.
We provide the details of experiment settings to reproduce our results in \cref{app:hypersearch}.
The code to reproduce all results reported in this work is made available for download at \url{https://github.com/LOG-postech/Sassha}.

\subsection{Image Classification}
\begin{table*}[t!]
    \vspace{-0.5em}
    \centering
    \caption{Image classification results of various optimization methods in terms of final validation accuracy (mean$\pm$std).
    \sassha consistently outperforms the other methods for all workloads.
    * means \emph{omitted} due to excessive computational requirements.}
    
    \vskip 0.1in
    \resizebox{0.8\linewidth}{!}{
        \begin{tabular}{clcccccc}
        \toprule
         & 
         & \multicolumn{2}{c}{CIFAR-10} 
         & \multicolumn{2}{c}{CIFAR-100} 
         & \multicolumn{2}{c}{ImageNet} \\
         \cmidrule(l{3pt}r{3pt}){3-4} \cmidrule(l{3pt}r{3pt}){5-6} \cmidrule(l{3pt}r{3pt}){7-8}
         \multicolumn{1}{c}{ Category }
         & \multicolumn{1}{c}{ Method }
         & \multicolumn{1}{c}{ ResNet-20 } 
         & \multicolumn{1}{c}{ ResNet-32 } 
         & \multicolumn{1}{c}{ ResNet-32 }  
         & \multicolumn{1}{c}{ WRN-28-10} 
         & \multicolumn{1}{c}{ ResNet-50 } 
         & \multicolumn{1}{c}{ ViT-s-32} \\ \midrule

        
       \multirow{4}{*}{First-order}  
       & SGD       & 
         $ 92.03 _{ \textcolor{black!60}{\pm 0.32} } $    &
         $ 92.69 _{\textcolor{black!60}{\pm 0.06} }  $    &
         $ 69.32 _{\textcolor{black!60}{\pm 0.19} }  $    &
         $ 80.06 _{\textcolor{black!60}{\pm 0.15} }  $    &
         $ 75.58 _{\textcolor{black!60}{\pm 0.05} }  $    &
         $ 62.90 _{\textcolor{black!60}{\pm 0.36} }  $   \\

        & AdamW      & 
        $ 92.04 _{\textcolor{black!60}{\pm 0.11} }  $     &
        $ 92.42 _{\textcolor{black!60}{\pm 0.13} }  $     &
        $ 68.78 _{\textcolor{black!60}{\pm 0.22} }  $     &
        $ 79.09 _{\textcolor{black!60}{\pm 0.35} }  $     &
        $ 75.38 _{\textcolor{black!60}{\pm 0.08} }  $     &
        $ 66.46 _{\textcolor{black!60}{\pm 0.15} }  $    \\
        
        & SAM $_{\text{SGD}}$  &
        $ 92.85 _{\textcolor{black!60}{\pm 0.07} }  $    &
        $ 93.89 _{\textcolor{black!60}{\pm 0.13} }  $    &
        $ 71.99 _{\textcolor{black!60}{\pm 0.20} }  $    &
        $ 83.14 _{\textcolor{black!60}{\pm 0.13} }  $    &
        $ 76.36 _{\textcolor{black!60}{\pm 0.16} }  $    &
        $ 64.54 _{\textcolor{black!60}{\pm 0.63} }  $    \\
        
        & SAM $_{\text{AdamW}}$  &
        $ 92.77 _{\textcolor{black!60}{\pm 0.29} }  $    &
        $ 93.45 _{\textcolor{black!60}{\pm 0.24} }  $    &
        $ 71.15 _{\textcolor{black!60}{\pm 0.37} }  $    &
        $ 82.88 _{\textcolor{black!60}{\pm 0.31} }  $    &
        $ 76.35 _{\textcolor{black!60}{\pm 0.16} }  $    &
        $ 68.31 _{\textcolor{black!60}{\pm 0.17} }  $    \\

        \midrule
        
        \multirow{4}{*}{Second-order} &
        AdaHessian &
        $ 92.00 _{\textcolor{black!60}{\pm 0.17} } $  &
        $ 92.48 _{\textcolor{black!60}{\pm 0.15} } $  &
        $ 68.06 _{\textcolor{black!60}{\pm 0.22} } $  &
        $ 76.92 _{\textcolor{black!60}{\pm 0.26} } $  &
        $ 73.64 _{\textcolor{black!60}{\pm 0.16} } $  &
        $ 66.42 _{\textcolor{black!60}{\pm 0.23} } $  \\
        
        & Sophia-H   & 
        $ 91.81 _{\textcolor{black!60}{\pm 0.27} } $  &
        $ 91.99 _{\textcolor{black!60}{\pm 0.08} } $  &
        $ 67.76 _{\textcolor{black!60}{\pm 0.37} } $  & 
        $ 79.35 _{\textcolor{black!60}{\pm 0.24} } $  & 
        $ 72.06 _{\textcolor{black!60}{\pm 0.49} } $  &
        $ 62.44 _{\textcolor{black!60}{\pm 0.36} } $  \\
        
        & Shampoo    & 
        $ 88.55 _ {\textcolor{black!60}{\pm 0.83}}$  &
        $ 90.23 _{\textcolor{black!60}{\pm 0.24}} $  &
        $ 64.08 _{\textcolor{black!60}{\pm 0.46}} $  &
        $ 74.06 _{\textcolor{black!60}{\pm 1.28}} $  &
        $*$                                          &
        $*$  \\
        
        \cmidrule(l{3pt}r{3pt}){2-8}
        
        \rowcolor{green!20} &
        \sassha    &
        $ \textbf{92.98} _{\textcolor{black!60}{\pm 0.05} }  $ &
        $ \textbf{94.09} _{\textcolor{black!60}{\pm 0.24} }  $ &
        $ \textbf{72.14} _{\textcolor{black!60}{\pm 0.16} }  $ & 
        $ \textbf{83.54} _{\textcolor{black!60}{\pm 0.08} }  $ &
        $ \textbf{76.43} _{\textcolor{black!60}{\pm 0.18} }  $ &
        $ \textbf{69.20} _{\textcolor{black!60}{\pm 0.30} }  $ \\
        
        \bottomrule
        \end{tabular}
    }
    \vskip 0.1in
    \label{tab:im_cls_results}
\end{table*}

\begin{figure*}[t!]
    \vspace{-0.5em}
    \centering
    \resizebox{0.8\linewidth}{!}{
    \includegraphics[width=0.325\linewidth]{figures/validation/Res32-CIFAR10-Acc.pdf}
    \includegraphics[width=0.325\linewidth]{figures/validation/WRN28-CIFAR100-Acc.pdf}
    \includegraphics[width=0.325\linewidth]{figures/validation/Res50-ImageNet-Acc.pdf}
    }
    \vspace{-0.5em}
    \caption{
    Validation accuracy curves along the training trajectory.
    We also provide loss curves in \cref{app:valloss}.
    }
    \label{fig:im_cls_results}
    \vspace{-0.7em}
\end{figure*}

\begin{table*}[ht!]
    \centering
    \caption{
    Language finetuning and pertraining results for various optimizers. For finetuning, \sassha achieves better results than AdamW and AdaHessian and compares competitively with Sophia-H. For pretraining, \sassha achieves the lowest perplexity among all optimizers.
    }
    \vskip 0.1in
    \resizebox{\linewidth}{!}{
        \begin{tabular}{lc}
            \toprule
             & \multicolumn{1}{c}{$\textbf{Pretrain} / $ GPT1-mini} \\
             \cmidrule(l{3pt}r{3pt}){2-2}
             & Wikitext-2 \\
             & \texttt{Perplexity}\\
            \midrule
            
            AdamW & $ 175.06 $ \\
            SAM $_{\text{AdamW}}$ & $ 158.06 $ \\
            AdaHessian & $ 407.69 $ \\
            Sophia-H & $ 157.60 $ \\
            
            \midrule 
            
            \rowcolor{green!20}
            \sassha &
            $ \textbf{122.40} $ \\
            
            \bottomrule
        \end{tabular}
        
        \begin{tabular}{|ccccccc}
            \toprule
                         \multicolumn{7}{|c}{ \textbf{Finetune} /  SqeezeBERT } \\
                         \cmidrule(l{3pt}r{3pt}){1-7}
                         SST-2 &  MRPC & STS-B & QQP & MNLI & QNLI & RTE \\
             \texttt{Acc} &  \texttt{Acc / F1}  & \texttt{S/P corr.} & \texttt{F1 / Acc} & \texttt{mat/m.mat} &  \texttt{Acc} &  \texttt{Acc} \\
            \midrule
            
            %AdamW         & 
            $ 90.29 _{\textcolor{black!60}{\pm 0.52}} $ 
            & $ 84.56 _{ \textcolor{black!60}{\pm 0.25} } $ / $ 88.99 _{\textcolor{black!60}{\pm 0.11}} $ 
            & $ 88.34 _{\textcolor{black!60}{\pm 0.15}} $ / $ 88.48 _{\textcolor{black!60}{\pm 0.20}} $ 
            & $ 89.92 _{\textcolor{black!60}{\pm 0.05}} $ / $ 86.58 _{\textcolor{black!60}{\pm 0.11}} $ 
            & $ 81.22 _{\textcolor{black!60}{\pm 0.07}} $ / $ 82.26 _{\textcolor{black!60}{\pm 0.05}} $ 
            & $ 89.93 _{\textcolor{black!60}{\pm 0.14}} $ 
            & $ 68.95 _{\textcolor{black!60}{\pm 0.72}} $  \\
    
            %SAM _{\text{AdamW}}   &
            $ \textbf{90.52} _{\textcolor{black!60}{\pm 0.27}} $ 
            & $ 83.25 _{\textcolor{black!60}{\pm 2.79}} $ / $ 87.90 _{\textcolor{black!60}{\pm 2.21}} $ 
            & $ 88.38 _{\textcolor{black!60}{\pm 0.01}} $ / $ 88.79 _{\textcolor{black!60}{\pm 0.99}} $ 
            & $ 90.26 _{\textcolor{black!60}{\pm 0.28}} $ / $ 86.99 _{\textcolor{black!60}{\pm 0.31}} $ 
            & $ 81.56 _{\textcolor{black!60}{\pm 0.18}} $ / $ \textbf{82.46} _{\textcolor{black!60}{\pm 0.19}} $ 
            & $ \textbf{90.38} _{\textcolor{black!60}{\pm 0.05}} $ 
            & $ 68.83 _{\textcolor{black!60}{\pm 1.46}} $  \\
    
            %AdaHessian    & 
            $ 89.64 _{\textcolor{black!60}{\pm 0.13}} $ 
            & $ 79.74 _{\textcolor{black!60}{\pm 4.00}} $ / $ 85.26 _{\textcolor{black!60}{\pm 3.50}} $ 
            & $ 86.08 _{\textcolor{black!60}{\pm 4.04}} $ / $ 86.46 _{\textcolor{black!60}{\pm 4.06}} $ 
            & $ 90.37 _{\textcolor{black!60}{\pm 0.05}} $ / $ 87.07 _{\textcolor{black!60}{\pm 0.05}} $ 
            & $ 81.33 _{\textcolor{black!60}{\pm 0.17}} $ / $ 82.08 _{\textcolor{black!60}{\pm 0.02}} $ 
            & $ 89.94 _{\textcolor{black!60}{\pm 0.12}} $ 
            & $ 71.00 _{\textcolor{black!60}{\pm 1.04}} $ \\
            
            % Sophia-H  &
            $ 90.44 _{\textcolor{black!60}{\pm 0.46}} $ 
            & $ 85.78 _{\textcolor{black!60}{\pm 1.07}} $ / $ 89.90 _{\textcolor{black!60}{\pm 0.82}} $ 
            & $ 88.17 _{\textcolor{black!60}{\pm 1.07}} $ / $ 88.53 _{\textcolor{black!60}{\pm 1.13}} $ 
            & $ 90.70 _{\textcolor{black!60}{\pm 0.04}} $ / $ 87.60 _{\textcolor{black!60}{\pm 0.06}} $ 
            & $ \textbf{81.77} _{\textcolor{black!60}{\pm 0.18}} $ / $ 82.36 _{\textcolor{black!60}{\pm 0.22}} $ 
            & $ 90.12_{\textcolor{black!60}{\pm 0.14}} $ 
            & $ 70.76 _{\textcolor{black!60}{\pm 1.44}} $  \\
            
            \midrule
            
            \rowcolor{green!20} 
            $ 90.44 _{\textcolor{black!60}{\pm 0.98}} $    &
            $ \textbf{86.28} _{\textcolor{black!60}{\pm 0.28}} $ / $ \textbf{90.13} _{\textcolor{black!60}{\pm 0.161}} $     &
            $ \textbf{88.72} _{\textcolor{black!60}{\pm 0.75}} $ / $ \textbf{89.10} _{\textcolor{black!60}{\pm 0.70}}  $     &
            $ \textbf{90.91} _{\textcolor{black!60}{\pm 0.06}} $ / $ \textbf{87.85}  _{\textcolor{black!60}{\pm 0.09}} $     &
            $ 81.61 _{\textcolor{black!60}{\pm 0.25}} $ / $ 81.71 _{\textcolor{black!60}{\pm 0.11}} $     &
            $ 89.85_{\textcolor{black!60}{\pm 0.20}} $    &
            $ \textbf{72.08} _{\textcolor{black!60}{\pm 0.55}} $  \\
            
            \bottomrule
        \end{tabular}
    }
    \vspace{-0.5em}
    \label{tab:language}
\end{table*}

We first evaluate \sassha for image classification on CIFAR-10, CIFAR-100, and ImageNet.
We train various models of the ResNet family \citep{he2016deep,zagoruyko2016wide} and an efficient variant of Vision Transformer \citep{beyer2022better}.
We adhere to standard inception-style data augmentations during training instead of making use of advanced data augmentation techniques \citep{devries2017improved} or regularization methods \citep{gastaldi2017shake}.
Results are presented in \cref{tab:im_cls_results} and \cref{fig:im_cls_results}.

We begin by comparing the generalization performance of adaptive second-order methods to that of first-order methods.
Across all settings, adaptive second-order methods consistently exhibit lower accuracy than their first-order counterparts.
This observation aligns with previous studies indicating that second-order optimization often result in poorer generalization compared to first-order approaches.
In contrast, \sassha, benefiting from sharpness minimization, consistently demonstrates superior generalization performance, outperforming both first-order and second-order methods in every setting.
Particularly, \sassha is up to 4\% more effective than the best-performing adaptive or second-order methods (\eg, WRN-28-10, ViT-s-32).
Moreover, \sassha continually surpasses SGD and AdamW, even when they are trained for twice as many epochs, achieving a performance margin of about 0.3\% to 3\%. 
Further details are provided in \cref{app:comp_fo_fair}.

Interestingly, \sassha also outperforms SAM.
Since first-order methods typically exhibit superior generalization performance compared to second-order methods, it might be intuitive to expect SAM to surpass \sassha if the two are viewed merely as the outcomes of applying sharpness minimization to first-order and second-order methods, respectively.
However, the results conflict with this intuition.
We attribute this to the careful design choices made in \sassha, stabilizing Hessian approximation under sharpness minimization, so as to unleash the potential of the second-order method, leading to its outstanding performance.
As a support, we show that naively incorporating SAM into other second-order methods does not yield these favorable results in \cref{app:samsophia}.
We also make more comparisons with SAM in \cref{sec:sassha_vs_sam}.

\subsection{Language Modeling}

Recent studies have shown the potential of second-order methods for pretraining language models.
Here, we first evaluate how \sassha performs on this task.
Specifically, we train GPT1-mini, a scaled-down variant of GPT1 \citep{radford2019language}, on Wikitext-2 dataset \citep{merity2022pointer} using various methods including \sassha and compare their results (see the left of \cref{tab:language}).
Our results show that \sassha achieves the lowest perplexity among all methods including Sophia-H \citep{sophia}, a recent method that is designed specifically for language modeling tasks and sets state of the art, which highlights generality in addition to the numerical advantage of \sassha.

We also extend our evaluation to finetuning tasks.
Specifically, we finetune SqueezeBERT \citep{iandola2020squeezebert} for diverse tasks in the GLUE benchmark \citep{wang2018glue}.
The results are on the right side of \cref{tab:language}.
It shows that \sassha compares competitively to other second-order methods.
Notably, it also outperforms AdamW---often the method of choice for training language models---on nearly all tasks.

\subsection{Comparison to SAM}\label{sec:sassha_vs_sam}

So far, we have seen that \sassha outperforms second-order methods quite consistently on both vision and language tasks.
Interestingly, we also find that \sassha often improves upon SAM.
In particular, it appears that the gain is larger for the Transformer-based architectures, \ie, ViT results in \cref{tab:im_cls_results} or GPT/BERT results in \cref{tab:language}.

We posit that this is potentially due to the robustness of \sassha to the block heterogeneity inherent in Transformer architectures, where the Hessian spectrum varies significantly across different blocks.
This characteristic is known to make SGD perform worse than adaptive methods like Adam on Transformer-based models \citep{zhang2024why}.
Since \sassha leverages second-order information via preconditioning gradients, it has the potential to address the ill-conditioned nature of Transformers more effectively than SAM with first-order methods.

To push further, we conducted additional experiments.
First, we allocate more training budgets to SAM to see whether it compares to \sassha.
% additionally compare \sassha to SAM with more training budgets.
The results are presented in \cref{tab:sam}.
We find that SAM still underperforms \sassha, even though it is given more budgets of training iterations over data or wall-clock time.
Furthermore, we also compare \sassha to more advanced variants of SAM including ASAM \citep{asam} and GSAM \citep{gsam}, showing that \sassha performs competitively even to these methods (\cref{app:samvariants_vs_sassha}).
Notably, however, these variants of SAM require a lot more hyperparameter tuning to be compared.


\section{Discussion}\label{sec:discussion}



\subsection{From Interactive Prompting to Interactive Multi-modal Prompting}
The rapid advancements of large pre-trained generative models including large language models and text-to-image generation models, have inspired many HCI researchers to develop interactive tools to support users in crafting appropriate prompts.
% Studies on this topic in last two years' HCI conferences are predominantly focused on helping users refine single-modality textual prompts.
Many previous studies are focused on helping users refine single-modality textual prompts.
However, for many real-world applications concerning data beyond text modality, such as multi-modal AI and embodied intelligence, information from other modalities is essential in constructing sophisticated multi-modal prompts that fully convey users' instruction.
This demand inspires some researchers to develop multimodal prompting interactions to facilitate generation tasks ranging from visual modality image generation~\cite{wang2024promptcharm, promptpaint} to textual modality story generation~\cite{chung2022tale}.
% Some previous studies contributed relevant findings on this topic. 
Specifically, for the image generation task, recent studies have contributed some relevant findings on multi-modal prompting.
For example, PromptCharm~\cite{wang2024promptcharm} discovers the importance of multimodal feedback in refining initial text-based prompting in diffusion models.
However, the multi-modal interactions in PromptCharm are mainly focused on the feedback empowered the inpainting function, instead of supporting initial multimodal sketch-prompt control. 

\begin{figure*}[t]
    \centering
    \includegraphics[width=0.9\textwidth]{src/img/novice_expert.pdf}
    \vspace{-2mm}
    \caption{The comparison between novice and expert participants in painting reveals that experts produce more accurate and fine-grained sketches, resulting in closer alignment with reference images in close-ended tasks. Conversely, in open-ended tasks, expert fine-grained strokes fail to generate precise results due to \tool's lack of control at the thin stroke level.}
    \Description{The comparison between novice and expert participants in painting reveals that experts produce more accurate and fine-grained sketches, resulting in closer alignment with reference images in close-ended tasks. Novice users create rougher sketches with less accuracy in shape. Conversely, in open-ended tasks, expert fine-grained strokes fail to generate precise results due to \tool's lack of control at the thin stroke level, while novice users' broader strokes yield results more aligned with their sketches.}
    \label{fig:novice_expert}
    % \vspace{-3mm}
\end{figure*}


% In particular, in the initial control input, users are unable to explicitly specify multi-modal generation intents.
In another example, PromptPaint~\cite{promptpaint} stresses the importance of paint-medium-like interactions and introduces Prompt stencil functions that allow users to perform fine-grained controls with localized image generation. 
However, insufficient spatial control (\eg, PromptPaint only allows for single-object prompt stencil at a time) and unstable models can still leave some users feeling the uncertainty of AI and a varying degree of ownership of the generated artwork~\cite{promptpaint}.
% As a result, the gap between intuitive multi-modal or paint-medium-like control and the current prompting interface still exists, which requires further research on multi-modal prompting interactions.
From this perspective, our work seeks to further enhance multi-object spatial-semantic prompting control by users' natural sketching.
However, there are still some challenges to be resolved, such as consistent multi-object generation in multiple rounds to increase stability and improved understanding of user sketches.   


% \new{
% From this perspective, our work is a step forward in this direction by allowing multi-object spatial-semantic prompting control by users' natural sketching, which considers the interplay between multiple sketch regions.
% % To further advance the multi-modal prompting experience, there are some aspects we identify to be important.
% % One of the important aspects is enhancing the consistency and stability of multiple rounds of generation to reduce the uncertainty and loss of control on users' part.
% % For this purpose, we need to develop techniques to incorporate consistent generation~\cite{tewel2024training} into multi-modal prompting framework.}
% % Another important aspect is improving generative models' understanding of the implicit user intents \new{implied by the paint-medium-like or sketch-based input (\eg, sketch of two people with their hands slightly overlapping indicates holding hand without needing explicit prompt).
% % This can facilitate more natural control and alleviate users' effort in tuning the textual prompt.
% % In addition, it can increase users' sense of ownership as the generated results can be more aligned with their sketching intents.
% }
% For example, when users draw sketches of two people with their hands slightly overlapping, current region-based models cannot automatically infer users' implicit intention that the two people are holding hands.
% Instead, they still require users to explicitly specify in the prompt such relationship.
% \tool addresses this through sketch-aware prompt recommendation to fill in the necessary semantic information, alleviating users' workload.
% However, some users want the generative AI in the future to be able to directly infer this natural implicit intentions from the sketches without additional prompting since prompt recommendation can still be unstable sometimes.


% \new{
% Besides visual generation, 
% }
% For example, one of the important aspect is referring~\cite{he2024multi}, linking specific text semantics with specific spatial object, which is partly what we do in our sketch-aware prompt recommendation.
% Analogously, in natural communication between humans, text or audio alone often cannot suffice in expressing the speakers' intentions, and speakers often need to refer to an existing spatial object or draw out an illustration of her ideas for better explanation.
% Philosophically, we HCI researchers are mostly concerned about the human-end experience in human-AI communications.
% However, studies on prompting is unique in that we should not just care about the human-end interaction, but also make sure that AI can really get what the human means and produce intention-aligned output.
% Such consideration can drastically impact the design of prompting interactions in human-AI collaboration applications.
% On this note, although studies on multi-modal interactions is a well-established topic in HCI community, it remains a challenging problem what kind of multi-modal information is really effective in helping humans convey their ideas to current and next generation large AI models.




\subsection{Novice Performance vs. Expert Performance}\label{sec:nVe}
In this section we discuss the performance difference between novice and expert regarding experience in painting and prompting.
First, regarding painting skills, some participants with experience (4/12) preferred to draw accurate and fine-grained shapes at the beginning. 
All novice users (5/12) draw rough and less accurate shapes, while some participants with basic painting skills (3/12) also favored sketching rough areas of objects, as exemplified in Figure~\ref{fig:novice_expert}.
The experienced participants using fine-grained strokes (4/12, none of whom were experienced in prompting) achieved higher IoU scores (0.557) in the close-ended task (0.535) when using \tool. 
This is because their sketches were closer in shape and location to the reference, making the single object decomposition result more accurate.
Also, experienced participants are better at arranging spatial location and size of objects than novice participants.
However, some experienced participants (3/12) have mentioned that the fine-grained stroke sometimes makes them frustrated.
As P1's comment for his result in open-ended task: "\emph{It seems it cannot understand thin strokes; even if the shape is accurate, it can only generate content roughly around the area, especially when there is overlapping.}" 
This suggests that while \tool\ provides rough control to produce reasonably fine results from less accurate sketches for novice users, it may disappoint experienced users seeking more precise control through finer strokes. 
As shown in the last column in Figure~\ref{fig:novice_expert}, the dragon hovering in the sky was wrongly turned into a standing large dragon by \tool.

Second, regarding prompting skills, 3 out of 12 participants had one or more years of experience in T2I prompting. These participants used more modifiers than others during both T2I and R2I tasks.
Their performance in the T2I (0.335) and R2I (0.469) tasks showed higher scores than the average T2I (0.314) and R2I (0.418), but there was no performance improvement with \tool\ between their results (0.508) and the overall average score (0.528). 
This indicates that \tool\ can assist novice users in prompting, enabling them to produce satisfactory images similar to those created by users with prompting expertise.



\subsection{Applicability of \tool}
The feedback from user study highlighted several potential applications for our system. 
Three participants (P2, P6, P8) mentioned its possible use in commercial advertising design, emphasizing the importance of controllability for such work. 
They noted that the system's flexibility allows designers to quickly experiment with different settings.
Some participants (N = 3) also mentioned its potential for digital asset creation, particularly for game asset design. 
P7, a game mod developer, found the system highly useful for mod development. 
He explained: "\emph{Mods often require a series of images with a consistent theme and specific spatial requirements. 
For example, in a sacrifice scene, how the objects are arranged is closely tied to the mod's background. It would be difficult for a developer without professional skills, but with this system, it is possible to quickly construct such images}."
A few participants expressed similar thoughts regarding its use in scene construction, such as in film production. 
An interesting suggestion came from participant P4, who proposed its application in crime scene description. 
She pointed out that witnesses are often not skilled artists, and typically describe crime scenes verbally while someone else illustrates their account. 
With this system, witnesses could more easily express what they saw themselves, potentially producing depictions closer to the real events. "\emph{Details like object locations and distances from buildings can be easily conveyed using the system}," she added.

% \subsection{Model Understanding of Users' Implicit Intents}
% In region-sketch-based control of generative models, a significant gap between interaction design and actual implementation is the model's failure in understanding users' naturally expressed intentions.
% For example, when users draw sketches of two people with their hands slightly overlapping, current region-based models cannot automatically infer users' implicit intention that the two people are holding hands.
% Instead, they still require users to explicitly specify in the prompt such relationship.
% \tool addresses this through sketch-aware prompt recommendation to fill in the necessary semantic information, alleviating users' workload.
% However, some users want the generative AI in the future to be able to directly infer this natural implicit intentions from the sketches without additional prompting since prompt recommendation can still be unstable sometimes.
% This problem reflects a more general dilemma, which ubiquitously exists in all forms of conditioned control for generative models such as canny or scribble control.
% This is because all the control models are trained on pairs of explicit control signal and target image, which is lacking further interpretation or customization of the user intentions behind the seemingly straightforward input.
% For another example, the generative models cannot understand what abstraction level the user has in mind for her personal scribbles.
% Such problems leave more challenges to be addressed by future human-AI co-creation research.
% One possible direction is fine-tuning the conditioned models on individual user's conditioned control data to provide more customized interpretation. 

% \subsection{Balance between recommendation and autonomy}
% AIGC tools are a typical example of 
\subsection{Progressive Sketching}
Currently \tool is mainly aimed at novice users who are only capable of creating very rough sketches by themselves.
However, more accomplished painters or even professional artists typically have a coarse-to-fine creative process. 
Such a process is most evident in painting styles like traditional oil painting or digital impasto painting, where artists first quickly lay down large color patches to outline the most primitive proportion and structure of visual elements.
After that, the artists will progressively add layers of finer color strokes to the canvas to gradually refine the painting to an exquisite piece of artwork.
One participant in our user study (P1) , as a professional painter, has mentioned a similar point "\emph{
I think it is useful for laying out the big picture, give some inspirations for the initial drawing stage}."
Therefore, rough sketch also plays a part in the professional artists' creation process, yet it is more challenging to integrate AI into this more complex coarse-to-fine procedure.
Particularly, artists would like to preserve some of their finer strokes in later progression, not just the shape of the initial sketch.
In addition, instead of requiring the tool to generate a finished piece of artwork, some artists may prefer a model that can generate another more accurate sketch based on the initial one, and leave the final coloring and refining to the artists themselves.
To accommodate these diverse progressive sketching requirements, a more advanced sketch-based AI-assisted creation tool should be developed that can seamlessly enable artist intervention at any stage of the sketch and maximally preserve their creative intents to the finest level. 

\subsection{Ethical Issues}
Intellectual property and unethical misuse are two potential ethical concerns of AI-assisted creative tools, particularly those targeting novice users.
In terms of intellectual property, \tool hands over to novice users more control, giving them a higher sense of ownership of the creation.
However, the question still remains: how much contribution from the user's part constitutes full authorship of the artwork?
As \tool still relies on backbone generative models which may be trained on uncopyrighted data largely responsible for turning the sketch into finished artwork, we should design some mechanisms to circumvent this risk.
For example, we can allow artists to upload backbone models trained on their own artworks to integrate with our sketch control.
Regarding unethical misuse, \tool makes fine-grained spatial control more accessible to novice users, who may maliciously generate inappropriate content such as more realistic deepfake with specific postures they want or other explicit content.
To address this issue, we plan to incorporate a more sophisticated filtering mechanism that can detect and screen unethical content with more complex spatial-semantic conditions. 
% In the future, we plan to enable artists to upload their own style model

% \subsection{From interactive prompting to interactive spatial prompting}


\subsection{Limitations and Future work}

    \textbf{User Study Design}. Our open-ended task assesses the usability of \tool's system features in general use cases. To further examine aspects such as creativity and controllability across different methods, the open-ended task could be improved by incorporating baselines to provide more insightful comparative analysis. 
    Besides, in close-ended tasks, while the fixing order of tool usage prevents prior knowledge leakage, it might introduce learning effects. In our study, we include practice sessions for the three systems before the formal task to mitigate these effects. In the future, utilizing parallel tests (\textit{e.g.} different content with the same difficulty) or adding a control group could further reduce the learning effects.

    \textbf{Failure Cases}. There are certain failure cases with \tool that can limit its usability. 
    Firstly, when there are three or more objects with similar semantics, objects may still be missing despite prompt recommendations. 
    Secondly, if an object's stroke is thin, \tool may incorrectly interpret it as a full area, as demonstrated in the expert results of the open-ended task in Figure~\ref{fig:novice_expert}. 
    Finally, sometimes inclusion relationships (\textit{e.g.} inside) between objects cannot be generated correctly, partially due to biases in the base model that lack training samples with such relationship. 

    \textbf{More support for single object adjustment}.
    Participants (N=4) suggested that additional control features should be introduced, beyond just adjusting size and location. They noted that when objects overlap, they cannot freely control which object appears on top or which should be covered, and overlapping areas are currently not allowed.
    They proposed adding features such as layer control and depth control within the single-object mask manipulation. Currently, the system assigns layers based on color order, but future versions should allow users to adjust the layer of each object freely, while considering weighted prompts for overlapping areas.

    \textbf{More customized generation ability}.
    Our current system is built around a single model $ColorfulXL-Lightning$, which limits its ability to fully support the diverse creative needs of users. Feedback from participants has indicated a strong desire for more flexibility in style and personalization, such as integrating fine-tuned models that cater to specific artistic styles or individual preferences. 
    This limitation restricts the ability to adapt to varied creative intents across different users and contexts.
    In future iterations, we plan to address this by embedding a model selection feature, allowing users to choose from a variety of pre-trained or custom fine-tuned models that better align with their stylistic preferences. 
    
    \textbf{Integrate other model functions}.
    Our current system is compatible with many existing tools, such as Promptist~\cite{hao2024optimizing} and Magic Prompt, allowing users to iteratively generate prompts for single objects. However, the integration of these functions is somewhat limited in scope, and users may benefit from a broader range of interactive options, especially for more complex generation tasks. Additionally, for multimodal large models, users can currently explore using affordable or open-source models like Qwen2-VL~\cite{qwen} and InternVL2-Llama3~\cite{llama}, which have demonstrated solid inference performance in our tests. While GPT-4o remains a leading choice, alternative models also offer competitive results.
    Moving forward, we aim to integrate more multimodal large models into the system, giving users the flexibility to choose the models that best fit their needs. 
    


\section{Conclusion}\label{sec:conclusion}
In this paper, we present \tool, an interactive system designed to help novice users create high-quality, fine-grained images that align with their intentions based on rough sketches. 
The system first refines the user's initial prompt into a complete and coherent one that matches the rough sketch, ensuring the generated results are both stable, coherent and high quality.
To further support users in achieving fine-grained alignment between the generated image and their creative intent without requiring professional skills, we introduce a decompose-and-recompose strategy. 
This allows users to select desired, refined object shapes for individual decomposed objects and then recombine them, providing flexible mask manipulation for precise spatial control.
The framework operates through a coarse-to-fine process, enabling iterative and fine-grained control that is not possible with traditional end-to-end generation methods. 
Our user study demonstrates that \tool offers novice users enhanced flexibility in control and fine-grained alignment between their intentions and the generated images.


% In the unusual situation where you want a paper to appear in the
% references without citing it in the main text, use \nocite
% \nocite{langley00}

\bibliography{sections/references}
\bibliographystyle{icml2025}

\subsection{Lloyd-Max Algorithm}
\label{subsec:Lloyd-Max}
For a given quantization bitwidth $B$ and an operand $\bm{X}$, the Lloyd-Max algorithm finds $2^B$ quantization levels $\{\hat{x}_i\}_{i=1}^{2^B}$ such that quantizing $\bm{X}$ by rounding each scalar in $\bm{X}$ to the nearest quantization level minimizes the quantization MSE. 

The algorithm starts with an initial guess of quantization levels and then iteratively computes quantization thresholds $\{\tau_i\}_{i=1}^{2^B-1}$ and updates quantization levels $\{\hat{x}_i\}_{i=1}^{2^B}$. Specifically, at iteration $n$, thresholds are set to the midpoints of the previous iteration's levels:
\begin{align*}
    \tau_i^{(n)}=\frac{\hat{x}_i^{(n-1)}+\hat{x}_{i+1}^{(n-1)}}2 \text{ for } i=1\ldots 2^B-1
\end{align*}
Subsequently, the quantization levels are re-computed as conditional means of the data regions defined by the new thresholds:
\begin{align*}
    \hat{x}_i^{(n)}=\mathbb{E}\left[ \bm{X} \big| \bm{X}\in [\tau_{i-1}^{(n)},\tau_i^{(n)}] \right] \text{ for } i=1\ldots 2^B
\end{align*}
where to satisfy boundary conditions we have $\tau_0=-\infty$ and $\tau_{2^B}=\infty$. The algorithm iterates the above steps until convergence.

Figure \ref{fig:lm_quant} compares the quantization levels of a $7$-bit floating point (E3M3) quantizer (left) to a $7$-bit Lloyd-Max quantizer (right) when quantizing a layer of weights from the GPT3-126M model at a per-tensor granularity. As shown, the Lloyd-Max quantizer achieves substantially lower quantization MSE. Further, Table \ref{tab:FP7_vs_LM7} shows the superior perplexity achieved by Lloyd-Max quantizers for bitwidths of $7$, $6$ and $5$. The difference between the quantizers is clear at 5 bits, where per-tensor FP quantization incurs a drastic and unacceptable increase in perplexity, while Lloyd-Max quantization incurs a much smaller increase. Nevertheless, we note that even the optimal Lloyd-Max quantizer incurs a notable ($\sim 1.5$) increase in perplexity due to the coarse granularity of quantization. 

\begin{figure}[h]
  \centering
  \includegraphics[width=0.7\linewidth]{sections/figures/LM7_FP7.pdf}
  \caption{\small Quantization levels and the corresponding quantization MSE of Floating Point (left) vs Lloyd-Max (right) Quantizers for a layer of weights in the GPT3-126M model.}
  \label{fig:lm_quant}
\end{figure}

\begin{table}[h]\scriptsize
\begin{center}
\caption{\label{tab:FP7_vs_LM7} \small Comparing perplexity (lower is better) achieved by floating point quantizers and Lloyd-Max quantizers on a GPT3-126M model for the Wikitext-103 dataset.}
\begin{tabular}{c|cc|c}
\hline
 \multirow{2}{*}{\textbf{Bitwidth}} & \multicolumn{2}{|c|}{\textbf{Floating-Point Quantizer}} & \textbf{Lloyd-Max Quantizer} \\
 & Best Format & Wikitext-103 Perplexity & Wikitext-103 Perplexity \\
\hline
7 & E3M3 & 18.32 & 18.27 \\
6 & E3M2 & 19.07 & 18.51 \\
5 & E4M0 & 43.89 & 19.71 \\
\hline
\end{tabular}
\end{center}
\end{table}

\subsection{Proof of Local Optimality of LO-BCQ}
\label{subsec:lobcq_opt_proof}
For a given block $\bm{b}_j$, the quantization MSE during LO-BCQ can be empirically evaluated as $\frac{1}{L_b}\lVert \bm{b}_j- \bm{\hat{b}}_j\rVert^2_2$ where $\bm{\hat{b}}_j$ is computed from equation (\ref{eq:clustered_quantization_definition}) as $C_{f(\bm{b}_j)}(\bm{b}_j)$. Further, for a given block cluster $\mathcal{B}_i$, we compute the quantization MSE as $\frac{1}{|\mathcal{B}_{i}|}\sum_{\bm{b} \in \mathcal{B}_{i}} \frac{1}{L_b}\lVert \bm{b}- C_i^{(n)}(\bm{b})\rVert^2_2$. Therefore, at the end of iteration $n$, we evaluate the overall quantization MSE $J^{(n)}$ for a given operand $\bm{X}$ composed of $N_c$ block clusters as:
\begin{align*}
    \label{eq:mse_iter_n}
    J^{(n)} = \frac{1}{N_c} \sum_{i=1}^{N_c} \frac{1}{|\mathcal{B}_{i}^{(n)}|}\sum_{\bm{v} \in \mathcal{B}_{i}^{(n)}} \frac{1}{L_b}\lVert \bm{b}- B_i^{(n)}(\bm{b})\rVert^2_2
\end{align*}

At the end of iteration $n$, the codebooks are updated from $\mathcal{C}^{(n-1)}$ to $\mathcal{C}^{(n)}$. However, the mapping of a given vector $\bm{b}_j$ to quantizers $\mathcal{C}^{(n)}$ remains as  $f^{(n)}(\bm{b}_j)$. At the next iteration, during the vector clustering step, $f^{(n+1)}(\bm{b}_j)$ finds new mapping of $\bm{b}_j$ to updated codebooks $\mathcal{C}^{(n)}$ such that the quantization MSE over the candidate codebooks is minimized. Therefore, we obtain the following result for $\bm{b}_j$:
\begin{align*}
\frac{1}{L_b}\lVert \bm{b}_j - C_{f^{(n+1)}(\bm{b}_j)}^{(n)}(\bm{b}_j)\rVert^2_2 \le \frac{1}{L_b}\lVert \bm{b}_j - C_{f^{(n)}(\bm{b}_j)}^{(n)}(\bm{b}_j)\rVert^2_2
\end{align*}

That is, quantizing $\bm{b}_j$ at the end of the block clustering step of iteration $n+1$ results in lower quantization MSE compared to quantizing at the end of iteration $n$. Since this is true for all $\bm{b} \in \bm{X}$, we assert the following:
\begin{equation}
\begin{split}
\label{eq:mse_ineq_1}
    \tilde{J}^{(n+1)} &= \frac{1}{N_c} \sum_{i=1}^{N_c} \frac{1}{|\mathcal{B}_{i}^{(n+1)}|}\sum_{\bm{b} \in \mathcal{B}_{i}^{(n+1)}} \frac{1}{L_b}\lVert \bm{b} - C_i^{(n)}(b)\rVert^2_2 \le J^{(n)}
\end{split}
\end{equation}
where $\tilde{J}^{(n+1)}$ is the the quantization MSE after the vector clustering step at iteration $n+1$.

Next, during the codebook update step (\ref{eq:quantizers_update}) at iteration $n+1$, the per-cluster codebooks $\mathcal{C}^{(n)}$ are updated to $\mathcal{C}^{(n+1)}$ by invoking the Lloyd-Max algorithm \citep{Lloyd}. We know that for any given value distribution, the Lloyd-Max algorithm minimizes the quantization MSE. Therefore, for a given vector cluster $\mathcal{B}_i$ we obtain the following result:

\begin{equation}
    \frac{1}{|\mathcal{B}_{i}^{(n+1)}|}\sum_{\bm{b} \in \mathcal{B}_{i}^{(n+1)}} \frac{1}{L_b}\lVert \bm{b}- C_i^{(n+1)}(\bm{b})\rVert^2_2 \le \frac{1}{|\mathcal{B}_{i}^{(n+1)}|}\sum_{\bm{b} \in \mathcal{B}_{i}^{(n+1)}} \frac{1}{L_b}\lVert \bm{b}- C_i^{(n)}(\bm{b})\rVert^2_2
\end{equation}

The above equation states that quantizing the given block cluster $\mathcal{B}_i$ after updating the associated codebook from $C_i^{(n)}$ to $C_i^{(n+1)}$ results in lower quantization MSE. Since this is true for all the block clusters, we derive the following result: 
\begin{equation}
\begin{split}
\label{eq:mse_ineq_2}
     J^{(n+1)} &= \frac{1}{N_c} \sum_{i=1}^{N_c} \frac{1}{|\mathcal{B}_{i}^{(n+1)}|}\sum_{\bm{b} \in \mathcal{B}_{i}^{(n+1)}} \frac{1}{L_b}\lVert \bm{b}- C_i^{(n+1)}(\bm{b})\rVert^2_2  \le \tilde{J}^{(n+1)}   
\end{split}
\end{equation}

Following (\ref{eq:mse_ineq_1}) and (\ref{eq:mse_ineq_2}), we find that the quantization MSE is non-increasing for each iteration, that is, $J^{(1)} \ge J^{(2)} \ge J^{(3)} \ge \ldots \ge J^{(M)}$ where $M$ is the maximum number of iterations. 
%Therefore, we can say that if the algorithm converges, then it must be that it has converged to a local minimum. 
\hfill $\blacksquare$


\begin{figure}
    \begin{center}
    \includegraphics[width=0.5\textwidth]{sections//figures/mse_vs_iter.pdf}
    \end{center}
    \caption{\small NMSE vs iterations during LO-BCQ compared to other block quantization proposals}
    \label{fig:nmse_vs_iter}
\end{figure}

Figure \ref{fig:nmse_vs_iter} shows the empirical convergence of LO-BCQ across several block lengths and number of codebooks. Also, the MSE achieved by LO-BCQ is compared to baselines such as MXFP and VSQ. As shown, LO-BCQ converges to a lower MSE than the baselines. Further, we achieve better convergence for larger number of codebooks ($N_c$) and for a smaller block length ($L_b$), both of which increase the bitwidth of BCQ (see Eq \ref{eq:bitwidth_bcq}).


\subsection{Additional Accuracy Results}
%Table \ref{tab:lobcq_config} lists the various LOBCQ configurations and their corresponding bitwidths.
\begin{table}
\setlength{\tabcolsep}{4.75pt}
\begin{center}
\caption{\label{tab:lobcq_config} Various LO-BCQ configurations and their bitwidths.}
\begin{tabular}{|c||c|c|c|c||c|c||c|} 
\hline
 & \multicolumn{4}{|c||}{$L_b=8$} & \multicolumn{2}{|c||}{$L_b=4$} & $L_b=2$ \\
 \hline
 \backslashbox{$L_A$\kern-1em}{\kern-1em$N_c$} & 2 & 4 & 8 & 16 & 2 & 4 & 2 \\
 \hline
 64 & 4.25 & 4.375 & 4.5 & 4.625 & 4.375 & 4.625 & 4.625\\
 \hline
 32 & 4.375 & 4.5 & 4.625& 4.75 & 4.5 & 4.75 & 4.75 \\
 \hline
 16 & 4.625 & 4.75& 4.875 & 5 & 4.75 & 5 & 5 \\
 \hline
\end{tabular}
\end{center}
\end{table}

%\subsection{Perplexity achieved by various LO-BCQ configurations on Wikitext-103 dataset}

\begin{table} \centering
\begin{tabular}{|c||c|c|c|c||c|c||c|} 
\hline
 $L_b \rightarrow$& \multicolumn{4}{c||}{8} & \multicolumn{2}{c||}{4} & 2\\
 \hline
 \backslashbox{$L_A$\kern-1em}{\kern-1em$N_c$} & 2 & 4 & 8 & 16 & 2 & 4 & 2  \\
 %$N_c \rightarrow$ & 2 & 4 & 8 & 16 & 2 & 4 & 2 \\
 \hline
 \hline
 \multicolumn{8}{c}{GPT3-1.3B (FP32 PPL = 9.98)} \\ 
 \hline
 \hline
 64 & 10.40 & 10.23 & 10.17 & 10.15 &  10.28 & 10.18 & 10.19 \\
 \hline
 32 & 10.25 & 10.20 & 10.15 & 10.12 &  10.23 & 10.17 & 10.17 \\
 \hline
 16 & 10.22 & 10.16 & 10.10 & 10.09 &  10.21 & 10.14 & 10.16 \\
 \hline
  \hline
 \multicolumn{8}{c}{GPT3-8B (FP32 PPL = 7.38)} \\ 
 \hline
 \hline
 64 & 7.61 & 7.52 & 7.48 &  7.47 &  7.55 &  7.49 & 7.50 \\
 \hline
 32 & 7.52 & 7.50 & 7.46 &  7.45 &  7.52 &  7.48 & 7.48  \\
 \hline
 16 & 7.51 & 7.48 & 7.44 &  7.44 &  7.51 &  7.49 & 7.47  \\
 \hline
\end{tabular}
\caption{\label{tab:ppl_gpt3_abalation} Wikitext-103 perplexity across GPT3-1.3B and 8B models.}
\end{table}

\begin{table} \centering
\begin{tabular}{|c||c|c|c|c||} 
\hline
 $L_b \rightarrow$& \multicolumn{4}{c||}{8}\\
 \hline
 \backslashbox{$L_A$\kern-1em}{\kern-1em$N_c$} & 2 & 4 & 8 & 16 \\
 %$N_c \rightarrow$ & 2 & 4 & 8 & 16 & 2 & 4 & 2 \\
 \hline
 \hline
 \multicolumn{5}{|c|}{Llama2-7B (FP32 PPL = 5.06)} \\ 
 \hline
 \hline
 64 & 5.31 & 5.26 & 5.19 & 5.18  \\
 \hline
 32 & 5.23 & 5.25 & 5.18 & 5.15  \\
 \hline
 16 & 5.23 & 5.19 & 5.16 & 5.14  \\
 \hline
 \multicolumn{5}{|c|}{Nemotron4-15B (FP32 PPL = 5.87)} \\ 
 \hline
 \hline
 64  & 6.3 & 6.20 & 6.13 & 6.08  \\
 \hline
 32  & 6.24 & 6.12 & 6.07 & 6.03  \\
 \hline
 16  & 6.12 & 6.14 & 6.04 & 6.02  \\
 \hline
 \multicolumn{5}{|c|}{Nemotron4-340B (FP32 PPL = 3.48)} \\ 
 \hline
 \hline
 64 & 3.67 & 3.62 & 3.60 & 3.59 \\
 \hline
 32 & 3.63 & 3.61 & 3.59 & 3.56 \\
 \hline
 16 & 3.61 & 3.58 & 3.57 & 3.55 \\
 \hline
\end{tabular}
\caption{\label{tab:ppl_llama7B_nemo15B} Wikitext-103 perplexity compared to FP32 baseline in Llama2-7B and Nemotron4-15B, 340B models}
\end{table}

%\subsection{Perplexity achieved by various LO-BCQ configurations on MMLU dataset}


\begin{table} \centering
\begin{tabular}{|c||c|c|c|c||c|c|c|c|} 
\hline
 $L_b \rightarrow$& \multicolumn{4}{c||}{8} & \multicolumn{4}{c||}{8}\\
 \hline
 \backslashbox{$L_A$\kern-1em}{\kern-1em$N_c$} & 2 & 4 & 8 & 16 & 2 & 4 & 8 & 16  \\
 %$N_c \rightarrow$ & 2 & 4 & 8 & 16 & 2 & 4 & 2 \\
 \hline
 \hline
 \multicolumn{5}{|c|}{Llama2-7B (FP32 Accuracy = 45.8\%)} & \multicolumn{4}{|c|}{Llama2-70B (FP32 Accuracy = 69.12\%)} \\ 
 \hline
 \hline
 64 & 43.9 & 43.4 & 43.9 & 44.9 & 68.07 & 68.27 & 68.17 & 68.75 \\
 \hline
 32 & 44.5 & 43.8 & 44.9 & 44.5 & 68.37 & 68.51 & 68.35 & 68.27  \\
 \hline
 16 & 43.9 & 42.7 & 44.9 & 45 & 68.12 & 68.77 & 68.31 & 68.59  \\
 \hline
 \hline
 \multicolumn{5}{|c|}{GPT3-22B (FP32 Accuracy = 38.75\%)} & \multicolumn{4}{|c|}{Nemotron4-15B (FP32 Accuracy = 64.3\%)} \\ 
 \hline
 \hline
 64 & 36.71 & 38.85 & 38.13 & 38.92 & 63.17 & 62.36 & 63.72 & 64.09 \\
 \hline
 32 & 37.95 & 38.69 & 39.45 & 38.34 & 64.05 & 62.30 & 63.8 & 64.33  \\
 \hline
 16 & 38.88 & 38.80 & 38.31 & 38.92 & 63.22 & 63.51 & 63.93 & 64.43  \\
 \hline
\end{tabular}
\caption{\label{tab:mmlu_abalation} Accuracy on MMLU dataset across GPT3-22B, Llama2-7B, 70B and Nemotron4-15B models.}
\end{table}


%\subsection{Perplexity achieved by various LO-BCQ configurations on LM evaluation harness}

\begin{table} \centering
\begin{tabular}{|c||c|c|c|c||c|c|c|c|} 
\hline
 $L_b \rightarrow$& \multicolumn{4}{c||}{8} & \multicolumn{4}{c||}{8}\\
 \hline
 \backslashbox{$L_A$\kern-1em}{\kern-1em$N_c$} & 2 & 4 & 8 & 16 & 2 & 4 & 8 & 16  \\
 %$N_c \rightarrow$ & 2 & 4 & 8 & 16 & 2 & 4 & 2 \\
 \hline
 \hline
 \multicolumn{5}{|c|}{Race (FP32 Accuracy = 37.51\%)} & \multicolumn{4}{|c|}{Boolq (FP32 Accuracy = 64.62\%)} \\ 
 \hline
 \hline
 64 & 36.94 & 37.13 & 36.27 & 37.13 & 63.73 & 62.26 & 63.49 & 63.36 \\
 \hline
 32 & 37.03 & 36.36 & 36.08 & 37.03 & 62.54 & 63.51 & 63.49 & 63.55  \\
 \hline
 16 & 37.03 & 37.03 & 36.46 & 37.03 & 61.1 & 63.79 & 63.58 & 63.33  \\
 \hline
 \hline
 \multicolumn{5}{|c|}{Winogrande (FP32 Accuracy = 58.01\%)} & \multicolumn{4}{|c|}{Piqa (FP32 Accuracy = 74.21\%)} \\ 
 \hline
 \hline
 64 & 58.17 & 57.22 & 57.85 & 58.33 & 73.01 & 73.07 & 73.07 & 72.80 \\
 \hline
 32 & 59.12 & 58.09 & 57.85 & 58.41 & 73.01 & 73.94 & 72.74 & 73.18  \\
 \hline
 16 & 57.93 & 58.88 & 57.93 & 58.56 & 73.94 & 72.80 & 73.01 & 73.94  \\
 \hline
\end{tabular}
\caption{\label{tab:mmlu_abalation} Accuracy on LM evaluation harness tasks on GPT3-1.3B model.}
\end{table}

\begin{table} \centering
\begin{tabular}{|c||c|c|c|c||c|c|c|c|} 
\hline
 $L_b \rightarrow$& \multicolumn{4}{c||}{8} & \multicolumn{4}{c||}{8}\\
 \hline
 \backslashbox{$L_A$\kern-1em}{\kern-1em$N_c$} & 2 & 4 & 8 & 16 & 2 & 4 & 8 & 16  \\
 %$N_c \rightarrow$ & 2 & 4 & 8 & 16 & 2 & 4 & 2 \\
 \hline
 \hline
 \multicolumn{5}{|c|}{Race (FP32 Accuracy = 41.34\%)} & \multicolumn{4}{|c|}{Boolq (FP32 Accuracy = 68.32\%)} \\ 
 \hline
 \hline
 64 & 40.48 & 40.10 & 39.43 & 39.90 & 69.20 & 68.41 & 69.45 & 68.56 \\
 \hline
 32 & 39.52 & 39.52 & 40.77 & 39.62 & 68.32 & 67.43 & 68.17 & 69.30  \\
 \hline
 16 & 39.81 & 39.71 & 39.90 & 40.38 & 68.10 & 66.33 & 69.51 & 69.42  \\
 \hline
 \hline
 \multicolumn{5}{|c|}{Winogrande (FP32 Accuracy = 67.88\%)} & \multicolumn{4}{|c|}{Piqa (FP32 Accuracy = 78.78\%)} \\ 
 \hline
 \hline
 64 & 66.85 & 66.61 & 67.72 & 67.88 & 77.31 & 77.42 & 77.75 & 77.64 \\
 \hline
 32 & 67.25 & 67.72 & 67.72 & 67.00 & 77.31 & 77.04 & 77.80 & 77.37  \\
 \hline
 16 & 68.11 & 68.90 & 67.88 & 67.48 & 77.37 & 78.13 & 78.13 & 77.69  \\
 \hline
\end{tabular}
\caption{\label{tab:mmlu_abalation} Accuracy on LM evaluation harness tasks on GPT3-8B model.}
\end{table}

\begin{table} \centering
\begin{tabular}{|c||c|c|c|c||c|c|c|c|} 
\hline
 $L_b \rightarrow$& \multicolumn{4}{c||}{8} & \multicolumn{4}{c||}{8}\\
 \hline
 \backslashbox{$L_A$\kern-1em}{\kern-1em$N_c$} & 2 & 4 & 8 & 16 & 2 & 4 & 8 & 16  \\
 %$N_c \rightarrow$ & 2 & 4 & 8 & 16 & 2 & 4 & 2 \\
 \hline
 \hline
 \multicolumn{5}{|c|}{Race (FP32 Accuracy = 40.67\%)} & \multicolumn{4}{|c|}{Boolq (FP32 Accuracy = 76.54\%)} \\ 
 \hline
 \hline
 64 & 40.48 & 40.10 & 39.43 & 39.90 & 75.41 & 75.11 & 77.09 & 75.66 \\
 \hline
 32 & 39.52 & 39.52 & 40.77 & 39.62 & 76.02 & 76.02 & 75.96 & 75.35  \\
 \hline
 16 & 39.81 & 39.71 & 39.90 & 40.38 & 75.05 & 73.82 & 75.72 & 76.09  \\
 \hline
 \hline
 \multicolumn{5}{|c|}{Winogrande (FP32 Accuracy = 70.64\%)} & \multicolumn{4}{|c|}{Piqa (FP32 Accuracy = 79.16\%)} \\ 
 \hline
 \hline
 64 & 69.14 & 70.17 & 70.17 & 70.56 & 78.24 & 79.00 & 78.62 & 78.73 \\
 \hline
 32 & 70.96 & 69.69 & 71.27 & 69.30 & 78.56 & 79.49 & 79.16 & 78.89  \\
 \hline
 16 & 71.03 & 69.53 & 69.69 & 70.40 & 78.13 & 79.16 & 79.00 & 79.00  \\
 \hline
\end{tabular}
\caption{\label{tab:mmlu_abalation} Accuracy on LM evaluation harness tasks on GPT3-22B model.}
\end{table}

\begin{table} \centering
\begin{tabular}{|c||c|c|c|c||c|c|c|c|} 
\hline
 $L_b \rightarrow$& \multicolumn{4}{c||}{8} & \multicolumn{4}{c||}{8}\\
 \hline
 \backslashbox{$L_A$\kern-1em}{\kern-1em$N_c$} & 2 & 4 & 8 & 16 & 2 & 4 & 8 & 16  \\
 %$N_c \rightarrow$ & 2 & 4 & 8 & 16 & 2 & 4 & 2 \\
 \hline
 \hline
 \multicolumn{5}{|c|}{Race (FP32 Accuracy = 44.4\%)} & \multicolumn{4}{|c|}{Boolq (FP32 Accuracy = 79.29\%)} \\ 
 \hline
 \hline
 64 & 42.49 & 42.51 & 42.58 & 43.45 & 77.58 & 77.37 & 77.43 & 78.1 \\
 \hline
 32 & 43.35 & 42.49 & 43.64 & 43.73 & 77.86 & 75.32 & 77.28 & 77.86  \\
 \hline
 16 & 44.21 & 44.21 & 43.64 & 42.97 & 78.65 & 77 & 76.94 & 77.98  \\
 \hline
 \hline
 \multicolumn{5}{|c|}{Winogrande (FP32 Accuracy = 69.38\%)} & \multicolumn{4}{|c|}{Piqa (FP32 Accuracy = 78.07\%)} \\ 
 \hline
 \hline
 64 & 68.9 & 68.43 & 69.77 & 68.19 & 77.09 & 76.82 & 77.09 & 77.86 \\
 \hline
 32 & 69.38 & 68.51 & 68.82 & 68.90 & 78.07 & 76.71 & 78.07 & 77.86  \\
 \hline
 16 & 69.53 & 67.09 & 69.38 & 68.90 & 77.37 & 77.8 & 77.91 & 77.69  \\
 \hline
\end{tabular}
\caption{\label{tab:mmlu_abalation} Accuracy on LM evaluation harness tasks on Llama2-7B model.}
\end{table}

\begin{table} \centering
\begin{tabular}{|c||c|c|c|c||c|c|c|c|} 
\hline
 $L_b \rightarrow$& \multicolumn{4}{c||}{8} & \multicolumn{4}{c||}{8}\\
 \hline
 \backslashbox{$L_A$\kern-1em}{\kern-1em$N_c$} & 2 & 4 & 8 & 16 & 2 & 4 & 8 & 16  \\
 %$N_c \rightarrow$ & 2 & 4 & 8 & 16 & 2 & 4 & 2 \\
 \hline
 \hline
 \multicolumn{5}{|c|}{Race (FP32 Accuracy = 48.8\%)} & \multicolumn{4}{|c|}{Boolq (FP32 Accuracy = 85.23\%)} \\ 
 \hline
 \hline
 64 & 49.00 & 49.00 & 49.28 & 48.71 & 82.82 & 84.28 & 84.03 & 84.25 \\
 \hline
 32 & 49.57 & 48.52 & 48.33 & 49.28 & 83.85 & 84.46 & 84.31 & 84.93  \\
 \hline
 16 & 49.85 & 49.09 & 49.28 & 48.99 & 85.11 & 84.46 & 84.61 & 83.94  \\
 \hline
 \hline
 \multicolumn{5}{|c|}{Winogrande (FP32 Accuracy = 79.95\%)} & \multicolumn{4}{|c|}{Piqa (FP32 Accuracy = 81.56\%)} \\ 
 \hline
 \hline
 64 & 78.77 & 78.45 & 78.37 & 79.16 & 81.45 & 80.69 & 81.45 & 81.5 \\
 \hline
 32 & 78.45 & 79.01 & 78.69 & 80.66 & 81.56 & 80.58 & 81.18 & 81.34  \\
 \hline
 16 & 79.95 & 79.56 & 79.79 & 79.72 & 81.28 & 81.66 & 81.28 & 80.96  \\
 \hline
\end{tabular}
\caption{\label{tab:mmlu_abalation} Accuracy on LM evaluation harness tasks on Llama2-70B model.}
\end{table}

%\section{MSE Studies}
%\textcolor{red}{TODO}


\subsection{Number Formats and Quantization Method}
\label{subsec:numFormats_quantMethod}
\subsubsection{Integer Format}
An $n$-bit signed integer (INT) is typically represented with a 2s-complement format \citep{yao2022zeroquant,xiao2023smoothquant,dai2021vsq}, where the most significant bit denotes the sign.

\subsubsection{Floating Point Format}
An $n$-bit signed floating point (FP) number $x$ comprises of a 1-bit sign ($x_{\mathrm{sign}}$), $B_m$-bit mantissa ($x_{\mathrm{mant}}$) and $B_e$-bit exponent ($x_{\mathrm{exp}}$) such that $B_m+B_e=n-1$. The associated constant exponent bias ($E_{\mathrm{bias}}$) is computed as $(2^{{B_e}-1}-1)$. We denote this format as $E_{B_e}M_{B_m}$.  

\subsubsection{Quantization Scheme}
\label{subsec:quant_method}
A quantization scheme dictates how a given unquantized tensor is converted to its quantized representation. We consider FP formats for the purpose of illustration. Given an unquantized tensor $\bm{X}$ and an FP format $E_{B_e}M_{B_m}$, we first, we compute the quantization scale factor $s_X$ that maps the maximum absolute value of $\bm{X}$ to the maximum quantization level of the $E_{B_e}M_{B_m}$ format as follows:
\begin{align}
\label{eq:sf}
    s_X = \frac{\mathrm{max}(|\bm{X}|)}{\mathrm{max}(E_{B_e}M_{B_m})}
\end{align}
In the above equation, $|\cdot|$ denotes the absolute value function.

Next, we scale $\bm{X}$ by $s_X$ and quantize it to $\hat{\bm{X}}$ by rounding it to the nearest quantization level of $E_{B_e}M_{B_m}$ as:

\begin{align}
\label{eq:tensor_quant}
    \hat{\bm{X}} = \text{round-to-nearest}\left(\frac{\bm{X}}{s_X}, E_{B_e}M_{B_m}\right)
\end{align}

We perform dynamic max-scaled quantization \citep{wu2020integer}, where the scale factor $s$ for activations is dynamically computed during runtime.

\subsection{Vector Scaled Quantization}
\begin{wrapfigure}{r}{0.35\linewidth}
  \centering
  \includegraphics[width=\linewidth]{sections/figures/vsquant.jpg}
  \caption{\small Vectorwise decomposition for per-vector scaled quantization (VSQ \citep{dai2021vsq}).}
  \label{fig:vsquant}
\end{wrapfigure}
During VSQ \citep{dai2021vsq}, the operand tensors are decomposed into 1D vectors in a hardware friendly manner as shown in Figure \ref{fig:vsquant}. Since the decomposed tensors are used as operands in matrix multiplications during inference, it is beneficial to perform this decomposition along the reduction dimension of the multiplication. The vectorwise quantization is performed similar to tensorwise quantization described in Equations \ref{eq:sf} and \ref{eq:tensor_quant}, where a scale factor $s_v$ is required for each vector $\bm{v}$ that maps the maximum absolute value of that vector to the maximum quantization level. While smaller vector lengths can lead to larger accuracy gains, the associated memory and computational overheads due to the per-vector scale factors increases. To alleviate these overheads, VSQ \citep{dai2021vsq} proposed a second level quantization of the per-vector scale factors to unsigned integers, while MX \citep{rouhani2023shared} quantizes them to integer powers of 2 (denoted as $2^{INT}$).

\subsubsection{MX Format}
The MX format proposed in \citep{rouhani2023microscaling} introduces the concept of sub-block shifting. For every two scalar elements of $b$-bits each, there is a shared exponent bit. The value of this exponent bit is determined through an empirical analysis that targets minimizing quantization MSE. We note that the FP format $E_{1}M_{b}$ is strictly better than MX from an accuracy perspective since it allocates a dedicated exponent bit to each scalar as opposed to sharing it across two scalars. Therefore, we conservatively bound the accuracy of a $b+2$-bit signed MX format with that of a $E_{1}M_{b}$ format in our comparisons. For instance, we use E1M2 format as a proxy for MX4.

\begin{figure}
    \centering
    \includegraphics[width=1\linewidth]{sections//figures/BlockFormats.pdf}
    \caption{\small Comparing LO-BCQ to MX format.}
    \label{fig:block_formats}
\end{figure}

Figure \ref{fig:block_formats} compares our $4$-bit LO-BCQ block format to MX \citep{rouhani2023microscaling}. As shown, both LO-BCQ and MX decompose a given operand tensor into block arrays and each block array into blocks. Similar to MX, we find that per-block quantization ($L_b < L_A$) leads to better accuracy due to increased flexibility. While MX achieves this through per-block $1$-bit micro-scales, we associate a dedicated codebook to each block through a per-block codebook selector. Further, MX quantizes the per-block array scale-factor to E8M0 format without per-tensor scaling. In contrast during LO-BCQ, we find that per-tensor scaling combined with quantization of per-block array scale-factor to E4M3 format results in superior inference accuracy across models. 



\end{document}

