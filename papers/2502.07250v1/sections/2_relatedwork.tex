\section{Related Work}
\textbf{Complex Event Detection (CED)} has been extensively studied in traditional stream processing for structured databases. Early works \cite{CEPoverview_2012,Schultz_2009,Debar_2001} employed event engines such as Finite State Machines (FSMs) to identify complex events based on predefined patterns. More recently, research has shifted towards CED over unstructured, high-dimensional data. Approaches such as \cite{xing2020neuroplex,ROIGVILAMALA2023119376,deepproblog2} integrate neural detectors with symbolic rule-based systems like ProbLog~\cite{problog,deepproblog,deepproblog2}, enabling backpropagation through logical rules. However, these methods suffer from computational scalability issues. Moreover, most existing approaches focus on complex activities within short time spans \cite{khan2023hybridgraphnetworkcomplex,activitynet,THUMOS14}, limiting their applicability to longer sensor traces with extended temporal dependencies.

\textbf{Neural Algorithmic Reasoners (NAR)} are neural networks designed to learn algorithmic structures directly from data, enabling structured reasoning tasks such as sorting, shortest-path computation, and graph search \cite{narbenchmark,generalnar,naroverview}. \cite{transformernar} combines a transformer with a pre-trained NAR model to apply learned algorithmic knowledge to language-based reasoning tasks. Inspired by this, we view NAR as particularly well-suited for CED, where each complex event rule can be framed as an algorithm and learned in a data-driven manner. By leveraging NAR, we propose a framework that decouples complex event rule learning from raw sensor data, improving data efficiency and generalization.
