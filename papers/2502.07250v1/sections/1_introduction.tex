\section{Introduction}
Current work has achieved great performance in short-time perception machine learning tasks, such as human activity recognition or object detection, which typically require only a few seconds of sensor data for inference. However, many real-world applications critically depend on the ability to understand high-level contextual information over extended periods of time, as humans do—an ability that is crucial yet often overlooked. For example, consider a social robot that monitors the daily routine of the elderly with a camera of 30 frames per second rate that works all day. An 8-hour daytime video generates nearly one million frames; one needs to know how to compress and memorize key information to keep track of patterns that may span minutes or hours. To describe different situations, it is useful to use the concept of a \textbf{complex event (\emph{CE})}. A complex event represents a high-level scenario with spatiotemporal rules and patterns that require aggregating and reasoning over numerous short-term activities, which we call \textbf{atomic events (\emph{AE}s)}. Fig.~\ref{fig:ce_examples} shows examples of two common complex events, each featuring (spatial)temporal patterns of key atomic events, respectively. 

\begin{figure}[t]
    \centering
\includegraphics[width=\linewidth]{figs/ce_examples.png}
\caption{Examples of two common \emph{CE}s. (a) An intelligent assistant on a mobile device understands a sanitary protocol and alerts users of potential violations. (b) In a smart facility, a surveillance system detects potentially suspicious activity, such as an unusual parcel hand-off, by analyzing data across distributed cameras. }
    \label{fig:ce_examples}
    \vspace{-1em}
\end{figure}


\begin{figure*}[t]
    \centering
        \setlength{\abovecaptionskip}{0.cm}
    \setlength{\belowcaptionskip}{0.cm}
\includegraphics[width=0.87\textwidth]{figs/narce_overview.png}
\caption{Overview of the \narce{} Framework.  
    In \textbf{Stage 1}, an \textbf{LLM Synthesizer} generates pseudo concept \emph{AE} traces based on predefined complex event rules. These traces are tokenized and paired with online \emph{CE} labels to train the \textbf{Mamba-Based \emph{CE} NAR}.  
    In \textbf{Stage 2}, the trained \textbf{\emph{CE} NAR} is frozen, and a \textbf{Sensor Adapter $f'$} is trained to map sensor embeddings into its latent space, enabling online CE detection from raw sensor streams.}
    \label{fig:narce_overview}
    \vspace{-0.5em}
\end{figure*}

\emph{Complex events are challenging.} The reasons are three-fold. First, to recognize a complex event pattern, the model must identify key atomic event occurrences while ignoring irrelevant activities, which we call as ``don’t care'' elements or simply ``X''. For example, the sanitary protocol can be represented as ``Use restroom → X → Wash hands → X → Eat,'' where ``X'' includes other irrelevant activities like ``walking'' or ``sitting''. Incorporating ``X'' broadens the range of possible matching sequences, and the temporal duration further amplifies this space exponentially. Second, complex events have various and much longer time dependencies. In the sanitary protocol example, violation occurs when the person skips ``Wash hands'' after ``Use restroom'' and before ``Eat''. However, the time gap between those atomic events can be random, and in most cases significant. Third, complex events often require immediate attention. For instance, in a nursing monitor system, we should not wait to analyze data offline until the end of the day. Instead, we need online detection of \emph{CE}s to send out alerts in time when safety is violated, which is an aspect often ignored.

Two main approaches exist for detecting complex events. The first approach relies on data-driven methods, such as neural networks, which learn patterns directly from data without requiring human-defined rules. However, these methods demand massive labeled datasets, which are difficult to collect and annotate due to the long-span nature of complex events. For example, an inertial sensor at 100 Hz producing an hour-long sample generates 3.6 billion data points for 10,000 samples. Labeling such data is arduous, and whether neural networks can reliably uncover hidden rules and overcome memory fading issues for long-duration dependencies remains uncertain. Recent state-space model (SSM) architectures \cite{gu2022ssm, gu2024mamba, dao2024mamba2}, such as Mamba, show promise in tasks with long contextual dependencies, but they still face the limitations of being data-hungry, requiring vast amounts of labeled sensor data.

The second approach employs a neuro-symbolic framework that incorporates human knowledge as a rule reasoning component. Here, the neural network focuses on classifying atomic events, while reasoning over these events is handled separately. A human-written symbolic engine can be used, such as a state machine, can effectively represent the sequential stages of complex events, eliminating the need to explicitly manage memory or long-term dependencies. However, symbolic engines are often fragile to uncertainty and sensor errors. Probabilistic reasoning engines, as commonly seen in recent neurosymbolic works \cite{deepproblog, deepproblog2,scallop,neurasp,slash}, account for uncertainties and offer more robustness, but require expertise in languages like ProbLog or Answer Set Programming \cite{problog,asp} to define complex temporal rules efficiently, making their implementation challenging.


% \begin{figure}[tbp]
% \centerline{\includegraphics[width=1\textwidth]{figs/ce_overview.png}}
% \caption{An illustration of the real-time complex events detection task. The example on the right shows that ``\textit{Using Restroom}" and ``\textit{Eating}" without ``\textit{Washing hands}" triggers the complex event detection, but only at the last action ``\textit{Washing hands}" we attach the \textit{CE} label ``1" of this complex event.}
% \label{fig:ced-illustration}
% \end{figure}

% \begin{figure*}[t]
%     \centering
% \includegraphics[width=0.9\textwidth]{figs/ce_overview.png}
%     \caption{(a) Sanitary protocol violation in smart home health monitoring system. (b) Detecting coordinated terrorist attacks at different locations across the city using the surveillance system. (c) In a real-time complex event detection (CED) task, only the raw sensor streams and ground-truth complex event labels are provided.}
%     \label{fig:ce_overview}
% \end{figure*}

In this work, we propose \narce{}, a Neural Algorithmic Reasoner (NAR) framework for efficient and robust Online Complex Event Detection (CED) based on two hypotheses: 
\begin{itemize}[leftmargin=1em, nosep]
    \item \textbf{\emph{Hypothesis I}}: State-based methods are particularly effective for CED because they capture the progression of complex events through states and transitions, avoiding the need to manage long-term memory. 
    \item \textbf{\emph{Hypothesis II}}: Decoupling the learning of complex event rules from noisy sensor data improves data efficiency by reducing reliance on extensive labeled data.
\end{itemize}
We validated \emph{Hypothesis I} through baseline experiments, which showed that the state-space model Mamba outperforms other architectures for CED. Its state-machine-like architecture naturally models event progression, making it well-suited. Building on this, we further tested \emph{Hypothesis II} by proposing \narce{}, a framework inspired by the Neural Algorithmic Reasoner (NAR)\cite{narbenchmark,generalnar,naroverview}. NAR is designed to emulate classical algorithms using neural networks, enabling structured reasoning with noisy data. Following this, we treat each complex event rule as an algorithm and decouple the complexity of CED task into two components: (i) a reasoning module, using Mamba as its backbone, trained on synthetic concept traces to learn complex event rules independently of sensor data, and (ii) an adapter that maps sensor inputs to the reasoning module. By leveraging low-cost synthetic data generated by large language models (LLMs), \narce{} achieves higher data efficiency.

Experiments show that \narce{} outperforms baselines across three key metrics: higher online detection accuracy, better generalization to out-of-distribution and extended sensor data, and significantly reduced labeled data requirements. These findings confirm the effectiveness of our hypotheses and establish our framework as a robust and data-efficient solution for online CED. Code and dataset are available at:\href{https://anonymous.4open.science/r/narce-3344/}{/r/narce-3344/}.


% We present \narce{}, a \textbf{N}eural \textbf{A}lgorithmic \textbf{R}easoner(NAR) framework for effieicent and robust Online \textbf{C}omplex \textbf{E}vent Detection. Neural Algorithmic Reasoners (NAR), introduced by Veličković et al. (cite), are designed to emulate classical algorithms using neural networks, enabling structured reasoning with noisy data. Inspired by this, we treat each complex event rule as an algorithm and decouple the complexity of CED task into two components: (i) a reasoning module, NAR, trained on synthetic concept traces to learn these event rules independently of sensor data and (ii) an adapter that maps sensor inputs to the reasoning module. This approach, supported by synthetic data generated from large language models (LLMs) for (i), significantly reduces reliance on labeled sensor data.

% To evaluate \narce{}, we introduce a multimodal dataset with fine-grained event categories, addressing gaps in prior work. Through extensive experiments, we compare baseline models with different neural and neuro-symbolic architectures. These experiments reveal that the state-space model Mamba is particularly well-suited for the online CED task due to its similarity to state machines, which naturally model the progression of complex events. Building on these insights, \narce{} succeeds in three aspects, achieving higher detection accuracy, better generalization to out-of-distribution and extended sensor data, and requiring much less labeld sensor data for training, mitigating the heavy burden on sensor data collection and annotations.

% 



% While end-to-end neural architectures can effectively process noisy and uncertain sensor data, their limited context sizes and reasoning abilities restrict them from identifying complex events. Recent \textbf{Complex Event Processing (CEP)} systems \cite{ROIGVILAMALA2023119376} \cite{xing2020neuroplex} \cite{mao2021TOQ} based on neuro-symbolic methods combine data-driven neural models with symbolic models based on human knowledge. However, there is a lack of thorough comparison of these two approaches to complex event detection in performing temporal reasoning over noisy sensor data.

% To address this question, we explore in depth the performance of various neural and neurosymbolic architectures. Specifically, we formulate a multimodal CED task to recognize \textit{CE} patterns in real-time using IMU and acoustic streams. We evaluate (i) \textit{end-to-end neural} architectures, (ii) two-stage \textit{concept-based neural} architectures, and (iii) a \textit{neuro-symbolic} architecture on our synthesized 5-min \textit{CE} dataset. Our empirical analysis indicates the neuro-symbolic approach consistently outperforms purely neural architectures by $41\%$ on the average $F1$ score, even when the neural models were trained using a large training set of \textit{CE} examples, and had context sizes larger than the length of the longest temporal \textit{CE} patterns to detect the entire \textit{CE}.

% We present \narce{}, a \textbf{N}eural \textbf{A}lgorithmic \textbf{R}easoner(NAR) framework for Online \textbf{C}omplex \textbf{E}vent Detection. Inspired by Neural Algorithmic Reasoners (NAR), which integrate the structured reasoning of algorithms with the flexibility of neural networks, NARCE addresses CED by decoupling the task into two components: (i) a reasoning module trained on synthetic concept traces to learn event rules independently of sensor data and (ii) an adapter that maps sensor inputs to the reasoning module. By leveraging synthetic data from large language models (LLMs), NARCE reduces reliance on labeled sensor data.

% In this work, we address Online Complex Event Detection (CED) by formulating the task and introducing a multimodal dataset with inertial and acoustic data. Our dataset introduces fine-grained categories of complex events with corresponding patterns, addressing gaps in prior works that lack such structure or dedicated datasets.

% The key contribution of this work is \narce{}, a \textbf{N}eural \textbf{A}lgorithmic \textbf{R}easoner(NAR) framework for Online \textbf{C}omplex \textbf{E}vent Detection. \narce{} uses Mamba as its NAR backbone and simplifies the complex event learning task by decoupling it into two parts: a reasoning module that learns complex event rules and an adapter that links these rules to sensor data. The reasoning module, based on symbolic concept traces expressed as language tokens (e.g., sequences like [``walk", ``wash", ``wash", ...]), learns event rules independently of sensor data. This process is further enhanced using synthetic data generated by large language models (LLMs), significantly reducing the reliance on real sensor data. The sensor adapter, on the other hand, focuses on interpreting sensor inputs.

% Through extensive experiments, we compare baseline models with different neural and neuro-symbolic architectures. These experiments reveal that the state-space model Mamba is particularly well-suited for the online CED task due to its similarity to state machines, which naturally model the progression of complex events. Building on these insights, \narce{} succeeds in three aspects, achieving higher detection accuracy, better generalization to out-of-distribution and extended sensor data, and requiring much less labeld sensor data for training, mitigating the heavy burden on sensor data collection and annotations. These findings show \narce{}’s effectiveness in advancing robust and efficient CED solutions.