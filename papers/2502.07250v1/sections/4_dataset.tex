\subsection{Multimodal Complex Event Dataset}  
As no large-scale dataset exists for online CED, we develop a multimodal dataset in a smart health monitoring setting, reflecting the common use of multimodal sensors in realistic \emph{CE} tasks for richer information. The dataset includes 10 \emph{CE} classes spanning various categories, with detailed rule definitions provided in Table~\ref{tab:complex_events}.




To generate \emph{CE} sensor traces, we built a stochastic simulator that mimics daily human behaviors, producing random \emph{AE} labels following realistic distributions every 5 seconds. These \emph{AE} traces were then used to synthesize corresponding sensor traces by sampling from WISDM~\cite{wisdm} for IMU data and ESC50~\cite{esc50} for audio data. Details of the simulator are provided in Appendix~\ref{sec:simulator}.

Ground-truth \emph{CE} labels were generated using FSMs, one for each \emph{CE} class, which also serve as the FSM component in the \emph{Neural AE + FSM} model. The simulator first produces ground-truth \emph{AE} labels for each generated sequence, which are passed to the FSMs to determine the corresponding online \emph{CE} labels. Examples of these FSMs can be found in Appendix~\ref{sec:fsm}.

\textbf{Training Data.}
The training dataset consists of 5-minute \emph{CE} sensor traces synthesized using inertial and audio data from multiple subjects. Each \emph{CE} sequence contains 60 windows ($5min \times 60sec/min \div 5~sec = 60$). We generated 10,000 training examples and 2,000 validation examples. Some traces may not contain any \emph{CE} occurrences, reflecting real-world scenarios where complex events may be sparse.

\textbf{Test Data.}
The test dataset uses synthesized sensor traces from an unseen held-out subject. It includes 2,000 examples of 5-minute \emph{CE} sequences. Additionally, we created longer test datasets with 2,000 examples each for 15-minute and 30-minute sequences, which serve as out-of-distribution (OOD) datasets. These longer sequences follow the same \emph{CE} patterns defined in Table~\ref{tab:complex_events}, but with extended \emph{AE} durations and longer temporal gaps between key \emph{AEs}, introducing additional challenges for generalization.

