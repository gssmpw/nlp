\begin{abstract}
% Current machine learning models excel in short-span perception tasks but struggle to derive high-level insights from long-term observation - a capability central to human understanding of the world. To bridge this gap, we focus on the concept of \emph{complex events}, defined as sequences of short-term \emph{atomic events} governed by diverse (spatio)temporal rules. Online detection of \emph{CE}s is challenging, as it requires extracting meaningful patterns within long sensor data while ignoring irrelevant atomic events. In this work, we propose \narce{}, a Neural Algorithmic Reasoning (NAR) framework to address the challenges. Building on insights from evaluating various neural and neuro-symbolic architectures on a multimodal dataset we developed (combining inertial and acoustic sensor data), \narce{} leverages the state-space model Mamba as its backbone. It decouples the complexity of the task into two components: (i) a symbolic NAR module that learns complex event rules from concept traces synthesized by LLMs and (ii) an adapter that maps sensor data to the reasoning module’s latent space. Our results demonstrate that \narce{} outperforms baselines in accuracy and generalization to unseen, extended sensor traces, while using far less annotated data, significantly reducing the cost of complex event data collection and labeling.


Current machine learning models excel in short-span perception tasks but struggle to derive high-level insights from long-term observation, a capability central to understanding \emph{complex events} (\emph{CE}s). CEs, defined as sequences of short-term \emph{atomic events} (\emph{AE}s) governed by spatiotemporal rules, are challenging to detect online due to the need to extract meaningful patterns from long and noisy sensor data while ignoring irrelevant events. We hypothesize that state-based methods are well-suited for CE detection, as they capture event progression through state transitions without requiring long-term memory. Baseline experiments validate this, demonstrating that the state-space model Mamba outperforms existing architectures. However, Mamba's reliance on extensive labeled data, which are difficult to obtain, motivates our second hypothesis: decoupling CE rule learning from noisy sensor data can reduce data requirements. To address this, we propose \narce{}, a framework that combines Neural Algorithmic Reasoning (NAR) to split the task into two components: (i) learning CE rules independently of sensor data using synthetic concept traces generated by LLMs and (ii) mapping sensor inputs to these rules via an adapter. Our results show that \narce{} outperforms baselines in accuracy, generalization to unseen and longer sensor data, and data efficiency, significantly reducing annotation costs while advancing robust CE detection. 

% Perception over long-term horizon, do not use short-term. In particular, crucial to such understanding, is to detect complex events (italics) which we define as patterns of short-term or instanate to diverse spatiotemporal 
% state based model which is Mamba
% annotating with sensor data which is not directly interpretable to humans , and burdensome
% NARCE trained on a large amount of synthetic data and a small amount
% add one sentence numbers accuracy improvement and less data used
% need to add sensor data type (inertial + acoustic) abstract
% three metrics, generalization, and data hungry along all those dimension, narce succeeds
% do not use background, use irrelevant

\end{abstract}

% nature of complex events - state spaced model have advantage?
% learning neuram models are data-hungry, can we use NAR to clean the roles of rule learning part and sensor learning part (hypothesis)

% The dataset: describe why you choose those events

% write paper with two hypothesis

% Current machine learning models excel in short-span perception tasks but struggle to derive high-level insights from long-term observation, a capability central to understanding \emph{complex events} (\emph{CE}s). CEs, defined as sequences of short-term \emph{atomic events} (\emph{AE}s) governed by spatiotemporal rules, are challenging to detect online due to the need to extract meaningful patterns from long and noisy sensor data while ignoring irrelevant events. We propose to leverage Mamba, a state-space model for CE detection, based on our hypothesis that state-based methods are well-suited as they capture event progression through state transitions without requiring long-term memory, with baseline experiments validating this hypothesis. However, Mamba’s reliance on extensive labeled sensor data, which are challenging to obtain, motivates our second hypothesis: decoupling CE rule learning from noisy sensor data can reduce data requirements. To address this, we propose NARCE, a framework that combines Neural Algorithmic Reasoning (NAR) to split the task into two components: (i) learning CE rules independently of sensor data using synthetic concept traces generated by LLMs and (ii) mapping sensor inputs to these rules via an adapter. Our results show that NARCE outperforms baselines in accuracy, generalization to unseen and extended sensor data, and data efficiency, significantly reducing annotation costs while advancing robust CE detection. 