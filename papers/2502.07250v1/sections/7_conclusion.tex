\section{Conclusion}
In this work, we propose \narce{}, a Neural Algorithmic Reasoning framework for efficient and robust online Complex Event Detection. By decoupling complex event rule learning from sensor-specific variations, \narce{} reduces the reliance on large-scale labeled sensor data. Extensive experiments validate our \emph{Hypothesis I} and \emph{Hypothesis II}, showing that Mamba-Based \narce{} achieves comparable or superior performance to baseline models with significantly fewer labeled sensor samples. Our results demonstrate that training with low-cost synthetic pseudo \emph{AE} concept traces enhances generalization to out-of-distribution extended \emph{CE} sequences. These findings underscore the value of pretraining on structured \emph{CE} sequences before adapting to real-world sensor data. Future directions include exploring larger neural architectures, applying \narce{} to other datasets, real-world deployment, and improving the neurosymbolic approach through better probabilistic FSM design, potentially using probabilistic programming languages like ProbLog.

% \textbf{FC: Added further to a conversation with Mani. Please feel free to use, incorporate, or ignore as you see fit}
% \Cref{fig:ce_different_trainingsizes_boxplot} indicates that Neural AE + FSM demonstrates a strong capacity for accurately identifying complex events, surpassing both end-to-end neural methods and neural approaches that reuse the same Neural AE. Its performance appears comparable to that of a Mamba-based approach trained on 2000 data samples while benefiting from the symbolic representation of tasks, which may contribute to improved explainability and facilitate the adaptation to novel tasks. 

\section*{Acknowledgements}
The research reported in this paper was sponsored in part by the DEVCOM Army Research Laboratory (award \# W911NF1720196 ), the Air Force Office of Scientific Research (awards \#  FA95502210193 and FA95502310559), the National Institutes of Health (award \# 1P41EB028242), the National Science Foundation under award \#2325956, and the European Office of Aerospace Research \& Development (EOARD) under award number FA8655-22-1-7017. The views and conclusions contained in this document are those of the authors and should not be interpreted as representing the official policies, either expressed or implied, of the funding agencies.


\section*{Impact Statement}
This paper presents work whose goal is to advance the field of Machine Learning. There are many potential societal consequences of our work, none which we feel must be specifically highlighted here.