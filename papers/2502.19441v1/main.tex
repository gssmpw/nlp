\documentclass[lettersize,journal,compsoc]{IEEEtran}
\usepackage{amsmath,amsfonts}
\usepackage{algorithmic}
\usepackage{algorithm}
\usepackage{array}
\usepackage{caption}
\usepackage[caption=false,font=normalsize,labelfont=sf,textfont=sf]{subfig}%caption=false,
\usepackage{textcomp}
\usepackage{stfloats}
\usepackage{url}
\usepackage{ragged2e}
\usepackage{verbatim}
\usepackage{graphicx}
\usepackage{multirow} 
\usepackage{booktabs}
\usepackage{cite}
\usepackage{color}
\usepackage{xcolor} 
\hyphenation{op-tical net-works semi-conduc-tor IEEE-Xplore}
% updated with editorial comments 8/9/2021

\begin{document}

\title{AniGaussian: Animatable Gaussian Avatar with Pose-guided Deformation}

\author{Mengtian~Li, Shengxiang~Yao, Chen Kai, Zhifeng~Xie$^{*}$, Keyu~Chen$^{*}$, Yu-Gang~Jiang, \IEEEmembership{Fellow, IEEE}
        % <-this % stops a space

\thanks{\textbullet\ $^{*}$: Corresponding author}
\thanks{\textbullet\  Mengtian~Li is with Shanghai University, Fudan University. E-mail: mtli@$\left\{shu \setminus  fudan \right\}$.edu.cn}
\thanks{\textbullet\  Shengxiang~Yao, Chen Kai and Zhifeng~Xie are with Shanghai University.  E-mail:$\left\{ yaosx033 \setminus zhifeng \_ xie \right\}$ @shu.edu.cn,myckai@126.com}
\thanks{\textbullet\ Keyu~Chen is with Tavus Inc. E-mail: keyu@tavus.dev.}
\thanks{\textbullet\ Yu-Gang~Jiang is with the School
of Computer Science, Fudan University. E-mail: ygj@fudan.edu.cn}
}


% The paper headers
\markboth{Journal of \LaTeX\ Class Files,~Vol.~14, No.~8, August~2021}%
{Shell \MakeLowercase{\textit{et al.}}: A Sample Article Using IEEEtran.cls for IEEE Journals}

\IEEEpubid{0000--0000/00\$00.00~\copyright~2021 IEEE}
% Remember, if you use this you must call \IEEEpubidadjcol in the second
% column for its text to clear the IEEEpubid mark.




\IEEEtitleabstractindextext{

% \setcounter{figure}{0}
% \captionsetup{type=figure}
% \noindent\includegraphics[width=0.92\textwidth]{pic/fig1.png}
% \captionof{figure}{{\bfseries Infinite motion}: We propose a novel method for generating infinite motions, based on the timestamps featured in our HumanML3D-Extend dataset. This approach not only enables the generation of extremely long motions but also facilitates precise control over actions within specific time intervals.}
% \label{fig:1}

\vspace{1em}
\begin{abstract}
\justifying
% \begin{abstract}

% Recent works to jointly reconstruct 3D human and object from a single RGB image, are mostly model-based, that fail to capture the fine details of the clothed human body and object surface. In this paper, we introduce ReCHOR, a novel, model-free, first-method to produce realistic clothed human-object reconstructions from a monocular view. This is extremely challenging due to human-object occlusions, diverse interactions and depth ambiguity, as it needs to infer both 3D spatial awareness and high resolution details. Our core idea is based on estimating neural implicit representations for human and object respectively by an attention-based neural implicit model that attends to pixel-aligned features from both the global human-object image for spatial awareness and  the local separate view of human and object images for high quality details. Additionally, the network is conditioned on semantic features from an initial estimated human-object pose prior and a generative diffusion model that inpaints occluded regions, thus enabling the retrieval of details from them.
% We also propose a synthetic dataset with rendered scenes of diverse, inter-occluded 3D human and object scans, to train our network. We evaluate our method on the synthetic and real world BEHAVE dataset. Our experiments show that our method outperforms the SOTA in achieving realistic clothed human-object reconstructions.
Recent approaches to jointly reconstruct 3D humans and objects from a single RGB image represent 3D shapes with template-based or coarse models, which fail to capture details of loose clothing on human bodies. In this paper, we introduce a novel implicit approach for jointly reconstructing realistic 3D clothed humans and objects from a monocular view. For the first time, we model both the human and the object with an implicit representation, allowing to capture more realistic details such as clothing. This task is extremely challenging due to human-object occlusions and the lack of 3D information in 2D images, often leading to poor detail reconstruction and depth ambiguity. To address these problems, we propose a novel attention-based neural implicit model that leverages image pixel alignment from both the input human-object image for a global understanding of the human-object scene and from local separate views of the human and object images to improve realism with, for example, clothing details. Additionally, the network is conditioned on semantic features derived from an estimated human-object pose prior, which provides 3D spatial information about the shared space of humans and objects. To handle human occlusion caused by objects, we use a generative diffusion model that inpaints the occluded regions, recovering otherwise lost details. For training and evaluation, we introduce a synthetic dataset featuring rendered scenes of inter-occluded 3D human scans and diverse objects. Extensive evaluation on both synthetic and real-world datasets demonstrates the superior quality of the proposed human-object reconstructions over competitive methods.
\end{abstract}
Recent advancements in Gaussian-based human body reconstruction have achieved notable success in creating animatable avatars. However, there are ongoing challenges to fully exploit the SMPL model's prior knowledge and enhance the visual fidelity of these models to achieve more refined avatar reconstructions. In this paper, we introduce AniGaussian which addresses the above issues with two insights. First, we propose an innovative pose guided deformation strategy that effectively constrains the dynamic Gaussian avatar with SMPL pose guidance, ensuring that the reconstructed model not only captures the detailed surface nuances but also maintains anatomical correctness across a wide range of motions. Second, we tackle the expressiveness limitations of Gaussian models in representing dynamic human bodies. We incorporate rigid-based priors from previous works to enhance the dynamic transform capabilities of the Gaussian model. Furthermore, we introduce a split-with-scale strategy that significantly improves geometry quality. The ablative study experiment demonstrates the effectiveness of our innovative model design. Through extensive comparisons with existing methods, AniGaussian demonstrates superior performance in both qualitative result and quantitative metrics. 
\end{abstract}

\begin{IEEEkeywords}
3D gaussian splatting, avatar reconstruction, animatable avatar
\end{IEEEkeywords}
}
\maketitle

\section{Introduction}
\label{sec:intro}
% Image editing methods in diffusion models depend on user-defined control directions - users can unlock their creativity using these methods by specifying the desired manipulation through prompts~\cite{gandikota2023concept}, reference images~\cite{ruiz2022dreambooth, kumari2022customdiffusion, gal2022image, chen2024trainingfreeregionalpromptingdiffusion}, or attribute vectors~\cite{parmar2023zero,hertz2022prompt}. In this work, we ask a fundamentally different question: \emph{Can we automatically discover the underlying visual structure of a concept within diffusion model's knowledge?} %Rather than requiring user-specified controls, we aim to decompose the model's internal knowledge into meaningful directions.

% This question touches on a fundamental limitation in how we interact with diffusion models. Current control methods ~\cite{zhang2023addingconditionalcontroltexttoimage, gandikota2023concept, ye2023ipadaptertextcompatibleimage,ye2023ipadaptertextcompatibleimage, hertz2024stylealignedimagegeneration, li2023photomaker, shi2024instantbooth, chen2024trainingfreeregionalpromptingdiffusion} require users to specify their desired manipulations in advance, limiting interactive creativity. This contrasts with natural human artistic workflows, where creators dynamically explore creative ideas while jointly refining them toward meaningful artistic outcomes~\cite{hoffmann2016modeling}. This synergy between specification and exploration is not new to generative models. Early GAN architectures naturally developed disentangled latent spaces that enabled continuous\cite{harkonen2020ganspace,radford2015unsupervised, wu2021stylespace, shen2020interfacegan}, compositional control over generated images. Users could explore these spaces to discover interesting variations that would be difficult to describe in words~\cite{wu2021stylespace}, then combine them to achieve their creative goals~\cite{grabe2022towards}. 


% While diffusion models have largely superseded GANs in conditional image synthesis~\cite{dhariwal2021diffusion},  their underlying structure remains less understood. Diffusion models achieve remarkable diversity through high-dimensional latents, unlike GANs' compact latent spaces.  With a single prompt, diffusion models can generate radically different variations through different random initializations of input noise. We ask - Is it possible to discover interpretable structure within this vast space of variations?

Text-to-image diffusion models are capable of generating remarkable visual variations from a single prompt through different random initializations. However, this vast creative potential remains largely opaque to users---while we can generate diverse images, we lack understanding of the underlying structure of these variations. This presents a fundamental challenge: how can we discover and expose the latent visual capabilities encoded within these models?

\let\thefootnote\relax \footnote{$^{*}$Correspondence to \texttt{gandikota.ro@northeastern.edu}}

The challenge touches on a key limitation in how we interact with diffusion models today. Current control methods require users to explicitly specify their desired edits in advance through prompts~\cite{gandikota2023concept}, reference images~\cite{zhang2023addingconditionalcontroltexttoimage, chen2024trainingfreeregionalpromptingdiffusion, ruiz2022dreambooth,kumari2022customdiffusion, Ryu_lora, hu2021lora}, or attribute vectors~\cite{ye2023ipadaptertextcompatibleimage, hertz2024stylealignedimagegeneration, li2023photomaker, shi2024instantbooth,parmar2023zero,hertz2022prompt}. That contrasts sharply with natural human creative workflows, where artists dynamically explore creative ideas and jointly refine them toward meaningful artistic outcomes~\cite{hoffmann2016modeling}. The need for pre-specified controls creates a barrier between users and the full creative potential of these models.

Interestingly, earlier generative models like GANs~\cite{gans,karras2019style,brock2018large} naturally developed more interpretable internal structures. Their compact latent spaces often exhibited emergent disentanglement~\cite{harkonen2020ganspace,radford2015unsupervised, wu2021stylespace, shen2020interfacegan}, enabling continuous and compositional control over generated images. Users could explore these spaces to discover interesting variations that would be difficult to describe in words~\cite{wu2021stylespace}, then combine them to achieve their creative goals~\cite{grabe2022towards}.

Diffusion models have largely superseded GANs in conditional image synthesis~\cite{dhariwal2021diffusion}, achieving greater diversity through much higher-dimensional latents. And yet an understanding of the underlying structure of these larger latent spaces has remained elusive. In this work, we ask a fundamental question: \emph{Can we automatically discover the visual structure within a diffusion model's knowledge of a concept?} Rather than requiring user-specified controls, we aim to decompose the model's internal representations into expressive directions that users can explore and combine.

To address these needs, we present \textbf{SliderSpace}, a framework that brings systematic explorability to diffusion models. Given just a text prompt, SliderSpace discovers a canonical set of meaningful, diverse, and controllable directions within the model's knowledge of that concept. Each direction is implemented as a low-rank adapter~\cite{hu2021lora} that can be scaled and composed with others, allowing users to explore and smoothly combine different aspects of variation, as shown in Figure~\ref{fig:intro}.

We ground SliderSpace discovery in three key requirements for meaningful decomposition of a diffusion model's visual manifold: 
\begin{enumerate}
    \item \textbf{Unsupervised Discovery:} The decomposition process should emerge from the intrinsic structure of the model's learned representation, rather than being guided by predefined attributes. This ensures we capture the true topology of the model's knowledge space rather than projecting our assumptions onto it.
    
    \item \textbf{Semantic Orthogonality:} Each discovered control must represent a distinct semantic direction. This is enforced in a semantic feature space, like CLIP, where every slider has an orthogonal effect in embeddings. This prevents discovering multiple controls that create similar semantic effects, making the system more efficient and easier.
    
    \item \textbf{Distribution Consistency:} Directions must induce consistent transformations across both random seeds and prompt variations. 
\end{enumerate}

These requirements naturally lead to our proposed framework, which we formalize in Section~\ref{sec:method}. As we show in our experiments, SliderSpace is architecture-agnostic, working with both conventional U-Net based models like Stable Diffusion~\cite{rombach2022high, rombach2022sd20, podell2023sdxl, turbo, dmd} and recent transformer-based architectures like Flux~\cite{flux}.

We demonstrate the expressiveness of SliderSpace through three applications: First, we show how SliderSpace can decompose high-level concepts into diverse and expressive components, revealing the natural axes of variation in the model's understanding. Second, we explore artistic style variation, where SliderSpace discovers directions that match or exceed the diversity of manually curated artist lists while being judged more useful by human evaluators. Finally, we show how SliderSpace can help reverse the mode collapse commonly observed in distilled diffusion models, restoring diversity while maintaining generation speed.

Beyond providing practical creative control, SliderSpace opens new avenues for understanding and utilizing the latent capabilities of diffusion models. By mapping these models' visual potential into intuitive, composable directions, we take a step toward making their creative possibilities more accessible and interpretable to users.

% Image editing methods in diffusion models unlock the creativity of users. In this work we ask an alternate question: \emph{Can we organize and expose what of the diffusion model is already capable of?}.
% Existing methods for controlling image generation typically require users to manually specify edit directions for desired changes. This process is time-consuming, requires technical expertise, and limits the spontaneity of the creative process. For instance, if a user wants to adjust the smile of a generated person, they must explicitly request this edit, often through imprecise prompt engineering or model fine-tuning. This approach of predefined controls or manual specifications restricts users from fully exploring the latent capabilities of the model. There may be interesting stylistic variations or attributes that the model can generate, but users have no easy way to discover or utilize these.

% Natural visual disentanglement was an emergent property in the latent space of Generative Adversarial Models (GANs) \cite{harkonen2020ganspace,radford2015unsupervised, wu2021stylespace, shen2020interfacegan}. In particular, it has been observed that StyleGAN~\cite{karras2019style} stylespace neurons offer detailed control over many meaningful aspects of images that would be difficult to describe in words~\cite{wu2021stylespace}. However, diffusion models do not share such a compact latent space~\cite{park2023unsupervised}; and efforts to uncover such a space in the semantic embeddings of the text conditioning have met with limited success \nik{Nick - is there a specific citation you were thinking about?}.

% In this work we introduce \textbf{SliderSpace}, which takes a step towards uncovering an analogous low dimensional representation of diffusion models' visual breadth; in essence treating the diffusion model as many generators sharing parameters, where a particular generator is defined by a specific prompt. For a given prompt we sample many random seeds (and optionally prompt expansions using an LLM), generate the corresponding images, and apply an off the shelf feature extractor (in this work CLIP, but our method can be applied to any differentiable feature extractor). We use PCA to analyze these features, and for each of the leading $k$ principal components we train a LoRA \cite{} which causes the diffusion model to produces images which increase the feature magnitude along that component when passed back through the same feature extractor. This leads to a 'Slider' for each principal component, because each LoRA can be scaled and applied to the original diffusion model, continuously varying those visual features in the generated results (as measured, in our case, by CLIP).

% There are many other works that enhance the controllability of diffusion models. One common approach is enabling users to add spatial constraints to a generation either manually, or via a reference image \cite{zhang2023addingconditionalcontroltexttoimage, chen2024trainingfreeregionalpromptingdiffusion}, a second is leveraging more abstract embeddings (e.g. identity, style) extracted from a reference image \cite{ye2023ipadaptertextcompatibleimage, hertz2024stylealignedimagegeneration, li2023photomaker, shi2024instantbooth}, a third is finetuning a foundation model to better generate a concept important to the user \cite{ruiz2022dreambooth, kumari2022customdiffusion, Ryu_lora, hu2021lora}, and a fourth (most relevant to this work) is finding low-rank adaptors of the model based on a prompt or small training set which can be scaled to provide continous control over one aspect of generated image (e.g. night vs day, basic vs luxury, etc.) \cite{gandikota2023concept}. SliderSpace is complementary to all of these methods and offers something distinct. All of the other methods we are aware require the user (and / or model designer) to know in advance what type of control they want. In contrast SliderSpace assists users in discovering and controlling hidden capabilities present in the diffusion model's distribution of possible generations.

%We propose that truly intuitive creative control in a text-to-image model should meet three key criteria: \emph{discoverability}, \emph{intuitiveness}, and \emph{specificity}. The model should reveal controllable attributes that may not be immediately obvious, offer controls that are easy to understand and manipulate, and ensure each control affects a distinct attribute of the generated image.

% We demonstrate the utility and power of SliderSpace using three applications built on top of SDXL-DMD \cite{dmd}, because its fast generation speed lends itself well to the continuous control offered by SliderSpace.

% First, we study concept decomposition (Section \ref{sec:concept_exp}), where we learn sliders for a specific concept (e.g. 'monster', 'waterfall', 'car'). Through quantitative metrics of diversity and text alignment we demonstrate that the learned sliders dramatically boost the diversity of generations when randomly applied without harming text alignment; we also ask humans to qualitatively judge these results in a user study where they find the SliderSpace results to be more 'Diverse', 'Useful', and 'Creative' than our baselines.

% Second, we attempt to compare the automatic discoveries of SliderSpace to a large scale manual study of artistic styles (Section \ref{sec:art_exp}), open-sourced by ParrotZone \cite{parrotzone}. In this study SDXL was prompted with over 4300 artist names,  and based on visual inspection the cases of successful stylistic mimicry recorded. Quantitatively SliderSpace more closely matches the distribution of artistic variation discovered by ParrotZone than other baselines, and in our user studies was judged to be significantly more 'Diverse' and 'Useful' than the baselines. To our surprise humans even judged SliderSpace results to be slightly more 'Diverse' than the results generated by the manually discovered artist names of \cite{parrotzone}.

% Third, we attempt to use SliderSpace to reverse the mode collapse commonly observed in distilled few-step diffusion models relative to the original teacher model (Section \ref{sec:diverse_exp}). We quantitatively demonstrate that applying SliderSpace to SDXL-DMD leads to more closely matching the distribution of images by the original teacher, SDXL.

%Through extensive experiments on various state-of-the-art text-to-image models, we demonstrate that SliderSpace significantly enhances user control and creative expression in AI-assisted image generation tasks. Our method enables a range of applications, including concept decomposition and control, diversity improvement in generated images, customization dissection and edits, and the exploration of artistic styles inherent in the model.

% SliderSpace goes beyond providing a practical tool for enhanced creative control. By mapping the visual potential of diffusion models it can open new avenues for generative creativity and deepens our understanding of each model's hidden potential.
\section{Related Work}
\label{sec:related work}
% In this section, we review the existing literature on point cloud denoising and unsupervised image denoising.
%-------------------------------------------------------------------------
\subsection{Point cloud denoising}

    \subsubsection{Traditional methods}
Traditional approaches to point cloud denoising include statistical methods \cite{computingpointset2003,definingpointset2004,wlop2009HH}, filtering techniques\cite{pointsetsurfaces2001,Robustmoving2005, zaman2017density}, and optimization-based methods \cite{l1sparse2010,clop2014PR,digne2017bilateral,multi-projection2018duan,hu2020featuregraph} . These techniques often rely on handcrafted features and heuristics to distinguish signal from noise. For example, statistical methods may use distribution models to identify and remove outliers. Filtering methods, such as mean or median filters, operate under the assumption that noise is statistically different from the signal. Optimization-based methods formulate denoising as an energy minimization problem, where regularization terms constrain the solution to ensure certain smoothness cirterion or adherence to prior knowledge.

%-------------------------------------------------------------------------
    \subsubsection{Supervised learning based methods}
In recent years, several deep learning-based methods \cite{rakotosaona2020PCN,hermosilla2019TotalDenoising,luo2020DMR,luo_score-based_2021} have been proposed for point cloud denoising. NPD \cite{NPD2019} is the first learning-based point cloud denoising method that directly processes noisy data without requiring noise characteristics or neighboring point definitions. This approach combines local and global information by projecting noisy points onto estimated reference planes, effectively removing noise and enhancing robustness against variations in noise intensity and curvature. PointCleanNet\cite{rakotosaona2020PCN} first removes outlier points and then combines them with residual connectivity to predict the inverse displacement \cite{Guerrero2017PCPNetLL}, and iteratively shifts noisy points to remove noise. Pistilli \etal proposed GPDNet \cite{gpdnet2020}, which is a graph convolutional network to improve denoising robustness at high noise levels. Luo \etal also proposed  DMRDenoise \cite{luo2020DMR}, which filter
points by first downsampling the noisy inputs and reconstructing the local subsurface to perform point upsampling. However, this resampling method is difficult to maintain a good local shape. ScoreDenoise \cite{luo_score-based_2021} is proposed to tackle the aforementioned issues by iteratively updating the point position in implicit gradient fields learned by neural networks. For inference, they follows an iterative procedure with a decaying step size, which stabilizes point movement and prevents over-correction, allowing points to converge gradually toward the underlying geometry. The authors of \cite{de_Silva_Edirimuni_2023_CVPR} proposed IterativePFN - an iterative method that use a novel loss function that utilizes an adaptive ground truth target at each iteration to capture the relationship between intermediate filtering results during training. Zheng \etal proposed a end-to-end network for joint normal filtering-based point cloud denoising \cite{10173632}. They introduce an auxiliary normal filtering task to enhance the network's ability to remove noise while preserving geometric features more accurately.

Supervised methods can achieve impressive results, but are limited by the availability and quality of the training data, as they typically require paired noisy and clean point clouds to train the neural network. The absence of clean data in real-world scenario pose a significant challenge on applicability of these algorithms.

%-------------------------------------------------------------------------
    \subsubsection{Unsupervised learning methods}
Unsupervised learning-based methods for point cloud denoising do not require ground-truth clean data. Instead, these methods leverage the inherent structure or distribution of the point cloud to guide the denoising process. Unsupervised methods show promise in scenarios where clean data is absent or hard to obtain.

Hermosilla \etal first introduced Total Denoising (TotalDn) \cite{hermosilla2019TotalDenoising} as an unsupervised learning approach for point cloud denoising, relying solely on noisy data without requiring clean ground truth. TotalDn approximates the underlying surfaces by regressing points from the distribution of unstructured total noise, utilizing a spatial prior term to refine the estimation of geometry. 

In DMRDenoise \cite{luo2020DMR}, an unsupervised version is proposed which utilizes a loss function that identify local neighborhoods using a probabilistic Gaussian mask on the k-nearest neighbors, which selectively retains points likely to represent the underlying surface. By leveraging an Earth Mover's Distance (EMD) assignment, it achieves a one-to-one correspondence between input and predicted points, aligning them to reduce noise within local neighborhoods.
This approach enhances robustness to noise and adapts well to varied surface geometries. However, the probabilistic masking and EMD calculation add computational complexity, which can slow down inference in dense or noisy point clouds. 

ScoreDenoise \cite{luo_score-based_2021} also introduced an unsupervised version that employs ensemble score function and an adaptive neighborhood-covering loss for model training.  
Score-u is probably the most relevant work to our method. However, the defined score in \cite{luo_score-based_2021} is only an displacement-alike version of the score function and there is no explicit formula relating the estimated score to the final denoising result. Also, the iterative process is computationally expensive, and can suffer from fluctuation, leading to perturbed and unstable solution.

Most recently, Noise4Denoise \cite{noise4Wang2024} method is proposed that use an additional doubly-noisy point cloud to learn noise displacement by comparing the two noise levels. This approach effectively leverages synthetic noise for training, allowing the model to estimate residuals without relying on clean data.
However, in practical applications, noise parameters are often unknown and variable, making it challenging to replicate the exact conditions assumed during training. This reliance on predefined noise characteristics can limit the model's applicability to real-world scenarios where noise distributions may differ significantly from synthetic settings. 
%-------------------------------------------------------------------------
\subsection{Unsupervised image denoising}
Recently unsupervised image denoising has made significant progress. Non-Bayesian methods include PURE \cite{luisier2010image}, SURE \cite{SURE2018} \textit{etc.}, which are based on various unbiased risk estimator under certain noise distribution. Other methods explore self-similarity in natural images \cite{xu2015patch, doi:10.1137/23M1614456} or exploits the statistical properties of noise to achieve denoising effect \cite{gravel2004method}.  

Noise2Noise \cite{2018Noise2NoiseLI} is a pioneering method that does not require clean data, but it requires multiple noisy versions of the same image for training. To address this limitation, methods such as Noise2Void \cite{2018Noise2VoidL}, Noise2Self \cite{2019Noise2SelfBD}, \textit{etc.}, have been developed, which use only a single noisy image. These methods are particularly important for practical applications where obtaining clean images or multiple noisy realizations of the same image is difficult or impossible. Neighbor2Neighbor \cite{huang2021neighbor2neighbor} proposed a two-step method with a a random neighbor sub-sampler that generates training  pairs and a denosing network. Kim \etal proposed Noise2Score\cite{kim_noise2score_2021}, a novel Bayesian framework for self-supervised image denoising without clean data. The core of Noise2Score is the usage of Tweedie's formula, which provides an explicit representation of the denoised image through a score function. Combined with score function estimation, Noise2Score can be applied to image denoising with any exponential family noise. Kim \etal also proposed the Noise Distribution Adaptive Self-Supervised Image Denoising method \cite{kim_noise_2022}, which further extends the application of Noise2Score by combining the Tweedie distribution with score matching. This enables adaptive handling of various noise distributions and dynamically adjusts the denoising process by estimating noise parameters. On the other hand, Xie \etal \cite{scoreXie2024} broadened the denoising scope of Noise2Score by allowing it to handle complex noise models, such as multiplicative and structurally correlated noise, demonstrating broad applicability to diverse noise models.

These development of unsupervised image denoising method motivate us to explore similar ideas in 3D point cloud denoising.




\section{\methodname{}: Automatic Functionality Annotation Pipeline}
\label{sec: annotation pipeline}
This section introduces \methodname{}, an annotation pipeline (Fig.~\ref{fig: anno pipeline}) that automatically produces contextual element functionality annotations used to enhance VLMs' GUI grounding capabilities.


\begin{table}[t]
\tiny
\centering
\caption{\textbf{Comparing our \methodname{} dataset with existing large-scale UI datasets.} Multi-Res means the samples are collected on devices with various resolutions. Auto Anno. means the samples are collected autonomously. \#Anno. means the number of annotated samples provided by the datasets.}
\label{tab:data comparison}
\begin{tabular}{@{}cccccccc@{}}
\toprule
Dataset & UI Type & \begin{tabular}[c]{@{}c@{}}Multi\\ Res.\end{tabular} & \begin{tabular}[c]{@{}c@{}}Real-world\\ Scenario\end{tabular} & \begin{tabular}[c]{@{}c@{}}Auto\\ Anno. \end{tabular} & \begin{tabular}[c]{@{}c@{}}Contextual\\ Functionality\\ Semantics\end{tabular} & \#Anno. & Task \\ \midrule
WebShop~\citep{yao2022webshop} & Web & \cross & \cross & \cross & \cross & 12k & Web Navigation \\
Mind2Web~\citep{deng2024mind2web} & Web & \cross & \cmark & \cross & \cross & 2.4k & Web Navigation \\
WebArena~\citep{zhou2023webarena} & Web & \cross & \cmark & \cross & \cross & 812 & Web Navigation \\
\midrule
S2W~\citep{Wang2021Screen2WordsAM} & Mobile & \cross & \cmark & \cross & \cross & 112k & Screen Summarization \\
Wid. Cap.~\citep{Li2020WidgetCG} & Mobile & \cross & \cmark & \cross & \cross & 163k & Element Captioning \\
PixelHelp~\citep{Li2020MappingNL} & Mobile & \cross & \cmark & \cross & \cross & 187 & Element Grounding \\
RICOSCA~\citep{Li2020MappingNL} & Mobile & \cross & \cmark & \cross & \cross & 295k & Action Grounding \\
MoTIF~\citep{Burns2022ADF} & Mobile & \cross & \cmark & \cross & \cross & 6k & Mobile Navigation \\
AITW~\citep{rawles2023android} & Mobile & \cross & \cmark & \cross & \cross & 715k & Mobile Navigation \\
RefExp~\citep{Bai2021UIBertLG} & Mobile & \cross & \cmark & \cross & \cross & 20.8k & Element Grounding \\
VWB~\citep{liu2024visualwebbench} & Web & \cross & \cmark & \cross & \cross & 1.5k & Elem. Ground \& Ref. \\
SeeClick Web~\citep{cheng2024seeclick} & Web & \cross & \cmark & \cmark & \cross & 271k & Element Grounding \\
UI REC/REG~\citep{hong2023cogagent} & Web & \cmark & \cmark & \cmark & \cross & 400k & Box2DOM, DOM2Box \\
Ferret-UI~\citep{you2024ferretui} & Mobile & \cmark & \cmark & \cmark & \cross & 250k & Elem. Ground \& Ref. \\
\methodname{} (ours) & Web, Mobile & \cmark & \cmark & \cmark & \cmark & 704k & Functionality Ground \& Ref. \\ \bottomrule
\end{tabular}
\end{table}



\begin{figure}[t]
    \centering
    \includegraphics[width=0.95\linewidth]{figure/AnnoPipeline3.pdf}
    \caption{\textbf{The proposed pipeline for automatic UI functionality annotation.} An LLM is utilized to predict element functionality based on the UI content changes observed during the interaction. LLM-aided rejection and verification are introduced to improve data quality. Finally, the high-quality functionality annotations will be converted to instruction-following data by applying task templates.}
    \label{fig: anno pipeline}
\end{figure}


\subsection{Collecting UI Interaction Trajectories}
Our pipeline initiates by collecting interaction trajectories, which are sequences of UI contents captured by interacting with UI elements. Each trajectory step captures all interactable elements and the accessibility tree (AXTree) that briefly outlines the UI structure, which will be used to generate functionality annotations. To amass these trajectories, we utilize the latest Common Crawl repository as the data source for web UIs and Android Emulator for mobile UIs. Note that illegal websites and Apps are excluded manually from the sources to ensure no pornographic or violent content is included in our dataset. Please refer to Sec.~\ref{sec:supp:record traj detail} for collecting details and data license.

\subsection{Functionality Annotation Based on UI Dynamics}
Subsequently, the pipeline generates functionality annotations for elements in the collected trajectories. Interacting with an element $e$, by clicking or hovering over it, triggers content changes in the UI. In turn, these changes can be used to predict the functionality $f$ of the interacted element. For instance, if clicking an element causes new buttons to appear in a column, we can predict that the element likely functions as a dropdown menu activator (an example in Fig.~\ref{fig: funcpred diff case}).
With this observation, we utilize a capable LLM (i.e., Llama-3-70B~\citep{llama3modelcard}) as a surrogate for humans to summarize an element's functionality based on the UI content changes resulting from interaction. Concretely, we generate compact content differences for AXTrees before ($s_t$) and after ($s_{t+1}$) the interaction using a file-comparing library\footnote{https://docs.python.org/3/library/difflib.html}. Then, we prompt the LLM to thoroughly analyze the UI content changes (addition, deletion, and unchanged lines), present a detailed Chain-of-Thoughts~\citep{wei2022chain} reasoning process explaining how the element affects the UI, and finally summarize the element's functionality.

In cases where element interactions significantly transform the UI and cause lengthy differences—such as navigating to a new screen—we adjust our approach by using UI description changes instead of the AXTree differences. Specifically, we prompt the same LLM to discern the UI hierarchy, describe UI regions, and finally describe the entire UI functionality. After describing the UIs before and after the interaction, the LLM analyzes the description differences, presents reasoning, and summarizes the element's functionality. This annotation process is formulated as:
\begin{equation}
    f = \text{LLM}(p_{\text{anno}}, s_t, s_{t+1})
\end{equation}

where $f$ is the predicted functionality, $p_{\text{anno}}$ is the annotation prompt (Tab.~\ref{tab:supp:funcpred manip prompt} and Tab.~\ref{tab:supp:funcpred nav prompt}). Examples of annotated elements are depicted in Fig.~\ref{fig: our dataset} and more annotation details are explained in Sec.~\ref{sec:supp:anno details}.

\subsection{Removing Invalid Samples via LLM-Aided Rejection}
The collected trajectories may contain invalid samples due to broken UIs, such as incomplete UI loading. These samples are meaningless as they contain corrupted UI content and can mislead the models trained with them.

To filter out these invalid samples, we introduce an LLM-aided rejection approach. Initially, hand-written rules are used to detect obvious broken cases, such as blank UI contents, UIs containing elements indicating content loading, and interaction targets outside of UIs. While these obvious cases constitute a large portion of the invalid samples, there are a few types that are difficult to detect with hand-written rules. For instance, interacting with a “view more” button might unexpectedly redirect the user to a login page instead of the desired information page due to website login restrictions. To identify these challenging samples, we prompt the annotating LLM to also act as a rejector. Specifically, the LLM takes the UI content changes, generated using a file-comparing library, as input, provides detailed reasoning on whether the changes are meaningful for predicting the element's functionality, and finally outputs predictability scores ranging from 0 to 3. This process is formulated as follows:
\begin{equation}
 score = \text{LLM}(p_{\text{reject}}, e, s_t, s_{t+1})
\end{equation}
where $p_{\text{reject}}$ is the rejection prompt (Tab.~\ref{tab:supp:rejection prompt}).

This approach ensures that clear and predictable samples receive higher scores, while those that are ambiguous or unpredictable receive lower scores. For instance, if a button labeled "Show More", upon interaction, clearly adds new content, this sample will considered to provide sufficient changes that can anticipate the content expansion functionality and will get a score of 3. Conversely, if clicking on a "View Profile" link fails to display the profile possibly due to web browser issues, this unpredictable sample will get a score less than 3.

After implementing empirical experiments, we deploy this LLM-based rejector to discard the bottom 30\% of samples based on their scores to strike a balance between the elimination of invalid samples and the preservation of valid ones (More details in Sec.~\ref{suc:supp:reject details}). The samples that pass the hand-written rules and the LLM rejector are subsequently submitted for functionality annotation. Please see representative rejection examples in Fig.~\ref{fig: rejection examples}.

\subsection{Improving Annotation Quality via LLM-Based Verification}
The functionality annotations produced by the LLM probably contain incorrect, ambiguous, and hallucinated samples (See a case in Fig.~\ref{fig: anno pipeline}), which probably misleads the trained VLMs and compromises evaluation accuracy. To improve dataset quality, we prompt LLMs to verify the annotations by checking whether the targeted element $e$ fulfills the intent of the annotated functionality $f$. This process presents the LLMs with the interacted element, its UI context, the UI changes induced by this element, and the functionality generated in the previous annotation process. The LLMs are then tasked with analyzing the UI content changes before predicting whether the interacted element aligns with the given functionality. If the LLMs determine that the interacted element fulfills the functionality given its UI context, the LLMs will grant a full score (An example in Fig.~\ref{fig: verif diff case}). If the interacted element is considered to mismatch the functionality, this functionality can be seen as incorrect as this mismatch indicates that it may not accurately reflect the element's actual role within the UI context.

To mitigate the potential biases in LLMs~\citep{panickssery2024llm, zheng2023judging, bai2024benchmarking}, two different LLMs (i.e., Llama-3-70B~\citep{llama3modelcard} and Mistral-7B-Instruct-v0.2~\citep{mistral}) are employed as verifiers and prompted to output 0-3 scores. The scoring process is formulated as follows:
\begin{equation}
 score = \text{LLM}(p_{\text{verify}}, e, f, s_t, s_{t+1})
\end{equation}
where $p_{\text{verify}}$ denotes the verification prompt (Tab.~\ref{tab:supp:verif prompt}). Only if the two scores are both 3s do we consider the functionality label correct (More details in Sec.~\ref{suc:supp:verif details}). Although this filtering approach seems stringent, we can make up the number of annotations through scaling. 

\begin{figure}[t]
    \centering
    \includegraphics[width=0.9\linewidth]{figure/our_dataset_img.pdf}
    \caption{Element functionality annotations generated by the proposed AutoGUI pipeline for both web and mobile viewpoints.}
    \label{fig: our dataset}
    \vspace{-5mm}
\end{figure}

\subsection{Functionality Grounding and Referring Task Generation}
\vspace{-2mm}
After rejecting, annotating, and verifying, we obtain a high-quality UI functionality dataset containing triplets of \{UI screenshot, Interacted element, Functionality\}. To convert this dataset into an instruction-following dataset for training and evaluation, we generate functionality grounding and referring tasks using diverse prompt templates (see Tab.~\ref{tab:task templates}). To mitigate the difficulty of predicting absolute values for various resolutions, the coordinates of element bounding boxes are all normalized within the range $[0,99]$ (see Fig.~\ref{fig: our dataset} for examples).

\subsection{Explore the \methodname{} Dataset}

\begin{table}[]
\centering
\small
\caption{\textbf{The statistics of the AutoGUI datasets.} The Anno. Tokens and Avg. Words columns show the total number of tokens and the average number of words for the functionality annotations regardless of task templates. The Domains/Apps column shows the number of unique web domains/mobile Apps involved in each split.}
\label{tab:simple data stats}
\begin{tabular}{@{}ccccccc@{}}
\toprule
Split & \#Tasks & Anno. Tokens & Avg. Words & Domains/Apps & Device Ratio   \\                                                                   \midrule
Train & 702k  & 17.9M        & 23.1       & 916     & Web: $54.6\%$, Mobile: $45.4\%$                                              \\ \cmidrule(r){1-6}
Test  & 2k    & 53.4k        & 22.5       & 299     & Web: $50\%$, Mobile: $50\%$                                                                                                               \\ \bottomrule
\end{tabular}
\end{table}

\begin{figure}[t]
    \centering
    \includegraphics[width=1.0\linewidth]{figure/wordcloud_token-dist-comparison.pdf}
    \caption{\textbf{Diversity of the AutoGUI dataset.} \textbf{Left}: The word cloud illustrates the ratios of the verbs representing the main intents in the functionality annotations. \textbf{Right}: Comparing the distributions of the annotation token numbers for our AutoGUI training split, SeeClick Web training data~\citep{cheng2024seeclick}, and Widget Captioning~\citep{Li2020WidgetCG}. The comparison demonstrates that our dataset covers significantly more diverse task lengths.}
    \label{fig: wordcloud and tokdistrib}
\end{figure}
\vspace{-2mm}

The \methodname{} pipeline finally collects 22.4k trajectories, from which we select 2k grounding samples (evenly divided between web and smartphone views) as the test set and remove the trajectories to which these samples belong. Subsequently, 702k samples are randomly selected from the remaining instances to constitute the training set. The statistics of our dataset in Tab.~\ref{tab:simple data stats} and Sec.~\ref{sec:supp:data stats} show that our dataset covers diverse UIs and exhibits variety in lengths and functional semantics of the annotations. Moreover, our dataset presents a unique ensemble of research challenges for developing generalist web agents in real-world settings. As shown in Tab.~\ref{tab:data comparison} and Fig.~\ref{fig: functionality vs others}, our dataset distinguishes itself from existing literature by providing functionality-rich data as well as tasks that require VLMs to discern the contextual functionalities of elements to achieve high grounding accuracy.

\section{Analysis of Data Quality}
This section analyzes the reliability of the proposed annotation pipeline and data quality.

\noindent{\textbf{Comparison with Human Annotation}} To demonstrate the superiority of the proposed automatic annotation pipeline based on open-source LLMs, $N=145$ samples (99 valid and 46 invalid) are randomly selected as a testbed for comparing the annotation correctness of a trained human annotator and the pipeline. Here, correctness is defined as $Correctness = C / (N - R)$, where $C$ and $R$ denote the numbers of correctly annotated and rejected samples, respectively. The denominator subtracts the number of rejected samples as we are more interested in the percentage of correct samples after rejecting the samples considered invalid by the annotator. The authors thoroughly check the annotation results according to the three criteria in Fig.~\ref{fig: check criteria}: 1. Context-specificity. The functionality annotations must include context-specific descriptions to ensure one-to-one mapping between the element and its annotation. 2. Appropriate details. Avoid detailing unnecessary aspects of the UIs to keep the description focused on functionality. 3. No hallucination. The annotations must not include information not grounded in the visual context of the UIs. See more details in Sec.~\ref{sec:supp:humaneval details}.

After experimenting with three runs, Tab.~\ref{tab:ablate autogui} shows that the proposed AutoGUI pipeline achieves high correctness comparable to the trained human annotator (r6 vs. r1). Without rejection and verification (r2), AutoGUI is inferior as it cannot recognize invalid samples. Notably, simply using the rules written by the authors can improve the correctness, which is further enhanced with the LLM-aided rejector (r4 vs. r3). Moreover, utilizing the annotating LLM itself to self-verify its annotations helps AutoGUI surpass the trained annotator (r5 vs. r1). Introducing another LLM verifier (i.e., Mistral-7B-Instruct-v0.2) brings a slight increase which results from Mistral recognizing Llama-3-70B’s incorrect descriptions of how dropdown menu options work. Overall, these results justify the efficacy of the AutoGUI annotation pipeline.

Qualitatively comparing the annotation patterns of the human and AutoGUI (Fig.~\ref{fig: autogui vs human}), we find that AutoGUI employs the strong LLM to generate more detailed and clear annotations which would take significantly more time for the human annotator. This result suggests that the AutoGUI pipeline can lessen the burden of collecting data for training UI-VLMs.

\noindent{\textbf{Impact of LLM Output Uncertainty}} The uncertainty of LLM outputs manifests in annotation, rejection, and verification, possibly impacting the quality of the AutoGUI dataset. To evaluate this impact, we first sample 100 valid samples to test the AutoGUI pipeline for three runs. The consistency rate is 94.5\%, indicating that 94.5\% of the samples possess consistent annotation outcomes (i.e. correct or incorrect) across the runs. We also test the LLM-aided rejector with 46 invalid samples and find that the rejection consistency over three runs is 79.3\%. This indicates that LLM uncertainty impacts this rejection process. Nevertheless, this impact is minor due to the low prevalence of invalid samples (4\% of all samples) that fail the hand-written rules.

In summary, AutoGUI exhibits annotation correctness comparable to that of human annotators and LLM output uncertainty poses a minor impact on the AutoGUI annotation process.



\begin{figure}[t]
    \centering
    \includegraphics[width=0.85\linewidth]{figure/check_criteria_img.pdf}
    \caption{The checking criteria used for comparing AutoGUI pipeline and the human annotator.}
    \label{fig: check criteria}
\end{figure}


\begin{table}[]
\small
\centering
\caption{\textbf{Comparing the AutoGUI and human annotator.} AutoGUI with the proposed rejection and verification achieves annotation correctness comparable to trained human annotators. One LLM means Llama-3-70B and Two LLMs include Mistral-7B-Instruct-v0.2 as well.}
\label{tab:ablate autogui}
\begin{tabular}{@{}ccccc@{}}
\toprule
No. & Annotator  & Rejector   & Verifier              & Correctness \\ \midrule
r1 & Human      & -          & -                     & 95.5\%      \\
r2 & Llama-3-70B & -          & -                     & 64.5\%      \\
r3 & Llama-3-70B & Rules      & -                     & 83.1\%      \\
r4 & Llama-3-70B & Rules+LLM  & -                     & 94.4\%      \\
r5 & Llama-3-70B & Rules+LLM  & One LLM            & 96.0\%      \\
r6 & Llama-3-70B & Rules+LLM & Two LLMs & \textbf{96.7\%}      \\ \bottomrule
\end{tabular}
\end{table}
\vspace{-2mm}


\section{Experiment}
\label{sec:Experiment}

\subsection{Dataset}
Outside Knowledge Visual Question Answering (OK-VQA)~\cite{marino2019ok} is a benchmark dataset designed to evaluate VQA systems that require leveraging external knowledge sources beyond the information present in an image. The dataset consists of 14,055 knowledge-based questions paired with 14,031 images from the COCO dataset~\cite{lin2014microsoft}. These questions span 10 diverse knowledge categories, including domains such as Science and Technology, Geography, Cooking and Food, and Vehicles and Transportation. The questions were crowdsourced via Amazon Mechanical Turk, ensuring they require real-world knowledge to answer, making this dataset significantly more challenging than conventional VQA datasets. 

The dataset is split into 9,009 training samples and 5,046 testing samples, with each question associated with 10 ground-truth answers annotated by human annotators. This multi-answer format helps address ambiguity and variability in responses. Table~\ref{tab:okvqa_details} outlines key statistics and the distribution of questions across various knowledge categories in the Ok-VQA dataset. Baseline evaluations on OK-VQA using state-of-the-art models like MUTAN and Bilinear Attention Networks (BAN) reveal a significant drop in performance compared to traditional VQA datasets. This performance degradation underscores the need for models with enhanced retrieval and reasoning capabilities to incorporate unstructured, open-domain knowledge effectively.

\begin{table}[h!]
    \centering
    \footnotesize
    \setlength{\tabcolsep}{4pt}
    \renewcommand{\arraystretch}{1.2}
    \caption{Key Details of the OK-VQA Dataset}
    \begin{tabular}{|p{3.2cm}|p{4.8cm}|}
        \hline
        \textbf{Attribute}                & \textbf{Details} \\
        \hline
        \textbf{Name}                     & OK-VQA (Outside Knowledge VQA) \\
        \hline
        \textbf{Source}                   & COCO Image Dataset \\
        \hline
        \textbf{Number of Questions}      & 14,055 \\
        \hline
        \textbf{Number of Images}         & 14,031 \\
        \hline
        \textbf{Question Categories}      & 10 Categories \\
        \hline
        \textbf{Categories Breakdown}     & Vehicles \& Transportation (16\%) \newline Brands, Companies \& Products (3\%) \newline Objects, Materials \& Clothing (8\%) \newline Sports \& Recreation (12\%) \newline Cooking \& Food (15\%) \newline Geography, History, Language \& Culture (3\%) \newline People \& Everyday Life (9\%) \newline Plants \& Animals (17\%) \newline Science \& Technology (2\%) \newline Weather \& Climate (3\%) \newline Other (12\%) \\
        \hline
        \textbf{Average Question Length}  & 8.1 words \\
        \hline
        \textbf{Average Answer Length}    & 1.3 words \\
        \hline
        \textbf{Unique Questions}         & 12,591 \\
        \hline
        \textbf{Unique Answers}           & 14,454 \\
        \hline
        \textbf{Answer Annotations}       & 10 answers per question \\
        \hline
        \textbf{Answer Types}             & Open-ended \\
        \hline
        \textbf{Requires External Knowledge} & Yes (e.g., Wikipedia, Common Sense, etc.) \\
        \hline
        \textbf{Typical Knowledge Sources}& Unstructured Text (Wikipedia) \\
        \hline
    \end{tabular}
    \label{tab:okvqa_details}
\end{table}

\subsection{Implementation Details}
The experiments are conducted on Google Colab using a T4 GPU. The NVIDIA T4 GPU features 16 GB of GDDR6 memory, 320 Tensor Cores, and supports mixed-precision computation, making it suitable for deep learning tasks. Due to computational constraints, we evaluate our model on a subset of 100 samples from the OK-VQA dataset~\cite{marino2019ok}.

\subsection{OOD and ID Category Splits}
In our experiments, we evaluate our approach using the OK-VQA dataset~\cite{marino2019ok}, which we split into OOD and ID subsets based on knowledge categories. The OOD categories include Vehicles and Transportation, Brands, Companies and Products, Sports and Recreation, Science and Technology, and Weather and Climate. The ID categories comprise Objects, Materials and Clothing, Cooking and Food, Geography, History, Language and Culture, People and Everyday Life, Plants and Animals, and Other. Using this split, we can assess how well the model generalizes to different categories of knowledge.

\subsection{Patch-Based Image Preprocessing}
For VQA processing, we preprocess each input image by dividing it into patches of various sizes, specifically 2×2, 3×3 and 4x4 grids. This patch-based approach captures fine-grained visual details, which can enhance feature extraction for complex queries. We then employ the BLIP-VQA model~\cite{li2022blip} to extract image representations and generate initial contextual information based on the image and the associated question.

\subsection{Retrieval-Augmented Knowledge Integration}
To incorporate external knowledge, we use  RAG~\cite{lewis2020retrieval} with external knowledge sources such as Wikipedia and DBpedia. RAG retrieves relevant information based on the question and the visual features extracted by BLIP-VQA~\cite{li2022blip}. This retrieval process supplies the model with real-world context beyond the image, which is crucial for correctly answering questions that depend on external knowledge.

\subsection{State-of-the-Art Performance Comparison}
We evaluate our proposed FilterRAG framework on the OK-VQA dataset and compare it to state-of-the-art methods (Table~\ref{table:SOTA-OK-VQA}). The baseline models, Base1 and Base2, use the BLIP-VQA model with the VQA v2~\cite{goyal2017making} and OK-VQA datasets~\cite{marino2019ok}, achieving 83.0\% and 40.0\% accuracy, respectively. The drop highlights the challenge of knowledge-based questions in OK-VQA. Our FilterRAG framework, which integrates BLIP-VQA, RAG, and external knowledge sources like Wikipedia and DBpedia, achieves 36.5\% accuracy in OOD settings. This result demonstrates the effectiveness of grounding VQA responses with external knowledge, especially for OOD scenarios. 

Compared to state-of-the-art methods, KRISP~\cite{marino2021krisp}  achieves 38.35\% with Wikipedia and ConceptNet, while MAVEx~\cite{wu2022multi} reaches 41.37\% using Wikipedia, ConceptNet, and Google Images. The highest performance comes from KAT (ensemble)~\cite{gui2021kat} at 54.41\% with Wikipedia and Frozen GPT-3. Although these models achieve higher accuracy, they often require significant computational resources. 

FilterRAG balances performance and efficiency, making it suitable for resource-constrained environments. As shown in Figure~\ref{fig:plot1_accuracy}, it achieves 37.0\% accuracy in ID settings, 36.0\% in OOD settings, and 36.5\% when combining ID and OOD data. This highlights its robustness for knowledge-intensive VQA tasks.

\begin{figure}[h!]
    \centering
    \includegraphics[width=\linewidth]{figures/plot1_accuracy_v2.pdf}
    \caption{Comparison of Model Accuracy Across Different Settings.}
    \label{fig:plot1_accuracy}
\end{figure}

\begin{table*}[t]
    \centering
    \footnotesize
    \caption{Performance Comparison of State-of-the-Art Methods on the OK-VQA Dataset}
    \label{tab:okvqa_results}
    \renewcommand{\arraystretch}{1.2}
    \setlength{\tabcolsep}{10pt}
    \begin{tabular}{l l c}
        \toprule
        \textbf{Method}                                & \textbf{External Knowledge Sources}                          & \textbf{Accuracy (\%)} \\
        \midrule
        Q-only (Marino et al., 2019)~\cite{marino2019ok}                  & —                                                          & 14.93                  \\
        MLP (Marino et al., 2019)~\cite{marino2019ok}                     & —                                                          & 20.67                  \\
        BAN (Marino et al., 2019)~\cite{marino2019ok}              & —                                                          & 25.1                  \\
        MUTAN (Marino et al., 2019)~\cite{marino2019ok}               & —                                                          & 26.41                  \\
        ClipCap (Mokady et al., 2021)~\cite{mokady2021clipcap}                 & —                                                          & 22.8                   \\
        \midrule
        BAN + AN (Marino et al., 2019~\cite{marino2019ok}                  & Wikipedia                                                  & 25.61                  \\
        BAN + KG-AUG (Li et al., 2020)~\cite{li2020boosting}        & Wikipedia + ConceptNet                                     & 26.71                  \\
        Mucko (Zhu et al., 2020)~\cite{zhu2020mucko}                      & Dense Caption                                              & 29.2                   \\
        ConceptBERT (Gardères et al., 2020)~\cite{garderes2020conceptbert}           & ConceptNet                                                 & 33.66                  \\
        KRISP (Marino et al., 2021)~\cite{marino2021krisp}                   & Wikipedia + ConceptNet                                     & 38.35                  \\
        RVL (Shevchenko et al., 2021)~\cite{shevchenko2021reasoning}                 & Wikipedia + ConceptNet                                     & 39.0                   \\
        Vis-DPR (Luo et al., 2021)~\cite{luo2021weakly}                    & Google Search                                              & 39.2                   \\
        MAVEx (Wu et al., 2022)~\cite{wu2022multi}                       & Wikipedia + ConceptNet + Google Images                    & 41.37                  \\
        PICa-Full (Yang et al., 2022)~\cite{yang2022empirical}                 & Frozen GPT-3 (175B)                                        & 48.0                   \\
        KAT (Gui et al., 2022) (Ensemble)~\cite{gui2021kat}             & Wikipedia + Frozen GPT-3 (175B)                           & 54.41                  \\
        REVIVE (Lin et al., 2022) (Ensemble)~\cite{lin2022revive}          & Wikipedia + Frozen GPT-3 (175B)                           & 58.0                   \\
        RASO (Fu et al., 2023)~\cite{fu2023generate}                        & Wikipedia + Frozen Codex                                   & 58.5                   \\
        \midrule
        \textbf{FilterRAG (Ours)}                     & Wikipedia + DBpedia (\textbf{Frozen} BLIP-VQA and GPT-Neo 1.3B)    & \textbf{36.5}          \\
        \bottomrule
    \end{tabular}
    \label{table:SOTA-OK-VQA}
\end{table*}


\subsection{Hallucination Detection via Grounding Scores}
We evaluate the grounding scores of our FilterRAG framework against baseline models to assess its ability to mitigate hallucinations by aligning answers with external knowledge. As shown in Figure~\ref{fig:plot2_grounding_score}, Base1 achieves the highest grounding score of 94.60\% on the VQA v2 dataset~\cite{goyal2017making}, indicating that BLIP performs effectively when answering general-domain questions that do not require external knowledge. In contrast, Base2, evaluated on the OK-VQA dataset~\cite{marino2019ok}, shows a significant drop to 71.70\%, highlighting the challenge of answering knowledge-based questions without access to external information, thereby increasing the likelihood of hallucinations.

\begin{figure}[h!]
    \centering
    \includegraphics[width=\linewidth]{figures/plot2_grounding_score_v2.pdf}
    \caption{Grounding Score Comparison Across Baselines and Proposed Methods.}
    \label{fig:plot2_grounding_score}
\end{figure}

To address this limitation, our proposed method integrates BLIP-VQA, RAG, and external knowledge sources such as Wikipedia and DBpedia. The grounding scores for our method are 70.06\% for In-Distribution (ID) data, 70.68\% for Out-of-Distribution (OOD) data, and 70.37\% when combining both settings. These consistent scores demonstrate that FilterRAG effectively grounds answers in retrieved knowledge, reducing hallucinations even in challenging OOD scenarios.

Although our method does not achieve the grounding performance of Base1, it provides reliable results for knowledge-intensive tasks by leveraging external knowledge sources. This makes FilterRAG a robust and efficient solution for real-world VQA applications, particularly where external knowledge and OOD generalization are critical.

\subsection{Ablation Study}
We evaluate the effect of different image grid sizes on the performance of our FilterRAG framework with BLIP-VQA and RAG in OOD scenarios. We consider three grid configurations, 2x2, 3x3, and 4x4, and evaluate their influence on accuracy and grounding score. As shown in Figure~\ref{fig:plot5_measure_grid_size}, accuracy decreases slightly as the grid size increases. The accuracy is 37.00\% for the 2x2 grid, declines to 35.00\% for the 3x3 grid, and further drops to 34.00\% for the 4x4 grid. This downward trend indicates that larger grid sizes lead to excessive fragmentation, making it challenging for the model to extract coherent and meaningful visual features.

\begin{figure}[h!]
    \centering
    \includegraphics[width=\linewidth]{figures/plot5_measure_grid_size_v2.pdf}
    \caption{Effect of Grid Sizes on Accuracy and Grounding Score.}
    \label{fig:plot5_measure_grid_size}
\end{figure}

Similarly, the grounding score follows a declining trend with increasing grid size. The grounding score is 70.06\% for the 2x2 grid, reducing to 69.20\% for the 3x3 grid and 68.07\% for the 4x4 grid. This decline suggests that finer grid divisions hinder the model’s ability to align generated answers with retrieved external knowledge, likely due to the loss of contextual coherence when images are broken into smaller patches.

Overall, the 2x2 grid size achieves the best trade-off between accuracy and grounding score. It maintains both visual coherence and effective knowledge alignment, thereby reducing the risk of hallucinations. Consequently, for OOD scenarios in the FilterRAG framework, the 2x2 grid configuration is the most effective for ensuring robust and reliable performance.

\subsection{Qualitative Analysis}
We perform a qualitative analysis of FilterRAG on the OK-VQA dataset~\cite{marino2019ok}, evaluating its performance in both In-Domain (ID) and Out-of-Distribution (OOD) settings. As illustrated in Figure~\ref{fig:Qualitative_Analysis}, FilterRAG generates accurate answers in ID scenarios where the retrieved knowledge is relevant and aligns well with the visual context. In these cases, the model effectively combines visual cues and external knowledge, resulting in well-grounded responses. These errors are frequently caused by misalignment between the visual context and the retrieved information, reflecting the challenge of handling ambiguous or novel queries outside the training distribution.

In OOD settings, FilterRAG struggles when relevant knowledge of unfamiliar concepts cannot be effectively retrieved. This often leads to hallucinations, where the model produces plausible but incorrect answers that are not supported by the retrieved evidence. This analysis highlights the critical role of reliable knowledge retrieval and precise multimodal alignment in mitigating hallucinations. Improving the quality of knowledge retrieval and refining visual-textual alignment are essential steps toward making FilterRAG more reliable in OOD contexts. Future improvements in these areas can help ensure more accurate and context-aware responses in real-world VQA applications.

\section{Conclusion}

%In this paper, w
We propose a new PEFT method called DiffoRA, which enables efficient and adaptive LLM fine-tuning based on LoRA. 
Instead of adjusting every interior rank, 
%of the decomposition matrices 
%of all modules, 
we argue that adopting LoRA module-wisely is sufficient. 
To achieve this, we construct a DAM to select the modules that are most suitable and essential to fine-tune. We theoretically analyze how the DAM impacts the convergence rate and generalization capability.
%of the pre-trained model. 
Furthermore, we adopt continuous relaxation and discretization to establish DAM.
%for each task. 
To alleviate the issue of discretization discrepancy, we utilize the weight-sharing strategy for optimization. 
%We fully implement our method and t
The experimental results demonstrate that our DiffoRA works consistently better than the baselines across all benchmarks. 

\bibliographystyle{IEEEtran}
\bibliography{references}

\newpage

\begin{IEEEbiography}
[{\includegraphics[width=1in,height=1.25in,clip,keepaspectratio]{photo/mtli.jpg}}]{Mengtian Li} currently hold a position as a Lecturer of Shanghai University, while simultaneously fulfilling the responsibilities of a Post-doc of Fudan University. She received Ph.D. degree from East China Normal University, Shanghai, China, in 2022. She serves as reviewers for CVPR, ICCV, ECCV, ICML, ICLR, NeurIPS , IEEE TIP and PR, etc. Her research lies in 3D vision and computer graphics, focuses at human avatar animating and 3D scene understanding, reconstruction, generation.
\end{IEEEbiography}

% \begin{IEEEbiography}
% [{\includegraphics[width=1in,height=1.25in,clip,keepaspectratio]{photo/zhai.jpg}}]{Chengshuo Zhai} received her Bachelor's degree in Digital Media Technology from Yanbian University in 2018 and is currently pursuing a Master's degree at the Shanghai Film Academy, Shanghai University. Her primary research focus is on human motion, specifically the generation of cross-modal human motion sequences.
% \end{IEEEbiography}
\vspace{-33pt}
\begin{IEEEbiography}
[{\includegraphics[width=1in,height=1.25in,clip,keepaspectratio]{photo/ysx.jpg}}]{Yaosheng Xiang} received his Bachelor's degree from Huaqiao University and is currently pursuing a Master's degree at the Shanghai Film Academy, part of Shanghai University. His research interests primarily focus on digital human reconstruction, with an emphasis on creating and editing animatable avatars from video footage.
\end{IEEEbiography}
\vspace{-33pt}
\begin{IEEEbiography}
[{\includegraphics[width=1in,height=1.25in,clip,keepaspectratio]{photo/ck.jpg}}]{Chen Kai} is currently a graduate supervisor at the Shanghai Film Academy of Shanghai University. He is the Director of the Shanghai Film Special Effects Engineering Technology Research Center and the Director of the Shanghai University film-producing workshop. He obtained a Master of Fine Arts (MFA) degree from the École Nationale Supérieure des Beaux-Arts de Le Mans in France, majoring in Contemporary Art. He participated in developing animation software Miarmy won the 70th Tech Emmy Awards presented by NATAS (National Academy of Television Arts \& Sciences) in 2018. His creative pursuits include experimental cinema, photography, digital interactive installations, and other forms of art.
\end{IEEEbiography}
\vspace{-33pt}
\begin{IEEEbiography}
[{\includegraphics[width=1in,height=1.25in,clip,keepaspectratio]{photo/xie.jpg}}]{Zhifeng Xie} received the Ph.D. degree in computer application technology from Shanghai Jiao Tong University, Shanghai, China. He was a Research Assistant with the City University of Hong Kong, Hong Kong. He is currently an Associate Professor with the Department of Film and Television Engineering, Shanghai University, Shanghai. He has published several works on CVPR, ECCV, IJCAI, IEEE Transactions on Image Processing, IEEE Transactions on Neural Networks and Learning Systems, and IEEE Transactions on Circuits and Systems for Video Technology. His current research interests include image/video processing and computer vision.
\end{IEEEbiography}
\vspace{-33pt}
\begin{IEEEbiography}[{\includegraphics[width=1in,height=1.25in,clip,keepaspectratio]{photo/keyu.png}}]{Keyu Chen} is a senior AI researcher affiliated with Tavus Inc.. He obtained the master and bachelor degree from University of Science and Technology of China in 2021 and 2018. His research interests are mainly focused on digital human modeling, animation, and affective analysis.
\end{IEEEbiography}

\vspace{-33pt}
\begin{IEEEbiography}
[{\includegraphics[width=1in,height=1.25in,clip,keepaspectratio]{photo/Yu-GangJiang.jpg}}]{Yu-Gang Jiang} received the Ph.D. degree in Computer Science from City University of Hong Kong in 2009 and worked as a Postdoctoral Research Scientist at Columbia University, New York, from 2009 to 2011. He is currently Vice President and Chang Jiang Scholar Distinguished Professor of Computer Science at Fudan University, Shanghai, China. His research lies in the areas of multimedia, computer vision, and trustworthy AGI. He is a fellow of the IEEE and the IAPR.
\end{IEEEbiography}


\end{document}


