\documentclass[lettersize,journal,compsoc]{IEEEtran}
\usepackage{amsmath,amsfonts}
\usepackage{algorithmic}
\usepackage{algorithm}
\usepackage{array}
\usepackage{caption}
\usepackage[caption=false,font=normalsize,labelfont=sf,textfont=sf]{subfig}%caption=false,
\usepackage{textcomp}
\usepackage{stfloats}
\usepackage{url}
\usepackage{ragged2e}
\usepackage{verbatim}
\usepackage{graphicx}
\usepackage{multirow} 
\usepackage{booktabs}
\usepackage{cite}
\usepackage{color}
\usepackage{xcolor} 
\hyphenation{op-tical net-works semi-conduc-tor IEEE-Xplore}
% updated with editorial comments 8/9/2021

\begin{document}

\title{AniGaussian: Animatable Gaussian Avatar with Pose-guided Deformation}

\author{Mengtian~Li, Shengxiang~Yao, Chen Kai, Zhifeng~Xie$^{*}$, Keyu~Chen$^{*}$, Yu-Gang~Jiang, \IEEEmembership{Fellow, IEEE}
        % <-this % stops a space

\thanks{\textbullet\ $^{*}$: Corresponding author}
\thanks{\textbullet\  Mengtian~Li is with Shanghai University, Fudan University. E-mail: mtli@$\left\{shu \setminus  fudan \right\}$.edu.cn}
\thanks{\textbullet\  Shengxiang~Yao, Chen Kai and Zhifeng~Xie are with Shanghai University.  E-mail:$\left\{ yaosx033 \setminus zhifeng \_ xie \right\}$ @shu.edu.cn,myckai@126.com}
\thanks{\textbullet\ Keyu~Chen is with Tavus Inc. E-mail: keyu@tavus.dev.}
\thanks{\textbullet\ Yu-Gang~Jiang is with the School
of Computer Science, Fudan University. E-mail: ygj@fudan.edu.cn}
}


% The paper headers
\markboth{Journal of \LaTeX\ Class Files,~Vol.~14, No.~8, August~2021}%
{Shell \MakeLowercase{\textit{et al.}}: A Sample Article Using IEEEtran.cls for IEEE Journals}

\IEEEpubid{0000--0000/00\$00.00~\copyright~2021 IEEE}
% Remember, if you use this you must call \IEEEpubidadjcol in the second
% column for its text to clear the IEEEpubid mark.




\IEEEtitleabstractindextext{

% \setcounter{figure}{0}
% \captionsetup{type=figure}
% \noindent\includegraphics[width=0.92\textwidth]{pic/fig1.png}
% \captionof{figure}{{\bfseries Infinite motion}: We propose a novel method for generating infinite motions, based on the timestamps featured in our HumanML3D-Extend dataset. This approach not only enables the generation of extremely long motions but also facilitates precise control over actions within specific time intervals.}
% \label{fig:1}

\vspace{1em}
\begin{abstract}
\justifying
% \begin{abstract}


The choice of representation for geographic location significantly impacts the accuracy of models for a broad range of geospatial tasks, including fine-grained species classification, population density estimation, and biome classification. Recent works like SatCLIP and GeoCLIP learn such representations by contrastively aligning geolocation with co-located images. While these methods work exceptionally well, in this paper, we posit that the current training strategies fail to fully capture the important visual features. We provide an information theoretic perspective on why the resulting embeddings from these methods discard crucial visual information that is important for many downstream tasks. To solve this problem, we propose a novel retrieval-augmented strategy called RANGE. We build our method on the intuition that the visual features of a location can be estimated by combining the visual features from multiple similar-looking locations. We evaluate our method across a wide variety of tasks. Our results show that RANGE outperforms the existing state-of-the-art models with significant margins in most tasks. We show gains of up to 13.1\% on classification tasks and 0.145 $R^2$ on regression tasks. All our code and models will be made available at: \href{https://github.com/mvrl/RANGE}{https://github.com/mvrl/RANGE}.

\end{abstract}


Recent advancements in Gaussian-based human body reconstruction have achieved notable success in creating animatable avatars. However, there are ongoing challenges to fully exploit the SMPL model's prior knowledge and enhance the visual fidelity of these models to achieve more refined avatar reconstructions. In this paper, we introduce AniGaussian which addresses the above issues with two insights. First, we propose an innovative pose guided deformation strategy that effectively constrains the dynamic Gaussian avatar with SMPL pose guidance, ensuring that the reconstructed model not only captures the detailed surface nuances but also maintains anatomical correctness across a wide range of motions. Second, we tackle the expressiveness limitations of Gaussian models in representing dynamic human bodies. We incorporate rigid-based priors from previous works to enhance the dynamic transform capabilities of the Gaussian model. Furthermore, we introduce a split-with-scale strategy that significantly improves geometry quality. The ablative study experiment demonstrates the effectiveness of our innovative model design. Through extensive comparisons with existing methods, AniGaussian demonstrates superior performance in both qualitative result and quantitative metrics. 
\end{abstract}

\begin{IEEEkeywords}
3D gaussian splatting, avatar reconstruction, animatable avatar
\end{IEEEkeywords}
}
\maketitle

\section{Introduction}
Backdoor attacks pose a concealed yet profound security risk to machine learning (ML) models, for which the adversaries can inject a stealth backdoor into the model during training, enabling them to illicitly control the model's output upon encountering predefined inputs. These attacks can even occur without the knowledge of developers or end-users, thereby undermining the trust in ML systems. As ML becomes more deeply embedded in critical sectors like finance, healthcare, and autonomous driving \citep{he2016deep, liu2020computing, tournier2019mrtrix3, adjabi2020past}, the potential damage from backdoor attacks grows, underscoring the emergency for developing robust defense mechanisms against backdoor attacks.

To address the threat of backdoor attacks, researchers have developed a variety of strategies \cite{liu2018fine,wu2021adversarial,wang2019neural,zeng2022adversarial,zhu2023neural,Zhu_2023_ICCV, wei2024shared,wei2024d3}, aimed at purifying backdoors within victim models. These methods are designed to integrate with current deployment workflows seamlessly and have demonstrated significant success in mitigating the effects of backdoor triggers \cite{wubackdoorbench, wu2023defenses, wu2024backdoorbench,dunnett2024countering}.  However, most state-of-the-art (SOTA) backdoor purification methods operate under the assumption that a small clean dataset, often referred to as \textbf{auxiliary dataset}, is available for purification. Such an assumption poses practical challenges, especially in scenarios where data is scarce. To tackle this challenge, efforts have been made to reduce the size of the required auxiliary dataset~\cite{chai2022oneshot,li2023reconstructive, Zhu_2023_ICCV} and even explore dataset-free purification techniques~\cite{zheng2022data,hong2023revisiting,lin2024fusing}. Although these approaches offer some improvements, recent evaluations \cite{dunnett2024countering, wu2024backdoorbench} continue to highlight the importance of sufficient auxiliary data for achieving robust defenses against backdoor attacks.

While significant progress has been made in reducing the size of auxiliary datasets, an equally critical yet underexplored question remains: \emph{how does the nature of the auxiliary dataset affect purification effectiveness?} In  real-world  applications, auxiliary datasets can vary widely, encompassing in-distribution data, synthetic data, or external data from different sources. Understanding how each type of auxiliary dataset influences the purification effectiveness is vital for selecting or constructing the most suitable auxiliary dataset and the corresponding technique. For instance, when multiple datasets are available, understanding how different datasets contribute to purification can guide defenders in selecting or crafting the most appropriate dataset. Conversely, when only limited auxiliary data is accessible, knowing which purification technique works best under those constraints is critical. Therefore, there is an urgent need for a thorough investigation into the impact of auxiliary datasets on purification effectiveness to guide defenders in  enhancing the security of ML systems. 

In this paper, we systematically investigate the critical role of auxiliary datasets in backdoor purification, aiming to bridge the gap between idealized and practical purification scenarios.  Specifically, we first construct a diverse set of auxiliary datasets to emulate real-world conditions, as summarized in Table~\ref{overall}. These datasets include in-distribution data, synthetic data, and external data from other sources. Through an evaluation of SOTA backdoor purification methods across these datasets, we uncover several critical insights: \textbf{1)} In-distribution datasets, particularly those carefully filtered from the original training data of the victim model, effectively preserve the model’s utility for its intended tasks but may fall short in eliminating backdoors. \textbf{2)} Incorporating OOD datasets can help the model forget backdoors but also bring the risk of forgetting critical learned knowledge, significantly degrading its overall performance. Building on these findings, we propose Guided Input Calibration (GIC), a novel technique that enhances backdoor purification by adaptively transforming auxiliary data to better align with the victim model’s learned representations. By leveraging the victim model itself to guide this transformation, GIC optimizes the purification process, striking a balance between preserving model utility and mitigating backdoor threats. Extensive experiments demonstrate that GIC significantly improves the effectiveness of backdoor purification across diverse auxiliary datasets, providing a practical and robust defense solution.

Our main contributions are threefold:
\textbf{1) Impact analysis of auxiliary datasets:} We take the \textbf{first step}  in systematically investigating how different types of auxiliary datasets influence backdoor purification effectiveness. Our findings provide novel insights and serve as a foundation for future research on optimizing dataset selection and construction for enhanced backdoor defense.
%
\textbf{2) Compilation and evaluation of diverse auxiliary datasets:}  We have compiled and rigorously evaluated a diverse set of auxiliary datasets using SOTA purification methods, making our datasets and code publicly available to facilitate and support future research on practical backdoor defense strategies.
%
\textbf{3) Introduction of GIC:} We introduce GIC, the \textbf{first} dedicated solution designed to align auxiliary datasets with the model’s learned representations, significantly enhancing backdoor mitigation across various dataset types. Our approach sets a new benchmark for practical and effective backdoor defense.



\section{Related Work}
\label{sec:relatedwork}

\subsection{Current AI Tools for Social Service}
\label{subsec:relatedtools}
% the title I feel is quite broad

Harnessing technology for social good has always been a grand challenge in social service \cite{berzin_practice_2015}. As early as the 90s, artificial neural networks and predictive models have been employed as tools for risk assessments, decision-making, and workload management in sectors like child protective services and mental health treatment \cite{fluke_artificial_1989, patterson_application_1999}. The recent rise of generative AI is poised to further advance social service practice, facilitating the automation of administrative tasks, streamlining of paperwork and documentation, optimisation of resource allocation, data analysis, and enhancing client support and interventions \cite{fernando_integration_2023, perron_generative_2023}.

Today, AI solutions are increasingly being deployed in both policy and practice \cite{goldkind_social_2021, hodgson_problematising_2022}. In clinical social work, AI has been used for risk assessments, crisis management, public health initiatives, and education and training for practitioners \cite{asakura_call_2020, gillingham2019can, jacobi_functions_2023, liedgren_use_2016, molala_social_2023, rice_piloting_2018, tambe_artificial_2018}. AI has also been employed for mental health support and therapeutic interventions, with conversational agents serving as on-demand virtual counsellors to provide clinical care and support \cite{lisetti_i_2013, reamer_artificial_2023}.
% commercial solutions include Woebot, which simulates therapeutic conversation, and Wysa, an “emotionally intelligent” AI coach, powered by evidenced-based clinical techniques \cite{reamer_artificial_2023}. 
% Non-clinical AI agents like Replika and companion robots can also provide social support and reduce loneliness amongst individuals \cite{ahmed_humanrobot_2024, chaturvedi_social_2023, pani_can_2024, ta_user_2020}.

Present research largely focuses on \textit{\textbf{AI-based decision support tools}} in social service \cite{james_algorithmic_2023, kawakami2022improving}, especially predictive risk models (PRMs) used to predict social service risks and outcomes \cite{gillingham2019can, van2017predicting}, like the Allegheny Family Screening Tool (AFST), which assesses child abuse risk using data from US public systems \cite{chouldechova_case_2018, vaithianathan2017developing}. Elsewhere, researchers have also piloted PRMs to predict social service needs for the homeless using Medicaid data\cite{erickson_automatic_2018, pourat_easy_2023}, and AI-powered algorithms to promote health interventions for at-risk populations, such as HIV testing among Californian homeless \cite{rice_piloting_2018, yadav_maximizing_2017}.

\subsection{Generative AI and Human-AI Collaboration}
\label{subsec:relatedworkhaicollaboration}
Beyond decision-making algorithms and PRMs, advancements in generative AI, such as large language models (LLMs), open new possibilities for human-AI (HAI) collaboration in social services. 
LLMs have been called "revolutionary" \cite{fui2023generative} and a "seismic shift" \cite{cooper2023examining}, offering "content support" \cite{memmert2023towards} by generating realistic and coherent responses to user inputs \cite{cascella2023evaluating}. Their vastly improved capabilities and ubiquity \cite{cooper2023examining} makes them poised to revolutionise work patterns \cite{fui2023generative}. Generative AI is already used in fields like design, writing, music, \cite{han2024teams, suh2021ai, verheijden2023collaborative, dhillon2024shaping, gero2023social} healthcare, and clinical settings \cite{zhang2023generative, yu2023leveraging, biswas2024intelligent}, with promising results. However, the social service sector has been slower in adopting AI \cite{diez2023artificial, kawakami2023training}.

% Yet, the social service sector is one that could perhaps stand to gain the most from AI technologies. As Goldkind \cite{goldkind_social_2021} writes, social service, as a "values-centred profession with a robust code of ethics" (p. 372), is uniquely placed to inform the development of thoughtful algorithmic policy and practice. 
Social service, however, stands to benefit immensely from generative AI. SSPs work in time-poor environments \cite{tiah_can_2024}, often overwhelmed with tedious administrative work \cite{meilvang_working_2023} and large amounts of paperwork and data processing \cite{singer_ai_2023, tiah_can_2024}. 
% As such, workers often work in time-poor environments and are burdened with information overload and administrative tasks \cite{tiah_can_2024, meilvang_working_2023}. 
Generative AI is well-placed to streamline and automate tasks like formatting case notes, formulating treatment plans and writing progress reports, which can free up valuable time for more meaningful work like client engagement and enhance service quality \cite{fernando_integration_2023, perron_generative_2023, tiah_can_2024, thesocialworkaimentor_ai_nodate}. 

Given the immense potential, there has been emerging research interest in HAI collaboration and teamwork in the Human-Computer Interaction and Computer Supported Cooperative Work space \cite{wang_human-human_2020}. HAI collaboration and interaction has been postulated by researchers to contribute to new forms of HAI symbiosis and augmented intelligence, where algorithmic and human agents work in tandem with one another to perform tasks better than they could accomplish alone by augmenting each other's strengths and capabilities  \cite{dave_augmented_2023, jarrahi_artificial_2018}.

However, compared to the focus on AI decision-making and PRM tools, there is scant research on generative AI and HAI collaboration in the social service sector \cite{wykman_artificial_2023}. This study therefore seeks to fill this critical gap by exploring how SSPs use and interact with a novel generative AI tool, helping to expand our understanding of the new opportunities that HAI collaboration can bring to the social service sector.

\subsection{Challenges in AI Use in Social Service}
\label{subsec:relatedworkaiuse}

% Despite the immense potential of AI systems to augment social work practice, there are multiple challenges with integrating such systems into real-life practice. 
Despite its evident benefits, multiple challenges plague the integration of AI and its vast potential into real-life social service practice.
% Numerous studies have investigated the use of PRMs to help practitioners decide on a course of action for their clients. 
When employing algorithmic decision-making systems, practitioners often experience tension in weighing AI suggestions against their own judgement \cite{kawakami2022improving, saxena2021framework}, being uncertain of how far they should rely on the machine. 
% Despite often being instructed to use the tool as part of evaluating a client, 
Workers are often reluctant to fully embrace AI assessments due to its inability to adequately account for the full context of a case \cite{kawakami2022improving, gambrill2001need}, and lack of clarity and transparency on AI systems and limitations \cite{kawakami2022improving}. Brown et al. \cite{brown2019toward} conducted workshops using hypothetical algorithmic tools 
% to understand service providers' comfort levels with using such tools in their work,
and found similar issues with mistrust and perceived unreliability. Furthermore, introducing AI tools can  create new problems of its own, causing confusion and distrust amongst workers \cite{kawakami2022improving}. Such factors are critical barriers to the acceptance and effective use of AI in the sector.

\citeauthor{meilvang_working_2023} (2023) cites the concept of \textit{boundary work}, which explores the delineation between "monotonous" administrative labour and "professional", "knowledge based" work drawing on core competencies of SSPs. While computers have long been used for bureaucratic tasks like client registration, the introduction of decision support systems like PRMs stirred debate over AI "threatening professional discretion and, as such, the profession itself" \cite{meilvang_working_2023}. Such latent concerns arguably drive the resistance to technology adoption described above. Generative AI is only set to further push this boundary, 
% these concerns are only set to grow in tandem with the vast capabilities of generative and other modern AI systems. Compared to the relatively primitive AI systems in past years, perceived as statistical algorithms \cite{brown2019toward} turning preset inputs like client age and behavioural symptoms \cite{vaithianathan2017developing} into simple numerical outputs indicating various risk scores, modern AI systems are vastly more capable: LLMs 
with its ability to formulate detailed reports and assessments that encroach upon the "core" work of SSPs.
% accept unrestricted and unstructured inputs and return a range of verbose and detailed evaluations according to the user's instructions. 
Introducing these systems exacerbate previously-raised issues such as understanding the limitations and possibilities of AI systems \cite{kawakami2022improving} and risk of overreliance on AI \cite{van2023chatgpt}, and requires a re-examination of where users fall on the algorithmic aversion-bias scale \cite{brown2019toward} and how they detect and react to algorithmic failings \cite{de2020case}. We address these critical issues through an empirical, on-the-ground study that to our knowledge is the first of its kind since the new wave of generative AI.

% W 

% Yet, to date, we have limited knowledge on the real-world impacts and implications of human-AI collaboration, and few studies have investigated practitioners’ experiences working with and using such AI systems in practice, especially within the social work context \cite{kawakami2022improving}. A small number of studies have explored practitioner perspectives on the use of AI in social work, including Kawakami et al. \cite{kawakami2022improving}, who interviewed social workers on their experiences using the AFST; Stapleton et al. \cite{stapleton_imagining_2022}, who conducted design workshops with caseworkers on the use of PRMs in child welfare; and Wassal et al. \cite{wassal_reimagining_2024}, who interviewed UK social work professionals on the use of AI. A common thread from all these studies was a general disregard for the context and users, with many practitioners criticising the failure of past AI tools arising from the lack of participation and involvement of social workers and actual users of such systems in the design and development of algorithmic systems \cite{wassal_reimagining_2024}. Similarly, in a scoping review done on decision-support algorithms in social work, Jacobi \& Christensen \cite{jacobi_functions_2023} reported that the majority of studies reveal limited bottom-up involvement and interaction between social workers, researchers and developers, and that algorithms were rarely developed with consideration of the perspective of social workers.
% so the \cite{yang_unremarkable_2019} and \cite{holten_moller_shifting_2020} are not real-world impacts? real-world means to hear practitioner's voice? I feel this is quite important but i didnt get this point in intro!

% why mentioning 'which have largely focused on existing ADS tools (e.g., AFST)'? i can see our strength is more localized, but without basic knowledge of social work i didnt get what's the 'departure' here orz
% the paragraph is great! do we need to also add one in line 20 21?

\subsection{Designing AI for Social Service through Participatory Design}
\label{subsec:relatedworkpd}
% i think it's important! but maybe not a whole subsection? but i feel the strong connection with practitioners is indeed one of our novelties and need to highlight it, also in intro maybe
% Participatory design (PD) has long been used extensively in HCI \cite{muller1993participatory}, to both design effective solutions for a specific community and gain a deep understanding of that community. Of particular interest here is the rich body of literature on PD in the field of healthcare \cite{donetto2015experience}, which in this regard shares many similarities and concerns with social work. PD has created effective health improvement apps \cite{ryu2017impact}, 

% PD offers researchers the chance to gather detailed user requirements \cite{ryu2017impact}...

Participatory design (PD) is a staple of HCI research \cite{muller1993participatory}, facilitating the design of effective solutions for a specific community while gaining a deep understanding of its stakeholders. The focus in PD of valuing the opinions and perspectives of users as experts \cite{schuler_participatory_1993} 
% In recent years, the tech and social work sectors have awakened to the importance of involving real users in designing and implementing digital technologies, developing human-centred design processes to iteratively design products or technologies through user feedback 
has gained importance in recent years \cite{storer2023reimagining}. Responding to criticisms and failures of past AI tools that have been implemented without adequate involvement and input from actual users, HCI scholars have adopted PD approaches to design predictive tools to better support human decision-making \cite{lehtiniemi_contextual_2023}.
% ; accordingly, in social service, a line of research has begun studying and designing for human-AI collaboration with real-world users (e.g. \cite{holten_moller_shifting_2020, kawakami2022improving, yang_unremarkable_2019}).
Section \ref{subsec:relatedworkaiuse} shows a clear need to better understand SSP perspectives when designing and implementing AI tools in the social sector. 
Yet, PD research in this area has been limited. \citeauthor{yang2019unremarkable} (2019), through field evaluation with clinicians, investigated reasons behind the failure of previous AI-powered decision support tools, allowing them to design a new-and-improved AI decision-support tool that was better aligned with healthcare workers’ workflows. Similarly, \citeauthor{holten_moller_shifting_2020} (2020) ran PD workshops with caseworkers, data scientists and developers in public service systems to identify the expectations and needs that different stakeholders had in using ADS tools.

% Indeed, it is as Wise \cite{wise_intelligent_1998} noted so many years ago on the rise of intelligent agents: “it is perhaps when technologies are new, when their (and our) movements, habits and attitudes seem most awkward and therefore still at the forefront of our thoughts that they are easiest to analyse” (p. 411). 
Building upon this existing body of work, we thus conduct a study to co-design an AI tool \textit{for} and \textit{with} SSPs through participatory workshops and focus group discussions. In the process, we revisit many of the issues mentioned in Section \ref{subsec:relatedworkaiuse}, but in the context of novel generative AI systems, which are fundamentally different from most historical examples of automation technologies \cite{noy2023experimental}. This valuable empirical inquiry occurs at an opportune time when varied expectations about this nascent technology abound \cite{lehtiniemi_contextual_2023}, allowing us to understand how SSPs incorporate AI into their practice, and what AI can (or cannot) do for them. In doing so, we aim to uncover new theoretical and practical insights on what AI can bring to the social service sector, and formulate design implications for developing AI technologies that SSPs find truly meaningful and useful.
% , and drive future technological innovations to transform the social service sector not just within [our country], but also on a global scale.

 % with an on-the-ground study using a real prototype system that reflects the state of AI in current society. With the presumption that AI will continue to be used in social work given the great benefits it brings, we address the pressing need to investigate these issues to ensure that any potential AI systems are designed and implemented in a responsible and effective manner.

% Building upon these works, this study therefore seeks to adopt a participatory design methodology to investigate social workers’ perspectives and attitudes on AI and human-AI collaboration in their social work practice, thus contributing to the nascent body of practitioner-centred HCI research on the use of AI in social work. Yet, in a departure from prior work, which have largely focused on existing ADS tools (e.g., AFST) and were situated in a Western context, our paper also aims to expand the scope by piloting a novel generative AI tool that was designed and developed by the researchers in partnership with a social service agency based in Singapore, with aims of generating more insights on wider use cases of AI beyond what has been previously studied.

% i may think 'While the current lacunae of research on applications of AI in social work may appear to be a limitation, it simultaneously presents an exciting opportunity for further research and exploration \cite{dey_unleashing_2023},' this point is already convincing enough, not sure if we need to quote here
% I like this end! it's a good transition to our study design, do we need to mention the localization in intro as well? like we target at singapore

% Given the increasing prominence and acceptance of AI in modern society, 

% These increased capabilities vastly exacerbate the issues already present with a simpler tool like the AFST: the boundaries and limitations of an LLM system are significantly more difficult to understand and its possible use cases are exponentially greater in scope. 

% Put this in discussion section instead?
% Kawakami et al's work "highlights the importance of studying how collaborative decision-making... impacts how people rely upon and make sense of AI models," They conclude by recommending designing tools that "support workers in understanding the boundaries of [an AI system's] capabilities", and implementing design procedures that "support open cultures for critical discussion around AI decision making". The authors outline critical challenges of implementing AI systems, elucidating factors that may hinder their effectiveness and even negatively affect operations within the organisation.


% Is this needed?:
% talk about the strengths of PD in eliciting user viewpoints and knowledge, in particular when it is a field that is novel or where a certain system has not been used or developed or tested before


\begin{figure*}[!t]
	\centering
	\includegraphics[width=\linewidth]{Fig/flow.png}

	\caption{Method overview includes (a) a formative understanding of current personhood verification and related challenges through competitive analysis  (b) users' perception, preferences, and design through an interview study}
\label{fig:method}
\end{figure*}
\vspace{-2mm}
\section{Method Overview}
\label{sec:method}
\vspace{-2mm}
Building on the existing literature, it is clear that while significant progress has been made, a critical gap remains in understanding the key factors to operationalize personhood credentials that balance privacy, security, and trustworthiness online. 
%This challenge becomes even more pressing with the rise of increasingly advanced AI, which enables bad actors to scale their operations, exacerbating issues such as impersonation, fake identities, and non-human interactions. 
As outlined in Figure~\ref{fig:method}, our study comprises: (1) a competitive analysis of current personhood/identity verification tools to identify challenges. These insights inform the design of a user study aimed at (2) investigating users’ perceptions (RQ1), identifying factors influencing their preferences for personhood credentials (RQ2), and conceptualizing designs (RQ3) to address these challenges.

%Please add the flow digram / RQs of different methods with a method overview. see here https://arxiv.org/pdf/2410.01817?}


\vspace{-2mm}
\section{Formative Understanding of PHCs}
\vspace{-2mm}
In this section, we outline our formative analysis of existing personhood verification systems, which informed the design rationale for developing our user study (Section~\ref{user-study}).

%\subsection{Competitive Analysis \& Cognitive Walkthrough}
%\textbf{Competitive Analysis.}
%No prior studies have explored personhood credentials systems' usability and security issues. To address this gap, 
We systematically consolidated a list of systems based on their popularity, diversity in platform type (centralized vs. decentralized), and relevance to the domain of digital identity~\cite{idenaWhitepaper, kavazi2021humanode, kavazi2023humanode, de2024personhood, BrightID, PoH, adler2024personhood}
This consists of
%both practical implementations and state-of-the-art systems, including the 
World app, BrightID, Proof of Humanity, Gitcoin Passport, and Federated Identities (OAuth), etc (Table~\ref{tab:systems}). 
%as well as collected public user's review from Google Playstore. We chose these systems based on their popularity, diversity in platform type (centralized vs. decentralized), and relevance to the domain of digital identity\fixme{add citations of research papers from lit review}. 
Table~\ref{tab:identity_verification} provides an overview of different attributes of how existing systems operate and their design trade-offs. We found 15 apps categorized into six groups. Five of these were centralized, primarily government-based personhood verification systems. This initial categorization is based on the data requirements for issuing credentials varied, including behavior filters, biometrics (such as face, selfie, iris, or video), social graph and vouching mechanisms, physical ID verification, and, in some cases, combinations of these methods. 
\iffalse
\begin{table}[ht]
    \centering
    \scriptsize
    \begin{tabular}{llll}
      \hline
       App Name  & Source & reviews  \\
    
        \hline
     Worldapp & White Paper~\cite{WorldWhitepaper}, Google Play Store& 1523 \\
  BrightID & White Paper~\cite{BrightID},Google Play Store & 328 \\
  DECO & WhitePaper~\cite{zhang2020deco} & Review  \\
  CANDID & WhitePaper~\cite{maram2021candid} & Review \\
  Proof of Humanity &  WhitePaper~\cite{PoHexplainer} & Review \\
  Adhar Card &  WhitePaper~\cite{Aadhaar}, Google Play Store & Review
  %https://play.google.com/store/apps/details?id=in.gov.uidai.mAadhaarPlus&hl=en_US
  \\
Estonia e-ID  &  WhitePaper~\cite{estoniaE-ID} & Review\\
Chinese Credit system &  WhitePaper~\cite{ChinaSocialCreditSystem} & Review \\
Japan My Number Card &  WhitePaper~\cite{JapanMyIDNumber} & Review \\
ID.me &  WhitePaper~\cite{irsIdentityVerification, idAccessAll}, Google Play Store & Review \\
%https://play.google.com/store/apps/details?id=me.id.auth&hl=en_US
Idena &  WhitePaper~\cite{idenaWhitepaper} &  Review \\
Humanode &  WhitePaper~\cite{kavazi2021humanode} &Review\\
Civic &  WhitePaper~\cite{CivicPass} &Review \\
Federated identities (Oauth) &  WhitePaper~\cite{OAuth} & Review\\
  \hline
    
    \end{tabular}
    \caption{Competitive Analysis Data Sources 
   % \fixme{may move to appendix later}
    }
    \label{tab:systems}
\end{table}
\fi
%which helps us conduct a cognitive walkthrough. 

%we analyzed 15 popular systems in terms of their features, such as issuance system (centralized vs decentralized), types of data requirements for issuing credentials, types of  service providers of those systems. 
%Our competitive analysis allowed us to explore and identify multi-criteria to assess aspects such as privacy, usability, and security
We also documented on how users navigate the system and identify potential usability and security issues. Two UI/UX in out team evaluated whether users could successfully sign up and obtain personhood credentials. We independently compiled an initial list of evaluation results based on key questions. This includes- \textit{``How intuitive is the verification process?; How effectively does the platform provide feedback during different steps of registration and verification?; How do we as users feel regarding the data requirements in the verification systems?; How does the platform manage users' data?; What are the potential risks regarding users' privacy in the platform?''}
%about user workflows, task completion, and potential points of failure. 
%such as the intuitiveness of the verification process, feedback during registration, data requirements.
%data management, and privacy risks. 
%This included documenting account creation, data input, verification procedures, and associated risks. 
Given the limited access to systems like Estonia’s digital ID, Civic, and China’s social credit system, we used available white papers and documentation to reconstruct their workflows. Finally, we synthesized our observations and conducted qualitative coding to identify recurring themes.



\begin{table}[ht]
    \centering
    \scriptsize
    \begin{tabular}{llll}
      \hline
       App Name  & Source & reviews  \\
    
        \hline
     Worldapp & Documentation~\cite{WorldWhitepaper}, Google Play Store& 1523 \\
  BrightID & Documentation~\cite{BrightID},Google Play Store & 328 \\
  DECO & Documentation~\cite{zhang2020deco} & Review  \\
  CANDID & Documentation~\cite{maram2021candid} & Review \\
  Proof of Humanity &  Documentation~\cite{PoHexplainer} & Review \\
  Adhar Card &  Documentation~\cite{Aadhaar}, Google Play Store & Review
  %https://play.google.com/store/apps/details?id=in.gov.uidai.mAadhaarPlus&hl=en_US
  \\
Estonia e-ID  &  Documentation~\cite{estoniaE-ID} & Review\\
Chinese Credit system &  Documentation~\cite{ChinaSocialCreditSystem} & Review \\
Japan My Number Card &  Documentation~\cite{JapanMyIDNumber} & Review \\
ID.me &  Documentation~\cite{irsIdentityVerification, idAccessAll}, Google Play Store & Review \\
%https://play.google.com/store/apps/details?id=me.id.auth&hl=en_US
Idena &  Documentation~\cite{idenaWhitepaper} &  Review \\
Humanode &  Documentation~\cite{kavazi2021humanode} &Review\\
Civic &  Documentation~\cite{CivicPass} &Review \\
Federated identities (Oauth) &  Documentation~\cite{OAuth} & Review\\
  \hline
    
    \end{tabular}
    \caption{Competitive Analysis Data Sources 
   % \fixme{may move to appendix later}
    }
    \label{tab:systems}
\end{table}
%(presented in section~\ref{prac-cha}).

%\textbf{Cognitive Walkthough.}
%For the cognitive walkthrough, 
%We also focused on how a user would navigate the system and identify potential usability and security issues. Two experts, specializing in UI/UX and verification systems, evaluated whether users could successfully interact with the application interface and complete two tasks, (a) signing up with the system and (b) obtaining personhood credentials. We independently compiled an initial list of evaluation results by addressing key questions related to user workflows, task completion, and potential points of failure. This includes- \textit{``How intuitive is the verification process?; How effectively does the platform provide feedback during different steps of registration and verification?; How do we as users feel regarding the data requirements in the verification systems?; How does the platform manage users' data?; What are the potential risks regarding users' privacy in the platform?''}
%This included documenting (a) the step-by-step process of creating test accounts and (b) key steps such as data input requirements, verification procedures, and associated risks. Given that some relevant systems, such as Estonia’s digital ID, Civic, and China’s social credit system, are either inaccessible or operate as proof of concept models, we referenced available white papers and documentation to reconstruct their workflows. Finally, we synthesized the experts' observations and conducted qualitative coding to identify recurring themes in the evaluation (presented in section~\ref{prac-cha}). 
%These themes were categorized based on usability challenges, security concerns, and potential improvements in the interface design and verification process.
%Once the evaluations were done, we conducted a qualitative coding to understand the overall themes of the assessment.
%of the user interface and user experience, 

%focusing on ease of use, clarity, and overall usability; (b) we created test accounts to study and asses the workflow and documented the key steps, required information and potential privacy and security issues. Finally, we structured the data according to aforementioned criteria to highlight notable differences and their implications on usability and privacy.
%For evaluating the current verification process of some applications, we have utilized cognitive analysis of UI/UX, data requirement and privacy issue 
%We have selected some popular centralized and decentralized platforms such as World app, Bright ID, Proof of Humanity, Passport Gitcoin, Federated Identities (OAuth), Aadhar Card, Estonia's digital ID and China's social credit system . 

%For cognitive analysis of UI/UX, we have considered a few questions set: 
%\tanusree{from where did we get these questions? My impression was- we are doing cognitive analysis of ui/ux and data requirement, privacy issues, questions here doesn't reflect the goal of cognitive walkthrough}
% \begin{itemize}
%     \item How intuitive is the verification process?
%     \item How effectively does the platform provide feedback during different steps of registration and verification?
%     \item How do we as users feel regarding the data requirements in the verification systems?
%      \item How does the platform manage users' data?
%     \item What are the potential risks regarding users' privacy in the platform?
% %\end{itemize}
% %The following 2 questions have been utilized for data requirement analysis
% %\begin{itemize}
%     %\item What type of data (e.g., personal and biometric, etc) are required for issuing the credentials?
%     %\item In which stage, are these credentials requested from users? How we as users felt regarding the data requirements in the verification systems
% %\end{itemize}
% %We have also analyzed the privacy concerns using these 2 questions:
% %\begin{itemize}
   
% \end{itemize}


 %  \begin{figure*}
 % 	\centering
 % 	\includegraphics[width=0.8\linewidth]{Fig/worldapp.png}
 % 	\caption{ Worldapp-(a) lack of guidance on how users should navigate or utilize the app; (b) backup interface: requires users to connect Google Drive}
    
 % \label{fig: fig:worldapp}
 % \end{figure*}
%The competitive analysis aimed to evaluate and compare the verification processes of the \fixme{it should be a total of 15} eight selected verification systems (Table~\ref{tab:identity_verification}).
%The following predefined criteria were utilized to ensure a structured and consistent evaluation of the platforms:

% \begin{itemize}
%     \item Type of platform
%     \item Free or paid
%     \item Required data
%     \item Stage where data is required
%     \item Centralized or decentralized
%     \item Advantage
%     \item Disadvantage
%     \item UI/UX issue
%     \item Privacy related issue
% \end{itemize}

% We collected data for analysis using the following approach:
% \begin{itemize}
%     \item We analyzed the user interface and the user experience qualitatively and focused on ease of use, clarity and usability.
%     \item We created test accounts to study and asses the whole account creation workflow and documented the key steps and required information.
% \end{itemize}


  %  \item We reviewed official resources such as documentation and privacy policy to evaluate privacy concerns. 


\begin{table*}[h!]
    \centering
    \caption{Comparison of Existing Personhood Verification Systems}
    \label{tab:identity_verification}
    \resizebox{\textwidth}{!}{ 
    \begin{tabular}{l >{\small}l >{\small}l >{\small}l >{\small}p{3cm} >{\small}p{2.5cm} >{\small}l} 
        \hline
        \textbf{Category} & \textbf{Service Name} & \textbf{Architecture} & \textbf{Issuer} & \textbf{Credential} & \textbf{Platform} & \textbf{Free/Paid} \\
        \hline
        \hline
        \multirow{3}{*}{Behavioral Filter} 
        & CAPTCHA & Centralized & open-source, vendor & Recognize distorted texts, images, sounds etc. & Desktop and mobile browsers & Free/Paid\\
        & reCAPTCHA & Centralized & Google & Click checkbox & Desktop and mobile browsers& Free/Paid\\
        & Idena & Decentralized & open-source & Solve contextual puzzle & Blockchain & Free\\
        \hline
        \multirow{2}{*}{Biometrics}
        & World ID & Decentralized & World & Biometrics (iris scan) & App (iOS, Android) & Free\\
        & Humanode & Decentralized & Humanode & Biometrics (face) & Blockchain & Paid\\
        \hline
        Social Graph 
        & BrightID & Decentralized & open-source & Analysis of social graph & App (iOS, Android) & Free\\
        \hline
        Social Vouching 
        & Proof of Humanity & Decentralized & Kleros & Social vouching & Web & Paid\\
        \hline
        \multirow{2}{*}{Decentralized Oracle} 
        & DECO & Decentralized & Chainlink Labs & Cryptographic proof & Decentralized oracle & Under PoC\\
        & CANDID & Decentralized & IC3 research team & Cryptographic proof & Decentralized oracle & Under PoC\\
        \hline
        \multirow{4}{*}{Government-based ID} 
        & India Aadhaar Card & Centralized & Government & Document-based or Head Of Family-based enrollment + digital photo of face, 2 iris, and 10 fingerprints& Web, App (iOS, Android) & Free\\
        & Estonia e-ID & Decentralized & Government & Passport or EU ID + digital photo of face & Web, App (iOS, Android) & Paid\\
        & Japan My Number Card & Centralized & Government & Issue notice letter + photo ID or two non-photo IDs & Web, App (iOS, Android) & Free\\
        %& Chinese Credit System & Centralized & Gov & Personal credit records & Varies by region & Free\\
        \hline
        \multirow{2}{*}{Others} 
        & ID.me & Centralized & ID.me & Government-issued ID & Web & Free\\
        & Civic Pass & Decentralized & Civic & Government-issued ID, Biometrics (face), Humanness, Liveness & Web & Free\\
        \hline
    \end{tabular}
    }
\end{table*}

\begin{figure*}[h]
    \centering
    \begin{subfigure}{0.48\textwidth}
        \centering
        \raisebox{0.5\height}{
        \includegraphics[width=\textwidth]{Fig/idena.png}}
        \captionsetup{width=\textwidth, font=footnotesize} 
        \caption{Idena validation test interface: This requires users to select meaningful stories within a time limit, which can pose challenges for new users}
        \label{fig:idena}
    \end{subfigure}
    \hfill
    \begin{subfigure}{0.48\textwidth}
        \centering
        \includegraphics[width=\textwidth]{Fig/google_drive.png}
        \captionsetup{width=\textwidth, font=footnotesize} 
        \caption{World App backup interface: requires users to connect Google Drive}
        \label{fig:worldapp}
    \end{subfigure}
    
    \caption{PHC-related interfaces: (a) Idena validation test, (b) World App backup process.}
    \label{fig:phc_interfaces}
\end{figure*}

\vspace{-2mm}
\subsection{Challenges in Identity Verification}
\vspace{-2mm}
\label{prac-cha}
\textbf{Demanding Cognitive and Social Efforts for Verification Workflow.}
We found platforms such as World App and BrightID developed on decentralized technologies, 
including zero-knowledge proofs and social connections, may confuse non-technical users. For instance, user review from playstore suggested-many having issues understanding how to receive BrightID scores to prove they are sufficiently connected with others and verified within the graph. In their words \textit{``It's hard for me to connect with people to create the social graph.''} 
%\textbf{Usability Issue.}
%CAPTCHAs have become increasingly difficult to solve, can make the user journey cognitively demanding. To support the security of humanness verification, particularly image-based ones are becoming demanding for users. 
From experts' evaluation of UI/UX, we found Proof of Humanity lacks options to correct or update mistakes, which can make the registration process less user-friendly. %Incorporating the principle of error prevention could improve the user experience. 
Similarly, Idena's validation test (flip test) (Figure~\ref{fig:idena}) was challenging as new users as it required to create a meaningful story within the allotted time and earn enough points for validation. Simialrly, World App's(Figure~\ref{fig:worldapp}) account creation process to get an identifier doesn't inform users how and why to navigate the app can undermine intended functionality,  or underutilization of the app’s capabilities.


% \begin{figure*}[h]
%     \centering
%     \begin{minipage}{0.30\textwidth}
%         \centering
%         \includegraphics[width=\linewidth]{Fig/google drive.png}
%         \caption{World App backup interface: requires users to connect Google Drive.}
%         \label{fig:worldapp}
%     \end{minipage}
%     \hfill
%     \begin{minipage}{0.48\textwidth}
%         \centering
%         \includegraphics[width=\linewidth]{Fig/wordl1.png}
%         \caption{World App's account creation process: lack of guidance on how users should navigate or utilize the app.}
%         \label{fig:Worldapp1}
%     \end{minipage}
% \end{figure*}

\textbf{New or Complex System Rule to Recover ID. }
Both from UI/UX task and playstore review, we found the BrighID recovery process tedious and the rules unclear. A representative user review stated-\textit{``If you create an account and do not set up recovery connections you cannot get your account back. This forces you to create a new account which defeats the purpose of the app.''}
Another workflow of World App that requires users to connect their Google Drive to back up their accounts. However, this process may confuse users and create challenges during account recovery if they fail to complete the backup(Figure~\ref{fig:worldapp}).
 

%  \begin{figure}
%  	\centering
%  	\includegraphics[width=\linewidth]{Fig/wordl1.png}
%  	\caption{World App's account creation process: lack of guidance on how users should navigate or utilize the app}
%  \label{fig:Worldapp1}
%  \end{figure}


\textbf{Privacy and Data Requirement Issue. }
From our competitive analysis (Table~\ref{tab:litcomparison}), Data requirements across the systems vary significantly in scope and sensitivity. Decentralized platforms like World App, and BrightID required minimal data collection to issue ID while Proof of Humanity require video submission to receive a credential for was quite invasive when the videos were open to the public with clear faces.
%Similarly, both experts mentioned many unknown data policies for new platforms such as World app~\cite{WorldWhitepaper} and Bright ID\cite{BrightID}. 
While there is benefit of decentralization, often it is not clear how exactly service providers will handle the data in their policies and white papers.
%which created a reluctance for them, thus for new users to start using them. 
In contrast, Federated Identities OAuth\cite{OAuth} login process streamlines and this contributed to using known third-party service providers. This ensures ease of use as users need to specify the identity provider during the login or authentication process and grant access to their specific data. This reflects the importance of known entities and level of trust in data handling.
%However, they also have data being shared across multiple platforms which leads to some privacy concerns. 
Centralized systems, including Aadhaar and Estonia digital ID, require extensive personal and biometric data—fingerprints and iris scans—to ensure verification services while experts expressed privacy concerns towards china’s Social Credit personhood System, especially the use of it in measuring social scores.
%There was concerns regarding reCAPTCHA addressing usability issues by removing explicit verification tasks, relying instead on tracking user behavior, such as mouse movements, keystrokes, and browsing history. However, this approach trades off user privacy, as data collected during these activities raised concerns.


\textbf{Requirement of Optimal Device or Physical Presence.}\\
Government-supported systems like Aadhaar and Estonia e-Card feature structured interfaces but come with limitations: Aadhaar’s biometric registration may challenge rural populations, while Estonia’s dependence on smart-card hardware might exclude those without the necessary devices. Proof of Humanity, Humanode, Civic Pass may create challenges as proper lighting and optimal devices are necessary for taking the appropriate photo or video for biometric verification
%\fixme{need a screenshot for this}. 
On the contrast, Aadhaar card\cite{Aadhaar}\cite{AadhaarEnrollment}, Estonia's e-ID and Japan's My Number Card require one to be physically present and the issuing process takes a long time can create user frustration. 
%The existing systems and platforms that we have evaluated can hardly strike a balance between privacy, functionality and usability.  



%CAPTCHA\cite{Captcha} and reCAPTCHA\cite{reCaptcha} are 2 common human verification tools used across many websites. While CAPTCHAs add an additional step for users when they are trying to access a website, reCAPTCHAS come into play by removing any external verification. Rather, reCAPTCHAs track users' activities which has raised privacy concerns as there is lack of transparency between user and reCAPTCHA authority. Users are not sure how the tracking data will be used. 

\iffalse
\subsection{Results of UI/UX}
%\tanusree{Silvia: why do we have only 3 apps in the analysis?Ayae created a list a long ago. please complete the analysis for all the apps from this list}  \tanusree{I am not sure why facebook is in the analysis. we talked about only including verification apps, facebook is not one of them} \fixme{look at the Suggetsions in comment}
The eight \fixme{15 systems} systems evaluated manifest diverse approaches to user experience, emphasizing accessibility, intuitiveness, and transparency\fixme{write in active sentence or active voice, it reads like chatGPT and reviewer will think the same}. Platforms such as World App and BrightID developed on decentralized technologies, 
%though their intricate verification methods, 
including zero-knowledge proofs\fixme{add citation} and social connections \fixme{add as footnote what social connection means here and citation}, may confuse non-technical users. Proof of Humanity requires video submissions \fixme{what kind of video, is it their face? then talk about privacy, this doesn't seem to be a blockchain issue rather privacy issue}, a process potentially intimidating for individuals less familiar with blockchain platforms. 

In contrast, Federated Identities (OAuth) streamlines login processes via well-known third-party providers\fixme{who is the third-party provider for them}, ensuring ease of use for most users \fixme{is that all? }. 

Government-supported systems like Aadhaar and Estonia e-Card feature structured interfaces but come with limitations: Aadhaar’s biometric registration may challenge rural populations, while Estonia’s dependence on smart-card hardware might exclude those without the necessary devices. \fixme{add about Japan My Number Card.} 

Passport Gitcoin, focused on Web3 integration, struggles with clarity for users new to decentralized identity concepts. Finally, China’s Social Credit System delivers a seamless yet opaque experience, leaving users uncertain about the data influencing their scores.\par
Data requirements across the systems vary significantly in scope and sensitivity. Decentralized platforms like World App, BrightID, and Proof of Humanity emphasize minimal data collection but still require sensitive information, such as Ethereum addresses, social graphs, or video proofs, to ensure authenticity. 

Centralized systems, including Aadhaar and Estonia digital ID, require extensive personal and biometric data—fingerprints and iris scans—to ensure seamless service delivery. 

Passport Gitcoin, designed for Web3 wallet integration, relies on centralized storage, demanding significant user trust. Federated Identities (OAuth) achieves a balance by sharing limited data through third-party providers but this comes with the risk of overexposure. China’s Social Credit System stands out for its vast data collection, encompassing financial, social, and daily activities, raising alarm over pervasive monitoring and privacy intrusion.\par
Privacy concerns are critical across the eight systems, influenced by their data management practices. Decentralized platforms like World App and BrightID prioritize privacy, yet linking personal data to public blockchains—as seen in Proof of Humanity—poses inherent risks. Centralized systems like Aadhaar and Estonia e-Card depend on centralized databases, making them vulnerable to surveillance risks. Federated Identities (OAuth) simplifies access but could expose user data to third-party providers without explicit consent. Passport Gitcoin presents privacy challenges because users' information can be shared with third-party service providers. Meanwhile, China’s Social Credit System exemplifies extreme privacy erosion, extensively monitoring citizen behavior with minimal transparency about data use. Striking a balance between privacy and functionality remains a universal challenge for all these systems.

\fixme{citations to be added} We have evaluated 15 systems to present diverse approaches to user experience, emphasizing usability, accessibility, intuitiveness and transparency.
\fixme{citation didn't work} CAPTCHA\cite{Captcha} and reCAPTCHA\cite{reCaptcha} are 2 common human verification tools used across many websites. While CAPTCHAs add an additional step for users when they are trying to access a website, reCAPTCHAS come into play by removing any external verification. Rather, reCAPTCHAs track users' activities which has raised privacy concerns as there is lack of transparency between user and reCAPTCHA authority. Users are not sure how the tracking data will be used. 

\tanusree{no good content}
Platforms such as World app\cite{WorldWhitepaper} and Bright ID\cite{BrightID} are developed on decentralized technologies which include zero-knowledge proofs but do not present a clear and concise terms and conditions and privacy policy, which may create reluctance for new users to start using them. In figure 1(a), the on-boarding screen of World App appears with a consent checkbox to obtain explicit consent from the users that they agree to the "Terms and Conditions" and acknowledge the "Privacy Notice" of World App. But the terms and conditions and privacy notice are not mentioned in the same screen, tapping on the link buttons redirects users to a different screen, thus creating an obstacle in their user journey. If the necessary terms and conditions were presented clearly and concisely on the on-boarding screen, it would have informed users about the app's policies and ensure a smoother user journey. 1(b) represents the Bright ID license agreement, but it is too long to read. Users may not have enough patience to go through the details as it is time consuming and tap the agree button to continue. But this action may create privacy risks as users don't know what type of access they are providing to the application.
\begin{figure}[h]
     \centering
     \begin{subfigure}[b]{0.2\textwidth}
         \centering
         \includegraphics[width=\textwidth]{Fig/world app t&c.png}
         \caption{The terms and conditions and privacy notice are not mentioned in the World App's on-boarding screen}
         \label{fig:The terms and conditions and privacy policy are not mentioned in the World App's on-boarding screen}
     \end{subfigure}
     \hfill
     % \begin{subfigure}[b]{0.3\textwidth}
     %     \centering
     %     \includegraphics[width=\textwidth]{Fig/google drive.png}
     %     \caption{World App requires users to connect Google Drive for enabling backup}
     %     \label{fig:five over x}
     % \end{subfigure}
     % \hfill
     \begin{subfigure}[b]{0.3\textwidth}
         \centering
         \includegraphics[width=\textwidth]{Fig/bright id t&c.png}
         \caption{Bright ID's license agreement contains a long description which users may not want to read}
         \label{fig:three sin x}
     \end{subfigure}
     \hfill
        \caption{On-boarding screens of World App and Bright ID}
        \label{fig:three graphs}
\end{figure}
In figure 2, we can see World App requires users to connect their Google Drive to back up their world app accounts but this may lead users to providing access to their sensitive information.
\begin{figure}[h]
    \centering
    \includegraphics[width=0.5\linewidth]{Fig/google drive.png}
    \caption{World App requires users to connect Google Drive for enabling backup}
    \label{fig:World App requires users to connect Google Drive for enabling backup}
\end{figure}
% \iffalse
% \begin{figure}
%  	\centering
%  	\includegraphics[width=0.5\linewidth]{Fig/world app t&c.png}
%  	\caption{The terms and conditions and privacy policy are not mentioned in the World App's on-boarding screen}   
%  \label{fig:The terms and conditions and privacy policy is not clearly mentioned}
%  \end{figure}
%  \begin{figure}
%  	\centering
%  	\includegraphics[width=\linewidth]{Fig/bright id t&c.png}
%  	\caption{The license agreement and privacy policy is too long to read}   
%  \label{fig:The license agreement and privacy policy is too long to read}
%  \end{figure}
% . \par
%  \begin{figure}
%  	\centering
%  	\includegraphics[width=\linewidth]{Fig/google drive.png}
%  	\caption{World App requires users to connect Google Drive for enabling backup}
    
%  \label{fig:World App asking to connect Google Drive}
%  \end{figure}


 


Proof of Humanity\cite{PoH}\cite{PoHexplainer} offers a unique approach to verification with a social identification system. But the verification process requires users to connect their cryptocurrency wallet which will be publicly linked to users' account. Thus, users' wallet holdings and transaction history will be linked to users' identity which users may not prefer. 

In contrast, Federated Identities OAuth\cite{OAuth} provides streamlined login process via well known third-party service provides, also known as identity providers such as Google, Facebook etc. It ensures ease of use as users need to specify the identity provider during the login or authentication process and grant access to their specific data. But, data is shared across multiple platform which may raise privacy concerns among users. 

DECO\cite{zhang2020deco} and CanDID\cite{maram2021candid} are decentralized and privacy preserving oracle protocols where DECO allows users to prove the authenticity of website data obtained over TLS (Transport Layer Security) without revealing sensitive information. But Oracle has access to users' data which pose as a privacy risk. CanDID provides users with control of their own credentials but privacy depends on the honesty and integrity of verifiers and decentralized identity validators. 

Idena\cite{idenaWhitepaper}, Humanode\cite{Humanode} and Civic Pass\cite{CivicPass} - all are blockchain based person identification system where Idena performs validation by conducting flip tests and Humanode and Civic Pass are developed on crypto-biometric network. Though Idena does not collect any personally identifiable information, the behavioral data collected can be used in future for pattern analysis. 

Humanode and Civic pass both require biometric verification (face scan) which can create concerns among users about how their sensitive credential (face) will be managed by the systems. It is noteworthy that, most of the platforms are decentralized (World App, Bright ID, Proof of Humanity, Idena, Humanode, Civic), some requiring cryptocurrency wallet (Proof of Humanity, Civic Pass) and some requiring biometric verification (Proof of Humanity, Humanode, Civic Pass).    %citations to be added
\par
Government issued identity documents such as Aadhaar Card, Estonia's e-ID, China's social credit system and Japan's My Number Card are controlled and managed by central government. Citizens' sensitive credential can be at high risk if the government's security system is not robust enough to prevent any kind of hacking or data breaching. China's social credit system monitors citizen data extensively without maintaining complete transparency about data use and management. 

ID.me is another online identity network that enables individuals to verify their legal identity digitally. But privacy concerns arises as a single company holds a large amount of personal data and users have limited control over their data. %citations to be added
\par
Usability across these different platforms are critical. CAPTCHAs have become increasingly difficult to solve, often leading users to leave the website or platform without completing their user journey. Accessibility remains another issue as visually impaired users are unable to solve any CAPTCHA that is text or image based. reCAPTCHA comes with the solution of these problems but trading of users' privacy as users' data is being tracked down by the authority. 

From Figure 3 and 4, it is apparent that World app and Bright ID provide a simple and intuitive account creation form but an introductory video or step by step guide would be more helpful to guide users to navigate throughout the applications and perform necessary actions.
 \begin{figure}
 	\centering
 	\includegraphics[width=\linewidth]{Fig/world app account creation.png}
 	\caption{World App's account creation process is simple but doesn't inform users about how they should navigate or use the app \fixme{silvia, is there a reason you added all these UIs? why all of the uis are randomly placed, I shared examples so many times, i am not seeing anything I gave instruction.}}
 \label{fig:World App's on-boarding process}
 \end{figure}
 
 \begin{figure}
 	\centering
 	\includegraphics[width=\linewidth]{Fig/bright id account creation.png}
 	\caption{The "Create my BrightID" process in the Bright ID app is straightforward but lacks guidance on how users should navigate or utilize the app effectively. \fixme{explain why these screenshots are important to add from cognitive walkthrough. caption itself should be self explanatory with text explaining in the main body}}
    
 \label{fig:Bright ID's on-boarding process}
 \end{figure}
The principle of error prevention could make the user journey of registration in Proof of Humanity more user-friendly. As there is no option to correct or update any mistake, it may increase user frustration. Idena's validation test (flip test) (Figure 4) can be inconvenient for new users as they may struggle to find the meaningful story in the provided time and collect points to validate them.
 \begin{figure}
 	\centering
 	\includegraphics[width=\linewidth]{Fig/idena.png}
 	\caption{Idena validation test interface requiring users to select meaningful stories within a time limit which can be challenging for new users \fixme{anyone reading this caption would not understand anything}}
    
 \label{fig:Selecting meaningful story for validation process on Idena}
 \end{figure}
The platforms requiring video selfie or face scan (Proof of Humanity, Humanode, Civic Pass) may create another challenging situation for users as proper lighting and optimal devices are necessary for taking the appropriate photo or video for biometric verification. 

Aadhaar card\cite{Aadhaar}\cite{AadhaarEnrollment}, Estonia's e-ID and Japan's My Number Card are all government based credentials but completing all the formalities and getting the card takes a long time, sometimes creating user frustration. The existing systems and platforms that we have evaluated can hardly strike a balance between privacy, functionality and usability.   %citations to be added


% \begin{figure}[!t]
% 	\centering
% 	\includegraphics[width=\linewidth]{Fig/world app.png}
% 	\caption{New account creation process in  World App}
    
% \label{fig:New account creation process in  World App}
% \end{figure}
% \begin{figure}[!t]
% 	\centering
% 	\includegraphics[width=\linewidth]{Fig/bright id.png}
% 	\caption{New account creation process in  Bright ID}
    
% \label{fig:New account creation process in  Bright ID}
% \end{figure}


\subsection{Reddit Analysis}
%\tanusree{ishan to add}
We first collected \fixme{X} posts and  \fixme{X} comments on December 24th, 2024, using the Python Reddit API Wrapper (PRAW)~\footnote{https://praw.readthedocs.io/en/stable/}. We gathered the data from various relevant subreddits, ensuring a broad and comprehensive understanding of what users discuss on identify verification or personhood verification. Through qualitative analysis of this Reddit data, we were able to uncover detailed insights into the rich and prevalent usage of verification systems. This analysis highlighted users' current usage, potential challenges and risks they encounter. These findings provide a solid foundation to explore these themes further in subsequent in-depth interviews.

\paragraph{Data Collection}
 To comprehensively cover content related to our research questions on personhood verification, we first created a list of search keywords by identifying close terminologies related to \textit{``personhood verification''} (general keywords) and \textit{``bot check''} (technology-focused keywords), etc. We utilized a combination of general and technology-focused keywords in our search. We employed general terms such as Personhood Verification, Identity Proof, Human Check and Bot Check. These keywords were designed to capture posts authored by or discussing personhood verification. For the technology focus, we used terms such as \fixme{add}. These keywords targeted discussions specifically about the use of popular tools and platforms. We conducted open searches combining these keywords across Reddit to gather data from various subreddits.
 Other than open searches, we also applied specific criteria to select subreddits, ensuring comprehensive coverage of relevant discussions: these subreddits should focus either on the personhood verification community or technology. We chose subreddits with the most active users online during our browsing sessions. The full list of subreddits and search keywords used is detailed in Table\fixme{need to find out the subreddit most prevalent discussing these topic}. 

\paragraph{Analysis}
Two researchers reviewed each post and categorized related posts or comments into five overarching high-level themes: \fixme{need to add after data analysis}. Within these categories, 53 level-2 themes were identified, such as \fixme{need to add after data analysis}. During the analysis process, researchers regularly convene to discuss discrepancies and emerging themes in the codebook, aiming to reach a consensus. These categories allowed us to investigate RQ2 and partially address RQ1. 

\subsection{Results}
% \tanusree{ishan to add}
\fi
\vspace{-2mm}
\section{ User Study Method}
\vspace{-2mm}
\label{user-study}
This section outlines the method for exploring users' perceptions and preferences of personhood credentials. We conducted semi-structured interviews with 23 participants from the US, and the EU/UK in October 2024.
%We started with a round of pilot studies (n=5) to validate the interview protocol. Based on the findings of pilot studies, we revised the interview protocol and conducted the final round of interviews (n=17). 
The study was approved by the Institutional Review Board (IRB).
\vspace{-2mm}
\subsection{Participant Recruitment}
\vspace{-2mm}
We recruited participants through (1) social media posts, (2) online crowdsourcing platforms, including CloudResearch and Prolific. Respondents were invited to our study if they met the selection criteria: a) 18 years or older and b) living in the US or the EU/UK. Participation was voluntary, and participants were allowed to quit anytime. Each participant received a \$30 Amazon e-gift card upon completing an hour-and-a-half interview.

\subsection{Participants}
%\tanusree{check for final count} \ayae{updated percentage with final 23 counts} 
We interviewed 23 participants, 10 from the US and 12 from the EU/UK. The majority of the participants (61\%) were in the age range of 25-34, followed by 22\% were 35-44 years old. The participants were from the United States and various countries, namely Spain, Sweden, Germany, Hungary, and the United Kingdom. Participants had different backgrounds of education levels, with 87\% of participants holding a Bachelor’s degree and 65\% holding a graduate degree. 65\% of participants had a technology background, while 48\% of them had a CS background. All participants reported using online services that required them to verify their personhood. Table~\ref{table:demographics} presents the demographics of our participants. We refer to participants as P1,. . . ,23.
\begin{table*}[h!]
\centering
%\scriptsize
\caption{Overview of PHC Application Scenarios}
\label{table:scenario}
%\resizebox{\textwidth}{!}{%
\begin{tabular}{lll}
\hline
\textbf{Scenario} & \textbf{Service} & \textbf{Credential} \\
\hline
Financial service & Bank, Financial institutions & Passport or Driver’s license, Face scan \cite{yousefi2024digital}\\
% \hline
Healthcare service & Hospitals, Clinics & Health insurance card,  Fingerprint \cite{chen2012non,fatima2019biometric,jahan2017robust}\\
% \hline
Social media & Tech companies & National identity card, Video selfie \cite{instagramWaysVerify, metaTypesID,instagramTypesID} \\
% \hline
LLM application & Tech companies & Iris scan \cite{WorldWhitepaper, worldHumanness}\\
% \hline
Government service & Government & Driver’s license or National identity card \cite{LogingovVerify}\\
% \hline
Employment background check & Background check companies & Tax identification card, Fingerprint\cite{cole2009suspect}\\
\hline
\end{tabular}%
%}
% \vspace{0.5em}
\label{tab:scenarios}
\end{table*}
\begin{table*}[h]
\centering
\caption{Participant demographics and background.}
%\fixme{add the participants you completed so far}
\resizebox{\textwidth}{!}{%
\begin{tabular}{l l l l l l l l}
\hline
\textit{Participant ID} & \textit{Gender} & \textit{Age} & \textit{Country of residence} & \textit{Education} & \textit{Technology background}  & \textit{CS background} &\textit{Residency duration} \\
\hline
P1 & Male & 25-34 & the US & Master's degree & Yes & Yes &3-5 years\\
P2 & Female & 25-34 & the US & Master's degree & Yes & Yes & 1-3 years\\
P3 & Female & 25-34 & the UK & Master's degree & Yes & No & 1-3 years\\
P4 & Female & 35-44 & the UK & Some college, but no degree & Yes & Yes & Over 10 years \\
P5 & Male & 25-34 & the US & Doctoral degree & Yes & Yes & 5-10 years \\
P6 & Male & 35-44 & the US & Less than a high school diploma & No & No & Over 10 years \\
P7 & Male & 25-34 & the US & Doctoral degree & Yes & Yes & 3-5 years\\
P8 & Male & 45-54 & the US & Bachelor's degree & Yes & Yes & Over 10 years \\
P9 & Female & 25-34 & New Zealand & Master's degree & No  &  No &  Over 10 years\\
P10 & Male & 25-34 & the US & Master's degree & No & No & Over 10 years\\
P11 & Female & 25-34 & the UK & Bachelor's degree & No & No & Over 10 years\\
P12 & Male & 18-24 & the UK & Master's degree & Yes & Yes & 1-3 years\\
P13 & Male & 35-44 & the UK & Bachelor's degree & Yes & No & Over 10 years\\
P14 & Male & 25-34 & Sweden & High school graduate & No & No & Over 10 years \\
P15 & Female & 25-34 & Spain & Master's degree & Yes & Yes & Over 10 years \\
P16 & Female & 25-34 & Germany & Master's degree & Yes & Yes & Over 10 years \\
P17 & Female & 25-34 & Spain & Doctoral degree & No & No & Over 10 years \\
P18 & Female & 35-44 & the US & Bachelor's degree & No & No & Over 10 years \\
P19 & Female & 25-34 & Germany & Master's degree & Yes & Yes & 3-5 years \\
P20 & Male & 25-34 & Hungary & Master's degree & Yes & No & 3-5 years \\
P21 & Male & 35-44 & the US & Bachelor's degree & Yes & No & 5-10 years \\
P22 & Female & 18-24 & France & Master's degree & Yes & Yes & Less than 1 year\\
P23 & Male & 45-52 & the US & Master's degree & No & No & Over 10 years\\
\hline
\end{tabular}%
}
\label{table:demographics}
\end{table*}


\vspace{-2mm}
\subsection{Semi-Structured Interview Procedure} \label{sec:study_protocol}
\vspace{-2mm}
%\fixme{explain in details why the study designed in a certain way. please read papers to learn more. data minimization and advertisement paper. The method section is too bland. We have a wonderful study design. Scenario-specific study design, describe scenarios and why chose this scenario. Mainly method should include all design rationale, and example questions when necessary to clarify your rational}

We started with a round of pilot 
%(Appendix~\ref{pilot}) 
studies (n=5) to validate the interview protocol. Based on the findings of pilot studies, we revised the interview protocol.

\textbf{Open Ended Discussion.} We designed the interview script based on our research questions outlined in the introduction section~\ref{sec:introduction}. 
%We added the interview script to the section~\autoref{protocol}. 
At the beginning of the study, we received the participants’ consent to conduct the study. Once they agreed, we proceeded with a semi-structured interview. The study protocol was structured according to the following sections: (1) Current practices regarding digital identity verification; (2) Users' perception of PHC before and after watching the informational video; (3) Scenario-based session to investigate factors that influence users' preferences of PHC; 
%(4) Users' preference of PHC; 
(4) Design session to conceptualize users' expectations; (5) A brief post-survey on Users' Preference of PHCs.
%of PHCs in different scenarios.

In the first section, we first asked a set of questions to understand participants' current practices of online platforms and the types of identity verification methods they had experience with. This is to understand their familiarity with different types of verification, such as biometrics, physical IDs, etc.
%and methods that might have worked well based on their prior experience.

%of online identity verification. When participants mentioned certain types of online services that required identity verification, we inquired about their experience with verification method. Was it easy to use, or did you run into any issues?"} We further inquired about any challenges participants faced with identity verification - \textit{"Did you encounter any challenges when using this method?"} 
%If biometrics didn’t naturally come up in prior discussions, we prompted to consider them- \textit{"Have you ever used services where you had to verify yourself through face, fingerprints, or iris scans, or other biometrics?"} If they mentioned any experience with biometric verification, we followed up with questions like- \textit{"What worked well? Were there any concerns you had?"}
In the second section, we then asked about participants' current understanding and perception of personhood credentials either from prior knowledge or from intuition by just hearing the term. %We also asked if they knew how personhood credentials work, particularly how it has been handled by the different services they use. 
%As all participants had never heard of PHC, we prompted them to interpret the term based solely on its wording. 
While the majority recognized this as unfamiliar terminology, most inferred that it referred to a form of personal identification, often associating it with biometric verification.
%In the pilot interviews, The majority of the participants could not provide substantial responses on their understanding of how personhood credentials work, before starting the second part of the interview, we showed them an informational video on personhood credentials.
%Most of the participants were unfamiliar with this term, so we then asked \textit{ Can you explain what you think it means by just hearing the term?"} 
%Before proceeding with the third section of the interview, we assessed participants' understanding of PHC with knowledge questions.
Then, we showed them an introduction video on PHC \footnote{https://anonymous.4open.science/r/PHC-user-study-14BB/}, %\fixme{create an anonymous GitHub, upload the video and add a footnote here} \ayae{reflected}. 
%The video provides an overview of PHCs, 
covering their definition, 
%the steps involved in issuing and using them, 
and implications of it in online services. Based on former literature\cite{adler2024personhood}, we designed the video with easy-to-understand text, visuals, and audio to make the concepts accessible to average users. We created a set of knowledge questions to assess participants' understanding of PHC before and after showing the video. %as attached in Appendix~\ref{knowledge_questions}.

%including the same knowledge questions. 
%Most participants correctly responded to knowledge questions, which ask the basic understanding of digital identity crisis and personhood credentials. 
%Even before showing the introduction video, regarding the question \textit{"What could happen if online identities are poorly verified?"}, 95\% correctly selected \textit{"Fake accounts, bots, and fraud could increase significantly."} For the question \textit{"What are Personhood Credentials (PHCs)?"}, 90\% correctly choose the option \textit{"Digital credentials that confirm a person’s identity."} 
For instance, we observed an improve in correct response rate for the question, such as, \textit{``What is the primary goal of PHC?''} from 85\% to 100\% after watching the video.
%where the correct answer was \textit{"To verify a person's identity without exposing personal information."} 
%However, regarding the question \textit{"To whom do you provide minimal personal information during the PHC process?"}, only 35\% selected the correct answer \textit{"PHC issuers (e.g., governments or trusted organizations)"}, while the most frequent response was \textit{"Online service providers (e.g., social media)"} at 45\%.
%\ayae{KQ results reflected}
%We also asked some open-ended questions to evaluate whether our introduction video helped participants better understand PHC \textit{''How would you explain your understanding of personhood credentials?''} 
%We further asked what benefits and concerns came to mind for them.
In the third section, we focused on scenario-based discussions, exploring specific applications of PHC to understand factors that influence participants' preferences towards PHCs as well as identify challenges to leverage in PHC design for various services. We examined the following six scenarios: (1) Financial service, (2) Healthcare service, (3) Social Media, (4) LLM applications, (5) Government Portal, and (6) Employment Background Check.
%We covered a wide range of use cases of online personhood verification via these six scenarios since they encompass diverse user needs, security and usability, and privacy requirements. %\fixme{please see the comment with iffalse tag and make it concise, we talked about it before}
\iffalse
%Firstly, financial system is a critical scenario for identity verification where high level of security protections are expected as exemplified by KYC. Thus, such services continue to develop transformative digital identity verification to ensure the security and integrity of financial transactions\cite{parate2023digital}. The second scenario is healthcare systems, which also have high privacy requirements due to the confidentiality of medical data. The pandemic has accelerated the adoption of online healthcare services and in response to this digital transformation, the recent study has proposed blockchain-based decentralized identity management systems \cite{javed2021health}. Thirdly, we consider the scenario of social media, which faces the critical challenges of online identity as shown in spreading misinformation and harmful content from fake or anonymous accounts \cite{ceylan2023sharing}. The fourth scenario is designed with a specific context of interacting with Large language models (LLMs). The former study discussed vulnerability in dialog-based systems where adversaries can exploit the training process to introduce toxicity into responses \cite{weeks2023first}. Thus, such vulnerabilities indicate identity verification may also be important for LLM applications. Fifth, government services are familiar situations that require people to verify their identity. Various countries have developed their own electronic ID schemes \cite{stalla2018gdpr}. Lastly, we also cover the scenario of employment background checks needing precise identity verification to ensure the reliability of applicants. The current background check system involves vulnerable processes that increase the risk of identity theft and unauthorized data access.\cite{blowers2013national}. Such challenges underline the relevance of PHCs, which can mitigate risks by providing a secure framework for verification.
%\ayae{included citation}
\fi
We have also incorporated various types of data or credentials requirements (e.g. physical id, biometrics, etc) across scenarios to maintain diversity in our discussion with participants as shown in Table.\ref{table:scenario}. %For instance,
%we  We have multiple existing verification methods, including 
%humanness verification (e.g., selfie, video call), document-based verification (e.g., government-issued ID), and biometrics information. 
We selected types of credentials for each scenario based on former literature and existing PHC as explained in the section \ref{subsec:verification_practice}. %\fixme{cite worldcoin, and other app and literature}. \fixme{from here to end of this paragraph ---These needs to go to the literature review section on the current usecase of PHC. And only 2 line summarizing why you chose the diverse type of credential data and refer to the literature section}


%% Let me find the former literature to explain why we select these credentials
For each of the six scenarios, we explored participants' perceptions of using PHC in hypothetical situations that align with the research focus as well as to help participants can relate PHC concepts to real-world applications. This is particularly useful for this study where where user perceptions and expectation under specific conditions are crucial to devising solutions \cite{carroll2003making}.
%\fixme{cite scenario method paper from jack caroll}.
%\ayae{reflected}
We asked about their feelings, perceived benefit and risks. We also nudge them to think about any privacy and security perception around using PHC and types of data (e.g., iris, face, government id, etc) involved in issuing PHC. 

\fixme{
%\textbf{Pre-understanding: Guessed it as one of the verification methods} 
%The majority of the participants were not familiar with the term ``Personhood Credential'', although most of them used some forms of such credentials. 
%As all participants have never heard of PHC, we prompted them to interpret the term based solely on its wording. Most of them inferred that it referred to another type of person identification. 
%For instance, P3 commented \textit{``It can be anything that would point to one single individual that would differentiate that individual from others.''} When participants expressed how PHC identifies a person's uniqueness, their understanding ranged from verifying basic information such as address or age, and certain eligibility to advanced identification of digital identity (e.g., behavioral, economical, etc) with Multi-factor authentication or knowledge-based questions.

%\textbf{Post-understanding: Involvement of trusted entity} When asked to explain their understanding of PHC, P13 noted, \textit{"So it sounds like, basically, you it's similar to how you verify things before. Like you use a biometrics and your government Id. But then you get a personal key. You do it with like a trusted organization rather than each individual. And then you can use that key for all the different services you use."} P1 elaborated PHC process as a shift of the verification entity, \textit{" I'd say we are sort of moving the verification burden from the user side to a service provider side where they have access to our data, and they have access to the token that's assigned to each person that's unique. And that's easily like traceable across online platforms. and this token is used for verification with 3rd parties, where they don't get access to your personal data, but they only use this service provider to give them the authenticity that you are a real user."} These suggest that the role of the PHC issuer is recognized as a crucial component of PHC.}
%began by asking \textit{"How did you feel about using PHC to verify your identity when opening your bank account?"} To dive deeper, we also asked about potential benefits:\textit{"What potential benefits do you see in using PHC in this online banking context?"}. We also inquired about these aspects- \textit{"Do you think using PHC improves the security of your bank account? Why?", "Did this method of identity verification make you feel more confident about your privacy? why?"} Additionally, we discussed their comfort levels for providing credentials (e.g., Government-issued ID, biometric information) and asked about any concerns about data collection-\textit{"Were you comfortable providing your government-issued ID and using facial recognition? Why?"}
}

\iffalse
%%% column: scenario, credential, service providers.
\begin{table*}[h!]
\centering
\caption{Overview of PHC Application Scenarios}
\label{table:scenario}
%\resizebox{\textwidth}{!}{%
\begin{tabular}{lll}
\hline
\textbf{Scenario} & \textbf{Service Provider} & \textbf{Types of Credential} \\
\hline
Financial Service & Bank, Financial Institutions & Passport or Driver’s license, Face scan \cite{yousefi2024digital}\\
% \hline
Healthcare Service & Hospitals, Clinics & Health insurance card,  Fingerprint \cite{chen2012non,fatima2019biometric,jahan2017robust}\\
% \hline
Social Media & Tech Companies & National identity card, Video selfie \cite{instagramWaysVerify, metaTypesID,instagramTypesID} \\
% \hline
LLM Application & Tech Companies & Iris scan \cite{WorldWhitepaper, worldHumanness}\\
% \hline
Government Service & Government & Driver’s license or National identity card \cite{LogingovVerify}\\
% \hline
Employment Background Check & Background Check Companies & Tax identification card, Fingerprint\cite{cole2009suspect}\\
\hline
\end{tabular}%
%}
% \vspace{0.5em}
\label{tab:scenarios}
\end{table*}
\fi


%\textbf{Design Session.}
%\fixme{need to explain how and why you design the design session, where you designed, how participants were unstructured and so on.} \ayae{reflected in the following paragraph}

In the fourth section, we began by refreshing participants’ memories of the various risks and concerns discussed in the earlier scenario-based section. Following this, we guided participants to brainstorm potential design solutions by sketching their ideas to address these concerns. To facilitate the sketching process, we developed sketch notes in Zoom as prompts to help participants generate ideas, particularly when starting from scratch is challenging. 
%on Zoom whiteboard or pen and paper, using a think-aloud protocol.  
%Nevertheless, it is difficult to develop new ideas from scratch, so 
%Additionally, we described the main issues or concerns that the participants identified during the interview at the top of the sketch notes. 
%Participants can develop their ideas at the center of the whiteboard by locating the above components or creating new shapes, lines, or text boxes for their sketches. 
We also investigated participants' preferences for PHC regarding the issuers and issuance systems of PHCs, as well as the types of data required for issuing PHCs. 
%in the context of who issues PHC or type of issuance systems, and what types of data are needed to issue PHC to address RQ2. 
%An example includes- \textit{``What types of credential would you prefer to use as personhood verification? ; Which organizations or stakeholders would you prefer to issue and manage your PHC?''} 
We encourage participants to explain their reasoning. These questions were informed by insights from the pilot study, where participants expressed preferences for different types of data, system architecture, and various stakeholders involved in PHC issuing.
%However, these questions alone can only find optimal ways within the scope of currently existing options and cannot generate new design implications. Therefore,

\iffalse
\tanusree{we can cut this section as this didn't give any result and doesn't answer RQs directly.}Lastly, to understand preference on issuance system, we introduced the decentralized PHC system architecture with another instructional video. Following the video, we asked participants to explain their understanding of the decentralized PHC system and their preference for the issuance system (centralized or decentralized). We introduced it after the sketch session is that participants may organically come up with the idea of decentralized systems on their own, and we intended to avoid priming them. 
\fi
%Then, we asked them to explain their understanding of the decentralized PHC and preferred issuance system (centralized or decentralized.)- \textit{`` Could you explain why you would prefer decentralized system in managing your PHCs?''}
%\textit{"Would you prefer to get multiple PHCs from different issuers depending on the situation or application you're using, or would you rather have a single PHC from one issuer?"}

\textbf{Post-Survey.}
%%\fixme{need to explain how and why you design the design session, where you designed, how participants were unstructured and so on.}
We conducted a post-survey to obtain participants' PHC preference quantitatively. It included questions on participants' preference on credential type, issuer and issuance system  for the scenarios (e.g., financial, medical, etc) we considered in our interview.

\vspace{-2mm}
\subsection{Data Analysis}
\vspace{-2mm}
Once we got permission from the participants, we obtained interview data through the audio recording and transcription on Zoom. We analyzed these transcribed scripts through thematic analysis \cite{Braun2012-sz, Fereday2006-yv}. Firstly, all of the pilot interview data was coded by two researchers independently. Then, we compared and developed new codes until we got a consistent codebook. Following this, both coders coded 20\% of the interview data of the main study. We finalized the codebook by discussing the coding to reach agreements. Lastly, we divided the remaining data and coded them. After both researchers completed coding for all interviews, they cross-checked each other’s coded transcripts and found no inconsistencies. Lower-level codes were then grouped into sub-themes, from which main themes were identified. Lastly, these codes were organized into broader categories. Our inter-coder reliability (0.90) indicated a reasonable agreement between the researchers.
\iffalse

\begin{table*}[h]
\centering
\caption{Participant demographics and background.}
%\fixme{add the participants you completed so far}
\resizebox{\textwidth}{!}{%
\begin{tabular}{l l l l l l l l}
\hline
\textit{Participant ID} & \textit{Gender} & \textit{Age} & \textit{Country of residence} & \textit{Education} & \textit{Technology background}  & \textit{CS background} &\textit{Residency duration} \\
\hline
P1 & Male & 25-34 & the US & Master's degree & Yes & Yes &3-5 years\\
P2 & Female & 25-34 & the US & Master's degree & Yes & Yes & 1-3 years\\
P3 & Female & 25-34 & the UK & Master's degree & Yes & No & 1-3 years\\
P4 & Female & 35-44 & the UK & Some college, but no degree & Yes & Yes & Over 10 years \\
P5 & Male & 25-34 & the US & Doctoral degree & Yes & Yes & 5-10 years \\
P6 & Male & 35-44 & the US & Less than a high school diploma & No & No & Over 10 years \\
P7 & Male & 25-34 & the US & Doctoral degree & Yes & Yes & 3-5 years\\
P8 & Male & 45-54 & the US & Bachelor's degree & Yes & Yes & Over 10 years \\
P9 & Female & 25-34 & New Zealand & Master's degree & No  &  No &  Over 10 years\\
P10 & Male & 25-34 & the US & Master's degree & No & No & Over 10 years\\
P11 & Female & 25-34 & the UK & Bachelor's degree & No & No & Over 10 years\\
P12 & Male & 18-24 & the UK & Master's degree & Yes & Yes & 1-3 years\\
P13 & Male & 35-44 & the UK & Bachelor's degree & Yes & No & Over 10 years\\
P14 & Male & 25-34 & Sweden & High school graduate & No & No & Over 10 years \\
P15 & Female & 25-34 & Spain & Master's degree & Yes & Yes & Over 10 years \\
P16 & Female & 25-34 & Germany & Master's degree & Yes & Yes & Over 10 years \\
P17 & Female & 25-34 & Spain & Doctoral degree & No & No & Over 10 years \\
P18 & Female & 35-44 & the US & Bachelor's degree & No & No & Over 10 years \\
P19 & Female & 25-34 & Germany & Master's degree & Yes & Yes & 3-5 years \\
P20 & Male & 25-34 & Hungary & Master's degree & Yes & No & 3-5 years \\
P21 & Male & 35-44 & the US & Bachelor's degree & Yes & No & 5-10 years \\
P22 & Female & 18-24 & France & Master's degree & Yes & Yes & Less than 1 year\\
P23 & Male & 45-52 & the US & Master's degree & No & No & Over 10 years\\
\hline
\end{tabular}%
}
\label{table:demographics}
\end{table*}
\fi
\begin{figure}[tbh]
\centering
\includegraphics[width=0.49\textwidth]{sec/Figures/quali_2d.pdf}
\caption{
    \textbf{Qualitative comparisons of generated brushes for surface details.} Our method captures geometry details guided by texts, effectively preserving surface structure and avoiding mesh distortions.
}
\label{Fig: Qualitative 2D}
\end{figure}
\vspace{-0.6 cm}
\section{Experiments}
\label{sec:Experiment}
In this section, we conduct experiments to evaluate the various capabilities of Text2VDM both quantitatively and qualitatively for text-to-VDM brush generation.
% in ~\Cref{Qualitative} and ~\Cref{Quantitative}.
We then present an ablation study that validates the significance of our key insight into CFG-weighted SDS, as well as the effect of the region control and shape control.
% in ~\Cref{Ablation}.

\begin{figure*}[tbh]
\centering
\includegraphics[width=1\textwidth]{sec/Figures/quali_3d.pdf}
\caption{
    \textbf{Qualitative comparisons of generated brushes for geometric structures.}  Our method accurately presents key geometric features described by text, facilitating downstream applications in modeling software.
}
\label{Fig: Qualitative 3D}
\end{figure*}

\subsection{Qualitative Evaluation}
\label{Qualitative}
To the best of our knowledge, Text2VDM is the first framework to generate VDM brushes from text.
We adapted three existing methods for comparison and classified them into two categories. The first category includes Text2Mesh~\cite{single12-text2mesh} and TextDeformer~\cite{SIGGRAPH:TextDeformer:2023}, which generate a brush mesh through text-guided mesh deformation on a planar mesh, following a process similar to ours. For the second category, we opt to directly generate VDM via Paint-it~\cite{paintit}. Notably, this method originally uses SDS to optimize a UNet for generating PBR textures. We reframed it to suit our VDM brush generation task, modifying it to generate VDM through SDS optimization of the UNet. We compared the visual results in \Cref{Fig: Qualitative 2D} and \Cref{Fig: Qualitative 3D}.

Compared to other methods, Text2VDM can generate more vivid and better-quality VDM brushes. Text2Mesh applies displacement to each vertex along normal directions, resulting in limited mesh deformation. TextDeformer indicates the accumulation of local deformations in the Jacobians, which results in global mesh drift, making it challenging to bake these meshes into VDM.
Reframed Paint-it VDM generation is equivalent to optimizing the three-axis displacement of each vertex on the mesh with SDS. Although the UNet reduces noise from the SDS~\cite{paintit}, geometric regularization is still required to ensure mesh quality. The generated mesh must compromise between solving the problem and maintaining smoothness, which makes achieving high-quality mesh generation quite challenging.
% Using a UNet to generate VDM is equivalent to optimizing the three-axis displacement of each vertex on the mesh with SDS. Although the UNet reduces noise from the SDS~\cite{paintit}, geometric regularization is still required to ensure mesh quality. The generated mesh must compromise between solving the problem and being smooth, which results in low-quality mesh generation.

\subsection{Quantitative Evaluation}
\label{Sec: Quantitative}

We quantitatively evaluated our framework regarding generation consistency with text input and mesh quality. We used 40 distinctive text prompts for VDM generation.

\noindent\textbf{Generation Consistency with Text.} We initially assessed the relevance of the generated results to the text descriptions~\cite{CLIP:CORR:2021}. 12 different views were rendered for average scores respectively, as presented in Table~\ref{tab:quantitative comp}. Our approach achieves the highest scores compared to baseline methods.


\begin{table}[h!]
\caption{Quantitative evaluation of state-of-the-art methods. The geometry CLIP score is calculated on shaded images with uniform albedo colors~\cite{Richdreamer:CVPR:2024}, and self-intersection is quantified as the ratio of self-intersected mesh faces to the total number of faces.}
\centering
\footnotesize   % incase not overflow
% TADA & TextDeformer & Fantasia3D
\begin{tabular}{*{10}{c}}
         \hline
           & Geometry CLIP Score $\uparrow$ & Mesh Self-Intersection $\downarrow$\\
         \hline 
         Paintit & $0.2375$ & $19.42\%$  \\
         Text2Mesh & \underline{0.2497}  & $7.18\%$\\
         TextDeformer & $0.2477$  & \textbf{0.04\%} \\
         Ours & \textbf{0.2556} &  \underline{0.77\%}\\
         \hline
\end{tabular}
\label{tab:quantitative comp}
\end{table}

\begin{figure*}[tbh!]
\centering
\includegraphics[width=1\textwidth]{sec/Figures/ablation_sds.pdf}
\caption{
    \textbf{Effect of CFG-weighted SDS.} CFG-weighted SDS effectively mitigates semantic coupling issues in SDS, such as generating the tortoise’s tail and head or the snail’s head, by providing more focused semantic guidance. In contrast, CSD adds extra negative terms that fail to decouple semantics, resulting in a less stable and more time-consuming optimization process.
} 
\label{Fig: Effect of CFG-weighted SDS}
\end{figure*}

\noindent\textbf{Mesh Quality.} We evaluated mesh quality by examining self-intersection. Paint-it and Text2Mesh, which utilize direct vertex displacement, often converge to a local minimum and disregard the mesh triangulation. While TextDeformer exhibits the lowest self-intersection, its tendency to produce over-smoothed results frequently results in losing object features described in text prompts. 

\begin{table}[h!]
\caption{User evaluation of generated VDMs.}
\centering
\footnotesize
    \begin{tabular}[width=1.0\textwidth]{*{10}{c}}
         \hline 
         User Preference $\uparrow$  & Geometry Quality & Consistency  with Text\\
         \hline 
         Paintit & $3.1\%$  & $1.7\%$ \\
         Text2Mesh & \underline{$18.3\%$} & \underline{$27.3\%$} \\
         TextDeformer & $3.3\%$ & $3.4\%$ \\
         Ours & \textbf{75.3\%} & \textbf{67.6\%} \\
         \hline
    \end{tabular}
\label{tab:user comp}
\end{table}


\noindent\textbf{User Study.} We further conducted a user study to evaluate the effectiveness and expressiveness of our method. A Google Form was utilized to assess 1) geometry quality and 2) consistency with text. We recruited 32 participants, of whom 14 are graduate students majoring in media arts, and 18 are company employees specializing in AI content generation. In this form, the participants were instructed to choose the preferred renderings of VDM from different methods in randomized order, as shown in Table~\ref{tab:user comp}. The results show participants preferred our method by a significant margin. 
% For practical evaluation, we invited 5 participants to use VDMs generated by our methods in Blender to sculpt 3D models that aligned with their expectations (Figures~\ref{Fig: Local to Global Mesh Stylization} and~\ref{Fig: Coarse to Fine Interactive Modeling}).

\subsection{Ablation Study}
\label{Ablation}
\textbf{Effects of CFG-Weighted SDS.}  %为了验证CFG-weighted SDS的有效性,我们设置的实验对比了直接使用SDS,使用CFG-Weighted SDS 以及使用三种不同negative prompt权重的CSD。As mentioned in ~\Cref{sec: tesds}, SDS在没有全局语义作为reference的情况下进行local component生成时会有语义耦合的情况,生成出来的mesh会有明显的瑕疵。另外我们发现使用negative prompt这种直观的做法并不能有效的解耦语义,并且增大负文本的权重时会使得优化过程变得不稳定,更难收敛,导致了低质量的mesh。相比之下,我们的方法不需要进行额外的Unet推理,并且能够有效的对语义进行解耦生成符合要求的mesh。
We conducted experiments to compare the generated results of directly using SDS~\cite{DreamFusion:ICLR:2022}, CFG-weighted SDS, and CSD~\cite{CSD:Arxiv:2023} with three different annealed weights of negative prompt (\Cref{Fig: Effect of CFG-weighted SDS}). As discussed in \Cref{sec: tesds}, SDS can result in semantic coupling when generating sub-object structure, leading to artifacts like the tortoise's tail and head or the snail's head. We also found that using negative prompts was ineffective at decoupling semantics. Increasing the initial weight of negative prompts further makes the optimization unstable, resulting in low-quality results. In contrast, our method effectively mitigates semantic coupling to produce high-quality meshes without requiring additional UNet inference.
\begin{figure}[tbh!]
\centering
\includegraphics[width=0.48\textwidth]{sec/Figures/ablation_masks2.pdf}
\caption{
    \textbf{Effect of region control.} Region masks can effectively control the shape of surface details based on different text inputs.
}
\label{Fig: Effect of region mask}
\end{figure}

\noindent\textbf{Effects of Region Control.} %我们在图中展示了两组region mask在不同text prompt下对surface detailed brush生成的控制能力。我们生成的结果可以很好得match text生成cloth,metal,stone等不同质感,同时符合region mask所限制的形状
\Cref{Fig: Effect of region mask} demonstrates two sets of region masks and their control over surface details generation under different text prompts. Without using a region mask, the results lack a specific shape, which may not satisfy the desired stylized effect. By using a region mask, our generated results effectively conform to the user's desired shapes while also aligning with the styles specified by the text, such as metal and stone.


\noindent\textbf{Effects of Shape Control.} % 我们的方法在图中展示了生成结果与通过shape map进行初始化体积与方向保持一致的能力,在不同的local component生成中,比如beard,pauldron,elf ear,我们的方法能够在保持体积和方向大致稳定的情况下生成多样的符合文本描述的结果
Our method demonstrates that user-specified VDMs can effectively control the volume and direction of generated geometric structures. As shown in \Cref{Fig: Effect of shape map}, various generated geometric structures, such as elf ears and pauldrons, are high-quality and align with the text descriptions. We also found that without volume initialization, it is challenging to generate desired results. It indicates that this initialization is crucial for steering the gradient flow of geometric structure generation via adjusting the Laplacian term.
\begin{figure}[!htb]
\centering
\includegraphics[width=0.49\textwidth]{sec/Figures/ablation_shape.pdf}
\caption{
    \textbf{Effect of shape control.} User-specified VDMs can help achieve the intended final effect of geometric structures by initializing the brush's volume and direction.
}
\label{Fig: Effect of shape map}
\end{figure}
\vspace{-0.3 cm}
\subsection{Applications}
\label{Application}
Once various VDM brushes are generated, users can directly use these brushes to meet diverse creative needs in mainstream modeling software. For example, they can apply VDM brushes for mesh stylization and engage in a real-time iterative modeling process.

\noindent\textbf{Local-to-Global Mesh Stylization.} Although mesh stylization is a complex task even for professional artists, combining different surface details allows users to achieve stylization quickly. For instance, users can apply a variety of wall-damage brushes to specific areas of a stone pillar, creating a style of damage (~\Cref{Fig: Local to Global Mesh Stylization}).
% Similarly, they can use different rust-effect brushes on a helmet to give it an aged style, 

\begin{figure}[!htb]
\centering
\includegraphics[width=0.48\textwidth]{sec/Figures/application.pdf}
\caption{\textbf{Local to global mesh stylization.} Applying various surface details brushes can create a damaged-style stone pillar model.} 
\label{Fig: Local to Global Mesh Stylization}
\end{figure}

\noindent\textbf{Coarse-to-Fine Interactive Modeling.} Unlike previous methods~\cite{magiclay,tipeditor} that require a lengthy optimization process for each edit and result in non-reusable outcomes, our generated VDM brushes can be directly used in modeling software. This enables users to apply the generated brushes easily and interactively. For example, \Cref{Fig: Coarse to Fine Interactive Modeling} shows that users can combine various brushes, such as skeleton hand, rose pattern, and pauldron to refine a coarse cloth model into a highly detailed one.


\section{Conclusion}
We introduce a novel approach, \algo, to reduce human feedback requirements in preference-based reinforcement learning by leveraging vision-language models. While VLMs encode rich world knowledge, their direct application as reward models is hindered by alignment issues and noisy predictions. To address this, we develop a synergistic framework where limited human feedback is used to adapt VLMs, improving their reliability in preference labeling. Further, we incorporate a selective sampling strategy to mitigate noise and prioritize informative human annotations.

Our experiments demonstrate that this method significantly improves feedback efficiency, achieving comparable or superior task performance with up to 50\% fewer human annotations. Moreover, we show that an adapted VLM can generalize across similar tasks, further reducing the need for new human feedback by 75\%. These results highlight the potential of integrating VLMs into preference-based RL, offering a scalable solution to reducing human supervision while maintaining high task success rates. 

\section*{Impact Statement}
This work advances embodied AI by significantly reducing the human feedback required for training agents. This reduction is particularly valuable in robotic applications where obtaining human demonstrations and feedback is challenging or impractical, such as assistive robotic arms for individuals with mobility impairments. By minimizing the feedback requirements, our approach enables users to more efficiently customize and teach new skills to robotic agents based on their specific needs and preferences. The broader impact of this work extends to healthcare, assistive technology, and human-robot interaction. One possible risk is that the bias from human feedback can propagate to the VLM and subsequently to the policy. This can be mitigated by personalization of agents in case of household application or standardization of feedback for industrial applications. 

\bibliographystyle{IEEEtran}
\bibliography{references}

\newpage

\begin{IEEEbiography}
[{\includegraphics[width=1in,height=1.25in,clip,keepaspectratio]{photo/mtli.jpg}}]{Mengtian Li} currently hold a position as a Lecturer of Shanghai University, while simultaneously fulfilling the responsibilities of a Post-doc of Fudan University. She received Ph.D. degree from East China Normal University, Shanghai, China, in 2022. She serves as reviewers for CVPR, ICCV, ECCV, ICML, ICLR, NeurIPS , IEEE TIP and PR, etc. Her research lies in 3D vision and computer graphics, focuses at human avatar animating and 3D scene understanding, reconstruction, generation.
\end{IEEEbiography}

% \begin{IEEEbiography}
% [{\includegraphics[width=1in,height=1.25in,clip,keepaspectratio]{photo/zhai.jpg}}]{Chengshuo Zhai} received her Bachelor's degree in Digital Media Technology from Yanbian University in 2018 and is currently pursuing a Master's degree at the Shanghai Film Academy, Shanghai University. Her primary research focus is on human motion, specifically the generation of cross-modal human motion sequences.
% \end{IEEEbiography}
\vspace{-33pt}
\begin{IEEEbiography}
[{\includegraphics[width=1in,height=1.25in,clip,keepaspectratio]{photo/ysx.jpg}}]{Yaosheng Xiang} received his Bachelor's degree from Huaqiao University and is currently pursuing a Master's degree at the Shanghai Film Academy, part of Shanghai University. His research interests primarily focus on digital human reconstruction, with an emphasis on creating and editing animatable avatars from video footage.
\end{IEEEbiography}
\vspace{-33pt}
\begin{IEEEbiography}
[{\includegraphics[width=1in,height=1.25in,clip,keepaspectratio]{photo/ck.jpg}}]{Chen Kai} is currently a graduate supervisor at the Shanghai Film Academy of Shanghai University. He is the Director of the Shanghai Film Special Effects Engineering Technology Research Center and the Director of the Shanghai University film-producing workshop. He obtained a Master of Fine Arts (MFA) degree from the École Nationale Supérieure des Beaux-Arts de Le Mans in France, majoring in Contemporary Art. He participated in developing animation software Miarmy won the 70th Tech Emmy Awards presented by NATAS (National Academy of Television Arts \& Sciences) in 2018. His creative pursuits include experimental cinema, photography, digital interactive installations, and other forms of art.
\end{IEEEbiography}
\vspace{-33pt}
\begin{IEEEbiography}
[{\includegraphics[width=1in,height=1.25in,clip,keepaspectratio]{photo/xie.jpg}}]{Zhifeng Xie} received the Ph.D. degree in computer application technology from Shanghai Jiao Tong University, Shanghai, China. He was a Research Assistant with the City University of Hong Kong, Hong Kong. He is currently an Associate Professor with the Department of Film and Television Engineering, Shanghai University, Shanghai. He has published several works on CVPR, ECCV, IJCAI, IEEE Transactions on Image Processing, IEEE Transactions on Neural Networks and Learning Systems, and IEEE Transactions on Circuits and Systems for Video Technology. His current research interests include image/video processing and computer vision.
\end{IEEEbiography}
\vspace{-33pt}
\begin{IEEEbiography}[{\includegraphics[width=1in,height=1.25in,clip,keepaspectratio]{photo/keyu.png}}]{Keyu Chen} is a senior AI researcher affiliated with Tavus Inc.. He obtained the master and bachelor degree from University of Science and Technology of China in 2021 and 2018. His research interests are mainly focused on digital human modeling, animation, and affective analysis.
\end{IEEEbiography}

\vspace{-33pt}
\begin{IEEEbiography}
[{\includegraphics[width=1in,height=1.25in,clip,keepaspectratio]{photo/Yu-GangJiang.jpg}}]{Yu-Gang Jiang} received the Ph.D. degree in Computer Science from City University of Hong Kong in 2009 and worked as a Postdoctoral Research Scientist at Columbia University, New York, from 2009 to 2011. He is currently Vice President and Chang Jiang Scholar Distinguished Professor of Computer Science at Fudan University, Shanghai, China. His research lies in the areas of multimedia, computer vision, and trustworthy AGI. He is a fellow of the IEEE and the IAPR.
\end{IEEEbiography}


\end{document}


