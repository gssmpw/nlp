\section{Related Work}
\label{relatedwork}
The state of the art for black-box learning is summarized, followed by an overview of hybrid-learning directions with a focus on physics-informed ML for (soft) robots. 
Our contributions and claims complete this section.
\subsection{Black-Box Learning}
Within soft robotics, data-driven approaches for control are still in their infancy~\cite{Wang.2021b}, with much progress having been made in recent years. 
In~\cite{Braganza.2007}, \textit{feedforward neural networks} (NNs) were first applied to feedforward control of soft extensible continuum manipulators. 
Many works also use feedforward NNs for MPC, e.g., of a soft actuator with one degree of freedom (DoF)~\cite{Gillespie.2018}, a soft continuum joint~\cite{Cheney.2024}, and a six-DoF soft robot~\cite{Hyatt.2019,Hyatt.2020}.
In~\cite{Habich.2023}, Gaussian processes are trained to realize learning-based position and stiffness feedforward control of a soft actuator.

Nonlinear material behavior and friction cause hysteresis effects that can be captured using \textit{recurrent neural networks}~\cite{Lipton.29.05.2015}. 
These are used in~\cite{Thuruthel.2017} to learn the direct dynamics of a soft manipulator with a nonlinear autoregressive exogenous model. 
Only an open-loop predictive control policy was implemented, which was further developed in~\cite{Thuruthel.2019} for closed-loop control.
Data-driven MPC was successfully realized for a single actuator by using long short-term memory (LSTM) units~\cite{Luong.2021} and for the ASR from Fig.~\ref{fig:cover}(b) via gated recurrent units (GRUs)~\cite{Schafke.2024}.

Recording real-world data requires a lot of effort, and trained networks only achieve \textit{good accuracy within the seen data space}.
In case of system changes, training must be repeated with new data due to\textit{ poor extrapolation}. 
In addition, convergence is more difficult for high-dimensional NNs required to model multi-DoF robots. 
For this reason, the prediction of the entire state vector was not possible with the chosen network architecture and the existing data in~\cite{Hyatt.2019}.
Thus, only the velocity at the next time step was learned. 
Data-driven control of a soft robot using the Koopman operator is realized in~\cite{Bruder.2021}, and, more recently, a data-efficient method based on neural ODEs~\cite{Chen.2018} was used to model a soft manipulator~\cite{Kasaei.2023}. 
However, the authors do not use physical knowledge, which results in poor generalization.

For \textit{different payloads}, reinforcement learning using an LSTM network as a forward-dynamics model was conducted in~\cite{Centurelli.2022}.
However, even for only $\nu{=}1$ varying model parameter (payload), different real-world datasets with variable payload conditions are necessary as training data.
We do not believe that such black-box approaches are scalable for real-world applications, i.e., $\nu{>}1$: 
For the minimum requirement of two levels per model parameter (e.g., minimum and maximum payload), $2^{\nu}$~datasets are already required.
This implies that for variations of several parameters, a time-consuming and expensive data acquisition is necessary.
In addition, some parameters require finer variations and, therefore, more than two levels due to poor interpolation.  
%However, the authors conclude that with such NNs, no hysteresis effects can be learned. These occur strongly with soft robots because of the nonlinear material behavior and also due to joint friction within ASRs. 
%State-space models of nonlinear systems can be transformed into recurrent neural networks (RNNs) and vice versa \cite{Schussler.2019}. 
 \subsection{Directions of Hybrid Learning}
The problems above can be solved using hybrid approaches by applying \textit{physical knowledge in combination with black-box learning}. 
Within the pioneering work of Nguyen-Tuong and Peters for rigid robots~\cite{NguyenTuong.2010}, knowledge of the parametric dynamics model is incorporated into the nonparametric Gaussian process via mean or kernel function. 
%Unlike NNs, the prediction time of GPs scales linearly in the number of training examples. 
Finite-element models have been used to analyze soft robotic segments, thus generating vast amounts of (simulated) training data for feedforward NNs~\cite{Runge.2017}. 
%However, this requires a highly accurate simulation, which depends to a large extent on material parameters and manufacturing tolerances that are difficult to determine. 
External loads such as gravity or contact forces are not taken into account. 

\textit{Residual/Error learning} of physical models using data is one main direction of hybrid learning.
This has been done for a soft continuum joint~\cite{Johnson.2021}, a soft parallel robot~\cite{Huang.2024}, and soft continuum robots~\cite{Reinhart.2017, Gao.2024, Lou.2024, Jiahao.2024}. 
The main requirement for these approaches is that \textit{the physical model must be efficiently evaluated} during real-time control or estimation, which is often not the case for accurate models of soft robots.
Another possible disadvantage is the need for real experimental data from all domains.
This indicates, due to poor extrapolation, that a new error model must be learned if the system changes after training. 
Consequently, the authors of~\cite{Lou.2024} collect training data in several domains (different payloads between $\SI{0}{\gram}$ and~$\SI{300}{\gram}$) in order to ensure generalization within this data region (interpolation).
However, we argue that trained models should extrapolate across the available training data and that data acquisition for every possible system change is not practical.
In line with this, it was recently concluded that the \textit{generalization beyond bounded training data and the handling of novel dynamical events} should be examined in future work~\cite{Gao.2024}.
Also, the authors of~\cite{Huang.2024} declare robust control with learned models in changing environments as future work.
%A sufficiently accurate model is generally required so that error learning does not become too complex and may not converge for multi-DoF systems. 
%Further, constant joint stiffnesses and damping are assumed in~\cite{Johnson.2021}. This is not true for antagonistic actuators with nonlinear material behavior and joint angle limits~\cite{Vanderborght.2013}. Learning those latent, nonlinear quantities would be more useful. This must be done unsupervised, since \textit{no measured data} for damping or stiffness are available.

Alternatively to error learning, \textit{prior knowledge for constraining NN training} results in less data required. 
This was realized in~\cite{NEUMANN.2013} using extreme learning machines with only one hidden layer. 
Due to the simple network structure, linear constraints, such as the monotonic behavior of the drives or physical restrictions of the soft robot, can be taken into account using quadratic optimization. 
In~\cite{Tariverdi.2021}, a soft manipulator was discretized according to continuum-mechanics principles, and recurren neural networks (RNNs) were trained for each node, taking applied forces and moments for every node as inputs. 
The authors conclude that the supervised learning used might not entirely respect underlying physics, and \textit{trained models are not applicable to changing dynamics without retraining}.

The aforementioned problems are countered by \textit{physics-informed machine learning}~\cite{Karniadakis.2021}, also known as scientific machine learning~\cite{Cuomo.14.01.2022}. 
This is done by including physical knowledge into the training process so that \textit{trained regressors will be aware of governing physical laws}. %Existing NN approaches can be extended to PINNs~\cite{Raissi.2019}, allowing resulting models to plausibly predict outside seen data.
Regarding modeling and control of dynamical systems, physics-informed ML is an alternative approach compared to classical gray-box modeling: 
Instead of (or in addition to) identifying parameters of white-box models consisting of simplified modeling assumptions, physics-informed ML comes from the black-box side without modeling simplifications and is equipped with additional constraints such as symmetries, causal relationships, or conservation laws~\cite{Nghiem.2023}.
Underlying ODEs/PDEs can act as learning bias during the training process, limiting the high-dimensional search space and, therefore, the necessary amount of real data while still maintaining a high generalization. 
This is beneficial since real-world experiments are expensive, and \textit{the training data will rarely represent all possible system conditions during inference}.
Besides high generalizability despite less data, trained surrogate models are considerably faster to evaluate since they do not rely on fine spatiotemporal discretization required for numerical integration of conventional ODEs/PDEs. 
This significantly speeds up the solution of hard optimization problems, such as during nonlinear MPC, which is outlined in a recent overview~\cite{Nghiem.2023} as one of the main opportunities for physics-informed~ML.

%The unsupervised learning of latent nonlinear state variables such as stiffness and damping is also possible, although no measured data for these quantities are available. This is realized by the differential equations encoded in the network~\cite{Zhang.2020,Raissi.2020}.
\subsection{Physics-Informed Machine Learning in (Soft) Robotics}
For rigid systems, physics-inspired networks using Lagrangian and Hamiltonian mechanics were proposed~\cite{Lutter.2019,Lutter.05.10.2021}. 
Mass matrix, centrifugal, Coriolis, gravitational, and frictional forces as well as (potential and kinetic) energies are learned unsupervised by minimizing the ODE residual. 
In~\cite{Nicodemus.2022}, PINNs are employed for the nonlinear MPC of a simulated rigid robot with two degrees of freedom (DoF).

Within soft robotics, hybrid RNNs using responses from the physical model as additional inputs for state observation of one-DoF soft actuators (McKibben actuator and soft finger) are utilized in~\cite{Sun.2022}. 
A physics-informed simulation model is trained from finite-element (FE) data and is used as \textit{efficient-to-evaluate surrogate model} within MPC in~\cite{Lahariya.2022}. 
A simulated one-DoF soft finger is considered, and the authors identify the transfer to physical prototypes and extension to multi-DoF tasks as a future research direction. 
Similarly, PINNs are trained as fast surrogate models of soft robotic fingers in~\cite{Wang.2024,Beaber.2024}.
To enable fast prediction of Cosserat rod statics, PINNs are trained for one simulated tendon-driven segment of a continuum robot~\cite{Bensch.2024}.
A real one-segment tendon-driven soft robot was modeled in~\cite{Liu.2024} using Lagrangian neural networks with measurement data. 
Since the network is trained with offline-recorded data, the generalization to changes of the system dynamics would be poor.

The authors of~\cite{Yoon.2024} propose a kinematics-informed neural network for pneumatic and tendon-driven soft robots with one segment. 
The generalization ability is examined for motor commands, which are unseen during network training. 
Although this is already an important study, generalization to changed systems would be even more crucial. 
Learning system dynamics instead of pure kinematics would also be advantageous for optimal control, which is realized in~\cite{Chhatoi.2023} for an ASR using differential dynamic programming. 
Thereby, forward dynamics are simulated via inversion of the inertia matrix and they declare the realization of MPC as future work. 
For our robot, forward simulation via inversion of the inertia matrix and numerical integration of the \textit{stiff ODE} is very time-consuming. 
Therefore, a fast and accurate surrogate model is necessary for MPC. 

In~\cite{Mendenhall.2024}, experimental data is extended with simulated FE data of \textit{out-of-distribution} (unseen) loading/deformation cases to train a PINN of a single-bellows actuator. 
The main motivation here is to build a fast surrogate of a computationally expensive FE model.
Although the NN is 435 times faster compared to the FEM, one simulation still takes ${>}\SI{20}{\second}$ for the parallel actuator consisting of five bellows. 
This computation time further increases for soft robots with more bellows making it unsuitable for real-time applications. 
Also, only statics are modeled to predict deformations.

Most similar to our proposed approach,~\cite{Wang.2024c} formulate a physics-informed LSTM network with a \textit{variable system property as model input}. 
This allows the training of the model for unseen system properties, which are not present when recording training data. 
However, it is only applied to a simple one-DoF mass-spring-damper system.
Control is also not considered.

\subsection{Contributions}
Laschi et al.~\cite{Laschi.2023} conclude that\textit{ traditional modeling techniques must be incorporated into existing learning strategies} for novel advancements in the soft-robotics field.
Recently, Falotico et al.~\cite{Falotico.2024} also declare the integration of physical principles into machine-learning approaches as a perspective for solving current issues in terms of accuracy and computational efficiency, \textit{explicitly mentioning PINNs}.  
Related soft-robotics work only considers simple (often simulated) actuators and not PINNs for control (PINCs)~\cite{Antonelo.2024}, which are specifically designed for real-time MPC.
To the best of our knowledge, no published work considers PINNs for fast MPC of -- either real nor multi-DoF -- soft robotic systems.
Dynamics changes after training are also predominantly neglected.

Our work \textit{contributes} to this field: 
\textbf{1)}~A first-principles model of our ASR is derived and identified, which is very accurate, but the forward simulation requires significant computing time. 
This justifies the need for an alternative model approach with low computational cost during inference.
\textbf{2)}~For the first time, the original form of physics-informed neural networks -- introduced by Raissi et al.~\cite{Raissi.2019} and formulated for state-space modeling in~\cite{Antonelo.2024,Krauss.2024} -- is applied to a real multi-DoF soft robot.
This is especially relevant for model-based estimation/control, where \textit{models with low computational cost and high accuracy even for changed dynamics} are required.
\textbf{3)}~Generalizability is achieved by extending existing work on PINNs for control~\cite{Antonelo.2024,Krauss.2024} to consider system changes after training.
We perform a systematic hyperparameter optimization (HPO) during the PINN training, which is often neglected due to excessive training times of several days.
\textbf{4)}~Two hours of experimental data of the soft robot are used for evaluation, whereby the dynamics of the robot are changed. 
The proposed model is compared against the gold standard of both modeling worlds, namely a hyperparameter-optimized RNN, and an identified FP model, regarding prediction speed and accuracy.
\textbf{5)}~For the first time, learning-based MPC with PINNs is realized for multi-DoF soft robots and validated in several dynamic real-world experiments, including comparison with a baseline controller.
\textbf{6)}~The whole codebase for learning and control is published as open source\footnote{\label{foot:pinn}After acceptance, the link to the public repository will be placed here.} for possible applications beyond soft robotics. 
All datasets are also available to enable reproducibility.

In sum, we make \textit{three claims}: 
First, our surrogate model outperforms an accurate physics-driven model in terms of prediction speed by orders of magnitude with only slightly reduced accuracy.
Second, by incorporating physical knowledge during training, the PINN achieves higher prediction accuracies and generalization to out-of-distribution data compared to a recurrent neural network.
Throughout this article, all models receive data from only one domain during training/identification and are tested in unseen test domains, as depicted in Fig.~\ref{fig:cover}(b).
Third, accurate learning-based nonlinear MPC with PINNs is enabled for different soft-robot dynamics without the need of retraining the system model or retuning the controller.
This is intended to represent one practical scenario with high demands on prediction speed and accuracy. 
All claims are proven experimentally.