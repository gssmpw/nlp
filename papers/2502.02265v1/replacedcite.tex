\section{Related Works}
To address the challenges of robotic control, researchers have developed several methodologies to enhance reinforcement learning (RL), improve control precision, and increase robustness. Three prominent techniques are Reward Shaping, Integrator Feedback, and Hindsight Experience Replay (HER).

\textbf{Reward Shaping}:
Reward Shaping modifies the reward function to help the agent learn faster and more effectively. It adds an extra reward to encourage minimizing the steady-state error, the difference between the desired and actual states. This additional feedback helps the agent perform better relative to the target state____.

\textbf{Integrator Feedback}:
Inspired by classical control theory, Integrator Feedback incorporates the integral of past error signals into the agent's observations. This reduces steady-state errors by considering both the current error and its integral.  However, it can introduce complexity and instability, leading to overshoot and oscillatory behavior. While it enhances control precision, it should be implemented carefully to avoid negative impacts on system stability ____.

\textbf{Hindsight Experience Replay (HER)}:
HER is an experience replay technique that reinterprets failed trajectories as successful ones by considering different, actually achieved goals. This method is particularly useful in goal-conditioned RL problems. By treating failures as learning opportunities, HER improves the efficiency and effectiveness of RL in complex and varied settings, especially for precise control and goal-oriented tasks ____.

Both HER and Reward Shaping can be effectively integrated into the Adviser-Actor-Critic (AAC) framework, enhancing its performance. These strategies complement AAC by enriching the quality of experience. However, integral feedback does not serve a purposeful role within the AAC architecture. AAC offers a sophisticated alternative to integral feedback, especially for complex control tasks.