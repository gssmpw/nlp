Recall the Graph Connectivity problem, where the input is an unknown undirected graph on $n$ labeled vertices. In each query, the algorithm picks an unordered pair $e=(u,v)\in \binom V2$, and the oracle returns whether $e$ is an edge of $G$, flipped independently with probability $0<p<\frac 12$. The goal of the algorithm is to determine whether $G$ is connected or not.


In this section, we prove \cref{thm:graph-conn-hard}, which we recall below:
\GraphConnectivity*

\subsection{Preliminaries} \label{sec:conn:prelim}

\begin{lemma}[Stirling's formula] \label{lem:stirling}\
  \begin{enumerate}
    \item There exists an absolute constant $\epsilon>0$ such that for all $n\ge 1$,
    \begin{align*}
      \epsilon \sqrt n\left(\frac ne\right)^n \le n! \le \epsilon^{-1} \sqrt n\left(\frac ne\right)^n.
    \end{align*}
    \item Fix $m\ge 2$. There exists an constant $\epsilon=\epsilon(m)>0$ such that for all $n\ge 1$ and all $k_1,\ldots,k_m\ge 1$ with $k_1+\cdots+k_m=n$,
    \begin{align*}
      \epsilon \cdot \frac{n^{n+1/2}}{\prod_{i\in [m]} k_i^{k_i+1/2}} \le \binom{n}{k_1,\ldots,k_m} \le \epsilon^{-1} \cdot \frac{n^{n+1/2}}{\prod_{i\in [m]} k_i^{k_i+1/2}}.
    \end{align*}
  \end{enumerate}
\end{lemma}

Let $(\bP_n)_n$ and $(\bQ_n)_n$ be two sequences of probability measures such that $\bP_n$ and $\bQ_n$ are defined on the same measurable spaces $\Omega_n$.
We say $\bQ_n$ is contiguous with respect to $\bP_n$, denoted by $\bQ_n\contig \bP_n$, if for every sequence $(\cA_n)_n$ of measurable sets $\cA_n\subseteq \Omega_n$, $\bP_n(\cA_n) \to 0$ implies $\bQ_n(\cA_n)\to 0$.

\begin{lemma}[Cayley's formula] \label{lem:cayley}
The number of spanning trees on $n$ labeled vertices is $n^{n-2}$.
\end{lemma}


\subsection{Structure of a uniform spanning tree} \label{sec:conn:ust}
Our proof of \cref{thm:graph-conn-hard} uses several properties of the uniform random spanning tree (UST) of the complete graph.
In this section we state and prove these properties.

The notion of balanced edges is crucial to our construction and analysis.
\begin{definition}[Balanced edges] \label{defn:conn:balanced-edge}
  Let $\beta>0$ be a constant.
  Let $T$ be a spanning tree on $n$ labeled vertices.
  An edge $e\in T$ is called \emph{$\beta$-balanced} if both sides of $e$ has at least $\beta n$ vertices.
  Let $B_\beta(T)$ denote the set of $\beta$-balanced edges of $T$.
\end{definition}

\begin{proposition} \label{prop:conn:ust-structure-1}
  Let $T$ be a UST on $n$ labeled vertices.
  There exist absolute constants $\epsilon,\gamma_1,\gamma_2,\gamma_3>0$ such that with probability at least $\epsilon$, the following are true simultaneously.
  \begin{enumerate}[label=(\roman*)]
    \item\label{item:prop:conn:ust-structure-1:i}
    \begin{align*}
      \gamma_1 \sqrt n \le \left| B_{1/3}(T) \right| \le \gamma_2 \sqrt n.
    \end{align*}
    \item\label{item:prop:conn:ust-structure-1:ii} For all $e\in B_{1/3}(T)$, if $T_1$, $T_2$ are the two connected components of $T\backslash e$, then
    \begin{align*}
      \min\{\left| B_{1/7}(T_1) \right|,\left| B_{1/7}(T_2) \right|\} \ge \gamma_3 \sqrt n.
    \end{align*}
  \end{enumerate}
\end{proposition}
Our proof of \cref{prop:conn:ust-structure-1} uses the following lemmas.

\begin{lemma}[Balanced edges form a chain] \label{lem:conn:balanced-edge-chain}
  Let $T$ be a tree on $n$ labeled vertices.
  Then the subgraph formed by all edges in $B_{1/3}(T)$ is either empty or a chain.
\end{lemma}
\begin{proof}
  Suppose $B_{1/3}(T)$ is non-empty.
  Let $H$ be the subgraph formed by all edges in $B_{1/3}(T)$.

  \paragraph{Step 1.} We prove that $H$ is connected.
  Let $e_1,e_2\in B_{1/3}(T)$ and $e_3$ be an edge on the path (in $T$) between $e_1$ and $e_2$.
  Let $T_{i,1},T_{i,2}$ ($i=1,2,3$) be the two connected components of $T\backslash e_i$,
  such that $e_3\in T_{1,1}$, $e_3\in T_{2,1}$, $e_1\in T_{3,1}$, $e_2\in T_{3,2}$.
  Then $|T_{i,j}|\ge \frac n3$ for $i,j\in \{1,2\}$.
  We have $T_{3,i} \supseteq T_{i,2}$ ($i=1,2$).
  So $|T_{3,i}| \ge \frac n3$ for $i=1,2$ and $e_3$ is $\frac 13$-balanced.
  This shows that $H$ is connected.

  \paragraph{Step 2.} We prove that $H$ is a chain.
  By Step 1, $H$ is a connected subgraph of $T$, so it is a tree.
  Suppose $H$ has a vertex $v$ of degree at least three.
  Then there exist three distinct $\frac 13$-balanced edges $e_1,e_2,e_3$ containing $v$.
  Say $e_i=(v,u_i)$ ($i=1,2,3$).
  Let $T_i$ ($i=1,2,3$) be the connected component of $T\backslash e_i$ not containing $v$.
  Then $|T_i|\ge \frac n3$ and $T_i$ ($i=1,2,3$) are all disjoint.
  This implies
  \begin{align*}
    n\ge |T_1|+|T_2|+|T_3|+1 \ge n+1,
  \end{align*}
  which is a contradiction.
  So all vertices of $H$ have degree at most two, and $H$ must be a chain.
\end{proof}

The following statement about typical distances in a UST is well-known (e.g., \cite{aldous1991continuum}).
\begin{lemma}[Typical distance in UST] \label{lem:conn:ust-dist}
  Let $T$ be a UST on $n$ labeled vertices and $u,v$ be two fixed vertices (not dependent on $T$).
  Then for any $\epsilon>0$, there exist absolute constants $\gamma_1,\gamma_2>0$, such that
  \begin{align*}
    \bP\left[ \gamma_1 \sqrt n \le \dist_T(u,v) \le \gamma_2 \sqrt n \right] \ge 1-\epsilon,
  \end{align*}
  where $\dist_T$ denotes the graph distance in $T$.
\end{lemma}

To prove \cref{prop:conn:ust-structure-1}, we define a probability measure $\bQ$ on the space of spanning trees on $n$ labeled vertices that is contiguous with respect to the UST measure, and \cref{prop:conn:ust-structure-1} holds under $\bQ$.

\begin{definition}[Measure $\bQ$] \label{defn:conn:ust-measure-q}
  We define the measure $\bQ$ as follows.
  A \emph{tree tuple} $\cD$ is a tuple which consists of the following data:
  \begin{enumerate}[label=(\roman*)]
    \item Six distinct vertices $u_0,u_1,u_2,v_0,v_1,v_2\in [n]$.
    \item Integers $\frac n3-\frac n{100}\le L_0,R_0 \le \frac n3-1$.
    Integers $\frac n3-L_0\le L_1 \le \frac{n}{100}$, $\frac n3-R_0\le R_1 \le \frac{n}{100}$.
    Let $W = n-L_0-L_1-R_0-R_1$.
    \item A partition of $V=[n]$ into five subsets $V=V_{L_0}\sqcup V_{L_1}\sqcup V_{R_0}\sqcup V_{R_1} \sqcup V_W$, where $u_i\in V_{L_i}$, $v_i\in V_{R_i}$, $u_2,v_2\in V_W$, and $|V_{L_i}|=L_i$, $|V_{R_i}|=R_i$, $|V_W|=W$ ($i=1,2$).
    \item Five spanning trees: $T_{L_i}$ on $V_{L_i}$, $T_{R_i}$ on $V_{R_i}$, $T_W$ on $V_W$ ($i=1,2$).
  \end{enumerate}
  Given a tree tuple $\cD$, it produces a spanning tree
  \begin{align*}
    T(\cD) = T_{L_0} \cup T_{L_1} \cup T_{R_0} \cup T_{R_1} \cup T_W \cup \{(u_0,u_1),(u_1,u_2),(v_0,v_1),(v_1,v_2)\}.
  \end{align*}
  The measure $\bQ$ is the distribution of $T(\cD)$, where $\cD$ is uniformly chosen from all tree tuples.
\end{definition}

\begin{lemma}[Contiguity] \label{lem:conn:ust-contig}
  Let $\bP$ be the UST measure on $n$ labeled vertices.
  Let $\bQ$ be as defined in \cref{defn:conn:ust-measure-q}.
  Then $\bQ \contig \bP$.
\end{lemma}
\begin{proof}
  Let us study the support of $\bQ$.
  By \cref{lem:conn:balanced-edge-chain}, $\frac 13$-balanced edges in $T$ form a chain.
  By the construction (\cref{defn:conn:ust-measure-q}), for $T=T(\cD)$, $u_1$ and $v_1$ are the two endpoints of the chain $B_{1/3}(T)$, and $u_2$ (resp.~$v_2$) is the second vertex on the path from $u_1$ to $v_1$ (resp.~$v_1$ to $u_1$).
  Furthermore, $u_0$ (resp.~$v_0$) is the only neighbor $x$ of $u_1$ (resp.~$v_1$) not on the path between $u_1$ and $v_1$, such that $u_0$'s connected component in $T\backslash (u_0,x)$ (resp.~$v_0$'s connected component in $T\backslash (v_0,x)$) has size at least $\frac n3-\frac{n}{100}$.
  This means that $\cD$ can be reconstructed given $T(\cD)$, up to swapping $(u_0,u_1,u_2)$ and other data with $(v_0,v_1,v_2)$ and the corresponding data.
  Specifically, every $T$ in the support of $\bQ$ is realized by exactly two $\cD$s.
  Therefore, $\bQ$ is equal to the conditional measure $\bP(\cdot \mid T\in \supp \bQ)$.

  Now, to prove that $\bQ \contig \bP$, it suffices to prove that $\left| \supp \bQ \right| = \Theta(\left| \supp \bP \right|) = \Theta(n^{n-2})$.
  Because every $T$ is realized exactly twice, it suffices to prove that the number of different tree tuples is $\Theta(n^{n-2})$.
  We construct a tree tuple according to the following procedure.
  \begin{enumerate}[label=(\roman*)]
    \item Choose six distinct vertices $u_0,u_1,u_2,v_0,v_1,v_2\in [n]$.
    There are $n!/(n-6)!=\Theta(n^6)$ ways to do this.
    \item Choose $L_1$ and $R_1$ from $[1, \frac{n}{100}]$.
    Choose $L_0$ from $\left[\frac n3-L_1, \frac n3-1\right]$ and $R_0$ from $\left[\frac n3-R_1, \frac n3-1\right]$.
    \item Choose the partition $V=V_{L_0}\sqcup V_{L_1}\sqcup V_{R_0}\sqcup V_{R_1} \sqcup V_W$.
    There are
    \begin{align*}
      \binom{n-6}{L_0-1,L_1-1,R_0-1,R_1-1,W-2}
    \end{align*}
    ways to do this.
    \item Choose the five spanning trees, $T_{L_0}$, $T_{L_1}$, $T_{R_0}$, $T_{R_1}$, $T_W$.
    There are
    \begin{align*}
      L_0^{L_0-2} L_1^{L_1-2} R_0^{R_0-2} R_1^{R_1-2} W^{W-2}
    \end{align*}
    ways to do this.
  \end{enumerate}
  Summarizing the above, for large enough $n$, the number of different tree tuples is
  \begin{align*}
    &~\frac{n!}{(n-6)!} \cdot \sum_{1\le L_1,R_1 \le \frac{n}{100}} \sum_{\substack{\frac n3-L_1 \le L_0 \le \frac n3-1\\ \frac n3-R_1 \le R_0 \le \frac n3-1}} \binom{n-6}{L_0-1,L_1-1,R_0-1,R_1-1,W-2} \\
    \nonumber &~\cdot L_0^{L_0-2} L_1^{L_1-2} R_0^{R_0-2} R_1^{R_1-2} W^{W-2} \\
    \nonumber \asymp &~ \sum_{1\le L_1,R_1 \le \frac{n}{100}} \sum_{\substack{\frac n3-L_1 \le L_0 \le \frac n3-1\\ \frac n3-R_1 \le R_0 \le \frac n3-1}} \binom{n}{L_0,L_1,R_0,R_1,W} L_0^{L_0-1} L_1^{L_1-1} R_0^{R_0-1} R_1^{R_1-1} W^W \\
    \nonumber \asymp &~ \sum_{1\le L_1,R_1 \le \frac{n}{100}} \sum_{\substack{\frac n3-L_1 \le L_0 \le \frac n3-1\\ \frac n3-R_1 \le R_0 \le \frac n3-1}} \frac{n^{n+1/2}}{(L_0 L_1 R_0 R_1)^{3/2} W^{1/2}} \\
    \nonumber \asymp &~ \sum_{1\le L_1,R_1 \le \frac{n}{100}} \sum_{\substack{\frac n3-L_1 \le L_0 \le \frac n3-1\\ \frac n3-R_1 \le R_0 \le \frac n3-1}} \frac{n^{n-3}}{(L_1 R_1)^{3/2}} \\
    \nonumber \asymp &~ \sum_{1\le L_1,R_1 \le \frac{n}{100}} \frac{n^{n-3}}{(L_1 R_1)^{1/2}} \\
    \nonumber \asymp &~ n^{n-2}.
  \end{align*}
  This finishes the proof.
\end{proof}

\begin{lemma}[\cref{prop:conn:ust-structure-1}, \cref{item:prop:conn:ust-structure-1:i} for $\bQ$] \label{lem:conn:ust-structure-q-i}
  Let $T$ be sampled from $\bQ$ (\cref{defn:conn:ust-measure-q}).
  For any $\epsilon>0$, there exist $\gamma_1,\gamma_2>0$ such that with probability at least $1-\epsilon$, \cref{prop:conn:ust-structure-1}, \cref{item:prop:conn:ust-structure-1:i} holds for $T$.
\end{lemma}
\begin{proof}
  As we discussed in the proof of \cref{lem:conn:ust-contig}, all edges on the path between $u_1$ and $v_1$ are $\frac 13$-balanced.
  Conditioned on $V_W$, $T_W$ being a UST on $V_W$,
  by \cref{lem:conn:ust-dist},
  \begin{align*}
    \bP\left[ \gamma_1 \sqrt W \le \dist_{T_W}(u_1,v_1) \le \gamma_2 \sqrt W \right] \ge 1-\epsilon.
  \end{align*}
  Noting that $\dist_T(u_1,v_1)=\dist_{T_W}(u_1,v_1)$ and $W = \Theta(n)$, we finish the proof.
\end{proof}

\begin{corollary}[\cref{prop:conn:ust-structure-1}, \cref{item:prop:conn:ust-structure-1:i}] \label{coro:conn:ust-structure-i}
  Let $T$ be a UST on $n$ labeled vertices.
  There exist absolute constants $\epsilon,\gamma_1,\gamma_2>0$ such that with probability at least $\epsilon$, \cref{prop:conn:ust-structure-1}, \cref{item:prop:conn:ust-structure-1:i} holds for $T$.
\end{corollary}
\begin{proof}
  By \cref{lem:conn:ust-structure-q-i,lem:conn:ust-contig}.
\end{proof}
Using \cref{coro:conn:ust-structure-i}, we can prove that \cref{prop:conn:ust-structure-1} holds for $\bQ$.
\begin{lemma}[\cref{prop:conn:ust-structure-1} for $\bQ$] \label{lem:conn:ust-structure-q-all}
  Let $T$ be sampled from $\bQ$ (\cref{defn:conn:ust-measure-q}).
  There exist absolute constants $\epsilon,\gamma_1,\gamma_2,\gamma_3>0$ such that with probability at least $\epsilon$, both items in \cref{prop:conn:ust-structure-1} hold for $T$.
\end{lemma}
\begin{proof}
  Let $\cE_{L_0}$ (resp.~$\cE_{R_0}$, $\cE_W$) be the event that $\left| B_{1/3}(T_{L_0}) \right|$ (resp.~$\left| B_{1/3}(T_{R_0}) \right|$, $\dist_{T_W}(u_1,v_1)$) is in $[\gamma_1 \sqrt n, \gamma_2 \sqrt n]$.
  Let $\cE = \cE_{L_0} \cap \cE_{R_0} \cap \cE_W$.
  Let $\cA$ be the $\sigma$-algebra generated by $V_{L_0}$, $V_{R_0}$, $V_W$.
  Conditioned on $\cA$, the three subtrees $T_{L_0}$, $T_{R_0}$, $T_W$ are independent and are USTs on the respective vertex sets.
  By \cref{lem:conn:ust-dist}, \cref{coro:conn:ust-structure-i}, and $L_0,R_0,W=\Theta(n)$, there exist $\epsilon,\gamma_1,\gamma_2>0$ such that
  \begin{align*}
    \bP(\cE_{L_0} |\cA), \bP(\cE_{R_0} |\cA), \bP(\cE_W |\cA) \ge \epsilon^{1/3}.
  \end{align*}
  By independence, we have $\bP(\cE | \cA) \ge \epsilon$.
  Therefore $\bP(\cE) \ge \epsilon$.

  Now we show that when $\cE$ happens, both items in \cref{prop:conn:ust-structure-1} hold.
  As we discussed in the proof of \cref{lem:conn:ust-contig}, $B_{1/3}(T)$ is the set of edges on the path between $u_1$ and $v_1$.
  So \cref{item:prop:conn:ust-structure-1:i} holds because $\cE_W$ happens.
  Now consider \cref{item:prop:conn:ust-structure-1:ii}.
  Let $e$ be an edge on the path between $u_1$ and $v_1$.
  Let $T_1$, $T_2$ be the two components of $T\backslash e$, where $u_1\in T_1$, $v_1\in T_2$.
  Then $T_{L_0} \subseteq T_1$, $T_{R_0} \subseteq T_2$.
  Every $\frac 13$-balanced edge of $T_{L_0}$ divides $T_{L_0}$ into components of size at least $\frac{|T_{L_0}|}3$,
  so it is a $\beta$-balanced edge of $T_1$ with $\beta = \frac{|T_{L_0}|}{3 |T_1|} \ge \frac{\frac{n}{3} - \frac{n}{100}}{3 \cdot \frac{2}{3} n} \ge \frac 17$.
  The same discussion holds for $T_{R_0}$ and $T_2$.
  Because $\cE_{L_0}$ and $\cE_{R_0}$ happen, \cref{item:prop:conn:ust-structure-1:ii} holds.
\end{proof}


By \cref{lem:conn:ust-structure-q-all,lem:conn:ust-contig}, we complete the proof of \cref{prop:conn:ust-structure-1}.

\begin{lemma} \label{lem:conn:ust-structure-2}
  Let $T$ be a UST. Then for any $\epsilon>0$, $0<\beta<\frac 12$, there exists $\gamma>0$ such that
  \begin{align*}
    \bP\left[ \left| B_\beta(T) \right| \le \gamma \sqrt n \right] \ge 1-\epsilon.
  \end{align*}
\end{lemma}
\begin{proof}
  The proof is by the first moment method.
  Let $e=(u,v)$ be an unordered pair of vertices.
  For any integer $\beta n \le k \le (1-\beta) n$, the number of spanning trees containing $e$ such that $u$'s side contains exactly $k$ (including $u$) vertices is
  \begin{align*}
    \binom{n-2}{k-1} k^{k-2} (n-k)^{n-k-2}
    \asymp n^{-2} \binom nk k^{k-1} (n-k)^{n-k-1}
    \asymp \frac{n^{n-3/2}}{(k(n-k))^{3/2}}
    \asymp n^{n-9/2}.
  \end{align*}
  So the number of trees $T$ with $e\in B_\beta(T)$ is
  \begin{align*}
    \asymp \sum_{k\in [\beta n, (1-\beta) n]} n^{n-9/2} \asymp n^{n-7/2}.
  \end{align*}
  Therefore,
  \begin{align*}
    \bE \left[B_\beta(T)\right] \asymp \frac{\binom n2 \cdot n^{n-7/2}}{n^{n-2}} \asymp \sqrt n.
  \end{align*}
  The result then follows from Markov's inequality.
\end{proof}

\begin{lemma} \label{lem:conn:ust-structure-3}
  Let $T$ be a UST.
  For $e\in T$, let $s_T(e)$ denote the size of the smaller component of $T\backslash e$.
  Then for any $\epsilon>0$, there exists $\gamma>0$ such that
  \begin{align*}
    \bP\left[ \sum_{e\in T} s_T(e) \le \gamma n^{3/2} \right] \ge 1-\epsilon.
  \end{align*}
\end{lemma}
\begin{proof}
  The proof is by first moment method.
  Let $e=(u,v)$ be an unordered pair of vertices.
  We define $s_T(e)=0$ if $e\not \in T$.
  Then
  \begin{align*}
    \bE \left[s_T(e)\right] &\asymp \frac 1{n^{n-2}} \cdot \sum_{1\le k\le \frac n2} k \cdot \binom{n-2}{k-1} k^{k-2} (n-k)^{n-k-2} \\
    \nonumber &\asymp \frac 1{n^n} \sum_{1\le k\le \frac n2} \binom nk k^k (n-k)^{n-k-1} \\
    \nonumber &\asymp \sum_{1\le k\le \frac n2} \frac{n^{1/2}}{k^{1/2} (n-k)^{3/2}} \\
    \nonumber &\asymp \sum_{1\le k\le \frac n2} \frac{1}{k^{1/2} n} \\
    \nonumber &\asymp n^{-1/2}.
  \end{align*}
  Therefore,
  \begin{align*}
    \bE\left[ \sum_{e\in T} s_T(e)\right] \asymp \binom n2 \cdot n^{-1/2} \asymp n^{3/2}.
  \end{align*}
  The result then follows from Markov's inequality.
\end{proof}

Combining \cref{prop:conn:ust-structure-1}, \cref{lem:conn:ust-structure-2} and \cref{lem:conn:ust-structure-3}, we get the following corollary.
\begin{corollary} \label{coro:conn:ust-structure-final}
  Let $T$ be a UST on $n$ labeled vertices.
  There exist absolute constants $\epsilon,\gamma_1,\gamma_2,\gamma_3,\gamma_4,\gamma_5>0$ such that the following are true simultaneously.
  \begin{enumerate}[label=(\roman*)]
    \item\label{item:coro:conn:ust-structure-final:i}
    \begin{align*}
      \gamma_1\sqrt n \le \left| B_{1/3}(T) \right| \le \left| B_{1/42}(T) \right| \le \gamma_2 \sqrt n.
    \end{align*}
    \item\label{item:coro:conn:ust-structure-final:ii} For all $e\in B_{1/3}(T)$, if $T_1$ and $T_2$ are the two connected components of $T\backslash e$, then
    \begin{align*}
      \gamma_3 \sqrt n \le \min\{\left| B_{1/7}(T_1) \right|,\left| B_{1/7}(T_2) \right|\} \le \max\{\left| B_{1/14}(T_1) \right|,\left| B_{1/14}(T_2) \right|\} \le \gamma_4 \sqrt n.
    \end{align*}
    \item\label{item:coro:conn:ust-structure-final:iii}
    \begin{align*}
      \sum_{e\in T} s_T(e) \le \gamma_5 n^{3/2}.
    \end{align*}
  \end{enumerate}
\end{corollary}
\begin{proof}
  By \cref{prop:conn:ust-structure-1}, there exist $\epsilon,\gamma_1,\gamma_3>0$ such that with probability at least $\epsilon$, the lower bounds in \cref{coro:conn:ust-structure-final}, \cref{item:coro:conn:ust-structure-final:i,item:coro:conn:ust-structure-final:ii} hold.
  By \cref{lem:conn:ust-structure-2}, there exists $\gamma_2>0$ such that
  \begin{align*}
    \bP\left[\left| B_{1/42}(T) \right| \le \gamma_2 \sqrt n\right] \ge 1-\epsilon/3.
  \end{align*}
  By \cref{lem:conn:ust-structure-3}, there exists $\gamma_5>0$ such that
  \begin{align*}
    \bP\left[\sum_{e\in T} s_T(e) \le \gamma_5 n^{3/2} \right] \ge 1-\epsilon/3.
  \end{align*}
  By union bound, with probability at least $\epsilon/3$, \cref{coro:conn:ust-structure-final}, \cref{item:coro:conn:ust-structure-final:i,item:coro:conn:ust-structure-final:iii}, and the lower bound in \cref{coro:conn:ust-structure-final}, \cref{item:coro:conn:ust-structure-final:ii} hold.
  Now notice that $B_{1/14}(T_1) \cup B_{1/14}(T_2) \subseteq B_{1/42}(T)$, so the upper bound in \cref{coro:conn:ust-structure-final}, \cref{item:coro:conn:ust-structure-final:ii} holds with $\gamma_4=\gamma_2$.
\end{proof}

\subsection{Hard distribution and three-phase problem} \label{sec:conn:init}
We now start the proof of \cref{thm:graph-conn-hard}.
Our input distribution is defined as follows.
\begin{definition}[Hard distribution for graph connectivity] \label{defn:conn:hard-dist}
  Let $\beta_0=\frac 1{21}$.
  We generate the input graph $G$ using the following procedure.
  \begin{enumerate}[label=(\arabic*)]
    \item Let $T$ be a UST on vertex set $V=[n]$.
    \item If $B_{\beta_0}(T)=\emptyset$, return to the previous step.
    Otherwise, let $e_0\sim \Unif(B_{\beta_0}(T))$.
    \item Throw a fair coin $z\sim \Ber(1/2)$.
    If $z=1$, output $G=T$; if $z=0$, output $G=T\backslash e_0$.
  \end{enumerate}
\end{definition}

We define the following three-phase problem, where the algorithm has more power than in the noisy query model.
\begin{definition}[Three-phase problem] \label{defn:conn:three-phase}
  Let $G$ be generated from \cref{defn:conn:hard-dist}.
  Let $c_1,c_2>0$ be constants to be determined later.
  Consider the following three-phase problem.
  \begin{enumerate}[label=\arabic*.]
    \item \label{item:sec:conn:phase-1} Phase 1: The algorithm makes $m_1=c_1\log n$ noisy queries to every unordered pair of vertices $(u,v)\in \binom V2$.
    \item \label{item:sec:conn:phase-2} Phase 2: The oracle reveals some edges and non-edges of $G$. The choice of these edges and non-edges will be described later.
    \item \label{item:sec:conn:phase-3} Phase 3: The algorithm makes up to $m_2=c_2 n^2$ (adaptive) exact queries.
  \end{enumerate}
  The goal of the algorithm is to determine whether the input graph $G$ is connected.
\end{definition}

\begin{lemma} \label{lem:conn:reduction}
  If no algorithm can solve the three-phase problem (\cref{defn:conn:three-phase}) with error probability $\epsilon>0$, then no algorithm can solve the graph connectivity problem with error probability $\epsilon$ using at most $m_1 m_2 = c_1 c_2 n^2 \log n$ noisy queries.
\end{lemma}
The proof is similar to \cref{lem:inf:reduction} and omitted.



\begin{proposition}[Hardness of the three-phase problem] \label{prop:conn:three-phase-hard}
  For some choices of $c_1$, $c_2$, and Phase 2 strategy (\cref{defn:conn:three-phase}), the following is true:
  there exists $\epsilon>0$ such that no algorithm can solve the three-phase problem (\cref{defn:conn:three-phase}) with error probability $\epsilon$.
\end{proposition}

\begin{proof}[Proof of \cref{thm:graph-conn-hard}]
  Combining \cref{prop:conn:three-phase-hard} and \cref{lem:conn:reduction}.
\end{proof}

The following sections are devoted to the proof of \cref{prop:conn:three-phase-hard}.

\subsection{Phase 1} \label{sec:conn:phase-1}
In Phase 1, the algorithm makes $m_1=c_1 \log n$ noisy queries to every potential edge $e\in \binom V2$.
Let $a_e$ denote the number of times where a query to $e$ returns $1$.
Then for $e\in G$, $a_e\sim \Bin(m_1, 1-p)$; for $e\not \in G$, $a_e \sim \Bin(m_1, p)$.
For $0\le k\le m_1$, define
\begin{align*}
  p_k = \bP(\Bin(m_1, 1-p) = k) = \binom{m_1}{k} (1-p)^k p^{m_1 - k}.
\end{align*}

Let $I = \left[p m_1 - \log^{0.6}n, p m_1 + \log^{0.6}n\right]$.
\begin{lemma} \label{lem:conn:binom-concentrate}
  Let $x\sim \Bin(m_1, 1-p)$, $y\sim \Bin(m_1, p)$.
  Then
  \begin{align}
    \label{eqn:lem-conn-binom-concentrate:i} \bP(x\in I) &= n^{-c_3 \pm o(1)}, \\
    \label{eqn:lem-conn-binom-concentrate:ii} \bP(y\in I) &= 1-o(1),
  \end{align}
  where $c_3 = c_1 (1-2p) \log \frac{1-p}p$.
\end{lemma}
The proof is the same as \cref{lem:inf:binom-concentrate} and omitted.


Let $\bP^{(0)}$ denote the graph distribution in \cref{defn:conn:hard-dist}.
Let $\bP^{(1)}$ denote the posterior distribution of $G$ conditioned on observations in Phase 1.
Let $\cC^{(0)}$ (resp.~$\cC^{(1)}$) deonte the support of $\bP^{(0)}$ (resp.~$\bP^{(1)}$).
Then $\cC^{(1)} = \cC^{(0)}$ and for any graph $H \in \cC^{(0)}$, we have
\begin{align} \label{eqn:conn-phase-1:post}
  \bP^{(1)}(H) &\propto \bP\left((a_e)_{e\in \binom V2}| H\right) \bP^{(0)}(H),\\
  \nonumber &= \left(\prod_{e\in H} p_{a_e}\right)\left(\prod_{e\in H^c} p_{m_1 - a_e}\right) \bP^{(0)}(H).
\end{align}
where $H^c$ denotes the complement $\binom V2\backslash H$.

\subsection{Phase 2} \label{sec:conn:phase-2}
In Phase 2, the oracle reveals some edges and non-edges of $G$ as follows.
\begin{enumerate}[label=2\alph*.]
  \item In Step 2a, the oracle reveals potential edges $e$ with $a_e\not \in I$.
  \item In Step 2b, the oracle reveals every $e\in G$ independently with probability $q_{a_e}$. We choose $q_j = 1-\frac{p_{m_1-j} p_{j_l}}{p_j p_{m_1-j_l}}$ for $j\in I$ where $j_l = p m_1 - m_1^{0.6}$.
  \item In Step 2c, the oracle reveals $n-2$ edges of $G$ as follows. If $G$ is disconnected, reveal all edges of $G$. Otherwise, $G$ is connected and is some tree $T$. If $G$ has a $\beta_0$-balanced edge that is not revealed yet, uniformly randomly choose an edge $e^*$ from all such edges, and reveal all edges of $G\backslash e^*$. If all $\beta_0$-balanced edges of $G$ have been revealed, report failure.
\end{enumerate}







\subsubsection{Step 2a and Step 2b} \label{sec:conn:phase-2:step-ab}

By the same analysis as in \cref{sec:inf:phase-2:step-a,sec:inf:phase-2:step-b}, observations up to Step 2b have the same effect as the following procedure:
\begin{definition}[Alternative observation procedure] \label{defn:conn:alt-obs}
  Let $G$ be  generated as in \cref{defn:conn:hard-dist}.
  \begin{enumerate}[label=(\arabic*)]
    \item Observe every edge $e\in G$ independently with probability
    \begin{align*}
      p_+ &= 1-\sum_{j\in I} p_j (1-q_j)
      = 1 - \frac{p_{j_l}}{p_{m_1-j_l}} \cdot \sum_{j\in I} p_{m_1-j} \\
      \nonumber &= 1-(1\pm o(1)) \frac{p_{j_l}}{p_{m_1-j_l}} = 1-n^{-c_3\pm o(1)}.
    \end{align*}
    \item Observe every non-edge $e\in G^c$ independently with probability
    \begin{align*}
      p_- = \bP(\Bin(m_1,p)\not \in I) = o(1).
    \end{align*}
  \end{enumerate}
\end{definition}


Let $\bP^{(2b)}$ be the posterior distribution of $G$ after Step 2b and $\cC^{(2b)}$ be its support.
Let $E^{(2a)}_+$ (resp.~$E^{(2a)}_-$) denote the set of edges (resp.~non-edges) revealed in Step 2a.
Let $E^{(2b)}_+$ be the set of edges revealed in Step 2b that were not revealed in Step 2a.
Define $E^{(\le 2b)}_+ = E^{(2a)}_+ \cup E^{(2b)}_+$.
Then $\cC^{(2b)}$ is the set of graphs $H\in \cC^{(0)}$ satisfying $E^{(\le 2b)}_+ \subseteq H$ and $E^{(2a)}_-\cap H = \emptyset$.

By the same analysis as in \cref{sec:inf:phase-2:step-a,sec:inf:phase-2:step-b}, the posterior distribution satisfies
\begin{align} \label{eqn:conn-phase-2-step-b:post}
  \bP^{(2b)}(H) \propto \left(\frac{p_{k_l}}{p_{m_1-k_l}}\right)^{|H|-(n-2)} \bP^{(0)}(H).
\end{align}
for $H\in \cC^{(2b)}$.

\subsubsection{Step 2c} \label{sec:conn:phase-2:step-c}
Let $E^{(2c)}_+$ be the set of edges revealed in Step 2c that were not revealed in previous steps.
Let $\bP^{(2c)}$ be the posterior distribution of $G$ and $\cC^{(2c)}$ be the support of $\bP^{(2c)}$.

\begin{lemma} \label{lem:conn:phase-2:step-c:failure}
  Conditioned on $G$ being connected, with probability $\Omega(1)$, Step 2c does not report failure, and $e^*$ is $\frac 13$-balanced.
\end{lemma}
\begin{proof}
  Step 2c reports failure when $G$ is connected and all $\beta_0$-balanced edges have been revealed in previous steps.
  Let $T$ be a UST.
  By \cref{coro:conn:ust-structure-final}, for some $\epsilon,\gamma_1,\gamma_2>0$, we have
  \begin{align*}
    \bP\left[\gamma_1 \sqrt n \le \left| B_{1/3}(T) \right| \le \left| B_{\beta_0}(T) \right| \le \gamma_2 \sqrt n\right] > \epsilon.
  \end{align*}
  When constructing the input distribution (\cref{defn:conn:balanced-edge}), conditioned on $G$ being connected, the distribution of $G$ is uniform over all spanning trees with at least one $\beta_0$-balanced edge.
  Let $\cE_1$ be the event that $G$ is connected, has at least $\gamma_1 \sqrt n$ $\frac 13$-balanced edges, and at most $\gamma_2 \sqrt n$ $\beta_0$-balanced edges.
  By the above discussion, $\cP(\cE_1) = \Omega(1)$.
  In \cref{defn:conn:alt-obs}, every edge is observed independently with probability $p_+ = 1-n^{-c_3\pm o(1)}$.
  By choosing $c_1>0$ small enough, we can let $c_3>0$ be arbitrarily small.
  Let $\wt B_{1/3}$ (resp.~$\wt B_{\beta_0}$) be the set of $\frac 13$-balanced (resp.~$\beta_0$-balanced) edges of $G$ not revealed in previous steps.
  Then
  \begin{align*}
    \wt B_{1/3} \subseteq \wt B_{\beta_0}, \qquad
    \left|\wt B_{1/3}\right| \sim \Bin\left(\left| B_{1/3}(G) \right|, 1-p_+\right), \qquad
    \left|\wt B_{\beta_0}\right| \sim \Bin\left(\left| B_{\beta_0}(G) \right|, 1-p_+\right).
  \end{align*}
  Note that
  \begin{align*}
    \bE\left[\Bin\left(\left| B_{1/3}(G) \right|, 1-p_+\right)\right] &= \left| B_{1/3}(G) \right| \cdot (1-p_+) = n^{1/2-c_3 \pm o(1)}, \\
    \bE\left[\Bin\left(\left| B_{\beta_0}(G) \right|, 1-p_+\right)\right] &= \left| B_{\beta_0}(G) \right| \cdot (1-p_+) = n^{1/2-c_3 \pm o(1)}.
  \end{align*}
  By Bernstein's inequality, there exists $\gamma_3,\gamma_4>0$ such that conditioned on $\cE_1$, we have
  \begin{align*}
    \bP\left[ \left|\wt B_{1/3}\right| \ge \gamma_3 \left| B_{1/3}(G) \right| \cdot (1-p_+) \mid \cE_1\right] \ge 1-\exp\left(-n^{1/2-c_3 \pm o(1)}\right),\\
    \bP\left[ \left|\wt B_{\beta_0}\right| \le \gamma_4 \left| B_{\beta_0}(G) \right| \cdot (1-p_+) \mid \cE_1\right] \ge 1-\exp\left(-n^{1/2-c_3 \pm o(1)}\right).
  \end{align*}
  Let $\cE_2$ be the event that $\left|\wt B_{1/3}\right| \ge \gamma_3 \left| B_{1/3}(G) \right| \cdot (1-p_+)$ and $\left|\wt B_{\beta_0}\right| \le \gamma_4 \left| B_{\beta_0}(G) \right| \cdot (1-p_+)$.
  The above discussion shows that $\bP(\cE_2 | \cE_1) = 1-o(1)$.

  When $\cE_1$ and $\cE_2$ both happen, we have
  $
    \frac{\left|\wt B_{1/3}\right|}{\left|\wt B_{\beta_0}\right|} \ge \frac{\gamma_3 \gamma_1}{\gamma_4 \gamma_2} > 0
  $.
  So conditioned on $\cE_1\cap \cE_2$, the probability that $e^*$ is $\frac 13$-balanced is $\Omega(1)$.
  This finishes the proof.
\end{proof}

From now on we condition on the event that Step 2c does not report failure, and $e^*$ is $\frac 13$-balanced.
Let us consider the posterior distribution $\bP^{(2c)}$.
Let $E^{(\le 2c)}_+ = E^{(\le 2b)}_+ \cup E^{(2c)}_+$ be the set of observed edges at the end of Step 2c.
Then $E^{(\le 2c)}_+$ consists of $n-2$ edges, which is a forest with two components $T_1$ and $T_2$, each containing at least $\frac n3$ and at most $\frac{2n}3$ vertices.
The support $\cC^{(2c)}$ is easy to describe.
Let $G_0 = T_1 \cup T_2$ and $G_e = T_1 \cup T_2 \cup \{e\}$ for $e\in E(T_1, T_2)$.
Then
\begin{align*}
  \cC^{(2c)} = \left\{G_0\right\} \cup \left\{G_e : e\in E(T_1,T_2)\setminus E^{(2a)}_-\right\}.
\end{align*}
The posterior distribution $\bP^{(2c)}$ is not simply the distribution $\bP^{(2b)}$ restricted to $\cC^{(2c)}$.
For $H\in \cC^{(2c)}$, we have
\begin{align} \label{eqn:sec:conn:phase-2:step-c:post-step}
  \bP^{(2c)}(H) &\propto \bP\left( E^{(2c)}_+ \mid H, E^{(\le 2b)}_+\right) \bP^{(2b)}(H) \\
  \nonumber &\propto \bP\left( E^{(2c)}_+ \mid H, E^{(\le 2b)}_+\right) \left(\frac{p_{k_l}}{p_{m_1-k_l}}\right)^{|H|-(n-2)} \bP^{(0)}(H).
\end{align}
For $H=G_0$, \cref{eqn:sec:conn:phase-2:step-c:post-step} simplifies to
\begin{align} \label{eqn:sec:conn:phase-2:step-c:post-g0}
  \bP^{(2c)}(G_0) \propto \bP^{(0)}(G_0).
\end{align}
For $H=G_e$, \cref{eqn:sec:conn:phase-2:step-c:post-step} simplifies to
\begin{align} \label{eqn:sec:conn:phase-2:step-c:post-ge}
  \bP^{(2c)}(G_e) \propto \frac 1{\left| B_{\beta_0}(G_e) \backslash E^{(\le 2b)}_+ \right|} \cdot \frac{p_{k_l}}{p_{m_1-k_l}} \bP^{(0)}(G_e).
\end{align}
Note that the $\propto$ symbols in \cref{eqn:sec:conn:phase-2:step-c:post-g0,eqn:sec:conn:phase-2:step-c:post-ge} hide the same factor.

Further simplifying \cref{eqn:sec:conn:phase-2:step-c:post-g0,eqn:sec:conn:phase-2:step-c:post-ge}, we get
\begin{align}
  \label{eqn:sec:conn:phase-2:step-c:post-g0-2} \bP^{(2c)}(G_0) &\propto \sum_{e\in E(T_1,T_2)} \frac 1{\left| B_{\beta_0}(G_e) \right|}, \\
  \label{eqn:sec:conn:phase-2:step-c:post-ge-2} \bP^{(2c)}(G_e) &\propto \frac 1{\left| B_{\beta_0}(G_e) \backslash E^{(\le 2b)}_+ \right|} \cdot \frac{p_{k_l}}{p_{m_1-k_l}}, \qquad \forall e\in E(T_1,T_2) \backslash E^{(2a)}_-.
\end{align}

We now consider the set $B_{\beta_0}(G_e)$ for $e\in E(T_1,T_2)$.
Let $\beta_1 = \frac{\beta_0 n}{|T_1|}$, $\beta_2 = \frac{\beta_0 n}{|T_2|}$.
Then $\frac 1{14}\le \beta_1,\beta_2\le \frac 17$ and $B_{\beta_1}(T_1) \cup B_{\beta_2}(T_2) \subseteq B_{\beta_0}(G_e)$ for all $e\in E(T_1,T_2)$.
For $e_i\in T_i$ ($i=1,2$), let $S_{T_i}(e_i)$ be the set of vertices in the smaller component of $T_i\backslash e_i$. (If the two components have the same size, choose a side arbitrarily.)
For $e=(u_1,u_2)\in E(T_1,T_2)$ (with $u_i\in T_i$, $i=1,2$), an edge $e_i\in T_i\backslash B_{\beta_i}(T_i)$ ($i=1,2$) is in $B_{\beta_0}(G_e)$ if and only if $u_i\in S_{T_i}(e_i)$.
For $i\in \{1,2\}$ and $e=(u_1,u_2)\in E(T_1,T_2)$, define
\begin{align*}
  B'_{T_i}(e) = \{e_i : e_i\in T_i\backslash B_{\beta_i}(T_i), u_i\in S_{T_i}(e_i)\}.
\end{align*}
Then for $e\in E(T_1,T_2)$, we have
\begin{align} \label{eqn:sec:conn:phase-2:step-c:new-balanced}
  B_{\beta_0}(G_e) = \{e\} \cup B_{\beta_1}(T_1) \cup B_{\beta_2}(T_2) \cup B'_{T_1}(e) \cup B'_{T_2}(e).
\end{align}
Note that the union is a disjoint union.

\begin{lemma} \label{lem:conn:phase-2:step-c:structure-1}
  Conditioned on $G$ being connected, there exist constants $\epsilon,\gamma_1,\gamma_2,\gamma_3>0$ such that with probability at least $\epsilon$, the following are true simultaneously.
  \begin{enumerate}[label=(\roman*)]
    \item\label{item:lem:conn:phase-2:step-c:structure-1:i} Step 2c does not report failure and $e^*\in B_{1/3}(G)$.
    \item\label{item:lem:conn:phase-2:step-c:structure-1:ii}
    \begin{align*}
      \gamma_1 \sqrt n \le \left| B_{\beta_1}(T_1) \right| + \left| B_{\beta_2}(T_2) \right| \le \gamma_2 \sqrt n.
    \end{align*}
    \item\label{item:lem:conn:phase-2:step-c:structure-1:iii}
    \begin{align*}
      \sum_{e_1\in T_1\backslash B_{\beta_1}(T_1)} \left| S_{T_1}(e_1) \right| + \sum_{e_2\in T_2\backslash B_{\beta_2}(T_2)} \left| S_{T_2}(e_2) \right| \le \gamma_3 n^{3/2}.
    \end{align*}
  \end{enumerate}
\end{lemma}
\begin{proof}
  Let $T$ be a UST.
  Let $\cE_1$ be the event that all items in \cref{coro:conn:ust-structure-final} hold.
  Then conditioned on $G$ being connected, $\cE_1$ happens with probability $\Omega(1)$.
  In the following, condition on that $G$ is connected and $\cE_1$ happens.

  Let $\cE_2$ be the event that \cref{lem:conn:phase-2:step-c:structure-1}, \cref{item:lem:conn:phase-2:step-c:structure-1:i} holds.
  By the proof of \cref{lem:conn:phase-2:step-c:failure}, conditioned on $\cE_1$, $\cE_2$ happens with probability $\Omega(1)$.
  In the following, condition on that $\cE_1$ and $\cE_2$ both happen.

  By \cref{coro:conn:ust-structure-final}, \cref{item:coro:conn:ust-structure-final:ii}, and because $e^*\in B_{1/3}(T)$,
  \begin{align*}
    \left| B_{\beta_1}(T_1) \right| + \left| B_{\beta_2}(T_2) \right|&\ge \left| B_{1/7}(T_1) \right| + \left| B_{1/7}(T_2) \right| \ge \gamma_3 \sqrt n,\\
    \left| B_{\beta_1}(T_1) \right| + \left| B_{\beta_2}(T_2) \right|&\le \left| B_{1/14}(T_1) \right| + \left| B_{1/14}(T_2) \right| \le \gamma_4 \sqrt n.
  \end{align*}
  Therefore conditioned on $\cE_1$ and $\cE_2$, \cref{lem:conn:phase-2:step-c:structure-1}, \cref{item:lem:conn:phase-2:step-c:structure-1:ii} holds.

  By \cref{coro:conn:ust-structure-final}, \cref{item:coro:conn:ust-structure-final:iii}, we have
  \begin{align*}
    \sum_{e_1\in T_1\backslash B_{\beta_1}(T_1)} \left| S_{T_1}(e_1) \right| + \sum_{e_2\in T_2\backslash B_{\beta_2}(T_2)} \left| S_{T_2}(e_2) \right|
    \le \sum_{e\in T} \left| S_T(e) \right| \le \gamma_5 n^{3/2}.
  \end{align*}
  Therefore conditioned on $\cE_1$ and $\cE_2$, \cref{lem:conn:phase-2:step-c:structure-1}, \cref{item:lem:conn:phase-2:step-c:structure-1:iii} holds.
\end{proof}

\begin{corollary} \label{coro:conn:phase-2:step-c:structure-2}
  Conditioned on $G$ being connected, there exist constants $\epsilon,\gamma_1,\gamma_2,\gamma_3,\gamma_4,\gamma_5>0$ such that with probability at least $\epsilon$, the following are true simultaneously.
  \begin{enumerate}[label=(\roman*)]
    \item\label{item:coro:conn:phase-2:step-c:structure-2:i} For all $e\in E(T_1,T_2)$,
    \begin{align*}
      \left| B_{\beta_0}(G_e) \right| \ge \gamma_1 \sqrt n.
    \end{align*}
    \item\label{item:coro:conn:phase-2:step-c:structure-2:ii} For all $e\in E(T_1,T_2)$,
    \begin{align} \label{eqn:item:coro:conn:phase-2:step-c:structure-2:ii}
      \gamma_2 \left| B_{\beta_0}(G_e) \right| \cdot (1-p_+)\le \left| B_{\beta_0}(G_e) \backslash E^{(\le 2b)}_+ \right| \le \gamma_3 \left| B_{\beta_0}(G_e) \right| \cdot (1-p_+).
    \end{align}
    \item\label{item:coro:conn:phase-2:step-c:structure-2:iii}
    \begin{align*}
      \gamma_4 n^{3/2} \le \sum_{e\in E(T_1,T_2)} \frac 1{\left| B_{\beta_0}(G_e) \right|} \le \gamma_5 n^{3/2}.
    \end{align*}
  \end{enumerate}
\end{corollary}
\begin{proof}
  Let $T$ be a UST. Let $\cE$ be the event that all items in \cref{lem:conn:phase-2:step-c:structure-1} hold.
  Then conditioned $G$ being connected, $\cE$ happens with probability $\Omega(1)$.
  In the following, condition on that $G$ is connected and $\cE$ happens.

  By \cref{lem:conn:phase-2:step-c:structure-1}, \cref{item:lem:conn:phase-2:step-c:structure-1:ii} and \cref{eqn:sec:conn:phase-2:step-c:new-balanced}, we have
  \begin{align*}
    \left| B_{\beta_0}(G_e) \right| \ge \left| B_{\beta_1}(T_1) \right| + \left| B_{\beta_2}(T_2) \right| \ge \gamma_1 \sqrt n.
  \end{align*}
  So \cref{coro:conn:phase-2:step-c:structure-2}, \cref{item:coro:conn:phase-2:step-c:structure-2:i} holds.
  This implies the upper bound in \cref{coro:conn:phase-2:step-c:structure-2}, \cref{item:coro:conn:phase-2:step-c:structure-2:iii} as
  \begin{align*}
    \sum_{e\in E(T_1,T_2)} \frac 1{\left| B_{\beta_0}(G_e) \right|} \le n^2 \cdot \frac 1{\gamma_1 \sqrt n} = \gamma_1^{-1} n^{3/2}.
  \end{align*}

  By \cref{coro:conn:phase-2:step-c:structure-2}, \cref{item:coro:conn:phase-2:step-c:structure-2:i} and Bernstein's inequality, for every $e\in E(T_1,T_2)$, with probability $1-\exp\left(-n^{1/2-c_3\pm o(1)}\right)$, \cref{eqn:item:coro:conn:phase-2:step-c:structure-2:ii} holds.
  By union bound, with probability $1-o(1)$, \cref{eqn:item:coro:conn:phase-2:step-c:structure-2:ii} holds for all $e\in E(T_1,T_2)$.
  This proves \cref{coro:conn:phase-2:step-c:structure-2}, \cref{item:coro:conn:phase-2:step-c:structure-2:ii}.

  It remains to prove the lower bound in \cref{coro:conn:phase-2:step-c:structure-2}, \cref{item:coro:conn:phase-2:step-c:structure-2:iii}.
  By \cref{lem:conn:phase-2:step-c:structure-1}, \cref{item:lem:conn:phase-2:step-c:structure-1:ii,item:lem:conn:phase-2:step-c:structure-1:iii}, and \cref{eqn:sec:conn:phase-2:step-c:new-balanced}, we have
  \begin{align*}
    &~\sum_{e\in E(T_1,T_2)} \left| B_{\beta_0}(G_e) \right| \\
    \nonumber =&~ \sum_{e\in E(T_1,T_2)} \left(1 + \left| B_{\beta_1}(T_1) \right| + \left| B_{\beta_2}(T_2) \right| + \left| B'_{T_1}(e) \right| + \left| B'_{T_2}(e) \right|\right) \\
    \nonumber \le&~ \sum_{e\in E(T_1,T_2)} \left( 1 + \gamma_2 \sqrt n + \left| B'_{T_1}(e) \right| + \left| B'_{T_2}(e) \right|\right) \\
    \nonumber \le&~ (1+\gamma_2) \sqrt n |T_1||T_2| + \sum_{e_1\in T_1\backslash B_{\beta_1}(T_1)} \left| S_{T_1}(e_1) \right| \cdot |T_2|
    + \sum_{e_2\in T_2\backslash B_{\beta_2}(T_2)} \left| S_{T_2}(e_2) \right| \cdot |T_1| \\
    \nonumber \le&~ (1 + \gamma_2+\gamma_3) n^{5/2}.
  \end{align*}
  On the other hand, by Cauchy-Schwarz inequality,
  \begin{align*}
    \left( \sum_{e\in E(T_1,T_2)} \frac 1{\left| B_{\beta_0}(G_e) \right|} \right)\left( \sum_{e\in E(T_1,T_2)} \left| B_{\beta_0}(G_e) \right| \right)
    \ge (|T_1||T_2|)^2 \ge \frac 29 n^2.
  \end{align*}
  So
  \begin{align*}
    \sum_{e\in E(T_1,T_2)} \frac 1{\left| B_{\beta_0}(G_e) \right|} \ge \frac 2{9(\gamma_2+\gamma_3)} n^{3/2}.
  \end{align*}
\end{proof}

Now we are able to further utilize \cref{eqn:sec:conn:phase-2:step-c:post-g0-2} and \cref{eqn:sec:conn:phase-2:step-c:post-ge-2}.
Write
\begin{align*}
  Z^{(2c)} = \sum_{e\in E(T_1,T_2)} \frac 1{\left| B_{\beta_0}(G_e) \right|} + \sum_{e\in E(T_1,T_2) \backslash E^{(2a)}_-} \frac 1{\left| B_{\beta_0}(G_e) \backslash E^{(\le 2b)}_+ \right|} \cdot \frac{p_{k_l}}{p_{m_1-k_l}}.
\end{align*}
Conditioned on that all items in \cref{coro:conn:phase-2:step-c:structure-2} hold, for all $e\in E(T_1,T_2) \backslash E^{(2a)}_-$, we have
\begin{align*}
  \frac 1{\left|B_{\beta_0}(G_e) \backslash E^{(\le 2b)}_+\right|} \cdot \frac{p_{k_l}}{p_{m_1-k_l}}
  \asymp \frac 1{\left|B_{\beta_0}(G_e)\right|(1-p_+)} \cdot \frac{p_{k_l}}{p_{m_1-k_l}}
  \asymp\frac 1{\left|B_{\beta_0}(G_e)\right|},
\end{align*}
where the second step holds because
\begin{align*}
  \frac{p_{m_1-k_l}}{p_{k_l}}\cdot (1-p_+)
  = \frac{p_{m_1-k_l}}{p_{k_l}} \cdot \sum_{k\in I} p_k(1-q_k)
  = \sum_{k\in I} p_{m_1-k} = 1-p_- = 1-o(1).
\end{align*}
Therefore,
\begin{align*}
  Z^{(2c)} &\asymp \sum_{e\in E(T_1,T_2)} \frac 1{\left| B_{\beta_0}(G_e) \right|} \asymp n^{3/2},\\
  \bP^{(2c)}(G_0) &= \frac 1{Z^{(2c)}} \sum_{e\in E(T_1,T_2)} \frac 1{\left| B_{\beta_0}(G_e) \right|} \asymp 1.
\end{align*}

By Bernstein's inequality, for any $\delta_1>0$, we have
\begin{align*}
  \bP\left[\left| E^{(2a)}_- \right| \ge \delta_1 n^2\right] = o(1).
\end{align*}
So for small enough $\delta_1>0$, with probability $1-o(1)$, we have
\begin{align*}
  &~\sum_{e\in E(T_1,T_2) \backslash E^{(2a)}_-} \bP^{(2c)}(G_e) \\
  \nonumber =&~ \frac 1{Z^{(2c)}} \sum_{e\in E(T_1,T_2) \backslash E^{(2a)}_-} \frac 1{\left|B_{\beta_0}(G_e) \backslash E^{(\le 2b)}_+\right|} \cdot \frac{p_{k_l}}{p_{m_1-k_l}} \\
  \nonumber \asymp &~ n^{-3/2} \sum_{e\in E(T_1,T_2) \backslash E^{(2a)}_-} \frac 1{\left|B_{\beta_0}(G_e)\right|} \\
  \nonumber \asymp &~ n^{-3/2} \left(\sum_{e\in E(T_1,T_2)} \frac 1{\left|B_{\beta_0}(G_e)\right|} - \sum_{e\in E^{(2a)}_-} \frac 1{\left|B_{\beta_0}(G_e)\right|} \right) \\
  \nonumber \asymp &~ n^{-3/2} \left(n^{3/2} - \delta_1 n^2 \cdot n^{-1/2} \right) \\
  \nonumber \asymp &~ 1,
\end{align*}
where in the second-to-last step we used \cref{coro:conn:phase-2:step-c:structure-2}, \cref{item:coro:conn:phase-2:step-c:structure-2:i,item:coro:conn:phase-2:step-c:structure-2:iii}, and $\left| E^{(2a)}_- \right| \le \delta_1 n^2$.

Summarizing the above, at the end of Step 2c, with probability $\Omega(1)$, we have
\begin{align*}
  \bP^{(2c)}(\text{disconnected}) &= \bP^{(2c)}(G_0) = \Theta(1), \\
  \bP^{(2c)}(\text{connected}) &= \sum_{e\in E(T_1,T_2) \backslash E^{(2a)}_-} \bP^{(2c)}(G_e) = \Theta(1).
\end{align*}

\subsection{Phase 3} \label{sec:conn:phase-3}
In Phase 3, the algorithm makes $c_2 n^2$ adaptive exact queries.
We show that for $c_2>0$ small enough, with probability $\Omega(1)$, the algorithm will not be able to return the correct answer.

Let $E^{(3)}$ be the set of edges queried in Phase 3.
We can w.l.o.g.~assume that $E^{(3)} \subseteq E(T_1,T_2) \backslash E^{(2a)}_-$, because only queries in this set are useful.
Conditioned on $G$ being connected, the probability that $E^{(3)}$ hits the edge $e^*$ is
\begin{align*}
  \frac{\sum_{e\in E^{(3)}} \bP^{(2c)}(G_e)}{\sum_{e\in E(T_1,T_2) \backslash E^{(2a)}_-} \bP^{(2c)}(G_e)}
  \le c_2 n^2 \cdot \Theta(n^{-2}) \asymp c_2,
\end{align*}
which is $1-\Omega(1)$ for $c_2>0$ small enough.
Therefore, for small enough $c_2$, with probability $\Omega(1)$, $E^{(3)}$ does not hit the edge $e^*$.

Conditioned on $E^{(3)}$ does not hit $e^*$, let $\bP^{(3)}$ denote the posterior distribution of the original graph $G$ given all observations.
We have
\begin{align*}
  \bP^{(3)}(\text{disconnected}) &= \bP^{(3)}(G_0),\\
  \bP^{(3)}(\text{connected}) &= \sum_{e\in E(T_1,T_2) \backslash \left(E^{(2a)}_-\cup E^{(3)}\right)} \bP^{(3)}(G_e),
\end{align*}
where
\begin{align*}
  \bP^{(2c)}(G_0) &=\frac 1{Z^{(3)}} \sum_{e\in E(T_1,T_2)} \frac 1{\left| B_{\beta_0}(G_e) \right|}, \\
  \bP^{(2c)}(G_e) &=\frac 1{Z^{(3)}} \frac 1{\left| B_{\beta_0}(G_e) \backslash E^{(\le 2b)}_+ \right|} \cdot \frac{p_{k_l}}{p_{m_1-k_l}},\\
  Z^{(3)} &= \sum_{e\in E(T_1,T_2)} \frac 1{\left| B_{\beta_0}(G_e) \right|} + \sum_{e\in E(T_1,T_2) \backslash \left(E^{(2a)}_-\cup E^{(3)}\right)} \frac 1{\left| B_{\beta_0}(G_e) \backslash E^{(\le 2b)}_+ \right|} \cdot \frac{p_{k_l}}{p_{m_1-k_l}}.
\end{align*}

By the same discussion as in the end of Phase 2, Step 2c, for $c_2>0$ small enough, with probability $\Omega(1)$, we have
\begin{align*}
  \bP^{(3)}(\text{disconnected}) &= \Theta(1), \\
  \bP^{(3)}(\text{connected}) &= \Theta(1).
\end{align*}
In this case, any return value would lead to an error probability of $\Omega(1)$.

This concludes the proof of \cref{prop:conn:three-phase-hard}.

\subsection{\texorpdfstring{$s$-$t$}{s-t} Connectivity} \label{sec:conn:s-t}
In this section we modify the proof of \cref{thm:graph-conn-hard} to show hardness of $s$-$t$ Connectivity.
Recall the $s$-$t$ Connectivity problem, where the input is an unknown undirected graph on $n$ labeled vertices, and a pair of vertices $s,t \in V$. An algorithm can make noisy queries to edge membership and the goal is to determine whether $s$ and $t$ are in the same connected component of $G$.

\begin{proposition}[Hardness of $s$-$t$ Connectivity] \label{prop:s-t-conn-hard}
  Any algorithm that solves the $s$-$t$ Connectivity problem with $\frac 13$ error probability uses $\Omega(n^2\log n)$ noisy queries in expectation.
\end{proposition}

\begin{proof}
  As discussed in \cref{sec:conn:overview}, the error probability in the proposition statement can be replaced with any $0<\epsilon<\frac 12$, and the expected number of queries can be replaced with worst-case number of queries.

  We design an input distribution for $s$-$t$ Connectivity by generating $G$ from \cref{defn:conn:hard-dist}, and choosing $s,t$ i.i.d.~$\sim \Unif(V)$.

  Then we run the same proof as \cref{thm:graph-conn-hard}.
  That is, we define a three-phase problem for $s$-$t$ Connectivity, where the oracle uses the same strategy in Phase 2 as in Graph Connectivity.
  Because $s$ and $t$ are independent with $G$, in the end of Phase 2, conditioned on Step 2c does not report failure, with probability $\Omega(1)$, $s\in T_1$ and $t\in T_2$.
  In this case, $s$-$t$ Connectivity is equivalent to Graph Connectivity.
  In the proof of Graph Connectivity, we have shown that with probability $\Omega(1)$ (over the randomness of the graph and Phase 1 and 2), any algorithm that uses at most $c_2 n^2$ queries in Phase 3 has $\Omega(1)$ error probability.
  This implies that the same holds for $s$-$t$ Connectivity.
\end{proof}
