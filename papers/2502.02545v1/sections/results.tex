In order to characterize the weak recovery of the proposed spectral methods, we define two message passing schemes tailored to respectively have eigenvectors of $\tens{L}$ and $\tens{T}$ as fixed points. Similarly to these previous works, we leverage the \textit{state evolution} associated to the algorithms in order to quantify the alignment between the spectral estimators and the weight matrix $\mat{W}_\star$, tracking the overlap matrices
\begin{equation}\label{eq:def:overlaps_amp}
    \mat{M}^t \coloneqq \frac{1}{d}\left(\matwhat^t\right)^T\bW_\star,\quad\mat{Q}^t\coloneqq \frac{1}{d}\left(\matwhat^t\right)^T\matwhat^t,
\end{equation}
and their value $\bM$, $\bQ$, at convergence. The state evolution equations for generic linear GAMP algorithms are presented in Appendix \ref{app:sec:SE_LGAMP}.
\subsection{Asymmetric spectral method}
\begin{definition}\label{def:sp_GAMP_asymm}
Consider the linear GAMP algorithm (\ref{app:def:eq:GAMP_Omega},\ref{app:def:eq:GAMP_B}) with denoiser functions
\begin{equation}
    \gout^t(\bomega,y) =\dgout(y)\bomega\coloneqq\left(\Cov[\rdmvect{z}\big|y]-\bI\right)\bomega,\quad\boldf_{in}^t(\vect{b}) = \gamma^{-1}\vect{b},
\end{equation}
with $\gamma > 0$ a parameter to be fixed. Its iterations read
\begin{align}\label{eq:def:sp_GAMP_Omega}
    &\bOmega^{t} =\rdmmat{X}\matwhat^t - \gamma^{-1}\tens{G}\cdot\bOmega^{t-1},\\
    &\matwhat^{t+1} = \gamma^{-1}\rdmmat{X}^T\tens{G}\cdot\bOmega^t,\label{eq:est:gamp}
\end{align}
\end{definition}

\begin{figure*}[t]
    \centering
    \includegraphics[width=0.99\linewidth]{plots/z1z2_n1e4_alpha04_d1e4_alpha1_eigenvalues_light.pdf}
    \caption{Distribution of the eigenvalues (dots) $\lambda\in\mathbb C$ of $\tens{L}$ at finite $n = 10^4$, for $g(z_1,z_2) = z_1z_2$,  $\alpha_c \approx 0.59375$. (\textbf{Left}) $\alpha = 0.4 < \alpha_c$. (\textbf{Right}) $\alpha = 1 > \alpha_c$. The dashed blue circle has radius equal to $\sqrt{\nicefrac{\alpha}{\alpha_c}}$, {\it i.e.} the value $\gamma_b$ predicted in Lemma \ref{result:2}. The dashed orange vertical line corresponds to $\operatorname{Re}\lambda = \nicefrac{\alpha}{\alpha_c}$, the eigenvalue $\gamma_s$ defined in Lemma \ref{result:1}. As predicted by the state evolution equations for this problem, two significant eigenvalues (highlighted in orange) are observed near this vertical line.}
    \label{fig:eigenvalues_product_p2}
\end{figure*}

When $\gamma$ is chosen as an eigenvalue of $\tens{L}$, the correspondent eigenvector is a fixed point of eq. (\ref{eq:def:sp_GAMP_Omega}). In the high-dimensional limit, the asymptotic overlap of the estimator $\matwhat$ in \eqref{eq:est:gamp} can be tracked thanks to the state evolution equations, which follow from an immediate application of the general result of \cite{javanmard2013state} to the GAMP algorithm \eqref{def:sp_GAMP}.
\begin{proposition}[State evolution  \cite{javanmard2013state}] 
Let $\mat{M}^{t}$ and $\mat{Q}^{t}$ denote the overlaps defined in eq.~\eqref{eq:def:overlaps_amp} for the iterative algorithm \eqref{def:sp_GAMP_asymm}. Then, in the proportional high-dimensional limit $n,d\to\infty$ at fixed $\alpha = \nicefrac{n}{d}$, they satisfy the following \emph{state evolution}: 
\label{def:sp_SE}
\begin{align}
\mat{M}^{t+1} &= \alpha\gamma^{-1}\cF(\mat{M}^t)\\
    \mat{Q}^{t+1} &= \mat{M}^t(\mat{M}^t)^T + \alpha\gamma^{-2}  \left(\cG(\mat{M}^t) + \cF(\mat{Q}^{t}) \right) \label{app:eq:se_sp_gamp_Q}
\end{align}
where
\begin{align}\label{app:eq:def:operator_F}\cF(\mat{M}) &\coloneqq \E_{\rdm{y}}[\dgout(\rdm{y})\mat{M} \dgout(\rdm{y})],\\
\cG(\mat{M}) &\coloneqq \E_{\rdm{y}}[\dgout(\rdm{y})\mat{M} \dgout(\rdm{y})\mat{M}^T \dgout(\rdm{y})].
\end{align}
\end{proposition}
With the state evolution equations in hand, one can derive a sharp characterization of the asymptotic weak recovery threshold in terms of the spectral properties of the estimator by a linear stability argument \cite{strogatz2001nonlinear}
\begin{lemma}
    \label{result:1}
    For $\alpha>\alpha_c$,  $\gamma_s = \nicefrac{\alpha}{\alpha_c}$ is the largest value of $\gamma$ such that the state evolution \ref{def:sp_SE} has a stable fixed point ($\mat{M},\mat{Q}$) with
    \begin{equation}
        \mat{M}\neq\bzero,\quad\mat{Q} \in \mathbb{S}_+^p \setminus \{\bzero\}.
    \end{equation}
   Additionally, $\mat{M}\in\mathbb{S}_+^p \setminus \{\bzero\}$.
\end{lemma}
\begin{lemma}
    \label{result:2}
For all $\alpha\in\R$,  $\gamma_b = \sqrt{\nicefrac{\alpha}{\alpha_c}}$ is the largest value for $\gamma$ such that the state evolution \ref{def:sp_SE} has a fixed point ($\mat{M},\mat{Q}$) with
    \begin{equation}
        \mat{M} = \bzero,\quad\mat{Q} \in \mathbb{S}_+^p \setminus \{\bzero\}.
    \end{equation}
    The fixed point is stable for $\alpha<\alpha_c$ and unstable otherwise.
\end{lemma}
The derivation of the Lemmas is outlined in Appendix \ref{app:derivation_asymmetric}.
This two results indicate a change of behavior of the operator norm of $\tens{L}$ at the critical value $\alpha_c$, leading to the following conjecture. 
\begin{conjecture}
\label{conjecture:1}
     In the high-dimensional limit $n,d\to\infty$, $\nicefrac{n}{d}\to\alpha$, the empirical spectral distribution associated to the $pn$ eigenvalues of $\tens{L}$ converges weakly almost surely to a density whose support is strictly contained in a disk of radius $\gamma_b = \sqrt{\nicefrac{\alpha}{\alpha_c}}$ centered at the origin. 
     Moreover
     \begin{itemize}
         \item  for $\alpha<\alpha_c$, $\normop{\tens{L}} \xrightarrow[n \to \infty]{a.s.} \gamma_b$ and the associated eigenvector is not correlated with $\matw$;
         \item for $\alpha>\alpha_c$, $\normop{\tens{L}} \xrightarrow[n \to \infty]{a.s.} \gamma_s > \gamma_b$ and the associated eigenvector defined in (\ref{eq:def:spectral_asymmetric}), weakly recovers the signal. 
     \end{itemize}
\end{conjecture}


This conjecture, motivated by the results (\ref{result:1}-\ref{result:2}) is perfectly supported by extensive simulations, as illustrated in Fig.~\ref{fig:eigenvalues_product_p2}, \ref{fig:eigenvalues_vs_alpha_z1z2}. A rigorous proof requires, however, a finer control of the spectral norm of these operators.

\subsection{Symmetric spectral method}
In order to simplify the notation, define the symmetric $p \times p$ matrix
\begin{equation}\label{eq:def:preprocessing}
    \bcT(y) \coloneqq \bI - \Cov^{-1}[\rdmvect{z}\big|y].
\end{equation}
\begin{definition}\label{def:sp_GAMP} 
Consider the linear GAMP algorithm (\ref{app:def:eq:GAMP_Omega},\ref{app:def:eq:GAMP_B}) with denoiser functions
\begin{align}
    \gout^t(\bomega,y) &= \bcT(y)\bV_t\left(a_t\bI - \bV_t\bcT(\rdm{y})\bV_t\right)^{-1}\bomega,\\
    \boldf_{in}^{t+1}(\vect{b}) &= \bV_{t+1}\vect{b},
\end{align}
with
\begin{equation}\label{eq:symmetric_se_V}
    \bV_{t+1} \!=\! \left(\gamma_t\bV_t -\alpha\E_{\rdm{y}\sim\Zout}\left[\bcT(\rdm{y})\bV_t\left(a_t\bI - \bV_t\bcT(\rdm{y})\bV_t\right)^{-1}\right]\right)^{-1},
\end{equation}
$a_t$ a parameter to be fixed  and $\gamma_t$ chosen, a posteriori, such that $\normop{\bV_{t+1}} = 1$.
Note that $\bV_t$ is symmetric at all times given a symmetric initialization $\bV_0$.
\end{definition}
In Appendix \ref{app:derivation_simmetric} we show that, for properly chosen $a_t$, the fixed point for the iterate $\matwhat^t\bV_t$ is the eigenvector of $\tens{T}$ with eigenvalue $\lim_{t\to\infty}a_t\gamma_t$. The denoiser functions chosen for this GAMP algorithm are derived as a generalization of the ones in \cite{zhang24c}, where a similar approach has been used to characterize the recovery properties of spectral algorithms for structured single-index models. \\  
\begin{proposition}[State evolution  \cite{javanmard2013state,mondelli18a}]\label{result:se_symm}
Let $\mat{M}^{t}$ and $\mat{Q}^{t}$ denote the overlaps defined in eq.~\eqref{eq:def:overlaps_amp} for the iterative algorithm \eqref{def:sp_GAMP}. Then, in the proportional high-dimensional limit $n,d\to\infty$ at fixed $\alpha = \nicefrac{n}{d}$, they satisfy the following \emph{state evolution} equations:
\begin{equation}
\begin{aligned}
    \mat{M}^{t+1} &= \alpha\cF(\bM^t;a_t,\bV_t),\\
    \mat{Q}^{t+1} &=\bM^t(\bM^t)^T + \alpha (\cG(\bM^t;a_t,\bV_t) + \Tilde{\cF}(\bQ^t;a_t,\bV_t)),
\end{aligned}
\end{equation}
where we have defined the symmetric operator
\begin{equation}
\begin{aligned}
    &\cF(\bM; a_t, \bV_t) := &\E_{\rdm{y}\sim\Zout}\left[\bV_t\bcT(\rdm{y})\bV_t\left(a_t - \bV_t\bcT(\rdm{y})\bV_t\right)^{-1}\mat{M}(\Cov[\rdmvect{z}\big|\rdm{y}]-\bI)\right],
\end{aligned}
\end{equation}
while $\tilde{\cF}$ and $\cG$ are given in eq. (\ref{app:eq:def:tilde_cF},\ref{app:eq:def:G_operator}).
\end{proposition}
The complete set of state evolutions equation is displayed in Appendix \ref{app:symmetric_state_evolution}. For $a = 1$, the fixed point for eq. (\ref{eq:symmetric_se_V}) is $\bV = \bI$ and the operators $\cF(\bM; 1, \bI)$ and $\tilde{\cF}(\bM;1,\bI)$ coincide with $\cF(\bM)$ defined in (\ref{eq:def:operator_F}), having the largest eigenvalue correspondent to $\alpha_c^{-1}$. As $a\to\infty$, $\bV\to\bI$ and $\cF(\;\cdot\;;a,\bV(a))\to\bzero$. Therefore, since the operator depends continuously on $a\in[1,\infty)$, there exists a continuous function $\nu_1^{\cF}(a)$ for the largest eigenvalue of the operator as a function of the parameter $a$, with $\nu_1^{\cF}(1) = \alpha_c^{-1}$ and $\nu_1^{\cF}(a\to\infty)\to0$. Hence, for all $\alpha > \alpha_c$, there exists $a>1$ such that $\nu_1^{\cF}(a) = \alpha^{-1}$. A similar argument applies to $\tilde{\cF}$ and its largest eigenvalue $\nu_1^{\tilde{\cF}}(a)$.
\begin{lemma} \label{result:3}For $\alpha > \alpha_c$, consider $a_t=a>1$ solution of $\nu_1^{\cF}(a_t) = \alpha^{-1}$ and $\gamma_t$ such that $\normop{\bV_{t+1}} = 1$,
with $\bV_{t+1}$ given by eq. (\ref{eq:symmetric_se_V}). Then, the state evolution has a stable fixed point $(\bM,\bQ)$ with
\begin{equation}
    \bM \neq \bzero,\quad\bQ\in\mathbb{S}_+^p \setminus \{\bzero\},
\end{equation}
correspondent to the eigenvalue $\lambda_s = a\gamma$.
\end{lemma}
\begin{lemma} \label{result:4}For $\alpha \geq \alpha_c$, $a_t = a\geq1$ solution to $\nu_1^{\tilde{\cF}}(a) = \alpha^{-1}$, $\gamma_t$ such that $\normop{\bV_{t+1}}=1$, with $\bV_{t+1}$ given by eq. (\ref{eq:symmetric_se_V}), the state evolution has an unstable fixed point $(\bM,\bQ)$ with
\begin{equation}
    \bM = \bzero,\quad\bQ\in\mathbb{S}_+^p \setminus \{\bzero\},
\end{equation}
correspondent to the eigenvalue $\lambda_b = a\gamma$.
\end{lemma}
The derivation of the Lemmas is outlined in Appendix \ref{app:derivation_simmetric}.
\begin{conjecture}
\label{conjecure:2}
    In the high-dimensional limit $n,d\to\infty$, $\nicefrac{n}{d}\to\alpha$, for $\alpha>\alpha_c$, the largest eigenvalue of $\tens{T}$ converges to $\lambda_s$, defined in Lemma \ref{result:3}. In this regime, the symmetric spectral method, defined in (\ref{eq:def:spectral_symmetric}), weakly recovers the signal. Moreover, the empirical spectral distribution of the $pd$ eigenvalues of $\tens{T}$ converges weakly almost surely to a density upper bounded by $\lambda_b<\lambda_s$ defined in Lemma \ref{result:4}.
\end{conjecture}
By construction, in multi-index models (\( p > 1 \)), the tensor \( \tens{T} \), or equivalently its flattened matrix representation \( \mat{flat}(\tens{T}) \), exhibits a highly structured form due to the presence of \emph{repeated} entries from the measurement matrix \( \rdmmat{X} \). This intrinsic redundancy complicates its analysis using standard random matrix theory tools. Conjecture \ref{conjecure:2} based on results (\ref{result:3}-\ref{result:4}) and further supported by numerical simulations (see Fig.~\ref{fig:eigenvalues_vs_alpha_z1z2}, \ref{fig:eigenvalues_ditribution_TAP}), offers a novel framework for understanding the spectral properties of such matrices. Moreover, for any link function such that $\Cov[\rdmvect{z}\big|y]$ admits a common basis for all $y$, with real eigenvalues $\{\lambda_k(y)+1\}_{k=1}^p$, the analysis of spectrum of $\mat{flat}(\tens{T})$ can be simplified following the arguments in Appendix \ref{app:examples_joinlty_diag}. Indeed, the symmetric spectral method reduces to the diagonalization of $p$ matrices \begin{equation}
\sum_{i\in\integset{n}}\frac{\lambda_k(\rdm{y}_i)}{\lambda_k(\rdm{y}_i)+1}\rdmvect{x}_i\rdmvect{x}_i^T\in\R^{d\times d}, \quad k\in\integset{p}.\end{equation} 
Their structure allows the use of the techniques in \cite{mondelli18a,Lu2019} to analyze the spectrum, and could support the formalization of the results in Conjecture \ref{conjecure:2} for this subset of problems.
\begin{figure}[h]
    \centering
    \vspace{-0.2cm} 
\includegraphics[width=0.5\linewidth]{plots/final_cropped_z1z2_eigenvalues_vs_alpha_5000.pdf}
    \caption{Largest eigenvalues (magnitude, if complex) of the tensors $\tens{L}$ (triangles), $n = 5000$, and $\tens{T}$ (circles), $d = 5000$, for the function $g(z_1,z_2) = z_1z_2$, versus the sample complexity $\alpha$.     The orange and blue lines, respectively represent the values of the largest eigenvector $\nicefrac{\alpha}{\alpha_c}$ and the  edge of the bulk $\sqrt{\nicefrac{\alpha}{\alpha_c}}$ in Conj. \ref{conjecture:1} for the asymmetric spectral method. The green and purple line correspond to the values of the largest eigenvalue $\lambda_s$ and the edge of the bulk $\lambda_b$ in Conj. \ref{conjecure:2} for the symmetric spectral method. 
    \label{fig:eigenvalues_vs_alpha_z1z2}}
        \vspace{-0.2cm} 
\end{figure}

\begin{figure*}[t]
    \centering
    \includegraphics[width=0.99\linewidth]{plots/squares_p4_eig_distr_TAP_d1e4_three_light.pdf}
    \caption{Distribution of the eigenvalues of $\tens{T}$, $d = 10^4$, for the link function $g(\bz)=p^{-1}\norm{\bz}^2$, $p=4$. The critical threshold in $\alpha_c = 2$. The distribution is truncated on the left. (\textbf{Left}) $\alpha = 1 < \alpha_c$. (\textbf{Center}) $\alpha = \alpha_c$.  (\textbf{Right}) $\alpha = 6 > \alpha_c$. As predicted by the state evolution framework, in this regime we observe four eigenvalues (highlighted in green) separated from the main bulk, centered around $\lambda_s$ (green vertical line) obtained in Lemma \ref{result:3}. The vertical purple line correspond to the value $\lambda_b$ provided in Lemma \ref{result:4} as a bound for the bulk.}
    \label{fig:eigenvalues_ditribution_TAP}
\end{figure*}
\subsection{Relation between the two spectral methods}
 As in the single-index setting, the tensors $\tens{L}$ and $\tens{T}$ are related by the following Proposition (proven in App. \ref{app:proof_proposition}).
\begin{proposition}
\label{proposition:maillard}
~
\begin{enumerate}[leftmargin=2em,wide=1pt]
    \item Given an eigenpair $\gamma_\tens{L} \geq 1, \bOmega$ of $\tens{L}$, and defining 
    $
        \bW := \bX^T\tens{G}\cdot\bOmega
   $,    then:
\begin{equation}       \tens{T}_{\gamma_\tens{L}}\cdot\bW\coloneqq  \rdmmat{X}^T\tens{G}\cdot\left((\tens{G}+\gamma_\tens{L}\tens{I})^{-1}\cdot(\rdmmat{X}\mat{W})\right)= \bW.
    \end{equation}
    \item Given an eigenpair $\gamma_\tens{T}, \bW$ of $\tens{T}$ and defining
    \begin{equation}
    \label{eq:mat_omega}
        \bOmega := (\tens{I} + \tens{G})^{-1}\cdot(\bX\bW),
    \end{equation}
    then:
    \begin{equation}
        \tens{L}\cdot\bOmega = \gamma_\tens{T}\bOmega + (\gamma_\tens{T} - 1)\tens{G}\cdot\bOmega.
    \end{equation}
\end{enumerate}
Consequently, if there exists an eigenvector $\bW$ of $\tens{T}$ with eigenvalue $\gamma_\tens{T} = 1$, then $\bOmega$ in Eq.~\eqref{eq:mat_omega} is an eigenvector of $\tens{L}$ with eigenvalue $\gamma_{\tens{L}} = \gamma_{\tens{T}} = 1$.
\end{proposition}