\section{Related work}
Prior work has developed methods for adapting LiDAR sensing to external environmental factors. In \cite{b3, b4, b5, b6}, the point cloud density is increased in regions of interest (RoI), with the RoI typically determined through external camera systems. For example, in \cite{b3}, moving targets are identified within the RoI by co-locating the LiDAR with a camera that shares the same field of view. In \cite{b4}, the RoI is based on probable pedestrian coordinates identified from sparse LiDAR scans. In \cite{b5}, regions of high activity are detected using an event camera to define the RoI. Additionally, \cite{b6} optimized both sampling coordinates and the inpainting algorithm for images captured by a co-located camera. More recently, \cite{b7} proposed a system that adapts both the range and resolution of the LiDAR over the field of view. However, the adaptations in \cite{b7} are performed using offline location-based static topology maps, without considering real-time environmental conditions.

In parallel with the progress in adaptive LiDAR systems, intelligent driver monitoring systems (DMS) have been developed for ADAS to ensure the attentiveness of the human driver (who is fully or partially in responsible for the driving task). DMS often center around detecting the gaze direction of the driver. For instance, prior work has leveraged driver’s gaze within distraction warning systems \cite{b8, b9, b10, b11}. These systems monitor the driver’s gaze in real-time to issue alerts when the driver is deemed distracted. 

Overall, both adaptive LiDARs and driver monitoring systems are being actively used in advanced driver support systems. However, there is a lack of research exploring the potential for combining LiDAR with DMS. In particular, existing adaptive LiDAR systems for ADAS do not take into account information on human driver's gaze, which leads to LiDAR perceiving often the same objects as the human. This duplication might provide for better reliability of the joint driver-vehicle perception but risks missing out on improvements of accuracy of LiDAR sensing in the areas that are not  perceived by the driver. At the same time, DMS can readily provide the LiDAR with the information on regions currently perceived by the driver, such that the LiDAR can complement the driver's sensing instead of duplicating it.