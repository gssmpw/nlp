\documentclass[conference, final, letterpaper, 10pt]{ieeeconf}
\IEEEoverridecommandlockouts
% The preceding line is only needed to identify funding in the first footnote. If that is unneeded, please comment it out.
\overrideIEEEmargins
%\usepackage{cite}
\usepackage{amsmath,amssymb,amsfonts}
\usepackage{algorithmic}
\usepackage{graphicx}
\usepackage{textcomp}
\usepackage{xcolor}
\usepackage{booktabs}
\usepackage{cite}
\usepackage{jabbrv}
\usepackage{url}

\def\BibTeX{{\rm B\kern-.05em{\sc i\kern-.025em b}\kern-.08em
    T\kern-.1667em\lower.7ex\hbox{E}\kern-.125emX}}

\title{Adapting LiDAR sensing to driver gaze \\for enhanced automotive perception}

\author{Federico Scarì$^1$, Nitin Jonathan Myers$^2$, Chen Quan$^2$, Arkady Zgonnikov$^1$
\thanks{$^1$Department of Cognitive Robotics, $^2$Delft Center for Systems and Control,
        TU Delft, 2628 CB Delft, The Netherlands
        {\tt\small \{F.Scari@, N.J.Myers@, C.Quan@, A.Zgonnikov@\}tudelft.nl}}%
}

\begin{document}

\maketitle

\begin{abstract}
Autonomous vehicles rely on accurate trajectory prediction to inform decision-making processes related to navigation and collision avoidance. However, current trajectory prediction models show signs of overfitting, which may lead to unsafe or suboptimal behavior. To address these challenges, this paper presents a comprehensive framework that categorizes and assesses the definitions and strategies used in the literature on evaluating and improving the robustness of trajectory prediction models. This involves a detailed exploration of various approaches, including data slicing methods, perturbation techniques, model architecture changes, and post-training adjustments. In the literature, we see many promising methods for increasing robustness, which are necessary for safe and reliable autonomous driving.

\end{abstract}

% \begin{IEEEkeywords}
% Autonomous driving, trajectory prediction, robustness
% \end{IEEEkeywords}

\section{Introduction} \label{Introduction}

\end{document}