\pdfoutput=1
%% This is file `sample-acmsmall-conf.tex',
%% generated with the docstrip utility.
%%
%% The original source files were:
%%
%% samples.dtx  (with options: `all,proceedings,bibtex,acmsmall-conf')
%% 
%% IMPORTANT NOTICE:
%% 
%% For the copyright see the source file.
%% 
%% Any modified versions of this file must be renamed
%% with new filenames distinct from sample-acmsmall-conf.tex.
%% 
%% For distribution of the original source see the terms
%% for copying and modification in the file samples.dtx.
%% 
%% This generated file may be distributed as long as the
%% original source files, as listed above, are part of the
%% same distribution. (The sources need not necessarily be
%% in the same archive or directory.)
%%
%%
%% Commands for TeXCount
%TC:macro \cite [option:text,text]
%TC:macro \citep [option:text,text]
%TC:macro \citet [option:text,text]
%TC:envir table 0 1
%TC:envir table* 0 1
%TC:envir tabular [ignore] word
%TC:envir displaymath 0 word
%TC:envir math 0 word
%TC:envir comment 0 0
%%
%%
%% The first command in your LaTeX source must be the \documentclass
%% command.
%%
%% For submission and review of your manuscript please change the
%% command to \documentclass[manuscript, screen, review]{acmart}.
%%
%% When submitting camera ready or to TAPS, please change the command
%% to \documentclass[sigconf]{acmart} or whichever template is required
%% for your publication.
%%
%%
\documentclass[acmsmall]{acmart}

%%
%% \BibTeX command to typeset BibTeX logo in the docs
\AtBeginDocument{%
  \providecommand\BibTeX{{%
    Bib\TeX}}}

%% Rights management information.  This information is sent to you
%% when you complete the rights form.  These commands have SAMPLE
%% values in them; it is your responsibility as an author to replace
%% the commands and values with those provided to you when you
%% complete the rights form.

%% These commands are for a PROCEEDINGS abstract or paper.
%%
%%  Uncomment \acmBooktitle if the title of the proceedings is different
%%  from ``Proceedings of ...''!
%%
%%\acmBooktitle{Woodstock '18: ACM Symposium on Neural Gaze Detection,
%%  June 03--05, 2018, Woodstock, NY}

\setcopyright{cc}
\setcctype{by}
\acmDOI{10.1145/3715727}
\acmYear{2025}
\acmJournal{PACMSE}
\acmVolume{2}
\acmNumber{FSE}
\acmArticle{FSE012}
\acmMonth{7}
\received{2024-09-13}
\received[accepted]{2025-01-14}


%%
%% Submission ID.
%% Use this when submitting an article to a sponsored event. You'll
%% receive a unique submission ID from the organizers
%% of the event, and this ID should be used as the parameter to this command.
%%\acmSubmissionID{123-A56-BU3}

%%
%% For managing citations, it is recommended to use bibliography
%% files in BibTeX format.
%%
%% You can then either use BibTeX with the ACM-Reference-Format style,
%% or BibLaTeX with the acmnumeric or acmauthoryear sytles, that include
%% support for advanced citation of software artefact from the
%% biblatex-software package, also separately available on CTAN.
%%
%% Look at the sample-*-biblatex.tex files for templates showcasing
%% the biblatex styles.
%%

%%
%% The majority of ACM publications use numbered citations and
%% references.  The command \citestyle{authoryear} switches to the
%% "author year" style.
%%
%% If you are preparing content for an event
%% sponsored by ACM SIGGRAPH, you must use the "author year" style of
%% citations and references.
%% Uncommenting
%% the next command will enable that style.
%%\citestyle{acmauthoryear}

\newcommand{\kwcost}[1]{\mathbf{cost}\left(  {#1} \right)}
\newcommand{\circuitcon}[2]{{#1} + {#2}}

\newcommand{\bigomega}{\mathbf{\Omega}}

%% Semantics frame
\newcommand{\sfbox}[1]
{
\cfbox{blue}{#1}
}
\newcommand{\srule}{\vspace{2mm}\rule{\columnwidth}{1pt}\vspace{2mm}}

\newcommand{\lang}{\textsc{Laqe}}
%% General syntax
\newcommand{\dom}[1]{\mathop{\text{dom}}(#1)}
\newcommand{\codom}[1]{\mathop{\text{cod}}(#1)}
\newcommand{\kw}[1]{\mbox{\ttt{#1}}}
\newcommand{\cdparens}[1]{({#1})}
\newcommand{\cd}[1]{{\lstinline!#1!}}
\newcommand{\hmm}{\textsf{HMM}}
\newcommand{\rulename}[1]{\textsc{#1}}
\newcommand{\ruleref}[1]{Rule~\rulename{#1}}

\newcommand{\true}{\ensuremath{\kw{true}}}
\newcommand{\false}{\ensuremath{\kw{false}}}
\newcommand{\ttrue}{\kw{t}}
\newcommand{\ffalse}{\kw{f}}
\newcommand{\prog}{\ensuremath{P}}
\newcommand{\pred}{\ensuremath{\mathcal{P}}}
\newcommand{\predf}[2]{\ensuremath{\pred(#1,#2)}}
\newcommand{\defeq}{\triangleq}

\newcommand{\type}[2]{\ensuremath{#1 : #2}}
\newcommand{\typed}[4]{\ensuremath{#1 \vdash_{#2} \type{#3}{#4}}}

\newcommand{\btype}{\beta}
\newcommand{\utype}{\theta}
\newcommand{\val}{v}
\newcommand{\uval}{u}

\newcommand{\algname}{\textsf{OAC}}
\newcommand{\algnameminus}{\textsf{OACMinus}}
\newcommand{\coam}{\textsf{OAC}}
\newcommand{\lopt}{\textsf{Lopt}}
\newcommand{\coamwith}[1]{\ensuremath{\mathsf{SOAM}[{#1}]}}
\newcommand{\queso}{{\textsf{Queso}}}
\newcommand{\voqc}{{\textsf{VOQC}}}
\newcommand{\pyzx}{{\textsf{PyZX}}}
\newcommand{\quartz}{{\textsf{Quartz}}}
\newcommand{\quartztool}{$\mathsf{Quartz}$}
\newcommand{\quesotool}{$\mathsf{Queso}$}
\newcommand{\feyntool}{\textsf{FeynOpt}}

\newcommand{\quartzt}[1]{\ensuremath{\mathsf{Quartz}_{\,#1}}}
\newcommand{\quesot}[1]{\ensuremath{\mathsf{Queso}_{\,#1}}}
\newcommand{\coamt}[1]{\ensuremath{\mathsf{SOAM}[#1]}}
\newcommand{\clifft}{Clifford+T}

\newcommand{\compat}[2]{\ensuremath{{#1} \mathbin{\scaleobj{1.2}{\diamond}} {#2}}}
\newcommand{\notcompat}[2]{\ensuremath{{#1} \mathbin{\scaleobj{1.2}{\centernot{\diamond}}} {#2}}}
\newcommand{\windowopt}[2]{{#2}~\textsf{\textbf{segment-optimal}}_{#1}}
\newcommand{\wopttext}{segment optimal}
\newcommand{\compressed}[1]{{#1}~\textsf{\textbf{compact}}}
\newcommand{\locallyopt}[2]{{#2}~\textsf{\textbf{locally-optimal}}_{#1}}
\newcommand{\qubits}[1]{\mathsf{qubits}({#1})}
%% Terms
\newcommand{\tmemprog}{memory-progress\xspace}
\newcommand{\tmempres}{memory-preservation\xspace}


%% imperative serial types
\newcommand{\kwint}{\kw{int}}
\newcommand{\kwnat}{\kw{nat}}
\newcommand{\kwfut}{\kw{fut}}
\newcommand{\kwprod}[2]{\ensuremath{{#1} \times {#2}}}
\newcommand{\kwarr}[2]{\ensuremath{{#1} \ra {#2}}}
\newcommand{\kwloc}[1]{\ensuremath{{#1}~\kw{loc}}}
\newcommand{\kwref}[1]{\ensuremath{{#1}~\kw{ref}}}

%% imperative multithreaded types
\newcommand{\kwtid}{\kw{tid}}
\newcommand{\kwunit}{\kw{unit}}
\newcommand{\kwok}{\kw{ok}}

% space, heap, store
\newcommand{\heap}{H}
\newcommand{\spc}{H}
\newcommand{\empspc}{\emptyset}
\newcommand{\spaceext}[3]{{#1}[{#2} \mapsto {#3}]}

\newcommand{\catspace}{\uplus}
\newcommand{\catheap}{\uplus}

\newcommand{\heapun}[1]{\langle #1 \rangle}
\newcommand{\heapbi}[2]{\langle #1 ; #2 \rangle}
\newcommand{\heaptri}[3]{\langle #1 ; #2; #3 \rangle}
\newcommand{\heapquad}[4]{\langle #1 ; #2 ; #3 ; #4 \rangle}
\newcommand{\restctx}[2]{\ensuremath{#1 \upharpoonright_{#2}}}
\newcommand{\freeloc}[1]{\ensuremath{\mathsf{FL}(#1)}}
\newcommand{\locs}[1]{\ensuremath{\mathsf{Loc}(#1)}}
\newcommand{\diff}[1]{\ensuremath{\mathsf{Diff}(#1)}}

\newcommand{\rename}[3]{[#2 \mapsto #3](#1)}
\newcommand{\kwt}{\kw{t}}

\newcommand{\estore}{[~]}
\newcommand{\mkstore}[2]{\ensuremath{{#1}::{#2}}}

%% imperative serial syntax
\newcommand{\kwn}{\kw{n}}
\newcommand{\kwlet}[3]{\kw{let}~{#1}={#2}~\kw{in}~{#3}~\kw{end}}
\newcommand{\kwfun}[3]{\ensuremath{\kw{fun}~{#1}~{#2}~\kw{is}~{#3}~\kw
{end}}}
\newcommand{\kwpair}[2]{\ensuremath{\langle{#1},{#2}}\rangle}
\newcommand{\kwapply}[2]{\ensuremath{{#1}~{#2}}}
\newcommand{\kwfst}[1]{\ensuremath{\kw{fst}\cdparens{#1}}}
\newcommand{\kwsnd}[1]{\ensuremath{\kw{snd}\cdparens{#1}}}
%\newcommand{\kwfst}[1]{\ensuremath{\kw{fst}~#1}}
%\newcommand{\kwsnd}[1]{\ensuremath{\kw{snd}~#1}}
\newcommand{\gcing}[1]{\ensuremath{[#1]}}
\newcommand{\kwnew}[1]{\ensuremath{\kw{ref}(#1)}}
\newcommand{\kwderef}[1]{\ensuremath{\mathop{!}#1}}
\newcommand{\kwwrite}[2]{\ensuremath{#1 \mathop{:=} #2}}

\newcommand{\kwletrec}[2]{\ensuremath{#1 \mathop{\cdot} #2}}
\newcommand{\kwtask}[3]{\ensuremath{#1 \mathop{\cdot} #2 \mathop{\cdot} #3}}
\newcommand{\kwtaskalt}[2]{\ensuremath{#1 \mathop{\cdot} #2}}
\newcommand{\halt}{\bot}
\newcommand{\tree}{T}
\newcommand{\trace}{t}
%% imperative multithreaded syntax
\newcommand{\kwfork}[1]{{\ensuremath{\kw{fork}\cdparens{#1}}}}
\newcommand{\kwjoin}[1]{{\ensuremath{\kw{join}\cdparens{#1}}}}
\newcommand{\kwunitv}{\ensuremath{(\,)}}
\newcommand{\kwtidv}{\ensuremath{\kw{t}}}
\newcommand{\gheap}{G}
\newcommand{\lheap}{\heap}
\newcommand{\tolheap}[1]{\ensuremath{\Delta(#1)}}
\newcommand{\theap}[2]{\left(#1, #2\right)}

%% hierarchical syntax
\newcommand{\cdpar}{\texttt{par}}
%\newcommand{\kwpar}[2]{\ensuremath{#1 \mathop{\|} #2}}
\newcommand{\kwpar}[2]
           {\ensuremath{\mathop{\vartriangleleft \hspace{-0.1em} #1, #2
               \hspace{-0.1em} \vartriangleright}}}
\newcommand{\kwpara}[2]
           {\ensuremath{\mathop{\blacktriangleleft \hspace{-0.1em} #1, #2
               \hspace{-0.1em} \blacktriangleright}}}
%\newcommand{\kwpara}[2]{\ensuremath{#1 \mathop{\overline{\|}} #2}}
\newcommand{\kwparl}[2]{\ensuremath{#1 \overset{\leftarrow}{\|} #2}}
\newcommand{\kwparr}[2]{\ensuremath{#1 \overset{\rightarrow}{\|} #2}}
%\newcommand{\gcing}[1]{\ensuremath{[#1]}}
\newcommand{\task}{T}
\newcommand{\config}{\mathcal{C}}

%% flattening
\newcommand{\flate}[1]{\hat{#1}}
\newcommand{\flatten}[3]{\left\| #2 \right\|_{#1} \leadsto #3}
\newcommand{\fstep}{\step}%{\step_F}

%% shorthands
\renewcommand{\a}{\ensuremath{\alpha}}
\renewcommand{\b}{\ensuremath{\beta}}
\newcommand{\h}{\ensuremath{\eta}}
\renewcommand{\r}{\ensuremath{\rho}}
%\newcommand{\s}{\ensuremath{\sigma}}
\newcommand{\p}{\ensuremath{P}}
\newcommand{\s}{\p}
\newcommand{\om}{\ensuremath{\Omega}}
\renewcommand{\l}{\ensuremath{l}}
\newcommand{\sig}{\ensuremath{\Sigma}}
\newcommand{\empctx}{\ensuremath{\cdot}}


% Relations
%\newcommand{\red}{\Downarrow}
%\newcommand{\redgc}{\stackrel{gc?}{\Longrightarrow}}
%\newcommand{\alloc}{\stackrel{alloc}\Longrightarrow}
\newcommand{\la}{\leftarrow}
\newcommand{\ra}{\rightarrow}
\newcommand{\pstep}{\Rightarrow}
\newcommand{\tstep}{\Rightarrow}
\newcommand{\optstep}{\longmapsto}
\newcommand{\compstep}{\longmapsto_{\delta}}
\newcommand{\localstep}[3]{{#2} \overset{#1}{\optstep} {#3}}
\newcommand{\globstep}[2]{{#1} \compstep {#2}}
%\newcommand{\stepr}[1]{\xra{#1}}
\newcommand{\step}{\ra}
\newcommand{\stepgc}[1]{\xra[{\mbox{\tiny GC}}]{#1}}
\newcommand{\gcstep}{\ra_{\mbox{\tiny GC}}}
\newcommand{\pgcstep}{\pstep_{\mbox{\tiny GC}}}
\newcommand{\cgcstep}{\rightarrow_{\mbox{\tiny CGC}}}


%\newcommand{\sunion}[2]{{#1} \stackrel{?}{\bigcup} {#2}}
%\newcommand{\spush}[2]{{#1} \stackrel{?}{\downarrow} {#2}}

% Other judgments
\newcommand{\fresh}{\ensuremath{\; \mathsf{fresh}}}
%\newcommand{\leaf}{\ensuremath{\; \mathsf{leaf}}}
%\newcommand{\starrow}[1]{\stackrel{\mbox{\tiny #1}}{\xrightarrow}}
\newcommand{\starrow}[1]{\xrightarrow{#1}}
%\newcommand{\alloc}[4]{\mathit{alloc}\left(#1, #2\right) = \left(#3, #4\right)}
\newcommand{\alloc}[4]{#1; #2 \starrow{alloc} #3; #4}
%\newcommand{\update}[4]{\mathit{update}\left(#1, #2 \la #3\right) = #4}
\newcommand{\update}[4]{#1; #2; #3 \starrow{update} #4}
%\newcommand{\lookup}[3]{\mathit{lookup}\left(#1, #2\right) = #3}
\newcommand{\lookup}[3]{#1; #2 \starrow{lookup} #3}
\newcommand{\newtask}[2]{#1 \starrow{new} #2}
\newcommand{\isdone}[3]{#1 \starrow{done} #2; #3}
\newcommand{\diffs}[1]{\mathit{diff}(#1)}
\newcommand{\initial}{\ensuremath{\;\mathsf{initial}}}
\newcommand{\htyped}[3]{\vdash_{#3} #1 : #2}

% Multilevel heap judgments
\newcommand{\heaptype}[3]{\left(#1, #2\right) : #3}
\newcommand{\allocg}[4]{\mathit{allocg}\left(#1, #2\right) = \left(#3, #4\right)}
\newcommand{\allocl}[4]{\mathit{allocl}\left(#1, #2\right) = \left(#3, #4\right)}
%\newcommand{\promote}[6]{\mathit{promote}\left(#1, #2, #3\right) =
%  \left(#4, #5, #6\right)}
\newcommand{\promote}[6]{#1; #2; #3 \starrow{promote} #4; #5; #6}
%\newcommand{\promotebrl}[3]{\mathit{promote}\left(#1, #2, #3\right)}
%\newcommand{\promotebrr}[3]{\left(#1, #2, #3\right)}
\newcommand{\promotebrl}[3]{#1; #2; #3}
\newcommand{\promotebra}{\starrow{promote}}
\newcommand{\promotebrr}[3]{#1; #2; #3}
\newcommand{\pmap}{M}
\newcommand{\greachable}[1]{\mathsf{greachable}\left(#1\right)}

% Theorems
\newtheorem{thm}[theorem]{Theorem}
\newtheorem{lem}[theorem]{Lemma}
% \newtheorem{corollary}[theorem]{Corollary}
% \newtheorem{claim}{Claim}


%% Rule Array
\newenvironment{rulearray}
{
\newcommand{\newcol}{\qquad}
\newcommand{\newcolhalf}{\quad}
\newcommand{\newrow}{\\[4ex]}
\newcommand{\newrowhalf}{\\[2ex]}
\[
\begin{array}{c}
}
{
\end{array}
\]
\let\newcol\undefined
\let\newrow\undefined
}

% Author-specific todo notes
\newcommand{\ramtodo}[2][]
{\todo[color=magenta,author=Ram,size=\small,#1]{#2}}


\newcommand{\defn}[1]{\emph{\textbf{#1}}}
\newcommand{\mpl}{\textsf{MPL}}

\newcommand{\rulereftwo}[2]{rules~\rulename{#1} and \rulename{#2}}
\newcommand{\with}{\ensuremath{\mathbin;}}

\newcommand{\highlight}[1]{\colorbox{gray!20}{\ensuremath{#1}}}
\newcommand{\hred}[1]{\colorbox{red!10}{\ensuremath{#1}}}
\newcommand{\hblue}[1]{\colorbox{blue!10}{\ensuremath{#1}}}
\newcommand{\hgreen}[1]{\colorbox{green!10}{\ensuremath{#1}}}
\newcommand{\sizeof}[1]{\ensuremath{\lvert #1 \rvert}}
\newcommand{\costof}[1]{\ensuremath{\mathbf{cost} ({#1})}}
\newcommand{\cost}{\ensuremath{\mathbf{cost}}}
\newcommand{\oracle}{\ensuremath{\mathbf{oracle}}}


% inline "math highlight" to make it easier to read inline judgements
\definecolor{darkblue}{HTML}{0007C9}
% \newcommand{\mh}[1]{{\color{darkblue}\ensuremath{\mbox{\ensuremath{#1}}}}}
\newcommand{\mh}[1]{{\ensuremath{\mbox{\ensuremath{#1}}}}}


%% variable context, location signature
% \newcommand{\ctxvar}{\Gamma}
% \newcommand{\ctxloc}{\Sigma}
\newcommand{\ctxemp}{\ensuremath{\cdot}}
\newcommand{\ctxext}[3]{\ensuremath{#1,#2\!:\!#3}} % extend context
\newcommand{\etyped}[4]{\ensuremath{{#1} \vdash_{#2} {#3} : {#4}}}
\newcommand{\memtyped}[3]{\ensuremath{{#1} \vdash {#2} : {#3}}}
\newcommand{\gtyped}[3]{\ensuremath{{#1} \vdash {#2} : {#3}}}
\newcommand{\httyped}[6]{\ensuremath{{#1} \with {#2} \with {#3} \vdash {#4}\!\cdot\!{#5} : {#6}}}
\newcommand{\ttyped}[5]{\ensuremath{{#1} \with {#2} \with {#3} \vdash {#4} : {#5}}}

\newcommand{\sttyped}[6]{\ensuremath{{\vdash_{#1} {#2} \with {#3} \with {#4} \with {#5} : {#6}}}}
\newcommand{\getyped}[6]{\ensuremath{{#1} \vdash_{#2, #3} {#4} \with {#5} : {#6}}}




%% types
\newcommand{\typnat}{\kw{nat}}
\newcommand{\typint}{\kw{int}}
\newcommand{\typbool}{\kw{bool}}
\newcommand{\typchar}{\kw{char}}
\newcommand{\typfloat}{\kw{float}}
\newcommand{\typprod}[2]{\ensuremath{{#1} \times {#2}}}
\newcommand{\typfun}[2]{\ensuremath{{#1}\!\rightarrow\!{#2}}}
\newcommand{\typref}[1]{\ensuremath{{#1}~\kw{ref}}}
\newcommand{\typfut}[1]{\ensuremath{{#1}~\kw{fut}}}
\newcommand{\futs}[1]{\mathsf{Fut}(#1)}
\newcommand{\futsmem}[2]{\mathsf{Fut}(#1, #2)}




% expression syntax
\newcommand{\enat}[1]{\ensuremath{#1}}
\newcommand{\efun}[3]{\ensuremath{\kw{fun}~{#1}~{#2}~\kw{is}~{#3}}}
\newcommand{\epair}[2]{\ensuremath{\langle {#1}, {#2} \rangle}}
\newcommand{\eapp}[2]{\ensuremath{{#1}~{#2}}}
\newcommand{\efst}[1]{\ensuremath{\kw{fst}~{#1}}}
\newcommand{\esnd}[1]{\ensuremath{\kw{snd}~{#1}}}
\newcommand{\eref}[1]{\ensuremath{\kw{ref}~{#1}}}
\newcommand{\ebang}[1]{\ensuremath{\mathop{!}#1}}
\newcommand{\eupd}[2]{\ensuremath{#1 \mathop{:=} #2}}
\newcommand{\elet}[3]{\kw{let}~{#1}={#2}~\kw{in}~{#3}}
\newcommand{\epar}[2]{\ensuremath{\langle {#1}\mathbin\|{#2} \rangle}}

\newcommand{\purelang}{{\sc $\lambda^{P}$}}
\newcommand{\reflang}{{\sc $\lambda^{U}$}}

% task syntax
% \newcommand{\tleaf}[2]{\ensuremath{{#1}\!\cdot\!{#2}}}
% \newcommand{\tpar}[4]{\ensuremath{\dblangle{{#1}\!\cdot\!{#2}\mathbin\|{#3}\!\cdot\!{#4}}}}
% \newcommand{\tpar}[4]{\ensuremath{\llparenthesis\,{#1}\!\cdot\!{#2}\mathbin\|{#3}\!\cdot\!{#4}\,\rrparenthesis}}
% \newcommand{\ttpar}[2]{\ensuremath{\llparenthesis\,{#1}\mathbin\|{#2}\,\rrparenthesis}}
% \newcommand{\tparg}[6]{\ensuremath{\llparenthesis\,{#1}\!\cdot\!{#2}\!\cdot\!{#3}\mathbin\|{#4}\!\cdot\!{#5}\!\cdot\!{#6}\,\rrparenthesis}}
% \newcommand{\tpar}[3]{\ensuremath{{#1}\!\cdot\!\llparenthesis\,{#2}\mathbin\|{#3}\,\rrparenthesis}}
% \newcommand{\taskhpe}[3]{\ensuremath{{#1}\!\cdot\!{#2}\!\cdot\!{#3}}}

% \newcommand{\mem}{\mu}
\newcommand{\mememp}{\emptyset}
\newcommand{\memext}[3]{\ensuremath{#1}[{#2} \!\hookrightarrow\! {#3}]}

\newcommand{\actarrow}{\blacktriangleright}
\newcommand{\pasarrow}{\vartriangleright}
\newcommand{\fmap}{\Delta}
\newcommand{\femp}{\emptyset}
\newcommand{\fmapactive}[3]{\ensuremath{#1} [{#2} \!\actarrow\! {#3}]}
\newcommand{\fmapjoined}[3]{\ensuremath{#1} [{#2} \!\pasarrow\! {#3}]}


\newcommand{\futctxt}{\Knownctxt}
\newcommand{\Futctxt}{\Knownctxt}
\newcommand{\ReadLocs}{\mathsf{R}}
\newcommand{\Knownctxt}{{K}}
\newcommand{\fut}[2]{\kw{fut}(#1; #2)}
\newcommand{\harpfut}[1]{\kw{fut}(#1)}
\newcommand{\futctxtemp}{\emptyset}
\newcommand{\te}[1]{\{#1\}}
\newcommand{\hemp}{\emptyset}
\newcommand{\hcat}{\cup}
\newcommand{\hext}[2]{{#1},{#2}}

\newcommand{\tack}{\oplus}
\newcommand{\plug}{\bowtie}

\newcommand{\omparam}{step length}
\newcommand{\actwrite}[2]{\textbf{U}{#1}\!\Leftarrow\!{#2}}
\newcommand{\actalloc}[2]{\textbf{A}{#1}\!\Leftarrow\!{#2}}
\newcommand{\actread}[2]{\textbf{R}{#1}\!\Rightarrow\!{#2}}
\newcommand{\actsync}[2]{\textbf{F}{#1}\!\Rightarrow\!{#2}}
\newcommand{\actnone}{\textbf{N}}

% Relations
\newcommand{\stepstar}{\longmapsto^*}
\newcommand{\tstepstar}{\tstep^*}
\newcommand{\drfstep}[2]{\xmapsto[{#2}]{{\,#1\,}}}
\newcommand{\drfstepstar}[1]{\xmapsto{{\,#1\,}}\joinrel\mathrel{^*}}


% computation graph
\newcommand{\gt}[2]{\ensuremath{\mathsf{GT}({#1},{#2})}}
\newcommand{\gemp}{\bullet}
\newcommand{\gseq}[2]{{#1}\oplus{#2}}
\newcommand{\gseqnamed}[3]{{#1}\oplus_{#2}{#3}}
\newcommand{\gseqa}[2]{\gseqnamed{#1}{a}{#2}}
\newcommand{\gseqb}[2]{\gseqnamed{#1}{b}{#2}}
\newcommand{\gspawn}[1]{\mathsf{spawn}\ {#1}}
\newcommand{\gsync}[1]{\mathsf{sync}\ {#1}}
\newcommand{\ghead}[1]{\mathsf{hd}(#1)}
\newcommand{\gtail}[1]{\mathsf{tl}(#1)}

% \def\ojoin{\setbox0=\hbox{$\bowtie$}%
%   \rule[\bt0]{.25em}{.4pt}\llap{\rule[\ht0]{.25em}{.4pt}}}


\newcommand{\fcpar}[3]{\ensuremath{\gseq{#1}{(\gpar{#2}{#3})}}}

% \def\rightouterjoin{\mathbin{\bowtie\mkern-5.8mu\ojoin}}

\newcommand{\gmerge}[2]{\bowtie_F ({#1}, {#2})}
\newcommand{\gmergerel}[3]{\bowtie_R ({#1}, {#2}) \downarrow {#3}}
% \newcommand{\gcseq}[1]{\ensuremath{\gseq{#1}{\raisebox{-1pt}{$\square$}}}}
% \newcommand{\gcseq}[1]{\ensuremath{\fbox{$#1$}}}
% \newcommand{\gcseq}[1]{\ensuremath{\llparenthesis #1 \rrparenthesis}}
\newcommand{\gcseq}[1]{\ensuremath{[#1]}}
% \newcommand{\gcpar}[3]{\ensuremath{{#1}\!\cdot\!({#2}\mathbin\|{#3})}}
\newcommand{\gcpar}[3]{\ensuremath{\gseq{#1}{(\gpar{#2}{#3})}}}
\newcommand{\gcparnamed}[4]{\ensuremath{\gseq{#1}{({#2}\otimes_{#3}{#4})}}}
\newcommand{\gcspawn}[4]{\ensuremath{\gseq{#1}{\gseq{#2}{(\gpar{#3}{#4})}}}}


\newcommand{\gpar}[2]{{#1}\otimes_{a}{#2}}
\newcommand{\gw}[1]{\ensuremath{\mathsf{W}({#1})}}
\newcommand{\ga}[1]{\ensuremath{\mathsf{A}({#1})}}
\newcommand{\greads}[1]{\ensuremath{\ReadLocs({#1})}}
\newcommand{\gaw}[1]{\ensuremath{\mathsf{AW}({#1})}}
\newcommand{\lw}[1]{\ensuremath{\mathsf{LW}({#1})}}
\newcommand{\alw}[1]{\ensuremath{\mathsf{A}({#1}) \cup \mathsf{LW}({#1})}}
\newcommand{\gabw}[1]{\ensuremath{\gaw{#1}}}
% \newcommand{\gabw}[1]{\ensuremath{\mathsf{A}({#1}) \cup \lw{#1}}}


\newcommand{\saw}[1]{\ensuremath{\mathsf{SP}({#1})}}
% \newcommand{\gf}[1]{\ensuremath{\llbracket{#1}\rrbracket}}
\newcommand{\gf}[1]{\ensuremath{\overline{#1}}}

\newcommand{\extendsfj}[2]{\ensuremath{{#1}~\textsf{extends}~{#2}~\textsf{with f/j}}}
\newcommand{\extendswith}[3]{\ensuremath{{#1}~\textsf{extends}~{#2}~\textsf{with}~{#3}}}

% \newcommand{\tpardag}[4]{\ensuremath{\llparenthesis\,{#1}\!\cdot\!{#2}\mathbin\|{#3}\!\cdot\!{#4}\rrparenthesis}}
% \newcommand{\dagobv}[2]{#1~\textsf{obv}~#2}
% \newcommand{\dagread}[2]{#1~\textsf{reads}~#2}
% \newcommand{\dagwrite}[2]{#1~\textsf{writes}~#2}
% \newcommand{\dagdrf}[2]{#1~\textsf{drf}~@~#2}

\newcommand{\geok}[2]{{#1} \with {#2}~\textit{ok}}
\newcommand{\loc}[1]{{#1}~\textit{loc}}
\newcommand{\gleaf}[1]{{#1}~\textit{leaf}}
\newcommand{\gnode}[1]{{#1}~\textit{node}}

\newcommand{\drf}[2]{{#1} \vdash {#2}~{\textit{drf}}}
\newcommand{\drfb}[2]{{#1} \vdash {#2}~{\textit{wrf}}}
\newcommand{\drft}{\textit{drf}}
% \newcommand{\typed}[2]{{#1} \vdash }

% Theorems
\theoremstyle{plain}
\newtheorem{property}{Property}

% \theoremstyle{definition}
% \newtheorem{definition}{Definition}

%% Rules description
\newcommand{\flushLR}[3]{\hspace*{#3}\makebox[0em][l]{#1}\hspace*{\fill}\makebox[0em][r]{#2}\hspace*{#3}}
% \newcommand{\rulesdesc}[2]{\flushLR{\textbf{#1}}{\fbox{#2}}{1em}}
\newcommand{\rulesdesc}[2]{\textbf{#1}\hspace*{1em}{\fbox{#2}}}
\newcommand{\desc}[1]{\textbf{#1}}

% Syntax highlighting
\newdimen\zzlistingsize
\newdimen\zzlistingsizedefault
\zzlistingsizedefault=9pt
\newdimen\kwlistingsize
\kwlistingsize=9pt
\zzlistingsize=\zzlistingsizedefault
\gdef\lco{black}
%\newcommand{\keywordstyle}{\fontsize{0.9\zzlistingsize}{1.0\zzlistingsize}\bf}
%\newcommand{\keywordstyle}{\fontsize{\kwlistingsize}{1.1\kwlistingsize}\normalfont\bf\color{\lco}}
%\settowidth{\zzlstwidth}{{\Lstbasicstyle~}}




%%
%% end of the preamble, start of the body of the document source.
\begin{document}

%%
%% The "title" command has an optional parameter,
%% allowing the author to define a "short title" to be used in page headers.
\title{\bench: A \underline{Co}de E\underline{ff}ici\underline{e}ncy Benchmark for Code Generation}

%%
%% The "author" command and its associated commands are used to define
%% the authors and their affiliations.
%% Of note is the shared affiliation of the first two authors, and the
%% "authornote" and "authornotemark" commands
%% used to denote shared contribution to the research.

%%
%% By default, the full list of authors will be used in the page
%% headers. Often, this list is too long, and will overlap
%% other information printed in the page headers. This command allows
%% the author to define a more concise list
%% of authors' names for this purpose.


\author{Yun Peng}
\orcid{0000-0003-1936-5598}
\affiliation{%
  \institution{The Chinese University of Hong Kong}
  \city{Hong Kong}
  \country{China}
}
\email{ypeng@cse.cuhk.edu.hk}

\author{Jun Wan}
\orcid{0009-0006-3294-688X}
\affiliation{%
  \institution{Zhejiang University}
  \city{Hangzhou}
  \country{China}
}
\email{22451014@zju.edu.cn}

\author{Yichen Li}
\orcid{0009-0009-8370-644X}
\affiliation{%
  \institution{The Chinese University of Hong Kong}
  \city{Hong Kong}
  \country{China}
}
\email{ycli21@cse.cuhk.edu.hk}

\author{Xiaoxue Ren}
\authornote{Corresponding author.}
\authornote{Also with Hangzhou High-Tech Zone (Binjiang) Institute of Blockchain and Data Security.}
\orcid{0000-0002-5526-1617}
\affiliation{%
  \institution{The State Key Laboratory of Blockchain and Data Security, Zhejiang University}
  \city{Hangzhou}
  \country{China}
}
\email{xxren@zju.edu.cn}

%%
%% The abstract is a short summary of the work to be presented in the
%% article.

%%
%% The code below is generated by the tool at http://dl.acm.org/ccs.cfm.
%% Please copy and paste the code instead of the example below.
%%


%%
%% Keywords. The author(s) should pick words that accurately describe
%% the work being presented. Separate the keywords with commas.
%% A "teaser" image appears between the author and affiliation
%% information and the body of the document, and typically spans the
%% page.

%%
%% This command processes the author and affiliation and title
%% information and builds the first part of the formatted document.
% \begin{figure*}[!htp]
%     \centering
%     \includegraphics[width=\textwidth]{icml2024/figures/intro.pdf}
%     \caption{SayAnything performs audio-driven lip synchronization through video editing, demonstrating zero-shot generalization to in-the-wild and various style domains without fine-tuning. Our fusion scheme eliminates the dependency on additional supervision signals like SyncNet for lip synchronization. More video results are available in the supplementary materials.}
%     \label{fig:teaser}
% \end{figure*}

\begin{abstract}
Recent advances in diffusion models have led to significant progress in audio-driven lip synchronization. However, existing methods typically rely on constrained audio-visual alignment priors or multi-stage learning of intermediate representations to force lip motion synthesis. This leads to complex training pipelines and limited motion naturalness. In this paper, we present SayAnything, a conditional video diffusion framework that directly synthesizes lip movements from audio input while preserving speaker identity. Specifically, we propose three specialized modules, including an identity preservation module, an audio guidance module, and an editing control module. Our novel design effectively balances different condition signals in the latent space, enabling precise control over appearance, motion, and region-specific generation without requiring additional supervision signals or intermediate representations. Extensive experiments demonstrate that SayAnything generates highly realistic videos with improved lip-teeth coherence, enabling unseen characters to \textbf{say anything} while effectively generalizing to animated characters.
\end{abstract}
\vspace{-1cm}

\begin{CCSXML}
<ccs2012>
   <concept>
       <concept_id>10011007.10011074.10011092.10011782</concept_id>
       <concept_desc>Software and its engineering~Automatic programming</concept_desc>
       <concept_significance>500</concept_significance>
       </concept>
 </ccs2012>
\end{CCSXML}

\ccsdesc[500]{Software and its engineering~Automatic programming}

\keywords{Code Generation, Benchmark, Code Efficiency, Time}

\maketitle


\section{Introduction}


\begin{figure}[t]
\centering
\includegraphics[width=0.6\columnwidth]{figures/evaluation_desiderata_V5.pdf}
\vspace{-0.5cm}
\caption{\systemName is a platform for conducting realistic evaluations of code LLMs, collecting human preferences of coding models with real users, real tasks, and in realistic environments, aimed at addressing the limitations of existing evaluations.
}
\label{fig:motivation}
\end{figure}

\begin{figure*}[t]
\centering
\includegraphics[width=\textwidth]{figures/system_design_v2.png}
\caption{We introduce \systemName, a VSCode extension to collect human preferences of code directly in a developer's IDE. \systemName enables developers to use code completions from various models. The system comprises a) the interface in the user's IDE which presents paired completions to users (left), b) a sampling strategy that picks model pairs to reduce latency (right, top), and c) a prompting scheme that allows diverse LLMs to perform code completions with high fidelity.
Users can select between the top completion (green box) using \texttt{tab} or the bottom completion (blue box) using \texttt{shift+tab}.}
\label{fig:overview}
\end{figure*}

As model capabilities improve, large language models (LLMs) are increasingly integrated into user environments and workflows.
For example, software developers code with AI in integrated developer environments (IDEs)~\citep{peng2023impact}, doctors rely on notes generated through ambient listening~\citep{oberst2024science}, and lawyers consider case evidence identified by electronic discovery systems~\citep{yang2024beyond}.
Increasing deployment of models in productivity tools demands evaluation that more closely reflects real-world circumstances~\citep{hutchinson2022evaluation, saxon2024benchmarks, kapoor2024ai}.
While newer benchmarks and live platforms incorporate human feedback to capture real-world usage, they almost exclusively focus on evaluating LLMs in chat conversations~\citep{zheng2023judging,dubois2023alpacafarm,chiang2024chatbot, kirk2024the}.
Model evaluation must move beyond chat-based interactions and into specialized user environments.



 

In this work, we focus on evaluating LLM-based coding assistants. 
Despite the popularity of these tools---millions of developers use Github Copilot~\citep{Copilot}---existing
evaluations of the coding capabilities of new models exhibit multiple limitations (Figure~\ref{fig:motivation}, bottom).
Traditional ML benchmarks evaluate LLM capabilities by measuring how well a model can complete static, interview-style coding tasks~\citep{chen2021evaluating,austin2021program,jain2024livecodebench, white2024livebench} and lack \emph{real users}. 
User studies recruit real users to evaluate the effectiveness of LLMs as coding assistants, but are often limited to simple programming tasks as opposed to \emph{real tasks}~\citep{vaithilingam2022expectation,ross2023programmer, mozannar2024realhumaneval}.
Recent efforts to collect human feedback such as Chatbot Arena~\citep{chiang2024chatbot} are still removed from a \emph{realistic environment}, resulting in users and data that deviate from typical software development processes.
We introduce \systemName to address these limitations (Figure~\ref{fig:motivation}, top), and we describe our three main contributions below.


\textbf{We deploy \systemName in-the-wild to collect human preferences on code.} 
\systemName is a Visual Studio Code extension, collecting preferences directly in a developer's IDE within their actual workflow (Figure~\ref{fig:overview}).
\systemName provides developers with code completions, akin to the type of support provided by Github Copilot~\citep{Copilot}. 
Over the past 3 months, \systemName has served over~\completions suggestions from 10 state-of-the-art LLMs, 
gathering \sampleCount~votes from \userCount~users.
To collect user preferences,
\systemName presents a novel interface that shows users paired code completions from two different LLMs, which are determined based on a sampling strategy that aims to 
mitigate latency while preserving coverage across model comparisons.
Additionally, we devise a prompting scheme that allows a diverse set of models to perform code completions with high fidelity.
See Section~\ref{sec:system} and Section~\ref{sec:deployment} for details about system design and deployment respectively.



\textbf{We construct a leaderboard of user preferences and find notable differences from existing static benchmarks and human preference leaderboards.}
In general, we observe that smaller models seem to overperform in static benchmarks compared to our leaderboard, while performance among larger models is mixed (Section~\ref{sec:leaderboard_calculation}).
We attribute these differences to the fact that \systemName is exposed to users and tasks that differ drastically from code evaluations in the past. 
Our data spans 103 programming languages and 24 natural languages as well as a variety of real-world applications and code structures, while static benchmarks tend to focus on a specific programming and natural language and task (e.g. coding competition problems).
Additionally, while all of \systemName interactions contain code contexts and the majority involve infilling tasks, a much smaller fraction of Chatbot Arena's coding tasks contain code context, with infilling tasks appearing even more rarely. 
We analyze our data in depth in Section~\ref{subsec:comparison}.



\textbf{We derive new insights into user preferences of code by analyzing \systemName's diverse and distinct data distribution.}
We compare user preferences across different stratifications of input data (e.g., common versus rare languages) and observe which affect observed preferences most (Section~\ref{sec:analysis}).
For example, while user preferences stay relatively consistent across various programming languages, they differ drastically between different task categories (e.g. frontend/backend versus algorithm design).
We also observe variations in user preference due to different features related to code structure 
(e.g., context length and completion patterns).
We open-source \systemName and release a curated subset of code contexts.
Altogether, our results highlight the necessity of model evaluation in realistic and domain-specific settings.





\section{Bellman Error Centering}

Centering operator $\mathcal{C}$ for a variable $x(s)$ is defined as follows:
\begin{equation}
\mathcal{C}x(s)\dot{=} x(s)-\mathbb{E}[x(s)]=x(s)-\sum_s{d_{s}x(s)},
\end{equation} 
where $d_s$ is the probability of $s$.
In vector form,
\begin{equation}
\begin{split}
\mathcal{C}\bm{x} &= \bm{x}-\mathbb{E}[x]\bm{1}\\
&=\bm{x}-\bm{x}^{\top}\bm{d}\bm{1},
\end{split}
\end{equation} 
where $\bm{1}$ is an all-ones vector.
For any vector $\bm{x}$ and $\bm{y}$ with a same distribution $\bm{d}$,
we have
\begin{equation}
\begin{split}
\mathcal{C}(\bm{x}+\bm{y})&=(\bm{x}+\bm{y})-(\bm{x}+\bm{y})^{\top}\bm{d}\bm{1}\\
&=\bm{x}-\bm{x}^{\top}\bm{d}\bm{1}+\bm{y}-\bm{y}^{\top}\bm{d}\bm{1}\\
&=\mathcal{C}\bm{x}+\mathcal{C}\bm{y}.
\end{split}
\end{equation}
\subsection{Revisit Reward Centering}


The update (\ref{src3}) is an unbiased estimate of the average reward
with  appropriate learning rate $\beta_t$ conditions.
\begin{equation}
\bar{r}_{t}\approx \lim_{n\rightarrow\infty}\frac{1}{n}\sum_{t=1}^n\mathbb{E}_{\pi}[r_t].
\end{equation}
That is 
\begin{equation}
r_t-\bar{r}_{t}\approx r_t-\lim_{n\rightarrow\infty}\frac{1}{n}\sum_{t=1}^n\mathbb{E}_{\pi}[r_t]= \mathcal{C}r_t.
\end{equation}
Then, the simple reward centering can be rewrited as:
\begin{equation}
V_{t+1}(s_t)=V_{t}(s_t)+\alpha_t [\mathcal{C}r_{t+1}+\gamma V_{t}(s_{t+1})-V_t(s_t)].
\end{equation}
Therefore, the simple reward centering is, in a strict sense, reward centering.

By definition of $\bar{\delta}_t=\delta_t-\bar{r}_{t}$,
let rewrite the update rule of the value-based reward centering as follows:
\begin{equation}
V_{t+1}(s_t)=V_{t}(s_t)+\alpha_t \rho_t (\delta_t-\bar{r}_{t}),
\end{equation}
where $\bar{r}_{t}$ is updated as:
\begin{equation}
\bar{r}_{t+1}=\bar{r}_{t}+\beta_t \rho_t(\delta_t-\bar{r}_{t}).
\label{vrc3}
\end{equation}
The update (\ref{vrc3}) is an unbiased estimate of the TD error
with  appropriate learning rate $\beta_t$ conditions.
\begin{equation}
\bar{r}_{t}\approx \mathbb{E}_{\pi}[\delta_t].
\end{equation}
That is 
\begin{equation}
\delta_t-\bar{r}_{t}\approx \mathcal{C}\delta_t.
\end{equation}
Then, the value-based reward centering can be rewrited as:
\begin{equation}
V_{t+1}(s_t)=V_{t}(s_t)+\alpha_t \rho_t \mathcal{C}\delta_t.
\label{tdcentering}
\end{equation}
Therefore, the value-based reward centering is no more,
 in a strict sense, reward centering.
It is, in a strict sense, \textbf{Bellman error centering}.

It is worth noting that this understanding is crucial, 
as designing new algorithms requires leveraging this concept.


\subsection{On the Fixpoint Solution}

The update rule (\ref{tdcentering}) is a stochastic approximation
of the following update:
\begin{equation}
\begin{split}
V_{t+1}&=V_{t}+\alpha_t [\bm{\mathcal{T}}^{\pi}\bm{V}-\bm{V}-\mathbb{E}[\delta]\bm{1}]\\
&=V_{t}+\alpha_t [\bm{\mathcal{T}}^{\pi}\bm{V}-\bm{V}-(\bm{\mathcal{T}}^{\pi}\bm{V}-\bm{V})^{\top}\bm{d}_{\pi}\bm{1}]\\
&=V_{t}+\alpha_t [\mathcal{C}(\bm{\mathcal{T}}^{\pi}\bm{V}-\bm{V})].
\end{split}
\label{tdcenteringVector}
\end{equation}
If update rule (\ref{tdcenteringVector}) converges, it is expected that
$\mathcal{C}(\mathcal{T}^{\pi}V-V)=\bm{0}$.
That is 
\begin{equation}
    \begin{split}
    \mathcal{C}\bm{V} &= \mathcal{C}\bm{\mathcal{T}}^{\pi}\bm{V} \\
    &= \mathcal{C}(\bm{R}^{\pi} + \gamma \mathbb{P}^{\pi} \bm{V}) \\
    &= \mathcal{C}\bm{R}^{\pi} + \gamma \mathcal{C}\mathbb{P}^{\pi} \bm{V} \\
    &= \mathcal{C}\bm{R}^{\pi} + \gamma (\mathbb{P}^{\pi} \bm{V} - (\mathbb{P}^{\pi} \bm{V})^{\top} \bm{d_{\pi}} \bm{1}) \\
    &= \mathcal{C}\bm{R}^{\pi} + \gamma (\mathbb{P}^{\pi} \bm{V} - \bm{V}^{\top} (\mathbb{P}^{\pi})^{\top} \bm{d_{\pi}} \bm{1}) \\  % 修正双重上标
    &= \mathcal{C}\bm{R}^{\pi} + \gamma (\mathbb{P}^{\pi} \bm{V} - \bm{V}^{\top} \bm{d_{\pi}} \bm{1}) \\
    &= \mathcal{C}\bm{R}^{\pi} + \gamma (\mathbb{P}^{\pi} \bm{V} - \bm{V}^{\top} \bm{d_{\pi}} \mathbb{P}^{\pi} \bm{1}) \\
    &= \mathcal{C}\bm{R}^{\pi} + \gamma (\mathbb{P}^{\pi} \bm{V} - \mathbb{P}^{\pi} \bm{V}^{\top} \bm{d_{\pi}} \bm{1}) \\
    &= \mathcal{C}\bm{R}^{\pi} + \gamma \mathbb{P}^{\pi} (\bm{V} - \bm{V}^{\top} \bm{d_{\pi}} \bm{1}) \\
    &= \mathcal{C}\bm{R}^{\pi} + \gamma \mathbb{P}^{\pi} \mathcal{C}\bm{V} \\
    &\dot{=} \bm{\mathcal{T}}_c^{\pi} \mathcal{C}\bm{V},
    \end{split}
    \label{centeredfixpoint}
    \end{equation}
where we defined $\bm{\mathcal{T}}_c^{\pi}$ as a centered Bellman operator.
We call equation (\ref{centeredfixpoint}) as centered Bellman equation.
And it is \textbf{centered fixpoint}.

For linear value function approximation, let define
\begin{equation}
\mathcal{C}\bm{V}_{\bm{\theta}}=\bm{\Pi}\bm{\mathcal{T}}_c^{\pi}\mathcal{C}\bm{V}_{\bm{\theta}}.
\label{centeredTDfixpoint}
\end{equation}
We call equation (\ref{centeredTDfixpoint}) as \textbf{centered TD fixpoint}.

\subsection{On-policy and Off-policy Centered TD Algorithms
with Linear Value Function Approximation}
Given the above centered TD fixpoint,
 mean squared centered Bellman error (MSCBE), is proposed as follows:
\begin{align*}
    \label{argminMSBEC}
 &\arg \min_{{\bm{\theta}}}\text{MSCBE}({\bm{\theta}}) \\
 &= \arg \min_{{\bm{\theta}}} \|\bm{\mathcal{T}}_c^{\pi}\mathcal{C}\bm{V}_{\bm{{\bm{\theta}}}}-\mathcal{C}\bm{V}_{\bm{{\bm{\theta}}}}\|_{\bm{D}}^2\notag\\
 &=\arg \min_{{\bm{\theta}}} \|\bm{\mathcal{T}}^{\pi}\bm{V}_{\bm{{\bm{\theta}}}} - \bm{V}_{\bm{{\bm{\theta}}}}-(\bm{\mathcal{T}}^{\pi}\bm{V}_{\bm{{\bm{\theta}}}} - \bm{V}_{\bm{{\bm{\theta}}}})^{\top}\bm{d}\bm{1}\|_{\bm{D}}^2\notag\\
 &=\arg \min_{{\bm{\theta}},\omega} \| \bm{\mathcal{T}}^{\pi}\bm{V}_{\bm{{\bm{\theta}}}} - \bm{V}_{\bm{{\bm{\theta}}}}-\omega\bm{1} \|_{\bm{D}}^2\notag,
\end{align*}
where $\omega$ is is used to estimate the expected value of the Bellman error.
% where $\omega$ is used to estimate $\mathbb{E}[\delta]$, $\omega \doteq \mathbb{E}[\mathbb{E}[\delta_t|S_t]]=\mathbb{E}[\delta]$ and $\delta_t$ is the TD error as follows:
% \begin{equation}
% \delta_t = r_{t+1}+\gamma
% {\bm{\theta}}_t^{\top}\bm{{\bm{\phi}}}_{t+1}-{\bm{\theta}}_t^{\top}\bm{{\bm{\phi}}}_t.
% \label{delta}
% \end{equation}
% $\mathbb{E}[\delta_t|S_t]$ is the Bellman error, and $\mathbb{E}[\mathbb{E}[\delta_t|S_t]]$ represents the expected value of the Bellman error.
% If $X$ is a random variable and $\mathbb{E}[X]$ is its expected value, then $X-\mathbb{E}[X]$ represents the centered form of $X$. 
% Therefore, we refer to $\mathbb{E}[\delta_t|S_t]-\mathbb{E}[\mathbb{E}[\delta_t|S_t]]$ as Bellman error centering and 
% $\mathbb{E}[(\mathbb{E}[\delta_t|S_t]-\mathbb{E}[\mathbb{E}[\delta_t|S_t]])^2]$ represents the the mean squared centered Bellman error, namely MSCBE.
% The meaning of (\ref{argminMSBEC}) is to minimize the mean squared centered Bellman error.
%The derivation of CTD is as follows.

First, the parameter  $\omega$ is derived directly based on
stochastic gradient descent:
\begin{equation}
\omega_{t+1}= \omega_{t}+\beta_t(\delta_t-\omega_t).
\label{omega}
\end{equation}

Then, based on stochastic semi-gradient descent, the update of 
the parameter ${\bm{\theta}}$ is as follows:
\begin{equation}
{\bm{\theta}}_{t+1}=
{\bm{\theta}}_{t}+\alpha_t(\delta_t-\omega_t)\bm{{\bm{\phi}}}_t.
\label{theta}
\end{equation}

We call (\ref{omega}) and (\ref{theta}) the on-policy centered
TD (CTD) algorithm. The convergence analysis with be given in
the following section.

In off-policy learning, we can simply multiply by the importance sampling
 $\rho$.
\begin{equation}
    \omega_{t+1}=\omega_{t}+\beta_t\rho_t(\delta_t-\omega_t),
    \label{omegawithrho}
\end{equation}
\begin{equation}
    {\bm{\theta}}_{t+1}=
    {\bm{\theta}}_{t}+\alpha_t\rho_t(\delta_t-\omega_t)\bm{{\bm{\phi}}}_t.
    \label{thetawithrho}
\end{equation}

We call (\ref{omegawithrho}) and (\ref{thetawithrho}) the off-policy centered
TD (CTD) algorithm.

% By substituting $\delta_t$ into Equations (\ref{omegawithrho}) and (\ref{thetawithrho}), 
% we can see that Equations (\ref{thetawithrho}) and (\ref{omegawithrho}) are formally identical 
% to the linear expressions of Equations (\ref{rewardcentering1}) and (\ref{rewardcentering2}), respectively. However, the meanings 
% of the corresponding parameters are entirely different.
% ${\bm{\theta}}_t$ is for approximating the discounted value function.
% $\bar{r_t}$ is an estimate of the average reward, while $\omega_t$ 
% is an estimate of the expected value of the Bellman error.
% $\bar{\delta_t}$ is the TD error for value-based reward centering, 
% whereas $\delta_t$ is the traditional TD error.

% This study posits that the CTD is equivalent to value-based reward 
% centering. However, CTD can be unified under a single framework 
% through an objective function, MSCBE, which also lays the 
% foundation for proving the algorithm's convergence. 
% Section 4 demonstrates that the CTD algorithm guarantees 
% convergence in the on-policy setting.

\subsection{Off-policy Centered TDC Algorithm with Linear Value Function Approximation}
The convergence of the  off-policy centered TD algorithm
may not be guaranteed.

To deal with this problem, we propose another new objective function, 
called mean squared projected centered Bellman error (MSPCBE), 
and derive Centered TDC algorithm (CTDC).

% We first establish some relationships between
%  the vector-matrix quantities and the relevant statistical expectation terms:
% \begin{align*}
%     &\mathbb{E}[(\delta({\bm{\theta}})-\mathbb{E}[\delta({\bm{\theta}})]){\bm{\phi}}] \\
%     &= \sum_s \mu(s) {\bm{\phi}}(s) \big( R(s) + \gamma \sum_{s'} P_{ss'} V_{\bm{\theta}}(s') - V_{\bm{\theta}}(s)  \\
%     &\quad \quad-\sum_s \mu(s)(R(s) + \gamma \sum_{s'} P_{ss'} V_{\bm{\theta}}(s') - V_{\bm{\theta}}(s))\big)\\
%     &= \bm{\Phi}^\top \mathbf{D} (\bm{TV}_{\bm{{\bm{\theta}}}} - \bm{V}_{\bm{{\bm{\theta}}}}-\omega\bm{1}),
% \end{align*}
% where $\omega$ is the expected value of the Bellman error and $\bm{1}$ is all-ones vector.

The specific expression of the objective function 
MSPCBE is as follows:
\begin{align}
    \label{MSPBECwithomega}
    &\arg \min_{{\bm{\theta}}}\text{MSPCBE}({\bm{\theta}})\notag\\ 
    % &= \arg \min_{{\bm{\theta}}}\big(\mathbb{E}[(\delta({\bm{\theta}}) - \mathbb{E}[\delta({\bm{\theta}})]) \bm{{\bm{\phi}}}]^\top \notag\\
    % &\quad \quad \quad\mathbb{E}[\bm{{\bm{\phi}}} \bm{{\bm{\phi}}}^\top]^{-1} \mathbb{E}[(\delta({\bm{\theta}}) - \mathbb{E}[\delta({\bm{\theta}})]) \bm{{\bm{\phi}}}]\big) \notag\\
    % &=\arg \min_{{\bm{\theta}},\omega}\mathbb{E}[(\delta({\bm{\theta}})-\omega) \bm{\bm{{\bm{\phi}}}}]^{\top} \mathbb{E}[\bm{\bm{{\bm{\phi}}}} \bm{\bm{{\bm{\phi}}}}^{\top}]^{-1}\mathbb{E}[(\delta({\bm{\theta}}) -\omega)\bm{\bm{{\bm{\phi}}}}]\\
    % &= \big(\bm{\Phi}^\top \mathbf{D} (\bm{TV}_{\bm{{\bm{\theta}}}} - \bm{V}_{\bm{{\bm{\theta}}}}-\omega\bm{1})\big)^\top (\bm{\Phi}^\top \mathbf{D} \bm{\Phi})^{-1} \notag\\
    % & \quad \quad \quad \bm{\Phi}^\top \mathbf{D} (\bm{TV}_{\bm{{\bm{\theta}}}} - \bm{V}_{\bm{{\bm{\theta}}}}-\omega\bm{1}) \notag\\
    % &= (\bm{TV}_{\bm{{\bm{\theta}}}} - \bm{V}_{\bm{{\bm{\theta}}}}-\omega\bm{1})^\top \mathbf{D} \bm{\Phi} (\bm{\Phi}^\top \mathbf{D} \bm{\Phi})^{-1} \notag\\
    % &\quad \quad \quad \bm{\Phi}^\top \mathbf{D} (\bm{TV}_{\bm{{\bm{\theta}}}} - \bm{V}_{\bm{{\bm{\theta}}}}-\omega\bm{1})\notag\\
    % &= (\bm{TV}_{\bm{{\bm{\theta}}}} - \bm{V}_{\bm{{\bm{\theta}}}}-\omega\bm{1})^\top {\bm{\Pi}}^\top \mathbf{D} {\bm{\Pi}} (\bm{TV}_{\bm{{\bm{\theta}}}} - \bm{V}_{\bm{{\bm{\theta}}}}-\omega\bm{1}) \notag\\
    &= \arg \min_{{\bm{\theta}}} \|\bm{\Pi}\bm{\mathcal{T}}_c^{\pi}\mathcal{C}\bm{V}_{\bm{{\bm{\theta}}}}-\mathcal{C}\bm{V}_{\bm{{\bm{\theta}}}}\|_{\bm{D}}^2\notag\\
    &= \arg \min_{{\bm{\theta}}} \|\bm{\Pi}(\bm{\mathcal{T}}_c^{\pi}\mathcal{C}\bm{V}_{\bm{{\bm{\theta}}}}-\mathcal{C}\bm{V}_{\bm{{\bm{\theta}}}})\|_{\bm{D}}^2\notag\\
    &= \arg \min_{{\bm{\theta}},\omega}\| {\bm{\Pi}} (\bm{\mathcal{T}}^{\pi}\bm{V}_{\bm{{\bm{\theta}}}} - \bm{V}_{\bm{{\bm{\theta}}}}-\omega\bm{1}) \|_{\bm{D}}^2\notag.
\end{align}
In the process of computing the gradient of the MSPCBE with respect to ${\bm{\theta}}$, 
$\omega$ is treated as a constant.
So, the derivation process of CTDC is the same 
as for the TDC algorithm \cite{sutton2009fast}, the only difference is that the original $\delta$ is replaced by $\delta-\omega$.
Therefore, the updated formulas of the centered TDC  algorithm are as follows:
\begin{equation}
 \bm{{\bm{\theta}}}_{k+1}=\bm{{\bm{\theta}}}_{k}+\alpha_{k}[(\delta_{k}- \omega_k) \bm{\bm{{\bm{\phi}}}}_k\\
 - \gamma\bm{\bm{{\bm{\phi}}}}_{k+1}(\bm{\bm{{\bm{\phi}}}}^{\top}_k \bm{u}_{k})],
\label{thetavmtdc}
\end{equation}
\begin{equation}
 \bm{u}_{k+1}= \bm{u}_{k}+\zeta_{k}[\delta_{k}-\omega_k - \bm{\bm{{\bm{\phi}}}}^{\top}_k \bm{u}_{k}]\bm{\bm{{\bm{\phi}}}}_k,
\label{uvmtdc}
\end{equation}
and
\begin{equation}
 \omega_{k+1}= \omega_{k}+\beta_k (\delta_k- \omega_k).
 \label{omegavmtdc}
\end{equation}
This algorithm is derived to work 
with a given set of sub-samples—in the form of 
triples $(S_k, R_k, S'_k)$ that match transitions 
from both the behavior and target policies. 

% \subsection{Variance Minimization ETD Learning: VMETD}
% Based on the off-policy TD algorithm, a scalar, $F$,  
% is introduced to obtain the ETD algorithm, 
% which ensures convergence under off-policy 
% conditions. This paper further introduces a scalar, 
% $\omega$, based on the ETD algorithm to obtain VMETD.
% VMETD by the following update:
% \begin{equation}
% \label{fvmetd}
%  F_t \leftarrow \gamma \rho_{t-1}F_{t-1}+1,
% \end{equation}
% \begin{equation}
%  \label{thetavmetd}
%  {{\bm{\theta}}}_{t+1}\leftarrow {{\bm{\theta}}}_t+\alpha_t (F_t \rho_t\delta_t - \omega_{t}){\bm{{\bm{\phi}}}}_t,
% \end{equation}
% \begin{equation}
%  \label{omegavmetd}
%  \omega_{t+1} \leftarrow \omega_t+\beta_t(F_t  \rho_t \delta_t - \omega_t),
% \end{equation}
% where $\rho_t =\frac{\pi(A_t | S_t)}{\mu(A_t | S_t)}$ and $\omega$ is used to estimate $\mathbb{E}[F \rho\delta]$, i.e., $\omega \doteq \mathbb{E}[F \rho\delta]$.

% (\ref{thetavmetd}) can be rewritten as
% \begin{equation*}
%  \begin{array}{ccl}
%  {{\bm{\theta}}}_{t+1}&\leftarrow& {{\bm{\theta}}}_t+\alpha_t (F_t \rho_t\delta_t - \omega_t){\bm{{\bm{\phi}}}}_t -\alpha_t \omega_{t+1}{\bm{{\bm{\phi}}}}_t\\
%   &=&{{\bm{\theta}}}_{t}+\alpha_t(F_t\rho_t\delta_t-\mathbb{E}_{\mu}[F_t\rho_t\delta_t|{{\bm{\theta}}}_t]){\bm{{\bm{\phi}}}}_t\\
%  &=&{{\bm{\theta}}}_t+\alpha_t F_t \rho_t (r_{t+1}+\gamma {{\bm{\theta}}}_t^{\top}{\bm{{\bm{\phi}}}}_{t+1}-{{\bm{\theta}}}_t^{\top}{\bm{{\bm{\phi}}}}_t){\bm{{\bm{\phi}}}}_t\\
%  & & \hspace{2em} -\alpha_t \mathbb{E}_{\mu}[F_t \rho_t \delta_t]{\bm{{\bm{\phi}}}}_t\\
%  &=& {{\bm{\theta}}}_t+\alpha_t \{\underbrace{(F_t\rho_tr_{t+1}-\mathbb{E}_{\mu}[F_t\rho_t r_{t+1}]){\bm{{\bm{\phi}}}}_t}_{{b}_{\text{VMETD},t}}\\
%  &&\hspace{-7em}- \underbrace{(F_t\rho_t{\bm{{\bm{\phi}}}}_t({\bm{{\bm{\phi}}}}_t-\gamma{\bm{{\bm{\phi}}}}_{t+1})^{\top}-{\bm{{\bm{\phi}}}}_t\mathbb{E}_{\mu}[F_t\rho_t ({\bm{{\bm{\phi}}}}_t-\gamma{\bm{{\bm{\phi}}}}_{t+1})]^{\top})}_{\textbf{A}_{\text{VMETD},t}}{{\bm{\theta}}}_t\}.
%  \end{array}
% \end{equation*}
% Therefore, 
% \begin{equation*}
%  \begin{array}{ccl}
%   &&\textbf{A}_{\text{VMETD}}\\
%   &=&\lim_{t \rightarrow \infty} \mathbb{E}[\textbf{A}_{\text{VMETD},t}]\\
%   &=& \lim_{t \rightarrow \infty} \mathbb{E}_{\mu}[F_t \rho_t {\bm{{\bm{\phi}}}}_t ({\bm{{\bm{\phi}}}}_t - \gamma {\bm{{\bm{\phi}}}}_{t+1})^{\top}]\\  
%   &&\hspace{1em}- \lim_{t\rightarrow \infty} \mathbb{E}_{\mu}[  {\bm{{\bm{\phi}}}}_t]\mathbb{E}_{\mu}[F_t \rho_t ({\bm{{\bm{\phi}}}}_t - \gamma {\bm{{\bm{\phi}}}}_{t+1})]^{\top}\\
%   &=& \lim_{t \rightarrow \infty} \mathbb{E}_{\mu}[{\bm{{\bm{\phi}}}}_tF_t \rho_t ({\bm{{\bm{\phi}}}}_t - \gamma {\bm{{\bm{\phi}}}}_{t+1})^{\top}]\\   
%   &&\hspace{1em}-\lim_{t \rightarrow \infty} \mathbb{E}_{\mu}[ {\bm{{\bm{\phi}}}}_t]\lim_{t \rightarrow \infty}\mathbb{E}_{\mu}[F_t \rho_t ({\bm{{\bm{\phi}}}}_t - \gamma {\bm{{\bm{\phi}}}}_{t+1})]^{\top}\\
%   && \hspace{-2em}=\sum_{s} d_{\mu}(s)\lim_{t \rightarrow \infty}\mathbb{E}_{\mu}[F_t|S_t = s]\mathbb{E}_{\mu}[\rho_t\bm{{\bm{\phi}}}_t(\bm{{\bm{\phi}}}_t - \gamma \bm{{\bm{\phi}}}_{t+1})^{\top}|S_t= s]\\   
%   &&\hspace{1em}-\sum_{s} d_{\mu}(s)\bm{{\bm{\phi}}}(s)\sum_{s} d_{\mu}(s)\lim_{t \rightarrow \infty}\mathbb{E}_{\mu}[F_t|S_t = s]\\
%   &&\hspace{7em}\mathbb{E}_{\mu}[\rho_t(\bm{{\bm{\phi}}}_t - \gamma \bm{{\bm{\phi}}}_{t+1})^{\top}|S_t = s]\\
%   &=& \sum_{s} f(s)\mathbb{E}_{\pi}[\bm{{\bm{\phi}}}_t(\bm{{\bm{\phi}}}_t- \gamma \bm{{\bm{\phi}}}_{t+1})^{\top}|S_t = s]\\   
%   &&\hspace{1em}-\sum_{s} d_{\mu}(s)\bm{{\bm{\phi}}}(s)\sum_{s} f(s)\mathbb{E}_{\pi}[(\bm{{\bm{\phi}}}_t- \gamma \bm{{\bm{\phi}}}_{t+1})^{\top}|S_t = s]\\
%   &=&\sum_{s} f(s) \bm{\bm{{\bm{\phi}}}}(s)(\bm{\bm{{\bm{\phi}}}}(s) - \gamma \sum_{s'}[\textbf{P}_{\pi}]_{ss'}\bm{\bm{{\bm{\phi}}}}(s'))^{\top}  \\
%   &&-\sum_{s} d_{\mu}(s) {\bm{{\bm{\phi}}}}(s) * \sum_{s} f(s)({\bm{{\bm{\phi}}}}(s) - \gamma \sum_{s'}[\textbf{P}_{\pi}]_{ss'}{\bm{{\bm{\phi}}}}(s'))^{\top}\\
%   &=&{\bm{\bm{\Phi}}}^{\top} \textbf{F} (\textbf{I} - \gamma \textbf{P}_{\pi}) \bm{\bm{\Phi}} - {\bm{\bm{\Phi}}}^{\top} {d}_{\mu} {f}^{\top} (\textbf{I} - \gamma \textbf{P}_{\pi}) \bm{\bm{\Phi}}  \\
%   &=&{\bm{\bm{\Phi}}}^{\top} (\textbf{F} - {d}_{\mu} {f}^{\top}) (\textbf{I} - \gamma \textbf{P}_{\pi}){\bm{\bm{\Phi}}} \\
%   &=&{\bm{\bm{\Phi}}}^{\top} (\textbf{F} (\textbf{I} - \gamma \textbf{P}_{\pi})-{d}_{\mu} {f}^{\top} (\textbf{I} - \gamma \textbf{P}_{\pi})){\bm{\bm{\Phi}}} \\
%   &=&{\bm{\bm{\Phi}}}^{\top} (\textbf{F} (\textbf{I} - \gamma \textbf{P}_{\pi})-{d}_{\mu} {d}_{\mu}^{\top} ){\bm{\bm{\Phi}}},
%  \end{array}
% \end{equation*}
% \begin{equation*}
%  \begin{array}{ccl}
%   &&{b}_{\text{VMETD}}\\
%   &=&\lim_{t \rightarrow \infty} \mathbb{E}[{b}_{\text{VMETD},t}]\\
%   &=& \lim_{t \rightarrow \infty} \mathbb{E}_{\mu}[F_t\rho_tR_{t+1}{\bm{{\bm{\phi}}}}_t]\\
%   &&\hspace{2em} - \lim_{t\rightarrow \infty} \mathbb{E}_{\mu}[{\bm{{\bm{\phi}}}}_t]\mathbb{E}_{\mu}[F_t\rho_kR_{k+1}]\\  
%   &=& \lim_{t \rightarrow \infty} \mathbb{E}_{\mu}[{\bm{{\bm{\phi}}}}_tF_t\rho_tr_{t+1}]\\
%   &&\hspace{2em} - \lim_{t\rightarrow \infty} \mathbb{E}_{\mu}[  {\bm{{\bm{\phi}}}}_t]\mathbb{E}_{\mu}[{\bm{{\bm{\phi}}}}_t]\mathbb{E}_{\mu}[F_t\rho_tr_{t+1}]\\ 
%   &=& \lim_{t \rightarrow \infty} \mathbb{E}_{\mu}[{\bm{{\bm{\phi}}}}_tF_t\rho_tr_{t+1}]\\
%   &&\hspace{2em} - \lim_{t \rightarrow \infty} \mathbb{E}_{\mu}[ {\bm{{\bm{\phi}}}}_t]\lim_{t \rightarrow \infty}\mathbb{E}_{\mu}[F_t\rho_tr_{t+1}]\\  
%   &=&\sum_{s} f(s) {\bm{{\bm{\phi}}}}(s)r_{\pi} - \sum_{s} d_{\mu}(s) {\bm{{\bm{\phi}}}}(s) * \sum_{s} f(s)r_{\pi}  \\
%   &=&\bm{\bm{\bm{\Phi}}}^{\top}(\textbf{F}-{d}_{\mu} {f}^{\top}){r}_{\pi}.
%  \end{array}
% \end{equation*}


\section{Methodology}
\label{sec:meth}

\subsection{Baseline ZS-CIR Framework}
Composed Image Retrieval (CIR) enables users to search for target images $I_{target}$ by providing a reference image, $I_{ref}$, and a text prompt, $P$, describing desired modifications. Zero-shot composed image retrieval (ZS-CIR) leverages Vision-Language (VL) models, $\Psi$, such as CLIP \cite{radford2021learning}, whose vision branch, $\Psi_{I}$, and text branch, $\Psi_{T}$, are trained to learn a shared embedding space where semantically similar image and text pairs are mapped close to each other. In this framework, as show in Figure \ref{fig:motiv}\red{a}, target images are encoded using the vision branch, $\Phi_{I}$, while the query is composed by processing both $I_{ref}$ and $P$ through the text branch $\Psi_{T}$, as composition operations are more naturally handled in the text modality. 

Recent ZS-CIR approaches generate the composed text embedding from $I_{ref}$ and $P$ using one of two methods: direct image captioning (CIReVL, LDRE and SEIZE) or pseudo tokenization (Pic2Word, LinCIR, SEARLE and KEDs). We denote this composition process as $\mathcal{F}$, resulting in a composed text embedding $\Psi_T(\mathcal{F}(I_{ref}, P))$.

In an ideal ZS-CIR scenario, the target image $I_{target}$ should appear within the top-k results retrieved from the gallery $\mathcal{D}$. This retrieval is formalized as:
\begin{equation}
\label{eq:zscir}
\mathbb{I}_{top-k} = \argmax_{I \in \mathcal{D}} \frac{\Psi_T(\mathcal{F}(I_{ref}, P))^T \cdot \Psi_I(I)}{\lVert\Psi_T(\mathcal{F}(I_{ref}, P))\rVert \cdot \lVert\Psi_I(I)\rVert}.
\end{equation}
If $I_{target} \notin \mathbb{I}_{top-k}$, the user must reformulate the prompt and repeat the feature extraction process to obtain alternative retrieval results, incurring time and computational resource costs. Notably, as shown in Eq. \ref{eq:zscir}, only the composed feature embedding $\Psi_T(\mathcal{F}(I_{ref}, P))$ directly influences the computation of $\mathbb{I}_{top-k}$ results. Although the gallery images are represented by their image embeddings, the image embedding of the reference image $\Psi_I(I_{ref})$ does not contribute to the retrieval process.



\subsection{Our Approach: Prompt Directional Vector}

Rather than simply employing the composed embedding alone, $\Psi_T(\mathcal{F}(I_{ref}, P))$, as depicted in Figure \ref{fig:motiv}\red{b},
we propose a generalized formulation of composed text embeddings by considering the embedding modification direction, $\Delta_{PDV}$, which is derived from the difference between the provided prompt, $P$, and the reference image, $I_{ref}$. Formally, we define $\Delta_{PDV}$ as,
\begin{equation}
\Delta_{PDV} = \Psi_T(\mathcal{F}(I_{ref}, P)) - \Psi_T(\mathcal{F}(I_{ref})).
\label{eq:pdv}
\end{equation}
We then form the composed text embedding as follows,
\begin{equation}
	\Psi_T(\mathcal{F}(I_{ref}, P)) = \Psi_T(\mathcal{F}(I_{ref})) + \alpha_T\Delta_{PDV}, \label{eqn:text_embedding}
\end{equation}
{\noindent}where $\alpha$ controls the movement along the prompt vector $\Delta_{PDV}$ and $\Psi_T(\mathcal{F}(I_{ref}))$ is the original text embedding.

\subsection{Strategies for Using PDV}

We explore three strategies for using $\Delta_{PDV}$:

\textbf{(1) Prompt Directional Vector for Text (PDV-T)}, which enhances controllability in ZS-CIR. While baseline ZS-CIR approaches represent a special case where $\alpha=1$, varying $\alpha$ provides users with additional control over the retrieval process (refer to Figure \ref{fig:motiv}\red{c}). Setting $\alpha>1$ amplifies the modification specified by the prompt, while $\alpha<1$ reduces its effect. This approach offers a more efficient alternative to modifying the prompt directly, as it requires neither new feature extraction nor prompt reformulation. Note that we use the notation $\Phi_{PDV-T}$ to represent the composed text embedding.

\begin{figure}[!tbh]
\vspace*{-0.4cm}
\centerline{
\includegraphics[width=1.1\linewidth]{images/CIR_vs_I+T.drawio.pdf}}
	\caption{Comparison of Image + Text (a) vs PDV (b).}
	\label{fig:vtvspv}
\end{figure}

\textbf{(2) Prompt  Directional Vector for Image (PDV-I)}, which extends the modification principle to visual embeddings. While previous approaches primarily relied on composed text embeddings, experimental results show that direct fusion of image and text features yields inferior performance compared to composed features. We hypothesize that this performance gap arises because the direct text embedding, $\Phi_{T}(P)$, differs significantly from the prompt vector $\Delta_{PDV}$, as illustrated in Figure \ref{fig:vtvspv}. This difference occurs because the semantic meaning of natural language is context-sensitive, where in our case the context is provided by the reference image embedding $\Psi_T(\mathcal{F}(I_{ref}))$. To address this limitation, we propose combining $\Delta_{PDV}$ with visual embeddings. Specifically, we compute the composed visual embedding $\Phi_{PDV-I}$
as $\Psi_{I}(I_{ref}) + \alpha_I\Delta_{PDV}$, where $\Psi_{I}(I_{ref})$ represents the original visual embedding obtained from the reference image, and the same prompt vector obtained via Eq. \ref{eq:pdv} is used to modify this visual representation.

\textbf{(3) Prompt  Directional Vector Fusion (PDV-F)}, which calculates the final similarity score between a query and target image which combines both composed embeddings. This fusion embedding, $\Phi_{PDV-F}$, can be defined as,
\begin{equation}
	\Phi_{PDV-F} = (1-\beta)\Phi_{PDV-I} +\beta\Phi_{PDV-T},
\end{equation}
where $\beta$ is a weighting parameter balancing the contribution of the composed visual and textual embeddings.








\newcommand{\tabincell}[2]{\begin{tabular}{@{}#1@{}}#2\end{tabular}}
\newcommand{\rowstyle}[1]{\gdef\currentrowstyle{#1}%
	#1\ignorespaces
}

\newcommand{\className}[1]{\textbf{\sf #1}}
\newcommand{\functionName}[1]{\textbf{\sf #1}}
\newcommand{\objectName}[1]{\textbf{\sf #1}}
\newcommand{\annotation}[1]{\textbf{#1}}
\newcommand{\todo}[1]{\textcolor{blue}{\textbf{[[TODO: #1]]}}}
\newcommand{\change}[1]{\textcolor{blue}{#1}}
\newcommand{\fetch}[1]{\textbf{\em #1}}
\newcommand{\phead}[1]{\vspace{1mm} \noindent {\bf #1}}
\newcommand{\wei}[1]{\textcolor{blue}{{\it [Wei says: #1]}}}
\newcommand{\peter}[1]{\textcolor{red}{{\it [Peter says: #1]}}}
\newcommand{\zhenhao}[1]{\textcolor{dkblue}{{\it [Zhenhao says: #1]}}}
\newcommand{\feng}[1]{\textcolor{magenta}{{\it [Feng says: #1]}}}
\newcommand{\jinqiu}[1]{\textcolor{red}{{\it [Jinqiu says: #1]}}}
\newcommand{\shouvick}[1]{\textcolor{violet(ryb)}{{\it [Shouvick says: #1]}}}
\newcommand{\pattern}[1]{\emph{#1}}
%\newcommand{\tool}{{{DectGUILag}}\xspace}
\newcommand{\tool}{{{GUIWatcher}}\xspace}


\newcommand{\guo}[1]{\textcolor{yellow}{{\it [Linqiang says: #1]}}}

\newcommand{\rqbox}[1]{\begin{tcolorbox}[left=4pt,right=4pt,top=4pt,bottom=4pt,colback=gray!5,colframe=gray!40!black,before skip=2pt,after skip=2pt]#1\end{tcolorbox}}



\section{Implementation and Evaluation}
\label{sec:evaluation}

We prototype our proposal into a tool \toolName, using approximately 5K lines of OCaml (for the program analysis) and 5K lines of Python code (for the repair). 
In particular, we employ Z3~\cite{DBLP:conf/tacas/MouraB08} as the SMT solver, clingo~\cite{DBLP:books/sp/Lifschitz19} as the ASP solver, and Souffle~\cite{scholz2016fast} as the Datalog engine. %, respectively.
To show the effectiveness, 
we design the experimental evaluation to answer the 
following research questions (RQ):
(Experiments ran on a server with an Intel® Xeon® Platinum 8468V, 504GB RAM, and 192 cores. All the dataset are publicly available from \cite{zenodo_benchmark})

\begin{itemize}[align=left, leftmargin=*,labelindent=0pt]
\item \textbf{RQ1:} How effective is \toolName in verifying CTL properties for relatively small but complex programs, compared to the state-of-the-art tool  \function~\cite{DBLP:conf/sas/UrbanU018}?


\item \textbf{RQ2:} What is the effectiveness of \toolName in detecting real-world bugs, which can be encoded using both CTL and linear temporal logic (LTL), such as non-termination gathered from GitHub \cite{DBLP:conf/sigsoft/ShiXLZCL22} and unresponsive behaviours in protocols  \cite{DBLP:conf/icse/MengDLBR22}, compared with \ultimate~\cite{DBLP:conf/cav/DietschHLP15}?

\item \textbf{RQ3:} How effective is \toolName in repairing CTL violations identified in RQ1 and RQ2? which has not been achieved by any existing tools. 


 

\end{itemize}



% \begin{itemize}[align=left, leftmargin=*,labelindent=0pt]
% \item \textbf{RQ1:} What is the effectiveness of \toolName in verifying CTL properties in a set of relatively small yet challenging programs, compared to the state-of-the-art tools, T2~\cite{DBLP:conf/fmcad/CookKP14},  \function~\cite{DBLP:conf/sas/UrbanU018}, and \ultimate~\cite{DBLP:conf/cav/DietschHLP15}?


% \item \textbf{RQ2:} What is the effectiveness of \toolName in finding  real-world bugs, which can be encoded using CTL properties, such as non-termination 
% gathered from GitHub \cite{DBLP:conf/sigsoft/ShiXLZCL22} and unresponsive behaviours in protocol implementations \cite{DBLP:conf/icse/MengDLBR22}?

% \item \textbf{RQ3:} What is the effectiveness of \toolName in repairing CTL bugs from RQ1--2?

% \end{itemize}

%The benchmark programs are from various sources. More specifically, termination bugs from real-world projects \cite{DBLP:conf/sigsoft/ShiXLZCL22} and CTL analysis \cite{DBLP:conf/fmcad/CookKP14} \cite{DBLP:conf/sas/UrbanU018}, and temporal bugs in real-world protocol implementations \cite{DBLP:conf/icse/MengDLBR22}. 



% \ly{are termination bugs ok? Do we need to add new CTL bugs?}
\subsection{RQ1: Verifying CTL Properties}

% Please add the following required packages to your document preamble:
%  \Xhline{1.5\arrayrulewidth}

\hide{\begin{figure}[!h]
\vspace{-8mm}
\begin{lstlisting}[xleftmargin=0.2em,numbersep=6pt,basicstyle=\footnotesize\ttfamily]
(*@\textcolor{mGray}{//$EF(\m{resp}{\geq}5)$}@*)
int c = *; int resp = 0;
int curr_serv = 5; 
while (curr_serv > 0){ 
 if (*) {  
   c--; 
   curr_serv--;
   resp++;} 
 else if (c<curr_serv){
   curr_serv--; }}
\end{lstlisting} 
\vspace{-2mm}
\caption{A possibly terminating loop} 
\label{fig:terminating_loop}
\vspace{-2mm}
\end{figure}}


%loses precision due to a \emph{dual widening} \cite{DBLP:conf/tacas/CourantU17}, and 

The programs listed in \tabref{tab:comparewithFuntionT2} were obtained from the evaluation benchmark of \function, which includes a total of 83 test cases across over 2,000 lines of code. We categorize these test cases into six groups, labeled according to the types of CTL properties. 
These programs are short but challenging, as they often involve complex loops or require a more precise analysis of the target properties. The \function tends to be conservative, often leading it to return ``unknown" results, resulting in an accuracy rate of 27.7\%. In contrast, \toolName demonstrates advantages with improved accuracy, particularly in \ourToolSmallBenchmark. 
%achieved by the novel loop summaries. 
The failure cases faced by \toolName highlight our limitations when loop guards are not explicitly defined or when LRFs are inadequate to prove termination. 
Although both \function and \toolName struggle to obtain meaningful invariances for infinite loops, the benefits of our loop summaries become more apparent when proving properties related to termination, such as reachability and responsiveness.  




\begin{table}[!t]
\vspace{1.5mm}
\caption{Detecting real-world CTL bugs.}
\normalsize
\label{tab:comparewithCook}
\renewcommand{\arraystretch}{0.95}
\setlength{\tabcolsep}{4pt}  
\begin{tabular}{c|l|c|cc|cc}
\Xhline{1.5\arrayrulewidth}
\multicolumn{1}{l|}{\multirow{2}{*}{\textbf{}}} & \multirow{2}{*}{\textbf{Program}}        & \multirow{2}{*}{\textbf{LoC}} & \multicolumn{2}{c|}{\textbf{\ultimateshort}}   & \multicolumn{2}{c}{\textbf{\toolName}}             \\ \cline{4-7} 
  \multicolumn{1}{l|}{}                           &                                          &                               & \multicolumn{1}{c|}{\textbf{Res.}} & \textbf{Time} & \multicolumn{1}{c|}{\textbf{Res.}} & \textbf{Time} \\ \hline
  1 \xmark                                      & \multirow{2}{*}{\makecell[l]{libvncserver\\(c311535)}}   & 25                            & \multicolumn{1}{c|}{\xmark}      & 2.845         & \multicolumn{1}{c|}{\xmark}      & 0.855         \\  
  1 \cmark                                      &                                          & 27                            & \multicolumn{1}{c|}{\cmark}      & 3.743         & \multicolumn{1}{c|}{\cmark}      & 0.476         \\ \hline
  2 \xmark                                      & \multirow{2}{*}{\makecell[l]{Ffmpeg\\(a6cba06)}}         & 40                            & \multicolumn{1}{c|}{\xmark}      & 15.254        & \multicolumn{1}{c|}{\xmark}      & 0.606         \\  
  2 \cmark                                      &                                          & 44                            & \multicolumn{1}{c|}{\cmark}      & 40.176        & \multicolumn{1}{c|}{\cmark}      & 0.397         \\ \hline
  3 \xmark                                      & \multirow{2}{*}{\makecell[l]{cmus\\(d5396e4)}}           & 87                            & \multicolumn{1}{c|}{\xmark}      & 6.904         & \multicolumn{1}{c|}{\xmark}      & 0.579         \\  
  3 \cmark                                      &                                          & 86                            & \multicolumn{1}{c|}{\cmark}      & 33.572        & \multicolumn{1}{c|}{\cmark}      & 0.986         \\ \hline
  4 \xmark                                      & \multirow{2}{*}{\makecell[l]{e2fsprogs\\(caa6003)}}      & 58                            & \multicolumn{1}{c|}{\xmark}      & 5.952         & \multicolumn{1}{c|}{\xmark}      & 0.923         \\  
  4 \cmark                                      &                                          & 63                            & \multicolumn{1}{c|}{\cmark}      & 4.533         & \multicolumn{1}{c|}{\cmark}      & 0.842         \\ \hline
  5 \xmark                                      & \multirow{2}{*}{\makecell[l]{csound-an...\\(7a611ab)}} & 43                            & \multicolumn{1}{c|}{\xmark}      & 3.654         & \multicolumn{1}{c|}{\xmark}      & 0.782         \\  
  5 \cmark                                      &                                          & 45                            & \multicolumn{1}{c|}{TO}          & -             & \multicolumn{1}{c|}{\cmark}      & 0.648         \\ \hline
  6 \xmark                                      & \multirow{2}{*}{\makecell[l]{fontconfig\\(fa741cd)}}     & 25                            & \multicolumn{1}{c|}{\xmark}      & 3.856         & \multicolumn{1}{c|}{\xmark}      & 0.769         \\  
  6 \cmark                                      &                                          & 25                            & \multicolumn{1}{c|}{Error}       & -             & \multicolumn{1}{c|}{\cmark}      & 0.651         \\ \hline
  7 \xmark                                      & \multirow{2}{*}{\makecell[l]{asterisk\\(3322180)}}       & 22                            & \multicolumn{1}{c|}{\unk}        & 12.687        & \multicolumn{1}{c|}{\unk}        & 0.196         \\  
  7 \cmark                                      &                                          & 25                            & \multicolumn{1}{c|}{\unk}        & 11.325        & \multicolumn{1}{c|}{\unk}        & 0.34          \\ \hline
  8 \xmark                                      & \multirow{2}{*}{\makecell[l]{dpdk\\(cd64eeac)}}          & 45                            & \multicolumn{1}{c|}{\xmark}      & 3.712         & \multicolumn{1}{c|}{\xmark}      & 0.447         \\  
  8 \cmark                                      &                                          & 45                            & \multicolumn{1}{c|}{\cmark}      & 2.97          & \multicolumn{1}{c|}{\unk}        & 0.481         \\ \hline
  9 \xmark                                      & \multirow{2}{*}{\makecell[l]{xorg-server\\(930b9a06)}}   & 19                            & \multicolumn{1}{c|}{\xmark}      & 3.111         & \multicolumn{1}{c|}{\xmark}      & 0.581         \\  
  9 \cmark                                      &                                          & 20                            & \multicolumn{1}{c|}{\cmark}      & 3.101         & \multicolumn{1}{c|}{\cmark}      & 0.409         \\ \hline
  10 \xmark                                      & \multirow{2}{*}{\makecell[l]{pure-ftpd\\(37ad222)}}      & 42                            & \multicolumn{1}{c|}{\cmark}      & 2.555         & \multicolumn{1}{c|}{\xmark}      & 0.933         \\  
  10 \cmark                                      &                                          & 49                            & \multicolumn{1}{c|}{\cmark}        & 2.286         & \multicolumn{1}{c|}{\cmark}      & 0.383         \\ \hline
  11 \xmark  & \multirow{2}{*}{\makecell[l]{live555$_a$\\(181126)}} & 34  & \multicolumn{1}{c|}{\cmark} &  2.715         & \multicolumn{1}{c|}{\xmark}    & 0.513   \\  
  11 \cmark  &     &   37    & \multicolumn{1}{c|}{\cmark} &  2.837         & \multicolumn{1}{c|}{\cmark}      & 0.341 \\ \hline
  12 \xmark  & \multirow{2}{*}{\makecell[l]{openssl\\(b8d2439)}} & 88  & \multicolumn{1}{c|}{\xmark} &  4.15          & \multicolumn{1}{c|}{\xmark}    & 0.78   \\
  12 \cmark  &     &  88     & \multicolumn{1}{c|}{\cmark} &  3.809         & \multicolumn{1}{c|}{\cmark}      & 0.99 \\ \hline
  13 \xmark  & \multirow{2}{*}{\makecell[l]{live555$_b$\\(131205)}} & 83  & \multicolumn{1}{c|}{\xmark} & 2.838         & \multicolumn{1}{c|}{\xmark}    & 0.602     \\  
  13 \cmark  &    &   84     & \multicolumn{1}{c|}{\cmark} &  2.393         & \multicolumn{1}{c|}{\cmark}      & 0.565 \\ \Xhline{1.5\arrayrulewidth}
                                                   & {\bf{Total}}                                  & 1249  & \multicolumn{1}{c|}{\bestBaseLineReal}          & $>$180       & \multicolumn{1}{c|}{\ourToolRealBenchmark}              & 16.01        \\ \Xhline{1.5\arrayrulewidth}
  \end{tabular}
  \end{table}

\subsection{RQ2: CTL Analysis on  Real-world Projects}




Programs in \tabref{tab:comparewithCook} are from real-world repositories, each associated with a Git commit number where developers identify and fix the bug manually. 
In particular, the property used for programs 1-9 (drawn from \cite{DBLP:conf/sigsoft/ShiXLZCL22}) is  \code{AF(Exit())}, capturing non-termination bugs. The properties used for programs 10-13 (drawn from \cite{DBLP:conf/icse/MengDLBR22}) are of the form \code{AG(\phi_1{\rightarrow}AF(\phi_2))}, capturing unresponsive behaviours from the protocol implementation. 
We extracted the main segments of these real-world bugs into smaller programs (under 100 LoC each), preserving features like data structures and pointer arithmetic. Our evaluation includes both buggy (\eg 1\,\xmark) and developer-fixed (\eg 1\,\cmark) versions.
After converting the CTL properties to LTL formulas, we compared our tool with the latest release of UltimateLTL (v0.2.4), a regular participant in SV-COMP \cite{svcomp} with competitive performance. 
Both tools demonstrate high accuracy in bug detection, while \ultimateshort often requires longer processing time. 
This experiment indicates that LRFs can effectively handle commonly seen real-world loops, and \toolName performs a more lightweight summary computation without compromising accuracy. 



%Following the convention in \cite{DBLP:conf/sigsoft/ShiXLZCL22}, t
%Prior works \cite{DBLP:conf/sigsoft/ShiXLZCL22} gathered such examples by extracting 
%\toolName successfully identifies the majority of buggy and correct programs, with the exception of programs 7 and 8. 







{
\begin{table*}[!h]
  \centering
\caption{\label{tab:repair_benchmark}
{Experimental results for repairing CTL bugs. Time spent by the ASP solver is separately recorded. 
}
}
\small
\renewcommand{\arraystretch}{0.95}
  \setlength{\tabcolsep}{9pt}
\begin{tabular}{l|c|c|c|c|c|c|c|c}
  \Xhline{1.5\arrayrulewidth}
  \multicolumn{1}{c|}{\multirow{2}{*}{\textbf{Program}}} & \multicolumn{1}{c|}{\multirow{2}{*}{\shortstack{\textbf{LoC}\\\textbf{(Datalog)}}}} & \multicolumn{3}{c|}{\textbf{Configuration}}                                 & \multicolumn{1}{c|}{\multirow{2}{*}{\textbf{Fixed}}} & \multicolumn{1}{c|}{\multirow{2}{*}{\textbf{\#Patch}}} & \multicolumn{1}{c|}{\multirow{2}{*}{\textbf{ASP(s)}}} & \multirow{2}{*}{\textbf{Total(s)}} \\ \cline{3-5}

  \multicolumn{1}{c|}{}                                  & \multicolumn{1}{c|}{}                              & \multicolumn{1}{c|}{\textbf{Symbols}} & \multicolumn{1}{c|}{\textbf{Facts}} & \multicolumn{1}{c|}{\textbf{Template}} & \multicolumn{1}{c|}{} & \multicolumn{1}{c|}{} & \multicolumn{1}{c|}{}  &                                      \\ \hline

AF\_yEQ5 (\figref{fig:first_Example})                                           & 115                           & 3+0                   & 0+1                & Add                & \cmark     & 1                   & 0.979                              & 1.593                                \\
test\_until.c                                         & 101                            & 0+3                   & 1+0                & Delete                & \cmark     & 1                   & 0.023                              & 0.498                                \\
next.c                                                & 87                            & 0+4                   & 1+0                & Delete                & \cmark     & 1                   & 0.023                              & 0.472                                \\
libvncserver                                          & 118                            & 0+6                   & 1+0                & Delete                & \cmark     & 3                   & 0.049                              & 1.081                                \\
Ffmpeg                                                & 227                           & 0+12                  & 1+0                & Delete                & \cmark     & 4                   & 13.113                              & 13.335                                \\
cmus                                                  & 145                           & 0+12                  & 1+0                & Delete                & \cmark     & 4                   & 0.098                              & 2.052                                \\
e2fsprogs                                             & 109                           & 0+8                   & 1+0                & Delete                & \cmark     & 2                   & 0.075                              & 1.515                                \\
csound-android                                        & 183                           & 0+8                   & 1+0                & Delete                & \cmark     & 4                   & 0.076                              & 1.613                                \\
fontconfig                                            & 190                           & 0+11                  & 1+0                & Delete                & \cmark     & 6                   & 0.098                              & 2.507                                \\
dpdk                                                  & 196                           & 0+12                  & 1+0                & Delete                & \cmark     & 1                   & 0.091                              & 2.006                                \\
xorg-server                                           & 118                            & 0+2                   & 1+0                & Delete                & \cmark     & 2                   & 0.026                              & 0.605                                \\
pure-ftpd                                             & 258                           & 0+21                  & 1+0                & Delete                & \cmark     & 2                   & 0.069                              & 3.590                               \\
live$_a$                                              & 112                            & 3+4                   & 1+1                & Update                & \cmark     & 1                   & 0.552                              & 0.816                                \\
openssl                                               & 315                           & 1+0                   & 0+1                & Add.                & \cmark     & 1                   & 1.188                              & 2.277                                \\
live$_b$                                              & 217                           & 1+0                   & 0+1                & Add                & \cmark     & 1                   & 0.977                              & 1.494                                 \\
  \Xhline{1.5\arrayrulewidth}
\textbf{Total}                                                 & 2491                          &                       &                    &                   &           &                     & 17.437                              & 35.454                               \\ 
  \Xhline{1.5\arrayrulewidth}           
\end{tabular}

\vspace{-2mm}
\end{table*}
}


\subsection{RQ3: Repairing CTL Property Violations} 


\tabref{tab:repair_benchmark} gathers all the program instances (from \tabref{tab:comparewithFuntionT2} and \tabref{tab:comparewithCook}) that violate their specified CTL properties and are sent to \toolName for repair.   
The \textbf{Symbols} column records the number of symbolic constants + symbolic signs, while the \textbf{Facts} column records the number of facts allowed to be removed + added. 
We gradually increase the number of symbols and the maximum number of facts that can be added or deleted. 
The \textbf{Configuration} column shows the first successful configuration that led to finding patches, and we record the total searching time till reaching such configurations. 
We configure \toolName to apply three atomic templates in a breadth-first manner with a depth limit of 1, \ie, \tabref{tab:repair_benchmark} records the patch result after one iteration of the repair. 
The templates are applied sequentially in the order: delete, update, and add. The repair process stops when a correct patch is found or when all three templates have been attempted. 
%without success. 
% Because of this configuration, \toolName only finds one patch for Program 1 (AF\_yEQ5). 
% The patch inserting \plaincode{if (i>10||x==y) \{y=5; return;\}} mentioned in \figref{fig:Patched-program} cannot be found in current configuration, as it requires deleting facts then adding new facts on the updated program.
% The `Configuration' column in \tabref{tab:repair_benchmark} shows the number of symbolic constants and signs, the number of facts allowed to be removed and added, and the template used when a patch is found.

Due to the current configuration, \toolName only finds patch (b) for Program 1 (AF\_yEQ5), while the patch (a) shown in \figref{fig:Patched-program} can be obtained by allowing two iterations of the repair: the first iteration adds the conditional than a second iteration to add a new assignment on the updated program. 
Non-termination bugs are resolved within a single iteration by adding a conditional statement that provides an earlier exit. 
For instance, \figref{fig:term-Patched-program} illustrates the main logic of 1\,\xmark, which enters an infinite loop when \code{\m{linesToRead}{\leq}0}. 
\toolName successfully 
provides a fix that prevents \code{\m{linesToRead}{\leq}0} from occurring before entering the loop. Note that such patches are more desirable which fix the non-termination bug without dropping the loops completely. 
%much like the example shown in  \figref{fig:term-Patched-program}. At the same time, 
Unresponsive bugs involve adding more function calls or assignment modifications. 
%Most repairs were completed within seconds. 

On average, the time taken to solve ASP accounts for 49.2\% (17.437/35.454) of the total repair time. We also keep track of the number of patches that successfully eliminate the CTL violations. More than one patch is available for non-termination bugs, as some patches exit the entire program without entering the loop. 
While all the patches listed are valid, those that intend to cut off the main program logic can be excluded based on the minimum change criteria. 
After a manual inspection of each buggy program shown in \tabref{tab:repair_benchmark}, we confirmed that at least one generated patch is semantically equivalent to the fix provided by the developer. 
As the first tool to achieve automated repair of CTL violations, \toolName successfully resolves all reported bugs. 



\begin{figure}[!t]
\begin{lstlisting}[xleftmargin=6em,numbersep=6pt,basicstyle=\footnotesize\ttfamily]
void main(){ //AF(Exit())
  int lines ToRead = *;
  int h = *;
  (*@\repaircode{if ( linesToRead <= 0 )  return;}@*)
  while(h>0){
    if(linesToRead>h)  
        linesToRead=h; 
    h-=linesToRead;} 
  return;}
\end{lstlisting}
\caption{Fixing a Possible Hang Found in libvncserver \cite{LibVNCClient}}
\label{fig:term-Patched-program}
\end{figure}



\noindent\textbf{Developer Perspective:}
Our study shows that 21.44\% to 26.77\% of newly introduced libraries are abandoned within their first year, with a decreasing survival rate in the following years. We observed slowdown patterns in later stages which could potentially serve as an abandonment flag. Developers should monitor release activity patterns of their dependencies, such as sudden slowdowns and prolonged inactivity in the within life cycle as warning signs of potential abandonment. Developers should consider the survival rate of libraries when selecting dependencies. Additionally, the high abandonment rate after two years emphasizes the need for proper monitoring and preparedness to handle dependency shifts.






\noindent\textbf{Research Perspective:} 
Our findings suggest that release patterns and library lifecycles are important factors to explore. Future research should focus on identifying the indicators that lead to abandonment and developing tools to predict which libraries are at risk. These tools could help developers track the age and release patterns of libraries, allowing them to identify at-risk projects early.


\noindent\textbf{Ecosystem Perspective:}
The steady rate of project abandonment in the Maven ecosystem underscores the need for ecosystem-level interventions to support maintainers and reduce the risk of abandonment. Platforms should consider offering automatic alerts for libraries showing signs of abandonment.

\section{Discussion of Assumptions}\label{sec:discussion}
In this paper, we have made several assumptions for the sake of clarity and simplicity. In this section, we discuss the rationale behind these assumptions, the extent to which these assumptions hold in practice, and the consequences for our protocol when these assumptions hold.

\subsection{Assumptions on the Demand}

There are two simplifying assumptions we make about the demand. First, we assume the demand at any time is relatively small compared to the channel capacities. Second, we take the demand to be constant over time. We elaborate upon both these points below.

\paragraph{Small demands} The assumption that demands are small relative to channel capacities is made precise in \eqref{eq:large_capacity_assumption}. This assumption simplifies two major aspects of our protocol. First, it largely removes congestion from consideration. In \eqref{eq:primal_problem}, there is no constraint ensuring that total flow in both directions stays below capacity--this is always met. Consequently, there is no Lagrange multiplier for congestion and no congestion pricing; only imbalance penalties apply. In contrast, protocols in \cite{sivaraman2020high, varma2021throughput, wang2024fence} include congestion fees due to explicit congestion constraints. Second, the bound \eqref{eq:large_capacity_assumption} ensures that as long as channels remain balanced, the network can always meet demand, no matter how the demand is routed. Since channels can rebalance when necessary, they never drop transactions. This allows prices and flows to adjust as per the equations in \eqref{eq:algorithm}, which makes it easier to prove the protocol's convergence guarantees. This also preserves the key property that a channel's price remains proportional to net money flow through it.

In practice, payment channel networks are used most often for micro-payments, for which on-chain transactions are prohibitively expensive; large transactions typically take place directly on the blockchain. For example, according to \cite{river2023lightning}, the average channel capacity is roughly $0.1$ BTC ($5,000$ BTC distributed over $50,000$ channels), while the average transaction amount is less than $0.0004$ BTC ($44.7k$ satoshis). Thus, the small demand assumption is not too unrealistic. Additionally, the occasional large transaction can be treated as a sequence of smaller transactions by breaking it into packets and executing each packet serially (as done by \cite{sivaraman2020high}).
Lastly, a good path discovery process that favors large capacity channels over small capacity ones can help ensure that the bound in \eqref{eq:large_capacity_assumption} holds.

\paragraph{Constant demands} 
In this work, we assume that any transacting pair of nodes have a steady transaction demand between them (see Section \ref{sec:transaction_requests}). Making this assumption is necessary to obtain the kind of guarantees that we have presented in this paper. Unless the demand is steady, it is unreasonable to expect that the flows converge to a steady value. Weaker assumptions on the demand lead to weaker guarantees. For example, with the more general setting of stochastic, but i.i.d. demand between any two nodes, \cite{varma2021throughput} shows that the channel queue lengths are bounded in expectation. If the demand can be arbitrary, then it is very hard to get any meaningful performance guarantees; \cite{wang2024fence} shows that even for a single bidirectional channel, the competitive ratio is infinite. Indeed, because a PCN is a decentralized system and decisions must be made based on local information alone, it is difficult for the network to find the optimal detailed balance flow at every time step with a time-varying demand.  With a steady demand, the network can discover the optimal flows in a reasonably short time, as our work shows.

We view the constant demand assumption as an approximation for a more general demand process that could be piece-wise constant, stochastic, or both (see simulations in Figure \ref{fig:five_nodes_variable_demand}).
We believe it should be possible to merge ideas from our work and \cite{varma2021throughput} to provide guarantees in a setting with random demands with arbitrary means. We leave this for future work. In addition, our work suggests that a reasonable method of handling stochastic demands is to queue the transaction requests \textit{at the source node} itself. This queuing action should be viewed in conjunction with flow-control. Indeed, a temporarily high unidirectional demand would raise prices for the sender, incentivizing the sender to stop sending the transactions. If the sender queues the transactions, they can send them later when prices drop. This form of queuing does not require any overhaul of the basic PCN infrastructure and is therefore simpler to implement than per-channel queues as suggested by \cite{sivaraman2020high} and \cite{varma2021throughput}.

\subsection{The Incentive of Channels}
The actions of the channels as prescribed by the DEBT control protocol can be summarized as follows. Channels adjust their prices in proportion to the net flow through them. They rebalance themselves whenever necessary and execute any transaction request that has been made of them. We discuss both these aspects below.

\paragraph{On Prices}
In this work, the exclusive role of channel prices is to ensure that the flows through each channel remains balanced. In practice, it would be important to include other components in a channel's price/fee as well: a congestion price  and an incentive price. The congestion price, as suggested by \cite{varma2021throughput}, would depend on the total flow of transactions through the channel, and would incentivize nodes to balance the load over different paths. The incentive price, which is commonly used in practice \cite{river2023lightning}, is necessary to provide channels with an incentive to serve as an intermediary for different channels. In practice, we expect both these components to be smaller than the imbalance price. Consequently, we expect the behavior of our protocol to be similar to our theoretical results even with these additional prices.

A key aspect of our protocol is that channel fees are allowed to be negative. Although the original Lightning network whitepaper \cite{poon2016bitcoin} suggests that negative channel prices may be a good solution to promote rebalancing, the idea of negative prices in not very popular in the literature. To our knowledge, the only prior work with this feature is \cite{varma2021throughput}. Indeed, in papers such as \cite{van2021merchant} and \cite{wang2024fence}, the price function is explicitly modified such that the channel price is never negative. The results of our paper show the benefits of negative prices. For one, in steady state, equal flows in both directions ensure that a channel doesn't loose any money (the other price components mentioned above ensure that the channel will only gain money). More importantly, negative prices are important to ensure that the protocol selectively stifles acyclic flows while allowing circulations to flow. Indeed, in the example of Section \ref{sec:flow_control_example}, the flows between nodes $A$ and $C$ are left on only because the large positive price over one channel is canceled by the corresponding negative price over the other channel, leading to a net zero price.

Lastly, observe that in the DEBT control protocol, the price charged by a channel does not depend on its capacity. This is a natural consequence of the price being the Lagrange multiplier for the net-zero flow constraint, which also does not depend on the channel capacity. In contrast, in many other works, the imbalance price is normalized by the channel capacity \cite{ren2018optimal, lin2020funds, wang2024fence}; this is shown to work well in practice. The rationale for such a price structure is explained well in \cite{wang2024fence}, where this fee is derived with the aim of always maintaining some balance (liquidity) at each end of every channel. This is a reasonable aim if a channel is to never rebalance itself; the experiments of the aforementioned papers are conducted in such a regime. In this work, however, we allow the channels to rebalance themselves a few times in order to settle on a detailed balance flow. This is because our focus is on the long-term steady state performance of the protocol. This difference in perspective also shows up in how the price depends on the channel imbalance. \cite{lin2020funds} and \cite{wang2024fence} advocate for strictly convex prices whereas this work and \cite{varma2021throughput} propose linear prices.

\paragraph{On Rebalancing} 
Recall that the DEBT control protocol ensures that the flows in the network converge to a detailed balance flow, which can be sustained perpetually without any rebalancing. However, during the transient phase (before convergence), channels may have to perform on-chain rebalancing a few times. Since rebalancing is an expensive operation, it is worthwhile discussing methods by which channels can reduce the extent of rebalancing. One option for the channels to reduce the extent of rebalancing is to increase their capacity; however, this comes at the cost of locking in more capital. Each channel can decide for itself the optimum amount of capital to lock in. Another option, which we discuss in Section \ref{sec:five_node}, is for channels to increase the rate $\gamma$ at which they adjust prices. 

Ultimately, whether or not it is beneficial for a channel to rebalance depends on the time-horizon under consideration. Our protocol is based on the assumption that the demand remains steady for a long period of time. If this is indeed the case, it would be worthwhile for a channel to rebalance itself as it can make up this cost through the incentive fees gained from the flow of transactions through it in steady state. If a channel chooses not to rebalance itself, however, there is a risk of being trapped in a deadlock, which is suboptimal for not only the nodes but also the channel.

\section{Conclusion}
This work presents DEBT control: a protocol for payment channel networks that uses source routing and flow control based on channel prices. The protocol is derived by posing a network utility maximization problem and analyzing its dual minimization. It is shown that under steady demands, the protocol guides the network to an optimal, sustainable point. Simulations show its robustness to demand variations. The work demonstrates that simple protocols with strong theoretical guarantees are possible for PCNs and we hope it inspires further theoretical research in this direction.
The advancement of artificial intelligence in the legal domain has led to the development of various tools that assist in legal research, document retrieval, and automated legal reasoning. Several studies have explored the use of Natural Language Processing (NLP)\cite{khurana2023natural}, machine learning models, and vector-based search mechanisms to enhance the efficiency of legal chatbots. The primary focus of this literature review is on retrieval-augmented generation (RAG) models, FAISS-based document retrieval, deep learning for legal applications, and the use of large language models (LLMs) in legal AI.  

Recent research on Retrieval-Augmented Generation (RAG)\cite{gao2023retrieval} for legal AI has demonstrated its potential in enhancing legal text retrieval and summarization. S. S. Manathunga, Y. and A. Illangasekara\cite{manathunga2023retrieval} proposed a RAG-based model that improves legal text summarization by dynamically fetching relevant documents before generating responses. Similarly, Lee and Ryu \cite{ryu-etal-2023-retrieval} explored the application of RAG in case law retrieval, demonstrating its superiority over traditional keyword-based search engines. The introduction of RAG has significantly improved response accuracy by grounding AI-generated text in authoritative legal documents, reducing hallucinations in AI-driven legal assistance.  

% \begin{figure}[h]
%     \centering
%     \includegraphics[width=8cm]{FAISS.png}
%     \caption{Faiss: Efficient Similarity Search and Clustering of Dense Vectors}
%     \label{Overall Result of comparing FAISS and Chroma with different number of top documents}
% \end{figure}

The efficiency of FAISS (Facebook AI Similarity Search) in legal document retrieval has also been widely studied. Zhao et al. \cite{devlin-etal-2019-bert} implemented FAISS to enhance large-scale legal question answering systems, achieving significant improvements in retrieval speed and relevance. N. Goyal and D. Chen \cite{inbook} demonstrated that FAISS-based vector search mechanisms outperform conventional database searches in legal information retrieval, reducing query response time while maintaining high accuracy. The integration of FAISS with transformer-based models, as seen in the work of Hsieh and Wu, further enhances semantic retrieval, ensuring that chatbot responses align with actual legal texts.  

Transformer-based models such as BERT and GPT-based architecture have also contributed to the evolution of AI-driven legal research. Devlin et al. introduced BERT (Bidirectional Encoder Representations from Transformers), which significantly improved the understanding of legal language. RoBERTa, an optimized version of BERT, was later developed by Liu et al. \cite{liu2019roberta} to enhance contextual understanding and document similarity matching in legal queries. These models have been integrated into legal chatbots for contract analysis and legal decision-making, as demonstrated in the studies of Li et al. and Jin and Liu, where fine-tuned transformers improved legal text comprehension and summarization.  
The role of deep learning in legal AI has also been investigated extensively. Radford et al. introduced GPT-3, which paved the way for legal AI assistants capable of generating human-like responses. However, researchers such as Firth and Lee emphasized the limitations of LLMs in legal reasoning, arguing that these models require external verification mechanisms to prevent misinformation. The use of contrastive learning and fine-tuning for legal text retrieval has been explored by Arabi and Akbari \cite{article}, who demonstrated that embedding-based retrieval significantly improves chatbot response accuracy.  

Another significant area of research involves evaluating AI-generated legal responses using automated metrics. Zhang and Wu introduced BLEU\cite{10.3115/1073083.1073135} and ROUGE\cite{lin-2004-rouge} scores as a means to evaluate AI-generated legal text summaries, ensuring their quality and relevance. Similarly, Zhao et al. \cite{yuan2024rag} examined the effectiveness of RAG-based models in handling complex legal queries, highlighting the importance of legal consistency scores (LCS) in evaluating AI-driven responses.  

The practical applications of legal AI chatbots have been studied extensively in the context of access to justice and AI ethics. Wang and Cheng et al. \cite{xue2024bias} highlighted the potential of AI-driven legal assistants in bridging the justice gap, particularly in countries where legal resources are not easily accessible. Chan conducted a systematic review of retrieval-based legal chatbots, noting that while these systems improve accessibility, they also raise ethical concerns regarding legal misinformation and bias. Research by Min \cite{Min2023ARTIFICIALIA} explored methods for bias detection and mitigation in legal AI, ensuring fairness in AI-generated legal advice.  

Comparative studies between rule-based legal bots, keyword-driven legal search engines, and AI-powered legal chatbots further illustrate the superiority of retrieval-augmented approaches. In a study conducted by Zeng \cite{zeng2024scalable}, FAISS-based retrieval mechanisms significantly outperformed traditional Boolean keyword searches, reducing irrelevant document retrieval by 40\%. Singh \cite{10760929} further demonstrated that AI-powered legal research tools using NLP provide faster and more contextually accurate responses compared to standard legal databases.  

Despite these advancements, challenges remain in AI-driven legal research. Existing chatbots still struggle with multi-jurisdictional legal queries, as noted by Weichbroth \cite{Weichbroth2025AIAT}, who emphasized the need for jurisdiction-aware legal AI models. Additionally, legal AI models often lack the ability to process long-context legal arguments effectively, a limitation discussed by Gupta, who proposed memory-based retrieval techniques to improve long-form legal text processing.  

Research continues to refine AI-driven legal assistance, particularly in retrieval-augmented generation, FAISS-based search, transformer models, and deep learning techniques for legal research. However, further improvements are needed in bias mitigation, jurisdiction-specific adaptations, and long-context legal understanding. Future developments in multilingual legal AI, enhanced retrieval mechanisms, and AI-powered contract analysis will be crucial in making legal AI tools more accessible, reliable, and widely applicable in legal practice.
\section{Conclusion}
In this work, we propose a simple yet effective approach, called SMILE, for graph few-shot learning with fewer tasks. Specifically, we introduce a novel dual-level mixup strategy, including within-task and across-task mixup, for enriching the diversity of nodes within each task and the diversity of tasks. Also, we incorporate the degree-based prior information to learn expressive node embeddings. Theoretically, we prove that SMILE effectively enhances the model's generalization performance. Empirically, we conduct extensive experiments on multiple benchmarks and the results suggest that SMILE significantly outperforms other baselines, including both in-domain and cross-domain few-shot settings.

\section{Data Availability}
The code and data of \tool and \bench are available at https://github.com/JohnnyPeng18/Coffe.

\section*{Acknowledgment}
The authors would like to thank the anonymous reviewers who have provided insightful and constructive comments on this paper.
This work is supported by the National Nature Science Foundation of China (No. 62302437).


\newpage
\bibliographystyle{ACM-Reference-Format}
\bibliography{ref}

\end{document}
\endinput
%%
%% End of file `sample-acmsmall-conf.tex'.
