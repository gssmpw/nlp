\pdfoutput=1
%% This is file `sample-acmsmall-conf.tex',
%% generated with the docstrip utility.
%%
%% The original source files were:
%%
%% samples.dtx  (with options: `all,proceedings,bibtex,acmsmall-conf')
%% 
%% IMPORTANT NOTICE:
%% 
%% For the copyright see the source file.
%% 
%% Any modified versions of this file must be renamed
%% with new filenames distinct from sample-acmsmall-conf.tex.
%% 
%% For distribution of the original source see the terms
%% for copying and modification in the file samples.dtx.
%% 
%% This generated file may be distributed as long as the
%% original source files, as listed above, are part of the
%% same distribution. (The sources need not necessarily be
%% in the same archive or directory.)
%%
%%
%% Commands for TeXCount
%TC:macro \cite [option:text,text]
%TC:macro \citep [option:text,text]
%TC:macro \citet [option:text,text]
%TC:envir table 0 1
%TC:envir table* 0 1
%TC:envir tabular [ignore] word
%TC:envir displaymath 0 word
%TC:envir math 0 word
%TC:envir comment 0 0
%%
%%
%% The first command in your LaTeX source must be the \documentclass
%% command.
%%
%% For submission and review of your manuscript please change the
%% command to \documentclass[manuscript, screen, review]{acmart}.
%%
%% When submitting camera ready or to TAPS, please change the command
%% to \documentclass[sigconf]{acmart} or whichever template is required
%% for your publication.
%%
%%
\documentclass[acmsmall]{acmart}

%%
%% \BibTeX command to typeset BibTeX logo in the docs
\AtBeginDocument{%
  \providecommand\BibTeX{{%
    Bib\TeX}}}

%% Rights management information.  This information is sent to you
%% when you complete the rights form.  These commands have SAMPLE
%% values in them; it is your responsibility as an author to replace
%% the commands and values with those provided to you when you
%% complete the rights form.

%% These commands are for a PROCEEDINGS abstract or paper.
%%
%%  Uncomment \acmBooktitle if the title of the proceedings is different
%%  from ``Proceedings of ...''!
%%
%%\acmBooktitle{Woodstock '18: ACM Symposium on Neural Gaze Detection,
%%  June 03--05, 2018, Woodstock, NY}

\setcopyright{cc}
\setcctype{by}
\acmDOI{10.1145/3715727}
\acmYear{2025}
\acmJournal{PACMSE}
\acmVolume{2}
\acmNumber{FSE}
\acmArticle{FSE012}
\acmMonth{7}
\received{2024-09-13}
\received[accepted]{2025-01-14}


%%
%% Submission ID.
%% Use this when submitting an article to a sponsored event. You'll
%% receive a unique submission ID from the organizers
%% of the event, and this ID should be used as the parameter to this command.
%%\acmSubmissionID{123-A56-BU3}

%%
%% For managing citations, it is recommended to use bibliography
%% files in BibTeX format.
%%
%% You can then either use BibTeX with the ACM-Reference-Format style,
%% or BibLaTeX with the acmnumeric or acmauthoryear sytles, that include
%% support for advanced citation of software artefact from the
%% biblatex-software package, also separately available on CTAN.
%%
%% Look at the sample-*-biblatex.tex files for templates showcasing
%% the biblatex styles.
%%

%%
%% The majority of ACM publications use numbered citations and
%% references.  The command \citestyle{authoryear} switches to the
%% "author year" style.
%%
%% If you are preparing content for an event
%% sponsored by ACM SIGGRAPH, you must use the "author year" style of
%% citations and references.
%% Uncommenting
%% the next command will enable that style.
%%\citestyle{acmauthoryear}

\newcommand{\kwcost}[1]{\mathbf{cost}\left(  {#1} \right)}
\newcommand{\circuitcon}[2]{{#1} + {#2}}

\newcommand{\bigomega}{\mathbf{\Omega}}

%% Semantics frame
\newcommand{\sfbox}[1]
{
\cfbox{blue}{#1}
}
\newcommand{\srule}{\vspace{2mm}\rule{\columnwidth}{1pt}\vspace{2mm}}

\newcommand{\lang}{\textsc{Laqe}}
%% General syntax
\newcommand{\dom}[1]{\mathop{\text{dom}}(#1)}
\newcommand{\codom}[1]{\mathop{\text{cod}}(#1)}
\newcommand{\kw}[1]{\mbox{\ttt{#1}}}
\newcommand{\cdparens}[1]{({#1})}
\newcommand{\cd}[1]{{\lstinline!#1!}}
\newcommand{\hmm}{\textsf{HMM}}
\newcommand{\rulename}[1]{\textsc{#1}}
\newcommand{\ruleref}[1]{Rule~\rulename{#1}}

\newcommand{\true}{\ensuremath{\kw{true}}}
\newcommand{\false}{\ensuremath{\kw{false}}}
\newcommand{\ttrue}{\kw{t}}
\newcommand{\ffalse}{\kw{f}}
\newcommand{\prog}{\ensuremath{P}}
\newcommand{\pred}{\ensuremath{\mathcal{P}}}
\newcommand{\predf}[2]{\ensuremath{\pred(#1,#2)}}
\newcommand{\defeq}{\triangleq}

\newcommand{\type}[2]{\ensuremath{#1 : #2}}
\newcommand{\typed}[4]{\ensuremath{#1 \vdash_{#2} \type{#3}{#4}}}

\newcommand{\btype}{\beta}
\newcommand{\utype}{\theta}
\newcommand{\val}{v}
\newcommand{\uval}{u}

\newcommand{\algname}{\textsf{OAC}}
\newcommand{\algnameminus}{\textsf{OACMinus}}
\newcommand{\coam}{\textsf{OAC}}
\newcommand{\lopt}{\textsf{Lopt}}
\newcommand{\coamwith}[1]{\ensuremath{\mathsf{SOAM}[{#1}]}}
\newcommand{\queso}{{\textsf{Queso}}}
\newcommand{\voqc}{{\textsf{VOQC}}}
\newcommand{\pyzx}{{\textsf{PyZX}}}
\newcommand{\quartz}{{\textsf{Quartz}}}
\newcommand{\quartztool}{$\mathsf{Quartz}$}
\newcommand{\quesotool}{$\mathsf{Queso}$}
\newcommand{\feyntool}{\textsf{FeynOpt}}

\newcommand{\quartzt}[1]{\ensuremath{\mathsf{Quartz}_{\,#1}}}
\newcommand{\quesot}[1]{\ensuremath{\mathsf{Queso}_{\,#1}}}
\newcommand{\coamt}[1]{\ensuremath{\mathsf{SOAM}[#1]}}
\newcommand{\clifft}{Clifford+T}

\newcommand{\compat}[2]{\ensuremath{{#1} \mathbin{\scaleobj{1.2}{\diamond}} {#2}}}
\newcommand{\notcompat}[2]{\ensuremath{{#1} \mathbin{\scaleobj{1.2}{\centernot{\diamond}}} {#2}}}
\newcommand{\windowopt}[2]{{#2}~\textsf{\textbf{segment-optimal}}_{#1}}
\newcommand{\wopttext}{segment optimal}
\newcommand{\compressed}[1]{{#1}~\textsf{\textbf{compact}}}
\newcommand{\locallyopt}[2]{{#2}~\textsf{\textbf{locally-optimal}}_{#1}}
\newcommand{\qubits}[1]{\mathsf{qubits}({#1})}
%% Terms
\newcommand{\tmemprog}{memory-progress\xspace}
\newcommand{\tmempres}{memory-preservation\xspace}


%% imperative serial types
\newcommand{\kwint}{\kw{int}}
\newcommand{\kwnat}{\kw{nat}}
\newcommand{\kwfut}{\kw{fut}}
\newcommand{\kwprod}[2]{\ensuremath{{#1} \times {#2}}}
\newcommand{\kwarr}[2]{\ensuremath{{#1} \ra {#2}}}
\newcommand{\kwloc}[1]{\ensuremath{{#1}~\kw{loc}}}
\newcommand{\kwref}[1]{\ensuremath{{#1}~\kw{ref}}}

%% imperative multithreaded types
\newcommand{\kwtid}{\kw{tid}}
\newcommand{\kwunit}{\kw{unit}}
\newcommand{\kwok}{\kw{ok}}

% space, heap, store
\newcommand{\heap}{H}
\newcommand{\spc}{H}
\newcommand{\empspc}{\emptyset}
\newcommand{\spaceext}[3]{{#1}[{#2} \mapsto {#3}]}

\newcommand{\catspace}{\uplus}
\newcommand{\catheap}{\uplus}

\newcommand{\heapun}[1]{\langle #1 \rangle}
\newcommand{\heapbi}[2]{\langle #1 ; #2 \rangle}
\newcommand{\heaptri}[3]{\langle #1 ; #2; #3 \rangle}
\newcommand{\heapquad}[4]{\langle #1 ; #2 ; #3 ; #4 \rangle}
\newcommand{\restctx}[2]{\ensuremath{#1 \upharpoonright_{#2}}}
\newcommand{\freeloc}[1]{\ensuremath{\mathsf{FL}(#1)}}
\newcommand{\locs}[1]{\ensuremath{\mathsf{Loc}(#1)}}
\newcommand{\diff}[1]{\ensuremath{\mathsf{Diff}(#1)}}

\newcommand{\rename}[3]{[#2 \mapsto #3](#1)}
\newcommand{\kwt}{\kw{t}}

\newcommand{\estore}{[~]}
\newcommand{\mkstore}[2]{\ensuremath{{#1}::{#2}}}

%% imperative serial syntax
\newcommand{\kwn}{\kw{n}}
\newcommand{\kwlet}[3]{\kw{let}~{#1}={#2}~\kw{in}~{#3}~\kw{end}}
\newcommand{\kwfun}[3]{\ensuremath{\kw{fun}~{#1}~{#2}~\kw{is}~{#3}~\kw
{end}}}
\newcommand{\kwpair}[2]{\ensuremath{\langle{#1},{#2}}\rangle}
\newcommand{\kwapply}[2]{\ensuremath{{#1}~{#2}}}
\newcommand{\kwfst}[1]{\ensuremath{\kw{fst}\cdparens{#1}}}
\newcommand{\kwsnd}[1]{\ensuremath{\kw{snd}\cdparens{#1}}}
%\newcommand{\kwfst}[1]{\ensuremath{\kw{fst}~#1}}
%\newcommand{\kwsnd}[1]{\ensuremath{\kw{snd}~#1}}
\newcommand{\gcing}[1]{\ensuremath{[#1]}}
\newcommand{\kwnew}[1]{\ensuremath{\kw{ref}(#1)}}
\newcommand{\kwderef}[1]{\ensuremath{\mathop{!}#1}}
\newcommand{\kwwrite}[2]{\ensuremath{#1 \mathop{:=} #2}}

\newcommand{\kwletrec}[2]{\ensuremath{#1 \mathop{\cdot} #2}}
\newcommand{\kwtask}[3]{\ensuremath{#1 \mathop{\cdot} #2 \mathop{\cdot} #3}}
\newcommand{\kwtaskalt}[2]{\ensuremath{#1 \mathop{\cdot} #2}}
\newcommand{\halt}{\bot}
\newcommand{\tree}{T}
\newcommand{\trace}{t}
%% imperative multithreaded syntax
\newcommand{\kwfork}[1]{{\ensuremath{\kw{fork}\cdparens{#1}}}}
\newcommand{\kwjoin}[1]{{\ensuremath{\kw{join}\cdparens{#1}}}}
\newcommand{\kwunitv}{\ensuremath{(\,)}}
\newcommand{\kwtidv}{\ensuremath{\kw{t}}}
\newcommand{\gheap}{G}
\newcommand{\lheap}{\heap}
\newcommand{\tolheap}[1]{\ensuremath{\Delta(#1)}}
\newcommand{\theap}[2]{\left(#1, #2\right)}

%% hierarchical syntax
\newcommand{\cdpar}{\texttt{par}}
%\newcommand{\kwpar}[2]{\ensuremath{#1 \mathop{\|} #2}}
\newcommand{\kwpar}[2]
           {\ensuremath{\mathop{\vartriangleleft \hspace{-0.1em} #1, #2
               \hspace{-0.1em} \vartriangleright}}}
\newcommand{\kwpara}[2]
           {\ensuremath{\mathop{\blacktriangleleft \hspace{-0.1em} #1, #2
               \hspace{-0.1em} \blacktriangleright}}}
%\newcommand{\kwpara}[2]{\ensuremath{#1 \mathop{\overline{\|}} #2}}
\newcommand{\kwparl}[2]{\ensuremath{#1 \overset{\leftarrow}{\|} #2}}
\newcommand{\kwparr}[2]{\ensuremath{#1 \overset{\rightarrow}{\|} #2}}
%\newcommand{\gcing}[1]{\ensuremath{[#1]}}
\newcommand{\task}{T}
\newcommand{\config}{\mathcal{C}}

%% flattening
\newcommand{\flate}[1]{\hat{#1}}
\newcommand{\flatten}[3]{\left\| #2 \right\|_{#1} \leadsto #3}
\newcommand{\fstep}{\step}%{\step_F}

%% shorthands
\renewcommand{\a}{\ensuremath{\alpha}}
\renewcommand{\b}{\ensuremath{\beta}}
\newcommand{\h}{\ensuremath{\eta}}
\renewcommand{\r}{\ensuremath{\rho}}
%\newcommand{\s}{\ensuremath{\sigma}}
\newcommand{\p}{\ensuremath{P}}
\newcommand{\s}{\p}
\newcommand{\om}{\ensuremath{\Omega}}
\renewcommand{\l}{\ensuremath{l}}
\newcommand{\sig}{\ensuremath{\Sigma}}
\newcommand{\empctx}{\ensuremath{\cdot}}


% Relations
%\newcommand{\red}{\Downarrow}
%\newcommand{\redgc}{\stackrel{gc?}{\Longrightarrow}}
%\newcommand{\alloc}{\stackrel{alloc}\Longrightarrow}
\newcommand{\la}{\leftarrow}
\newcommand{\ra}{\rightarrow}
\newcommand{\pstep}{\Rightarrow}
\newcommand{\tstep}{\Rightarrow}
\newcommand{\optstep}{\longmapsto}
\newcommand{\compstep}{\longmapsto_{\delta}}
\newcommand{\localstep}[3]{{#2} \overset{#1}{\optstep} {#3}}
\newcommand{\globstep}[2]{{#1} \compstep {#2}}
%\newcommand{\stepr}[1]{\xra{#1}}
\newcommand{\step}{\ra}
\newcommand{\stepgc}[1]{\xra[{\mbox{\tiny GC}}]{#1}}
\newcommand{\gcstep}{\ra_{\mbox{\tiny GC}}}
\newcommand{\pgcstep}{\pstep_{\mbox{\tiny GC}}}
\newcommand{\cgcstep}{\rightarrow_{\mbox{\tiny CGC}}}


%\newcommand{\sunion}[2]{{#1} \stackrel{?}{\bigcup} {#2}}
%\newcommand{\spush}[2]{{#1} \stackrel{?}{\downarrow} {#2}}

% Other judgments
\newcommand{\fresh}{\ensuremath{\; \mathsf{fresh}}}
%\newcommand{\leaf}{\ensuremath{\; \mathsf{leaf}}}
%\newcommand{\starrow}[1]{\stackrel{\mbox{\tiny #1}}{\xrightarrow}}
\newcommand{\starrow}[1]{\xrightarrow{#1}}
%\newcommand{\alloc}[4]{\mathit{alloc}\left(#1, #2\right) = \left(#3, #4\right)}
\newcommand{\alloc}[4]{#1; #2 \starrow{alloc} #3; #4}
%\newcommand{\update}[4]{\mathit{update}\left(#1, #2 \la #3\right) = #4}
\newcommand{\update}[4]{#1; #2; #3 \starrow{update} #4}
%\newcommand{\lookup}[3]{\mathit{lookup}\left(#1, #2\right) = #3}
\newcommand{\lookup}[3]{#1; #2 \starrow{lookup} #3}
\newcommand{\newtask}[2]{#1 \starrow{new} #2}
\newcommand{\isdone}[3]{#1 \starrow{done} #2; #3}
\newcommand{\diffs}[1]{\mathit{diff}(#1)}
\newcommand{\initial}{\ensuremath{\;\mathsf{initial}}}
\newcommand{\htyped}[3]{\vdash_{#3} #1 : #2}

% Multilevel heap judgments
\newcommand{\heaptype}[3]{\left(#1, #2\right) : #3}
\newcommand{\allocg}[4]{\mathit{allocg}\left(#1, #2\right) = \left(#3, #4\right)}
\newcommand{\allocl}[4]{\mathit{allocl}\left(#1, #2\right) = \left(#3, #4\right)}
%\newcommand{\promote}[6]{\mathit{promote}\left(#1, #2, #3\right) =
%  \left(#4, #5, #6\right)}
\newcommand{\promote}[6]{#1; #2; #3 \starrow{promote} #4; #5; #6}
%\newcommand{\promotebrl}[3]{\mathit{promote}\left(#1, #2, #3\right)}
%\newcommand{\promotebrr}[3]{\left(#1, #2, #3\right)}
\newcommand{\promotebrl}[3]{#1; #2; #3}
\newcommand{\promotebra}{\starrow{promote}}
\newcommand{\promotebrr}[3]{#1; #2; #3}
\newcommand{\pmap}{M}
\newcommand{\greachable}[1]{\mathsf{greachable}\left(#1\right)}

% Theorems
\newtheorem{thm}[theorem]{Theorem}
\newtheorem{lem}[theorem]{Lemma}
% \newtheorem{corollary}[theorem]{Corollary}
% \newtheorem{claim}{Claim}


%% Rule Array
\newenvironment{rulearray}
{
\newcommand{\newcol}{\qquad}
\newcommand{\newcolhalf}{\quad}
\newcommand{\newrow}{\\[4ex]}
\newcommand{\newrowhalf}{\\[2ex]}
\[
\begin{array}{c}
}
{
\end{array}
\]
\let\newcol\undefined
\let\newrow\undefined
}

% Author-specific todo notes
\newcommand{\ramtodo}[2][]
{\todo[color=magenta,author=Ram,size=\small,#1]{#2}}


\newcommand{\defn}[1]{\emph{\textbf{#1}}}
\newcommand{\mpl}{\textsf{MPL}}

\newcommand{\rulereftwo}[2]{rules~\rulename{#1} and \rulename{#2}}
\newcommand{\with}{\ensuremath{\mathbin;}}

\newcommand{\highlight}[1]{\colorbox{gray!20}{\ensuremath{#1}}}
\newcommand{\hred}[1]{\colorbox{red!10}{\ensuremath{#1}}}
\newcommand{\hblue}[1]{\colorbox{blue!10}{\ensuremath{#1}}}
\newcommand{\hgreen}[1]{\colorbox{green!10}{\ensuremath{#1}}}
\newcommand{\sizeof}[1]{\ensuremath{\lvert #1 \rvert}}
\newcommand{\costof}[1]{\ensuremath{\mathbf{cost} ({#1})}}
\newcommand{\cost}{\ensuremath{\mathbf{cost}}}
\newcommand{\oracle}{\ensuremath{\mathbf{oracle}}}


% inline "math highlight" to make it easier to read inline judgements
\definecolor{darkblue}{HTML}{0007C9}
% \newcommand{\mh}[1]{{\color{darkblue}\ensuremath{\mbox{\ensuremath{#1}}}}}
\newcommand{\mh}[1]{{\ensuremath{\mbox{\ensuremath{#1}}}}}


%% variable context, location signature
% \newcommand{\ctxvar}{\Gamma}
% \newcommand{\ctxloc}{\Sigma}
\newcommand{\ctxemp}{\ensuremath{\cdot}}
\newcommand{\ctxext}[3]{\ensuremath{#1,#2\!:\!#3}} % extend context
\newcommand{\etyped}[4]{\ensuremath{{#1} \vdash_{#2} {#3} : {#4}}}
\newcommand{\memtyped}[3]{\ensuremath{{#1} \vdash {#2} : {#3}}}
\newcommand{\gtyped}[3]{\ensuremath{{#1} \vdash {#2} : {#3}}}
\newcommand{\httyped}[6]{\ensuremath{{#1} \with {#2} \with {#3} \vdash {#4}\!\cdot\!{#5} : {#6}}}
\newcommand{\ttyped}[5]{\ensuremath{{#1} \with {#2} \with {#3} \vdash {#4} : {#5}}}

\newcommand{\sttyped}[6]{\ensuremath{{\vdash_{#1} {#2} \with {#3} \with {#4} \with {#5} : {#6}}}}
\newcommand{\getyped}[6]{\ensuremath{{#1} \vdash_{#2, #3} {#4} \with {#5} : {#6}}}




%% types
\newcommand{\typnat}{\kw{nat}}
\newcommand{\typint}{\kw{int}}
\newcommand{\typbool}{\kw{bool}}
\newcommand{\typchar}{\kw{char}}
\newcommand{\typfloat}{\kw{float}}
\newcommand{\typprod}[2]{\ensuremath{{#1} \times {#2}}}
\newcommand{\typfun}[2]{\ensuremath{{#1}\!\rightarrow\!{#2}}}
\newcommand{\typref}[1]{\ensuremath{{#1}~\kw{ref}}}
\newcommand{\typfut}[1]{\ensuremath{{#1}~\kw{fut}}}
\newcommand{\futs}[1]{\mathsf{Fut}(#1)}
\newcommand{\futsmem}[2]{\mathsf{Fut}(#1, #2)}




% expression syntax
\newcommand{\enat}[1]{\ensuremath{#1}}
\newcommand{\efun}[3]{\ensuremath{\kw{fun}~{#1}~{#2}~\kw{is}~{#3}}}
\newcommand{\epair}[2]{\ensuremath{\langle {#1}, {#2} \rangle}}
\newcommand{\eapp}[2]{\ensuremath{{#1}~{#2}}}
\newcommand{\efst}[1]{\ensuremath{\kw{fst}~{#1}}}
\newcommand{\esnd}[1]{\ensuremath{\kw{snd}~{#1}}}
\newcommand{\eref}[1]{\ensuremath{\kw{ref}~{#1}}}
\newcommand{\ebang}[1]{\ensuremath{\mathop{!}#1}}
\newcommand{\eupd}[2]{\ensuremath{#1 \mathop{:=} #2}}
\newcommand{\elet}[3]{\kw{let}~{#1}={#2}~\kw{in}~{#3}}
\newcommand{\epar}[2]{\ensuremath{\langle {#1}\mathbin\|{#2} \rangle}}

\newcommand{\purelang}{{\sc $\lambda^{P}$}}
\newcommand{\reflang}{{\sc $\lambda^{U}$}}

% task syntax
% \newcommand{\tleaf}[2]{\ensuremath{{#1}\!\cdot\!{#2}}}
% \newcommand{\tpar}[4]{\ensuremath{\dblangle{{#1}\!\cdot\!{#2}\mathbin\|{#3}\!\cdot\!{#4}}}}
% \newcommand{\tpar}[4]{\ensuremath{\llparenthesis\,{#1}\!\cdot\!{#2}\mathbin\|{#3}\!\cdot\!{#4}\,\rrparenthesis}}
% \newcommand{\ttpar}[2]{\ensuremath{\llparenthesis\,{#1}\mathbin\|{#2}\,\rrparenthesis}}
% \newcommand{\tparg}[6]{\ensuremath{\llparenthesis\,{#1}\!\cdot\!{#2}\!\cdot\!{#3}\mathbin\|{#4}\!\cdot\!{#5}\!\cdot\!{#6}\,\rrparenthesis}}
% \newcommand{\tpar}[3]{\ensuremath{{#1}\!\cdot\!\llparenthesis\,{#2}\mathbin\|{#3}\,\rrparenthesis}}
% \newcommand{\taskhpe}[3]{\ensuremath{{#1}\!\cdot\!{#2}\!\cdot\!{#3}}}

% \newcommand{\mem}{\mu}
\newcommand{\mememp}{\emptyset}
\newcommand{\memext}[3]{\ensuremath{#1}[{#2} \!\hookrightarrow\! {#3}]}

\newcommand{\actarrow}{\blacktriangleright}
\newcommand{\pasarrow}{\vartriangleright}
\newcommand{\fmap}{\Delta}
\newcommand{\femp}{\emptyset}
\newcommand{\fmapactive}[3]{\ensuremath{#1} [{#2} \!\actarrow\! {#3}]}
\newcommand{\fmapjoined}[3]{\ensuremath{#1} [{#2} \!\pasarrow\! {#3}]}


\newcommand{\futctxt}{\Knownctxt}
\newcommand{\Futctxt}{\Knownctxt}
\newcommand{\ReadLocs}{\mathsf{R}}
\newcommand{\Knownctxt}{{K}}
\newcommand{\fut}[2]{\kw{fut}(#1; #2)}
\newcommand{\harpfut}[1]{\kw{fut}(#1)}
\newcommand{\futctxtemp}{\emptyset}
\newcommand{\te}[1]{\{#1\}}
\newcommand{\hemp}{\emptyset}
\newcommand{\hcat}{\cup}
\newcommand{\hext}[2]{{#1},{#2}}

\newcommand{\tack}{\oplus}
\newcommand{\plug}{\bowtie}

\newcommand{\omparam}{step length}
\newcommand{\actwrite}[2]{\textbf{U}{#1}\!\Leftarrow\!{#2}}
\newcommand{\actalloc}[2]{\textbf{A}{#1}\!\Leftarrow\!{#2}}
\newcommand{\actread}[2]{\textbf{R}{#1}\!\Rightarrow\!{#2}}
\newcommand{\actsync}[2]{\textbf{F}{#1}\!\Rightarrow\!{#2}}
\newcommand{\actnone}{\textbf{N}}

% Relations
\newcommand{\stepstar}{\longmapsto^*}
\newcommand{\tstepstar}{\tstep^*}
\newcommand{\drfstep}[2]{\xmapsto[{#2}]{{\,#1\,}}}
\newcommand{\drfstepstar}[1]{\xmapsto{{\,#1\,}}\joinrel\mathrel{^*}}


% computation graph
\newcommand{\gt}[2]{\ensuremath{\mathsf{GT}({#1},{#2})}}
\newcommand{\gemp}{\bullet}
\newcommand{\gseq}[2]{{#1}\oplus{#2}}
\newcommand{\gseqnamed}[3]{{#1}\oplus_{#2}{#3}}
\newcommand{\gseqa}[2]{\gseqnamed{#1}{a}{#2}}
\newcommand{\gseqb}[2]{\gseqnamed{#1}{b}{#2}}
\newcommand{\gspawn}[1]{\mathsf{spawn}\ {#1}}
\newcommand{\gsync}[1]{\mathsf{sync}\ {#1}}
\newcommand{\ghead}[1]{\mathsf{hd}(#1)}
\newcommand{\gtail}[1]{\mathsf{tl}(#1)}

% \def\ojoin{\setbox0=\hbox{$\bowtie$}%
%   \rule[\bt0]{.25em}{.4pt}\llap{\rule[\ht0]{.25em}{.4pt}}}


\newcommand{\fcpar}[3]{\ensuremath{\gseq{#1}{(\gpar{#2}{#3})}}}

% \def\rightouterjoin{\mathbin{\bowtie\mkern-5.8mu\ojoin}}

\newcommand{\gmerge}[2]{\bowtie_F ({#1}, {#2})}
\newcommand{\gmergerel}[3]{\bowtie_R ({#1}, {#2}) \downarrow {#3}}
% \newcommand{\gcseq}[1]{\ensuremath{\gseq{#1}{\raisebox{-1pt}{$\square$}}}}
% \newcommand{\gcseq}[1]{\ensuremath{\fbox{$#1$}}}
% \newcommand{\gcseq}[1]{\ensuremath{\llparenthesis #1 \rrparenthesis}}
\newcommand{\gcseq}[1]{\ensuremath{[#1]}}
% \newcommand{\gcpar}[3]{\ensuremath{{#1}\!\cdot\!({#2}\mathbin\|{#3})}}
\newcommand{\gcpar}[3]{\ensuremath{\gseq{#1}{(\gpar{#2}{#3})}}}
\newcommand{\gcparnamed}[4]{\ensuremath{\gseq{#1}{({#2}\otimes_{#3}{#4})}}}
\newcommand{\gcspawn}[4]{\ensuremath{\gseq{#1}{\gseq{#2}{(\gpar{#3}{#4})}}}}


\newcommand{\gpar}[2]{{#1}\otimes_{a}{#2}}
\newcommand{\gw}[1]{\ensuremath{\mathsf{W}({#1})}}
\newcommand{\ga}[1]{\ensuremath{\mathsf{A}({#1})}}
\newcommand{\greads}[1]{\ensuremath{\ReadLocs({#1})}}
\newcommand{\gaw}[1]{\ensuremath{\mathsf{AW}({#1})}}
\newcommand{\lw}[1]{\ensuremath{\mathsf{LW}({#1})}}
\newcommand{\alw}[1]{\ensuremath{\mathsf{A}({#1}) \cup \mathsf{LW}({#1})}}
\newcommand{\gabw}[1]{\ensuremath{\gaw{#1}}}
% \newcommand{\gabw}[1]{\ensuremath{\mathsf{A}({#1}) \cup \lw{#1}}}


\newcommand{\saw}[1]{\ensuremath{\mathsf{SP}({#1})}}
% \newcommand{\gf}[1]{\ensuremath{\llbracket{#1}\rrbracket}}
\newcommand{\gf}[1]{\ensuremath{\overline{#1}}}

\newcommand{\extendsfj}[2]{\ensuremath{{#1}~\textsf{extends}~{#2}~\textsf{with f/j}}}
\newcommand{\extendswith}[3]{\ensuremath{{#1}~\textsf{extends}~{#2}~\textsf{with}~{#3}}}

% \newcommand{\tpardag}[4]{\ensuremath{\llparenthesis\,{#1}\!\cdot\!{#2}\mathbin\|{#3}\!\cdot\!{#4}\rrparenthesis}}
% \newcommand{\dagobv}[2]{#1~\textsf{obv}~#2}
% \newcommand{\dagread}[2]{#1~\textsf{reads}~#2}
% \newcommand{\dagwrite}[2]{#1~\textsf{writes}~#2}
% \newcommand{\dagdrf}[2]{#1~\textsf{drf}~@~#2}

\newcommand{\geok}[2]{{#1} \with {#2}~\textit{ok}}
\newcommand{\loc}[1]{{#1}~\textit{loc}}
\newcommand{\gleaf}[1]{{#1}~\textit{leaf}}
\newcommand{\gnode}[1]{{#1}~\textit{node}}

\newcommand{\drf}[2]{{#1} \vdash {#2}~{\textit{drf}}}
\newcommand{\drfb}[2]{{#1} \vdash {#2}~{\textit{wrf}}}
\newcommand{\drft}{\textit{drf}}
% \newcommand{\typed}[2]{{#1} \vdash }

% Theorems
\theoremstyle{plain}
\newtheorem{property}{Property}

% \theoremstyle{definition}
% \newtheorem{definition}{Definition}

%% Rules description
\newcommand{\flushLR}[3]{\hspace*{#3}\makebox[0em][l]{#1}\hspace*{\fill}\makebox[0em][r]{#2}\hspace*{#3}}
% \newcommand{\rulesdesc}[2]{\flushLR{\textbf{#1}}{\fbox{#2}}{1em}}
\newcommand{\rulesdesc}[2]{\textbf{#1}\hspace*{1em}{\fbox{#2}}}
\newcommand{\desc}[1]{\textbf{#1}}

% Syntax highlighting
\newdimen\zzlistingsize
\newdimen\zzlistingsizedefault
\zzlistingsizedefault=9pt
\newdimen\kwlistingsize
\kwlistingsize=9pt
\zzlistingsize=\zzlistingsizedefault
\gdef\lco{black}
%\newcommand{\keywordstyle}{\fontsize{0.9\zzlistingsize}{1.0\zzlistingsize}\bf}
%\newcommand{\keywordstyle}{\fontsize{\kwlistingsize}{1.1\kwlistingsize}\normalfont\bf\color{\lco}}
%\settowidth{\zzlstwidth}{{\Lstbasicstyle~}}




%%
%% end of the preamble, start of the body of the document source.
\begin{document}

%%
%% The "title" command has an optional parameter,
%% allowing the author to define a "short title" to be used in page headers.
\title{\bench: A \underline{Co}de E\underline{ff}ici\underline{e}ncy Benchmark for Code Generation}

%%
%% The "author" command and its associated commands are used to define
%% the authors and their affiliations.
%% Of note is the shared affiliation of the first two authors, and the
%% "authornote" and "authornotemark" commands
%% used to denote shared contribution to the research.

%%
%% By default, the full list of authors will be used in the page
%% headers. Often, this list is too long, and will overlap
%% other information printed in the page headers. This command allows
%% the author to define a more concise list
%% of authors' names for this purpose.


\author{Yun Peng}
\orcid{0000-0003-1936-5598}
\affiliation{%
  \institution{The Chinese University of Hong Kong}
  \city{Hong Kong}
  \country{China}
}
\email{ypeng@cse.cuhk.edu.hk}

\author{Jun Wan}
\orcid{0009-0006-3294-688X}
\affiliation{%
  \institution{Zhejiang University}
  \city{Hangzhou}
  \country{China}
}
\email{22451014@zju.edu.cn}

\author{Yichen Li}
\orcid{0009-0009-8370-644X}
\affiliation{%
  \institution{The Chinese University of Hong Kong}
  \city{Hong Kong}
  \country{China}
}
\email{ycli21@cse.cuhk.edu.hk}

\author{Xiaoxue Ren}
\authornote{Corresponding author.}
\authornote{Also with Hangzhou High-Tech Zone (Binjiang) Institute of Blockchain and Data Security.}
\orcid{0000-0002-5526-1617}
\affiliation{%
  \institution{The State Key Laboratory of Blockchain and Data Security, Zhejiang University}
  \city{Hangzhou}
  \country{China}
}
\email{xxren@zju.edu.cn}

%%
%% The abstract is a short summary of the work to be presented in the
%% article.

%%
%% The code below is generated by the tool at http://dl.acm.org/ccs.cfm.
%% Please copy and paste the code instead of the example below.
%%


%%
%% Keywords. The author(s) should pick words that accurately describe
%% the work being presented. Separate the keywords with commas.
%% A "teaser" image appears between the author and affiliation
%% information and the body of the document, and typically spans the
%% page.

%%
%% This command processes the author and affiliation and title
%% information and builds the first part of the formatted document.
\begin{abstract}
The performance of legged locomotion is closely tied to the accuracy and comprehensiveness of state observations. ``Blind policies", which rely solely on proprioception, are considered highly robust due to the reliability of proprioceptive observations. However, these policies significantly limit locomotion speed and often require collisions with the terrain to adapt. In contrast, ``Vision policies" allows the robot to plan motions in advance and respond proactively to unstructured terrains with an online perception module. However, perception is often compromised by noisy real-world environments, potential sensor failures, and the limitations of current simulations in presenting dynamic or deformable terrains. Humanoid robots, with high degrees of freedom and inherently unstable morphology, are particularly susceptible to misguidance from deficient perception, which can result in falls or termination on challenging dynamic terrains. To leverage the advantages of both vision and blind policies, we propose VB-Com, a composite framework that enables humanoid robots to determine when to rely on the vision policy and when to switch to the blind policy under perceptual deficiency. We demonstrate that VB-Com effectively enables humanoid robots to traverse challenging terrains and obstacles despite perception deficiencies caused by dynamic terrains or perceptual noise.
\end{abstract}      

\begin{CCSXML}
<ccs2012>
   <concept>
       <concept_id>10011007.10011074.10011092.10011782</concept_id>
       <concept_desc>Software and its engineering~Automatic programming</concept_desc>
       <concept_significance>500</concept_significance>
       </concept>
 </ccs2012>
\end{CCSXML}

\ccsdesc[500]{Software and its engineering~Automatic programming}

\keywords{Code Generation, Benchmark, Code Efficiency, Time}

\maketitle


\section{Introduction}
\label{sec:intro}

\begin{figure*}[tb]
    \centering
    \includegraphics[width=0.848\linewidth]{figs/circuitnn.pdf} 
    \caption{Illustration of differentiable CircuitNN. CircuitNN is designed based on differentiable NAND gates. After DAS is guided by PI and PO pairs of the truth table, CircuitNN can get the precise circuit architecture logic equivalent to the truth table.}
    \label{fig:circuitnn}
\end{figure*}

% 1. Describe the importance of logic synthesis
% 2. Existing Problems
% (a) Neural Architecture Search: Unstable, Predefined Setting, etc.
% (b) Circuit Generation: Probabilistic Model, Logic Equivalence

With the rapid advancement of technology, the scale of integrated circuits (ICs) has expanded exponentially. 
This expansion has introduced significant challenges in chip manufacturing, particularly concerning power and area metrics.
A primary objective in IC design is achieving the same circuit function with fewer transistors, thereby reducing power usage and area occupancy.

Logic synthesis~\cite{hachtel2005logicsynth}, a critical step in electronic design automation (EDA), transforms behavioral-level circuit designs into optimized gate-level circuits, ultimately yielding the final IC layout. 
The primary goal of logic synthesis is to identify the physical implementation with the fewest gates for a given circuit function. 
This task constitutes a challenging NP-hard combinatorial optimization problem. 
Current logic synthesis tools~\cite{brayton2010abc, wolf2013yosys} rely on human-designed heuristics, often leading to sub-optimal outcomes.

Differentiable architecture search (DAS) techniques~\cite{liu2018darts, chu2020darts} offer novel perspectives on addressing challenges in this problem.
Circuit functions can be represented through truth tables, which map binary inputs to their corresponding outputs. 
Truth tables provide a precise representation of input-output relationships, ensuring the design of functionally equivalent circuits.
Inspired by this, researchers~\cite{deepmind2024ai4sys, wang2024tnet} have begun exploring the application of DAS to synthesize circuits directly from truth tables.
Specifically, \citet{deepmind2024ai4sys} proposed CircuitNN, a framework that learns differentiable connection structures with logic gates, enabling the automatic generation of logic circuits from truth tables.
This approach significantly reduces the complexity of traditional circuit generation. 
Building on this, \citet{wang2024tnet} introduced T-Net, a triangle-shaped variant of CircuitNN, incorporating regularization techniques to enhance the efficiency of DAS.

Despite these advancements, several challenges remain. 
The computational complexity of DAS grows quadratically with the number of gates, posing scalability issues.
Although triangle-shaped architecture~\cite{wang2024tnet} partially mitigates this problem, redundancy persists. 
%Additionally, DAS is susceptible to converging to local optima, limiting the ability to search architectures that satisfy the given truth tables~\cite{liu2018darts}. 
%Furthermore, hyperparameters (network depth and layer width) require extensive searches, introducing complexity and prolonging the synthesis process. 
Additionally, DAS is susceptible to converging to local optima~\cite{liu2018darts} and hyperparameters (network depth and layer width) require extensive searches. 
The challenges arise from the vast search space in DAS. 
% Even with predefined settings for CircuitNN, finding a configuration that meets the truth table requires extensive trial and error during the DAS process. 
Intuitively, limiting the search space through predefined parameters (network depth, gates per layer, and connection probabilities) can significantly reduce the complexity.

Recent advances~\cite{openai2023gpt4, abramson2024alphafold3, esser2024sd3, li2024mar} in conditional generative models have demonstrated remarkable performance across language, vision, and graph generation tasks. 
Motivated by these developments, we propose a novel approach to circuit generation that generates preliminary circuit structures to guide DAS in generating refined circuits matching specified truth tables. 
Firstly, we introduce CircuitVQ, a tokenizer with a discrete codebook for circuit tokenization. 
Built upon our Circuit AutoEncoder framework~\cite{hou2022graphmae,li2023maskgae,wu2025mgvga}, CircuitVQ is trained through a circuit reconstruction task. 
Specifically, the CircuitVQ encoder encodes input circuits into discrete tokens using a learnable codebook, while the decoder reconstructs the circuit adjacency matrix based on these tokens.
Subsequently, the CircuitVQ encoder serves as a circuit tokenizer for CircuitAR pretraining, which employs a masked autoregressive modeling paradigm~\cite{chang2022maskgit, li2023mage}. 
In this process, the discrete codes function as supervision signals. 
After training, CircuitAR can generate discrete tokens progressively, which can be decoded into initial circuit structures by the decoder of the CircuitVQ. 
These prior insights can guide DAS in producing refined circuits that match the target truth tables precisely.

Our key contributions can be summarized as follows:
\begin{itemize}
\item We introduce CircuitVQ, a circuit tokenizer that facilitates graph autoregressive modeling for circuit generation, based on our Circuit AutoEncoder framework;
\item Develop CircuitAR, a model trained using masked autoregressive modeling, which generates initial circuit structures conditioned on given truth tables;
\item Propose a refinement framework that integrates differentiable architecture search to produce functionally equivalent circuits guided by target truth tables;
\item Comprehensive experiments demonstrating the scalability and capability emergence of our CircuitAR and the superior performance of the proposed circuit generation approach.
\end{itemize}

% Motivation
% (a) Diffusion (Vision, Graph), Autoregressive (Language, Vision)
% (b) Circuit Generation for Predefined Setting
% (c) Neural Architecture Search for Strict Logic Equivalence

% Contribution
% (a) Circuit Tokenizer (new transformer arch, training strategy)
% (b) CircuitAR (train and gen strategies, post-ar strategy)
% (c) Extensive Evaluation including BitD (Bit Distance) for Scalability

\section{Motivation}
\label{sec:motivation}



% In LLM inference, not only does weight matter, but the memory requirements of the KV Cache are also considerable.
In this section, we first demonstrate that the emerging paradigm of group quantization demands a high level of adaptivity, which current adaptive methods lack.
We then discuss how adapting these methods to group quantization could compromise their efficiency.
Given that LLMs generate KV caches during runtime, real-time quantization capability is crucial.
These challenges lead to our proposal of a mathematical adaptive numerical type (\texttt{MANT}), which we will detail later.



\begin{figure}[t]
    \centering
    \begin{minipage}[t]{0.48\columnwidth}
      \centering
      \includegraphics[width=\columnwidth]{fig/moti_group_ppl.pdf}
      \caption{LLM accuracy with different quantization granularities. We report the perplexity (PPL) metric (lower is better).}\label{fig:moti_group_ppl} 
    \end{minipage}
    \hspace{2pt}
    \begin{minipage}[t]{0.48\columnwidth}
      \centering
      \includegraphics[width=\columnwidth]{fig/motivation_adaptive_ppl.pdf}
      \caption{Accuracy loss for \texttt{INT}, \texttt{ANT}, and Ideal (clustering algorithm K-Means) adaptive methods in group quantization. }\label{fig:moti_ppl} 
    \end{minipage}
    % \vspace*{-0.3cm}
\end{figure}




\subsection{Group Quantization Accuracy Analysis}
\label{sec:acc_analysis}

In this subsection, we begin by comparing the accuracy of traditional channel-wise quantization with group-wise quantization~\cite{shao2024omniquant,zhao2023atom,liu2024kivi,sheng2023flexgen,lin2023awq,zhao2023atom}, establishing the baseline for group-wise quantization in this study.
We then delve into the use of various adaptive data types in group quantization, emphasizing the necessity for full adaptivity.



\Fig{fig:moti_group_ppl} illustrates the perplexity when quantizing the LLaMA-7B model~\cite{touvron2023llama} with various granularities using the \texttt{INT4}-based symmetric quantization.
Channel-wise quantization significantly worsens the perplexity of the examined LLM, increasing it from 5.68 to 6.85.
Conversely, group-wise quantization mitigates this loss in perplexity with a group size of 128, corresponding to an average of 4.125 bits per element (16-bit scaling factor).
Additionally, we observe that a smaller group size of 32 offers only a slight improvement in perplexity, but the scaling factor overhead increases by $4\times$.



Given this analysis, we adopt a group size of 128 as our standard configuration for the remainder of this section.
Previous research indicates that the \texttt{INT} data type is not optimal for accuracy since tensors or channels exhibit varied distributions, leading to the proposal of various adaptive data types~\cite{guo2022ant, guo2023olive, zadeh2020gobo, zadeh2022mokey}.
We evaluate their efficacy in the context of group quantization, which falls into two main categories: data-type-based and clustering-based.



\textbf{Data-type-based adaptive methods} select data types from discrete sets based on tensor data distribution.
ANT~\cite{guo2022ant} is a representative example of the data-type-based method.
ANT packages several different data types for selection, including \texttt{INT} for the uniform distribution, \texttt{PoT} (Power of Two) for the Laplace distribution, and \texttt{flint} for the Gaussian distribution.
%ANT designed \texttt{flint} for Gaussian distributions.

\textbf{Clustering-based adaptive methods} utilize clustering algorithms to generate centroids that align with the data distribution and provide considerable adaptivity. 
Mokey~\cite{zadeh2022mokey} and GOBO~\cite{zadeh2020gobo} exemplify this approach, though they focus on tensor- or channel-wise quantization. In our study, we adapt them to group quantization through per-group clustering.

%Clustering-based methods employ clustering algorithms to generate centroids that fit the data distribution, demonstrating sufficient adaptivity.
%Mokey~\cite{zadeh2022mokey} and GOBO~\cite{zadeh2020gobo} are such presentative works, but only target tensor- or channel-wise quantization.
%In our work, we modify those works to support group quantization by performing per-group clustering.
\Fig{fig:moti_ppl} compares the accuracy of the methods described above for the LLaMA-7B model under 4-bit group-wise quantization. 
The group-wise \texttt{ANT} method outperforms the \texttt{INT} type by dynamically selecting from three data types to better match the value distribution, resulting in reduced perplexity (PPL) loss. 
Moreover, per-group clustering adjusts more effectively to the value distribution of each group, establishing itself as the accuracy-optimal and ideal adaptive method. 
This approach achieves nearly lossless 4-bit quantization, equivalent to 16 centroids per group. 
However, this ideal scenario is impractical due to the significant overhead associated with storing per-group centroids, effectively rendering it a 6-bit quantization.

\begin{figure}[t] 
    \centering 
    \includegraphics[width=1.0\linewidth]{fig/intro_cdf.pdf}  
    \caption{The cumulative distribution function (CDF) of the tensor, channel, and group, respectively. The tensor data were taken from layers 8 to 23, while the 16 channel and group data were sampled from one tensor with specific strides.}\label{fig:moti_dist} 
    %  \vspace*{-0.3cm}
\end{figure}

To illustrate the group-wise diversity in data distribution, we sampled the weights of the Q and V tensors in LLaMA-7B model. 
We normalized all sampled data to their absolute maximum values, which ranged from -1 to 1. \Fig{fig:moti_dist} displays the cumulative distribution function (CDF) for the tensor, channel, and group levels, respectively. 
We observed that the diversity at the group level is significantly higher than at the tensor level. 
In simpler terms, while different tensors exhibit similar distributions, groups can have markedly different distributions. This finding underscores the necessity for full adaptivity in group quantization to fully realize its potential.
\paragraph{Takeaway 1.} The group quantization is an emerging paradigm to accelerate LLMs, and the significant group-level diversity requires a high level of adaptivity to fully unleash its potential.

\subsection{Group Quantization Efficiency Analysis}
\label{subsec:efficiency}


In this subsection, we provide a detailed efficiency analysis for the above adaptive quantization methods.
In \Tbl{intro:dtype}, we compare OliVe~\cite{guo2023olive}, ANT~\cite{guo2022ant}, GOBO~\cite{zadeh2020gobo}, and Mokey~\cite{zadeh2022mokey} with \texttt{INT} regarding the efficiency of computation, encoding, and decoding. 
In this paper, we use the term encoding (decoding) interchangeably with quantization (dequantization).
 

Data-type-based adaptive methods such as ANT~\cite{guo2022ant} and Olive~\cite{guo2023olive} achieve computational efficiency comparable to \texttt{INT}. 
Both utilize specialized decoders that decode these data types prior to computation, resulting in high decoding efficiency. 
However, as previously demonstrated, these methods suffer from limited adaptivity in the group quantization paradigm. 
A straightforward approach to enhance adaptivity is to expand their set of data types. 
However, incorporating new data types necessitates additional decoders, escalating hardware design costs. 
Additionally, compatibility issues between new and existing data types may reduce computational efficiency. 
For instance, the \texttt{NF4} data type~\cite{dettmers2023qlora} requires an FP16 MAC unit, which is incompatible with existing \texttt{ANT} data types.


\paragraph{Takeaway 2.} Enhancing the data-type-based adaptive method for group quantization is challenging and requires a careful balance for the computation and decoding efficiency.

Clustering-based adaptive methods like GOBO~\cite{zadeh2020gobo} and Mokey~\cite{zadeh2022mokey} can sufficiently adapt to various distributions at the group level. 
However, they require codebooks for quantization and dequantization, leading to high adaptivity at the expense of encoding and computational efficiency. 
For instance, a 16-entry codebook with 8 bits per entry requires 128 bits per group, creating an inevitable trade-off between adaptivity and memory overhead. GOBO~\cite{zadeh2020gobo} employs the K-means algorithm to quantize weights and requires dequantization to \texttt{FP16} using a codebook lookup table before computation, resulting in high adaptivity but low computational efficiency. 
Conversely, Mokey~\cite{zadeh2022mokey} enhances the computation of clustering-based methods by using indices for centroid values via approximate calculations, though matrix multiplication still relies on floating-point units, increasing overhead compared to integer units. 
Furthermore, Mokey creates one \texttt{golden dictionary} for all activations and weights, akin to using a single data type in quantization, thus reducing adaptivity.


\paragraph{Takeaway 3.} Deploying the clustering-based adaptive methods under group quantization is challenging owing to the low encoding and computation efficiency. 


\begin{table}[t]
    \centering
    \small
    \renewcommand{\arraystretch}{1.2}
    \caption[]{Features of DNN accelerators with adaptive and flexible data types are summarized. Here, `Effi.' stands for efficiency, `Med.' for medium, and `LUT' for lookup table.}
  
    \resizebox{1.0\columnwidth}{!}{
      \begin{tabular}{c|cc|ccc|cc|c}
        \Xhline{1.2pt}
        \multirow{2}{*}{Architecture} & \multicolumn{2}{c|}{Encode} & \multicolumn{3}{c|}{Computation} & \multicolumn{2}{c|}{Decode} & \multirow{2}{*}{Adaptivity} \\ \cline{2-8}
        & Method & Effi. & Method & Bit & Effi. & Method & Effi. \\
        \Xhline{1.2pt}
        \texttt{INT} & Round & High & INT & 4 \& 8 & High & Calculation & High & Low \\ 
        OliVe~\cite{guo2023olive} & Search & Med. & INT & 4 \& 8 & High & Decoder & High & Med. \\ 
        ANT~\cite{guo2022ant} & Search & Med. & INT & 4 \& 8 & High & Decoder & High & Med. \\ 
        Mokey~\cite{zadeh2022mokey} & Cluster & Med. & Float & 4 \& 8 & Med. & Calculation & Med. & Low \\ 
        GOBO~\cite{zadeh2020gobo} & Cluster & Low & Float & 16 & Low & LUT & Med. & High \\ 
        \hline
        \multirow{2}{*}{\proj}  & Search  & Med.  & \multirow{2}{*}{INT} & \multirow{2}{*}{4 \& 8} & \multirow{2}{*}{High} & \multirow{2}{*}{Calculation} & \multirow{2}{*}{High} & \multirow{2}{*}{High} \\ \cline{2-3}
        &  Map &  High &  &&&\\ 
        \Xhline{1.2pt}
    \end{tabular}
    }
    \vspace*{0.1cm}
    \label{intro:dtype}
    \vspace*{-0.2cm}
  \end{table}

\subsection{Support for Real-time Quantization}
\label{sec:moti_kvcache}

The above group-wise diversity presents a challenge for both weights and KV cache.
In addition, KV cache faces challenges in real-time group-wise quantization because the KV cache is generated dynamically during LLM inference.


To facilitate low-precision computation in group-wise quantization, it is necessary to quantize K and V along the inner dimension. 
This requirement stems from the support for matrix inner product operations in most GPUs and TPUs. 
During these operations, the group-wise scaling factor can be extracted from the multiply-accumulate process. 
\Fig{fig:kv_process} depicts the computation process of K and V during the decode stage. We define the dimension used for matrix inner product operations as the inner dimension. 
The inner dimensions of the K and V caches differ; the K cache requires a transpose operation, whereas the V cache does not, complicating the situation.


In the prefill stage, K and V can easily compute the scaling factor for each group. 
During the decode stage, the newly generated K vector is concatenated along the inner dimension of the K cache, enabling immediate quantization. 
However, the newly generated V vector is associated with different groups, with only one element per group produced per iteration. This process prevents the scaling factor for the entire group from being obtained in a single iteration, posing a significant challenge for the real-time quantization of the V cache.


\begin{figure}[t] 
  \centering 
  % \includegraphics[width=1.0\linewidth]{fig/dse_kv_process.pdf}  
  \includegraphics[width=0.9\linewidth]{fig/moti_kv_dimension.pdf}  
  \caption{\small Comparison of group-wise K and V cache quantization. They have different inner dimensions due to the transposition of K (key).}

  \label{fig:kv_process}
  % \vspace*{-0.4cm}
\end{figure}


Given those challenges, we propose \proj with a mathematical encoding format that can fuse with integer computation and enhance the decoding efficiency.
In addition, this encoding format provides sufficient adaptivity for group-wise quantization.
Regarding the challenge in KV cache, \proj employs a real-time quantization engine that ensures efficient encoding and decoding for KV cache.
By addressing these challenges, \proj enables efficient low-bit group-wise quantization.


\section{Methodology}\label{sec:meth}

This section describes how we build the benchmark \bench, including selecting the coding problems, proposing \tool to generate stressful test cases for function-level and file-level code generation, and designing a novel time efficiency metric efficient@k.

\subsection{Data Preparation}\label{sec:data}


To construct \bench, we collect problems in the test splits of two existing function-level correctness benchmarks (i.e., HumanEval~\cite{humaneval} and MBPP~\cite{mbpp}), and two existing file-level correctness benchmarks (i.e., APPS~\cite{apps} and CodeContests~\cite{codecontests}). Each benchmark contains multiple coding problems and provides each problem with a description that explains the requirements in natural language, several ground truth solutions that address the problem, and several test cases that evaluate the correctness of generated code solutions. As there are multiple versions for MBPP, we choose the common subset of the sanitized version~\cite{mbppsantized} and the MBPP+ benchmark verified by EvalPlus~\cite{mbppplus} as our base benchmark to ensure the highest quality. 

With the selected benchmarks, we first validate the problems by checking the potential conflicts of provided test cases and ground truth solutions. Secondly, we select problems that most LLMs could correctly answer to reduce the difficulty of problems for the two file-level benchmarks since a problem is not useful in time efficiency evaluation if no LLM can answer it. We show the statistics of four benchmarks in Table~\ref{tab:datasets}.

\begin{table}[h] \footnotesize  \centering\resizebox{0.48\textwidth}{!}{\begin{tabular}{c|l|c|c|c}
\toprule
\textbf{Task} & \textbf{Dataset} & \textbf{N-shot} & \multirowcell{\textbf{Train texts} \\ \textbf{for STMD}} & \multirowcell{\textbf{Evaluation} \\ \textbf{texts}} \\
\midrule
\multirow{3}{*}{\multirowcell{Text \\ Summarization}} & CNN/DailyMail & 0 & 2,000 & 2,000 \\
& XSum & 0 & 2,000 & 2,000 \\
& SamSum & 0 & 2,000 & 819 \\
\midrule
\multirow{4}{*}{\multirowcell{QA \\ Long answer}} & PubMedQA & 0 & 2,000 & 2,000 \\
& MedQUAD & 5 & 2,000 & 2,000 \\
& TruthfulQA & 5 & 408 & 409 \\
& GSM8k & 5 & 2,000 & 1,319 \\
\midrule
\multirow{4}{*}{\multirowcell{QA \\ Short answer}} & SciQ & 0 & 5,000 & 1,000 \\
& CoQA & \multirowcell{all preceding \\ questions} & 5,000 & 2,000 \\
& TriviaQA & 5 & 5,000 & 2,000 \\
\midrule
\multirow{1}{*}{\multirowcell{MCQA}} & MMLU & 5 & 5,000 & 2,000 \\
\bottomrule
\end{tabular}
}\caption{\label{tab:dataset_stat} The statistics of the datasets used for evaluation.}
\end{table}

\subsubsection{Problem Validation}
To ensure the quality of test cases and ground truth solutions in the four benchmarks, we run the ground truth solutions in the provided test cases and remove 1) ground truth solutions that cannot pass the provided test cases to ensure consistency, 2) ground truth solutions with file operations to keep safety, and 3) problems without valid ground truth solutions and test cases. We show the number of validated problems in each benchmark in the third column of Table~\ref{tab:datasets}. All problems in HumanEval and MBPP can be successfully validated, so no problem is removed. For the Code Contests benchmark, we identify five problems with file operations, and we remove them to guarantee the safety of testing environments. For the APPS benchmark, we identify 1,894 problems whose ground truth solutions conflict with the provided test cases. The reason for such conflicts is that the APPS benchmark does not require the output of a code solution to exactly match the expected outputs in test cases to be correct, which differs from the other three benchmarks. We remove the 1,894 problems without exact matches in the APPS benchmark to maintain consistent evaluation standards.

\subsubsection{Problem Selection}
Current LLMs are quite effective in function-level code generation by achieving a pass@1 of more than 80\% in the HumanEval benchmark, as discussed in Sec.~\ref{sec:intro}. However, they perform much worse in file-level code generation since the most powerful LLM has a Pass@1 of 28.5\% on the Code Contests benchmark and a Pass@1 of less than 10\% on the APPS benchmark~\cite{codecontestleaderboard,appspsc}. This limits the usage of the full set of the Code Contests and APPS benchmarks because a problem that no LLM can correctly answer does not contribute to the time efficiency evaluation.
Therefore, for the validated problems in the two benchmarks, we sample one code solution with temperature 0 on 14 LLMs used in our experiments described in Table~\ref{tab:models} and remove 48 and 2,223 problems that code solutions from all LLMs failed in the Code Contests and APPS benchmark, respectively. To balance the number of problems in the function-level split and file-level split of \bench, we further select 300 problems in the APPS benchmark for which more LLMs can generate correct code solutions. We show the number of selected problems from the four benchmarks and associated statistics in the 4$\sim$6 columns of Table~\ref{tab:datasets}.



\subsection{Stressful Test Case Generation: \tool}
\begin{figure*}[tbp]
    \centering
    \includegraphics[width=1.0\textwidth]{figures/overview.pdf}
    \vspace{-20pt}
    \caption{The overview workflow of \tool.} 
    \label{fig: overview}
    \vspace{-10pt}
\end{figure*}

With the selected problems, we propose a novel LLM-based approach \tool to generate stressful test cases automatically. In contrast to current LLM-based test case generation methods~\cite{liu2023is, li24large,ouédraogo24large,bhatia23unit,max24an,sami24a}, \tool aims to generate test cases to evaluate the time efficiency of code solutions under extreme conditions rigorously. This inherently requires constructing exceptionally long and intricate inputs that can hardly be handled by LLMs directly, leading to unsatisfactory accuracy, i.e., the proportion of correctly generated stressful test cases is low.


\subsubsection{Overview}

To improve the accuracy of stressful test case generation, \tool introduces contracts to guide the test case generation and validate the generated test cases. Contracts are collections of assertion statements that record the type, scale, and internal constraints between the inputs. Providing contracts in the test case generation process can help LLMs understand the dependencies between test inputs. Besides, \tool can easily identify incorrect test cases from the assertion errors contracts raise. To avoid overlong stressful test cases that hinder the performance of LLMs, we design two new formats of test cases by reformulating the test case generation task into a code generation task: \textit{expression test cases} and \textit{generator test cases}. Different from raw test cases that directly provide test inputs, expression and generator test cases contain code to generate test inputs, which greatly shortens the length of test cases. 


We present the overview of \tool in Figure \ref{fig: overview}. \tool does not directly generate stressful test cases. Instead, it decomposes the task into three phases: 1) contract generation, 2) stressful test case generation, and 3) test case-contract pair check. In the first phase, \tool generates contracts by analyzing the target program, i.e., the ground truth solution for each problem in the benchmark. The generated contracts are then provided as demonstrations for stressful test case generation in the second phase, in which \tool generates expression and generator test cases instead of raw test cases. Since contracts are also generated and there is no guarantee of their correctness, \tool enters the third phase if the number of \textit{AssertionError} occurrences for a certain contract exceeds a threshold. In the third phase, \tool implements an LLM judge to determine the responsibility for conflicts between generated contracts and test cases. The contracts or test cases that are judged to be incorrect will be sent back for regeneration. This iterative process allows the generation of contracts and stressful test cases to mutually reinforce each other.

\subsubsection{Phase I: Contract Generation}\label{sec:contract}
In the first phase, \tool inserts assertion statements that check the preconditions of inputs as contracts into the target program, such as \textit{assert n > 0}. The contracts ensure that the inputs meet the required specifications in format (e.g., variable type), scale (e.g., input length, order of magnitude), and intrinsic constraints (e.g., right triangle side lengths).

The benefits of inserting contracts before stressful test case generation are twofold: 1) \textbf{Knowledge Enrichment.} Contracts explicitly indicate the functionality of the target program and the dependencies between inputs, which can help LLM better understand natural language descriptions provided in problems~\cite{liu2023is,endres2024can}; 2) \textbf{Early Validation.} Contracts can identify invalid inputs in test cases at the beginning of program execution and stop the execution-based test case validation process early, which largely improves the efficiency of the test case generation process.

In contract generation, \tool generates one assertion statement in an iteration and combines all assertion statements into a contract. When generating assertion statements, \tool prompts LLMs to consider the type, scale, and intrinsic constraints between inputs given the target program, existing correctness test cases, and previously generated assertion statements as demonstrations. \tool implements the same methodology to generate assertion statements for function-level and file-level target programs. However, \tool employs different strategies to insert assertion statements into target programs, given the differences between the code solutions in function-level and file-level code generation illustrated in Sec.~\ref{sec:mot}.

\textbf{Function-level Contract Insertion.} For function-level target programs with a determined number of inputs, \tool generates and inserts assertion statements for function parameters at the beginning of the function body. For example, \tool inserts assertion statements right before the return statements in the function \textit{add()} in Figure ~\ref{fig:probdef}(a). 


\textbf{File-level Contract Insertion.} For file-level target programs with an unknown number of inputs and multiple input locations, \tool reformulates the contract generation problem into a code editing problem. It first identifies all input locations by checking the related system APIs such as \textit{input()} and then inserts assertion statements for each identified input location sequentially. \tool inserts assertion statements right after the input locations in most cases. However, as input locations in loops generally assign values for generic types such as \textit{list} and \textit{dict}, \tool inserts assertion statements after the entire loop where the assignments are complete to check the fully assigned types. For example, \tool identifies two input locations highlighted in blue in Figure~\ref{fig:probdef}(b). \tool first generates assertion statements for the input that assigns values to variable $n$ and inserts them right after the assignment. \tool then generates assertion statements for the second input in the loop, and this time, it inserts assertion statements after the entire \textit{for} loop. 


To improve the correctness of generated assertion statements, \tool tests all generated assertion statements against the correctness test cases each time it inserts a new assertion statement. If the current assertion statement fails on the test cases, \tool only removes the current assertion statement and regenerates a new one while maintaining the assertion statements correctly generated in previous iterations. The iteration ends until no new assertion statements are generated or a maximum iteration number is reached.



\subsubsection{Phase II: Stressful Test Case Generation}
With the generated contracts as demonstrations, in the second phase, \tool generates stressful test cases. Unlike correctness test case generation, it is quite challenging to generate stressful test cases because LLMs must generate test cases of maximal length and complexity within the constraints of its finite context window while simultaneously ensuring adherence to intrinsic input constraints specified by contracts. Correctness test cases in current benchmarks~\cite{humaneval,mbpp,codecontests,apps} are raw test cases that directly provide the values for test inputs. However, due to the limited context window size, it is infeasible to directly generate overlong raw test cases for time efficiency evaluation. For example, it is hard for LLMs to generate a list with more than a million numbers for stressful tests. To address this challenge, we introduce two new formats of stressful test cases:


\textbf{Expression Test Cases.} Expression test cases utilize Python expressions to generate test cases, allowing for more complex input generation while maintaining a compact representation within the LLM's context window. For instance, a list with a million numbers could be easily generated by an expression ``\textit{[random.randint(1, 100000) for \_ in range(1000000)]}'', which is much shorter than listing a million numbers. Expression test cases offer a balance between complexity and conciseness, enabling the creation of structured inputs. They are suitable for function-level test case generation with a determined number of test inputs. To evaluate code solutions on expression test cases, we just need to execute the expressions to get the real test inputs before the code execution. 



\textbf{Generator Test Cases.} Generator test cases are Python functions that output the test inputs. It is quite useful for creating stressful test cases that require intricate logical relationships or patterns that are difficult to express in single expressions. For example, it is suitable for file-level code generation where the number of inputs is undetermined. Expression test cases cannot handle this since we do not know how many expressions should be generated.

To generate expression and generator test cases, \tool prompts the LLMs with contract, verified generated test cases as demonstrations, so LLMs can learn the dependencies between inputs as well as the specific formats of the expected test cases. The generated test cases are then verified against the previously generated contracts and the target program. Test cases that pass the validation of contracts and the execution of the target program are collected to build \bench. Verified stressful test cases are also used as demonstrations to help generate the following stressful test cases.

\subsubsection{Phase III: Test Case-Contract Pair Check}

Although the generated contracts are verified against the existing correctness test cases, correctness test cases do not cover all possible cases and dependencies among inputs, especially in stressful scenarios. Contracts can still make mistakes and induce false positives. During the test case validation, if a generated test case violates the inserted contract, it triggers an \textit{AssertionError}. If the \textit{AssertionError} consistently occurs for multiple test cases, the contract may be incorrect and thereby hinder the entire stressful test case generation procedure. To mitigate this, when the number of conflicts between contracts and test cases (i.e., \textit{AssertionError} occurrences during execution) exceeds a predefined threshold (5 in this paper), the generated test case and the violated contract are paired for further check by an LLM judge checker in the third phase.


The LLM judge takes all accumulated contract-related execution failure pairs as inputs, along with the target program, to analyze and determine the validity of the contracts and the test cases. The judge reviews the violated contract with exact stressful inputs, rethinks the correctness of the generated contract, and determines the root cause of conflicts. Once the root cause is identified, the relevant judgment results and corresponding failure pairs are sent back to the previous phases for regeneration. By providing feedback for incorrect contracts or test cases, \tool enhances the robustness of the test case validation and enables the improvements between contract generation and test case generation. To prevent duplicate judgments, once the LLM judge determines that a contract is valid in the third phase, it will not be checked again, and test cases that fail the validation of this contract will be directly rejected in the future.



\subsection{Time Efficiency Metric: Efficient@k}\label{sec:metric}

Previous work~\cite{shypula2024learning,effibench} intuitively adopts execution time as the performance measurement to evaluate the time efficiency of LLM-generated code. However, execution time measurements could be affected by many factors, such as process scheduling and disk I/O, so it is not stable enough to make a solid comparison between the time efficiency of different code solutions. In this section, we propose to use \textit{CPU instruction count} to replace execution time to measure the time efficiency of code solutions stably. Based on CPU instruction count measurements, we propose a new metric \textit{efficient@k} to evaluate both the correctness and time efficiency of code solutions.


\subsubsection{CPU Instruction Count} To find a more stable measurement to replace execution time, we first look into the factors contributing to the execution time. Patterson and Hennessy~\cite{arch} define the CPU time cost by a program through the following equation.
\begin{equation}\label{eq:cic}
    \text{CPU Time} = [\textbf{\text{Instruction Count}}] \times [\text{Clock per Instruction}] \times [\text{Clock Cycle Time}]
\end{equation}

From the equation, the CPU time of a program is determined by three factors. While \textit{Clock per Instruction} and \textit{Clock Cycle Time} depend on the physical machine where the program runs, the only factor related to the program is \textit{Instruction Count}. Therefore, if a program has a higher CPU instruction count on the same machine, it is less efficient, and vice versa. Unlike the execution time measurements that could be affected by many factors, CPU instruction count measurements are more stable as CPU instruction count for a program does not increase even if the program execution is slowed or stalled by external factors. It is also straightforward to measure CPU instruction count using the system APIs. For example, Linux provides a command tool named \textit{perf}~\cite{perf} to support CPU instruction count measurements.


\subsubsection{Efficient@k} CPU instruction count is a stable measurement for the time efficiency evaluation of different code solutions. However, its absolute value is not meaningful as the same code solution has different CPU instruction counts in different machines. Besides, it is not comprehensive as it does not measure the correctness of generated code solutions. To address these problems, we propose a new metric named \textit{Efficient@k}, inspired by the design of \textit{pass@k}~\cite{humaneval}. We show the original definition of pass@k in Equation~\ref{eq:pass} and the definition of proposed efficient@k in Equation~\ref{eq:eff}.

\begin{equation}\label{eq:pass}
    \text{pass@k}:=\underset{\text { Problems }}{\mathbb{E}}\left[1-\frac{\binom{n-c}{k}}{\binom{n}{k}}\right]
\end{equation}

\begin{equation}\label{eq:eff}
    \text{efficient@k}:=\underset{\text { Problems }}{\mathbb{E}}\left[1-\frac{\binom{n-c_f}{k}}{\binom{n}{k}}\right]
\end{equation}

Pass@k is an expectation over all problems in the benchmark for the probability that at least one solution in $k$ samples can pass all test cases. In equation~\ref{eq:pass}, total $n$ solutions are sampled from LLMs instead of only $k$ samples to reduce the variance. By running the sampled code solutions on correctness test cases, we can get the solutions $c$ that can pass all the test cases to estimate the probability of correctness. Pass@k is a solid metric with low variance and can be easily reproduced under different platforms. 

We follow the idea of pass@k when designing efficient@k. Pass@k requires the correct code solutions $c$ to contribute, while in efficient@k, we collect the number $c_f$ of the correct solutions faster than the best ground truth solution to replace $c$ in pass@k. Therefore, efficient@k evaluates the probability of LLMs to generate correct and fast enough code solutions. Efficient@k compares the CPU instruction count of code solutions and ground truth solutions to determine which runs faster. By doing so, efficient@k does not consider the absolute values of CPU instruction counts to avoid the impacts of specific systems or machines. With a value range from 0 to pass@k, efficient@k combines correctness and time efficiency evaluation to comprehensively evaluate the quality of code solutions.


\subsection{Code Efficiency Benchmark: \bench}

Frontier language models demonstrate a remarkable mismatch between their problem-solving capabilities and poor out-of-box verification capabilities.
These limitations have largely been attributed to the inability of current language models to self-diagnose hallucinations or enforce rigour \citep{zhang_how_2023,orgad_llms_2024,snyder_early_2024,kamoi_evaluating_2024, tyen_llms_2024, DBLP:conf/iclr/0009CMZYSZ24}.
However, our findings that models can be directed to accurately perform verifications at scale suggest that these out-of-box limitations can be addressed with standard methods like instruction tuning.
We compiled a set of challenging reasoning problems and candidate solutions to provide a benchmark for these deficits.

Each entry in this benchmark consists of a question, a correct candidate response, and an incorrect candidate response, and is manually curated from the residuals of our sampling-based search experiments (Section~\ref{section:pipeline}).
An example entry from this benchmark can be found below (see Appendix~\ref{app:examplebenchmark} for more).

\vspace{0.4cm}
\begin{tcolorbox}[title=Question from LiveBench Reasoning (Web-of-Lies Puzzle), breakable]
In this question, assume each person either always tells the truth or always lies. The person at the campground thinks their friend is lying. Mateo is at the aquarium. The person at the restaurant says the person at the hotel lies. Farid is at the movie theater. The person at the movie theater says the person at the campground lies. Ryan is at the shopping mall. The person at the cafe says the person at the campground lies. The person at the observatory says the person at the museum lies. The person at the museum says the person at the restaurant tells the truth. The person at the campground says the person at the airport tells the truth. The person at the campground tells the truth. Luna is at the restaurant. The person at the movie theater saw a firetruck. Isabella is at the observatory. The person at the amusement park says the person at the museum tells the truth. The person at the restaurant says the person at the shopping mall tells the truth. Liam is at the beach. Ayaan is at the airport. The person at the restaurant says the person at the beach tells the truth. Maya is at the amusement park. The person at the restaurant says the person at the aquarium lies. The person at the restaurant says the person at the movie theater lies. The person at the library says the person at the campground tells the truth. Elowen is at the hotel. Priya is at the library. The person at the gym says the person at the restaurant tells the truth. Kehinde is at the cafe. Anika is at the museum. Quan is at the campground. Soren is at the gym. Does the person at the museum tell the truth? Does the person at the restaurant tell the truth? Does the person at the movie theater tell the truth? Think step by step, and then put your answer in **bold** as a list of three words, yes or no (for example, **yes, no, yes**). If you don't know, guess.
\end{tcolorbox}
\definecolor{mydarkgreen}{rgb}{0.0, 0.5, 0.0} %


\begin{paracol}{2}

\switchcolumn[0]
\begin{tcolorbox}[breakable, colframe=mydarkgreen!70, colback=mydarkgreen!10, title=Correct response]
Here's how we can break down this logic puzzle:

1. **Quan is at the campground and tells the truth.** This is a given fact.

...

4. **The person at the restaurant (Luna) says the person at the hotel (Elowen) lies.**  We don't know yet if Luna is lying or telling the truth.

...

10. **Since Luna tells the truth about Farid, Luna *always* tells the truth.**  This unlocks several other pieces of information.

...

13. **The person at the cafe (Kehinde) says the person at the campground lies.** Since Quan tells the truth, Kehinde is lying.

Therefore:
* **Anika (museum):** Tells the truth.
* **Luna (restaurant):** Tells the truth.
* **Farid (movie theater):** Lies.

**yes, yes, no**
\end{tcolorbox}

\switchcolumn
\begin{tcolorbox}[breakable, colframe=red!70, colback=red!10, title=Incorrect Response Due to Hallucination]
Here's how we can solve this logic puzzle:

1. **Quan is at the campground and tells the truth.** This is given.

2. **The person at the airport (Ayaan) tells the truth.** Quan (campground) says the person at the airport tells the truth, and Quan tells the truth.

...

4. **Luna (restaurant) lies.**  The person at the restaurant says the person at the hotel (Elowen) lies. \textbf{\color{red} Later, the person at the restaurant says the person at the campground (Quan) lies.} Since Quan tells the truth, Luna (restaurant) must be lying.

...

Therefore:
* **Anika (museum): lies**
* **Luna (restaurant): lies**
* **Farid (movie theater): lies**

**Answer: no, no, no** 
\end{tcolorbox}
\end{paracol}

\noindent\begin{minipage}{\textwidth}
\captionof{figure}{Example of an entry in our verification benchmark. The question is sourced from the LiveBench Reasoning benchmark, and the two responses are generated by Gemini v1.5 Pro. The green response has the correct final answer; the red response has the wrong final answer due to hallucinating a non-existent clause.}
\label{fig:example}
\vspace{0.4cm}
\end{minipage}

\noindent
On each entry, our benchmark studies verification accuracy on two tasks:
\begin{enumerate}
    \item \textbf{Scoring task.} When given only the question and one of the responses, is the model able to discern the correctness of the response?
    \item \textbf{Comparison task.} When provided the whole tuple with the correctness labels of the responses masked and a guarantee that at least one response is correct, is the model able to discern which response is correct and which is incorrect?
\end{enumerate}

\noindent
The scoring task is also evaluated over a separate set of (question, response) pairs where the response reaches the correct final answer by coincidence but contains fatal errors and should be labeled by a reasonable verifier as being incorrect; an example can be found in Appendix~\ref{app:examplebenchmark}.
In the scoring task, models are provided only with the task description; in the comparison task, models are provided only with the task description and a suggestion to identify disagreements between responses in its reasoning.

Table~\ref{tab:benchmark} lists the baseline performances of current commercial model offerings on this benchmark.
Gemini v1.5 Pro is omitted from the benchmark as the entries in the benchmark are curated from the residuals of Gemini v1.5 Pro.
The prompts used in Table~\ref{tab:benchmark} are provided in Appendix~\ref{app:benchmarkprompts}.

As we previously observed, and has been noted in prior works \citep{tyen_llms_2024, kamoi_evaluating_2024}, verification errors are typically due to low recall.
Even the easier comparison task, models perform only marginally better---and often worse---than random chance.
In many cases, Consistency@5 performs worse than one-shot inference because Consistency simply averages out noise from an output distribution, meaning that a model biased towards producing an incorrect answer will do so with higher probability under Consistency.
Addressing these deficits in verification capabilities---which we see as low-hanging fruit for post-training---would enable not only better sampling-based search, but also other downstream applications of verification including reinforcement learning \citep[e.g.][]{o1-preview,deepseekai2025deepseekr1incentivizingreasoningcapability}, data flywheeling \citep[e.g.,][]{welleck_generating_2022}, and end-user experience (see Section~\ref{sec:related} for further discussion).


\begin{table}[htbp]
\centering
\begin{tabular}{llcccccc}
\toprule
\textbf{Model} & \textbf{Metric} & \multicolumn{3}{c}{\textbf{Scoring Accuracy}} & \multicolumn{1}{c}{\textbf{Comparison Accuracy}} \\
\cmidrule(lr){3-5} \cmidrule(lr){6-6}
 &  & \textbf{Correct} & \textbf{Wrong} & \textbf{Flawed} &  \\
\midrule
\multirow{2}{*}{GPT-4o} & Pass@1    & 76.5\%  & 31.0\% & 22.2\% & 43.2\%\\
 & Consistency@5 & 77.4\% & 30.0\% & 11.1\% & 35.4\% \\
\midrule
\multirow{2}{*}{Claude 3.5 Sonnet} & Pass@1 & 89.6\% & 22.5\% & 33.3\% & 56.1\% \\
 & Consistency@5 & 90.3\% & 17.5\% & 33.3\% & 61.2\% \\
\midrule
\multirow{2}{*}{o1-preview} & Pass@1 & 100\% & 68.8\% & 80.0\% & 84.5\% \\
 & Consistency@5 & 100\% & 79.4\% & 88.8\% & 92\% \\
\midrule
\multirow{2}{*}{Gemini 2.0 Flash} & Pass@1 & 73.5\% & 44.5\% & 60\% & 58\%  \\
 & Consistency@5 & 77.4\% & 42.5\% & 66.6\% & 58.7\% \\
\midrule
\multirow{2}{*}{Gemini 2.0 Thinking Flash} & Pass@1 & 75.4\% & 56.5\% & 53.3\%  & 80\%  \\
 & Consistency@5 & 77.4\%  & 55\% & 55.5\%  & 89.1\% \\
\midrule
\multicolumn{2}{c}{Random guessing}  & 80\% & 20\% & 20\% & 50\% \\
\bottomrule
\end{tabular}
\caption{Accuracy rates of commercial language models on our verification benchmark. For the task of response scoring (Scoring Accuracy), accuracy rates are broken down for entries that require identifying a correct response as being correct (Correct), entries that require identifying a wrong response as being wrong (Wrong), and entries that require identifying a wrong response that coincidentally reaches the correct answer as being wrong (Flawed).
GPT-4o and Claude 3.5 Sonnet only perform marginally better than random guessing across all tasks. o1-Preview performs better, but still fails to identify 20-30\% of wrong responses.
}
\label{tab:benchmark}
\end{table}


With the stressful test case generation approach \tool, we add stressful test cases for each problem selected in Sec.~\ref{sec:data}. Specifically, we generate 20 stressful test cases for each problem and measure the CPU instruction count each test case costs. We conduct the measurements 12 times and remove the highest and lowest measurements before calculating the average to ensure the most stable results. In the CPU instruction measurements, we limit the execution time of one single measurement to five seconds so that the measurements for one test case will not exceed one minute. We then rank the average CPU instruction count of each test case and include the five test cases with the highest CPU instruction counts in \bench. We do not include all generated stressful test cases in \bench to avoid large time costs in time efficiency evaluation, since stressful test cases generally take much longer time than correctness test cases to execute. We reserve all existing correctness in \bench to validate the correctness of generated code solutions. We show the statistics of \bench in Table~\ref{tab:benchmark}.
\iffalse
\begin{table*}[htbp]
\tiny
\begin{center}
\begin{tabular}{lccccccccccccc}\toprule
Model, ft setting & mem & \#param & ARC-c & ARC-e & BoolQ & HS & OBQA & PIQA & rte & SIQA & WG & Avg
%\\\cmidrule(lr){2-3}\cmidrule(lr){4-5} \cmidrule(lr){6-7} \cmidrule(lr){8-9}\cmidrule(lr){10-11} \cmidrule(lr){12-13} \cmidrule(lr){14-15} \cmidrule(lr){16-17} 
\\\cmidrule(lr){1-13}
Llama2(7B), LoRA, $r=64$ & 23.46GB & 159.9M(2.37\%) & \textbf{44.97} & 77.02 & 77.43 & \textbf{57.75} & 32.0 & \textbf{78.45} & 62.09 & \textbf{47.75} & 68.75 & 60.69\\
Llama2(7B), SPruFT, $r=128$ & \textbf{17.62GB} & 145.8M(2.16\%) & 43.60 & \textbf{77.26} & \textbf{77.77} & 57.47 & \textbf{32.6} & 78.07 & \textbf{64.98} & 46.67 & \textbf{69.30} & \textbf{60.86} \\\cmidrule(lr){2-13}
Llama2(7B), FA-LoRA, $r=64$ & 17.25GB & 92.8M(1.38\%) & 43.77 & \textbf{77.57} & 77.74 & \textbf{57.45} & 31.0 & 77.86 & \textbf{66.06} & \textbf{47.13} & 69.06 & 60.85\\
Llama2(7B), FA-SPruFT, $r=128$ & \textbf{15.21GB} & 78.6M(1.17\%) & \textbf{43.94} & 77.22 & \textbf{77.83} & 57.11 & \textbf{32.0} & \textbf{78.18} & 65.70 & 46.47 & \textbf{69.38} & \textbf{60.87}\\\midrule
Llama3(8B), LoRA, $r=64$ & 30.37GB & 167.8M(2.09\%) & \textbf{53.07} & \textbf{81.40} & \textbf{82.32} & \textbf{60.67} & 34.2 & \textbf{79.98} & 69.68 & \textbf{48.52} & \textbf{73.56} & \textbf{64.82}\\
Llama3(8B), SPruFT, $r=128$ & \textbf{24.49GB} & 159.4M(1.98\%) & 52.47 & 81.10 & 81.28 & 60.29 & \textbf{34.6} & 79.76 & \textbf{70.04} & 47.75 & 73.24 & 64.50 \\\cmidrule(lr){2-13}
Llama3(8B), FA-LoRA, $r=64$ & 24.55GB & 113.2M(1.41\%) & \textbf{52.47} & \textbf{81.36} & \textbf{82.23} & 60.17 & \textbf{35.0} & \textbf{79.76} & \textbf{70.04} & \textbf{48.31} & \textbf{73.56} & \textbf{64.77}\\
Llama3(8B), FA-SPruFT, $r=128$ & \textbf{22.41GB} & 92.3M(1.15\%) & 52.22 & 81.19 & 81.35 & \textbf{60.20} & 34.2 & 79.71 & 69.31 & 47.13 & 73.01 & 64.26 \\\bottomrule
\end{tabular}
%\vspace{-0.2cm}
\caption{Fine-tuning Llama on Alpaca dataset for 5 epochs and evaluating on 9 tasks from EleutherAI LM Harness. "mem" represents the memory usage, with further details provided in Appendix~\ref{apdx:measure}. \#param is the number of trainable parameters, where the difference of \#param between the two approaches depends on the architecture of Llama, as some layers have $d_{in} \neq d_{out}$. Note that 10 million trainable parameters only account for less than 0.15GB of memory requirement. FA indicates that we freeze attention layers, but not including MLP layers followed by attention blocks. HS, OBQA, and WG represent HellaSwag, OpenBookQA, and WinoGrande datasets. More details of datasets can be found in Appendix~\ref{apdx:data}. The ablation study for different $r$ and the comparison with other LoRA variants can be found in Appendix~\ref{apdx:ablation}. All reported results are accuracies on the corresponding tasks. \textbf{Bold} indicates the best results of two approaches on the same task.} \label{tab:llm} 
\end{center}
\end{table*}
\fi

\begin{table*}[htbp]
\tiny
\begin{center}
\begin{tabular}{lccccccccccccc}\toprule
Model, ft setting & mem & \#param & ARC-c & ARC-e & BoolQ & HS & OBQA & PIQA & rte & SIQA & WG & Avg
\\\cmidrule(lr){1-13}
Llama2(7B)\\ \cmidrule(lr){1-1} 
LoRA, $r=64$ & 23.46GB & 159.9M(2.37\%) & \textbf{44.97} & 77.02 & 77.43 & 57.75 & 32.0 & \textbf{78.45} & 62.09 & 47.75 & 68.75 & 60.69\\
VeRA, $r=64$ & 22.97GB & 1.374M(0.02\%) & 43.26 & 76.43 & 77.40 & 57.26 & 31.6 & 78.02 & 62.09 & 45.85 & 68.75 & 60.07\\
DoRA, $r=64$ & 44.85GB & 161.3M(2.39\%) & 44.71 & 77.02 & 77.55 & \textbf{57.79} & 32.4 & 78.29 & 61.73 & \textbf{47.90} & 68.98 & 60.71\\
RoSA, $r=32, d=1.2\%$ & 44.69GB & 157.7M(2.34\%) & 43.86 & \textbf{77.48} & \textbf{77.86} & 57.42 & 32.2 & 77.97 & 63.90 &  47.29 & 69.06 & 60.78\\
SPruFT, $r=128$ & \textbf{17.62GB} & 145.8M(2.16\%) & 43.60 & 77.26 & 77.77 & 57.47 & \textbf{32.6} & 78.07 & \textbf{64.98} & 46.67 & \textbf{69.30} & \textbf{60.86} %\\\cmidrule(lr){2-13}
%FA-LoRA, $r=64$ & 17.25GB & 92.8M(1.38\%) & 43.77 & \textbf{77.57} & 77.74 & \textbf{57.45} & 31.0 & 77.86 & 66.06 & \textbf{47.13} & 69.06 & 60.85\\
%FA-DoRA, $r=64$ & 30.61GB & 93.6M(1.39\%) & 43.94 & 77.44 & 77.49 & 57.44 & 31.0 & 77.86 & \textbf{66.43} & 46.98 & 69.14 & 60.86\\
%FA-RoSA, $r=32, d=1.2\%$ & 38.34GB & 98.3M(1.46\%) & \textbf{44.28} & 77.02 & 77.68 & 57.22 & 31.0 & 77.97 & 64.26 & 46.32 & 69.22 & 60.55\\
%FA-SPruFT, $r=128$ & \textbf{15.21GB} & 78.6M(1.17\%) & 43.94 & 77.22 & \textbf{77.83} & 57.11 & \textbf{32.0} & \textbf{78.18} & 65.70 & 46.47 & \textbf{69.38} & \textbf{60.87}
\\\midrule
Llama3(8B)\\ \cmidrule(lr){1-1} 
LoRA, $r=64$ & 30.37GB & 167.8M(2.09\%) & 53.07 & 81.40 & 82.32 & 60.67 & 34.2 & 79.98 & 69.68 & 48.52 & 73.56 & 64.82\\
VeRA, $r=64$ & 29.49GB & 1.391M(0.02\%) & 50.26 & 80.30 & 81.41 & 60.16 & 34.4 & 79.60 & 69.31 & 46.93 & 72.77 & 63.90\\
DoRA, $r=64$ & 51.45GB & 169.1M(2.11\%) & \textbf{53.33} & \textbf{81.57} & \textbf{82.45} & \textbf{60.71} & 34.2 & \textbf{80.09} & 69.31 & \textbf{48.67} & \textbf{73.64} & \textbf{64.88}\\
RoSA, $r=32, d=1.2\%$ & 48.40GB & 167.6M(2.09\%) & 51.28 & 81.27 & 81.80 & 60.18 & 34.4 & 79.87 & 69.31 & 47.95 & 73.16 & 64.36\\
SPruFT, $r=128$ & \textbf{24.49GB} & 159.4M(1.98\%) & 52.47 & 81.10 & 81.28 & 60.29 & \textbf{34.6} & 79.76 & \textbf{70.04} & 47.75 & 73.24 & 64.50 %\\\cmidrule(lr){2-13}
%FA-LoRA, $r=64$ & 24.55GB & 113.2M(1.41\%) & 52.47 & 81.36 & 82.23 & 60.17 & \textbf{35.0} & 79.76 & 70.04 & 48.31 & \textbf{73.56} & 64.77\\
%FA-DoRA, $r=64$ & 40.62GB & 114.3M(1.42\%) & \textbf{52.56} & \textbf{81.69} & \textbf{82.26} & \textbf{60.20} & 34.4 & \textbf{79.82} & \textbf{70.40} & \textbf{48.46} & 73.40 & \textbf{64.80}\\
%FA-RoSA, $r=32, d=1.2\%$ & 42.31GB & 124.3M(1.55\%) & 52.22 & 81.19 & 82.05 & 60.11 & 34.4 & 79.76 & 69.31 & 47.70 & 73.16 & 64.43\\
%FA-SPruFT, $r=128$ & \textbf{22.41GB} & 92.3M(1.15\%) & 52.22 & 81.19 & 81.35 & \textbf{60.20} & 34.2 & 79.71 & 69.31 & 47.13 & 73.01 & 64.26 
\\\bottomrule
\end{tabular}
%\vspace{-0.2cm}
\caption{Fine-tuning Llama on Alpaca dataset for 5 epochs and evaluating on 9 tasks from EleutherAI LM Harness. ``mem" represents the memory usage, with further details provided in Appendix~\ref{apdx:measure}. \#param is the number of trainable parameters, where the difference of \#param between the two approaches depends on the architecture of Llama, as some layers have $d_{in} \neq d_{out}$. %FA indicates that we freeze attention layers, but not including MLP layers followed by attention blocks. 
HS, OBQA, and WG represent HellaSwag, OpenBookQA, and WinoGrande datasets. %More details of datasets can be found in Appendix~\ref{apdx:data}. 
The ablation study for different $r$ can be found in Appendix~\ref{apdx:ranks}. All reported results are accuracies on the corresponding tasks. \textbf{Bold} indicates the best result on the same task. } \label{tab:llm} 
\end{center}
\end{table*}

\section{Experimental Setup}\label{sec:setup}

%(0.5 page)
%Why the chosen framework?
%Some prior approaches

%- parameter settings
%- uniform across layers vs greedy ... 
%- potential transformer-specific details

%Equations about what these methods do.. 

%(0.5 page)
%Which NN architectures are used, why?
%Number of parameters, layers, ...

%(Potential prior work on compression -- )

\subsection{Datasets} \label{subsec:dataset}
We use multiple datasets for different tasks. For image classification, we fine-tune models on the training split and evaluate it on the validation split of Tiny-ImageNet~\citep{tavanaei2020embedded}, CIFAR100~\citep{alex2009learning}, and Caltech101~\citep{li_andreeto_ranzato_perona_2022}. For text generation, we fine-tune LLMs on 256 samples from Stanford-Alpaca~\citep{alpaca} and assess zero-shot performance on nine EleutherAI LM Harness tasks~\citep{gao2021framework}. See Appendix~\ref{apdx:data} for details.

\subsection{Models and Baselines} \label{subsec:models}

We fine-tune full-precision Llama-2-7B and Llama-3-8B (float32) using our SPruFT, LoRA~\citep{hulora}, VeRA~\citep{kopiczko2024vera}, DoRA~\citep{liu2024dora}, and RoSA~\citep{nikdan2024rosa}. RoSA is chosen as the representative SFT method and is the only SFT due to the high memory demands of other SFT approaches, while full fine-tuning is excluded for the same reason. We freeze Llama’s classification layers and fine-tune only the linear layers in attention and MLP blocks.

Next, we evaluate importance metrics by fine-tuning Llamas and image models, including DeiT~\citep{touvron2021training}, ViT~\citep{dosovitskiy2020image}, ResNet101~\citep{he2016deep}, and ResNeXt101~\citep{xie2017aggregated} on CIFAR100, Caltech101, and Tiny-ImageNet. For image tasks, we set the fine-tuning ratio at 5\%, meaning the trainable parameters are a total of 5\% of the backbone plus classification layers.

\subsection{Training Details} \label{subsec:training}
Our fine-tuning framework is built on torch-pruning\footnote{Torch-pruning is not required, all their implementations are based on PyTorch.}~\citep{fang2023depgraph}, PyTorch~\citep{paszke2019pytorch}, PyTorch-Image-Models~\citep{rw2019timm}, and HuggingFace Transformers~\citep{wolf2020transformers}. Most experiments run on a single A100-80GB GPU, while DoRA and RoSA use an H100-96GB GPU. We use the Adam optimizer~\citep{KingBa15} and fine-tune all models for a fixed number of epochs without validation-based model selection.

%Structured pruning often considers parameter dependencies in importance evaluation~\citep{liu2021group, fang2023depgraph, ma2023llmpruner}. This becomes the following process in our work: first, searching for dependencies by tracing the computation graph of gradient; next, evaluating the importance of parameter groups; and finally, fine-tuning the parameters within those important groups collectively. For instance, if $\W^{a}_{\cdot j}$ and $\W^{b}_{i\cdot}$ are dependent, where $\W^{a}_{\cdot j}$ is the $j$-th column in parameter matrix (or the $j$-th input channels/features) of layer $a$ and $\W^{b}_{i\cdot}$ is the $i$-th row in parameter matrix (or the $i$-th output channels/features) of layer $b$, then $\W^{a}_{\cdot j}$ and $\W^{b}_{i\cdot}$ will be fine-tuned simultaneously while the corresponding $\M^{a}_{dep}$ for $\W^{a}_{\cdot j}$ becomes column selection matrix and $\W^a_s$ becomes $\W^a_{f,dep}\M^a_{dep}$. Consequently, fine-tuning $2.5\%$ output channels for layer $b$ will result in fine-tuning additional $2.5\%$ input channels in each dependent layer. Therefore, for the $5\%$ of desired fine-tuning ratio, the fine-tuning ratio with considering dependencies is set to $2.5\%$\footnote{In some complex models, considering dependencies results in slightly more than twice the number of trainable parameters. However, in most cases, the factor is 2.} for the approach that includes dependencies. More details for dependencies of NN can be found in Appendix~\ref{apdx:dep}. 

\textbf{Image models}: The learning rate is set to $10^{-4}$ with cosine annealing decay~\citep{loshchilov2017sgdr}, where the minimum learning rate is $10^{-9}$. All image models used in this study are pre-trained on ImageNet. 

\textbf{Llama}: For LoRA and DoRA, we set $\alpha = 16$, a dropout rate of $0.1$, and a learning rate of $10^{-4}$  with linear decay (
$0.01$ decay rate). For SPruFT, we control trainable parameters using rank instead of fine-tuning ratio for direct comparison. The learning rate is $2 \cdot 10^{-5}$ with the same decay settings. Linear decay is applied after a warmup over the first $3$\% of training steps. The maximum sequence length is $2048$, with truncation for longer inputs and padding for shorter ones.


\section{Evaluation}


\begin{table}[t]
    \centering
    % \vspace{-0.1in}
    \scalebox{0.78}{
    % \begin{small}
        \begin{tabular}{lccc}
            \toprule
            \multirow{2}*{\textbf{MoE Models}} & \textbf{Parameters} & \textbf{Experts Per Layer} & \textbf{Num. of} \\
            & \textbf{(active / total)} & \textbf{(active / total)} & \textbf{Layers} \\
            \otoprule 
            \mixtral~\cite{jiang2024mixtral} & 12.9B / 46.7B & 2 / 8 & 32 \\
            % \hline
            \qwen~\cite{yang2024qwen2} & 2.7B / 14.3B & 4 / 60 & 24 \\
            \phimoe~\cite{abdin2024phi} & 6.6B / 42B & 2 / 16 & 32 \\
            \bottomrule 
        \end{tabular}
    % \end{small}
    }
    \caption{Characteristics of three \MoE models in evaluation.}
    \vspace{-0.2in}
    \label{table:eval-moe-models}
\end{table}








\subsection{Experimental Setup}
\label{subsec:eval-setup}


% \begin{figure*}[t]
%     \centering
%     \begin{subfigure}[t]{0.48\textwidth}
%         \centering
%         \includegraphics[width=.9\linewidth]{figs/eval-overall-lmsys.pdf}
%         \caption{Serving three \MoE models with LMSYS-Chat-1M dataset.}
%     \end{subfigure}
%     \begin{subfigure}[t]{0.48\textwidth}
%         \centering
%         \includegraphics[width=.9\linewidth]{figs/eval-overall-sharegpt.pdf}
%         \caption{Serving three \MoE models with ShareGPT dataset.}
%     \end{subfigure}
%     \caption{Overall performance of prefill and decode stages for \sys and other four baselines.}
%     \label{fig:eval-overall.pdf}
% \end{figure*}


\noindent \textbf{Testbed.}
We conduct all experiments on a six-GPU testbed, where each GPU is an NVIDIA GeForce RTX 3090 with 24 GB GPU memory. 
%
All GPUs are inter-connected using pairwise NVLinks and connected to the CPU memory using PCIe 4.0 with 32GB/s bandwidth. 
%
Additionally, the testbed has a total of 32 AMD Ryzen Threadripper PRO 3955WX CPU cores and 480 GB CPU memory.


\noindent \textbf{Models.}
We employ three popular \MoE-based \LLMs in our evaluation: \mixtral~\cite{jiang2024mixtral}, \qwen~\cite{yang2024qwen2}, and \phimoe~\cite{abdin2024phi}.
Table~\ref{table:eval-moe-models} describes the parameters, number of \MoE layers, and number of experts per layer for the three models.
Following the evaluation of existing works~\cite{song2024promoe}, we profile the models to set the optimal prefetch distance $d$ to three before evaluation.
% We set $d$ of \mixtral, \qwen, and \phimoe to \todo{$xxx$}, \todo{$xxx$}, and \todo{$xxx$}, respectively.


\noindent \textbf{Datasets and traces.}
We employ two real-world prompt datasets commonly used for \LLM evaluation: LMSYS-Chat-1M~\cite{zheng2023lmsys} and ShareGPT~\cite{sharegpt}.
%
For most experiments, we split the sampled datasets in a standard 7:3 ratio, where 70\% of the prompts' context data (\ie, semantic embeddings and expert maps) are stored in \sys's Expert Map Store, and 30\% of the prompts are used for testing. 
%
For online serving experiments, we empty the Expert Map Store and use real-world \LLM inference traces~\cite{patel2024splitwise,stojkovic2025dynamollm} released by Microsoft Azure to set input and generation lengths and drive invocations.

\noindent \textbf{Baselines.}
We compare \sys against four \SOTA \MoE serving baselines:
1) \textbf{MoE-Infinity}~\cite{xue2024moe} uses coarse-grained request-level expert activation patterns and synchronous expert prediction and prefetching for \MoE serving. 
We prepare the expert activation matrix collection for MoE-Infinity before evaluation for a fair comparison.
%
% However, the open-sourced MoE-Infinity codebase~\cite{moe-infinity-code} lacks some features described in its original paper, we had to modify
%y 
2) \textbf{ProMoE}~\cite{song2024promoe} employs a stride-based speculative expert prefetching approach for \MoE serving. Since the codebase of ProMoE is not open-sourced and requires training predictors for each \MoE model, we reproduced a prototype of ProMoE on top of MoE-Infinity in our best effort.
%
3) \textbf{Mixtral-Offloading}~\cite{eliseev2023fast} combines a layer-wise speculative expert prefetching and a \LRU-based expert cache. 
%
4) \textbf{DeepSpeend-Inference} employs an expert-agnostic layer-wise parameter offloading approach, which uses pure on-demand loading and does not support prefetching. 
%
We implement the offloading logic of DeepSpeed-Inference in the MoE-Infinity codebase and add an expert cache for a fair comparison.
We enable all baselines to serve \MoE models from HuggingFace Transformer~\cite{wolf2020huggingface}. 


\noindent \textbf{Metrics.}
Following the standard evaluation methodology of existing works~\cite{song2024promoe,xue2024moe,zhong2024distserve,agrawal2024taming} on \LLM serving, we report the performance of the prefill and decode stages separately. 
We measure Time-to-First-Token (TTFT) for the prefill stage and Time-Per-Output-Token (TPOT) for the decode stage.
Additionally, we also report other system metrics, such as expert hit rate and overheads, for detailed evaluation.


% \noindent \textbf{\sys's setting.}
% The hyperparameters of \sys containing the prefetch distance $d$ for each \MoE model, Expert Map Store capacity $C$, and Expert Cache memory limit $M$.
% For most experiments, we profile the \MoE models and set the prefetch distance $d$ to their optimal values. The Expert Map Store capacity $C$ is set to \todo{$xxx$} expert maps. We configure the Expert Cache memory limit to \todo{$xxx$} GB.
% The hyperparameter sensitivity is analyzed in \S\ref{subsec:eval-sensitivity}.


\begin{figure}[t]
  \centering
  \includegraphics[width=.95\linewidth]{figs/eval-overall-arxiv.pdf}
  \vspace{-0.15in}
  \caption{Overall performance of prefill and decode stages for \sys and other four baselines.}
  \vspace{-0.2in}
  \label{fig:eval-overall}
\end{figure}


\subsection{Overall Performance}
\label{subsec:eval-overall}



We first evaluate the performance of prefill and decode stages when running \sys and other baselines with the three \MoE models, where we measure Time-To-First-Token (TTFT) and Time-Per-Output-Token (TPOT) for each stage.
Note that the inference latency with expert offloading tends to be higher than no offloading due to two reasons: 
1) During inference, an excessive amount of parameters in \MoE models are loaded and offloaded, which prolongs the inference latency.
2) All baselines and \sys are implemented on top of the MoE-Infinity codebase~\cite{moe-infinity-code}, whose inference latency is inherently impacted by MoE-Infinity's implementation.
Nevertheless, comparing \sys and baselines is fair with the same experimental setup.

Figure~\ref{fig:eval-overall} shows the \TTFT, \TPOT, and expert hit rate of \sys and other four baselines when serving three \MoE models with LMSYS-Chat-1M and ShareGPT datasets, respectively.
DeepSpeed has both the worst \TTFT and \TPOT due to expert-agnostic offloading and lacking expert prefetching.
While Mixtral-Offloading, ProMoE, and MoE-Infinity perform better than DeepSpeed-Inference, they are underperformed by \sys because of coarse-grained offloading designs.
Compared to DeepSpeed-Inference, Mixtral-Offloading, ProMoE, and MoE-Infinity, our \sys reduces the average \TTFT by 44\%, 35\%, 33\%, 30\%, and reduces the average \TPOT by 70\%, 61\%, 55\%, 48\%, across three \MoE models.
%
% Figure~\ref{fig:eval-overall} also reports the expert hit rate of \sys and each baseline. 
For expert hit rate, Mixtral-Offloading achieves a higher hit rate than the other three baselines because of its synchronous speculative prefetching with a prefetch distance of 1. However, due to synchronous prefetching, its \TTFT and \TPOT are worse than others except DeepSpeed-Inference.
\sys improves the average expert hit rate by 147\%, 11\%, 34\%, and 63\% over DeepSpeed-Inference, Mixtral-Offloading, ProMoE, and MoE-Infinity, respectively.

% \begin{figure}[t]
%   \centering
%   \includegraphics[width=.9\linewidth]{figs/eval-overall-sharegpt.pdf}
%   % \vspace{-0.15in}
%   \caption{}
%   % \vspace{-0.25in}
%   \label{fig:eval-overall-sharegpt.pdf}
% \end{figure}




\subsection{Online Serving Performance}
\label{subsec:eval-online}


Except for the offline evaluation (\ie, Expert Map Store in full capacity before serving), we also evaluate \sys against other baselines in online serving settings.
We empty the Expert Map Store of \sys and the expert activation matrix collection of MoE-Infinity for the online serving experiment.
%
The request traces are derived from Azure \LLM inference traces~\cite{patel2024splitwise,stojkovic2025dynamollm}, with 64 requests randomly sampled to drive LMSYS-Chat-1M prompts for each \MoE model serving. 
To ensure consistency, \sys and all baselines input and generate the exact number of tokens specified in the traces.
%
Figure~\ref{fig:eval-online-serve} illustrates the CDF of end-to-end request latency across three \MoE models. The results demonstrate that \sys significantly reduces overall request latency compared to other baselines in online serving scenarios.


\begin{figure}[t]
  \centering
  \includegraphics[width=.95\linewidth]{figs/eval-online-serve-arxiv.pdf}
  \vspace{-0.15in}
  \caption{CDF of request latency for \MoE online serving.}
  \vspace{-0.2in}
  \label{fig:eval-online-serve}
\end{figure}



\subsection{Impact of Expert Cache Limits}



We measure the \TPOT of \sys and other baselines by limiting the expert cache memory budget to investigate their performance in the latency-memory trade-off (\S\ref{subsec:bg-latency-memory-tradeoff}).
We mainly focus on \TPOT to show the end-to-end performance impacted by varying cache limits.
Figure~\ref{fig:eval-cache-limit.pdf} shows the \TPOT of \sys and other four baselines when serving three \MoE models under different expert cache limits.
We gradually increase the GPU memory allocated for caching experts from 6 GB to 96 GB while employing the same experimental setting in \S\ref{subsec:eval-overall}.
Similarly, DeepSpeed-Inference has the worst \TPOT due to being expert-agnostic.
\sys consistently outperforms Mixtral-Offloading, ProMoE, and MoE-Infinity under varying expert cache limits.
Especially for limited GPU memory sizes (\eg, 6GB), \sys reduces the \TPOT by 32\%, 24\%, 18\%, and 18\%, compared to DeepSpeed-Inference, Mixtral-Offloading, ProMoE, and MoE-Infinity, across three \MoE models, respectively.
With fine-grained expert offloading, \sys significantly reduces the expert on-demand loading latency while maintaining a lower GPU memory footprint, therefore achieving a better spot in the latency-memory trade-off of \MoE serving.

% \subsection{Impact of Inference Batch Size}

\subsection{Ablation Study}
\label{subsec:eval-ablation}


% \begin{figure}[t]
%   \centering
%   \includegraphics[width=.95\linewidth]{figs/eval-expert-tracking.pdf}
%   % \vspace{-0.15in}
%   \caption{Expert hit rate of different expert pattern tracking approaches.}
%   % \vspace{-0.25in}
%   \label{fig:eval-expert-tracking}
% \end{figure}



We present the ablation study of \sys's design.


\textbf{Effectiveness of expert map search.}
One of \sys's key designs is the expert map, which tracks expert selection preferences in fine granularity.
We evaluate the effectiveness of the expert map against five expert pattern-tracking approaches as follows.
%
1) \textbf{Speculate}: speculative prediction used by Mixtral-Offloading~\cite{eliseev2023fast} and ProMoE~\cite{song2024promoe}, 
%
2) \textbf{Hit count}: request-level expert hit count used by MoE-Infinity~\cite{xue2024moe}, 
%
3) \textbf{Map (T)}: expert map with only trajectory similarity search,
4) \textbf{Map (T+S)}: expert map with both trajectory and semantic similarity search,
%
and
5) \textbf{Map (T+S+$\delta$)}: expert map with full features enabled, including trajectory and semantic similarity search (\S\ref{subsec:design-similarity-match}) and dynamic expert selection (\S\ref{subsec:design-expert-prefetch}).
%
We implement the above methods in \sys's Expert Map Matcher for a fair comparison.
Figure~\ref{fig:eval-expert-tracking} shows the expert hit rate of the above expert pattern tracking methods.
%
Speculative prediction is effective due to the widespread presence of residual connections in Transformer blocks. However, its effectiveness decreases drastically as prefetch distance increases~\cite{song2024promoe}.
%
The request-level expert activation count has the worst performance due to coarse granularity.
%
As features are incrementally restored to \sys's expert map, the expert hit rate gradually increases, demonstrating its effectiveness.

% \textbf{Effectiveness of asynchronous map matching.}




\begin{figure}[t]
  \centering
  \includegraphics[width=.9\linewidth]{figs/eval-cache-limit-arxiv.pdf}
  \vspace{-0.15in}
  \caption{Performance of \sys and other four baselines under varying expert cache limits.}
  \vspace{-0.1in}
  \label{fig:eval-cache-limit.pdf}
\end{figure}

\begin{figure}[!t]
    \centering
    \begin{subfigure}[t]{0.585\linewidth}
        \centering
        \includegraphics[width=\linewidth]{figs/eval-expert-tracking.pdf}
        \caption{Expert pattern tracking approaches.}
        \label{fig:eval-expert-tracking}
    \end{subfigure}
    % \hspace{0.02in}
    \begin{subfigure}[t]{0.385\linewidth}
        \centering
        \includegraphics[width=\linewidth]{figs/eval-prefetch-and-cache-arxiv.pdf}
        \caption{Prefetch and caching.}
        \label{fig:eval-prefetch-and-cache}
    \end{subfigure}
    \vspace{-0.1in}
    \caption{Ablation study of \sys.}
    \label{fig:eval-ablation}
    \vspace{-0.2in}
\end{figure}

\textbf{Effectiveness of expert prefetching and caching.}
We evaluate \sys's expert prefetching and caching against two caching algorithms:
1) \textbf{\LRU} used by Mixtral-Offloading~\cite{eliseev2023fast}
and 
2) \textbf{\LFU} used by MoE-Infinity~\cite{xue2024moe}.
%
Figure~\ref{fig:eval-prefetch-and-cache} depicts the expert hit rate of \sys and two baselines.
The results show that \LRU performs poorly in expert offloading scenarios. Though \LFU achieves a higher hit rate than \LRU, \sys surpasses both, achieving the highest expert hit rate.

\subsection{Sensitivity Analysis}
\label{subsec:eval-sensitivity}


\begin{figure}[t]
  \centering
  \includegraphics[width=.9\linewidth]{figs/eval-prefetch-distance.pdf}
  \vspace{-0.15in}
  \caption{Performance of \sys serving \MoE models with different prefetch distances.}
  \vspace{-0.1in}
  \label{fig:eval-prefetch-distance}
\end{figure}

% \begin{figure}[t]
%   \centering
%   \includegraphics[width=.9\linewidth]{figs/eval-store-capacity.pdf}
%   % \vspace{-0.15in}
%   \caption{Semantic and trajectory similarity lower bounds in \sys's serving with different Expert Map Store capacity.}
%   % \vspace{-0.25in}
%   \label{fig:eval-store-capacity}
% \end{figure}

\begin{figure}[t]
    \centering
    \begin{subfigure}[t]{0.55\linewidth}
        \centering
        \includegraphics[width=\linewidth]{figs/eval-store-capacity.pdf}
        \caption{Expert Map Store capacity.}
        \label{fig:eval-store-capacity}
    \end{subfigure}
    % \hspace{0.02in}
    \begin{subfigure}[t]{0.435\linewidth}
        \centering
        \includegraphics[width=\linewidth]{figs/eval-batch-size-arxiv.pdf}
        \caption{Inference batch size.}
        \label{fig:eval-batch-size}
    \end{subfigure}
    \vspace{-0.1in}
    \caption{Sensitivity analysis of \sys.}
    \vspace{-0.2in}
    \label{fig:eval-sensitivity}
\end{figure}


We analyze the sensitivity of three hyperparameters: prefetch distance of \MoE models, the capacity of Expert Map Store, and inference batch size.


\textbf{Prefetch distance of \MoE models.}
Figure~\ref{fig:eval-prefetch-distance} shows the \TTFT and \TPOT of \sys when serving three \MoE models with different prefetch distances.
%
We have demonstrated that the expert hit rate decreases when gradually increasing the prefetch distance (Figure~\ref{fig:bg-hit-distance}).
%
When the prefetch distance is small ($<3$), \sys cannot perfectly hide its system delay from the inference process, such as the map matching and expert prefetching, leading to the increase of inference latency.
%
With larger prefetch distances ($>3$), \sys has worse expert hit rates that also degrade the performance. 
Therefore, we set the prefetch distance $d$ to 3 for evaluating \sys.


\textbf{Capacity of Expert Map Store.}
We measure the mean semantic and trajectory similarity scores searched in \sys's expert map matching for \MoE model serving.
%
Figure~\ref{fig:eval-store-capacity} presents the mean semantic and trajectory similarity scores of \sys with different Expert Map Store capacity sizes.
%
Both semantic and trajectory similarity scores improve as the store capacity increases.
%
While the similarity scores exhibit a significant increase with capacities below 1K, further capacity expansion yields diminishing similarity gains. 
To minimize \sys's memory overhead, we set \sys's Expert Map Store capacity to 1K in evaluation.


\textbf{Inference batch size.}
We investigate the impact of inference batch size on \sys and three baselines using \mixtral with LMSYS-Chat-1M.
%
Figure~\ref{fig:eval-batch-size} presents the performance of \sys, Mixtral-Offloading, ProMoE, and MoE-Infinity as the batch size increases from one to four. \sys achieves the lowest \TTFT and \TPOT in most cases.


% \textbf{Inference batch size.}


% \subsection{Scalability}
% \label{subsec:eval-scalability}
% From one to six GPUs


\begin{figure}[t]
  \centering
  \includegraphics[width=.92\linewidth]{figs/eval-overhead-latency.pdf}
  \vspace{-0.15in}
  \caption{Latency breakdown of \sys's one inference iteration with three \MoE models.}
  \vspace{-0.1in}
  \label{fig:eval-overhead-latency.pdf}
\end{figure}





\subsection{System Overheads}
\label{subsec:eval-overhead}


\noindent \textbf{Latency overheads of \sys's operations.}
Figure~\ref{fig:eval-overhead-latency.pdf} shows the latency breakdown of one inference iteration in \sys when serving the three \MoE models.
We report any operations of \sys in \S\ref{subsec:eval-overall} that may incur a significant latency delay, including context collection, map matching, expert on-demand loading, expert prefetching, and map update after the iteration completes.
\qwen has lower end-to-end iteration latency than \mixtral and \phimoe because of significantly fewer parameters.
Note that expert prefetching, map matching, and map update tasks are executed asynchronously, aside from the inference process. Hence, they do not contribute to the end-to-end iteration latency.
Excluding three asynchronous tasks, the total delay incurred by other operations is consistently less than 30ms (5\% of the iteration) across three \MoE models, which is negligible compared to the inference latency.


\noindent \textbf{Memory overheads of \sys's Expert Map Store.}
Figure~\ref{fig:eval-overhead-memory.pdf} shows the CPU memory footprint of \sys's Expert Map Store when varying the store capacity from 1K to 32K maps.
The memory needed to store expert maps for \qwen is more than \mixtral and \phimoe because it has more experts per layer over the other two models, which increases the map shape.
Even for the largest capacity (32K), the Expert Map Store requires less than 200MB of memory to store the maps, which is trivial since modern GPU servers usually have abundant CPU memory (\eg, p4d.24xlarge on AWS EC2~\cite{aws-ec2} has over 1100 GB of CPU memory).
In the evaluation, \sys's map store capacity with 1K maps is sufficient for maintaining performance (\S\ref{subsec:eval-sensitivity}), resulting in minimal memory overhead.



\begin{figure}[t]
  \centering
  \includegraphics[width=.85\linewidth]{figs/eval-overhead-memory.pdf}
  % \vspace{-0.1in}
  \caption{CPU memory footprint of \sys's Expert Map Store with different capacity.}
  \vspace{-0.1in}
  \label{fig:eval-overhead-memory.pdf}
\end{figure}

\section{Implications}

Based on the findings we conclude in Sec.~\ref{sec:eval}, we provide some implications for researchers who build LLMs and practitioners who use LLMs in software development.

\textbf{LLM Researchers.} We identify that there is a large gap between correct code generation and efficient code generation. This indicates that the current LLM-generated code is correct but sub-optimal, and generating efficient code remains a great challenge, especially for file-level code generation. This challenge cannot be effectively mitigated by just increasing the model size of current LLMs. We recommend that LLM researchers consider the code structure and semantics when improving the time efficiency of LLM-generated code. Besides, LLM researchers should also focus more on file-level code generation since current LLMs perform much worse on it than function-level code generation.



\textbf{Software Practitioners.}  As LLMs are gradually adopted in software development in product environments, software practitioners face the problem of choosing LLMs. In function-level and file-level code generation, generally, code solutions generated by DeepSeek V2 Coder and Llama3.1-405b obtain the best time efficiency, respectively. However, we also find that some LLMs with middle sizes, such as Llama3-70b and CodeLlama-34b, achieve competitive performance. We recommend software practitioners adopt middle-sized LLMs to obtain similar performance on efficient code generation with much lower computational costs.
This work identifies signal collapse as a critical bottleneck in one-shot neural network pruning. Performance loss in pruned networks is due to \textbf{signal collapse} in addition to the removal of critical parameters. We propose \textbf{REFLOW} (\textbf{Re}storing \textbf{F}low of \textbf{Low}-variance signals), a simple yet effective method that mitigates signal collapse without computationally expensive weight updates. By focusing on signal preservation, REFLOW highlights the importance of mitigating signal collapse in sparse networks and enables magnitude pruning to match or surpass state-of-the-art one-shot pruning methods such as CHITA, CBS, and WF.

REFLOW consistently achieves state-of-the-art accuracy across diverse architectures, restoring ResNeXt-101 from under 4.1\% to 78.9\% top-1 accuracy at 80\% sparsity on ImageNet. Its lightweight design makes it a practical solution for both research and deployment, delivering high-quality sparse models without the overhead of traditional approaches. These findings challenge the traditional emphasis on weight selection strategies and underscore the critical role of signal propagation for achieving high-quality sparse networks in the context of one-shot pruning.



The advancement of artificial intelligence in the legal domain has led to the development of various tools that assist in legal research, document retrieval, and automated legal reasoning. Several studies have explored the use of Natural Language Processing (NLP)\cite{khurana2023natural}, machine learning models, and vector-based search mechanisms to enhance the efficiency of legal chatbots. The primary focus of this literature review is on retrieval-augmented generation (RAG) models, FAISS-based document retrieval, deep learning for legal applications, and the use of large language models (LLMs) in legal AI.  

Recent research on Retrieval-Augmented Generation (RAG)\cite{gao2023retrieval} for legal AI has demonstrated its potential in enhancing legal text retrieval and summarization. S. S. Manathunga, Y. and A. Illangasekara\cite{manathunga2023retrieval} proposed a RAG-based model that improves legal text summarization by dynamically fetching relevant documents before generating responses. Similarly, Lee and Ryu \cite{ryu-etal-2023-retrieval} explored the application of RAG in case law retrieval, demonstrating its superiority over traditional keyword-based search engines. The introduction of RAG has significantly improved response accuracy by grounding AI-generated text in authoritative legal documents, reducing hallucinations in AI-driven legal assistance.  

% \begin{figure}[h]
%     \centering
%     \includegraphics[width=8cm]{FAISS.png}
%     \caption{Faiss: Efficient Similarity Search and Clustering of Dense Vectors}
%     \label{Overall Result of comparing FAISS and Chroma with different number of top documents}
% \end{figure}

The efficiency of FAISS (Facebook AI Similarity Search) in legal document retrieval has also been widely studied. Zhao et al. \cite{devlin-etal-2019-bert} implemented FAISS to enhance large-scale legal question answering systems, achieving significant improvements in retrieval speed and relevance. N. Goyal and D. Chen \cite{inbook} demonstrated that FAISS-based vector search mechanisms outperform conventional database searches in legal information retrieval, reducing query response time while maintaining high accuracy. The integration of FAISS with transformer-based models, as seen in the work of Hsieh and Wu, further enhances semantic retrieval, ensuring that chatbot responses align with actual legal texts.  

Transformer-based models such as BERT and GPT-based architecture have also contributed to the evolution of AI-driven legal research. Devlin et al. introduced BERT (Bidirectional Encoder Representations from Transformers), which significantly improved the understanding of legal language. RoBERTa, an optimized version of BERT, was later developed by Liu et al. \cite{liu2019roberta} to enhance contextual understanding and document similarity matching in legal queries. These models have been integrated into legal chatbots for contract analysis and legal decision-making, as demonstrated in the studies of Li et al. and Jin and Liu, where fine-tuned transformers improved legal text comprehension and summarization.  
The role of deep learning in legal AI has also been investigated extensively. Radford et al. introduced GPT-3, which paved the way for legal AI assistants capable of generating human-like responses. However, researchers such as Firth and Lee emphasized the limitations of LLMs in legal reasoning, arguing that these models require external verification mechanisms to prevent misinformation. The use of contrastive learning and fine-tuning for legal text retrieval has been explored by Arabi and Akbari \cite{article}, who demonstrated that embedding-based retrieval significantly improves chatbot response accuracy.  

Another significant area of research involves evaluating AI-generated legal responses using automated metrics. Zhang and Wu introduced BLEU\cite{10.3115/1073083.1073135} and ROUGE\cite{lin-2004-rouge} scores as a means to evaluate AI-generated legal text summaries, ensuring their quality and relevance. Similarly, Zhao et al. \cite{yuan2024rag} examined the effectiveness of RAG-based models in handling complex legal queries, highlighting the importance of legal consistency scores (LCS) in evaluating AI-driven responses.  

The practical applications of legal AI chatbots have been studied extensively in the context of access to justice and AI ethics. Wang and Cheng et al. \cite{xue2024bias} highlighted the potential of AI-driven legal assistants in bridging the justice gap, particularly in countries where legal resources are not easily accessible. Chan conducted a systematic review of retrieval-based legal chatbots, noting that while these systems improve accessibility, they also raise ethical concerns regarding legal misinformation and bias. Research by Min \cite{Min2023ARTIFICIALIA} explored methods for bias detection and mitigation in legal AI, ensuring fairness in AI-generated legal advice.  

Comparative studies between rule-based legal bots, keyword-driven legal search engines, and AI-powered legal chatbots further illustrate the superiority of retrieval-augmented approaches. In a study conducted by Zeng \cite{zeng2024scalable}, FAISS-based retrieval mechanisms significantly outperformed traditional Boolean keyword searches, reducing irrelevant document retrieval by 40\%. Singh \cite{10760929} further demonstrated that AI-powered legal research tools using NLP provide faster and more contextually accurate responses compared to standard legal databases.  

Despite these advancements, challenges remain in AI-driven legal research. Existing chatbots still struggle with multi-jurisdictional legal queries, as noted by Weichbroth \cite{Weichbroth2025AIAT}, who emphasized the need for jurisdiction-aware legal AI models. Additionally, legal AI models often lack the ability to process long-context legal arguments effectively, a limitation discussed by Gupta, who proposed memory-based retrieval techniques to improve long-form legal text processing.  

Research continues to refine AI-driven legal assistance, particularly in retrieval-augmented generation, FAISS-based search, transformer models, and deep learning techniques for legal research. However, further improvements are needed in bias mitigation, jurisdiction-specific adaptations, and long-context legal understanding. Future developments in multilingual legal AI, enhanced retrieval mechanisms, and AI-powered contract analysis will be crucial in making legal AI tools more accessible, reliable, and widely applicable in legal practice.
\section*{Conclusion}
This paper aims to enhance our understanding of the computational complexity of computing various Shapley value variants. We found that for various ML models --- including decision trees, regression tree ensembles, weighted automata, and linear regression --- both local and global interventional and baseline SHAP can be computed in polynomial time under HMM modeled distributions. This extends popular algorithms, such as TreeSHAP, beyond their empirical distributional scope. We also establish strict complexity gaps between the various SHAP variants (baseline, interventional, and conditional) and prove the intractability of computing SHAP for tree ensembles and neural networks in simplified scenarios. Overall, we present SHAP as a versatile framework whose complexity depends on four key factors: \begin{inparaenum}[(i)] \item model type, \item SHAP variant, \item distribution modeling approach, \item and local vs. global explanations\end{inparaenum}. We believe this perspective provides deeper insight into the computational complexity of SHAP, paving the way for future work.




%We believe that our framework provides a more intricate understanding of SHAP computation complexity across different models, distributions, and variants, paving the way for further research.

Our work opens promising directions for future research. First, expanding our computational analysis to other SHAP-related metrics, such as asymmetric SHAP~\citep{frye20} and SAGE~\citep{covert2020understanding}, would be valuable. Additionally, we aim to explore more expressive distribution classes and relaxed assumptions beyond those in Section \ref{sec:tractable} while maintaining tractable SHAP computation. Finally, when exact computation is intractable (Section \ref{sec:intractable}), investigating the approximability of SHAP metrics through approximation and parameterized complexity theory~\citep{downey2012parameterized} is an important direction.

%Our work opens several promising avenues for future research on the computational properties of explainable AI methods, with a particular focus on SHAP. First, it would be interesting to broaden the computational analysis conducted in this work to include other popular SHAP-related metrics in the literature, such as asymmetric SHAP \cite{frye20} and SAGE \cite{covert2020understanding}. Also, in the future, we aim to explore more expressive distribution classes and relaxed distributional assumptions—extending beyond those examined in Section \ref{sec:tractable} —that still yield tractable SHAP computation. Finally, when exact computation proves intractable (Section \ref{sec:intractable}), it is worthwhile to theoretically investigate the question of the approximability of computing the SHAP metrics across various configurations, through the lens of approximation and parametrized complexity theory \cite{arora2009computational}.

%This paper aims to deepen our understanding of the computational complexity involved in obtaining different Shapley value variants. We found that for a variety of ML models, including decision trees, tree ensembles for regression, weighted automata, and linear regression models — computing both local and global interventional and baseline SHAP can be done in polynomial time when distributions are modeled by HMMs. This extends the distributional scope of popular algorithms like TreeSHAP, which is limited to empirical distributions. Additionally, we demonstrate a strict complexity gap between SHAP variants, showing that interventional and baseline SHAP can be strictly easier to compute than conditional SHAP. Despite these positive results, we uncovered intractability for various SHAP variants in neural networks and tree ensembles. Finally, we provided generalized complexity relations across SHAP variants. We believe that our framework offers a deeper understanding of the complexity involved in computing SHAP across various variants, models, distributions, as well as in both local and global computations, laying the groundwork for future research.

\section{Data Availability}
The code and data of \tool and \bench are available at https://github.com/JohnnyPeng18/Coffe.

\section*{Acknowledgment}
The authors would like to thank the anonymous reviewers who have provided insightful and constructive comments on this paper.
This work is supported by the National Nature Science Foundation of China (No. 62302437).


\newpage
\bibliographystyle{ACM-Reference-Format}
\bibliography{ref}

\end{document}
\endinput
%%
%% End of file `sample-acmsmall-conf.tex'.
