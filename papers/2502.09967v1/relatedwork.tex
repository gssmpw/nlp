\section{Related Work}
\label{sec:related}
\subsection{Fully Supervised Group Activity Recognition}
Existing GAR algorithms primarily rely on extracting visual information about individuals in the scene to infer group activities. 
Early works utilized hand-crafted features to recognize various activities through probabilistic graphical models~\cite{lan2012social,Amer2013,amer2014hirf,Li2016Multiview} or AND-OR models~\cite{amer2012cost,choi2013understanding,shu2015joint}.

With the rapid advancement of deep learning technology, GAR algorithms based on Convolutional Neural Networks have emerged as the primary focus of research.
Some approaches \cite{deng2016structure,ibrahim2018hierarchical,ibrahim2016hierarchical,li2017sbgar,qi2018stagnet,shu2017cern,wang2017recurrent,yan2018participation,ibrahim2016hierarchical,shu2017cern,CCG,li2019nonlocal} achieved satisfactory results in exploring the individual spatial-temporal relations within the scene based on Recurrent Neural Networks or Long Short-Term Memory structures. 
Recent developments in graph neural networks and transformers have improved the capability to model relations between individuals. 
Wu et al. ~\cite{ARG} devised an Actor Relation Graph (ARG) that constructed actor relation graphs to capture both appearance and position relations among actors.
Gavrilyuk et al. ~\cite{gavrilyuk2020actor} proposed an Actor-Transformer using RGB, optical flow, and pose features as input to model actors. 
Yuan et al. ~\cite{yuan2021learning} enhanced individual representations by incorporating global contextual information and aggregated the relation between individuals through a Spatial-Temporal Bi-linear Pooling module. 
Liu et al. ~\cite{liu2021multimodal} utilized individual action label embeddings to create a semantic graph that refines visual representations. 
Tang et al. ~\cite{tang2019learning} proposed to align individual visual representations with semantic representations derived from action labels through knowledge distillation. 

\begin{figure*}
    \centering
    \includegraphics[width=0.97\linewidth]{figs_pdf/fig_overview.pdf}
    \caption{Overview of the proposed framework. 
    } 
    \label{fig_overview}
\end{figure*} 

\subsection{Weakly Supervised Group Activity Recognition}
Some algorithms aimed to overcome the limitations of individual annotations and explore group activity recognition in a weakly supervised setting.
Bagautdinov et al. ~\cite{bagautdinov2017social} simultaneously performed individual detection and feature extraction using a fully convolutional network, then fed the results into an RNN to recognize group activities in conjunction with individual actions.
Zhang et al. ~\cite{zhang2019fast} made the individual detection and weakly supervised group activity recognition collaborate in an end-to-end framework by sharing convolutional layers between them.
Yan et al. ~\cite{yan2020sam} addresses the issue of missing bounding boxes by generating actor boxes from detectors trained on external datasets and learning to prune irrelevant suggestions, thereby eliminating the need for actor-level labels during both training and inference. 

Apart from detector based algorithms, some methods utilize attention mechanisms to extract regions relevant to group activities.
Wu et al. \cite{wu2022active} utilized attention mechanisms to obtain masks that identify the spatial locations of scene activities and eliminate background information, using these masks as visual markers to construct spatial-temporal relations at different scales.
Kim et al. ~\cite{kim2022DF} proposed the Detector-Free method, which encodes the context of group activity as a set of visual embeddings, thereby bypassing the explicit target detection. 
Chappa et al. ~\cite{chappa2023spartan} employed self-distillation to learn frame-level and patch-level objectives in the latent space, aligning global spatio-temporal features from the entire sequence with local spatio-temporal features from the sampled sequence.
Wu et al. ~\cite{wu2024learning} embedded the specific label semantics to extract corresponding fine-grained information based on the hierarchy inherent in group-level labels, approaching GAR as a multi-label classification task.

However, these approaches lack an explicit connection between the visual information of individual actions and their semantic concepts, which has been demonstrated by fully supervised methods to be beneficial for recognizing group activities. 
To address these issues, we introduce visual conceptual knowledge that provides general visual representations of individual actions, and capture key areas based on actions through image correlation theorem.