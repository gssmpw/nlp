\documentclass[pdflatex,sn-mathphys-num, twoside]{sn-jnl}% 

\usepackage{graphicx}%
\usepackage{multirow}%
\usepackage{amsmath,amssymb,amsfonts}%
\usepackage{amsthm}%
\usepackage{mathrsfs}%
\usepackage[title]{appendix}%
\usepackage{xcolor}%
\usepackage{textcomp}%
\usepackage{manyfoot}%
\usepackage{booktabs}%
\usepackage{algorithm}%
\usepackage{algorithmicx}%
\usepackage{algpseudocode}%
\usepackage{listings}%

\theoremstyle{thmstyleone}%
\newtheorem{theorem}{Theorem}%
\newtheorem{proposition}[theorem]{Proposition}% 
\theoremstyle{thmstyletwo}%
\newtheorem{example}{Example}%
\newtheorem{remark}{Remark}%
\theoremstyle{thmstylethree}%
\newtheorem{definition}{Definition}%
\raggedbottom

\begin{document}

\title[Quantum-enabled framework for the Advanced Encryption Standard in the post-quantum era]{Quantum-enabled framework for the Advanced Encryption Standard in the post-quantum era}

\author[1]{\fnm{Albert} \sur{Nieto Morales}}
\author*[1]{\fnm{Arit Kumar} \sur{Bishwas}}\email{arit.kumar.bishwas@pwc.com}
\author[1]{\fnm{Joel Jacob} \sur{Varghese}}

\affil*[1]{\orgdiv{Innovation Hub}, \orgname{PricewaterhouseCoopers}, \orgaddress{\country{USA}}}

\abstract{Quantum computers create new security risks for today's encryption systems. This paper presents an improved version of the Advanced Encryption Standard (AES) that uses quantum technology to strengthen protection. Our approach offers two modes: a fully quantum-based method for maximum security and a hybrid version that works with existing infrastructure. The system generates encryption keys using quantum randomness instead of predictable computer algorithms, making keys virtually impossible to guess. It regularly refreshes these keys automatically to block long-term attacks, even as technology advances. Testing confirms the system works seamlessly with current security standards, maintaining fast performance for high-volume data transfers. The upgraded AES keeps its original security benefits while adding three key defenses: quantum-powered key creation, adjustable security settings for different threats, and safeguards against attacks that exploit device vulnerabilities. Organizations can implement this solution in stages—starting with hybrid mode for sensitive data while keeping older systems operational. This phased approach allows businesses to protect financial transactions, medical records, and communication networks today while preparing for more powerful quantum computers in the future. The design prioritizes easy adoption, requiring no costly replacements of existing hardware or software in most cases.}


\keywords{quantum random number generator, advanced encryption standard, post-quantum cryptography, quantum-safe encryption, quantum-enabled AES}


\maketitle

\section{Introduction}\label{sec1}

This section outlines the fundamental motivations driving this research, the core problem posed by quantum-era vulnerabilities in AES, and the key contributions of our proposed quantum-enabled AES solution.

\subsection{Motivation}

The Advanced Encryption Standard (AES) remains the cornerstone of symmetric-key cryptography, securing everything from financial transactions to government communications. Its efficiency and standardized design have made it ubiquitous in classical computing environments \cite{sharma2016novel, raghu2015application}. However, the rise of quantum computing threatens to erode AES’s security guarantees. Quantum algorithms like Grover’s reduce the effective security of AES-128 by half \cite{bonnetain2019quantum}, while hybrid quantum-classical adversaries could exploit vulnerabilities in key generation or implementation \cite{jang2020grover}. Although AES-256 retains stronger theoretical resistance, its reliance on classical entropy sources and static key schedules leaves it exposed to sophisticated attacks in the quantum era. This underscores the urgent need to re-engineer AES with quantum-safe primitives while preserving its operational efficiency.

\subsection{Problem statement}

Current AES implementations face three critical challenges in a post-quantum landscape:

\begin{itemize}
    \item Grover's algorithm reduces the brute-force search complexity for AES-$n$ from \(O(2^n)\) to \(O(2^{n/2})\) \cite{dong2021automatic}, effectively halving security margins.
    
    \item Classical pseudo-random number generators (PRNGs) lack true randomness, creating exploitable patterns in key material \cite{vaudenay2005classical}.
    
    \item Adversaries combining quantum brute-force with classical side-channel analysis could circumvent traditional defenses \cite{yan2019quantum}.
\end{itemize}


We assume an adversary with access to a large-scale quantum computer capable of running Grover’s algorithm, thereby reducing the complexity of brute-force key searches from \(O(2^n)\) to approximately \(O(2^{n/2})\). In addition, the adversary can adopt a “harvest-now-decrypt-later” strategy \cite{mosca2017cybersecurity}, collecting ciphertext today in anticipation of future quantum resources to decrypt sensitive information. Side-channel attacks are also considered, where adversaries may exploit physical leakage (e.g., electromagnetic emissions or timing information) to infer partial key material. This model further allows for coordinated multi-vector approaches, such as combining quantum-accelerated brute force with classical side-channel analysis to maximize cryptanalytic success \cite{krachenfels2021automated}. By articulating this threat landscape, we establish the baseline assumptions under which our quantum-enabled AES architecture is designed to operate securely.

While post-quantum cryptography (PQC) standardization efforts focus on asymmetric algorithms \cite{stebila2016post,alagic2020status}, symmetric systems like AES require parallel innovations to resist quantum-capable adversaries. Existing proposals often neglect holistic threat models or fail to integrate seamlessly with legacy infrastructure. In this paper, we propose a quantum-enabled AES framework designed to counteract quantum-era threats effectively. Our main contributions include the integration of a QRNG to supply key materials with entropy higher than that of classical pseudorandom approaches, thus mitigating the vulnerabilities posed by Grover's algorithm \cite{wang2022quantumsafe,basu2019nist}. We demonstrate how quantum-generated keys can preserve or enhance the ``bits of security'' for AES-256 through rigorous entropy validation protocols, even as large-scale quantum computers become a reality \cite{han2014improving,dong2021automatic}. Using genuine quantum randomness from photon sampling and quantum walk processes, our solution addresses key expansion limitations while resisting prospective cryptanalytic attacks including brute-force, linear, and differential cryptanalysis \cite{yan2019quantum,vaudenay2005classical,ma2016quantum}. Finally, we detail an architectural blueprint featuring hardware-software co-design principles \cite{sadr2022hardware} and discuss practical methods for integrating QRNG-based key material into classical AES workflows through backward compatible API extensions and FIPS-compliant entropy mixing \cite{dworkin2018nist}, facilitating ease of adoption and immediate deployment in existing cryptographic infrastructure .

\section{Background}
This section merges foundational concepts and state-of-the-art research on AES, quantum threats, the post-quantum landscape, and random number generation methods. 

\subsection{Quantum computing}

Quantum computing operates on principles fundamentally distinct from classical computing, drawing on quantum mechanics to process information in ways that defy classical intuition \cite{nielsen2010quantum}. At its core are qubits, which differ from classical bits by exploiting superposition, the ability to exist in multiple states (0 and 1) simultaneously, and entanglement, a phenomenon where qubits share interconnected states regardless of physical separation \cite{preskill2018nisq}. This duality enables quantum systems to explore vast computational possibilities in parallel, as illustrated in Fig.~\ref{fig:random_behavior}. For example, while a classical computer evaluates solutions sequentially, a quantum machine can manipulate entangled qubits to test many solutions at once, offering exponential speedups for specific problems \cite{arute2019quantum}. 

\begin{figure}[h!]
    \centering
    \includegraphics[width=0.7\textwidth]{img/random_behavior.png}
    \caption{Classical vs. quantum bit representation. Classical bits (left) are binary (0/1), while qubits (right) leverage superposition to exist in probabilistic states.}
    \label{fig:random_behavior}
\end{figure}

However, current quantum hardware remains fragile and error-prone. Most devices operate in the Noisy Intermediate-Scale Quantum (NISQ) era \cite{preskill2018nisq}, limited to dozens or hundreds of qubits plagued by decoherence and noise \cite{terhal2015quantum}. Practical applications today focus on niche tasks like simulating molecular interactions \cite{castelvecchi2017quantum} or optimizing logistics, with error rates still too high for broader use. Yet, progress in error-correction techniques, such as topological qubits \cite{fowler2012surface} or fault-tolerant architectures \cite{nist2021}, hints at a future where stable, large-scale quantum systems could revolutionize fields like cryptography \cite{shor1999polynomial}, materials science \cite{cao2019quantum}, and artificial intelligence \cite{biamonte2017quantum}.


\subsection{Advanced encryption standard}

Standardized by the U.S. National Institute of Standards and Technology (NIST) in 2001 \cite{nist2001fips197}, AES emerged as the successor to the aging Data Encryption Standard (DES). Designed by Joan Daemen and Vincent Rijmen \cite{daemen2002design}, AES operates on 128-bit data blocks and supports symmetric keys of 128, 192, or 256 bits. Its structure employs a substitution-permutation network (SPN) that iteratively transforms plaintext into ciphertext through a series of mathematical operations \cite{biryukov2009structural}.

\begin{figure}[h!]
    \centering
    \includegraphics[width=0.4\textwidth]{img/structureaes.png}
    \caption{Overview of the AES encryption process. The secret key and plaintext undergo multiple rounds of substitution, permutation, and mixing to generate ciphertext. Secure key generation, reliant on high-entropy randomness, is critical to prevent brute-force attacks.}
    \label{fig:aes-key-generation}
\end{figure}

As illustrated in Figure~\ref{fig:aes-key-generation}, the encryption process begins with plain text and a secret key. These inputs undergo multiple rounds of transformations designed to systematically obscure and scramble data. Each round applies four core operations \cite{daemen2002design}:
\begin{itemize}
    \item Non-linear byte substitutions via S-boxes, lookup tables that replace each byte with a predetermined value to break visible patterns \cite{keliher2001modeling}.
    \item Row-wise cyclic shifts, rotating bytes within each row to disperse data relationships.
    \item Columnar mixing using finite field arithmetic, mathematical blending of column values to propagate changes across the entire block.
    \item Round-key integration through XOR operations, merging processed data with unique subkeys derived from the main secret.
\end{itemize}


The number of transformation rounds—10, 12, or 14—scales with the key length (128, 192, or 256 bits respectively), creating a tunable balance between cryptographic strength and computational demands \cite{dworkin2018nist}. This layered approach ensures even minor changes to the input dramatically alter the ciphertext—a property called the avalanche effect \cite{liu2019avalanche}—while remaining efficient for widespread use \cite{schneier2015applied}.

AES became the global encryption standard by overcoming DES's vulnerabilities\textemdash notably its short 56-bit keys crackable by 1990s hardware \cite{coppersmith1994data, eff1998cracking}\textemdash through robust 128\textendash 256-bit keys and a substitution-permutation network (SPN) that resists pattern analysis. While DES fell to brute-force attacks in hours, AES's mathematical structure and expanded key space require infeasible $2^{128}$ operations to breach \cite{grassl2016grover}, even with quantum-accelerated methods. Its efficiency in hardware/software led to ubiquitous adoption in TLS, Wi-Fi, and disk encryption \cite{barker2020nist}, but security hinges on truly random keys: weak key generation undermines AES's strength regardless of algorithmic soundness \cite{bonneau2012science}. The transition from DES highlighted cryptography's arms race with computing power\textemdash AES's larger blocks (128 vs.\ 64 bits) and SPN design prevent the specialized hardware attacks that doomed its predecessor \cite{shamir2000efficient}, ensuring relevance in modern security ecosystems.


\subsection{Post-quantum cryptography}

The cryptographic implications of quantum computing are profound, necessitating the development of post-quantum cryptography—a field dedicated to creating algorithms resistant to both classical and quantum attacks. Shor’s algorithm \cite{shor1997polynomial} threatens to dismantle widely used encryption methods like RSA and elliptic curve cryptography (ECC) by solving integer factorization and discrete logarithms exponentially faster than classical methods \cite{dong2021automatic}. Similarly, Grover’s algorithm \cite{grover1996fast} weakens symmetric-key security, reducing the effective strength of AES-128 by half \cite{bonnetain2019quantum}. These vulnerabilities are not merely theoretical; experimental demonstrations have already factored small integers using quantum principles \cite{arute2019quantum}. While large-scale attacks remain years away, the urgency to adopt post-quantum cryptography—algorithms resistant to quantum brute-force searches—has spurred global standardization efforts, with organizations like NIST evaluating candidates for future protocols \cite{nist2022pqc}. 

The risks extend beyond brute-force key searches. Quantum algorithms like Simon’s enable collision attacks on cryptographic modes, as demonstrated by Chang et al. in their analysis of AES-OTR, an authenticated encryption scheme. By exploiting periodicity in associated data, attackers can forge authentication tags with quantum acceleration, compromising data integrity even without recovering the key \cite{chang2022collision}. Such attacks underscore the need for quantum-resistant designs in both symmetric and asymmetric cryptography. To mitigate these threats, NIST launched a post-quantum cryptography standardization initiative in 2016, evaluating candidates for encryption and digital signatures. Leading proposals include lattice-based schemes like CRYSTALS-Kyber \cite{bos2018crystals}, which rely on the hardness of learning-with-errors (LWE) problems; hash-based signatures such as SPHINCS+ \cite{bernstein2019sphincs}, which leverage cryptographic hash functions; and code-based systems like Classic McEliece \cite{mceliece1978public}, rooted in error-correcting code theory. These algorithms aim to replace quantum-vulnerable primitives while maintaining compatibility with existing infrastructure. Hybrid systems, which combine AES-256 with post-quantum key encapsulation \cite{braithwaite2016hybrid}, offer a transitional solution, ensuring backward compatibility while hardening against quantum adversaries.

To systematically evaluate and categorize the security of post-quantum cryptographic algorithms, NIST has established five distinct security levels within its PQC standardization framework \cite{nist2020ir8309}. These levels correspond to equivalent classical security strengths, ensuring that post-quantum algorithms provide robust protection against both classical and quantum adversaries. Specifically, NIST defines the following security levels:

\begin{itemize}
    \item Level 1 targets security equivalent to AES-128, suitable for applications requiring protection against near-term quantum threats.
    \item Level 2 aligns with AES-192, offering enhanced security for more sensitive data.
    \item Level 3 corresponds to AES-256, intended for high-security needs, safeguarding against advanced quantum attacks.
    \item Level 4 provides security beyond AES-256, addressing applications with exceptionally high confidentiality requirements and anticipating future quantum advancements.
    \item Level 5 represents the highest security tier, designed for the most critical systems where maximum resistance to both current and foreseeable quantum threats is imperative.
\end{itemize}


Adopting PQC is not merely a technical challenge but a strategic imperative. Adversaries can exploit “harvest-now, decrypt-later” tactics \cite{mosca2018harvest}, stockpiling encrypted data today for decryption once quantum computers mature. This has led to various frameworks and roadmaps aimed at guiding industries in preparing for the quantum threat at both technical and operational levels \cite{bishwas2024strategicroadmapquantumresistant}. Industries handling sensitive data—healthcare, defense, finance—must prioritize PQC integration to safeguard long-term confidentiality. However, challenges remain, including performance overheads for lattice-based schemes \cite{ducas2018crystals} and the logistical complexity of migrating legacy systems. Bonnetain et al. emphasize that even AES-256’s security margins depend on implementation rigor, as quantum-optimized attacks on reduced-round variants demonstrate vulnerabilities in suboptimal configurations \cite{bonnetain2019quantum}. Thus, PQC adoption must be holistic, addressing cryptographic agility, entropy quality, and protocol-level integration to withstand the dual threats of classical and quantum adversaries.


\subsection{Random number generators}

Computers typically generate random numbers through deterministic algorithms known as pseudo-random number generators (PRNGs). These systems rely on initial seed values and mathematical functions to produce sequences that mimic randomness \cite{rarity1994quantum, liu2018device}. While modern PRNGs exhibit long periods and pass statistical randomness tests, their outputs remain predictable if an adversary gains access to the seed or internal state—a critical flaw in security-sensitive contexts. For non-cryptographic applications, such as simulations or gaming, PRNGs offer sufficient randomness. However, their deterministic nature poses risks in cryptography, where predictability undermines secrecy.

True random number generators (TRNGs) address this limitation by deriving randomness from physical processes, such as thermal noise or quantum fluctuations \cite{ghosh2022quantum, furst2010high}. TRNGs measure inherently unpredictable phenomena, making their outputs resistant to reverse engineering or state reconstruction. For example, classical hardware-based TRNGs might sample electrical noise, but these methods can still exhibit subtle correlations under specific environmental conditions. In contrast, quantum random number generators (QRNGs) exploit the intrinsic probabilistic nature of quantum mechanics—such as photon polarization or arrival times—to generate randomness immune to classical predictability (Fig.~\ref{fig:photons}). QRNGs operate by measuring quantum events with no deterministic precursor. 

\begin{figure}[h!]
\centering
\includegraphics[width=\textwidth]{img/photons.png}
\caption{A photon source emits light particles toward a semi-transparent mirror, which randomly reflects or transmits each photon. Two detectors record the path taken by each photon, encoding reflections as '0' and transmissions as '1'. The sequence of detections generates a random bitstring derived from quantum-mechanical probabilities.}
\label{fig:photons}
\end{figure}

A common implementation involves directing photons toward a semi-transparent mirror, where each photon randomly follows one of two paths. Detectors register the path as a '0' or '1', producing a random bitstream. Unlike classical physical processes, such as coin tossing or electrical noise, quantum measurements cannot be predicted even with complete knowledge of initial conditions. This arises from the quantum principle that measurement outcomes are fundamentally probabilistic, not merely chaotic or sensitive to hidden variables \cite{yan2019quantum, ma2016quantum}.

For instance, while classical systems like coin tosses appear random due to complex initial conditions (e.g., force, air resistance), their outcomes could theoretically be predicted with sufficient precision. Quantum systems, however, lack such determinism, as environmental interactions cannot fully explain measurement results, a distinction rooted in the mathematical framework of quantum theory. The security advantage of QRNGs lies in their ability to provide verifiable entropy. Classical TRNGs may inadvertently introduce bias or correlations due to hardware imperfections or environmental interference. QRNGs, in contrast, derive randomness from quantum processes that are provably nondeterministic, offering robust defenses against both classical and quantum adversaries \cite{rarity1994quantum}. This makes QRNGs indispensable for cryptographic key generation, where long-term secrecy depends on irreproducible randomness.

In practice, several QRNG designs exist beyond the single-photon split approach. For example, the so-called device-independent methods aim to certify randomness through loophole-free Bell tests \cite{bierhorst2018experiment,pironio2010random}, minimizing the need to trust physical assumptions. Continuous-variable implementations measure optical field quadratures \cite{gabriel2016continuous} or phase fluctuations at higher throughput, but can involve more complex calibration to ensure truly quantum origins. Commercial vendors offer different hardware solutions with varying bit rates, form factors, and certifications, making trade-offs between cost and security assurances \cite{stucki2010high}. In this work, we selected a compact, single-photon-based QRNG to balance ready availability with robust entropy generation. Its hardware design is well suited for common computing environments, offering sufficient bit rates for typical AES key generation while keeping system integration overhead low. This choice also helps align the experimental framework with real-world deployment scenarios, where factors such as device certification, USB connectivity, and straightforward software APIs are often critical \cite{herrero2017quantum}. Although device-independent QRNGs offer stronger security guarantees by removing trust assumptions about the hardware, they can be more expensive and are not yet commercially available.

\subsection{Previous implementations}

Several research teams have developed quantum-adapted AES implementations to assess its vulnerability in quantum computing scenarios, implementing AES inside a quantum computer. Almazrooie et al. designed a compact AES-128 quantum circuit using only 928 qubits while preserving cryptographic properties such as uniqueness of keys, critical to resist quantum brute-force attacks \cite{almazrooie2018quantum}. Wang et al. demonstrated a 30\% reduction in the required quantum resources (qubits and operations) for AES-128 by optimizing how subkeys are integrated during encryption rounds \cite{wang2022quantumcircuit}. Huang and colleagues introduced methods to simplify quantum circuits for AES components such as S-boxes, significantly reducing computational steps without compromising security \cite{huang2022synthesizing}. Zou et al. further improved efficiency by reusing intermediate values during encryption, enabling AES-256 execution with 768 qubits - a 40\% reduction compared to earlier designs \cite{zou2020quantum}. Recent work by Sharma et al. hybridized classical and quantum processes, using quantum randomness for key generation while offloading bulk encryption to classical hardware, ensuring backward compatibility \cite{sharma2023post}. Although these advances demonstrate progress, current implementations remain theoretical due to hardware limitations: full AES-128 still requires $\sim$900 error-corrected qubits, exceeding the capabilities of today's noisy quantum computers. Nevertheless, broader research efforts have examined the feasibility of building quantum cryptographic systems to mitigate post-quantum threats, including methodologies that address implementation aspects from design to deployment \cite{9698591}.


\section{Methodology}\label{sec4}
This section details the design and implementation of two quantum-enhanced AES frameworks: a pure quantum entropy architecture (QE-P) and a hybrid quantum-classical variant (QE-H). We refer collectively to both modes (QE-P and QE-H) as QE-AES in formal proofs. The methodology addresses critical challenges in post-quantum symmetric cryptography. Finally, we describe lattice-based augmentation for postquantum transition security, implementation benchmarks, and side-channel resistance strategies. 

\subsection{Experimental setup}
The experiments were conducted using a Quantis USB 4M QRNG \cite{IDQuantique2025}, manufactured by ID Quantique. The following tables list the key specifications:


\begin{table}[h!]
\caption{Quantis USB 4M QRNG specifications.}
\centering
\begin{tabular}{ll}
\toprule
\textbf{Specification} & \textbf{Value} \\
\midrule
Random Bit Rate & 4 Mbit/s $\pm$ 10\% \\
Thermal Noise Contribution & $<$ 1\% \\
Storage Temperature & -25°C to +85°C \\
Dimensions & 61 mm $\times$ 31 mm $\times$ 114 mm \\
USB Specification & 2.0 \\
Requirement & PC with USB connector \\
Power & Via USB port \\
\bottomrule
\end{tabular}

\label{table:quantis}
\end{table}


The Quantis USB 4M serves as the quantum entropy source in our framework. It delivers raw quantum bits at a throughput of 4 Mbit/s (with a 10\% tolerance) prior to any extraction or post-processing. The fact that less than 1\% of the output arises from thermal noise confirms that the device's randomness is predominantly based on quantum effects. Our system is based on an Apple M3 Pro.


\begin{table}[h!]
\caption{Computing environment specifications.}
\centering
\begin{tabular}{ll}
\toprule
\textbf{System specification} & \textbf{Value} \\
\midrule
Processor & Apple M3 Pro, 3.8 GHz, 12 cores \\
Memory (RAM) & 36 GB \\
Operating system & macOS 15.2 \\
\bottomrule
\end{tabular}

\label{table:system}
\end{table}

\newpage
\subsection{Pure quantum randomness architecture}

In the pure quantum mode (QE-P), the AES derives all of its cryptographic material directly from a quantum source. The seed for the AES master key, as well as the subkeys used in each encryption round, come from random quantum phenomena rather than from classical or pseudo-random methods. A QRNG is used to generate bits by measuring random quantum events, which yields a stream of raw bits,

\begin{equation}
    \{b_i\}_{i=1}^{N},
\end{equation}

that exhibit inherent quantum unpredictability.

Due to minor device imperfections, the raw quantum bitstream may have slight biases. A straightforward and widely used technique is the pairwise approach (sometimes called Von Neumann extraction), which processes bits in pairs to remove any skew. Conceptually, if
\begin{equation}
(b_{2j}, b_{2j+1}) = (0,1)
\end{equation}
output ‘0’; if
\begin{equation}
(b_{2j}, b_{2j+1}) = (1,0)
\end{equation}
output ‘1’; otherwise discard. 

Let the result of this bias-correction stage be \(R\), a high-entropy random string.

From \(R\), a 256-bit AES master key is extracted via a simple hash or keyed-hash function:
\begin{equation}
K_{\text{master}} = \text{Hash}(R) \in \{0,1\}^{256}.
\end{equation}
This step condenses the quantum bitstream into a fixed-length key for AES-256, ensuring uniform distribution over the \(2^{256}\) key space.

To further increase robustness against cryptanalysis (including potential quantum side-channel or related-key attacks), each encryption round can incorporate a small block of fresh quantum randomness. In AES-256, the classical key schedule expands \(K_{\text{master}}\) into round subkeys:
\begin{equation}
K_0, K_1, K_2, \dots, K_{14}.
\end{equation}
In the QE-P design, additional quantum bits \(Q_i\) are injected into each round subkey to disrupt any predictive structure:
\begin{equation}
K_i' = K_i \oplus Q_i,
\end{equation}
where each \(Q_i\) is, for example, 128 bits drawn from the same QRNG (and hashed or bias-corrected for uniformity). This ensures that even if an attacker partially learns something about \(K_i\) through side-channel leakage, they must still guess the fresh quantum part \(Q_i\).

During encryption, the Round Key Addition uses \(K_i'\) in place of the classical subkey \(K_i\). The other AES steps (SubBytes, ShiftRows, MixColumns) remain unchanged from a standard AES-256 implementation. Because the only modification is in how the subkeys are formed, the rest of AES’s workflow stays fully compatible with existing libraries and hardware accelerators. The final ciphertext is thus indistinguishable from a standard AES-256 ciphertext, but it benefits from a master key and subkeys that are far more resistant to quantum-enabled brute force and related-key attacks.

Let us now consider the mathematical logic of the QE-P. The QRNG outputs a sequence \(R\) with near-perfect min-entropy per bit (close to 1 bit of entropy for each physical bit generated). This randomness is not merely hard to predict; it is totally unpredictable. By hashing (or similarly condensing) \(R\) into a 256-bit string, a uniform distribution over the space \(\{0,1\}^{256}\) is ensured. This uniformity is crucial: if an attacker does not know \(R\), then \(K_{\text{master}}\) is infeasible to guess, even with quantum resources.

Each subkey \(K_i'\) is formed by XORing in fresh quantum bits \(Q_i\). This process blinds the round key so that if an adversary attempts a side-channel or related-key attack (which often relies on predicting or correlating round keys), they must also confront a genuinely random component that changes each time. Formally,
\begin{equation}
\Pr(\text{guess all } K_i') \approx \frac{1}{2^{256+\sum (\text{size of } Q_i)}}.
\end{equation}
Even a partial leak of \(K_i\) does not compromise the subkey if \(Q_i\) remains secret.

Backward security is achieved because if a subkey \(K_i'\) is revealed at some time, fresh quantum randomness in subsequent rounds prevents an attacker from deducing future subkeys. Forward security is maintained because if an old subkey is revealed, it does not directly enable predicting earlier quantum bits used in other rounds unless the attacker can break the QRNG’s quantum process (which is assumed infeasible under standard quantum-mechanical assumptions). Through these mechanisms, the QE-P architecture enhances AES-256 with additional layers of unpredictability. Keys are no longer derived exclusively from classical sources that might be vulnerable to pseudo-random weaknesses or advanced quantum cryptanalysis. Instead, each step in AES key generation and scheduling is regularly refreshed by truly random quantum data, thereby making attacks—classical or quantum—dramatically more difficult in practice.

\subsection{Hybrid quantum-classical architecture}

In the hybrid quantum–classical mode (QE-H), AES key material is derived from both quantum and classical sources of randomness. This approach combines the strong unpredictability of quantum bits with the established reliability of classical cryptographic methods, minimizing single points of failure and facilitating integration with existing infrastructures. QE-H aggregates entropy from multiple origins, such as a quantum generator and one or more operating-system or hardware-based classical random number generators. The outputs of these distinct streams are combined into a single hybrid seed:
\begin{equation}
E_{\text{hybrid}} = \text{Mix}(E_{\text{quantum}} \parallel E_{\text{classical}}),
\end{equation}
where \(E_{\text{quantum}}\) is a block obtained from a quantum process, \(E_{\text{classical}}\) is derived from vetted classical sources, and \(\text{Mix}(\cdot)\) is a cryptographic mixing function designed to preserve and diffuse entropy across the final output. 

From the hybrid entropy \(E_{\text{hybrid}}\), the system extracts the AES master key by condensing it into a fixed-length 256-bit string:
\begin{equation}
K_{\text{master}} = \text{Hash}(E_{\text{hybrid}}) \in \{0,1\}^{256}.
\end{equation}
This step guarantees a uniform distribution of the master key over the 256-bit space. Each round key from the classical AES-256 key schedule is further enhanced by incorporating additional bits from \(E_{\text{hybrid}}\). Specifically, a chunk of mixed randomness, denoted by \(\Delta_i \subset E_{\text{hybrid}}\), is XORed with the round key:
\begin{equation}
K_i' = K_i \oplus \Delta_i.
\end{equation}

The design offers layered unpredictability; even if one source of randomness fails or is compromised, the other continues to secure the key material. This duality means that an attacker would need to compromise both quantum and classical sources to undermine security. Furthermore, the approach maintains backward compatibility, allowing organizations to incorporate quantum entropy into existing systems without altering ciphertext formats or decryption procedures. The architecture also reduces dependence on any single randomness source, ensuring continuity if either the quantum or classical component is temporarily weakened.



\subsection{Entropy management and validation}

To maintain a high level of cryptographic assurance, both the pure quantum and hybrid architectures should continually verify the quality of their randomness. This entails monitoring the output from the quantum and/or classical entropy sources in real-time and triggering corrective actions (such as re-seeding) if anomalous patterns are detected.

A typical framework for managing entropy quality includes a dedicated subsystem collects batches of random bits at periodic intervals (for example, every few milliseconds) to ensure that the bitstream remains uniform and unpredictable. Statistical checks are performed on these batches, including tests for frequency to confirm that 0’s and 1’s appear in expected proportions, correlation tests to identify unwanted dependencies among consecutive bits or blocks, and tests for runs or sub-blocks to detect long sequences of repeated values that might indicate a breakdown in the randomness source.

Acceptable statistical bounds are defined in accordance with recognized guidelines such as FIPS or the NIST SP 800 series. Should any metric deviate substantially from its normal range, the system may automatically reseed by switching to a backup entropy source or injecting fresh quantum material to restore randomness. In addition, security logs record these events and notifications are sent to operators for further investigation. Whenever reseeding occurs, new high-entropy material is integrated into the AES key-generation process. This ensures that both the master key and round keys maintain their unpredictability, even in the event of temporary degradation in the randomness source.


\subsection{Key lifecycle}

A robust key lifecycle is essential to ensure that high-entropy quantum randomness protects data at every stage of encryption and decryption. Initially, a master key \(K_{\text{root}}\) is derived by combining quantum-generated bits with necessary contextual data, such as system identifiers. A secure hash or key-derivation function then condenses this input into a uniform 256-bit key, ensuring that the initial key is both unpredictable and uniquely tied to its environment.

Once AES expands \(K_{\text{root}}\) into round subkeys, each subkey is enhanced by blending it with a short quantum nonce. In particular, for each round \(i\), the modified subkey is defined as:
\begin{equation}
k'_i = k_i \oplus q_i,
\end{equation}
where \(q_i\) represents fresh quantum randomness allocated to that round. This whitening process hinders side-channel and related-key attacks by preventing correlation of round keys across different sessions or devices.

To further reduce exposure in the event of a key compromise or partial leakage, the system periodically generates entirely new keys through dynamic rekeying. This regeneration, which may be triggered after a set number of encrypted data blocks or after a fixed time interval, reintroduces high-entropy quantum bits so that previous key material provides no advantage to an attacker in predicting future keys.

When a key reaches the end of its lifecycle or is replaced, its sensitive material is securely erased by overwriting it with fresh random data, preferably derived from quantum entropy. This secure erasure complicates forensic attempts to recover any remnants of the old key from memory or storage media.

\subsection{Practical implementation}

Organizations can integrate the proposed quantum-enabled AES approach with existing hardware security modules (HSMs), key management services (KMS), and popular cryptographic libraries such as OpenSSL without requiring extensive modifications. HSMs are widely used to store and manage cryptographic keys within secure, tamper-resistant hardware. In a typical setup, the QRNG streams raw quantum bits into a conditioning process that refines the randomness, which is then passed to the HSM via a secure interface. The HSM subsequently uses these high-entropy values to seed internal random number generation, derive symmetric AES keys, or update master key material. By relying on the HSM’s standard APIs and protocols, administrators can configure the device to replace or augment its default entropy source with quantum-derived bits while preserving existing key-generation workflows. 

A similar pattern applies to software-based KMS solutions, where a trusted daemon or service receives the QRNG output, performs optional post-processing, and injects the quantum entropy into the key lifecycle. Implementations can expose this capability through well-known API endpoints, allowing existing applications to request quantum-safe keys without altering higher-level business logic. For integration with cryptographic libraries like OpenSSL, developers can register a custom entropy callback that invokes the QRNG source whenever new randomness is needed. In this workflow, the library continues to function as usual, except that critical operations such as AES key generation now draw from quantum randomness rather than relying solely on classical pseudorandom number generators. This design path ensures compatibility with existing toolchains and preserves compliance with standards such as FIPS 140, provided that the QRNG and related processes meet relevant certification requirements.


\section{Evaluation}\label{sec6}

In this section, we evaluate the quantum-enhanced AES framework through three complementary methodologies: (1) entropy quantification, (2) NIST SP 800-22 compliance testing, and (3) Dieharder battery testing. We then discuss how these findings translate into practical security gains.

\subsection{Entropy analysis}
We first compare the entropy of 100\,MB random bitstreams produced by our QRNG and by CTR-DRBG as a high-quality pseudorandom number. Table~\ref{table:entropy} summarizes the results from the \texttt{ENT} test suite.

\begin{table}[h!]
\caption{Comparison of randomness metrics between a QRNG and a PRNG.}
\centering
\small
\begin{tabular}{l c c}
\toprule
Metric & QRNG & PRNG \\
\midrule
Bits per byte & 7.999998 & 7.999998 \\
Chi-square ($\chi^2$) & 326.75 & 282.97 \\
Chi-square p-value & 0.16\% & 11.02\% \\
Monte Carlo $\pi$ error & 0.01\% & 0.00\% \\
Serial correlation & -0.000029 & -0.000039 \\
\bottomrule
\end{tabular}
\label{table:entropy}
\end{table}

Both generators achieve essentially maximal Shannon entropy (7.999998 bits/byte). A noteworthy difference appears in the $\chi^2$ p-values: the quantum source exhibits a relatively small p-value (0.16\%), whereas the PRNG shows a higher p-value (11.02\%). In typical randomness testing, extremely small p-values (\(<0.01\%\)) can indicate non-random patterns; however, p-values within roughly \((0.0001, 99.9999)\%\) are often considered acceptable for such tests, and slight shifts can be attributed to sampling variance, test quirks, or hardware idiosyncrasies. Both distributions here remain within a generally acceptable range for cryptographic purposes, passing frequency-based criteria without strong signs of bias.

\subsection{NIST SP 800-22 compliance}
Next, we subjected 800 separate 1\,MB samples to the NIST SP 800-22 test suite, covering 15 primary tests (many having multiple subtests). Table~\ref{table:nist} shows a selection of pass rates in key categories.

\begin{table}[h!]
\caption{Pass rates for selected NIST SP 800-22 tests applied to QRNG and PRNG outputs, each tested on 800 samples of size 1\,MB.}
\centering
\begin{tabular}{l c c}
\toprule
Test category & QRNG pass rate & PRNG pass rate \\
\midrule
Frequency                     & 100\% & 100\% \\
Block Frequency               & 100\% & 100\% \\
Cumulative Sums               & 100\% & 100\% \\
Runs                          & 100\% & 100\% \\
Non-overlapping Templates     & 98.65\% & 97.97\% \\
Overlapping Templates         & 100\% & 100\% \\
Universal Statistical         & 100\% & 100\% \\
Random Excursions             & 100\% & 100\% \\
Linear Complexity             & 100\% & 100\% \\
\bottomrule
\end{tabular}
\label{table:nist}
\end{table}

Both generators perform well, with perfect scores on frequency-based tests and near-perfect scores on template-based tests. The differences between 98.65\% and 97.97\% in non-overlapping templates are minor but suggest that the QRNG exhibits slightly fewer detectable patterns. Overall, both sources comfortably pass NIST’s recommended thresholds for cryptographically secure random number generation.

\subsection{Dieharder test battery results}
To further validate randomness, we employed the Dieharder battery of tests, which includes additional assessments beyond NIST SP 800-22. Table~\ref{table:dieharder} lists representative results. Each p-value reflects how well the bitstream fits the expected statistical distribution for a given test; pass rates indicate the fraction of repeated test instances that fell within acceptable bounds.

\begin{table}[h!]
\caption{Dieharder test results. Pass rates below 100\% can arise from natural statistical variance.}
\centering
\small
\begin{tabular}{l c c c c}
\toprule
\multirow{2}{*}{Test name} & \multicolumn{2}{c}{p-value} & \multicolumn{2}{c}{Pass rate} \\
 & QRNG & PRNG & QRNG & PRNG \\
\midrule
\texttt{dab\_bytedistrib}      & 0.4097 & 0.5363 & 100\% & 100\% \\
\texttt{dab\_dct}              & 0.8348 & 0.3837 & 100\% & 100\% \\
\texttt{diehard\_birthdays}    & 0.2528 & 0.2515 & 100\% & 100\% \\
\texttt{diehard\_count\_1s\_str} & 0.5529 & 0.6921 & 100\% & 100\% \\
\texttt{diehard\_operm5}       & 0.0421 & 0.8881 & 100\% & 100\% \\
\texttt{rgb\_lagged\_sum}      & 0.5620 & 0.5654 & 97\%  & 91\%  \\
\texttt{rgb\_minimum\_distance}& 0.7169 & 0.5996 & 100\%  & 100\% \\
\texttt{sts\_serial}           & 0.4947 & 0.5760 & 100\% & 100\% \\
\bottomrule
\end{tabular}

\label{table:dieharder}
\end{table}

Both the QRNG and PRNG pass the majority of tests with robust p-values. Small pass-rate variations, are not uncommon across multiple test repetitions, especially when analyzing large data volumes. Given that neither generator fails consistently on any specific test, the results are consistent with high-quality randomness suitable for cryptographic applications.

\subsection{Security enhancement}

A primary concern for symmetric-key systems in the post-quantum era is Grover's algorithm, which reduces brute-force key-search complexity from $O(2^n)$ to $O(2^{n/2})$. For a 256-bit key, this theoretical speedup implies an effective complexity on the order of $2^{128}$. While this level remains large, augmenting AES with genuine quantum randomness makes it more difficult for an adversary to guess or predict the key schedule, thus adding an additional barrier to practical quantum attacks.

A second vulnerability involves "harvest-now-decrypt-later" (HNDL) strategies, in which attackers capture encrypted data today and wait for advances in quantum hardware to decrypt it later. By regularly refreshing keys with high-entropy quantum bits---and keeping key lifetimes short---organizations limit the amount of data any single key can protect. Consequently, an attacker's computational burden grows, further discouraging attempts to breach the system. Additionally, side-channel attacks often rely on deducing partial information about a key schedule. If a device uses purely classical randomness, learning a few bits of one round key can help predict others. In contrast, injecting fresh quantum randomness into each round key frustrates such predictions: partial knowledge of one round no longer helps an adversary infer future or past keys. This reduction in key predictability markedly strengthens defenses against hardware-based or software-based side-channel threats.


\subsection{Summary of evaluation}
Table~\ref{table:threats} highlights how each threat vector is addressed by quantum-derived keys and frequent rekeying. The statistical tests confirm that the QRNG consistently provides high-quality randomness, meeting (and sometimes exceeding) the standards set by NIST SP 800-22 and Dieharder. While the raw security margin gain against Grover’s algorithm may appear small in isolation, in practice it synergizes with AES-256 strength and short key lifetimes to form a meaningful additional barrier. 

\begin{table}[h!]
\caption{Key quantum threats and how they are mitigated in our framework.}
\centering
\small
\begin{tabular}{l l}
\toprule
Threat vector               & Mitigation                                \\
\midrule
Grover's algorithm          & Slightly increased effective key space    \\
Harvest-now-decrypt-later   & Frequent quantum-based key rotation       \\
Quantum side-channels       & Entropy masking at each round derivation  \\
PRNG state recovery         & No deterministic seed, purely quantum     \\
\bottomrule
\end{tabular}
\label{table:threats}
\end{table}


The evaluation indicates that augmenting AES with quantum-derived keys yields statistically verifiable improvements in randomness and provides concrete\textemdash though incremental\textemdash gains in resistance to quantum-capable adversaries. The real security benefit emerges from combining this higher-entropy key material with best practices such as dynamic rekeying, side-channel hardening, and potential post-quantum key exchange. 


\subsection{Use cases}\label{sec7}

The quantum-enhanced AES framework delivers strategic value across industries by addressing sector-specific vulnerabilities while maintaining operational continuity. In financial services, institutions leverage quantum-derived keys to secure high-frequency trading platforms and cross-border payment networks, mitigating risks from quantum-accelerated attacks on transaction data. This aligns with evolving PCI-DSS quantum-readiness requirements while protecting against future decryption of stored customer financial records. Healthcare systems adopt the framework to safeguard sensitive electronic health records and pharmaceutical research data, where quantum-resistant encryption prevents industrial espionage targeting drug patents and clinical trial results. Government agencies integrate the solution into classified communication networks and critical infrastructure monitoring systems, countering state-sponsored quantum cryptanalysis capabilities that threaten national security assets.

Cloud providers differentiate offerings through quantum-augmented encryption-as-a-service, enabling enterprises to meet GDPR mandates for post-quantum data protection in multi-tenant environments. Industrial IoT operators in energy and manufacturing sectors deploy the framework to harden SCADA systems against ransomware attacks exploiting quantum-weakened encryption in legacy devices. Telecommunications companies implement quantum-secure VPNs for 5G network backhaul, combining lattice-based key exchange with dynamically refreshed AES session keys to prevent quantum-enabled eavesdropping. Automotive manufacturers embed the technology in connected vehicles to establish tamper-proof firmware updates, addressing growing concerns about quantum-capable attacks on vehicular communication buses. These applications demonstrate how the framework transforms quantum security from technical compliance into competitive advantage, enabling organizations to future-proof operations while maintaining backward compatibility with existing infrastructure. Early adopters gain market leadership through enhanced customer trust, reduced liability from quantum-decryptable data breaches, and streamlined migration to post-quantum standards.

\subsection{Quantum technologies integration}

Organizations can bolster their security framework by integrating other quantum-based techniques for key exchange and secure communication. By fusing PQC protocols with AES keys generated from quantum randomness, an organization not only secures the symmetric key generation process but also safeguards the asymmetric components involved in key establishment protocols against quantum threats. This multi-layered defense strategy significantly complicates an adversary’s efforts to access or manipulate sensitive key material, as every stage of the key lifecycle—from creation to exchange—is thoroughly protected against potential future cryptanalytic advances.

In addition to PQC, \emph{quantum key distribution} (QKD) introduces another formidable security layer by ensuring that any attempt to intercept key material is immediately detectable. While attackers in classical channels might monitor communications undetected, QKD exploits quantum mechanics to expose tampering through observable disturbances in photon states. Once a QKD link is established, the communicating parties can generate a symmetric key that is, in theory, immune to eavesdropping, since any interference irreversibly alters the quantum states involved. This QKD-derived key can then be merged with the high-entropy quantum bits employed by AES, creating a “quantum-protected” data channel that leverages both the tamper-detection capabilities of QKD and the post-quantum robustness provided by PQC-based exchanges (or classical channels) that support routine operations.


\section{Conclusion and future work}\label{sec8}

This work has shown that integrating quantum random number generators with AES offers a viable path toward stronger resistance against quantum-enabled adversaries. By injecting high-entropy quantum randomness into both initial key generation and intermediate steps of the key schedule, the proposed design mitigates the classical predictability that undermines many existing encryption solutions. Benchmarks and statistical analyses of the quantum-derived randomness confirm its suitability for cryptographic operations, while side-channel evaluations suggest meaningful reductions in the leakage that typically arises from deterministic key expansions. The overall system maintains backward compatibility with current infrastructures and can be introduced in stages, making it immediately relevant for securing critical data in finance, healthcare, government, and cloud environments. Although large-scale quantum computers remain in development, the demonstrated approach helps organizations prepare for the practical reality of Grover’s algorithm and the harvest-now-decrypt-later strategies that threaten the long-term confidentiality of valuable datasets.

Despite these advancements, several limitations and open challenges merit further investigation. The requirement for specialized QRNG hardware can pose scalability and cost concerns for some organizations, and the uncertain timeline for mature quantum computing leaves ambiguity regarding when these enhanced defenses will become indispensable. Moreover, while our evaluation includes preliminary side-channel leakage analyses, more rigorous testing is needed to establish comprehensive resilience against hardware-based or quantum-assisted attacks. Future work could explore cost-effective QRNG alternatives, develop refined integration strategies for environments with constrained computing resources, and extend research into dynamic side-channel countermeasures. Incorporating quantum key distribution may also provide an additional layer of assurance for ultra-sensitive domains. Ultimately, continued collaboration with hardware manufacturers, standardization bodies, and the broader post-quantum research community will be vital to ensuring that quantum-enabled AES deployments remain robust, efficient, and adaptable as the technology matures.

Future research can extend these findings by pursuing tighter integrations with lattice-based post-quantum cryptography, where quantum-derived keys support both symmetric and asymmetric primitives. Experimental testing can expand to more sophisticated hardware implementations, exploring real-time constraints under high-throughput conditions and applying the approach to edge devices with limited computational capabilities. In combination with the aforementioned side-channel enhancements and continuous leakage monitoring, these steps would enrich the end-to-end encryption model—offering a complementary blend of quantum-safe key establishment and quantum-augmented symmetric protection—ultimately paving the way for a cryptographic ecosystem robust enough to withstand the advent of large-scale quantum computation.


\bibliography{sn-bibliography}% common bib file
\newpage
\begin{appendices}

\section*{Security proofs}\label{app:proofs}
We consider a QIND-CPA game (Quantum Indistinguishability under Chosen Plaintext Attack) in which an adversary \(A\) interacts with an encryption oracle \(E_k\). The oracle implements an AES-based scheme enhanced with quantum randomness for key generation and subkey derivation (QE-AES). Let
\begin{equation}
  \mathrm{Adv}_{\mathrm{QE\text{-}AES}}^{\mathrm{QIND\text{-}CPA}}(A)
\end{equation}
denote \(A\)'s advantage in distinguishing encryptions of chosen plaintexts under QE-AES. The following arguments outline why this advantage remains negligible under standard assumptions about AES and the unpredictability of the quantum entropy source.

\subsection*{Entropy and key space}

Let \(K \in \{0,1\}^{n}\) be the QE-AES key for \(n=256\). We assume each bit of \(K\) enjoys min-entropy \(H_{\infty} \approx 1\). Formally, if \(p_{\mathrm{max}}\) is the maximum probability of any key value,
\begin{equation}
  H_{\infty}(K) \;=\; -\log_2 \bigl( p_{\mathrm{max}} \bigr),
\end{equation}
and for near-perfect quantum randomness, \(p_{\mathrm{max}}\) is close to \(2^{-n}\). Hence the effective key space approaches \(2^n\). Minor biases reduce the exponent slightly, but the distribution still remains extremely hard to guess or correlate.

\subsection*{Impact of Grover's algorithm}

Under Grover's algorithm, the complexity of brute-force searching an \(n\)-bit key space drops from \(O(2^n)\) to \(O(2^{n/2})\). For \(n = 256\), this naively implies \(O(2^{128})\) operations. However, adding quantum randomness \emph{throughout} key expansion and subkey derivation can slightly raise the attacker’s effective complexity or require more sophisticated attacks. Let:
\begin{equation}
  \mathcal{O} \;=\; 2^{n/2 + \delta},
\end{equation}
where \(\delta \ge 0\) captures small increases in difficulty (e.g.\ from round-key modifications or side-channel obfuscation). For instance, frequent rekeying (\(\Delta T\)) and per-round quantum “whitening” each add overhead for any adversary attempting to apply Grover-like strategies on every derived subkey.

\subsection*{Hybrid argument and security reduction}

A common reduction transforms any QIND-CPA adversary \(A\) against QE-AES into an IND-CPA adversary \(B\) against classical AES-256. Concretely:

\begin{equation}
  \mathrm{Adv}_{\mathrm{QE\text{-}AES}}^{\mathrm{QIND\text{-}CPA}}(A)
  \;\;\le\;\;
  f(\delta, n)
  \,\times\,
  \mathrm{Adv}_{\mathrm{AES\text{-}256}}^{\mathrm{IND\text{-}CPA}}(B)
  \;+\;
  \epsilon(\lambda),
\end{equation}
where \(f(\delta,n)\) accounts for the relationship between Grover’s quantum complexity and the standard AES-256 key space, and \(\epsilon(\lambda)\) denotes negligible terms that can arise from quantum oracle imperfections. If AES-256 remains secure against classical attacks, the overall advantage of \(A\) cannot exceed negligible bounds.

\subsection*{Dynamic rekeying}

Let \(\Delta T\) be the rekeying interval (either time-based or block-based):
\begin{equation}
  \Delta T \;=\; \min\bigl(T_{\text{block}}, T_{\text{time}}\bigr).
\end{equation}
Each time \(\Delta T\) is reached, the system derives a new quantum-enhanced key. This practice:
\begin{itemize}
  \item Splits data streams into multiple encryption keys, reducing the ``harvest-now-decrypt-later'' window for any single key.
  \item Forces attackers to re-apply Grover-like strategies repeatedly, greatly compounding the overall resources needed to compromise multiple ephemeral keys.
\end{itemize}

\subsection*{Security argument}

In practical terms, an adversary must do one of the following to break QE-AES:

\begin{itemize}
  \item Exploit an unforeseen weakness in AES-256 itself, giving a success probability significantly better than \(2^{-128}\) in the quantum regime.
  \item Compromise the quantum entropy source to predict or bias the generated keys.
\end{itemize}

Even under these conditions, frequent key rotations ensure that no single key protects too much data. Collectively, these measures make it \emph{exceedingly difficult} for a quantum adversary to decrypt all captured ciphertexts, especially if each key is derived with high-entropy quantum bits and used only for a limited interval \(\Delta T\).

\begin{table}[h!]
\caption{High-level comparison of classical AES-256 vs.\ quantum-augmented AES. Values are illustrative rather than experimentally confirmed.}
\centering
\begin{tabular}{lcc}
\toprule
\textbf{Parameter (conceptual)} & \textbf{Classical AES-256} & \textbf{QE-AES} \\
\midrule
Effective key bits (Grover scale) & $\approx 128$ & $\approx 128 + \delta$ \\
Rekeying interval & Infrequent & Frequent, at $\Delta T$ \\
Side-channel resilience & Standard & Enhanced by random subkey infusion \\
\bottomrule
\end{tabular}
\label{table:conceptual-comparison}
\end{table}




\newpage
\section*{Acronyms}
\addcontentsline{toc}{section}{List of acronyms}%
\label{sec:acronyms}

\begin{description}
  \item[\textbf{AES}] Advanced Encryption Standard
  \item[\textbf{API}] Application Programming Interface
  \item[\textbf{ARX}] Add-Rotate-Xor
  \item[\textbf{CCA}] Chosen Ciphertext Attack
  \item[\textbf{CPA}] Chosen Plaintext Attack
  \item[\textbf{DES}] Data Encryption Standard
  \item[\textbf{ECDH}] Elliptic Curve Diffie--Hellman
  \item[\textbf{ECC}] Elliptic Curve Cryptography
  \item[\textbf{FIPS}] Federal Information Processing Standards
  \item[\textbf{GCM}] Galois/Counter Mode
  \item[\textbf{GDPR}] General Data Protection Regulation
  \item[\textbf{HKDF}] HMAC-based Key Derivation Function
  \item[\textbf{HNDL}] Harvest-Now Decrypt-Later
  \item[\textbf{HSM}] Hardware Security Module
  \item[\textbf{IoT}] Internet of Things
  \item[\textbf{IND-CCA}] Indistinguishability under Chosen Ciphertext Attack
  \item[\textbf{IND-CPA}] Indistinguishability under Chosen Plaintext Attack
  \item[\textbf{KEM}] Key Encapsulation Mechanism
  \item[\textbf{KMS}] Key Management Service
  \item[\textbf{LWE}] Learning With Errors
  \item[\textbf{MI}] Mutual Information
  \item[\textbf{MLWE}] Module Learning With Errors
  \item[\textbf{NISQ}] Noisy Intermediate-Scale Quantum
  \item[\textbf{NIST}] National Institute of Standards and Technology
  \item[\textbf{PCI-DSS}] Payment Card Industry Data Security Standard
  \item[\textbf{PQC}] Post-Quantum Cryptography
  \item[\textbf{PRNG}] Pseudorandom Number Generator
  \item[\textbf{QRNG}] Quantum Random Number Generator
  \item[\textbf{QIND-CCA}] Quantum Indistinguishability under Chosen Ciphertext Attack
  \item[\textbf{QIND-CPA}] Quantum Indistinguishability under Chosen Plaintext Attack
  \item[\textbf{RSA}] Rivest--Shamir--Adleman
  \item[\textbf{SCADA}] Supervisory Control and Data Acquisition
  \item[\textbf{SHA}] Secure Hash Algorithm
  \item[\textbf{SHA3}] Secure Hash Algorithm 3
  \item[\textbf{SHAKE}] Secure Hash Algorithm Keccak-based
  \item[\textbf{SPN}] Substitution--Permutation Network
  \item[\textbf{TLS}] Transport Layer Security
  \item[\textbf{USB}] Universal Serial Bus
  \item[\textbf{VPN}] Virtual Private Network
\end{description}

\end{appendices}
\end{document}
