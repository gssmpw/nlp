\section{Literature Review}
\label{sec2}

This literature review provides an in-depth analysis of various aspects related to classical AES, cryptanalytic resistance of AES over cyber-attacks, the importance of key randomness in AES security, and recent advancements in enhancing AES key strength. Through a comprehensive exploration of the existing research, methodologies, and advancements, this review aims to contribute to the understanding of these critical components of cryptography and their impact on secure communication \cite{dong2021automatic}\cite{stefanov2000optical}. 


In a work by Jang et. al., the authors have undertaken the objective of evaluating the security of secret-key ciphers against potential quantum adversaries, considering the significant advancements anticipated in quantum computing. Although a fully functional quantum computer remains a future prospect, there is a growing need to assess the vulnerability of secret-key ciphers to quantum attacks. In this research, the authors explore the key recovery attack utilizing Grover's search algorithm on the three variants of AES (-128, -192, -256) in the context of quantum implementation and quantum key search using Grover's algorithm \cite{wang2022quantumsafe}. A pool of implementations is developed by reducing the circuit depth metrics, primarily focusing on optimization strategies, and incorporating state-of-the-art advancements in relevant fields \cite{ma2016quantum}\cite{rarity1994quantum}. 


The authors present the least Toffoli depth and full-depth implementations of AES, surpassing the work done by Zou et al. in their Asiacrypt'20 paper by over 98 percent for all variants of AES \cite{zou2020quantum}. The improvement in the product of qubit count and Toffoli depth is more than 75 percent compared to the results presented by Zou et al. Furthermore, the authors conduct a detailed analysis of the implementations proposed by Jaques et al. in their Eurocrypt'20 paper, addressing identified bugs and providing corrected benchmarks. Based on their findings, the authors conclude that their work demonstrates improvements in Toffoli/full depth and Toffoli depth - qubit count product compared to all previous research, including the recent Eprint'22 paper by Huang and Sun \cite{huang2022synthesizing}\cite{wang2022quantumcircuit}. 


Addressing the widespread use of the Advanced Encryption Standard (AES) as a block cipher, Ze. Wang et. al. ideated a design of quantum circuits for deciphering AES-128 and the sample-AES (S-AES). In the quantum circuit of AES-128, an affine transformation is performed for the SubBytes part to resolve the issue of the initial state of the output qubits in SubBytes not being the $|0\rangle \otimes 8$ state. This improvement allows for the encoding of the new round sub-key on the qubits that encode the previous round sub-key \cite{wang2022quantumcircuit}. As a result, the number of utilized qubits is reduced by 224 compared to the implementation by Langenberg et al. For S-AES, the authors present a complete quantum circuit requiring only 48 qubits. This qubit count is within the achievable range of existing noisy intermediate-scale quantum computers \cite{zou2020quantum}\cite{rao2017aes}. 


To summarize, the authors' work focuses on designing quantum circuits for deciphering AES-128 and S-AES. In the AES-128 quantum circuit, an affine transformation is applied to resolve the issue of the initial state of the output qubits in SubBytes. This enables the encoding of the new round sub-key on the qubits encoding the previous round sub-key, resulting in a significant reduction in the number of utilized qubits compared to previous implementations. Furthermore, the authors demonstrate the feasibility of implementing S-AES by presenting a complete quantum circuit that utilizes only 48 qubits, a count well within the capabilities of existing noisy intermediate-scale quantum computers \cite{langenberg2020reducing}. 


A study by S. Rao et. al. highlights the importance of cryptographic techniques in maintaining information privacy. The authors emphasize the reliability of AES as a trusted cryptographic algorithm in symmetric key cryptography \cite{rao2017aes}. They acknowledge the potential threats posed by advancements in quantum computing and aim to address these concerns by demonstrating that AES-256, when utilized with Grover's algorithm, can withstand quantum attacks \cite{bonnetain2019quantum}\cite{vanHeesch2019quantum}. 
 

Scientists Chang et. al. investigated the security of AES-OTR, a third-round algorithm in the CAESAR competition, under the Q2 model. The authors construct periodic functions of the associated data by manipulating the associated data based on the parallel and serial structure characteristics of the AES-OTR algorithm during associated data processing \cite{chang2022collision}. Utilizing the Simon quantum algorithm, they repeatedly construct the periodic functions of the associated data. Using collision pairs, two successful collision forgery attacks on the AES-OTR algorithm are executed, resulting in obtaining the period s with a high probability \cite{almazrooie2018quantum}. 
 

The attacks presented in this study pose a significant threat to the security of the AES-OTR algorithm. By exploiting the relative independence between different modules of the AES-OTR authentication encryption algorithm, as well as the parallel and serial structures in associated data processing and the simplicity of intermediate variable generation, the authors propose a collision forgery attack on the AES-OTR algorithm under quantum computing \cite{chang2022collision}. They demonstrate that the AES-OTR algorithm is highly vulnerable when associated data and Nonce are reused. By analysing the collision form of the associated data and combining it with the Simon quantum algorithm, they construct the periodic function f of the associated data and solve the collision period s with a high success rate, resulting in generating the same authentication Tag when employing fake associated data for processing. This enables the achievement of monitoring or tampering effects without detection by a third party, demonstrating feasibility and efficiency. 


The authors emphasize the need for addressing the cryptographic crisis in the post-quantum era and highlight the current focus on cryptographic structure and working modes under the quantum security model in anti-quantum cryptography design and research \cite{chen2016postquantum}\cite{bernstein2009introduction}. They stress the importance of conducting quantum security analysis of instantiated cryptographic algorithms. This research on collision forgery attacks under quantum computing of AES-OTR contributes to the security analysis of specific authentication encryption algorithms, offering insights for enhancing the design scheme of related quantum-resistant cryptographic algorithms \cite{bonnetain2019quantum}. The authors plan to develop a polynomial time quantum discriminator with as many rounds as possible and explore the combination of the Simon algorithm and the Grover algorithm to enhance the speed of the exhaustive search key for quantum key recovery attacks \cite{raghu2015application}\cite{chang2022collision}. 
 

In a research work by M. Almazrooie et al., authors present an explicit quantum design of AES-128 that utilizes the minimum number of qubits \cite{almazrooie2018quantum}. The authors structure the design by constructing quantum circuits for the main components of AES-128 and then combining them to form the quantum version of AES-128. To ensure efficiency, they adopt some of the most efficient approaches used in classical hardware implementations for constructing the circuits of the multiplier and multiplicative inverse in F2[x]./(x8+x4+x3+x+1) \cite{almazrooie2018grover}. The results demonstrate that a quantum circuit for AES-128 can be implemented using 928 qubits. This efficient implementation showcases the feasibility of utilizing quantum technology for AES-128. Additionally, to maintain key uniqueness when employing quantum AES-128 as a Boolean function within a Black-box in other key searching quantum algorithms, the authors propose a method that requires 930 qubits \cite{sharma2023post}\cite{yan2016improved}.
 

In summary, the authors present an explicit quantum design of AES-128, focusing on minimizing the required number of qubits. By constructing quantum circuits for the main components of AES-128 and adopting efficient approaches for the multiplier and multiplicative inverse, they achieve an implementation that utilizes only 928 qubits. Furthermore, they propose a method to preserve key uniqueness in other key searching quantum algorithms, which requires 930 qubits. These findings contribute to the advancement of quantum implementations of AES-128 and offer insights into maintaining key uniqueness in quantum algorithms utilizing AES-128 as a Boolean function. 


In a research work by Z. Huang et al., several generic synthesis and optimization techniques for circuits implementing the quantum oracles of iterative symmetric-key ciphers used in quantum attacks are presented. The objective is to improve the efficiency and performance of these circuits, specifically in relation to Grover and Simon's algorithms \cite{huang2022synthesizing}. 
 

Initially, a general structure for implementing the round functions of block ciphers in-place is proposed. Additionally, novel techniques for synthesizing efficient quantum circuits of both linear and non-linear cryptographic building blocks are introduced. These techniques are applied to the Advanced Encryption Standard (AES), and a thorough investigation of depth-width trade-offs is conducted \cite{strand2021status}. 
 

Throughout the study, a quantum circuit for the AES S-box is derived, guaranteeing minimal T-depth based on new observations made on its classical circuit. This leads to a significant reduction in the required T-depth and width (number of qubits) for implementing the quantum circuits of AES. In comparison to the circuit proposed in EUROCRYPT 2020, the T-depth is reduced from 60 to 40 without increasing the width or 30 with a slight increase in width \cite{jaques2020grover}. Moreover, when compared to the circuit proposed in ASIACRYPT 2020, one of the authors' circuits achieves a reduction in width from 512 to 371, while simultaneously reducing the Toffoli-depth from 2016 to 1558. It is worth noting that further reductions in width to 270 are possible, albeit with an increased depth. The authors provide a comprehensive range of depth-width trade-offs, setting new records for the synthesis and optimization of quantum circuits of AES \cite{huang2022synthesizing}. 
 

In summary, the authors' work focuses on improving the efficiency of circuits implementing quantum oracles for symmetric-key ciphers used in quantum attacks. By proposing novel synthesis and optimization techniques and conducting a thorough exploration of depth-width trade-offs, they achieve significant reductions in T-depth and width for the AES circuits. These findings contribute to the advancement of quantum circuit synthesis and optimization, particularly for AES, and the provided source code enables further exploration and application of these circuits \cite{sharma2023post}\cite{dunne2023analysis}. 
 

A study by J Zou et al. demonstrate quantum circuits for the AES S-box and S-box-1 that require fewer qubits compared to previous work. This improvement enhances the efficiency of the AES quantum circuits. Secondly, the authors introduce the S-box-1 operation into their quantum circuits of AES, resulting in a reduction in the number of qubits in the zig-zag method. This modification contributes to further qubit savings and optimizes the quantum circuit implementation. 
 

Thirdly, a method is presented to reduce the number of qubits in the key schedule of AES. By implementing this method, the authors achieve substantial reductions in the number of qubits required for the quantum circuits of AES-128, AES-192, and AES-256. Specifically, the previous quantum circuits necessitated at least 864, 896, and 1232 qubits for AES-128, AES-192, and AES-256, respectively. In contrast, the authors' quantum circuit implementations of AES-128, AES-192, and AES-256 only require 512, 640, and 768 qubits, respectively. This signifies a reduction in the number of qubits by more than 30\% \cite{yan2016improved}. 


In summary, the authors' work focuses on improving quantum circuit implementations of AES through several advancements. Their proposed quantum circuits for the AES S-box and S-box-1 require fewer qubits, while introducing the S-box-1 operation optimizes the zig-zag method. Furthermore, their method to reduce the number of qubits in the key schedule significantly reduces the overall qubit requirements for AES-128, AES-192, and AES-256. These findings contribute to the efficiency and optimization of AES quantum circuit implementations, showcasing reductions in the required number of qubits by more than 30\% compared to previous approaches. 

 
In a research work by X. Bonnetain and team, the authors undertake a novel analysis to assess the post-quantum security of AES for the first time. Their objective is to determine the new security margin by identifying the minimum number of non-attacked rounds within a time frame of less than 2128 encryptions \cite{chen2016postquantum}\cite{mohammad2015innovative}. To achieve this, the authors provide generalized and quantized versions of the most effective known cryptanalysis techniques on reduced-round AES, along with a discussion on attacks that do not appear to benefit significantly from a quantum speed-up. To facilitate efficient computation of the attacks' complexity, the authors propose a new framework for structured search that encompasses both classical and quantum attacks. They anticipate that this framework will prove valuable for future analyses in the field. 


The authors present their best attack, which is a quantum Demirci-Selçuk meet-in-the-middle attack. Surprisingly, the design principles underlying this attack also led to new and counter-intuitive classical trade-offs known as Time-Memory-Data (TMD) trade-offs. These trade-offs enable the reduction of memory requirements in certain attacks against AES-256 and AES-128, which is an unexpected finding \cite{sharma2023post}.