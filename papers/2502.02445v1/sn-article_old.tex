%%Note: the following reference styles support Namedate and Numbered referencing. By default the style follows the most common style. To switch between the options you can add or remove “Numbered” in the optional parenthesis. 
%%The option is available for: sn-basic.bst, sn-vancouver.bst, sn-chicago.bst%  
 
%%\documentclass[pdflatex,sn-nature]{sn-jnl}% Style for submissions to Nature Portfolio journals
%%\documentclass[pdflatex,sn-basic]{sn-jnl}% Basic Springer Nature Reference Style/Chemistry Reference Style
\documentclass[pdflatex,sn-mathphys-num]{sn-jnl}% Math and Physical Sciences Numbered Reference Style 
%%\documentclass[pdflatex,sn-mathphys-ay]{sn-jnl}% Math and Physical Sciences Author Year Reference Style
%%\documentclass[pdflatex,sn-aps]{sn-jnl}% American Physical Society (APS) Reference Style
%%\documentclass[pdflatex,sn-vancouver,Numbered]{sn-jnl}% Vancouver Reference Style
%%\documentclass[pdflatex,sn-apa]{sn-jnl}% APA Reference Style 
%%\documentclass[pdflatex,sn-chicago]{sn-jnl}% Chicago-based Humanities Reference Style

%%%% Standard Packages
%%<additional latex packages if required can be included here>

\usepackage{graphicx}%
\usepackage{multirow}%
\usepackage{amsmath,amssymb,amsfonts}%
\usepackage{amsthm}%
\usepackage{mathrsfs}%
\usepackage[title]{appendix}%
\usepackage{xcolor}%
\usepackage{textcomp}%
\usepackage{manyfoot}%
\usepackage{booktabs}%
\usepackage{algorithm}%
\usepackage{algorithmicx}%
\usepackage{algpseudocode}%
\usepackage{listings}%
%%%%
%% as per the requirement new theorem styles can be included as shown below
\theoremstyle{thmstyleone}%
\newtheorem{theorem}{Theorem}%  meant for continuous numbers
%%\newtheorem{theorem}{Theorem}[section]% meant for sectionwise numbers
%% optional argument [theorem] produces theorem numbering sequence instead of independent numbers for Proposition
\newtheorem{proposition}[theorem]{Proposition}% 
%%\newtheorem{proposition}{Proposition}% to get separate numbers for theorem and proposition etc.

\theoremstyle{thmstyletwo}%
\newtheorem{example}{Example}%
\newtheorem{remark}{Remark}%

\theoremstyle{thmstylethree}%
\newtheorem{definition}{Definition}%

\raggedbottom

\begin{document}

\title[Quantum-enabled Framework for Symmetric Key Cryptography]{Quantum-enabled Framework for Symmetric Key Cryptography}

\author*[1]{\fnm{Arit Kumar} \sur{Bishwas}}\email{arit.kumar.bishwas@pwc.com}

\author[1]{\fnm{Nivedita} \sur{Dey}}\email{nivedita.dey@pwc.com}

\author[1]{\fnm{Albert} \sur{Nieto Morales}}\email{albert.morales@pwc.com}

\affil*[1]{\orgdiv{Innovation Hub}, \orgname{PricewaterhouseCoopers}, \orgaddress{\country{USA}}}


%%==================================%%
%% Sample for unstructured abstract %%
%%==================================%%

\abstract{This research article explores the interrelationship among classical communication, symmetric key cryptographic algorithms such as Advanced Encryption Standard (AES), and the theoretical threat posed by quantum computers in terms of cryptanalytic attacks. The paper begins by elucidating the fundamental principles of classical communication and its relevance to secure data transmission. The article examines the potential impact of quantum computing on current cryptographic algorithms, emphasizing the need for future-proof encryption methods. A framework for enhancing the randomness of AES keys using quantum-based random number generation is proposed in this article which will strengthen the security of AES encryption as well as mitigate the risk of cryptographic attacks. Finally, the article delves into the specific applications of AES in various industries, highlighting its role in security-sensitive domains. Through a comprehensive analysis of these interconnected concepts, this article aims to provide valuable insights into the evolving landscape of secure communication.}


\keywords{Quantum Random Number Generator, Advanced Encryption Standard, Post-Quantum Cryptography, Quantum-Safe Encryption, Quantum-enabled AES}


\maketitle

\section{Introduction}\label{sec1}

Classical communication forms the foundation of secure data transmission, enabling information exchange across various channels. The deployment of symmetric key cryptographic algorithms, such as AES, ensures confidentiality, integrity, and authentication of data in classical communication systems \cite{sharma2016novel}. AES, recognized as a leading encryption standard, finds extensive use in diverse industries, including finance, healthcare, defense, and telecommunications. Its widespread adoption stems from its robustness, efficiency, and ability to protect sensitive information from unauthorized access. This paper examines AES as a prominent example of a symmetric key cryptographic algorithm and its critical role in securing communication networks \cite{chen2015quantum}\cite{dong2021automatic}. 

Advanced Encryption Standard (AES) is one of the most popular encryption algorithms which is extremely fast with less computational overhead and highly secure form of encryption that is a favorite of businesses and governments worldwide \cite{raghu2015application}\cite{vaudenay2005classical}. The versatility of AES makes it indispensable in numerous industries that handle security-sensitive information. In the finance sector, AES is utilized to secure online banking transactions, protect digital assets, and safeguard sensitive customer data. Likewise, the healthcare industry relies on AES to ensure the privacy and integrity of electronic health records, enabling secure sharing of patient information among healthcare providers. Furthermore, AES is integral to secure communication in defense applications, safeguarding military command and control systems, classified information, and strategic communications. The telecommunications sector benefits from AES by securing voice and data transmissions, securing mobile communication protocols, and ensuring the confidentiality of sensitive customer information. The National Security Agency (NSA) as well as other governmental bodies utilize AES encryption and keys to protect classified or other sensitive information. Apart from defense and finance, AES is often included in commercial based products. In Wi-Fi, AES can be used as part of WPA2 (Wi-Fi Protected Access 2). Mobile apps, native processor support, software development libraries, VPN (Virtual Private Network) implementations and file systems in operating system have noticeably incorporated AES into their technology. These use cases demonstrate the wide-ranging applicability and significance of AES in today's security landscape \cite{bonnetain2019quantum}\cite{jang2020grover}. 


While classical cryptographic algorithms like AES provide a robust defense against traditional computing resources, the rise of quantum computing poses a potential threat to their security. Quantum computers harness the power of quantum phenomena to perform calculations at an exponentially faster rate compared to classical computers. This increased computational power can potentially compromise the effectiveness of current cryptographic algorithms. As quantum computing advances, it is crucial to explore post-quantum cryptographic algorithms that can resist quantum-based attacks, ensuring long-term security for classical communication systems. Quantum-safe cryptography products and solutions assure unconditional security of critical data available on the internet and cloud across all industry verticals \cite{stebila2016post}\cite{joseph2022transitioning}. Breaking an AES-256 standard which uses 256-bits of key would require 2256 possible combinations to decipher, which is so computation-intensive that a computer able to perform quadrillions of instructions per second cannot make it in real-time \cite{wang2022quantumsafe}\cite{basu2019nist}. 


In order to quantify the security of cryptosystem, “bits of security” is used which can be thought of as a function of number of steps needed to crack a system. Algorithms approved by NIST (National Institute of Standards and Technology) use 112-bits of security which means 2112 steps are needed to crack the algorithm. The security of encryption depends on the length of the key and the cryptosystem used \cite{han2014improving}. But, with the advent of quantum computing, AES algorithm standards are already under threat. There is a famous quantum algorithm known as Grover’s algorithm which is able to perform any unstructured database search operation in quadratically faster than its classical counterpart, which implies that Grover’s algorithm can quadratically reduce the security of symmetric key cryptosystem like AES. AES-128 on a classical computer provides 128-bits of security, whereas on a quantum computer, it will provide only 64-bits of security \cite{dong2021automatic}. So, AES-128 is definitely not quantum safe as only a 64-qubit fault-tolerant quantum hardware can break the cipher. But, AES-256 offers 128-bits of security on a quantum computer, which is not theoretically but acceptable bits of security keeping in mind the current and near-term quantum era. Thus, doubling the key can ensure better security in classical AES. This key expansion is done classically with the help of strong mathematical operations and thus opening up vulnerability for future quantum attacks when full-scale quantum computers will be available. Key generation from a true source of randomness like a ‘quantum random number generator’ and a ‘quantum key expansion’ would add more power to AES cryptosystem \cite{yan2019quantum}. Bits generated from a quantum random number generator offers more extractable entropy or randomness than any pseudo random number generator. Thus, it will ensure stricter security against any future quantum attacks \cite{vaudenay2005classical}\cite{ma2016quantum}. 


In this article, we have proposed a quantum enabled AES solution which is able to provide concrete improvements over classical AES in terms of (i) better source of entropy, (ii) enhanced key strength, (iii) quantum safe solution, and (iv) resistance to cryptanalytic attacks.  


The paper is organized as follows. Section~\ref{sec1} and Section~\ref{sec2} cover introduction and literature review respectively. Section~\ref{sec3} provides a comprehensive overview of classical AES. It discusses the key principles, algorithm structure, and encryption/ decryption processes involved in classical AES. Section~\ref{sec4} delves into quantum threats associated to classical AES and possible cryptanalytic attacks. It explores the vulnerabilities arising from advancements in quantum computing and analyses different attack scenarios like linear cryptanalytic attack, differential cryptanalytic attack, brute force attack, and side channel attack. Section~\ref{sec5} focuses on the generation of random numbers, types, and evolution of random number generators (RNGs) commonly used in cryptographic applications. It compares and analyses the strengths and weaknesses of different RNGs. Section~\ref{sec6} explores quantum random number generation techniques, highlighting their unique properties and advantages over classical RNGs. It discusses how quantum RNGs (QRNGs) can enhance the security of cryptographic systems. The proposed solution architecture is detailed in Section~\ref{sec7}. It outlines how quantum-enabled AES can be implemented to address the future quantum threats faced by classical AES \cite{jang2022aes}. The design and integration of quantum random number generation into the AES framework are discussed. Section~\ref{sec8} presents the results of statistical tests conducted to evaluate the performance and reliability of the quantum bits used as key in proposed solution \cite{herrero2017quantum}. The specifications and metrics used to assess the effectiveness of the quantum-enabled AES are also discussed. Section 9 explores potential applications and use cases where quantum-enabled AES can provide enhanced security. It discusses scenarios where the integration of quantum technologies can strengthen encryption practices. The article concludes by summarizing the key findings and contributions of the research as mentioned in section 10. It highlights the significance of quantum-enabled AES in countering quantum threats and provides insights into future research directions for further research and development in this field. 


\section{Literature Review}\label{sec2}

This literature review provides an in-depth analysis of various aspects related to classical AES, cryptanalytic resistance of AES over cyber-attacks, the importance of key randomness in AES security, and recent advancements in enhancing AES key strength. Through a comprehensive exploration of the existing research, methodologies, and advancements, this review aims to contribute to the understanding of these critical components of cryptography and their impact on secure communication \cite{dong2021automatic}\cite{stefanov2000optical}. 


In a work by Jang et. al., the authors have undertaken the objective of evaluating the security of secret-key ciphers against potential quantum adversaries, considering the significant advancements anticipated in quantum computing. Although a fully functional quantum computer remains a future prospect, there is a growing need to assess the vulnerability of secret-key ciphers to quantum attacks. In this research, the authors explore the key recovery attack utilizing Grover's search algorithm on the three variants of AES (-128, -192, -256) in the context of quantum implementation and quantum key search using Grover's algorithm \cite{wang2022quantumsafe}. A pool of implementations is developed by reducing the circuit depth metrics, primarily focusing on optimization strategies, and incorporating state-of-the-art advancements in relevant fields \cite{ma2016quantum}\cite{rarity1994quantum}. 


The authors present the least Toffoli depth and full-depth implementations of AES, surpassing the work done by Zou et al. in their Asiacrypt'20 paper by over 98 percent for all variants of AES \cite{zou2020quantum}. The improvement in the product of qubit count and Toffoli depth is more than 75 percent compared to the results presented by Zou et al. Furthermore, the authors conduct a detailed analysis of the implementations proposed by Jaques et al. in their Eurocrypt'20 paper, addressing identified bugs and providing corrected benchmarks. Based on their findings, the authors conclude that their work demonstrates improvements in Toffoli/full depth and Toffoli depth - qubit count product compared to all previous research, including the recent Eprint'22 paper by Huang and Sun \cite{huang2022synthesizing}\cite{wang2022quantumcircuit}. 


Addressing the widespread use of the Advanced Encryption Standard (AES) as a block cipher, Ze. Wang et. al. ideated a design of quantum circuits for deciphering AES-128 and the sample-AES (S-AES). In the quantum circuit of AES-128, an affine transformation is performed for the SubBytes part to resolve the issue of the initial state of the output qubits in SubBytes not being the $|0\rangle \otimes 8$ state. This improvement allows for the encoding of the new round sub-key on the qubits that encode the previous round sub-key \cite{wang2022quantumcircuit}. As a result, the number of utilized qubits is reduced by 224 compared to the implementation by Langenberg et al. For S-AES, the authors present a complete quantum circuit requiring only 48 qubits. This qubit count is within the achievable range of existing noisy intermediate-scale quantum computers \cite{zou2020quantum}\cite{rao2017aes}. 


To summarize, the authors' work focuses on designing quantum circuits for deciphering AES-128 and S-AES. In the AES-128 quantum circuit, an affine transformation is applied to resolve the issue of the initial state of the output qubits in SubBytes. This enables the encoding of the new round sub-key on the qubits encoding the previous round sub-key, resulting in a significant reduction in the number of utilized qubits compared to previous implementations. Furthermore, the authors demonstrate the feasibility of implementing S-AES by presenting a complete quantum circuit that utilizes only 48 qubits, a count well within the capabilities of existing noisy intermediate-scale quantum computers \cite{langenberg2020reducing}. 


A study by S. Rao et. al. highlights the importance of cryptographic techniques in maintaining information privacy. The authors emphasize the reliability of AES as a trusted cryptographic algorithm in symmetric key cryptography \cite{rao2017aes}. They acknowledge the potential threats posed by advancements in quantum computing and aim to address these concerns by demonstrating that AES-256, when utilized with Grover's algorithm, can withstand quantum attacks \cite{bonnetain2019quantum}\cite{vanHeesch2019quantum}. 
 

Scientists Chang et. al. investigated the security of AES-OTR, a third-round algorithm in the CAESAR competition, under the Q2 model. The authors construct periodic functions of the associated data by manipulating the associated data based on the parallel and serial structure characteristics of the AES-OTR algorithm during associated data processing \cite{chang2022collision}. Utilizing the Simon quantum algorithm, they repeatedly construct the periodic functions of the associated data. Using collision pairs, two successful collision forgery attacks on the AES-OTR algorithm are executed, resulting in obtaining the period s with a high probability \cite{almazrooie2018quantum}. 
 

The attacks presented in this study pose a significant threat to the security of the AES-OTR algorithm. By exploiting the relative independence between different modules of the AES-OTR authentication encryption algorithm, as well as the parallel and serial structures in associated data processing and the simplicity of intermediate variable generation, the authors propose a collision forgery attack on the AES-OTR algorithm under quantum computing \cite{chang2022collision}. They demonstrate that the AES-OTR algorithm is highly vulnerable when associated data and Nonce are reused. By analysing the collision form of the associated data and combining it with the Simon quantum algorithm, they construct the periodic function f of the associated data and solve the collision period s with a high success rate, resulting in generating the same authentication Tag when employing fake associated data for processing. This enables the achievement of monitoring or tampering effects without detection by a third party, demonstrating feasibility and efficiency. 


The authors emphasize the need for addressing the cryptographic crisis in the post-quantum era and highlight the current focus on cryptographic structure and working modes under the quantum security model in anti-quantum cryptography design and research \cite{chen2016postquantum}\cite{bernstein2009introduction}. They stress the importance of conducting quantum security analysis of instantiated cryptographic algorithms. This research on collision forgery attacks under quantum computing of AES-OTR contributes to the security analysis of specific authentication encryption algorithms, offering insights for enhancing the design scheme of related quantum-resistant cryptographic algorithms \cite{bonnetain2019quantum}. The authors plan to develop a polynomial time quantum discriminator with as many rounds as possible and explore the combination of the Simon algorithm and the Grover algorithm to enhance the speed of the exhaustive search key for quantum key recovery attacks \cite{raghu2015application}\cite{chang2022collision}. 
 

In a research work by M. Almazrooie et al., authors present an explicit quantum design of AES-128 that utilizes the minimum number of qubits \cite{almazrooie2018quantum}. The authors structure the design by constructing quantum circuits for the main components of AES-128 and then combining them to form the quantum version of AES-128. To ensure efficiency, they adopt some of the most efficient approaches used in classical hardware implementations for constructing the circuits of the multiplier and multiplicative inverse in F2[x]./(x8+x4+x3+x+1) \cite{almazrooie2018grover}. The results demonstrate that a quantum circuit for AES-128 can be implemented using 928 qubits. This efficient implementation showcases the feasibility of utilizing quantum technology for AES-128. Additionally, to maintain key uniqueness when employing quantum AES-128 as a Boolean function within a Black-box in other key searching quantum algorithms, the authors propose a method that requires 930 qubits \cite{sharma2023post}\cite{yan2016improved}.
 

In summary, the authors present an explicit quantum design of AES-128, focusing on minimizing the required number of qubits. By constructing quantum circuits for the main components of AES-128 and adopting efficient approaches for the multiplier and multiplicative inverse, they achieve an implementation that utilizes only 928 qubits. Furthermore, they propose a method to preserve key uniqueness in other key searching quantum algorithms, which requires 930 qubits. These findings contribute to the advancement of quantum implementations of AES-128 and offer insights into maintaining key uniqueness in quantum algorithms utilizing AES-128 as a Boolean function. 


In a research work by Z. Huang et al., several generic synthesis and optimization techniques for circuits implementing the quantum oracles of iterative symmetric-key ciphers used in quantum attacks are presented. The objective is to improve the efficiency and performance of these circuits, specifically in relation to Grover and Simon's algorithms \cite{huang2022synthesizing}. 
 

Initially, a general structure for implementing the round functions of block ciphers in-place is proposed. Additionally, novel techniques for synthesizing efficient quantum circuits of both linear and non-linear cryptographic building blocks are introduced. These techniques are applied to the Advanced Encryption Standard (AES), and a thorough investigation of depth-width trade-offs is conducted \cite{strand2021status}. 
 

Throughout the study, a quantum circuit for the AES S-box is derived, guaranteeing minimal T-depth based on new observations made on its classical circuit. This leads to a significant reduction in the required T-depth and width (number of qubits) for implementing the quantum circuits of AES. In comparison to the circuit proposed in EUROCRYPT 2020, the T-depth is reduced from 60 to 40 without increasing the width or 30 with a slight increase in width \cite{jaques2020grover}. Moreover, when compared to the circuit proposed in ASIACRYPT 2020, one of the authors' circuits achieves a reduction in width from 512 to 371, while simultaneously reducing the Toffoli-depth from 2016 to 1558. It is worth noting that further reductions in width to 270 are possible, albeit with an increased depth. The authors provide a comprehensive range of depth-width trade-offs, setting new records for the synthesis and optimization of quantum circuits of AES \cite{huang2022synthesizing}. 
 

In summary, the authors' work focuses on improving the efficiency of circuits implementing quantum oracles for symmetric-key ciphers used in quantum attacks. By proposing novel synthesis and optimization techniques and conducting a thorough exploration of depth-width trade-offs, they achieve significant reductions in T-depth and width for the AES circuits. These findings contribute to the advancement of quantum circuit synthesis and optimization, particularly for AES, and the provided source code enables further exploration and application of these circuits \cite{sharma2023post}\cite{dunne2023analysis}. 
 

A study by J Zou et al. demonstrate quantum circuits for the AES S-box and S-box-1 that require fewer qubits compared to previous work. This improvement enhances the efficiency of the AES quantum circuits. Secondly, the authors introduce the S-box-1 operation into their quantum circuits of AES, resulting in a reduction in the number of qubits in the zig-zag method. This modification contributes to further qubit savings and optimizes the quantum circuit implementation. 
 

Thirdly, a method is presented to reduce the number of qubits in the key schedule of AES. By implementing this method, the authors achieve substantial reductions in the number of qubits required for the quantum circuits of AES-128, AES-192, and AES-256. Specifically, the previous quantum circuits necessitated at least 864, 896, and 1232 qubits for AES-128, AES-192, and AES-256, respectively. In contrast, the authors' quantum circuit implementations of AES-128, AES-192, and AES-256 only require 512, 640, and 768 qubits, respectively. This signifies a reduction in the number of qubits by more than 30\% \cite{yan2016improved}. 


In summary, the authors' work focuses on improving quantum circuit implementations of AES through several advancements. Their proposed quantum circuits for the AES S-box and S-box-1 require fewer qubits, while introducing the S-box-1 operation optimizes the zig-zag method. Furthermore, their method to reduce the number of qubits in the key schedule significantly reduces the overall qubit requirements for AES-128, AES-192, and AES-256. These findings contribute to the efficiency and optimization of AES quantum circuit implementations, showcasing reductions in the required number of qubits by more than 30\% compared to previous approaches. 

 
In a research work by X. Bonnetain and team, the authors undertake a novel analysis to assess the post-quantum security of AES for the first time. Their objective is to determine the new security margin by identifying the minimum number of non-attacked rounds within a time frame of less than 2128 encryptions \cite{chen2016postquantum}\cite{mohammad2015innovative}. To achieve this, the authors provide generalized and quantized versions of the most effective known cryptanalysis techniques on reduced-round AES, along with a discussion on attacks that do not appear to benefit significantly from a quantum speed-up. To facilitate efficient computation of the attacks' complexity, the authors propose a new framework for structured search that encompasses both classical and quantum attacks. They anticipate that this framework will prove valuable for future analyses in the field. 


The authors present their best attack, which is a quantum Demirci-Selçuk meet-in-the-middle attack. Surprisingly, the design principles underlying this attack also led to new and counter-intuitive classical trade-offs known as Time-Memory-Data (TMD) trade-offs. These trade-offs enable the reduction of memory requirements in certain attacks against AES-256 and AES-128, which is an unexpected finding \cite{sharma2023post}. 


\section{Overview of AES}\label{sec3}
AES encryption was established by the United States National Institute of Standards and Technology (NIST) in 2001 and it aims to offer a specification for electronic data encryption. AES is characterized as being a symmetric block cipher, in other words it uses the same key for encryption and decryption \cite{mohammad2015innovative}.


AES is an iterative rather than Feistel cipher. It is based on a ‘substitution-permutation network’. It comprises a series of linked operations, some of which involve replacing inputs with specific outputs (substitutions), and others involve shuffling bits around (permutations). AES performs all its computations on bytes rather than bits. However, the number of rounds in AES is variable and depends on the length of the key. Each of these rounds uses a different 128-bit round key, which is calculated from the original AES key as described in Figure~\ref{fig:aes-key-generation}.


Types of AES encryption (based on the number of rounds):
\begin{enumerate}
    \item \textbf{128-bit keys encryption}: 10 rounds
    \item \textbf{192-bit keys encryption}: 12 rounds
    \item \textbf{256-bit keys encryption}: 14 rounds
\end{enumerate}

\begin{figure}[h!]
\centering
\includegraphics[width=\textwidth]{img/structureaes.png}
\caption{AES key generation process.}
\label{fig:aes-key-generation}
\end{figure}


\subsection{Quantum Threats to Classical AES and Possible Cryptanalytic Attacks}\label{subsec31}

Firstly, asymmetric encryption algorithms like RSA and Diffie-Hellman have been mathematically proven to be impacted and be broken by quantum computers using Shor’s algorithm.


However, that is not the case for symmetric encryption algorithm mostly talking about AES (AES-256 to be precise). As unlike asymmetric encryption algorithms, symmetric encryption algorithms do not rely on prime factorization which was once believed to be extremely safe as factorizing is a tedious, time taking and extremely complex problem to be tackled by a classical computer and could take thousands of years or more, but, with the help of quantum computers and quantum computational algorithms like Shor’s algorithm it can speed up the whole process exponentially. It has already been theoretically stated that a fault tolerant quantum computer with 4000 qubits will be able to easily compromise encryption algorithms under a minute which are extremely hard and would take several years for a classical computer. On the contrary, no quantum algorithmic insights have yet been sighted to break a symmetric key cryptographic algorithm with significant speed-up \cite{wang2022quantumsafe}\cite{strand2021status}.


However, the best possible known approach to compromise the AES algorithm is brute forcing which is nothing but searching through the possible key streams, and there exists a very well-known quantum computational algorithm that could speed up the whole process of searching quadratically known as Grover’s algorithm. Grover’s unstructured database search quantum algorithm can reduce the brute force attack time to its square root. So, for AES-128 the attack time becomes reduced to 264 (which is not secure in near term quantum era). While, computational complexity of AES-256 becomes reduced to 2128 which is extremely secured in current quantum scenario. Due to extremely low qubit quality, millions of noisy qubits will be required to break AES-256. So, it can be concluded that “AES-256 is not quantum-breakable” \cite{bonnetain2019quantum}.


But there exists a problem. Keeping an eye on recent quantum hardware highlights, if millions of qubits can be incorporated or qubit fidelity is enhanced then it might create a concern for classical AES-256. Now, the reason behind the fact that it is not completely secured from the hands of the era of full-scale quantum computation is because most of the classical encryption algorithms use random numbers for their key generation. The following table represents different cryptanalytic attacks on classical AES which are compute intensive with current hardware, but a quantum computer and a quantum cryptanalytic algorithm might be able to reduce the cryptanalytic resistance of current AES. All these attack experiments have been performed on single byte key for demonstration purpose, as shown in the following table.

\begin{sidewaystable}
\caption{Cryptanalytic Attack Scenarios on Classical AES}\label{tab3}
\begin{tabular*}{\textheight}{@{\extracolsep\fill}p{4cm}p{6cm}p{8cm}@{}}
\toprule
\textbf{Attack Genre} & \textbf{Description} & \textbf{Attack Experiment (Snippet) and Possible Output Pattern} \\
\midrule
Brute-Force Attack & 
Brute force attack typically involves systematically trying all possible keys until the key is found. The resistance of AES to a brute force attack is based on the large key space of \(2^{256}\), making it computationally infeasible to try all possible keys in a reasonable timeframe. & 
\texttt{def brute\_force\_attack(ciphertext):}\\
\texttt{  for i in range(256):}\\
\texttt{    key = bytes([i]) * 32}\\
\texttt{    try:}\\
\texttt{      cipher = AES.new(key, AES.MODE\_ECB)}\\
\texttt{      decrypted = cipher.decrypt(ciphertext)}\\
\texttt{      print("Key found:", key)}\\
\texttt{      print("Decrypted plaintext:", unpad(decrypted, AES.block\_size).decode())}\\
\texttt{      return}\\
\texttt{    except ValueError:}\\
\texttt{      continue}\\
Ciphertext: \texttt{15fe3059d8d...}\\
Key found: \texttt{b\{\textbackslash x00...\textbackslash x00\}}. \\
\midrule
Linear Cryptanalytic Attack & 
Linear cryptanalysis is a chosen plaintext attack aiming to exploit linear approximations in the behavior of a cipher to deduce information about the encryption key. Linear relationships between plaintext, ciphertext, and key bits uncover patterns and biases leading to key recovery. Information about the secret key can be extracted by analyzing many plaintext-ciphertext pairs. &
\texttt{def linear\_cryptanalysis\_attack(ciphertext, known\_plaintext):}\\
\texttt{  for i in range(256):}\\
\texttt{    key = bytes([i]) * 32}\\
\texttt{    cipher = AES.new(key, AES.MODE\_ECB)}\\
\texttt{    decrypted = cipher.decrypt(ciphertext)}\\
\texttt{    if decrypted == known\_plaintext:}\\
\texttt{      print("Key found:", key.hex())}\\
\texttt{      print("Decrypted plaintext:", decrypted.decode())}\\
\texttt{      return}\\
Ciphertext: \texttt{e1619fe2...}\\
Key found: \texttt{6f. Decrypted plaintext: Hello, world!} \\
\midrule
Differential Cryptanalytic Attack & 
Differential cryptanalysis analyzes the differences in input-output pairs of ciphertext to identify patterns that can lead to key recovery. This aims to deduce information about the secret key used in encryption by observing statistical patterns or differences. &
\texttt{def differential\_cryptanalysis\_attack(ciphertext\_1, ciphertext\_2):}\\
\texttt{  for i in range(256):}\\
\texttt{    key = bytes([i]) * 32}\\
\texttt{    cipher = AES.new(key, AES.MODE\_ECB)}\\
\texttt{    decrypted\_1 = cipher.decrypt(ciphertext\_1)}\\
\texttt{    decrypted\_2 = cipher.decrypt(ciphertext\_2)}\\
\texttt{    if decrypted\_1 == decrypted\_2:}\\
\texttt{      print("Key found:", key.hex())}\\
\texttt{      return}\\
Ciphertext 1: \texttt{616d757a...}\\
Ciphertext 2: \texttt{646f6c6c...}\\
Key found: \texttt{6f.} \\
\bottomrule
\end{tabular*}
\end{sidewaystable}


\section{Generation of Random Numbers}\label{sec4}


\subsection{Determinism in machine and Pseudo Random Numbers (PRNs)}\label{subsec41}

In computational science, a machine is devised to act deterministically with the help of an algorithm inbuilt in it. The random numbers generated under an algorithm in a machine is called Pseudo-Random Numbers (PRNs), But they can offer deterministic randomness only, which is completely predictable once its internal state is known \cite{rarity1994quantum}\cite{liu2018device}. This nature of PRNs does not guarantee unpredictability or randomness all the time, if simulated for a longer period with huge data set, they are susceptible to convergence after a particular cycle, called period. In security-sensitive applications, this kind of algorithmic approach using mathematical element is proven to be highly unacceptable, as with an increase in computational power, the pattern behind generated random numbers under PRNGs will become fully deterministic. This has led the researchers to opt for sources of true randomness.

\subsection{Non-determinism in nature and true Random Numbers (TRNs)}\label{subsec42}

When we observe our physical world, we find natural fluctuation everywhere, as nature is intrinsically random. Thus, we can generate pure random numbers by measuring random fluctuations, known as noise. Noise, if can be measured through a process called sampling, can generate numbers, for example, if we measure the electric current of TV static over time, we will generate a truly random sequence. These random numbers are called True Random numbers (TRNs) and generating devices are called True Random Number generators (TRNGs). The random sequence generated using TRNGs can be visualized by drawing a path that changes direction according to each number, this phenomenon is known as a random walk, where there exists lack of pat-tern at all scales. At each point in the sequence, the next move is always unpredictable. Since random processes rely on the physical element, they are said to be non-deterministic as impossible determine in advance machines, on the other hand, is completely deterministic. The randomness of a sequence is dependent on the random-ness of the initial seed only. Same seed will eventually produce the same sequence and hence predictability of random numbers increases in accordance with the increase in the chance of guessing the initial seed. In many cases, a random bit generator is constructed from a source of true randomness, which used to seed the internal state of a PRNG. These type of RNGs are called Hybrid Random Number generators (HRNGs) \cite{ghosh2022quantum}\cite{furst2010high}.

\subsection{Comparison of Pseudo vs True Random Numbers}\label{subsec43}

Time-variant biased sequences of pseudo RNGs vs Unbiased sequences of pure RNGs

When numbers are generated pseudo-randomly many sequences cannot occur, for example, if Alice generates a truly random sequence of 20 shifts, it is equivalent to a uniform selection form a stack of all possible sequence of shifts. This stack will contain 2620 pages, which is astronomical in size, exploring which might take more than 200 million years. Compared to this, if Alice generates a 20 - digit pseudo-random sequence using a 4-digit pure random seed, this is equivalent to a uniform selection from 10,000 possible initial seeds. Generation of 10,000 different sequences is vanishingly small in a fraction of all possible sequences. When we move from random to pseudo-random shifts, there is a significant amount of shrinkage of key space into a much smaller seed space. So, for a pseudo-random sequence to be indistinguishable from a randomly generated sequence, it must be impractical for a computer to try all seeds and look for a match. Using advanced computing power, if random numbers generated using a pseudo RNG and a true RNG are simulated over a longer period, PRNs will exhibit convergence through a pattern and lose its unpredictable nature.


Random number generators based on microscopic phenomena to generate low level, statistically random noise signals such as thermal noise photoelectric effect and other quantum phenomena. These stochastic processes are in theory completely unpredictable and random sampling of a large set of data using Quantum Random Number generators produce evenly distribution of numbers and zero correlation among numbers.

\subsection{Partial determinism in hardware RNGs}\label{subsec51}

Evolution of a macroscopic system described by classical physics is predictable due to its sensitivity towards initial conditions, though apparently appear to be random and unpredictable. Measurement of a coin tossing can be predicted from several factors like speed with which the coin has been tossed, initial force applied on the coin, angular momentum of the coin, etc. Another example of physical RNG based on input sensitive noisy system include monitoring of electrical noise. Generation of random numbers by physical devices could be undermined by factors like predictable sources of noise as shown in Figure~\ref{fig:random_behavior}. 


\begin{figure}[h!]
\centering
\includegraphics[width=\textwidth]{img/random_behavior.png}
\caption{Random behavior for quantum superposition.}
\label{fig:random_behavior}
\end{figure}


The remarkable difference between classical physics and quantum physics lies in the fact of the inability of initial state environment effects to explain the occurrence of a definite experimental outcome. The environment cannot choose from possibilities in the wavefunction as an environment is also described by the same quantum laws. This advantage of quantum theory can be prominently used for generation of pure uniformly distributed series of random numbers. 

 
According to Prof. Hugo Zbinden of University of Geneva in Switzerland, " Quantum random-number generators profit from intrinsic randomness of quantum physics, whereas classical true random generators are based on chaotic systems, which are deterministic and, in theory, to some extent predictable". 

\subsection{Architecture of Proposed Solution}\label{subsec52}

Random numbers generated through classical software methods are not quantum-proof, as a quantum machine learning algorithm could exist to speed up the whole process of finding out the bitstream due to the lack of randomness of the system. Quantum cipher key generated from a Quantum Random Number Generator ensures true randomness and makes it secured from best known attacks. For the quantum random number generator, which generates random bits used in AES encryption, the key distribution function (KDF) is essential for securely distributing encryption keys to ensure that only authorized parties can access the encrypted data. This process is illustrated in Figure~\ref{fig:random_behavior}. This function is needed to maintain the confidentiality and integrity of the cryptographic process. 

\begin{figure}[h!]
\centering
\includegraphics[width=\textwidth]{img/key_distribution.png}
\caption{Random behavior for quantum superposition.}
\label{fig:key_distribution}
\end{figure}


In Figure~\ref{fig:block_qenabled}, a block diagram for quantum-enabled AES encryption is presented, beginning with a quantum random number generator. The process includes a randomness verifier and a quantum-classical integrator, ensuring the robustness of the encryption process. This leads to the transformation of plaintext into quantum-proof ciphertext, which is transmitted via a classical public channel. Finally, the AES modules for both encryption and decryption work together to generate the decrypted plaintext, ensuring secure communication. 


\begin{figure}[h!]
\centering
\includegraphics[width=\textwidth]{img/block_qenabled.png}
\caption{Random behavior for quantum superposition.}
\label{fig:block_qenabled}
\end{figure}

\section{Statistical Test Result and Analysis}\label{sec6}

Quantum Random Number Generators exhibit lack of pattern in comparison to conventional classical or Pseudo Random Number Generators during time-driven simulation on a huge data set. Figures~\ref{fig:Idqrandom1},~\ref{fig:param_adj} show NIST List of Tests and Parameter Initialization to detect non-randomness of a given sequence of binary data. Figures~\ref{fig:param_adj} and~\ref{fig:test_fail_2} show failed outcome of NIST test when fed with random bits generated from classical RNGs to demonstrate pseudo RNGs with small period are susceptible to predictability. During the design, security evaluation and certification process, principle of random number generator and its implementation are evaluated statistically using standard statistical test suites like National Institute of Standards and Testing (NIST) SP 800-22.


Although standard statistical tests can only evaluate the statistical quality of random bits generated instead of guaranteeing unpredictability of randomness source, we have performed NIST tests to check entropy of the generated random bits from Quantis USB QRNG by IDQuantique with 4 Mbps of entropy source. Each individual test is done on 10 bitstreams of 1 million binary digits. Different sets of random bits generated by our random number generator are rigorously tested through several iterations. Figures~\ref{fig:ido_qrng} and~\ref{fig:ido_qrng_partial} represent output results obtained by a classical Pseudo RNG (Linear Congruential Generator or LCG) and Quantis QRNG respectively on testing over a sample data of more than 10 million bits. 


\begin{figure}[h!]
\centering
\includegraphics[width=0.65\textwidth]{img/Idqrandom1_enhanced.png}
\caption{Idqrandom1.txt file containing 10M bits, is fed as input to NIST test suite, which has 15 statistical tests for verifying randomness.}
\label{fig:Idqrandom1}
\end{figure}


\begin{figure}[h!]
\centering
\includegraphics[width=0.65\textwidth]{img/param_adj_enhanced.png}
\caption{Parameter adjustments provide custom-specific testing for certain statistical tests.}
\label{fig:param_adj}
\end{figure}


\begin{figure}[h!]
\centering
\includegraphics[width=\textwidth]{img/test_fail.png}
\caption{Partial Screenshot(1) of Pseudo RNG - NIST Test Failed.}
\label{fig:test_fail}
\end{figure}


\begin{figure}[h!]
\centering
\includegraphics[width=\textwidth]{img/test_fail_2.png}
\caption{Partial Screenshot(1) of Pseudo RNG - NIST Test Failed.}
\label{fig:test_fail_2}
\end{figure}

\begin{figure}[h!]
\centering
\includegraphics[width=\textwidth]{img/ido_qrng.png}
\caption{For IDQ QRNG generated random bits, instantaneous entropy has been found, without discarding any bits or without any need of sampling/ whitening - NIST Test Passed.}
\label{fig:ido_qrng}
\end{figure}

\begin{figure}[h!]
\centering
\includegraphics[width=\textwidth]{img/ido_qrng_partial.png}
\caption{Partial Screenshot of IDQ QRNG with acceptable threshold of entropy - NIST Test Passed.}
\label{fig:ido_qrng_partial}
\end{figure}


From the above experiments and analysis, it can be concluded that - (i) AES key/ cryptographic key generated from quantum RNGs will work both for small and large periods/ cycles, (ii) key generated with pseudo RNG with significantly large periods will also fail, if trillions of data are fed, and (iii) patterns can never be found in any quantum key stream. 


\section{Use Cases of Quantum enabled AES Encryption}\label{sec7}

\subsection{Online Payments}\label{subsec71}

One of the most sensitive areas where encryption is used is in online payment. Even PCI-DSS standards mandate payment card data (stored as well as in-transit forms) to be encrypted using algorithms such as AES-256 \cite{raghu2015application}. However as mentioned before with the upcoming quantum computing era such algorithms will not aid in securing our data any longer.  However, a quantum computing aided cryptographic algorithm helps in such cases and makes it quantum safe. 

\subsection{Databases}\label{subsec72}

Database is another sensitive area handling all of our confidential data (from transaction passwords to health data) making encryption a must have. Encrypting databases help to restrict external hackers as well as insiders from seeing specific organizational data. Transparent database encryption (TDE) is a popular database encryption technique that helps to encrypt all “data at rest” in one go. Making it quantum safe in the quantum era is a must. 

\subsection{Virtual Private Networks}\label{subsec73}

VPNs or Virtual Private Networks are extremely useful and are used by many people for their business, personal data transaction, online payment and are also used by governments.

\begin{itemize}
    \item Many businesses use this so that their remote employees can connect to the office and connect to file servers, email servers and other services without having to make them directly available over the internet and hence creating a secured environment.
    \item For own Data privacy like sharing sensitive information over the internet, online transactions and many more.
    \item While using a public wireless network where it is really insecure to surf the web.
    \item By governments for sharing secretive and confidential information and requires a secured environment for data transfer.
\end{itemize}

VPNs rely on encryption algorithms however most VPNs rely on algorithms which are not quantum safe and are susceptible to quantum.


\subsection{SSL}\label{subsec74}

An SSL certificate (or TLS certificate) is a digital certificate that binds a cryptographic key to your organization's details. Secure Socket Layer (SSL) is cryptographic protocol designed to encrypt communication between a server and a web browser. Encryption data reduces the cybersecurity risk of man-in-the-middle attacks or many other forms of cyber-attack. Traditionally, SSL has been used to secure credit card information on e-commerce sites, personal data transfer and to secure social media sites. However a quantum computing aided encryption algorithm would make the whole protocol quantum safe. 


\subsection{Data in Cloud}\label{subsec75}

In public and hybrid cloud models, sensitive data resides at a third-party data centre. Any attack on co-tenants can result in that data getting exposed too. Encrypting your data in the cloud prevents hackers from being able to read it correctly. Quantum safe encryption algorithm helps in securing the data in the cloud \cite{sharma2016novel}. 

\subsection{Online Voting in Elections}\label{subsec76}

Secured online voting is another attractive aspect of cryptography since it aims at achieving security and privacy simultaneously. The available schemes rely on public key cryptography and pseudonyms rosters to conceal voters’ identities without ensuring privacy. Existing traditional voting schemes are susceptible to security threats caused by quantum algorithms as underlying mathematical assumptions are made upon integer factoring or discrete logarithm. Proposing voting schemes with inherent resistivity to quantum adversaries is of utter importance. One of the plausible solutions would be to encrypt the votes with a quantum key (post quantum cryptography) or to apply a quantum encryption algorithm to encrypt the votes as quantum messages to ensure their secrecy throughout transmission and ballot counting \cite{joseph2022transitioning}\cite{xu2023overview}. 

\subsection{Satellite Communications}\label{subsec77}

The classical channels used for satellite communications rely on one time pads (OTPs). The challenge with OTP is to ensure that only the transmitter and receiver have that key (set of random numbers) to encode and decode a message respectively. No party can assure the absence of any eavesdropper to copy the key while being distributed between sender and receiver. This problem can be solved by sending the key as quantum particles (photons) as any attempt to measure a quantum state would elapse its original state, thus both parties can detect the presence of an intruder. Further research on quantum invasion in satellite communications will allow a satellite to be securely re-keyed on orbit for providing encryption to uplinked command path and downlinked data path, distributing keys between widely separated ground stations with satellite relay, achieving communication over inter-continental distances. 

\subsection{Emails}\label{subsec78}

Email encryption helps to protect sensitive information sent through email channels. Public key encryption methods along with digital certificates are usually the methods used for securing email communications. Most email providers rely on quantum breakable algorithms making it susceptible to hacks. However, quantum computing aided AES cryptographic algorithm makes it quantum safe and much more secured. 

\section{Conclusion and Future Scope}\label{sec8}
In this groundbreaking research article, the utilization of quantum computing in conjunction with AES encryption has been proposed, presenting a paradigm shift in the field of secure communication. By harnessing the unique properties of quantum mechanics, such as quantum randomness, this innovative approach demonstrates the potential to significantly enhance the security of current encryption standards in the impending quantum era. Despite the limitations of quantum hardware's current capabilities, our solution has successfully leveraged the power of quantum mechanics to bolster AES encryption, ensuring robustness against emerging cryptographic threats. The introduction of quantum key-enabled AES encryption marks a pivotal advancement in the quest for quantum-secure communication protocols. With the implementation of quantum keys, the vulnerability of classical cryptographic systems to quantum-based attacks is significantly mitigated. Quantum computing-enabled AES opens a multitude of compelling use cases across various industries, as discussed in this article. While acknowledging the current limitations of quantum hardware, our solution emphasizes the immense potential of quantum computing in the near-term quantum era. The application of quantum randomness in current encryption standards represents a significant milestone in quantum-enhanced cryptography. The power of quantum mechanics imbues our solution with unparalleled unpredictability, rendering it highly resilient against brute-force and probabilistic attacks. As the field of quantum computing continues to evolve, our research lays the foundation for future developments in quantum-secure cryptographic algorithms and protocols. The successful integration of quantum computing with AES encryption demonstrates the viability of quantum-enhanced security solutions in real-world applications. It also serves as a clarion call to the cryptographic community, urging a collective effort to advance quantum-secure techniques to safeguard sensitive data in the face of quantum computing's rapid progress. In conclusion, the fusion of quantum computing with AES encryption represents a significant leap forward in securing communication channels in the quantum era. The quantum key-enabled AES encryption technique showcases the potential to fortify the existing cryptographic infrastructure against impending quantum threats. Despite the current constraints of quantum hardware, the utilization of quantum randomness in the present encryption standards demonstrates a glimpse of the immense power quantum mechanics can offer. With promising use cases across industries, our solution paves the way for future research and development in quantum-secure cryptography, empowering us to embrace the forthcoming quantum computing revolution with confidence. As we forge ahead, it is imperative for researchers, industry leaders, and policymakers to collectively embrace quantum-enhanced cryptography and collaboratively shape a quantum-safe future for our interconnected world. 



\bibliography{sn-bibliography}% common bib file


\end{document}
