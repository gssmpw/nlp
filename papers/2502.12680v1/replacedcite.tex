\section{Related Work}
This work builds on prior work on teleoperation and, in general, on cooperation between humans and automation.

\subsection{Scenarios for Teleoperation}
____ analyzed possible scenarios for teleoperation in road traffic. These scenarios were gathered via interviews and observations with control center staff, video analyses from real-world road events, and interviews with AV safety operators. ____ categorized use cases into eight clusters, including passenger interaction, technical malfunctions, and interactions between the environment, the RO, the AV, and others. 

In our study, the scenario consists of road works too complex or out of the legal boundary for our AV to handle independently and, therefore, is out of its ODD, relating to the unmapped construction site____. %They further propose that a vehicle could navigate through the road works on its own using road-site units as support. However, in our scenario, this option is not feasible as it surpasses the vehicle's ODD.


\subsection{Teleoperation Concepts}
\label{ssec:conceptsofteleop}
AV teleoperation intensities can be divided into three categories. ____ defined %collected multiple terms with the same usage to define a universal taxonomy for remote operation systems. The terms 
\textit{remote monitoring}, \textit{remote assistance}, and \textit{remote driving} in ascending order of the interaction intensity (see also ____). \textit{Remote monitoring} refers to a one-sided stream of information from the vehicle to the RO. \textit{Remote assistance} requires the RO to aid the AV in its decisions but not take over to perform actual driving tasks, contrary to \textit{remote driving}. ____ define \textit{remote driving} "as 'real-time performance of part or all of the DDT [dynamic driving task] and/or DDT fallback [...] by a remote driver.'"~\cite[p. 2]{bogdoll_taxonomy_2022}, while directly controlling the vehicle "in the form of steering and throttle/brake commands"~\cite[p. 10]{bogdoll_taxonomy_2022}.

Using ____'s taxonomy, ____ surveyed prior work. They split \textit{remote driving} and \textit{remote assistance} into concepts. We evaluated concepts of \textit{trajectory guidance} (building a path by dragging the mouse cursor), \textit{waypoint guidance} (generate a connected path by placing single points), and \textit{interactive path planning} (user chooses one of multiple system-offered paths) and compared them. We exclude other concepts as they are either strongly tied to \textit{remote driving} or too use case specific. We specifically exclude the concepts of explicit \textit{remote driving} due to their limitations in teleoperation (see Section~\ref{sec:limitations}). %, as well as to provide a fair comparability between the analyzed concepts.

Although ____ classify \textit{trajectory guidance} as a form of \textit{remote driving}, based on the practical implementation we are proposing, the fluid transition between those categories and the corresponding definition in the taxonomy by ____, for further discussion, we categorized it as \textit{remote assistance}.
Additionally, the interaction type for actions in remote assistance is very different compared to active control due to the different levels of interaction abstraction (direct movement vs., e.g., point guidance) and the event-driven interaction requests in remote assistance____. 
%Furthermore, there are specific use cases in the teleoperation of vehicles____ where remote assistance is unavailable due to, for example, the unavailability of parts or subsystems crucial for autonomous driving. ____ also restrict the usage domain of \textit{remote assistance} to SAE-Level 4 vehicles. 
%Thus, when to use which modality is still to be examined and discussed in future work. These considerations should also consider the applied limitations to both categories (e.g., latency requirements and part or subsystem availability).

____ evaluated several teleoperation concepts (direct control, shared control, trajectory guidance, waypoint guidance, collaborative planning, and perception modification) from an ego perspective with eight experts in the automotive and teleoperation industry. Such experts might not be the final users of these teleoperation concepts, so we had participants without a background in these industries. They found that ``a holistic teleoperation system should be composed of implementations of the Shared Control, Collaborative Planning and Perception Modification concepts''~\cite[p. 639]{brecht2024evaluation}.



%Complementary to this study, 
____ conducted a literature survey regarding the "final 100 meters problem"____, which relates to the variability of the final destination in a journey based on user preferences and technical limitations. The survey was conducted on "possible interaction modalities and modes"~\cite[p. 2]{colley_systematic_2022}, creating an overview of possibilities to communicate with the vehicle. %The findings include different experimental methods for controlling the vehicle directly. Derived from this, a study was conducted with VR simulation, which, among other objectives, compared different methods of interaction modalities for actively controlling an AV. 
The results of a VR study indicated a preference for a steering wheel, followed by using a joystick. %As ____ point out, the preference towards the steering wheel could be due to the familiarity with this modality from the driving experience. % contrary to the general novelty towards the other modalities in this aspect.
%Our work follows a similar approach but in a different context. 
While ____ evaluated interaction concepts from inside the AV, we compare concepts for remote assistance. This opens new possibilities and interaction perspectives (e.g., handling multiple requests simultaneously or simplified representations of the environment).

%While parts of our work rely on the findings of ____, we investigate the steering of AVs in a different context. This context is characterized by different HMIs but the same modality (i.e., a mouse) in only a remote assistance manner, parallel handling of multiple requests, different visual perspectives (e.g., computer screen instead of VR), two-way delays between control inputs and visualizations of their actions (see section \ref{sec:limitations}), expected vehicle-sided perceptual or control errors or very complex situations creating exceptionally high cognitive load. This provides a different environment for such operations, adding the prospect that future ROs will be confronted with such situations several times a day, as opposed to the situational 100m problem that arises in the context of more private demand. Therefore, there may be different potentials for the modalities in terms of \textit{(1) cognitive load, (2) trust, (3) perceived safety, (4) preference, (5) subjective situation awareness, and (6) behavior}, from the perspective of, in this case, a RO.



%\subsection{Cooperative Interfaces in Automated Vehicles}
%Switching to manual driving is often employed as a solution to the limitations of AVs____. However, this method has raised several concerns. Situations that an automated system finds difficult to handle may also be challenging for a human driver. As a result, completely disabling the automated system, even if some supportive functions are available, is viewed with skepticism____. Additionally, the transition from automated to manual driving might negatively impact the human user's performance on the road____.

%A potential solution is driver-vehicle cooperation, where both the system and the driver work together like teammates to compensate for each other's weaknesses____. Humans can assist AVs in identifying unexpected situations and determining how to address them____, forecasting pedestrian actions (such as crossing the road)____, and giving the go-ahead for specific maneuvers____.

%Currently, if an AV is unable to recognize objects confidently, the standard procedure is to transfer control to the human driver. But the human passenger could also assume responsibility for object classification. ____ introduced two interaction techniques: a \textit{free text} system allowing the driver to identify an unrecognized object vocally, and a \textit{choice} system, a more advanced approach where the AV proposes selectable options without definitively selecting the correct one. This can occur with objects that look very similar or have minor changes, such as in the case of ____. A follow-up work by ____ explored this possibility more. While the cooperation approach was evaluated for users \textit{within} an AV, this could also have implications for teleoperation interfaces.

\subsection{Human Factors}
\label{sec:human_factors}
\subsubsection{Human Factors in Remote Operation}
____ identified the human factors related to remote operations. Although they focus on remote \textit{ship} operations, many factors apply in an automotive context. Especially situational awareness (SA)____, the lack of physical sensing and soundscape____ and the implied delays____ are important for ROs. %This is especially true for driving tasks but also for remote assistance, which requires consideration in the design of remote control interfaces for AVs.

%Another aspect to be considered is presented by ____. 
Additionally, ____ raised the question of what the maximum feasible count N of vehicles managed concurrently is. Further challenges involved are determining the difficulty levels when one RO has to grasp the situation for multiple vehicles in case of emergencies or determining "how many additional operators (up to N-1 carts) are required when one cart is in the takeover?"~\cite[p. 6]{hashimoto_human_2022} (cart $\widehat{=}$ vehicle).
Additionally, switching between vehicles and switching costs shall be considered____. Thus, our study addresses the applicability of one operator managing multiple vehicles concurrently. It takes performance measurements on parallel incoming vehicle requests, tied with the currently used interaction modality. To support parallel request handling, our proposed UI provides an additional secondary display for a selected vehicle to display its front camera to monitor the vehicle.

\subsubsection{Organization and Roles in Remote Operation Centers (ROCs)}
____ proposed a "first conceptual step" for an organizational structure for ROCs. While being a preliminary concept, it lays a foundation for future discussions. This proposal defines three central roles, the Remote Coordinator (RC), the Remote Driving Operator (RDO), and the Remote System Operator (RSO), along with multiple Peripheral Roles that support and complement core operations. 
The primary task of an RC is to maintain an overview and monitor all incidents and how they are handled. RDOs are responsible for remotely controlling individual AVs, while RSOs manage a fleet of AVs. 

%There is much more detail to this concept in the respective work by ____, however, the provided overview suffices for the scope of this paper.
%____ TODO::

%ovfrom differenterspecified?
%Collected factors there with a directly matching viewpoint in the automotive remote operation context are (1) situational awareness(SA)____, (2) high workload, (3) trust, (4) boredom, (5) fatigue, (6) skill degradation, (7) human-machine interface, (8) lack of direct sensory information in the SCC, (9) communication challenges, (10) decision-making, (11) linguistic problems, (12) cultural problems, (13) teamwork


%%DO NOT DELETE============================================================
%überhaupt wirklich nötig? betrifft nicht direkt die Fernsteuerung im Endeffekt und ist nur information drum herum
%application domains: ____, ____, ____
% \subsection{Application in Different Domains}
% While the pure problem-solving aspect of remote controlling still has application when even SAE Level 5____ will become the standard (e.g., critical system failures), there are multiple more application domains for automotive remote control until this milestone is reached. Vreeswijk et al.____ present a pilot, where a shuttle, bound to an operational domain but without additional infrastructure, actively participated in public transport operations. In case of yet for the AD undecidable situations, in the latest increment of the pilot, an RO could intervene and solve the situation remotely. Upcoming passengers also could order specified routes to possible locations inside the operational domain, just as in the conceptual "Robo-Taxi", Cummings et al.____ outlined.

% Further, Cummings et al.____ mention different additional Concepts of Operations for AVs, whereby aspects of remote operations play a role primarily in the context of remote monitoring and remote assistance. One of those concepts is "Original Equipment Manufacturing (OEM) AV Dispatch Support", which refers to remote monitoring and support operations on vehicles by their manufacturer.
%//TODO!!



\subsection{Interface Requirements}
\label{ssec:interfacereq}
Several studies were conducted to determine the requirements for remote control interfaces for AVs, for example, the HAVOC Project____\footnote{\url{http://tinyurl.com/havocproject}; Accessed by: 29.11.2024}. 15 participants monitored 10 automated trucks and had to assist in five occurring events actively. Two of those events were active control tasks conducted with a steering wheel, pedals, and a computer screen. %Several remarks (user needs) were collected concerning remote monitoring, assistance, and remote driving. 
The feedback focused on the lack of information to consciously control the vehicle, including the need for a 360° overview of the surroundings and the uncertainty of when the AV can take over again, which we aimed to resolve in our prototype. To achieve this within the interface, we included a projected birds-eye view of the vehicle, added information on collision avoidance detection, and notified the RO when the vehicle took over again.
%These aspects, therefore, were considered when designing the apparatus for the user study, as partly described in section~\ref{sec:study}.

Comparably, ____ studied remote driving.  %Although remote driving has different interface requirements than remote assistance, some overlap exists.
The requirements we identified and adapted for our use case concerning remote assistance included emphasizing the takeover reason, giving contextual road information, and giving AI suggestions for operations____. We included automated suggestions only in \pathPlanning for differentiation. %However, we kept automated suggestions reserved for \pathPlanning as this already embodies a selection from given suggestions and otherwise would break the fair comparison of the pure interaction concepts.

____ used a demonstrational VR remote driving setup, whose interface was based on the requirements of ____.
To gather these requirements, ____ conducted 18 interviews. Half of them had some remote controlling experiences. %, although not necessarily with vehicles. 
80 requirements were collected. In a second interview with participants from industry, these requirements were clustered and rated, resulting in a final set of 20 requirements (e.g., 360-degree remote view, vehicle position, vehicle issues, traffic rules), where we adopted the relevant ones. % While these requirements were collected for a remote driving background, not all may apply to remote assistance, especially regarding the necessity rate. Nevertheless, we identified and adapted the relevant requirements for our interface, such as a 360-degree (top-down) view from the vehicle, information concerning the reason for the takeover, projected control actions, and vehicle speed.

%Similar to our study and apparatus, 
____ designed a Human Machine Interface (HMI) for the teleoperation of vehicles in a public transport control center context. %They gathered data from a study with 12 participants, all employees in public transport control centers in Germany. 
They wanted to receive "a first impression on whether the development is in the right direction, particularly whether its overall setup is valid"~\cite[p. 19]{kettwich_teleoperation_2021}.
The findings included that the proposed HMI establishes a "suitable interface design for the teleoperation of highly automated vehicles in public transport"~\cite[p. 15]{kettwich_teleoperation_2021}. However, some suggested improvements were related to overwhelming information. Participants dealt with 6 monitors and operated with a keyboard and mouse and an additional touchscreen. Thus, we created a more minimalist design to maintain focus on the most important elements for the given task (see Section~\ref{sec:study}).

Further, ____ defined seven HMI evaluation criteria: (1) Features, (2) Information, (3) SA, (4) Usability, (5) User Acceptance, (6) Attention, (7) Capacity. We used questionnaires to evaluate our prototypes. We used the SUS____ to measure user usability, the van der Laan acceptance scale____ to measure User Acceptance, and NASA-TLX____ to measure user capacity. Additionally, participants were asked for liked and missed features. However, we did not include dedicated scales for measuring the remaining criteria: Information, SA, and Attention, but instead relied on logged data to provide recognition to them. %, sufficient to address our research question.






\subsection{Limitations of Remote Controlling}
\label{sec:limitations}
While controlling AVs remotely brings many advantages, limitations also exist. ____ and ____ raise critical issues addressing the remote driving of AVs. Delays concerning the technical side of communication and the human side of reorientation and reaction time narrow the applicability of remote operations. Therefore, ____ point out that requiring an RO to take over the vehicle at high speeds in a short time frame in the range of seconds would be unsafe. Taking these limitations into account, the expected safe situations that remain open for a takeover are those in which the vehicle is traveling at slow speeds, less than 10 mph____, or has already pulled to the shoulder by performing a DDT as specified by the SAE____ in a critical situation.
The technical delay in communication was further elaborated by ____. They present a case study further investigating the imposed delay by comparing the effect of different levels of resolutions in LTE____ and Wi-Fi____, which represent "wireless networks used by vehicles"~\cite[p. 22]{kang_augmenting_2018}.
% \url{https://www.telekom.com/de/konzern/details/5g-geschwindigkeit-ist-datenkommunikation-in-echtzeit-544496#:~:text=Die%20durchschnittliche%20Latenz%20im%20LTE,Zeit%20von%20drei%20Millisekunden%20erzielt.}
Their results show that higher resolutions, i.e., higher image sizes, lead to higher image latencies. %Subsequently, two optimization techniques are proposed. The first is to dynamically adjust the intervals of the intra-coded frames (I-frames) relative to the predicted frames (P-frames) based on the different vehicle dynamics. The second is to minimize the transmitted data by transmitting only the differences between two 3D maps on both sides.%Ergebnis

Concerning the given limitations on remote driving operations, we propose that the effects of limitations on remote assistance are less intensive and, therefore, bearable. As the task of speed and brake control is delegated to the vehicle, many involved security issues caused by latency and human factors (see  Section~\ref{sec:human_factors}) are minimized or entirely mitigated. Additionally, enabling the RO to plan vehicle routes for limited distances within remote assistance also enables them to assist multiple vehicles simultaneously, contrary to basic remote driving.
Therefore, our study focuses on remote assistance to take advantage of the benefits an SAE 4 vehicle presents to mitigate security risks.

%Statement re. research/dev. gap that we address