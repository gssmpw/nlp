\section{Methods}
\label{sec:methods}
Inspired by ongoing work,
we have two aims in
developing an ethical framework for African data ethics: 1) honoring the existing scholarship of African data ethicists by preserving diverging perspectives in the discourse, and 2) re-engaging with classic African philosophies to productively expand on African data ethics principles. To achieve the balance between a respectfully discursive and productively expansive framework, we combined methods from qualitative document analysis and literature reviews to develop our qualitative analysis protocol
\cite{morgan2022conducting, birhane2022forgotten, battle2024what}. 

\subsection{Data Collection}
The first author seeded the search by reading \textit{The African Philosophy Reader} \cite{coetzee2004african} due to prior exposure.
This text informed subsequent keyword searches in established academic databases:
Google Scholar, Web of Science, Scispace,\footnote{
Scispace is a language model search engine for literature reviews: \url{https://typeset.io}} and the publication repositories of ACM and IEEE.
The first author also searched their institutional library and online African philosophy libraries accessible through their institution.
They used key phrases such as \textbf{``African AI ethics''}, \textbf{``African philosophy''} and \textbf{``African data ethics''} to identify relevant literature. 
Documents from database searches were excluded if African values and data science practices were not the primary topic. For example, \citet{stahl2023ai} 
was excluded because its focus on North African data policy. \citet{okolo2023responsible} was included because the document primarily discussed the AI climate in SSA countries. All identified papers from the ACM Digital Library were ultimately excluded because African data values were often only discussed within a global overview of data ethics. 
In addition, documents were excluded if they were not a full document. Full documents were understood as non-archival and archival papers (no extended abstracts), reports, or book chapters. No range was set on the publishing year to permit the inclusion of foundational texts from African philosophy and African information ethics.

In parallel to keyword searches, we requested literature recommendations from other scholars in the field. 
 In addition, we used reference and citation tracking to identify relevant documents that were missed in searches. By the end of our iterative data collection, 47 documents were collected. Details of inclusions/exclusions can be found in Appendix~\ref{apdx:method}.


\subsection{Thematic Analysis}
 The first author reflexively coded the documents through a practice iteration followed by two rounds of coding to surface themes of African data ethics. 
     First, the first author began a grounded reflection process by selecting six documents as a representative set of corpus topics \cite{coetzee2004particularity, mhlambi2020from, african_union2024continental, ndjungu2020blood, segun2021critically, day2023data}.
     From this selection, they highlighted and recorded meaningful excerpts from each document. Then they annotated a reflection about how the excerpt answers the research question. These reflections enabled the first author to focus the coding on what is considered an ethical principle. In subsequent coding iterations, we consider an African ethical principle to be a \textbf{moral value, understanding, or standard prioritized by or derived from SSA communities}. 
     The first author then repeated the above process for each document in the full corpus, where they coded for statements that answer the research question. Then, they organized the resulting codes into themes through affinity diagramming \cite{scupin2008kj} using Miro.
     After the first full round of coding and principle clustering, several minor principles had conceptual gaps 
     requiring the analysis of additional documents. For example, several documents mentioned centering women but
     lacked details in terms of responsible data science practices. To answer the open question, work from Pollicy\footnote{\url{https://pollicy.org/}} (an Afro-feminist technology collective based in Uganda) was collected to bolster the principle. In this iteration, excerpts the minor principles served as a codebook.
