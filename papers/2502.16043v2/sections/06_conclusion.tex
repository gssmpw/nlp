\section{Limitations}
Our reflexive thematic analysis should not be conflated with a systematic literature review. While we did cast a wide net to build our corpus (including gray literature) there are impactful African data ethics documents not included in our framework. Therefore, the resulting framework should not be considered comprehensive but an introduction to notable ideas. Future work will aim to incorporate more African data ethics documents.

\section{Conclusion}
Through a thematic analysis of 47 documents, we derived an African data ethics framework that encompasses seven major principles: \textbf{1)} decolonize \& challenge internal power asymmetry, \textbf{2)} center all communities, \textbf{3)} uphold universal good, \textbf{4)} communalism in practice, \textbf{5)} data self-determination, \textbf{6)} invest in data institutions \& infrastructures and \textbf{7)} prioritize education \& youth. Our framework scratches the surface of African data ethics discourse, and the surface is rich with historically grounded, communitarian, and pragmatic insights for RDS.

A comparative analysis of our framework with six other data ethics frameworks highlights African perspectives as progressive and needed voices in global data ethics discourse. For truly pluralistic and responsible data science, we urge the data ethics community to readily seek the perspectives of Africans and other practitioners of the Global Majority. Such inclusion will not only enrich the theoretical foundation of data ethics but can inform more equitable and culturally responsive approaches to data governance, algorithmic fairness, and technological development. 

Finally, a discussion of two case studies demonstrates the utility of our framework for evaluating RDS practices. The growing interest in developing and adopting AI tools for use in African contexts highlights the potential for our framework to be integrated into real-world data science workflows. In view of this, future work will engage African practitioners to evaluate the merits, gaps, and usability of our framework. 

