\section{Background}
\label{sec:background}

\subsection{Pluralistic Efforts in Global Data Ethics}
Data ethics encompasses the normative frameworks and moral principles governing the collection, processing, storage, and deployment of data
\cite{floridi2016data}. 
An ethical principle, in a data ethics context, represents a \textbf{fundamental normative guide for action}—such as respect for persons, beneficence, or justice—that transcends mere technical guidelines or professional best practices. What distinguishes genuine ethical principles from operational procedures is their foundation in moral philosophy and their universal applicability across contexts.
Literature within this field has proposed new data ethics paradigms \cite{krijger2023ai, verhulst2021reimagining, burr2023ethical}, critiqued current approaches to responsible data science \cite{taylor2019responsible, leonelli2016locating, gansky2022counterfacctual}, and experimented with algorithmic or other technical approaches to mitigate bias \cite{hort2024bias, wang2023mitigating, smith2020mitigating}. While scholarly communities such as FAccT and AIES serve as platforms for engaging data ethics discourse, a significant portion of the scholarship produced in these venues has its roots in Western philosophy. 

While Western philosophical traditions have dominated data ethics discourse through frameworks such as utilitarianism and deontological ethics \cite{moran2022towards, heavin2024digital}, emerging scholarship increasingly recognizes the necessity of incorporating diverse ethical paradigms and epistemological frameworks from Global Majority perspectives. For example, Indigenous knowledge systems provide alternative conceptualizations of data sovereignty and stewardship that challenge Western individualistic notions of privacy and ownership \cite{carroll2023care, hudson2023indigenous, walter2021indigenous, lovett2019good, rainie2019indigenous}. East Asian philosophical traditions, including Confucian ethics, contribute valuable insights regarding social harmony and collective responsibility in technological development, offering nuanced frameworks for balancing individual rights with communal interests in data-driven systems \cite{d2023ai, ess2006ethical, yeh2010effect}. Traditional African philosophy offers Ubuntu-based approaches that emphasize collective well-being and communal responsibility in data ethics \cite{reviglio2020datafied, okyere2023place}.

\subsection{African Philosophy: Foundation of African Data Ethics}
 \textbf{African Philosophy is a sub-domain of philosophy meant to reclaim and generate philosophical theories from Sub-Saharan Africans (SSA)} \cite{ramose2004struggle}.\footnote{Throughout the rest of the document, Africa will refer to Sub-Saharan Africa.} 
At the time of rapid decolonization, African philosophers pivoted from primarily engaging with Western philosophy to focus on uncovering the rich intellectual theories and practices of pre-colonial Africans \cite{wiredu2004akan}. In addition to examining the past, African philosophers work to cultivate an intellectual home for new philosophies that speak to the realities of modern African life \cite{ramose2004struggle, kohnert2022machine}. 

Several African philosophical frameworks offer valuable insights for data ethics and responsible technology development. The Yoruba concept of "iwa" (character/moral behavior) emphasizes the ethical implications of one's actions on the collective community, suggesting approaches to data governance that prioritize communal benefit over individual gain \cite{oyeshile2021yoruba}. Similarly, the Akan concept of "onipa" (personhood) \cite{wingo2006akan, owusu2019onipa} and the Zulu notion of Ubuntu, often translated as "umuntu ngumuntu ngabantu" (a person is a person through other persons) \cite{reviglio2020datafied, gwagwa2022role}, provides frameworks for understanding human dignity and agency in technological contexts, particularly relevant for issues of consent and data sovereignty. The Ethiopian philosophy of "Medemer" (synergy/coming together) offers a model for collaborative data sharing and governance that balances individual autonomy with collective benefit \cite{assefa2024critical}. Additionally, the Igbo concept of "omenala" (customs/traditions) \cite{nwoye2011igbo, nwala1985igbo} and the Swahili principle of "ujamaa" (familyhood) \cite{nyerere1962ujamaa} suggest that data governance should align with existing social structures and cultural practices rather than imposing external frameworks. Overall, these philosophies collectively lay the foundation for ethical frameworks that challenge Western individualistic approaches to data privacy, data sharing, and ownership while promoting interdependence and cultural alignment.
\subsection{Current AI Ethics Discourse in Africa}
A growing number of scholars are leveraging African philosophy to assess technological development and AI \cite{ewuoso2021african, capurro2008information, ruttkampbloem2023epistemic, gwagwa2022role, attoe2023conversations}. 
These works, along with foundational African philosophy, can inform practical approaches to data collection protocols that respect communal ownership and decision-making \cite{buhler2023unlocking}, inspire algorithmic fairness metrics that incorporate African conceptions of justice and equity \cite{asiedu2024case}, and help shape privacy frameworks that balance individual rights with community interests \cite{jimoh2023quest}. Additionally, African philosophy can be valuable in informing ethics review processes that consider local cultural contexts and values and help build data governance structures that reflect African leadership models and decision-making practices \cite{chinweuba2019philosophy}. Our work aims to help bridge these issues and expand discourse on how African philosophies can contribute to advancing responsible data science practices within the continent.


\subsubsection{Practitioner-led}
Sábëlo Mhlambi's pioneering work in African data ethics is grounded in the principles of Ubuntu \cite{mhlambi2020from}. He draws from the epistemic, ontological, and ethical theories of African philosophers like Mogobe Ramose \cite{ramose2004ethicsofubuntu} to critique prevailing understandings in AI ethics such as agency \cite{mhlambi2023decolonizing}. Since Mhlambi's Ubuntu data ethics contribution, a significant amount of African scholarship on data ethics highlights Ubuntu as an African philosophy that should be engaged with more in global data ethics work \cite{gwagwa2022role, segun2021critically, langat2020how, kiemde2022towards, goffi2023teaching}.  
This body of work makes a general appeal to African communitarian ethics to critique current ethical paradigms \cite{metz2021african}, report on data science work in Africa \cite{day2023data}, and survey perceptions and concerns among African data science practitioners \cite{eke2022forgotten}. There is also a plethora of African data ethics gray literature, information published outside of academic venues, published through African data organizations (often in partnership or funded by Western institutions) by way of blog posts \cite{shilongo2023creativity}, reports \cite{sinha2023principlesafrofeminist}, and formalized briefs \cite{gwagwa2019recommendations}. 
It must be noted that many contributions are made by Western, non-African data ethicists and disproportionately fail to include other African philosophies beyond Ubuntu. 

\subsubsection{Continental \& Local Policy}
The development of local and continental policies for AI and data regulation can help African countries improve adherence to data ethics frameworks while steering responsible AI development. 
% Over the past decade, there have been an increasing number of efforts focused on implementing data regulation in African countries. 
Currently, 38/55 African Union (AU) member states have enacted formal data protection regulations, with Malawi and Ethiopia recently enacting data protection laws in mid-2024 \cite{okolo2024operationalizing}. The African Union has also released continental frameworks, such as the African Union Convention on Cyber Security and Personal Data Protection (Malabo Convention), which mandates national cybersecurity policies and strategies while addressing personal data protection and cybercrime \cite{malaboconvention}. 
In parallel, efforts to regulate AI are gaining momentum, with 14 countries adopting national AI strategies \cite{ECDPM-Africa}, complemented by the AU’s AU-AI Continental Strategy published in August 2024 \cite{african_union2024continental}. However, systematic gaps in data regulation persist \cite{okolo2024operationalizing, eke2022responsible, john2021technology, osakwe2021strengthening}, and if left unaddressed, these shortcomings may undermine AI regulatory efforts. To advance data ethics and safeguard African communities, governments must prioritize enforcement of existing data privacy laws while ensuring regulations provide adequate protections.


