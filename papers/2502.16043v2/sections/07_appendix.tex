\section{PRISMA Diagram of Document Collection}
To supplement the description of the data collection process in \nameref{sec:methods} section, the following PRISMA diagram illustrates the corpus-building process including the identification, screening and inclusion stages. 
\label{apdx:method}
\begin{figure}[!ht]
  \centering
  \label{fig:prisma}
  \includegraphics[width=\linewidth]{figs/PRISMA_Diagram.pdf}
  \caption{PRISMA Diagram of document collection process}
  \Description{A woman and a girl in white dresses sit in an open car.}
\end{figure}

\section{African Data Ethics Framework Case Studies}
\label{apdx:more_case_studies}
The following section discusses five more case studies from data science work in the African context.
\subsection{Decolonize \& Challenge Internal Power Asymmetry Case Study: DRC Mineral Extraction}
The violent exploitation of miners in the Democratic Republic of Congo (DRC) is a harrowing example of how unchecked power from the West and within Africa corrupts the data science ecosystem. The DRC is home to an abundance of minerals necessary for data science. Specifically, cobalt and silicon are foundational components of all technology, especially the massive amount of computers that store and process data in global data centers \cite{ndjungu2020blood}. To keep up with computing demand, multinational companies collect a copious amount of minerals from the DRC. Silicon and cobalt are often referred to as ``blood minerals'' because Western companies are able to make billions of dollars from the technology industry while the DRC continues to experience violent internal displacement, ineffective interventions, and a minuscule fraction of the value of their mining labor \cite{ndjungu2020blood}. The extractive relationship between the Congolese and large Western companies has direct parallels to DRC's colonial relationship with Belgium \cite{ndjungu2020blood}. 
The monarch of Belgium, King Leopold, violently claimed the DRC to extract and sell raw materials so Belgium could be a major player in meeting the material demands of an industrializing Europe without any concern for the humanity of the Congolese people. This neocolonial relationship is furthered by African leaders who have assumed the role of middlemen in the mineral trade. The Dodd-Frank Section 1502 is a United States law passed to address the dehumanizing mining labor practices \cite{ndjungu2020blood}. This law required companies to execute due diligence to ensure they were not selling DRC minerals mined from conflict. Rather than adhering to this law, multinational companies pulled out of direct agreements with the DRC and joined new partnerships with neighboring countries such as Rwanda. These other African countries serve as middlemen to buy blood minerals from the DRC and sell to the multinational companies so the companies could keep their hands clean in the eyes of the law \cite{ndjungu2020blood}. Understanding the colonial and modern-day political background of DRC blood minerals is key to contextualizing calls for reducing the scale of datasets and demand for new technology \footnote{\url{https://newint.org/violence/2024/its-time-hold-big-tech-accountable-violence-drc}}. Every leader within the data science ecosystem has the responsibility to challenge and not perpetuate colonial and asymmetric power. 

\subsection{Uphold Universal Good: International Partnership to Increase Access to COVID-19 Information}
At the beginning of the COVID-19 pandemic, the Rwandan government partnered with the German technology company GIZ to develop a chatbot for remote communities to access tailored COVID-19 information and guidance \cite{kohnert2022machine}. To meet the needs of Rwandan users, the chatbot can communicate in the local language of Kinyarwanda; the medical advice is based on the Rwandan medical databases, and the project is open source \footnote{\url{https://github.com/Digital-Umuganda/Mbaza-chatbot}}. Beyond the features of the product, both organizations worked together to develop Rwanda's technical infrastructure to not only host the chatbot software but also maintain local technologies in the future \footnote{\url{https://www.giz.de/en/workingwithgiz/KI-Ruanda-Digitalisierung.html}}. The RBC chatbot is the product of an equitable partnership of African and Western data organizations that were committed to promoting the well-being of their community in the face of a catastrophic pandemic that impacted the world. The collaboration was successful because their decisions were attuned to each stakeholder's capabilities and limitations \cite{kohnert2022machine}. By all accounts, this is an example of a harmonious, dignified, and socially good data science practice.

\subsection{Communalism in Practice Case Study: Challenging Utilitarian Data Ethics with a Communitarian Analysis}
 Another aspect of practicing communalism in data science is applying communitarian theories as a lens for evaluating data ethics. The Western concept of utilitarianism is a predominant paradigm in data ethics. Utilitarian data science aims to construct AI and other DDS that maximize the amount of social good and minimize the amount of social harm at scale (see effective altruists). African communitarian theories provide novel and strong critiques of utilitarian data ethics as well. One African data ethicist, in particular, applies African values to outline why utilitarianism: 1) trivializes human dignity through rationality, 2) justifies the suppression of non-dominant people and values, 3) ignores the role relationality plays in human-AI interaction, and 4) misinterprets the nature of self-sacrifice \cite{metz2021african}. Respectful debate with diverging perspectives is essential to the progress of RDS.

\subsection{Invest in Data Institutions \& Infrastructures Case Study: Building the Capacity of National Statistical Offices}
As a formal data collective, the National Institute of Statistics of Rwanda (NISR) developed a report to guide their management procedures for administrative data \cite{habimana2018guidelines}. Their report recognizes the data sharing network they are a part of, raises concerns with data quality specific to Rwanda, and proposes new administrative standards for assessing data quality. While the authors recognize that their data management infrastructures need to progress, they view collaborations between local data practitioners as the key to development. Development collaborations include technical workshops, conferences, dissemination of data quality frameworks, and supporting staff in their respective data work \cite{habimana2018guidelines}. African data scientists are eager to develop their communities' capacity to manage data science projects. Organizations such as NISR recognize that this development requires a comprehensive assessment of the status quo, supportive collaboration, and incremental development of data standards.

\subsection{Prioritize Education \& Youth Case Study: African Data Science Research Mentorship}
There is a growing community of African researchers, engineers, and technologists ready to utilize data science to make meaningful changes in their communities. However, the eagerness of young data science researchers must be met with mentorship and education on how to consider their positionality in their work. A set of fictional narratives inspired by real African data science projects illustrate the importance of education in the development of caring and successful data scientists \cite{abebe2021narratives}. Young researchers are interested in tackling large-scale issues to help their communities. However, this passion can lead to overstepping boundaries with data subjects, especially in cases of data refusal. Therefore, if a young data scientist is unsure, they should seek the advice of trusted advisors and inform themselves about the customs, languages, and history of the data communities they want to work with before collecting data. Even though within a data science team a young data scientist has lesser power due to seniority, they yield a lot of power over their data subjects. As such, it's important to recognize the multiple positionalities a data scientist holds and adapt to feedback from all stakeholders. 

\section{Comparative Analysis Table}
As a supplement to the \nameref{sec:compare}, the following table shows the coding of minor African data ethics principles to each of the six particularist frameworks.
\label{apdx:compare_table}
\begin{figure}[]
  \centering
  \includegraphics[width=0.9\linewidth]{figs/Comparison_Coding.pdf}
  \caption{Table of African Data Ethics Principles covered in six particularist data ethics frameworks}
  \Description{A table of minor African data ethics principles on the y-axis and six comparison papers on the x-axis (\cite{klein2024data,floridi2018ai4people,carroll2021operationalizing,jobin2019global,younas2024proposing, raquib2022islamic}. Between these axes white and green cell colors indicate if a principle is present or not. At the last row there are fractions to count each green cell (14/23, 14/23, 15/23, 13/23, 10/23, 6/23). }
\end{figure}