\section{African Data Science Ethical Framework}


In this section, we summarize the key components of our framework, shown in \autoref{tab:framework-overview}.
\begin{table}[!h]
\centering
\resizebox{\columnwidth}{!}{%
\begin{tabular}{|c|c|}
\hline
\rowcolor[HTML]{9B9B9B} 
\textbf{Major Principle} & \textbf{Minor Principles} \\ \hline
\rowcolor[HTML]{FFFFFF} 
\cellcolor[HTML]{FFFFFF} & \cellcolor[HTML]{EFEFEF}Challenge Colonial Power \\
\rowcolor[HTML]{FFFFFF} 
\multirow{-2}{*}{\cellcolor[HTML]{FFFFFF}\begin{tabular}[c]{@{}c@{}}Decolonize \& \\ Challenge Internal Power Asymmetry\end{tabular}} & Challenge Internal Power Asymmetry \\ \hline
\rowcolor[HTML]{FFFFFF} 
\cellcolor[HTML]{FFFFFF} & \cellcolor[HTML]{EFEFEF}Community in Everything \\
\rowcolor[HTML]{FFFFFF} 
\cellcolor[HTML]{FFFFFF} & Solidarity \\
\rowcolor[HTML]{FFFFFF} 
\cellcolor[HTML]{FFFFFF} & \cellcolor[HTML]{EFEFEF}Inclusion of the Marginalized \\
\rowcolor[HTML]{FFFFFF} 
\cellcolor[HTML]{FFFFFF} & Center Remote \& Rural Communities \\
\rowcolor[HTML]{FFFFFF} 
\multirow{-5}{*}{\cellcolor[HTML]{FFFFFF}Center All Communities} & \cellcolor[HTML]{EFEFEF}Center Women \\ \hline
\rowcolor[HTML]{FFFFFF} 
\cellcolor[HTML]{FFFFFF} & Universal Dignity \\
\rowcolor[HTML]{FFFFFF} 
\cellcolor[HTML]{FFFFFF} & \cellcolor[HTML]{EFEFEF}Common Good \\
\rowcolor[HTML]{FFFFFF} 
\multirow{-3}{*}{\cellcolor[HTML]{FFFFFF}Uphold Universal Good} & Harmony \\ \hline
\rowcolor[HTML]{FFFFFF} 
\cellcolor[HTML]{FFFFFF} & \cellcolor[HTML]{EFEFEF}Consensus-Building \\
\rowcolor[HTML]{FFFFFF} 
\cellcolor[HTML]{FFFFFF} & Reciprocity \\
\rowcolor[HTML]{FFFFFF} 
\cellcolor[HTML]{FFFFFF} & \cellcolor[HTML]{EFEFEF}Resolving Data-Driven Harms \\
\rowcolor[HTML]{FFFFFF} 
\multirow{-4}{*}{\cellcolor[HTML]{FFFFFF}Communalism in Practice} & Fair Collaboration \\ \hline
\rowcolor[HTML]{FFFFFF} 
\cellcolor[HTML]{FFFFFF} & \cellcolor[HTML]{EFEFEF}"For Africans, By Africans" \\
\rowcolor[HTML]{FFFFFF} 
\cellcolor[HTML]{FFFFFF} & Treasure Indigenous Knowledge \\
\rowcolor[HTML]{FFFFFF} 
\multirow{-3}{*}{\cellcolor[HTML]{FFFFFF}Data Self-Determination} & \cellcolor[HTML]{EFEFEF}Data Sovereignty \& Privacy \\ \hline
\rowcolor[HTML]{FFFFFF} 
\cellcolor[HTML]{FFFFFF} & Measured Development \\
\rowcolor[HTML]{FFFFFF} 
\cellcolor[HTML]{FFFFFF} & \cellcolor[HTML]{EFEFEF}Technical Infrastructure \\
\rowcolor[HTML]{FFFFFF} 
\cellcolor[HTML]{FFFFFF} & Governance Infrastructure \\
\rowcolor[HTML]{FFFFFF} 
\multirow{-4}{*}{\cellcolor[HTML]{FFFFFF}\begin{tabular}[c]{@{}c@{}}Invest in Data Institution \\ \& Infrastructures\end{tabular}} & \cellcolor[HTML]{EFEFEF}Support Formal \& Informal Collectives \\ \hline
\rowcolor[HTML]{FFFFFF} 
\cellcolor[HTML]{FFFFFF} & Holistic Education \\
\rowcolor[HTML]{FFFFFF} 
\multirow{-2}{*}{\cellcolor[HTML]{FFFFFF}Prioritize Education \& Youth} & \cellcolor[HTML]{EFEFEF}Youth Empowerment and Protection \\ \hline
\end{tabular}%
}
\caption{Overview of the major principles and minor principles of our proposed African data ethics framework.}
\label{tab:framework-overview}
\end{table}

\subsection{Decolonize \& Challenge Internal Power Asymmetry}
Challenging power structures in technological development is not only necessary to mitigate the perpetuation of colonial power legacies, but also misuse and exploitation by any authority. 

\textbf{Challenge Colonial Power.}
\label{sec:chall_colo}
RDS practices from the West do not seamlessly transfer to the African context
because these practices are developed within colonial contexts disconnected from the realities of African practitioners and users \cite{eke2022forgotten, shilongo2023creativity,adelani2022masakhaner,gwagwa2019recommendations,eke2023introducing,okolo2023responsible, eke2023towards, goffi2023teaching,carman2023applying}. African practitioners identify three dimensions in which colonialism and imperialism limit RDS: epistemic injustice, dehumanizing extraction, and dependent partnerships. Firstly, African scholars identify trends in philosophical epistemic injustice permeating global data ethics paradigms \cite{eke2022forgotten, metz2021african, olojede2023towards}. As many African philosophers agree, Enlightenment ideals (a premier part of the Western philosophical canon) were predicated on colonialism and racism \cite{gwagwa2019recommendations}. Africans were deemed incapable of rational thinking by Western colonizers, so through the Enlightenment principle of rationality, anti-blackness was justified \cite{lauer2017african}. Furthermore, colonization was not only excused but encouraged by rationality so colonizers could develop Africans through Western instruction. Under colonial rule, Africans were taught to abandon their Indigenous knowledge to adopt the rational knowledge of the West. The legacy of colonialism is why African data scientists encourage casting aside Western perspectives to develop African RDS perspectives \cite{mhlambi2020from}. Additionally, an over reliance on performance metrics encourages the same colonial blindspot that excuses and encourages the marginalization of Africans in technology such as facial recognition \cite{mhlambi2020from, buolamwini2018gender, cisse2018look, gwagwa2022role}. 

Secondly, many documents recognize that most African contributions to data science disproportionately benefit Western corporations like OpenAI, Google, Meta, and Microsoft \cite{ndjungu2020blood,chan2021limits,abebe2021narratives, nwankwo2019africa, kiemde2022towards}. The computing demand of large-data systems such as AI proliferates neocolonialism to new heights in Africa \cite{eke2023towards}. The work of Africans within the data science ecosystem should benefit Africans first \cite{kohnert2022machine}. The fact that it currently does not is connected to the legacy of colonialism and chattel slavery in which Africans were forced to extract their raw materials so colonialists could fuel industrialization and capitalism in their home countries \cite{mhlambi2020from, ndjungu2020blood, day2023data, shilongo2023creativity, birhane2020algorithmic}.

Finally, the last vestige of colonial power 
to be challenged in African RDS is dependent partnership. Africa currently lacks the technical infrastructure for large-scale DDS, which pushes data scientists towards unfair agreements with powerful organizations to gain technology \cite{shilongo2023creativity, hountondji2004producing, osaghae2004rescuing}. Even worse, companies such as Amazon, Google, Meta, and Uber use savior language such as ``liberating the bottom million'' to describe their digital services in Africa \cite{abebe2021narratives}. 

\textbf{Challenge Internal Power Asymmetry.} 
\label{sec:chall_in}
DDS should not be used to oppress the freedoms of citizens or perpetuate government corruption. 
Critical African philosophers view authoritarianism as governing to accumulate wealth and power rather than serving the needs of their citizens and Africa as a whole \cite{nyerere1962ujamaa}.  
Even after liberation from colonial rule, some African philosophers accuse their governments of being primarily concerned with replacing the colonial ruling class instead of dismantling it \cite{coetzee2004laterMarx, kohnert2022machine}. To maintain their position, government officials focus on maintaining dependent relationships with the West and enforcing cultural nationalism to suppress dissent \cite{gwagwa2019recommendations}. 

African governments have already harnessed their control of national technology through internet shutdowns \cite{okolo2023responsible}. Therefore, to many authoritarian actors, powerful data technology is just another tool for suppression. Of particular concern to many African practitioners is China as a neocolonial collaborator with African authoritarians. Chinese companies have been found to provide the data technology Ethiopia, Uganda, and Zimbabwe have used to surveil their citizens \cite{okolo2023responsible}.

While authoritarian uses of data technology are resolutely unethical, more widely accepted uses of government DDS are scrutinized as well. 
The ubiquitous deployment of a digital ID system forces citizens to choose between access to important services or preserving their privacy from a system they have no control over \cite{gwagwa2019recommendations}. 
As governments consider adopting data technology, they need to be accountable to their citizens \cite{ade-ibijola2023artificial,osaghae2004rescuing}. 
To combat the misuse of government power, DDS should improve government efficiency, transparency, and enforcement of citizens' freedom \cite{gwagwa2019recommendations, mabe2007security, eke2023towards}. 

\subsection{Center All Communities} 
Community involvement ensures DDS consider the needs and potential impacts of communities beyond the end-users. 

\textbf{Community in Everything.}
\label{sec:com_every}
Akan philosophies regard the community as an invaluable resource that guides how every individual lives \cite{wiredu2004akan,metz2021african, coetzee2004particularity, mhlambi2023decolonizing,gwagwa2022role}. Therefore community input is crucial for constructing a full picture of technical requirements, especially in high-stakes domains \cite{sinha2023principlesafrofeminist,mhlambi2020from, eke2023towards}.
The concept of community can be misappropriated to deem any collection of stakeholders as sufficient community representatives. African communitarian ethics define a community as individuals with a shared identity who are emotionally invested in each other \cite{nwankwo2019africa,sinha2023principlesafrofeminist, ruttkampbloem2023epistemic, gyekye2004person}. 
With this more narrow definition of community, involving affected communities in all stages of the lifecycle requires building trust and respecting boundaries by gaining an understanding of cultural norms \cite{abebe2021narratives, ade-ibijola2023artificial}. Additionally, community members should be sufficiently trained or educated on the nature of the technology so they can provide well-informed input \cite{shilongo2023creativity,adelani2022masakhaner, plantinga2024responsible}. Rather than checking off a list, community-centered data science work should be conducted as a co-creation process in which all stakeholders depend on each other \cite{langat2020how, kohnert2022machine, abebe2021narratives, adelani2022masakhaner,nwankwo2019africa, lauer2017african, kiemde2022towards}.

\textbf{Solidarity.}
Solidarity is understood as looking out for other diverse communities based on mutual respect and the goal of social cohesion \cite{gwagwa2019recommendations, mhlambi2020from}. In Ubuntu understanding, solidarity is a deep care for others, including people of the past, present, future, and the environment \cite{mhlambi2023decolonizing,gwagwa2019recommendations, okolo2023responsible, dignum2023responsible, gwagwa2022role}. With this perspective, DDS should be developed not just with the end user in mind but all the other communities who could be impacted by the technology
\cite{gwagwa2022role,olojede2023towards,gyekye2004person}.
Solidarity violations between African countries is of particular concern. The success of one African community should not be predicated on the suffering of another \cite{ndjungu2020blood, biko2004black}. Upholding solidarity means that all actions made in the data science lifecycle should explicitly protect or improve the lives of vulnerable or marginalized communities. 

Exploiting the vulnerability of another is not only unethical but unsustainable due to our interconnected nature. The suffering of one community will eventually lead to the destruction of all communities \cite{nwankwo2019africa}.  
Banding together, ``watching one another's back'', and developing DDS as one big family is key to mitigating harm \cite{olojede2023towards,nyerere1962ujamaa}. 

\textbf{Inclusion of the Marginalized.}
\label{sec:inclusion}
While Africa needs to be included in global data science efforts, Africa itself is full of diverse communities that should also be represented in African data science efforts \cite{adelani2022masakhaner,gwagwa2019recommendations,goffi2023teaching}. 
African communities' underrepresentation in datasets across all data science tasks is due to, Gwagwa as described, being uncounted, unaccounted, and discounted \cite{gwagwa2019recommendations}. Leaving communities out of data also excludes them from the benefits DDS provide \cite{gwagwa2022role}. Given the need to build explicitly African DDS, the lack of African datasets is a threat to efficacy \cite{okolo2023responsible,olojede2023towards, ade-ibijola2023artificial}.
Including marginalized communities requires mutual respect for diverse perspectives and creating procedures such as impact assessments to provide opportunities for inclusive input \cite{african_union2024continental, abebe2021narratives, goffi2023teaching, mhlambi2020from}. In addition, it’s important to challenge the social, political, and economic dynamics that push communities to the margins in the first place \cite{kiemde2022towards, olojede2023towards,segun2021critically,day2023data,uzomah2023african,abebe2021narratives}.
 
\textbf{Center Remote \& Rural Communities.}
Development, especially technical development, is usually focused in urban centers and excludes remote and rural communities \cite{ade-ibijola2023artificial, sinha2023principlesafrofeminist, osaghae2004rescuing}. Given the lack of infrastructure in remote and rural communities, data technology should be used to develop and optimize infrastructures and public services for these regions \cite{carman2023applying, african_union2024continental}. 
However, it’s important to keep in mind that RDS done on behalf of rural and remote communities that do not consider their culture, livelihoods, and direct input can lead to harm \cite{african_union2024continental, ndjungu2020blood, carman2023applying}.

\textbf{Center Women.}
\label{sec:center_women}
Due to the prevalence of patriarchy in many African societies, there is a need to encourage the agency of women in DDS efforts. A few documents suggest that women-led technology businesses and the education of women and girls should be incentivized \cite{african_union2024continental}. However, open questions remain about how to maintain African women's participation in a field known to be male-dominated and antagonistic to women \cite{african_union2024continental, gwagwa2019recommendations}. 

Afro-feminists have a response to the techno-chauvinism that dominates data science \cite{sinha2023principlesafrofeminist}. Rather than centering women in general, there must be a recognition of the intersectional status of African women \cite{sinha2023principlesafrofeminist}. As articulated by Rosebell Kagumire, African women experience domination through systems of patriarchy, race, sexuality, and global imperialism \cite{dieng2023speaking, coetzee2004particularity}. Therefore, DDS should be developed with the complex needs of African women in mind, because their compounded experiences of marginalization provide insight into the needs of various oppressed populations \cite{sinha2023principlesafrofeminist}. 
There are numerous examples of African women harnessing the internet to fill in the gaps of an oppressive society and data technology holds similar potential \cite{dieng2023speaking}.
For example, Chil AI Lab Group is a women-led data science collective that is successfully using data technology to address the often neglected health needs of women in Africa \cite{eke2023towards}. 

\subsection{Uphold Universal Good}
Ethical development and deployment of DDS requires a commitment to upholding fundamental human dignity and ensuring these technologies benefit all. 

\textbf{Universal Dignity.}
Every human and community deserves humane treatment, and DDS should never violate their dignity \cite{olojede2023towards,mhlambi2020from}. The African Charter on Human and Peoples’ Rights and the Universal Declaration on Human Rights set precedent for the just treatment of humans \cite{african_union2024continental}. Regardless of these laws, African philosophies necessitate respect for human dignity because humans should be inherently valued for their existence and connection to others \cite{segun2021critically,metz2021african, dignum2023responsible}. Every human must be treated with respect, care, and concern for their well-being \cite{wiredu2004akan,dieng2023speaking,coetzee2004particularity, gyekye2004person, wiredu2004moralfoundations}. 
In applying the principle of universal dignity to RDS practices, every person involved in the data lifecycle should be respected.
Individuals should not be used as a means to execute data work \cite{metz2021african,ramose2004struggle}. Rather, all efforts should be taken to ensure their well-being and dignity are 
preserved when asked to contribute to DDS \cite{gwagwa2019recommendations,abebe2021narratives}. This same respect also extends to communities. Collective agreements need to be honored, and collective work or resources should not be used in a manner that threatens the well-being of the community \cite{moahi2007globalization}. While this principle is self-evident, there are many cases in which the rights of Africans were violated for large-scale DDS
\cite{african_union2024continental, segun2021critically, kohnert2022machine, moahi2007globalization}.

\textbf{Common Good.}
DDS should contribute to maintaining the safety, health, and goodness of all \cite{olojede2023towards}.
In various African philosophies, a person is defined by their commitment to acting for the benefit of those around them \cite{african_union2024continental, nwankwo2019africa, mhlambi2023decolonizing,coetzee2004particularity,ruttkampbloem2023epistemic, gyekye2004person,abdul2023transhumanism, wiredu2004moralfoundations}. 
In terms of RDS, they have to be made with the explicit goal of improving society and dismantling systemic harms \cite{sinha2023principlesafrofeminist, okolo2023responsible, eke2023towards}.  
In Africa, improving the efficacy of agriculture practices, healthcare access, responsiveness of public services, and the security of financial services are over-arching priorities \cite{carman2023applying,kohnert2022machine}. Achieving common good involves incorporating collective values early in the process \cite{langat2020how,dignum2023responsible}, guiding development with regulatory toolkits \cite{olojede2023towards}, not focusing on individualistic profit maximization \cite{gwagwa2019recommendations,segun2021critically,mabe2007security,nyerere1962ujamaa, mhlambi2020from,dieng2023speaking}, and encouraging the open sharing of data \cite{gwagwa2019recommendations,abebe2021narratives,day2023data}. Building DDS toward the common good should be the ultimate goal for RDS \cite{carman2023applying, metz2021african, olojede2023towards, mabe2007security}. 

\textbf{Harmony.}
DDS should further the mutual well-being of all stakeholders. In addition, data standards and frameworks will be most effective when they harmonize with each other \cite{gwagwa2019recommendations, kiemde2022towards, wareham2021artificial, mabe2007security}.
In many African philosophies, harmony is not a state but a dynamic and reciprocal process of calibrating one's actions in response to changes in the environment. In Ubuntu ethics, dogmatism is rejected because it impedes individuals from acting in harmony with the changing world \cite{ramose2004ethicsofubuntu}. In Akan philosophy, morality is defined as acting in line with collective human interests \cite{wiredu2004moralfoundations}. 
Upholding harmony in data science can be understood on two dimensions: impact and practice. 
Data should be harnessed to bring people closer to their environment so they can act in the best interests of not only themselves but also those around them. In terms of practice, data ethics frameworks are most effective when all the elements of data science work are accounted for \cite{kiemde2022towards,gwagwa2019recommendations}. Also, acknowledging the unique ethical needs at each stage of the data science lifecycle can inform an adaptable practice of RDS. As Gwagwa et al. assert, the harmonious practice of RDS in Africa requires country-level data ethics frameworks to be in alignment with frameworks developed at the continental level \cite{gwagwa2022role}. 
\subsection{Communalism in Practice} 
The development and deployment of DDS should mitigate harms, involve communities in decision-making, and ensure reciprocal benefits for African stakeholders. 

\textbf{Consensus-Building.} 
If data scientists want to develop responsible practices and encourage effective collaboration, consensus-building is a well-practiced strategy from African communities \cite{wiredu2004moralfoundations}. Rather than the majority rule common in Western societies, African elders discuss issues until they all agree on a final decision \cite{wiredu2004akan,carman2023applying}. Achieving consensus requires the final decision to be 1) the dominant view of the group, 2) in line with the common good, and 3) aligned with the morals of the individual parties \cite{coetzee2004particularity}. Consensus should be broached in an environment of trust, practical reason, humility, openness, and respect for the viewpoints of all involved parties \cite{coetzee2004particularity,nwankwo2019africa,gwagwa2019recommendations,mhlambi2020from, okolo2023responsible, gwagwa2022role}. 

Community engagement provides spaces for consensus in the data science lifecycle to include more perspectives \cite{day2023data,mabe2007security}. Consensus processes should also include procedures for documentation to keep track of disagreements, dissenting opinions, and the progression of project values \cite{kling2023role}. It is not easy to achieve these conditions, so conflict management, negotiation, and reasonable bargaining are helpful mechanisms to fully consider and resolve contradicting positions \cite{gwagwa2019recommendations, osaghae2004rescuing, dieng2023speaking}. Consensus should be a dynamic feedback loop to ensure every contributor is on the same page about the team's approach to RDS \cite{segun2021critically, nwankwo2019africa, uzomah2023african, kohnert2022machine, abebe2021narratives, gwagwa2019recommendations, dignum2023responsible}. The actual process of consensus-building is also a helpful mechanism to build trust between data collaborators and develop informed consent from future users \cite{nwankwo2019africa,kohnert2022machine,abebe2021narratives}.

\textbf{Reciprocity.}
In many African philosophies, reciprocity is the foundation of a healthy society. In Akan society, practicing reciprocity ensures that community needs are met, while building deep social bonds \cite{wiredu2004moralfoundations}. African perfectionist proponents go as far as to assert that assisting others in achieving their goals makes someone more of a person \cite{wareham2021artificial,wiredu2004moralfoundations}. Without reciprocity, society becomes imbalanced and co-dependent \cite{coetzee2004particularity,mhlambi2023decolonizing,nyerere1962ujamaa}. 
There are numerous examples of African data subjects not reaping any benefits from the data collaborations they participate in \cite{gwagwa2019recommendations,abebe2021narratives}. This often leads to technically mediated harms while the controllers of data amass profits \cite{gwagwa2019recommendations}. 
Therefore, sustainable DDS should practice reciprocity on several dimensions \cite{wiredu2004moralfoundations}. If someone contributes to a DDS they should meaningfully benefit from the system or project \cite{mhlambi2020from, gyekye2004person, sinha2023principlesafrofeminist}. Inspired by philosophies such as Ubuntu or Ujamaa, DDS should operate in a manner that benefits the society in which they are created and deployed \cite{eke2023towards, adelani2022masakhaner,dignum2023responsible}. 

\textbf{Resolving Data-Driven Harms.}
When disagreements, conflict, or harm occur at any stage of the data science lifecycle, we need mechanisms of accountability and reconciliation to correct wrongs and empower those impacted. In African societies, harm is not just actively making someone's life worse but also neglecting obligations to the community \cite{wiredu2004akan, gyekye2004person}. A person who causes harm is viewed as a moral failure who must be corrected by their community through sanctions and even mental rehabilitation to correct deeper issues connected to their poor actions \cite{wiredu2004moralfoundations, coetzee2004particularity}. Even the most powerful members of society, such as chiefs, are subject to correction and even dismissal by their community \cite{wiredu2004akan}. 

The adoption of AI and other DDS have already caused harm to African populations by way of data bias, socio-economic risk, and privacy violations \cite{ade-ibijola2023artificial}. 
There are African data ethicists who stress the need to develop procedures for communities and individuals harmed by DDS to seek restitution \cite{gwagwa2019recommendations,mhlambi2023decolonizing}. These solutions are dependent on African governments and external multinational organizations committing to transparency, equality, and restorative practices \cite{gwagwa2019recommendations,african_union2024continental,okolo2023responsible, dignum2023responsible, kiemde2022towards}. African governments can mitigate data harm by being transparent about their potential data collaborations, outlining their plans for data protection before, during, and after the deployment of DDS, and enforcing mechanisms of accountability and dissent from their citizens \cite{shilongo2023creativity, sinha2023principlesafrofeminist}. 

Similar to the dismissal of chiefs, powerful stakeholders acting outside of their agreed duties, must experience restorative consequences, not just a slap on the wrist \cite{mandaza2004reconciliationzimbabwe, ndjungu2020blood, shilongo2023creativity, mhlambi2023decolonizing, langat2020how, gwagwa2019recommendations, mhlambi2020from, biko2004black, coetzee2004laterMarx}.

\textbf{Fair Collaboration.}
Given the current gap between Africa's AI readiness and growing interest in AI adoption, many concede external partnership as a necessity\cite{eke2023introducing, african_union2024continental}. However, exploitative external relationships set a precedent that curtails African self-determination in data science work \cite{ndjungu2020blood,sinha2023principlesafrofeminist}. When building relationships, there are established obligations that each collaborator owes to the other \cite{coetzee2004particularity, metz2021african}. Data collaborations need to be predicated on trust, fair attribution of work, and a commitment to prioritizing the agency of African collaborators \cite{adelani2022masakhaner,nwankwo2019africa,abebe2021narratives,gwagwa2019recommendations, wareham2021artificial, wiredu2004moralfoundations}. 

\subsection{Data Self-Determination}
\label{sec:Data Self-Determination}
African data science should be an avenue for bolstering the self-determination of Indigenous African communities.

\textbf{``For Africans, By African''.}
\label{sec:fubu} 
This principle is inspired by the concerted efforts of African data scientists to reclaim leadership in African data science work \cite{chan2021limits}.
To combat deficit-based narratives about Africa, African data scientists need to reclaim and celebrate their strength, rich cultures, and scientific achievements in conducting RDS \cite{abebe2021narratives,adelani2022masakhaner,hountondji2004producing, coetzee2004particularity, lauer2017african,gwagwa2019recommendations, carman2023applying}. The diverse values and perspectives of African communities should ground the development of African data ethics \cite{coetzee2004african,segun2021critically, african_union2024continental, ruttkampbloem2023epistemic, gwagwa2022role, dignum2023responsible, olojede2023towards}. 
Given the thousands of cultures that comprise Africa, the potential for novel approaches to data science must be explored \cite{eke2022forgotten,goffi2023teaching,coetzee2004laterMarx, shilongo2023creativity,day2023data,kohnert2022machine}.  

African data should not be primarily collected for Western tech powers or published for immediate and uncontrolled use \cite{birhane2020algorithmic,hountondji2004producing}. African data practitioners do not need tech superpowers to speak for Africans on the global stage, provide pre-trained models, or ensure work meets the standards of Western data institutions, Africans are more than capable of leading without interference \cite{ndjungu2020blood,abebe2021narratives,goffi2023teaching,mhlambi2023decolonizing, okolo2023responsible, ade-ibijola2023artificial, biko2004black}. This does not mean Africans should not collaborate with external data practitioners and vice versa \cite{hountondji2004producing, eke2023introducing}. 
Rather, local African data practitioners must lead data science work so its development is properly situated in the communities it will be used \cite{nwankwo2019africa,lauer2017african,kiemde2022towards, eke2023towards}.  


\textbf{Treasure Indigenous Knowledge.}
\label{sec:treasure_ik}
DDS should preserve, center, and continue the development of Indigenous knowledge. With the legacy of colonial epistemic injustice, African modernization and Indigenous knowledge preservation are often viewed as at odds with each other \cite{african_union2024continental,kohnert2022machine,eke2023introducing}. On the contrary, many documents hold Indigenous knowledge as a pivotal component of RDS in Africa. 

DDS can be used to store Indigenous languages, customs, and history in close consultation with Indigenous communities. However, some are concerned that joining a globalized data ecosystem will lead to a loss of culture and identity \cite{african_union2024continental, abebe2021narratives,eke2023towards, ade-ibijola2023artificial}. As elders, griots, and other stewards of Indigenous knowledge pass,
younger generations have to take on the responsibility of preserving their community's culture
\cite{kotut2024griot,ramose2004struggle}. 
There are over 1500 languages indigenous to Africa, but very few are represented in data technology, such as natural language processing (NLP), which leaves out large portions of Africans from using technology \cite{shilongo2023creativity}. Pre-colonial Indigenous knowledge needs to be reclaimed to develop African data values that reflect local communities \cite{abdul2023transhumanism, chan2021limits}. Local communities can never fully be represented if there is not an understanding of their roots or history \cite{ramose2004struggle}. Building datasets that represent Indigenous languages for inclusive models opens a whole set of new users who can digitally store and analyze Indigenous knowledge that is typically shared orally for future generations \cite{shilongo2023creativity,moahi2007globalization}. 
The boundaries on what Indigenous knowledge should be a part of DDS must be understood by consulting with the community before proceeding on any project \cite{kotut2024griot,moahi2007globalization}.

African philosophers emphasize Indigenous knowledge isn’t limited to the past \cite{hountondji2004producing}. Investing in African RDS is an investment in creating new Indigenous knowledge \cite{uzomah2023african,lauer2017african}. Local talent does not have to reinvent the wheel to explore open questions in the more recent field of data science \cite{lauer2017african,mabe2007security, moahi2007globalization}. The richness of African knowledge can develop new RDS practices and understandings \cite{abdul2023transhumanism, mhlambi2020from, biko2004black, coetzee2004laterMarx}. 

\textbf{Data Sovereignty \& Privacy.}
\label{sec:data_ sov}
Mechanisms must be developed to protect African creativity and privacy in the development of DDS. Given the legacies of extractive colonialism, ownership is viewed as the key to data sovereignty in Africa \cite{gwagwa2019recommendations,shilongo2023creativity,kiemde2022towards}. African ownership in the data science process can be achieved by codifying intellectual property rights \cite{african_union2024continental}, enforcing data ownership \cite{shilongo2023creativity}, and exploring Indigenous conceptions of collective privacy \cite{nwankwo2019africa,goffi2023teaching,mabe2007security,langat2020how, moahi2007globalization}. 

Africans are often regarded as ``simply'' data subjects \cite{shilongo2023creativity}. The role of a data subject is materially essential to data science work (without data, nothing can be done). The narratives of collecting as much data as possible to achieve generalizability devalue data subjects as dehumanized resources \cite{gwagwa2019recommendations,birhane2020algorithmic,olojede2023towards, mhlambi2020from, ndjungu2020blood}. This devaluing encourages data collectors to share and use data without any knowledge, consent, or compensation of data subjects \cite{sinha2023principlesafrofeminist, nyerere1962ujamaa}. Many African data ethicists call for a correction of this narrative to recognize data subjects as the proper owners of their data by shifting power and access control to data subjects \cite{day2023data,ruttkampbloem2023epistemic, abdul2023transhumanism}. Achieving this shift in ownership should be done by demanding data-sharing terms and not working with data collaborators who do not honor these terms \cite{biko2004black, okolo2023responsible}. 
African ownership of data, resources for data science, and technical contributions should be non-negotiable for RDS.


\subsection{Invest in Data Institutions \& Infrastructures.}
Prioritizing infrastructure, investing in people, and establishing sound policy and governance frameworks should be measured to not deter social progress in the name of technological progress.

\textbf{Measured Development.}
The development of data science ecosystems should be balanced, measured, inclusive, community-minded, and holistic \cite{kohnert2022machine, eke2023towards, ade-ibijola2023artificial}. Without this approach, the adoption of AI and other data technologies can lead to more unrest and inequality across Africa.
To many, the potential of DDS is profound and would change the trajectory of African development \cite{african_union2024continental, mabe2007security}. Data are viewed as the driving resource for the Fourth Industrial Revolution \cite{carman2023applying, gwagwa2019recommendations}. There are African data scientists and governments who insist joining the AI boom will provide Africa the quality of life benefits afforded to the major players of past industrial revolutions \cite{okolo2023responsible, kohnert2022machine, coetzee2004laterMarx}. 
However, there is skepticism about wholeheartedly diving into large-scale data science adoption \cite{uzomah2023african, olojede2023towards}. 
There is a need to quell the AI hype as the solution for all African problems and consider who will actually be served: Africans or the external powers propelling the AI boom \cite{wareham2021artificial,birhane2020algorithmic, sinha2023principlesafrofeminist}. 
Through the paradigm of measured development, technical development should move at the pace of social development \cite{kiemde2022towards, nyerere1962ujamaa}. Paulin Hountondji's critique of science in Africa applies well to data science development. Development should not be driven by ``scientific extroversion'' or catching up with the West \cite{hountondji2004producing,goffi2023teaching}. Rather, the development of data science should be an investment in the progress of African people based on African intellect, priorities, and visions of the future \cite{shilongo2023creativity,biko2004black}.

\textbf{Technical Infrastructure.}
\label{sec:tech_infra}
To implement DDS in Africa, practitioners call for investment in physical data science infrastructure, assessment of the current capacities of technical infrastructure, and development of responsible data management practices \cite{moahi2007globalization, ruttkampbloem2023epistemic}. Achieving this principle in Africa is a big feat when electricity and broadband access is not only sparse but one of the most costly to access in the world \cite{okolo2023responsible, ade-ibijola2023artificial}.  
Nigeria, Mozambique, and Rwanda have recognized the need to invest in technical infrastructure and have partnered with external tech companies and international financial institutions to build their respective capacity to host DDS \cite{okolo2023responsible}. 
There are also innovative ways to work with current technical infrastructure to lessen reliance on external investment \cite{mhlambi2020from}. 
Technical infrastructure development should also coincide with the development of responsible data management protocols so that African data and data science work are not vulnerable to dispossession \cite{abebe2021narratives, gwagwa2019recommendations}. 

\textbf{Governance Infrastructure.}
We need sustainable and measured governance infrastructure to guide the development of DDS \cite{african_union2024continental, chan2021limits}. Policy measures and regulations are major priorities for African data science communities to guide RDS practices. African Union member states are slowly developing data protection regulations, but many documents stress the urgency for African data policy \cite{kiemde2022towards, plantinga2024responsible}.
Without clear policies and legal standards for RDS, African data scientists lack guidance in their practices leaves African communities vulnerable to data exploitation from external and internal actors alike \cite{abebe2021narratives,cisse2018look, mandaza2004reconciliationzimbabwe}. 
Governance infrastructures include incremental regulations \cite{gwagwa2019recommendations}, monitoring bodies \cite{goffi2023teaching}, continental commitments \cite{african_union2024continental}, and algorithmic impact assessments \cite{sinha2023principlesafrofeminist}. 


\textbf{Support Formal \& Informal Collectives.}
Capacity-building in Africa necessitates the support of diverse data science collectives~\cite{okolo2023responsible, abebe2021narratives}. 
81\% of jobs in Africa are based in informal economies \cite{shilongo2023creativity}. As such, only focusing on supporting data science research (in which wider recognition and acceptance is a common issue related to epistemic injustice \cite{eke2022forgotten,chan2021limits}) neglects a large portion of potential data collaborators \cite{hountondji2004producing}. There should be efforts to connect Africans interested in using data science for entrepreneurship \cite{biko2004black,shilongo2023creativity} and accessible data science job training \cite{abebe2021narratives}. However, these collectives should not be siloed. The boundary between formal and informal data organizations should be dismantled to exchange technical knowledge, coordinate work, and pool resources \cite{dieng2023speaking, kling2023role}. 
Both forms of data collectives have important functions and need to rely on each other to flourish. One form of collective is not meant to replace the other \cite{osaghae2004rescuing}. If both of these collectives are not supported, African data scientists will have to seek support outside of their communities, which furthers the ``brain drain'' of highly skilled Africans to the West \cite{okolo2023responsible}. Investing in collectives also builds a workforce for in-house development which reduces foreign dependence \cite{kiemde2022towards, carman2023applying, plantinga2024responsible}. 


\subsection{Prioritize Education \& Youth}
Youth involvement and education are essential for ensuring the continued development and implementation of ethical data science practices that respect cultural contexts, African philosophy, and Indigenous knowledge.

\textbf{Holistic Education.}
The African population has low attainment of digital skills \cite{okolo2023responsible, ade-ibijola2023artificial}. As large foreign technology companies set root in Africa, policymakers stress the need for monumental efforts to train local talent \cite{african_union2024continental,shilongo2023creativity, mhlambi2020from, cisse2018look}. Providing technical skills early in education will help prepare a strong cohort of future data scientists \cite{sinha2023principlesafrofeminist,nwankwo2019africa}. There should also be investments in integrating AI curricula in informal organizations like the Data Values Project to reduce educational barriers \cite{shilongo2023creativity}. 
Importantly, an indispensable part of a comprehensive data science education is data ethics \cite{goffi2023teaching, kiemde2022towards, ramose2004struggle}. An United Nations Educational, Scientific and Cultural Organization (UNESCO) survey found that very few African countries feel equipped to contend with the ethical implications of AI \cite{kiemde2022towards}. Teaching data ethics in Africa should involve centering the lived experiences and culture of the students \cite{goffi2023teaching,kiemde2022towards}. Students should be educated about the common dangers of data science and also develop their ethical discernment to prepare them for the sociotechnical complexities of data science.  


\textbf{Youth Empowerment \& Protection.}
Africa is a young continent with a large population of educated and digitally native youth \cite{nwankwo2019africa, goffi2023teaching}. 
Prioritizing the youth of Africa is a two-pronged principle: 1) protect young people from harm and 2) empower youth to lead data science agendas. 
The youngest generation has a tech-savviness that can be transferable to data science \cite{african_union2024continental,birhane2020algorithmic,abebe2021narratives}. 
If youth are expected to be the first adopters of African DDS on a large scale then these systems should be designed to protect youth so they cannot be taken advantage of. DDS should enrich the development of African youth and empower them to innovate, imagine, and contribute to bettering the communities they are a part of. 
Their comfort with technology may lead them to uncritically adopt a ``move fast and break things'' approach \cite{abebe2021narratives, ruttkampbloem2023epistemic}. To address these concerns, data science work should be intergenerational. 
