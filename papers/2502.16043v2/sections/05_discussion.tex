\section{Comparative Analysis with Particularist Data Ethics Frameworks}
\label{sec:compare}
We compare our framework to six particularist frameworks to gauge the breadth of our work alongside:
intersectional feminists \cite{klein2024data}, Western technology powers \cite{floridi2018ai4people}, Indigenous communities of Turtle Island \cite{carroll2021operationalizing}, policymakers from global superpowers \cite{jobin2019global}, Central Asian data scientists \cite{younas2024proposing}, and Muslim data scientists \cite{raquib2022islamic}. The first author reviewed each framework and used the minor principles of the African data ethics framework as a codebook (see Appendix \ref{apdx:compare_table} for comparative coding results).

Each framework covered, at most, 65\% of our proposed 23 principles. Delving into the diverging and converging principles provides insight into where African data ethics fits in the global data ethics discourse. 
The six frameworks did not discuss \textit{centering remote and rural communities} or \textit{prioritization of education \& youth}. 
Our framework may have highlighted these communities because they represent a significant portion of Africa and have unique needs not fully met by the status quo \cite{sanny2023africas, barrett2017structural}.\footnote{\url{https://www.theeastafrican.co.ke/tea/news/east-africa/police-army-block-ugandan-opposition-headquarters-4698696}} 
In line with African philosophical conceptions of particularism, every community has distinct shared experiences that inform their values and normative understandings of the world. Engaging with data ethics from different cultural standpoints exposes data scientists to approaches or potential harms they would have never considered \cite{adamu2021rethinking}.

\textit{Common good} was the only principle covered by all the frameworks. This finding also falls in line with a popular understanding of philosophy: universalism. An example of universal theory is natural law. Natural law is the legal manifestation of universalism and asserts that there are certain rights afforded to every human being\cite{gordon2022universalism}. The variety of charters, committees, and trials led by the international legal ecosystem are guided by a commitment to upholding natural law universalism. 
The whole RDS community is broadly guided by a universalist commitment to social good \cite{floridi2018ai4people}. However, upon closer review, references to the common good in non-African frameworks often remain abstract, lacking the depth and specificity seen in the work of African data ethicists.
African appeals to universalism are grounded in the tradition of speaking truth to power, emphasizing the need to address systemic inequities perpetuated by RDS practitioners. 
Drawing from critiques of international human rights law, African scholars highlight the importance of pairing narrative restoration—the recognition and reclamation of African humanity—with material restoration, including tangible reparative actions \cite{biko2004black, oyowe2014african, gordon2022universalism}. African critiques of universalism informs why African data ethicists engage with the concept of social good in ways that are both urgent and deeply pragmatic. 


\section{African RDS Case Studies: An Exercise of Framework Application}
The primary aim of our framework is to translate its normative values and guidelines into an effective moral decision-making procedure for RDS. Moral decision-making procedures enable individuals to assess a range of choices and select actions that align with intended outcomes while upholding core values \cite{smith2022making}. Our framework is oriented toward advancing effective and RDS practices in Africa. The major principles describe the overarching values that should be upheld throughout the whole data science lifecycle (community benefit, challenging exploitation, and prioritizing African agency). The minor principles outline the normative practices to yield intended results and maintain the outlined values. For example, to \textit{Uphold the Universal Good} in African RDS, our framework recommends proactively instituting humane work practices (\textit{Universal Dignity}), evaluating the collective impact of technology (\textit{Common Good}), and encouraging balance in RDS design and practices (\textit{Harmony}). With our framework, we can evaluate how decisions made in African data science contexts align with the principles derived from our thematic analysis. The following section discusses two case studies from data science collectives in Africa to demonstrate how our framework can be used to evaluate responsible decision-making.
To view more case study discussions, see Appendix \ref{apdx:more_case_studies}. 

\subsection{The Promise of African-led Data Science}
Masakhane is an African natural language processing (NLP) collective that builds language datasets and models in Indigenous African languages. By all reports across documents, Masakhane practices all the principles proposed in this framework and especially upholds the major principle \textit{Data Self-Determination} \cite{eke2023responsible, chan2021limits, shilongo2023creativity}.
They uphold these principles with a commitment to centering African values in their founding principles, working with existing public datasets to not infringe on data privacy, and building language datasets to preserve indigenous African languages for the future to come \cite{adelani2022masakhaner}. Masakhane also has a very welcoming and communal organizational structure to include any interested party in weekly meetings and communications. Once a member wants to contribute to a Masakhane project, they must undergo in-house training to maintain quality of the their dataset and model. They also explicitly prohibit ``parachute research from the Global North'' to ensure the time and resources of their collective provide direct benefit to their communities.\footnote{\url{https://www.masakhane.io/}} Finally, every project plan includes a discussion of data privacy considerations to guide their work. While there are many other parts to Masakhane's work, the practices described in their public documents demonstrate an African data science community closely aligned with the perspectives represented in our framework. Masakhane is part of the growing grassroots efforts to imagine what African-led responsible data science can achieve. 

\subsection{Limits of Inclusive Representation in African Facial Recognition Technology}
African people, and Black people in general, are severely underrepresented in facial recognition datasets which has led to a performance bias against Black users. African technology companies are committed to addressing this bias so their primarily African users can rely on their products. For example, a woman-led facial recognition start-up called BACE curated a diverse dataset from the local community so their facial recognition system could better detect Black subjects \cite{eke2023towards}. Users upload photos of their IDs and short videos on their phones to confirm their identity.\footnote{\url{https://www.bacegroup.com/}} For accessibility, they developed a secure mobile API. The biometric technology was first developed to aid financial fraud investigation efforts in Ghana. The current government efforts were hampered by the lack of citizens with formal identification documents. Their use cases have expanded to include identity verification for the public and private sectors. While the facial recognition technology developed by BACE achieves privacy and inclusive performance, the principles described in our framework call for more direct and sustained community involvement. Inclusion in datasets is not the only part of the data science lifecycle communities should be involved in. Additionally, BACE was created to meet the needs of financial institutions but did the team gauge if the local community had privacy or surveillance concerns during the development process \cite{birhane2020algorithmic}? The work of BACE is monumental, however, our framework raises important considerations about the sustained involvement of community members.
