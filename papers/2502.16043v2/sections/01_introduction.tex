\section{Introduction}
Artificial intelligence (AI), machine learning (ML), and data science research and development are primarily conducted in economically powerful countries such as the United States, China, the United Kingdom, France, and Germany \cite{bengio2024international}. 
Consequently, data-driven systems (DDS) leveraging these methodologies often reflect the values of large corporations and Western ideals \cite{birhane2022forgotten}. 
When ethical dilemmas arise within data science development, mitigation strategies are often limited by the perspectives of these actors who rarely experience the direct consequences of data harm \cite{mhlambi2023decolonizing,cisse2018look,gwagwa2022role}. 


To balance the dominant Western perspective, pluralistic data ethics calls for perspectives from the Global Majority to %develop better strategies to 
better mitigate systemic concerns \cite{rifat2023many,carman2023applying}. 
Embracing pluralism in global data ethics acknowledges the limits of any single ethical perspective in guiding comprehensive responsible data science (RDS) practices. 
Unfortunately, African voices have routinely been left out of Western-centric data science ethics discussions \cite{eke2023responsible}. 
This is a glaring omission---African data collaborators have an intimate experience with how colonialism, anti-blackness, and data dispossession operate in data science work \cite{abebe2021narratives,eke2023responsible}.
 In addition, in the development of DDS, African people are often exploited as data workers, their resources are extracted for computing infrastructure, and even when deploying systems in their own communities, many components are managed and owned by external actors \cite{birhane2020algorithmic}.

To address the underrepresentation of African perspectives in global data ethics, African data scientists have returned to African philosophies such as Ubuntu to articulate the current ethical dilemmas in data science work on the continent \cite{gwagwa2022role}. 
However, African data ethics texts are spread across a variety of publication venues and are %only 
rarely synthesized into a cohesive review or framework
\cite{eke2022responsible}. 
For this reason, it can be difficult to 
grasp
common topics, differing fields of thought, and how proposed principles translate to current practices of RDS in Africa. 

In this work, we synthesize the current African data ethics discourse and contextualize data ethics theories through 
the analysis of African philosophies. 
We conduct a thematic analysis of 47 documents from African philosophy, information ethics, AI ethics, and human-computer interaction research to answer the following research question: \textbf{What are the African values and ethical theories that can inform responsible data science (RDS) practices in Africa?} 
As a result of our inquiry, we derived an ethical framework consisting of seven major principles rooted in the realities of data science work in Africa (Table~\ref{tab:framework-overview}). 
For example, one major principle we surfaced (\textit{decolonize \& challenge internal power asymmetry}) contends with the legacy of African colonialism and the current administration of neocolonialism to articulate the inefficacy of Western-centric data science in African contexts. 
Several principles also engage with a range of African communitarian theories to recontextualize common paradigms of social good, participatory design, and data ethics education. 


We place our framework in conversation with six other particularist data ethics frameworks to gauge its unique contributions to global efforts. While every framework is concerned with implementing technology for ``common good'',  we find African data ethics adds a pragmatic and nuanced perspective on how to uphold this ideal.
To demonstrate the utility of our framework, we conclude with two case studies taken from ongoing African data science efforts.
Overall, our work bridges rich African philosophical traditions with contemporary data science practices, developing a framework to guide responsible technology development within African contexts while contributing to global discussions on data ethics and justice.

