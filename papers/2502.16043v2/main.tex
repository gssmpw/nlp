%%
%% This is file `sample-sigconf.tex',
%% generated with the docstrip utility.
%%
%% The original source files were:
%%
%% samples.dtx  (with options: `all,proceedings,bibtex,sigconf')
%% 
%% IMPORTANT NOTICE:
%% 
%% For the copyright see the source file.
%% 
%% Any modified versions of this file must be renamed
%% with new filenames distinct from sample-sigconf.tex.
%% 
%% For distribution of the original source see the terms
%% for copying and modification in the file samples.dtx.
%% 
%% This generated file may be distributed as long as the
%% original source files, as listed above, are part of the
%% same distribution. (The sources need not necessarily be
%% in the same archive or directory.)
%%
%%
%% Commands for TeXCount
%TC:macro \cite [option:text,text]
%TC:macro \citep [option:text,text]
%TC:macro \citet [option:text,text]
%TC:envir table 0 1
%TC:envir table* 0 1
%TC:envir tabular [ignore] word
%TC:envir displaymath 0 word
%TC:envir math 0 word
%TC:envir comment 0 0
%%
%% The first command in your LaTeX source must be the \documentclass
%% command.
%%
%% For submission and review of your manuscript please change the
%% command to \documentclass[manuscript, screen, review]{acmart}.
%%
%% When submitting camera ready or to TAPS, please change the command
%% to \documentclass[sigconf]{acmart} or whichever template is required
%% for your publication.
%%
%%
%
\documentclass[sigconf]{acmart}
%\documentclass[manuscript,screen,review,anonymous]{acmart}
\usepackage{tabularx,booktabs}
\usepackage{graphicx} % Required for inserting image
\usepackage{colortbl}
\usepackage{multirow}
\pdfoutput=1
%% \BibTeX command to typeset BibTeX logo in the docs
\AtBeginDocument{%
  \providecommand\BibTeX{{%
    Bib\TeX}}}

%% Rights management information.  This information is sent to you
%% when you complete the rights form.  These commands have SAMPLE
%% values in them; it is your responsibility as an author to replace
%% the commands and values with those provided to you when you
%% complete the rights form.
\setcopyright{acmlicensed}
\copyrightyear{2025}
\acmYear{2025}
\acmDOI{XXXXXXX.XXXXXXX}
%% These commands are for a PROCEEDINGS abstract or paper.
\acmConference[Conference acronym 'XX]{Make sure to enter the correct
  conference title from your rights confirmation email}{June 03--05,
  2018}{Woodstock, NY}
%%
%%  Uncomment \acmBooktitle if the title of the proceedings is different
%%  from ``Proceedings of ...''!
%%
%%\acmBooktitle{Woodstock '18: ACM Symposium on Neural Gaze Detection,
%%  June 03--05, 2018, Woodstock, NY}
\acmISBN{978-1-4503-XXXX-X/2018/06}

%%
%% Submission ID.
%% Use this when submitting an article to a sponsored event. You'll
%% receive a unique submission ID from the organizers
%% of the event, and this ID should be used as the parameter to this command.
%%\acmSubmissionID{123-A56-BU3}

%%
%% For managing citations, it is recommended to use bibliography
%% files in BibTeX format.
%%
%% You can then either use BibTeX with the ACM-Reference-Format style,
%% or BibLaTeX with the acmnumeric or acmauthoryear sytles, that include
%% support for advanced citation of software artefact from the
%% biblatex-software package, also separately available on CTAN.
%%
%% Look at the sample-*-biblatex.tex files for templates showcasing
%% the biblatex styles.
%%

%%
%% The majority of ACM publications use numbered citations and
%% references.  The command \citestyle{authoryear} switches to the
%% "author year" style.
%%
%% If you are preparing content for an event
%% sponsored by ACM SIGGRAPH, you must use the "author year" style of
%% citations and references.
%% Uncommenting
%% the next command will enable that style.
%%\citestyle{acmauthoryear}


%%
%% end of the preamble, start of the body of the document source.
\begin{document}

%%
%% The "title" command has an optional parameter,
%% allowing the author to define a "short title" to be used in page headers.
\title{African Data Ethics: A Discursive Framework for Black Decolonial Data Science}

%%
%% The "author" command and its associated commands are used to define
%% the authors and their affiliations.
%% Of note is the shared affiliation of the first two authors, and the
%% "authornote" and "authornotemark" commands
%% used to denote shared contribution to the research.
\author{Teanna Barrett}
\affiliation{%
  \institution{University of Washington}
  \city{Seattle}
  \state{Washington}
  \country{USA}
}
\email{tb23@cs.washington.edu}

\author{Chinasa T. Okolo}
\affiliation{%
  \institution{The Brookings Institution}
  \city{Washington}
  \state{D.C.}
  \country{USA}}
\email{cokolo@brookings.edu}

\author{B. Biira}
\affiliation{%
  \institution{University of Washington}
  \city{Seattle}
  \state{Washington}
  \country{USA}
}
\email{bbiira@uw.edu}

\author{Eman Sherif}
\affiliation{%
 \institution{University of Washington}
  \city{Seattle}
  \state{Washington}
  \country{USA}}
\email{emans@uw.edu}


\author{Amy X. Zhang}
\affiliation{%
  \institution{University of Washington}
  \city{Seattle}
  \state{Washington}
  \country{USA}}
\email{axz@cs.uw.edu}

\author{Leilani Battle}
\affiliation{%
  \institution{University of Washington}
  \city{Seattle}
  \state{Washington}
  \country{USA}}
\email{leibatt@cs.washington.edu}

%%
%% By default, the full list of authors will be used in the page
%% headers. Often, this list is too long, and will overlap
%% other information printed in the page headers. This command allows
%% the author to define a more concise list
%% of authors' names for this purpose.
\renewcommand{\shortauthors}{Barrett et al.}

%%
%% The abstract is a short summary of the work to be presented in the
%% article.
End-to-end imitation learning offers a promising approach for training robot policies. However, generalizing to new settings—such as unseen scenes, tasks, and object instances—remains a significant challenge. Although large-scale robot demonstration datasets have shown potential for inducing generalization, they are resource-intensive to scale. In contrast, human video data is abundant and diverse, presenting an attractive alternative. Yet, these human-video datasets lack action labels, complicating their use in imitation learning. Existing methods attempt to extract grounded action representations (e.g., hand poses), but resulting policies struggle to bridge the embodiment gap between human and robot actions.
% our approach
We propose an alternative approach: leveraging language-based reasoning from human videos - essential for guiding robot actions - to train generalizable robot policies. Building on recent advances in reasoning-based policy architectures, we introduce Reasoning through Action-free Data (RAD). RAD learns from both robot demonstration data (with reasoning and action labels) and action-free human video data (with only reasoning labels). The robot data teaches the model to map reasoning to low-level actions, while the action-free data enhances reasoning capabilities. Additionally, we will release a new dataset of 3,377 human-hand demonstrations compatible with the Bridge V2 benchmark. This dataset includes chain-of-thought reasoning annotations and hand-tracking data to help facilitate future work on reasoning-driven robot learning.
% experiments
Our experiments demonstrate that RAD enables effective transfer across the embodiment gap, allowing robots to perform tasks seen only in action-free data. Furthermore, scaling up action-free reasoning data significantly improves policy performance and generalization to novel tasks. These results highlight the promise of reasoning-driven learning from action-free datasets for advancing generalizable robot control. 
% releasing dataset
Website: \href{https://rad-generalization.github.io}{here}.


%%
%% The code below is generated by the tool at http://dl.acm.org/ccs.cfm.
%% Please copy and paste the code instead of the example below.
%%
\begin{CCSXML}
<ccs2012>
   <concept>
       <concept_id>10003120.10003121.10003126</concept_id>
       <concept_desc>Human-centered computing~HCI theory, concepts and models</concept_desc>
       <concept_significance>500</concept_significance>
       </concept>
   <concept>
       <concept_id>10003456.10003462.10003588.10003589</concept_id>
       <concept_desc>Social and professional topics~Governmental regulations</concept_desc>
       <concept_significance>300</concept_significance>
       </concept>
 </ccs2012>
\end{CCSXML}

\ccsdesc[500]{Human-centered computing~HCI theory, concepts and models}
\ccsdesc[300]{Social and professional topics~Governmental regulations}

%%
%% Keywords. The author(s) should pick words that accurately describe
%% the work being presented. Separate the keywords with commas.
\keywords{responsible data science, data ethics, african philosophy, decolonial AI, pluralistic alignment}
%% A "teaser" image appears between the author and affiliation
%% information and the body of the document, and typically spans the
%% page.
\begin{comment}
    \begin{teaserfigure}
  \includegraphics[width=\textwidth]{sampleteaser}
  \caption{Seattle Mariners at Spring Training, 2010.}
  \Description{Enjoying the baseball game from the third-base
  seats. Ichiro Suzuki preparing to bat.}
  \label{fig:teaser}
\end{teaserfigure}



\received{20 February 2007}
\received[revised]{12 March 2009}
\received[accepted]{5 June 2009}
\end{comment}
%%
%% This command processes the author and affiliation and title
%% information and builds the first part of the formatted document.
\maketitle

\begin{figure}[ht]
    \centering
    \includegraphics[width=0.8\linewidth]{graphs/greater_than_naive.pdf}
    \vspace{0.5cm}
    \includegraphics[width=0.8\linewidth]{graphs/p1_bottom.png}
    \vspace{-5pt}
    \caption{\textcolor{positional}{Positional} vs.\ \textcolor{nonpositional}{non-positional} circuits. In a \textcolor{nonpositional}{non-positional} circuit, the same edges must be included at all positions. A \textcolor{positional}{positional} circuit can distinguish between the same edge at different positions. This specificity yields better trade-offs between circuit size and faithfulness. It can also increase both precision and recall.}
    \label{fig:p1}
    \vspace{-5pt}
\end{figure}

\section{Introduction}

\looseness=-1
A primary goal of interpretability research is to characterize the internal mechanisms in language models (LMs) and other NLP models. 
A core approach in this area is \textbf{circuit discovery}---identifying the minimal subgraph within the model's computation graph that performs a specific task \citep{olah2021framework,olah-mech}.
Typically, the nodes of a circuit represent model components (e.g., attention heads, neurons, or layers).
While manual circuit discovery methods can yield position-specific insights \citep{wanginterpretability,goldowskydill2023localizingmodelbehaviorpath}, \emph{automatic methods often overlook positional information}, treating components as uniformly relevant across all input token positions \citep{conmytowards,syed2023attribution}. 
For instance, if an attention head is included in a circuit, it is assumed to contribute equally to the computation for every position in the input sequence.
The assumption that circuits are position-invariant ignores the fact that different positions often require distinct computations.
By ignoring positions, current methods limit their ability to capture mechanisms that operate across positions, such as interactions between attention heads across positions.

In this study, we start by demonstrating that positional agnosticism is a significant limitation (\S\ref{sec:motivating}). Then, to address these limitations, we introduce a new approach: position-aware edge attribution patching (PEAP; \S\ref{sec:full_circ_discovery}; Figure~\ref{fig:p1}). Current approaches  assume that if an edge is in a circuit, then the same edge will be in the circuit at all positions, thus leading to low precision. It is also assumed that an edge's importance should be aggregated across positions before deciding whether it should be included in the circuit; this can lead to cancellation effects, and thus low recall. PEAP instead allows us to compute the importance of cross-positional edges, and separately evaluates edge importance at each position. We show that this leads to smaller and more accurate circuits; see Figure~\ref{fig:p1}.

Incorporating positional information into circuit discovery is straightforward when inputs have the same length and structure across examples.

However, realistic datasets are not nearly this templatic.
How, then, can we incorporate positional information into automatic circuit discovery?
To address this challenge, we propose \textbf{schemas} (\S\ref{sec:schema}). 
Schemas assign semantic labels to spans of tokens, enabling information aggregation across examples even when the spans differ in length.

For example, in the input ``The \textcolor{positional}{war} lasted from 1453 to 14\underline{\hspace{1em}},'' the span ``\textcolor{positional}{war}'' could be labeled as ``\emph{Subject}''.
This enables handling spans with varying lengths: the phrase ``\textcolor{positional}{Black Plague}'' in another example can be treated as a single positional span with the same role as ``\textcolor{positional}{war}''.
In experiments with two LMs and three tasks, we find that circuits discovered using schemas achieve a better trade-off between circuit size and faithfulness to the model's behavior than position-agnostic circuits.
Importantly, position-aware circuits offer a more precise representation of the underlying mechanisms, providing a more concise foundation for mechanistic explanations.

We also present a fully automated pipeline for schema generation and application (\S\ref{sec:schema-generation}) using large language models (LLMs). 
We evaluate the quality of the generated schemas and their utility in discovering position-aware circuits (\S\ref{sec:schema-eval}).
Notably, circuits derived using automatically generated and applied schemas achieve comparable faithfulness scores to circuits discovered with human-designed and manually applied schemas.

We summarize our contributions as follows:
\begin{itemize}[noitemsep,leftmargin=*,topsep=1pt,parsep=1pt]
    \item Introduce a position-aware circuit discovery method, which obtains better faithfulness than position-agnostic discovery.  
    \item Introduce dataset schemas,  facilitating positional circuit discovery in more naturalistic settings. 
    \item Develop an automated schema generation and application pipeline with LLMs, yielding schemas that are comparable to manually-annotated ones.
\end{itemize}

\section{Background of Cost Estimation} \label{sec:background}
This section first gives a brief overview of classical and learned cost estimation. 
Afterwards, we describe the learning procedure of \lcms and provide a taxonomy that guides our selection of recent \lcms for this study in \Cref{sec:methodology}.

\subsection{Traditional \& Learned Cost Estimation}
\textbf{Traditional Cost Estimation.} 
Precise cost estimates for different plan candidates in a database are crucial for the query optimizer to select optimal plans from a large search space.
Thus, a lot of engineering effort has been spent since the beginning of database development to estimate the execution costs of a query plan.
Most database systems such as MySQL \cite{widenius2002}, Oracle, PostgreSQL, or System R \cite{astrahan1976} use hand-crafted cost models to reason about the execution costs of a query plan.
These models typically provide a cost function for each physical operator in a query plan that estimates its runtime costs according to CPU usage, I/O operations, memory consumption, expected tuples, and random or sequential page accesses.
However, due to the wide variety of data, queries, and data layouts, traditional cost models need to make simplifying assumptions (e.g., independence of attributes).
These often lead to incorrect predictions of the execution cost. 
Consequently, the query optimizer makes sub-optimal decisions that degrade the query performance by increasing its runtime \cite{leis_how_2015}.
%-------------------------------------------------------

\noindent\textbf{Learned Cost Estimation.}
The need to improve prediction accuracy and the rise of machine learning motivated the idea of \lcms. 
The main idea is to approximate the complex cost functions with a learned model.
Generally, a typical model learns from previous query executions to predict execution costs like runtime.
In contrast to traditional cost models, the promise of \lcms is that they can better learn arbitrarily complex functions.
Thus, improved prediction accuracy can be expected in contrast to traditional approaches based on simplifying assumptions.
Overall, the higher accuracy is expected to lead to a selection of query plans with improved query performance.

%-----------------------------------------
\subsection{Learning Procedure of \lcms}
For our study, we look at effects that also result from the learning procedure of \lcms.
As such, we briefly review the traditional procedure as depicted in \Cref{fig:learning_procedure} to provide the necessary background:
\circles{A}~At first, a workload generator is used to create a large set of randomized, synthetic SQL-Strings that involve a variety of representative query properties such as filter predicates, joins, or aggregation types.
\circles{B}~These queries are executedses (e.g., an airline or movie database) to collect the actual costs of queries.
An important aspect here is that training procedures of many \lcms leads to biases in the dataset due to timeouts and pre-optimized queries, as discussed later.
\circles{C}~Next, various information is extracted from the workload execution.
Most importantly, the physical query plans are extracted, which serve as input to cost models.
In addition to physical plans, \lcms require different information, such as data characteristics like histograms or sample bitmaps.
\circles{D}+\circles{E}~Finally, the workload (i.e., plans and runtime) is then split for training and testing the \lcms. 

\begin{figure*}
    \centering
    \includegraphics[width=\linewidth]{./figures/training_procedure.pdf}
    \caption{
    Learning procedure of \lcms. 
    \circles{\textsc{A}} Generation of synthetic training queries. \circles{\textsc{B}} Query execution on training databases. 
    \circles{\textsc{C}} Feature (query plans, data characteristics, and sample bitmaps) and label (query runtimes) extraction to generate the training and test dataset. 
    \circles{\textsc{D}} Training of the \lcm with supervised learning. 
    \circles{\textsc{E}} Evaluation of the \lcm against unseen test data.}
    \label{fig:learning_procedure}
\end{figure*}

%------------------------------------------------------
\subsection{Taxonomy of \lcms} \label{subsec:taxonomy}
\lcms developed in the last years differ in various dimensions.
This section provides a brief taxonomy of recent \lcms to structure the different methodological approaches. 
This taxonomy will guide the selection of \lcms that we use in this study and ensure that we cover the different methodologies to analyze how they affect the ability of \lcms to support query optimization.

\noindent\textbf{Input Features.}
The first crucial dimension is the input features that a \lcm learns from.
The input features are extracted from the executed workloads (cf. \Cref{fig:learning_procedure}\circles{C}).
The query plan and the underlying data distribution need to be modeled so that a \lcm is informed to make reasonable predictions about the execution costs, which in turn affects query optimization, as we will show.
However, \lcms make use of different information for cost estimation.

\begin{enumerate}[leftmargin=*, nosep]
\item \textbf{SQL-String vs. Query Plans}: 
Some of the first models rely on the SQL string to describe a query, as it gives insights about the tables, predicates, and joins. 
However, details of the execution plan, such as physical operators or the order of joins, are not described there. 
Thus, most \lcms utilize the physical query plan, which includes the operators (e.g., scans, joins) and physical operator types  (e.g., nested loop vs. hash join).
As we will see later, this is fundamental for query optimization.

\item \textbf{Cardinalities}:
Intermediate cardinalities are an important input signal for the overall cost of a plan as they denote the number of tuples an operator needs to process \cite{leis_how_2015}.
Thus, many \lcms leverage intermediate cardinalities as input features, which are either annotated by the databases' cardinality estimator or obtained through an additional learned estimator from related work \cite{hilprecht2020deepdb, kipf2019, yang2020}.
While some \lcms also ignore cardinalities as input for cost prediction, we show in our study that they, in fact, improve the usefulness of cost estimates from \lcms for various query optimization tasks.

\item\textbf{Data Distribution}:
Another helpful factor in estimating cost is understanding the data distribution in the base tables, especially if no cardinalities are used.
For instance, the fact of how many distinct values exist in a column might influence the efficiency of physical operators (e.g., hash join). 
As such, some \lcms use data distribution represented as database statistics and histograms or sample bitmaps (which we explain later) from the base tables as inputs.
However, as we will show in our study, their effect on query optimization tasks remains unclear.

\item \textbf{Cost Estimates}: 
Finally, some of the most recent \lcms even leverage the cost estimates provided by a classical cost estimator as an input feature, which serves as a strong input signal.
This idea renders these \lcms to \textit{hybrid} as they combine a traditional cost model with a learned approach.
The study shows that this provides significant benefits.
\end{enumerate}

%-----------------------------------------------------
\noindent \textbf{Query Representation.}
Many \lcms use model architectures use a graph-based representation to encode query plans as input to the models\footnote{The graph-based representation of the queries refers to the fact whether a model leverages the query graph structure and not to the model learning architecture itself.}.
These approaches thus explicitly leverage information about the order (parent-child relationships) of operators in plans.
However, other \lcms \cite{kipf2019, akdere2012} represent a query plan (or the SQL string) as a flat vector of fixed size without modeling the operator dependencies, which we refer to as \textit{flat} representation in this paper.
While intuitively, capturing the structure and not using a flat representation should be beneficial for \lcms, the results of using graph structure in this study are not that clear. 

%-------------------------------------------------------
\noindent \textbf{Database Dependency.}
Furthermore, an important aspect is whether \lcms can generalize to unseen databases (i.e., a new set of tables) or not.
\textit{Database-agnostic} \lcms were designed \cite{hilprecht2022, zibo_liang_dace_2024} to enable cost predictions for unseen databases that were not part of the training data.
This approach has the advantage of directly providing results without requiring database-specific training data. 
In contrast, \textit{database-specific} \cite{sun2019, zhao2022, marcus2019} models cannot generalize for unknown databases.
For this study, an interesting question is if one of these classes is better suited to support query optimization tasks as database-specific can better adapt to one single database while database-agnostic models can generalize better.

%----------------------------------------
\noindent \textbf{Model Architecture.}
Finally, the presented \lcms differ largely in their learning approach.
Various learning architectures were proposed, including decision trees, tree-structured neural networks, neural units, graph neural networks, and transformer architectures.
While different architectures show different results on the cost estimation tasks, it is still open to see which architecture provides the best results for query optimization.
%----------------------------------------
\section{Proposed Method}
\label{sec:method}

In this section, we introduce our homotopy-based multi-objective framework for face parsing and its integration with both \textbf{GAN-based} and \textbf{diffusion-based} face editing models. We outline the problem formulation, dataset preparation, model architecture, training strategy, and evaluation pipeline, emphasizing \textbf{fairness}, \textbf{robustness}, and \textbf{semantic alignment}.

\subsection{Problem Formulation}
\label{subsec:problem_formulation}

We define the dataset \(\mathbf{X} = \{x_i\}\), where each face image is paired with a segmentation mask \( y_i \in \mathbf{Y} \), mapping to 19 facial components (e.g., hair, eyes, mouth). Demographic attributes are denoted as \(\mathbf{a}\) (e.g., \texttt{Male}, \texttt{Young}, \texttt{Wearing Hat}). Our objective is to train a segmentation function \( f_\theta(\cdot) \) that predicts \(\hat{y}_i\) while optimizing for accuracy, fairness, and robustness. Accuracy is maximized by aligning \(\hat{y}_i\) with \(y_i\) using Dice loss~\cite{sudre2017generalised}. Fairness is enforced by minimizing variance \(\mathrm{Var}(\mathrm{mIoU}_g)\) across demographic groups, ensuring equitable segmentation quality. Robustness is maintained by penalizing performance degradation (\(\mathrm{mIoU}\) drop) under input perturbations such as noise and occlusion.

\begin{algorithm}[h][t]
\caption{Multi-Objective Face Parsing (Pseudo-code)}
\label{alg:multi_objective_pseudocode}
\begin{algorithmic}[1]
\Require Homotopy function \(h(t)\) providing \((\alpha, \beta, \gamma)\) for epoch \(t\)
\For{epoch \(t = 1 \dots T\)}
    \State \((\alpha, \beta, \gamma) = h(t)\)
    \For{each batch in DataLoader}
        \State \textbf{Load} images \(\{x\}\), masks \(\{m\}\), attributes \(\{a\}\)
        \State outputs \(= f_{\theta}(x)\) \Comment{U-Net forward pass}
        \State \(\mathcal{L}_{\mathrm{acc}} = \mathrm{DiceLoss}(outputs, m)\)
        \State outputs\(_{\mathrm{noisy}} = outputs + \text{random\_noise}()\)
        \State \(\mathcal{L}_{\mathrm{rob}} = -\mathrm{mIoU}(\mathrm{softmax}(outputs_{\mathrm{noisy}}), m)\)
        \State \(\mathcal{L}_{\mathrm{fair}} = \mathrm{Var}\left[\mathrm{mIoU}_g\right]\)
        \State \(\mathcal{L}_{\text{total}} = \alpha\,\mathcal{L}_{\mathrm{acc}} + \beta\,\mathcal{L}_{\mathrm{rob}} + \gamma\,\mathcal{L}_{\mathrm{fair}}\)
        \State \textbf{Backward} and \textbf{update} \(\theta\)
    \EndFor
\EndFor
\end{algorithmic}
\end{algorithm}

\subsection{Dataset Preparation}
\label{subsec:dataset_preparation}

We employ the CelebAMask-HQ dataset \cite{CelebAMask-HQ}, divided into training, validation, and test sets. Each image and mask are resized to \(256 \times 256\) for compatibility with our U-Net architecture. Demographic attributes are extracted from annotations to compute fairness metrics.

\subsection{Model Architecture}
\label{subsec:model_architecture}

Our segmentation model utilizes a U-Net architecture with a ResNet-34 encoder pre-trained on ImageNet. It outputs 19 channels corresponding to distinct facial regions, balancing computational efficiency with high segmentation accuracy.

\subsection{Multi-Objective Training}
\label{subsec:multi_objective}

We train the U-Net segmentation models by optimizing a weighted sum of accuracy, fairness, and robustness losses, dynamically adjusted using homotopy-based scheduling. The training process is outlined in Algorithm~\ref{alg:multi_objective_pseudocode}.

\paragraph{Loss Components}
\begin{itemize}
    \item \textbf{Accuracy Loss (\(\mathcal{L}_{\mathrm{acc}}\)):} Dice loss measures the overlap between predicted and ground truth masks.
    \item \textbf{Robustness Loss (\(\mathcal{L}_{\mathrm{rob}}\)):} Negative \(\mathrm{mIoU}\) under perturbed predictions to ensure stability.
    \item \textbf{Fairness Loss (\(\mathcal{L}_{\mathrm{fair}}\)):} Variance of \(\mathrm{mIoU}\) across demographic groups to promote equitable performance.
\end{itemize}

\textbf{Alternative Fairness Computation:} We also compute per-group \(\mathrm{mIoU}\) for each demographic attribute, enabling detailed analysis of performance disparities (see Section~\ref{subsec:fairness_comparison}).

\subsection{Homotopy-Based Loss Scheduling}
\label{subsec:homotopy}

We dynamically balance the three loss components using epoch-dependent weights \(\alpha(t)\), \(\beta(t)\), and \(\gamma(t)\), ensuring \(\alpha(t) + \beta(t) + \gamma(t) = 1\). Initially, accuracy is prioritized, with weights shifting towards robustness and fairness over time. We explore three scheduling strategies:

\begin{itemize}
    \item \textbf{Linear:} \(\alpha(t)\) decreases linearly, while \(\beta(t)\) and \(\gamma(t)\) increase proportionally.
    \item \textbf{Sigmoid:} Smooth logistic transitions for gradual emphasis shifts.
    \item \textbf{Piecewise:} Abrupt changes in weight distribution at predefined training stages.
\end{itemize}

\begin{figure}[t]
    \centering
    \includegraphics[width=\columnwidth]{figures/parameters_by_homotopy-crop.pdf}
    \caption{Comparison of \(\alpha\), \(\beta\), and \(\gamma\) schedules across three homotopy methods (Linear, Sigmoid, and Piecewise) over 30 epochs. Each subplot illustrates the evolution of a parameter (\(\alpha\), \(\beta\), or \(\gamma\)) as it adapts during training, highlighting the differences in transition dynamics across homotopy strategies. The legend below the figure identifies the homotopy method for each curve.}
    \label{fig:homotopy-schedules}
\end{figure}


Figure~\ref{fig:homotopy-schedules} illustrates the evolution of these weights across training epochs for each homotopy method.

\subsection{Integration with Generative Models}

\subsubsection{GAN-Based Face Editing}
\label{subsec:gan_integration}

We utilize the trained U-Nets to generate segmentation maps for the training and validation sets, which are then used to train a Pix2PixHD GAN. The GAN architecture comprises:

\begin{itemize}
    \item \textbf{Generator} \(G\): Transforms segmentation maps into RGB images.
    \item \textbf{Discriminator} \(D\): Distinguishes real images from generated ones.
\end{itemize}

The GAN training involves a combination of adversarial loss and pixel-level \(L_1\) reconstruction loss:
\[
\mathcal{L}_{\mathrm{GAN}} = \mathcal{L}_{\mathrm{adv}}(G, D) + \lambda \, \|\hat{x} - x\|_1,
\]
where \(\hat{x} = G(\text{segmentation\_map})\) and \(x\) is the real image.

During testing, the GAN generates images using segmentation maps from the test set produced by both single-objective and multi-objective U-Nets, enabling evaluation of how segmentation quality impacts generative performance.

\subsubsection{ControlNet-Based Face Editing}
\label{subsec:controlnet_integration}

In addition to GANs, we integrate \textbf{ControlNet} \cite{zhang2023adding} for diffusion-based face editing. ControlNet leverages segmentation maps to guide the diffusion process, enhancing image fidelity and semantic alignment. Our setup includes:

\begin{itemize}
    \item \textbf{ControlNet Model:} Pre-trained on Stable Diffusion, fine-tuned on our segmentation maps.
    \item \textbf{Diffusion Pipeline:} Combines ControlNet with a text encoder and U-Net backbone to generate photorealistic faces conditioned on segmentation maps.
\end{itemize}

\textbf{Training Procedure:} ControlNet is fine-tuned for a single epoch using segmentation maps from the training set. In diffusion-based experiments, we compare only the single-objective model with the multi-objective linear homotopy model to manage computational resources effectively. The training minimizes the standard denoising loss:
\[
\mathcal{L}_{\mathrm{ControlNet}} = \mathcal{L}_{\mathrm{denoise}},
\]
where \(\mathcal{L}_{\mathrm{denoise}}\) is the Mean Squared Error between predicted and actual noise. During testing, ControlNet generates images using test set segmentation maps from both U-Net models, allowing assessment of segmentation quality's effect on diffusion-based generation.

\subsection{Evaluation Metrics and Setup}
\label{subsec:evaluation}

\paragraph{Segmentation Metrics}  
We evaluate segmentation performance using the mean Intersection-over-Union (\(\mathrm{mIoU}\)) across 19 facial classes. Fairness is quantified by the variance \(\mathrm{Var}(\mathrm{mIoU}_g)\) across demographic groups, and robustness is assessed through performance under Gaussian noise, occlusions, and blur.

\paragraph{Generative Metrics}  
For GAN outputs, we evaluate image quality using \textbf{Fréchet Inception Distance (FID)}, which quantifies realism by comparing feature distributions between generated and real images. Additionally, \textbf{Learned Perceptual Image Patch Similarity (LPIPS)} measures perceptual similarity, where lower scores indicate greater visual resemblance to real images.

\paragraph{Implementation Details}  
All experiments are implemented in PyTorch and trained on four NVIDIA A10 GPUs using the Adam optimizer with a learning rate of \(10^{-4}\). For ControlNet, we fine-tune the pre-trained \texttt{control\_v11p\_sd15\_seg} model based on Stable Diffusion v1.5. Our pipeline supports gradient accumulation and mixed precision (FP16) for computational efficiency. Homotopy-based loss scheduling is configurable (\texttt{linear}, \texttt{sigmoid}, \texttt{piecewise}). Detailed training configurations will be released alongside our code and models to ensure reproducibility.

\paragraph{Workflow Summary}  
\begin{enumerate}
    \item \textbf{Train U-Nets:} Train single-objective and multi-objective U-Nets on the training set, validate on the validation set.
    \item \textbf{Generate Segmentation Maps:} Use trained U-Nets to produce segmentation maps for training, validation, and test sets.
    \item \textbf{Train GAN:} Train the Pix2PixHD GAN using segmentation maps from the training and validation sets.
    \item \textbf{Fine-Tune ControlNet:} Fine-tune ControlNet on training set segmentation maps for one epoch.
    \item \textbf{Generate and Evaluate Images:} Generate images using GAN and ControlNet with test set segmentation maps from both U-Net models; evaluate using FID and LPIPS.
\end{enumerate}

In the following section, we present quantitative and qualitative results demonstrating the effectiveness of our approach across various conditions and demographic groups.

\section{Results}\label{sect:results}

\subsection{Established Benchmarks}\label{sect:results_west}
We begin by evaluating all vision-language models on established benchmarks, based on ImageNet and COCO Captions, among other datasets. As revealed in Table~\ref{tab:west_standard_setup}, increasing the dataset size from 10 billion to 100 billion examples does not improve performance substantially. This is statistically supported by Wilcoxon's signed rank test~\cite{wilcoxon1992individual}, which gives a $p$-value of 0.9, indicating that differences are not significant.


In addition, we also fit data scaling laws for every combination of model and dataset following the recipe proposed in~\citet{alabdulmohsin2022revisiting}. This allows us to evaluate whether or not the performance gap is expected to increase or decrease in the infinite-compute regime. We report the resulting scaling exponents and asymptotic performance limits in the tables. Again, we do not observe  significant differences at the 95\% confidence level ($p$-value of 0.09).


\subsection{Cultural Diversity}
Unlike the Western-oriented metrics reported in Section~\ref{sect:results_west}, cultural diversity metrics present an entirely different picture. We observe \emph{notable} gains when scaling the size of the dataset from 10 billion to 100 billion examples in Table~\ref{tab:culture_standard_setup}. 
For example, scaling training data from 10 billion to 100 billion examples yields substantial gains on Dollar Street 10-shot classification task, where ViT-L and ViT-H see absolute improvements of 5.8\% and 5.4\%, respectively. These gains outperform the typical improvements (less than 1\%) observed on Western-oriented 10-shot metrics by a large margin.
Using Wilcoxon's signed rank test, we obtain a $p$-value of 0.002, indicating a statistically significant evidence at the 99\% confidence level.


\subsection{Multilinguality}

Our multilingual benchmark, Crossmodal-3600 zero-shot retrieval~\cite{thapliyal2022crossmodal}, shows a disparity in performance gains: low-resource languages benefit more from the 100 billion scale than the high-resource ones. The disparity, illustrated in Figure~\ref{fig:multilinguality}, which not only exists in all model sizes but also widens as the models become larger. Detailed results for each language can be found in Appendix~\ref{appendix:data_scale}.

% source: https://colab.corp.google.com/drive/1AKgGDITZqTC2hQjVc-Iv8xuysh5giP0i#scrollTo=2EtEXMbly8dB&line=1&uniqifier=1
\begin{figure}[h!]
    % \includegraphics[width=\linewidth]{figures/multilang-Average_Multilingual__Low-Resource_Lang.pdf}
    % \includegraphics[width=0.86\linewidth]{figures/multilang-Average_Multilingual__High-Resource_Lang.pdf}
    \includegraphics[width=\linewidth]{figures/multilang-Average_XM3600_Retrieval.pdf}
    \caption{Scaling up to 100B examples leads to more notable improvements in low-resource languages. $\Delta$ denotes the improved accuracy when scaling from 10B examples to 100B.}
    \label{fig:multilinguality}
\end{figure}


\subsection{Fairness}
For fairness, we report on 3 metrics discussed in Section~\ref{sect:evals}. 

\paragraph{Representation Bias.} The first metric is representation bias (RB), with results detailed in Table~\ref{tab:rb}. We observe that models trained on unbalanced web data have a significantly higher preference to associate a randomly chosen image from ImageNet~\cite{deng2009imagenet} with the label ``Male'' over the label ``Female.'' 

In fact, this occurs nearly 85\% of the time. Training on 100B examples does not mitigate this effect. This finding aligns with previous research highlighting the necessity of bias mitigation strategies, such as data balancing~\cite{alabdulmohsin2024clip}, to address inherent biases in web-scale datasets.

% \begin{table}[h]
%     \centering\scriptsize
%     \caption{representation bias with respect to gender using imagenet. Here, a value of 0.8, for example, indicates that the model would prefer to associate a randomly chosen image from ImageNet with the label ``Male'' over the label ``Female''.}
%     \label{tab:rb}
%     \begin{tabularx}{\columnwidth}{@{}c|YYY@{}}
%     \toprule
%     \bf Model&\bf1B &\bf10B &\bf100B\\
%     \midrule
% B & 83.2&84.5&85.2
% \\
% L & 88.2&86.4&85.5\\
% H & 86.8&85.0&86.6\\
% \bottomrule
%     \end{tabularx}
% \end{table}

\begin{table}[h]
\begin{tabularx}{\columnwidth}{c|YYY@{}}
    \toprule
    \bf Model&\bf1B &\bf10B &\bf100B\\
    \midrule
B & 83.2&84.5&85.2\\
L & 88.2&86.4&85.5\\
H & 86.8&85.0&86.6\\
\bottomrule
\end{tabularx}
\captionof{table}{Representation bias w.r.t. gender (see Section~\ref{sect:results}). Here,  values [\%] indicate how often the model prefers to associate a random  image with the label ``Male'' over ``Female''.} \label{tab:rb}
\end{table}



\paragraph{Association Bias.} Second, Figure~\ref{fig:ab} shows the association bias in SigLIP-H/14 between gender and occupation as we scale the data from 10 to 100 billion examples. Specifically, we plot the probability that the model would prefer a particular occupation label, such as ``{\fontfamily{lmodern}\selectfont secretary}'' over another label, such as ``{\fontfamily{lmodern}\selectfont manager}'' when images correspond to males or females. In this evaluation, we use the Fairface~\cite{karkkainen2021fairface} dataset. The labels we compare are: ``{\fontfamily{lmodern}\selectfont librarian}'' vs. ``{\fontfamily{lmodern}\selectfont scientist}'', ``{\fontfamily{lmodern}\selectfont nurse}'' vs. ``{\fontfamily{lmodern}\selectfont doctor}'', ``{\fontfamily{lmodern}\selectfont housekeeper}'' vs. ``{\fontfamily{lmodern}\selectfont homeowner}'', ``{\fontfamily{lmodern}\selectfont receptionist}'' vs. ``{\fontfamily{lmodern}\selectfont executive}'' and ``{\fontfamily{lmodern}\selectfont secretary}'' vs. ``{\fontfamily{lmodern}\selectfont manager}''. Again, we do not see a reduction in association bias by simply increasing the size of the training data. %Full results are in Appendix~\ref{appendix:ab}.

%Additionally, we are unable to evaluate cultural diversity and fairness in PaliGemma's transfer tasks due to the lack of appropriate benchmarks. This is an open question that we hope to address in the future.

\paragraph{Performance Disparity.} Finally, one common definition of fairness in machine learning is maintaining similar performance across different groups. See, for instance,~\citet{dehghani2023scaling} and the related notions of ``Equality of Opportunity'' and ``Equalized Odds''~\cite{hardt2016equalityopportunitysupervisedlearning}. Table~\ref{tab:perf_disparity} show that scaling the data to 100 billion examples improves performance disparity, which is consistent with the improvement in cultural diversity.

%  to show on top of page
% \begin{table}[h]
    \centering\scriptsize
    \begin{tabularx}{\columnwidth}{l|YYY@{}}
    \toprule
    \bf Model & \bf 1B & \bf 10B & \bf 100B\\ \midrule
    &\multicolumn{3}{c}{\em 0-shot Dollar Street}\\[2pt]
B & 32.5 & 29.9 & \bf29.0\\
L & \bf29.7 & 29.8 & 30.4 \\
H & 32.2 & 33.0 & \bf32.1\\
\midrule 
    &\multicolumn{3}{c}{\em 0-shot GeoDE}\\[2pt]
B & 4.7 & 5.5 & \bf4.4\\
L & 3.2 & 4.0 & \bf2.8 \\
H & 3.6 & 3.0 & \bf2.7\\
\bottomrule
 
    \end{tabularx}
    \caption{Performance disparity (lower is better) for models pretrained on 100B seen examples of different data scales. Pretraining on 100B examples tends to lower disparity.}
    \label{tab:per_disp_mini}
\end{table}
% \FloatBarrier

\begin{table*}[h]
    \centering\scriptsize
    \caption{Performance disparity results for various SigLIP models pretrained on 100 billion seen examples of 1B, 10B, and 100B datasets. Here, disparity corresponds to the maximum gap across subgroups in Dollar Street (by income level) and GeoDE (by geographic region). Pretraining on 100B examples tends to improve disparity overall.}
    \label{tab:perf_disparity}
    \begin{tabularx}{2\columnwidth}{ll|YYYYYY|Y}
    \toprule
    \bf Model & \bf Data Scale &\multicolumn{6}{c}{\bf Performance per Subgroup} & \bf Disparity\\ \midrule
    \multicolumn{8}{c}{\em 0-shot Dollar Street}\\[2pt]
    & & \bf 0-200	& \bf 200-685	& \bf 685-1998	& \bf $>$1998
    & & & \\ \midrule
B&1B&29.4&43.9&56.5&62.0&&&32.5\\
B&10B&31.6&44.0&55.4&61.5&&&29.9\\
B&100B&32.0&44.3&56.3&61.0&&&\bf29.0\\[3pt]
L&1B&33.7&44.7&57.3&63.4&&&\bf29.7\\
L&10B&35.7&47.8&58.7&65.5&&&29.8\\
L&100B&33.7&46.6&59.5&64.1&&&30.4\\[3pt]
H&1B&32.3&44.9&58.4&64.5&&&32.2\\
H&10B&33.9&46.3&58.6&66.9&&&33.0\\
H&100B&34.1&48.2&62.2&66.1&&&\bf32.1\\ \midrule

    \multicolumn{8}{c}{\em 0-shot GeoDE}\\[2pt]
    & & \bf Africa	& \bf Americas	& \bf East-Asia	& \bf Europe & \bf South-East Asia & \bf West Asia
    & \\ \midrule
B&1B&89.4&92.1&91.8&94.1&92.5&93.4&4.7\\
B&10B&88.4&91.8&91.4&94.0&92.2&93.0&5.5\\
B&100B&88.8&91.4&91.0&93.3&91.7&92.2&\bf4.4\\[3pt]
L&1B&92.0&94.0&94.0&95.2&94.2&94.9&3.2\\
L&10B&91.8&94.4&94.0&95.8&94.2&94.7&4.0\\
L&100B&93.5&95.1&95.4&96.2&95.0&95.8&\bf2.8\\[3pt]
H&1B&91.5&94.4&94.7&95.2&94.1&94.5&3.6\\
H&10B&93.4&95.4&95.0&96.5&95.1&95.6&3.0\\
H&100B&93.6&95.1&95.3&96.3&95.2&95.8&\bf2.7\\

 \bottomrule
    \end{tabularx}
\end{table*}


\subsection{Transfer To Generative Models}
\label{sec:transfer}

\begin{table}[h!]
\centering
\footnotesize

% Note: We removed 1b result to avoid confusion to readers. See https://docs.google.com/document/d/1YxRpUO7elSaviOQ5XIXtnUWajgYAF7FfRO0vmXQCUtU/edit?resourcekey=0-pPjeeIrYEXRuvnuBFZn5Uw&tab=t.0#heading=h.gaczi2wqv0go.
\begin{tabular}{p{0pt}l|rrrrr}
\toprule
& Data & Semantics & OCR & Multiling & RS & Avg \\
\midrule
% % source: https://docs.google.com/spreadsheets/d/1W5_VNitkO6k-HSKBV6sGX81EGXgma877m0gGsg8zPzw/edit?resourcekey=0-2zH_U5z5kL9I5Nnhg7SChQ&gid=1972601917#gid=1972601917
\includegraphics[width=8pt]{images/snowflake_2744-fe0f.png} & 1B & 76.0 & 66.8 & 67.0 & 92.3 & 73.6 \\
\includegraphics[width=8pt]{images/snowflake_2744-fe0f.png} & 10B & 75.4 & 65.2 & 66.3 & 91.9 & 72.7 \\
\includegraphics[width=8pt]{images/snowflake_2744-fe0f.png} & 100B & 76.4 & 67.0 & 66.9 & 92.1 & 73.9 \\
\includegraphics[width=8pt]{images/fire_1f525.png} & 1B & 77.1 & 69.5 & 66.9 & 92.0 & 75.1 \\
\includegraphics[width=8pt]{images/fire_1f525.png} & 10B & 76.4 & 66.9 & 66.0 & 91.8 & 73.7 \\
\includegraphics[width=8pt]{images/fire_1f525.png} & 100B & {77.2} & {70.0} & {67.0} & {91.8} & {75.3} \\
\bottomrule
\end{tabular}

\caption{
The PaliGemma transfer results of ViT-L/16 models pretrained on 10B and 100B examples, with both frozen (top) %({\includegraphics[width=8pt]{images/snowflake_2744-fe0f.png}}) 
and unfrozen (bottom) %({\includegraphics[width=8pt]{images/fire_1f525.png}}) 
vision components. Results are aggregated.
}
\label{tab:transfer_avg}
\end{table}


We use PaliGemma~\citep{beyer2024paligemma} with both frozen and unfrozen vision component to assess the transferability of our vision models, which were contrastively pre-trained on datasets of different scales. In Table~\ref{tab:transfer_avg}, when taking the noise level into consideration, we do not observe consistent performance gains across downstream tasks as we scale the pre-training dataset. More details can be found in Appendix~\ref{appendix:transfer}.


%\paragraph{Recognizing Sensitive Attributes.}
%Finally, we also report the performance of the models in recognizing sensitive attributes, following a similar evaluation in~\citet{radford2021learning}. We report the accuracy in predicting perceived gender in Fairface~\cite{karkkainen2021fairface} and predicting perceived race in UTK~\cite{utkface_url}. Overall, we observe that scaling the data to 100 billion examples improves this aspect of fairness as well. Table~\ref{tab:fairness_pred} provides the full results. We do not observe a particular pattern in this type of evaluation.
%\begin{table}[t]
    \centering\scriptsize
    \begin{tabularx}{\columnwidth}{@{}ll|YY@{}}
    \toprule
    \bf Model & \bf Data Scale & \bf Gender & \bf Race\\ \midrule
B	&	1B	& 91.0	& 58.6\\
B	&	10B	& 91.7 &	\bf59.5\\
B	&	100B &	\bf91.9	& 53.1\\[3pt]
L	&	1B	& 0.94.8	& \bf55.4\\
L	&	10B	& 93.9	& 53.1\\
L	&	100B &	\bf95.0	& 54.0\\[3pt]
H	&	1B	& 94.5	& \bf54.5\\
H	&	10B	& \bf95.4	& 54.3\\
H	&	100B	& \bf95.4	& 50.2 \\
 \bottomrule
    \end{tabularx}
    \caption{Accuracy in recognizing sensitive attributes using Fairface and UTK datasets. See Section~\ref{sect:results} for details.}
    \label{tab:fairness_pred}
\end{table}

\section{Discussion}
\subsection{Case Study}


Fig. \ref{fig:casestudy} shows 2-D UMAP \cite{mcinnes2020umapuniformmanifoldapproximation} projections of embedding vectors for PetClinic Microservices \cite{microapps2024petclinic} using VoyageAI, ME-unixcoder-340K, and ME-llm2vec-340K. ME-unixcoder and ME-llm2vec show clearer microservice clusters compared to VoyageAI and Fig. \ref{fig:mexample}. For instance, \textit{API-Gateway} service classes are split in VoyageAI's representation but closer in the other models. ME-llm2vec demonstrates the closest grouping within microservices and clearest separation between them. In fact, ME-llm2vec's figure shows only 6 clear outliers which we review in detail and display their names and neighbors.



The two \textit{MetricConfig} classes, \textit{ResourceNotFoundException} and \textit{CacheConfig} lack domain-specific terms since they are utility classes, which highlights the importance of separating them from domain-related ones during the decomposition. However, ME-llm2vec was able correctly represent classes with even slight domain hints. For instance, most models struggle to differentiate between the nearly identical entry-point classes (e.g. \textit{ConfigServerApplication}), as seen in Fig. \ref{fig:mexample} and \ref{fig:casestudy} while ME-llm2vec managed to correctly place them within their services. On the other hand, the class \textit{PetRequest}, which was closer to \textit{API-Gateway} instead of \textit{Customers}, shows an intriguing outlier. Despite ME-llm2vec correctly matching the "Pet" related classes, it failed with \textit{PetRequest}. its function as a Request object, which is typically associated with the Gateway pattern, is a potential reason. Notably, ME-llm2vec successfully identified \textit{API-Gateway} classes, differentiating them from \textit{Customers}. We find this interesting because \textit{API-Gateway} includes classes representing various bounded contexts, often causing confusion in other models. ME-llm2vec recognized these classes' distinct purpose, grouping them together despite their diverse domains.

% Both \textit{API-Gateway} and \textit{Customers} services contain a "PetType" class. But in \textit{Customers}'s case, this class was closer to the "Specialty" class from \textit{Vets}, which is likely due to nearly identical source code they have. 

\subsection{Discussion}


We designed the analysis component to be as abstract as possible to accommodate the rapidly evolving representation learning landscape. As new and improved embedding models are published, they can be integrated with minimal effort. While our evaluation results show that with ME-LLM2Vec, we can generate highly cohesive and consistent decompositions, one of our objectives is to highlight the potential of Language Models in generating more efficient representations than traditional approaches for the decomposition problem. In fact, MonoEmbed is both a decomposition approach (when considering the full approach) and an embedding model (when using models such as ME-LLM2Vec). These models can be used to enrich existing decomposition approaches. For example, MicroMiner's CodeBERT \cite{trabelsi2023microminer} can be replaced with ME-LLM2Vec and the GNN based methods \cite{desai2021cogcn,yedida2023deeply,mathai2022chgnn,qian2023gdcdvf} can be extended by using ME-LLM2Vec as the encoder. In fact, it can be used as an additional representation type in approaches such as \cite{khaled2022hydecomp,qian2023gdcdvf}. These models can be even extended further by incorporating unstructured inputs (e.g. resources, configurations, documentation) and different PLs.




\subsection{Threats to Validity}
\subsubsection{Internal Validity}
Clustering algorithms and decomposition approaches have hyper-parameters that can affect performance on evaluation benchmarks. To mitigate this threat, we compared their performance with different hyper-parameter inputs across a varied set of evaluation applications.

\subsubsection{External Validity}
To address the threat of our approach to generalize on monolithic applications and PLs, we used a large set of monolithic and microservices applications from related work \cite{kalia2021mono2micro,khaled2022hydecomp,yedida2023deeply,jin2021fosci} to benchmark decomposition approaches. 

\subsubsection{Construct Validity}
This threat can potentially be in the form of the evaluation metrics used in our experiments. In order to mitigate this threat, we employ established metrics in supervised learning tasks (RQ1-3) and different metrics from decomposition research \cite{khaled2022hydecomp,kalia2021mono2micro,jin2021fosci,yedida2023deeply,mathai2022chgnn} (RQ4). 
\section{Conclusion and future work}
In this study, we examined the ability of LLMs to produce self-generated counterfactual explanations (SCEs).
We design a prompt-based setup for evaluating the efficacy of \SCEs.
Our results show that LLMs consistently struggle with generating valid \SCEs. In many cases model prediction on a \SCE does not yield the same target prediction for which the model crafted the \SCE.
Surprisingly, we find that LLMs put significant emphasis on the context---the prediction on \SCE is significantly impacted by the presence of original prediction and instructions for generating the \SCE.
Based on this empirical evidence, we argue that LLMs are still far from being able to explain their own predictions counterfactually.
Our findings add to similar insights from recent studies on other forms of self-explanations~\cite{lanham2023measuring,tanneru2024quantifying}.



Our work opens several avenues for future work. Inspired by counterfactual data augmentation~\cite{sachdeva2023catfood}, one could include the counterfactual explanation capabilities a part of the LLM training process. This inclusion may enhance the counterfactual reasoning capabilities of the LLM. Follow ups should also explore the effect of prompt tuning, specifically, model-tailored prompts for generating \SCEs. These approaches might lead to better quality \SCEs.


We limited our investigation to open source models of upto 70B parameters. Extending our analysis to larger and more recent models, \eg, DeepSeek R1 671B, and closed source models like OpenAI o3 would be an interesting avenue for future work.

Finally, our experiments were limited to relatively simple tasks: classification and mathematics problems where the solution is an integer. This limitation was mainly due to the fact that it is difficult to automatically judge validity of answers for more open-ended language generation tasks like search and information retrieval. Scaling our analysis to such tasks would require significant human-annotation resources, and is an important direction for future investigations.


%%
%% The acknowledgments section is defined using the "acks" environment
%% (and NOT an unnumbered section). This ensures the proper
%% identification of the section in the article metadata, and the
%% consistent spelling of the heading.
\begin{acks}
Thank you to Joseph Asike, George Obaido, Innocent Obi Jr., and Andrew Shaw for additional advice and support. The work presented was funded in part by the NSF (award \# 2141506), the Sloan Foundation, and Google.
\end{acks}

%%
%% The next two lines define the bibliography style to be used, and
%% the bibliography file.
\bibliographystyle{ACM-Reference-Format}
\bibliography{main}


%%
%% If your work has an appendix, this is the place to put it.
\appendix
\newpage
\appendix
\onecolumn

\renewcommand{\thetable}{A\arabic{table}} % Prefix table numbers with 'A'
\renewcommand{\thefigure}{A\arabic{figure}} % Prefix figure numbers with 'A'
\renewcommand{\theequation}{A\arabic{equation}} % Prefix equation numbers with 'A'

\setcounter{table}{0} % Reset table counter
\setcounter{figure}{0} % Reset figure counter
\setcounter{equation}{0} % Reset equation counter

\section*{Appendix}

\section{Optimal Brain Surgeon Derivation}
\label{OBS_ALGORITHM}

In the original setup in OBS, we have a local quadratic model for the loss $L$ given by:
$$
    \delta L = L(w + \delta w) \approx L(w) + \nabla_w L^T \delta w + \frac{1}{2} \delta w^T H \delta w
$$
Since OBS is a pruning-after-training approach, they discarded the 1-st order component. Reducing the expression for saliency as:
$$
    \delta L = \frac{1}{2} \delta w^T H \delta w
$$
To remove a single parameter, the authors of OBS introduced the constraint $e_q^T \delta w + w_q = 0$, with $e_q$ being the $q^{\text{th}}$ canonical basis vector. The pruning is defined as a constrained optimization problem of the form:
$$
    \min_{\delta w \in \mathbb{R^d}} \left( \frac{1}{2} \delta w^T H \delta w\right),
    ~~\text{s.t}~~
    e_q^T \delta w + w_q = 0.
$$
And the choice of which parameter to remove becomes:
$$
    \min_{q \in \mathcal{Q}} \left\{
        \min_{\delta w \in \mathbb{R^d}} \left( \frac{1}{2} \delta w^T H \delta w\right),
        ~~\text{s.t}~~
        e_q^T \delta w + w_q = 0
    \right\}.
$$
To solve the internal problem, we use a Lagrange multiplier $\lambda$ to write the problem as an unconstrained optimization case as follows:
$$
    \mathcal{L}(\delta w, \lambda) =
    \frac{1}{2} \delta w^T H \delta w +
    \lambda(e_q^T \delta w + w_q).
$$
Then, to find the stationary conditions, we compute the partial derivatives with respect to $\delta w$ and $\lambda$, and equate them to 0, obtaining:
$$
    \nabla_{\delta w} \mathcal{L} = 
    H \delta w + \lambda e_q = 0 
    \rightarrow
    \delta w = - \lambda H^{-1} e_q
$$
$$
    \nabla_{\lambda} \mathcal{L} =
    e_q^T \delta w + w_q = 0
    \rightarrow
    e_q^T \delta w = -w_q
$$
With some replacements, we get:
$$
    e_q^T \delta w = -w_q
    \rightarrow
    e_q^T \left( 
        - \lambda H^{-1} e_q
    \right) = -w_q
    \rightarrow
    - \lambda e_q^T H^{-1} e_q = -w_q
    \rightarrow
    \lambda = \frac{w_q}{e_q^T H^{-1} e_q} = \frac{w_q}{[H^{-1}]_{qq}}
$$
$$
    \delta w = - \frac{w_q H^{-1} e_q}{[H^{-1}]_{qq}}
$$
Replacing the expression for $\delta w$ in the saliency expression, we have:
\begin{align*}
    \delta L = \frac{1}{2} \delta w^T H \delta w
    &= \frac{1}{2}\left(
        - \frac{w_q H^{-1} e_q}{[H^{-1}]_{qq}}
    \right)^T
    H
    \left(
        - \frac{w_q H^{-1} e_q}{[H^{-1}]_{qq}}
    \right)
    \nonumber \\
    &= 
    \frac{w_q^2}{2[H^{-1}]_{qq}^2}
    \left(
        H^{-1} e_q
    \right)^T
    H
    \left(
        H^{-1} e_q
    \right)
    \nonumber \\
    &= 
    \frac{w_q^2}{2[H^{-1}]_{qq}^2}
    e_q ^T
    H^{-1}
    e_q
    = 
    \frac{w_q^2}{2[H^{-1}]_{qq}^2}
    [H^{-1}]_{qq}
    = 
    \frac{w_q^2}{2[H^{-1}]_{qq}}
    \nonumber \\
\end{align*}
%------------------------------------------------------------------------------------------------
\newpage
\section{Fisher Brain Surgeon Sensitivity Derivation}
\label{FBSS_ALGORITHM}
As we considered a PBT setting, it is not possible to ignore the first-order term in the local quadratic approximation of the error as it could still be informative. In this case, our model for sensitivity is given by: 
$$
    \delta L = \nabla_w L^T \delta w + \frac{1}{2} \delta w^T H \delta w
$$
The process to remove a single parameter remains similar; the constraint $e_q^T \delta w + w_q = 0$, with $e_q$ is still valid, redefining the optimization problem as:
$$
    \min_{\delta w \in \mathbb{R^d}} \left(
        \nabla_w L^T \delta w +  \frac{1}{2} \delta w^T H \delta w
    \right),
    ~~\text{s.t}~~
    e_q^T \delta w + w_q = 0.
$$
And the choice of which parameter to remove becomes:
$$
    \min_{q \in \mathcal{Q}} \left\{
        \min_{\delta w \in \mathbb{R^d}} \left(
            \nabla_w L^T \delta w + \frac{1}{2} \delta w^T H \delta w
        \right),
        ~~\text{s.t}~~
        e_q^T \delta w + w_q = 0
    \right\}.
$$
Using a Lagrange multiplier $\lambda$ as in the reference case, we solve the following unconstrained optimization problem:
$$
    \mathcal{L}(\delta w, \lambda) =
    \nabla_w L^T \delta w + 
    \frac{1}{2} \delta w^T H \delta w +
    \lambda(e_q^T \delta w + w_q).
$$
With the following stationary conditions:
$$
    \nabla_{\delta w} \mathcal{L} = 
    \nabla_w L + H \delta w + \lambda e_q = 0 
    \rightarrow
    \delta w = - (\lambda H^{-1}e_q + H^{-1} \nabla_w L)
$$
$$
    \nabla_{\lambda} \mathcal{L} =
    e_q^T \delta w + w_q = 0
    \rightarrow
    e_q^T \delta w = -w_q
$$
The expression for $\lambda$ is redefined as follows:
\begin{align*}
    e_q^T \left(
        - (\lambda H^{-1}e_q + H^{-1} \nabla_w L)
    \right) 
    &= -w_q
    \nonumber \\
    \lambda e_q^T H^{-1} e_q + e_q^T H^{-1} \nabla_w L
    &= w_q
    \nonumber \\
    \lambda [H^{-1}]_{qq} 
    &= w_q - e_q^T H^{-1} \nabla_w L
    \nonumber \\
    \lambda
    &= \frac{w_q - e_q^T H^{-1} \nabla_w L}{[H^{-1}]_{qq}}
\end{align*}
Replacing the expression for $\delta w$ in our sensitivity expression, we have:
\begin{align*}
    \delta L = \nabla_w L^T \delta w + \frac{1}{2} \delta w^T H \delta w
    &= 
    \nabla_w L^T \left[
        - (\lambda H^{-1}e_q + H^{-1} \nabla_w L)
    \right]
    \nonumber \\
    &+
    \frac{1}{2}\left[
        - (\lambda H^{-1}e_q + H^{-1} \nabla_w L)
    \right]^T
    H
    \left[
        - (\lambda H^{-1}e_q + H^{-1} \nabla_w L)
    \right]
    \nonumber \\
    &= 
    - \lambda \nabla_w L^T H^{-1}e_q - \nabla_w L^T H^{-1} \nabla_w L
    \nonumber \\
    &+
    \frac{1}{2}\left[
        (\lambda H^{-1}e_q)^T + (H^{-1} \nabla_w L)^T
    \right]
    \left[
        \lambda H H^{-1}e_q + H H^{-1} \nabla_w L)
    \right]
    \nonumber \\
    &= 
    - \lambda \nabla_w L^T H^{-1}e_q - \nabla_w L^T H^{-1} \nabla_w L
    \nonumber \\
    &+
    \frac{1}{2}\left[
        (\lambda H^{-1}e_q)^T + (H^{-1} \nabla_w L)^T
    \right]
    \left[
        \lambda e_q + \nabla_w L
    \right]
    \nonumber \\
    &= 
    - \lambda \nabla_w L^T H^{-1}e_q - \nabla_w L^T H^{-1} \nabla_w L
    \nonumber \\
    &+
    \frac{1}{2}\left[
        (\lambda H^{-1}e_q)^T \lambda e_q
        + (H^{-1} \nabla_w L)^T \lambda e_q
        + (\lambda H^{-1}e_q)^T \nabla_w L
        + (H^{-1} \nabla_w L)^T \nabla_w L
    \right]
    \nonumber \\
    &= 
    - \lambda \nabla_w L^T H^{-1}e_q - \nabla_w L^T H^{-1} \nabla_w L
    \nonumber \\
    &+
    \frac{1}{2}\left[
        \lambda^2 e_q^T H^{-1} e_q
        + \lambda \nabla_w L^T H^{-1} e_q
        + \lambda e_q^T H^{-1} \nabla_w L
        + \nabla_w L^T H^{-1} \nabla_w L
    \right]
    \nonumber \\
    &= 
    \frac{1}{2}\left[
        \lambda^2 [H^{-1}]_{qq}
        - \lambda \nabla_w L^T H^{-1} e_q
        + \lambda e_q^T H^{-1} \nabla_w L
        - \nabla_w L^T H^{-1} \nabla_w L
    \right]
    \nonumber \\
\end{align*}
Finally, replacing the $\lambda$:
\begin{align*}
    \delta L 
    &= 
    \frac{1}{2}\left[
        \lambda^2 [H^{-1}]_{qq}
        - \lambda \nabla_w L^T H^{-1} e_q
        + \lambda e_q^T H^{-1} \nabla_w L
        - \nabla_w L^T H^{-1} \nabla_w L
    \right]
    \nonumber \\
    &= 
    \frac{1}{2[H^{-1}]_{qq}}\left[
        (w_q - e_q^T H^{-1} \nabla_w L)^2 
        + (w_q - e_q^T H^{-1} \nabla_w L)(e_q^T H^{-1} \nabla_w L - \nabla_w L^T H^{-1} e_q)
        - \nabla_w L^T H^{-1} \nabla_w L
    \right]
    \nonumber \\
    &= 
    \frac{1}{2[H^{-1}]_{qq}}[
        w_q^2
        - 2 w_q (e_q^T H^{-1} \nabla_w L)
        + (e_q^T H^{-1} \nabla_w L)^2
        + w_q (e_q^T H^{-1} \nabla_w L)
    \nonumber \\
        &- w_q (\nabla_w L^T H^{-1} e_q)
        - (e_q^T H^{-1} \nabla_w L)(e_q^T H^{-1} \nabla_w L)
        + (e_q^T H^{-1} \nabla_w L)(\nabla_w L^T H^{-1} e_q)
        - \nabla_w L^T H^{-1} \nabla_w L
    ]
    \nonumber \\
    &= 
    \frac{1}{2[H^{-1}]_{qq}}[
        w_q^2
        - w_q (e_q^T H^{-1} \nabla_w L)
        + (e_q^T H^{-1} \nabla_w L)^2
    \nonumber \\
        &- w_q (\nabla_w L^T H^{-1} e_q)
        - (e_q^T H^{-1} \nabla_w L)^2
        + (e_q^T H^{-1} \nabla_w L)(\nabla_w L^T H^{-1} e_q)
        - \nabla_w L^T H^{-1} \nabla_w L
    ]
    \nonumber \\
    &= 
    \frac{1}{2[H^{-1}]_{qq}}\left[
        w_q^2
        - 2 w_q (e_q^T H^{-1} \nabla_w L)
        + (e_q^T H^{-1} \nabla_w L)^2
        - \nabla_w L^T H^{-1} \nabla_w L
    \right]
    \nonumber \\
    &= 
    \frac{1}{2[\hat{F}^{-1}]_{qq}}
    \left[
        w_q - (e_q^T \hat{F}^{-1} \nabla \mathcal{L}(w_0))
    \right]^2
\end{align*}

%------------------------------------------------------------------------------------------------

\newpage
\section{Training and Testing Details}
\label{appendix:training_parameters}

We perform an 80:20 stratified split, with a constant seed, on the CIFAR10/100 training dataset to obtain a validation set with the same class distribution. For both datasets, we have a training set with 40,000 samples, a validation set with 10,000 samples, and a testing set of 10,000 samples. Validation is performed after each training step, and the weights of the best-performing validation step (based on top-1 accuracy) are utilized for the final evaluation on the testing set. Table \ref{tab:table_training_parameters} summarizes the training parameters.

\begin{table}[h]
\caption{Training parameters used for ResNet18 and VGG19 on the CIFAR-10/100 datasets.}
\label{tab:table_training_parameters}
\vskip 0.15in
\begin{center}
\begin{small}
\begin{sc}
\begin{tabular}{lcc}
\toprule
Parameter & ResNet18 & VGG19 \\
\midrule
Number of steps       & 160 & 160 \\
Criterion             & CE & CE \\
Optimizer             & SGD & SGD \\
Learning rate         & 0.01 & 0.1 \\
Momentum              & 0.9 & 0.9 \\
Weight decay          & $5 \times 10^{-4}$ & $1 \times 10^{-4}$ \\
Learning rate drops   & [60, 120] & [60, 120] \\
Learning rate drop factor & 0.2 & 0.1 \\
\bottomrule
\end{tabular}
\end{sc}
\end{small}
\end{center}
\vskip -0.1in
\end{table}

%------------------------------------------------------------------------------------------------

\newpage
\section{Results CIFAR10}
\subsection{ResNet18}
\label{appendix:CIFAR10_ResNet18}

\begin{table}[h]
\caption{Performance of different sensitivity methods for pruning evaluated using ResNet18 on the CIFAR-10 testset. The right side of the table presents our proposed criteria. The mean accuracy and standard deviation are reported across three initialization seeds for various sparsity levels. Baseline, no pruning: $91.78 \pm 0.09$.}
\label{tab:resnet18_cifar10_compressors}
\vskip 0.15in
\begin{center}
\begin{small}
\begin{sc}
\resizebox{\textwidth}{!}{%
\begin{tabular}{lccccc|cccc}
\toprule
Sparsity  & Random & Magnitude & GN & SNIP & GraSP & FD & FP & FTS & FBSS \\
\midrule
0.10  & 91.71 ± 0.21 & 91.72 ± 0.07 & 91.57 ± 0.15 & 91.72 ± 0.07 & 89.16 ± 0.05 & 91.87 ± 0.13 & 91.63 ± 0.21 & 91.53 ± 0.12 & 91.76 ± 0.08 \\
0.20  & 91.63 ± 0.11 & 91.42 ± 0.12 & 91.51 ± 0.09 & 91.64 ± 0.16 & 88.69 ± 0.34 & 91.50 ± 0.12 & 91.65 ± 0.14 & 91.53 ± 0.15 & 91.54 ± 0.13 \\
0.30  & 91.45 ± 0.18 & 91.61 ± 0.13 & 91.68 ± 0.20 & 91.65 ± 0.08 & 88.67 ± 0.26 & 91.65 ± 0.18 & 91.44 ± 0.27 & 91.49 ± 0.05 & 91.62 ± 0.07 \\
0.40  & 91.59 ± 0.18 & 91.06 ± 0.16 & 91.61 ± 0.09 & 91.55 ± 0.08 & 88.24 ± 0.33 & 91.51 ± 0.05 & 91.38 ± 0.13 & 91.56 ± 0.28 & 91.39 ± 0.05 \\
0.50  & 91.60 ± 0.06 & 91.32 ± 0.13 & 91.44 ± 0.13 & 91.22 ± 0.07 & 87.69 ± 0.15 & 91.30 ± 0.18 & 91.58 ± 0.16 & 91.46 ± 0.19 & 91.41 ± 0.05 \\
0.60  & 91.10 ± 0.16 & 91.18 ± 0.16 & 91.59 ± 0.13 & 91.24 ± 0.04 & 87.48 ± 0.55 & 91.34 ± 0.07 & 91.35 ± 0.16 & 91.40 ± 0.11 & 91.38 ± 0.18 \\
0.70  & 91.17 ± 0.04 & 91.07 ± 0.07 & 91.19 ± 0.17 & 91.33 ± 0.18 & 87.26 ± 0.34 & 91.34 ± 0.23 & 91.42 ± 0.23 & 91.18 ± 0.18 & 91.27 ± 0.14 \\
0.80  & 90.78 ± 0.08 & 91.10 ± 0.12 & 90.95 ± 0.35 & 90.74 ± 0.10 & 87.18 ± 0.51 & 90.95 ± 0.11 & 91.08 ± 0.06 & 90.94 ± 0.22 & 90.73 ± 0.33 \\
0.90  & 89.35 ± 0.13 & 89.88 ± 0.28 & 90.39 ± 0.23 & 90.36 ± 0.34 & 86.60 ± 0.51 & 90.04 ± 0.21 & 90.20 ± 0.08 & 90.55 ± 0.23 & 89.22 ± 0.30 \\
0.95  & 87.59 ± 0.11 & 89.23 ± 0.19 & 89.00 ± 0.05 & 89.31 ± 0.17 & 86.50 ± 0.05 & 88.61 ± 0.28 & 89.50 ± 0.18 & 89.47 ± 0.32 & 87.58 ± 0.25 \\
0.98  & 83.47 ± 0.20 & 85.70 ± 0.33 & 86.43 ± 0.05 & 87.26 ± 0.28 & 85.99 ± 0.08 & 85.61 ± 0.20 & 86.97 ± 0.22 & 87.24 ± 0.32 & 83.40 ± 0.74 \\
0.99  & 78.28 ± 0.45 & 71.99 ± 0.28 & 83.47 ± 0.15 & 84.54 ± 0.04 & 84.56 ± 0.46 & 82.13 ± 0.28 & 83.74 ± 0.48 & 84.85 ± 0.18 & 77.60 ± 1.02 \\
\bottomrule
\end{tabular}}
\end{sc}
\end{small}
\end{center}
\vskip -0.1in
\end{table}

%------------------------------------------------------------------------------------------------
\clearpage
\subsection{VGG19}
\label{appendix:CIFAR10_VGG19}

As discussed earlier, introducing a warm-up phase effectively mitigates layer collapse in data-dependent pruning methods. Here, we evaluate the impact of different warm-up durations by comparing no warm-up, a single warm-up epoch, and five warm-up epochs. Table \ref{tab:VGG19_cifar10_compressors} demonstrates how performance drastically degrades with increasing sparsity, ultimately leading to layer collapse at 0.90 sparsity. However, as shown in the results, a single warm-up epoch is sufficient to prevent collapse and stabilize pruning performance. Moreover, as seen in Table \ref{tab:VGG19_cifar10_compressors_warmup5}, increasing the warm-up period to five epochs provides no substantial additional improvement. This indicates that prolonged warm-up training is not necessary; a single training step is enough to achieve gradient stabilization and overcome layer collapse.

\begin{table}[h]
\caption{Performance of different sensitivity methods for pruning evaluated using VGG19 on the CIFAR-10 test set. The right side of the table presents our proposed criteria. The mean accuracy and standard deviation are reported across three initialization seeds for various sparsity levels. Baseline, no pruning: $89.21 \pm 0.22$.}
\label{tab:VGG19_cifar10_compressors}
\vskip 0.15in
\begin{center}
\begin{small}
\begin{sc}
\resizebox{\textwidth}{!}{%
\begin{tabular}{lccccc|cccc}
\toprule
Sparsity  & Random & Magnitude & GN & SNIP & GraSP & FD & FP & FTS & FBSS \\
\midrule
0.10  & 88.40 ± 0.95 & 89.12 ± 0.55 & 90.14 ± 0.10 & 90.16 ± 0.18 & 87.81 ± 1.66 & 90.20 ± 0.29 & 90.21 ± 0.37 & 90.25 ± 0.38 & 89.06 ± 0.75 \\
0.20  & 89.19 ± 0.22 & 89.65 ± 0.60 & 89.59 ± 0.69 & 90.06 ± 0.04 & 89.57 ± 0.34 & 89.91 ± 0.28 & 90.28 ± 0.55 & 89.80 ± 0.28 & 88.89 ± 0.76 \\
0.30  & 88.93 ± 0.83 & 88.77 ± 1.07 & 90.23 ± 0.09 & 89.88 ± 0.59 & 89.14 ± 0.19 & 90.25 ± 0.09 & 89.97 ± 0.26 & 90.46 ± 0.41 & 89.06 ± 0.36 \\
0.40  & 88.28 ± 1.08 & 89.38 ± 0.53 & 90.50 ± 0.23 & 89.79 ± 0.67 & 88.20 ± 0.31 & 90.51 ± 0.12 & 90.37 ± 0.24 & 90.23 ± 0.14 & 10.00 ± 0.00 \\
0.50  & 88.96 ± 0.82 & 89.03 ± 0.59 & 90.46 ± 0.60 & 90.38 ± 0.25 & 88.67 ± 0.23 & 89.54 ± 0.86 & 90.47 ± 0.52 & 90.19 ± 0.31 & 10.00 ± 0.00 \\
0.60  & 88.15 ± 0.68 & 89.47 ± 0.18 & 89.95 ± 0.30 & 90.32 ± 0.25 & 88.82 ± 0.32 & 90.02 ± 0.40 & 90.18 ± 0.33 & 90.14 ± 0.36 & 10.00 ± 0.00 \\
0.70  & 88.02 ± 0.53 & 89.63 ± 0.44 & 89.69 ± 0.42 & 89.23 ± 0.19 & 89.62 ± 0.81 & 89.85 ± 0.08 & 90.01 ± 0.34 & 10.00 ± 0.00 & 10.00 ± 0.00 \\
0.80  & 88.28 ± 0.34 & 89.62 ± 0.91 & 85.72 ± 0.63 & 89.39 ± 0.43 & 88.82 ± 0.14 & 10.00 ± 0.00 & 88.29 ± 0.11 & 10.00 ± 0.00 & 10.00 ± 0.00 \\
0.90  & 85.82 ± 0.19 & 89.29 ± 0.79 & 10.00 ± 0.00 & 80.85 ± 0.62 & 24.28 ± 20.2 & 10.00 ± 0.00 & 10.00 ± 0.00 & 10.00 ± 0.00 & 10.00 ± 0.00 \\
0.95  & 84.41 ± 0.05 & 10.00 ± 0.00 & 10.00 ± 0.00 & 10.00 ± 0.00 & 10.00 ± 0.00 & 10.00 ± 0.00 & 10.00 ± 0.00 & 10.00 ± 0.00 & 10.00 ± 0.00 \\
0.98  & 80.04 ± 0.90 & 10.00 ± 0.00 & 10.00 ± 0.00 & 10.00 ± 0.00 & 10.00 ± 0.00 & 10.00 ± 0.00 & 10.00 ± 0.00 & 10.00 ± 0.00 & 10.00 ± 0.00 \\
0.99  & 76.89 ± 0.26 & 10.00 ± 0.00 & 10.00 ± 0.00 & 10.00 ± 0.00 & 10.00 ± 0.00 & 10.00 ± 0.00 & 10.00 ± 0.00 & 10.00 ± 0.00 & 10.00 ± 0.00 \\
\bottomrule
\end{tabular}}
\end{sc}
\end{small}
\end{center}
\vskip -0.1in
\end{table}
\newpage
%------------------------------------------------------------------------------------------------
\begin{table*}[h]
\caption{Performance of different compression methods evaluated after 1 warmup epoch using VGG19 on the CIFAR-10 dataset. We report the mean accuracy between three initialization seeds across various sparsity levels. Baseline, no pruning: $89.21 \pm 0.22$.}
\label{tab:VGG19_cifar10_compressors_warmup1}
\vskip 0.15in
\begin{center}
\begin{small}
\begin{sc}
\resizebox{\textwidth}{!}{%
\begin{tabular}{lccccc|cccc}
\toprule
Sparsity  & Random & Magnitude & GN & SNIP & GraSP & FD & FP & FTS & FBSS \\
\midrule
0.80  & 88.73 ± 0.38 & 88.35 ± 0.54 & 86.76 ± 0.27 & 87.39 ± 0.66 & 87.24 ± 0.25 & 87.14 ± 0.45 & 87.00 ± 0.87 & 87.68 ± 0.33 & 64.33 ± 15.91 \\
0.90  & 87.26 ± 0.42 & 88.62 ± 0.49 & 85.96 ± 0.75 & 86.75 ± 0.76 & 87.47 ± 0.33 & 86.69 ± 0.72 & 87.09 ± 0.31 & 87.42 ± 0.21 & 46.16 ± 7.62 \\
0.95  & 85.47 ± 0.64 & 87.68 ± 0.49 & 86.66 ± 0.27 & 86.00 ± 1.10 & 86.71 ± 1.24 & 85.71 ± 1.35 & 86.73 ± 0.36 & 87.56 ± 0.62 & 46.30 ± 5.32 \\
0.98  & 80.44 ± 0.30 & 86.61 ± 0.62 & 84.72 ± 1.69 & 87.22 ± 0.23 & 86.45 ± 0.64 & 80.34 ± 6.43 & 86.07 ± 0.39 & 86.36 ± 0.29 & 49.05 ± 4.31 \\
0.99  & 77.24 ± 0.73 & 83.69 ± 1.36 & 80.28 ± 2.04 & 83.49 ± 1.77 & 85.39 ± 0.43 & 75.11 ± 7.80 & 84.40 ± 1.27 & 85.35 ± 1.05 & 47.10 ± 4.41 \\
\bottomrule
\end{tabular}}
\end{sc}
\end{small}
\end{center}
\vskip -0.1in
\end{table*} 
%------------------------------------------------------------------------------------------------

\begin{table}[h]
\caption{Performance of different sensitivity methods for pruning evaluated after 5 warmup epochs using VGG19 on the CIFAR-10 testset. The right side of the table presents our proposed criteria. The mean accuracy and standard deviation are reported across three initialization seeds for various sparsity levels. Baseline, no pruning: $89.21 \pm 0.22$.}
\label{tab:VGG19_cifar10_compressors_warmup5}
\vskip 0.15in
\begin{center}
\begin{small}
\begin{sc}
\resizebox{\textwidth}{!}{%
\begin{tabular}{lccccc|cccc}
\toprule
Sparsity  & Random & Magnitude & GN & SNIP & GraSP & FD & FP & FTS & FBSS \\
\midrule
0.80  & 88.84 ± 0.43 & 88.41 ± 0.47 & 87.58 ± 0.52 & 88.15 ± 1.09 & 86.77 ± 1.14 & 87.28 ± 0.90 & 88.22 ± 0.82 & 86.68 ± 0.61 & 70.52 ± 9.25 \\
0.90  & 87.56 ± 0.62 & 88.60 ± 0.93 & 86.73 ± 0.37 & 87.89 ± 0.25 & 87.10 ± 0.47 & 87.50 ± 1.42 & 88.18 ± 0.47 & 86.98 ± 0.14 & 47.78 ± 1.26 \\
0.95 & 85.51 ± 0.69 & 87.66 ± 1.19 & 87.44 ± 0.46 & 87.71 ± 0.82 & 87.05 ± 0.16 & 86.83 ± 1.47 & 87.36 ± 0.52 & 87.00 ± 0.74 & 48.83 ± 2.52 \\
0.98 & 82.09 ± 0.17 & 86.24 ± 0.52 & 84.66 ± 1.33 & 86.55 ± 0.84 & 86.04 ± 0.66 & 85.44 ± 0.64 & 86.64 ± 0.13 & 84.89 ± 0.51 & 49.48 ± 0.85 \\
0.99 & 77.22 ± 1.03 & 83.93 ± 1.80 & 81.62 ± 2.17 & 84.53 ± 0.70 & 81.33 ± 5.77 & 81.71 ± 1.41 & 85.02 ± 0.69 & 83.78 ± 0.80 & 41.24 ± 1.55 \\
\bottomrule
\end{tabular}}
\end{sc}
\end{small}
\end{center}
\vskip -0.1in
\end{table}

%------------------------------------------------------------------------------------------------

\newpage
\section{Results CIFAR100}
\subsection{ResNet18}
\label{sec:resnet_cifar-100}

CIFAR-100 results exhibit a similar trend to those observed on CIFAR-10, further reinforcing the robustness of our proposed Fisher-Taylor Sensitivity (FTS) criterion. Across all evaluated sparsity levels, FTS consistently maintains strong performance, frequently ranking among the top-performing methods. This trend is particularly evident at extreme sparsities, where many pruning approaches suffer significant performance degradation. The stability of FTS across both datasets highlights its effectiveness in preserving network expressivity despite aggressive pruning.

\begin{table}[h]
\caption{Performance of different compression methods evaluated using ResNet18 on the CIFAR-100 dataset. We report the mean accuracy between three initialization seeds across various sparsity levels. Baseline, no pruning: $69.57 \pm 0.19$.}
\label{tab:resnet18_cifar100_compressors}
\vskip 0.15in
\begin{center}
\begin{small}
\begin{sc}
\resizebox{\textwidth}{!}{%
\begin{tabular}{lccccc|cccc}
\toprule
Sparsity  & Random & Magnitude & GN & SNIP & GraSP & FD & FP & FTS & FBSS \\
\midrule
0.10  & 69.16 ± 0.11 & 69.37 ± 0.14 & 69.63 ± 0.34 & 69.42 ± 0.07 & 64.26 ± 0.27 & 69.66 ± 0.30 & 69.08 ± 0.21 & 69.16 ± 0.11 & 69.07 ± 0.10 \\
0.20  & 69.16 ± 0.30 & 69.06 ± 0.24 & 69.19 ± 0.11 & 69.30 ± 0.08 & 63.28 ± 0.58 & 69.60 ± 0.30 & 69.35 ± 0.35 & 69.41 ± 0.43 & 69.07 ± 0.20 \\
0.30  & 69.36 ± 0.18 & 68.58 ± 0.36 & 69.37 ± 0.13 & 68.82 ± 0.17 & 62.02 ± 0.43 & 69.24 ± 0.40 & 68.84 ± 0.13 & 68.80 ± 0.55 & 68.96 ± 0.11 \\
0.40  & 69.41 ± 0.20 & 68.50 ± 0.29 & 69.16 ± 0.26 & 68.95 ± 0.19 & 61.18 ± 0.19 & 69.17 ± 0.16 & 68.88 ± 0.25 & 69.02 ± 0.21 & 68.92 ± 0.25 \\
0.50  & 69.12 ± 0.46 & 68.17 ± 0.20 & 68.94 ± 0.20 & 68.63 ± 0.11 & 61.11 ± 0.40 & 69.13 ± 0.13 & 68.68 ± 0.12 & 68.71 ± 0.12 & 68.71 ± 0.57 \\
0.60  & 68.66 ± 0.27 & 67.78 ± 0.35 & 68.77 ± 0.17 & 68.63 ± 0.42 & 61.40 ± 0.78 & 68.34 ± 0.43 & 67.98 ± 0.23 & 68.41 ± 0.14 & 68.60 ± 0.15 \\
0.70  & 67.95 ± 0.43 & 67.51 ± 0.24 & 68.29 ± 0.39 & 68.08 ± 0.18 & 59.43 ± 0.76 & 68.03 ± 0.46 & 67.96 ± 0.15 & 68.29 ± 0.06 & 68.16 ± 0.07 \\
0.80  & 67.26 ± 0.48 & 66.55 ± 0.19 & 67.20 ± 0.37 & 67.21 ± 0.38 & 59.08 ± 0.22 & 66.70 ± 0.05 & 67.05 ± 0.06 & 66.77 ± 0.65 & 66.62 ± 0.43 \\
0.90  & 64.75 ± 0.16 & 64.48 ± 0.18 & 64.87 ± 0.27 & 65.70 ± 0.08 & 59.16 ± 0.91 & 64.74 ± 0.44 & 65.46 ± 0.30 & 65.41 ± 0.13 & 63.90 ± 0.31 \\
0.95  & 61.01 ± 0.32 & 62.20 ± 0.06 & 62.20 ± 0.23 & 63.20 ± 0.20 & 57.91 ± 0.09 & 62.14 ± 0.42 & 63.22 ± 0.25 & 63.21 ± 0.47 & 61.25 ± 0.44 \\
0.98  & 54.72 ± 0.22 & 55.44 ± 0.18 & 57.34 ± 0.31 & 58.83 ± 0.35 & 54.85 ± 0.35 & 55.57 ± 0.17 & 58.05 ± 0.18 & 58.59 ± 0.12 & 55.02 ± 0.34 \\
0.99  & 45.62 ± 0.55 & 40.39 ± 0.36 & 50.46 ± 0.61 & 52.96 ± 0.10 & 49.13 ± 0.19 & 48.02 ± 0.32 & 49.98 ± 0.60 & 52.85 ± 0.24 & 44.91 ± 0.52 \\
\bottomrule
\end{tabular}}
\end{sc}
\end{small}
\end{center}
\vskip -0.1in
\end{table}

%------------------------------------------------------------------------------------------------
\clearpage
\subsection{VGG19}
The results on VGG19 with CIFAR-100 exhibit a similar trend to those observed on CIFAR-10, reinforcing the effectiveness of our proposed approach. Once again, we identify the occurrence of layer collapse at extreme sparsities when no warm-up is applied, leading to a significant drop in accuracy. Introducing a single warm-up epoch effectively resolves this issue, restoring pruning performance across all evaluated criteria. However, increasing the warm-up phase to five epochs does not yield any additional advantage, indicating that a brief warm-up period is sufficient to stabilize gradient-based importance scores and prevent collapse.

\label{sec:vgg_cifar-100}

\begin{table}[h]
\caption{Performance of different compression methods evaluated using VGG19 on the CIFAR-100 dataset. We report the mean accuracy between three initialization seeds across various sparsity levels. Baseline, no pruning: $58.96 \pm 2.30$.}
\label{tab:VGG19_cifar100_compressors}
\vskip 0.15in
\begin{center}
\begin{small}
\begin{sc}
\resizebox{\textwidth}{!}{%
\begin{tabular}{lccccc|cccc}
\toprule
Sparsity & Random & Magnitude & GN & SNIP & GraSP & FD & FP & FTS & FBSS \\
\midrule
0.10  & 60.31 ± 0.40 & 59.13 ± 1.29 & 61.93 ± 0.48 & 61.98 ± 0.29 & 59.32 ± 0.63 & 62.13 ± 0.61 & 60.45 ± 3.47 & 61.56 ± 1.04 & 58.79 ± 0.98 \\
0.20  & 60.43 ± 1.14 & 59.27 ± 0.34 & 62.64 ± 0.21 & 62.68 ± 0.24 & 61.21 ± 0.41 & 63.04 ± 0.43 & 62.71 ± 1.02 & 62.24 ± 0.44 & 60.48 ± 0.48 \\
0.30  & 58.32 ± 0.60 & 59.35 ± 1.43 & 62.61 ± 0.23 & 63.11 ± 0.35 & 59.30 ± 0.43 & 62.85 ± 0.42 & 61.43 ± 0.61 & 62.65 ± 0.54 & 58.77 ± 1.02 \\
0.40  & 56.50 ± 3.20 & 60.04 ± 1.02 & 62.36 ± 0.02 & 62.39 ± 0.55 & 56.34 ± 1.49 & 62.38 ± 0.75 & 61.56 ± 1.25 & 62.67 ± 0.06 & 1.00 ± 0.00 \\
0.50  & 58.47 ± 1.49 & 61.49 ± 1.22 & 62.02 ± 0.64 & 62.76 ± 0.50 & 54.43 ± 0.84 & 62.84 ± 0.33 & 62.25 ± 0.33 & 62.47 ± 0.42 & 1.00 ± 0.00 \\
0.60  & 57.54 ± 0.74 & 61.50 ± 0.30 & 62.55 ± 0.13 & 63.08 ± 0.55 & 56.76 ± 0.69 & 62.40 ± 0.57 & 62.70 ± 0.63 & 62.17 ± 0.23 & 1.00 ± 0.00 \\
0.70  & 57.63 ± 0.80 & 61.71 ± 0.25 & 60.85 ± 0.79 & 60.58 ± 0.39 & 57.76 ± 0.84 & 60.44 ± 0.34 & 60.92 ± 0.41 & 60.51 ± 1.67 & 1.00 ± 0.00 \\
0.80  & 57.84 ± 0.57 & 61.89 ± 1.02 & 55.09 ± 0.49 & 59.84 ± 0.29 & 58.39 ± 0.74 & 1.00 ± 0.00 & 43.16 ± 1.02 & 58.66 ± 2.28 & 1.00 ± 0.00 \\
0.90  & 58.41 ± 0.41 & 62.60 ± 0.91 & 1.00 ± 0.00 & 8.35 ± 10.39 & 42.88 ± 1.64 & 1.00 ± 0.00 & 1.00 ± 0.00 & 8.87 ± 11.13 & 1.00 ± 0.00 \\
0.95  & 54.84 ± 1.08 & 1.00 ± 0.00 & 1.00 ± 0.00 & 1.00 ± 0.00 & 1.00 ± 0.00 & 1.00 ± 0.00 & 1.00 ± 0.00 & 1.00 ± 0.00 & 1.00 ± 0.00 \\
0.98  & 50.21 ± 0.72 & 1.00 ± 0.00 & 1.00 ± 0.00 & 1.00 ± 0.00 & 1.00 ± 0.00 & 1.00 ± 0.00 & 1.00 ± 0.00 & 1.00 ± 0.00 & 1.00 ± 0.00 \\
0.99  & 46.69 ± 0.45 & 1.00 ± 0.00 & 1.00 ± 0.00 & 1.00 ± 0.00 & 1.00 ± 0.00 & 1.00 ± 0.00 & 1.00 ± 0.00 & 1.00 ± 0.00 & 1.00 ± 0.00 \\
\bottomrule
\end{tabular}}
\end{sc}
\end{small}
\end{center}
\vskip -0.1in
\end{table}

%------------------------------------------------------------------------------------------------

\begin{table}[h]
\caption{Performance of different compression methods evaluated after 1 warmup epoch using VGG19 on the CIFAR-100 dataset. We report the mean accuracy between three initialization seeds across various sparsity levels. Baseline, no pruning: $58.96 \pm 2.30$.}
\label{tab:VGG19_cifar100_compressors_warmup1}
\vskip 0.15in
\begin{center}
\begin{small}
\begin{sc}
\resizebox{\textwidth}{!}{%
\begin{tabular}{lccccc|cccc}
\toprule
Sparsity & Random & Magnitude & GN & SNIP & GraSP & FD & FP & FTS & FBSS \\
\midrule
0.80  & 60.39 ± 1.16 & 58.91 ± 0.41 & 52.81 ± 1.32 & 55.62 ± 2.27 & 55.15 ± 2.25 & 56.71 ± 0.31 & 58.03 ± 0.93 & 52.41 ± 3.07 & 52.74 ± 5.16 \\
0.90  & 58.90 ± 0.98 & 60.95 ± 0.81 & 50.56 ± 4.59 & 55.89 ± 2.05 & 56.01 ± 1.58 & 52.07 ± 3.24 & 53.65 ± 0.57 & 52.45 ± 3.75 & 19.65 ± 1.68 \\
0.95  & 56.10 ± 0.85 & 57.64 ± 2.63 & 50.34 ± 1.00 & 53.70 ± 3.60 & 56.16 ± 0.41 & 54.44 ± 1.38 & 53.24 ± 3.54 & 53.56 ± 1.26 & 17.24 ± 0.44 \\
0.98  & 50.97 ± 0.40 & 54.66 ± 2.56 & 43.43 ± 5.32 & 50.19 ± 1.59 & 54.64 ± 1.50 & 42.75 ± 1.91 & 50.59 ± 3.39 & 48.56 ± 5.25 & 16.42 ± 0.64 \\
0.99  & 46.52 ± 0.45 & 43.33 ± 5.83 & 33.90 ± 5.35 & 42.65 ± 5.32 & 45.98 ± 4.48 & 29.67 ± 8.49 & 49.11 ± 3.46 & 48.70 ± 2.59 & 13.25 ± 0.84 \\
\bottomrule
\end{tabular}}
\end{sc}
\end{small}
\end{center}
\vskip -0.1in
\end{table}


%------------------------------------------------------------------------------------------------

\begin{table}[h]
\caption{Performance of different compression methods evaluated after 5 warmup epochs using VGG19 on the CIFAR-100 dataset. We report the mean accuracy between three initialization seeds across various sparsity levels. Baseline, no pruning: $58.96 \pm 2.30$.}
\label{tab:VGG19_cifar100_compressors_warmup5}
\vskip 0.15in
\begin{center}
\begin{small}
\begin{sc}
\resizebox{\textwidth}{!}{%
\begin{tabular}{lccccc|cccc}
\toprule
Sparsity & Random & Magnitude & GN & SNIP & GraSP & FD & FP & FTS & FBSS \\
\midrule
0.80  & 60.41 ± 1.39 & 58.38 ± 0.85 & 60.86 ± 0.79 & 61.63 ± 0.45 & 56.25 ± 0.49 & 59.59 ± 0.76 & 59.37 ± 3.50 & 60.86 ± 0.53 & 46.93 ± 9.04 \\
0.90  & 60.32 ± 0.09 & 57.74 ± 1.64 & 57.77 ± 2.41 & 58.23 ± 4.07 & 56.27 ± 1.02 & 60.19 ± 0.63 & 61.23 ± 0.50 & 60.52 ± 0.37 & 21.66 ± 1.95 \\
0.95 & 57.86 ± 0.53 & 59.55 ± 1.15 & 56.09 ± 0.97 & 58.83 ± 0.65 & 55.26 ± 1.25 & 55.80 ± 2.77 & 59.83 ± 0.94 & 58.52 ± 1.32 & 19.98 ± 2.62 \\
0.98 & 51.75 ± 0.43 & 47.75 ± 7.63 & 52.26 ± 4.06 & 55.27 ± 1.69 & 54.59 ± 0.96 & 49.46 ± 4.98 & 57.40 ± 1.26 & 56.00 ± 1.08 & 17.59 ± 1.36 \\
0.99 & 47.59 ± 0.80 & 42.46 ± 7.95 & 46.58 ± 2.00 & 53.13 ± 0.84 & 53.91 ± 1.53 & 42.87 ± 4.63 & 53.17 ± 1.18 & 53.05 ± 2.14 & 13.92 ± 0.14 \\
\bottomrule
\end{tabular}}
\end{sc}
\end{small}
\end{center}
\vskip -0.1in
\end{table}


%------------------------------------------------------------------------------------------------
\clearpage

\section{Mask Batch Size for Other Sparsities}
The Effect of batch size on pruning performance across different sparsities. 
As sparsity increases, the effect of batch size on pruning performance becomes more pronounced. 
At lower sparsities (0.90, 0.95), the differences across batch sizes are less evident, suggesting that even smaller batches provide a reasonable estimation of parameter importance. However, at extreme sparsities (0.98, 0.99), we observe a clear trend where larger batch sizes consistently lead to better parameter selection, ultimately improving accuracy. This aligns with our hypothesis that larger batches help reduce variance in gradient estimation, leading to more stable and effective pruning decisions. 
\label{batch_size_heatmaps}

\begin{figure}[h]
    \centering
    \includegraphics[width=0.8\linewidth]{imgs/cifar10_resnet18_heatmap_warmup_0.png}
    \caption{Effect of batch size on pruning performance at increasing sparsities.}
    \label{fig:enter-label}
\end{figure}

%------------------------------------------------------------------------------------------------

\clearpage
\section{Comparison of our criteria with magnitude-based pruning}

Figure \ref{fig:our_criterion_vs_magnitude} illustrates the relationship between parameter magnitude and different sensitivity-based pruning metrics. Each point represents a model parameter, with red points indicating the top-ranked parameters selected for retention by each criterion. The green dashed line marks the 99th percentile of parameter magnitudes.

A key observation is that the most effective pruning criteria, such as Fisher-Taylor Sensitivity, tend to retain parameters with a broad range of magnitudes, including many that are relatively small (left of the green line). This shows that the estimated importance does not always prioritize parameters based on their magnitude. 


\begin{figure}[htp]
    \centering
    \includegraphics[width=0.9\linewidth]{imgs/cifar_10_mag_vs_criteria_s_99.png}
    \caption{Our criteria vs. Magnitude parameter selection for 99\% sparsity (ResNet18, CIFAR-10, Seed 0)} 
    \label{fig:our_criterion_vs_magnitude}
\end{figure}


\end{document}
\endinput
%%
%% End of file `sample-sigconf.tex'.
