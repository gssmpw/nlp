\section{Experimental evaluation}
\label{sec:experimental}

\subsection{Test procedure}
\label{sec:experiment}
We conducted experimental tests to assess the effectiveness
of data augmentation using MoireDB, comparing the results
against robustness values obtained via several alternative
models: data augmentation using Fractal arts and FVis, as originally
proposed for PixMix, and PixMix with data augmentation
images taken from FractalDB and VisualAtom.

The training model we use is WideResNet~\cite{PIXMIX,WideResNet}.
We use CIFAR as a training-image dataset.
For each of the various data augmentation strategies, we create
an augmented version of the CIFAR training-image dataset,
then train WideResNet on the augmented dataset for 100 epochs
and measure the robustness of the trained model.
Robustness is measured on the CIFAR-C dataset of test images
using the Corruptions and Adversaries evaluation tasks~\cite{PIXMIX}.

The Corruptions task involves using CIFAR-C to measure
robustness against image corruption~\cite{ImageNet-C}.
The metric for this assessment is the previously mentioned mCE, which is smaller for greater robustness.
mCE is computed as the mean image identification accuracy
for the 15 types of image corruption represented by CIFAR-C.

The Adversaries task involves measuring robustness
against adversarial attack~\cite{Adrobust}.
The metric for this assessment is the image identification accuracy,
with lower values indicating better performance.
Adversarial attacks are applied to CIFAR test images.


\subsection{Results of robustness tests}
\label{sec:evaluate}

Table~\ref{tabA} shows the results of tests to assess
the impact of MoireDB-based data augmentation on
the robustness of image classification.
The column labeled ``Baseline'' lists
results from the original PixMix paper~\cite{PIXMIX}.

From Table~\ref{tabA} we see that 
data augmentation using MoireDB achieves better
image identification robustness than any other method---including
data augmentation using Fractal arts---for both CIFAR-10-C and 
CIFAR-100-C.
Comparing results for the FDSL datasets 
FractalDB, VisualAtom, and MoireDB,
we see that, in \textit{every} test of robustness,
the largest robustness improvement
is achieved for data augmentation using MoireDB.

These results demonstrate that MoireDB-based data augmentation
can yield robustness improvements comparable to or greater than
data augmentation using Fractal arts or FVis.

To analyze the test results in greater detail,
we consider image classification accuracies
for the various types of image corruption in CIFAR-100-C.
From Table~\ref{tabNoise} we see that, for
all forms of image corruption caused by noise,
the greatest improvement in image classification robustness
is achieved by data augmentation using FVis.
On the other hand, from Table~\ref{tabBlur} we see that,
for various forms of blurring,
MoireDB tends to yield greater robustness improvements
than other image datasets. According to Table~\ref{tabBlur}, we can see MoireDB with PixMix performaed better results on the blurred noise types on the validation of CIFAR-100-C dataset.


Similarly, from Tables~\ref{tabWea} and~\ref{tabDig}
we see that, for image corruption due to snow or frost,
as well as for image corruption due to elastic deformation
or pixelation, data augmentation using MoireDB achieves
the greatest improvement in robustness.
