\section{Data augmentation using formulagenerated Moir´e images}
\label{sec:method}

In the present study, we propose MoireDB, a formula-generated image
dataset for data augmentation. Our goal is to ensure that
training on MoireDB-augmented image datasets
increases the robustness of image recognition models
for classification tasks. Section~\ref{sec:MoireDB-Render}
describes our procedure for generating the Moir\'e images
comprising MoireDB, while Section~\ref{sec:MoireMIX-Render}
discusses our strategy for data augmentation using
generated Moir\'e images.



As noted above, the idea of generating Moir\'e images
for use in data augmentation is motivated by the hypothesis
that using and against illusory images for data augmentation should
improve robustness.

\subsection{Generation of Moir\'e images}
\label{sec:MoireDB-Render}

% \figPIXMIXmoire
% Use figure* for multi-column figure

\begin{figure*}[t]
    \vspace{-12pt}
    \centering
    % \includegraphics[width=14.0cm]{figure/PIXMIX_moireDB.pdf}
    \includegraphics[width=14.0cm]{figs/en-fig5-.pdf}
    \vspace{-12pt} 
    \caption{Example illustrating PIXMIX-style data augmentation
             with Moir\'e images~\cite{ImageNet-C,PIXMIX}.}
    \label{figPIXMIXmoire}
    \vspace{-12pt}
\end{figure*}





\begin{table*}[t]
    \vspace{12pt}
    \caption{Adjustable parameters for auto-generated Moir\'e images}
    \centering
    \vspace{-8pt}
    \begin{tabular}{lc|c} \toprule[0.8pt]
        Parameter                    & Symbol     & Range \\\midrule[0.8pt]
%
        Interval frequency           & $\nu$      & $0.01\leq \nu < 0.05$ \\[1ex]
%
        Center-point coordinates           & $x_c, y_c$ & $0\leq (x_c, y_c) < 600$  \\[1ex]
%
        %\begin{minipage}[c]{0.5\columnwidth}
         Number of superposed concentric-circle patterns & $Q_n$       & $Q_n = \{1, 2, 3\}$  \\
        %\end{minipage}               
        \bottomrule[0.8pt] \\
    \end{tabular}
    \vspace{-20pt}
    \label{tab:MoireDBparam}
\end{table*}


Our algorithm for generating the Moir\'e images constituting
MoireDB is depicted schematically in Fig.~\ref{figMoire}.
The starting point is a simple procedure
(Fig.~\ref{figMoire}, far left) for generating a
concentric-circle pattern; this procedure is described
by a formula, discussed below, containing multiple
adjustable parameters such as the coordinates 
$(x_c,y_c)$ of the common center point.
To generate a single Moir\'e image, we invoke this 
formula multiple times---with randomly chosen values
for the adjustable parameters---to yield a set 
of multiple distinct concentric-circle patterns 
(Fig.~\ref{figMoire}, center), then simply \textit{superpose}
these to yield the Moir\'e image (Fig.~\ref{figMoire}, far right).
The superposition of randomly generated
concentric-circle patterns gives rise to the characteristic
interference fringes of Moir\'e images, and varying the
adjustable parameters defining the concentric-circle
patterns allows a wide range of distinct fringe patterns to be
realized.

The image representing each concentric-circle pattern is generated
by a formula that computes a brightness value for each pixel in the image.
Each Moir\'e image depends on several adjustable parameters:
the number $Q_n$ of concentric-circle patterns superposed, and,
for each of these patterns, the center-point coordinates $(x_c, y_c)$
and an interval frequency parameter $\nu$ described below.
Values for all of these parameters are chosen randomly within
the ranges listed in Table~\ref{tab:MoireDBparam}.
 
Each concentric-circle pattern may be described as a superposition
of circles of the form

\begin{equation}
f_{Q_n}=\frac{1}{Q_n}\sum^{m}_{k=1}\eta_k\in\mathbb{R}^2
\end{equation}
%
where $m$ is the number of circles drawn in the pattern
and $\eta_k$ represents the $k$-th circle. Denoting the radius of 
this circle by $r_k$, and recalling that the circle is centered
at $(x_c,y_c)$, we may express $\eta_k$ in the form 

\begin{equation}
  \eta_k=
  \left\{
    \begin{array}{l}
      x=(r_k\cos{\theta}+x_c)\times g \\
      y=(r_k\sin{\theta}+y_c)\times g
    \end{array}
  \right.
  (0\leq\theta<2\pi)
\end{equation}
%
The center-point coordinates ($x_c$,$y_c$) are chosen at random from 
a uniform distribution.
The quantity $g$ in this expression, representing the brightness
at point $(x,y)$, is a sinusoidally varying function of the
radial distance $r$:

\begin{equation}
g=(V_M(cos(\nu \times \pi \times r)) + 1)\times255,
\end{equation}
%
where $V_M$ is the amplitude of the sinusoidal brightness variation.
Using the brightness $g$ to define a grayscale value for each pixel
yields an image representing the concentric-circle pattern. Choosing
the number of concentric-circle patterns $Q_n>1$ then ensures
interference between the patterns, yielding the desired Moir\'e image.

We set the size of generated images to be $512\times 512$ [pixel];
the number of circles $m$ drawn for each concentric-circle pattern
is determined as appropriate based on the image size and the 
interval frequency $\nu$.

\tabA
\tabNoise
\tabBlur

\subsection{Data augmentation using Moir\'e images}
\label{sec:MoireMIX-Render}

Our strategy for data augmentation using Moir\'e images
is outlined schematically in Fig.~\ref{figPIXMIXmoireConcept},
and Fig.~\ref{figPIXMIXmoire} shows a detailed diagram
of the operational pipeline of our PixMix implementation
with Moir\'e images, in this case for an example involving
1 additive mixing operation and 2 multiplicative mixing
operations. Our data augmentation procedure is the same as
that used in PixMix.
The number of Moir\'e images we generate for data augmentation
is 14,230, chosen to match the number of Fractal arts used in PixMix.
For each image, the parameter values in the image-generation
formulas are chosen at random from the ranges listed in
Table~\ref{tab:MoireDBparam}.
For each mixing step, we choose an image at random from the 
set of generated images and mix it either additively or 
multiplicatively with the selected Moir\'e image or with 
the input image.

% \tabA
% \tabNoise
% \tabBlur
\tabWea
\tabDig
