%%%%%%%% ICML 2025 EXAMPLE LATEX SUBMISSION FILE %%%%%%%%%%%%%%%%%

\documentclass{article}

% Recommended, but optional, packages for figures and better typesetting:
\usepackage{microtype}
\usepackage{graphicx}
\usepackage{subfigure}
\usepackage{booktabs} % for professional tables
\usepackage{amssymb}
\usepackage{multirow}
% \usepackage{apacite}

% hyperref makes hyperlinks in the resulting PDF.
% If your build breaks (sometimes temporarily if a hyperlink spans a page)
% please comment out the following usepackage line and replace
% \usepackage{icml2025} with \usepackage[nohyperref]{icml2025} above.
\usepackage{hyperref}

\usepackage{color}
\usepackage{xspace}
\newcommand{\name}[0]{ \xspace} 
% general commands
\newcommand{\TODO}[1]{{\textbf{\textcolor{red}{TODO: #1}}}}
\newcommand{\NOTE}[1]{{\textbf{\textcolor{red}{NOTE: #1}}}}
\newcommand{\HK}[1]{\textcolor{red}{[KunHe: #1]}}
\newcommand{\WJR}[1]{\textcolor{blue}{[WJR: #1]}}

% Attempt to make hyperref and algorithmic work together better:
\newcommand{\theHalgorithm}{\arabic{algorithm}}

% Use the following line for the initial blind version submitted for review:
% \usepackage{icml2025}

% If accepted, instead use the following line for the camera-ready submission:
\usepackage[accepted]{icml2025}

% For theorems and such
\usepackage{amsmath}
\usepackage{amssymb}
\usepackage{mathtools}
\usepackage{amsthm}

% if you use cleveref..
\usepackage[capitalize,noabbrev]{cleveref}

%%%%%%%%%%%%%%%%%%%%%%%%%%%%%%%%
% THEOREMS
%%%%%%%%%%%%%%%%%%%%%%%%%%%%%%%%
\theoremstyle{plain}
\newtheorem{theorem}{Theorem}[section]
\newtheorem{proposition}[theorem]{Proposition}
\newtheorem{lemma}[theorem]{Lemma}
\newtheorem{corollary}[theorem]{Corollary}
\theoremstyle{definition}
\newtheorem{definition}[theorem]{Definition}
\newtheorem{assumption}[theorem]{Assumption}
\theoremstyle{remark}
\newtheorem{remark}[theorem]{Remark}

% Todonotes is useful during development; simply uncomment the next line
%    and comment out the line below the next line to turn off comments
%\usepackage[disable,textsize=tiny]{todonotes}
\usepackage[textsize=tiny]{todonotes}


% The \icmltitle you define below is probably too long as a header.
% Therefore, a short form for the running title is supplied here:
\icmltitlerunning{LocalEscaper: A Weakly-supervised Framework with Regional Reconstruction for Scalable Neural TSP Solvers}

\begin{document}

\twocolumn[
%\icmltitle{LocalEscaper: A Weakly-Supervised Neural Solver for solving Large-Scale Traveling Salesman Problems}
\icmltitle{LocalEscaper: A Weakly-supervised Framework with Regional Reconstruction for Scalable Neural TSP Solvers}

% It is OKAY to include author information, even for blind
% submissions: the style file will automatically remove it for you
% unless you've provided the [accepted] option to the icml2025
% package.

% List of affiliations: The first argument should be a (short)
% identifier you will use later to specify author affiliations
% Academic affiliations should list Department, University, City, Region, Country
% Industry affiliations should list Company, City, Region, Country

% You can specify symbols, otherwise they are numbered in order.
% Ideally, you should not use this facility. Affiliations will be numbered
% in order of appearance and this is the preferred way.
%\icmlsetsymbol{equal}{*}

\begin{icmlauthorlist}
\icmlauthor{Junrui Wen}{hust}
\icmlauthor{Yifei Li}{hust}
\icmlauthor{Bart Selman}{Cornell} %selman@cs.cornell.edu
\icmlauthor{Kun He}{hust}
%\icmlauthor{}{sch}
%\icmlauthor{}{sch}
\end{icmlauthorlist}

\icmlaffiliation{hust}{School of Computer Science, Huazhong University of Science and Technology, China}
\icmlaffiliation{Cornell}{Department of Computer Science, Cornell University, United States}

\icmlcorrespondingauthor{Kun He}{brooklet60@hust.edu.cn}


% You may provide any keywords that you
% find helpful for describing your paper; these are used to populate
% the "keywords" metadata in the PDF but will not be shown in the document
\icmlkeywords{combinatorial optimization, neural solver, weakly-supervised learning, TSP}

\vskip 0.3in
]

% this must go after the closing bracket ] following \twocolumn[ ...

% This command actually creates the footnote in the first column
% listing the affiliations and the copyright notice.
% The command takes one argument, which is text to display at the start of the footnote.
% The \icmlEqualContribution command is standard text for equal contribution.
% Remove it (just {}) if you do not need this facility.

% \printAffiliationsAndNotice{}  % leave blank if no need to mention equal contribution
% \printAffiliationsAndNotice{\icmlEqualContribution} % otherwise use the standard text.
\footnotetext[1]{School of Computer Science, Huazhong University of Science and Technology, China. ${}^2$Department of Computer Science, Cornell University, United States. Correspondence to: Kun He $<$brooklet60@hust.edu.cn$>$.\\\\Preprint.}

% \footnote{Department of Computer Science, Cornell University, United States}

\begin{abstract}
Neural solvers have shown significant potential in solving the Traveling Salesman Problem (TSP), yet current approaches face significant challenges.  Supervised learning (SL)-based solvers require large amounts of high-quality labeled data, while reinforcement learning (RL)-based solvers, though less dependent on such data, often suffer from inefficiencies. 
%Current approaches rely primarily on supervised learning (SL) or reinforcement learning (RL). While SL-based solvers require large volumes of high-quality labeled data, RL-based solvers, though free from this dependency, often suffer from lower efficiency. 
To address these limitations, we propose \textit{LocalEscaper}, a novel weakly-supervised learning framework for large-scale TSP. 
LocalEscaper effectively combines the advantages of both SL and RL, enabling effective training on datasets with low-quality labels. To further enhance solution quality, we introduce a \textit{regional reconstruction strategy}, which %effectively escapes local optima,
mitigates the problem of local optima, a common issue in existing local reconstruction methods. 
Additionally, %to alleviate the computational burden of large-scale TSPs, we develop an attention mechanism with linear complexity, significantly reducing overhead without sacrificing performance. 
we propose a linear-complexity attention mechanism that reduces computational overhead, enabling the efficient solution of large-scale TSPs without sacrificing performance. 
Experimental results on both synthetic and real-world datasets demonstrate that LocalEscaper outperforms existing neural solvers, achieving state-of-the-art results. Notably, %it efficiently solves TSP instances with up to 50,000 cities, setting a new benchmark for scalability and effectiveness in the field.  
 it sets a new benchmark for scalability and efficiency, solving TSP instances with up to 50,000 cities.
\end{abstract}

\section{Introduction}
\label{01Intro}

The Traveling Salesman Problem (TSP) is a classic combinatorial optimization (CO) problem with widespread applications in fields such as transportation~\cite{wang2021deep}, logistics~\cite{castaneda2022optimizing}, and circuit design~\cite{alkaya2013application}. Due to its NP-hard nature, finding the optimal solution for large-scale TSP instances remains a significant challenge. Over the past few decades, researchers have focused on mathematical programming and heuristic methods to find locally optimal solutions~\cite{applegate2009certification,lin1973effective,helsgaun2000effective}. However, these methods tend to be time-consuming, making them unsuitable for real-world, large-scale applications.


Recently, neural combinatorial optimization (NCO) methods have shown considerable promise in solving TSP~\cite{kwon2020pomo,jin2023pointerformer,drakulic2024bq}. 
Current NCO solvers typically rely on either supervised learning (SL)~\cite{joshi2019efficient,hottung2021learning,luo2023neural} or reinforcement learning (RL)~\cite{bello2016neural,kwon2020pomo,jin2023pointerformer}.  
%SL-based methods train neural network models using a large dataset of labeled instances. These labels represent high-quality solutions, typically generated by other methods, such as exact solvers or heuristics. 
SL-based methods require large datasets of labeled instances, where labels represent high-quality solutions typically obtained from exact solvers or heuristics. 
%Although SL-based methods are efficient in learning, their efficiency depends on large amounts of data. Moreover, generating high-quality labels typically involves significant computational cost, especially for large-scale instances. 
While SL-based methods are efficient in training, they depend on large amounts of labeled data, and generating high-quality labels for large-scale instances is computationally expensive.
On the other hand, RL-based methods do not rely on labeled data, learning to generate solutions through reward signals. However, 
%due to the lack of optimal solution guidance, RL methods encounter severe sparse reward issues when solving large-scale TSP and are prone to getting stuck in local optima
RL methods face issues such as sparse rewards and a tendency to get stuck in local optima~\cite{bengio2021machine, min2024unsupervised}, which can reduce learning efficiency.

%Whether based on SL or RL, NCO solvers face the scalability challenge due to device memory constraints, with most solvers plateauing at 500 to 1,000 scales.
In addition to these challenges, scalability is a major concern for both SL- and RL-based NCO solvers, as most solvers plateau at problem sizes of around 500 to 1,000 nodes due to memory constraints. 
Moreover, most NCO solvers employ Transformer network architectures~\cite{vaswani2017attention}, which exhibit quadratic computational complexity as the problem scale increases. This makes it particularly difficult to train models capable of generating high-quality solutions for large-scale TSP.

To address the scalability issue, a number of NCO solvers have adopted divide-and-conquer strategies based on the Partial Optimization Metaheuristic Under Special Intensification Conditions (POPMUSIC) framework~\cite{taillard2019popmusic}, improving the solution quality by dividing the TSP tour into smaller, non-overlapping subsequences and reconstructing each subsequence independently~\cite{cheng2023select,luo2023neural,ye2024glop,zheng2024udc}. 
%They transform the challenging task of solving large-scale problems into the relatively simpler task of solving smaller sub-problems. However, these techniques only reconstruct the visiting order of nodes within subsequences while keeping the starting and destination nodes fixed, without altering the global relationships between subsequences, thereby limiting their ability to escape local optima.
While these methods perform local search using subsequence reconstruction and reduce the complexity of solving large-scale problems, they are limited in their ability to escape local optima, as they only reconstruct the visiting order within subsequences while leaving the global relationships between subsequences unchanged. 

Given these challenges, research on NCO methods for TSP faces three key obstacles: (1) Overcoming the limitations of SL and RL approaches, (2) reducing computational costs, and (3) escaping  local optima.
To address these challenges, we introduce LocalEscaper, a novel model designed to enhance TSP solving. 

For the first challenge, %SL is efficient but relies on high-quality labels, while RL avoids this requirement but is less efficient. 
%We propose a weakly-supervised learning framework that begins SL training with low-quality labels and progressively refines them using RL approaches. This framework combines the efficiency of SL with the flexibility of RL, avoiding the need for high-quality labels and leveraging their complementary strengths.
we propose a weakly-supervised learning framework that combines SL and RL to leverage the strengths of both. Our method begins with SL using low-quality labels and progressively refines them through RL, enabling effective training without requiring high-quality labeled data.

For the second challenge, some researchers~\cite{yang2023memory, luo2024self} argue that pairwise attention computation among tokens in Transformer-based NCO solvers is not always necessary. %, and some efforts have been made to reduce computational complexity. 
We address the computational burden by designing a lightweight neural network architecture with linear complexity. %This approach aggregates node information in a TSP graph through virtual nodes, enabling both computational complexity and memory usage to scale linearly with the problem size.
This approach reduces both memory usage and computational overhead, enabling scalability to large-scale TSP instances.

For the third challenge, %we consider that POPMUSIC-based methods are prone to local optima because they restrict changes in node order to nodes with close sequence distances, neglecting spatial relationships between nodes. This subsequence reconstruction approach makes it difficult to adjust relationships between nodes that are close in spatial distance but far in sequence distance. To address this, we propose a regional reconstruction approach, which focuses on densely breaking edges within a specific region of the solution and reconnecting them to obtain a better feasible solution. We implement the regional reconstruction and subsequence reconstruction approaches using RL to construct an improver. This improver efficiently enhances solution quality during both training and inference stages for our weakly-supervised learning framework. 
we introduce a regional reconstruction strategy to overcome the limitations of existing divide-and-conquer approaches. Unlike traditional methods, which only adjust the visiting order within subsequences, our regional reconstruction method focuses on breaking and reconnecting edges within specific regions of the solution. This approach, coupled with an RL-based heuristic improver, enhances the solution quality by better escaping local optima during both training and inference stages.

%Experimental results demonstrate that LocalEscaper achieves state-of-the-art performance among existing NCO solvers on large-scale TSP instances ranging from 1,000 to 10,000 cities.
Experimental results demonstrate that LocalEscaper achieves state-of-the-art performance among existing NCO solvers on large-scale TSP instances ranging from 1,000 to 10,000 cities. In particular, LocalEscaper can handle large-scale instances with up to 50,000 cities quickly, while most existing NCO solvers either run out of memory or take too long.
The main contributions of our work are as follows: 
\begin{itemize}
\item %We propose a novel weakly-supervised learning framework to train the neural network model for TSP. Our method does not require high-quality labeled datasets for training. Instead, it continuously refines the label quality throughout the training process.
We propose a novel weakly-supervised learning framework that enables training without high-quality labeled datasets, progressively refining label quality during the training process.
\item We design a lightweight neural network architecture with linear complexity% to scale up to large-scale problems. Our model significantly reduces memory usage and improves the forward propagation speed.
, which scales efficiently to large-scale problems and significantly reduces memory usage and computation time.
\item  %We propose a regional reconstruction approach and construct an RL-based heuristic improver to enhance the solution quality during training and inference stages. Experimental results show that the regional reconstruction is highly effective in escaping local optima caused by subsequence reconstruction.
 We introduce a regional reconstruction approach, coupled with an RL-based improver, to enhance solution quality and effectively escape local optima during both training and inference stages. 
 \item  Experimental results demonstrate the superior performance of the proposed LocalEscaper over  existing NCO solvers on TSP instances ranging from 1,000 to 50,000 cities. 
\end{itemize}


\section{Related Work}
\label{02RW}
 In this section, we provide literature review from the following three perspectives on NCO Solvers. %, basing on supervised learning, reinforcement learning, and refinement search. 
 
%\subsection{Supervised Learning for NCO}
\subsection{SL-based NCO Solvers}
Supervised learning (SL)-based NCO solvers rely on large labeled datasets to train models.  
Early work in this domain was pioneered by \citet{vinyals2015pointer}, who proposed the Pointer Network.  Their approach used SL techniques to solve small-scale TSP problems, employing a recurrent neural network (RNN) with attention mechanisms to iteratively construct solutions. 
\citet{joshi2019efficient} advanced this by incorporating graph convolutional networks (GCNs) \cite{kipf2016semi} to predict the probability of each edge being part of the optimal solution. This helped the model better capture the underlying graph structure of the TSP. 

\citet{sun2023difusco} utilized an anisotropic graph neural network \cite{bresson2018experimental} to construct a diffusion model, which iteratively denoises the solution and predicts a heatmap to guide the Monte Carlo Tree Search (MCTS) algorithm. 
\citet{luo2023neural} introduced the Light
Encoder and Heavy Decoder (LEHD) model, which combines a lightweight encoder and a heavy decoder, making it particularly suitable for SL-based training. 
These SL-based methods have shown success for small-scale problems but still face challenges when scaling to larger TSP instances due to their reliance on large labeled datasets. 

%\subsection{Reinfoecement Learning for NCO}
\subsection{RL-based NCO Solvers}
Reinforcement learning (RL)-based NCO solvers, unlike SL-based methods, do not require labeled instances for training. 
Instead, they rely on reward signals to guide learning.  
\citet{kool2018attention} introduced a self-attention-based \cite{vaswani2017attention} NCO solver,  trained using RL, and demonstrated that it outperforms earlier SL-based methods in solving TSP instances. 
Building on this, \citet{kwon2020pomo} introduced the POMO solver, which generates multiple trajectories for a single TSP instance by starting from different nodes. 
During inference, POMO augments the input data with techniques like flipping and folding, enabling the model to produce diverse solutions and select the optimal one.  

\citet{jin2023pointerformer} improved POMO by incorporating a reversible residual network architecture to reduce memory consumption, enabling the model to scale to 500-node TSP instances with promising results. 
\citet{drakulic2024bq} proposed the BQ-NCO model, which reduces the state space of Markov Decision Processes (MDPs), thereby  improving RL efficiency. 

Despite these advancements, RL-based methods often struggle with issues such as getting stuck in local optima during training and the problem of sparse rewards \cite{bengio2021machine, min2024unsupervised}.
These challenges  hinder the ability of RL-based models to generalize to larger-scale problems. 

%\subsection{Improvement Approaches for NCO}
\subsection{Search-based NCO Solvers}
Search-based NCO solvers typically begin with an initial feasible solution and iteratively refine it to improve the outcome. Neural network models often play a direct or indirect role in guiding this refinement process. 
For instance, \citet{xin2021neurolkh} and \citet{zheng2023reinforced} integrated SL and RL into the classical heuristic Lin-Kernighan-Helsgaun (LKH) solver \cite{helsgaun2000effective,helsgaun2009general,helsgaun2017extension}, improving its efficiency by using learning-based methods to construct candidate node sets. 

Some search-based approaches leverage MCTS to enhance solution quality~\cite{fu2021generalize,qiu2022dimes,sun2023difusco,xia2024position}. These methods typically use GCNs to generate heatmaps that guide MCTS in finding better results. However, they are heavily reliant on MCTS, and constructing solutions greedily based solely on these heatmaps often leads to suboptimal performance. 
Additionally, MCTS is computationally expensive, making it difficult to scale for large TSP instances. 

Other search-based solvers use neural-based heuristics to directly refine solutions. 
For instance, \citet{luo2023neural} applied subsequence reconstruction on top of a constructive model to improve solutions. They also employed this approach in a self-improved learning solver \cite{luo2024self}, where the model is trained on a dataset with low-quality labels and then refines these labels using subsequence reconstruction. 
\citet{cheng2023select} and \citet{ye2024glop} generated initial solutions using random insertion and applied subsequence reconstruction via RL-based models to enhance them. 
However, subsequence reconstruction methods have limited ability to modify the global relationships between nodes, often leading to the solver getting stuck in local optima. This limitation is a significant challenge for search-based NCO solvers and motivates our proposed approach, which aims to improve solution quality by addressing global relationships and escaping local optima. 
% In future work, we will focus on enhancing the scalability of LocalEscaper and extending it to more combinatorial optimization problems.

% Cheng~\cite{cheng2023select} introduces a destroy-and-repair operator to escape local optima. The operator removes long-connection and random selected edges to make the fragment sizes more uniform. Then, they apply the Lin–Kernighan algorithm~\cite{helsgaun2000effective} repair the broken fragments to a complete solution. However, this method randomly breaks edges, often resulting in solutions that resemble the original, making it inefficient at both improvement and escaping local optima. 
% Luo et al. enhance solutions by applying subsequence reconstruction on top of end-to-end solution construction. 


\section{Preliminaries}
\label{03Pre}
\subsection{The TSP Problem}
The focus of current NCO research is on the classic 2D Euclidean Traveling Salesman Problem (TSP). 
In this problem, we are given an undirected, fully connected graph $\mathcal{G}(\mathcal{V}, \mathcal{E})$ with $n$ nodes.  The node set $\mathcal{V}=\{v_i | 1\leq i \leq n\}$ represents the cities and the edge set $\mathcal{E}=\{e_{ij} | 1\leq i, j\leq n\}$ represents all possible connections between the cities. The cost, denoted as $\text{cost}(v_i, v_j)$, represents the Euclidean distance between nodes $v_i$ and $v_j$. 

The objective of the TSP is to find a Hamiltonian circuit (or tour) $\boldsymbol{\tau} = (\tau_1, \tau_2, \dots, \tau_n)$, which minimizes the total cost $L_{\text{total}}(\boldsymbol{\tau})$. The total cost is the sum of the Euclidean distances between consecutive nodes in the tour, as well as the distance from the last node back to the first, formulated as: 
\begin{equation}
{L_{\text{total}}(\boldsymbol{\tau})=\text{cost}(\tau_n, \tau_1)+\sum\limits_{i=1}^{n-1}{\text{cost}(\tau_i, \tau_{i+1})}},
% \label{eq:Balance}
\end{equation}
where $\tau_n$ is followed by $\tau_1$ to form the closed circuit.
  
\subsection{Subsequence Reconstruction Task}
In the subsequence reconstruction task, we are given a subsequence $\boldsymbol{\tau}^{\prime} = (\tau_1^{\prime}, \tau_2^{\prime}, \dots, \tau_{m}^{\prime})$ of length $m$, where $\boldsymbol{\tau}^{\prime} \subseteq \boldsymbol{\tau}$ and $m \leq n$. 
The goal is to reorder the intermediate nodes of the subsequence to form a new sequence $\boldsymbol{\tau}^{\prime\prime} = (\tau_1^{\prime\prime}, \tau_2^{\prime\prime}, \dots, \tau_{m}^{\prime\prime})$, while keeping the endpoints fixed: $\tau_1^{\prime\prime}=\tau_1^{\prime}$ and $\tau_m^{\prime\prime}=\tau_m^{\prime}$. 

The objective of subsequence reconstruction is to minimize the subsequence cost $L_{\text{sub}}(\boldsymbol{\tau}^{\prime\prime})$, which is given by:
\begin{equation}
{L_{\text{sub}}(\boldsymbol{\tau}^{\prime\prime})=\sum\limits_{i=1}^{m-1}{\text{cost}(\tau_i^{\prime\prime}, \tau_{i+1}^{\prime\prime})}},
% \label{eq:Balance}
\end{equation}
where the goal is to reorder the subsequence to reduce the total cost, improving the overall tour quality.

\subsection{Regional Reconstruction Task}
The regional reconstruction task aims to improve the overall TSP solution by focusing on local regions within the tour. Given a tour $\boldsymbol{\tau}$, the corresponding set of edges $\mathcal{E}_{\boldsymbol{\tau}}=\{e_{\tau_{i},\tau_{i+1}} | 1\leq i \leq n-1\} \cup \{e_{\tau_{n},\tau_{1}}\}$ represents the set of connections in the tour. 

For each city $\tau_i$, we consider a 2D coordinate $c = (x, y)$ and define the set of $k$-nearest neighbors $\mathcal{V}_{c}$ in graph $\mathcal{G}$ as the nodes closest to $c$.
The edges in $\mathcal{E}_{\boldsymbol{\tau}}$ that have nodes from $\mathcal{V}_{c}$ as their predecessors are defined as $\mathcal{E}_{\boldsymbol{\tau}}^{\prime}=\{e_{\tau_{i},\tau_{i+1}} | \tau_{i} \in \mathcal{V}_{c}\}$. Specifically, define $\tau_{n+1}=\tau_{1}$ to denote the successor of $\tau_{n}$. 

The objective of the regional reconstruction task is to improve the current tour by modifying the edges of the regional set $\mathcal{E}_{\boldsymbol{\tau}}^{\prime}$.  
Specifically, we aim to find an edge set $\mathcal{E}^{\text{add}} \subset \mathcal{E}$ such that the updated edge set $\mathcal{E}_{\boldsymbol{\tau}}^{\text{new}} = \mathcal{E}^{\text{add}} \cup (\mathcal{E}_{\boldsymbol{\tau}} \setminus \mathcal{E}_{\boldsymbol{\tau}}^{\prime})$ forms a new Hamiltonian circuit $\boldsymbol{\tau}^\text{new}$. The goal is to minimize the total cost of the new tour $L_{\text{total}}(\boldsymbol{\tau}^\text{new})$, 
which directly corresponds to minimizing the cost of the newly added edges in $\mathcal{E}^{\text{add}}$, thus improving the overall solution quality.


\section{Methodology}
\label{04Method}
\begin{figure}[t]
\centering
\includegraphics[width=82mm]{Z_Framework1.pdf}
\caption{Overview of the proposed framework, which includes a dataset and three tasks that can be executed in parallel. 
%In this paper, we adopt a sequential execution approach for Task 1 and Task 2.
}
\label{fig:Framework}
\end{figure}

In this section, we introduce a weakly-supervised learning framework to address the challenges of large-scale TSP discussed in the Introduction. An overview of the proposed framework is presented in Figure \ref{fig:Framework}. Our framework is designed to operate on a dataset consisting of three main tasks: 
Task 1 for dataset improvement, Task 2 for training reconstruction models, and Task 3 for training the constructive model. 
These tasks can be executed in parallel. For practical purposes, we adopt a sequential execution strategy where Task 1 and Task 2 alternate, while Task 3 runs continuously. 

\subsection{Weakly-supervised Learning Framework}
We define the dataset as $\mathcal{D}^t(\boldsymbol{\mathcal{G}}, \boldsymbol{\mathcal{T}}^t)$, where 
% $\boldsymbol{\mathcal{G}}=\{\mathcal{G}_1,\mathcal{G}_2,\dots,\mathcal{G}_N\}$ 
$\boldsymbol{\mathcal{G}}$ 
represents the set of TSP graphs, 
% $\boldsymbol{\mathcal{T}}^t = \{ \boldsymbol{\tau}_1^t, \boldsymbol{\tau}_2^t, \dots, \boldsymbol{\tau}_N^t\}$ 
$\boldsymbol{\mathcal{T}}^t$ 
corresponds to the set of tour labels, with  $t \in \{0,1,\dots,T\}$ indicating the iteration counter. 
The initial label set, $\boldsymbol{\mathcal{T}}^0$, is generated using a random insertion algorithm based on the graph set $\boldsymbol{\mathcal{G}}$.

We employ a weakly-supervised learning approach, where the dataset $\mathcal{D}$ is used to train a constructive model parameterized by $\boldsymbol{\theta}$, following the "learning to construct partial solutions" paradigm proposed in \citet{luo2023neural}.
This model generates TSP solutions by sequentially selecting the next node to visit. 

At each training step, we sample a subsequence $\boldsymbol{\tau}^{\prime}$ from a given tour $\boldsymbol{\tau}$, where $\boldsymbol{\tau}^\prime$ provides a binary indication $b_i$ for each available node $v_i$ (1 for selection, 0 for non-selection). 
The model then predicts the selection probability $p_i$ for each available node $v_i$. 
The objective function is the cross-entropy loss:
% \begin{equation}
${\mathcal{L}(\boldsymbol{\theta}) = -\sum_{i=1}^{u-1}{b_i\text{log}(p_i)}}$,
% \label{eq:Balance}
% \end{equation}
where $u$ is the number of available nodes in the subsequence. 

 
To improve the label quality of the dataset, we introduce an improver $\mathcal{I}$, which is independent of the model $\boldsymbol{\theta}$, 
to enhance the label quality of the dataset $\mathcal{D}$. 
The label improvement process is defined as $\mathcal{D}^{t+1} = \mathcal{I}(\mathcal{D}^t)$, and we impose the constraint that the total loss for the updated labels satisfies: 
$L_{\text{total}}(\boldsymbol{\tau}^{t+1}) \leq L_{\text{total}}(\boldsymbol{\tau}^{t})$, $\forall \boldsymbol{\tau}^{t} \in \boldsymbol{\mathcal{T}}^{t}$. 
% \begin{equation}
% L_{\text{total}}(\boldsymbol{\tau}^{t+1}) \leq L_{\text{total}}(\boldsymbol{\tau}^{t}) ~\text{for all}~ \boldsymbol{\tau}^{t} \in \boldsymbol{\mathcal{T}}^{t}.
% \end{equation}

After each improvement, we retrain the model $\boldsymbol{\theta}$ using supervised learning on the updated dataset $\mathcal{D}^{t+1}$.
Simultaneously, we train the improver $\mathcal{I}$ using reinforcement learning. 
% Thus, our weakly supervised learning framework consists of two main processes: 1) training the constructive model $\boldsymbol{\theta}$, and 2)improving the accuracy of the dataset $\mathcal{D}$ and training the improver $\mathcal{I}$. 
% Since the improver $\mathcal{I}$ is independent of the model $\boldsymbol{\theta}$ in terms of parameters, these two processes can be performed in parallel. 

Next, we provide a detailed description of the improver.

\subsection{Improver for Escaping Local Optima}
A crucial component of our framework is the improver $\mathcal{I}$, designed to help escape local optima by refining solutions through regional reconstruction. 
The improver operates both during training (to improve the label quality of the dataset) and during inference (to enhance the quality of the generated solutions). %Its primary role is to address escape local optima encountered in subsequence reconstruction by leveraging regional reconstruction.

The improver follows a three-step process: Subsequence Reconstruction, 2-opt, and Regional Reconstruction.

\textbf{Subsequence Reconstruction}: Building upon the approach in \citet{ye2024glop}, we randomly decompose a TSP solution of length $n$ into $\lfloor \frac{n}{m} \rfloor$ subsequences, each of length $m$. 
To improve the homogeneity of the input, we apply a coordinate transformation to the $x$ and $y$ coordinates of the nodes: 
$\boldsymbol{x} \leftarrow \frac{\boldsymbol{x}-\bar{x}}{\sigma},  ~~\boldsymbol{y}\leftarrow\frac{\boldsymbol{y}-\bar{y}}{\sigma}, $
where $\sigma$ is the maximum absolute value among the $x$ and $y$ coordinates. 
Each subsequence is then reconstructed using a neural model parameterized by $\boldsymbol{\psi}$. 
The model replaces the original subsequence $\boldsymbol{\tau}^{\prime}$ with a newly reconstructed subsequence \textbf{$\boldsymbol{\tau}^{\prime\prime}$},   if $L_{\text{sub}}(\boldsymbol{\tau}^{\prime\prime}) \leq L_{\text{sub}}(\boldsymbol{\tau}^{\prime})$.

Our model follows the multiple trajectories approach of POMO~\cite{kwon2020pomo}, with the distinction that we employ a non-autoregressive neural model. Specifically, the encoder generates a heatmap 
% $\boldsymbol{\mathcal{H}}_{\boldsymbol{\psi}}$ 
of edge selection probabilities in a single pass, and the decoder constructs new subsequences iteratively based on 
% $\boldsymbol{\mathcal{H}}_{\boldsymbol{\psi}}$. 
the heatmap. 
Non-autoregressive models offer fast construction speeds. When the subsequence length $m$ is small (e.g., less than 100), the drawback of such coarse construction~\cite{ye2024glop} is minimal. %much less pronounced.

\textbf{2-opt}: We apply the classical 2-opt algorithm~\cite{lin1973effective}, which iteratively checks all possible  edge swaps to improve the solution. 
If a swap leads to a shorter solution, it is executed, and the process continues  until no further improvements are possible. This helps fine-tune the solution by removing suboptimal edges.

\textbf{Regional Reconstruction}: 
In this step, we remove a set of edges $\mathcal{E}_{\boldsymbol{\tau}}^{\prime}$ from the original TSP solution $\mathcal{E}_{\boldsymbol{\tau}}$.  Let $|\mathcal{E}_{\boldsymbol{\tau}}^{\prime}|=k$, the remaining edges $\mathcal{E}_{\boldsymbol{\tau}} \setminus \mathcal{E}_{\boldsymbol{\tau}}^{\prime}$ can compose $k$ subsequences 
$\mathcal{S}=\{ \boldsymbol{\tau}_{1}^{\prime}, \boldsymbol{\tau}_{2}^{\prime}, \dots, \boldsymbol{\tau}_{k}^{\prime} \}$% (Insensitive to the direction of each subsequence in $\mathcal{S}$).  
, where each subsequence is independent of the direction of traversal.

To ensure that the reconstructed solution is a valid Hamiltonian circuit, we transform the reconstruction process into a permutation problem. We permute the subsequences in $\mathcal{S}$ to form a subsequence list $\mathcal{X}=(\boldsymbol{\eta}_{1}, \boldsymbol{\eta}_{2}, \dots, \boldsymbol{\eta}_{k})$. 
During the permutation, we have the option to reverse the direction of any subsequence.  
For instance, for a subsequence $\boldsymbol{\tau}^{\prime} = (\tau_1^{\prime}, \tau_2^{\prime}, \dots, \tau_{m}^{\prime})$, its reverse is denoted as $\neg \boldsymbol{\tau}^{\prime} = (\tau_{m}^{\prime}, \tau_{m-1}^{\prime}, \dots, \tau_{1}^{\prime})$. 

Once the subsequences are permuted and reversed as necessary, they are concatenated to form a new Hamiltonian circuit $\boldsymbol{\tau}^{\text{new}}$. This reconstructed solution provides a refined path that potentially escapes local optima by incorporating a broader set of possible configurations.

This permutation process can be modeled as a Markov Decision Process (MDP), detailed in the next subsection. 

\subsection{The Permutation as an MDP}
We model the permutation task as a Markov Decision Process (MDP) to optimally select subsequences and reconstruct the final solution. 
In this framework, %the subsequence list $\mathcal{X}$ is initialized as an empty list, and the set of available subsequences is initialized as $\mathcal{C}=\{\boldsymbol{\tau}_{1}^{\prime}, \neg\boldsymbol{\tau}_{1}^{\prime}, \boldsymbol{\tau}_{2}^{\prime}, \neg\boldsymbol{\tau}_{2}^{\prime},\dots,\boldsymbol{\tau}_{k}^{\prime}, \neg\boldsymbol{\tau}_{k}^{\prime}\}$. Then, we gradually append subsequence from $\mathcal{C}$ to $\mathcal{X}$.
the subsequence list $\mathcal{X}$ is gradually built by appending subsequences from a set of available subsequences $\mathcal{C}$.

The MDP  formulation consists of the following elements: states, actions, transition, and rewards. 

\textit{States}: The state is characterized by the current subsequence list $\mathcal{X}$ and the set of available subsequences $\mathcal{C}$.
Initially, $\mathcal{X}$ is an empty list, and $\mathcal{C}$ contains the subsequences as well as their reverses:
$\mathcal{C}=\{\boldsymbol{\tau}_{1}^{\prime}, \neg\boldsymbol{\tau}_{1}^{\prime}, \boldsymbol{\tau}_{2}^{\prime}, \neg\boldsymbol{\tau}_{2}^{\prime},\dots,\boldsymbol{\tau}_{k}^{\prime}, \neg\boldsymbol{\tau}_{k}^{\prime}\}$. 
At each step, subsequences are selected from $\mathcal{C}$ and added to $\mathcal{X}$.

\textit{Actions}: An action is the selection of a subsequence $\boldsymbol{\tau}_{\text{act}}$ from the available set $\mathcal{C}$ to be appended to the subsequence list $\mathcal{X}$.

\textit{Transition}: After performing an action, the subsequence $\boldsymbol{\tau}_{\text{act}}$ is appended to $\mathcal{X}$, and both $\boldsymbol{\tau}_{\text{act}}$ and its reverse $\neg\boldsymbol{\tau}_{\text{act}}^{\prime}$ are removed from $\mathcal{C}$.
The process continues until all subsequences are appended, and  $\mathcal{C}$ becomes empty, signaling the completion of the reconstruction. 

\textit{Rewards}: The reward function is designed to evaluate the quality of the reconstructed solution. 
Once the reconstruction is complete, the total reward is defined as  the negative of the solution's cost, which is the sum of the edge costs between adjacent subsequences: 
\begin{equation}
{R(\mathcal{X})=-\text{cost}(\eta_{1}^{\text{first}}, \eta_{k}^{\text{last}})-\sum\limits_{i=1}^{k-1}\text{cost}(\eta_{i}^{\text{last}}, \eta_{i+1}^{\text{first}})},
\label{eq:Reward}
\end{equation}
where $\eta_{i}^{\text{first}}$ and $\eta_{i}^{\text{last}}$ represent the first and last nodes of subsequence $\boldsymbol{\eta}_i$, respectively. 
The goal is to minimize this total cost by carefully selecting subsequences during the reconstruction process. 

%We follow the multiple trajectory approach~\cite{kwon2020pomo} to 
We employ the REINFORCE algorithm~\cite{williams1992simple} to train a non-autoregressive policy model $\boldsymbol{\phi}$ to select subsequences during regional reconstruction.  
The policy model is trained by sampling multiple trajectory lists $\boldsymbol{\mathcal{X}}=\{\mathcal{X}^{1}, \mathcal{X}^{2}, \dots, \mathcal{X}^{N} \}$, where each $\mathcal{X}^i$ represents a possible subsequence list generated by the policy. 

The gradient of the policy network is approximated as: 
\begin{equation}
\begin{aligned}
\triangledown_{\boldsymbol{\phi}}J(\boldsymbol{\phi}) &\approx \frac{1}{N}\sum\limits_{i=1}^{N} \frac{R(\mathcal{X}^{i}) - \mu(\boldsymbol{\mathcal{X}})}{\delta(\boldsymbol{\mathcal{X}})} \triangledown_{\boldsymbol{\phi}} \text{log} ~ p_{\boldsymbol{\phi}}(\mathcal{X}^{i}| \mathcal{S}),
\\
\mu(\boldsymbol{\mathcal{X}}) &= \frac{1}{N}\sum\limits_{i=1}^{N} R(\mathcal{X}^{i}), \\
\delta(\boldsymbol{\mathcal{X}}) &= \sqrt{\frac{1}{N}\sum\limits_{i=1}^{N} \left(R(\mathcal{X}^{i})-\mu(\boldsymbol{\mathcal{X}})\right)^2} . 
\label{eq:Regional Reconstruction Gradient}
\end{aligned}
\end{equation}
Here, $p_{\boldsymbol{\phi}}(\mathcal{X}^{i}| \mathcal{S})$ is the probability of the policy model $\boldsymbol{\phi}$ constructing the solution $\mathcal{X}^{i}$ given the regional reconstruction problem $\mathcal{S}$. 

%We use the same coordinate transformation in subsequence reconstruction, and the new tour replaces the original one if its cost is lower.
The training process aims to optimize the policy so that subsequences are selected in a way that minimizes the total cost of the final reconstructed solution. During this process, the same coordinate transformation used in subsequence reconstruction is applied, and the new solution replaces the original one if its cost is lower.

\subsection{Linear Attention for TSP}
The vanilla attention mechanism~\cite{vaswani2017attention} is computationally expensive, especially for large-scale problems like the TSP, where pairwise attention calculations between nodes can become prohibitively costly. 
%We argue that pairwise attention calculations between nodes may be unnecessary in TSP, and therefore we propose an attention mechanism with linear computational complexity to build a lightweight network architecture. 
To address this issue, we propose a linear attention mechanism that significantly reduces computational complexity while still effectively capturing the relevant information for TSP. 

Our attention mechanism operates by aggregating the global features of the graph into representative nodes,  which are then broadcast to all other nodes.  This method reduces the need for pairwise attention between all nodes, as illustrated in Figure \ref{fig:LinearAttention}. 
The representative nodes consist of anchor nodes, the starting node, and the destination node. 

Each node $v \in \mathcal{V}$ (city node) is assigned a coordinate $\mathbf{c}_v \in [0,1]^2$, representing its position in 2D space. 
Anchor nodes $\mathcal{A}$ are placed in a grid-like, evenly spaced manner within this coordinate space. 
%To distinguish from the anchor nodes $\mathcal{A}$, we refer to the nodes $\mathcal{V}$ as the city nodes. 
 Following the "learning to construct partial solutions" paradigm~\cite{luo2023neural}, the constructive model $\boldsymbol{\theta}$ uses the last and first nodes of the current route as the starting and destination nodes for constructing the next route. 
 
We now describe the encoder and decoder components of the network.

\textbf{Encoder}: The encoder transforms the coordinates of both city nodes and anchor nodes into embeddings. 
For each city node $v \in \mathcal{V}$ and anchor node $a \in \mathcal{A}$, we use distinct 2-layer perceptrons to map their coordinates ${\mathbf{c}}_{v},{\mathbf{c}}_{a} \in [0,1]^{2}$ into embeddings $\mathbf{h}_{v},\mathbf{h}_{a} \in \mathbb{R}^d$, where 
$d$ is the embedding dimension. 

\textbf{Decoder}: 
The decoder takes in the anchor nodes $\mathcal{A}$ and a subset of the city nodes $\mathcal{V}^{\prime} \subseteq \mathcal{V}$, which includes both available nodes as well as the starting and destination nodes. 
The number of available nodes is denoted as $\tilde{n}$ and the number of anchor nodes is $|\mathcal{A}|=A$. 

The decoder consists of $L$ linear attention decoding layers, indexed by $l$. 
Each decoding layer comprises two phases: aggregation and broadcasting.

\textit{Aggregation Phase}: In this phase, the embeddings of the representative nodes (anchor nodes, starting nodes, and destination nodes) serve as queries, and the embeddings of the city nodes $\mathcal{V}^{\prime}$ serve as keys and values. The aggregation is computed as: 
\begin{equation}
{\tilde{H}^{l}=\text{softmax}\left(\frac{\mathbf{Q}_{}^{l} W_{Q}^{l}(\mathbf{K}_{}^{l} W_{K}^{l})}{\sqrt{d}}\right)\mathbf{V}^{l}W_{V}^{l}},
\label{eq:Aggregation}
\end{equation}
where $W_{Q}^{l}$, $W_{K}^{l}$ and $W_{V}^{l}$ are learnable  weight matrices. The output $\tilde{H}^{l} \in \mathbb{R}^{(A+2) \times d}$ represents the aggregated embeddings of the representative nodes. 

\textit{Broadcasting Phase}: In the broadcasting phase, we concatenate the embeddings of the anchor nodes $\mathcal{A}$ and the city nodes $\mathcal{V}^{\prime}$ as the query $\hat{\mathbf{Q}}_{}^{l} \in \mathbb{R}^{(A+\tilde{n}+2) \times d}$. 
The aggregated embeddings $\tilde{H}^{l} \in \mathbb{R}^{(A+2) \times d}$serve as both the key and value. The broadcasting is calculated as: 
\begin{equation}
{\hat{H}^{l} = \text{softmax}\left(\frac{\hat{\mathbf{Q}}_{}^{l} \hat{W}_{Q}^{l}(\tilde{H}^{l} \hat{W}_{K}^{l})}{\sqrt{d}}\right)\tilde{H}^{l}\hat{W}_{V}^{l}},
\label{eq:Broadcasting}
\end{equation}
where $\hat{W}_{V}^{l}$ are learnable weight matrices. 
The result, $\hat{H}^{l} \in \mathbb{R}^{(A+\tilde{n}+2) \times d}$,  represents the broadcasted embeddings of both anchor nodes and city nodes.  

After broadcasting, we transform $\hat{H}^{l}$ into the output $H^{l} \in \mathbb{R}^{(A+\tilde{n}+2) \times d}$ through the same feed-forward and residual connections used in vanilla attention. These embeddings are then updated for both the anchor nodes $\mathcal{A}$ and the city nodes $\mathcal{V}^{\prime}$ as their updated embeddings. 

The total computational cost for each decoding layer, as derived from the aggregation and broadcasting phases, involves tensor multiplications on the order of $(2A+4)\tilde{n}+A^2+6A+8$  where $A$ is constant. 
Thus, the overall computational complexity of our attention mechanism is ${O}(\tilde{n})$, which is linearly dependent on the number of available nodes $\tilde{n}$.

\begin{figure}[t]
\centering
\includegraphics[width=82mm]{Z_LinearAttention.pdf}
% \vspace{-2em}
\caption{Illustration of the proposed linear attention architecture.}
\label{fig:LinearAttention}
\end{figure}


\section{Experiments}
\label{05Exp}
%\label{08Tab_random}
\begin{table*}[ht]
\caption{Comparison of results on large-scale TSP instances. RL, SL, G, S, NH and MCTS denote Reinforcement Learning, Supervised Learning, Greedy Decoding, Sampling Decoding, Neural-based Heuristics and Monte Carlo Tree Search, respectively. * indicates the baseline used for calculating the performance gap.}
\label{Table: TSP random}
\centering
% \hspace{-2em}
\resizebox{1.0\linewidth}{!}{
\begin{tabular}{cc|ccr|ccr|ccr}%四个c代表有四列且内容居中
\toprule%第一道横线
\multirow{2}{*}{Method} & \multirow{2}{*}{Type}
& \multicolumn{3}{c|}{TSP-1000} & \multicolumn{3}{c|}{TSP-5000} & \multicolumn{3}{c}{TSP-10000} \\
&&Obj.&Gap&Time&Obj.&Gap&Time&Obj.&Gap&Time \\
\midrule%横线

Concorde &Exact &23.12* &0.00\% &~2.21h &N/A &N/A &N/A &N/A &N/A &N/A \\

LKH-3 &Heuristic &23.12 &0.00\% &~4.41h &51.03* &0.00\% &~~2.62h &71.78* &0.00\% &~~~~6h \\
\midrule%横线 
% AM &RL+G &20.020 &20.99 &1.51m &31.15 &34.75 &3.18m &141.68 &97.39 &5.99m \\

POMO      &RL+S &32.51 &40.61\% &~~8.6m  &\multicolumn{3}{c|}{OOM} &\multicolumn{3}{c}{OOM} \\

Pointerformer   &RL+S &24.80 &7.27\% &~6.33m  &\multicolumn{3}{c|}{OOM} &\multicolumn{3}{c}{OOM} \\

H-TSP &RL+S &24.66 &6.62\% &~3.14m  &55.31 &8.39\% &~~2.42m  &77.75 &8.32\% &~~\textbf{6.60m} \\

INViT-3V &RL+G &24.47 &5.82\% &37.30m  &54.47 &6.75\% &29.93m  &76.88 &7.09\% &55.53m \\

LEHD greedy &SL+G &23.84 &3.11\% &~1.64m  &59.90 &17.39\% &33.63m  &91.33 &27.2\% &~~3.02h \\

\textbf{LocalEscaper} greedy &SL+G &\textbf{23.81} &\textbf{2.98\%} &~~~~\textbf{37s} &\textbf{53.45} &\textbf{4.75\%} &~~\textbf{2.08m} &\textbf{75.95} &\textbf{5.81\%} &~~7.45m \\

\midrule%横线 
Att-GCN &SL+MCTS &23.69 &2.47\% &~~~~4.3m  &52.72 &3.32\% &~~8.35m  &74.34 &3.57\% &42.68m \\

DIFUSCO &SL+MCTS &23.38 &1.12\% &13.73m &52.32 &2.53\% &22.95m  &73.62 &2.58\% &~~1.07h \\

GLOP &RL+NH &23.84 &3.11\% &~~~~3.3m  &53.42 &4.69\% &~~~~\textbf{47s}  &75.29 &4.90\% &~~\textbf{1.33m} \\

LEHD RRC100 &SL+NH &23.52 &1.73\% &42.73m  &N/A &N/A &N/A  &N/A &N/A &N/A \\

% \textbf{LocalEscaper} Rec100 &SL+RL+NH &23.36 &1.04 &\textbf{2.05m}  &51.83 &1.57 &2.92m  &73.18 &1.95 &9.8m \\
% \textbf{LocalEscaper} Rec500 &SL+RL+NH &\textbf{23.31} &\textbf{0.82} &7.77m  &\textbf{51.75} &\textbf{1.41} &5.45m  &\textbf{72.98} &\textbf{1.67} &16.43m \\
\textbf{LocalEscaper} Rec100 &SL+RL+NH &23.33 &0.91\% &~~~~\textbf{2.8m}  &51.81 &1.53\% &~~~~~~3m  &73.17 &1.94\% &~~10.2m \\

\textbf{LocalEscaper} Rec500 &SL+RL+NH &\textbf{23.29} &\textbf{0.74\%} &11.42m  &\textbf{51.69} &\textbf{1.29\%} &~~6.03m  &\textbf{72.99} &\textbf{1.69\%} &16.07m \\
\bottomrule%横线
\end{tabular}
}
\end{table*}

\subsection{Experimental Setup}
\textbf{Problem Settings}. We adopt the standard data generation process from the previous work~\cite{kool2018attention}. 
The training set consists of 8192 instances, each with 1000 nodes randomly sampled from a uniform distribution. 
The test set includes both synthetic data and real-world data from TSPLIB~\cite{reinelt1991tsplib}. 
The synthetic data includes instances with three problem sizes: 1000, 5000, and 10000 nodes, referred to as TSP-1000, TSP-5000, and TSP-10000, respectively.  
The TSP-1000 and TSP-10000 test sets use the same instances as in \citet{fu2021generalize}, with 128 and 16 instances, respectively. The TSP-5000 test set is generated similarly, with 16 instances. 
For subsequence reconstruction problems, each subsequence length is set to $m=100$. For regional reconstruction problems, the center point $c$ is sampled from the $[0,1]^2$ space, and the number of removed edges is set to $k=60$. 

\textbf{Model Settings}. The constructive model $\boldsymbol{\theta}$ consists of  $L=6$ decoding layers with $A = 5 \times 5$ anchor nodes. Both the subsequence reconstruction model $\boldsymbol{\psi}$ and the regional reconstruction model $\boldsymbol{\phi}$ have 6 encoding layers.  
We evaluate LocalEscaper in three modes: 
(1) \textbf{LocalEscaper greedy}: A greedy construction using the constructive model; 
(2) \textbf{LocalEscaper Rec100}: After greedy construction, the solution is refined with 100 improvement iterations by our improver; 
(3) \textbf{LocalEscaper Rec500}: Similar to Rec100, but with up to 500 improvement iterations. 

\textbf{Training}. We follow a curriculum learning approach~\cite{wang2021survey}, starting with shorter subsequences and progressively increasing their length as training progresses.  Specifically, the subsequence length $|\boldsymbol{\tau}^{\prime}|$ is sampled from the range $[30,130]$ for each batch. 
After each epoch, the bounds of the sampling range are increased by 5, continuing until the upper bound reaches 1000 nodes. 
The optimizer is Adam~\cite{kingma2014adam}, with an initial learning rate of 1e-4 and a decay rate of 0.99 per epoch. Both the reconstruction models $\boldsymbol{\psi}$ and $\boldsymbol{\phi}$ are trained with the same optimizer and learning rate. 

\textbf{Baselines}. We compare our method with 
(1) \textbf{Classical solvers}: Concorde~\cite{applegate2009certification} and LKH-3~\cite{helsgaun2017extension}; 
(2) \textbf{Constructive solver}s: POMO~\cite{kwon2020pomo}, Pointerformer~\cite{jin2023pointerformer}, H-TSP~\cite{pan2023h}, INViT-3V\cite{fang2024invit} and LEHD greedy~\cite{luo2023neural}; (3) \textbf{Search-based solvers}: Att-GCN~\cite{fu2021generalize}, DIFUSCO~\cite{sun2023difusco}, GLOP~\cite{ye2024glop} and LEHD RRC100~\cite{luo2023neural}. 

\textbf{Metrics and Inference}.
We evaluate the methods based on three metrics: average tour length (Obj.), average optimality gap (Gap) and total run-time (Time). 
Since classical solvers are run on a single CPU, their run-time should not be directly compared with methods running on a GPU. 
All neural solvers are run on an RTX 3090 GPU for inference. We set a maximum inference time of 6 hours per problem size, and any method exceeding this time limit is marked as "N/A". 


\begin{table*}
\caption{Comparison of results on TSPLIB benchmark instances. TSPLIB1$\sim$1000 contains 48 instances, TSPLIB1001$\sim$5000 contains 22 instances, and TSPLIB5001+ contains 7 instances.}
\label{Table: TSPLIB}
% \label{08Tab_TSPLIB}
\centering
% \hspace{-2em}
\resizebox{1.0\linewidth}{!}
{
\begin{tabular}{cc|ccr|ccr|ccr}
\toprule
\multirow{2}{*}{Method} & \multirow{2}{*}{Type}
& \multicolumn{3}{c|}{TSPLIB1 $\sim$ 1000} & \multicolumn{3}{c|}{TSPLIB1001 $\sim$ 5000} & \multicolumn{3}{c}{TSPLIB5001+} \\
&&Obj.&Gap&Time&Obj.&Gap&Time&Obj.&Gap&Time \\
\midrule

OPT & N/A & 710.60 & 0.00\% & N/A & 7197.74 & 0.00\% & N/A & 504396.51 & 0.00\% & N/A \\

% Pointerformer & RL+S & 38559.17 & 13.05 & 1.23 & 200113.36 & 26.37 & 55.41 &\multicolumn{3}{c}{OOM} \\

LEHD RRC100 & SL+NH & 713.91 & 0.47\% & 1.07h  &7556.21 &4.94\% &5.24h &\multicolumn{3}{c}{OOM} \\

GLOP & RL+NH & 719.63 & 1.27\% & 2.46m & 7617.18 & 5.83\% & 1.46m & 536643.24 & 6.39\% & 1.75m \\

Att-GCN & SL+MCTS & 716.05 & 0.77\% & 19.83m & 7445.14 & 3.44\% & 1.18h &\multicolumn{3}{c}{OOM} \\

DIFUSCO & SL+MCTS & 713.39 & 0.39\% & 21.78m & 7380.36 & 2.54\% & 1.24h & 552009.53 & 9.44\% & 2.64h \\

\textbf{LocalEscaper} Rec100 & SL+RL+NH &713.56 &0.39\% &7.72m  &7301.02 &1.61\% &3.80m  &518770.37 &2.86\% &4.03m \\

\textbf{LocalEscaper} Rec500 & SL+RL+NH &\textbf{712.68} &\textbf{0.29\%} &38.60m  &\textbf{7278.70} &\textbf{1.25\%} &18.99m  &\textbf{517878.90} &\textbf{2.61\%} &10.95m \\

\bottomrule
\end{tabular}
}

% \begin{tabular}{cc|ccr|ccr|ccr}
% \toprule
% \multirow{2}{*}{Method} & \multirow{2}{*}{Type}
% & \multicolumn{3}{c|}{TSPLIB1 \sim 1000} & \multicolumn{3}{c|}{TSPLIB1001 \sim 5000} & \multicolumn{3}{c}{TSPLIB5001+ } \\
% &&Obj.&Gap&Time&Obj.&Gap&Time&Obj.&Gap&Time \\
% \midrule

% OPT & N/A & 34108.69 & 0.00\% & N/A & 158350.18 & 0.00\% & N/A & 3530775.57 & 0.00\% & N/A \\

% % Pointerformer & RL+S & 38559.17 & 13.05 & 1.23 & 200113.36 & 26.37 & 55.41 &\multicolumn{3}{c}{OOM} \\

% LEHD RRC100 & SL+NH & 34267.88 & 0.47\% & 1.07h  &166236.59 &4.94\% &5.24h &\multicolumn{3}{c}{OOM} \\

% GLOP & RL+NH & 34542.40 & 1.27\% & 2.46m & 167577.95 & 5.83\% & 1.46m & 3756502.71 & 6.39\% & 1.75m \\

% Att - GCN & SL+MCTS & 34370.29 & 0.77\% & 19.83m & 163793.09 & 3.44\% & 1.18h &\multicolumn{3}{c}{OOM} \\

% DIFUSCO & SL+MCTS & 34242.58 & 0.39\% & 21.78m & 162367.82 & 2.54\% & 1.24h & 3864066.71 & 9.44\% & 2.64h \\

% \textbf{LocalEscaper} Rec100 & SL+RL+NH &34250.67 &0.39\% &7.72m  &160622.45 &1.61\% &3.80m  &3631392.57 &2.86\% &4.03m \\

% \textbf{LocalEscaper} Rec500 & SL+RL+NH &\textbf{34208.60} &\textbf{0.29\%} &38.60m  &\textbf{160131.45} &\textbf{1.25\%} &18.99m  &\textbf{3625152.29} &\textbf{2.61\%} &10.95m \\

% \bottomrule
% \end{tabular}

% \begin{tabular}{cc|cr|cr|cr}
% \toprule
% \multirow{2}{*}{Method} & \multirow{2}{*}{Type}
% & \multicolumn{2}{c|}{TSPLIB1 \sim 1000} & \multicolumn{2}{c|}{TSPLIB1001 \sim 5000} & \multicolumn{2}{c}{TSPLIB5001+ } \\
% &&Gap&Time&Gap&Time&Gap&Time \\
% \midrule

% OPT & N/A & 0.00\% & N/A & 0.00\% & N/A & 0.00\% & N/A \\

% % Pointerformer & RL+S & 38559.17 & 13.05 & 1.23 & 200113.36 & 26.37 & 55.41 &\multicolumn{3}{c}{OOM} \\

% LEHD RRC100 & SL+NH & 0.47\% & 1.07h &4.94\% &5.24h &\multicolumn{2}{c}{OOM} \\

% GLOP & RL+NH & 1.27\% & 2.46m & 5.83\% & 1.46m & 6.39\% & 1.75m \\

% Att - GCN & SL+MCTS & 0.77\% & 19.83m & 3.44\% & 1.18h &\multicolumn{2}{c}{OOM} \\

% DIFUSCO & SL+MCTS & 0.39\% & 21.78m & 2.54\% & 1.24h & 9.44\% & 2.64h \\

% \textbf{LocalEscaper} Rec100 & SL+RL+NH &0.39\% &7.72m &1.61\% &3.80m &2.86\% &4.03m \\

% \textbf{LocalEscaper} Rec500 & SL+RL+NH &\textbf{0.29\%} &38.60m  &\textbf{1.25\%} &18.99m &\textbf{2.61\%} &10.95m \\

% \bottomrule
% \end{tabular}

\end{table*}

\begin{figure}[t]
\centering
\includegraphics[width=60mm]{Z_Decomposition.pdf}
\caption{Ablation study on the three steps of improver. %for TSP-1000.
}
\label{fig:Decomposition}
% \vspace{-2em}
\end{figure}

\subsection{Results and Analysis}
The main experimental results on the uniform distribution instances are presented in Table \ref{Table: TSP random}. 
% We report the total time taken to solve all instances. 
In comparison with constructive solvers, LocalEscaper demonstrates  fast inference speed under greedy decoding and achieves the lowest gap across all three instance sizes. Our constructive model $\boldsymbol{\theta}$, based on LEHD, performs similarly to LEHD on TSP-1000 but outperforms it on instances with more than 1000 nodes. % Due to LEHD is trained on small-scale instances with 100 nodes, while our model employs a lightweight linear attention mechanism that reduces computational overhead, enabling training on instances with up to 1000 nodes. As a result, it performs better  generalization on large-scale instances. 
While LEHD is trained on small-scale instances (with 100 nodes), our model uses a lightweight linear attention mechanism that reduces computational overhead, enabling training on larger instances (up to 1000 nodes), resulting in better generalization to large-scale instances.

Compared to search-based solvers, LocalEscaper also achieves the lowest gap across all instance sizes. LocalEscaper Rec100 delivers  competitive solutions in a short amount of time, while LocalEscaper Rec500 further improves the solution quality by reducing the gap for TSP-1000 to $0.74\%$, and for TSP-5000 and TSP-10000 to $1.29\%$ and $1.69\%$. 


Experimental results on TSPLIB are shown in Table \ref{Table: TSPLIB}. We categorize instances into three groups based on their size: $1\sim1000$ nodes, $1001\sim5000$ nodes, and over 5000 nodes. For each group, we solve all instances sequentially. 
Emphasizing the optimality gap, we compare LocalEscaper with other search-based methods. LocalEscaper Rec500 achieves the lowest gap across all instance categories. LocalEscaper Rec100 outperforms Att-GCN, LEHD RRC100, and DIFUSCO in inference speed, while achieving similar or lower gaps compared to other methods. 



\begin{table}
\caption{Effect of weakly SL vs. SL on the average optimality gap in training the constructive model.}
\label{Table: weakly SL vs. SL}
\centering
% \hspace{-2em}
% \resizebox{1.0\linewidth}{!}{
\begin{tabular}{c|ccc}%四个c代表有四列且内容居中
\toprule%第一道横线
% \multirow{2}{*}{Method} & \multicolumn{3}{c}{Gap}\\
Method&TSP-1000&TSP-5000&TSP-10000\\
\midrule%横线

SL &3.03\% &3.94\% &5.22\% \\

Weakly SL &2.98\% &4.75\% &5.81\% \\
\bottomrule%横线
\end{tabular}
% }
\label{08Tab_WSL_vs_SL}
\end{table}
% \begin{table}
\caption{Effect of the scale of regional reconstruction on LocalEscaper Rec500.}
\label{Table: Regional_Reconstruction_Scale}
\centering
% \hspace{-2em}
\resizebox{1.0\linewidth}{!}{
\begin{tabular}{c|cc|cc|cc}%四个c代表有四列且内容居中
\toprule%第一道横线
\multirow{2}{*}{Scale} 
& \multicolumn{2}{c|}{TSP-1000} & \multicolumn{2}{c|}{TSP-5000} & \multicolumn{2}{c}{TSP-10000}\\
&Gap&Time &Gap&Time &Gap&Time\\
\midrule%横线

$k=30$ &0.87\% &7.93m  &1.43\% &5.43m  &1.81\% &15.40m \\

$k=60$ &0.74\% &12.13m  &1.21\% &6.00m  &1.69\% &16.12m \\

$k=90$ &0.74\% &15.75m  &1.21\% &6.62m  &1.67\% &17.07m \\
\bottomrule%横线
\end{tabular}
}
\label{08Tab_RR}
\end{table}
\begin{table}
\caption{Comparison of results on larger-scale TSP instances.}
\label{Table: TSP-5W}
\centering
% \hspace{-2em}
\resizebox{1.0\linewidth}{!}{
\begin{tabular}{c|cc|cc}%四个c代表有四列且内容居中
\toprule%第一道横线
\multirow{2}{*}{Method} 
& \multicolumn{2}{c|}{TSP-20000} & \multicolumn{2}{c}{TSP-50000}\\
&Gap&Time &Gap&Time\\
\midrule%横线

% Concorde &Exact &23.12* &0.00 &2.21h &- &- &- &- &- &- \\

LKH-3 &0.00\%* &29.18h &0.00\%* &91.13h \\
\midrule%横线 

\textbf{LocalEscaper} greedy &7.59\% &28.25m  &13.41\% &2.88h \\

\textbf{LocalEscaper} Rec100 &2.53\% &43.70m &5.09\% &2.95h \\

\textbf{LocalEscaper} Rec500 &2.27\% &58.58m &4.46\% &3.19h \\
\bottomrule%横线
\end{tabular}
}
\end{table}

\subsection{Ablation Study} 
\textbf{Weakly-supervised Learning vs. Supervised Learning}.  
To assess the difference between weakly-supervised learning and fully-supervised learning for training the constructive model, we present a comparison in Table \ref{Table: weakly SL vs. SL}. 
Both methods were trained on the same dataset of 1000-node instances, with LKH-3 generating the labels for SL.  
Although weakly-supervised learning slightly outperforms supervised learning on TSP-1000, the difference is only $0.05\%$, indicating that the performance is %essentially comparable.
nearly identical.  For instances with more than 1000 nodes, supervised learning exhibits better generalization.

\textbf{Effect of the Improver's Components}. 
We conducted an ablation study on TSP-1000 to investigate the impact of each component of the improver.  
Figure \ref{fig:Decomposition} shows the solution improvement process over 500 iterations, with one or two components of the improver removed. Removing either regional reconstruction or both regional reconstruction and 2-opt significantly reduces performance, indicating that relying solely on subsequence reconstruction leads to a fast convergence to low-quality local optima.

\subsection{Performance on Larger Scale Instances} 
Table \ref{Table: TSP-5W} presents the evaluation results of LocalEscaper on TSP-20000 and TSP-50000, each with 16 instances. We removed the 2-opt step from the improver in TSP-50000 to adapt to the scale of these instances. The results show that while the generalization of the constructive model decreases on TSP-50000,  the improver still efficiently enhances solution quality. 




\section{Conclusion}
In this work, we propose LocalEscaper, a weakly-supervised framework with regional reconstruction for scalable neural TSP solvers.
Our method leverages RL-based heuristics to refine solutions and iteratively enhance low-quality training labels, thereby reducing reliance on expert-annotated data. 
For solution improvement, we introduce a regional reconstruction strategy that enhances subsequence reconstruction by improving node reordering and escaping local optima.
Furthermore, we introduce a TSP-specific linear attention mechanism that reduces computational complexity, enabling our model to efficiently handle large-scale instances.
The LocalEscaper framework achieves state-of-the-art performance among neural combinatorial optimization solvers, outperforming previous methods in solution accuracy, inference speed, and generalization to large-scale TSP instances.
%In future work, we aim to further enhance the scalability of LocalEscaper and explore its applicability to other combinatorial optimization problems.



%换页
\newpage

% \section*{Accessibility}
% Authors are kindly asked to make their submissions as accessible as possible for everyone including people with disabilities and sensory or neurological differences.
% Tips of how to achieve this and what to pay attention to will be provided on the conference website \url{http://icml.cc/}.

%\section*{Software and Data}

%If a paper is accepted, we strongly encourage the publication of software and data with the camera-ready version of the paper whenever appropriate. This can be done by including a URL in the camera-ready copy. However, \textbf{do not} include URLs that reveal your institution or identity in your submission for review. Instead, provide an anonymous URL or upload the material as ``Supplementary Material'' into the OpenReview reviewing system. Note that reviewers are not required to look at this material when writing their review.

% Acknowledgements should only appear in the accepted version.
%\section*{Acknowledgements}

\section*{Impact Statement}
This paper presents work whose goal is to %advance the field of Machine Learning. 
Explore approaches to combine deep learning techniques in solving combinatorial optimization problems in large scale, that could be applicable for real world applications. There are many potential societal consequences 
of our work, none of which we feel must be specifically highlighted here.


% In the unusual situation where you want a paper to appear in the
% references without citing it in the main text, use \nocite
\nocite{langley00}

\bibliography{main}
\bibliographystyle{icml2025}


%%%%%%%%%%%%%%%%%%%%%%%%%%%%%%%%%%%%%%%%%%%%%%%%%%%%%%%%%%%%%%%%%%%%%%%%%%%%%%%
%%%%%%%%%%%%%%%%%%%%%%%%%%%%%%%%%%%%%%%%%%%%%%%%%%%%%%%%%%%%%%%%%%%%%%%%%%%%%%%
% APPENDIX
%%%%%%%%%%%%%%%%%%%%%%%%%%%%%%%%%%%%%%%%%%%%%%%%%%%%%%%%%%%%%%%%%%%%%%%%%%%%%%%
%%%%%%%%%%%%%%%%%%%%%%%%%%%%%%%%%%%%%%%%%%%%%%%%%%%%%%%%%%%%%%%%%%%%%%%%%%%%%%%
\iffalse 
\newpage
\appendix
\onecolumn
\section{You \emph{can} have an appendix here.}

You can have as much text here as you want. The main body must be at most $8$ pages long.
For the final version, one more page can be added.
If you want, you can use an appendix like this one.  

The $\mathtt{\backslash onecolumn}$ command above can be kept in place if you prefer a one-column appendix, or can be removed if you prefer a two-column appendix.  Apart from this possible change, the style (font size, spacing, margins, page numbering, etc.) should be kept the same as the main body.
\fi 
%%%%%%%%%%%%%%%%%%%%%%%%%%%%%%%%%%%%%%%%%%%%%%%%%%%%%%%%%%%%%%%%%%%%%%%%%%%%%%%
%%%%%%%%%%%%%%%%%%%%%%%%%%%%%%%%%%%%%%%%%%%%%%%%%%%%%%%%%%%%%%%%%%%%%%%%%%%%%%%


\end{document}


% This document was modified from the file originally made available by
% Pat Langley and Andrea Danyluk for ICML-2K. This version was created
% by Iain Murray in 2018, and modified by Alexandre Bouchard in
% 2019 and 2021 and by Csaba Szepesvari, Gang Niu and Sivan Sabato in 2022.
% Modified again in 2023 and 2024 by Sivan Sabato and Jonathan Scarlett.
% Previous contributors include Dan Roy, Lise Getoor and Tobias
% Scheffer, which was slightly modified from the 2010 version by
% Thorsten Joachims & Johannes Fuernkranz, slightly modified from the
% 2009 version by Kiri Wagstaff and Sam Roweis's 2008 version, which is
% slightly modified from Prasad Tadepalli's 2007 version which is a
% lightly changed version of the previous year's version by Andrew
% Moore, which was in turn edited from those of Kristian Kersting and
% Codrina Lauth. Alex Smola contributed to the algorithmic style files.
