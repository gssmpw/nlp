\subsection{The TSP Problem}
The focus of current NCO research is on the classic 2D Euclidean Traveling Salesman Problem (TSP). 
In this problem, we are given an undirected, fully connected graph $\mathcal{G}(\mathcal{V}, \mathcal{E})$ with $n$ nodes.  The node set $\mathcal{V}=\{v_i | 1\leq i \leq n\}$ represents the cities and the edge set $\mathcal{E}=\{e_{ij} | 1\leq i, j\leq n\}$ represents all possible connections between the cities. The cost, denoted as $\text{cost}(v_i, v_j)$, represents the Euclidean distance between nodes $v_i$ and $v_j$. 

The objective of the TSP is to find a Hamiltonian circuit (or tour) $\boldsymbol{\tau} = (\tau_1, \tau_2, \dots, \tau_n)$, which minimizes the total cost $L_{\text{total}}(\boldsymbol{\tau})$. The total cost is the sum of the Euclidean distances between consecutive nodes in the tour, as well as the distance from the last node back to the first, formulated as: 
\begin{equation}
{L_{\text{total}}(\boldsymbol{\tau})=\text{cost}(\tau_n, \tau_1)+\sum\limits_{i=1}^{n-1}{\text{cost}(\tau_i, \tau_{i+1})}},
% \label{eq:Balance}
\end{equation}
where $\tau_n$ is followed by $\tau_1$ to form the closed circuit.
  
\subsection{Subsequence Reconstruction Task}
In the subsequence reconstruction task, we are given a subsequence $\boldsymbol{\tau}^{\prime} = (\tau_1^{\prime}, \tau_2^{\prime}, \dots, \tau_{m}^{\prime})$ of length $m$, where $\boldsymbol{\tau}^{\prime} \subseteq \boldsymbol{\tau}$ and $m \leq n$. 
The goal is to reorder the intermediate nodes of the subsequence to form a new sequence $\boldsymbol{\tau}^{\prime\prime} = (\tau_1^{\prime\prime}, \tau_2^{\prime\prime}, \dots, \tau_{m}^{\prime\prime})$, while keeping the endpoints fixed: $\tau_1^{\prime\prime}=\tau_1^{\prime}$ and $\tau_m^{\prime\prime}=\tau_m^{\prime}$. 

The objective of subsequence reconstruction is to minimize the subsequence cost $L_{\text{sub}}(\boldsymbol{\tau}^{\prime\prime})$, which is given by:
\begin{equation}
{L_{\text{sub}}(\boldsymbol{\tau}^{\prime\prime})=\sum\limits_{i=1}^{m-1}{\text{cost}(\tau_i^{\prime\prime}, \tau_{i+1}^{\prime\prime})}},
% \label{eq:Balance}
\end{equation}
where the goal is to reorder the subsequence to reduce the total cost, improving the overall tour quality.

\subsection{Regional Reconstruction Task}
The regional reconstruction task aims to improve the overall TSP solution by focusing on local regions within the tour. Given a tour $\boldsymbol{\tau}$, the corresponding set of edges $\mathcal{E}_{\boldsymbol{\tau}}=\{e_{\tau_{i},\tau_{i+1}} | 1\leq i \leq n-1\} \cup \{e_{\tau_{n},\tau_{1}}\}$ represents the set of connections in the tour. 

For each city $\tau_i$, we consider a 2D coordinate $c = (x, y)$ and define the set of $k$-nearest neighbors $\mathcal{V}_{c}$ in graph $\mathcal{G}$ as the nodes closest to $c$.
The edges in $\mathcal{E}_{\boldsymbol{\tau}}$ that have nodes from $\mathcal{V}_{c}$ as their predecessors are defined as $\mathcal{E}_{\boldsymbol{\tau}}^{\prime}=\{e_{\tau_{i},\tau_{i+1}} | \tau_{i} \in \mathcal{V}_{c}\}$. Specifically, define $\tau_{n+1}=\tau_{1}$ to denote the successor of $\tau_{n}$. 

The objective of the regional reconstruction task is to improve the current tour by modifying the edges of the regional set $\mathcal{E}_{\boldsymbol{\tau}}^{\prime}$.  
Specifically, we aim to find an edge set $\mathcal{E}^{\text{add}} \subset \mathcal{E}$ such that the updated edge set $\mathcal{E}_{\boldsymbol{\tau}}^{\text{new}} = \mathcal{E}^{\text{add}} \cup (\mathcal{E}_{\boldsymbol{\tau}} \setminus \mathcal{E}_{\boldsymbol{\tau}}^{\prime})$ forms a new Hamiltonian circuit $\boldsymbol{\tau}^\text{new}$. The goal is to minimize the total cost of the new tour $L_{\text{total}}(\boldsymbol{\tau}^\text{new})$, 
which directly corresponds to minimizing the cost of the newly added edges in $\mathcal{E}^{\text{add}}$, thus improving the overall solution quality.
