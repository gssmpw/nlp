\subsection{TSP Problem}
Current NCO research focuses on the classic 2D Euclidean TSP.
Given an $n$ nodes undirected fully connected graph $\mathcal{G}(\mathcal{V}, \mathcal{E})$, where $\mathcal{V}=\{v_i | 1\leq i \leq n\}$ represents the set of city nodes and $\mathcal{E}=\{e_{ij} | 1\leq i, j\leq n\}$ represents the set of all edges. Let $\text{cost}(v_i, v_j)$ denote the cost of moving from  $v_i$ to $v_j$, which equals the distance between $v_i$ and $v_j$. The objective of the TSP is to find a Hamiltonian circuit (or a tour), represented as a sequence $\boldsymbol{\tau} = (\tau_1, \tau_2, \dots, \tau_n)$, that minimizes the total cost $L_{\text{total}}(\boldsymbol{\tau})$, calculated as:
\begin{equation}
{L_{\text{total}}(\boldsymbol{\tau})=\text{cost}(\tau_n, \tau_1)+\sum\limits_{i=1}^{n-1}{\text{cost}(\tau_i, \tau_{i+1})}},
\label{eq:Balance}
\end{equation}

\subsection{Subsequence Reconstruction Problem}
Given a subsequence $\boldsymbol{\tau}^{\prime} = (\tau_1^{\prime}, \tau_2^{\prime}, \dots, \tau_{m}^{\prime})$ with length $m$, where $\boldsymbol{\tau}^{\prime} \subseteq \boldsymbol{\tau}$ and $m \leq n$. 
The task of subsequence reconstruction aims to reorder its intermediate nodes to produce a new sequence $\boldsymbol{\tau}^{\prime\prime} = (\tau_1^{\prime\prime}, \tau_2^{\prime\prime}, \dots, \tau_{m}^{\prime\prime})$, while keeping the endpoints fixed such that $\tau_1^{\prime\prime}=\tau_1^{\prime}$ and $\tau_m^{\prime\prime}=\tau_m^{\prime}$. The objective is to minimizes the subsequence cost $L_{\text{sub}}(\boldsymbol{\tau}^{\prime\prime})$, which has:
\begin{equation}
{L_{\text{sub}}(\boldsymbol{\tau}^{\prime\prime})=\sum\limits_{i=1}^{m-1}{\text{cost}(\tau_i^{\prime\prime}, \tau_{i+1}^{\prime\prime})}},
\label{eq:Balance}
\end{equation}

\subsection{Regional Reconstruction Problem}
We define the novel regional reconstruction problem. Given a tour $\boldsymbol{\tau}$, which corresponds to a set of edges $\mathcal{E}_{\boldsymbol{\tau}}=\{e_{\tau_{i},\tau_{i+1}} | 1\leq i \leq n-1\} \cup \{e_{\tau_{n},\tau_{1}}\}$. Consider a 2D coordinate $c$, where the set of $k$-nearest neighbor nodes of $c$ in the graph $\mathcal{G}$ is denoted by $\mathcal{V}_{c}$.
The edges in $\mathcal{E}_{\boldsymbol{\tau}}$ with nodes from $\mathcal{V}_{c}$ as their predecessors are represented as $\mathcal{E}_{\boldsymbol{\tau}}^{\prime}=\{e_{\tau_{i},\tau_{i+1}} | \tau_{i} \in \mathcal{V}_{c}\}$. Specifically, define $\tau_{n+1}=\tau_{1}$ to denote the successor of $\tau_{n}$. 
% Removing $E_{\boldsymbol{\tau}}^{\prime}$ from $E_{\boldsymbol{\tau}}$ results in $E_{\boldsymbol{\tau}} \setminus E_{\boldsymbol{\tau}}^{\prime}$.
% The objective of the regional reconstruction task is to find a edges set $E^{\text{add}} \subset E$, and add to $E_{\boldsymbol{\tau}} \setminus E_{\boldsymbol{\tau}}^{\prime}$, resulting in $E_{\boldsymbol{\tau}}^{\text{new}} = E^{\text{add}} \cup (E_{\boldsymbol{\tau}} \setminus E_{\boldsymbol{\tau}}^{\prime})$ 
The objective of the regional reconstruction task is to find a edges set $\mathcal{E}^{\text{add}} \subset \mathcal{E}$ such that the updated edge set $\mathcal{E}_{\boldsymbol{\tau}}^{\text{new}} = \mathcal{E}^{\text{add}} \cup (\mathcal{E}_{\boldsymbol{\tau}} \setminus \mathcal{E}_{\boldsymbol{\tau}}^{\prime})$ forms a new Hamiltonian circuit $\boldsymbol{\tau}^\text{new}$ and minimize the total cost $L_{\text{total}}(\boldsymbol{\tau}^\text{new})$, equivalent to minimizing $\sum_{e_{i,j}\in \mathcal{E}^{\text{add}}} {\text{cost}(i,j)}$. 
% Let $q_{i,j}$ denotes whether the edge $e_{i,j}$ in $E_{\boldsymbol{\tau}}^{\text{new}}$. Specifically, $q_{i,j}=1$ if $e_{i,j} \in E_{\boldsymbol{\tau}}^{\text{new}}$, and $q_{i,j}=0$ otherwise. 
% The objective function of regional reconstruction can be formulated as: 
% \begin{subequations}\label{eq:Regional reconstruction problem}
% \begin{align}
% \min_{E^{\text{add}}} &\quad \sum_{e_{i,j}\in E^{\text{add}}} {\text{cost}(i,j)}
% \label{eq:Optimization function}\\
% & \quad E_{\boldsymbol{\tau}}^{\text{new}} = E^{\text{add}} \cup (E_{\boldsymbol{\tau}} \setminus E_{\boldsymbol{\tau}}^{\prime}), 
% \label{eq:Regional reconstruction constraints1}\\
% &{\quad \sum_{i =1}^{n} {q}_{i,j} = 1,~~~~\forall~1 \leq j \leq n,}
% \label{eq:Regional reconstruction constraints2}\\
% &{\quad \sum_{j =1}^{n} {q}_{i,j} = 1,~~~~\forall~1 \leq i \leq n,}
% \label{eq:Regional reconstruction constraints3}
% \end{align}
% \end{subequations}
