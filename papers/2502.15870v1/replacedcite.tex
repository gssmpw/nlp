\section{Literature Review}
The relationship between individual perceptions of AI limitations and organizational readiness is a multi-level phenomenon that cannot be fully understood through traditional technology adoption frameworks alone. An integrated theoretical approach incorporating sensemaking processes trust development mechanisms and organizational readiness dynamics is needed.

This literature review approaches AI adoption and readiness as a multi-level phenomenon. It addresses external pressures (such as competitive forces, policy frameworks, and societal discourse), organizational-level readiness (in terms of capabilities, culture, and infrastructure), and finally, individual-level factors (such as perceptions of AI limitations and sensemaking processes). The interplay across these levels highlights how organizations successfully adopt AI.


\subsection{AI Adoption in Organizations}

AI adoption has several unique characteristics that differentiate it from traditional technology adoption. ____ identify two characteristics: inscrutability and data dependency. Inscrutability manifests in the difficulty of predicting system behavior and explaining decision processes, while data dependency requires continuous system adjustments as organizational data evolves. These characteristics create increased variability in organizational decision-making processes that require new coordination mechanisms ____. The inscrutability concept is particularly interesting for understanding individual interactions with AI systems, as it directly influences how organizational members interpret and respond to AI-driven changes.

____ argue that AI adoption increases decision variation across interconnected organizational tasks, asking organizations to either reduce task interdependencies or implement strong coordination mechanisms. This finding challenges the typical focus on individual task-level AI adoption by highlighting the systemic nature of organizational AI adoption. ____ extend this understanding by showing how AI adoption introduces both technological affordances and constraints that vary significantly based on organizational size. Their study reveals that while larger firms perceive primarily operational affordances focused on efficiency and quality improvements, smaller firms see marketing affordances as more important, leading to different adoption patterns and outcomes.

With his Technology-Organization-Environment (TOE) framework, ____ emphasizes that innovation adoption depends on the interplay between technological features, organizational characteristics, and environmental conditions. Applied to AI adoption, ____ show how these elements are expressed through innovation management approaches, organizational AI readiness, and environmental pressures. This can also serve as a framework for looking at adoption barriers. ____ categorizes them across technical aspects (data availability, model reusability), organizational considerations (resource allocation, support infrastructure), and social dimensions (human-AI interaction, job security concerns, trust issues).

The organizational context shapes adoption patterns through what ____ identify as implementation capabilities. These capabilities encompass AI project planning, co-development, data management systems, and model lifecycle management processes. ____ demonstrate that AI adoption among management ranks, rather than just IT specialists, drives positive organizational outcomes. This finding highlights the importance of broad organizational involvement in the adoption process and suggests that successful AI implementation requires capabilities across different organizational levels.

Adopting AI technologies follows patterns that reflect unique characteristics and organizational implications. ____ emphasize the importance of human-machine teaming in successful AI adoption, noting that individuals build trust with AI systems through experience, expert endorsement, and systems designed to accommodate professional autonomy. This observation aligns with ____ identification of success factors, including effective communication channels, management support, training, and established reporting mechanisms for addressing AI-related concerns.

These implementation patterns reveal several sensitizing concepts important for understanding AI adoption. First, AI inscrutability is fundamental in shaping how organizational members interact with and interpret AI systems. Second, the system-wide impact concept captures the extensive organizational changes that AI adoption necessitates. Third, adoption barriers appear across technical, organizational, and social dimensions, providing observable indicators of adoption challenges. Fourth, capability distribution reflects the spread of AI-related competencies across organizational levels. Finally, implementation patterns capture organizations' observable approaches to managing AI adoption.

The complexity and uniqueness of AI adoption create distinct challenges requiring careful consideration of technical and organizational dimensions. These challenges are particularly evident in what ____ identifies as the social considerations of AI adoption, including increased dependence on non-human agents, job security concerns, and trust issues. Understanding these dynamics provides context for examining how organizations develop readiness for AI adoption, particularly considering the role of individual perceptions in shaping adoption outcomes.

Beyond these internal adoption challenges, external pressures - such as industry competition, global technology trends, and regulatory policy - further shape the path to AI adoption ____. ____ highlight the role of government support in the Saudi context, showing how initiatives like Vision 2030 influence adoption through multiple channels: directly through regulatory frameworks and policies and indirectly by shaping senior management support and competitive dynamics between organizations. Their study reveals that government support affects all aspects of the technology-organization-environment framework, creating opportunities and pressures for adopting organizational AI.

\subsection{Organizational Readiness for AI Adoption}


Organizational readiness is not only about having the right resources or leadership in place; it also mediates between broad external pressures (e.g., policy mandates and competitive landscapes) and how employees on the ground perceive and engage with AI. Organizational readiness serves as a bridge between broader external forces (such as policy requirements and market competition) and how individual employees actually engage with and implement AI in their daily work. It determines how well an organization can translate high-level strategic demands into successful adoption by its workforce. ____ emphasize that readiness involves aligning organizational assets, individual capabilities, and leadership commitment to support AI initiatives. They identify five core domains – strategic alignment, resources, knowledge, culture, and data - that collectively determine readiness.

AI readiness demands more than just technical infrastructure. ____ highlight that success depends on how well organizational processes can integrate AI. That includes adapting workflows, ensuring data compatibility, and developing continuous learning and refinement systems. Readiness is, therefore, an evolving state influenced by the organization's ability to adjust and respond to AI's changing demands—a classical organizational learning problem.

Organizations need to develop specific capabilities for successful AI adoption. ____ identified four concrete organizational capabilities: AI project planning, co-development of AI systems, data management, and AI model lifecycle management. This is a more process-oriented approach to readiness than the readiness factors ____ synthesized.

Leadership is vital for AI readiness. ____ argue that senior management support significantly affects individual attitudes and readiness to adopt AI. Leaders are important in allocating resources, prioritizing AI in strategic plans, and addressing resistance from change recipients ____.

Trust in AI systems is another determinant for organizational readiness, influencing adoption and sustained engagement ____. ____ emphasize that trust involves cognitive elements, like reliability and transparency, and emotional elements, such as the perceived human-likeness of AI. ____ highlight trustworthiness principles - beneficence, non-maleficence, autonomy, justice, and explicability - as critical to fostering trust. Building trust requires consistent system reliability, ethical alignment, and clear explanations of AI behavior. ____ emphasize the importance of expert endorsements, validation, and iterative user interactions in increasing trust. Similarly, ____ found that integrating AI into workflows while respecting user autonomy strengthens trust by framing AI as a collaborative tool rather than a replacement. Addressing these dimensions ensures individuals view AI as reliable and aligned with their roles, reducing resistance and enabling successful adoption.

Individual employees' cultural values collectively shape another critical dimension of organizational preparedness for AI adoption, as these personal orientations aggregate to influence the organization's overall cultural readiness for technological change. According to ____, individual cultural values significantly impact technology acceptance and readiness. Their research found that collectivism and long-term orientation positively influence the perceived usefulness and ease of use of new technologies at the individual level. Additionally, they found that a less masculine organizational culture helps reduce employee discomfort with technological change. ____ find that organizations, particularly in the exhibition industry, struggle with cultural barriers, such as risk aversion and resistance to change, which hinder readiness. Conversely, a culture of openness and collaboration can drive more effective adoption processes.

Organizational readiness is not developed in isolation but interacts with external pressures and opportunities. ____ highlight how competitive environments drive organizations to develop AI capabilities aggressively. Further, government and regulatory support play a significant role in shaping readiness. Indeed, policy-level initiatives can catalyze AI readiness by providing resources or mandating standards ____.

Although organizational readiness lays the strategic and cultural groundwork for AI adoption, its success ultimately depends on how individual employees perceive and integrate these technologies into their work. ____ highlight the role of workforce capabilities, particularly in developing skills and trust in AI systems. Individuals who view AI as threatening their autonomy or job security may resist its implementation. Addressing these concerns through communication, training, and involvement in AI projects can increase readiness. Hence, the following section turns to the micro-level factors that can accommodate or undermine readiness.


\subsection{Individual Perceptions of Limitations}


The successful adoption of AI technologies within organizations is not solely determined by technical capabilities but is significantly influenced by individual perceptions of AI limitations ____. These perceptions can act as barriers or facilitators to adoption, affecting organizational readiness and the overall implementation process ____.

Individuals' perceptions of AI limitations encompass a range of concerns that can hinder the adoption of AI technologies within organizations. One area of concern is the issue of trust and reliability. ____ and ____ highlight that individuals find building trust in AI systems difficult due to inconsistent or opaque outputs. This lack of trust is further impeded when AI systems fail to perform reliably in critical applications, leading to skepticism about their usefulness, as ____ noted. Additionally, ____ observe that fear and resistance to adoption can come from a misalignment between AI technologies and individuals' values and anxiety over dealing with complex IT systems.

Transparency and explainability of AI systems are also significant concerns among individuals. The "black box" nature of AI algorithms, particularly in complex models like large language models (LLMs), poses challenges for those who require clear and interpretable decision-making processes. ____ and ____ discuss how the lack of transparency can hinder individuals' understanding and acceptance of AI outputs. ____ and ____ further point out that AI systems often cannot provide meaningful, user-aligned explanations, which can decrease trust and confidence among users.

Furthermore, concerns about human-AI interaction play a role in shaping individuals' perceptions. ____ and ____ note that individuals may be cautious of over-reliance on AI and the potential for automation complacency, leading to skill degradation or reduced caution in their roles. Additionally, cognitive and self-serving biases can influence how individuals interpret AI capabilities. ____ demonstrate that when people lack information about how AI systems operate - specifically about what happens to machines' earnings in economic interactions - they tend to form self-serving beliefs that justify less cooperative behavior with the machines. The lack of emotional intelligence in AI systems, especially in contexts requiring empathy and nuanced human interaction, is another limitation that ____ cited.

Bias and fairness issues embedded in AI systems are significant concerns that affect individuals' willingness to adopt these technologies. ____ discuss how biases in training data can affect AI outputs, leading to unfair or discriminatory outcomes. ____ and ____ emphasize that amplifying societal stereotypes through AI systems poses ethical and legal risks, prompting individuals to question the fairness and appropriateness of AI-driven decisions within their organizations.

Technical limitations, such as inconsistent performance, contribute to individuals' skepticism about AI technologies. ____ and ____ report that perceived inconsistencies in AI performance can undermine individual confidence in these systems. ____ and ____ highlight that AI systems' difficulties in processing nuanced or context-specific information relevant to specific tasks can further diminish individuals' perceptions of AI effectiveness.

Ethical considerations also shape individuals' perceptions of AI limitations. ____ and ____ note that potential societal harm due to bias, misinformation, or unethical use can lead to resistance among individuals prioritizing ethical standards in their work. ____ and ____ observe that gaps between ethical principles and their implementation in AI technologies can result in individuals questioning the adoption of such systems.

Finally, practical challenges in implementing and integrating AI systems into existing workflows are perceived as significant limitations. ____ identify key challenges in interactive AI systems, including difficulties in defining appropriate roles between humans and AI, managing trade-offs between competing objectives like accuracy and interpretability, and dealing with multiple sources of uncertainty. The challenges of integrating AI can also vary by context - for instance, in academic writing, ____ found disagreement among academics about appropriate AI use and reporting requirements, with differences shaped by role, ethics perceptions, and language background. ____ found that while there was no broad aversion to AI systems, persistent human favoritism could affect integration efforts.

These perceived limitations align with key sensitizing concepts such as AI inscrutability and the adoption barriers across technical, organizational, and social dimensions discussed by ____, ____, and ____. Understanding these perceptions is important for organizations aiming to improve their readiness for AI adoption, as they influence individuals' willingness to engage with and support the integration of AI technologies.

According to ____, perceptions of adopting an information technology innovation are shaped by factors such as relative advantage, compatibility, complexity, trialability, and observability. In AI adoption, individuals' prior experiences with technology, individual innovativeness, and organizational communication channels contribute to perception formation ____. For instance, individuals with higher personal innovativeness are more likely to develop positive perceptions of AI technologies ____.

The domain of uncertainty framework suggests that uncertainties associated with change fit into four domains: conceptual uncertainty (What is the change?), functional value uncertainty (What is the value of the change?), process uncertainty (How will the change come about?), and impact uncertainty (What is the broader impact of the change?) ____. Individuals form perceptions based on how AI technologies address these uncertainties. For example, conceptual uncertainty arises from a lack of understanding of AI's functionalities, while impact uncertainty pertains to doubts about AI's long-term effects on job security and organizational practices.

External factors such as media representations, societal discourse, and organizational communication strategies influence perception formation ____. The perceived risks and uncertainties associated with AI, including job displacement and ethical concerns, are amplified or mitigated through these channels ____. Communication channels shape perceptions, as individuals rely on mass media and interpersonal communications to develop their understanding of AI technologies ____.

Finally, the literature highlights that individuals' perceptions of AI limitations are shaped through interpretation and meaning-making processes. Individuals attempt to understand how AI fits into their professional roles, organizational goals, and broader societal contexts, reconciling uncertainties about transparency, fairness, and ethical alignment ____. This interpretive process is both individual and collective, as organizational culture, peer interactions, and shared assumptions influence how individuals construct their understanding of AI technologies. ____ introduce the concept of "technological frames," highlighting how shared assumptions and knowledge within organizations shape people's perceptions and interactions with technology. Similarly, ____ demonstrate how informal networks and lateral employee interactions contribute to evolving interpretations during organizational change. These dynamics suggest that perceptions of AI are formed through ongoing collective processes at both personal and organizational levels. Understanding these shared interpretations offers valuable insights into how readiness and adoption are shaped, which will be examined in greater depth in the following section.

\vspace{-3mm}
\subsection{Sensemaking}

Sensemaking theory provides a valuable framework for understanding how individuals, teams, and organizations interpret and respond to AI. This approach is inherently multi-level, encompassing the personal sensemaking of employees, the collective sensemaking of groups or departments, and organizational sensemaking processes ____. These nested sensemaking processes also incorporate external cues - such as media stories, industry regulations, and competitive forces - reinforcing that AI adoption is shaped by influences from the macro-level to the micro-level.

Sensemaking is triggered by cues that disrupt individuals' existing understanding, prompting them to seek explanations and restore meaning ____. With their unique characteristics and potential implications, introducing AI technologies can serve as such a trigger, creating a need for individuals to make sense of these new realities ____. ____ identifies seven properties of sensemaking: identity construction, retrospection, enactment, social interaction, ongoing nature, extraction of cues, and plausibility over accuracy. These properties provide a framework for examining how individuals interpret AI limitations and construct their understanding of the technology's role in their work. Identity construction is central to sensemaking, as individuals interpret events in ways that maintain a consistent self-conception ____. In the context of AI adoption, individuals may perceive AI limitations in ways that align with their professional identities and values ____.

Sensemaking is inherently retrospective, as individuals make sense of events by drawing on past experiences and existing frameworks (Weick, 1995). Individuals' prior experiences with technology adoption and their exposure to the societal discourse around AI can shape their interpretations of AI limitations ____. This retrospective nature also suggests that individuals' perceptions may evolve as they accumulate experiences with AI technologies over time.

Enactment is another property of sensemaking, emphasizing that individuals actively construct the environments they face ____. In the context of AI adoption, individuals' actions and responses to the technology can shape the organizational reality surrounding AI. For instance, following what we know about confirmation bias, resistance, or avoidance behaviors based on perceived limitations could create self-fulfilling prophecies, reinforcing the challenges of AI integration ____.

Social interaction is important to sensemaking, as individuals rely on shared narratives and collective interpretations to construct meaning ____. Individuals' perceptions of AI limitations are not formed in isolation but are influenced by interactions with colleagues, organizational communication, and broader societal discourse ____. The social nature of sensemaking suggests that organizations can actively shape individuals' perceptions through strategic communication and promoting a supportive culture around AI adoption.

Sensemaking is an ongoing process, as individuals continuously update their interpretations based on new information and experiences ____. This ongoing nature is particularly relevant in the rapidly evolving AI landscape, where individuals' perceptions may shift as they encounter new applications, capabilities, and challenges. Organizations need to recognize the dynamic nature of sensemaking and provide ongoing support and communication to help individuals navigate the evolving realities of AI adoption.
The extraction of cues refers to the process by which individuals selectively attend to certain aspects of their environment to support their interpretations ____. In the context of AI adoption, individuals may focus on cues that reinforce their existing perceptions of AI limitations, such as instances of biased outputs or technical failures. Change agents could actively manage the cues available to individuals by highlighting successful AI implementations and providing transparent information about the technology's capabilities and limitations.

Finally, sensemaking prioritizes plausibility over accuracy, as individuals seek interpretations that are sufficiently coherent and credible to guide action ____. Individuals' perceptions of AI limitations may not always align with the technology's objective realities but are constructed in ways that make sense given their experiences, beliefs, and organizational context. This suggests that organizations must create narratives and experiences that promote positive and plausible interpretations of AI's role in the workplace.

Sensemaking perspective aligns with key insights from the previously discussed literature, such as the importance of addressing conceptual and impact uncertainties ____, the role of communication channels in shaping perceptions ____, and the influence of organizational culture on adoption readiness ____. Sensemaking theory extends these insights by providing a framework for understanding the cognitive and social processes through which individuals actively construct their perceptions of AI limitations.

Moreover, a multi-level sensemaking theory offers a dynamic and process-oriented view of perception formation, complementing the more static factors emphasized in technology acceptance models like TAM and UTAUT ____. By recognizing the ongoing and retrospective nature of sensemaking across different levels (macro to micro), organizations can develop more responsive and adaptive strategies for managing individuals' perceptions throughout the AI adoption process.

However, sensemaking theory also highlights the challenges of managing perceptions in the face of technological complexity and uncertainty. AI technologies' inscrutability and data dependency ____ can make it difficult for individuals to extract clear cues and construct plausible interpretations. The rapidly evolving capabilities of AI may also require continuous updating of sensemaking frameworks, placing demands on individuals and organizations to remain adaptable.

Bridging the gap between individual-level sensemaking and organizational-level readiness can be understood through organizational learning frameworks, such as the "4I" framework by ____. This framework conceptualizes learning as a multi-level process composed of four stages: intuiting, interpreting, integrating, and institutionalizing. At the individual level, individuals' intuit' and 'interpret' cues derived from their encounters with AI technologies and their perceived limitations. For example, employees may intuitively feel uncertainty or distrust when encountering opaque AI-driven decisions. Through personal interpretation, they construct a narrative that explains why the system behaves unpredictably. These individually held narratives converge as individuals engage in conversations and share experiences, moving from isolated interpretations to more collectively shared meanings.

Once collective interpretations solidify, the process shifts into 'integrating' at the group level. Teams develop a shared understanding of AI's limitations - its inscrutability, data dependencies, or fairness issues - and collectively decide how to respond. Over time, these group-level interpretations become 'institutionalized' into organizational practices, policies, and routines, shaping how the organization prepares for, manages, and leverages AI. Thus, individual sensemaking about AI limitations diffuses upward through group interactions and ultimately informs the organization's formal systems and culture, influencing organizational readiness for AI adoption. In this way, the alignment (or misalignment) between individual interpretations and organizational-level structures and strategies determines how effectively the organization can integrate AI technologies into its core operations.

While the sensemaking perspective provides valuable insights into how individuals interpret and make meaning of AI technologies, the next step is to distill key concepts that can guide the empirical investigation. Drawing from the literature reviewed above, several sensitizing concepts are particularly relevant for understanding how individuals' perceptions of AI limitations influence organizational readiness. These concepts serve not as rigid theoretical constructs but as flexible guides that orient the investigation while remaining open to emergent themes and patterns.

\subsection{Sensitizing Concepts}

The literature suggests several interconnected sensitizing concepts that operate across multiple levels - from individual cognition to organizational processes to external influences. These concepts guide the empirical investigation and anticipate the dynamic relationships that emerge in the findings. The concepts are organized to reflect how perceptions of AI limitations flow from individual interpretation through collective sensemaking to organizational adaptation. These concepts also guide the empirical inquiry into how organizations navigate AI adoption, from external demands and industry-wide influences to individual employees' daily interpretations and actions. The literature on AI adoption, organizational readiness, individual perceptions, and sensemaking suggests several interconnected concepts that inform exploration in further empirical investigation. These sensitizing concepts provide a basis for understanding how individuals' perceptions of AI limitations influence organizational readiness for adoption.

The literature suggests that how individuals form and develop their perceptions of AI limitations is a complex process influenced by individual and contextual factors. Understanding how people identify and categorize different types of limitations is crucial at the individual level ____. Professional background and expertise shape these interpretations, with individuals from different functional areas potentially perceiving limitations differently ____.

Contextual influences emerge as equally important in perception formation. The organizational environment, including existing technological infrastructure and support systems, shapes individuals' perceptions of limitations ____. Industry-specific challenges and opportunities create unique contexts influencing perception formation ____. External discourse, including media representation and professional networks, also contributes to how individuals understand and interpret AI limitations ____.

The literature highlights sensemaking at both individual and collective levels in how people interpret and respond to AI limitations. At the individual level, sensemaking involves personal interpretation and meaning-making processes ____. Individuals engage in retrospective reflection on their experiences with AI, drawing on their professional identity and past experiences to make sense of the limitations they encounter ____. This individual sensemaking process is ongoing, as people continuously update their interpretations based on new experiences and information.

The collective dimension of sensemaking emerges through social interactions and shared meaning construction. Knowledge sharing is an important mechanism, with groups developing shared understandings through formal and informal discussions ____. Informal networks are particularly important in sharing experiences and interpretations across organizational boundaries. These collective processes do not replace individual sensemaking but interact with it, as individuals draw on collective interpretations while contributing their understanding to the group's sensemaking process. Social dynamics within organizations influence both individual and collective sensemaking. Peer experiences and opinions shape interpretations, while leadership is essential in framing how limitations are understood and addressed ____.

The literature suggests that organizational readiness develops through distinct patterns influenced by organizational responses and cultural evolution. Organizations adapt to perceived limitations through various mechanisms, including resource allocation decisions and capability development ____. These responses shape the organization's overall readiness for AI adoption.

Cultural evolution appears to be an important aspect of readiness development. Organizations change work practices and routines as they adapt to AI technologies. Shifts in organizational attitudes and the development of learning processes emerge as important elements of this evolution. Patterns of resistance and acceptance also play a significant role in how readiness develops over time.

These sensitizing concepts suggest several key areas for exploration in this empirical investigation. They emphasize the importance of examining individual experiences and collective processes to understand how perceptions influence readiness. They also highlight the need to consider formal organizational responses and informal social dynamics. The concepts serve as a base for the interview guide (see Appendix 12.1), suggesting areas of inquiry while remaining open to emergent themes. These concepts remain deliberately broad to allow for unexpected findings and emerging patterns during data collection. They orient the investigation while maintaining flexibility to explore new directions as they emerge from the interviews.
\end{multicols}













\begin{figure}[!ht]
    \centering
    {\includegraphics[width=1\linewidth]{Images/1.png}}
    \caption{Sensitizing Concepts}
    \label{fig:fig1}\vspace{10mm}
\end{figure}

\begin{multicols}{2}