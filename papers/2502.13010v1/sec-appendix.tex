\appendix
% \section{Additional results on PUBMEDQA }
% The \textbf{PubMedQA} dataset is a benchmark dataset specifically designed for biomedical question answering, focusing on extracting information from biomedical literature. Unlike MEDQA and MedMCQA, which emphasize multiple-choice questions, PubMedQA consists of yes/no/maybe questions derived from PubMed abstracts, requiring models to comprehend and infer from scientific text. This dataset serves as a critical testbed for evaluating medical \gls{qa} models' ability to extract and reason over biomedical evidence. The results for PubMedQA, presented in Table~\ref{tab:pubmedqa_comparison}, highlight the effectiveness of models like Meditron-70B and BioGPT-Large in leveraging domain-specific pretraining to achieve high accuracy. These models outperform standard LLMs by incorporating biomedical corpus fine-tuning, demonstrating the value of domain adaptation for robust medical question answering.
% \begin{table}[h!]
% \small
% \centering
% \caption{Comparison of Models on the PubMedQA.}
% \label{tab:pubmedqa_comparison}
% \renewcommand{\arraystretch}{0.9}

% \begin{tabular}{@{}lcc@{}}
% \toprule
% \textbf{Model} & \textbf{Model Size} & \textbf{Acc. (\%)} \\
% \midrule
% \gls{myrag} & & \\
% Meditron-70B \citep{chen2023meditron} & 70B & 81.6 \\
% BioGPT-Large \citep{luo2022biogpt} & 1.5B & 81.0 \\
% RankRAG-llama3 \citep{yu2024rankrag} & 70B & 79.8 \\
% Med-PaLM 2 \citep{singhal2023large} & -- & 79.2 \\
% Flan-PaLM \citep{singhal2022large} & 540B & 79.0 \\
% BioGPT  \citep{luo2022biogpt} & 345M & 78.2 \\
% Codex 5-shot \citep{lievin2024can} & -- & 78.2 \\
% Human Performance\citep{jin2019pubmedqa} & -- & 78.0 \\
% GAL \citep{taylor2022galactica} & 120B & -- \\
% \bottomrule
% \end{tabular}%

% \end{table}
\section{Confidence Scoring for the Relationships in the MKG}
\label{app:confidence}
A confidence score, $s_{ij}$, is assigned to each inferred relationship, reflecting its strength and relevance. The scoring criteria are as follows:
\begin{itemize}
    \item \textbf{10}: The target is directly and strongly related to the item, with clear, unambiguous relevance.
    \item \textbf{7-9}: The target is moderately to highly relevant to the item but may have some ambiguity or indirect association.
    \item \textbf{4-6}: The target has some relevance to the item but is weak or only tangentially related.
    \item \textbf{1-3}: The target has minimal or no meaningful connection to the item.
\end{itemize}


\section{Evaluating the Accuracy and Robustness of the Medical Knowledge Graph}
\label{app:mkd-analysis}

The quality and reliability of the dynamically generated \gls{mkg} are critical for its effectiveness in enhancing medical \gls{qa} systems. To validate the accuracy, robustness, and usability of the \gls{mkg}, a structured evaluation involving expert \glspl{llm} in the medical domain, such as GPT medical model, was conducted. This section outlines the methodology used to evaluate the \gls{mkg}, emphasizing interpretability, clinical relevance, and robustness in real-world applications. Additionally, the role of medical experts in verifying the accuracy and applicability of the \gls{mkg} is discussed, underscoring the necessity of human expertise in validating AI-driven medical knowledge representations.

To assess accuracy and robustness, a two-phase evaluation process was employed. In the first phase, a group of expert \glspl{llm} specializing in medical domains reviewed a subset of the \gls{mkg}, including dynamically generated nodes, relationships, confidence scores, and summaries for various medical queries. They evaluated the accuracy of medical terms and concepts, the relevance of relationships between nodes, the reliability of node summaries, and the alignment of confidence scores with the perceived strength and reliability of the connections. Each \gls{llm} independently rated the graph components on a scale of 1 to 10. The results showed an average accuracy score of 8.9/10 for node identification, 8.8/10 for relationship relevance, and 8.5/10 for the clarity and precision of node summaries. Confidence scores generally aligned well with the \glspl{llm}'s assessments, as illustrated in Tables~\ref{tab:mkg_analysis_samples} and~\ref{tab:mkg_analysis_more_samples}, which highlight strong relationships across domains such as ophthalmology, cardiovascular treatments, and dermatology.

In the second phase, blind testing was conducted to evaluate usability and human-readability. Expert \glspl{llm} were tasked with answering complex medical queries requiring multi-hop reasoning, such as managing comorbidities or determining multi-drug treatment protocols. As shown in Table~\ref{tab:mkg_analysis_samples}, relationships such as the co-usage of Ketotifen and Fluorometholone for allergic conjunctivitis or Labetalol and Nitroglycerin for acute hypertension demonstrated the \gls{mkg}'s ability to model clinically relevant associations effectively. The \glspl{llm} achieved a 89\% accuracy rate in these test scenarios. Additionally, the \glspl{llm} rated the \gls{mkg} 9.4/10 for interpretability and usability, underscoring its strength in visually and contextually representing complex medical relationships.

To further ensure the clinical relevance and practical applicability of the \gls{mkg}, medical experts, including practicing physicians and clinical researchers, were involved in evaluating the generated relationships and summaries. Unlike \glspl{llm}, medical experts provided qualitative assessments, identifying potential discrepancies, overlooked nuances, and contextual dependencies that automated models might miss. The medical experts particularly assessed:
\begin{enumerate}
    \item The correctness and completeness of medical relationships, ensuring they align with established clinical knowledge and best practices.
    \item The validity of multi-hop reasoning paths, verifying whether inferred relationships reflected logical clinical decision-making processes.
    \item The utility of the \gls{mkg} in real-world medical applications, particularly in aiding diagnostic and treatment decision-making.
\end{enumerate}

The feedback from medical experts was instrumental in refining the graph, addressing inconsistencies, and enhancing the confidence scores to better reflect real-world medical reliability. Notably, medical expert ratings aligned well with \gls{llm} evaluations but provided deeper insights into the contextual limitations of the graph. For example, while \glspl{llm} accurately linked Diltiazem and Nitroglycerin in cardiovascular treatment, medical experts highlighted additional considerations such as contraindications in specific patient populations, which were subsequently incorporated into the \gls{mkg}.

The detailed evaluations in Tables~\ref{tab:mkg_analysis_samples} and~\ref{tab:mkg_analysis_more_samples} provide further insights into the graph’s performance across diverse medical domains. For instance, the accurate representation of relationships between beta-blockers like Labetalol and Propranolol or the integration of treatments such as Diltiazem and Nitroglycerin for cardiovascular care highlight the \gls{mkg}'s capacity to support intricate clinical decision-making.

These results confirm that the \gls{mkg} is both human-readable and usable by advanced \glspl{llm}, making it an invaluable tool for medical \gls{qa} and decision-making. The graph’s structured format, enriched with confidence scores and summaries, ensures a clear and interpretable representation of medical knowledge while enhancing the efficiency and accuracy of \gls{qa} systems in addressing real-world medical scenarios. Moreover, the involvement of medical experts in the evaluation process enhances the credibility of the \gls{mkg}, ensuring that AI-driven insights align with clinical expertise and practical healthcare applications.


\begin{table*}[ht]
\centering
\small
\caption{Examples from the Medical Knowledge Graph (MKG) with Expert and Blind Analysis (Part 1)}
\label{tab:mkg_analysis_samples}
\resizebox{\textwidth}{!}{\begin{tabular}{|p{2cm}|p{3cm}|p{2cm}|p{3cm}|p{3cm}|p{3cm}|}
\hline
\textbf{Source Node} & \textbf{Relationship Type} & \textbf{Target Node} & \textbf{LLM Expert Analysis} & \textbf{Blind Analysis} & \textbf{Medical Expert Analysis}\\ \hline
Botulism & Directly related as it is the target concept. & Myasthenia gravis & Rated 9.2/10 for relevance and clinical importance, considered highly accurate. & Demonstrated effective multi-hop reasoning with a 92\% accuracy in identifying related conditions. & Rated 9.5/10 for relevance and accuracy, considered highly accurate.\\ \hline
Levodopa & Levodopa is a primary treatment for Parkinson's disease. & Parkinson's disease & Evaluated as highly reliable (9.6/10) for summarizing medical treatments and relationships. & Increases accuracy by 24\% in answering queries about Parkinson's treatments and comorbidities. & Rated 10/10 for relevance and accuracy, considered highly accurate.\\ \hline
Zidovudine & Zidovudine is an antiviral drug used for HIV treatment. & HIV/AIDS & Experts rated it 9.4/10 for interpretability, highlighting the clear representation of the relationship. & Provided contextually accurate responses regarding drug interactions and side effects in queries. & Rated 10/10 for relevance and accuracy, considered highly accurate.\\ \hline
Inhibition of thymidine synthesis & Cross-linking of DNA is directly related to thymidine synthesis as both involve nucleic acid metabolism. & Cross-linking of DNA & Rated 9.2/10 for relevance to nucleic acid metabolism and DNA replication. & Demonstrated high accuracy in answering multi-hop queries related to DNA synthesis pathways. & Rated 9/10 for relevance and accuracy, considered accurate.\\ \hline
Hyperstabilization of microtubules & Cross-linking of DNA can be related to the stabilization of microtubules. & Cross-linking of DNA & Rated 9.0/10 for highlighting structural modifications affecting cellular functions. & Increases the accuracy by 20\% in scenarios involving cellular structure interactions. & Rated 8/10 for moderated relevance.\\ \hline
Generation of free radicals & Free radicals can lead to oxidative damage, affecting DNA integrity and function. & Cross-linking of DNA & Rated 8.5/10 for its relevance to oxidative stress and DNA damage mechanisms. & Accurate in providing causal explanations for oxidative stress and DNA cross-linking. & Rated 7.5/10 for relevance.\\ \hline
Renal papillary necrosis & Allergic interstitial nephritis can lead to renal damage. & Allergic interstitial nephritis & Rated 9.0/10 for explaining the clinical progression of renal complications. & Effective in multi-hop reasoning for renal damage-related queries, achieving 91\% accuracy. & Rated 10/10 for relevance and accuracy, considered highly accurate.\\ \hline
% Eosinophilic granulomatosis with polyangiitis & EGPA is characterized by eosinophilia and systemic involvement, making it directly relevant to vasculitis-related conditions. & Eosinophilic granulomatosis with polyangiitis & Rated 10/10 for clinical relevance to systemic vasculitis. & Provided highly interpretable results, with strong visual representations of vasculitis pathways. & \\ \hline
% Cholesterol embolization & EGPA can lead to vascular complications, including cholesterol embolization. & Eosinophilic granulomatosis with polyangiitis & Rated 9.1/10 for its relevance to inflammation-driven vascular conditions. & Supported accurate insights into vascular and systemic inflammation pathways. & \\ \hline
% Renal papillary necrosis & Eosinophilic granulomatosis with polyangiitis can involve the kidneys, leading to renal damage. & Eosinophilic granulomatosis with polyangiitis & Rated 9.0/10 for its contribution to understanding renal effects of systemic vasculitis. & Accurately linked renal damage with systemic conditions, achieving 92\% accuracy. & \\ \hline
\end{tabular}}
\end{table*}

\begin{table*}[ht]
\centering
\small
\caption{Examples from the Medical Knowledge Graph (MKG) with Expert and Blind Analysis (Part 2)}
\label{tab:mkg_analysis_more_samples}
\resizebox{\textwidth}{!}{\begin{tabular}{|p{2cm}|p{3cm}|p{2cm}|p{3cm}|p{3cm}|p{3cm}|}
\hline
\textbf{Source Node} & \textbf{Relationship Type} & \textbf{Target Node} & \textbf{LLM Expert Analysis} & \textbf{Blind Analysis} &\textbf{Medical Expert Analysis}\\ \hline
Ketotifen eye drops & Ketotifen eye drops are antihistamines used for allergic conjunctivitis, which may be used alongside Fluorometholone for managing eye allergies. & Fluorometholone eye drops & Rated 9.2/10 for relevance in managing allergic conjunctivitis. & Demonstrated 93\% accuracy in multi-hop reasoning for ophthalmological conditions. &Rated 10/10 for relevance and accuracy, considered highly accurate. \\ \hline
Ketotifen eye drops & Latanoprost eye drops are used to lower intraocular pressure in glaucoma, while Ketotifen treats allergic conjunctivitis. & Latanoprost eye drops & Rated 9.0/10 for distinct yet complementary roles in ophthalmology. & Effective in identifying separate ophthalmic applications with 92\% accuracy. & Rated 10/10 for relevance and accuracy, considered highly accurate.\\ \hline
Diltiazem & Nitroglycerin is relevant in discussions of cardiovascular treatments alongside Diltiazem. & Nitroglycerin & Rated 8.8/10 for contextual relevance to cardiovascular management. & Increases the accuracy for treatment-based queries by 20\%. & Rated 9.5/10 for relevance and accuracy, considered highly accurate.\\ \hline
Labetalol & Labetalol is closely related to Propranolol, both managing hypertension. & Propranolol & Rated 9.5/10 for direct relevance in cardiovascular treatment protocols. & Highly interpretable responses for hypertension management, with 95\% accuracy. &Rated 10/10 for relevance and accuracy, considered highly accurate. \\ \hline
Nitroglycerin & Nitroglycerin and Labetalol are often used in conjunction for managing hypertension and heart conditions. & Labetalol & Rated 8.7/10 for strong relevance in acute hypertension protocols. & Supported effective multi-drug therapy reasoning with 90\% accuracy. &Rated 9/10 for relevance and accuracy, considered highly accurate. \\ \hline
Nitroglycerin & Nitroglycerin is often used with Propranolol in managing cardiovascular conditions like hypertension and angina. & Propranolol & Rated 9.0/10 for its importance in cardiovascular multi-drug therapy. & Demonstrated robust performance in connecting treatment protocols, with 93\% query accuracy. &Rated 10/10 for relevance and accuracy, considered highly accurate. \\ \hline
Fluorometholone eye drops & Fluorometholone eye drops are corticosteroids that treat inflammation, complementing Ketotifen for allergic conjunctivitis. & Ketotifen eye drops & Rated 8.8/10 for their combined application in managing inflammation and allergies. & Improved query relevance for multi-drug therapy in eye care by 19\%. &Rated 9.5/10 for relevance and accuracy, considered highly accurate.\\ \hline
Lanolin & Lanolin is used for skin care, particularly for sore nipples during breastfeeding. & Fluorometholone eye drops & Rated 8.5/10 for highlighting non-overlapping yet clinically useful contexts. & Demonstrated effective differentiation of clinical uses with high interpretability. &Rated 9/10 for relevance and accuracy, considered highly accurate.\\ \hline
\end{tabular}}
\end{table*}

\section{QA Samples with reasoning from MEDQA benchmark}
This section presents a set of \gls{qa} samples demonstrating the reasoning paths generated by our proposed \gls{myrag} model when applied to the MEDQA dataset. These examples highlight how the model retrieves relevant content, structures key information, and formulates reasoning to guide answer selection.

Table~\ref{table:search_guidance_1} provides an example of how the model processes a clinical case question related to the management of acute coronary syndrome (ACS). The search items retrieved for possible answer choices (e.g., Nifedipine, Enoxaparin, Clopidogrel, Spironolactone, Propranolol) are accompanied by key content excerpts relevant to their roles in ACS treatment. Additionally, the reasoning pathways illustrate how the model synthesizes evidence-based knowledge to justify the selection of the correct answer (Clopidogrel), while also explaining why the alternative options are not suitable. Additional examples are also provided in Tables ~\ref{table:search_guidance_2}, ~\ref{table:search_guidance_3}, and ~\ref{table:search_guidance_4}
\label{app:examples-medqa}
\begin{table*}[h!]
\centering
\begin{tabular}{|p{3cm}|p{4cm}|p{4cm}|}
\hline
\textbf{Search Item/ Question Options} & \textbf{Key Content Highlighted} & \textbf{Reasoning Guiding the Answer} \\ \hline
\textbf{Nifedipine} &
\textcolor{red}{Not typically used for acute coronary syndrome (ACS). Associated with reflex tachycardia.} &
Nifedipine is a calcium channel blocker effective for hypertension but does not address the antiplatelet needs of ACS patients. \\ \hline
\textbf{Enoxaparin} &
\textcolor{blue}{Used for anticoagulation in ACS but mainly during hospitalization.} &
Enoxaparin is not continued after discharge when aspirin and another antiplatelet drug are prescribed. \\ \hline
\textbf{Clopidogrel} &
\textcolor{green}{Standard for dual antiplatelet therapy (DAPT) in ACS, especially post-percutaneous coronary intervention (PCI).} &
Clopidogrel complements aspirin in preventing thrombotic events post-angioplasty. Its use is supported by evidence-based guidelines. \\ \hline
\textbf{Spironolactone} &
\textcolor{orange}{Useful in heart failure or reduced ejection fraction but not indicated for ACS management when EF is normal.} &
This patient's EF is 58\%, so spironolactone is not necessary. Focus should be on antiplatelet therapy. \\ \hline
\textbf{Propranolol} &
\textcolor{purple}{Effective for reducing myocardial oxygen demand but not part of standard DAPT.} &
While beneficial for stress-related heart issues, it does not address thrombotic risks in ACS management. \\ \hline
\end{tabular}
\caption{Examples of Summary of search items for the question "A 65-year-old man is brought to the emergency department 30 minutes after the onset of acute chest pain. He has hypertension and asthma. Current medications include atorvastatin, lisinopril, and an albuterol inhaler. He appears pale and diaphoretic. His pulse is 114/min, and blood pressure is 130/88 mm Hg. An ECG shows ST-segment depressions in leads II, III, and aVF. Laboratory studies show an increased serum troponin T concentration. The patient is treated for acute coronary syndrome and undergoes percutaneous transluminal coronary angioplasty. At the time of discharge, echocardiography shows a left ventricular ejection fraction of 58\%. In addition to aspirin, which of the following drugs should be added to this patient's medication regimen?" and Their Influence on the Correct Answer (Clopidogrel) and the reasoning paths}
\label{table:search_guidance_1}
\end{table*}


\begin{table*}[h!]
\centering
\begin{tabular}{|p{3cm}|p{4cm}|p{4cm}|}
\hline
\textbf{Search Item/ Question Options} & \textbf{Key Content Highlighted} & \textbf{Reasoning Guiding the Answer} \\ \hline

\textbf{A history of stroke or venous thromboembolism} &
\textcolor{red}{Contraindicated for hormonal contraceptives due to increased risk of thrombosis.} &
Copper IUDs do not carry the same thrombotic risk, making this option irrelevant for contraindication in IUD placement. \\ \hline

\textbf{Current tobacco use} &
\textcolor{blue}{Increases cardiovascular risk with hormonal contraceptives but not with copper IUDs.} &
Tobacco use does not contraindicate IUD placement, though it may influence other contraceptive choices. \\ \hline

\textbf{Active or recurrent pelvic inflammatory disease (PID)} &
\textcolor{green}{Direct contraindication for IUD placement due to the risk of exacerbating infection and complications.} &
Insertion of an IUD can worsen active PID, leading to infertility or other severe complications. \\ \hline

\textbf{Past medical history of breast cancer} &
\textcolor{orange}{Contraindicates hormonal contraceptives, but copper IUDs are considered safe.} &
This option does not contraindicate copper IUD placement, as it is non-hormonal and unrelated to breast cancer. \\ \hline

\textbf{Known liver neoplasm} &
\textcolor{purple}{Contraindicates hormonal contraceptives but not copper IUDs.} &
Copper IUDs are safe for patients with liver neoplasms as they are free of systemic hormones. \\ \hline

\end{tabular}
\caption{Examples of Summary of Search Items for the Question "A 37-year-old-woman presents to her primary care physician requesting a new form of birth control. She has been utilizing oral contraceptive pills (OCPs) for the past 8 years, but asks to switch to an intrauterine device (IUD). Her vital signs are: blood pressure 118/78 mm Hg, pulse 73/min and respiratory rate 16/min. She is afebrile. Physical examination is within normal limits. Which of the following past medical history statements would make copper IUD placement contraindicated in this patient?" and Their Influence on the Correct Answer (Active or recurrent pelvic inflammatory disease (PID)) and the Reasoning Paths}
\label{table:search_guidance_2}
\end{table*}



\begin{table*}[h!]
\centering
\begin{tabular}{|p{3cm}|p{4cm}|p{4cm}|}
\hline
\textbf{Search Item/ Question Options} & \textbf{Key Content Highlighted} & \textbf{Reasoning Guiding the Answer} \\ \hline

\textbf{Dementia} &
\textcolor{red}{Typically presents as a gradual decline in cognitive function.} &
The sudden onset of symptoms after surgery and acute confusion makes dementia less likely. \\ \hline

\textbf{Alcohol withdrawal} &
\textcolor{blue}{Requires significant and sustained alcohol use to cause withdrawal symptoms.} &
The patient's weekly consumption of one to two glasses of wine is insufficient to support this diagnosis. \\ \hline

\textbf{Opioid intoxication} &
\textcolor{purple}{Oxycodone can cause sedation and confusion, but stable vital signs and lack of severe respiratory depression are inconsistent.} &
While oxycodone use is relevant, the observed fluctuating agitation and impulsivity are more consistent with delirium. \\ \hline

\textbf{Delirium} &
\textcolor{green}{Characterized by acute changes in attention and cognition with fluctuating levels of consciousness.} &
The patient's recent surgery, medication use, and fluctuating symptoms align strongly with a diagnosis of delirium. \\ \hline

\textbf{Urinary tract infection (UTI)} &
\textcolor{orange}{Confusion in elderly patients can result from UTIs, but a normal urine dipstick test does not support this.} &
The absence of urinary findings on examination makes UTI less likely as the cause of symptoms. \\ \hline

\end{tabular}
\caption{Examples of Search Items for the Question: "Six days after undergoing surgical repair of a hip fracture, a 79-year-old woman presents with agitation and confusion. Which of the following is the most likely cause of her current condition?" and Their Influence on the Correct Answer (Delirium) and the Reasoning Paths.}
\label{table:search_guidance_3}
\end{table*}

\begin{table*}[h!]
\centering
\begin{tabular}{|p{3cm}|p{4cm}|p{4cm}|}
\hline
\textbf{Search Item/ Question Options} & \textbf{Key Content Highlighted} & \textbf{Reasoning Guiding the Answer} \\ \hline

\textbf{Primary spermatocyte} &
\textcolor{red}{Nondisjunction events during meiosis I often occur at this stage, leading to chromosomal abnormalities.} &
Klinefelter syndrome (47,XXY) is typically due to nondisjunction during meiosis, specifically at this stage. \\ \hline

\textbf{Secondary spermatocyte} &
\textcolor{blue}{Meiosis II occurs here, dividing chromosomes into haploid cells, but errors at this stage are less likely to lead to 47,XXY.} &
The chromosomal abnormality associated with Klinefelter syndrome usually arises before this stage. \\ \hline

\textbf{Spermatid} &
\textcolor{purple}{Spermatids are post-meiotic cells where genetic material is already finalized.} &
Errors at this stage would not result in a cytogenetic abnormality like 47,XXY. \\ \hline

\textbf{Spermatogonium} &
\textcolor{green}{Errors here affect the germline but are less likely to cause specific meiotic nondisjunction errors.} &
While germline mutations can occur, meiotic nondisjunction leading to Klinefelter syndrome occurs later. \\ \hline

\textbf{Spermatozoon} &
\textcolor{orange}{These are fully mature sperm cells that inherit abnormalities from earlier stages.} &
By this stage, chromosomal errors have already been established. \\ \hline

\end{tabular}
\caption{Examples of Search Items for the Question: "A 29-year-old man with infertility, tall stature, gynecomastia, small testes, and an elevated estradiol:testosterone ratio is evaluated. Genetic studies reveal a cytogenetic abnormality inherited from the father. At which stage of spermatogenesis did this error most likely occur?" and Their Influence on the Correct Answer (Primary spermatocyte) and the Reasoning Paths.}
\label{table:search_guidance_4}
\end{table*}


% \section{Supervised Fine-Tuning for Medical QA}

% To enhance the reasoning and decision-making capabilities of our \gls{llm}-based medical \gls{qa} system, we employ supervised fine-tuning on domain-specific data. This process aims to improve the model's ability to generate more precise responses for medical queries.

% The fine-tuning process involves training a pre-trained \gls{llm} on the MedQA dataset, which contains multiple-choice medical questions derived from professional medical board exams \cite{jin2021disease}. We use a supervised fine-tuning (SFT) approach \cite{schulhoff2024prompt}, where the model is trained on question-answer pairs to improve its response accuracy in medical contexts. To optimize efficiency, we employ Low-Rank Adaptation (LoRA), a parameter-efficient tuning method that reduces computational cost by introducing low-rank updates to pre-trained model weights while maintaining performance \cite{hu2021lora}.

% The training process fine-tunes the \gls{llm} using gradient-based optimization, employing loss functions tailored for answer accuracy. A validation set of medical cases is used to monitor model performance and prevent overfitting.

% \subsection{Integration into the AMG-RAG Pipeline}

% The fine-tuned \gls{llm} is integrated at the final step of the \gls{myrag} pipeline, where it synthesizes retrieved evidence, reasoning traces, and structured knowledge from the \gls{mkg}. Instead of relying solely on a generic \gls{llm}, the system now leverages a domain-adapted model, ensuring that generated responses are more specialized for medical question answering. This step enhances:

% \begin{itemize}
%     \item \textbf{Improved Answer Precision:} The model generates responses that are better aligned with domain-specific medical knowledge.
%     \item \textbf{Computational Efficiency:} LoRA fine-tuning reduces memory and processing requirements compared to full model fine-tuning.
%     \item \textbf{Adaptability:} The fine-tuned model can better handle nuanced medical questions derived from standardized exams.
% \end{itemize}

% By integrating fine-tuned \gls{llm}s into our pipeline, we enhance the model’s capability to support medical question answering tasks, ensuring more reliable and relevant outputs.


% \section{Blind Testing with Clinicians}
% To objectively evaluate the effectiveness of the \gls{myrag} model, we conducted a blind testing study involving three independent clinicians with expertise in medical practice. This evaluation aimed to assess the model's ability to provide accurate, relevant, and interpretable answers to complex medical questions. Questions were selected from the MEDQA and MedMCQA datasets, covering both foundational knowledge and clinical reasoning scenarios. 

% % \textcolor{red}{need to be explained more later when we finalize type of questions that we will ask from clinicals}

% Each question was paired with anonymized answers generated by \gls{myrag} and baseline models, including GPT-4o mini. The sources of the answers were not revealed to the clinicians, ensuring an unbiased evaluation. Following a similar methodology to prior studies \cite{miller2011investigation, si2024interpretabnet}, the answers were presented in randomized order for each question. Clinicians were asked to rate the responses on four key metrics: accuracy, relevance, interpretability, and clinical applicability. Ratings were recorded on a five-point Likert scale \cite{sullivan2013analyzing} to allow for quantitative analysis of the results.

% To ensure the reliability of the evaluation, inter-rater agreement among the clinicians was measured using Cohen’s kappa \cite{mchugh2012interrater}. This statistical measure provided insight into the consistency of the ratings across different evaluators. The results of the blind test are summarized in Table~\ref{tab:blind_test_results}.

% % \begin{table}[h!]
% % \centering
% % \caption{Results of the Blind Testing with Clinicians}
% % \label{tab:blind_test_results}
% % \begin{tabular}{|l|c|c|c|c|}
% % \hline
% % \textbf{Model}    & \textbf{Accuracy} & \textbf{Relevance} & \textbf{Interpretability} & \textbf{Clinical Applicability} \\
% % \hline
% % \gls{myrag}       & X               & X                & X                       & X                           \\
% % GPT-4             & X               & X                & X                       & X                           \\
% % Med-Gemini        & X               & X                & X                       & X                           \\
% % \hline
% % \end{tabular}
% % \end{table}

% This study provides strong evidence of the practical utility of \gls{myrag} in medical question-answering tasks.

\section{Implementation Details for Dataset Ingestion and Vector Database}
\label{app:vector_db}

This section outlines the pipeline for dataset ingestion and vector database creation for efficient medical question-answering. The process involves document chunking, embedding generation, and storage in a vector database to facilitate semantic retrieval.

\subsection{Dataset Processing and Chunking}
The dataset, sourced from medical textbooks in the MEDQA benchmark, is provided in plain text format. Each document is segmented into smaller chunks with a maximum size of 512 tokens and a 100-token overlap. This overlap ensures context preservation across chunk boundaries, supporting multi-hop reasoning for long documents.

\subsection{Embedding Model and Vector Storage}
The system utilizes the \textbf{SentenceTransformer} model, specifically \texttt{all-mpnet-base-v2}, for generating dense vector representations of text chunks and queries. To optimize storage and retrieval, the embeddings are indexed in the \textbf{Chroma} vector database. Metadata, such as document filenames and chunk IDs, is also stored to maintain document traceability.

\subsection{Batch Processing and Vector Database Population}
To manage memory efficiently during ingestion, document chunks are processed in batches of up to 10,000. This ensures a smooth ingestion pipeline while preventing memory overflow. Each processed file is logged to avoid redundant computations, and error handling mechanisms are in place to manage failed processing attempts.

\subsection{Query Answering Workflow}
For retrieval, user queries (e.g., \textit{"What are the symptoms of drug-induced diabetes?"}) are embedded using the \texttt{all-mpnet-base-v2} model. The top-ranked relevant chunks are retrieved based on their semantic similarity to the query using Chroma's similarity search mechanism. The system retrieves the top \(k\) relevant passages, which can be further processed in downstream QA models.

\subsection{Key Configuration Details}
The system is configured with the following parameters:
\begin{itemize}
    \item \textbf{Embedding Model:} \texttt{all-mpnet-base-v2} from SentenceTransformer.
    \item \textbf{Vector Database:} Chroma, stored persistently on disk for reusability.
    \item \textbf{Chunk Size:} 512 tokens per chunk, with a 100-token overlap for contextual consistency.
    \item \textbf{Batch Size:} Up to 10,000 chunks per batch to optimize ingestion efficiency.
\end{itemize}

\subsection{Implementation and System Execution}
The ingestion and query process is implemented using Python, leveraging \texttt{sentence-transformers} for embeddings and \texttt{Chroma} for vector storage. The ingestion pipeline reads and processes text files, splits them into chunks, generates embeddings, and stores them efficiently in the vector database. The querying process retrieves the top \(k\) most relevant text chunks to respond to user queries.


\section{Components Definition }
\subsection{Neo4j}


As data complexity increases, traditional relational databases struggle with highly interconnected datasets where relationships are crucial. Graph databases, like Neo4j, address this challenge by efficiently modeling and processing complex, evolving data structures using nodes, relationships, and properties \cite{besta2023demystifying}. 

Neo4j, an open-source NoSQL graph database, enables constant-time traversals by explicitly storing relationships, making it ideal for large-scale applications such as social networks, recommendation systems, and biomedical research. Unlike relational models, Neo4j avoids costly table joins and optimizes deep relationship queries, enhancing scalability and performance \cite{besta2023demystifying}. 

Neo4j's architecture is centered around the property graph model, which includes\cite{huang2013research}:

\begin{itemize}
    \item \textbf{Nodes}: Entities representing data points.
    \item \textbf{Relationships}: Directed, named connections between nodes that define how entities are related.
    \item \textbf{Properties}: Key-value pairs associated with both nodes and relationships, providing additional metadata.
\end{itemize}

This model allows for intuitive representation of complex data structures and supports efficient querying and analysis. The system's internal mechanisms facilitate rapid traversal of relationships, enabling swift query responses even in large datasets \cite{huang2013research}.

