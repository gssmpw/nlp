\section{Related Work}

Medical \gls{qa} systems are essential for enhancing clinical decision-making, research, and patient care \cite{nazi2024large,liu2023utility,rezaei2024rag}. Over time, the field has evolved with various technological advancements addressing key challenges in processing medical information \cite{singhal2022large}. Domain-specific language models such as BioBERT \cite{lee2020biobert}, PubMedBERT \cite{gu2021domain}, and MedPaLM \cite{singhal2023large} have achieved significant success in biomedical tasks \cite{rohanian2024exploring}. However, these models often struggle with synthesizing complex relationships between medical entities and integrating data from diverse sources, particularly for rare conditions, drug interactions, and comorbidities \cite{zhou2023survey,yu2024large}.

To overcome these challenges, \gls{rag} frameworks \cite{rezaei2025vendirag,lewis2020retrieval} have enhanced \glspl{llm} by integrating external knowledge sources. Systems like MMED-RAG \cite{xia2024mmed} have extended this paradigm to include multimodal data. The introduction of \gls{cot} reasoning has further improved \gls{qa} performance, with IRCoT \cite{trivedi2022interleaving} combining \gls{cot} reasoning with \gls{rag} for more sophisticated inference. Recent advancements, such as Gemini's multimodal and long-context reasoning capabilities, have set new benchmarks in MedQA, surpassing GPT-4 in performance \cite{saab2024capabilities}. However, these systems often struggle to adapt to novel queries and dynamic data due to their rigid architectures.

\gls{kg}-based approaches provide another avenue for advancing medical information processing. Systems like KG-Rank \cite{huang2021knowledge} utilize structured knowledge representations and ontologies to enable hierarchical reasoning and inference. By combining knowledge graphs with ranking and re-ranking techniques, these systems enhance the factual accuracy of long-form \gls{qa} \cite{yang2024kg}. However, \gls{kg}-based systems face significant challenges in maintaining scalability and staying current with rapidly evolving biomedical discoveries.

\subsection*{Difference and Importance of AMG-RAG}

Our \gls{myrag} dynamically constructs relational medical \glspl{kg} integrated with advanced search capabilities. Unlike traditional static systems, our approach extracts medical terms from queries, enriches them with real-time data, and utilizes \glspl{llm} to infer relationships. This dynamic mechanism ensures continuous alignment with emerging medical knowledge, addressing the limitations of static knowledge bases and pre-trained models. By combining dynamic \glspl{kg} with \gls{cot} reasoning and \gls{rag}, our framework improves the adaptability and reliability of medical \gls{qa} systems.