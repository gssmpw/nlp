\section{Conclusion}
In this work, we introduce \gls{myrag}, an advanced \gls{qa} system that dynamically constructs \gls{mkg} while integrating sophisticated reasoning and external domain-specific search tools. The model exhibits significant improvements in accuracy and reasoning capabilities, particularly for medical question-answering tasks, outperforming other approaches of similar model size or 10 to 100 times larger. Using structured knowledge representations and advanced reasoning frameworks, our approach establishes a new benchmark for \gls{qa} in highly competitive and highly evolving domains such as medicine.

\section{Limitations}

Despite \gls{myrag} advancements, our approach has certain limitations. Firstly it relies on external search tools to introduce latency during the creation of \gls{mkg}. However, this occurs only once, when the \gls{mkg} is built from scratch for the first time. Additionally, while the model performs exceptionally well in medical domains, its applicability to non-medical tasks remains unexplored. 


Another limitation is the need for structured, authoritative sources of medical knowledge. Currently, \gls{myrag} retrieves information from diverse sources, including research articles and medical textbooks. However, as emphasized in clinical decision-making, treatment guidelines serve as essential references for standardized diagnosis and treatment protocols \cite{hager2024evaluation}. Future work on \gls{myrag} should focus on integrating structured access to these sources to ensure compliance with evidence-based medicine.


\section{Ethics Statement}
The development of \gls{llm}s for medical \gls{qa} requires careful ethical consideration due to risks of inaccuracy and bias. Ensuring the reliability of retrieved content is crucial, especially when integrating external knowledge sources. To mitigate these risks, we implement a confidence scoring mechanism into the \gls{mkg} to validate the information. However, bias detection and mitigation remain active research areas.
