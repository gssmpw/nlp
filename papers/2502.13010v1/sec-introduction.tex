\glsresetall
\section{Introduction}
Medical knowledge expands at a tremendous pace, with new research findings, clinical guidelines, and treatment protocols emerging constantly. \glspl{llm} have already demonstrated their value in harnessing this vast and evolving information for medical question answering by processing large corpora of domain-specific literature and data \cite{nazi2024large,liu2023utility}. 
\begin{figure}[t]
  \centering
  \includegraphics[width=.9\columnwidth]{latex/Figures/performance_vs_size.png}
  \caption{Performance vs parameter numbers for medical \gls{qa} on the MEDQA and MEDMCQA datasets. \gls{myrag} achieves an F1 score of 74.1\% on MEDQA and an accuracy of 66.34\% on MEDMCQA, surpassing both comparable models and those that are 10 to 100 times larger in size. See Tables \ref{tab:medqa_comparison} and \ref{tab:medmcqa_comparison} for more details.}
  \label{fig:performance_vs_size}
\end{figure}
However, a major challenge lies in ensuring that these models remain factually current and can accurately represent complex relationships among medical concepts \cite{rohanian2024exploring,yu2024large}. Traditional approaches to mitigating these issues involve the use of knowledge graphs, which offer structured, interconnected representations of medical information and can support more nuanced reasoning \cite{huang2021knowledge}. Yet, constructing and maintaining such graphs is labor-intensive, time-consuming, and expensive. These burdens are particularly acute in a domain as dynamic as medicine, where new insights quickly render old information out-of-date \cite{yang2024kg}.
To address this pressing issue, we propose an automated framework for constructing and continuously evolving \glspl{kg} specifically tailored to medical question answering. By leveraging \glspl{llm} agents and domain-specific search tools, our method autonomously generates graph \glspl{mkg}, enriched with descriptive metadata, confidence scores, and relevance indicators. In doing so, it drastically reduces the manual effort traditionally required to build and update knowledge graphs, while ensuring alignment with the latest medical advances. Unlike traditional \gls{rag} solutions that rely heavily on vector similarity for retrieval \cite{lewis2020retrieval}, our knowledge-graph-based approach provides more sophisticated reasoning capabilities through shared attributes and explicit relationships. This facilitates accurate synthesis of information across diverse medical domains, ranging from drug interactions and clinical trial data to patient histories and treatment guidelines.

A central component of our solution is the integration of these evolving knowledge graphs into a \gls{rag}-based pipeline. New and updated graph entities are continuously fed into an \gls{llm} question answering module, ensuring that responses draw upon the most up-to-date and contextually relevant medical information \cite{singhal2022large}. Building on this dynamic architecture, we introduce an iterative pipeline, \gls{myrag}, which combines insights from these automatically maintained graphs with traditional textual retrieval and multi-step chain-of-thought reasoning. By optimizing retrieval through confidence scoring and adaptive graph traversal, \gls{myrag} demonstrates significantly improved accuracy and completeness in medical \gls{qa} \cite{trivedi2022interleaving}.

We assess \gls{myrag} on the MEDQA and MEDMCQA benchmarks, which are designed to test evidence retrieval, complex reasoning, and multi-choice comprehension in the medical domain. Our model achieves an F1 score of 74.1\% on MEDQA and an accuracy of 66.34\% on MEDMCQA, outperforming both similarly sized \gls{rag} approaches and much larger state-of-the-art models (Fig.~\ref{fig:performance_vs_size}). Crucially, these gains do not necessitate additional fine-tuning or higher inference costs; instead, they result from seamlessly integrating knowledge graphs and domain-specific search tools. This efficient and scalable approach underscores the value of dynamically evolving knowledge retrieval in medical \gls{qa}, offering an avenue for enhancing clinical decision-making by delivering reliable, relationally enriched insights \cite{zhou2023survey}.