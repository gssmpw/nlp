% This must be in the first 5 lines to tell arXiv to use pdfLaTeX, which is strongly recommended.
\pdfoutput=1
% In particular, the hyperref package requires pdfLaTeX in order to break URLs across lines.

\documentclass[11pt]{article}

% Change "review" to "final" to generate the final (sometimes called camera-ready) version.
% Change to "preprint" to generate a non-anonymous version with page numbers.
\usepackage[preprint]{acl}

% Standard package includes
\usepackage{times}
\usepackage{latexsym}
\usepackage{algpseudocode}
\usepackage{algorithm}
% For proper rendering and hyphenation of words containing Latin characters (including in bib files)
\usepackage[T1]{fontenc}
% For Vietnamese characters
% \usepackage[T5]{fontenc}
% See https://www.latex-project.org/help/documentation/encguide.pdf for other character sets

% This assumes your files are encoded as UTF8
\usepackage[utf8]{inputenc}

% This is not strictly necessary, and may be commented out,
% but it will improve the layout of the manuscript,
% and will typically save some space.
\usepackage{microtype}

% This is also not strictly necessary, and may be commented out.
% However, it will improve the aesthetics of text in
% the typewriter font.
\usepackage{inconsolata}

%Including images in your LaTeX document requires adding
%additional package(s)
\usepackage{graphicx}
\usepackage[acronym]{glossaries}

% If the title and author information does not fit in the area allocated, uncomment the following
%
%\setlength\titlebox{<dim>}
%
% and set <dim> to something 5cm or larger.
\usepackage{inconsolata}


\usepackage{multicol}
\usepackage{booktabs}
\usepackage{makecell}
\usepackage{amsmath, amssymb}
\usepackage{multirow}
\usepackage{mathtools}
\usepackage{xcolor}
\usepackage{enumitem}
\usepackage{float}
\usepackage{graphicx}
\usepackage{upgreek}
\usepackage{seqsplit}
\usepackage{color,soul}
%\usepackage{ulem}
\usepackage{arydshln}
\usepackage{placeins}
\usepackage{pifont}
\usepackage{amssymb}
\newcommand{\cmark}{\ding{51}}%
\newcommand{\xmark}{\ding{55}}%
\usepackage{bbm}


\newcommand{\rsf}[1]{\textcolor{magenta}{#1}}


%%%%% NEW MATH DEFINITIONS %%%%%

\usepackage{amsmath,amsfonts,bm}
\usepackage{derivative}
% Mark sections of captions for referring to divisions of figures
\newcommand{\figleft}{{\em (Left)}}
\newcommand{\figcenter}{{\em (Center)}}
\newcommand{\figright}{{\em (Right)}}
\newcommand{\figtop}{{\em (Top)}}
\newcommand{\figbottom}{{\em (Bottom)}}
\newcommand{\captiona}{{\em (a)}}
\newcommand{\captionb}{{\em (b)}}
\newcommand{\captionc}{{\em (c)}}
\newcommand{\captiond}{{\em (d)}}

% Highlight a newly defined term
\newcommand{\newterm}[1]{{\bf #1}}

% Derivative d 
\newcommand{\deriv}{{\mathrm{d}}}

% Figure reference, lower-case.
\def\figref#1{figure~\ref{#1}}
% Figure reference, capital. For start of sentence
\def\Figref#1{Figure~\ref{#1}}
\def\twofigref#1#2{figures \ref{#1} and \ref{#2}}
\def\quadfigref#1#2#3#4{figures \ref{#1}, \ref{#2}, \ref{#3} and \ref{#4}}
% Section reference, lower-case.
\def\secref#1{section~\ref{#1}}
% Section reference, capital.
\def\Secref#1{Section~\ref{#1}}
% Reference to two sections.
\def\twosecrefs#1#2{sections \ref{#1} and \ref{#2}}
% Reference to three sections.
\def\secrefs#1#2#3{sections \ref{#1}, \ref{#2} and \ref{#3}}
% Reference to an equation, lower-case.
\def\eqref#1{equation~\ref{#1}}
% Reference to an equation, upper case
\def\Eqref#1{Equation~\ref{#1}}
% A raw reference to an equation---avoid using if possible
\def\plaineqref#1{\ref{#1}}
% Reference to a chapter, lower-case.
\def\chapref#1{chapter~\ref{#1}}
% Reference to an equation, upper case.
\def\Chapref#1{Chapter~\ref{#1}}
% Reference to a range of chapters
\def\rangechapref#1#2{chapters\ref{#1}--\ref{#2}}
% Reference to an algorithm, lower-case.
\def\algref#1{algorithm~\ref{#1}}
% Reference to an algorithm, upper case.
\def\Algref#1{Algorithm~\ref{#1}}
\def\twoalgref#1#2{algorithms \ref{#1} and \ref{#2}}
\def\Twoalgref#1#2{Algorithms \ref{#1} and \ref{#2}}
% Reference to a part, lower case
\def\partref#1{part~\ref{#1}}
% Reference to a part, upper case
\def\Partref#1{Part~\ref{#1}}
\def\twopartref#1#2{parts \ref{#1} and \ref{#2}}

\def\ceil#1{\lceil #1 \rceil}
\def\floor#1{\lfloor #1 \rfloor}
\def\1{\bm{1}}
\newcommand{\train}{\mathcal{D}}
\newcommand{\valid}{\mathcal{D_{\mathrm{valid}}}}
\newcommand{\test}{\mathcal{D_{\mathrm{test}}}}

\def\eps{{\epsilon}}


% Random variables
\def\reta{{\textnormal{$\eta$}}}
\def\ra{{\textnormal{a}}}
\def\rb{{\textnormal{b}}}
\def\rc{{\textnormal{c}}}
\def\rd{{\textnormal{d}}}
\def\re{{\textnormal{e}}}
\def\rf{{\textnormal{f}}}
\def\rg{{\textnormal{g}}}
\def\rh{{\textnormal{h}}}
\def\ri{{\textnormal{i}}}
\def\rj{{\textnormal{j}}}
\def\rk{{\textnormal{k}}}
\def\rl{{\textnormal{l}}}
% rm is already a command, just don't name any random variables m
\def\rn{{\textnormal{n}}}
\def\ro{{\textnormal{o}}}
\def\rp{{\textnormal{p}}}
\def\rq{{\textnormal{q}}}
\def\rr{{\textnormal{r}}}
\def\rs{{\textnormal{s}}}
\def\rt{{\textnormal{t}}}
\def\ru{{\textnormal{u}}}
\def\rv{{\textnormal{v}}}
\def\rw{{\textnormal{w}}}
\def\rx{{\textnormal{x}}}
\def\ry{{\textnormal{y}}}
\def\rz{{\textnormal{z}}}

% Random vectors
\def\rvepsilon{{\mathbf{\epsilon}}}
\def\rvphi{{\mathbf{\phi}}}
\def\rvtheta{{\mathbf{\theta}}}
\def\rva{{\mathbf{a}}}
\def\rvb{{\mathbf{b}}}
\def\rvc{{\mathbf{c}}}
\def\rvd{{\mathbf{d}}}
\def\rve{{\mathbf{e}}}
\def\rvf{{\mathbf{f}}}
\def\rvg{{\mathbf{g}}}
\def\rvh{{\mathbf{h}}}
\def\rvu{{\mathbf{i}}}
\def\rvj{{\mathbf{j}}}
\def\rvk{{\mathbf{k}}}
\def\rvl{{\mathbf{l}}}
\def\rvm{{\mathbf{m}}}
\def\rvn{{\mathbf{n}}}
\def\rvo{{\mathbf{o}}}
\def\rvp{{\mathbf{p}}}
\def\rvq{{\mathbf{q}}}
\def\rvr{{\mathbf{r}}}
\def\rvs{{\mathbf{s}}}
\def\rvt{{\mathbf{t}}}
\def\rvu{{\mathbf{u}}}
\def\rvv{{\mathbf{v}}}
\def\rvw{{\mathbf{w}}}
\def\rvx{{\mathbf{x}}}
\def\rvy{{\mathbf{y}}}
\def\rvz{{\mathbf{z}}}

% Elements of random vectors
\def\erva{{\textnormal{a}}}
\def\ervb{{\textnormal{b}}}
\def\ervc{{\textnormal{c}}}
\def\ervd{{\textnormal{d}}}
\def\erve{{\textnormal{e}}}
\def\ervf{{\textnormal{f}}}
\def\ervg{{\textnormal{g}}}
\def\ervh{{\textnormal{h}}}
\def\ervi{{\textnormal{i}}}
\def\ervj{{\textnormal{j}}}
\def\ervk{{\textnormal{k}}}
\def\ervl{{\textnormal{l}}}
\def\ervm{{\textnormal{m}}}
\def\ervn{{\textnormal{n}}}
\def\ervo{{\textnormal{o}}}
\def\ervp{{\textnormal{p}}}
\def\ervq{{\textnormal{q}}}
\def\ervr{{\textnormal{r}}}
\def\ervs{{\textnormal{s}}}
\def\ervt{{\textnormal{t}}}
\def\ervu{{\textnormal{u}}}
\def\ervv{{\textnormal{v}}}
\def\ervw{{\textnormal{w}}}
\def\ervx{{\textnormal{x}}}
\def\ervy{{\textnormal{y}}}
\def\ervz{{\textnormal{z}}}

% Random matrices
\def\rmA{{\mathbf{A}}}
\def\rmB{{\mathbf{B}}}
\def\rmC{{\mathbf{C}}}
\def\rmD{{\mathbf{D}}}
\def\rmE{{\mathbf{E}}}
\def\rmF{{\mathbf{F}}}
\def\rmG{{\mathbf{G}}}
\def\rmH{{\mathbf{H}}}
\def\rmI{{\mathbf{I}}}
\def\rmJ{{\mathbf{J}}}
\def\rmK{{\mathbf{K}}}
\def\rmL{{\mathbf{L}}}
\def\rmM{{\mathbf{M}}}
\def\rmN{{\mathbf{N}}}
\def\rmO{{\mathbf{O}}}
\def\rmP{{\mathbf{P}}}
\def\rmQ{{\mathbf{Q}}}
\def\rmR{{\mathbf{R}}}
\def\rmS{{\mathbf{S}}}
\def\rmT{{\mathbf{T}}}
\def\rmU{{\mathbf{U}}}
\def\rmV{{\mathbf{V}}}
\def\rmW{{\mathbf{W}}}
\def\rmX{{\mathbf{X}}}
\def\rmY{{\mathbf{Y}}}
\def\rmZ{{\mathbf{Z}}}

% Elements of random matrices
\def\ermA{{\textnormal{A}}}
\def\ermB{{\textnormal{B}}}
\def\ermC{{\textnormal{C}}}
\def\ermD{{\textnormal{D}}}
\def\ermE{{\textnormal{E}}}
\def\ermF{{\textnormal{F}}}
\def\ermG{{\textnormal{G}}}
\def\ermH{{\textnormal{H}}}
\def\ermI{{\textnormal{I}}}
\def\ermJ{{\textnormal{J}}}
\def\ermK{{\textnormal{K}}}
\def\ermL{{\textnormal{L}}}
\def\ermM{{\textnormal{M}}}
\def\ermN{{\textnormal{N}}}
\def\ermO{{\textnormal{O}}}
\def\ermP{{\textnormal{P}}}
\def\ermQ{{\textnormal{Q}}}
\def\ermR{{\textnormal{R}}}
\def\ermS{{\textnormal{S}}}
\def\ermT{{\textnormal{T}}}
\def\ermU{{\textnormal{U}}}
\def\ermV{{\textnormal{V}}}
\def\ermW{{\textnormal{W}}}
\def\ermX{{\textnormal{X}}}
\def\ermY{{\textnormal{Y}}}
\def\ermZ{{\textnormal{Z}}}

% Vectors
\def\vzero{{\bm{0}}}
\def\vone{{\bm{1}}}
\def\vmu{{\bm{\mu}}}
\def\vtheta{{\bm{\theta}}}
\def\vphi{{\bm{\phi}}}
\def\va{{\bm{a}}}
\def\vb{{\bm{b}}}
\def\vc{{\bm{c}}}
\def\vd{{\bm{d}}}
\def\ve{{\bm{e}}}
\def\vf{{\bm{f}}}
\def\vg{{\bm{g}}}
\def\vh{{\bm{h}}}
\def\vi{{\bm{i}}}
\def\vj{{\bm{j}}}
\def\vk{{\bm{k}}}
\def\vl{{\bm{l}}}
\def\vm{{\bm{m}}}
\def\vn{{\bm{n}}}
\def\vo{{\bm{o}}}
\def\vp{{\bm{p}}}
\def\vq{{\bm{q}}}
\def\vr{{\bm{r}}}
\def\vs{{\bm{s}}}
\def\vt{{\bm{t}}}
\def\vu{{\bm{u}}}
\def\vv{{\bm{v}}}
\def\vw{{\bm{w}}}
\def\vx{{\bm{x}}}
\def\vy{{\bm{y}}}
\def\vz{{\bm{z}}}

% Elements of vectors
\def\evalpha{{\alpha}}
\def\evbeta{{\beta}}
\def\evepsilon{{\epsilon}}
\def\evlambda{{\lambda}}
\def\evomega{{\omega}}
\def\evmu{{\mu}}
\def\evpsi{{\psi}}
\def\evsigma{{\sigma}}
\def\evtheta{{\theta}}
\def\eva{{a}}
\def\evb{{b}}
\def\evc{{c}}
\def\evd{{d}}
\def\eve{{e}}
\def\evf{{f}}
\def\evg{{g}}
\def\evh{{h}}
\def\evi{{i}}
\def\evj{{j}}
\def\evk{{k}}
\def\evl{{l}}
\def\evm{{m}}
\def\evn{{n}}
\def\evo{{o}}
\def\evp{{p}}
\def\evq{{q}}
\def\evr{{r}}
\def\evs{{s}}
\def\evt{{t}}
\def\evu{{u}}
\def\evv{{v}}
\def\evw{{w}}
\def\evx{{x}}
\def\evy{{y}}
\def\evz{{z}}

% Matrix
\def\mA{{\bm{A}}}
\def\mB{{\bm{B}}}
\def\mC{{\bm{C}}}
\def\mD{{\bm{D}}}
\def\mE{{\bm{E}}}
\def\mF{{\bm{F}}}
\def\mG{{\bm{G}}}
\def\mH{{\bm{H}}}
\def\mI{{\bm{I}}}
\def\mJ{{\bm{J}}}
\def\mK{{\bm{K}}}
\def\mL{{\bm{L}}}
\def\mM{{\bm{M}}}
\def\mN{{\bm{N}}}
\def\mO{{\bm{O}}}
\def\mP{{\bm{P}}}
\def\mQ{{\bm{Q}}}
\def\mR{{\bm{R}}}
\def\mS{{\bm{S}}}
\def\mT{{\bm{T}}}
\def\mU{{\bm{U}}}
\def\mV{{\bm{V}}}
\def\mW{{\bm{W}}}
\def\mX{{\bm{X}}}
\def\mY{{\bm{Y}}}
\def\mZ{{\bm{Z}}}
\def\mBeta{{\bm{\beta}}}
\def\mPhi{{\bm{\Phi}}}
\def\mLambda{{\bm{\Lambda}}}
\def\mSigma{{\bm{\Sigma}}}

% Tensor
\DeclareMathAlphabet{\mathsfit}{\encodingdefault}{\sfdefault}{m}{sl}
\SetMathAlphabet{\mathsfit}{bold}{\encodingdefault}{\sfdefault}{bx}{n}
\newcommand{\tens}[1]{\bm{\mathsfit{#1}}}
\def\tA{{\tens{A}}}
\def\tB{{\tens{B}}}
\def\tC{{\tens{C}}}
\def\tD{{\tens{D}}}
\def\tE{{\tens{E}}}
\def\tF{{\tens{F}}}
\def\tG{{\tens{G}}}
\def\tH{{\tens{H}}}
\def\tI{{\tens{I}}}
\def\tJ{{\tens{J}}}
\def\tK{{\tens{K}}}
\def\tL{{\tens{L}}}
\def\tM{{\tens{M}}}
\def\tN{{\tens{N}}}
\def\tO{{\tens{O}}}
\def\tP{{\tens{P}}}
\def\tQ{{\tens{Q}}}
\def\tR{{\tens{R}}}
\def\tS{{\tens{S}}}
\def\tT{{\tens{T}}}
\def\tU{{\tens{U}}}
\def\tV{{\tens{V}}}
\def\tW{{\tens{W}}}
\def\tX{{\tens{X}}}
\def\tY{{\tens{Y}}}
\def\tZ{{\tens{Z}}}


% Graph
\def\gA{{\mathcal{A}}}
\def\gB{{\mathcal{B}}}
\def\gC{{\mathcal{C}}}
\def\gD{{\mathcal{D}}}
\def\gE{{\mathcal{E}}}
\def\gF{{\mathcal{F}}}
\def\gG{{\mathcal{G}}}
\def\gH{{\mathcal{H}}}
\def\gI{{\mathcal{I}}}
\def\gJ{{\mathcal{J}}}
\def\gK{{\mathcal{K}}}
\def\gL{{\mathcal{L}}}
\def\gM{{\mathcal{M}}}
\def\gN{{\mathcal{N}}}
\def\gO{{\mathcal{O}}}
\def\gP{{\mathcal{P}}}
\def\gQ{{\mathcal{Q}}}
\def\gR{{\mathcal{R}}}
\def\gS{{\mathcal{S}}}
\def\gT{{\mathcal{T}}}
\def\gU{{\mathcal{U}}}
\def\gV{{\mathcal{V}}}
\def\gW{{\mathcal{W}}}
\def\gX{{\mathcal{X}}}
\def\gY{{\mathcal{Y}}}
\def\gZ{{\mathcal{Z}}}

% Sets
\def\sA{{\mathbb{A}}}
\def\sB{{\mathbb{B}}}
\def\sC{{\mathbb{C}}}
\def\sD{{\mathbb{D}}}
% Don't use a set called E, because this would be the same as our symbol
% for expectation.
\def\sF{{\mathbb{F}}}
\def\sG{{\mathbb{G}}}
\def\sH{{\mathbb{H}}}
\def\sI{{\mathbb{I}}}
\def\sJ{{\mathbb{J}}}
\def\sK{{\mathbb{K}}}
\def\sL{{\mathbb{L}}}
\def\sM{{\mathbb{M}}}
\def\sN{{\mathbb{N}}}
\def\sO{{\mathbb{O}}}
\def\sP{{\mathbb{P}}}
\def\sQ{{\mathbb{Q}}}
\def\sR{{\mathbb{R}}}
\def\sS{{\mathbb{S}}}
\def\sT{{\mathbb{T}}}
\def\sU{{\mathbb{U}}}
\def\sV{{\mathbb{V}}}
\def\sW{{\mathbb{W}}}
\def\sX{{\mathbb{X}}}
\def\sY{{\mathbb{Y}}}
\def\sZ{{\mathbb{Z}}}

% Entries of a matrix
\def\emLambda{{\Lambda}}
\def\emA{{A}}
\def\emB{{B}}
\def\emC{{C}}
\def\emD{{D}}
\def\emE{{E}}
\def\emF{{F}}
\def\emG{{G}}
\def\emH{{H}}
\def\emI{{I}}
\def\emJ{{J}}
\def\emK{{K}}
\def\emL{{L}}
\def\emM{{M}}
\def\emN{{N}}
\def\emO{{O}}
\def\emP{{P}}
\def\emQ{{Q}}
\def\emR{{R}}
\def\emS{{S}}
\def\emT{{T}}
\def\emU{{U}}
\def\emV{{V}}
\def\emW{{W}}
\def\emX{{X}}
\def\emY{{Y}}
\def\emZ{{Z}}
\def\emSigma{{\Sigma}}

% entries of a tensor
% Same font as tensor, without \bm wrapper
\newcommand{\etens}[1]{\mathsfit{#1}}
\def\etLambda{{\etens{\Lambda}}}
\def\etA{{\etens{A}}}
\def\etB{{\etens{B}}}
\def\etC{{\etens{C}}}
\def\etD{{\etens{D}}}
\def\etE{{\etens{E}}}
\def\etF{{\etens{F}}}
\def\etG{{\etens{G}}}
\def\etH{{\etens{H}}}
\def\etI{{\etens{I}}}
\def\etJ{{\etens{J}}}
\def\etK{{\etens{K}}}
\def\etL{{\etens{L}}}
\def\etM{{\etens{M}}}
\def\etN{{\etens{N}}}
\def\etO{{\etens{O}}}
\def\etP{{\etens{P}}}
\def\etQ{{\etens{Q}}}
\def\etR{{\etens{R}}}
\def\etS{{\etens{S}}}
\def\etT{{\etens{T}}}
\def\etU{{\etens{U}}}
\def\etV{{\etens{V}}}
\def\etW{{\etens{W}}}
\def\etX{{\etens{X}}}
\def\etY{{\etens{Y}}}
\def\etZ{{\etens{Z}}}

% The true underlying data generating distribution
\newcommand{\pdata}{p_{\rm{data}}}
\newcommand{\ptarget}{p_{\rm{target}}}
\newcommand{\pprior}{p_{\rm{prior}}}
\newcommand{\pbase}{p_{\rm{base}}}
\newcommand{\pref}{p_{\rm{ref}}}

% The empirical distribution defined by the training set
\newcommand{\ptrain}{\hat{p}_{\rm{data}}}
\newcommand{\Ptrain}{\hat{P}_{\rm{data}}}
% The model distribution
\newcommand{\pmodel}{p_{\rm{model}}}
\newcommand{\Pmodel}{P_{\rm{model}}}
\newcommand{\ptildemodel}{\tilde{p}_{\rm{model}}}
% Stochastic autoencoder distributions
\newcommand{\pencode}{p_{\rm{encoder}}}
\newcommand{\pdecode}{p_{\rm{decoder}}}
\newcommand{\precons}{p_{\rm{reconstruct}}}

\newcommand{\laplace}{\mathrm{Laplace}} % Laplace distribution

\newcommand{\E}{\mathbb{E}}
\newcommand{\Ls}{\mathcal{L}}
\newcommand{\R}{\mathbb{R}}
\newcommand{\emp}{\tilde{p}}
\newcommand{\lr}{\alpha}
\newcommand{\reg}{\lambda}
\newcommand{\rect}{\mathrm{rectifier}}
\newcommand{\softmax}{\mathrm{softmax}}
\newcommand{\sigmoid}{\sigma}
\newcommand{\softplus}{\zeta}
\newcommand{\KL}{D_{\mathrm{KL}}}
\newcommand{\Var}{\mathrm{Var}}
\newcommand{\standarderror}{\mathrm{SE}}
\newcommand{\Cov}{\mathrm{Cov}}
% Wolfram Mathworld says $L^2$ is for function spaces and $\ell^2$ is for vectors
% But then they seem to use $L^2$ for vectors throughout the site, and so does
% wikipedia.
\newcommand{\normlzero}{L^0}
\newcommand{\normlone}{L^1}
\newcommand{\normltwo}{L^2}
\newcommand{\normlp}{L^p}
\newcommand{\normmax}{L^\infty}

\newcommand{\parents}{Pa} % See usage in notation.tex. Chosen to match Daphne's book.

\DeclareMathOperator*{\argmax}{arg\,max}
\DeclareMathOperator*{\argmin}{arg\,min}

\DeclareMathOperator{\sign}{sign}
\DeclareMathOperator{\Tr}{Tr}
\let\ab\allowbreak


\usepackage{caption}
\usepackage{subcaption}
\captionsetup[figure]{font=small}


\title{Adaptive Knowledge Graphs Enhance Medical Question Answering: Bridging the Gap Between LLMs and Evolving Medical Knowledge}

% Author information can be set in various styles:
% For several authors from the same institution:
\author{Mohammad R. Rezaei$^{1\boldsymbol{*}}$, Reza Saadati Fard$^2$, Jayson L. Parker$^3$, Rahul G. Krishnan$^{4,5}$, Milad Lankarany$^{1}$\\ 
        $^1$ Department of Biomedical Engineering, University of Toronto \\
        $^2$ Department of Computer Science, Worcester Polytechnic Institute\\
        $^3$ Department of Biology, University of Toronto Mississauga\\
        $^4$ Department of Computer Science, University of Toronto\\
        $^5$ Vector Institute\\
        \textsuperscript{*}\texttt{mr.rezaei@mail.utoronto.ca}}

% \author{Reza Saadati Fard,\\ 
%         Department of Computer Science, Worcester Polytechnic Institute \\  \textsuperscript{*}\texttt{xx}}
% if the names do not fit well on one line use
%         Author 1 \\ {\bf Author 2} \\ ... \\ {\bf Author n} \\
% For authors from different institutions:
% \author{Author 1 \\ Address line \\  ... \\ Address line
%         \And  ... \And
%         Author n \\ Address line \\ ... \\ Address line}
% To start a separate ``row'' of authors use \AND, as in
% \author{Author 1 \\ Address line \\  ... \\ Address line
%         \AND
%         Author 2 \\ Address line \\ ... \\ Address line \And
%         Author 3 \\ Address line \\ ... \\ Address line}

% \author{First Author \\
%   Affiliation / Address line 1 \\
%   Affiliation / Address line 2 \\
%   Affiliation / Address line 3 \\
%   \texttt{email@domain} \\\And
%   Second Author \\
%   Affiliation / Address line 1 \\
%   Affiliation / Address line 2 \\
%   Affiliation / Address line 3 \\
%   \texttt{email@domain} \\}

%\author{
%  \textbf{First Author\textsuperscript{1}},
%  \textbf{Second Author\textsuperscript{1,2}},
%  \textbf{Third T. Author\textsuperscript{1}},
%  \textbf{Fourth Author\textsuperscript{1}},
%\\
%  \textbf{Fifth Author\textsuperscript{1,2}},
%  \textbf{Sixth Author\textsuperscript{1}},
%  \textbf{Seventh Author\textsuperscript{1}},
%  \textbf{Eighth Author \textsuperscript{1,2,3,4}},
%\\
%  \textbf{Ninth Author\textsuperscript{1}},
%  \textbf{Tenth Author\textsuperscript{1}},
%  \textbf{Eleventh E. Author\textsuperscript{1,2,3,4,5}},
%  \textbf{Twelfth Author\textsuperscript{1}},
%\\
%  \textbf{Thirteenth Author\textsuperscript{3}},
%  \textbf{Fourteenth F. Author\textsuperscript{2,4}},
%  \textbf{Fifteenth Author\textsuperscript{1}},
%  \textbf{Sixteenth Author\textsuperscript{1}},
%\\
%  \textbf{Seventeenth S. Author\textsuperscript{4,5}},
%  \textbf{Eighteenth Author\textsuperscript{3,4}},
%  \textbf{Nineteenth N. Author\textsuperscript{2,5}},
%  \textbf{Twentieth Author\textsuperscript{1}}
%\\
%\\
%  \textsuperscript{1}Affiliation 1,
%  \textsuperscript{2}Affiliation 2,
%  \textsuperscript{3}Affiliation 3,
%  \textsuperscript{4}Affiliation 4,
%  \textsuperscript{5}Affiliation 5
%\\
%  \small{
%    \textbf{Correspondence:} \href{mailto:email@domain}{email@domain}
%  }
%}
\usepackage{pifont}
\newcommand{\yesmarker}{\textcolor{black}{\ding{51}}} % Check mark
\newcommand{\nomarker}{\textcolor{red}{\ding{55}}}   % Cross mark
\begin{document}
\newacronym{myrag}{AMG-RAG}{Adaptive Medical Graph-RAG}
\newacronym{kg}{KG}{Knowledge Graph}
\newacronym{mkg}{MKG}{Medical Knowledge Graph}

\newacronym{llm}{LLM}{Large Language Model}
\newacronym{cot}{CoT}{Chain-of-Thought}
\newacronym{qa}{QA}{Question Answering}
\newacronym{rag}{RAG}{Retrieval Augmented Generation }
\maketitle
\begin{abstract}
\glspl{llm} have greatly advanced medical \gls{qa} by leveraging vast clinical data and medical literature. However, the rapid evolution of medical knowledge and the labor-intensive process of manually updating domain-specific resources can undermine the reliability of these systems. We address this challenge with \gls{myrag}, a comprehensive framework that automates the construction and continuous updating of \glspl{mkg}, integrates \gls{cot} reasoning, and retrieves current external evidence (e.g., PubMed, WikiSearch). By dynamically linking new findings and complex medical concepts, \gls{myrag} not only boosts accuracy but also enhances interpretability for medical queries.

Evaluations on the MEDQA and MEDMCQA benchmarks demonstrate the effectiveness of \gls{myrag}, achieving an F1 score of 74.1\% on MEDQA and an accuracy of 66.34\% on MEDMCQA—surpassing both comparable models and those 10 to 100 times larger. Importantly, these improvements are achieved without increasing computational overhead, underscoring the critical impact of automated knowledge graph generation and external evidence retrieval in delivering up-to-date, trustworthy medical insights.
\end{abstract}

% !TEX root = ../main.tex

Large Language Models (LLMs) have shown remarkable capabilities on numerous tasks in Natural Language Processing (NLP), 
ranging from language understanding to generation \cite{bubeck2023sparks, achiam2023gpt,team2023gemini, dubey2024llama}. The huge success of LLMs comes with important challenges to deploy them due to their massive size and computational costs. For instance,  Llama-3-405B \cite{dubey2024llama} requires 780GB of storage in half precision (FP16) and hence multiple high-end GPUs are needed just for inference. \textit{Model compression} has emerged as an important line of research to reduce the costs associated with deploying these foundation models. In particular, neural network pruning \cite{obd, hassibi1992second, benbaki2023fast}, where model weights are made to be sparse after training, has garnered significant attention. Different sparsity structures (Structured, Semi-Structured and Unstructured) obtained after neural network pruning result in different acceleration schemes. \textit{Structured pruning} removes entire structures such as channels, filters, or attention heads \cite{lebedev2016fast,wen2016learning,voita2019analyzing,el2022data} and readily results in acceleration as model weights dimensions are reduced. \textit{Semi-Structured pruning}, also known as, N:M sparsity \cite{zhou2021learning} requires that at most $N$ out of $M$ consecutive elements are non-zero elements. Modern NVIDIA GPUs provide support for 2:4 sparsity acceleration. \textit{Unstructured pruning} removes individual weights \cite{han2015learning, guo2016dynamic} from the model's weights and requires specialized hardware for acceleration. For instance, DeepSparse \cite{kurtic2022optimal, pmlr-v119-kurtz20a, DBLP:journals/corr/abs-2111-13445} provide CPU inference acceleration for unstructured sparsity.\\
Specializing to LLMs, one-shot pruning~\cite{meng2024alps, frantar2023sparsegpt, sun2023simple, zhang2023dynamic}, where one does a single forward pass on a small amount of calibration data, and prunes the model without expensive fine-tuning/retraining, is of particular interest. This setup requires less hardware requirements. For instance, \citet{meng2024alps} show how to prune an OPT-30B \cite{opt} using a single consumer-level V100 GPU with 32GB of CUDA memory, whereas full fine-tuning of such model using Adam \cite{kingma2014adam} at half-precision requires more than 220GB of CUDA memory.

Although one-shot pruning has desirable computational properties, it can degrade models' predictive and generative performance. To this end, recent work has studied extensions of model pruning to achieve smaller utility drop of model performance from compression. 
% Multiple one-shot methods have been developed in quantization \cite{frantar2022gptq, frantar2023sparsegpt, lin2024awq, behdin2023quantease, dettmers2023spqr} and neural network pruning \cite{frantar2023sparsegpt, meng2024alps, zhang2024oats}, which is closer to this paper's line of research. These one-shot methods do not require retraining--which is extremely expensive for models of the size of Llama-3-405B-- and work as resource-saving techniques that retain the model's performance. 

An interesting compression mechanism in the field of \textit{model compression} is the Sparse plus Low-Rank Matrix-Decomposition problem which aims to approximate model's weights by a sparse component plus a low-rank component~\cite{hintermuller2015robust, candes2011robust, lin2011linearized, 5394889, zhou2011godec, JMLR:v24:21-1130, NIPS2014_443cb001, yu2017compressing, li2023losparse}. Specializing to LLMs,~\citet{zhang2024oats} propose OATS 
%that addresses this type of %compression and 
that outperforms pruning methods for the same compression ratio (number of non-zero elements) on a wide range of LLM evaluation benchmarks (e.g. perplexity in Language generation). 

OATS \cite{zhang2024oats} is however a matrix decomposition algorithm inspired from a pruning algorithm Wanda \cite{sun2023simple}. Wanda has been designed as a relaxation/approximation of another state-of-the-art pruning algorithm SparseGPT \cite{frantar2023sparsegpt}. While Wanda has been found to be extremely useful and efficient in practice, recent work \cite{meng2024alps} show results where Wanda fails for high-sparsity regimes. In this paper, we provide a unified optimization framework to decompose pre-trained model weights into sparse plus low-rank components based on a layer-wise loss function. Our framework is modular and can incorporate different pruning and matrix-decomposition algorithms (developed independently in different contexts).
%under the umbrella of the local %layer-wise reconstruction error; 
Similar to~\cite{meng2024alps} we observe that our optimization-based framework results in models with better model utility-compression tradeoffs. The difference is particularly pronounced for higher compression regimes. 
%especially for higher compression %budgets, where SOTA methods 
% Our numerical results also show similar findings to \citet{meng2024alps} where high-sparsity significantly degrades the performance of approximation-based optimization methods like OATS.

Concurrently, in a different and complementary line of work,~\citet{mozaffari2024slope} have open-sourced highly-specialized CUDA kernels designed for N:M sparse \cite{zhou2021learning} plus low-rank matrix decompositions that result in significant acceleration and memory reduction for the pre-training of LLMs.
We note that our focus here is on improved algorithms for one-shot sparse plus low-rank matrix decompositions for foundation models with billions of parameters which is different from the work of \citet{mozaffari2024slope} that focuses on accelerating the pre-training of LLMs. The designed CUDA kernels \cite{mozaffari2024slope} can be exploited in our setting for faster acceleration and reduced memory footprint during inference.





% \textbf{Summary of approach and contributions:} We propose \ourmethod: an accurate and scalable framework for Sparse plus Low-Rank Matrix Decomposition for LLMs. Following the previous work on one-shot pruning and model compression, we pursue a layerwise approach. In particular, the reconstruction error resulting from compression in the output of each layer is minimized, under the compression constraints (i.e., sparsity and low-rank constraints).

\textbf{Summary of approach.\,\,\,\,} Our framework is coined \ourframework: \underline{H}ardware-\underline{A}ware (Semi-\underline{S}tructured) \underline{S}parse plus \underline{L}ow-rank \underline{E}fficient \& approximation-\underline{free} matrix decomposition for foundation models.

Hardware-aware refers to the fact that we mostly focus on a N:M sparse \cite{zhou2021learning} plus low-rank decomposition, for which acceleration on GPUs is possible, although \ourframework supports any type of sparsity pattern (unstructured, semi-structured, structured) in the sparsity constraint. Approximation-free refers to the fact that we directly minimize the local layer-wise reconstruction error introduced in \cref{eq:matrix-decomposition}, whereas we show prior work minimizes an approximation of this objective.

%Our unified framework introduces a well-posed 
%%As a part of our proposed framework, we consider an 
%%optimization form
We formulate the compression/decomposition task as a clear optimization problem; we minimize a local layer-wise reconstruction objective where the weights are given by the sum of a sparse and low-rank component. 
%%%of dense model weights under the  
%This optimization problem is decoupled into a sparse minimization subproblem and a low-rank minimization subproblem. 
We propose an efficient Alternating-Minimization approach that scales to models with billions of parameters relying on 
two key components: one involving sparse minimization (weight sparsity) and the other involving a low-rank optimization. 
For each of these subproblems 
we discuss how approximations to the optimization task can retrieve prior algorithms.
%the introduced subproblems, 
%we consider approximations to the minimization objective and retrieve different algorithms from related works given different %approximations.

% We provide an efficient and scalable algorithm based on Alternating-Minimization that does not rely on any approximation at the objective minimization level. 
% While \ourframework supports any sparsity pattern (unstructured, semi-structured, structured) in the sparsity constraint, we mostly focus on N:M sparsity \cite{zhou2021learning}, to make the decomposition Hardware-aware, as \citet{mozaffari2024slope} show how to get acceleration on modern GPUs for N:M sparse plus low-rank decomposition.

We note that \ourframework~differs from prior one-shot (sparse) pruning methods~\cite{frantar2023sparsegpt, meng2024alps, benbaki2023fast} as we seek a sparse plus low-rank decompositon of weights.
%%%%%introducing the low-rank component. 
Additionally, it differs from prior one-shot sparse plus low-rank matrix decomposition methods~\cite{zhang2024oats}
%by considering an approximation-free minimization approach of the 
as we directly minimize the local layer-wise reconstruction objective introduced in \cref{eq:matrix-decomposition}.

Our main \textbf{contributions} can be summarized as follows.
\begin{compactitem}
    \item We introduce \ourframework a unified one-shot LLM compression framework that scales to models with billions of parameters where we directly minimize the local layer-wise reconstruction error subject to  a sparse plus low-rank matrix decomposition of the pre-trained dense weights. 
    %    formulates a sparse plus low-rank matrix decomposition as an optimization problem with a local layer-wise reconstruction objective. We discuss approximations of this objective and show that OATS a popular method is recovered in a particular approximation.

    
    \item \ourframework uses an Alternating-Minimization approach that iteratively minimizes a Sparse and a Low-Rank component. \ourframework uses a given pruning method as a plug-in for the subproblem pertaining to the sparse component. Additionally, it uses Gradient-Descent type methods for the subproblem pertaining to the Low-Rank component.
    
    % \item In the subproblem pertaining to the sparse component, a rewrite of the optimization formulation shows that one can use any pruning algorithm, that minimizes the layer-wise reconstruction error, as a plug-in to sparsify the weights. We choose to show results for the algorithm SparseGPT.
    
    % In this pruning subproblem, we also enhance the performance of \ourmethod by exploiting the invariance of the Hessian--of the layer-wise reconstruction error--in each subproblem of the Alternating Minimization procedure, for a given layer. In particular, we use a pre-processing step that computes and stores the Hessian inverse--of the objective--, which is then passed to the deployed pruning algorithm (e.g. SparseGPT). 
    % \item In the subproblem  pertaining to the Low-Rank component, we give a theoretical closed form solution to the subproblem.
    % which does not scale to problems with billions of parameters. 
    % We also present a more tractable first-order optimization method for a reparametrization of the the low-rank problem, which is scalable to models with billions of parameters.
    
    % as $\bfUVt$ and use first-order optimization methods to minimize the layer-wise reconstruction objective.

    \item We discuss how special cases of our framework relying on specific approximations of the objective retrieve popular methods such as OATS, Wanda and MP --- \cite{zhang2024oats, sun2023simple,han2015learning, sze2020efficient}. This provides valuable insights into the underlying connections across different methods. 

    \item \ourframework improves upon state-of-the-art methods for one-shot sparse plus low-rank matrix decomposition. 
    For the Llama3-8B model with a 2:4 sparsity component plus a 64-rank component decomposition, \ourframework reduces the test perplexity by $12\%$ for the WikiText-2 dataset and reduces the gap (compared to the dense model) of the average of eight popular zero-shot tasks by $15\%$ compared to existing methods.
\end{compactitem}




\section{Background and Related Work}
Our work proposes to analyze how user involvement in the planning and execution stages of LLM agents shapes user trust in the LLM agents and the overall task performance of LLM agents. 
Thus, we position our work in \revise{three realms} of related literature: human-AI collaboration (\S~\ref{sec-rel-collaboration-LLM}), \revise{trust and reliance on AI systems (\S~\ref{sec-rel-trust-reliance}),} task support with LLMs and LLM agents (\S~\ref{sec-rel-LLM-agent}). %\glcomment{Be careful with the human-AI collaboration. I find it a bit distracting from our focus}

%\glcomment{the paper did not provide a strong review of literature surrounding AI trust. While I am not an expert in understanding trust in AI, I do know there is rich literature in this area. Because this review was not included, it is difficult to evaluate the originality of this work.}\ujcomment{Improve the section, but also point towards recent reviews for readers to get a more comprehensive view (e.g., Siddharth Mehrotra's recent trust review paper)}

\subsection{Human-AI Collaboration}
\label{sec-rel-collaboration-LLM}
% LLM agent in CHI~\cite{zhang2024s}
% \ujcomment{- General human-AI collaboration; delegation; AI-assisted decision making; trust/reliance; complementary performance; 
% - Metrics introduced in recent years;
% - What has actually worked or shown promise in facilitating optimal human-AI collaboration?
% - How do we position our work in this context?}

% \paratitle{Delegation, algorithm appreciation, algorithm aversion, control}. 
In recent decades, deep learning-based AI systems have shown promising performance across various domains~\cite{yang2022survey,fernando2021deep} and applications~\cite{pouyanfar2018survey,dong2021survey}. 
However, such AI systems are not good at dealing with out-of-distribution data~\cite{jia2017adversarial,mccoy-etal-2019-right}, and their intrinsic probabilistic nature brings much uncertainty in %practical service
practice~\cite{ghahramani2015probabilistic}. 
Such observations raise wide concerns about the accountability and reliability of AI systems~\cite{kaur2022trustworthy}. 
Under such circumstances, human-AI collaboration has been recognized as a well-suited approach %one promising approach 
to taking advantage of their promising predictive power and ensuring trustworthy outcomes~\cite{lai2021towards,jiang2021supporting}. 
While humans can provide more reliable and accountable task outcomes, too much user involvement to check and control AI outcomes is undesirable~\cite{lai2022human}. 
It goes against the premise that AI systems are introduced to reduce human workload. 
In that context, researchers have theorized and empirically analyzed when and where users could and should delegate to AI systems~\cite{lai2022human,lubars2019ask}. 

\paratitle{Task Delegation}. While humans prefer to play the leading role in human-AI collaboration~\cite{lubars2019ask}, delegating to AI systems can bring benefits like cost-saving and higher efficiency. 
Apart from manual delegation decisions, it is common to apply automatic rules for human delegation (\eg heuristics obtained from domain expertise or manually crafted rules~\cite{lai2022human}).
% Humans can delegate to AI decisions based on . 
Many user factors like trust~\cite{lubars2019ask}, human expertise domain~\cite{erlei2024understanding}, and AI knowledge~\cite{pinski2023ai}) have a substantial impact on human delegation behaviors. 
% With an empirical study, Erlei \etal~\cite{erlei2024understanding} found that the human expertise domain impacts human delegation behaviors, and the choice independence will be violated when users consider AI performance in an unrelated task.
% Erlei \etal conducted an empirical study to analyze the impact of choice independence and error type in the appropriate delegation behaviors~\cite{erlei2024understanding}. 
%Besides human delegation to AI systems, 
Another relevant stream of recent research has explored AI delegation to humans~\cite{madras2018predict,fugener2022cognitive,pinski2023ai}. 
Researchers have investigated the conditions under which AI systems should defer to a human decision maker, which may bring benefits of improved fairness~\cite{madras2018predict}, accuracy~\cite{narasimhan2022post}, and complementary teaming~\cite{ijcai2022p344}. 
Compared to human delegation, AI delegation has been observed to achieve more consistent benefits in team performance~\cite{fugener2022cognitive,hemmer2023human}. {In collaboration with LLM agents, users need to determine when they should be involved in high-level planning and real-time execution. Such involvement decisions are similar to the delegation choices made by users. While task delegation is not the focus of our study, future work can explore this further.}% within human-LLM agent collaboration.}


\paratitle{AI-assisted Decision Making} has attracted a lot of research focus in human-AI collaboration literature. 
Most existing work has conducted empirical studies~\cite{lai2021towards} and structured interviews~\cite{jiang2021supporting} to understand how factors surrounding the user, task, and AI systems affect human-AI collaboration. 
User factors like AI literacy~\cite{Chiang-IUI-2022}, cognitive bias~\cite{rastogi2022deciding}, and risk perception~\cite{fogliato2021impact,green2021algorithmic} have been observed to substantially impact user trust and reliance on the AI system. 
Similarly, task characteristics like task complexity and uncertainty~\cite{salimzadeh2023missing,salimzadeh2024dealing} and factors of the AI system (\eg performance feedback~\cite{bansal2019beyond,Lu-CHI-2021}, AI transparency~\cite{vossing2022designing} and confidence of AI advice~\cite{tomsett2020rapid,Zhang-FAT-2020}) also affect user trust and reliance on the AI system. 
For a more comprehensive survey of existing work on AI-assisted decision making, readers can refer to~\cite{lai2021towards}.

% Typically, user trust is operationalized as a subjective attitude toward AI systems/AI advice within the literature on human-AI collaboration. In comparison, user reliance on AI systems is based on user behaviors (\eg adoption of AI advice and modification of AI outcomes). 
% Such formulation can even be dated back to trust and reliance on automation systems~\cite{lee2004trust}.

% \paratitle{Calibrated Trust and Appropriate Reliance}. User trust in the context of human-AI collaboration is typically operationalized as a subjective attitude toward AI systems/AI advice~\cite{lee2004trust}. In comparison, user reliance on AI systems is based on user behaviors (\eg adoption of AI advice and modification of AI outcomes). 
% As pointed out by existing work on trust in algorithmic/automated systems, user trust can substantially affect user reliance~\cite{lee2004trust}. 
% While trust calibration is an important goal in human-AI collaboration, it may be not enough to ensure complementary team performance. 
% Through empirical user studies with different confidence levels of AI predictions, Zhang \etal~\cite{Zhang-FAT-2020} found that ``trust calibration alone is not sufficient to improve AI-assisted decision making''. 
% To achieve optimal human-AI collaboration, humans and AI systems need to play complementary roles~\cite{hemmer2021human,hemmer2024complementarity}, and humans need to know when they should adopt AI assistance. 
% In other words, humans should rely on AI advice when AI systems are correct and outperform them, and override AI advice when AI systems are incorrect or less capable than humans. 
% Such user reliance patterns are denoted as \textit{appropriate reliance}~\cite{schemmer2022should,schemmer2023appropriate}, which is the key to
% achieving complementary team performance. 

% Compared with human assistance, users can easily lose confidence in AI systems after seeing them make the same mistakes~\cite{dietvorst2015algorithm}. 
% Such algorithm aversion can be overcome by enabling users to modify the AI predictions~\cite{dietvorst2018overcoming}. 
% As a result of these 
% under-reliance (disuse AI assistance when AI systems outperform humans) and over-reliance (misuse AI assistance when AI systems are wrong or perform worse than humans).
% , users show contradicting attitudes towards AI assistance: algorithm appreciation~\cite{logg2019algorithm,hou2021expert} and algorithm aversion~\cite{dietvorst2015algorithm,dietvorst2018overcoming}. 

% \paratitle{User Trust}. 
% Most existing work has conducted empirical studies~\cite{lai2021towards} and structured interviews~\cite{jiang2021supporting} to understand user trust in AI systems. 



% As a result of uncalibrated trust, users also show sub-optimal reliance on the AI systems: under-reliance (disuse AI assistance when AI systems outperform humans) and over-reliance (misuse AI assistance when AI systems are wrong or perform worse than humans).\glcomment{Is this claim true: unexpected reliance due to uncalibrated trust?}

% To achieve optimal human-AI collaboration, humans and AI systems are supposed to play complementary roles~\cite{hemmer2021human,hemmer2024complementarity}, and humans know when they should adopt AI assistance. 
% In other words, humans should rely on AI advice when AI systems are correct and outperform them, and humans should override AI advice when AI systems are incorrect or less capable than humans. 
% Such user reliance patterns are denoted as appropriate reliance~\cite{schemmer2022should,schemmer2023appropriate}, which is the key to
% achieving complementary team performance. 
% Many factors like cognitive bias~\cite{he2023knowing}, 
% \paratitle{Interventions to Facilitate Calibrated Trust and Appropriate Reliance} 
% The main issues that lead to sub-optimal human-AI collaboration are: under-reliance (\ie disuse AI assistance when AI systems outperform humans) and over-reliance (\ie misuse AI assistance when AI systems are wrong or perform worse than humans)~\cite{schemmer2022should}. 
% %These reliance behaviors are also highly relevant to uncalibrated user trust. 
% Users with an uncalibrated trust in the AI system can be easily misled to disuse or misuse AI systems~\cite{jacovi2021formalizing}. 
%For example, compared with human assistance, users can easily develop a negative impression of AI systems and lose confidence in AI systems. Such phenomenon is called algorithm aversion~\cite{dietvorst2015algorithm}. 
%By contrast, some users were influenced more by algorithmic decisions instead of human decisions, and they first coined the notion of ``Algorithm Appreciation''~\cite{logg2019algorithm}. 
% Researchers have proposed various interventions to promote appropriate reliance~\cite{he2023knowing,Lu-CHI-2021,lu2024does,chiang2021you,Chiang-IUI-2022} and calibrate user trust in AI systems~\cite{buccinca2021trust,Zhang-FAT-2020}.  
% % We bring some representative interventions here.
% %\glcomment{Here is not good enough. I don't plan to bring too many examples. To check how to improve}
% For example, explainable AI methods have been shown to help reduce over-reliance~\cite{vasconcelos2023explanations} and under-reliance~\cite{wang2021explanations} in different scenarios albeit with little consistency across contexts. 
% Another example is tutorial interventions, which have shown effectiveness in user onboarding~\cite{lai2020chicago}, mitigating cognitive biases~\cite{he2023knowing} and developing AI literacy~\cite{Chiang-IUI-2022}. 
% For a more comprehensive overview of interventions to facilitate trust calibration and appropriate reliance, readers can refer to ~\cite{lai2021towards,eckhardt2024survey}.

% In this work, we analyze how user involvement in the planning and execution stages of LLM agents will shape user trust and affect overall task performance. 
% It is highly relevant to existing studies of human-AI collaboration about user trust and appropriate reliance. 
While machine learning and deep learning methods have been extensively analyzed in existing human-AI collaboration literature, to our knowledge, human-AI collaboration with LLM agents is still under-explored. 
Unlike previous studies where AI systems only follow a fixed mode to generate advice, LLM agents can be equipped with more logical clarity and can provide a step-wise plan and can follow a step-by-step execution. 
With such a plan-then-execute setup, LLM agents can bring high flexibility as well as uncertainty in high-level planning and real-time execution. Little is known about
%Meanwhile, it is unclear 
how well LLM agents can work as daily assistants while handling tasks entailing varying stakes and potential risks. %where wrong actions may cause a loss. 
In our study, we analyzed the impact of user involvement in such AI systems by adjusting their intermediate outcomes (plan and step-by-step execution) to calibrate their trust and improve task outcomes. 
Our findings and implications can help advance the understanding of the effectiveness of LLM agents in human-AI collaboration.
% with humans.

%\glcomment{I find the positioning of our work in the space of human-AI collaboration is a bit challenging. Our major claim is: human-AI collaboration with LLM agents is under-explored}

\subsection{\revise{Trust and Reliance on AI systems}}
\label{sec-rel-trust-reliance}
\revise{Trust and reliance have been important research topics since human adoption of automation systems~\cite{lee2004trust,dzindolet2003role}. Due to the widespread integration of AI systems in nearly all walks of society, %In recent years, 
there has been a growing interest in understanding user %researchers have developed a strong interest in 
trust~\cite{vereschak2021evaluate,mehrotra2024systematic} and reliance~\cite{eckhardt2024survey} on AI systems.}
User trust in the context of human-AI collaboration is typically operationalized as a subjective attitude toward AI systems/AI advice~\cite{lee2004trust}. 
In comparison, user reliance on AI systems is based on user behaviors (\eg adoption of AI advice and modification of AI outcomes). 
\revise{The two constructs have been shown to be highly related~\cite{lee1992trust,lee2004trust}: for example, user trust can substantially affect user reliance~\cite{lee2004trust}. 
However, they are intrinsically different and cannot be viewed as a direct reflection of each other~\cite{kahr2024understanding}. 
Most existing work has, therefore, studied the two constructs separately in terms of subjective trust and objective reliance.}
% As pointed out by existing work on trust in algorithmic/automated systems, user trust can substantially affect user reliance~\cite{lee2004trust}. 

\revise{%In an early analysis of human-AI trust, most literature 
Earlier work exploring human-AI trust primarily focused on the impact of different contextual factors surrounding user (\eg risk perception~\cite{green2021algorithmic}), task (\eg task complexity~\cite{salimzadeh2023missing}), and system (\eg stated accuracy~\cite{yin2019understanding,Zhang-FAT-2020}). 
% Among this literature, performance indicators of the AI system (\eg stated accuracy~\cite{yin2019understanding,Zhang-FAT-2020} and confidence~\cite{rechkemmer2022confidence}) have been extensively studied. 
% For trust calibration, researchers have proposed different interventions like explanations~\cite{Zhang-FAT-2020}, educational tutorials~\cite{Chiang-IUI-2022,chiang2021you,lai2020chicago}, competence comparison~\cite{ma2023should}. 
%According to empirical studies about AI-assisted decision making~\cite{yin2019understanding}, 
Empirical studies have shown that most users tend to trust AI systems that are perceived to be highly accurate~\cite{yin2019understanding}. 
Such trust is vulnerable, as the AI system may provide an illusion of competence with persuasive technology (\eg explanations~\cite{chromik2021think,He-IUI-2025}) or overclaimed performance~\cite{yin2019understanding}. 
Even if the AI systems are accurate on specific datasets, they still suffer from out-of-distribution data~\cite{liu2021understanding,chiang2021you}. 
The misplaced trust in the AI systems can lead to misuse of the systems.
% But such trust can be fragile. 
Several empirical studies~\cite{tolmeijer2021second} have shown that once users realize the AI system errs or performs worse than expected, their trust in the AI system can be violated, %In the extreme case, it can 
even resulting in the disuse of the AI system. 
Both the misuse and disuse of the AI system hinder optimal human-AI collaboration. 
}

\revise{%To calibrate user trust in the AI system, 
To address such concerns, researchers have explored how to help users calibrate their trust in the AI system. %researchers proposed 
Different techniques to help users realize the trustworthiness of the AI system have been proposed~\cite{kaur2022trustworthy,rechkemmer2022confidence,ma2023should}. 
For example, increasing the transparency of AI systems by providing confidence scores~\cite{Zhang-FAT-2020}, explanations~\cite{wang2021explanations}, trustworthiness cues~\cite{liao2022designing}, and uncertainty communication~\cite{Sunnie-FAccT-2024}. 
However, the actual trustworthiness of the AI system does not always align with user perception. 
As found by \citet{banovic2023being}, untrustworthy AI systems can deceive end users to gain their trust. 
Another example is that users can develop an illusion of explanatory depth brought by explainable AI techniques~\cite{chromik2021think}, which leads to uncalibrated trust in the AI system. 
Even if users have indicated trust in the AI system, they may turn to rely more on themselves in final decision-making. 
The reasons are complex, and many factors, such as accountability concerns~\cite{lima2021human,tolmeijer2022capable} and cognitive bias~\cite{he2023knowing}, may affect user reliance behaviors. %Much research effort is required to further our understanding of trust calibration in AI systems.
% On the other hand, research has dived deep into calibrating user trust in AI systems.
}

While trust calibration is an important goal in human-AI collaboration, it may be not enough to ensure complementary team performance. 
Through empirical user studies with different confidence levels of AI predictions, Zhang \etal~\cite{Zhang-FAT-2020} found that ``trust calibration alone is not sufficient to improve AI-assisted decision making''. 
To achieve optimal human-AI collaboration, humans and AI systems need to play complementary roles~\cite{hemmer2021human,hemmer2024complementarity}, and humans need to know when they should adopt AI assistance. 
In other words, humans should rely on AI advice when AI systems are correct and outperform them, and override AI advice when AI systems are incorrect or less capable than humans. 
Such user reliance patterns are denoted as \textit{appropriate reliance}~\cite{schemmer2022should,schemmer2023appropriate}, which is the key to
achieving complementary team performance. 

The main issues that lead to sub-optimal human-AI collaboration are: under-reliance (\ie disuse AI assistance when AI systems outperform humans) and over-reliance (\ie misuse AI assistance when AI systems are wrong or perform worse than humans)~\cite{schemmer2022should}. 
Users with an uncalibrated trust in the AI system can be easily misled to disuse or misuse AI systems~\cite{jacovi2021formalizing}. 
Researchers have proposed various interventions to promote appropriate reliance~\cite{he2023knowing,Lu-CHI-2021,lu2024does,chiang2021you,Chiang-IUI-2022} and calibrate user trust in AI systems~\cite{buccinca2021trust,Zhang-FAT-2020}.  
For example, explainable AI methods have been shown to help reduce over-reliance~\cite{vasconcelos2023explanations} and under-reliance~\cite{wang2021explanations} in different scenarios albeit with little consistency across contexts. 
Another example is tutorial interventions, which have shown effectiveness in user onboarding~\cite{lai2020chicago}, mitigating cognitive biases~\cite{he2023knowing} and developing AI literacy~\cite{Chiang-IUI-2022}. 
For a more comprehensive overview of interventions to facilitate trust calibration and appropriate reliance, readers can refer to ~\cite{lai2021towards,eckhardt2024survey,mehrotra2024systematic,kahr2024understanding}.

\revise{LLM agents~\cite{wang2024survey} have gained much popularity in recent years, distinguishing them from most prior AI systems. 
They can communicate through conversation, plan logically, and can be built to leverage powerful external tools to achieve complex functions.
% The interaction between human and AI systems is relatively limited. 
% Users mostly develop trust and reliance on the AI systems via provided information cues (\eg, explanations and references). 
While trust and reliance have been extensively analyzed in existing human-AI collaboration literature, it is still unclear how users trust and rely on AI systems powered by LLM agents. 
% This work addressed this gap with empirical studies to analyze user trust and reliance on the plan-then-execute LLM agents. 
In our work, calibrated trust is adopted as an important goal for human-AI collaboration in the planning and execution stage. 
Meanwhile, users are expected to fix potential errors in the planning and execution stages, reflecting their reliance on the AI system. 
Our work can substantially advance the understanding of trust and reliance on plan-then-execute LLM agents.
}

% \paratitle{position our work}.

\subsection{Task Support with LLMs and LLM Agents}
\label{sec-rel-LLM-agent}
% \glcomment{Do we need to give more context about LLM Agent? It seems more technical}
LLMs and LLM agents bring new opportunities and challenges to human-AI collaboration~\cite{bommasani2021opportunities}. 
%On the one hand, based on the text generation capability of LLMs, it would be possible for humans to directly give text responses and communicate with any AI systems that take LLMs as their backbone. 
%To this end, 
%LLMs and LLM agents bring new opportunities for more flexible and natural interactions with humans. 
% Meanwhile, the natural language understanding and learning capabilities also enable LLMs to evolve with user feedback. 
It is evident that their generation capabilities can help reduce the cognitive effort from humans. %On the other hand, the LLMs also bring new challenges like dealing with 
But LLMs are also riddled with challenges such as hallucination~\cite{ji2023survey} (\ie generated text seems plausible but is factually incorrect). 
% Fatal errors can be made if users get misled by such persuasive technology, resulting in unaffordable costs (\eg medical diagnosis and financial decisions).
Failure to handle such issues may bring fatal errors with unaffordable costs depending on the context (\eg medical diagnosis). 
%As our work is within the scope of human-AI collaboration, users can refer to corresponding literature reviews to obtain more technical background about LLMs~\cite{zhao2023survey} and LLM agents~\cite{xi2023rise,wang2024survey}.

\begin{figure*}[h]
    \centering
    \includegraphics[width=0.75\textwidth]{figures/Screenshot-planning.png}
    \caption{Screenshot of user-involved planning interface.} 
    \label{fig:planning}
    \Description{Screenshot of user-involved planning interface. At the top, we show the task description along with three buttons: show potential actions, plan edit instruction, and add one step. At the bottom, we show a step-wise plan for setting alarms. Users can click these buttons to achive the function we described in the user-involved planning to edit the plan.}
\end{figure*}

% In recent years, LLMs have gained an explosion of popularity among academia and the industry community. 
Due to the capability of generating coherent, knowledgeable, and high-quality responses to diverse human input~\cite{wei2022emergent}, a wide community of human-computer interaction researchers has paid attention to large language models~\cite{liao2023ai}.
% With large volumes of data, large language models can obtain capabilities to help humans in writing. 
% Ideally, any complex task that can be modularized into a chain of different functions can also be achieved with chaining LLMs.
% LLMs can be used to generate dynamic user interface ~\cite{wang2023enabling}, support scientific writing~\cite{shen2023convxai}, and obtain high-quality data annotation~\cite{gilardi2023chatgpt,wang2024human,he2024if}. 
Researchers have actively explored how LLMs can assist users in various tasks like data annotation~\cite{wang2024human,he2024if}, programming~\cite{omidvar2024evaluating}, scientific writing~\cite{shen2023convxai}, and fact verification~\cite{si2024large}. 
All the above functions can be achieved with elaborate prompt engineering using a single LLM. 
By chaining multiple LLMs with different functions, humans can customize task-specific workflows to solve complex tasks~\cite{wu2022ai}. 
Apart from obtaining answers with a one-shot text generation, LLMs also provide convenient conversational interactions. 
Through empirical studies, such conversational interactions have been shown to be effective in human-AI collaboration with multiple applications, such as decision making~\cite{slack2023explaining,lin2024decision,ma2024towards}, scientific writing~\cite{shen2023convxai}, and mental health support~\cite{sharma2023human}. 
With the growing popularity of LLMs, more and more humans have begun to adopt LLMs (\eg ChatGPT) to boost their work efficiency and productivity %in their everyday work
~\cite{zhao2023survey}.

% Meanwhile, researchers also analyzed different system factors associated with LLMs. 
% For example, Kim \etal~\cite{Sunnie-FAccT-2024} found that the LLM's uncertainty Expression can decrease user trust in wrong AI advice, which helps reduce over-reliance.


LLM agents have been shown to have good planning, memory, and toolkit usage capabilities~\cite{xi2023rise, wang2024survey}. 
% External toolkits greatly increase the impact of human-AI collaboration on the real world. 
When suitable toolkits are provided, LLM agents can readily generate a task-specific plan and solve the tasks using toolkits. 
Attracted by the promise of LLM agents, there have been some early explorations~\cite{geissler2024concept,zheng2023synergizing,zhang2024s} of adopting them in human-AI collaboration contexts. 
%To our knowledge, only a few works~\cite{geissler2024concept,zheng2023synergizing,zhang2024s} have explored human-AI collaboration with LLM agents. 
% These works mainly analyzed how LLM agents can serve specific use cases (\eg design creation~\cite{geissler2024concept}) or conducted structured interviews to obtain expert insights~\cite{zhang2024s,zheng2023synergizing}. 
These works were mostly analyzed in specific use cases (\eg design creation~\cite{geissler2024concept}). 
% There is a lack of empirical studies on user trust and team performance in collaboration with LLM agents.
It is unclear how user trust and team performance are affected by user interactions with LLM agents in a sequential decision making setup (\ie solving a task by executing a sequence of actions) where users can be in control of the execution. %\glcomment{It is the only mention of sequential decision making. In my current framing, I try to give a very specific definition of what type of tasks we are focusing on. I mainly motivate LLM agents to provide daily assistance and facilitate daily life. Do you think we want to highlight sequential decision making?}
To fill this research gap and advance our understanding of user control over LLM agents, we carried out a quantitative empirical study. %to obtain empirical evidence.



% \ujcomment{- How do we position our work in this context?}

% \subsection{Human-agent collaboration}
% \glcomment{More traditional work of human-autonomous agent collaboration? Is it too far from our topic?}

% \subsection{User Trust and Reliance in Human-AI Collaboration}
% \label{sec-rel-trust}

\section{NumericBench Generation}
In this section, we present our created  NumericBench, which is specifically designed to evaluate fundamental numerical capabilities of LLMs. 
NumericBench consists of diverse datasets and tasks, 
enabling a systematic and comprehensive evaluation.
We discuss the datasets included in NumericBench, the key abilities it evaluates, and the methodology for benchmark generation.

\begin{table*}[t]
	\caption{NumericBench statistics. R: contextual retrieval, C: comparison, S: summary, L: logical reasoning. The token count is calculated based on tiktoken, which is the tokenizer used by Llama3~\cite{grattafiori2024llama3herdmodels}. The sentences used for token calculation include both the context and the question.}
	\centering
	\renewcommand{\arraystretch}{1.15} % 设置行间距为默认的 1.15 倍
	\setlength{\tabcolsep}{1.5pt} % 将列间距设置为 1pt
\resizebox{\textwidth}{!}{
	\begin{tabular}{c|c|c|c|c}
		\toprule
		\textbf{Data} & \textbf{Format} & \textbf{Questions} & \textbf{\# Instance} & \textbf{Avg Token} \\ \midrule
		
		\multirow{3}{*}{} 
		& \multirow{3}{*}{} 
		& \begin{tabular}[c]{@{}c@{}}R: What is the index of the first occurrence\\ of the number -3095 in the list?\end{tabular} 
		& 1000 & 3704.23 \\ \cline{3-5}
		
		\textbf{\begin{tabular}[c]{@{}c@{}}Number\\ List\end{tabular}}
		& $[69, -1, 6.1, \ldots, 5.7]$
		& \begin{tabular}[c]{@{}c@{}}C: Which index holds the smallest number\\
			 in the list between the indices 20 and 80?\end{tabular} 
		& 1000 & 3685.57  \\ \cline{3-5}
		
		& & \begin{tabular}[c]{@{}c@{}}S: What is the average of the index of\\
			 top 30 largest numbers in the list?\end{tabular} 
		& 1000 & 3654.78 \\ \midrule
		
		\multirow{3}{*}{} 
		& \multirow{3}{*}{
		\begin{tabular}[c]{@{}c@{}}
			\{date: 2024-06-19,\\
			close\_price: 9.79, \\
			open\_price: 9.4, \\
			\ldots \\
			PE\_ratio: 4.5416\}
		\end{tabular}
		} 
		& \begin{tabular}[c]{@{}c@{}}
			R: On which date did the close price\\
			 of stock firstly reach 61.76 yuan?
		\end{tabular}
		& 1000 & 27585.35 \\ \cline{3-5}
		
		\textbf{Stock}
		& 
		& \begin{tabular}[c]{@{}c@{}}
			C: Among the top-45 trading value days, which\\
			 date did the stock have the lowest close price?
		\end{tabular}
		 & 1000 & 27595.40 \\ \cline{3-5}
		
		& & \begin{tabular}[c]{@{}c@{}} 
			S: How many days had the close price higher than\\
			 the open price from 2024-07-31 to 2024-12-13?
		\end{tabular}	
		& 1000 & 27561.29 \\ \midrule
		
		\multirow{3}{*}{} 
		& \multirow{3}{*}{
		\begin{tabular}[c]{@{}c@{}}
			\{date: 2024-07-21,\\
			pressure\_msl: 999.96,\\
			dew\_point\_2m: 26.25,\\
			\ldots \\
			cloud\_cover: 61.5\}
		\end{tabular}
		} 
		& \begin{tabular}[c]{@{}c@{}} 
			R: On which date did the dew point temperature\\
			 at two meters firstly drop below 9.15°C?
		\end{tabular}
		& 1000 & 27359.26 \\ \cline{3-5}
		
		\textbf{Weather}
		& & \begin{tabular}[c]{@{}c@{}} 
			C: On which date did the MSL pressure reach its\\
			highest value when the cloud cover was below 9\%?
		\end{tabular}
		& 1000 & 27368.19 \\ \cline{3-5}
		
		& & \begin{tabular}[c]{@{}c@{}} 
			S: What was the average temperature at two meters\\
			when the relative humidity exceeded 78.56\%?
		\end{tabular}
		& 1000 & 27331.21 \\ \midrule
		
		\textbf{Sequence} 
		& $[0.34, 3, 6, \ldots, 111]$ 
		& L: What is the next number in the sequence? & 500 & 677.57 \\ \midrule
		
		\textbf{\begin{tabular}[c]{@{}c@{}}Arithmetic \\Operation\end{tabular}} 
		& \begin{tabular}[c]{@{}c@{}} 
		$a: 6.755,
		b: -1.225$
		\end{tabular}
		& \begin{tabular}[c]{@{}c@{}} 
		 $Q_{oper}$: What is the result of $a + b$?\\
		 $Q_{context}$: What is the result of $a $ plus $b$?
		 
		\end{tabular}
		& 12000 & 112.00 \\ \midrule
		
		\textbf{\begin{tabular}[c]{@{}c@{}}Mixed-number-string\\ Sequence\end{tabular}} 
		& \begin{tabular}[c]{@{}c@{}} 
		$effV2\ldots x98o7Lo$
		\end{tabular}
		& \begin{tabular}[c]{@{}c@{}} 
		How many numbers are there in the string? Note\\
		that a sequence like 'a243b' counts as a single number.
		\end{tabular}
		& 2000 & 196.53 \\ \bottomrule

	\end{tabular}
}
	\label{tab:data_stat}
	
\end{table*}

 

\subsection{Numeric Dataset Collection}
NumericBench offers a diverse collection of numerical datasets and questions designed to reflect real-world scenarios. 
This variety ensures that LLMs are thoroughly tested on their fundamental  abilities on numerical data.

\noindent\textbf{Number List Dataset.}
The synthetic number list dataset consists of simple collections of numerical values (integer and floats) 
presented as ordered or unordered lists.
Numbers in lists are one of the simplest and most fundamental data representations encountered in real-world scenarios.
Despite their simplicity, retrieving, indexing,  comparison, and summary on numbers can verify the fundamental numerical ability of LLMs. 
This dataset serves as a fundamental dataset of how well LLMs understand numerical values as discrete entities.



\noindent\textbf{Stock Dataset.}
The time-series  stock dataset is crawled from Eastmoney website~\cite{eastmoney}, 
which has eighteen attributes, such as stock close prices, open price,  trading volumes, and price-earnings ratio, over time.
Stock  data reflects dynamic, real-world numerical reasoning challenges that involve trend analysis, comparison, and decision-making under uncertainty,  representing real-world financial workflows.
 




\noindent\textbf{Weather Dataset.}
The weather dataset is crawled from Open-Meteo python API~\citep{openmeteo}, which includes data related to weather metrics, such as temperature, precipitation, humidity, and wind speed. 
The data is presented across various longitude and latitude.
 
 




\noindent\textbf{Numeric Sequence  Dataset.}
The synthetic numeric sequence dataset comprises sequences of numbers generated by arithmetic or geometric progression, complex patterns, or noisy inputs. 
Tasks require identifying patterns, predicting the next number, or reasoning about relationships between numbers.
Numerical sequences test the logical reasoning capabilities of LLMs, requiring pattern recognition and multi-step reasoning. This dataset introduces structured challenges that are common in mathematics and algorithmic reasoning.


 
\noindent\textbf{Arithmetic Operation Dataset.}
The dataset comprises 12,000 pairs of simple numbers, each undergoing addition, subtraction, multiplication, and division operations. Each pair of numbers, $a$ and $b$, consists of $k$-digit integers with three decimal places, where $k \in \{1, 2, \cdots, 6\}$. 
For each value of $k$, there are 2,000 pairs, evenly distributed across the four basic operations (i.e, $+, -,  *, /$), with 500 pairs per operation. 
This dataset is to evaluate the fundamental mathematical operation capabilities of LLMs, simulating the majority of mathematical calculation requirements in real-world scenarios.

\noindent\textbf{Mixed-number-string Sequence Dataset.}
The dataset consists of alphanumeric strings of varying lengths $\{50, 100, 150, 200\}$, each containing a randomized mix of letters and digits. For each string length, 500 samples are generated, resulting in a total of 2,000 samples. Each sample includes a query asking for the count of contiguous numeric sequences within the string, where sequences like "a243b" count as a single number. This dataset is designed to assess the ability of LLMs to identify and count numeric sequences.
 







\subsection{Fundamental Numerical Ability}
NumericBench is designed to comprehensively evaluate six fundamental numerical reasoning abilities of LLMs, which is 
%These three fundamental abilities are 
essential for solving real-world numeric-related tasks.
%such as numeric data summary and financial price analysis.


\noindent\textbf{Contextual Retrieval Ability.}
Contextual retrieval ability evaluates how well LLMs can locate, extract, and identify specific numerical values or their positions within structured or unstructured data. 
This includes tasks like finding a specific number in a list, retrieving values , and indexing numbers based on their order.
For example, as shown in Table~\ref{tab:data_stat}, it evaluates LLMs on tasks such as retrieving stock prices and identifying key values within numerical lists or domain-specific data (e.g., stock market and weather-related information).
This ability is fundamental to numerical reasoning because it forms the foundation for higher-order tasks, such as comparison, aggregation, and logical reasoning. 
 
 



\noindent\textbf{Comparison Ability.}
Comparison ability evaluates how well LLMs can compare numerical values to determine relationships such as greater than, less than, or equal to, and identify trends or differences in datasets. 
Comparison is vital for logical reasoning and decision-making, as many real-world tasks depend on accurate numerical evaluation. 
For instance,  as shown in Table~\ref{tab:data_stat},   comparing prices is essential in stock  for assessing performance, while weather forecasting requires analysis of temperature or precipitation trends over time. 
 



\noindent\textbf{Summary Ability.}
Summary ability assesses the LLM’s capacity to aggregate numerical data into concise insights, such as calculating totals, averages, or other statistical metrics. 
Summarization is critical for condensing large datasets into actionable information, enabling decision-making based on aggregated insights rather than raw data. 
This ability is indispensable in domains like electricity usage analysis, where summarizing hourly or daily consumption helps forecast bills, in business reporting for aggregating sales and revenue data to evaluate performance, 
and in healthcare analytics to monitor trends in patient metrics over time.



\noindent\textbf{Logic Reasoning Ability.}
Logical Reasoning Ability measures the LLM’s ability to perform multi-step operations involving numerical data, 
such as recognizing patterns, inferring rules, and applying arithmetic or geometric reasoning to solve complex problems. Logical reasoning extends beyond simple numerical tasks and reflects the LLM’s capacity for deeper, structured thinking. 
This ability is crucial for algorithm design, where solving problems involving numeric sequences or patterns is essential, in scientific research for identifying relationships and correlations in data.

\noindent\textbf{Arithmetic Operation Ability.}
It reflects the LLM's capacity to perform mathematical calculations accurately. Such ability is essential for tasks involving numerical computations, such as  automated machine learning through LLMs.





\noindent\textbf{Number Recognition  Ability.}
This measures the LLM's proficiency in identifying and interpreting numerical information within a given context. It represents a fundamental requirement for handling numeric-based tasks effectively.




\subsection{NumericBench Generation}
We use the number list, stock, and weather datasets to evaluate the contextual retrieval, comparison, and summary abilities of LLMs. 
Specifically, for each ability and each dataset, we prepare a set of questions designed to assess the corresponding target ability.
As shown in Table~\ref{appx:number_question}, Table~\ref{appx:stock_question}, and Table~\ref{appx:weather_question} in Appendix, there are nine question sets in total, covering three abilities across three datasets. 
When evaluating a specific ability (e.g., contextual retrieval) on a specific dataset (e.g., stock data), we randomly select one question from the corresponding question set for each data instance (e.g., a stock instance) 
and manually label the answer. This approach enables us to generate question-answer pairs for each ability on the number list, stock, and weather datasets.

For arithmetic operations and number counting in the strings dataset, the question format is straightforward, as illustrated in Table~\ref{tab:data_stat}. These questions are designed to evaluate the basic arithmetic operation and number recognition abilities of LLMs.



\section{Experiments}

\subsection{Experiment Setting}

\noindent\textbf{Benchmarks and Evaluated Protocols.}
The statistic of  NumericBench is provided in Table~\ref{tab:data_stat}.
Also,
we set the exact answer for mixed-number-string dataset, 
set
the computed answer to two decimal places for arithmetic datasets, and  set the answer of each question as a single choice (e.g., A, B, or C) for other datasets to reliable evaluate LLMs~\citep{bai2024longbench}.
The evaluation metric is accuracy.

\begin{table*}[t]
	\centering
		\vspace{-2em}
	\setlength\tabcolsep{2pt}
	\footnotesize
	\caption{Evaluation of LLMs on numerical contextual retrieval, comparison, and summary tasks across number list, stock, and weather datasets. 
		Also, * indicates that scores are calculated based on a short subset of outputs, as these models cannot handle  long contexts and exhibit disruption when tested on longer instances.}
	\begin{tabular}{c|ccc|ccc|ccc|c}
		\toprule
		\multirow{2}{*}{\textbf{Model}} & \multicolumn{3}{c|}{\textbf{Retrieval}}                                                         & \multicolumn{3}{c|}{\textbf{Comparison}}                                                           & \multicolumn{3}{c|}{\textbf{Summary}}                                                           & \textbf{Logic}    \\ \cmidrule{2-11} 
		
		& \multicolumn{1}{c}{\textbf{Number}} & \multicolumn{1}{c}{\textbf{Stock}} & \textbf{Weather} & \multicolumn{1}{c}{\textbf{Number}} & \multicolumn{1}{c}{\textbf{Stock}} & \textbf{Weather} & \multicolumn{1}{c}{\textbf{Number}} & \multicolumn{1}{c}{\textbf{Stock}} & \textbf{Weather} & \textbf{Sequence} \\ \midrule
		
		\textbf{\texttt{Random}} & \multicolumn{1}{c}{12.5}                  & \multicolumn{1}{c}{12.5}               &          12.5        & \multicolumn{1}{c}{12.5}                  & \multicolumn{1}{c}{12.5}               &                 12.5 & \multicolumn{1}{c}{12.5}                  & \multicolumn{1}{c}{12.5}               &      12.5            &         12.5          \\ \midrule
		
		
		\textbf{\texttt{Llama-3.1-8B-Inst}}& \multicolumn{1}{c}{22.8}                  & \multicolumn{1}{c}{14.4}               &      13.5          & \multicolumn{1}{c}{19.5}                  & \multicolumn{1}{c}{11.7}               &     13.7            & \multicolumn{1}{c}{18.1}                  & \multicolumn{1}{c}{13.8}               &       13.9*          &       18.2            \\  
		
		\textbf{\texttt{Llama-3.1-70B-Inst}}& \multicolumn{1}{c}{37.3}                  & \multicolumn{1}{c}{17.4}               &     23.0             & \multicolumn{1}{c}{28.3}                  & \multicolumn{1}{c}{15.0}               &     28.7             & \multicolumn{1}{c}{24.7}                  & \multicolumn{1}{c}{16.4}               &       15.2           &     17.8              \\  
		
		\textbf{\texttt{Llama-3.3-70B-Inst}}& \multicolumn{1}{c}{44.4}                  & \multicolumn{1}{c}{19.4}               &      23.1            & \multicolumn{1}{c}{31.5}                  & \multicolumn{1}{c}{13.8}               &       35.8           & \multicolumn{1}{c}{26.3}                  & \multicolumn{1}{c}{16.8}               &   18.0               &     18.6              \\  
		
		\textbf{\texttt{Llama-3.1-405B-Inst}}& \multicolumn{1}{c}{44.6}                  & \multicolumn{1}{c}{26.8}               &          19.8        & \multicolumn{1}{c}{25.1}                  & \multicolumn{1}{c}{14.8}               &     29.8             & \multicolumn{1}{c}{32.9}                  & \multicolumn{1}{c}{17.0}               &    16.1              &     16.6              \\  
		
		\textbf{\texttt{Llama-3.1-Nemotron-70B-Inst}}& \multicolumn{1}{c}{41.6}                  & \multicolumn{1}{c}{19.3}               &        24.9          & \multicolumn{1}{c}{26.6}                  & \multicolumn{1}{c}{13.7}               &      33.6            & \multicolumn{1}{c}{29.4}                  & \multicolumn{1}{c}{16.5}               &     17.0             &     16.4              \\  
		
		\textbf{\texttt{Qwen2.5-7B-Inst}}& \multicolumn{1}{c}{20.2}                  & \multicolumn{1}{c}{17.3}               &    19.6              & \multicolumn{1}{c}{24.8}                  & \multicolumn{1}{c}{17.8}               &      18.8            & \multicolumn{1}{c}{18.5}                  & \multicolumn{1}{c}{11.7}               &     13.8             &    14.4               \\  
		\textbf{\texttt{Qwen2.5-72B-Inst}}& \multicolumn{1}{c}{28.8}                  & \multicolumn{1}{c}{41.4*}               &       12.4*           & \multicolumn{1}{c}{28.0}                  & \multicolumn{1}{c}{26.0*}               &       31.0*           & \multicolumn{1}{c}{31.9}                  & \multicolumn{1}{c}{18.8*}               &        16.4*          &      19.0             \\  
		\textbf{\texttt{GLM-4-Long}}& \multicolumn{1}{c}{26.5}                  & \multicolumn{1}{c}{19.5}               &       8.4           & \multicolumn{1}{c}{18.9}                  & \multicolumn{1}{c}{14.8}               &      21.6            & \multicolumn{1}{c}{20.8}                  & \multicolumn{1}{c}{10.8 }               &      10.5            &        17.6           \\  
		
				\textbf{\texttt{Deepseek-V3}}& \multicolumn{1}{c}{47.2}                  & \multicolumn{1}{c}{47.5}               &       10.9          & \multicolumn{1}{c}{27.0}                  & \multicolumn{1}{c}{22.5}               &       35.8          & \multicolumn{1}{c}{21.8}                  & \multicolumn{1}{c}{13.0}               &       15.1          &   15.8                \\  
		
		\textbf{\texttt{GPT-4o}}& \multicolumn{1}{c}{41.7}                  & \multicolumn{1}{c}{37.5}               &        15.4          & \multicolumn{1}{c}{30.6}                  & \multicolumn{1}{c}{33.0}               &       64.2           & \multicolumn{1}{c}{11.6}                  & \multicolumn{1}{c}{17.4}               &      16.5            &        14.6           \\ 
		

		
		\midrule
		
		\textbf{\texttt{Human Evaluation}}& \multicolumn{1}{c}{\textbf{100}}                  &  \multicolumn{1}{c}{\textbf{100}}               &      \textbf{100}            & \multicolumn{1}{c}{\textbf{100}}                  & \multicolumn{1}{c}{\textbf{100}}               &        \textbf{100}          & \multicolumn{1}{c}{\textbf{100}}                  & \multicolumn{1}{c}{\textbf{100}}               &          \textbf{100}        &               \textbf{52.6}   \\ \bottomrule
	\end{tabular}
	\label{tab:main_experiments}
\end{table*}
\begin{figure*}[t]
		\vspace{-1em}
	\centering 	
	\subfloat[Contextual  Retrieval]	
	{\centering\includegraphics[width=0.33\linewidth]{image/main_fig/retrieval-num-list.pdf}}
	\hfill
	\subfloat[Comparison]
	{\centering\includegraphics[width=0.33\linewidth]{image/main_fig/compare-num-list.pdf}}
	\subfloat[Summary]	
	{\centering\includegraphics[width=0.33\linewidth]{image/main_fig/summary-num-list.pdf}}
	\hfill
	\caption{Evaluation on short and long context on number list.}
	\label{fig:length_number}
	
\end{figure*}
\noindent\textbf{Evaluated Models.}
To comprehensively evaluate the retrieval and reasoning abilities of state-of-the-art and widely-used LLMs on numeric data, 
we benchmark over 10 popular LLMs with our constructed NumericBench, as follows.
\begin{itemize}[leftmargin=*]
	\item \textbf{The Llama Series~\citep{grattafiori2024llama3herdmodels}.} include Llama-3.1-8B Instruct, Llama-3.1-70B Instruct, Llama-3.1-405B Instruct, 
	Llama-3.3-70B-Instruct and Llama-3.1-Nemotron-70B-Instruct.
	%Deepseek-R1~\citep{deepseekai2025deepseekr1incentivizingreasoningcapability}, Deepseek R1-Zero, Deepseek-V3~\citep{liu2024deepseek}, GLM-4-Plus~\citep{glm2024chatglm}, GLM-4-Long~\citep{glm2024chatglm}, Claude Sonnet 3.5, Claude 3.5 Haiku, GPT-4o, GPT-4o-mini, GPT-o3 mini, Gemini 2.0 Pro, Llama-3.1-8B/70B/405B Instruct~\citep{grattafiori2024llama3herdmodels}, Llama-3.3-8B/70B Instruct, Llama-3.1-Nemotron-70B-Instruct, Qwen2.5-7B/72B Instruct, InternLM2.5-20B-Chat
	\item \textbf{The Qwen Series~\citep{qwen2025qwen25technicalreport}.} include the effective Qwen2.5-7B-Instruct and Qwen2.5-72B-Instruct. 
	%	\item \textbf{Math-oriented Models} include DeepSeek-Math-Instruct 7B~\citep{deepseek-math}, MetaMath-Llemma-7B~\citep{azerbayev2023llemma}~\citep{yu2023metamath} Mammoth-7B/14B~\citep{yue2023mammoth}	
	\item \textbf{The GLM Series~\citep{glm2024chatglm}.} We use GLM4-Long to run the benchmark, since it is the commonly used in GLM series.
	% due to the overly high price of GLM4-Plus. 
	
	\item \textbf{The Deepseek Series~\citep{liu2024deepseek}~\citep{deepseekai2025deepseekr1incentivizingreasoningcapability}.} We currently use Deepseek V3 to run the benchmark. 
	Deepseek R1 will be evaluated in the future, since its API is down and unavailable now . 
	\item  \textbf{The GPT Series~\cite{achiam2023gpt}.} We use GPT-4o to run the benchmark. 
\end{itemize}

 
	We attempted to conduct experiments   on various math-oriented LLMs, such as Metamath-Llemma-7B~\citep{yu2023metamath}, Deepseek-Math-7B-instruct~\citep{deepseek-math}, InternLM2-Math-7B~\citep{ying2024internlmmathopenmathlarge} and MAmmoTH-7B~\citep{yue2023mammoth}.
	 However, these models fail during experiments for various reasons such as overly long output sequence length and limited input sequence length. Fail cases are demonstrated in the Figure~\ref{fig:fail_internlm},  Figure~\ref{fig:fail_ds_math},  Figure~\ref{fig:fail_llemma}, and  Figure~\ref{fig:fail_mammoth} in Appendix. 



\subsection{Main Experiments}
\noindent \textbf{Evaluation on Contextual Retrieval, Comparison, Summary, and Logic Reasoning Abilities.}
As shown in Table~\ref{tab:main_experiments}, 
current popular and effective LLMs perform poorly on basic numerical tasks, 
including retrieval, comparison, summarization, and logical reasoning. 
The random baseline for each task is 12.5\%, as there are 8 choices, and the probability of randomly selecting the correct answer is 1/8. 
Human evaluation was conducted by three undergraduate students. 

Firstly,
LLMs particularly struggle with accurately retrieving numerical data.
This limitation arises from LLMs treating numbers as discrete tokens rather than continuous ones, coupled with insufficient exposure to structured numerical datasets during training, which restricts their ability to handle simple numeric retrieval tasks. 
Secondly, LLMs demonstrate weaknesses in recognizing numerical relationships, such as greater-than or less-than comparisons, due to a lack of numerical semantics and underdeveloped arithmetic reasoning capabilities. 
Thirdly,
LLMs also perform poorly in summarizing numerical data (e.g., calculating sums or means), reflecting their inability to execute multi-step numerical operations. 
Similarly, logical reasoning tasks, especially those involving patterns or sequences, are particularly challenging, with all models scoring below 20\%. 
These tasks require multi-step reasoning, pattern recognition, and arithmetic operations, which expose the architectural limitations of current LLMs.









\begin{figure*}[t]
	\vspace{-2em}
	\centering 	
	\subfloat[Contextual  Retrieval]	
	{\centering\includegraphics[width=0.33\linewidth]{image/noisy_dataset_fig/retrieval-noisy-stock.pdf}}
	\hfill
	\subfloat[Comparison]
	{\centering\includegraphics[width=0.33\linewidth]{image/noisy_dataset_fig/compare-noisy-stock.pdf}}
	\subfloat[Summary]	
	{\centering\includegraphics[width=0.33\linewidth]{image/noisy_dataset_fig/summary-noisy-stock.pdf}}
	\hfill
 
	\caption{Evaluation on noisy stock dataset. Due to the input sequence length limit of Qwen2.5-72B-Inst on the API platform, the data containing 6 irrelevant attributes cannot be evaluated using this model.}
	\label{fig:noisy_stock}
	
\end{figure*}




\begin{figure*}[t]
	\vspace{-1em}
	\centering 	
	\subfloat[Accuracy on $Q_{oper}$  (i.e., $a+b$)]	
	{\centering\includegraphics[width=0.32\linewidth]{image/arithmetic_fig/arith_bar.pdf}}
	\hfill
	\subfloat[ $Q_{oper}$  of different digits]
	{\centering\includegraphics[width=0.32\linewidth]{image/arithmetic_fig/arith_plot.pdf}}
	\subfloat[Accuracy on $Q_{context}$  (i.e., $a$ plus $b$)]
	{\centering\includegraphics[width=0.32\linewidth]{image/arithmetic_text_fig/arith_bar.pdf}}
 
	\caption{Evaluation on arithmetic operation.}
	\label{fig:arithmetic_fig}
		\vspace{-1em}
\end{figure*}



 



 
\noindent \textbf{Evaluation on  Different Context Length via Stock and Weather Datasets.}
We evaluate LLMs on varying context lengths.
Specifically, we categorize the contexts of number lists, stock data, and weather data into short and long contexts.
The average token numbers for the short and long contexts across the three datasets are listed in Table~\ref{tab:data_stat_short_long}.
As illustrated in Figure~\ref{fig:length_number}, Figure~\ref{fig:length_stock}, and Figure~\ref{fig:length_weather},
LLMs generally achieve lower accuracy on long contexts compared to short contexts. This is because long contexts require the model to have a stronger ability to capture long-range dependencies.
Furthermore, if an LLM fails to perform well on short contexts, it is unlikely to achieve good results on long contexts. 
It highlights the importance of the inherent capabilities of LLMs in understanding numeric data.



\noindent \textbf{Evaluation on Noisy Context  via Stock and Weather Datasets.}
To evaluate the numerical abilities of LLMs in  noisy contexts, we add $k\in\{2,4,6\}$ irrelevant attributes to each instance in the stock and weather. 
These irrelevant attributes are not used in the user queries.
As shown in Figure~\ref{fig:noisy_stock} and Figure~\ref{fig:noisy_weather} in Appendix, 
as $k$ increases, most LLMs exhibit degraded performance. This indicates that irrelevant context can  affect the LLM's numerical retrieval and reasoning abilities.

 

\noindent \textbf{Evaluation on Arithmetic Operations}
Similarly, 
we evaluate five strong LLMs on arithmetic operations.
Specifically, as illustrated in Figure~\ref{fig:arithmetic_fig}~(a), even for simple arithmetic operations involving two numbers, LLMs fail to achieve 100\% accuracy. 
Moreover, as the number of digits increases shown in Figure~\ref{fig:arithmetic_fig}~(b), the accuracy of LLMs decreases, highlighting their limited ability to handle arithmetic tasks effectively, which is also observed in~\citep{qiu2024dissecting}.
This poor performance stems from how LLMs generate responses. LLMs  predict the highest-order digit  before the lower-order digit~\citep{zhang2024reverse}, contradicting the standard arithmetic logic of progressing from lower- to higher-order digits.
In particular, Figure~\ref{fig:arithmetic_fig}~(a) and (c) shows that LLMs perform similarly on addition, subtraction, and division operations but achieve extremely low accuracy on multiplication tasks.






\noindent \textbf{Evaluation on Number Recognition via Mixed-number-string Dataset.}
We evaluate the number recognition ability of effective LLMs by identifying numbers from mixed-number-string sequences. For this evaluation, we select five  effective LLMs based on Table~\ref{tab:main_experiments}, including DeepSeek-v3, GLM-4-Long, LLaMA3.1-405B, and Qwen2.5-72B.
As shown in Table~\ref{tab:number_counting}, all LLMs achieve extremely low accuracy in counting numbers within strings. Moreover, as the length of the string increases from 50 to 100, the accuracy of the LLMs decreases further.
These results highlight that LLMs are significantly weak at distinguishing numbers from strings. The underlying reason is that current LLMs treat numbers as strings during training. 
This training paradigm inherently limits their ability to understand and process numbers effectively.
Also, the tokenizer can split a single number into multiple tokens, which can negatively affect the numeric meaning of each number.








\begin{table}[]
	\centering
	\small
	\caption{Evaluation on mixed-number-string data with lengths ranging from 50 (i.e., 50 L) to 200.}
	
	%	\footnotesize
	\begin{tabular}{c|cccc}
		\toprule
		\textbf{Model}    & \textbf{50 L} & \textbf{100 L} & \textbf{150 L} & \textbf{200 L} \\ \midrule
		
		
		\textbf{\texttt{LLama3.1-405B }}& 10.8      & 9.2        & 3.2        & 2.2        \\  
		
		\textbf{\texttt{Qwen2.5-72B}}   & 3.0         & 1.2        & 0.6        & 0.2        \\  
		
		\textbf{\texttt{GLM4-Long}  }   & 6.6       & 4.8        & 3.0          & 2.4        \\  
		
		\textbf{\texttt{GPT-4o }}       & 18.2      & 6.4        & 4.0          & 4.2        \\ 
		
		\textbf{\texttt{DeepSeek-V3}}   & 13.2      & 4.0          & 3.2        & 2.0          \\  
		\midrule
		\textbf{\texttt{Human Eval } }   & \textbf{100}      & \textbf{100}        & \textbf{100}         & \textbf{100}        \\ \bottomrule
	\end{tabular}
	\label{tab:number_counting}
\end{table}
\subsection{Discussions on Numeracy Gaps of LLMs}
In summary, extensive experimental results show that current state-of-the-art LLMs perform poorly on six fundamental numerical abilities.
% such as number recognition and arithmetic operations. 
Here we discuss five potential reasons behind their poor performance on numerical tasks.

\noindent \textbf{Tokenizer Limitation.}
LLMs use tokenizers to split input text into smaller units (tokens). Thus,
Numbers are split into chunks as strings, based on statistical patterns in the training data.
For example, $10000$ is split into $100$ and $00$ tokens\footnote{\url{https://gptforwork.com/tools/tokenizer}}.
These tokenizers do not considering  the real meaning of numbers and continuous magnitude of numbers.
Thus, LLMs do not perform well on simple number retrieval and comparison tasks.

\noindent \textbf{Training Corpora Limitation.}
LLMs are trained on extensive corpora, which also limits their ability to understand numerical-related symbols, such as $*$.
For example, the multiplication of 246 and 369 can be denoted as $246*369$.
However, $246*369$ may be interpreted as a password or encrypted text, since $*$ in text strings is often associated with encryption.
As a result, enabling LLMs to accurately interpret arithmetic symbols remains an open problem.


\noindent \textbf{Training Paradigm Limitation.}
The training of LLMs relies on the next-token prediction paradigm, which is inherently misaligned with the logic of numerical computation.
For example, when solving $16 + 56$ with the result being $72$, an LLM will first predict the highest-order digit of the answer (i.e., $7$) before predicting the lower-order digit (i.e., $2$). This approach contradicts the fundamental logic of arithmetic computation, which typically proceeds from the lower-order digit to the higher-order digit.
This discrepancy implies that LLMs effectively need to know the entire result upfront to generate digits sequentially in the correct order. As a result, LLMs struggle to perform well even on simple arithmetic operations.

\noindent \textbf{Positional Embedding Limitation.}
Note that LLMs incorporate positional embeddings for  tokens in sequence inputs. In arithmetic operations like $12 + 26$ and $26 + 12$, the order of the numbers does not affect the result. However, LLMs assign different positional embeddings to the number $12$ in each equation, as its position in the sequence differs. 
This lack of invariance in positional embeddings for numbers can influence the results.
Therefore, how to design the positional embedding that improves numerical ability of LLMs without affecting the text understanding  of LLMs is critical~\cite{mcleish2024transformers,golovneva2024contextual}.



\noindent \textbf{Transformer Architecture Limitation.}
LLMs use Transformer to process input sequence, which rely on pattern recognition rather than explicit algorithmic reasoning.
The computational power of transformers has upper bounds~\cite{merrill2023parallelism}. Considering the complexity of arithmetic operations in real-world applications, it still needs to be theoretically investigated whether transformers can perform well on numerical operations.

\section{Conclusion}
In this work, we introduce \gls{myrag}, an advanced \gls{qa} system that dynamically constructs \gls{mkg} while integrating sophisticated reasoning and external domain-specific search tools. The model exhibits significant improvements in accuracy and reasoning capabilities, particularly for medical question-answering tasks, outperforming other approaches of similar model size or 10 to 100 times larger. Using structured knowledge representations and advanced reasoning frameworks, our approach establishes a new benchmark for \gls{qa} in highly competitive and highly evolving domains such as medicine.

\section{Limitations}

Despite \gls{myrag} advancements, our approach has certain limitations. Firstly it relies on external search tools to introduce latency during the creation of \gls{mkg}. However, this occurs only once, when the \gls{mkg} is built from scratch for the first time. Additionally, while the model performs exceptionally well in medical domains, its applicability to non-medical tasks remains unexplored. 


Another limitation is the need for structured, authoritative sources of medical knowledge. Currently, \gls{myrag} retrieves information from diverse sources, including research articles and medical textbooks. However, as emphasized in clinical decision-making, treatment guidelines serve as essential references for standardized diagnosis and treatment protocols \cite{hager2024evaluation}. Future work on \gls{myrag} should focus on integrating structured access to these sources to ensure compliance with evidence-based medicine.


\section{Ethics Statement}
The development of \gls{llm}s for medical \gls{qa} requires careful ethical consideration due to risks of inaccuracy and bias. Ensuring the reliability of retrieved content is crucial, especially when integrating external knowledge sources. To mitigate these risks, we implement a confidence scoring mechanism into the \gls{mkg} to validate the information. However, bias detection and mitigation remain active research areas.

\section*{Acknowledgements}
% \section*{References}

\bibliography{main}
\newpage
\appendix
\clearpage
 
\definecolor{exampleblue}{RGB}{0, 114, 188} % Blue for header
\definecolor{exampleborder}{RGB}{0, 114, 188} % Blue for border
\definecolor{redtext}{RGB}{204, 0, 0}         % Red text for emphasis

\section{Appendix}
In this appendix, we provide additional details about the design of \textbf{NumericBench}, along with supplementary experimental results and case studies. The organization of the supplementary materials in this appendix is as follows:

\begin{enumerate}[leftmargin=*]
	
	\item \textbf{Question formats for contextual retrieval, comparison, and summary abilities.}  
As shown in Table~\ref{appx:number_question}, Table~\ref{appx:stock_question}, and Table~\ref{appx:weather_question}, 
we designed diverse question types tailored to each dataset to evaluate the three fundamental numerical abilities of LLMs: contextual retrieval, comparison, and summary. contextual retrieval  assesses the model’s capacity to accurately extract relevant numerical information from complex contexts; comparison tests the ability to analyze and compare numerical values;  Summary evaluates the synthesis of numerical information into concise and meaningful insights for tasks like reporting or trend analysis.


By designing tailored questions for each dataset, we ensure a comprehensive evaluation of LLMs’ numerical reasoning abilities across varying scenarios and complexities.
	\item \textbf{Basic numerical questions answered incorrectly by GPT-4o.}  
	As illustrated in Figure~\ref{fig:number_compare}, Figure~\ref{fig:multiplication}, and Figure~\ref{fig:number_couting}, GPT-4o failed to answer three basic numerical questions correctly. This result is surprising, considering GPT-4o's impressive performance in real-world applications. However, these findings highlight the weak fundamental numerical abilities of LLMs.
	
	\item \textbf{Token counts for short and long contexts.}  
	As shown in Table~\ref{tab:data_stat_short_long}, the token counts of long and short contexts differ significantly. This distinction enables a more thorough evaluation of LLM performance across scenarios involving varying context lengths. Short contexts are designed to test the model's ability to process and understand concise information, focusing on immediate comprehension and reasoning. In contrast, long contexts present a more complex challenge, requiring the model to handle extended sequences of information, maintain coherence over a larger context window, 
	and retrieve relevant details from earlier parts of the input. Such two type length can more comprehensively evaluate LLMs. 
	
	\item \textbf{Additional experimental results on noisy and varying-length contexts.}  
	As shown in Figure~\ref{fig:length_stock} and Figure~\ref{fig:length_weather}, existing LLMs perform poorly on the stock and weather datasets, although they achieve better performance compared to their results on short contexts. 
	Similarly, as shown in Figure~\ref{fig:noisy_weather}, LLMs perform poorly on noisy weather data.
 
 \item \textbf{Real failure cases of math-oriented LLMs.} In this paper, we do not compare existing math-oriented LLMs, such as Metamath-Llemma-7B~\citep{yu2023metamath}, Deepseek-Math-7B-Instruct~\citep{deepseek-math}, InternLM2-Math-7B~\citep{ying2024internlmmathopenmathlarge}, and MAmmoTH-7B~\citep{yue2023mammoth}. 
 This is primarily because these math-oriented LLMs are designed for specialized geometric and structured mathematical problems. They are unable to understand the tasks in NumericBench, fail to follow a correct reasoning process, and directly produce meaningless outputs. These failure cases are illustrated in Figure~\ref{fig:fail_internlm}, Figure~\ref{fig:fail_ds_math}, Figure~\ref{fig:fail_llemma}, and Figure~\ref{fig:fail_mammoth}.
 
\end{enumerate}

\noindent \textbf{The Use of AI Tools.} When writing  this paper, we use Grammarly\footnote{https://www.grammarly.com/} for automated spell checking and use GPT-4o\footnote{https://platform.openai.com/docs/models/gpt-4o} to refine several sentences.


\clearpage
 
 

\begin{table*}[!h]
	\centering
	
	\caption{Question format on number list dataset. R: contextual retrieval, C: comparison, S: summary. In the contextual retrieval task, a number $x$ is randomly selected from the given number list. For the comparison task, the $k$-th largest number is randomly generated within the range of one to the length of the number list. The indices $x$ corresponds to twenty percent of the length of the number list, while $y$ corresponds to eighty percent of the length. The number $z$ is randomly chosen within the range $(\min(\text{list}), \max(\text{list}))$. For the summary task, the top $k$ is set to thirty percent of the length of the number list.}
 
	\renewcommand{\arraystretch}{1.15}  
	\setlength{\tabcolsep}{1.5pt}  
	\begin{tabular}{c|c}
		\toprule
		\textbf{Ability}    & \textbf{Question Format} \\ \midrule
		\textit{\textbf{R}} &  \begin{tabular}[c]{@{}l@{}}
			$Q_0$: What is the index of the first occurrence of the number $x$ in the list?\\
			$Q_1$: What is the index of the last occurrence of the number $x$ in the list?\\
			$Q_2$: What is the number after the first occurrence of the number $x$ in the list?\\
			$Q_3$: What is the number before the last occurrence of the number $x$ in the list?\\
			$Q_4$: What is the index of the first even number in the list?\\
			$Q_5$: What is the index of the first odd number in the list?\\
			$Q_6$: What is the index of the last even number in the list?\\
			$Q_7$: What is the index of the last odd number in the list?
		\end{tabular} \\ \midrule
		\textit{\textbf{C}} &  \begin{tabular}[c]{@{}l@{}}
			$Q_8$: What is the index of the first occurrence of the $k$-th largest number in the given list?\\
			$Q_9$: Which index holds the greatest number in the list between the indices $x$ and $y$?\\
			$Q_{10}$: Which index holds the smallest number in the list between the indices $x$ and $y$?\\
			$Q_{11}$: Which number is closest to $z$ in the list between the indices $x$ and $y$?\\
			$Q_{12}$: Which number is furthest from $z$ in the list between the indices $x$ and $y$?\\
			$Q_{13}$: Which number is the largest among those less than $z$ in the list?\\
			$Q_{14}$: Which number is the smallest among those greater than $z$ in the list?
		\end{tabular} \\ \midrule
		\textit{\textbf{S}} &  \begin{tabular}[c]{@{}l@{}}
			$Q_{15}$: What is the maximum sum of any two consecutive items in the list?\\
			$Q_{16}$: What is the maximum sum of any three consecutive items in the list?\\
			$Q_{17}$: What is the maximum absolute difference between two consecutive items in the list?\\
			$Q_{18}$: What is the sum of the indices of the top $k$ largest numbers in the list?\\
			$Q_{19}$: What is the sum of the indices of the top $k$ smallest numbers in the list?\\
			$Q_{20}$: What is the average of the indices of the top $k$ largest numbers in the list?\\
			$Q_{21}$: What is the average of the indices of the top $k$ smallest numbers in the list?\\
			$Q_{22}$: How many times do numbers consecutively increase for more than five times?\\
			$Q_{23}$: How many times do numbers consecutively decrease for more than five times?\\
			$\cdots \cdots$ \\
		\end{tabular} \\ \bottomrule
	\end{tabular}	
	\label{appx:number_question}
\end{table*}
\clearpage


 

\begin{table*}[]
	
	\caption{Question format on stock dataset. R: contextual retrieval, C: comparison, S: summary. $x$ and $y$ lie within the minimum and maximum ranges of their respective attributes. The top $k$ corresponds to thirty percent of the number list. $date_1$ represents the day at the twentieth percentile of the stock history, while $date_2$ corresponds to the day at the eightieth percentile.}
	\centering
	\renewcommand{\arraystretch}{1.15} % 设置行间距为默认的 1.15 倍
	\setlength{\tabcolsep}{1.5pt} % 将列间距设置为 1pt
	\begin{tabular}{c|c}
		\toprule
		\textbf{Ability}    & \textbf{Question Format} \\ \midrule
		\textit{\textbf{R}} &  \begin{tabular}[c]{@{}l@{}}
			$Q_0$: On which date did the close price of the stock first reach $x$ yuan?\\
			$Q_1$: On which date did the highest price of the stock first reach $x$ yuan?\\
			$Q_2$: On which date did the volume of the stock first reach $x$ lots?\\
			$Q_3$: On which date did the value of the stock first reach $x$ thousand yuan?\\
			$Q_4$: On which date did the price change rate of the stock first reach $x$\%?\\
			$Q_5$: On which date did the price change of the stock first reach $x$ yuan?\\
		\end{tabular} \\ \midrule
		\textit{\textbf{C}} &  \begin{tabular}[c]{@{}l@{}}
			\begin{tabular}[c]{@{}l@{}}
				$Q_6$: On which date did the stock have the highest turnover rate when the close \\price was greater than $x$ yuan?
			\end{tabular}\\
			
			\begin{tabular}[c]{@{}l@{}}
				$Q_7$: On which date did the stock have the highest quantity relative ratio when \\the open price was less than $x$ yuan?
			\end{tabular}\\
			
			\begin{tabular}[c]{@{}l@{}}
				$Q_8$: On which date did the stock have the highest difference between the highest \\and lowest prices when the trading volume exceeded $x$ lots?
			\end{tabular}\\
			
			\begin{tabular}[c]{@{}l@{}}
				$Q_9$: On which date did the stock record the highest daily average price, calculated \\as 'value' divided by 'volume,' when the PE ratio was less than $x$?
			\end{tabular}\\
			
			\begin{tabular}[c]{@{}l@{}}
				$Q_{10}$: Among the top-$k$ trading value days, on which date did the stock have the \\lowest close price?
			\end{tabular}\\
			
			\begin{tabular}[c]{@{}l@{}}
				$Q_{11}$: When the quantity relative ratio exceeded $x$, on which date did the stock \\have the highest sum of the open price and close price?
			\end{tabular}\\
			
			\begin{tabular}[c]{@{}l@{}}
				$Q_{12}$: When the absolute price change rate exceeded $x$\%, on which date did the \\stock have the highest difference between the highest and lowest prices?
			\end{tabular}
		\end{tabular} \\ \midrule
		\textit{\textbf{S}} &  \begin{tabular}[c]{@{}l@{}}
			\begin{tabular}[c]{@{}l@{}}
				$Q_{13}$: How many days had a volume greater than $x$ from $date_1$ to $date_2$?
			\end{tabular}\\
			
			\begin{tabular}[c]{@{}l@{}}
				$Q_{14}$: How many days had the close price higher than the open price from \\$date_1$ to $date_2$?
			\end{tabular}\\
			
			\begin{tabular}[c]{@{}l@{}}
				$Q_{15}$: How many days had a close price higher than the open price, with the \\quantity relative ratio exceeding $x$\%?
			\end{tabular}\\
			
			\begin{tabular}[c]{@{}l@{}}
				$Q_{16}$: How many days had the close price reach $x$ yuan with the absolute price \\change rate exceeding $x$\%?
			\end{tabular}\\
			
			\begin{tabular}[c]{@{}l@{}}
				$Q_{17}$: What was the average trading volume when both the turnover rate \\exceeded $x$\% and the price change rate was greater than $y$\%?
			\end{tabular}\\
			
			\begin{tabular}[c]{@{}l@{}}
				$Q_{18}$: Excluding non-trading days, how many times did the open price of \\the stock rise for three or more consecutive days?
			\end{tabular}\\
			
			\begin{tabular}[c]{@{}l@{}}
				$Q_{19}$: Excluding non-trading days, how many times did the close price of \\the stock rise for three or more consecutive days?
			\end{tabular}\\
			
			\begin{tabular}[c]{@{}l@{}}
				$Q_{20}$: Excluding non-trading days, how many times did the open price and \\close price of the stock both rise for three or more consecutive days?
			\end{tabular}\\
		
			\begin{tabular}[c]{@{}l@{}}
			$\cdots \cdots$
		\end{tabular}
		
		\end{tabular} \\ \bottomrule
	\end{tabular}
\label{appx:stock_question}
\end{table*}
\clearpage
 

\begin{table*}[]
	\centering
	\caption{Question format on weather dataset.  R: contextual retrieval, C: comparison, S: summary. The value of $x$ falls within the minimum and maximum ranges of its respective attribute. $date_1$ represents the day at the twentieth percentile of the stock history, while $date_2$ represents the day at the eightieth percentile.}
	\begin{tabular}{c|c}
		\toprule
		\textbf{Ability}    & \textbf{Question Format} \\ \midrule
		\textit{\textbf{R}} &  \begin{tabular}[c]{@{}l@{}}
			$Q_0$: On which date did the temperature at two meters first reach $x$°C?\\
			$Q_1$: On which date did the relative humidity at two meters first exceed $x$\%?\\
			$Q_2$: On which date did the dew point temperature at two meters first drop below $x$°C?\\
			$Q_3$: On which date did the precipitation first exceed $x$ mm?\\
			$Q_4$: On which date did the sea-level air pressure first exceed $x$ hPa?\\
			$Q_5$: On which date did the cloud cover first reach $x$\%?\\
			$Q_6$: On which date did the wind speed at 10 meters first exceed $x$ m/s?
		\end{tabular} \\ \midrule
		\textit{\textbf{C}} &  \begin{tabular}[c]{@{}l@{}}
			\begin{tabular}[c]{@{}l@{}}
				$Q_7$: On which date did the temperature at two meters reach its highest value \\ 
				when the relative humidity was below $x$\%? 
			\end{tabular} \\
			
			\begin{tabular}[c]{@{}l@{}}
				$Q_8$: On which date did the relative humidity at two meters reach its lowest value \\ 
				when the temperature at two meters was above $x^\circ$C?
			\end{tabular} \\
			
			\begin{tabular}[c]{@{}l@{}}
				$Q_9$: On which date did the difference between the temperature and dew point \\ 
				at two meters reach its maximum when the cloud cover was below $x$\%? 
			\end{tabular} \\
			
			\begin{tabular}[c]{@{}l@{}}
				$Q_{10}$: On which date did the precipitation reach its highest value \\ 
				when the temperature at two meters was below $x^\circ$C? 
			\end{tabular} \\
			
			\begin{tabular}[c]{@{}l@{}}
				$Q_{11}$: On which date did the cloud cover reach its lowest value \\ 
				when the wind speed at 10 meters exceeded $x$ m/s? 
			\end{tabular} \\
			
			\begin{tabular}[c]{@{}l@{}}
				$Q_{12}$: On which date did the wind speed at 10 meters reach its highest value \\ 
				when the precipitation exceeded $x$ mm? 
			\end{tabular} \\
			
			\begin{tabular}[c]{@{}l@{}}
				$Q_{13}$: On which date did the sea-level air pressure reach its highest value \\ 
				when the cloud cover was below $x$\%? 
			\end{tabular}
		\end{tabular} \\ \midrule
		\textit{\textbf{S}} &  \begin{tabular}[c]{@{}l@{}}
			\begin{tabular}[c]{@{}l@{}}
				$Q_{14}$: How many days had a temperature at two meters greater than $x^\circ$C \\from $date_1$ to $date_2$? 
			\end{tabular} \\
			
			\begin{tabular}[c]{@{}l@{}}
				$Q_{15}$: How many days had a relative humidity at two meters exceeding $x$\% \\from $date_1$ to $date_2$? 
			\end{tabular} \\
			
			\begin{tabular}[c]{@{}l@{}}
				$Q_{16}$: How many days had a precipitation greater than $x$ mm from $date_1$ \\to $date_2$? 
			\end{tabular} \\
			
			\begin{tabular}[c]{@{}l@{}}
				$Q_{17}$: What was the average temperature at two meters when the relative \\humidity exceeded $x$\%? 
			\end{tabular} \\
			
			\begin{tabular}[c]{@{}l@{}}
				$Q_{18}$: What was the average wind speed at 10 meters when the precipitation \\exceeded $x$ mm? 
			\end{tabular} \\
			
			\begin{tabular}[c]{@{}l@{}}
				$Q_{19}$: How many times did the temperature at two meters rise for three or more \\consecutive days? 
			\end{tabular} \\
			
			\begin{tabular}[c]{@{}l@{}}
				$Q_{20}$: How many times did the relative humidity at two meters drop for \\three or more consecutive days? 
			\end{tabular} \\
		
					
		\begin{tabular}[c]{@{}l@{}}
			$\cdots \cdots$
		\end{tabular} \\
	
		\end{tabular} \\ \bottomrule
	\end{tabular}
\label{appx:weather_question}
\end{table*}

\clearpage
\begin{figure*}[t]
	\centering	
	\vspace{-1em}
	\frame{
		\includegraphics[width = 0.9\textwidth]{image/intro_example/number_compare.png}
	}
	%	\captionsetup{labelformat=empty}
	%	\addtocounter{figure}{-1}
	\caption{Number comparisons on GPT-4o. The correct answer is -9.11. }
	\label{fig:number_compare}
\end{figure*}

\begin{figure*}[t]
	\centering	
	\vspace{-1em}
	\frame{
		\includegraphics[width = 0.9\textwidth]{image/intro_example/multiplication.png}
	}
	%	\captionsetup{labelformat=empty}
	%	\addtocounter{figure}{-1}
	\caption{Number multiplication on GPT-4o. The correct answer is 102244.12. }
	\label{fig:multiplication}
\end{figure*}
\begin{figure*}[t]
	\centering	
	\vspace{-1em}
	\frame{
		\includegraphics[width = 0.9\textwidth]{image/intro_example/number_counting.jpg}
	}
	%	\captionsetup{labelformat=empty}
	%	\addtocounter{figure}{-1}
	\caption{Number counting on GPT-4o, which is required directly give answer. The correct answer is 4. }
	\label{fig:number_couting}
\end{figure*}

\clearpage


%\subsection{Additional Experiment Results}
%\subsubsection{Additional results on context length evaluation for stock and weather data}\label{appx:sssec:length}
		\begin{figure*}[t]
		\centering 	
		\subfloat[Contextual Retrieval]	
		{\centering\includegraphics[width=0.33\linewidth]{image/main_fig/retrieval-stock.pdf}}
		\hfill
		\subfloat[Comparison]
		{\centering\includegraphics[width=0.33\linewidth]{image/main_fig/compare-stock.pdf}}
		\subfloat[Summary]	
		{\centering\includegraphics[width=0.33\linewidth]{image/main_fig/summary-stock.pdf}}
		\hfill
		%	\subfloat[MUTAG]
		%	{\centering\includegraphics[width=0.25\linewidth, height=3.05cm]{image/g1-4.pdf}}	
		%	\hfill
		%	
		\caption{Evaluation on short and long context on stock dataset. Due to the input sequence length limit of Qwen2.5-72B-Inst on the API platform, the long dataset of all three abilities cannot be evaluated using this model.}
		\label{fig:length_stock}
	\end{figure*}
	
	
	
	\begin{figure*}[t]
		\centering 	
		\subfloat[Contextual Retrieval]	
		{\centering\includegraphics[width=0.33\linewidth]{image/main_fig/retrieval-weather.pdf}}
		\hfill
		\subfloat[Comparison]
		{\centering\includegraphics[width=0.33\linewidth]{image/main_fig/compare-weather.pdf}}
		\subfloat[Summary]	
		{\centering\includegraphics[width=0.33\linewidth]{image/main_fig/summary-weather.pdf}}
		\hfill
		%	\subfloat[MUTAG]
		%	{\centering\includegraphics[width=0.25\linewidth, height=3.05cm]{image/g1-4.pdf}}	
		%	\hfill
		%	
		\caption{Evaluation on short and long context on weather dataset. Due to the input sequence length limit of Qwen2.5-72B-Inst on the API platform, the long dataset of all three abilities cannot be evaluated using this model.}
		\label{fig:length_weather}
		
	\end{figure*}

		\begin{figure*}[t]
		
		\centering 	
		\subfloat[Contextual  Retrieval]	
		{\centering\includegraphics[width=0.33\linewidth]{image/noisy_dataset_fig/retrieval-noisy-weather.pdf}}
		\hfill
		\subfloat[Comparison]
		{\centering\includegraphics[width=0.33\linewidth]{image/noisy_dataset_fig/compare-noisy-weather.pdf}}
		\subfloat[Summary]	
		{\centering\includegraphics[width=0.33\linewidth]{image/noisy_dataset_fig/summary-noisy-weather.pdf}}
		\hfill
		%	\subfloat[MUTAG]
		%	{\centering\includegraphics[width=0.25\linewidth, height=3.05cm]{image/g1-4.pdf}}	
		%	\hfill
		%	
		\caption{Evaluation on  noisy weather dataset. Due to the input sequence length limit of Qwen2.5-72B-Inst on the API platform, the data containing 4 and 6 irrelevant attributes cannot be evaluated using this model.}
		\label{fig:noisy_weather}
		
	\end{figure*}
\clearpage
	
	
	
	
%	\begin{figure*}[t]
%		
%		\centering 	
%		\subfloat[Context Retrieval]	
%		{\centering\includegraphics[width=0.33\linewidth]{image/multi_dataset_fig/retrieval-multi-stock.pdf}}
%		\hfill
%		\subfloat[Comparison]
%		{\centering\includegraphics[width=0.33\linewidth]{image/multi_dataset_fig/compare-multi-stock.pdf}}
%		\subfloat[Summary]	
%		{\centering\includegraphics[width=0.33\linewidth]{image/multi_dataset_fig/summary-multi-stock.pdf}}
%		\hfill
%		%	\subfloat[MUTAG]
%		%	{\centering\includegraphics[width=0.25\linewidth, height=3.05cm]{image/g1-4.pdf}}	
%		%	\hfill
%		%	
%		\caption{Evaluation on multi-turn QA on stock dataset. Due to the input sequence length limit of Qwen2.5-72B-Inst on the API platform, the model cannot generate outputs in the third turn of the conversation. }
%		\label{fig:multurn_stock}
%		
%	\end{figure*}
	
	\begin{table*}[]
		\caption{The average token number on short and long instances for each data.}
		\centering
		\begin{tabular}{c|c|cc|cc}
			\toprule
			\multirow{2}{*}{\textbf{Dataset}}                                               & \multirow{2}{*}{\textbf{Ability}} & \multicolumn{2}{c|}{\textbf{Short}}                            & \multicolumn{2}{c}{\textbf{Long}}                             \\ \cmidrule{3-6} 
			
			&                                   & \multicolumn{1}{c|}{\textbf{\# Instance}} & \textbf{Avg Token} & \multicolumn{1}{c|}{\textbf{\# Instance}} & \textbf{Avg Token} \\ \midrule
			
			\multirow{3}{*}{\textbf{\begin{tabular}[c]{@{}c@{}}Number\\ List\end{tabular}}} & \textit{Contextual Retrieval}

                  & \multicolumn{1}{c|}{500}                     &        809.12     & \multicolumn{1}{c|}{500}                     &         6599.34      \\   
			
			& \textit{Comparison}                        & \multicolumn{1}{c|}{500}                     &     804.86     & \multicolumn{1}{c|}{500}                     &        6566.27      \\ 
			
			& \textit{Summary}



                  & \multicolumn{1}{c|}{500}                     &       822.49      & \multicolumn{1}{c|}{500}                     &       6487.07       \\ \midrule
			
			\multirow{3}{*}{\textbf{Stock}}                                                 & \textit{Contextual Retrieval}

                  & \multicolumn{1}{c|}{500}                     &        18529.07      & \multicolumn{1}{c|}{500}                     &      36641.63     \\  
			& \textit{Comparison}                        & \multicolumn{1}{c|}{500}                     &    18539.58     & \multicolumn{1}{c|}{500}                     &      36651.22      \\ 
			& \textit{Summary}

                  & \multicolumn{1}{c|}{500}                     &      18504.51      & \multicolumn{1}{c|}{500}                     &       36618.07      \\ \midrule
			
			\multirow{3}{*}{\textbf{Weather}}                                               & \textit{Contextual Retrieval}

                  & \multicolumn{1}{c|}{500}                     &        18362.38        & \multicolumn{1}{c|}{500}                     &        36356.13    \\  
			& \textit{Comparison}                        & \multicolumn{1}{c|}{500}                     &        18371.11    & \multicolumn{1}{c|}{500}                     &       36365.27     \\ 
			& \textit{Summary}



                  & \multicolumn{1}{c|}{500}                     &        18334.15     & \multicolumn{1}{c|}{500}                     &        36328.27    \\ \bottomrule
		\end{tabular}
		\label{tab:data_stat_short_long}
	\end{table*}
	\clearpage
%	\begin{figure*}[t]
%		
%		\centering 	
%		\subfloat[Context Retrieval]	
%		{\centering\includegraphics[width=0.33\linewidth]{image/multi_dataset_fig/retrieval-multi-weather.pdf}}
%		\hfill
%		\subfloat[Comparison]
%		{\centering\includegraphics[width=0.33\linewidth]{image/multi_dataset_fig/compare-multi-weather.pdf}}
%		\subfloat[Summary]	
%		{\centering\includegraphics[width=0.33\linewidth]{image/multi_dataset_fig/summary-multi-weather.pdf}}
%		\hfill
%		%	\subfloat[MUTAG]
%		%	{\centering\includegraphics[width=0.25\linewidth, height=3.05cm]{image/g1-4.pdf}}	
%		%	\hfill
%		%	
%		\caption{Evaluation on multi-turn QA on weather dataset.}
%		\label{fig:multurn_weather}
%		
%	\end{figure*}
	

	
	
%\noindent\textbf{Case Study}
	\newtcolorbox[auto counter, number within=section]{examplebox}[2][]{%
		colback=white,        % Background color
		colframe=exampleborder, % Border color
		coltitle=white,       % Header text color
		fonttitle=\bfseries,  % Header font
		colbacktitle=exampleblue, % Header background color
		title=Example~\thetcbcounter: #2, % Title format
	}
	\begin{figure*}
		\begin{examplebox}{InternLM2-Math-7B}
			\textbf{Failure:} \\Model repeatedly output the same number until the output sequence ends.
			
			\vspace{0.2cm}
			\textbf{User:} \\ Question: What is the maximum sum of any three consecutive items in the list? Options: A: 21501.9737231253, B: 21502.8144848529, C: 21501.9642201794, D: 21502.7354388437, E: 21502.0327435643, F: 21501.7526575621, G: 21501.7057960155, H: 21502 Data: [-5279, ....(100 numbers)]
			
			\vspace{0.3cm}
			\textbf{InternLM2-Math-7B Model:} \\
			Step 1: To find the maximum sum of any three consecutive items in the list, we need to identify the three largest numbers in the list and add them together. <PRM +> \\Step 2: We can start by sorting the list in ascending order to make it easier to identify the largest numbers. <PRM +> \\Step 3: After sorting, the list becomes: [-8509, -8509, -8509, -8509, -8509, -8509, -8509, -8509, -8509, -8509, -8509, -8509, -8509, -8509, -8509, -8509, -8509, -8509, -8509, -8509, -8509, -8509, -8509, -8509, -8509, -8509, -8509, -8509...(output too long)
		\end{examplebox}
		\caption{Failure type and conversation record of InternLM2-Math-7B. The model constantly generate the same "-8509" text until the output sequence was too long and truncated, leaving the problem unsolved.}
		\label{fig:fail_internlm}
	\end{figure*}
	
	\begin{figure*}
		\begin{examplebox}{DeepSeek-Math-7B}
				\textbf{Failure:} \\ Model outputs sequence without meaning, probably due to overly long input sequences. 
				
				\vspace{0.2cm}
				\textbf{User:} \\Question: What is the sum of the index of top 300 smallest numbers in the list? Options: A: 146734, B: 146738, C: 146733, D: 146724, E: 146728, F: 146727, G: 146721, H: 146729 Data: [3640, 1560, ...(100 numbers)]
				
				\vspace{0.3cm}
				\textbf{Deepseek-Math-7B Model:} \\
				31,3,9,3,1,1,3,3,3,3,3,3,3,3, 3,3,1,3,3,3,1,3,3,3, 3, 3,3,3, 1, 1,3,3, 3,1,3, 3,3,3, 1, 3,3,3, 3,3,1,3,1,3,91, 3,3,3, 3,31,3,3,3, 3, 339, 3, 3, 311, 1, 3, 1,3,3, 3,3,6,3,3, 3,31,3,3,1, 1, 1, 1, 3,1,1, 3,3, 1,1, 3,1,3,1, 1, 1, 1, 1, 19,1, 1, 1,191,1, 1,1, 19, 31,1,1, 1, 3,1, 1,3, 1133,3, 1, 1,1, 11, 11,3, 31, 3, 3,1,1,3,1, 1,33, 11,3,1911,3, 1,13,3131311119191,1, 3,1,11,1,19,11,1,1,1....(keeps outputting numbers)
		\end{examplebox}
		\caption{Failure type and conversation record of DeepSeek-Math-7B. The model only generates a series of numbers throughout its output, and we're unable to acquire any useful information from the output.}
		\label{fig:fail_ds_math}
	\end{figure*}	

	\begin{figure*}
		\begin{examplebox}{Metamath-Llemma-7B}
			\textbf{Failure:}\\ Model outputs ineffective chain of thoughts, and giving the wrong answer in the wrong format(ground truth is B))
			
			\vspace{0.2cm}
			\textbf{User:}\\ Question: What is the sum of the index of top 30 smallest numbers in the list? Options: A: 1566, B: 1572, C: 1575, D: 1573, E: 1577, F: 1564, G: 1569, H: 1567 Data: [-2568.1542995196, 1035.3645340299...(100 numbers)]
			
			\vspace{0.3cm}
			\textbf{Metamath-Llemma-7B Model:} \\ The data you receive is a list of numbers. To find the sum of the index of the top 30 smallest numbers, we need to find the 30 smallest numbers in the list and then calculate their sum. To find the 30 smallest numbers, we can use the min function in Python. The min function takes a list as input and returns the smallest number in the list. We can use the min function to find the 30 smallest numbers in the list. Now, we need to calculate the sum of these 30 smallest numbers. To do this, we can use the sum function in Python. The sum function takes a list as input and returns the sum of all the numbers in the list. We can use the sum function to calculate the sum of the 30 smallest numbers. Finally, we need to find the index of the sum. To do this, we can use the index function in Python. The index function takes a number as input and returns the index of that number in the list. We can use the index function to find the index of the sum. Therefore, the answer is (correct option). \#\#\#\# 1566 The answer is: 1566
		\end{examplebox}
		\caption{Failure type and conversation record of Metamath-Llemma-7B. The model attempts to use programming language to solve the problem, but it does not provide executable lines of code, only providing vague description of certain functions instead. After outputting the vague solution, the model outputs an answer, which is different from the ground truth and not in the expected format(We expect the model to provide the answer with a capital letter corresponding to the correct option).}
		\label{fig:fail_llemma}
	\end{figure*}

	\begin{figure*}
	\begin{examplebox}{MAmmoTH-13B}
		\textbf{Failure:}\\ Model outputs ineffective chain of thoughts, and giving the wrong answer in the wrong format(ground truth is B))
		
		\vspace{0.2cm}
		\textbf{User:}\\ Question: What is the sum of the index of top 30 smallest numbers in the list? Options: A: 1566, B: 1572, C: 1575, D: 1573, E: 1577, F: 1564, G: 1569, H: 1567 Data: [-2568.1542995196, 1035.3645340299...(100 numbers)]
		
		\vspace{0.3cm}
		\textbf{MAmmoTH-13B Model:}\\  Notice that there are two numbers in your list that have the smallest index: 4346.5174909575 and 8169.6394989672. They have the smallest index of 4346 and 8169 respectively. Since in the options the first number is the index of the smallest number, then the correct option is D.\\             Option: D
	\end{examplebox}
	\caption{Failure type and conversation record of Mammoth-13B. The model fail to understand the problem's instruction, and does not provide reasonable output. Moreover, it provides the wrong answer in incorrect format, just like Metamath-Llemma-7B(We expect the model to provide the answer with a capital letter corresponding to the correct option).}
	\label{fig:fail_mammoth}
	\end{figure*}
\end{document}
