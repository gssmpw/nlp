% Requires cvpr.sty, cvpr_eso.sty eso-pic.sty, everyshi.sty ieee_fullname.bst

%% You can choose to use two-column or single-column layout
%% Please update ConfName, ConfYear, PaperID accordingly

\documentclass[10pt,letterpaper]{article}
%\documentclass[10pt,letterpaper,twocolumn]{article} % Use two column 

\usepackage[review]{cvpr}      % To produce the REVIEW version
% \usepackage{cvpr}              % To produce the CAMERA-READY version
%\usepackage[pagenumbers]{cvpr} % To force page numbers, e.g. for an arXiv version

\usepackage{times}
\usepackage{epsfig}
\usepackage{graphicx}
\usepackage{amsmath}
\usepackage{amssymb}
\usepackage{gensymb} % for \degree
\usepackage{float}

% For table layout
\usepackage{xcolor}
\usepackage{booktabs}
\usepackage{colortbl}
\usepackage{multirow}
\usepackage{makecell}


% Setup caption
\usepackage[labelsep=period]{caption}
\captionsetup{font=small}
\RequirePackage{enumitem}
\setlist[itemize]{nosep}
\captionsetup[table]{aboveskip=3pt}
\captionsetup[table]{belowskip=2pt}
\captionsetup[figure]{aboveskip=5pt}
\captionsetup[figure]{belowskip=0pt}

% Reference commands
\newcommand{\Tref}[1]{Table~\ref{#1}}
\newcommand{\eref}[1]{Eq.~\eqref{#1}}
\newcommand{\Eref}[1]{Equation~\eqref{#1}}
\newcommand{\fref}[1]{Fig.~\ref{#1}}
\newcommand{\Fref}[1]{Figure~\ref{#1}}
\newcommand{\sref}[1]{Sec.~\ref{#1}}
\newcommand{\Sref}[1]{Section~\ref{#1}}

\renewcommand{\thetable}{S\arabic{table}} % Rename Fig. 1 to Fig. S1, Table 1 to Table S1
\renewcommand{\thefigure}{S\arabic{figure}}


% Define Macro Here 
%\renewcommand{\paragraph}[1]{\vspace{0.2em}\noindent \textbf{#1 \hspace{0.2em}}}
\newcommand{\todo}[1]{\textcolor{red}{{[TODO: #1]}}}

% It is strongly recommended to use hyperref, especially for the review version.
% hyperref with option pagebackref eases the reviewers' job.
% Please disable hyperref *only* if you encounter grave issues, e.g. with the
% file validation for the camera-ready version.
%
% If you comment hyperref and then uncomment it, you should delete
% ReviewTempalte.aux before re-running LaTeX.
% (Or just hit 'q' on the first LaTeX run, let it finish, and you
%  should be clear).
\usepackage[pagebackref,breaklinks,colorlinks]{hyperref}

%\usepackage[pagebackref=true,breaklinks=true,letterpaper=true,bookmarks=false,colorlinks]{hyperref}
% Variables
% Include other packages here, before hyperref.

% If you comment hyperref and then uncomment it, you should delete
% egpaper.aux before re-running latex.  (Or just hit 'q' on the first latex
% run, let it finish, and you should be clear).
%\usepackage[pagebackref=true,breaklinks=true,letterpaper=true,bookmarks=false,hidelinks]{hyperref}
%\usepackage[hidelinks]{hyperref}

%\camerareadycopy  %*** Uncomment this line for the final submission

% \def\confName{CVPR\xspace} % *** Enter the Conference Name
% \def\confYear{2025\xspace} % *** Enter conference Year 
% \def\PaperID{11104} % *** Enter the CVPR Paper ID here
% \def\httilde{\mbox{\tt\raisebox{-.5ex}{\symbol{126}}}}

\def\cvprPaperID{11104} % *** Enter the CVPR Paper ID here
\def\confName{CVPR}
\def\confYear{2025}

% Pages are numbered in submission mode, and unnumbered in camera-ready
%\ifcameraready\pagestyle{empty}\fi
\begin{document}

%%%%%%%%% TITLE
\title{Supplementary Material for \\ ``Learning Transformation-Isomorphic Latent Space for \\ Accurate Hand Pose Estimation''}

\author{Kaiwen Ren, Lei Hu, Zhiheng Zhang, Yongjing Ye, Shihong Xia\\
Institute of Computing Technology, CAS\\
Beijing, China\\
{\tt\small \{renkaiwen23s, hulei19z, zhangzhiheng20g, yeyongjing, xsh\}@ict.ac.cn}
}
\maketitle
% {
%     \hypersetup{linkcolor=black}
%     \tableofcontents
% }

\section{Prove the property of transformation}

We need to prove that the transformation set \( \mathsf G_{\mathbb I}=\{\mathcal H, \mathcal R_\theta, \mathcal{HR}_\theta \} \) satisfies the definition of group: closure for group combination operation, existence of identity element and existence of inverse for every element. The proof of the combination rule for \( \mathsf{G}_{\mathbb{I}} \) is simultaneously established through the proof of the closure property.

\paragraph{Proof of closure}

Before the formal proof, we first describe the properties of the transformations in \( \mathsf{G}_{\mathbb{I}} \).
\begin{align}
    \mathcal {HR}_\theta &= \mathcal H \circ \mathcal R_\theta, \label{eq1} \\
    \mathcal H \circ \mathcal R_\theta &= \mathcal R_{-\theta} \circ \mathcal H,
\end{align}
for any 2D coordinate $(x,y)$, by transformations $\mathcal H\circ\mathcal R_\theta$ we mean first flipping $(x,y)$ to $(-x,y)$ then rotate for $\theta$ radius to get $(-x\cos\theta-y\sin\theta,-x\sin\theta+y\cos\theta)$. Whereas if we first rotate for $-\theta$ radius we get $(x\cos\theta+y\sin\theta,-x\sin\theta+y\cos\theta)$, then horizontal flip produces $(-x\cos\theta-y\sin\theta,-x\sin\theta+y\cos\theta)$, which is the same as previous result. The proof for Eq.1 is similar.

According to the associative property of matrix multiplication, the same property applies to $\mathsf G_{\mathbb I}$ instantly.

Enumerate all possible pairs of elements in \( \mathsf{G}_{\mathbb{I}} \), and prove that their composition remains within \( \mathsf{G}_{\mathbb{I}} \):
\begin{itemize}
    \item Case 1: \( \mathcal H \circ \mathcal H = \mathcal I = \mathcal R_0 \),
    \item Case 2: \( \mathcal R_\alpha \circ \mathcal R_\beta = \mathcal R_{\alpha + \beta} \),
    \item Case 3: \( \mathcal {HR}_\alpha \circ \mathcal {HR}_\beta = \mathcal H \circ \mathcal R_\alpha \circ \mathcal H \circ \mathcal R_\beta = \mathcal R_{-\alpha}  \circ \mathcal R_\beta = \mathcal R_{-\alpha + \beta} \),
    \item Case 4: \( \mathcal H\circ\mathcal R_\theta=\mathcal{HR}_\theta, \mathcal R_{\theta}\circ\mathcal H=\mathcal H\circ\mathcal R_{-\theta}=\mathcal{HR}_{\theta} \),
    \item Case 5: \( \mathcal H\circ\mathcal {HR}_{\theta}=\mathcal R_{\theta}, \mathcal {HR}_{\theta}\circ\mathcal H=\mathcal R_{-\theta} \),
    \item Case 6: \( \mathcal R_{\alpha}\circ\mathcal {HR}_{\beta} = \mathcal {HR}_{-\alpha+\beta}, \mathcal {HR}_{\beta}\circ\mathcal R_{\alpha}=\mathcal {HR}_{\alpha+\beta} \),
\end{itemize}
in summary, we have completed the proof of closure.

\paragraph{Existence of identity}
\( \mathcal{R}_0 \) is the clearly identity element.

\paragraph{Existence of inverse}
The inverse elements of \( \mathcal{H} \), \( \mathcal{R}_{\theta} \), and \( \mathcal{HR}_{\theta} \) are \( \mathcal{H} \), \( \mathcal{R}_{-\theta} \), and \( \mathcal{HR}_{-\theta} \), respectively.

In conclusion, we have proven that \( \mathsf{G}_{\mathbb{I}} \) is a group.   \(\blacksquare\)

% \pagebreak
% {\small
% \bibliographystyle{ieee_fullname}
% \bibliography{ref}
% }
\end{document}
