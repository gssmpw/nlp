% ICCV 2025 Paper Template

\documentclass[10pt,twocolumn,letterpaper]{article}

%%%%%%%%% PAPER TYPE  - PLEASE UPDATE FOR FINAL VERSION
% \usepackage{iccv}              % To produce the CAMERA-READY version
% \usepackage[review]{iccv}      % To produce the REVIEW version
\usepackage[pagenumbers]{iccv} % To force page numbers, e.g. for an arXiv version

\usepackage{booktabs} % For formal tables

\usepackage{graphicx}
\usepackage{amsmath}

\usepackage[utf8]{inputenc} % allow utf-8 input
\usepackage[T1]{fontenc}    % use 8-bit T1 fonts
\usepackage{url}            % simple URL typesetting
\usepackage{amsfonts}       % blackboard math symbols
\usepackage{nicefrac}       % compact symbols for 1/2, etc.
\usepackage{microtype}      % microtypography
\usepackage{xcolor}         % colors
\usepackage{bm}         % colors
\usepackage{amsmath}         % colors
\usepackage{graphicx}

\usepackage{multirow}
\usepackage{makecell}
\usepackage{pifont}
\usepackage{wrapfig}

\usepackage{caption}
% \usepackage{subcaption}
\usepackage{comment}
\usepackage{makecell} 

\usepackage{xspace}

% Add a period to the end of an abbreviation unless there's one
% already, then \xspace.
\makeatletter
\DeclareRobustCommand\onedot{\futurelet\@let@token\@onedot}
\def\@onedot{\ifx\@let@token.\else.\null\fi\xspace}

\def\eg{\emph{e.g}\onedot} \def\Eg{\emph{E.g}\onedot}
\def\ie{\emph{i.e}\onedot} \def\Ie{\emph{I.e}\onedot}
\def\cf{\emph{c.f}\onedot} \def\Cf{\emph{C.f}\onedot}
\def\etc{\emph{etc}\onedot} \def\vs{\emph{vs}\onedot}
\def\wrt{w.r.t\onedot} \def\dof{d.o.f\onedot}
\def\etal{\emph{et al}\onedot}
\makeatother

\newcommand{\tabincell}[2]{\begin{tabular}{@{}#1@{}}#2\end{tabular}}
\newcommand{\xmark}{\ding{55}}
\newcommand{\cmark}{\ding{51}}


% Import additional packages in the preamble file, before hyperref
% \newcommand{\CG}{\mathcal{G}\xspace}
\newcommand{\CV}{\mathcal{V}\xspace}
\newcommand{\CE}{\mathcal{E}\xspace}
\newcommand{\CA}{\mathcal{A}\xspace}
\newcommand{\CF}{\mathcal{F}\xspace}
\newcommand{\CR}{\mathcal{R}\xspace}
\newcommand{\CB}{\mathcal{B}\xspace}
\newcommand{\CX}{\mathcal{X}\xspace}
\newcommand{\CK}{\mathcal{K}\xspace}
\newcommand{\CM}{\mathcal{M}\xspace}
\newcommand{\CC}{\mathcal{C}\xspace}
\newcommand{\CL}{\mathcal{L}\xspace}
\newcommand{\CI}{\mathcal{I}\xspace}
\newcommand{\CQ}{\mathcal{Q}\xspace}
\newcommand{\CO}{\mathcal{O}\xspace}
\newcommand{\CP}{\mathcal{P}\xspace}
\newcommand{\CS}{\mathcal{S}\xspace}
\newcommand{\CT}{\mathcal{T}\xspace}
\newcommand{\CJ}{\mathcal{J}\xspace}
\usepackage[para]{footmisc}
\usepackage{subfig}
% \usepackage{subcaption}
% \usepackage{array}
% \usepackage{colortbl}



% It is strongly recommended to use hyperref, especially for the review version.
% hyperref with option pagebackref eases the reviewers' job.
% Please disable hyperref *only* if you encounter grave issues, 
% e.g. with the file validation for the camera-ready version.
%
% If you comment hyperref and then uncomment it, you should delete *.aux before re-running LaTeX.
% (Or just hit 'q' on the first LaTeX run, let it finish, and you should be clear).
\definecolor{iccvblue}{rgb}{0.21,0.49,0.74}
\usepackage[pagebackref,breaklinks,colorlinks,allcolors=iccvblue]{hyperref}

%%%%%%%%% PAPER ID  - PLEASE UPDATE

\newcommand{\nickname}{HumanGif}
\newcommand\pl[1]{{\color{violet}#1}}

%%%%%%%%% TITLE - PLEASE UPDATE
\title{HumanGif: Single-View Human Diffusion with Generative Prior}

%%%%%%%%% AUTHORS - PLEASE UPDATE
\author{
Shoukang Hu\textsuperscript{\rm 1} \quad
Takuya Narihira\textsuperscript{\rm 1} \quad
Kazumi Fukuda\textsuperscript{\rm 1} \quad
Ryosuke Sawata\textsuperscript{\rm 1} \quad \\
Takashi Shibuya\textsuperscript{\rm 1} \quad
Yuki Mitsufuji\textsuperscript{\rm 1,2}\\
\textsuperscript{\rm 1}Sony AI \quad
\textsuperscript{\rm 2}Sony Group Corporation \\
}


% \author{Shoukang Hu\\
% Sony AI\\
% {\tt\small firstauthor@i1.org}
% For a paper whose authors are all at the same institution,
% omit the following lines up until the closing ``}''.
% Additional authors and addresses can be added with ``\and'',
% just like the second author.
% To save space, use either the email address or home page, not both
% \and
% Second Author\\
% Institution2\\
% First line of institution2 address\\
% {\tt\small secondauthor@i2.org}
% }

\begin{document}

\twocolumn[{
    \renewcommand\twocolumn[1][]{#1}%
    \maketitle
    \vspace{-20pt}
    % \setlength{\abovecaptionskip}{0cm}
    \begin{center}
        \centering
        \includegraphics[width=1.0\textwidth]{figures/teaser.pdf}
        \captionof{figure}{\nickname{} is a single-view human diffusion model that leverages generative priors to compensate for information missing from the input image. With generative prior, \nickname{} synthesizes perpetually realistic, view-consistent, and temporally coherent human avatars.}
        \label{fig:teaser}
    \end{center}
    % \vspace{-6pt}
}]

\maketitle
\begin{abstract}

% Recent works to jointly reconstruct 3D human and object from a single RGB image, are mostly model-based, that fail to capture the fine details of the clothed human body and object surface. In this paper, we introduce ReCHOR, a novel, model-free, first-method to produce realistic clothed human-object reconstructions from a monocular view. This is extremely challenging due to human-object occlusions, diverse interactions and depth ambiguity, as it needs to infer both 3D spatial awareness and high resolution details. Our core idea is based on estimating neural implicit representations for human and object respectively by an attention-based neural implicit model that attends to pixel-aligned features from both the global human-object image for spatial awareness and  the local separate view of human and object images for high quality details. Additionally, the network is conditioned on semantic features from an initial estimated human-object pose prior and a generative diffusion model that inpaints occluded regions, thus enabling the retrieval of details from them.
% We also propose a synthetic dataset with rendered scenes of diverse, inter-occluded 3D human and object scans, to train our network. We evaluate our method on the synthetic and real world BEHAVE dataset. Our experiments show that our method outperforms the SOTA in achieving realistic clothed human-object reconstructions.
Recent approaches to jointly reconstruct 3D humans and objects from a single RGB image represent 3D shapes with template-based or coarse models, which fail to capture details of loose clothing on human bodies. In this paper, we introduce a novel implicit approach for jointly reconstructing realistic 3D clothed humans and objects from a monocular view. For the first time, we model both the human and the object with an implicit representation, allowing to capture more realistic details such as clothing. This task is extremely challenging due to human-object occlusions and the lack of 3D information in 2D images, often leading to poor detail reconstruction and depth ambiguity. To address these problems, we propose a novel attention-based neural implicit model that leverages image pixel alignment from both the input human-object image for a global understanding of the human-object scene and from local separate views of the human and object images to improve realism with, for example, clothing details. Additionally, the network is conditioned on semantic features derived from an estimated human-object pose prior, which provides 3D spatial information about the shared space of humans and objects. To handle human occlusion caused by objects, we use a generative diffusion model that inpaints the occluded regions, recovering otherwise lost details. For training and evaluation, we introduce a synthetic dataset featuring rendered scenes of inter-occluded 3D human scans and diverse objects. Extensive evaluation on both synthetic and real-world datasets demonstrates the superior quality of the proposed human-object reconstructions over competitive methods.
\end{abstract}    
\section{Introduction}
\label{sec:intro}
% Image editing methods in diffusion models depend on user-defined control directions - users can unlock their creativity using these methods by specifying the desired manipulation through prompts~\cite{gandikota2023concept}, reference images~\cite{ruiz2022dreambooth, kumari2022customdiffusion, gal2022image, chen2024trainingfreeregionalpromptingdiffusion}, or attribute vectors~\cite{parmar2023zero,hertz2022prompt}. In this work, we ask a fundamentally different question: \emph{Can we automatically discover the underlying visual structure of a concept within diffusion model's knowledge?} %Rather than requiring user-specified controls, we aim to decompose the model's internal knowledge into meaningful directions.

% This question touches on a fundamental limitation in how we interact with diffusion models. Current control methods ~\cite{zhang2023addingconditionalcontroltexttoimage, gandikota2023concept, ye2023ipadaptertextcompatibleimage,ye2023ipadaptertextcompatibleimage, hertz2024stylealignedimagegeneration, li2023photomaker, shi2024instantbooth, chen2024trainingfreeregionalpromptingdiffusion} require users to specify their desired manipulations in advance, limiting interactive creativity. This contrasts with natural human artistic workflows, where creators dynamically explore creative ideas while jointly refining them toward meaningful artistic outcomes~\cite{hoffmann2016modeling}. This synergy between specification and exploration is not new to generative models. Early GAN architectures naturally developed disentangled latent spaces that enabled continuous\cite{harkonen2020ganspace,radford2015unsupervised, wu2021stylespace, shen2020interfacegan}, compositional control over generated images. Users could explore these spaces to discover interesting variations that would be difficult to describe in words~\cite{wu2021stylespace}, then combine them to achieve their creative goals~\cite{grabe2022towards}. 


% While diffusion models have largely superseded GANs in conditional image synthesis~\cite{dhariwal2021diffusion},  their underlying structure remains less understood. Diffusion models achieve remarkable diversity through high-dimensional latents, unlike GANs' compact latent spaces.  With a single prompt, diffusion models can generate radically different variations through different random initializations of input noise. We ask - Is it possible to discover interpretable structure within this vast space of variations?

Text-to-image diffusion models are capable of generating remarkable visual variations from a single prompt through different random initializations. However, this vast creative potential remains largely opaque to users---while we can generate diverse images, we lack understanding of the underlying structure of these variations. This presents a fundamental challenge: how can we discover and expose the latent visual capabilities encoded within these models?

\let\thefootnote\relax \footnote{$^{*}$Correspondence to \texttt{gandikota.ro@northeastern.edu}}

The challenge touches on a key limitation in how we interact with diffusion models today. Current control methods require users to explicitly specify their desired edits in advance through prompts~\cite{gandikota2023concept}, reference images~\cite{zhang2023addingconditionalcontroltexttoimage, chen2024trainingfreeregionalpromptingdiffusion, ruiz2022dreambooth,kumari2022customdiffusion, Ryu_lora, hu2021lora}, or attribute vectors~\cite{ye2023ipadaptertextcompatibleimage, hertz2024stylealignedimagegeneration, li2023photomaker, shi2024instantbooth,parmar2023zero,hertz2022prompt}. That contrasts sharply with natural human creative workflows, where artists dynamically explore creative ideas and jointly refine them toward meaningful artistic outcomes~\cite{hoffmann2016modeling}. The need for pre-specified controls creates a barrier between users and the full creative potential of these models.

Interestingly, earlier generative models like GANs~\cite{gans,karras2019style,brock2018large} naturally developed more interpretable internal structures. Their compact latent spaces often exhibited emergent disentanglement~\cite{harkonen2020ganspace,radford2015unsupervised, wu2021stylespace, shen2020interfacegan}, enabling continuous and compositional control over generated images. Users could explore these spaces to discover interesting variations that would be difficult to describe in words~\cite{wu2021stylespace}, then combine them to achieve their creative goals~\cite{grabe2022towards}.

Diffusion models have largely superseded GANs in conditional image synthesis~\cite{dhariwal2021diffusion}, achieving greater diversity through much higher-dimensional latents. And yet an understanding of the underlying structure of these larger latent spaces has remained elusive. In this work, we ask a fundamental question: \emph{Can we automatically discover the visual structure within a diffusion model's knowledge of a concept?} Rather than requiring user-specified controls, we aim to decompose the model's internal representations into expressive directions that users can explore and combine.

To address these needs, we present \textbf{SliderSpace}, a framework that brings systematic explorability to diffusion models. Given just a text prompt, SliderSpace discovers a canonical set of meaningful, diverse, and controllable directions within the model's knowledge of that concept. Each direction is implemented as a low-rank adapter~\cite{hu2021lora} that can be scaled and composed with others, allowing users to explore and smoothly combine different aspects of variation, as shown in Figure~\ref{fig:intro}.

We ground SliderSpace discovery in three key requirements for meaningful decomposition of a diffusion model's visual manifold: 
\begin{enumerate}
    \item \textbf{Unsupervised Discovery:} The decomposition process should emerge from the intrinsic structure of the model's learned representation, rather than being guided by predefined attributes. This ensures we capture the true topology of the model's knowledge space rather than projecting our assumptions onto it.
    
    \item \textbf{Semantic Orthogonality:} Each discovered control must represent a distinct semantic direction. This is enforced in a semantic feature space, like CLIP, where every slider has an orthogonal effect in embeddings. This prevents discovering multiple controls that create similar semantic effects, making the system more efficient and easier.
    
    \item \textbf{Distribution Consistency:} Directions must induce consistent transformations across both random seeds and prompt variations. 
\end{enumerate}

These requirements naturally lead to our proposed framework, which we formalize in Section~\ref{sec:method}. As we show in our experiments, SliderSpace is architecture-agnostic, working with both conventional U-Net based models like Stable Diffusion~\cite{rombach2022high, rombach2022sd20, podell2023sdxl, turbo, dmd} and recent transformer-based architectures like Flux~\cite{flux}.

We demonstrate the expressiveness of SliderSpace through three applications: First, we show how SliderSpace can decompose high-level concepts into diverse and expressive components, revealing the natural axes of variation in the model's understanding. Second, we explore artistic style variation, where SliderSpace discovers directions that match or exceed the diversity of manually curated artist lists while being judged more useful by human evaluators. Finally, we show how SliderSpace can help reverse the mode collapse commonly observed in distilled diffusion models, restoring diversity while maintaining generation speed.

Beyond providing practical creative control, SliderSpace opens new avenues for understanding and utilizing the latent capabilities of diffusion models. By mapping these models' visual potential into intuitive, composable directions, we take a step toward making their creative possibilities more accessible and interpretable to users.

% Image editing methods in diffusion models unlock the creativity of users. In this work we ask an alternate question: \emph{Can we organize and expose what of the diffusion model is already capable of?}.
% Existing methods for controlling image generation typically require users to manually specify edit directions for desired changes. This process is time-consuming, requires technical expertise, and limits the spontaneity of the creative process. For instance, if a user wants to adjust the smile of a generated person, they must explicitly request this edit, often through imprecise prompt engineering or model fine-tuning. This approach of predefined controls or manual specifications restricts users from fully exploring the latent capabilities of the model. There may be interesting stylistic variations or attributes that the model can generate, but users have no easy way to discover or utilize these.

% Natural visual disentanglement was an emergent property in the latent space of Generative Adversarial Models (GANs) \cite{harkonen2020ganspace,radford2015unsupervised, wu2021stylespace, shen2020interfacegan}. In particular, it has been observed that StyleGAN~\cite{karras2019style} stylespace neurons offer detailed control over many meaningful aspects of images that would be difficult to describe in words~\cite{wu2021stylespace}. However, diffusion models do not share such a compact latent space~\cite{park2023unsupervised}; and efforts to uncover such a space in the semantic embeddings of the text conditioning have met with limited success \nik{Nick - is there a specific citation you were thinking about?}.

% In this work we introduce \textbf{SliderSpace}, which takes a step towards uncovering an analogous low dimensional representation of diffusion models' visual breadth; in essence treating the diffusion model as many generators sharing parameters, where a particular generator is defined by a specific prompt. For a given prompt we sample many random seeds (and optionally prompt expansions using an LLM), generate the corresponding images, and apply an off the shelf feature extractor (in this work CLIP, but our method can be applied to any differentiable feature extractor). We use PCA to analyze these features, and for each of the leading $k$ principal components we train a LoRA \cite{} which causes the diffusion model to produces images which increase the feature magnitude along that component when passed back through the same feature extractor. This leads to a 'Slider' for each principal component, because each LoRA can be scaled and applied to the original diffusion model, continuously varying those visual features in the generated results (as measured, in our case, by CLIP).

% There are many other works that enhance the controllability of diffusion models. One common approach is enabling users to add spatial constraints to a generation either manually, or via a reference image \cite{zhang2023addingconditionalcontroltexttoimage, chen2024trainingfreeregionalpromptingdiffusion}, a second is leveraging more abstract embeddings (e.g. identity, style) extracted from a reference image \cite{ye2023ipadaptertextcompatibleimage, hertz2024stylealignedimagegeneration, li2023photomaker, shi2024instantbooth}, a third is finetuning a foundation model to better generate a concept important to the user \cite{ruiz2022dreambooth, kumari2022customdiffusion, Ryu_lora, hu2021lora}, and a fourth (most relevant to this work) is finding low-rank adaptors of the model based on a prompt or small training set which can be scaled to provide continous control over one aspect of generated image (e.g. night vs day, basic vs luxury, etc.) \cite{gandikota2023concept}. SliderSpace is complementary to all of these methods and offers something distinct. All of the other methods we are aware require the user (and / or model designer) to know in advance what type of control they want. In contrast SliderSpace assists users in discovering and controlling hidden capabilities present in the diffusion model's distribution of possible generations.

%We propose that truly intuitive creative control in a text-to-image model should meet three key criteria: \emph{discoverability}, \emph{intuitiveness}, and \emph{specificity}. The model should reveal controllable attributes that may not be immediately obvious, offer controls that are easy to understand and manipulate, and ensure each control affects a distinct attribute of the generated image.

% We demonstrate the utility and power of SliderSpace using three applications built on top of SDXL-DMD \cite{dmd}, because its fast generation speed lends itself well to the continuous control offered by SliderSpace.

% First, we study concept decomposition (Section \ref{sec:concept_exp}), where we learn sliders for a specific concept (e.g. 'monster', 'waterfall', 'car'). Through quantitative metrics of diversity and text alignment we demonstrate that the learned sliders dramatically boost the diversity of generations when randomly applied without harming text alignment; we also ask humans to qualitatively judge these results in a user study where they find the SliderSpace results to be more 'Diverse', 'Useful', and 'Creative' than our baselines.

% Second, we attempt to compare the automatic discoveries of SliderSpace to a large scale manual study of artistic styles (Section \ref{sec:art_exp}), open-sourced by ParrotZone \cite{parrotzone}. In this study SDXL was prompted with over 4300 artist names,  and based on visual inspection the cases of successful stylistic mimicry recorded. Quantitatively SliderSpace more closely matches the distribution of artistic variation discovered by ParrotZone than other baselines, and in our user studies was judged to be significantly more 'Diverse' and 'Useful' than the baselines. To our surprise humans even judged SliderSpace results to be slightly more 'Diverse' than the results generated by the manually discovered artist names of \cite{parrotzone}.

% Third, we attempt to use SliderSpace to reverse the mode collapse commonly observed in distilled few-step diffusion models relative to the original teacher model (Section \ref{sec:diverse_exp}). We quantitatively demonstrate that applying SliderSpace to SDXL-DMD leads to more closely matching the distribution of images by the original teacher, SDXL.

%Through extensive experiments on various state-of-the-art text-to-image models, we demonstrate that SliderSpace significantly enhances user control and creative expression in AI-assisted image generation tasks. Our method enables a range of applications, including concept decomposition and control, diversity improvement in generated images, customization dissection and edits, and the exploration of artistic styles inherent in the model.

% SliderSpace goes beyond providing a practical tool for enhanced creative control. By mapping the visual potential of diffusion models it can open new avenues for generative creativity and deepens our understanding of each model's hidden potential.
\section{Related Work}
\label{sec:related work}
% In this section, we review the existing literature on point cloud denoising and unsupervised image denoising.
%-------------------------------------------------------------------------
\subsection{Point cloud denoising}

    \subsubsection{Traditional methods}
Traditional approaches to point cloud denoising include statistical methods \cite{computingpointset2003,definingpointset2004,wlop2009HH}, filtering techniques\cite{pointsetsurfaces2001,Robustmoving2005, zaman2017density}, and optimization-based methods \cite{l1sparse2010,clop2014PR,digne2017bilateral,multi-projection2018duan,hu2020featuregraph} . These techniques often rely on handcrafted features and heuristics to distinguish signal from noise. For example, statistical methods may use distribution models to identify and remove outliers. Filtering methods, such as mean or median filters, operate under the assumption that noise is statistically different from the signal. Optimization-based methods formulate denoising as an energy minimization problem, where regularization terms constrain the solution to ensure certain smoothness cirterion or adherence to prior knowledge.

%-------------------------------------------------------------------------
    \subsubsection{Supervised learning based methods}
In recent years, several deep learning-based methods \cite{rakotosaona2020PCN,hermosilla2019TotalDenoising,luo2020DMR,luo_score-based_2021} have been proposed for point cloud denoising. NPD \cite{NPD2019} is the first learning-based point cloud denoising method that directly processes noisy data without requiring noise characteristics or neighboring point definitions. This approach combines local and global information by projecting noisy points onto estimated reference planes, effectively removing noise and enhancing robustness against variations in noise intensity and curvature. PointCleanNet\cite{rakotosaona2020PCN} first removes outlier points and then combines them with residual connectivity to predict the inverse displacement \cite{Guerrero2017PCPNetLL}, and iteratively shifts noisy points to remove noise. Pistilli \etal proposed GPDNet \cite{gpdnet2020}, which is a graph convolutional network to improve denoising robustness at high noise levels. Luo \etal also proposed  DMRDenoise \cite{luo2020DMR}, which filter
points by first downsampling the noisy inputs and reconstructing the local subsurface to perform point upsampling. However, this resampling method is difficult to maintain a good local shape. ScoreDenoise \cite{luo_score-based_2021} is proposed to tackle the aforementioned issues by iteratively updating the point position in implicit gradient fields learned by neural networks. For inference, they follows an iterative procedure with a decaying step size, which stabilizes point movement and prevents over-correction, allowing points to converge gradually toward the underlying geometry. The authors of \cite{de_Silva_Edirimuni_2023_CVPR} proposed IterativePFN - an iterative method that use a novel loss function that utilizes an adaptive ground truth target at each iteration to capture the relationship between intermediate filtering results during training. Zheng \etal proposed a end-to-end network for joint normal filtering-based point cloud denoising \cite{10173632}. They introduce an auxiliary normal filtering task to enhance the network's ability to remove noise while preserving geometric features more accurately.

Supervised methods can achieve impressive results, but are limited by the availability and quality of the training data, as they typically require paired noisy and clean point clouds to train the neural network. The absence of clean data in real-world scenario pose a significant challenge on applicability of these algorithms.

%-------------------------------------------------------------------------
    \subsubsection{Unsupervised learning methods}
Unsupervised learning-based methods for point cloud denoising do not require ground-truth clean data. Instead, these methods leverage the inherent structure or distribution of the point cloud to guide the denoising process. Unsupervised methods show promise in scenarios where clean data is absent or hard to obtain.

Hermosilla \etal first introduced Total Denoising (TotalDn) \cite{hermosilla2019TotalDenoising} as an unsupervised learning approach for point cloud denoising, relying solely on noisy data without requiring clean ground truth. TotalDn approximates the underlying surfaces by regressing points from the distribution of unstructured total noise, utilizing a spatial prior term to refine the estimation of geometry. 

In DMRDenoise \cite{luo2020DMR}, an unsupervised version is proposed which utilizes a loss function that identify local neighborhoods using a probabilistic Gaussian mask on the k-nearest neighbors, which selectively retains points likely to represent the underlying surface. By leveraging an Earth Mover's Distance (EMD) assignment, it achieves a one-to-one correspondence between input and predicted points, aligning them to reduce noise within local neighborhoods.
This approach enhances robustness to noise and adapts well to varied surface geometries. However, the probabilistic masking and EMD calculation add computational complexity, which can slow down inference in dense or noisy point clouds. 

ScoreDenoise \cite{luo_score-based_2021} also introduced an unsupervised version that employs ensemble score function and an adaptive neighborhood-covering loss for model training.  
Score-u is probably the most relevant work to our method. However, the defined score in \cite{luo_score-based_2021} is only an displacement-alike version of the score function and there is no explicit formula relating the estimated score to the final denoising result. Also, the iterative process is computationally expensive, and can suffer from fluctuation, leading to perturbed and unstable solution.

Most recently, Noise4Denoise \cite{noise4Wang2024} method is proposed that use an additional doubly-noisy point cloud to learn noise displacement by comparing the two noise levels. This approach effectively leverages synthetic noise for training, allowing the model to estimate residuals without relying on clean data.
However, in practical applications, noise parameters are often unknown and variable, making it challenging to replicate the exact conditions assumed during training. This reliance on predefined noise characteristics can limit the model's applicability to real-world scenarios where noise distributions may differ significantly from synthetic settings. 
%-------------------------------------------------------------------------
\subsection{Unsupervised image denoising}
Recently unsupervised image denoising has made significant progress. Non-Bayesian methods include PURE \cite{luisier2010image}, SURE \cite{SURE2018} \textit{etc.}, which are based on various unbiased risk estimator under certain noise distribution. Other methods explore self-similarity in natural images \cite{xu2015patch, doi:10.1137/23M1614456} or exploits the statistical properties of noise to achieve denoising effect \cite{gravel2004method}.  

Noise2Noise \cite{2018Noise2NoiseLI} is a pioneering method that does not require clean data, but it requires multiple noisy versions of the same image for training. To address this limitation, methods such as Noise2Void \cite{2018Noise2VoidL}, Noise2Self \cite{2019Noise2SelfBD}, \textit{etc.}, have been developed, which use only a single noisy image. These methods are particularly important for practical applications where obtaining clean images or multiple noisy realizations of the same image is difficult or impossible. Neighbor2Neighbor \cite{huang2021neighbor2neighbor} proposed a two-step method with a a random neighbor sub-sampler that generates training  pairs and a denosing network. Kim \etal proposed Noise2Score\cite{kim_noise2score_2021}, a novel Bayesian framework for self-supervised image denoising without clean data. The core of Noise2Score is the usage of Tweedie's formula, which provides an explicit representation of the denoised image through a score function. Combined with score function estimation, Noise2Score can be applied to image denoising with any exponential family noise. Kim \etal also proposed the Noise Distribution Adaptive Self-Supervised Image Denoising method \cite{kim_noise_2022}, which further extends the application of Noise2Score by combining the Tweedie distribution with score matching. This enables adaptive handling of various noise distributions and dynamically adjusts the denoising process by estimating noise parameters. On the other hand, Xie \etal \cite{scoreXie2024} broadened the denoising scope of Noise2Score by allowing it to handle complex noise models, such as multiplicative and structurally correlated noise, demonstrating broad applicability to diverse noise models.

These development of unsupervised image denoising method motivate us to explore similar ideas in 3D point cloud denoising.






\section{Methodology}
\paragraph{Preliminaries.}
We primarily focus on the homologous model merging, in which $\boldsymbol{\theta}_i$ all come from the same base model $\boldsymbol{\theta}_{\rm{base}}$. Given $K$ tasks $\{T_1,T_2,\cdots,T_K\}$ and $K$ corresponding fine-tuned models with parameters $\{\boldsymbol{\theta}_1,\boldsymbol{\theta}_2,\cdots,\boldsymbol{\theta}_K\}$, model merging aims to combine $K$ fine-tuned models into one single model simultaneously performing on $\{T_1,T_2,\cdots,T_K\}$ without post-training~\cite{method_p1_1,method_p1_2}.
Task vector~\cite{ilharco2023editing,yang2024adamerging} is a key element in merging method which could enhances the base model‘s ability or enable the model to handle other tasks. Specifically, for task $T_i$, the task vector $\boldsymbol\tau_i\in \mathbb{R}^D$ is defined as the vector obtained by subtracting the SFT weights $\boldsymbol{\theta}_i$ from the base model weight
$\boldsymbol{\theta}_{\rm{base}}$, \emph{i.e.}, $\boldsymbol\tau_i=\boldsymbol{\theta}_i-\boldsymbol{\theta}_{\rm{base}}$. The merged model could be denoted as $\boldsymbol{\theta}_m=\boldsymbol{\theta}_{\rm{base}}+\sum_i \lambda_i\boldsymbol{\tau}_i$, which $\lambda_i$ is the scaling factor measuring the importance of task vector. For clarification, we also denote the neuron set in $\boldsymbol{\theta}_i$ as $\mathcal{N}_i$, the neuron set in $\boldsymbol{\tau}_i$ as $\mathcal{T}_i$.



\begin{algorithm}[!ht]
    \caption{LED-Merging}
    \label{alg1}
    \begin{algorithmic}[1]
        \REQUIRE  base model $\boldsymbol{\theta}_{\rm{base}}$, SFT models $\{\boldsymbol{\theta}_{i}\mid i\in [K]\}$, mask ratios \{$r_{i} \mid i\in [K]\}$, scaling factors $\{\lambda_i\mid i\in[K]\}$, location datasets $\{\mathcal{X}_{i}\mid i\in[K]\}$
        \ENSURE merged parameter $\boldsymbol{\theta}_{m}$
        \STATE $\mathcal{M}\leftarrow\phi$
        \STATE $\boldsymbol{\theta}_{m}\leftarrow \boldsymbol{\theta}_{\rm{base}}$
        \FOR{$i\in [K]$}
        \STATE $I(\boldsymbol{\theta}_i)=\mathbb{E}_{x\sim \mathcal{X}_i}|\boldsymbol{\theta}_{i}\odot \nabla_{\boldsymbol{\theta}_i}\mathcal{L}(x)|$
        \STATE $I(\boldsymbol{\theta}_{\rm{base}})=\mathbb{E}_{x\sim \mathcal{X}_i}|\boldsymbol{\theta}_{\rm{base}}\odot \nabla_{\boldsymbol{\theta}_{\rm{base}}}\mathcal{L}(x)|$
        
        \STATE calculate $\mathcal{T}^{r_i}_{i}$ following Equation \ref{vote}
        \STATE  $\mathcal{M}\leftarrow \mathcal{M}\cup\{\mathcal{T}^{r_i}_i\}$
       
        
   
        
        
        \ENDFOR  
        \FOR{$i\in [K]$}
        
        \STATE calculate $\text{Disjoint}(\mathcal{T}_i^{r_i})$ use Equation~\ref{disjoint_safety}
        \STATE $\boldsymbol{m}_i \leftarrow \boldsymbol{0}$
        \FOR{$d\in \mathcal{T}_i^{r_i}$}
        \STATE $\boldsymbol{m}_{i,d}=1$
        \ENDFOR
        \STATE $\boldsymbol{\theta}_{m}\leftarrow \boldsymbol{\theta}_{m}+\lambda_i \boldsymbol{\tau}_i\odot \boldsymbol{m}_{i}$
        \ENDFOR
    \end{algorithmic}
\end{algorithm}
    %\vspace{-5pt}
\begin{figure*}[h!]
    \centering
    \includegraphics[width=\linewidth]{figs/pipeline_v2.pdf}
    \vspace{-40mm}
    \caption{Overview of our two-stage training pipeline {\ours}.}
    \label{fig:pipeline}
\end{figure*}


\paragraph{LED-Merging: Location, Election, and Disjoint Merging}
To address the neuron misidentification and interference issues in existing model merging methods, we propose LED-Merging (Location, Election, and Disjoint Merging). Specifically, previous studies \cite{modelstock, ilharco2023editing, tiesmerging} fail to accurately identify safety-related neurons in task vectors with a single magnitude score, namely \textit{neuron misidentification}. Meanwhile, there exists an interference between safety-related and utility-related task vector neurons during the merging process, namely \textit{neuron interference}. To address neuron misidentification, we first locate important neurons both in the base and fine-tuned models and then elect neurons from the task vector considering these two scores together. Subsequently, to mitigate the interference, we introduce a disjoint step, isolating these important neurons so that they influence different base neurons. The whole process is illustrated in Figure~\ref{fig:method}. 




In the location and election step, we consider the importance score from base and fine-tuned models simultaneously to locate task-specific neurons. In this way, it is more accurate than relying on the magnitude score alone because task-specific neurons with high importance score in the fine-tuned model may not necessarily score high in the base model, and vice versa.

{\textbf{Location}}.  We first calculate importance scores for each neuron in a base/fine-tuned model. Given a location dataset $\mathcal{X}_i=\{(x,y)_k\}$, where $x$ is the question and $y$ is the answer, we calculate the importance scores for the weight $\boldsymbol{\theta}_i\in\mathbb{R}^D$ in any  layer as follows~\cite{snip,spareseGPT,sun2024a}:
\begin{equation}
    I(\boldsymbol{\theta}_i)=\mathbb{E}_{x\sim \mathcal{X}_i}[\boldsymbol{\theta}_i\odot \nabla _{\boldsymbol{\theta}_i}\mathcal{L}(x)],
    \label{location}
\end{equation}
which $\mathcal{L}(x)=-\log p(y\mid x)$ is the conditional negative log-likelihood loss. We choose the SNIP score~\cite{snip} because it balances computational efficiency and performance~\cite{cq}. Please refer to Sec.~\ref{sec:ablation} for the comparison between different location methods. After computing importance scores, we choose top-$r_i$ neurons as the important neuron subset $\mathcal{N}_{i}^{r_i}$ from $I(\boldsymbol{\theta}_i)$.
 
 % After computing locating scores, we select the neurons scoring both high in base and fine-tuned models as important neurons in task vectors. Then in the disjoint step,  with preventing  polysemantic neurons  from receiving gradient updates towards different directions,
 % we use set difference to isolate the safety   and utility-related neurons  and construct corresponding masks for merging process,

{\textbf{Election}}. A natural question is how to select important neurons in the task vector $\boldsymbol{\tau}_i$ based on $I(\boldsymbol{\theta}_{\rm{base}})$ and $I(\boldsymbol{\theta}_{i})$. The important neurons in the base model may be different from neurons in the fine-tuned model. Therefore, we introduce the following election strategy to select neurons with high scores in both base and fine-tuned models:
\begin{equation}
    \mathcal{T}_i^{r_i}=\mathcal{N}_i^{r_i}\cap \mathcal{N}_{\rm{base}}^{r_i}.
    \label{vote}
\end{equation}
\emph{Remark}. We compare different choosing methods, including scoring low or high in base or fine-tuned model in Section~\ref{sec:ablation} and find that Equation \ref{vote} achieves the best performance.





{\textbf{Disjoint}}. As important neurons from different task vectors may conflict with each other at the same position, we use the set difference to disjoint the neurons from others to prevent interference:
\begin{equation}
    \text{Disjoint}(\mathcal{T}^{r_i}_{i})=\mathcal{T}^{r_i}_{i}-\mathop{\cup}\limits_{{J}\subsetneqq [K],|J|\geq 2}\mathop{\cap}\limits_{j\in {J}}\mathcal{T}^{r_j}_{j}.
    \label{disjoint_safety}
\end{equation}

Next, we construct a mask $\boldsymbol{m}_i\in\mathbb{R}^D$ to implement disjoint in the merging process. Specifically, this mask $\boldsymbol{m}_i$ is used to select neurons from $\mathcal{T}_i$. The mask ratio is $r_i$, where $r\in(0,1]$. The mask $\boldsymbol{m}_i$ can be derived from:
\begin{equation}
    \boldsymbol{m}_{i,d}=\begin{aligned} &\left\{ \begin{array}{ll} 1, & \text{if } d\in \text{Disjoint}(\mathcal{T}_{i}^{r_i}), \\ 0, & \text{otherwise}. \end{array} \right. \end{aligned}
    \label{mask_safety}
\end{equation}


% \subsection{Merging Models with Masks}
{\textbf{Merging}}. The final
merged task vector $\boldsymbol{\tau}_m$ is as follows:
\begin{equation}
    \boldsymbol{\tau}_m= \sum_i \lambda_i\boldsymbol{\tau}_{i}\odot\boldsymbol{m}_i.
    \label{merged_task_vector}
\end{equation}
We summarize the workflow in Algorithm \ref{alg1}.



\section{Experiments}
\label{sec:experiment}

Experiments are carried out on NVIDIA RTX4090 GPUs using PyTorch 2.2.0 \cite{paszke2019pytorch} and the rotation detection tool kits: MMRotate 1.0.0 \cite{zhou2022mmrotate}. All the experiments follow the same hyper-parameters (learning rate, batch size, optimizer, etc.).

Average precision (AP) is adopted as the primary metric. All the models are configured upon ResNet50 \cite{he2016deep} and trained with AdamW \cite{loshchilov2018decoupled}.
\textbf{1) Learning rate.} Initialized at 5e-5, warm-up for 500 iterations, and divided by ten at each decay step. 
\textbf{2) Epochs.} 72 for HRSC; 12 for the others.
\textbf{3) Augmentation.} Random rotation/flip for HRSC; random flip for the others.
\textbf{4) Image size.} Split into 1,024 $\times$ 1,024 with an overlap of 200 for DOTA/FAIR1M/STAR; scaled to 800 $\times$ 800 for others.
\textbf{5) Multi-scale.} All experiments evaluated without multi-scale technique \cite{zhou2022mmrotate}. 
\textbf{6) Datasets.} Six remote sensing and one retail scene datasets, covering all datasets used by the main counterparts \cite{yu2024point2rbox, luo2024pointobb, cao2023p2rbox}:

\begin{table*}[!tb]
\fontsize{8.5pt}{10pt}\selectfont
\setlength{\tabcolsep}{0.65mm}
\setlength{\aboverulesep}{0.4ex}
\setlength{\belowrulesep}{0.4ex}
\setlength{\abovecaptionskip}{1.5mm}
\centering
\begin{tabular}{l|c|c|c|c|c|c|c|c|c|c}
\toprule
{\textbf{Methods}} & {*} & {\textbf{\,DOTA-v1.0\,}} & {\textbf{\,DOTA-v1.5\,}} & {\textbf{\,DOTA-v2.0\,}} & {\textbf{~~DIOR~~}} & {\textbf{~~HRSC~~}} & {\textbf{\,FAIR1M\,}} & {\textbf{~~STAR~~}} & {\textbf{\,SKU110K\,}} & {\textbf{~~RSAR~~}} \\
\hline
\rowcolor{gray!20} \multicolumn{11}{l}{$\blacktriangledown$ \textit{RBox-supervised OOD}} \\ \hline
RetinaNet (2017) \cite{lin2017focal} & \checkmark & 68.69 & 60.57        & 47.00 & 54.96 & 84.49   & 37.67   & 21.80 & 78.50 & 57.67  \\
GWD (2021) \cite{yang2021rethinking} & \checkmark & 71.66 & 63.27        & 48.87 & 57.60 & 86.67   & 39.11   & 25.30 & 79.16 & 57.80 \\
FCOS (2019) \cite{tian2019fcos} & \checkmark & 72.44 & 64.53        & 51.77    &  59.83  & 88.99  & 41.25   & \textbf{28.10} & 80.09 & \textbf{66.66} \\
S$^2$A-Net (2022) \cite{han2022align} & \checkmark & \textbf{75.81} & \textbf{66.53} & \textbf{52.39} & \textbf{61.41} & \textbf{90.10} & \textbf{42.44}   & 27.30 & \textbf{80.36} & 66.47 \\
\hline
\rowcolor{gray!20} \multicolumn{11}{l}{$\blacktriangledown$ \textit{HBox-supervised OOD}} \\ \hline
Sun et al. (2021) \cite{sun2021oriented} & $\times$ & 38.60 & - & - & - & - & - & - & - & - \\
KCR (2023) \cite{zhu2023knowledge} & \checkmark & - & - & - & - &  79.10  & -  & - & - & -  \\
H2RBox (2023) \cite{yang2023h2rbox} & \checkmark & 70.05 & 61.70        & 48.68    & 57.80 &  7.03  & 35.94  & 17.20 & 57.15 & 49.92    \\
H2RBox-v2 (2023) \cite{yu2023h2rboxv2} & \checkmark & 72.31 & 64.76 & 50.33 & 57.64 & \textbf{89.66} & \textbf{42.27} & \textbf{27.30} & \textbf{70.70} & \textbf{65.16} \\
AFWS (2024) \cite{lu2024afws} & \checkmark & \textbf{72.55} & \textbf{65.92} & \textbf{51.73} & \textbf{59.07} & - & 41.80 & - & - & - \\
\hline
\rowcolor{gray!20} \multicolumn{11}{l}{$\blacktriangledown$ \textit{Point-supervised OOD}} \\ \hline
P2RBox (2024) \cite{cao2023p2rbox}$^\dagger$ & $\times$ & \underline{59.04} & -        & - & - & -   & -  & -  & - & -  \\
PointSAM (2024) \cite{liu2024pointsam}$^\dagger$ & $\times$ & - & - & - & \textbf{46.20} & -   & -  & -  & - & - \\
PointOBB (2024) \cite{luo2024pointobb} & $\times$ & 30.08 & 10.66        & 5.53     &  37.31  & -   & 11.19 & 9.19  & - & 13.80    \\
Point2RBox+SK (2024) \cite{yu2024point2rbox}$^\dagger$ & \checkmark & 40.27 & 30.51        & 23.43    & 27.34 & 79.40   & 20.03 & 7.86  & 3.41 & 27.81    \\
PointOBB-v2 (2025) \cite{ren2024pointobbv2} & $\times$ & 41.68 & 30.59        & 20.64    &  39.56  & -   & 13.36 & 9.00  & 56.63 & 18.99   \\
PointOBB-v3 (2025) \cite{zhang2025pointobbv3} & $\checkmark$ & 41.20 & 31.25 & 22.82 & 37.60 & - & 11.42  & 11.31 & - & 15.84 \\
PointOBB-v3 (2025) \cite{zhang2025pointobbv3} & $\times$ & 49.24 & 33.79 & 23.52 & 40.18 & - & 18.35 & \underline{12.85} & - & 22.60 \\
\rowcolor{gray!20} Point2RBox-v2 (ours) & \checkmark & 51.00 & \underline{39.45} & \underline{27.11} & 34.70 & \underline{82.67} & \underline{25.72} & 7.80 & \underline{64.00} & \underline{28.60}
 \\
\rowcolor{gray!20} Point2RBox-v2 (ours) & $\times$ & \textbf{62.61} & \textbf{54.06}        & \textbf{38.79}   & \underline{44.45}  & \textbf{86.15}   & \textbf{34.71}  & \textbf{14.20} & \textbf{65.64} & \textbf{30.90}    \\
\bottomrule
\specialrule{0pt}{2pt}{0pt}
\multicolumn{11}{l}{$^*$Comparison tracks: \checkmark = End-to-end training and testing; $\times$ = Generating pseudo labels to train the FCOS detector (two-stage training).} \\
\multicolumn{11}{l}{$^\dagger$Using additional priors. P2RBox/PointSAM: Pre-trained SAM model; Point2RBox+SK: One-shot sketches for each class.} \\
\bottomrule
\end{tabular}
\caption{Accuracy (AP$_{50}$) comparisons on the DOTA-v1.0/1.5/2.0, DIOR, HRSC, FAIR1M, STAR, SKU110K, and RSAR datasets.}
\label{tab:exp_other}
\vspace{-4pt}
\end{table*}

\begin{itemize}
    \item \textbf{DOTA \cite{xia2018dota}.} DOTA-v1.0 has 2,806 aerial images annotated with 15 categories, while DOTA-v1.5/2.0 are the extended versions with more small objects and categories.
    
    \item \textbf{DIOR \cite{cheng2022anchor}.} It is an aerial image dataset re-annotated with RBoxes based on its original HBox version \cite{li2020object}, with a high variation in object size and high intra‐class diversity. 

    \item \textbf{HRSC \cite{liu2017hrsc}.} It contains ship instances on the sea and inshore. The train/val/test set includes 436/181/444 images.

    \item \textbf{FAIR1M \cite{sun2022fair1m}.} It has more than 1 million instances and more than 40,000 images for fine-grained object recognition in remote sensing imagery, annotated with 37 categories. The results are evaluated on FAIR1M-1.0.

    \item \textbf{STAR \cite{li2024star}.} It is extensive for scene graph generation, covering more than 210,000 objects with diverse spatial resolutions, classified into 48 fine-grained categories and precisely annotated with oriented bounding boxes. 

    \item \textbf{SKU110K \cite{pan2020dynamic}.} It focuses on the detection of densely packed retail scenes with 110,712 objects in 11,762 images. The density reaches 86 instances per image. 

    \item \textbf{RSAR \cite{zhang2025rsar}.} It is a remote sensing dataset based on Synthetic Aperture Radar (SAR) imagery with 6 categories.

\end{itemize}

\begin{table*}[!tb]
\fontsize{8.5pt}{10pt}\selectfont
\setlength{\tabcolsep}{2.08mm}
\setlength{\aboverulesep}{0.4ex}
\setlength{\belowrulesep}{0.4ex}
\setlength{\abovecaptionskip}{1.5mm}
\hspace{1pt}
\begin{minipage}[t]{0.315\linewidth}
\centering
\begin{tabular}{c|cc|cc}
\toprule
\multirow{2}{*}{$w_\text{O}$} & \multicolumn{2}{c|}{\textbf{DOTA}} & \multicolumn{2}{c}{\textbf{HRSC}} \\
                  & {E2E} & {FCOS} & {E2E} & {FCOS} \\ \midrule
3  & 48.76 & 61.62 & 81.85 & 84.36 \\
5  & 49.81 & 62.44 & 82.46 & 85.76 \\
\rowcolor{gray!20} 10 & \textbf{51.00} & \textbf{62.61} & \textbf{82.67} & \textbf{86.15} \\
30 & 45.88 & 57.83 & 81.56 & 85.61 \\
\bottomrule
\end{tabular}
\caption{Ablation with the weight of $\mathcal{L}_\text{O}$.}
\label{tab:abl_lo}
\end{minipage}
\quad
\begin{minipage}[t]{0.315\linewidth}
\centering
\begin{tabular}{c|cc|cc}
\toprule
\multirow{2}{*}{$w_\text{W}$} & \multicolumn{2}{c|}{\textbf{DOTA}} & \multicolumn{2}{c}{\textbf{HRSC}} \\
                  & {E2E} & {FCOS} & {E2E} & {FCOS} \\ \midrule
3  & 50.85 & 56.78 & 78.42 & 83.49 \\
\rowcolor{gray!20} 5  & \textbf{51.00} & \textbf{62.61} & \textbf{82.67} & \textbf{86.15} \\
10 & 49.15 & 60.54 & 30.37 & 35.13 \\
30 & 42.84 & 52.53 & 23.89 & 25.91 \\
\bottomrule
\end{tabular}
\caption{Ablation with the weight of $\mathcal{L}_\text{W}$.}
\label{tab:abl_lw}
\end{minipage}
\quad
\begin{minipage}[t]{0.315\linewidth}
\setlength{\tabcolsep}{2.04mm}
\centering
\begin{tabular}{c|cc|cc}
\toprule
\multirow{2}{*}{$w_\text{E}$} & \multicolumn{2}{c|}{\textbf{DOTA}} & \multicolumn{2}{c}{\textbf{HRSC}} \\
                  & {E2E} & {FCOS} & {E2E} & {FCOS} \\ \midrule
0.1 & 48.75 & 57.62 & 34.71 & 39.45 \\
\rowcolor{gray!20} 0.3 & 51.00 & 62.61 & \textbf{82.67} & \textbf{86.15} \\
0.5 & \textbf{51.36} & \textbf{62.63} & 76.85 & 85.22 \\
1.0 & 49.05 & 60.63 & 56.59 & 59.59 \\
\bottomrule
\end{tabular}
\caption{Ablation with the weight of $\mathcal{L}_\text{E}$.}
\label{tab:abl_le}
\end{minipage}
\vspace{-4pt}
\end{table*}

\begin{table*}[!tb]
\fontsize{8.5pt}{10pt}\selectfont
\setlength{\tabcolsep}{2.04mm}
\setlength{\aboverulesep}{0.4ex}
\setlength{\belowrulesep}{0.4ex}
\setlength{\abovecaptionskip}{1.5mm}
\hspace{1pt}
\begin{minipage}[t]{0.315\linewidth}
\centering
\begin{tabular}{c|cc|cc}
\toprule
\multirow{2}{*}{$w_\text{ss}$} & \multicolumn{2}{c|}{\textbf{DOTA}} & \multicolumn{2}{c}{\textbf{HRSC}} \\
                  & {E2E} & {FCOS} & {E2E} & {FCOS} \\ \midrule
0.1 & 49.28 & 59.66 & 73.66 & 78.92 \\
\rowcolor{gray!20} 1.0 & \textbf{51.00} & \textbf{62.61} & \textbf{82.67} & \textbf{86.15} \\
3.0 & 49.15 & 59.20 & 1.30  & 1.65 \\
\bottomrule
\end{tabular}
\caption{Ablation with the weight of $\mathcal{L}_\text{ss}$.}
\label{tab:abl_lss}
\end{minipage}
\quad
\begin{minipage}[t]{0.647\linewidth}
\setlength{\tabcolsep}{3.05mm}
\centering
\begin{tabular}{c|c|c||c|c|c}
\toprule
{R / F / S} & {\textbf{DOTA}} & {\textbf{HRSC}} & {R / F / S} & {\textbf{DOTA}} & {\textbf{HRSC}} \\
 \midrule
90\% / 10\% / 0\% & 60.42 & 85.46 & 80\% / 20\% / 0\%  & 59.46 & 84.73 \\
75\% / 0\% / 25\% & 60.79 & 86.22 & 60\% / 15\% / 25\% & 62.38 & 84.21 \\
\cellcolor{gray!20}68\% / 7\% / 25\% & \cellcolor{gray!20}\textbf{62.61} & \cellcolor{gray!20}\textbf{86.15} & 38\% / 37\% / 25\% & 45.87 & 8.56  \\
45\% / 5\% / 50\% & 60.55 & 85.34 & 40\% / 10\% / 50\% & 60.49 & 10.74 \\
\bottomrule
\end{tabular}
\caption{Ablation with the proportion of augmented views in self-supervision.}
\label{tab:abl_pro}
\end{minipage}
\vspace{-10pt}
\end{table*}

\subsection{Main Results on DOTA-v1.0}
\label{sec:experiment-main}

Table \ref{tab:exp_dota} compares Point2RBox-v2 with the state-of-the-art methods, which can be categorized into two tracks: 

\textbf{1) End-to-end training.} These methods apply the trained weakly-supervised detector directly to the test set. Without relying on priors, our approach demonstrates an improvement of 16.93\% (51.00\% vs. 34.07\%) compared to Point2RBox. Even when compared to Point2RBox+SK, which incorporates additional data-side priors (i.e. one-shot examples for each class), our method still outperforms it by 10.73\% (51.00\% vs. 40.27\%).

\textbf{2) Two-stage training.} These methods generate RBox labels on train/val sets, with which the FCOS detector is trained. In this two-stage mode, Point2RBox-v2 achieves an accuracy of 62.61\%, considerably surpassing PointOBB series. Remarkably, it even outperforms the SAM-powered method P2RBox by 3.57\% (62.61\% vs. 59.04\%).

\textbf{Class-wise analysis.} The FCOS detector trained with labels generated by Point2RBox-v2 achieves accuracy nearly equivalent to RBox-supervised FCOS across six high-density categories: SH (86.9\% vs. 87.1\%), SV (79.6\% vs. 79.8\%), LV (76.3\% vs. 79.8\%), PL (88.0\% vs. 89.1\%), ST (82.9\% vs. 84.6\%), and TC (89.1\% vs. 90.4\%). Interestingly, these six high-density categories account for 88\% of DOTA instances. By annotating these categories with points and generating RBoxes using Point2RBox-v2 while labeling the other sparse categories with RBoxes, we can significantly reduce annotation labor without sacrificing much accuracy, highlighting the valuable role our method can play.

\begin{figure*}[t!]
\setlength{\abovecaptionskip}{1.2mm}
\centering
\includegraphics[width=0.96\linewidth]{figs/case.pdf}
\caption{Qualitative analysis on failed cases and overlap cases.}
\label{fig:case}
\vspace{-6pt}
\end{figure*}

\subsection{Results on More Datasets}

The results are displayed in Table \ref{tab:exp_other}.
On more challenging DOTA-v1.5/2.0, Point2RBox-v2 presents a similar trend, 23.47\%/18.15\% higher than PointOBB-v2 in the pseudo-generation track. 
On the ship detection dataset HRSC, the gap between Point2RBox-v2 and RBox-supervised FCOS is only 2.84\% (86.15\% vs. 88.99\%).
DIOR is relatively sparse, leading to less improvement with our methods---lower than PointSAM (44.45\% vs. 46.20\%) but still higher than methods that do not use SAM. 
Our method also provides competitive performance on fine-grained datasets FAIR1M and STAR. 
In addition to remote sensing scenarios, we carry out experiments on SKU110K for densely packed retail scenes. Existing point-supervised methods struggle in this case, whereas Point2RBox-v2 achieves performance on par with HBox-supervised H2RBox (65.64\% vs. 57.15\%).

\begin{table}[!tb]
\fontsize{8.5pt}{10pt}\selectfont
\setlength{\tabcolsep}{1.78mm}
\setlength{\aboverulesep}{0.4ex}
\setlength{\belowrulesep}{0.4ex}
\setlength{\abovecaptionskip}{1.5mm}
\centering
\begin{tabular}{ccccc|cc|cc}
\toprule
\multicolumn{5}{c|}{\textbf{Modules}} & \multicolumn{2}{c|}{\textbf{DOTA}} & \multicolumn{2}{c}{\textbf{HRSC}} \\
$\mathcal{L}_\text{O}$ & $\mathcal{L}_\text{W}$ & $\mathcal{L}_\text{ss}$ & $\mathcal{L}_\text{E}$ & \textit{CP} & {E2E} & {FCOS} & {E2E} & {FCOS} \\ \midrule
\checkmark & & & & & 0.00 & 0.00 & 0.00 & 0.00 \\
\checkmark & \checkmark & & & & 41.54 & 52.98 & 17.96 & 19.64 \\
\checkmark & \checkmark & \checkmark & & & 46.64 & 54.26 & 18.10 & 22.13 \\
\checkmark & \checkmark & \checkmark & \checkmark & & 49.55 & 61.88 & 78.79 & 83.79 \\
& \checkmark & \checkmark & \checkmark & \checkmark & 48.58 & 59.56 & 20.35 & 24.76 \\
\checkmark & & \checkmark & \checkmark & \checkmark & 38.94 & 48.44 & 11.64 & 14.93 \\
\checkmark & \checkmark & \checkmark & & \checkmark & 47.08 & 55.05 & 19.58 & 21.78 \\
\rowcolor{gray!20} \checkmark & \checkmark & \checkmark & \checkmark & \checkmark & \textbf{51.00} & \textbf{62.61} & \textbf{82.67} & \textbf{86.15} \\
\bottomrule
\end{tabular}
\caption{Ablation with incremental addition of modules.}
\label{tab:abl_mod}
\vspace{-4pt}
\end{table}

\begin{table}[!tb]
\fontsize{8.5pt}{10pt}\selectfont
\setlength{\tabcolsep}{2.85mm}
\setlength{\aboverulesep}{0.4ex}
\setlength{\belowrulesep}{0.4ex}
\setlength{\abovecaptionskip}{1.5mm}
\centering
\begin{tabular}{c|c|c||c|c|c}
\toprule
16 & \cellcolor{gray!20}$K\!=\!24$ & 32 & 1.2 & \cellcolor{gray!20}$\beta\!=\!1.6$ & 2.0 \\ \midrule
50.87 & \cellcolor{gray!20}\textbf{51.00} & 48.08 & 48.14 & \cellcolor{gray!20}51.00 & \textbf{51.33} \\
\bottomrule
\end{tabular}
\caption{Ablation with $K$ and $\beta$ in edge loss on DOTA (E2E).}
\label{tab:abl_edgeparam}
\vspace{-4pt}
\end{table}

\begin{table}[!tb]
\fontsize{8.5pt}{10pt}\selectfont
\setlength{\tabcolsep}{1.75mm}
\setlength{\aboverulesep}{0.4ex}
\setlength{\belowrulesep}{0.4ex}
\setlength{\abovecaptionskip}{1.5mm}
\centering
\begin{tabular}{c|cc|cc|cc}
\toprule
\multirow{2}{*}{$\sigma$} & \multicolumn{2}{c|}{Point2RBox} & \multicolumn{2}{c|}{PointOBB-v2} & \multicolumn{2}{c}{Point2RBox-v2} \\
 & {\textbf{DOTA}} & {\textbf{HRSC}} & {\textbf{DOTA}} & {\textbf{HRSC}} & {\textbf{DOTA}} & {\textbf{HRSC}} \\ \midrule
0\%  & 40.27 & 79.40 & 44.85 & - & 62.61 & 86.15 \\
10\% & 39.60 & 78.81 & 42.30 & - & 61.58 & 85.76 \\
30\% & 38.42 & 78.28 & 38.46 & - & 60.31 & 85.71 \\
\bottomrule
\end{tabular}
\caption{Ablation with the inaccuracy in point annotations.}
\label{tab:abl_noise}
\vspace{-10pt}
\end{table}

\subsection{Ablation Studies}
\label{sec:experiment-ablation}

Tables \ref{tab:abl_lo}-\ref{tab:abl_noise} display the ablation studies on DOTA-v1.0 and HRSC. ``E2E'' denotes end-to-end training; ``FCOS'' denotes two-stage training (i.e. generating pseudo labels to train FCOS). The final values adopted are highlighted in gray.

\textbf{Weight of each loss.} Tables \ref{tab:abl_lo}-\ref{tab:abl_le} determine the weights of the proposed losses. Based on these experiments, the weights $(w_\text{O},w_\text{W},w_\text{E},w_\text{ss})$ are set to $(10, 5, 0.3, 1)$.

\textbf{Proportion of augmented views.} Table \ref{tab:abl_pro} studies the proportion between rotation, flip, and scale. The results are reported with two-stage training (FCOS). Based on the results, the proportion is set to 68\%, 7\%, and 25\%.

\textbf{Incremental addition of modules.} Table \ref{tab:abl_mod} demonstrates the constraints from Gaussian and Voronoi achieve an accuracy of 52.98\% on DOTA. Adding consistency loss and edge loss further boosts it to 54.26\% and 61.88\%, respectively, whereas the improvement from copy-paste is 0.73\%. We also demonstrate the impact of omitting each core loss.

\textbf{Edge loss parameters.} We set $K=24$ and $\beta=1.6$ as they are observed to discern the correct edges during code development. Table \ref{tab:abl_edgeparam} provides a more precise ablation.

\textbf{Annotation inaccuracy.} We offset the annotated points by a noise from the uniform distribution $\left[-\sigma H, +\sigma H \right ]$, where $H$ is the height of objects. Table \ref{tab:abl_noise} shows that the AP$_{50}$ of Point2RBox-v2 decreases by less than 3\% when noise is added to point annotations, demonstrating the robustness of the proposed learning mechanisms.

\subsection{More Discussions}
\label{sec:experiment-discussions}

The qualitative analysis on the failed/overlap cases is shown in Fig. \ref{fig:case}. \textbf{1) Failed cases.} Although our method performs well overall, it struggles with certain categories that are sparse and not constrained by other objects. \textbf{2) Overlap cases.} 
Minimizing overlap as a soft constraint during training does not entirely eliminate overlap. Once trained, the model remains robust to some overlap during inference.

\section{Discussion}
\label{section:discussion}


\subsection{Practical Implications for Feedforward Prompting}

Of course, prompting an LLM continuously before the user submits their prompt is significantly most costly over submitting the prompt just once, once the user is ready.

% But user might not be ready, and the cognitive costs is pretty heavy.


\subsection{}


% Does this work well with Chain of Thought actually?
% Maybe this approach will actually incentivize self-prompt-chaining???
% What are the implications of this?


% A benefit of this is certainly more transparency in the LLM
% LLM is so flexible that adding this kind of structure is still okay for the LLM



% What's more costly, entering a prompt, then responding and saying, no i want this, or typing a prompt, and tuning the prompt/expected output to reduce message exchanges?

% Learning to become a better prompter. One is by trial and error experience. Perhaps another is through this feedforward that tells you what you might be able to anticipate.
{
    \small
    \bibliographystyle{ieeenat_fullname}
    \bibliography{main}
}


\end{document}
