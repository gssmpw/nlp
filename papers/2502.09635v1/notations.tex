\section{Mathematical Notations}
\label{sec:notations}

Here we provide a summary of main mathematical notations used in the main paper in Table \ref{table:notation_table}.

\begin{table*}[t]
	\centering
	\caption{Summary of mathematical notations.}
	%\vspace{-0.3cm}
	%\resizebox{\columnwidth}{!}{
		\begin{tabular}{c|l}
			\toprule
			Notation  & Description  \\
			\hline
			$ \mathcal{D} $ & a fact-checking dataset \\
			$ \mathcal{X} $ & a set of $ N=|\mathcal{X}| $ claims \\
			$ \mathcal{E} $ & a corpus of evidence sentences \\
			$ \mathcal{C} $ & a set of contextual documents \\
                $ \mathcal{R} $ & a set of referential documents \\
			$ \mathcal{N}_{\text{ref}}(e) $ & evidence sentence $ e $'s referential documents \\
                $ \mathcal{N}_{\text{evid}}(x) $ & claim $ x $'s retrieved evidence sentences \\
                $ \mathcal{Y} $ & a set of labels \\
                $ \textbf{h}_{e,\text{CLS}}^{(l)} $ & evidence sentence $ e $'s [CLS] token embedding at the $ l $-th step \\
                $ \hat{\textbf{h}}_{e}^{(l)} $ & evidence sentence $ e $'s embedding after intra-layer reasoning \\
                $ \hat{\textbf{h}}_{c}^{(l)} $ & aggregated contextual document embedding \\
                $ \hat{\textbf{h}}_{r}^{(l)} $ & aggregated referential document embedding \\
                $ \widehat{\textbf{H}}_{e}^{(l)} $ & evidence sentence $ e $'s augmented embedding matrix \\
                $ \bm{\pi}_{m,y} $ & the $ m $-th prompt embedding for class $ y $ \\
                $ \textbf{h}_{m,y} $ & the $ m $-th base prompt embedding for class $ y $ \\
			\bottomrule
		\end{tabular}
	%}
	%\vspace{-0.5cm}
	\label{table:notation_table}
\end{table*}