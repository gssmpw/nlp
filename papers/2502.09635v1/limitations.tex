\section*{Limitations}

Here we identify two limitations of our work in terms of dataset and evidence type.

\textbf{Dataset.} Our model is proposed to incorporate contextual and referential documents of evidence sentences. We assume that the contextual and referential documents of evidence sentences are available in the dataset, or the dataset provides identifiers for evidence sentences, such as PubMed ID, so that we can use these identifiers to search their contextual and referential documents online. In Appendix \ref{sec:dataset_preprocessing}, we provide details on how to use identifiers to obtain contextual and referential documents. If the given dataset does not provide contextual or referential documents, or the identifiers of evidence sentences are not available, our model will reason within evidence sentences for fact-checking.

\textbf{Evidence type.} Following existing textual fact-checking models, we propose our model to reason over textual evidence sentences only. Our model is not proposed for tabular or multi-modal evidence, thus cannot reason over these types of evidence for fact-checking. One potential future work would be to extend our three-layer evidence graph to a multi-modal graph for evidence reasoning.