%Version 3 December 2023
% See section 11 of the User Manual for version history
%
%%%%%%%%%%%%%%%%%%%%%%%%%%%%%%%%%%%%%%%%%%%%%%%%%%%%%%%%%%%%%%%%%%%%%%
%%                                                                 %%
%% Please do not use \input{...} to include other tex files.       %%
%% Submit your LaTeX manuscript as one .tex document.              %%
%%                                                                 %%
%% All additional figures and files should be attached             %%
%% separately and not embedded in the \TeX\ document itself.       %%
%%                                                                 %%
%%%%%%%%%%%%%%%%%%%%%%%%%%%%%%%%%%%%%%%%%%%%%%%%%%%%%%%%%%%%%%%%%%%%%

%%\documentclass[referee,sn-basic]{sn-jnl}% referee option is meant for double line spacing

%%=======================================================%%
%% to print line numbers in the margin use lineno option %%
%%=======================================================%%

%%\documentclass[lineno,sn-basic]{sn-jnl}% Basic Springer Nature Reference Style/Chemistry Reference Style

%%======================================================%%
%% to compile with pdflatex/xelatex use pdflatex option %%
%%======================================================%%

%%\documentclass[pdflatex,sn-basic]{sn-jnl}% Basic Springer Nature Reference Style/Chemistry Reference Style


%%Note: the following reference styles support Namedate and Numbered referencing. By default the style follows the most common style. To switch between the options you can add or remove “Numbered” in the optional parenthesis. 
%%The option is available for: sn-basic.bst, sn-vancouver.bst, sn-chicago.bst%  
 
%%\documentclass[pdflatex,sn-nature]{sn-jnl}% Style for submissions to Nature Portfolio journals
%%\documentclass[pdflatex,sn-basic]{sn-jnl}% Basic Springer Nature Reference Style/Chemistry Reference Style
\documentclass[pdflatex,sn-mathphys-num]{sn-jnl}% Math and Physical Sciences Numbered Reference Style 
%%\documentclass[pdflatex,sn-mathphys-ay]{sn-jnl}% Math and Physical Sciences Author Year Reference Style
%%\documentclass[pdflatex,sn-aps]{sn-jnl}% American Physical Society (APS) Reference Style
%%\documentclass[pdflatex,sn-vancouver,Numbered]{sn-jnl}% Vancouver Reference Style
%%\documentclass[pdflatex,sn-apa]{sn-jnl}% APA Reference Style 
%%\documentclass[pdflatex,sn-chicago]{sn-jnl}% Chicago-based Humanities Reference Style

%%%% Standard Packages
%%<additional latex packages if required can be included here>

\usepackage{graphicx}%
\usepackage{multirow}%
\usepackage{amsmath,amssymb,amsfonts}%
\usepackage{amsthm}%
\usepackage{mathrsfs}%
\usepackage[title]{appendix}%
\usepackage{xcolor}%
\usepackage{textcomp}%
\usepackage{manyfoot}%
\usepackage{booktabs}%
\usepackage{algorithm}%
\usepackage{algorithmicx}%
\usepackage{algpseudocode}%
\usepackage{listings}%
%%%%

%%%%%=============================================================================%%%%
%%%%  Remarks: This template is provided to aid authors with the preparation
%%%%  of original research articles intended for submission to journals published 
%%%%  by Springer Nature. The guidance has been prepared in partnership with 
%%%%  production teams to conform to Springer Nature technical requirements. 
%%%%  Editorial and presentation requirements differ among journal portfolios and 
%%%%  research disciplines. You may find sections in this template are irrelevant 
%%%%  to your work and are empowered to omit any such section if allowed by the 
%%%%  journal you intend to submit to. The submission guidelines and policies 
%%%%  of the journal take precedence. A detailed User Manual is available in the 
%%%%  template package for technical guidance.
%%%%%=============================================================================%%%%

%% as per the requirement new theorem styles can be included as shown below
\theoremstyle{thmstyleone}%
\newtheorem{theorem}{Theorem}%  meant for continuous numbers
%%\newtheorem{theorem}{Theorem}[section]% meant for sectionwise numbers
%% optional argument [theorem] produces theorem numbering sequence instead of independent numbers for Proposition
\newtheorem{proposition}[theorem]{Proposition}% 
%%\newtheorem{proposition}{Proposition}% to get separate numbers for theorem and proposition etc.

\theoremstyle{thmstyletwo}%
\newtheorem{example}{Example}%
\newtheorem{remark}{Remark}%

\theoremstyle{thmstylethree}%
\newtheorem{definition}{Definition}%

\raggedbottom
%%\unnumbered% uncomment this for unnumbered level heads

\begin{document}

% \title[Article Title]{A Multimodal Framework for Material Discovery: Leveraging Graphs and Large Language Models}


\title[Article Title]{Supplementary Information:A Multimodal Framework for Material Discovery: Infusing Large Language Models with Atomic Structures}

%%=============================================================%%
%% GivenName	-> \fnm{Joergen W.}
%% Particle	-> \spfx{van der} -> surname prefix
%% FamilyName	-> \sur{Ploeg}
%% Suffix	-> \sfx{IV}
%% \author*[1,2]{\fnm{Joergen W.} \spfx{van der} \sur{Ploeg} 
%%  \sfx{IV}}\email{iauthor@gmail.com}
%%=============================================================%%

\author*[1]{\fnm{Yingheng} \sur{Tang}}\email{ytang4@lbl.gov}

\author*[2]{\fnm{Wenbin} \sur{Xu}}\email{wenbinxu@lbl.gov}

\author[4]{\fnm{Jie} \sur{Cao}}

\author[3]{\fnm{Jianzhu} \sur{Ma}}

\author*[5]{\fnm{Weilu} \sur{Gao}}\email{weilugao@utah.edu}

\author[2]{\fnm{Steve} \sur{Farrell}}

\author[6,7]{\fnm{Benjamin} \sur{Erichson}}

\author[6,7,8]{\fnm{Michael W.} \sur{Mahoney}}

\author[1]{\fnm{Andy} \sur{Nonaka}}


\author*[1]{\fnm{Zhi} \sur{Yao}}\email{jackie\_zhiyao@lbl.gov}


\affil[1]{Applied Mathematics and Computational Research Division, Lawrence Berkeley National Laboratory, Berkeley, CA, USA}

\affil[2]{National Energy Research Scientific Computing Center, Lawrence Berkeley National Laboratory, Berkeley, CA, USA}

\affil[3]{Institute for AI Industry Research, Tsinghua University, Beijing, China}

\affil[4]{NSF National AI Institute for Student-AI Teaming, University of Colorado at Boulder, Boulder, USA}

\affil[5]{Department of Electrical and Computer Engineering, The University of Utah, Salt Lake City, UT, USA}

\affil[6]{Scientific Data Division, Lawrence Berkeley National Laboratory, Berkeley, CA, USA}


\affil[7]{International Computer Science Institute, Berkeley, CA, USA}

\affil[8]{Department of Statistics, University of California at Berkeley, Berkeley, CA, USA}

\keywords{Foundational model, Large Language Model, multi-modal learning, Material discovery}

%%\pacs[JEL Classification]{D8, H51}

%%\pacs[MSC Classification]{35A01, 65L10, 65L12, 65L20, 65L70}

\maketitle

\newpage
\begin{figure}[hbt]
    \centering
    \includegraphics[width = \textwidth]{figure/SI_SiO2.png}
    \caption{UMAP visualization of structural embeddings extracted from the bridge model. ($Si,$ $O$, $Si_xO_y$)}
    \label{SI_fig:xbar}
\end{figure}

\newpage
\begin{figure}[hbt]
    \centering
    \includegraphics[width = \textwidth]{figure/training_scheme.png}
    \caption{Training Scheme.}
    \label{SI_fig:xbar}
\end{figure}



\newpage
\section*{\quad Instruction Templates (descriptive tasks)}

\begin{table}[h!]
\centering
\renewcommand{\arraystretch}{1.2} % Adjust row height
\setlength{\tabcolsep}{5pt} % Adjust column separation
\begin{tabular}{|p{3cm}|p{12cm}|}
\hline
\textbf{Task} & \textbf{Instruction Template} \\
\hline
\textbf{Reduced Formula} & 
\begin{tabular}[t]{@{}l@{}}
\textless material structure\textgreater What is the chemical formula for this material? \\
\textless material structure\textgreater Can you tell me the chemical formula of this material? \\
\textless material structure\textgreater Please provide the chemical formula for the material. \\
\textless material structure\textgreater What is the formula for this material? \\
\textless material structure\textgreater Could you tell me the formula of the material? \\
\textless material structure\textgreater What elements make up this material? \\
\textless material structure\textgreater How would you write the chemical formula of this material? \\
\textless material structure\textgreater What is the exact chemical formula of this material? \\
\textless material structure\textgreater Can you provide the chemical formula for this material? \\
\end{tabular} \\
\hline
\textbf{Space Group} & 
\begin{tabular}[t]{@{}l@{}}
\textless material structure\textgreater What is the space group for this material? \\
\textless material structure\textgreater To which space group does this material belong? \\
\textless material structure\textgreater Can you tell me the space group of this material? \\
\textless material structure\textgreater Please provide the space group for the material. \\
\textless material structure\textgreater What is the crystallographic space group of this material? \\
\textless material structure\textgreater How is the space group of this material classified? \\
\textless material structure\textgreater Can you specify the space group for this material? \\
\textless material structure\textgreater Could you tell me the space group classification of this material? \\
\textless material structure\textgreater Can you provide the space group information for this material? \\
\textless material structure\textgreater What is the space group number of this material? \\
\end{tabular} \\
\hline
\textbf{Crystal System} & 
\begin{tabular}[t]{@{}l@{}}
\textless material structure\textgreater What is the crystal system of this material? \\
\textless material structure\textgreater Can you tell me the crystal system of this material? \\
\textless material structure\textgreater Please provide the crystal system for the material. \\
\textless material structure\textgreater What crystal system does this material belong to? \\
\textless material structure\textgreater How is the crystal system of this material classified? \\
\textless material structure\textgreater Can you specify the crystal system for this material? \\
\textless material structure\textgreater What is the crystallographic system of this material? \\
\textless material structure\textgreater Could you tell me the crystal system classification of this material? \\
\textless material structure\textgreater Which crystallographic system does this material belong to? \\
\end{tabular} \\
\hline
\textbf{Generate} & 
\begin{tabular}[t]{@{}l@{}}
\textless material structure\textgreater Can you provide another material similar to this material? \\
\textless material structure\textgreater Is there another material like this material that you can provide? \\
\textless material structure\textgreater Can you show me a different material similar to this one? \\
\textless material structure\textgreater Can you generate another material similar to this one? \\
\end{tabular} \\
\hline
\textbf{General} & 
\begin{tabular}[t]{@{}l@{}}
\textless material structure\textgreater Can you describe this material? \\
\textless material structure\textgreater \textless s\textgreater \\
\end{tabular} \\
\hline
\end{tabular}
\end{table}






\newpage
\section*{Answer Templates (descriptive tasks)}

\begin{table}[h!]
\centering
\renewcommand{\arraystretch}{1.2} % Adjust row height
\setlength{\tabcolsep}{5pt} % Adjust column separation
\begin{tabular}{|p{3cm}|p{12cm}|}
\hline
\textbf{Task} & \textbf{Instruction Template} \\
\hline
\textbf{Reduced Formula} & 
\begin{tabular}[t]{@{}l@{}}
The chemical formula for this material is \textless material attribute \textgreater. \\
The chemical formula of this material is \textless material attribute \textgreater. \\
The chemical formula for the material is \textless material attribute \textgreater. \\
The formula for this material is \textless material attribute \textgreater. \\
The formula of the material is \textless material attribute \textgreater. \\
The elements that make up this material are represented as \textless material attribute \textgreater. \\
The chemical formula of this material is written as \textless material attribute \textgreater. \\
The exact chemical formula of this material is \textless material attribute \textgreater. \\
The chemical formula for this material is \textless material attribute \textgreater. \\
\end{tabular} \\
\hline
\textbf{Space Group} & 
\begin{tabular}[t]{@{}l@{}}
The space group for this material is \textless material attribute \textgreater. \\
This material belongs to the space group \textless material attribute \textgreater. \\
The space group of this material is \textless material attribute \textgreater. \\
The space group for the material is \textless material attribute \textgreater. \\
The crystallographic space group of this material is \textless material attribute \textgreater. \\
The space group of this material is classified as \textless material attribute \textgreater. \\
The space group for this material is specified as \textless material attribute \textgreater. \\
The space group classification of this material is \textless material attribute \textgreater. \\
The space group information for this material is \textless material attribute \textgreater. \\
The space group number of this material is \textless material attribute \textgreater. \\
\end{tabular} \\
\hline
\textbf{Crystal System} & 
\begin{tabular}[t]{@{}l@{}}
The crystal system of this material is \textless material attribute \textgreater. \\
The crystal system of this material is \textless material attribute \textgreater. \\
The crystal system for the material is \textless material attribute \textgreater. \\
This material belongs to the \textless material attribute \textgreater crystal system. \\
The crystal system of this material is classified as \textless material attribute \textgreater. \\
The crystal system for this material is specified as \textless material attribute \textgreater. \\
The crystallographic system of this material is \textless material attribute \textgreater. \\
The crystal system classification of this material is \textless material attribute \textgreater. \\
This material belongs to the \textless material attribute \textgreater crystallographic system. \\
\end{tabular} \\
\hline
\end{tabular}
\end{table}





\newpage
\section*{\quad Instruction Templates (property part1)}

\begin{table}[h!]
\centering
\renewcommand{\arraystretch}{1.2} % Adjust row height
\setlength{\tabcolsep}{5pt} % Adjust column separation
\begin{tabular}{|p{3cm}|p{12cm}|}
\hline
\textbf{Task} & \textbf{Instruction Template} \\
\hline
\textbf{Is Metal} & 
\begin{tabular}[t]{@{}l@{}}
\textless material structure\textgreater Is this material metal or non-metal? \\
\textless material structure\textgreater Can you tell me if this material is metal or not? \\
\textless material structure\textgreater What is the classification of this material: metal or non-metal? \\
\textless material structure\textgreater Is this material considered a metal? \\
\textless material structure\textgreater How is this material categorized: metal or non-metal? \\
\textless material structure\textgreater Could you specify if this material is metal or non-metal? \\
\textless material structure\textgreater Is the material metallic or non-metallic? \\
\textless material structure\textgreater Can you provide the classification of this material: metal or non-metal? \\
\textless material structure\textgreater Is this material identified as a metal or non-metal? \\
\textless material structure\textgreater What type of material is this: metal or non-metal? \\
\end{tabular} \\
\hline
\textbf{Direct Bandgap} & 
\begin{tabular}[t]{@{}l@{}}
\textless material structure\textgreater Does the material have a direct bandgap or indirect bandgap? \\
\textless material structure\textgreater Is the bandgap of this material direct or indirect? \\
\textless material structure\textgreater Can you tell me if this material has a direct or indirect bandgap? \\
\textless material structure\textgreater What type of bandgap does this material have: direct or indirect? \\
\textless material structure\textgreater Is this material characterized by a direct or indirect bandgap? \\
\textless material structure\textgreater Could you specify if the bandgap of this material is direct or indirect? \\
\textless material structure\textgreater Does this material exhibit a direct or indirect bandgap? \\
\textless material structure\textgreater Is the bandgap in this material direct or indirect? \\
\textless material structure\textgreater How is the bandgap of this material classified: direct or indirect? \\
\textless material structure\textgreater Is this a direct or indirect bandgap material? \\
\end{tabular} \\
\hline
\textbf{Stability} & 
\begin{tabular}[t]{@{}l@{}}
\textless material structure\textgreater Is this material stable? \\
\textless material structure\textgreater Can you tell me if this material is stable? \\
\textless material structure\textgreater What is the stability of this material? \\
\textless material structure\textgreater Please provide the stability information for this material. \\
\textless material structure\textgreater Is the material stable under standard conditions? \\
\textless material structure\textgreater Is this material thermodynamically stable? \\
\end{tabular} \\
\hline
\textbf{Experimental Observation} & 
\begin{tabular}[t]{@{}l@{}}
\textless material structure\textgreater Is the material experimentally observed or not? \\
\textless material structure\textgreater Can you tell me if the material is observed in experiments? \\
\end{tabular} \\
\hline
\textbf{Is Magnetic} & 
\begin{tabular}[t]{@{}l@{}}
\textless material structure\textgreater Is the material magnetic or not? \\
\textless material structure\textgreater Is the material magnetic or non-magnetic? \\
\textless material structure\textgreater Can you tell me if this material is magnetic? \\
\textless material structure\textgreater What is the magnetic nature of this material? \\
\textless material structure\textgreater Is this material classified as magnetic? \\
\textless material structure\textgreater Does this material have magnetic properties? \\
\textless material structure\textgreater Is this a magnetic or non-magnetic material? \\
\end{tabular} \\
\hline
\textbf{Magnetic Order} & 
\begin{tabular}[t]{@{}l@{}}
\textless material structure\textgreater What is the magnetic order of the material? \\
\textless material structure\textgreater Can you tell me the magnetic order of this material? \\
\textless material structure\textgreater Could you specify the magnetic order of the material? \\
\textless material structure\textgreater What type of magnetic order does this material have? \\
\textless material structure\textgreater Please provide the magnetic ordering of the material. \\
\textless material structure\textgreater What is the magnetic arrangement in this material? \\
\textless material structure\textgreater Could you tell me the type of magnetic order of this material? \\
\end{tabular} \\
\hline
\end{tabular}
\end{table}



\newpage
\section*{Answer Templates (property part1)}

\begin{table}[h!]
\centering
\renewcommand{\arraystretch}{1.2} % Adjust row height
\setlength{\tabcolsep}{5pt} % Adjust column separation
\begin{tabular}{|p{3cm}|p{11cm}|}
\hline
\textbf{Task} & \textbf{Instruction Template} \\
\hline
\textbf{Is Metal} & 
\begin{tabular}[t]{@{}l@{}}
This material is classified as \textless material attribute \textgreater. \\
This material is a \textless material attribute \textgreater. \\
The classification of this material is \textless material attribute \textgreater. \\
This material is considered \textless material attribute \textgreater. \\
This material is categorized as \textless material attribute \textgreater. \\
This material is specified as \textless material attribute \textgreater. \\
This material is \textless material attribute \textgreater. \\
The classification of this material is \textless material attribute \textgreater. \\
This material is identified as \textless material attribute \textgreater. \\
This type of material is \textless material attribute \textgreater. \\
\end{tabular} \\
\hline
\textbf{Direct Bandgap} & 
\begin{tabular}[t]{@{}l@{}}
The material has a \textless material attribute \textgreater bandgap. \\
The bandgap of this material is \textless material attribute \textgreater. \\
This material has a \textless material attribute \textgreater bandgap. \\
This material has a \textless material attribute \textgreater type of bandgap. \\
This material is characterized by a \textless material attribute \textgreater bandgap. \\
The bandgap of this material is specified as \textless material attribute \textgreater. \\
This material exhibits a \textless material attribute \textgreater bandgap. \\
The bandgap in this material is \textless material attribute \textgreater. \\
The bandgap of this material is classified as \textless material attribute \textgreater. \\
This is a \textless material attribute \textgreater bandgap material. \\
\end{tabular} \\
\hline
\textbf{Stability} & 
\begin{tabular}[t]{@{}l@{}}
This material is \textless material attribute \textgreater. \\
Yes, this material is \textless material attribute \textgreater. \\
The stability of this material is \textless material attribute \textgreater. \\
The stability information for this material is \textless material attribute \textgreater. \\
This material is \textless material attribute \textgreater under standard conditions. \\
This material is \textless material attribute \textgreater. \\
\end{tabular} \\
\hline
\textbf{Experimental Observation} & 
\begin{tabular}[t]{@{}l@{}}
The material is \textless material attribute \textgreater. \\
The material is \textless material attribute \textgreater. \\
\end{tabular} \\
\hline
\textbf{Is Magnetic} & 
\begin{tabular}[t]{@{}l@{}}
The material is \textless material attribute \textgreater. \\
This material is \textless material attribute \textgreater. \\
Yes, this material is \textless material attribute \textgreater. \\
The magnetic nature of this material is \textless material attribute \textgreater. \\
This material is classified as \textless material attribute \textgreater. \\
This material has \textless material attribute \textgreater properties. \\
This is a \textless material attribute \textgreater material. \\
\end{tabular} \\
\hline
\textbf{Magnetic Order} & 
\begin{tabular}[t]{@{}l@{}}
The magnetic order of the material is \textless material attribute \textgreater. \\
The magnetic order of this material is \textless material attribute \textgreater. \\
The magnetic order of the material is specified as \textless material attribute \textgreater. \\
This material has a \textless material attribute \textgreater type of magnetic order. \\
The magnetic ordering of the material is \textless material attribute \textgreater. \\
The magnetic arrangement in this material is \textless material attribute \textgreater. \\
The type of magnetic order of this material is \textless material attribute \textgreater. \\
\end{tabular} \\
\hline
\end{tabular}
\end{table}






\newpage
\section*{\quad Instruction Templates (property part2)}

\begin{table}[h!]
\centering
\renewcommand{\arraystretch}{1.2} % Adjust row height
\setlength{\tabcolsep}{5pt} % Adjust column separation
\begin{tabular}{|p{3cm}|p{12cm}|}
\hline
\textbf{Task} & \textbf{Instruction Template} \\
\hline
\textbf{Bandgap} & 
\begin{tabular}[t]{@{}l@{}}
\textless material structure\textgreater What is the bandgap of the material? \\
\textless material structure\textgreater Can you tell me the bandgap of this material? \\
\textless material structure\textgreater What is the energy bandgap for this material? \\
\textless material structure\textgreater Could you specify the bandgap of the material? \\
\textless material structure\textgreater Could you tell me the bandgap energy level of this material? \\
\end{tabular} \\
\hline
\textbf{Formation Energy} & 
\begin{tabular}[t]{@{}l@{}}
\textless material structure\textgreater Can you tell me the formation energy of this material? \\
\textless material structure\textgreater Please provide the formation energy for the material. \\
\textless material structure\textgreater What is the formation energy value for this material? \\
\textless material structure\textgreater How much is the formation energy of this material? \\
\textless material structure\textgreater Can you specify the formation energy of this material? \\
\end{tabular} \\
\hline
\textbf{Energy Above Hull} & 
\begin{tabular}[t]{@{}l@{}}
\textless material structure\textgreater Can you tell me the energy above hull of this material? \\
\textless material structure\textgreater Please provide the energy above hull for the material. \\
\textless material structure\textgreater What is the energy above the hull for this material? \\
\textless material structure\textgreater How much is the energy above hull for this material? \\
\textless material structure\textgreater Can you specify the energy above hull of this material? \\
\textless material structure\textgreater Could you tell me the energy above hull of the material? \\
\end{tabular} \\
\hline
\end{tabular}
\end{table}



\newpage
\section*{Answer Templates (property part2)}

\begin{table}[h!]
\centering
\renewcommand{\arraystretch}{1.2} % Adjust row height
\setlength{\tabcolsep}{5pt} % Adjust column separation
\begin{tabular}{|p{3cm}|p{12cm}|}
\hline
\textbf{Task} & \textbf{Instruction Template} \\
\hline
\textbf{Bandgap} & 
\begin{tabular}[t]{@{}l@{}}
The bandgap of the material is \textless material attribute \textgreater. \\
The bandgap of this material is \textless material attribute \textgreater. \\
The energy bandgap for this material is \textless material attribute \textgreater. \\
The bandgap of the material is specified as \textless material attribute \textgreater. \\
The bandgap energy level of this material is \textless material attribute \textgreater. \\
\end{tabular} \\
\hline
\textbf{Formation Energy} & 
\begin{tabular}[t]{@{}l@{}}
The formation energy of this material is \textless material attribute \textgreater. \\
The formation energy for the material is \textless material attribute \textgreater. \\
The formation energy value for this material is \textless material attribute \textgreater. \\
The formation energy of this material is \textless material attribute \textgreater. \\
The formation energy of this material is specified as \textless material attribute \textgreater. \\
\end{tabular} \\
\hline
\textbf{Energy Above Hull} & 
\begin{tabular}[t]{@{}l@{}}
The energy above hull of this material is \textless material attribute \textgreater. \\
The energy above hull for the material is \textless material attribute \textgreater. \\
The energy above the hull for this material is \textless material attribute \textgreater. \\
The energy above hull for this material is \textless material attribute \textgreater. \\
The energy above hull of this material is specified as \textless material attribute \textgreater. \\
The energy above hull of the material is \textless material attribute \textgreater. \\
\end{tabular} \\
\hline
\end{tabular}
\end{table}





\end{document}