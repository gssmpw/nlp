%%%%%%%% ICML 2025 EXAMPLE LATEX SUBMISSION FILE %%%%%%%%%%%%%%%%%

\documentclass{article}

% Recommended, but optional, packages for figures and better typesetting:
\usepackage{microtype}
\usepackage{graphicx}
\usepackage{subfigure}
\usepackage{booktabs} % for professional tables
\usepackage{multirow}

% hyperref makes hyperlinks in the resulting PDF.
% If your build breaks (sometimes temporarily if a hyperlink spans a page)
% please comment out the following usepackage line and replace
% \usepackage{icml2025} with \usepackage[nohyperref]{icml2025} above.
\usepackage{hyperref}


% Attempt to make hyperref and algorithmic work together better:
\newcommand{\theHalgorithm}{\arabic{algorithm}}

% Use the following line for the initial blind version submitted for review:
% \usepackage{icml2025}

% If accepted, instead use the following line for the camera-ready submission:
\usepackage[accepted]{icml2025}

% For theorems and such
\usepackage{amsmath}
\usepackage{amssymb}
\usepackage{mathtools}
\usepackage{amsthm}

% if you use cleveref..
\usepackage[capitalize,noabbrev]{cleveref}


%% sid---package added %%
% \usepackage{enumitem}
% \usepackage{paralist}
\usepackage{wrapfig}
% \usepackage{algorithm}
\renewcommand{\algorithmiccomment}[1]{{/* #1 */}}

% \usepackage{algpseudocode}
\usepackage{booktabs}
\usepackage{xspace}
\usepackage{array}
\usepackage{titlesec}
\usepackage{hyperref}
%\usepackage[ruled,vlined]{algorithm2e}
\usepackage{tikz}
\usepackage{amsmath}
\usepackage{placeins}
\usepackage{enumitem}

% \usepackage[table,xcdraw]{xcolor}
% Beamer presentation requires 
\usepackage{colortbl}
% \usepackage{lmodern}

% \setcounter{secnumdepth}{3}

\newcommand{\payAttention}[3]{\textcolor{#1}{{{#2}: {#3}}}}
% \renewcommand{\payAttention}[3]{}
% \newcommand{\todo}[1]{\payAttention{red}{TODO}{#1}}
\newcommand{\xSays}[2]{\payAttention{blue}{#1 Says}{#2}}
\newcommand{\Alex}[1]{{\xSays{Alex}{#1}}}
\newcommand{\Sid}[1]{\xSays{Siddhartha}{#1}}
\newcommand{\Mahantesh}[1]{\xSays{Mahantesh}{#1}}
\newcommand{\smf}[1]{\xSays{Ferdous}{#1}}
\newcommand{\sgs}{\texttt{SGS-GNN}\xspace}
\newcommand{\mlp}{\texttt{MLP}\xspace}
\newcommand{\edgemlp}{\texttt{EdgeMLP}\xspace}
\newcommand{\gnn}{\texttt{GNN}\xspace}
% \newcommand{\sigmoid}{\texttt{Sigmoid}\xspace}
% \newcommand{\softmax}{\texttt{Softmax}\xspace}
\newcommand{\relu}{\texttt{ReLU}\xspace}
% \newcommand{\mlp}{\texttt{MLP}\xspace}

%----------------------end sid
\usepackage{placeins}
\usepackage{caption}
\usepackage{etoolbox}
\setlength{\abovedisplayskip}{1pt}  % Space above equations
\setlength{\belowdisplayskip}{1pt}  % Space below equations
\setlength{\abovedisplayshortskip}{1pt}  % Space above small equations
\setlength{\belowdisplayshortskip}{1pt}  % Space below small equations

\captionsetup{skip=4pt}

% Caption package adjustments
% \usepackage{caption}
% \captionsetup{skip=3pt}  % Adjust vertical space before/after captions
\captionsetup[figure]{aboveskip=3pt}
\captionsetup[figure]{belowskip=3pt}


%%% Sid- page limit hacks %%%%%%%
% squeeze
%%Remove the spacing between paragraphs and have a small paragraph indentation
% \addtolength{\parskip}{-0.3mm}
% \setlength{\textfloatsep}{6pt plus 1.0pt minus 2.0pt}
% \setlength{\floatsep}{6pt plus 1.0pt minus 2.0pt}

% \setlength{\textfloatsep}{2pt plus 1.0pt minus 2.0pt}
% \setlength{\floatsep}{2pt plus 1.0pt minus 2.0pt}

\setlength{\floatsep}{2pt} % Space between floats
\setlength{\textfloatsep}{2pt} % Space between floats and text
\setlength{\intextsep}{2pt} % Space between in-text floats and text


% \setlength{\abovecaptionskip}{0pt}
% \setlength{\belowcaptionskip}{0pt}
% \raggedbottom
%\flushbottom


%%Remove space around section headings.
% \usepackage[compact]{titlesec}
% \titlespacing{\section}{1pt}{3ex}{2ex}
% \titlespacing{\subsection}{1pt}{2ex}{1ex}
% \titlespacing{\subsubsection}{1pt}{1ex}{1ex}
% White space around figures & tables
% \setlength{\belowcaptionskip}{2pt}
% \setlength{\abovecaptionskip}{2pt}
% \setlength{\textfloatsep}{0.0cm}
% \setlength{\textfloatsep}{0.0cm}

% % %%When you get very desperate:
% % % \linespread{0.98}

% %%%%%%%% End sid page limit hack%%%%


%%%%%%%%%%%%%%%%%%%%%%%%%%%%%%%%
% THEOREMS
%%%%%%%%%%%%%%%%%%%%%%%%%%%%%%%%
\theoremstyle{plain}
\newtheorem{theorem}{Theorem}[section]
\newtheorem{proposition}[theorem]{Proposition}
\newtheorem{lemma}[theorem]{Lemma}
\newtheorem{corollary}[theorem]{Corollary}
\theoremstyle{definition}
\newtheorem{definition}[theorem]{Definition}
\newtheorem{assumption}[theorem]{Assumption}
\theoremstyle{remark}
\newtheorem{remark}[theorem]{Remark}

% --- sid
% \usepackage{subfig}
% \usepackage{subcaption}
\usepackage{graphicx}
% \usepackage{caption}
%%%%% NEW MATH DEFINITIONS %%%%%

\usepackage{amsmath,amsfonts,bm}
\usepackage{derivative}
% Mark sections of captions for referring to divisions of figures
\newcommand{\figleft}{{\em (Left)}}
\newcommand{\figcenter}{{\em (Center)}}
\newcommand{\figright}{{\em (Right)}}
\newcommand{\figtop}{{\em (Top)}}
\newcommand{\figbottom}{{\em (Bottom)}}
\newcommand{\captiona}{{\em (a)}}
\newcommand{\captionb}{{\em (b)}}
\newcommand{\captionc}{{\em (c)}}
\newcommand{\captiond}{{\em (d)}}

% Highlight a newly defined term
\newcommand{\newterm}[1]{{\bf #1}}

% Derivative d 
\newcommand{\deriv}{{\mathrm{d}}}

% Figure reference, lower-case.
\def\figref#1{figure~\ref{#1}}
% Figure reference, capital. For start of sentence
\def\Figref#1{Figure~\ref{#1}}
\def\twofigref#1#2{figures \ref{#1} and \ref{#2}}
\def\quadfigref#1#2#3#4{figures \ref{#1}, \ref{#2}, \ref{#3} and \ref{#4}}
% Section reference, lower-case.
\def\secref#1{section~\ref{#1}}
% Section reference, capital.
\def\Secref#1{Section~\ref{#1}}
% Reference to two sections.
\def\twosecrefs#1#2{sections \ref{#1} and \ref{#2}}
% Reference to three sections.
\def\secrefs#1#2#3{sections \ref{#1}, \ref{#2} and \ref{#3}}
% Reference to an equation, lower-case.
\def\eqref#1{equation~\ref{#1}}
% Reference to an equation, upper case
\def\Eqref#1{Equation~\ref{#1}}
% A raw reference to an equation---avoid using if possible
\def\plaineqref#1{\ref{#1}}
% Reference to a chapter, lower-case.
\def\chapref#1{chapter~\ref{#1}}
% Reference to an equation, upper case.
\def\Chapref#1{Chapter~\ref{#1}}
% Reference to a range of chapters
\def\rangechapref#1#2{chapters\ref{#1}--\ref{#2}}
% Reference to an algorithm, lower-case.
\def\algref#1{algorithm~\ref{#1}}
% Reference to an algorithm, upper case.
\def\Algref#1{Algorithm~\ref{#1}}
\def\twoalgref#1#2{algorithms \ref{#1} and \ref{#2}}
\def\Twoalgref#1#2{Algorithms \ref{#1} and \ref{#2}}
% Reference to a part, lower case
\def\partref#1{part~\ref{#1}}
% Reference to a part, upper case
\def\Partref#1{Part~\ref{#1}}
\def\twopartref#1#2{parts \ref{#1} and \ref{#2}}

\def\ceil#1{\lceil #1 \rceil}
\def\floor#1{\lfloor #1 \rfloor}
\def\1{\bm{1}}
\newcommand{\train}{\mathcal{D}}
\newcommand{\valid}{\mathcal{D_{\mathrm{valid}}}}
\newcommand{\test}{\mathcal{D_{\mathrm{test}}}}

\def\eps{{\epsilon}}


% Random variables
\def\reta{{\textnormal{$\eta$}}}
\def\ra{{\textnormal{a}}}
\def\rb{{\textnormal{b}}}
\def\rc{{\textnormal{c}}}
\def\rd{{\textnormal{d}}}
\def\re{{\textnormal{e}}}
\def\rf{{\textnormal{f}}}
\def\rg{{\textnormal{g}}}
\def\rh{{\textnormal{h}}}
\def\ri{{\textnormal{i}}}
\def\rj{{\textnormal{j}}}
\def\rk{{\textnormal{k}}}
\def\rl{{\textnormal{l}}}
% rm is already a command, just don't name any random variables m
\def\rn{{\textnormal{n}}}
\def\ro{{\textnormal{o}}}
\def\rp{{\textnormal{p}}}
\def\rq{{\textnormal{q}}}
\def\rr{{\textnormal{r}}}
\def\rs{{\textnormal{s}}}
\def\rt{{\textnormal{t}}}
\def\ru{{\textnormal{u}}}
\def\rv{{\textnormal{v}}}
\def\rw{{\textnormal{w}}}
\def\rx{{\textnormal{x}}}
\def\ry{{\textnormal{y}}}
\def\rz{{\textnormal{z}}}

% Random vectors
\def\rvepsilon{{\mathbf{\epsilon}}}
\def\rvphi{{\mathbf{\phi}}}
\def\rvtheta{{\mathbf{\theta}}}
\def\rva{{\mathbf{a}}}
\def\rvb{{\mathbf{b}}}
\def\rvc{{\mathbf{c}}}
\def\rvd{{\mathbf{d}}}
\def\rve{{\mathbf{e}}}
\def\rvf{{\mathbf{f}}}
\def\rvg{{\mathbf{g}}}
\def\rvh{{\mathbf{h}}}
\def\rvu{{\mathbf{i}}}
\def\rvj{{\mathbf{j}}}
\def\rvk{{\mathbf{k}}}
\def\rvl{{\mathbf{l}}}
\def\rvm{{\mathbf{m}}}
\def\rvn{{\mathbf{n}}}
\def\rvo{{\mathbf{o}}}
\def\rvp{{\mathbf{p}}}
\def\rvq{{\mathbf{q}}}
\def\rvr{{\mathbf{r}}}
\def\rvs{{\mathbf{s}}}
\def\rvt{{\mathbf{t}}}
\def\rvu{{\mathbf{u}}}
\def\rvv{{\mathbf{v}}}
\def\rvw{{\mathbf{w}}}
\def\rvx{{\mathbf{x}}}
\def\rvy{{\mathbf{y}}}
\def\rvz{{\mathbf{z}}}

% Elements of random vectors
\def\erva{{\textnormal{a}}}
\def\ervb{{\textnormal{b}}}
\def\ervc{{\textnormal{c}}}
\def\ervd{{\textnormal{d}}}
\def\erve{{\textnormal{e}}}
\def\ervf{{\textnormal{f}}}
\def\ervg{{\textnormal{g}}}
\def\ervh{{\textnormal{h}}}
\def\ervi{{\textnormal{i}}}
\def\ervj{{\textnormal{j}}}
\def\ervk{{\textnormal{k}}}
\def\ervl{{\textnormal{l}}}
\def\ervm{{\textnormal{m}}}
\def\ervn{{\textnormal{n}}}
\def\ervo{{\textnormal{o}}}
\def\ervp{{\textnormal{p}}}
\def\ervq{{\textnormal{q}}}
\def\ervr{{\textnormal{r}}}
\def\ervs{{\textnormal{s}}}
\def\ervt{{\textnormal{t}}}
\def\ervu{{\textnormal{u}}}
\def\ervv{{\textnormal{v}}}
\def\ervw{{\textnormal{w}}}
\def\ervx{{\textnormal{x}}}
\def\ervy{{\textnormal{y}}}
\def\ervz{{\textnormal{z}}}

% Random matrices
\def\rmA{{\mathbf{A}}}
\def\rmB{{\mathbf{B}}}
\def\rmC{{\mathbf{C}}}
\def\rmD{{\mathbf{D}}}
\def\rmE{{\mathbf{E}}}
\def\rmF{{\mathbf{F}}}
\def\rmG{{\mathbf{G}}}
\def\rmH{{\mathbf{H}}}
\def\rmI{{\mathbf{I}}}
\def\rmJ{{\mathbf{J}}}
\def\rmK{{\mathbf{K}}}
\def\rmL{{\mathbf{L}}}
\def\rmM{{\mathbf{M}}}
\def\rmN{{\mathbf{N}}}
\def\rmO{{\mathbf{O}}}
\def\rmP{{\mathbf{P}}}
\def\rmQ{{\mathbf{Q}}}
\def\rmR{{\mathbf{R}}}
\def\rmS{{\mathbf{S}}}
\def\rmT{{\mathbf{T}}}
\def\rmU{{\mathbf{U}}}
\def\rmV{{\mathbf{V}}}
\def\rmW{{\mathbf{W}}}
\def\rmX{{\mathbf{X}}}
\def\rmY{{\mathbf{Y}}}
\def\rmZ{{\mathbf{Z}}}

% Elements of random matrices
\def\ermA{{\textnormal{A}}}
\def\ermB{{\textnormal{B}}}
\def\ermC{{\textnormal{C}}}
\def\ermD{{\textnormal{D}}}
\def\ermE{{\textnormal{E}}}
\def\ermF{{\textnormal{F}}}
\def\ermG{{\textnormal{G}}}
\def\ermH{{\textnormal{H}}}
\def\ermI{{\textnormal{I}}}
\def\ermJ{{\textnormal{J}}}
\def\ermK{{\textnormal{K}}}
\def\ermL{{\textnormal{L}}}
\def\ermM{{\textnormal{M}}}
\def\ermN{{\textnormal{N}}}
\def\ermO{{\textnormal{O}}}
\def\ermP{{\textnormal{P}}}
\def\ermQ{{\textnormal{Q}}}
\def\ermR{{\textnormal{R}}}
\def\ermS{{\textnormal{S}}}
\def\ermT{{\textnormal{T}}}
\def\ermU{{\textnormal{U}}}
\def\ermV{{\textnormal{V}}}
\def\ermW{{\textnormal{W}}}
\def\ermX{{\textnormal{X}}}
\def\ermY{{\textnormal{Y}}}
\def\ermZ{{\textnormal{Z}}}

% Vectors
\def\vzero{{\bm{0}}}
\def\vone{{\bm{1}}}
\def\vmu{{\bm{\mu}}}
\def\vtheta{{\bm{\theta}}}
\def\vphi{{\bm{\phi}}}
\def\va{{\bm{a}}}
\def\vb{{\bm{b}}}
\def\vc{{\bm{c}}}
\def\vd{{\bm{d}}}
\def\ve{{\bm{e}}}
\def\vf{{\bm{f}}}
\def\vg{{\bm{g}}}
\def\vh{{\bm{h}}}
\def\vi{{\bm{i}}}
\def\vj{{\bm{j}}}
\def\vk{{\bm{k}}}
\def\vl{{\bm{l}}}
\def\vm{{\bm{m}}}
\def\vn{{\bm{n}}}
\def\vo{{\bm{o}}}
\def\vp{{\bm{p}}}
\def\vq{{\bm{q}}}
\def\vr{{\bm{r}}}
\def\vs{{\bm{s}}}
\def\vt{{\bm{t}}}
\def\vu{{\bm{u}}}
\def\vv{{\bm{v}}}
\def\vw{{\bm{w}}}
\def\vx{{\bm{x}}}
\def\vy{{\bm{y}}}
\def\vz{{\bm{z}}}

% Elements of vectors
\def\evalpha{{\alpha}}
\def\evbeta{{\beta}}
\def\evepsilon{{\epsilon}}
\def\evlambda{{\lambda}}
\def\evomega{{\omega}}
\def\evmu{{\mu}}
\def\evpsi{{\psi}}
\def\evsigma{{\sigma}}
\def\evtheta{{\theta}}
\def\eva{{a}}
\def\evb{{b}}
\def\evc{{c}}
\def\evd{{d}}
\def\eve{{e}}
\def\evf{{f}}
\def\evg{{g}}
\def\evh{{h}}
\def\evi{{i}}
\def\evj{{j}}
\def\evk{{k}}
\def\evl{{l}}
\def\evm{{m}}
\def\evn{{n}}
\def\evo{{o}}
\def\evp{{p}}
\def\evq{{q}}
\def\evr{{r}}
\def\evs{{s}}
\def\evt{{t}}
\def\evu{{u}}
\def\evv{{v}}
\def\evw{{w}}
\def\evx{{x}}
\def\evy{{y}}
\def\evz{{z}}

% Matrix
\def\mA{{\bm{A}}}
\def\mB{{\bm{B}}}
\def\mC{{\bm{C}}}
\def\mD{{\bm{D}}}
\def\mE{{\bm{E}}}
\def\mF{{\bm{F}}}
\def\mG{{\bm{G}}}
\def\mH{{\bm{H}}}
\def\mI{{\bm{I}}}
\def\mJ{{\bm{J}}}
\def\mK{{\bm{K}}}
\def\mL{{\bm{L}}}
\def\mM{{\bm{M}}}
\def\mN{{\bm{N}}}
\def\mO{{\bm{O}}}
\def\mP{{\bm{P}}}
\def\mQ{{\bm{Q}}}
\def\mR{{\bm{R}}}
\def\mS{{\bm{S}}}
\def\mT{{\bm{T}}}
\def\mU{{\bm{U}}}
\def\mV{{\bm{V}}}
\def\mW{{\bm{W}}}
\def\mX{{\bm{X}}}
\def\mY{{\bm{Y}}}
\def\mZ{{\bm{Z}}}
\def\mBeta{{\bm{\beta}}}
\def\mPhi{{\bm{\Phi}}}
\def\mLambda{{\bm{\Lambda}}}
\def\mSigma{{\bm{\Sigma}}}

% Tensor
\DeclareMathAlphabet{\mathsfit}{\encodingdefault}{\sfdefault}{m}{sl}
\SetMathAlphabet{\mathsfit}{bold}{\encodingdefault}{\sfdefault}{bx}{n}
\newcommand{\tens}[1]{\bm{\mathsfit{#1}}}
\def\tA{{\tens{A}}}
\def\tB{{\tens{B}}}
\def\tC{{\tens{C}}}
\def\tD{{\tens{D}}}
\def\tE{{\tens{E}}}
\def\tF{{\tens{F}}}
\def\tG{{\tens{G}}}
\def\tH{{\tens{H}}}
\def\tI{{\tens{I}}}
\def\tJ{{\tens{J}}}
\def\tK{{\tens{K}}}
\def\tL{{\tens{L}}}
\def\tM{{\tens{M}}}
\def\tN{{\tens{N}}}
\def\tO{{\tens{O}}}
\def\tP{{\tens{P}}}
\def\tQ{{\tens{Q}}}
\def\tR{{\tens{R}}}
\def\tS{{\tens{S}}}
\def\tT{{\tens{T}}}
\def\tU{{\tens{U}}}
\def\tV{{\tens{V}}}
\def\tW{{\tens{W}}}
\def\tX{{\tens{X}}}
\def\tY{{\tens{Y}}}
\def\tZ{{\tens{Z}}}


% Graph
\def\gA{{\mathcal{A}}}
\def\gB{{\mathcal{B}}}
\def\gC{{\mathcal{C}}}
\def\gD{{\mathcal{D}}}
\def\gE{{\mathcal{E}}}
\def\gF{{\mathcal{F}}}
\def\gG{{\mathcal{G}}}
\def\gH{{\mathcal{H}}}
\def\gI{{\mathcal{I}}}
\def\gJ{{\mathcal{J}}}
\def\gK{{\mathcal{K}}}
\def\gL{{\mathcal{L}}}
\def\gM{{\mathcal{M}}}
\def\gN{{\mathcal{N}}}
\def\gO{{\mathcal{O}}}
\def\gP{{\mathcal{P}}}
\def\gQ{{\mathcal{Q}}}
\def\gR{{\mathcal{R}}}
\def\gS{{\mathcal{S}}}
\def\gT{{\mathcal{T}}}
\def\gU{{\mathcal{U}}}
\def\gV{{\mathcal{V}}}
\def\gW{{\mathcal{W}}}
\def\gX{{\mathcal{X}}}
\def\gY{{\mathcal{Y}}}
\def\gZ{{\mathcal{Z}}}

% Sets
\def\sA{{\mathbb{A}}}
\def\sB{{\mathbb{B}}}
\def\sC{{\mathbb{C}}}
\def\sD{{\mathbb{D}}}
% Don't use a set called E, because this would be the same as our symbol
% for expectation.
\def\sF{{\mathbb{F}}}
\def\sG{{\mathbb{G}}}
\def\sH{{\mathbb{H}}}
\def\sI{{\mathbb{I}}}
\def\sJ{{\mathbb{J}}}
\def\sK{{\mathbb{K}}}
\def\sL{{\mathbb{L}}}
\def\sM{{\mathbb{M}}}
\def\sN{{\mathbb{N}}}
\def\sO{{\mathbb{O}}}
\def\sP{{\mathbb{P}}}
\def\sQ{{\mathbb{Q}}}
\def\sR{{\mathbb{R}}}
\def\sS{{\mathbb{S}}}
\def\sT{{\mathbb{T}}}
\def\sU{{\mathbb{U}}}
\def\sV{{\mathbb{V}}}
\def\sW{{\mathbb{W}}}
\def\sX{{\mathbb{X}}}
\def\sY{{\mathbb{Y}}}
\def\sZ{{\mathbb{Z}}}

% Entries of a matrix
\def\emLambda{{\Lambda}}
\def\emA{{A}}
\def\emB{{B}}
\def\emC{{C}}
\def\emD{{D}}
\def\emE{{E}}
\def\emF{{F}}
\def\emG{{G}}
\def\emH{{H}}
\def\emI{{I}}
\def\emJ{{J}}
\def\emK{{K}}
\def\emL{{L}}
\def\emM{{M}}
\def\emN{{N}}
\def\emO{{O}}
\def\emP{{P}}
\def\emQ{{Q}}
\def\emR{{R}}
\def\emS{{S}}
\def\emT{{T}}
\def\emU{{U}}
\def\emV{{V}}
\def\emW{{W}}
\def\emX{{X}}
\def\emY{{Y}}
\def\emZ{{Z}}
\def\emSigma{{\Sigma}}

% entries of a tensor
% Same font as tensor, without \bm wrapper
\newcommand{\etens}[1]{\mathsfit{#1}}
\def\etLambda{{\etens{\Lambda}}}
\def\etA{{\etens{A}}}
\def\etB{{\etens{B}}}
\def\etC{{\etens{C}}}
\def\etD{{\etens{D}}}
\def\etE{{\etens{E}}}
\def\etF{{\etens{F}}}
\def\etG{{\etens{G}}}
\def\etH{{\etens{H}}}
\def\etI{{\etens{I}}}
\def\etJ{{\etens{J}}}
\def\etK{{\etens{K}}}
\def\etL{{\etens{L}}}
\def\etM{{\etens{M}}}
\def\etN{{\etens{N}}}
\def\etO{{\etens{O}}}
\def\etP{{\etens{P}}}
\def\etQ{{\etens{Q}}}
\def\etR{{\etens{R}}}
\def\etS{{\etens{S}}}
\def\etT{{\etens{T}}}
\def\etU{{\etens{U}}}
\def\etV{{\etens{V}}}
\def\etW{{\etens{W}}}
\def\etX{{\etens{X}}}
\def\etY{{\etens{Y}}}
\def\etZ{{\etens{Z}}}

% The true underlying data generating distribution
\newcommand{\pdata}{p_{\rm{data}}}
\newcommand{\ptarget}{p_{\rm{target}}}
\newcommand{\pprior}{p_{\rm{prior}}}
\newcommand{\pbase}{p_{\rm{base}}}
\newcommand{\pref}{p_{\rm{ref}}}

% The empirical distribution defined by the training set
\newcommand{\ptrain}{\hat{p}_{\rm{data}}}
\newcommand{\Ptrain}{\hat{P}_{\rm{data}}}
% The model distribution
\newcommand{\pmodel}{p_{\rm{model}}}
\newcommand{\Pmodel}{P_{\rm{model}}}
\newcommand{\ptildemodel}{\tilde{p}_{\rm{model}}}
% Stochastic autoencoder distributions
\newcommand{\pencode}{p_{\rm{encoder}}}
\newcommand{\pdecode}{p_{\rm{decoder}}}
\newcommand{\precons}{p_{\rm{reconstruct}}}

\newcommand{\laplace}{\mathrm{Laplace}} % Laplace distribution

\newcommand{\E}{\mathbb{E}}
\newcommand{\Ls}{\mathcal{L}}
\newcommand{\R}{\mathbb{R}}
\newcommand{\emp}{\tilde{p}}
\newcommand{\lr}{\alpha}
\newcommand{\reg}{\lambda}
\newcommand{\rect}{\mathrm{rectifier}}
\newcommand{\softmax}{\mathrm{softmax}}
\newcommand{\sigmoid}{\sigma}
\newcommand{\softplus}{\zeta}
\newcommand{\KL}{D_{\mathrm{KL}}}
\newcommand{\Var}{\mathrm{Var}}
\newcommand{\standarderror}{\mathrm{SE}}
\newcommand{\Cov}{\mathrm{Cov}}
% Wolfram Mathworld says $L^2$ is for function spaces and $\ell^2$ is for vectors
% But then they seem to use $L^2$ for vectors throughout the site, and so does
% wikipedia.
\newcommand{\normlzero}{L^0}
\newcommand{\normlone}{L^1}
\newcommand{\normltwo}{L^2}
\newcommand{\normlp}{L^p}
\newcommand{\normmax}{L^\infty}

\newcommand{\parents}{Pa} % See usage in notation.tex. Chosen to match Daphne's book.

\DeclareMathOperator*{\argmax}{arg\,max}
\DeclareMathOperator*{\argmin}{arg\,min}

\DeclareMathOperator{\sign}{sign}
\DeclareMathOperator{\Tr}{Tr}
\let\ab\allowbreak


% Todonotes is useful during development; simply uncomment the next line
%    and comment out the line below the next line to turn off comments
%\usepackage[disable,textsize=tiny]{todonotes}
\usepackage[textsize=tiny]{todonotes}


% The \icmltitle you define below is probably too long as a header.
% Therefore, a short form for the running title is supplied here:
\icmltitlerunning{SGS-GNN: A Supervised Graph Sparsification method for Graph Neural Networks}

\begin{document}

\twocolumn[
\icmltitle{SGS-GNN: A Supervised Graph Sparsifier  for Graph Neural Networks}

% It is OKAY to include author information, even for blind
% submissions: the style file will automatically remove it for you
% unless you've provided the [accepted] option to the icml2025
% package.

% List of affiliations: The first argument should be a (short)
% identifier you will use later to specify author affiliations
% Academic affiliations should list Department, University, City, Region, Country
% Industry affiliations should list Company, City, Region, Country

% You can specify symbols, otherwise they are numbered in order.
% Ideally, you should not use this facility. Affiliations will be numbered
% in order of appearance and this is the preferred way.
\icmlsetsymbol{equal}{*}

\begin{icmlauthorlist}
\icmlauthor{Siddhartha Shankar Das}{purdue}
\icmlauthor{Naheed Anjum Arafat}{indep}
\icmlauthor{Muftiqur Rahman}{agni}
\icmlauthor{S M Ferdous}{pnnl}
\icmlauthor{Alex Pothen}{purdue}
\icmlauthor{Mahantesh M Halappanavar}{pnnl}
% \icmlauthor{Firstname7 Lastname7}{comp}
% %\icmlauthor{}{sch}
% \icmlauthor{Firstname8 Lastname8}{sch}
% \icmlauthor{Firstname8 Lastname8}{yyy,comp}
%\icmlauthor{}{sch}
%\icmlauthor{}{sch}
\end{icmlauthorlist}

\icmlaffiliation{purdue}{Department of Computer Science, Purdue University,  West Lafayette, IN 47907, USA}
\icmlaffiliation{indep}{Independent Researcher, USA}
\icmlaffiliation{agni}{Islamic University of Technology, Dhaka, Bangladesh}
\icmlaffiliation{pnnl}{Pacific Northwest National Lab, Richland, WA, USA}

\icmlcorrespondingauthor{Siddhartha Shankar Das}{das90@purdue.edu}
\icmlcorrespondingauthor{Naheed Anjum Arafat}{naheed\_anjum@u.nus.edu}

% You may provide any keywords that you
% find helpful for describing your paper; these are used to populate
% the "keywords" metadata in the PDF but will not be shown in the document
\icmlkeywords{Machine Learning, ICML}

\vskip 0.3in
 ]

% this must go after the closing bracket ] following \twocolumn[ ...

% This command actually creates the footnote in the first column
% listing the affiliations and the copyright notice.
% The command takes one argument, which is text to display at the start of the footnote.
% The \icmlEqualContribution command is standard text for equal contribution.
% Remove it (just {}) if you do not need this facility.

\printAffiliationsAndNotice{}  % leave blank if no need to mention equal contribution
% \printAffiliationsAndNotice{\icmlEqualContribution} % otherwise use the standard text.

Accelerating inference in Large Language Models (LLMs) is critical for real-time interactions, as LLMs have been widely incorporated into real-world services.  Speculative decoding, a fully algorithmic solution, has gained attention for improving inference speed by drafting and verifying tokens, thereby generating multiple tokens in a single forward pass. However, current drafting strategies usually require significant fine-tuning or have inconsistent performance across tasks.
To address these challenges, we propose \textbf{Hierarchy Drafting} (HD)\footnote{\scriptsize \url{https://github.com/zomss/Hierarchy_Drafting}}, a novel lossless drafting approach that organizes various token sources into multiple databases in a hierarchical framework based on temporal locality. 
In the drafting step, HD sequentially accesses multiple databases to obtain draft tokens from the highest to the lowest locality, ensuring consistent acceleration across diverse tasks and minimizing drafting latency.
Our experiments on Spec-Bench using LLMs with 7B and 13B parameters demonstrate that HD outperforms existing lossless drafting methods, achieving robust inference speedups across model sizes, tasks, and temperatures.
With the growing demand for accelerating Large Language Model (LLM) inference to enable efficient real-time human-LLM interactions, Speculative Decoding~\cite{BlockWise, SpecDecoding, SpecSampling} has gained attention for providing a fully algorithmic solution with minimal drawbacks.
While autoregressive decoding generates token by token, the decoding step in this method is divided into two substeps: \textit{drafting}, where likely tokens are sampled externally from a less complex model, and \textit{verifying}, where the sampled tokens are accepted or rejected by comparing with the LLM’s actual output.
By allowing the LLM to generate multiple accepted tokens in the verification phase, speculative decoding improves both the throughput and the latency of the LLM inference. 
Crucially, the efficiency of this approach depends on how draft tokens are generated, as performance gains hinge on the acceptance rate of these tokens~\cite{SpecSampling}.
Therefore, subsequent approaches to speculative decoding have focused on developing drafting strategies that sample tokens closely aligned with the target model.

\section{Bellman Error Centering}

Centering operator $\mathcal{C}$ for a variable $x(s)$ is defined as follows:
\begin{equation}
\mathcal{C}x(s)\dot{=} x(s)-\mathbb{E}[x(s)]=x(s)-\sum_s{d_{s}x(s)},
\end{equation} 
where $d_s$ is the probability of $s$.
In vector form,
\begin{equation}
\begin{split}
\mathcal{C}\bm{x} &= \bm{x}-\mathbb{E}[x]\bm{1}\\
&=\bm{x}-\bm{x}^{\top}\bm{d}\bm{1},
\end{split}
\end{equation} 
where $\bm{1}$ is an all-ones vector.
For any vector $\bm{x}$ and $\bm{y}$ with a same distribution $\bm{d}$,
we have
\begin{equation}
\begin{split}
\mathcal{C}(\bm{x}+\bm{y})&=(\bm{x}+\bm{y})-(\bm{x}+\bm{y})^{\top}\bm{d}\bm{1}\\
&=\bm{x}-\bm{x}^{\top}\bm{d}\bm{1}+\bm{y}-\bm{y}^{\top}\bm{d}\bm{1}\\
&=\mathcal{C}\bm{x}+\mathcal{C}\bm{y}.
\end{split}
\end{equation}
\subsection{Revisit Reward Centering}


The update (\ref{src3}) is an unbiased estimate of the average reward
with  appropriate learning rate $\beta_t$ conditions.
\begin{equation}
\bar{r}_{t}\approx \lim_{n\rightarrow\infty}\frac{1}{n}\sum_{t=1}^n\mathbb{E}_{\pi}[r_t].
\end{equation}
That is 
\begin{equation}
r_t-\bar{r}_{t}\approx r_t-\lim_{n\rightarrow\infty}\frac{1}{n}\sum_{t=1}^n\mathbb{E}_{\pi}[r_t]= \mathcal{C}r_t.
\end{equation}
Then, the simple reward centering can be rewrited as:
\begin{equation}
V_{t+1}(s_t)=V_{t}(s_t)+\alpha_t [\mathcal{C}r_{t+1}+\gamma V_{t}(s_{t+1})-V_t(s_t)].
\end{equation}
Therefore, the simple reward centering is, in a strict sense, reward centering.

By definition of $\bar{\delta}_t=\delta_t-\bar{r}_{t}$,
let rewrite the update rule of the value-based reward centering as follows:
\begin{equation}
V_{t+1}(s_t)=V_{t}(s_t)+\alpha_t \rho_t (\delta_t-\bar{r}_{t}),
\end{equation}
where $\bar{r}_{t}$ is updated as:
\begin{equation}
\bar{r}_{t+1}=\bar{r}_{t}+\beta_t \rho_t(\delta_t-\bar{r}_{t}).
\label{vrc3}
\end{equation}
The update (\ref{vrc3}) is an unbiased estimate of the TD error
with  appropriate learning rate $\beta_t$ conditions.
\begin{equation}
\bar{r}_{t}\approx \mathbb{E}_{\pi}[\delta_t].
\end{equation}
That is 
\begin{equation}
\delta_t-\bar{r}_{t}\approx \mathcal{C}\delta_t.
\end{equation}
Then, the value-based reward centering can be rewrited as:
\begin{equation}
V_{t+1}(s_t)=V_{t}(s_t)+\alpha_t \rho_t \mathcal{C}\delta_t.
\label{tdcentering}
\end{equation}
Therefore, the value-based reward centering is no more,
 in a strict sense, reward centering.
It is, in a strict sense, \textbf{Bellman error centering}.

It is worth noting that this understanding is crucial, 
as designing new algorithms requires leveraging this concept.


\subsection{On the Fixpoint Solution}

The update rule (\ref{tdcentering}) is a stochastic approximation
of the following update:
\begin{equation}
\begin{split}
V_{t+1}&=V_{t}+\alpha_t [\bm{\mathcal{T}}^{\pi}\bm{V}-\bm{V}-\mathbb{E}[\delta]\bm{1}]\\
&=V_{t}+\alpha_t [\bm{\mathcal{T}}^{\pi}\bm{V}-\bm{V}-(\bm{\mathcal{T}}^{\pi}\bm{V}-\bm{V})^{\top}\bm{d}_{\pi}\bm{1}]\\
&=V_{t}+\alpha_t [\mathcal{C}(\bm{\mathcal{T}}^{\pi}\bm{V}-\bm{V})].
\end{split}
\label{tdcenteringVector}
\end{equation}
If update rule (\ref{tdcenteringVector}) converges, it is expected that
$\mathcal{C}(\mathcal{T}^{\pi}V-V)=\bm{0}$.
That is 
\begin{equation}
    \begin{split}
    \mathcal{C}\bm{V} &= \mathcal{C}\bm{\mathcal{T}}^{\pi}\bm{V} \\
    &= \mathcal{C}(\bm{R}^{\pi} + \gamma \mathbb{P}^{\pi} \bm{V}) \\
    &= \mathcal{C}\bm{R}^{\pi} + \gamma \mathcal{C}\mathbb{P}^{\pi} \bm{V} \\
    &= \mathcal{C}\bm{R}^{\pi} + \gamma (\mathbb{P}^{\pi} \bm{V} - (\mathbb{P}^{\pi} \bm{V})^{\top} \bm{d_{\pi}} \bm{1}) \\
    &= \mathcal{C}\bm{R}^{\pi} + \gamma (\mathbb{P}^{\pi} \bm{V} - \bm{V}^{\top} (\mathbb{P}^{\pi})^{\top} \bm{d_{\pi}} \bm{1}) \\  % 修正双重上标
    &= \mathcal{C}\bm{R}^{\pi} + \gamma (\mathbb{P}^{\pi} \bm{V} - \bm{V}^{\top} \bm{d_{\pi}} \bm{1}) \\
    &= \mathcal{C}\bm{R}^{\pi} + \gamma (\mathbb{P}^{\pi} \bm{V} - \bm{V}^{\top} \bm{d_{\pi}} \mathbb{P}^{\pi} \bm{1}) \\
    &= \mathcal{C}\bm{R}^{\pi} + \gamma (\mathbb{P}^{\pi} \bm{V} - \mathbb{P}^{\pi} \bm{V}^{\top} \bm{d_{\pi}} \bm{1}) \\
    &= \mathcal{C}\bm{R}^{\pi} + \gamma \mathbb{P}^{\pi} (\bm{V} - \bm{V}^{\top} \bm{d_{\pi}} \bm{1}) \\
    &= \mathcal{C}\bm{R}^{\pi} + \gamma \mathbb{P}^{\pi} \mathcal{C}\bm{V} \\
    &\dot{=} \bm{\mathcal{T}}_c^{\pi} \mathcal{C}\bm{V},
    \end{split}
    \label{centeredfixpoint}
    \end{equation}
where we defined $\bm{\mathcal{T}}_c^{\pi}$ as a centered Bellman operator.
We call equation (\ref{centeredfixpoint}) as centered Bellman equation.
And it is \textbf{centered fixpoint}.

For linear value function approximation, let define
\begin{equation}
\mathcal{C}\bm{V}_{\bm{\theta}}=\bm{\Pi}\bm{\mathcal{T}}_c^{\pi}\mathcal{C}\bm{V}_{\bm{\theta}}.
\label{centeredTDfixpoint}
\end{equation}
We call equation (\ref{centeredTDfixpoint}) as \textbf{centered TD fixpoint}.

\subsection{On-policy and Off-policy Centered TD Algorithms
with Linear Value Function Approximation}
Given the above centered TD fixpoint,
 mean squared centered Bellman error (MSCBE), is proposed as follows:
\begin{align*}
    \label{argminMSBEC}
 &\arg \min_{{\bm{\theta}}}\text{MSCBE}({\bm{\theta}}) \\
 &= \arg \min_{{\bm{\theta}}} \|\bm{\mathcal{T}}_c^{\pi}\mathcal{C}\bm{V}_{\bm{{\bm{\theta}}}}-\mathcal{C}\bm{V}_{\bm{{\bm{\theta}}}}\|_{\bm{D}}^2\notag\\
 &=\arg \min_{{\bm{\theta}}} \|\bm{\mathcal{T}}^{\pi}\bm{V}_{\bm{{\bm{\theta}}}} - \bm{V}_{\bm{{\bm{\theta}}}}-(\bm{\mathcal{T}}^{\pi}\bm{V}_{\bm{{\bm{\theta}}}} - \bm{V}_{\bm{{\bm{\theta}}}})^{\top}\bm{d}\bm{1}\|_{\bm{D}}^2\notag\\
 &=\arg \min_{{\bm{\theta}},\omega} \| \bm{\mathcal{T}}^{\pi}\bm{V}_{\bm{{\bm{\theta}}}} - \bm{V}_{\bm{{\bm{\theta}}}}-\omega\bm{1} \|_{\bm{D}}^2\notag,
\end{align*}
where $\omega$ is is used to estimate the expected value of the Bellman error.
% where $\omega$ is used to estimate $\mathbb{E}[\delta]$, $\omega \doteq \mathbb{E}[\mathbb{E}[\delta_t|S_t]]=\mathbb{E}[\delta]$ and $\delta_t$ is the TD error as follows:
% \begin{equation}
% \delta_t = r_{t+1}+\gamma
% {\bm{\theta}}_t^{\top}\bm{{\bm{\phi}}}_{t+1}-{\bm{\theta}}_t^{\top}\bm{{\bm{\phi}}}_t.
% \label{delta}
% \end{equation}
% $\mathbb{E}[\delta_t|S_t]$ is the Bellman error, and $\mathbb{E}[\mathbb{E}[\delta_t|S_t]]$ represents the expected value of the Bellman error.
% If $X$ is a random variable and $\mathbb{E}[X]$ is its expected value, then $X-\mathbb{E}[X]$ represents the centered form of $X$. 
% Therefore, we refer to $\mathbb{E}[\delta_t|S_t]-\mathbb{E}[\mathbb{E}[\delta_t|S_t]]$ as Bellman error centering and 
% $\mathbb{E}[(\mathbb{E}[\delta_t|S_t]-\mathbb{E}[\mathbb{E}[\delta_t|S_t]])^2]$ represents the the mean squared centered Bellman error, namely MSCBE.
% The meaning of (\ref{argminMSBEC}) is to minimize the mean squared centered Bellman error.
%The derivation of CTD is as follows.

First, the parameter  $\omega$ is derived directly based on
stochastic gradient descent:
\begin{equation}
\omega_{t+1}= \omega_{t}+\beta_t(\delta_t-\omega_t).
\label{omega}
\end{equation}

Then, based on stochastic semi-gradient descent, the update of 
the parameter ${\bm{\theta}}$ is as follows:
\begin{equation}
{\bm{\theta}}_{t+1}=
{\bm{\theta}}_{t}+\alpha_t(\delta_t-\omega_t)\bm{{\bm{\phi}}}_t.
\label{theta}
\end{equation}

We call (\ref{omega}) and (\ref{theta}) the on-policy centered
TD (CTD) algorithm. The convergence analysis with be given in
the following section.

In off-policy learning, we can simply multiply by the importance sampling
 $\rho$.
\begin{equation}
    \omega_{t+1}=\omega_{t}+\beta_t\rho_t(\delta_t-\omega_t),
    \label{omegawithrho}
\end{equation}
\begin{equation}
    {\bm{\theta}}_{t+1}=
    {\bm{\theta}}_{t}+\alpha_t\rho_t(\delta_t-\omega_t)\bm{{\bm{\phi}}}_t.
    \label{thetawithrho}
\end{equation}

We call (\ref{omegawithrho}) and (\ref{thetawithrho}) the off-policy centered
TD (CTD) algorithm.

% By substituting $\delta_t$ into Equations (\ref{omegawithrho}) and (\ref{thetawithrho}), 
% we can see that Equations (\ref{thetawithrho}) and (\ref{omegawithrho}) are formally identical 
% to the linear expressions of Equations (\ref{rewardcentering1}) and (\ref{rewardcentering2}), respectively. However, the meanings 
% of the corresponding parameters are entirely different.
% ${\bm{\theta}}_t$ is for approximating the discounted value function.
% $\bar{r_t}$ is an estimate of the average reward, while $\omega_t$ 
% is an estimate of the expected value of the Bellman error.
% $\bar{\delta_t}$ is the TD error for value-based reward centering, 
% whereas $\delta_t$ is the traditional TD error.

% This study posits that the CTD is equivalent to value-based reward 
% centering. However, CTD can be unified under a single framework 
% through an objective function, MSCBE, which also lays the 
% foundation for proving the algorithm's convergence. 
% Section 4 demonstrates that the CTD algorithm guarantees 
% convergence in the on-policy setting.

\subsection{Off-policy Centered TDC Algorithm with Linear Value Function Approximation}
The convergence of the  off-policy centered TD algorithm
may not be guaranteed.

To deal with this problem, we propose another new objective function, 
called mean squared projected centered Bellman error (MSPCBE), 
and derive Centered TDC algorithm (CTDC).

% We first establish some relationships between
%  the vector-matrix quantities and the relevant statistical expectation terms:
% \begin{align*}
%     &\mathbb{E}[(\delta({\bm{\theta}})-\mathbb{E}[\delta({\bm{\theta}})]){\bm{\phi}}] \\
%     &= \sum_s \mu(s) {\bm{\phi}}(s) \big( R(s) + \gamma \sum_{s'} P_{ss'} V_{\bm{\theta}}(s') - V_{\bm{\theta}}(s)  \\
%     &\quad \quad-\sum_s \mu(s)(R(s) + \gamma \sum_{s'} P_{ss'} V_{\bm{\theta}}(s') - V_{\bm{\theta}}(s))\big)\\
%     &= \bm{\Phi}^\top \mathbf{D} (\bm{TV}_{\bm{{\bm{\theta}}}} - \bm{V}_{\bm{{\bm{\theta}}}}-\omega\bm{1}),
% \end{align*}
% where $\omega$ is the expected value of the Bellman error and $\bm{1}$ is all-ones vector.

The specific expression of the objective function 
MSPCBE is as follows:
\begin{align}
    \label{MSPBECwithomega}
    &\arg \min_{{\bm{\theta}}}\text{MSPCBE}({\bm{\theta}})\notag\\ 
    % &= \arg \min_{{\bm{\theta}}}\big(\mathbb{E}[(\delta({\bm{\theta}}) - \mathbb{E}[\delta({\bm{\theta}})]) \bm{{\bm{\phi}}}]^\top \notag\\
    % &\quad \quad \quad\mathbb{E}[\bm{{\bm{\phi}}} \bm{{\bm{\phi}}}^\top]^{-1} \mathbb{E}[(\delta({\bm{\theta}}) - \mathbb{E}[\delta({\bm{\theta}})]) \bm{{\bm{\phi}}}]\big) \notag\\
    % &=\arg \min_{{\bm{\theta}},\omega}\mathbb{E}[(\delta({\bm{\theta}})-\omega) \bm{\bm{{\bm{\phi}}}}]^{\top} \mathbb{E}[\bm{\bm{{\bm{\phi}}}} \bm{\bm{{\bm{\phi}}}}^{\top}]^{-1}\mathbb{E}[(\delta({\bm{\theta}}) -\omega)\bm{\bm{{\bm{\phi}}}}]\\
    % &= \big(\bm{\Phi}^\top \mathbf{D} (\bm{TV}_{\bm{{\bm{\theta}}}} - \bm{V}_{\bm{{\bm{\theta}}}}-\omega\bm{1})\big)^\top (\bm{\Phi}^\top \mathbf{D} \bm{\Phi})^{-1} \notag\\
    % & \quad \quad \quad \bm{\Phi}^\top \mathbf{D} (\bm{TV}_{\bm{{\bm{\theta}}}} - \bm{V}_{\bm{{\bm{\theta}}}}-\omega\bm{1}) \notag\\
    % &= (\bm{TV}_{\bm{{\bm{\theta}}}} - \bm{V}_{\bm{{\bm{\theta}}}}-\omega\bm{1})^\top \mathbf{D} \bm{\Phi} (\bm{\Phi}^\top \mathbf{D} \bm{\Phi})^{-1} \notag\\
    % &\quad \quad \quad \bm{\Phi}^\top \mathbf{D} (\bm{TV}_{\bm{{\bm{\theta}}}} - \bm{V}_{\bm{{\bm{\theta}}}}-\omega\bm{1})\notag\\
    % &= (\bm{TV}_{\bm{{\bm{\theta}}}} - \bm{V}_{\bm{{\bm{\theta}}}}-\omega\bm{1})^\top {\bm{\Pi}}^\top \mathbf{D} {\bm{\Pi}} (\bm{TV}_{\bm{{\bm{\theta}}}} - \bm{V}_{\bm{{\bm{\theta}}}}-\omega\bm{1}) \notag\\
    &= \arg \min_{{\bm{\theta}}} \|\bm{\Pi}\bm{\mathcal{T}}_c^{\pi}\mathcal{C}\bm{V}_{\bm{{\bm{\theta}}}}-\mathcal{C}\bm{V}_{\bm{{\bm{\theta}}}}\|_{\bm{D}}^2\notag\\
    &= \arg \min_{{\bm{\theta}}} \|\bm{\Pi}(\bm{\mathcal{T}}_c^{\pi}\mathcal{C}\bm{V}_{\bm{{\bm{\theta}}}}-\mathcal{C}\bm{V}_{\bm{{\bm{\theta}}}})\|_{\bm{D}}^2\notag\\
    &= \arg \min_{{\bm{\theta}},\omega}\| {\bm{\Pi}} (\bm{\mathcal{T}}^{\pi}\bm{V}_{\bm{{\bm{\theta}}}} - \bm{V}_{\bm{{\bm{\theta}}}}-\omega\bm{1}) \|_{\bm{D}}^2\notag.
\end{align}
In the process of computing the gradient of the MSPCBE with respect to ${\bm{\theta}}$, 
$\omega$ is treated as a constant.
So, the derivation process of CTDC is the same 
as for the TDC algorithm \cite{sutton2009fast}, the only difference is that the original $\delta$ is replaced by $\delta-\omega$.
Therefore, the updated formulas of the centered TDC  algorithm are as follows:
\begin{equation}
 \bm{{\bm{\theta}}}_{k+1}=\bm{{\bm{\theta}}}_{k}+\alpha_{k}[(\delta_{k}- \omega_k) \bm{\bm{{\bm{\phi}}}}_k\\
 - \gamma\bm{\bm{{\bm{\phi}}}}_{k+1}(\bm{\bm{{\bm{\phi}}}}^{\top}_k \bm{u}_{k})],
\label{thetavmtdc}
\end{equation}
\begin{equation}
 \bm{u}_{k+1}= \bm{u}_{k}+\zeta_{k}[\delta_{k}-\omega_k - \bm{\bm{{\bm{\phi}}}}^{\top}_k \bm{u}_{k}]\bm{\bm{{\bm{\phi}}}}_k,
\label{uvmtdc}
\end{equation}
and
\begin{equation}
 \omega_{k+1}= \omega_{k}+\beta_k (\delta_k- \omega_k).
 \label{omegavmtdc}
\end{equation}
This algorithm is derived to work 
with a given set of sub-samples—in the form of 
triples $(S_k, R_k, S'_k)$ that match transitions 
from both the behavior and target policies. 

% \subsection{Variance Minimization ETD Learning: VMETD}
% Based on the off-policy TD algorithm, a scalar, $F$,  
% is introduced to obtain the ETD algorithm, 
% which ensures convergence under off-policy 
% conditions. This paper further introduces a scalar, 
% $\omega$, based on the ETD algorithm to obtain VMETD.
% VMETD by the following update:
% \begin{equation}
% \label{fvmetd}
%  F_t \leftarrow \gamma \rho_{t-1}F_{t-1}+1,
% \end{equation}
% \begin{equation}
%  \label{thetavmetd}
%  {{\bm{\theta}}}_{t+1}\leftarrow {{\bm{\theta}}}_t+\alpha_t (F_t \rho_t\delta_t - \omega_{t}){\bm{{\bm{\phi}}}}_t,
% \end{equation}
% \begin{equation}
%  \label{omegavmetd}
%  \omega_{t+1} \leftarrow \omega_t+\beta_t(F_t  \rho_t \delta_t - \omega_t),
% \end{equation}
% where $\rho_t =\frac{\pi(A_t | S_t)}{\mu(A_t | S_t)}$ and $\omega$ is used to estimate $\mathbb{E}[F \rho\delta]$, i.e., $\omega \doteq \mathbb{E}[F \rho\delta]$.

% (\ref{thetavmetd}) can be rewritten as
% \begin{equation*}
%  \begin{array}{ccl}
%  {{\bm{\theta}}}_{t+1}&\leftarrow& {{\bm{\theta}}}_t+\alpha_t (F_t \rho_t\delta_t - \omega_t){\bm{{\bm{\phi}}}}_t -\alpha_t \omega_{t+1}{\bm{{\bm{\phi}}}}_t\\
%   &=&{{\bm{\theta}}}_{t}+\alpha_t(F_t\rho_t\delta_t-\mathbb{E}_{\mu}[F_t\rho_t\delta_t|{{\bm{\theta}}}_t]){\bm{{\bm{\phi}}}}_t\\
%  &=&{{\bm{\theta}}}_t+\alpha_t F_t \rho_t (r_{t+1}+\gamma {{\bm{\theta}}}_t^{\top}{\bm{{\bm{\phi}}}}_{t+1}-{{\bm{\theta}}}_t^{\top}{\bm{{\bm{\phi}}}}_t){\bm{{\bm{\phi}}}}_t\\
%  & & \hspace{2em} -\alpha_t \mathbb{E}_{\mu}[F_t \rho_t \delta_t]{\bm{{\bm{\phi}}}}_t\\
%  &=& {{\bm{\theta}}}_t+\alpha_t \{\underbrace{(F_t\rho_tr_{t+1}-\mathbb{E}_{\mu}[F_t\rho_t r_{t+1}]){\bm{{\bm{\phi}}}}_t}_{{b}_{\text{VMETD},t}}\\
%  &&\hspace{-7em}- \underbrace{(F_t\rho_t{\bm{{\bm{\phi}}}}_t({\bm{{\bm{\phi}}}}_t-\gamma{\bm{{\bm{\phi}}}}_{t+1})^{\top}-{\bm{{\bm{\phi}}}}_t\mathbb{E}_{\mu}[F_t\rho_t ({\bm{{\bm{\phi}}}}_t-\gamma{\bm{{\bm{\phi}}}}_{t+1})]^{\top})}_{\textbf{A}_{\text{VMETD},t}}{{\bm{\theta}}}_t\}.
%  \end{array}
% \end{equation*}
% Therefore, 
% \begin{equation*}
%  \begin{array}{ccl}
%   &&\textbf{A}_{\text{VMETD}}\\
%   &=&\lim_{t \rightarrow \infty} \mathbb{E}[\textbf{A}_{\text{VMETD},t}]\\
%   &=& \lim_{t \rightarrow \infty} \mathbb{E}_{\mu}[F_t \rho_t {\bm{{\bm{\phi}}}}_t ({\bm{{\bm{\phi}}}}_t - \gamma {\bm{{\bm{\phi}}}}_{t+1})^{\top}]\\  
%   &&\hspace{1em}- \lim_{t\rightarrow \infty} \mathbb{E}_{\mu}[  {\bm{{\bm{\phi}}}}_t]\mathbb{E}_{\mu}[F_t \rho_t ({\bm{{\bm{\phi}}}}_t - \gamma {\bm{{\bm{\phi}}}}_{t+1})]^{\top}\\
%   &=& \lim_{t \rightarrow \infty} \mathbb{E}_{\mu}[{\bm{{\bm{\phi}}}}_tF_t \rho_t ({\bm{{\bm{\phi}}}}_t - \gamma {\bm{{\bm{\phi}}}}_{t+1})^{\top}]\\   
%   &&\hspace{1em}-\lim_{t \rightarrow \infty} \mathbb{E}_{\mu}[ {\bm{{\bm{\phi}}}}_t]\lim_{t \rightarrow \infty}\mathbb{E}_{\mu}[F_t \rho_t ({\bm{{\bm{\phi}}}}_t - \gamma {\bm{{\bm{\phi}}}}_{t+1})]^{\top}\\
%   && \hspace{-2em}=\sum_{s} d_{\mu}(s)\lim_{t \rightarrow \infty}\mathbb{E}_{\mu}[F_t|S_t = s]\mathbb{E}_{\mu}[\rho_t\bm{{\bm{\phi}}}_t(\bm{{\bm{\phi}}}_t - \gamma \bm{{\bm{\phi}}}_{t+1})^{\top}|S_t= s]\\   
%   &&\hspace{1em}-\sum_{s} d_{\mu}(s)\bm{{\bm{\phi}}}(s)\sum_{s} d_{\mu}(s)\lim_{t \rightarrow \infty}\mathbb{E}_{\mu}[F_t|S_t = s]\\
%   &&\hspace{7em}\mathbb{E}_{\mu}[\rho_t(\bm{{\bm{\phi}}}_t - \gamma \bm{{\bm{\phi}}}_{t+1})^{\top}|S_t = s]\\
%   &=& \sum_{s} f(s)\mathbb{E}_{\pi}[\bm{{\bm{\phi}}}_t(\bm{{\bm{\phi}}}_t- \gamma \bm{{\bm{\phi}}}_{t+1})^{\top}|S_t = s]\\   
%   &&\hspace{1em}-\sum_{s} d_{\mu}(s)\bm{{\bm{\phi}}}(s)\sum_{s} f(s)\mathbb{E}_{\pi}[(\bm{{\bm{\phi}}}_t- \gamma \bm{{\bm{\phi}}}_{t+1})^{\top}|S_t = s]\\
%   &=&\sum_{s} f(s) \bm{\bm{{\bm{\phi}}}}(s)(\bm{\bm{{\bm{\phi}}}}(s) - \gamma \sum_{s'}[\textbf{P}_{\pi}]_{ss'}\bm{\bm{{\bm{\phi}}}}(s'))^{\top}  \\
%   &&-\sum_{s} d_{\mu}(s) {\bm{{\bm{\phi}}}}(s) * \sum_{s} f(s)({\bm{{\bm{\phi}}}}(s) - \gamma \sum_{s'}[\textbf{P}_{\pi}]_{ss'}{\bm{{\bm{\phi}}}}(s'))^{\top}\\
%   &=&{\bm{\bm{\Phi}}}^{\top} \textbf{F} (\textbf{I} - \gamma \textbf{P}_{\pi}) \bm{\bm{\Phi}} - {\bm{\bm{\Phi}}}^{\top} {d}_{\mu} {f}^{\top} (\textbf{I} - \gamma \textbf{P}_{\pi}) \bm{\bm{\Phi}}  \\
%   &=&{\bm{\bm{\Phi}}}^{\top} (\textbf{F} - {d}_{\mu} {f}^{\top}) (\textbf{I} - \gamma \textbf{P}_{\pi}){\bm{\bm{\Phi}}} \\
%   &=&{\bm{\bm{\Phi}}}^{\top} (\textbf{F} (\textbf{I} - \gamma \textbf{P}_{\pi})-{d}_{\mu} {f}^{\top} (\textbf{I} - \gamma \textbf{P}_{\pi})){\bm{\bm{\Phi}}} \\
%   &=&{\bm{\bm{\Phi}}}^{\top} (\textbf{F} (\textbf{I} - \gamma \textbf{P}_{\pi})-{d}_{\mu} {d}_{\mu}^{\top} ){\bm{\bm{\Phi}}},
%  \end{array}
% \end{equation*}
% \begin{equation*}
%  \begin{array}{ccl}
%   &&{b}_{\text{VMETD}}\\
%   &=&\lim_{t \rightarrow \infty} \mathbb{E}[{b}_{\text{VMETD},t}]\\
%   &=& \lim_{t \rightarrow \infty} \mathbb{E}_{\mu}[F_t\rho_tR_{t+1}{\bm{{\bm{\phi}}}}_t]\\
%   &&\hspace{2em} - \lim_{t\rightarrow \infty} \mathbb{E}_{\mu}[{\bm{{\bm{\phi}}}}_t]\mathbb{E}_{\mu}[F_t\rho_kR_{k+1}]\\  
%   &=& \lim_{t \rightarrow \infty} \mathbb{E}_{\mu}[{\bm{{\bm{\phi}}}}_tF_t\rho_tr_{t+1}]\\
%   &&\hspace{2em} - \lim_{t\rightarrow \infty} \mathbb{E}_{\mu}[  {\bm{{\bm{\phi}}}}_t]\mathbb{E}_{\mu}[{\bm{{\bm{\phi}}}}_t]\mathbb{E}_{\mu}[F_t\rho_tr_{t+1}]\\ 
%   &=& \lim_{t \rightarrow \infty} \mathbb{E}_{\mu}[{\bm{{\bm{\phi}}}}_tF_t\rho_tr_{t+1}]\\
%   &&\hspace{2em} - \lim_{t \rightarrow \infty} \mathbb{E}_{\mu}[ {\bm{{\bm{\phi}}}}_t]\lim_{t \rightarrow \infty}\mathbb{E}_{\mu}[F_t\rho_tr_{t+1}]\\  
%   &=&\sum_{s} f(s) {\bm{{\bm{\phi}}}}(s)r_{\pi} - \sum_{s} d_{\mu}(s) {\bm{{\bm{\phi}}}}(s) * \sum_{s} f(s)r_{\pi}  \\
%   &=&\bm{\bm{\bm{\Phi}}}^{\top}(\textbf{F}-{d}_{\mu} {f}^{\top}){r}_{\pi}.
%  \end{array}
% \end{equation*}



Recent efforts in speculative decoding have focused on developing effective drafting methods, using LM-based approaches, such as using smaller models than LLM~\cite{DistilSpec, SpecInfer} or incorporating specialized branches within the LLM architecture~\cite{MEDUSA, EAGLE2}.
However, their applicability in real-world scenarios is limited by the significant overhead associated with fine-tuning for optimization.
First, smaller models for drafting must be fine-tuned, such as by distillation, to generate tokens similar to LLMs to achieve optimal performance regardless of the given tasks~\cite{DistilSpec, multilingual}.
In addition, current LLM families~\cite{Llama2, vicuna} do not offer models of an appropriate size for drafting, often necessitating training from scratch.
In branch-based drafting, which modifies its original LLM architecture, the computational cost for training such branches within LLM is significant due to gradient calculations across the entire model, even though most parameters remain frozen~\cite{MEDUSA, EAGLE2, EAGLE}.
For example, EAGLE~\cite{EAGLE}, one of the leading methods, needs 1-2 days of training on 2-4 billion tokens using 4 A100 GPUs to train the 70B model.

To address these limitations, this paper explores a lightweight, lossless drafting strategy: \textit{Database Drafting}, eliminating the need for parameter updates~\cite{PLD, LAD, REST}. 
Database drafting constructs databases from various token sources and fetches draft tokens from the database using previous tokens.
However, as previous work relies on a single database from a single source, the coverage of draft tokens is restricted, leading to inconsistent acceleration across different tasks, as depicted in the left side of Figure~\ref{fig:motivation}. 
For example, PLD~\cite{PLD}, which uses previous tokens as its source, shows strengths in the summarization, highly repeating the tokens in the earlier texts, yet it achieves only marginal speedups in QA, where fewer promising tokens are included in the prior text. 
A straightforward solution to improve coverage is incorporating diverse sources into a single database. 
However, increasing the database scale leads to higher drafting latency, resulting in additional overhead.
As shown in the right side of Figure~\ref{fig:motivation}, REST~\cite{REST}, which uses the largest database, accurately predicts future tokens but suffers from significant latency, negating its high acceptance ratio benefits. 
Therefore, this paper proposes a solution to these limitations: \textit{Utilize diverse token sources simultaneously for robust performance and minimal overhead.}


With this objective in mind, we propose a simple yet effective solution: \textbf{Hierarchy Drafting} (HD), which integrates diverse token sources into a hierarchical framework. 
Our proposed method is inspired by the memory hierarchy system, which prioritizes data with high \textit{temporal locality} in the memory access for performance optimization~\cite{hierarchy}.
Therefore, HD groups draft tokens from diverse sources based on their temporal locality---the tendency for some tokens to reappear within or across generation processes. 
For example, when an LLM solves a math problem like, ‘\textit{The vertices of a triangle are at points (0, 0), (-1, 1), and (3, 3). What is the area of the triangle?}’, the coordinates frequently repeat within only a generation process for a given query but not across other generation processes.
In a related sense, phrases commonly generated by LLMs, such as ‘\textit{as an AI assistant}’, or frequent grammatical patterns exhibit relatively moderate locality, often appearing across different generation processes. 

Based on their temporal locality, the multiple databases of HD organize them into \textit{context-dependent database}, which stores tokens with high temporal locality for a given context; \textit{model-dependent database}, which captures frequently repeated phrases by LLMs across generations; and \textit{statistics-dependent database}, which contains statistically common phrases with slightly lower locality across processes than those in the model-dependent database.
During inference, HD accesses the databases in order of temporal locality, prioritizing tokens with high locality by starting with context-dependent, then model-dependent, and finally statistics-dependent databases until a sufficient number of draft tokens are obtained to convey to the LLM for verification.


This strategy has two benefits: firstly, increasing drafting accuracy by leveraging temporal locality and
secondly, reducing the overhead from drafting latency, as the scale of the databases is inversely correlated with the degree of locality—tokens with high locality are rarer. Thus, starting with the smaller context-dependent database for drafting tokens is more accurate and faster than using the larger statistics-dependent database alone.
Also, our hierarchical framework can encompass other database drafting methods owing to its \textit{plug-and-play} nature, making it easy to integrate diverse drafting sources based on their temporal locality.

We evaluate HD and other database drafting methods using widely adopted LLMs, Llama-2~\cite{Llama2} and Vicuna~\cite{vicuna}, on Spec-Bench~\cite{Spec_Survey}, a benchmark designed to assess effectiveness across diverse tasks.
Our proposed method, HD, outperforms other methods in our experiment and consistently achieves significant inference speedup across various settings, including model size, temperature, and tasks.
We also analyze how the hierarchical framework adaptively selects the appropriate database for each task while minimizing draft latency, aligning with our design goals.


Our contributions in this paper are threefold:
\vspace{-0.1in}
\begin{itemize}[itemsep=0.3mm, parsep=1pt, leftmargin=*]
    \item We identify the limitations of existing speculative decoding methods, which require additional fine-tuning or deliver inconsistent acceleration gains.
    \item We introduce a novel database drafting method, Hierarchy Drafting (HD), incorporating diverse token sources into the hierarchical framework for robust performance with minimizing overhead.
    \item We demonstrate that HD consistently achieves significant acceleration gains across various scenarios compared to other lossless methods.
\end{itemize}


\section{Preliminaries}
\label{sec:prelim}
$\gG \triangleq (\gV, \gE, \mX)$ is an undirected graph with its set of nodes and edges denoted by $\gV$ and $\gE$ respectively.  The node feature matrix $\mX \in \mathbb{R}^{|\gV| \times F}$ contains node feature $\vx_v \in \mathbb{R}^F$ as a row vector for every node $v \in \gV$. The adjacency matrix $\mA_{\gG}$ of size $|\gV| \times |\gV|$ captures the neighborhood of each node in $\gG$. In node classification, the goal is to predict a label $y_v \in C$ for each node $v \in \gV$ among the $\abs{C}$ possible class labels. The training uses labeled nodes $\gV_L \subset \gV$, while the unlabeled nodes $\gV_U = \gV \setminus \gV_L$ are used for validation and testing. A single-layer Graph Convolutional Network (GCN)~\cite{kipf2016semi} is defined as:
%%%%%%%%%%%%%%%%%%%%%%%%
\begin{equation}
\mH^{(l+1)} = \sigma(\hat{\mA}_\gG\mH^{(l)}\mW^{(l)}),
\label{eq:gcnlayer}
\end{equation}
%%%%%%%%%%%%%%%%%%%%%%%%
where $\mH^{(l)}$ represents the node embedding at layer $l$, with $\mH^{(0)}=\mX$, and $\mW^{(l)}$ is the learnable weight matrix in layer $l$. $\hat{\mA}_\gG$ denotes the normalized adjacency matrix and $\sigma$ is an activation function such as \relu. The predicted probabilities can be expressed as
$\hat{\mY} = \texttt{Softmax}(f_{\text{GNN}, \theta}(\gG))$ where $f_{\text{GNN}, \theta}(\gG)$ is a GCN model with $L$ layers and learnable parameters $\theta$. The dimension of $\hat{\mY}$ is $|\gV| \times \abs{C}$. The training objective is to find parameters $\theta$ that minimize the cross-entropy loss,
%%%%%%%%%%%%%%%%%%%%%%%%
\vspace{-6pt}
\begin{equation}
\label{eq:loss}
\mathcal{L}_\mathrm{CE} = - \frac{1}{|\gV_L|} \sum_{v \in \gV_L} \sum_{c = 1}^{\abs{C}} Y_{vc} \log \hat{Y}_{vc},
\end{equation}
%%%%%%%%%%%%%%%%%%%%%%%%
where $Y_{vc}$ indicates the true probability of node $v$ belonging to class $c \in C$. 

The \emph{homophily} of a graph characterizes the likelihood that nodes with the same labels are neighbors. Two commonly used measures are \emph{Node homophily} $\gH_n$~\citep{pei2020geom} and \emph{Edge homophily} $\gH_e$~\cite{zhu2020beyond} and defined as
%%%%%%%%%%%%%%%%%%%%%%%%
\vspace{-3pt}
\begin{align}
%\small
\gH_{n} = & \frac{1}{|\gV|} \sum_{u\in \gV} \frac{| \{v\in \gN(u) : y_v = y_u\}|}{|\gN(u)|},\\
\gH_{e} = & \frac{(u,v)\in \gE: y_u = y_v}{|\gE|}.
\end{align}\vspace{-3pt}
%%%%%%%%%%%%%%%%%%%%%%%%
The values of $\gH_n$ and $\gH_e$ range from $0$ to $1$, where a value close to $1$ indicates strong homophily, and a value close to $0$ indicates strong heterophily.
% 
%Graphs with $0 \leq \gH_{n} < 0.5$ are considered \emph{heterophilic}, while those with $0.5 \leq \gH_{n} \leq 1$ are considered \emph{homophilic}~\citep{pei2020geom}.


% The GNN iteratively computes node embeddings 
% $\vh_v^{(l)}$ through message passing. At the final layer $L$, the embeddings are denoted as:
% \[
% \mH^{(L)} = f_{\text{GNN},\theta}(\mX,\mA_{\gG}),
% \]
% where $f_{\text{GNN},\theta}$ encapsulates all message-passing and update steps across $L$ layers as well as learnable parameters $\theta$ and $\mH^{(L)}$ contains the final node embeddings. The predicted logits for node $v$ are obtained by applying a linear transformation to its final embedding:
% \[
% \vz_v = \mW^{(L)} \vh_v^{(L)},
% \]
% where $\mW^{(L)}$ is the learnable weight matrix for classification and $\vz_v$ represents the unnormalized scores (logits) for each class. The logits $\vz_v$ are converted to probabilities using the Softmax function:
% \[
% \hat{Y}_v = \text{Softmax}(\vz_v)
% \]
% Combining everything, the predicted probabilities can be expressed as a function of GNN as the following: 
% \[
% \hat{\mY} = \text{Softmax}(\mW^{(L)} f_{\text{GNN}, \theta}(\mX,\mA_{\gG}))
% \]
% Here $\hat{\mY}$ of dimension $|\gV| \times C$ and $C$ is the number of class labels of $\gG$.
% \begin{equation}
% \label{eq:loss}
% \mathcal{L}_\mathrm{CE} = - \frac{1}{|\gV_L|} \sum_{v \in \gV_L} \sum_{c = 1}^{C} Y_{vc} \log \tilde{Y}_{vc}
% \end{equation}
% where 
% \[
% \tilde{Y} = \text{Softmax}(\mW^{(L)} f_{\text{GNN}, \theta}(\mX,\mA_{\tilde{\gG}}))
% \]
%\subsection{Theoretical Motivation.} 
%As mentioned earlier, 
% We are interested in the space of sparse subgraphs $\gG_q$ where every subgraph in that space has the same sparsity. 
%It is clear that this space consist of all possible $\floor{\frac{q|\gE|}{100}}$-length subset of edges of $\gE$. 

\subsection{Problem Statement and Theoretical Motivation}
Given $q>0$, this paper aims to construct a sparse subgraph $\tilde{\gG} \triangleq (\gV, \tilde{\gE},\mX)$ with $q\%$ of original edges in $\gE$. Let $\gG_q$ be the space of all distinct sparse subgraphs of $\gG$ where every subgraph contains exactly $k = \floor{\frac{q|\gE|}{100}}$ edges, $\gG_q = \{\tilde{\gE} \subset \gE : \abs{\tilde{\gE}} = k\}$.
The objective of our supervised sparse graph construction is to find the parameters $\theta$ along with a sparse subgraph $\tilde{\gG} \in \gG_q$ that minimize $\gL_\mathrm{CE}$.

We can define a probability space $(\Omega,\gF,p)$ by considering $\gE$ as the sample space, $\gF = \gG_q \subseteq 2^{\Omega}$ as the event space and a suitable probability measure $p: \gF \rightarrow [0,1]$. 
%An important question is what kind of probability measure is suitable and how to obtain it. 
The probability measure is determined by which subgraphs result in a node representation that minimizes the loss in Eq.~\ref{eq:loss}, which, in turn, depends on the downstream task. As a result, it is unknown which probability distribution is suitable as a choice for $p$. Specifically, we can perform the following decomposition to predict the probability that a node $v$ belongs to a class $c\in C$,
% 
% \[
% P(Y|G) = \E_{\tilde{g}} P(Y|\tilde{g})  P(\tilde{g}|G)
% \]
\vspace{-3pt}
\begin{align}
\label{eq:theo1}
P(\tilde{Y}_{vc}|\gG) &= \sum_{\tilde{\gG} \in \gG_q} P(\tilde{Y}_{vc}|\tilde{\gG})  P(\tilde{\gG}|\gG).
    % & =  \E_{\tilde{G}\sim p^*} P(Y|\tilde{G}) f_\phi (p^*|G)
\end{align}
% \vspace{-4pt}
There are two issues with the above decomposition. First, it requires enumerating all possible candidate subgraphs $\tilde{\gG} \in \gG_q$. This is computationally challenging because there are $\binom{|\gE|}{k}$ subgraphs, and it is not possible to estimate the probabilities $P(\tilde{Y}_{vc}|\tilde{\gG})$,  $P(\tilde{\gG}|\gG)$ due to their dependence on the downstream task under consideration.

We address these issues by encoding $P(\tilde{\gG}|\gG)$ as a learnable neural network module $\tilde{p} = f_{\edgemlp,\phi}(\gG)$ that explicitly learns to estimate the probability measure $p$ for every edge based on the downstream task. The neural network searches the space $\gG_q$ by adjusting its learned probability estimate $\tilde{p}$ based on the gradient of the loss. Finally, we model $P(\tilde{Y}_{vc}|\tilde{\gG})$ as a GNN that takes the sparsified sample $\tilde{\gG}$ sampled from the learned distribution $\tilde{p}$. Hence, Eq.~\ref{eq:theo1} can be approximated by,
% sid: Naheed, can you check the following equation?
\vspace{-3pt}
% \begin{equation}
% \label{eq:theo_motiv}
% P(\tilde{Y}_{vc}|\gG) \approx (\E_{\tilde{\gG}\sim f_{\edgemlp,\phi}(\gG)} f_{\gnn,\theta}(\tilde{\gG}))f_{\edgemlp,\phi} (\gG).
% \end{equation}
\begin{equation}
\label{eq:theo_motiv}
P(\tilde{Y}_{vc}|\gG) \approx \E_{\tilde{\gG}\sim f_{\edgemlp,\phi}(\gG)} [f_{\gnn,\theta}(\tilde{\gG})f_{\edgemlp,\phi} (\gG)].
\end{equation}
% \vspace{-1pt}
Equation~\ref{eq:theo_motiv} follows since 
%is due to the fact that 
instead of directly searching over the space $\gG_q$, we rely on the distribution $\tilde{p}$ approximated via a neural network. Once this distribution is learned, the law of large numbers indicates that a sufficient number of samples from $\tilde{\gG} \sim \tilde{p}$ can estimate $P(\tilde{Y})$. Thus the summation can be replaced with the expected value from a learned GNN model using sparse subgraph samples. 
% This insight also guides us in designing aggregate node representations from ensembles of sparse subgraphs.
% Here $p^*$ is the population distribution we attempt to learn using parameterized encoding function $f_\phi$ with $\phi$ being the learnable parameters.
\begin{figure*}[t]
	\centering
	\includegraphics[width=1.0\linewidth]{Figures/SGS-GNN2.pdf}
	\caption{Illustration of the three modules in \sgs. The edge probability encoding module computes a probability distribution, the sampler module samples the subgraph, and downstream GNN makes predictions using that sparse subgraph.
		% Solid lines indicate forward pass and dashed lines indicate backpropagation pathways.
	}
	\label{fig:sgsarchitecture}
\end{figure*}
\section{Related Work}
\label{sec:related}

%Graph sparsification techniques~\cite{hashemi2024comprehensive, chen2023demystifying} are commonly used to improve the training speed of GNNs for large graphs to improve the prediction quality. Unsupervised sparsification relies on heuristics or domain knowledge but does not include downstream task information; in contrast, supervised sparsification is tailored to the specific downstream task.


\noindent\textbf{Unsupervised Graph Sparsification.} 
%Some unsupervised methods sparsify graphs based solely on graph structure while maintaining theoretical guarantees on the preserved properties. 
Effective resistance (ER)~\cite{spielman2011graph} based sampling generates spectral sparse subgraphs while bounding the eigenvalues of the original graph's Laplacian. FastGAT~\cite{srinivasa2020fast} uses ER to improve GNN efficiency, but the high computational cost of ER makes it impractical for large graphs.  However, a major advantage of ER is its ability to produce multiple sparse subgraphs, minimizing information loss and improving GNN performance compared to single sparse graph methods such as $k$-NN.
The random sparsifier is the fastest approach to get sparse subgraphs and is widely used in GNNs such as DropEdge~\cite{rong2019dropedge} and GraphSAGE~\cite{hamilton2017inductive}. GraphSAINT~\cite{zeng2019graphsaint} uses the normalized \emph{degree} as edge weight to assign low sampling probability to edges in denser clusters. Later, it is used for sampling subgraphs and often produces better results than random sparsifiers. 
%Our \edgemlp learns the probability distribution of sampling to generate multiple subgraphs as well.
Spanner (e.g., $t$-spanner)~\cite{dragan2011spanners} is a topology-based sparsifier that preserves distances between nodes in a sparse subgraph by a factor of $t$. 
%For GNNs with $t$-hops, using $t$-spanner subgraphs is beneficial as it retains information during aggregation. 
Spanning Tree and Forest are useful sparsifiers, as they preserve node connectivity, which helps message propagation in GNNs. Although both of these lack control over sparsity, the notion of connectivity is essential. %Thus, we use \emph{degree} based edge-weight \emph{prior} to encourage connectivity among components.
Some other topology-based sparsifiers include  Rank Degree Sparsifier~\cite{voudigari2016rank}, Local Degree Sparsifier~\cite{hamann2016structure},
Forest Fire~\cite{leskovec2007graph}, and Degree-based sparsification~\cite{su2024generic, liu2023dspar}.
Another class of unsupervised sparsifiers first computes similarities between two nodes as edge weight and then 
samples. It could be structural similarity, such as \emph{Jaccard distance} on a portion of shared neighbors (SCAN~\cite{xu2007scan}) or feature similarity (SimSparse~\cite{wu2023alleviating}, AGS-GNN~\cite{das2024ags}).

% Reinforcement learning-based sparsification, SparL~\cite{wickman2021sparrl}
% Bayesian Edge Sampling~\cite{hasanzadeh2020bayesian}

\noindent\textbf{Supervised Graph Sparsification.}
Supervised graph sparsification may have computational overhead due to their training phase but is often compensated by prediction quality in noisy or heterophilic graphs. Methods like SparseGAT~\cite{sparsegat} and SuperGAT~\cite{kim2022find} use the entire graph information during GNN training and learn sparse subgraphs through regularizers. 
%These \textit{implicit sparsifiers} methods are not scalable for large graphs. 
SGCN~\cite{li2020sgcn} is another sparsification method that optimizes the runtime by alternating the sparsifier's and GNN's learning. 
NeuralSparse~\cite{zheng2020robust}, PDTNet~\cite{luo2021learning}, explicitly learn to sample sparse subgraphs. Ours \sgs falls into a similar category with distinctions. NeuralSparse samples $k$-neighbors from each node to generate the sparse graphs, whereas \sgs globally learns the sampling distribution of all edges and then samples from that distribution, which is significantly faster. 
%In addition, sampling neighbors $k$ limits the exact control over sparsity, and the sparser are bound to choose some edges of the neighborhood that might be useless.
\sgs uses a prior probability distribution to narrow the sparse subgraph search space; this notion of prior is useful~\cite{wang2024probability} and recent work like Mixture of Graphs (MOG)~\cite{zhang2024graph} uses \textit{Jaccard similarity}, \textit{gradient magnitude}, and \textit{effective resistance} as prior for sparse subgraph selection.
Additionally, LAGCN~\cite{chen2020label} employs an edge classifier to modify graphs based on training nodes, while our \edgemlp coupled with regularizer uses training edges to foster homophily in the sampled subgraph.
%
% \naheed{@Sid, Discuss Sparsification works (in general)}
%
% \naheed{@Sid, Discuss UnSupervised sparsification and their limitations}
%
% \naheed{@Sid, Supervised sparsification and their limitations. How our method fits there.}
%
% \naheed{@Sid, Any other works that has similar working principle as us but may not be tailored for graph sparsification.} perhaps~\cite{wang2024probability}?
% 
%\cite{wang2024probability}, Generic graph sparsification, unsupervised \cite{su2024generic}, 
%%%%%%%%%%%%%%%%%%%%%%%%%%%%%%%%%%%%%%%%%%%%
%%%%%%%%%%%%%%%%%%%%%%%%%%%%%%%%%%%%%%%%%%%%
% %%%%%%%%%%%%%%%%%%%%%%%%%%%%%%%%%%%%%%%%%%%%
% \begin{figure}[!htbp]
% \centering
% \includegraphics[width=1.0\linewidth]{Figures/EdgeMLP.png}
% \caption{(Placeholder) Mechanism of \edgemlp}
% \label{fig:edgemlp}
% \end{figure}
% %%%%%%%%%%%%%%%%%%%%%%%%%%%%%%%%%%%%%%%%%%%%
% \FloatBarrier
\section{Proposed method: \sgs}
\label{sec:Method}
% \paragraph{}
Figure~\ref{fig:sgsarchitecture} depicts our proposed method \sgs. In the following, we discuss its major components.
%\paragraph{Edge Probability Encoding Module (module I).}
\subsection{Module I: Edge Probability Encoding}
Given input $\gG$, the edge probability encoding module (\edgemlp) maps the node features to edge weights in the range $[0,1]$ followed by normalization to turn the learned weights into probabilities. The learned edge weights represent the model's unnormalized confidence in the existence of each edge. 
%However, the sum of these scores across all edges may not necessarily sum to 1. Hence, the normalization is required.
% We use a Multi-layer Perceptron to learn the probability $p(e_{uv})$ for each edge $e_{uv}\in E$. $p(e_{uv})$ represents the likelihood of sampling  edge $(u,v)$ for sparsification. 
% 
%%%%%%% sid before
% \edgemlp learns the edge weights as a function of node encodings $\vh^{(i)}_u,\vh^{(i)}_v$:
% \begin{equation*}
%     \vh_{u}^{(i)} = \relu(\Phi^{(i)}, \vh_u^{(i-1)}\oplus\mathrm{AGG}(\{\vh_{r}^{(i-1)}:r\in \gN(u)\})),
% \end{equation*}
% \begin{equation}
% \label{eq:w_uv}
% w(e_{uv}) =  \texttt{Sigmoid}(\mlp_{\phi}((\vh^{(i)}_u - \vh^{(i)}_v) \oplus (\vh^{(i)}_u \odot \vh^{(i)}_v))).
% \end{equation}
% Here, $h^{(0)}_u = \vx_u$, $\gN(u)$ denotes the set of neighbors of node $u$, $\mathrm{AGG}$ indicates the aggregation operation from \texttt{GraphSAGE}~\cite{hamilton2017inductive}, $\oplus$ indicates concatenation and $\odot$ represents elementwise multiplication.
% There are several options available for encoding \(\vh^{(i)}_u\), such as \texttt{GCN} and \mlp. Special consideration is needed when using convolutional layers for large graphs. At this stage, we can create a random sparse subgraph or draw from a prior probability distribution (eq.~\ref{eq:prior}). This subgraph will differ from the one utilized later in the downstream GNN module. 
% Additionally, among other ways, concatenating the subtraction and multiplication of two embeddings~\cite{reimers2019sentence} is very effective.
% 
%%%%%%% sid after
% \edgemlp learns the edge weights of $(u,v)$ as a function of node embeddings $\vh^{(i)}_u,\vh^{(i)}_v$:
% \begin{equation}
% \label{eq:w_uv}
% w(e_{uv}) = \sigmoid(\mlp_{\phi}((\vh^{(i)}_u - \vh^{(i)}_v) \oplus (\vh^{(i)}_u \odot \vh^{(i)}_v))).
% \end{equation}
% Here, $\sigmoid$ refers to \texttt{Sigmoid} activation function, $\oplus$ indicates concatenation, and $\odot$ represents element-wise multiplication. The embedding $\vh^{(i)}_u$ of node $u$, can be computed from \texttt{MLP} or using convolutional layers such as SAGE~\cite{hamilton2017inductive} or GCN~\cite{kipf2016semi}. For example, if \texttt{SAGE} layer is used then, 
% \begin{equation}
% \label{eq:embh_u}
%     \vh_{u}^{(i)} = \relu(\mW^{(i)}, \vh_u^{(i-1)}\oplus\mathrm{AGG}(\{\vh_{r}^{(i-1)}:r\in \gN(u)\})).
% \end{equation}
% Here, $h^{(0)}_u = \vx_u$, $\mW^{(i)}$ is the learnable weight matrix at $i$-th layer, $\gN(u)$ denotes the set of neighbors of node $u$, $\mathrm{AGG}$ indicates the aggregation operation (e.g., \texttt{mean, max}).
% 
% \textcolor{blue}{
% For large-scale graphs, the neighborhood expansion $r \in \gN(u)$ in equation~\ref{eq:embh_u} is computationally heavy on memory. Thus instead of considering all neighbors, we propose to take a subset of neighbors $\tilde{\gN}(u) \subset \gN(u)$ following a fixed prior probability distribution, $q_u(\gN(u))$. We considered two such distributions: (i) Uniform distribution:  $\forall r \in \gN(u), q_u(r) = \frac{1}{\abs{\gN(u)}}$. and (ii) Degree-proportionate distribution: $\forall r \in \gN(u), q_u(r) \propto (\frac{1}{d_u} + \frac{1}{d_r})$
% }
\edgemlp learns the edge weights of $(u,v)$ as a function of node embeddings $\vh_u,\vh_v$:
\begin{equation}
\label{eq:w_uv}
w(e_{uv}) = \sigmoid(\mlp_{\phi}((\vh_u - \vh_v) \oplus (\vh_u \odot \vh_v))).
\end{equation}
Here, $\sigmoid$ refers to \texttt{Sigmoid} activation function, $\oplus$ indicates concatenation, and $\odot$ represents element-wise multiplication. Let us assume $\vh_u$ indicates the node embedding in matrix $\mH$ corresponding to node $u$. Thus the node embedding matrix $\mH$ can be computed from an \mlp in the following manner: 
$\mH = \relu(\mlp_\mW(\mX))$, where $\mlp_\mW$ is an MLP with weights $\mW$. 
% $h_u$ indicates the node embedding in $\mH$ corresponding to node $u$. 
\mlp is computationally efficient; however, it does not exploit the graph structural information. As a result, \mlp is not necessarily the most effective choice, as we have shown later in the Ablation study (section \ref{app:ablationstudy}).

A common way to incorporate graph structural information  is to use graph convolutions such as vanilla GCN~\cite{kipf2016semi} or SAGE convolution~\cite{hamilton2017inductive}.
For instance, one can compute graph-structure aware node embedding matrix using a single-layer \texttt{GCN} as the following: $\mH = \sigma(\hat{\mA}_\gG\mX\mW)$,  where, $\mW$ is the learnable weight matrix. However, considering the entire graph $\mA_\gG$ is memory intensive for large graphs. 

Thus \sgs takes a length $\floor{\frac{q|\gE|}{100})}$ subset of edges $\gE_\mathrm{sp} \subseteq \gE$ following a fixed prior probability distribution $p_\mathrm{prior}$ and uses the induced subgraph $\gG[\gE_\mathrm{sp}]$ for computing node embedding $\mH$. In order to maintain good connectivity in $\gG[\gE_{sp}]$, the prior distribution is defined as the following:
% We have considered two such prior distributions: (i) uniform: $\forall_{(u,v) \in \gE}~p_\mathrm{prior}(u,v) = \frac{1}{\abs{\gE}}$ and (ii) degree-proportionate:
$\forall_{(u,v) \in \gE}~p_\mathrm{prior}(u,v) \propto (\frac{1}{d_u} + \frac{1}{d_v})$, where $d_u,d_v$ are the degrees of nodes $u,v$.
% 
% The sampled sparse subgraph $\gE_\mathrm{sp}$ is used for node embedding. 
% This sparse graph is different from the one used later in the downstream GNN module, which is learned.
% Sid: I think mentioning SAGE here would be overkill; GCN explains our case.
% The only purpose of this sparse subgraph is to aid the efficient computation of embeddings, which is different from the one utilized later in the downstream GNN module.

% Furthermore, among other ways, concatenating the subtraction and multiplication of two embeddings~\cite{reimers2019sentence} is very effective.

% \subsubsection{Normalization.}
%\paragraph{Normalization.}
\noindent\textbf{Normalization.} 
Normalization turns the learned edge weights into a valid probability distribution. One simple choice is \emph{sum-normalization} computed as following: $\tilde{p}(e_{uv}) = {w(e_{uv})}/{\sum_{(u,v)\in \gE} w(e_{uv})}$.
Another choice is \emph{softmax-normalization} with temperature annealing,
\begin{equation}
\label{eq:softmaxnorm}
\tilde{p}(e_{uv}) = \frac{\exp(w(e_{uv})/T)}{\sum_{(u,v)\in \gE} \exp(w(e_{uv})/T)}.
\end{equation}
Here, $T>0$ is the temperature parameter.
% that interpolates the distribution from one-hot to uniform.
% When $T$ is large, the sampling distribution becomes close to uniform, while a small value of $T$ makes the samples identical to those from the categorical distribution~\cite{jang2016categorical}. 
When $T$ is large, the learned distribution $\tilde{p}$ approaches uniform distribution over edges, whereas the learned distribution $\tilde{p}$ approaches categorical distribution when $T$ is small~\cite{jang2016categorical}. 
% 
% A high and low $T$ value corresponds to the exploration and exploitation behavior of the distribution.
As the learned distribution approaches uniform distribution, the model tends to explore more diverse subgraphs from subgraph space $\gG_q$. On the other hand, as the learned distribution approaches categorical distribution, the model tends to explore less in $\gG_q$.
Hence, we vary $T$ as a function of training iterations such that in early iterations, the algorithm explores more while narrowing down to its preferred search space later on. 
We execute such an annealing mechanism with the following equation:
%where $t$ is the temperature that varies as a function of epoch in the following manner:
\begin{equation}
T = \max (T_\mathrm{min},T_0 - \mathrm{epoch} \cdot r),
\end{equation}
% 
where $r = (T_0 - T_\mathrm{min})/{\mathrm{max\_epochs}}$ is the annealing rate, $T_\mathrm{min}$ is the minimum allowable temperature, and $T_0$ is the initial temperature. The temperature linearly decreases from the initial value $T_0$ to its final value $T_\mathrm{min}$ with the epochs. We keep track of the $T$-value that gives the best validation accuracy and use it later during inference.
 
Alg.~\ref{alg:edgmlp} shows the pseudocode for \edgemlp. 

% %%%%%%%%%%%%%%%%%%
\begin{algorithm}[!hbt]
\caption{\edgemlp Module}
\begin{algorithmic}[1] % The [1] here is for line numbering
\small
\STATE \textbf{Input:} $\gG (\gV, \gE, \mX)$, sample \% $q$, \#layers $L$, $\mathrm{epoch}$,  $\mathrm{max\_epochs}$
\STATE $\forall_{(u,v)\in\gE}~p_\mathrm{prior}(u,v) \gets \frac{1/d_u + 1/d_v}{\sum_{i,j\in \gE} (1/d_i + 1/d_j)}$
\STATE $\gE_\mathrm{sp} \gets \text{Multinomial}(\gE, p_\mathrm{prior}, \floor{\frac{q|\gE|}{100}})$ %\COMMENT{if $\gG$ is large otherwise use $\gE$}
\STATE $\mH \gets \texttt{GCN}_\mW(\gE_\mathrm{sp},\mX, L)$
\STATE $\forall_{(u,v)\in \gE}~\vw(u,v) = \sigma(\mlp_{\phi}((\vh^{(i)}_u - \vh^{(i)}_v) \oplus (\vh^{(i)}_u \odot \vh^{(i)}_v))$
\STATE $T \gets \max (T_\mathrm{min},T_0 - \mathrm{epoch} \cdot \frac{T_0 - T_\mathrm{min}}{\mathrm{max}\_\mathrm{epochs}})$
%\STATE $p(e_{uv}) = \frac{\exp(w(e_{uv}))/t)}{\sum_{(u,v)\in \gE} \exp(w(e_{uv}))/t)}: \forall (u,v) \in \gE $
\STATE $\tilde{p} \gets \mathrm{Softmax}(\vw/T)$    
\STATE \textbf{Return} $\tilde{p}, \vw$
\end{algorithmic}
\label{alg:edgmlp}
\end{algorithm}
% %%%%%%%%%%%%%%%%%%
% 
% As we will show later that, this learned probabilities $\tilde{p}$ can approximate the true unknown probabilities with arbitrarily small error due to the universal approximation property of $f_{\text{MLP},\phi}$.
% \paragraph{Theoretical analysis.} Let us assume that (i) there is an idealized learning ORACLE that knows the true probability distribution over the edges, (ii) the true distribution $p^*$ is a continuous function of node features $\mX$ and (iii) $\mX$ is a compact subset (bounded and closed) of Euclidean space $\mathbb{R}^n$.
% Now, the Universal Approximation Theorem states that a feed-forward neural network with at least one hidden layer and a finite number of neurons can approximate any continuous function $f: \mathbb{R}^n \rightarrow \mathbb{R}$ on a compact subset of $\mathbb{R}^n$, given a suitable choice of weights and activation functions. We use this result to state the following proposition under the assumptions I-III.
% \begin{proposition} For any error $\epsilon>0$, there exists an \mlp $f_{\mlp,\phi} = \tilde{p}$ that approximates the function $p^*$.
% \begin{equation}
% \label{eq:uapp}
% \sup_{e \in \mathcal{E}} \|\tilde{p}(\vx_e) - p^*(\vx_e)\|_1 \leq \epsilon
% \end{equation}
% where both $\tilde{p}$ and $p^*$ are functions of edge features $\vx_e = \vx_e =  (\vh^{(i)}_u - \vh^{(i)}_v | \vh^{(i)}_u \odot \vh^{(i)}_v)$ as used in equation~\ref{eq:w_uv}.
% \end{proposition}

% More discussion on this result can be found in Appendix~\ref{theo:uap}.

%\paragraph{Sparse subgraph sampling (module II).}
\subsection{Module II: Sparse Subgraph Sampling}
Given the learned distribution $\tilde{p}$ over the edges of the input graph, sparse subgraph sampling aims to construct a sparse graph with the user-given sparsity constraint $q$. We do not know which discrete distribution has $\tilde{p}$ as parameters. A natural choice is to construct $\tilde{\gG} = (\gV,\tilde{\gE},\mX)$ by assuming that $\tilde{p}$ is a parameter of a \emph{Multinomial} distribution. Hence we can sample $k=\floor{\frac{q|\gE|}{100}}$ edges as
$\tilde{\gE} \sim \text{Multinomial}(\tilde{p},k)$.

We can also construct $\tilde{\gG}$ by assuming that $\tilde{p}$ is a parameter of some categorical distribution and use \emph{Gumbel Softmax trick}~\cite{jang2016categorical}. The idea is to induce \emph{Gumbel noise} $g_{uv}\sim Gumbel(0,1)$ to the edges and select Top-$K$ edges with the highest probabilities.
% the $k$ edges using $\texttt{TopK}$ function. 
In order to sample edges according to categorical distribution, we replace our softmax-normalization (Equation~\ref{eq:softmaxnorm}) with the following:
\begin{equation}
\tilde{p}(e_{uv}) = \frac{\exp(({\log w(e_{uv})+g_{uv}})/{T})}{\sum_{(u,v)\in \gE} \exp({(\log w(e_{uv})+g_{uv})}/{T
})}.    
\end{equation}
Adding noise ensures that we are taking different samples at each time, and with low temperatures ($T=0.1, T=0.5$), the samples become identical to samples from a categorical distribution~\cite{jang2016categorical}. 
%We can also select the top-$\floor{\frac{q|\gE|}{100}}$ edges with the highest probabilities but lack exploration capability during training. However, it can give us one of the best sparse subgraphs during inference. 
%\naheed{@Sid, Discuss how we sampled from the categorical distribution with noise using Gumbel-softmax trick.}
%Sid: resource: https://docs.google.com/document/d/1HrslfnxNP6dso6DX9OXFvXSzQYF9Q-mz/edit
%We have empirically evaluated the effect of these distribution choices on the performance of \sgs in the section on ablation studies.

%\paragraph{Theoretical analysis.}
\noindent\textbf{Theoretical analysis I.} 
Let $\mathcal{E}^*$ and $\mathcal{\tilde{E}}$ denote the ordered collection of edges sampled by the idealized learning ORACLE according to true distribution $p^*$ and by \sgs according to learned probability $\tilde{p}$ respectively. For analytical convenience, let us assume that the algorithm samples $k$ edges with replacement. We have the following theorem that lower-bounds the \#edges common between sampled subgraphs from \sgs and idealized learning ORACLE.

\begin{theorem}[Lower-bound] The expected number of edges sampled by both \sgs and idealized learning ORACLE satisfies
\vspace{-10pt}
\begin{equation} 
\mathbb{E}[|\mathcal{E}^* \cap \mathcal{\tilde{E}}|] \geq k \sum_{j=1}^{|\mathcal{E}|} \frac{(p^*_j + \tilde{p}_j - \epsilon)^2}{4},
\end{equation}
where $k = \floor{q|\mathcal{E}|/100}$ with $0 \leq q \leq 100$ as a user-specified parameter and $\epsilon\in [0,1]$ is the error.
\end{theorem}
The proof is in Appendix~\ref{theo:commonedges}. The implications are:
\begin{enumerate}[wide, labelwidth=!, labelindent=2pt,itemsep=1pt,topsep=1pt]
    \item Let the true distribution be uniform. In the best-case scenario $\epsilon \rightarrow 0$ and $\tilde{p} = p^* = \frac{1}{|\mathcal{E}|}$. Then there are at least $\frac{k}{|\mathcal{E}|}$ common edges between $\tilde{\gG}$ and $\gG^*$. 
    % However, since $k < \abs{\mathcal{E}}$, the lower-bound of $\mathbb{E}[|\mathcal{E}^* \cap \mathcal{\tilde{E}}|] \geq \frac{k}{|\mathcal{E}|}$ is not very useful even though the learned distribution is accurate. 
    Since $k << \abs{\mathcal{E}}$, this specific scenario suggests that the learned sparse subgraph may not overlap much with the true one even after we have learned the true distribution. When the true distribution is uniform, every subgraph from $\gG_q$ is a global minimizer of the task-specific loss $\mathcal{L}_{CE}$. Otherwise, the learning ORACLE would have put more mass on certain edges and the distribution $p^*$ would not have been uniform. As individual subgraphs are indistinguishable in terms of performance, this case beats the purpose of supervised sparsification.

    \item Let the true distribution be one-hot. In other words, suppose $\tilde{p} = p^* = \delta_{ij}$, where $\delta_{ij}$ is the \emph{Kronecker-delta}. In this case, as $\epsilon \rightarrow 0$, the lower bound reduces to 
    \vspace{-8pt}
    \begin{equation*}
    \mathbb{E}[|\mathcal{E}^* \cap \mathcal{\tilde{E}}|] \geq k \sum_{j=1}^{|\mathcal{E}|} (\tilde{p}_j)^2 = k.
    \end{equation*}
    % \vspace{-4pt}
    This identity suggests that the sampled edges are expected to completely overlap with the true sparse subgraph. 
    
\end{enumerate}

% In practice, it would be surprising if the true distribution is uniform. If it was, the loss landscape would have been flat. 
For strong heterophilic graphs ($\gH_n$ is small), the true distribution is less likely to be uniform. Because a uniform edge sample would retain a similar node homophily as in the input graph, and such a subgraph would not be able to minimize $\mathcal{L}_{CE}$~\cite{das2024ags}. Thus, it is important for the learned probability distribution to approximate $p^*$ so that the sampled subgraph is close enough to the true one.
 
We have analyzed $\tilde{\gG}$ generated by \sgs on a synthetic graph in Appendix~\ref{app:toymoon}.

\begin{comment}
\begin{theorem}[Upper-bound] The expected number of edges sampled by both \sgs and idealized learning ORACLE satisfies
\begin{equation}
\mathbb{E}[|\mathcal{E}^* \cap \mathcal{\tilde{E}}|] \leq k (1 - \frac{\|p^* - \tilde{p}\|_1}{2}) 
\end{equation}
where $k = \floor{q|\mathcal{E}|/100}$ with $0 \leq q \leq 100$ as a user-specified parameter.
\end{theorem}

\textit{The implication of the upper bound.} When $\tilde{p} \rightarrow p^*$, the norm $\|p^* - \tilde{p}\|_1 \rightarrow 0$; therefore, the number of common edges could be close to $k$.

\end{comment}

%\paragraph{GNN module (module III).}
\subsection{Module III: Downstream GNN and Loss Functions}
At this stage, we input the sampled subgraph to a downstream GNN that supports edge weights as computed in Equation~\ref{eq:w_uv}; since the edge weights of the sampled edges are one of the ways we optimize \edgemlp via backpropagation. An example \gnn would be 
\begin{equation}
\hat{\mY} = \texttt{Softmax}(f_{\gnn,\theta}(\gV, \tilde{\gE}, \mX, \tilde{\vw})),
\end{equation}
where $\tilde{\gE}$ refers to the edges of the sampled sparse subgraph $\gG'$ and $\tilde{\vw} = \vw[\tilde{\gE}]$ contains the edge weights. 
%One of the ways gradient optimizers update the parameters of \edgemlp is by backpropagating through these sampled weights.

% We do not explicitly construct the sparse adjacency matrix $A_{\tilde{G}}$

%\paragraph{Loss functions.}
\noindent\textbf{Loss functions.} 
We introduce two regularizers to engrain various inductive biases to \sgs and combined these functions with the Cross-Entropy loss $\gL_\mathrm{CE}$ as follows:
\begin{equation}
\mathcal{L} = \alpha_1\mathcal{L}_\mathrm{CE} + \alpha_2 \mathcal{L}_\mathrm{assor} + \alpha_3 \mathcal{L}_\mathrm{cons},
\end{equation}
where $0 \leq \alpha_1,\alpha_2,\alpha_3 \leq 1$ are regularizer coefficients.

The \textbf{Assortativity loss} $\mathcal{L}_\mathrm{assor}$ uses the labels of the training nodes to force nodes with similar labels to have higher edge weights while forcing dissimilarly labeled nodes to have a small nonzero weight. This regularizer encourages edge homophily in the sampled sparse graph.
\vspace{-7pt}
\begin{equation}
\small
     \mathcal{L}_\mathrm{assor} \triangleq -\sum_{(u,v) \in \gE:u \land \gV_L \land v \in \gV_L} \mathbb{I}(y_u=y_v)\cdot \log w(e_{uv}),
\end{equation}
% \vspace{-20pt}
where $\mathbb{I}(.)$ is an indicator function that returns $0$ or $1$. 

The \textbf{Consistency loss} defined below encourages learned edge probabilities to reflect the similarity between node embeddings or features:
\vspace{-8pt}
\begin{equation}
\mathcal{L}_\mathrm{cons} \triangleq \sum_{(u,v) \in \tilde{\gE}} \|w(e_{uv}) - \mathrm{cosine}(\vh_u^l,\vh_v^l)\|,
\end{equation}
where $\mathrm{cosine}(\vh_u^l,\vh_v^l) = {\vh_u^l\cdot \vh_v^l}/{\|\vh_u^l\|\|\vh_v^l\|}$ is the cosine similarity of the learned GNN embeddings $\vh_u^l,\vh_v^l$ of nodes $u$, $v$ from layer $l$, and $w(e_{uv})$ is the learned probability for edge $(u,v)$ in the sparse graph $\tilde{\gG}$. This mechanism aligns the edge probabilities with the global graph structure and ensures that the sparsifier learns to preserve edges consistent with the broader graph relationships. 
% In ablation studies, we have evaluated the utility of these additional regularizing functions.
% 
% The dashed arrows in Fig.~\ref{fig:sgsarchitecture} illustrate the pathways of \edgemlp and \gnn model parameter updates through these losses during backpropagation. 
%Alg.~\ref{alg:sgstraining} in Appendix~\ref{app:algorithm} shows the pseudocode of our base training algorithm.
%These loss functions allow backpropagation to update \edgemlp, using only the training edges with $\gL_\mathrm{assor}$, and all the edges with $\gL_\mathrm{cons}$.


%\paragraph{Theoretical analysis.}
\noindent\textbf{Theoretical analysis II.} 
We consider vanilla GCN as a downstream GNN to examine how the sparse subgraph, $\tilde{\gG}$ from \sgs, affects node embeddings compared to the ideal subgraph $\gG^*$ from a learning ORACLE. Suppose an $L$-layer GCN produces embeddings $\tmH^{(L)}$ and $\mH^{*(L)}$ when taking $\tilde{\gG}$ and $\gG^*$ as input, respectively.
%Our goal is to analyze the respective encodings produced by a $L$-layer GCN when the input subgraphs are $\gG^*$ and , respectively. 
%$\gG^*$ (corresponding to adjacency matrix $\mA_{\gG^*}$) and $\tilde{\gG}$ (corresponding to adjacency matrix $\mA_{\tilde{\gG}}$) respectively. 
% For simplicity, we will denote the ideal and our sampled sparse matrices, $\mA_{\gG^*}$ as $\mA^*$ and $\mA_{\tilde{\gG}}$ as $\tmA$ respectively.
% 
% From eq.~\ref{eq:gcnlayer}, a single GCN layer is defined as $\mH^{(l+1)} =\sigma(\hat{\mA}\mH^{(l)}\mW^{(l)})$,
% % where $\hat{\mA} = \mD^{-1/2}\mA\mD^{-1/2}$ is the normalized adjacency matrix, $\mH^{(l)}$ is the input to the $l$-th layer with $\mH^{(0)} = \mX$, $\mW^{(l)}$ is the learnable weight matrix for $l$-th layer and $\sigma$ is non-linear activation function. 
% and an $L$-layer GCN produces embeddings $\tmH^{(L)}$ and $\mH^{*(L)}$ when it takes sparse matrices $\tmA$ and $\mA^*$ as input. 
% 
Is there an upper bound of the difference in the downstream node encodings $\mathbb{E}[\normLtwo{\tmH^{(L)} - \mH^{*(L)}}]$, due to the use of a learned subgraph?

To that end, we assume for all $l\in L$, $\normLtwo{\mW} \leq \alpha < 1$ where $\alpha$ is a constant. This is reasonable since each $\mW^{(l)}$ is typically controlled during training using regularization techniques, e.g., weight decay. As input features in $\mX$ are bounded, we also assume that there exists a constant $\beta$ such that $\forall l>0$, $\normLtwo{\mH}^{(l)} \leq \beta$. We also assume that $\sigma$ is \textit{Lipschitz continuous} with \textit{Lipschitz constant} $L_\sigma$. 
% For instance,  activation functions such as \relu, \texttt{Sigmoid}, or \texttt{TanH} are \textit{Lipschitz continuous}. 
% In particular, 
We assume \relu activation to simplify our analysis since \relu has a Lipschitz constant $L_\sigma = 1$. Under these assumptions, we have the following theorem (proof in Appendix~\ref{theo:gcnembed}). 

\begin{theorem}[Error in GCN encodings]
For sufficiently deep L-layer GCN, the error in node embeddings  

\vspace{-15pt}
{\scriptsize
\[
\mathbb{E}[\lim_{L \to \infty} \normLtwo{\tmH^{(L)} - \mH^{*(L)}}] < \frac{\beta}{1-\alpha}\sqrt{2k (1 - \sum_{j=1}^{|\mathcal{E}|} \frac{(p^*_j + \tilde{p}_j - \epsilon)^2}{4})}.
\]
}
\vspace{-15pt}
\end{theorem}
% The proof is given in Appendix~\ref{theo:gcnembed}.
 % 
% 
\subsection{\sgs Training and Additional Details}
\label{subsec:largescale}
\begin{algorithm}[!ht]
\caption{\sgs Training}
\begin{algorithmic}[1] % The [1] here is for line numbering
\small
\STATE \textbf{Input:} $\gG (\gV, \gE, \mX)$, sample \% $q$, \#layers $L$, METIS Parts $n$
% \STATE \textbf{Output:} \texttt{EdgeMLP}, \texttt{GNN}
\STATE $p_\mathrm{prior}(u,v) \gets \frac{1/d_u + 1/d_v}{\sum_{i,j\in \gE} (1/d_i + 1/d_j)}$

\STATE $\gG_\mathrm{parts} \gets \{\gG_1,\gG_2,\cdots,\gG_n\}= \mathrm{METIS} (\gG(\gV,\gE, p_\mathrm{prior}), n)$

\FOR{$\mathrm{epoch}$ in $\mathrm{max\_epochs}$}

    \FOR {$\gG_i(\gV_i,\gE_i,\mX_i,p^i_\mathrm{prior}) \in \gG_\mathrm{parts}$}
        \STATE $\tilde{p}, \vw \gets \edgemlp(\gE_i, \mX_i, L)$ \COMMENT{\textbf{Algorithm~\ref{alg:edgmlp}}}    
        \STATE $\tilde{p}_a \gets \lambda \tilde{p}+(1-\lambda)p^i_\mathrm{prior}$/*\textbf{Augmenting $\tilde{p}$ with prior}*/
        \STATE $\tilde{\gE}, \tilde{\vw} \gets \mathrm{Sample}(\tilde{p}_a, \vw, \floor{\frac{q|\gE|}{100}})$   \COMMENT{\textbf{Module II}}
        \STATE $\hat{\mY}, \tilde{\mH} \gets \mathrm{GNN}_\theta(\tilde{\gE},\mX_i,\tilde{\vw})$ \COMMENT{\textbf{Module III}}

        \STATE Compute $\gL_{CE}, \gL_\mathrm{assor}$, and $\gL_\mathrm{cons}$ using $\hat{\mY},\tilde{\mH}$
        
        \STATE $\gL \gets \alpha_1\cdot \gL_\mathrm{CE}+ \alpha_2\cdot \gL_\mathrm{assor}+ \alpha_3\cdot \gL_\mathrm{cons}$
        % 
        \STATE Backward Propagate through $\gL$
        % 
    \ENDFOR
    
\ENDFOR
%\STATE \textbf{Return} \texttt{EdgeMLP}, \texttt{GNN} 
\end{algorithmic}
\label{alg:sgstraining}
\end{algorithm}

Alg.~\ref{alg:sgstraining} outlines the pseudocode for training \sgs. \sgs starts with two precomputation steps:
i) computing the degree-proportionate edge weight as a \emph{prior} to enhance the learned distribution $\tilde{p}$ (line 1), and
ii) partitioning the input graph using METIS~\cite{karypis1997metis} for batch processing (line 2). Towards computing the loss for every partition at each iteration,  \sgs executes Edge probability encoding, Learned distribution augmentation with a prior, Sparse subgraph sampling and node embedding via GNN. Finally, the loss is backpropagated, the update pathways of which have been illustrated in Figure~\ref{fig:sgsarchitecture} earlier.

\textbf{Batch processing.} 
We can use \edgemlp from Alg.~\ref{alg:edgmlp} to compute edge weights in large-scale graphs, but efficient batch processing on edges is necessary for stochastic training of GNNs so as to reduce the risk of getting stuck in local minima.
It is crucial to select a batch of edges that have high locality, preferably from within a cluster, and we utilize METIS to achieve this. We could have made partitions small enough to fit GPUs and then applying GNN without any sparsification, similar to ClusterGCN~\cite{chiang2019cluster}. However, certain edges, such as task-irrelevant edges, may negatively impact performance, particularly in heterophilic or noisy graphs. In such cases, a high-quality learned sparse subgraph performs better than full graph, as validated in our experiments (\S\ref{subsubsec:fixedsampler}). 
% Sparsification also enables larger partitions without significant information loss.

\textbf{Augmenting $\tilde{p}$ with prior.} 
% We begin training for each graph partition by learning the probability distribution and edge weights from \edgemlp (line 5). 
While $\tilde{p}$ can be directly used to sample sparse subgraphs, the resulting subgraph may be suboptimal for message passing due to missing bridge edges connecting low-degree node pairs.
Thus augmenting the sampler with  $p_\mathrm{prior}$, which favors such edges, results in better quality sparse subgraph. $p_\mathrm{prior}$ is defined as
% Although \emph{effective resistance} can be an option, it is impractical for large graphs. 
\vspace{-8pt}
\begin{equation}
\label{eq:prior}
 p_\mathrm{prior}(u,v) \triangleq \frac{1/d_u + 1/d_v}{\sum_{i,j\in \gE} (1/d_i + 1/d_j)},
\end{equation}
where $d_u,d_v$ are degrees of nodes $u,v$. We control the emphasis of prior on the learned distribution with a parameter $\lambda \in [0,1]$, resulting in the \emph{augmented probability distribution}: $\tilde{p}_{a}(u,v) = \lambda \tilde{p}(u,v) + (1-\lambda) p_\mathrm{prior}(u,v)$ (line 7). The impact of $p_\mathrm{prior}$ on \sgs is discussed in Appendix~\ref{app:parameters}.

Another enhancement we consider is the \textbf{conditional updates} to \edgemlp. Since backpropagation is computationally expensive, we only update \edgemlp when the training F1-score from the learned sparse subgraph exceeds the baseline subgraph from $p_\mathrm{prior}$. The detailed algorithm for \sgs with conditional updates is in Appendix~\ref{app:algorithm}.

During inference, we use the learned probability distribution from \edgemlp, sample an ensemble of sparse subgraphs, and mean-aggregate their representations to produce final prediction on a test node. The pseudocode for inference (Alg.~\ref{alg:sgsinference}) is in Appendix~\ref{app:algorithm}.

\begin{comment}%\paragraph{Degree bias augmentation with a learned probability distribution.}
\noindent\textbf{Degree bias augmentation with a prior.} 
The search space for learning algorithms to identify optimal sparse graphs can be extensive in large-scale graphs. To address this, we can use a prior probability distribution $p_\mathrm{prior}$ based on \emph{degree}~\cite{zeng2019graphsaint}. This distribution favors edges connected to vertices with low degrees and assigns lower selection probabilities to edges within denser clusters. Due to computational efficiency, we chose degree-proportionate prior over \emph{effective resistance}. We control the bias between the learned and prior probability distributions using the control parameter $\lambda \in [0,1]$. The probability distribution becomes, $\tilde{p}(u,v) = \lambda \tilde{p}(u,v) + (1-\lambda) p_\mathrm{prior}(u,v)$, where $\tilde{p}$ is the learned probability distribution and degree norm $p_\mathrm{prior}$ for edge ($u,v$) is,
\vspace{-8pt}
 \begin{equation}
\label{eq:prior}
 p_\mathrm{prior}(u,v) = \frac{1/d_u + 1/d_v}{\sum_{i,j\in \gE} (1/d_i + 1/d_j)}.
\end{equation}
%\vspace{8pt}
% \vspace{-10pt}
% Alg.~\ref{alg:sgstraining} outlines the pseudocode of \sgs.

\noindent\textbf{METIS graph partitioning.}
We can use \edgemlp from Alg.~\ref{alg:edgmlp} for edge weight computation in large graphs. However, we want efficient batch processing on edges, avoid gradient accumulation, and make the training process stochastic. This helps train GNN models with fewer epochs by reducing the likelihood of getting stuck in local minima.
One crucial step is selecting a batch of edges such that the vertices in a batch of edges have high locality and preferably from within a cluster. To achieve this, we consider very fast graph partitioning methods like METIS~\cite{karypis1997metis}.

For inference, we use the probability distribution from \edgemlp, sample multiple sparse subgraphs, and aggregate their predictions. The pseudocode of inference (Alg.~\ref{alg:sgsinference}) is provided in Appendix~\ref{app:algorithm} for brevity.
\end{comment}


% \paragraph{Dynamic temperature annealing.}
% \[
% t = \max (t_\mathrm{min},t_0 - \mathrm{epoch} \cdot r)
% \]
% where $r = \frac{(t_0 - t_\mathrm{min})}{\mathrm{total}\_\mathrm{epochs}}$ is the annealing rate, $t_\mathrm{min}$ is minimum allowable temperature, and $t_0$ is the initial temperature. 
% \paragraph{Consistency loss.}

% \[
% L_\mathrm{cons} = \sum_{(u,v) \in \tilde{\gG}} \|\tilde{p}(e_{uv}) - \mathrm{cosine}(\vh_u,\vh_v)\|
% \]
% where $\mathrm{cosine}(\vh_u,\vh_v) = \frac{\vh_u\cdot \vh_v}{\|\vh_u\|\|\vh_v\|}$ is the cosine similarity of the learned GNN embeddings $\vh_u,\vh_v$ corresponding to nodes $u$ and $v$ and $p(e_{uv})$ is the learned probability for edge $(u,v)$ in the sparse graph $\tilde{\gG}$.

% \paragraph{Performance-Gated Training.}
% %%%%%%%%%%%%%%%%%%

% \noindent\textbf{Conditional update of \edgemlp.}
% Backward propagation is often the most computationally intensive part of training, so we employ a conditional mechanism to update \edgemlp selectively. We evaluate the learned sparse subgraph against a subgraph from the prior probability distribution $p_\mathrm{prior}$. If the training F1-score from the learned sparse subgraph is better than the baseline, parameters of \edgemlp are updated; otherwise, the update is skipped. Detailed algorithm for \sgs with conditional updates (Alg.~\ref{alg:sgstrainingpriorfull}) is provided in Appendix~\ref{app:algorithm}.

% \subsection{Theoretical results}
% \label{subsec:theory}
% The final objective of our theoretical analysis is to obtain an upper bound on the learned node embedding produced by module III (section 4.4.4). To obtain this upper bound we consider as a reference an idealized learning ORACLE that knows the true sampling probability distribution. Toward our final theoretical goal, first, we obtain an upper bound on the learned probability distribution (section 4.4.1). Afterwards, we obtain an upper bound on the number of common edges (section 4.4.2) which helps us bound the error in the learned Adjacency matrix (section 4.4.3). Finally, in section 4.4.4, we obtain an upper bound on the learned node embedding produced by \sgs.


% \subsection{Computational Complexity}
\textbf{Computational Complexity.}
Suppose the number of hidden dimension $H\approx F$, where $F$ is the dimension of the node features. The cost of an $L$-layer GCN is $\bigO(L(|\gE|\cdot H + |\gV| \cdot H^2))$~\cite{chiang2019cluster}. 
The cost of Alg~\ref{alg:edgmlp} is $\bigO(L(|\gE_\mathrm{sp}|\cdot H + |\gV| \cdot H^2)+ \abs{\gE}\cdot H^2)$, since computing node-embedding (line 4, Alg.~\ref{alg:edgmlp}) using sparse graph $\gE_\mathrm{sp}$ costs $\bigO(L(|\gE_\mathrm{sp}|\cdot H + |\gV| \cdot H^2))$, and edge weight computation using \mlp (line 5, Alg.~\ref{alg:edgmlp}) costs $\bigO(\abs{\gE}\cdot H^2)$. 
% 
With an $L$-layer GCN used as downstream GNN acting on the sparse subgraph $\tilde{\gE}$, the downstream GNN costs $\bigO(L(|\tilde{\gE}|\cdot H + |\gV| \cdot H^2)$. 
% For loss, $L_\mathrm{assor}$, $L_\mathrm{cons}$ is $\bigO(|\tilde{\gE}$ and $\bigO(|\tilde{\gE}|\cdot H)$ respectively. 
Since, $\abs{\gE_\mathrm{sp}}=\abs{\tilde{\gE}}$ the total complexity of \sgs  (Alg.~\ref{alg:sgstraining}) is $\bigO(L(|\tilde{\gE}|\cdot H + |\gV| \cdot H^2)+ \abs{\gE}\cdot H^2)$.

\textbf{Space complexity.} Let $n$ partitions from METIS have similar sizes. The memory requirement for \sgs with $L$-layer GCN is  $\bigO\left(\frac{|\gE| + |\gV|\cdot H}{n} + L \cdot H^2 \right)$. 
% The former part is for graph-related information, and the latter is for learnable parameters.



% $\bigO(L \cdot H^2)$ for edge weight, 
% There is a precomputation cost of $\bigO(|\gE|)$ for $p_\mathrm{prior}$, and for large-scale graphs, there is an additional one-time partition cost from METIS.

% Then computation complexity of \edgemlp 

% If $H\approx F$, then the overall complexity of $L$ layer GCN can be approximated as $\bigO(L(|\gE|\cdot H + |\gV| \cdot H^2))$. If GCN is used at \edgemlp and \gnn, then the complexity of the computation using a sparse graph of size $|\tilde{\gE}|=\floor{\frac{q|\gE|}{100}}$ becomes $\bigO(L(|\tilde{\gE}| \cdot H + |\gV| \cdot H^2)$. The cost of \mlp for the edge weight computation is $\bigO(\abs{\gE}\cdot H^2)$. There is a pre-computation cost of $\bigO(|\gE|)$ for $p_\mathrm{prior}$, and for large-scale graphs, there is an additional one-time partition cost from METIS.
% Memory requirements:
% sid: maybe remove this or put this into appendix
% The memory requirements for considering the entire graph include several components: $\bigO(|\gE|)$ for edges, $\bigO(|\gV| \cdot F)$ for features, $\bigO(L|\gV| \cdot H)$ for the hidden state, $\bigO(L \cdot H^2)$ for trainable parameters, and $\bigO(L \cdot \gV \cdot H)$ for activations and gradients. If a partitioning method like METIS is employed, resulting in $n$ partitions, each partition will require approximately $\bigO\left((|\gE| + |\gV| F + L \cdot |\gV| \cdot H)/{n}\right) + \bigO(L \cdot H^2)$ of memory.
\section{Experiments}
\begin{table*}[ht]
    \footnotesize
    \centering
    \renewcommand{\arraystretch}{1.1} % Adjusts the row spacing
    \resizebox{16cm}{!} 
    { 
    \begin{tblr}{hline{1,2,Z} = 0.8pt, hline{3-Y} = 0.2pt,
                 colspec = {Q[l,m, 13em] Q[l,m, 6em] Q[c,m, 8em] Q[c,m, 5em] Q[l,m, 14em]},
                 colsep  = 4pt,
                 row{1}  = {0.4cm, font=\bfseries, bg=gray!30},
                 row{2-Z} = {0.2cm},
                 }
\textbf{Dataset}       & \textbf{Table Source} & \textbf{\# Tables / Statements} & \textbf{\# Words / Statement} & \textbf{Explicit Control}\\ 
\SetCell[c=5]{c} \textit{Single-sentence Table-to-Text}\\
ToTTo \cite{parikh2020tottocontrolledtabletotextgeneration}   & Wikipedia        & 83,141 / 83,141                  & 17.4                          & Table region      \\
LOGICNLG \cite{chen2020logicalnaturallanguagegeneration} & Wikipedia        & 7,392 / 36,960                  & 14.2                          & Table regions      \\ 
HiTab \cite{cheng-etal-2022-hitab}   & Statistics web   & 3,597 / 10,672                  & 16.4                          & Table regions \& reasoning operator \\ 
\SetCell[c=5]{c} \textit{Generic Table Summarization}\\
ROTOWIRE \cite{wiseman2017challengesdatatodocumentgeneration} & NBA games      & 4,953 / 4,953                   & 337.1                         & \textbf{\textit{X}}                   \\
SciGen \cite{moosavi2021scigen} & Sci-Paper      & 1,338 / 1,338                   & 116.0                         & \textbf{\textit{X}}                   \\
NumericNLG \cite{suadaa-etal-2021-towards} & Sci-Paper   & 1,355 / 1,355                   & 94.2                          & \textbf{\textit{X}}                    \\
\SetCell[c=5]{c} \textit{Table Question Answering}\\
FeTaQA \cite{nan2021fetaqafreeformtablequestion}     & Wikipedia      & 10,330 / 10,330                 & 18.9                          & Queries rewritten from ToTTo \\
\SetCell[c=5]{c} \textit{Query-Focused Table Summarization}\\
QTSumm \cite{zhao2023qtsummqueryfocusedsummarizationtabular}                        & Wikipedia      & 2,934 / 7,111                   & 68.0                          & Queries from real-world scenarios\\ 
\textbf{eC-Tab2Text} (\textit{ours})                           & e-Commerce products      & 1,452 / 3,354                   & 56.61                          & Queries from e-commerce products\\
    \end{tblr}
    }
\caption{Comparison between \textbf{eC-Tab2Text} (\textit{ours}) and existing table-to-text generation datasets. Statements and queries are used interchangeably. Our dataset specifically comprises tables from the e-commerce domain.}
\label{tab:datasets}
\end{table*}

\subsection{Datasets}

%We utilize three real-world datasets to evaluate the performance of \model. The JiNan dataset is derived from traffic flow statistics in a real city. Similar to the processing in \cite{song2020STSGCNaaai}, we set the time interval to 5 minutes. However, unlike \cite{song2020STSGCNaaai}, our constructed adjacency matrix is not based on distance but on connectivity. Specifically, if two detection points are on the same road, an edge is constructed between them. The remaining two datasets are based on California's highways, which differ geographically from JiNan dataset. The detailed information is shown in Table~\ref{tab.datasets}.

To evaluate the performance of \model, we utilize three real-world datasets, each offering unique traffic flow dynamics. The dataset JiNan, which is first released by us, derived from actual traffic flow statistics in a city, mirrors the setup in \cite{song2020STSGCNaaai}, where the time interval is set to 5 minutes. 
% When we construct the experimental dataset, we establish edges between neighbor detection points if they are connected on the road. The other two datasets PeMS providing a geographical contrast to the JiNan dataset. 
Detailed descriptions of these datasets are provided in Table~\ref{tab.datasets}.

% \begin{table*}[t]

% % \vspace{-6mm}
% \centering

% % \vspace{-2mm}
% \footnotesize

% % \fontsize{9}{12} \selectfont
% \setlength{\tabcolsep}{1.2mm}{}

% \begin{tabular}{ccccccccccccc}
% \toprule
% \midrule
% \multirow{2}{*}{Model} & \multicolumn{3}{c}{PeMS04} & \multicolumn{3}{c}{JiNan} & \multicolumn{3}{c}{SD} & \multicolumn{3}{c}{GBA} \\
% \cline{2-13}
%  & MAE & RMSE & MAPE & MAE & RMSE & MAPE & MAE & RMSE & MAPE & MAE & RMSE & MAPE\\
% \midrule

% HA &38.03  &59.24  &27.88\%    \\
% ARIMA &33.73  &48.80  &24.18\%    \\
% VAR &24.54  &38.61  &17.24\%    \\
% SVR &28.70  &44.56  &19.20\%    \\
% DCRNN &21.22  &33.44  &14.17\%  &11.40 &20.54 &41.74\% &21.03 &33.37 &14.13\%  &23.13 &36.35 &20.84\%\\
% STGCN &21.16  &34.89  &13.83\%  &9.41 &16.08 &36.42\% &19.67 &34.14 &13.86\% &23.42 &38.57 &18.46\%\\
% GWNet &24.89  &39.66  &17.29\%  &9.39 &16.13 &35.83\% &\textbf{17.74} &29.62 &11.88\% &20.91 &\textbf{33.41} &17.66\%\\
% GMAN &19.14  &31.60  &13.19\%  &&&  \\
% ASTGCN(r) &22.93  &35.22  &16.56\%  &&& &23.70 &37.63 &15.65\% &26.47 &40.99 &23.65\%\\
% AGCRN &19.83  &32.26  &12.97\%  &9.52 &16.15 &40.27\% &18.09 &32.01 &13.28\% &21.01 &34.25 &16.90\%\\
% % ASTGNN &18.66  &31.13  &12.47\%  &0.7481  &0.7562  &0.7633  \\
% DMSTGCN   \\
% STGODE &20.84  &32.82  &13.77\%  &9.25 &15.75 &37.68\% &19.55 &33.57 &13.22\% &21.79 &35.37 &18.26\%\\
% STGNCDE &19.21  &31.09  &12.76\%  &&& \\
% D2STGNN &19.55  &31.99  &12.82\%  &&&  &17.85 &\textbf{29.51} &11.54\% &\textbf{20.71} &33.65 &15.04\% \\
% DSTAGNN &19.30 &31.46 &12.70\%  &&&  &21.82 &34.68 &14.40\% &23.82 &37.29 &20.16\% \\
% \midrule
% % PDFormer & 18.32  &29.97  & 12.10\%  &{0.8216}  &{0.8032}  &{0.8156}  \\ 
% SSTBAN &20.04  &31.89  &14.96\%  &8.41	&14.62	&33.79\%  &18.29	&31.72	&\textbf{11.43\%}\\ 
% STWave &21.20  &34.47  &14.32\%  & 9.23 &16.10  &36.12\% &17.78 &29.64 &11.77\% \\
% \midrule
% \textbf{\model(GRU)} &\textbf{18.40}	&\textbf{30.41}	&\textbf{12.21\%}  &\textbf{8.32}	&\textbf{14.52}	&\textbf{32.19\%}  &19.26	&33.48	&11.99\% &23.24	&39.18	&18.39\%\\
% \textbf{\model(Transformer)} &19.46	&31.40	&14.75\%  &8.67	&15.03	&37.25\% &18.38	&31.31	&12.11\% &22.31	&37.28	&\textbf{14.19\%} \\
% \midrule
% \bottomrule
% \end{tabular}
% \vspace{-1mm}
% \caption{Performance comparison of all models on four real-world datasets.}
% \label{tab:perform_compared}
% \vspace{-3mm}
% \end{table*}

\begin{table*}[t]

%\vspace{-6mm}
\centering

% \vspace{-2mm}
%\footnotesize

\fontsize{9.5}{10} \selectfont
\setlength{\tabcolsep}{3.5mm}{}

\begin{tabular}{cccccccccc}
\toprule
\midrule
\multirow{2}{*}{Model} & \multicolumn{3}{c}{PeMS04} & \multicolumn{3}{c}{PeMS07} & \multicolumn{3}{c}{JiNan}\\
%\cline{2-10}
% & MAE & RMSE & MAPE & MAE & RMSE & MAPE & MAE & RMSE & MAPE\\
\cmidrule(lr){2-4} \cmidrule(lr){5-7} \cmidrule(lr){8-10} % 使用 cmidrule 替代 cline
 & \raisebox{-0.1ex}{MAE} & \raisebox{-0.1ex}{RMSE} & \raisebox{-0.1ex}{MAPE} 
 & \raisebox{-0.1ex}{MAE} & \raisebox{-0.1ex}{RMSE} & \raisebox{-0.1ex}{MAPE} 
 & \raisebox{-0.1ex}{MAE} & \raisebox{-0.1ex}{RMSE} & \raisebox{-0.1ex}{MAPE} \\
\midrule

HA &38.03  &59.24  &27.88\%  &45.12 &65.64 &24.51\%  &13.23 &22.85 &46.83\% \\
ARIMA &33.73  &48.80  &24.18\%   &38.17 &59.27 &19.46\% &14.89 &26.22 &48.53\% \\
VAR &24.54  &38.61  &17.24\%    &50.22 &75.63 &32.22\% &12.56 &20.60 &40.54\% \\
SVR &28.70  &44.56  &19.20\%   &32.49 &50.22 &14.26\%  &11.98 &20.82 &39.56\%\\
LSTM &26.77  &40.65  &18.23\%  &29.98 &45.94 &13.20\% &11.30 &19.40 &38.02\%\\
DCRNN &21.02  &33.44  &14.17\%   &25.22 &38.61 & 11.82\% &11.40 &20.54 &41.74\%\\
STGCN &21.16  &34.89  &13.83\%   &25.33 &39.34 &11.21\% &9.41 &16.08 &36.42\%\\
GWNet &24.89  &39.66  &17.29\%   &26.39 &41.50 &11.97\% &9.39 &16.13 &35.83\%\\
GMAN &19.14  &31.60  &13.19\%   &20.97 &34.02 &9.05\%  &10.15 &16.98 &38.32\%\\
ASTGCN(r) &22.93  &35.22  &16.56\%   &24.01 &37.87 &10.73\% &10.06 &17.23 &38.95\%\\
AGCRN &19.83  &32.26  &12.97\%  &22.37 &36.55 &9.12\%  &9.52 &16.15 &40.27\%\\
% ASTGNN &18.66  &31.13  &12.47\%  &0.7481  &0.7562  &0.7633  \\
DMSTGCN  &20.01 &32.18 &14.50\%  &23.73 &36.01 &12.21\% &8.78 &\underline{14.61} &34.52\%\\
STGODE &20.84  &32.82  &13.77\%   &22.59 &37.54 &10.14\% &9.25 &15.75 &37.68\%\\
STGNCDE &19.21  &\underline{31.09}  &12.76\%   &20.53 &\underline{33.84} &\underline{8.80\%} &9.14 &16.60 &36.89\%\\
D\textsuperscript{2}STGNN &19.55  &31.99  &12.82\%    &21.55 &34.83 &9.39\% &9.12 &15.97 &35.65\%\\
DSTAGNN &19.30 &31.46 &12.70\% &21.42 &34.51 &9.01\% &8.95 &14.99 &36.87\% \\
\midrule
% PDFormer & 18.32  &29.97  & 12.10\%  &{0.8216}  &{0.8032}  &{0.8156}  \\ 
SSTBAN &\underline{18.89}  &31.21  &\underline{12.62\%}    &20.45 &37.55 &9.31\% &\underline{8.41}	&14.62	&\underline{33.79\%}\\ 
STWave &21.20  &34.47  &14.32\%   &\underline{20.33} &34.03 &\textbf{8.58\%} &9.23 &16.10  &36.12\%\\
\midrule
\textbf{\model} &\textbf{18.40$^*$}	&\textbf{30.41$^*$}	&\textbf{12.21\%$^*$}    &\textbf{20.08$^*$}	&\textbf{33.73$^*$}	&9.29\% &\textbf{8.32$^*$}	&\textbf{14.52$^*$}	&\textbf{32.19\%$^*$}\\
% \textbf{\model(Transformer)} &19.46	&31.40	&14.75\%  &8.67	&15.03	&37.25\% &18.38	&31.31	&12.11\%  \\
\midrule
\bottomrule
\end{tabular}
\vspace{-1mm}
\caption{Performance comparison of all models on three real-world datasets. Marker $*$ indicates the
results are statistically significant (t-test with p-value $<$ 0.01).}
\label{tab:perform_compared}
\vspace{-6mm}
\end{table*}

\subsection{Baseline Methods}

To assess the performance of \model, we compare it against a diverse range of established baselines:

\begin{itemize}
    \item \textbf{HA} \cite{hamilton2020HA} uses the historical average 
    of input data for prediction.
    \item \textbf{ARIMA} \cite{box2015ARIMA} is a well-known statistical model widely employed for time series forecasting.
    \item \textbf{VAR} \cite{lutkepohl2005VAR}: is another traditional method for time series forecasting.
    \item \textbf{SVR} \cite{wu2004SVR} employs support vector regression for predictive modeling.
    \item \textbf{LSTM} \cite{graves2012LSTM} is a deep learning model that captures temporal dependencies but does not account for spatial correlations.
    \item \textbf{DCRNN} \cite{li2018DCRNN} integrates diffusion convolution into GRU layers to enhance spatial-temporal correlation capture.
    \item \textbf{STGCN} \cite{yu2017STGCN} combines graph and temporal convolutions to handle spatial-temporal data.
    \item \textbf{GWNet} \cite{wu2019GWNet} combines dilated convolution with diffusion graph convolution and introduces a self-adaptive adjacency matrix.
    \item \textbf{GMAN} \cite{zheng2020GMANaaai} employs multi-attention to capture both spatial and temporal dynamics.
    \item \textbf{ASTGCN} \cite{guo2019ASTGCN} applies attention mechanisms on both temporal and spatial convolutions to dynamically capture spatio-temporal correlations.
    \item \textbf{AGCRN} \cite{bai2020AGCRN} focuses on extracting node-specific features and uncovering hidden node interdependencies.
    % \item \textbf{ASTGNN} \cite{guo2021lASTGNN}: designs trend-aware self-attention module and a dynamic graph convolution module to capture spatialtemporal dynamics.
    \item \textbf{DMSTGCN} \cite{han2021DMSTGCN}  learns dynamic spatial dependencies and builds a multi-faceted fusion module for complex traffic data features.
    \item \textbf{STGODE} \cite{fang2021STGODE} adopts ordinary differential equations for traffic flow forecasting.
    \item \textbf{STGNCDE} \cite{choi2022STGNCDE} designs two neural controlled differential equations for prediction.
    \item \textbf{D\textsuperscript{2}}\textbf{STGNN} \cite{D2STGNN} models traffic flow by separating it into the diffusion component and the inherent component.
    \item \textbf{DSTAGNN} \cite{lan2022dstagnn} constructs a spatio-temporal graph and utilizes multi-head attention to represent dynamic spatial relevance.
    \item \textbf{SSTBAN} \cite{guo2023SSTBAN} implements a self-supervised learning approach with a masking method for prediction.
    \item \textbf{STWave} \cite{fang2023STWave} employs wavelets to decompose traffic data into stable trends and fluctuating events.
\end{itemize}

\subsection{Evaluation Metrics and Experimental Settings}
In our evaluation,  we employ the mean absolute error (MAE), root mean square error (RMSE) and mean absolute percentage error (MAPE) to quantify the performance of different methods. 

Our experiments are conducted on a server with NVIDIA RTX 4090 GPU cards, running CUDA version 12.2. All the models are implemented using PyTorch.
The datasets are split in a 6:2:2 ratio for training, validation, and testing, respectively. We use historical data from the past hour to predict the traffic flow for the next hour, corresponding to using the past 12 time steps to forecast the next 12 steps. 


To prevent overfitting, an early-stopping strategy is employed with a patience setting of 10. We use Adam optimizer with an initial learning rate of 0.001. The standard batch size for all experiments is set to 64. If GPU memory constraints occur, the batch size is reduced to 32, and further to 16 if necessary, until the programs can run efficiently. The number of dimensions of node attribute on three datasets is $C$ = 1. Totally, there are 3 hyperparameters in our model, \ie, the numbers of bottleneck transformer block $L$, the number of attention heads $h$, and the dimensionality $d$ of each attention head, where the total number of features $D = h \times d$. The optimal settings for our model  on PeMS04 and JiNan datasets are $L = 2$, $h = 8$, $d = 16$ ($D = 128$). For the PeMS07 dataset, the best performance is achieved with $L = 2$, $h = 8$, $d = 12$ ($D = 96$). All source code and data are available at \url{https://github.com/roarer008/STDN}

%6:2:2 超参设置,比如d,k,l,adam,batch_size,GPU,learning rate
\subsection{Performance Comparison}
Table~\ref{tab:perform_compared} presents the results from graph-based baselines and grid-based baselines. The best results are highlighted in bold, and the second-best results are underlined. Based on these results, several key conclusions can be drawn:
\begin{itemize}
    \item \model achieves state-of-the-art performance, particularly evident in the PeMS04 and JiNan datasets. Traditional machine learning methods such as ARIMA typically perform poorly, as they are unable to capture the non-linear correlations present in the spatio-temporal traffic data.%With the exception of the MAPE metric, \model  surpasses other baselines on the PeMS07 dataset. %We posit that GRU is more beneficial for less-nodes datasets. 
    
    %\item Traditional machine learning methods such as ARIMA typically perform poorly, as they are unable to capture the non-linear correlations present in the spatio-temporal traffic data.
    
    \item  Among the GCN-based models, AGCRN demonstrates strong performance. Compared to other models, \model excels in capturing the structure of the road network by effectively integrating eigenvalues from the Laplacian matrix with traffic flow data. 
    %Besides that, attention-based models generally perform near optimally among all baselines. Notably, SSTBAN demonstrates strong performance.
    \item Attention-based models generally perform near optimally among all baselines. Notably, \model distinguishes itself by integrating spatio-temporal embeddings with traffic flow data and the decoder, significantly enhancing traffic forecasting accuracy.
    \item Compared to the baseline models, we incorporate multi-resolution temporal features, such as "time of day" and "day of week", for temporal embeddings, alongside a geospatial directed graph for spatial embeddings. Based on these spatiotemporal embeddings, traffic flow is disentangled into trend and seasonality parts. This novel disentangling method significantly enhances the traffic forecasting accuracy.
    
\end{itemize}
% (1) \model achieves state-of-the-art performances and the advantages are more evident in the PeMS04 and JiNan datasets. We argue that GRU is more beneficial for less-nodes datasets. (2) Among the GCN-based models, AGCRN performs well. Compared to other GCN-based models, \model effectively captures the road network's structure by efficiently integrating eigenvalues from the Laplacian matrix with traffic series data. (3) Among the attention-based models, GMAN and DSTAGNN are the best baselines. Compared to other attention-based models, \model excels by incorporating both spatial and temporal information in the encoder’s objective for prediction.

\subsection{Ablation Study}

% \begin{table}[!t]
% \centering
% \scalebox{0.68}{
%     \begin{tabular}{ll cccc}
%       \toprule
%       & \multicolumn{4}{c}{\textbf{Intellipro Dataset}}\\
%       & \multicolumn{2}{c}{Rank Resume} & \multicolumn{2}{c}{Rank Job} \\
%       \cmidrule(lr){2-3} \cmidrule(lr){4-5} 
%       \textbf{Method}
%       &  Recall@100 & nDCG@100 & Recall@10 & nDCG@10 \\
%       \midrule
%       \confitold{}
%       & 71.28 &34.79 &76.50 &52.57 
%       \\
%       \cmidrule{2-5}
%       \confitsimple{}
%     & 82.53 &48.17
%        & 85.58 &64.91
     
%        \\
%        +\RunnerUpMiningShort{}
%     &85.43 &50.99 &91.38 &71.34 
%       \\
%       +\HyReShort
%         &- & -
%        &-&-\\
       
%       \bottomrule

%     \end{tabular}
%   }
% \caption{Ablation studies using Jina-v2-base as the encoder. ``\confitsimple{}'' refers using a simplified encoder architecture. \framework{} trains \confitsimple{} with \RunnerUpMiningShort{} and \HyReShort{}.}
% \label{tbl:ablation}
% \end{table}
\begin{table*}[!t]
\centering
\scalebox{0.75}{
    \begin{tabular}{l cccc cccc}
      \toprule
      & \multicolumn{4}{c}{\textbf{Recruiting Dataset}}
      & \multicolumn{4}{c}{\textbf{AliYun Dataset}}\\
      & \multicolumn{2}{c}{Rank Resume} & \multicolumn{2}{c}{Rank Job} 
      & \multicolumn{2}{c}{Rank Resume} & \multicolumn{2}{c}{Rank Job}\\
      \cmidrule(lr){2-3} \cmidrule(lr){4-5} 
      \cmidrule(lr){6-7} \cmidrule(lr){8-9} 
      \textbf{Method}
      & Recall@100 & nDCG@100 & Recall@10 & nDCG@10
      & Recall@100 & nDCG@100 & Recall@10 & nDCG@10\\
      \midrule
      \confitold{}
      & 71.28 & 34.79 & 76.50 & 52.57 
      & 87.81 & 65.06 & 72.39 & 56.12
      \\
      \cmidrule{2-9}
      \confitsimple{}
      & 82.53 & 48.17 & 85.58 & 64.91
      & 94.90&78.40 & 78.70& 65.45
       \\
      +\HyReShort{}
       &85.28 & 49.50
       &90.25 & 70.22
       & 96.62&81.99 & \textbf{81.16}& 67.63
       \\
      +\RunnerUpMiningShort{}
       % & 85.14& 49.82
       % &90.75&72.51
       & \textbf{86.13}&\textbf{51.90} & \textbf{94.25}&\textbf{73.32}
       & \textbf{97.07}&\textbf{83.11} & 80.49& \textbf{68.02}
       \\
   %     +\RunnerUpMiningShort{}
   %    & 85.43 & 50.99 & 91.38 & 71.34 
   %    & 96.24 & 82.95 & 80.12 & 66.96
   %    \\
   %    +\HyReShort{} old
   %     &85.28 & 49.50
   %     &90.25 & 70.22
   %     & 96.62&81.99 & 81.16& 67.63
   %     \\
   % +\HyReShort{} 
   %     % & 85.14& 49.82
   %     % &90.75&72.51
   %     & 86.83&51.77 &92.00 &72.04
   %     & 97.07&83.11 & 80.49& 68.02
   %     \\
      \bottomrule

    \end{tabular}
  }
\caption{\framework{} ablation studies. ``\confitsimple{}'' refers using a simplified encoder architecture. \framework{} trains \confitsimple{} with \RunnerUpMiningShort{} and \HyReShort{}. We use Jina-v2-base as the encoder due to its better performance.
}
\label{tbl:ablation}
\end{table*}
To evaluate the effectiveness of different components in \model, we conducted the ablation study with several variants of the \model:
\begin{itemize}
    \item \textbf{w/o TE}: This variant removes the temporal embedding modeling, meaning the decoder operates solely with spatial embedding cues.
    \item \textbf{w/o SE}: This variant removes the spatial embedding modeling, meaning the decoder operates solely with temporal embedding cues.
    \item \textbf{w/o STE}: This variant eliminates spatio-temporal embedding, thus the traffic flow is not decomposed into trend-seasonality components. As a result, the decoder does not incorporate any spatio-temporal cues.
    \item \textbf{w/o DRG}: This variant eliminates the dynamic relationship graph learning.
    \item \textbf{w/o STD}: Instead of using spatiotemporal-aware decomposition to disentangle the traffic sequence data, this variant adopts the decomposition method utilized by Autoformer \cite{wu2021Autoformer}.
\end{itemize}
% (1) \textit{w/o TE}: this variant removes the temporal embedding modeling, meaning the decoder operates solely with spatial embedding cues. (2) \textit{w/o SE}: this variant removes the spatial embedding modeling, meaning the decoder operates solely with temporal embedding cues. (3) \textit{w/o STE}: this variant eliminates spatio-temporal embedding, thus the traffic flow is not decomposed into trend-seasonality components. As a result, the decoder does not incorporate any spatio-temporal cues. (4) \textit{w/o DRG}: this variant removes the dynamic relationship graph learning. (5) \textit{w/o STD}: instead of using spatiotemporal-aware decomposition to disentangle the traffic sequence data, this variant adopts the decomposition method utilized by Autoformer \cite{wu2021Autoformer}.

Table~\ref{tab.ablation} presents the comparison of \model and its variants on PeMS04 and JiNan datasets. From this comparison, we can draw several conclusions: (1) The origin \model consistently achieves the best performance relative to its variants, underscoring the effectiveness of its full configuration. (2) The results show that the variant ``\textit{w/o TE}'' generally outperforms ``\textit{w/o SE}'' across most tasks. This suggests that temporal information, particularly periodic information, plays a more critical role than spatial information. (3) The ``\textit{w/o DRG}'' underperforms \model, indicating the importance of the dynamic relationship graph learning module. (4) The performance of ``\textit{w/o STD}'' underlines the necessity of the trend-seasonality decomposition module aware of spatio-temporal embeddings.
% % \begin{table}[!t]
% \centering
% \scalebox{0.68}{
%     \begin{tabular}{ll cccc}
%       \toprule
%       & \multicolumn{4}{c}{\textbf{Intellipro Dataset}}\\
%       & \multicolumn{2}{c}{Rank Resume} & \multicolumn{2}{c}{Rank Job} \\
%       \cmidrule(lr){2-3} \cmidrule(lr){4-5} 
%       \textbf{Method}
%       &  Recall@100 & nDCG@100 & Recall@10 & nDCG@10 \\
%       \midrule
%       \confitold{}
%       & 71.28 &34.79 &76.50 &52.57 
%       \\
%       \cmidrule{2-5}
%       \confitsimple{}
%     & 82.53 &48.17
%        & 85.58 &64.91
     
%        \\
%        +\RunnerUpMiningShort{}
%     &85.43 &50.99 &91.38 &71.34 
%       \\
%       +\HyReShort
%         &- & -
%        &-&-\\
       
%       \bottomrule

%     \end{tabular}
%   }
% \caption{Ablation studies using Jina-v2-base as the encoder. ``\confitsimple{}'' refers using a simplified encoder architecture. \framework{} trains \confitsimple{} with \RunnerUpMiningShort{} and \HyReShort{}.}
% \label{tbl:ablation}
% \end{table}
\begin{table*}[!t]
\centering
\scalebox{0.75}{
    \begin{tabular}{l cccc cccc}
      \toprule
      & \multicolumn{4}{c}{\textbf{Recruiting Dataset}}
      & \multicolumn{4}{c}{\textbf{AliYun Dataset}}\\
      & \multicolumn{2}{c}{Rank Resume} & \multicolumn{2}{c}{Rank Job} 
      & \multicolumn{2}{c}{Rank Resume} & \multicolumn{2}{c}{Rank Job}\\
      \cmidrule(lr){2-3} \cmidrule(lr){4-5} 
      \cmidrule(lr){6-7} \cmidrule(lr){8-9} 
      \textbf{Method}
      & Recall@100 & nDCG@100 & Recall@10 & nDCG@10
      & Recall@100 & nDCG@100 & Recall@10 & nDCG@10\\
      \midrule
      \confitold{}
      & 71.28 & 34.79 & 76.50 & 52.57 
      & 87.81 & 65.06 & 72.39 & 56.12
      \\
      \cmidrule{2-9}
      \confitsimple{}
      & 82.53 & 48.17 & 85.58 & 64.91
      & 94.90&78.40 & 78.70& 65.45
       \\
      +\HyReShort{}
       &85.28 & 49.50
       &90.25 & 70.22
       & 96.62&81.99 & \textbf{81.16}& 67.63
       \\
      +\RunnerUpMiningShort{}
       % & 85.14& 49.82
       % &90.75&72.51
       & \textbf{86.13}&\textbf{51.90} & \textbf{94.25}&\textbf{73.32}
       & \textbf{97.07}&\textbf{83.11} & 80.49& \textbf{68.02}
       \\
   %     +\RunnerUpMiningShort{}
   %    & 85.43 & 50.99 & 91.38 & 71.34 
   %    & 96.24 & 82.95 & 80.12 & 66.96
   %    \\
   %    +\HyReShort{} old
   %     &85.28 & 49.50
   %     &90.25 & 70.22
   %     & 96.62&81.99 & 81.16& 67.63
   %     \\
   % +\HyReShort{} 
   %     % & 85.14& 49.82
   %     % &90.75&72.51
   %     & 86.83&51.77 &92.00 &72.04
   %     & 97.07&83.11 & 80.49& 68.02
   %     \\
      \bottomrule

    \end{tabular}
  }
\caption{\framework{} ablation studies. ``\confitsimple{}'' refers using a simplified encoder architecture. \framework{} trains \confitsimple{} with \RunnerUpMiningShort{} and \HyReShort{}. We use Jina-v2-base as the encoder due to its better performance.
}
\label{tbl:ablation}
\end{table*}
% Table~\ref{tab.ablation} shows the comparison of these variants on the PeMS04 and JiNan datasets. Based on the results, we can draw the following conclusions: (1) The results show the superiority of SSA over GCN in capturing dynamic and long-range spatial dependencies. (2) PDFormer leads to a large performance improvement over w/o Mask, highlighting the value of using the mask matrices to identify the significant node pairs. In addition, w/o SemSAH and w/o GeoSAH perform worse than PDFormer, indicating that both local and global spatial dependencies are significant for traffic prediction. (3) w/o Delay performs worse than PDFormer because this variant ignores the spatial propagation delay between nodes but considers the spatial message passing as immediate.
\vspace{-2mm}
\subsection{Parameter Sensitivity Study}
% \begin{table}[!t]
% \centering
% \scalebox{0.68}{
%     \begin{tabular}{ll cccc}
%       \toprule
%       & \multicolumn{4}{c}{\textbf{Intellipro Dataset}}\\
%       & \multicolumn{2}{c}{Rank Resume} & \multicolumn{2}{c}{Rank Job} \\
%       \cmidrule(lr){2-3} \cmidrule(lr){4-5} 
%       \textbf{Method}
%       &  Recall@100 & nDCG@100 & Recall@10 & nDCG@10 \\
%       \midrule
%       \confitold{}
%       & 71.28 &34.79 &76.50 &52.57 
%       \\
%       \cmidrule{2-5}
%       \confitsimple{}
%     & 82.53 &48.17
%        & 85.58 &64.91
     
%        \\
%        +\RunnerUpMiningShort{}
%     &85.43 &50.99 &91.38 &71.34 
%       \\
%       +\HyReShort
%         &- & -
%        &-&-\\
       
%       \bottomrule

%     \end{tabular}
%   }
% \caption{Ablation studies using Jina-v2-base as the encoder. ``\confitsimple{}'' refers using a simplified encoder architecture. \framework{} trains \confitsimple{} with \RunnerUpMiningShort{} and \HyReShort{}.}
% \label{tbl:ablation}
% \end{table}
\begin{table*}[!t]
\centering
\scalebox{0.75}{
    \begin{tabular}{l cccc cccc}
      \toprule
      & \multicolumn{4}{c}{\textbf{Recruiting Dataset}}
      & \multicolumn{4}{c}{\textbf{AliYun Dataset}}\\
      & \multicolumn{2}{c}{Rank Resume} & \multicolumn{2}{c}{Rank Job} 
      & \multicolumn{2}{c}{Rank Resume} & \multicolumn{2}{c}{Rank Job}\\
      \cmidrule(lr){2-3} \cmidrule(lr){4-5} 
      \cmidrule(lr){6-7} \cmidrule(lr){8-9} 
      \textbf{Method}
      & Recall@100 & nDCG@100 & Recall@10 & nDCG@10
      & Recall@100 & nDCG@100 & Recall@10 & nDCG@10\\
      \midrule
      \confitold{}
      & 71.28 & 34.79 & 76.50 & 52.57 
      & 87.81 & 65.06 & 72.39 & 56.12
      \\
      \cmidrule{2-9}
      \confitsimple{}
      & 82.53 & 48.17 & 85.58 & 64.91
      & 94.90&78.40 & 78.70& 65.45
       \\
      +\HyReShort{}
       &85.28 & 49.50
       &90.25 & 70.22
       & 96.62&81.99 & \textbf{81.16}& 67.63
       \\
      +\RunnerUpMiningShort{}
       % & 85.14& 49.82
       % &90.75&72.51
       & \textbf{86.13}&\textbf{51.90} & \textbf{94.25}&\textbf{73.32}
       & \textbf{97.07}&\textbf{83.11} & 80.49& \textbf{68.02}
       \\
   %     +\RunnerUpMiningShort{}
   %    & 85.43 & 50.99 & 91.38 & 71.34 
   %    & 96.24 & 82.95 & 80.12 & 66.96
   %    \\
   %    +\HyReShort{} old
   %     &85.28 & 49.50
   %     &90.25 & 70.22
   %     & 96.62&81.99 & 81.16& 67.63
   %     \\
   % +\HyReShort{} 
   %     % & 85.14& 49.82
   %     % &90.75&72.51
   %     & 86.83&51.77 &92.00 &72.04
   %     & 97.07&83.11 & 80.49& 68.02
   %     \\
      \bottomrule

    \end{tabular}
  }
\caption{\framework{} ablation studies. ``\confitsimple{}'' refers using a simplified encoder architecture. \framework{} trains \confitsimple{} with \RunnerUpMiningShort{} and \HyReShort{}. We use Jina-v2-base as the encoder due to its better performance.
}
\label{tbl:ablation}
\end{table*}
Figure \ref{fig:abla} illustrates the results of hyper-parameter sensitivity analysis for our \model on PeMS04 and PeMS07 datasets. This study involved varying the number of decoder layers and the number of features in \model, exploring options within the ranges of [1, 2, 3, 4] for layers and [32, 64, 96, 128, 160] for features. From this analysis, we can draw several conclusions: (1) The performance of our model improves with an increasing in the number of decoder layers but stabilizes at 2 layers. (2) Optimal performance is achieved with 128 features on the PeMS04 dataset and 96 features on the PeMS07 dataset. This finding highlights that while increasing the number of features generally enhances the model's capability to represent complex traffic patterns, but excessive features may introduce noise and degrade model performance.


\subsection{Model Efficiency Study}
\begin{figure}
    %\vspace{-5mm}
    \begin{center}
    \includegraphics[width=0.46\textwidth]{figure/time.pdf}
    %\vspace{-4mm}
    \caption{The computational time cost on PeMS04 and JiNan datasets.}
    \vspace{-3mm}
    \label{fig:time}
    \end{center}
\end{figure}
% \begin{table}[t!]
\centering
    \scriptsize
    \setlength{\tabcolsep}{0.0035\linewidth}
    \caption{\textbf{Computational efficiency of EvSSC across different datasets.} Memory denotes training memory usage.}
    %\vskip-1ex
\setlength{\tabcolsep}{4pt} %5pt 设定列之间的宽度
\resizebox{\columnwidth}{!}{%
\begin{tabular}{l|>{\columncolor{gray!10}}l>{\columncolor{blue!8}}l|>{\columncolor{gray!10}}l>{\columncolor{blue!8}}l}
\toprule
\textbf{ } & \textbf{VoxFormer-S} & \textbf{EvSSC (VoxFormer)} & \textbf{SGN-S} & \textbf{EvSSC (SGN)}\\
\midrule\midrule
\multicolumn{5}{c}{\textit{DSEC-SSC}} \\ \midrule 
\textbf{mIoU} &25.62 & 26.34 & 29.06 & 29.55\\ 
\textbf{IoU}  & 47.25 & 47.29 & 43.70 & 43.99\\ 
\textbf{Memory}  & 9.74G & 10.52G & 10.19G & 10.70G\\ 
\textbf{Latency} & 0.732s & 0.836s & 0.941s & 1.193s \\\midrule 
\multicolumn{5}{c}{\textit{SemanticKITTI-E}} \\ \midrule 
\textbf{mIoU} & 12.86 & 13.61 & 14.55 & 15.15\\ 
\textbf{IoU}  & 44.42 & 45.01 & 43.60 & 43.17\\ 
\textbf{Memory} & 14.87G & 15.78G & 15.29G & 17.79G \\ 
\textbf{Latency}  & 0.996s & 1.005s & 0.855s &1.005s\\ \midrule 
\multicolumn{5}{c}{\textit{SemanticKITTI-C Shot Noise}} \\ \midrule 
\textbf{mIoU} & 8.29 & 12.64 & 13.62 & 14.32\\ 
\textbf{IoU}  & 44.26 & 45.04 & 42.05 & 42.54\\ 
\textbf{Memory} & 14.87G & 15.78G & 15.29G & 17.79G \\ 
\textbf{Latency}  & 0.996s & 1.005s & 0.855s &1.005s\\
\bottomrule
\end{tabular}
}
\label{table:efficiency}
%\vskip-3ex
\end{table}


% \begin{figure}
    \centering
    \includegraphics[width=0.98\linewidth]{figure/Epoch_time_with_spe.png}
    \vspace{-3mm}
    \caption{The MAE on the validation part of PeMS04 dataset during the training process}
    \vspace{-6mm}
    \label{fig:epoch}
\end{figure}
\begin{figure}
    \centering
    \vspace{-1mm}
    \includegraphics[width=0.98\linewidth]{figure/Epoch_time_with_spe.pdf}
    \vspace{-0.5mm}
    \caption{The MAE on the validation part of PeMS04 dataset during the training process.}
    \vspace{-4mm}
    \label{fig:epoch}
\end{figure}
To demonstrate the efficiency of our model, we benchmark \model against DMSTGCN, SSTBAN, and STWave, which have achieved suboptimal results at the PeMS04 and JiNan datasets. Figure~\ref{fig:time} displays the average training time per epoch and inference time for each model. Figure~\ref{fig:epoch} illustrates the MAE curves on the validation part of PeMS04 dataset during the training process. The following observations can be made: (1) \model not only trains faster but also infers quicker than the compared models. (2) \model demonstrates a faster convergence rate, achieving better performance in fewer epochs. In our experiments, while STWave reaches its best performance at epoch 100, its MAE is still higher than the lowest MAE achieved by our \model at just epoch 19. 

The computational complexity of our \model encoder-decoder module is $O(TD + LND)$, with the encoder and decoder contributing complexities of $O(TD)$ and $O(LND)$, respectively. Here, $L$ denotes the number of bottleneck transformer blocks. The complexity of the spatio-temporal embedding module is given by $O((T + N)D + N^3)$. Although calculating the eigenvectors and eigenvalues of the graph Laplacian is computationally intensive, marked by a complexity of  $O(N^3)$, this process can be efficiently handled through preprocessing prior to training. Therefore, \model maintains comparable time and memory complexity during training, ensuring efficiency without compromising performance.
\section{Conclusion}\label{sec:conclusion}

%%% Summarize main points and give brief overview of results %%%
Model-based approaches for time series classification can be effectively utilized when a model of the underlying dynamical process is available~\cite{shen2017classification}. 
Using structural identifiability (SI) analysis, structurally identifiable parameter combinations of the dynamical model can be obtained. 
Individual time series observations may then be represented as point estimates in the original parameter space or in the space of structurally identifiable parameter combinations. 
We introduced a novel method \eco{dubbed} \textbf{S}tructural-\textbf{I}dentifiability \textbf{M}apping (\myMethod{}) and demonstrated that \myMethod{} improves classification performance for the classification of time series data when taking a model-based approach and the underlying dynamical model is structurally unidentifiable.

% improved performance on classification task
Furthermore, it has been shown on a set of relevant example systems that classification performance is significantly improved when learning with data represented in the space of structurally identifiable parameter combinations. 
The increase in performance also persists when time series data of varying quality \mjc{are} \eco{produced}: for all types of time grids (dense, sparse and irregular) as well as for \eco{varying levels of} the observation\eco{al} noise introduced, learning in the space of structurally identifiable parameter combinations outperforms learning in the space of the original model parameters.

This work presents a first success in incorporating SI analysis directly into the learning process for classification. 
The \myMethod{} approach is straightforward and can be applied whenever a SI analysis can be carried out. 
An explicit reparametrisation of a given dynamical model in terms of fewer, structurally identifiable parameters is not needed in order to benefit from SI analysis. 
\pt{This is especially important in situations where explicit expressions for structurally identifiable parameter combinations are available following a SI analysis, but suitable model reparametrizations are not possible.}

Finally, outcomes of the learning process stay interpretable: while interpretation in the space of structurally identifiable parameter combinations is not straightforward, any insight in this space may be translated back to the space of the original model parameters $g^{-1}(\boldsymbol{\Phi})$, which, in turn, are meaningful in the domain-specific context.

% 
% 
% \section*{Software and Data}

% Source code neural sparse taken from, \href{https://github.com/flyingdoog/PTDNet/tree/main/NeuralSparse/NeuralSparseGraphSAGE}{NeuralSparse}.
% Mixture of Graphs, \href{https://github.com/yanweiyue/MoG}{MOG}, %GraphSAINT \href{}{Pytorch geometric}, DropEDGE \href{}{link needed}, SparseGAT \href{}{Link Needed}.
% 
% \clearpage
\section*{Impact Statement}
% This paper presents work whose goal is to advance the field of 
% Machine Learning. There are many potential societal consequences 
% of our work, none which we feel must be specifically highlighted here.

This paper presents a scalable supervised graph sparsification method that aims to support large-scale graph machine learning for Graph Neural Networks (GNNs).
There are many potential societal consequences of our work. Positive consequences include making AI more accessible to smaller organizations and researchers with limited resources, as well as reducing the carbon footprint of AI models. 
%{\color{blue}There are also few potential negative consequences such as bias, fairness, security, and interoperability. }
% There are other potential societal consequences 
% of our work, none of which we feel must be specifically highlighted here.
\section*{Acknowledgement}
This work was supported in part by the U.S. Department of Energy, Office of Science, Office of Advanced Scientific Computing Research (ASCR) grant SC-0022260 and Computer Science Competitive Portfolios program at Pacific Northwest National Laboratory (PNNL); and by the Laboratory Directed Research and Development Program at PNNL. PNNL is a multi-program national laboratory operated for the U.S. Department of Energy (DOE) by Battelle Memorial Institute under Contract No. DE-AC05-76RL01830.
%This work was supported in part by the U.S. Department of Energy, Office of Science, Office of Advanced Scientific Computing Research's Computer Science Competitive Portfolios program at Pacific Northwest National Laboratory (PNNL).  PNNL is a multi-program national laboratory operated for the U.S. Department of Energy (DOE) by Battelle Memorial Institute under Contract No. DE-AC05-76RL01830.
%
%Siddhartha Shankar Das and Alex Pothen were supported by the Advanced Scientific Computing Research program of the U.S. Department of Energy  through grant SC-0022260. 
%
%S M Ferdous was supported by the Laboratory Directed Research and Development Program at PNNL. 
% \begingroup
% \setlength{\itemsep}{0pt}
\bibliography{GNNbib1}
\bibliographystyle{icml2025}
% \endgroup

%%%%%%%%%%%%%%%%%%%%%%%%%%%%%%%%%%%%%%%%%%%%%%%%%%%%%%%%%%%%%%%%%%%%%%%%%%%%%%%
%%%%%%%%%%%%%%%%%%%%%%%%%%%%%%%%%%%%%%%%%%%%%%%%%%%%%%%%%%%%%%%%%%%%%%%%%%%%%%%
% APPENDIX
%%%%%%%%%%%%%%%%%%%%%%%%%%%%%%%%%%%%%%%%%%%%%%%%%%%%%%%%%%%%%%%%%%%%%%%%%%%%%%%
%%%%%%%%%%%%%%%%%%%%%%%%%%%%%%%%%%%%%%%%%%%%%%%%%%%%%%%%%%%%%%%%%%%%%%%%%%%%%%%
% \newpage
\appendix
\onecolumn
\section{Theoretical Analysis}
\subsection{Notations}
We dedicate Table~\ref{tab:Notation} to index the notations used in this paper. Note that every notation is also defined when it is introduced.
\begin{table*}[h!]
\caption{Notations.}\label{tab:Notation}%\\
\centering  
% \resizebox{\textwidth}{!}{
\begin{tabular}{l l l}
\toprule
 $\gG$ &$\triangleq$ & Input graph with a vertex set $\gV$, an edge set $\gE$, and features $\mX$\\
 $\boldsymbol{A}$ &$\triangleq$ & Adjacency matrix of $\gG$\\
 $\gE$ & $\triangleq$ & Edges of $\gG$\\
 $\gV$ & $\triangleq$ & Nodes of $\gG$\\
 $\mX$ & $\triangleq$ & Matrix containing node features of $\gG$\\
 $\vy$ & $\triangleq$ & Vector of node labels of $\gG$\\
 $C$ & $\triangleq$ & An ordered set containing all possible node labels of $\gG$\\
 $F$ & $\triangleq$ & Dimension of node features in $\gG$\\
 $L$ & $\triangleq$ & Number of GNN layers\\
 $H$ & $\triangleq$ & Node embedding dimension\\ 
 ${\mH}$ & $\triangleq$ & Node embedding matrix\\
 $\vh_u$ & $\triangleq$ & Embedding of node u\\
 $\vw$ & $\triangleq$ & Vector of edge weights in  $\gG$\\
 $q$ & $\triangleq$ & Ratio of \# edges in sparse graph and \# edges in input graph in \%\\
 $k$ & $\triangleq$ & \# edges in the sparse graph, $k\triangleq\floor{\frac{q|\gE|}{100}}$\\ 
 $\tilde{p}$ & $\triangleq$ & Learned probability distribution by \sgs \\
 $\tilde{\gE}$ & $\triangleq$ & Set of edges sampled from $\gE$ by \sgs following $\tilde{p}$\\
$\tilde{\gG}$ &$\triangleq$ & Sparse subgraph $(\gV,\tilde{\gE},\mX)$ constructed by \sgs \\  
 $\mA_{\tilde{\gG}}$ or $\tmA$ & $\triangleq$ & Adjacency matrix of $\tilde{\gG}$\\
 $\tilde{\vw}$ & $\triangleq$ & Edge weight of sparse graph learned by \sgs \\
 $p_\mathrm{prior}$ & $\triangleq$ & Probability distribution of a fixed prior on $\gG$ \\
  $\tilde{p}_a$ & $\triangleq$ & Augmented learned probability distribution  \\
 $p^*$ & $\triangleq$ & True probability distribution known by the idealized learning ORACLE\\
 ${\gE^*}$ & $\triangleq$ & Set of edges sampled from $\gE$ by the learning ORACLE following distribution $p^*$\\   
 $\gG^*$ &$\triangleq$ & True sparse subgraph $(\gV,\gE^*,\mX)$ constructed by the learning ORACLE \\
 $\mA_{\gG^*}$ or $\mA^*$ & $\triangleq$ & Adjacency matrix of $\gG^*$\\
 % $\gG^*$ &$\triangleq$ & True sparse subgraph constructed by the idealized learning ORACLE \\
 % $\mA_{\gG^*}$ or $\mA^*$ & $\triangleq$ & Adjacency matrix of $\gG^*$\\
 $\gL_\mathrm{CE}$ & $\triangleq$ & Cross entropy loss\\
 $\gL_\mathrm{assor}$ & $\triangleq$ & Assortative loss\\
 $\gL_\mathrm{cons}$ & $\triangleq$ & Consistency loss\\
$\gL$ & $\triangleq$ & Total loss\\ 
 
 
 
 
 
 
 \bottomrule
\end{tabular}
% }
\end{table*}
\subsection{Bounding \#common edges wrt. true subgraph}
\label{theo:commonedges}
Let $\mathcal{E}^*$ and $\mathcal{\tilde{E}}$ denote the ordered collection of edges sampled by the idealized learning ORACLE according to true distribution $p^*$ and by \sgs according to learned probability $\tilde{p}$ respectively. For analytical convenience, let us assume that both learning algorithms sample $k = \floor{q|\mathcal{E}|/100}$ edges with replacement independently.
 
% We will first show that $\mathbf{Pr}(\mathcal{E}_i^* = \mathcal{\tilde{E}}_i) \geq \min_j (p^*_i, \tilde{p}_i)$ and $\min(p^*_i,\tilde{p}_i) = 1- \frac{1}{2} \|\tilde{p} - p^*\|_1$. These two results will lead us to prove that $\mathbf{Pr}(\mathcal{E}_i^* = \mathcal{\tilde{E}}_i) \geq 1 - \frac{\epsilon}{2}$. 
First, we will prove lemma~\ref{lem:singleedge}, which show that the probability of an edge chosen by \sgs coincides with that chosen by the ORACLE has a lower bound. Finally, we will prove one of the main results (Theorem~\ref{theo:commonedges}), which shows that given $q \in [0,100]$, we can lower-bound the expected number of common edges between \sgs and the learning ORACLE. 

\begin{lemma} 
\label{lem:singleedge}
For any arbitrarily chosen $i \in \{1,2,\ldots, k\}$
\[
\mathbf{Pr}(\mathcal{E}_i^* = \mathcal{\tilde{E}}_i) \geq \sum_{j=1}^{|\mathcal{E}|} \frac{(p^*_j + \tilde{p}_j - \epsilon)^2}{4},
\]
where $k = \floor{q|\mathcal{E}|/100}$ and $0 \leq q \leq 100$ is a user-specified parameter and $\epsilon\in [0,1]$ is the error.
\end{lemma}

\begin{proof} We prove the above lemma in two parts.


\paragraph{Part 1: Universal approximation of probability distribution over edges.}
%\label{tho:uap}
The Universal Approximation Theorem~\cite{cybenko1989approximation,augustine2024survey} states that a feed-forward neural network with at least one hidden layer and a finite number of neurons can approximate any continuous function $f: \mathbb{R}^n \rightarrow \mathbb{R}$ on a compact subset of $\mathbb{R}^n$, given a suitable choice of weights and activation functions. 

In our case, $p^* = f$ is the true edge probability distribution for the downstream task, $\tilde{p} = f_{\text{MLP},\phi}$ is the learned approximate distribution and $\vx_e$ is a vector of edge features, for instance, $\vx_e =  ((\vh_u - \vh_v) \oplus (\vh_u \odot \vh_v))$ as used in equation~\ref{eq:w_uv}. The following universal approximation property holds for the module I component of \sgs,
\begin{equation}
\label{eq:uapp}
\sup_{e \in \mathcal{E}} \|\tilde{p}(\vx_e) - p^*(\vx_e)\|_1 \leq \epsilon.
\end{equation}
 Here, we have two underlying assumptions: (i) the optimal distribution $p^*$ is a function of node features $\mX$ and (ii) $\mX$ is a compact subset (bounded and closed) of Euclidean space $\mathbb{R}^n$. The first assumption is made to simplify the problem. The second assumption is quite practical since the node features are typically normalized. Hence, we can show that the embeddings $\vh_u,\vh_v$, which are continuous images of $\mX$, are also compact due to the extreme value theorem. As a result, the edge features $\vx_e$ which, in a sense, \emph{lifts} the end-point node features into higher-dimensional Euclidean space are also compact. The approximation error $\epsilon$ can be made arbitrarily small by increasing the capacity of the MLP, e.g., adding more neurons or layers. 

\paragraph{Part 2: Common edges wrt. optimal subgraph.}

The event $\mathcal{E}_i^* = \mathcal{\tilde{E}}_i$ means that both $\mathcal{E}_i^*$ and $\mathcal{\tilde{E}}_i$ contain the same edge. But there are $|\mathcal{E}|$ such candidates. Hence, the probability of this event is given by,

\begin{align*}
    \mathbf{Pr}(\mathcal{E}_i^* = \mathcal{\tilde{E}}_i) &= \sum_{j=1}^{|\mathcal{E}|} \mathbf{Pr}(\mathcal{E}_i^* = \mathcal{E}_j \land \mathcal{\tilde{E}}_i = \mathcal{E}_j), \\
    &= \sum_{j=1}^{|\mathcal{E}|} \mathbf{Pr}(\mathcal{E}_i^* = \mathcal{E}_j) \cdot \mathbf{Pr}(\mathcal{\tilde{E}}_i = \mathcal{E}_j), \\
    & = \sum_{j=1}^{|\mathcal{E}|} p^*_j \cdot \tilde{p}_j, \\
    &\geq \sum_{j=1}^{|\mathcal{E}|} \frac{(p^*_j + \tilde{p}_j - |p^*_j - \tilde{p}_j|)^2}{4}, \\
    & \geq \sum_{j=1}^{|\mathcal{E}|} \frac{(p^*_j + \tilde{p}_j - \epsilon)^2}{4}.
\end{align*}
The second line follows since the optimal sampler is a different algorithm independent from the sampler used in \sgs. The last line follows because $\|p^*_j - \tilde{p}_j\|_1 \leq \epsilon \implies |p^*_j - \tilde{p}_j| \leq \epsilon$ (from eq.~\ref{eq:uapp}). 
\end{proof}

% \begin{lemma}
% \[
%     \min(p^*_i,\tilde{p}_i) = 1- \frac{1}{2} \|\tilde{p} - p^*\|_1
% \]
% \end{lemma}
% \begin{proof}
%     It is known (for instance, see~\cite{xie2024distributionally}) that the total variation distance $d_{TV}(\tilde{p},p^*)$ satisfies
%     \[
%     d_{TV}(\tilde{p},p^*) = \frac{1}{2}\|\tilde{p} - p^*\|_1
%     = 1 - \min({\tilde{p},p^*})
%     \]
% \end{proof}

We have the following theorem that lower-bounds the number of common edges with respect to the optimal sampler $|\mathcal{E} ^* \cap \mathcal{\tilde{E}}|$: 
\begin{theorem}[Lower-bound]
\begin{equation}
\mathbb{E}[|\mathcal{E}^* \cap \mathcal{\tilde{E}}|] \geq k \sum_{j=1}^{|\mathcal{E}|} \frac{(p^*_j + \tilde{p}_j - \epsilon)^2}{4},
\end{equation}
where $k = \floor{q|\mathcal{E}|/100}$ and $0 \leq q \leq 100$ is a user-specified parameter.
\end{theorem}
\begin{proof}
    Since we are drawing $k$ edges independently at random, the theorem follows by applying the linearity of expectation on the following:
\begin{align*}
\mathbb{E}[|\mathcal{E}^* \cap \mathcal{\tilde{E}}|] = \mathbb{E}[\sum_{i=1}^k \mathbb{I}(\mathcal{E}_i^* = \mathcal{\tilde{E}}_i)] &= \sum_{i=1}^k \mathbf{Pr}(\mathcal{E}_i^* = \mathcal{\tilde{E}}_i) \\
& = k\cdot \mathbf{Pr}(\mathcal{E}_i^* = \mathcal{\tilde{E}}_i)\\
& \geq k \sum_{j=1}^{|\mathcal{E}|} \frac{(p^*_j + \tilde{p}_j - \epsilon)^2}{4}
\end{align*}
\end{proof}
This theorem shows that the expected number of common edges between the sample subgraph obtained by \sgs $\mathcal{\tilde{G}}$ and the true optimal sample subgraph $\mathcal{G}^*$ is non-trivial. 

% \paragraph{The implication of the lower-bound.} 
% (1) Suppose, the true distribution is uniform. In the best case scenario $\epsilon \rightarrow 0$ and $\tilde{p} = p^* = \frac{1}{|\mathcal{E}|}$. Thus there are at least $\frac{k}{|\mathcal{E}|}$ common edges between $\tilde{\gG}$ and $\gG^*$. However, since $k < \abs{\mathcal{E}}$, the lower-bound of $\mathbb{E}[|\mathcal{E}^* \cap \mathcal{\tilde{E}}|] \geq \frac{k}{|\mathcal{E}|}$ is not very useful even though the learned distribution is accurate. This suggests that \emph{learning the optimum uniform distribution is less likely to produce the optimum sparse subgraph}. 

% (2) Suppose, the true distribution is the Dirac distribution (often called the $\delta$ distribution) where all probability mass is concentrated on a single edge. In other words, suppose $\tilde{p} = p^* = \delta_{ij}$ where $\delta_{ij}$ is the Kronecker-delta. In such a skewed distribution, as $\epsilon \rightarrow 0$, the lower bound reduces to 
% \[
% \mathbb{E}[|\mathcal{E}^* \cap \mathcal{\tilde{E}}|] \geq k \sum_{j=1}^{|\mathcal{E}|} (\tilde{p}_j)^2 = k.
% \]
% This identity suggests that the sampled edges are expected to 100\% overlap with the true, optimal sparse subgraph.

\begin{theorem}[Upper-bound]
\begin{equation}
\mathbb{E}[|\mathcal{E}^* \cap \mathcal{\tilde{E}}|] \leq k (1 - \frac{\|p^* - \tilde{p}\|_1}{2}), 
\end{equation}
where $k = \floor{q|\mathcal{E}|/100}$ and $0 \leq q \leq 100$ is a user-specified parameter.
\end{theorem}
\begin{proof}
\begin{align*}
    \mathbf{Pr}(\mathcal{E}_i^* = \mathcal{\tilde{E}}_i) &= \sum_{j=1}^{|\mathcal{E}|} p^*_j \cdot \tilde{p}_j \\
    & \leq \sum_{j=1}^{|\mathcal{E}|} \min(p^*_j,\tilde{p}_j) \\
    &= 1 - d_{TV}(p^*,\tilde{p}) \\
    &= 1 - \frac{1}{2} \|p^* - \tilde{p}\|_1    
\end{align*}
\end{proof}
Here $d_{TV}$ is the total variation distance. The result used in the last line regarding $d_{TV}$ can be found in~\citet{xie2024distributionally}. 

\paragraph{The implication of the upper-bound.} 
When $\tilde{p} \rightarrow p^*$, the norm $\|p^* - \tilde{p}\|_1 \rightarrow 0$; therefore, the number of common edges could be close to $k$.

\subsection{Upper-bounding the error in the learned Adjacency matrix} 
With the bound proven earlier on the \#common edges by the sparse subgraph of \sgs with that by a learning ORACLE, in this section, we want to obtain an upper-bound on the error in terms of the norm of the Adjacency matrices. As adjacency matrices are used by GNNs for computing node embeddings, such result is important for obtaining error bound on the embeddings later on.

Let $\mA_{\tilde{\gG}}$ and $\mA_{\gG^*}$ be the corresponding adjacency matrices of the learned sparse graph $\tilde{\gG}$ and true optimal sparse graph $\gG^*$. The dimension of these matrices is the same as the input adjacency matrix $\mA_{\mathcal{G}}$ except that $\mA_{\mathcal{G}}$ is denser. Let us also denote the Frobenius norm of a matrix $\mA$ as $\|\mA\|_F$ and the spectral norm of $\mA$ as $\|\mA\|_2$. The Frobenius norm of $\mA$ is defined as $\sqrt{\sum_{ij} \mA^2_{ij}}$, whereas the spectral norm of $\mA$ is the largest singular value $\sigma_{max}(\mA)$ of $\mA$.


Since \sgs do not know the true probability distribution $p^*$, error is introduced in the learned adjacency matrix $\mA_{\tilde{\gG}}$ of the downstream sparse subgraph. We are interested in analyzing the expected error introduced in $\mA_{\tilde{\gG}}$ in terms of the spectral norm, to be precise, $\mathbb{E}[\|\mA_{\tilde{\gG}} - \mA_{\gG^*}\|_2]$. To this end, we will exploit the lower bound derived in Theorem 1 and the fact that $\|\mA\|_2 \leq \|\mA\|_F$. 

\begin{lemma}[Error in Adjacency matrix approximation] Let $\mA_{\tilde{\gG}}$ and $\mA_{\gG^*}$ be the corresponding adjacency matrices of the learned sparse graph $\tilde{\gG}$ and true optimal sparse graph $\gG^*$. If the downstream sampler sampled $k$ edges independently at random (with replacement) to construct those matrices following their respective distributions $\tilde{p}$ and $p^*$, then 
    \[
    \mathbb{E}[\|\mA_{\tilde{\gG}} - \mA_{\gG^*}\|_2] \leq \sqrt{2k(1-\sum_{j=1}^{\abs{\mathcal{E}}} \frac{(p^*_j + \tilde{p}_j - \epsilon)^2}{4})},
    \]
    where $k = \floor{q|\mathcal{E}|/100}$ and $0 \leq q \leq 100$ is a user-specified parameter.
\end{lemma}
\begin{proof}
Since the entries in adjacency matrices are either $0$ or $1$, the difference $\mA_{\tilde{\gG}}(i,j) - \mA_{\gG^*}(i,j)$ are in $\{-1,0,1\}$ for all $i,j$. The following holds by definition of Frobenus norm,

\[
\|\mA_{\tilde{G}} - \mA_{G^*}\|^2_F = \sum_{ij}(\mA_{\tilde{\gG}}(i,j) - \mA_{\gG^*}(i,j))^2.
\] 
As a result, only the non-zero entries in $\mA_{\tilde{\gG}} - \mA_{\gG^*}$ contribute to the square of Frobenius norm $\|\mA_{\tilde{G}} - \mA_{G^*}\|^2_F$.
The expected number of non-zero entries in $\|\mA_{\tilde{\gG}} - \mA_{\gG^*}\|^2_F$ corresponds to the expected cardinality $\abs{(\mathcal{\tilde{E}} \setminus \mathcal{E}^*) \cup (\mathcal{E}^* \setminus \mathcal{\tilde{E}})}$. Thus

\begin{align*}
    \mathbb{E}[\|\mA_{\tilde{\gG}} - \mA_{\gG^*}\|^2_F] &= \mathbb{E}[\abs{(\mathcal{\tilde{E}} \setminus \mathcal{E}^*) \cup (\mathcal{\tilde{E}} \setminus \mathcal{E}^*)}] \\ 
    &= \mathbb{E}[\abs{\mathcal{\tilde{E}}} + \abs{\mathcal{E}^*} - 2 \abs{\mathcal{\tilde{E}} \cap \mathcal{E}^*}] \\
    &= 2k - 2\mathbb{E}[\abs{\mathcal{\tilde{E}} \cap \mathcal{E}^*}] \\
    &\leq 2k - 2k \sum_{j=1}^{|\mathcal{E}|} \frac{(p^*_j + \tilde{p}_j - \epsilon)^2}{4} \\
    &= 2k (1 - \sum_{j=1}^{|\mathcal{E}|} \frac{(p^*_j + \tilde{p}_j - \epsilon)^2}{4}).
\end{align*}
Applying Jensen's inequality for convex functions, in particular, applying $(\mathbb{E}[\rX])^2 \leq \mathbb{E}[\rX^2])$ yields,
\begin{align*}
     (\mathbb{E}[\|\mA_{\tilde{\gG}} - \mA_{\gG^*}\|_F])^2 &\leq  \mathbb{E}[\|\mA_{\tilde{\gG}} - \mA_{\gG^*}\|^2_F] \\
     &\leq 2k (1 - \sum_{j=1}^{|\mathcal{E}|} \frac{(p^*_j + \tilde{p}_j - \epsilon)^2}{4}).
\end{align*}
Taking square-root on both sides yields,
\[
 \mathbb{E}[\|\mA_{\tilde{\gG}} - \mA_{\gG^*}\|_F] \leq \sqrt{2k (1 - \sum_{j=1}^{|\mathcal{E}|} \frac{(p^*_j + \tilde{p}_j - \epsilon)^2}{4})}.
\]
We obtain the theorem using the following relation between the Frobenius and spectral norms.
\begin{align*}
    \|\mA_{\tilde{\gG}} - \mA_{\gG^*}\|_2 &\leq \|\mA_{\tilde{\gG}} - \mA_{\gG^*}\|_F \\
    \implies \mathbb{E}[\|\mA_{\tilde{\gG}} - \mA_{\gG^*}\|_2] &\leq \mathbb{E}[\|\mA_{\tilde{\gG}} - \mA_{\gG^*}\|_F] \\
    &= \sqrt{2k (1 - \sum_{j=1}^{|\mathcal{E}|} \frac{(p^*_j + \tilde{p}_j - \epsilon)^2}{4})}
\end{align*}
\end{proof}
% \begin{corollary} Let us assume $\epsilon \rightarrow 0$ and the model learned the true pmf. Then the spectral norm approximation error is
%     \[
%     \mathbb{E}[\|\mA_{\tilde{\gG}} - \mA_{\gG^*}\|_2] \leq \sqrt{2k(1-\frac{1}{4\abs{\mathcal{E}}})}
%     \]
%     where $k = \floor{q|\mathcal{E}|/100}$ and $0 \leq q \leq 100$ is a user-specified parameter.
% \end{corollary}
% \begin{proof}
% Since we assumed $\epsilon \rightarrow 0$ and $p^*_j = \tilde{p}_j$, Theorem 3 reduces to
% \[
% \mathbb{E}[\|\mA_{\tilde{\gG}} - \mA_{\gG^*}\|_2] \leq \sqrt{2k(1- \frac{\sum_{j=1}^{\abs{\mathcal{E}}} (p^*_j)^2}{4})}
% \]
% For any probability mass function the following inequality holds
% $\sum_{i=1}^n p^2_i \geq 1/n$. This holds with equality when the distribution is uniform. Thus,
% \[
% \sum_{j=1}^{\abs{\mathcal{E}}} (p^*_j)^2 \geq \frac{1}{\abs{\mathcal{E}}}\\
% \implies \mathbb{E}[\|\mA_{\tilde{\gG}} - \mA_{\gG^*}\|_2] \leq \sqrt{2k(1-\frac{1}{4\abs{\mathcal{E}}})}
% \]
% \end{proof}
\subsection{Upper-bounding the error in the predicted node embeddings}
\label{theo:gcnembed}
We consider vanilla GCN as proof of concept to understand how the changes in the sparse subgraph affect the node embeddings produced by a trained GCN. Our goal is to analyze the respective encodings produced by an $L$-layer GCN when the input subgraphs are $\gG^*$ (corresponding to $\mA_{\gG^*}$) and $\tilde{\gG}$ (corresponding to $\mA_{\tilde{\gG}}$) respectively. For simplicity, we will shorten the matrices $\mA_{\gG^*}$ as $\mA^*$ and $\mA_{\tilde{\gG}}$ as $\tmA$. 

A single GCN layer is defined as,

\[
\mH^{(l+1)} = \sigma(\hat{\mA}\mH^{(l)}\mW^{(l)}),
\]
where $\hat{\mA} = \mD^{-1/2}\mA\mD^{-1/2}$ is the normalized adjacency matrix, $\mH^{(l)}$ is the input to the $l$-th layer with $\mH^{(0)} = \mX$, $\mW^{(l)}$ is the learnable weight matrix for $l$-th layer and $\sigma$ is non-linear activation function. Let us suppose an $L$-layer GCN produces embeddings $\tmH^{(L)}$ and $\mH^{*(L)}$ when it takes sparse matrices $\tmA$ and $\mA^*$ as input. We want to upper-bound,
\[
\mathbb{E}[\normLtwo{\tmH^{(L)} - \mH^{*(L)}}],
\]
in other words, the loss in the downstream node encodings is due to using our learned subgraph. 

\paragraph{Assumptions.} We assume that for all $l$, $\normLtwo{\mW} \leq \alpha < 1$ where $\alpha$ is a constant no more than 1. This is reasonable since each $\mW^{(l)}$ is typically controlled during training using regularization techniques, e.g., weight decay. Assuming that the input features in $\mX$ are bounded, we can also assume that there exists a constant $\beta$ such that $\forall l$, $\normLtwo{H}^{(l)} \leq \beta$. We also assume that $\sigma$ is \emph{Lipschitz continuous} with \emph{Lipschitz constant} $L_\sigma$; for instance,  activation functions such as \relu, sigmoid, or tanH are Lipschitz continuous. In particular, we assume \relu activation for our theoretical analysis because \relu has \emph{Lipschitz constant} $L_\sigma = 1$, which simplifies our analysis.

Under these assumptions, we have the following theorem,
\begin{theorem}[Error in GCN encodings]
For sufficiently deep L-layer GCN (large L), the error 
{
\[
\mathbb{E}[\lim_{L \to \infty} \normLtwo{\tmH^{(L)} - \mH^{*(L)}}] < \frac{\beta}{1-\alpha}\sqrt{2k (1 - \sum_{j=1}^{|\mathcal{E}|} \frac{(p^*_j + \tilde{p}_j - \epsilon)^2}{4})}.
\]
}
\end{theorem}
\begin{proof}
{
\[
\tmH^{(L)} - \mH^{*(L)} = \sigma(\hat{\tmA}\tmH^{(L-1)}\mW^{(L-1)}) - \sigma(\hat{\mA}^*\mH^{*(L-1)}\mW^{(L-1)})
\]
}
Since $\sigma$ is a Lipschitz continuous function, we have
{
\begin{align*}
\normLtwo{\tmH^{(L)} - \mH^{*(L)}} \leq L_\sigma\normLtwo{\hat{\tmA}\tmH^{(L-1)}\mW^{(L-1)} - \hat{\mA}^*\mH^{*(L-1)}\mW^{(L-1)}} \\
= \normLtwo{\hat{\tmA}\tmH^{(L-1)}\mW^{(L-1)} - \hat{\mA}^*\mH^{*(L-1)}\mW^{(L-1)}}\\
= \normLtwo{(\hat{\tmA} -\hat{\mA}^*) \tmH^{(L-1)}\mW^{(L-1)} + \hat{\mA}^*(\tmH^{(L-1)}- \mH^{*(L-1)})\mW^{(L-1)}}
\end{align*}
}
For notational convenience, let us suppose $D^{(L)} = \normLtwo{\tmH^{(L)} - \mH^{*(L)}}$. Applying the sub-multiplicative property of the spectral norm and triangle inequality, we obtain the following recurrence relation
{
\begin{align*}
    D^{(L)} &\leq \normLtwo{(\hat{\tmA} -\hat{\mA}^*)}\normLtwo{\tmH^{(L-1)}}\normLtwo{\mW^{(L-1)}} +   \normLtwo{\hat{\mA}^*}D^{(L-1)}\normLtwo{\mW^{(L-1)}} \\
    &\leq \normLtwo{(\hat{\tmA} -\hat{\mA}^*)} \beta\alpha + \normLtwo{\hat{\mA}^*}D^{(L-1)}\alpha \\
    &\leq \normLtwo{(\hat{\tmA} -\hat{\mA}^*)} \beta\alpha + D^{(L-1)}\alpha 
\end{align*}
}
The last inequality holds because normalized adjacency matrix satisfies $\normLtwo{\hat{\mA}^*} \leq 1$. This is because $\hat{\mA}^*$ is symmetric, row-stochastic matrix. Thus the singular values of $\hat{\mA}^*$ is the absolute values of eigenvalues of $\hat{\mA}^*$ and the largest singular value of $\hat{\mA}^*$ is the largest eigenvalue of $\hat{\mA}^*$. But $\hat{\mA}^*$ being row-stochastic, its largest eigenvalue is at most 1 hence $\normLtwo{\hat{\mA}^*} = \sigma_{max}(\hat{\mA}^*) \leq 1$.

By unrolling the recursion from earlier inequality:
\begin{align*}
     D^{(L)} &\leq \normLtwo{(\hat{\tmA} -\hat{\mA}^*)} \beta \alpha\sum_{l=0}^{L-1} \alpha^{l} + D^{(0)}\alpha^L
\end{align*}
$D^{(0)} = \normLtwo{\tmH^{(0)} - \mH^{*(0)}} = \normLtwo{\mX - \mX} = 0$. Since $\alpha < 1$, The geometric series simplifies to:
\begin{align*}
\sum_{l=0}^{L-1} \alpha^{l} = \frac{1-\alpha^L}{1-\alpha} \\
\lim_{L \to \infty} \sum_{l=0}^{L-1} \alpha^{l} = \frac{1}{1-\alpha}
\end{align*}
Thus our earlier inequality becomes:
\[
\lim_{L \to \infty} D^{(L)} \leq \frac{\beta\alpha}{1-\alpha}\normLtwo{(\hat{\tmA} -\hat{\mA}^*)} < \frac{\beta}{1-\alpha}\normLtwo{(\hat{\tmA} -\hat{\mA}^*)}
\]
Taking expectation on both sides gives us our desired result:
\small{
\begin{align*}
    \mathbb{E}[\lim_{L \to \infty} \normLtwo{\tmH^{(L)} - \mH^{*(L)}}] = \mathbb{E}[D^{(L}] < \frac{\beta}{1-\alpha}\mathbb{E}[\normLtwo{(\hat{\tmA} -\hat{\mA}^*)}] \\
    < \frac{\beta}{1-\alpha}\mathbb{E}[\normLtwo{(\hat{\mA} - \mA^*)}] \\
    = \frac{\beta}{1-\alpha}\sqrt{2k (1 - \sum_{j=1}^{|\mathcal{E}|} \frac{(p^*_j + \tilde{p}_j - \epsilon)^2}{4})}
\end{align*}
}
% Expanding the difference in the RHS:
% {
% \[
% \normLtwo{\hat{\tmA}\tmH^{(L-1)}\mW^{(L-1)} - \hat{\mA}^*\mH^{*(L-1)}\mW^{(L-1)}} = 
% \normLtwo{(\hat{\tmA} -\hat{\mA}^*) \tmH^{(L-1)}\mW^{(L-1)} - \hat{\mA}^*\mH^{*(L-1)}\mW^{(L-1)}}
% \]
% }
\end{proof}
% \begin{corollary}[Condition for convergence in embedding]
%     Assuming infinite depth of GCN layer $L$, the learned node embeddings converge to the `true embeddings' if the number of sampled edges satisfies
%     \[
%     k < \frac{1}{2}(\frac{1-\alpha}{\beta})^2
%     \]
% Here, by `true embedding', we mean the embedding generated by applying GCN on a subgraph sampled following the optimal probability distribution.
% \end{corollary}
% \begin{proof}
    
% \end{proof}
% We know that the total variation distance between two probability distributions is 1/2 of the $\text{L}_1$-distance between them. Hence, the following holds by applying universal approximation theorem (equation~\ref{eq:uapp})

% \begin{align}
% \text{TVD}(\tilde{p} , p^*) = \frac{1}{2}\|\tilde{p} - p^*\|_1 \leq \frac{\epsilon}{2}
% \end{align}

% \paragraph{2. The induced sparse subgraphs spectrum is a good approximation to the true graphs spectrum}
% $\forall \vx, \exists \epsilon$ such that
% \[
% (1-\epsilon) \leq \frac{\vx^{\top}\tilde{L}\vx}{\vx^{\top}L\vx} \leq (1+\epsilon)
%  \]
%  where $\tilde{L}$ is the Laplacian of the sparse graph sampled following the learned distribution $\tilde{p}$ and L is the original graph Laplacian.

% How to bound the ratio of the quadratic forms. Some results here: https://arxiv.org/pdf/2403.13268

\FloatBarrier



\clearpage
\section{Analyzing the Effectiveness of \sgs with a Synthetic Graph}
\label{app:toymoon}
In this section, we demonstrate and analyze the effectiveness of \sgs with a synthetically generated heterophilic graph.

\paragraph{Synthetic Graph: Moon.}
The moon dataset has the following properties: number of nodes $|\gV|=150$, number of edges $|\gE|=870$, average degree $d=5.8$, node homophily $\gH_n=0.2$, edge homophily $\gH_e = 0.32$, training/test split = $30\%/70\%$, and 2D coordinates of the points representing the nodes are the node features $\mX$.
The dataset comprises two half-moons representing two communities with $68\%$ edges connecting them as bridge edges.

%n_samples=150, degree=4, train=0.3, h = 0.2


\paragraph{Explaining the Effectiveness of \sgs on Heterophilic graph.} Fig.~\ref{fig:moongraph} juxtaposes the input moon graph (Fig.~\ref{fig:moongraph}, left) and the sparsified moon graph by \sgs (Fig.~\ref{fig:moongraph}, right). \sgs removes a significant portion of bridge edges, causing an increase in edge homophily from $0.32$ to $1.0$. As a result, the accuracy of vanilla GCN increased from $80\%$ on the full graph to $100\%$ on the sparsified graph. Since heterophilous edges significantly hinder the node representation learning, \sgs identifies them during training and learns to put less probability mass on such edges for downstream node classification. 
% \sgs identifies such edges that are detrimental to a downstream task by learning to put less probability mass on them. 
Due to this learning dynamics, \sgs is more effective on heterophilic graphs such as the Moon graph.

%%%%%%%%%%%%%%%%%%%%%%%%%%%%%%%%%%%%%%%%%%%%
\begin{figure}[!htbp]
\centering
\includegraphics[width=\linewidth]{Figures/SGS-moon.png}
\caption{Toy example with two half moon demonstrates the effectiveness of \sgs. The original graph has $68\%$ edges with different node labels; in contrast, the learned sparse subgraph from \sgs contains no such bridge edges.}
\label{fig:moongraph}
\end{figure}

%%%%%%%%%%%%%%%%%%%%%%%%%%%%%%%%%%%%%%%%%%%%

\clearpage
\section{Additional algorithmic details of \sgs }
\label{app:algorithm}

\paragraph{Conditional update of \edgemlp.}
Backward propagation is often the most computationally intensive part of training, so we employ a conditional mechanism to update \edgemlp selectively. 
We evaluate the learned sparse subgraph (line 9, Alg.~\ref{alg:sgstrainingpriorfull}) against a subgraph from the prior probability distribution $p_\mathrm{prior}$ (line 11, Alg.~\ref{alg:sgstrainingpriorfull}). If the training F1-score from the learned sparse subgraph is better than the baseline, parameters of \edgemlp are updated (line 19, Alg.~\ref{alg:sgstrainingpriorfull}). Otherwise, the update to \edgemlp is skipped (line 22, Alg.~\ref{alg:sgstrainingpriorfull}). 

The detailed algorithm for \sgs with conditional updates is in Alg.~\ref{alg:sgstrainingpriorfull}.


% If the size of $\gG$ is large, we compute a sparse subgraph from a prior probability distribution $p_\mathrm{prior}$ for \edgemlp. Later, we sample another sparse subgraph from the learned distribution to use with downstream GNN. The degree-proportionate prior is,

%  \begin{equation}
%  p_\mathrm{prior}(u,v) = \frac{1/d_u + 1/d_v}{\sum_{i,j\in \gE} (1/d_i + 1/d_j)}.
% \end{equation}

% % %%%%%%%%%%%%%%%%%%
% % \begin{wrapfigure}{c}{0.8\textwidth}
% \begin{center}
% \begin{algorithm}[!htbp]
% \caption{\sgs Training}
% \label{alg:sgstraining}
% \begin{algorithmic}[1] % The [1] here is for line numbering
% \STATE \textbf{Input:} $\gG (\gV, \gE, \mX)$, sample percent $q$, $num\_hops$
% \STATE \textbf{Output:} \texttt{EdgeMLP}, \texttt{GNN}

% %\STATE Sample size, $Q =\frac{q\gE}{100}$
% %\STATE Degree Norm, $p(u,v) = \frac{1/d_u + 1/d_v}{\sum_{i,j\in \gE} (1/d_i + 1/d_j)}$

% \FOR{$\mathrm{epochs}$ in $\mathrm{max\_epochs}$}

%     \STATE $\tilde{p},\vw =\edgemlp(\gE,p_\mathrm{prior})$
    
%     \STATE $\gE',\vw' = \mathrm{Sample}(\tilde{p}, \vw, \floor{\frac{q|\gE|}{100})}$ \COMMENT{Sparse graph for downstream GNN}
%     \STATE $\hat{\mY}, \mH' = \mathrm{GNN}_\theta(\gE',\vw',\mX)$

%     \STATE $\gL_\mathrm{CE} = \mathrm{CrossEntropy} (\mY\mathrm{[train]}, \hat{\mY}\mathrm{[train]})$
%     \STATE $\mathrm{mask[u,v]}=True:\forall_{(u,v)\in \gE} u \in \gV_L \land v \in \gV_L$
%     \STATE $\gL_\mathrm{assor} = \mathrm{CrossEntropy}(\gE \mathrm{[mask]},\vw \mathrm{[mask]})$

%     \STATE $\gL_\mathrm{cons} = \mathrm{Similarity} (\vw, \mathrm{Cosine}(\mH'[\gE[s]],\mH'[\gE[t]]))$

%     \STATE $\gL = \alpha_1\cdot \gL_\mathrm{CE}+ \alpha_2\cdot \gL_\mathrm{assor}+ \alpha_3\cdot \gL_\mathrm{cons}$
%     \STATE Backward propagate through $\gL$ and optimize $\edgemlp_\phi, \mathrm{GNN}_\theta$.
% \ENDFOR

% \STATE \textbf{Return} \texttt{EdgeMLP}, \texttt{GNN} 
% \end{algorithmic}
% \end{algorithm}
% \end{center}
% % \end{wrapfigure}
% % %%%%%%%%%%%%%%%%%%

% \sgs also supports conditional updates of \edgemlp, and the pseudocode is outlined in Algorithm~\ref{alg:sgstrainingpriorfull}. 


% %%%%%%%%%%%%%%%%%%
\begin{algorithm}[!htbp]
\caption{\sgs Training with conditional updates}
\begin{algorithmic}[1] % The [1] here is for line numbering
\STATE \textbf{Input:} $\gG (\gV, \gE, \mX)$, sample percent $q$, $\mathrm{hops}$, METIS Parts, $n$
\STATE \textbf{Output:} \texttt{EdgeMLP}, \texttt{GNN}
\STATE Compute $p_\mathrm{prior}(u,v) \gets \frac{1/d_u + 1/d_v}{\sum_{i,j\in \gE} (1/d_i + 1/d_j)}$

\STATE $\gG_\mathrm{parts}=\{\gG_1,\gG_2,\cdots,\gG_n\}\gets \mathrm{METIS} (\gG(\gV,\gE, p), n)$

\FOR{$\mathrm{epochs}$ in $\mathrm{max\_epochs}$}

    \FOR {$\gG_i(\gV_i,\gE_i,\mX_i,p^i_\mathrm{prior}) \in \gG_\mathrm{parts}$}
        \STATE $\tilde{p},\vw \gets \edgemlp(\gE_i,p^i_\mathrm{prior}, \mX_i,\mathrm{hops})$        
        \STATE $\tilde{p}_a \gets \lambda \tilde{p}+(1-\lambda)p^i_\mathrm{prior}$
        \STATE $\tilde{\gE},\tilde{\vw} \gets \mathrm{Sample}(\tilde{p}_a,\vw,\floor{\frac{q|\gE_i|}{100}})$ \COMMENT{\textbf{Learned sparse subgraph}}
        
        \STATE $\hat{\mY}, \tilde{\mH} \gets \mathrm{GNN}_\theta(\tilde{\gE},\tilde{\vw},\mX)$
        
        \STATE $\gE_\mathrm{prior} \gets \mathrm{Sample}({p_i},\floor{\frac{q|\gE_i|}{100}})$ \COMMENT{\textbf{Sparse subgraph from prior}}

        \STATE $\hat{\mY}_\mathrm{prior}  \gets \mathrm{GNN}_\theta(\gE_\mathrm{prior},\mX)$

        \IF {Evaluate$(\hat{\mY}) \ge $ Evaluate$(\hat{\mY}_\mathrm{prior})$}
            \STATE $\gL_\mathrm{CE} \gets \mathrm{CrossEntropy} (\mY_{\gV_L}, \hat{\mY}_{\gV_L})$

            \STATE $\forall_{(u,v)\in \gE_i} u \in \gV_L \land v \in \gV_L : \mathrm{mask[u,v]} \gets \text{True}$
        
            \STATE $\gL_\mathrm{assor} \gets \mathrm{CrossEntropy}(\gE \mathrm{[mask]},\vw \mathrm{[mask]})$
        
            \STATE $\gL_\mathrm{cons} \gets \mathrm{Sim} (\vw, \mathrm{Cosine}(\vh_u,\vh_v): \forall_{(u,v)\in \gE})$ 
        
            \STATE $\gL \gets \alpha_1\cdot \gL_\mathrm{CE}+ \alpha_2\cdot \gL_\mathrm{assor}+ \alpha_3\cdot \gL_\mathrm{cons}$
            \STATE Backward Propagate through $\gL$ and optimize EdgeMLP$_\phi, \mathrm{GNN}_\theta$.
        
        \ELSE
            \STATE $\gL_\mathrm{CE} \gets \mathrm{CrossEntropy} (\mY_{\gV_L}, \hat{\mY}_{\gV_L})$
            \STATE Backward Propagate through $\gL_\mathrm{CE}$ and optimize $\mathrm{GNN}_\theta$.            
        \ENDIF
    
    \ENDFOR
    
\ENDFOR
\STATE \textbf{Return} \texttt{EdgeMLP}, \texttt{GNN} 
\end{algorithmic}
\label{alg:sgstrainingpriorfull}
\end{algorithm}
% %%%%%%%%%%%%%%%%%%

\clearpage
\paragraph{Inference.} 
During inference, we use the learned probability distribution from \edgemlp. We keep track of the best temperature $T$ that gave the best validation accuracy and use that to sample an ensemble of sparse subgraphs. Then, we mean-aggregate their representations to produce the final
prediction on a test node. 

The reason we consider ensemble of subgraphs is because there are variability in the edges of the sample subgraphs even if they are all sampled from the same distribution. Thus mean-aggregation of node embeddings is an effective way to improve the robustness of the learned node embeddings. 

The inference pseudocode is provided in Algorithm~\ref{alg:sgsinference}.

% %%%%%%%%%%%%%%%%%%
\begin{algorithm}[!htbp]
\caption{\sgs Inference}
\begin{algorithmic}[1] % The [1] here is for line numbering
\STATE \textbf{Input:} Graph $\gG (\gV, \gE, \mX)$, sample \% $q$, Ensemble size, $R$.
\STATE \textbf{Output:} Prediction, $\hat{\mY}$

%\STATE Sample size, $Q =\frac{q\gE}{100}$
%\STATE Degree Norm, $p(u,v) = \frac{1/d_u + 1/d_v}{\sum_{i,j\in \gE} (1/d_i + 1/d_j)}$
    \STATE $\vw, \tilde{p} = \edgemlp(\gE, \mX, T_\mathrm{best})$ \COMMENT{\textbf{Use $T$ that gave best validation accuracy}.}   
    \STATE $S_y \gets \emptyset$ \COMMENT{Predictions}
    
    \FOR {$i$ in $R$}
        \STATE $\tilde{\gE}, \tilde{\vw} \gets \mathrm{Sample}(\tilde{p},\floor{\frac{q|\gE|}{100}})$        
        \STATE $\hat{\mY}_i \gets \mathrm{GNN}_\theta(\tilde{\gE},\tilde{\vw},\mX)$
        \STATE $S_y \gets S_y \cup \hat{\mY}_i$
    \ENDFOR

    \STATE Predict, $\hat{\mY} \gets \mathrm{Mean} (S_y)$
    
\STATE \textbf{Return} $\hat{\mY}$
\end{algorithmic}
\label{alg:sgsinference}
\end{algorithm}
% %%%%%%%%%%%%%%%%%%
\clearpage


% \FloatBarrier
\section{Dataset Description}
\label{app:dataset}
\begin{table}[!htbp]
\caption{Additional details of the dataset are provided. $\gH_\mathrm{adj}$ refers to adjusted homophily. $d$ corresponds to the average degree, $C$ number of classes, and $F$ is the feature dimension. \textit{Tr.} is the training label rate.}
\label{tab:datasetdescription}
% \begin{wrapfigure}{c}{1.0\textwidth}
\centering
\begin{sc}
\resizebox{1.0\linewidth}{!}
{
\def\arraystretch{1.0}
\begin{tabular}{@{}crrrccrcccl@{}}
\toprule
\textbf{Dataset} &
  $|\gV|$ &
  $|\gE|$ &
  \textbf{$d$} &
  \textbf{$\gH_\mathrm{adj}$} &
  \textbf{$C$} &
  \textbf{$F$} &
  \textbf{Tr.} &
  \textbf{Self-Loop} &
  \textbf{Isolated} &
  \textbf{Context} \\ \midrule
Cornell        & 183       & 557         & 3.04   & -0.42 & 5  & 1703 & 0.48 & TRUE  & FALSE & Web Pages           \\
Texas          & 183       & 574         & 3.14   & -0.26 & 5  & 1703 & 0.48 & TRUE  & FALSE & Web Pages           \\
Wisconsin      & 251       & 916         & 3.65   & -0.20 & 5  & 1703 & 0.48 & TRUE  & FALSE & Web Pages           \\
reed98         & 962       & 37,624      & 39.11  & -0.10 & 3  & 1001 & 0.6  & FALSE & FALSE & Social Network      \\
amherst41      & 2,235     & 181,908     & 81.39  & -0.07 & 3  & 1193 & 0.6  & FALSE & FALSE & Social Network      \\
penn94         & 41,554    & 2,724,458   & 65.56  & -0.06 & 2  & 4814 & 0.47 & FALSE & FALSE & Social Network      \\
Roman-empire   & 22,662    & 65,854      & 2.91   & -0.05 & 18 & 300  & 0.5  & FALSE & FALSE & Wikipedia           \\
cornell5       & 18,660    & 1,581,554   & 84.76  & -0.04 & 3  & 4735 & 0.6  & FALSE & FALSE & Web pages           \\
Squirrel       & 5,201     & 396,846     & 76.30  & -0.01 & 5  & 2345 & 0.48 & TRUE  & FALSE & Wikipedia           \\
johnshopkins55 & 5,180     & 373,172     & 72.04  & 0.00  & 3  & 2406 & 0.6  & FALSE & FALSE & Web Pages           \\
Actor          & 7,600     & 53,411      & 7.03   & 0.01  & 5  & 932  & 0.48 & TRUE  & FALSE & Actors in Movies    \\
Minesweeper    & 10,000    & 78,804      & 7.88   & 0.01  & 2  & 7    & 0.5  & FALSE & FALSE & Synthetic           \\
Questions      & 48,921    & 307,080     & 6.28   & 0.02  & 2  & 301  & 0.5  & FALSE & FALSE & Yandex Q            \\
Chameleon      & 2,277     & 62,792      & 27.58  & 0.03  & 5  & 2581 & 0.48 & TRUE  & FALSE & Wiki Pages          \\
Tolokers       & 11,758    & 1,038,000   & 88.28  & 0.09  & 2  & 10   & 0.5  & FALSE & FALSE & Toloka Platform     \\
Amazon-ratings & 24,492    & 186,100     & 7.60   & 0.14  & 5  & 556  & 0.5  & FALSE & FALSE & Co-purchase network \\
genius         & 421,961   & 1,845,736   & 4.37   & 0.17  & 2  & 12   & 0.6  & FALSE & TRUE  & Social Network      \\
pokec          & 1,632,803 & 44,603,928  & 27.32  & 0.42  & 3  & 65   & 0.6  & FALSE & FALSE & Social Network      \\
arxiv-year     & 169,343   & 2,315,598   & 13.67  & 0.26  & 5  & 128  & 0.6  & FALSE & FALSE & Citation            \\
snap-patents   & 2,923,922 & 27,945,092  & 9.56   & 0.21  & 5  & 269  & 0.6  & TRUE  & TRUE  & Citation            \\
ogbn-proteins  & 132,534   & 79,122,504  & 597.00 & 0.05  & 94 & 8    & 0.2  & FALSE & FALSE & Protein Network     \\\midrule \midrule
Cora           & 19,793    & 126,842     & 6.41   & 0.56  & 70 & 8710 & 0.2  & FALSE & FALSE & Citation Network    \\
DBLP           & 17,716    & 105,734     & 5.97   & 0.68  & 4  & 1639 & 0.2  & FALSE & FALSE & Citation Network    \\
Computers      & 13,752    & 491,722     & 35.76  & 0.68  & 10 & 767  & 0.6  & FALSE & TRUE  & Co-purchase Network \\
PubMed         & 19,717    & 88,648      & 4.50   & 0.69  & 3  & 500  & 0.2  & FALSE & FALSE & Social Network      \\
Cora\_ML        & 2,995     & 16,316      & 5.45   & 0.75  & 7  & 2879 & 0.2  & FALSE & FALSE & Citation Network    \\
SmallCora      & 2,708     & 10,556      & 3.90   & 0.77  & 7  & 1433 & 0.05 & FALSE & FALSE & Citation Network    \\
CS             & 18,333    & 163,788     & 8.93   & 0.78  & 15 & 6805 & 0.2  & FALSE & FALSE & Co-author Network   \\
Photo          & 7,650     & 238,162     & 31.13  & 0.79  & 8  & 745  & 0.2  & FALSE & TRUE  & Co-purchase Network \\
Physics        & 34,493    & 495,924     & 14.38  & 0.87  & 5  & 8415 & 0.2  & FALSE & FALSE & Co-author Network   \\
CiteSeer       & 4,230     & 10,674      & 2.52   & 0.94  & 6  & 602  & 0.2  & FALSE & FALSE & Citation Network    \\
wiki           & 11,701    & 431,726     & 36.90  & 0.58  & 10 & 300  & 0.99 & TRUE  & TRUE  & Wikipedia           \\
Reddit         & 232,965   & 114,615,892 & 491.99 & 0.74  & 41 & 602  & 0.66 & FALSE & FALSE & Social Network      \\ \bottomrule
\end{tabular}
}
\end{sc}
\end{table}
% \end{wrapfigure}
Table~\ref{tab:datasetdescription} shows the details of the characteristics of the graph datasets, including the splits used throughout the experimentation.

Along with synthetic dataset, for heterophily, we used, 
\textit{Cornell, Texas}, \textit{Wisconsin} from the \textit{WebKB}~\cite{pei2020geom}; \textit{Chameleon}, \textit{Squirrel} ~\cite{rozemberczki2021multi}; \textit{Actor} ~\cite{pei2020geom}; \textit{Wiki, ArXiv-year, Snap-Patents, Penn94, Pokec, Genius, reed98, amherst41, cornell5}, and \textit{Yelp}~\cite{lim2021large}. 
We also experiment on some recent benchmark datasets, \textit{Roman-empire, Amazon-ratings, Minesweeper, Tolokers}, and \textit{Questions} from~\cite{platonov2023critical}.

For homophily, we used
\textit{Cora}~\cite{sen2008collective}; \textit{Citeseer}~\cite{giles1998citeseer}; \textit{pubmed} \cite{namata2012query}; \textit{Coauthor-cs}, \textit{Coauthor-physics}~\cite{shchur2018pitfalls}; \textit{Amazon-computers},  \textit{Amazon-photo} ~\cite{shchur2018pitfalls}; \textit{Reddit}~\cite{hamilton2017inductive}; and, \textit{DBLP}~\cite{fu2020magnn}. 

\noindent\textbf{Heterophily Characterization.} The term \emph{homophily} in a graph describes the likelihood that nodes with the same labels are neighbors. Although there are several ways to measure homophily, three commonly used measures are {\em homophily of the nodes} ($\gH_{n}$), {\em homophily of the edges} ($\gH_{e}$), and {\em adjusted homophily} ($\gH_\mathrm{adj}$).
The {\em node homophily}~\cite{pei2020geom} is defined as,  
\begin{align}
\gH_{n} & = \frac{1}{|\gV|} \sum_{u\in \gV}\frac{| \{v\in \gN(u) : y_v = y_u\}|}{|\gN(u)|}.
\end{align}
The {\em edge homophily}~\cite{zhu2020beyond} of a graph is,

\begin{equation}
    \gH_{e} = \frac{|\{(u,v)\in \gE : y_u = y_v\}|}{|\gE|}. 
\end{equation}

The 
{\em adjusted homophily}~\cite{platonov2022characterizing} is defined as,
\begin{equation}
    \gH_\mathrm{adj} = \frac{\gH_{e}-\sum_{k=1}^{c} D_k^2/(2|\gE|^2)}{1-\sum_{k=1}^c D_k^2/2|\gE|^2}. 
\end{equation}

Here, $D_k = \sum_{v:y_v=k}d_v$ denote the sum of degrees of the nodes belonging to class $k$. 

The values of the node homophily and the edge homophily range from $0$ to $1$, and the adjusted homophily ranges from $-\frac{1}{3}$ to $+1$ (Proposition 1 in~\cite{platonov2022characterizing}). 
Among these measures, adjusted homophily considers the class imbalance. Thus, this work classifies graphs with adjusted homophily, $\gH_\mathrm {adj} \le 0.50$ as heterophilic.

% \paragraph{Dataset used in experiments.} 

\clearpage


% \FloatBarrier
\section{Runtime Comparison}
\label{app:runtime}

\subsection{Impact of Conditional Updates on Runtime}
Table.~\ref{tab:largescaleruntime} compares the runtime of \sgs with and without conditional updates for large-scale graphs (with $|\gE| \ge 1M$). The results indicate that conditional updates are similar to our standard training algorithm in terms of computational efficiency while providing improvements in F1-score under identical conditions. The additional computational costs of evaluation with prior get compensated by fewer updates of \edgemlp.

% Please add the following required packages to your document preamble:
% \usepackage{booktabs}
\begin{table}[!htbp]
\caption{Comparison of runtime of \sgs with and without conditional updates on large-scale graphs (with $|\gE| \ge 1M$). Here, \textit{Runtime (s)} refers to the mean training time per epoch. The terms \edgemlp/\gnn represent the proportion of time the \edgemlp module is updated relative to the \gnn. The results indicate that conditional updates are not significantly slower than our standard training algorithm, yet provide performance improvements to \sgs under similar conditions.}
\label{tab:largescaleruntime}
% \begin{wrapfigure}{c}{1.0\textwidth}
\centering
\begin{sc}
\resizebox{1.0\columnwidth}{!}
{
\def\arraystretch{1.0}
\begin{tabular}{@{}crrr|cc|cc|c@{}}
\toprule
\multirow{2}{*}{\textbf{Dataset}} & \multirow{2}{*}{\textbf{Node}} & \multirow{2}{*}{\textbf{Edges}} & \multirow{2}{*}{\textbf{Degree}} & \multicolumn{2}{c|}{\textbf{\sgs Runtime (s)}} & \multicolumn{2}{c|}{\textbf{\sgs F1-Score}} & \multirow{2}{*}{\textbf{\#EdgeMLP/\#GNN}} \\
 &  &  &  & \textbf{w/o. cond} & \textbf{w. cond} & \textbf{w/o. cond} & \textbf{w. cond} &  \\\midrule
cornell5 & 18,660 & 1,581,554 & 84.76 & \textbf{0.3625} & 0.3795 & 69.02 $\pm$ 0.09 & \textbf{69.12 $\pm$ 0.20} & 0.94 \\
Tolokers & 11,758 & 1,038,000 & 88.28 & 0.1743 & \textbf{0.1630} & 78.12 $\pm$ 0.13 & \textbf{78.13 $\pm$ 0.17} & 0.42 \\
genius & 421,961 & 1,845,736 & 4.37 & \textbf{0.3884} & 0.4799 & 79.92 $\pm$ 0.08 & \textbf{80.07 $\pm$ 0.11} & 0.43 \\
pokec & 1,632,803 & 44,603,928 & 27.32 & 6.7984 & \textbf{6.4885} & 62.05 $\pm$ 0.33 & \textbf{62.20 $\pm$ 0.10} & 0.75 \\
arxiv-year & 169,343 & 2,315,598 & 13.67 & \textbf{0.4571} & 0.4580 & \textbf{36.99 $\pm$ 0.11} & 36.98 $\pm$ 0.13 & 0.23 \\
snap-patents & 2,923,922 & 27,945,092 & 9.56 & \textbf{6.3470} & 7.1236 & 34.86 $\pm$ 0.15 & \textbf{34.95 $\pm$ 0.16} & 0.84 \\
Reddit & 232,965 & 114,615,892 & 491.99 & \textbf{8.0892} & 8.2960 & \textbf{91.45 $\pm$ 0.07} & 91.43 $\pm$ 0.02 & 0.44\\\bottomrule
\end{tabular}
}
\end{sc}
\end{table}
% \end{wrapfigure}

\subsection{Comparison with Baseline GNN based Sparsifiers}
\label{app:runtimerelated}
Table~\ref{tab:runtimerelated} shows related algorithms' mean training time (s). Although \sgs is slower than the unsupervised sparsification-based GNNs, it is significantly faster than supervised sparsifiers.
% Please add the following required packages to your document preamble:
% \usepackage{booktabs}
\begin{table}[!htbp]
% \begin{wrapfigure}{c}{1.0\textwidth}
\caption{Mean training time (s) per epoch of related methods. OOM refers to out-of-memory.}
\label{tab:runtimerelated}
\centering
\begin{sc}
\resizebox{1.0\columnwidth}{!}
{
\def\arraystretch{1.0}
\begin{tabular}{@{}c|ccccccc@{}}
\toprule
\textbf{Method} & \textbf{ClusterGCN} & \textbf{GraphSAINT} & \textbf{DropEdge} & \textbf{MOG} & \textbf{SparseGAT} & \textbf{Neural Sparse} & \textbf{SGS-GNN} \\ \midrule
CS & 0.0095 & 0.0089 & 0.0146 & OOM & 0.1009 & 0.1515 & 0.0221 \\
Questions & 0.0082 & 0.0072 & 0.0290 & 0.1263 & 0.0236 & 0.1221 & 0.0261 \\
Amazon-ratings & 0.0068 & 0.0062 & 0.0169 & 0.1054 & 0.0152 & 0.0499 & 0.0178 \\
johnshopkins55 & 0.0071 & 0.0061 & 0.0207 & OOM & 0.0102 & 0.1234 & 0.0244 \\
amherst41 & 0.0062 & 0.0058 & 0.0101 & OOM & 0.0053 & 0.0368 & 0.0162 \\ \bottomrule
\end{tabular}
}
\end{sc}
\end{table}
% \end{wrapfigure}
\clearpage


\section{Ablation Studies}
\label{app:ablationstudy}

This section investigates how different components of \sgs behave and contribute to overall performance. We organize this section as follows,

\begin{enumerate}
    \item Section~\ref{subsec:ab_edgemlpgnn} investigates $\gL_\mathrm{assor}, \gL_{cons}$, \edgemlp, \gnn, and Conditional Updates mechanism. We also compare its runtime against standard \sgs training vs \sgs with conditional updates. We also show \sgs can be used with other GNNs in Sec~\ref{app:othergnn}.

    \item Section~\ref{app:parameters} explores parameter settings with/without prior, different normalization and sampling methods, and inference with/without an ensemble of subgraphs.

    \item Section~\ref{app:gridsearch} shows ideal settings for regularizer coefficients $\alpha_1, \alpha_2, \alpha_3$. We also show the impact of $\lambda$ for augmenting the learned probability distribution $p$ using $p_\mathrm{prior}$.
\end{enumerate}


\subsection{$\gL_\mathrm{assor}, \gL_{cons}$, \edgemlp, \gnn, and Conditional Updates}
\label{subsec:ab_edgemlpgnn}

Table~\ref{tab:ablationgnn} illustrates the performance of \sgs with various combinations of regularizers, embedding layers in \edgemlp, and convolutional layers in \gnn. 

\begin{enumerate}
    \item $\gL_\mathrm{assor}$: Case 1, 2 shows improvement in results when $\gL_\mathrm{assor}$ is used.

    \item $\gL_\mathrm{cons}$: From cases 4, 6, 8 shows $L_\mathrm{cons}$ improves results when $\texttt{GCN}$ module is used in the GNN.

    \item \edgemlp: In general, we found that the \texttt{GCN} layers for \edgemlp encodings performs best (cases 5-6, 11-12). 
    
    \item \gnn: Both \texttt{GCN} and  \texttt{GAT} modules yielded overall the best results (case 6, 11).
    
    \item Conditional updates: Case 3 shows that conditional updates can benefit some graphs.     
    
    We also investigated the runtime and quality of \sgs with and without conditional updates for large-scale graphs. We found both have similar runtime as the condition check expense gets compensated by fewer updates of \edgemlp. Detailed comparisons of conditional updates in large graphs ($|\gE|\ge 1M$) are included in the Table~\ref{tab:largescaleruntime}.   
\end{enumerate}







\begin{table}[!htbp]
\caption{Combination of \edgemlp, \gnn, Conditional update and $L_\mathrm{cons}.$}
\label{tab:ablationgnn}
% \begin{wrapfigure}{c}{1.0\linewidth}
\centering
\begin{sc}
\resizebox{0.9\linewidth}{!}
{
\def\arraystretch{1.0}
\begin{tabular}{cccccc|ccc}
\toprule
% \rowcolor[HTML]{B7B7B7} 
\textbf{} & {$\mathbf{L_\mathrm{assor}}$} & {$\mathbf{L_\mathrm{cons}}$} & \textbf{\edgemlp} & \textbf{\gnn} & \textbf{Cond.} & {\textbf{SmallCora}} & {\textbf{CoraFull}} & {\textbf{johnshopkin}} \\ \midrule
1 & N & N & \cellcolor[HTML]{F4CCCC}MLP & \cellcolor[HTML]{FFF2CC}GCN & N & 73.80 $\pm$ 0.67 & 61.78 $\pm$ 0.20 & 66.12 $\pm$ 1.38 \\
2 & Y & N & \cellcolor[HTML]{F4CCCC}MLP & \cellcolor[HTML]{FFF2CC}GCN & N & 74.88 $\pm$ 0.15 & 63.99 $\pm$ 0.24 & 66.18 $\pm$ 1.05 \\
3 & Y & N & \cellcolor[HTML]{F4CCCC}MLP & \cellcolor[HTML]{FFF2CC}GCN & Y & 75.82 $\pm$ 0.46 & 64.07 $\pm$ 0.31 & 66.87 $\pm$ 0.93 \\
4 & Y & Y & \cellcolor[HTML]{F4CCCC}MLP & \cellcolor[HTML]{FFF2CC}GCN & Y & 76.58 $\pm$ 0.47 & 65.33 $\pm$ 0.28 & 69.25 $\pm$ 0.76 \\
5 & Y & N & \cellcolor[HTML]{D0E0E3}GCN & \cellcolor[HTML]{FFF2CC}GCN & Y & 75.80 $\pm$ 0.77 & 65.66 $\pm$ 0.14 & 71.06 $\pm$ 0.32 \\
\rowcolor[HTML]{D9D9D9} 
6 & Y & Y & \cellcolor[HTML]{D0E0E3}GCN & \cellcolor[HTML]{FFF2CC}GCN & Y & 77.50 $\pm$ 0.62 & \textbf{66.56 $\pm$ 0.22} & 70.79 $\pm$ 0.18 \\
7 & Y & N & \cellcolor[HTML]{CFE2F3}GSAGE & \cellcolor[HTML]{FFF2CC}GCN & Y & 75.82 $\pm$ 0.44 & 63.70 $\pm$ 0.09 & 67.53 $\pm$ 0.80 \\
8 & Y & Y & \cellcolor[HTML]{CFE2F3}GSAGE & \cellcolor[HTML]{FFF2CC}GCN & Y & 77.48 $\pm$ 0.61 & 65.12 $\pm$ 0.11 & 68.63 $\pm$ 0.66 \\
9 & Y & N & \cellcolor[HTML]{F4CCCC}MLP & \cellcolor[HTML]{D9EAD3}GAT & Y & 77.72 $\pm$ 1.63 & 66.40 $\pm$ 0.08 & 67.92 $\pm$ 0.73 \\
10 & Y & Y & \cellcolor[HTML]{F4CCCC}MLP & \cellcolor[HTML]{D9EAD3}GAT & Y & 75.78 $\pm$ 3.22 & 66.46 $\pm$ 0.16 & 68.17 $\pm$ 0.33 \\
\rowcolor[HTML]{D9D9D9} 
11 & Y & N & \cellcolor[HTML]{D0E0E3}GCN & \cellcolor[HTML]{D9EAD3}GAT & Y & \textbf{78.18 $\pm$ 0.74} & 66.33 $\pm$ 0.20 & \textbf{71.97 $\pm$ 0.59} \\
12 & Y & Y & \cellcolor[HTML]{D0E0E3}GCN & \cellcolor[HTML]{D9EAD3}GAT & Y & 76.94 $\pm$ 2.76 & 66.39 $\pm$ 0.18 & 71.00 $\pm$ 0.96 \\
13 & Y & N & \cellcolor[HTML]{CFE2F3}GSAGE & \cellcolor[HTML]{D9EAD3}GAT & Y & 77.98 $\pm$ 0.79 & 66.38 $\pm$ 0.23 & 69.29 $\pm$ 1.56 \\
14 & Y & Y & \cellcolor[HTML]{CFE2F3}GSAGE & \cellcolor[HTML]{D9EAD3}GAT & Y & 75.74 $\pm$ 2.02 & 66.41 $\pm$ 0.25 & 68.82 $\pm$ 0.24\\\bottomrule
\end{tabular}
}
\end{sc}
\end{table}
% \end{wrapfigure}




\subsubsection{\sgs with other GNN modules}
\label{app:othergnn}
The sampled sparse subgraphs from \edgemlp can be fed into any downstream GNNs and demonstrate a couple of variants of \sgs. Chebnet from Chebyshev~\cite{he2022convolutional}, Graph Attention Network (GAT)~\cite{velivckovic2017graph}, Graph Isomorphic Network (GIN)~\cite{xu2018powerful}, Graph Convolutional Network (GCN)~\cite{kipf2016semi} are some of the GNNs used for demonstration. 

Fig.~\ref{fig:sparsityvsgnn} shows the performance of these GNNs on homophilic and heterophilic datasets. \texttt{SGS-GCN} and \texttt{SGS-GAT} are two best performing models.


%%%%%%%%%%%%%%%%%%%%%%%%%%%%%%%%%%%%%%%%%%%%
\begin{figure}[!htbp]
\centering
% \includegraphics[width=0.5\linewidth]{Figures/SparsityvsGNN.png}
\includegraphics[width=0.6\linewidth]{Figures/SGS-differentgnn.pdf}

\caption{Performance of \sgs with different GNN modules using $20\%$ edges.}
\label{fig:sparsityvsgnn}
\end{figure}
%%%%%%%%%%%%%%%%%%%%%%%%%%%%%%%%%%%%%%%%%%%%


% \subsection{Additional Ablation studies}
% \label{app:moreablation}

\subsection{Impact of  $p_\mathrm{prior}$, Normalization \& Sampling schemes, and Ensembling on \sgs}
\label{app:parameters}

Table~\ref{tab:ablation} highlights the impact of the following components: 

% Please add the following required packages to your document preamble:
% \usepackage{booktabs}
\begin{table}[t]
\caption{Ablation Studies different components of \sgs.}
\label{tab:ablation}
% \begin{wrapfigure}{c}{1.0\textwidth}
\centering
\begin{sc}
\resizebox{1.0\linewidth}{!}
{
\def\arraystretch{1.0}
\begin{tabular}{@{}cccccc|cccccc@{}}
\toprule
\textbf{Case} & \textbf{Prior} & \textbf{Norm.} & \textbf{Sampl.}  & \multicolumn{1}{c|}{\textbf{Ensem.}} & \textbf{SmallCora} & \textbf{Cora\_ML} & \textbf{CiteSeer} & \textbf{Squirrel} & \textbf{johnshopkins55} & \textbf{Roman-empire} \\ \midrule
1 & N  & Sum & Mult  & \multicolumn{1}{c|}{N} & 69.30 $\pm$ 1.20 & 81.05 $\pm$ 0.74 & 82.84 $\pm$ 0.47 & 48.90 $\pm$ 1.06 & 63.86 $\pm$ 0.58 & 63.27 $\pm$ 0.31 \\
2 & N  & Sum & Mult  & \multicolumn{1}{c|}{Y} & 72.84 $\pm$ 0.91 & 82.92 $\pm$ 0.73 & \textbf{87.42 $\pm$ 0.42} & 46.30 $\pm$ 1.18 & 65.14 $\pm$ 1.14 & \textbf{64.31 $\pm$ 0.13} \\
% 3 & N  & Sum & Mult & $L_\mathrm{a*}$ & \multicolumn{1}{c|}{Y} & 73.48 $\pm$ 1.11 & \textbf{85.01 $\pm$ 0.33} & 86.43 $\pm$ 0.23 & 49.68 $\pm$ 0.73 & \textbf{74.73 $\pm$ 0.51} & 63.20 $\pm$ 0.21 \\
%4 & N  & Sum & Mult & $L_\mathrm{a*}$, $L_\mathrm{c*}$ & \multicolumn{1}{c|}{Y} & 75.86 $\pm$ 0.74 & 84.82 $\pm$ 0.24 & 86.55 $\pm$ 0.33 & 48.03 $\pm$ 0.79 & 72.08 $\pm$ 1.16 & 63.16 $\pm$ 0.30 \\
% 5 & Y & 0 & Sum & Mult & $L_\mathrm{a*}$, $L_\mathrm{c*}$ & \multicolumn{1}{c|}{Y} & 74.82 $\pm$ 0.64 & 84.21 $\pm$ 0.60 & 86.55 $\pm$ 0.25 & 47.53 $\pm$ 0.32 & 68.44 $\pm$ 0.46 & 63.06 $\pm$ 0.11 \\
% 6 & Y & 1 & Sum & Mult & $L_\mathrm{a*}$, $L_\mathrm{c*}$ & \multicolumn{1}{c|}{Y} & 75.54 $\pm$ 0.23 & 84.01 $\pm$ 0.74 & 86.34 $\pm$ 0.21 & 48.63 $\pm$ 0.44 & 70.77 $\pm$ 0.40 & 63.27 $\pm$ 0.12 \\
3 & Y  & Sum & Mult & \multicolumn{1}{c|}{Y} & 75.54 $\pm$ 0.41 & \textbf{83.87 $\pm$ 0.69} & 86.31 $\pm$ 0.26 & 47.97 $\pm$ 0.60 & 72.68 $\pm$ 0.51 & 62.88 $\pm$ 0.19 \\
4 & Y & Softmax & Mult & \multicolumn{1}{c|}{Y} & 75.44 $\pm$ 0.51 & 83.81 $\pm$ 0.72 & 86.31 $\pm$ 0.26 & 47.90 $\pm$ 0.42 & \textbf{72.97 $\pm$ 0.20} & 62.98 $\pm$ 0.16 \\
5 & Y & Gumbel & TopK & \multicolumn{1}{c|}{Y} & \textbf{76.24 $\pm$ 0.43} & 83.36 $\pm$ 0.34 & 86.44 $\pm$ 0.16 & \textbf{51.49 $\pm$ 0.72} & 71.83 $\pm$ 1.00 & 63.00 $\pm$ 0.11 \\ \midrule
\multicolumn{11}{l}{\textbf{Prior:} Use of prior, \textbf{Sum:} Sum-Normalization, \textbf{Softmax:} \textit{Softmax} with temperature annealing}\\
\multicolumn{11}{l}{\textbf{Mult:} \textit{Multinonmial} Sampling, \textbf{Gumbel:} \textit{Gumbel-Softmax} with \textit{TopK}}\\
\end{tabular}
}
\end{sc}
\end{table}
% \end{wrapfigure}

\begin{enumerate}
    \item Prior $p_\mathrm{prior}$: Cases 2-3 show that augmenting the learned probability distribution $\tilde{p}$ with prior $p_\mathrm{prior}$ can benefit some datasets. We have also conducted an in-depth comparison between the distributions $\tilde{p}$ and augmented distribution $\tilde{p}_a$. Figure~\ref{fig:augment_p} shows that there $\tilde{p}_a$ is left skewed whereas $\tilde{p}$ is not. Since rare edges still get some negligible mass, it is possible for $\tilde{\gG}$ constructed from $\tilde{p}_a$ to retain some bridge edges from these tails, if there are any. 
    
\begin{figure}[!htbp]
    \centering
    %\subfigure{\includegraphics[width=0.48\linewidth]{Figures/SparsityvsHomophily.png}
    \subfigure{\includegraphics[width=0.4\linewidth]{Figures/SGS-learnedp.png}
    \label{subfig:learnedp}} 
    %\hfill
    % \subfigure{\includegraphics[width=0.48\linewidth]{Figures/SparsityvsAccuracy2.png}
    %\hfill     
     \subfigure{\includegraphics[width=0.4\linewidth]{Figures/SGS-learnedp_a.png}
     \label{subfig:priorpa}} 
     \subfigure{\includegraphics[width=0.4\linewidth]{Figures/SGS-prior.png}
     \label{subfig:priorp}}    
    \caption{The learned probability distribution $\tilde{p}$ (top-left), augmented distribution $\tilde{p}_a$(top-right) and fixed prior $p_\mathrm{prior}$ (bottom). Augmentation puts negligible mass on some rare yet critical edges in the left tail of $\tilde{p}_a$.}
    \label{fig:augment_p}
\end{figure}
    \item Normalization and Sampling: We considered three normalization and sampling techniques. i) sum-normalization with multinomial sampling, ii) softmax-normalization with temperature with multinomial sampling, and iii) Gumbel softmax normalization with Topk selection. Cases 3-5 show that each of these techniques can improve results in certain datasets, and thus, it is difficult to nominate a single one as best. However, in our experiments, we opted for multinomial sampling with softmax temperature annealing for training to encourage exploration in early iterations.

    \item Ensemble subgraphs during inference: Case 2 demonstrates that using multiple subgraphs for ensemble prediction yields better results than a single subgraph (Case 1).    
\end{enumerate}

 % The addition of assortative loss $L_\mathrm{assor}$ in Case 3 enhances performance on heterophilic graphs by promoting homophily in sampled subgraphs. %Case 4 shows that incorporating consistency loss $L_\mathrm{cons}$ benefits certain graphs. 
%While our base training method updates \edgemlp at each iteration, conditional update with a prior distribution speeds up results by reducing the search space. Cases 5-7 indicate that hop size at \edgemlp influences performance. 

% % Please add the following required packages to your document preamble:
% \usepackage{booktabs}
\begin{table}[t]
\caption{Ablation Studies different components of \sgs.}
\label{tab:ablation}
% \begin{wrapfigure}{c}{1.0\textwidth}
\centering
\begin{sc}
\resizebox{1.0\linewidth}{!}
{
\def\arraystretch{1.0}
\begin{tabular}{@{}cccccc|cccccc@{}}
\toprule
\textbf{Case} & \textbf{Prior} & \textbf{Norm.} & \textbf{Sampl.}  & \multicolumn{1}{c|}{\textbf{Ensem.}} & \textbf{SmallCora} & \textbf{Cora\_ML} & \textbf{CiteSeer} & \textbf{Squirrel} & \textbf{johnshopkins55} & \textbf{Roman-empire} \\ \midrule
1 & N  & Sum & Mult  & \multicolumn{1}{c|}{N} & 69.30 $\pm$ 1.20 & 81.05 $\pm$ 0.74 & 82.84 $\pm$ 0.47 & 48.90 $\pm$ 1.06 & 63.86 $\pm$ 0.58 & 63.27 $\pm$ 0.31 \\
2 & N  & Sum & Mult  & \multicolumn{1}{c|}{Y} & 72.84 $\pm$ 0.91 & 82.92 $\pm$ 0.73 & \textbf{87.42 $\pm$ 0.42} & 46.30 $\pm$ 1.18 & 65.14 $\pm$ 1.14 & \textbf{64.31 $\pm$ 0.13} \\
% 3 & N  & Sum & Mult & $L_\mathrm{a*}$ & \multicolumn{1}{c|}{Y} & 73.48 $\pm$ 1.11 & \textbf{85.01 $\pm$ 0.33} & 86.43 $\pm$ 0.23 & 49.68 $\pm$ 0.73 & \textbf{74.73 $\pm$ 0.51} & 63.20 $\pm$ 0.21 \\
%4 & N  & Sum & Mult & $L_\mathrm{a*}$, $L_\mathrm{c*}$ & \multicolumn{1}{c|}{Y} & 75.86 $\pm$ 0.74 & 84.82 $\pm$ 0.24 & 86.55 $\pm$ 0.33 & 48.03 $\pm$ 0.79 & 72.08 $\pm$ 1.16 & 63.16 $\pm$ 0.30 \\
% 5 & Y & 0 & Sum & Mult & $L_\mathrm{a*}$, $L_\mathrm{c*}$ & \multicolumn{1}{c|}{Y} & 74.82 $\pm$ 0.64 & 84.21 $\pm$ 0.60 & 86.55 $\pm$ 0.25 & 47.53 $\pm$ 0.32 & 68.44 $\pm$ 0.46 & 63.06 $\pm$ 0.11 \\
% 6 & Y & 1 & Sum & Mult & $L_\mathrm{a*}$, $L_\mathrm{c*}$ & \multicolumn{1}{c|}{Y} & 75.54 $\pm$ 0.23 & 84.01 $\pm$ 0.74 & 86.34 $\pm$ 0.21 & 48.63 $\pm$ 0.44 & 70.77 $\pm$ 0.40 & 63.27 $\pm$ 0.12 \\
3 & Y  & Sum & Mult & \multicolumn{1}{c|}{Y} & 75.54 $\pm$ 0.41 & \textbf{83.87 $\pm$ 0.69} & 86.31 $\pm$ 0.26 & 47.97 $\pm$ 0.60 & 72.68 $\pm$ 0.51 & 62.88 $\pm$ 0.19 \\
4 & Y & Softmax & Mult & \multicolumn{1}{c|}{Y} & 75.44 $\pm$ 0.51 & 83.81 $\pm$ 0.72 & 86.31 $\pm$ 0.26 & 47.90 $\pm$ 0.42 & \textbf{72.97 $\pm$ 0.20} & 62.98 $\pm$ 0.16 \\
5 & Y & Gumbel & TopK & \multicolumn{1}{c|}{Y} & \textbf{76.24 $\pm$ 0.43} & 83.36 $\pm$ 0.34 & 86.44 $\pm$ 0.16 & \textbf{51.49 $\pm$ 0.72} & 71.83 $\pm$ 1.00 & 63.00 $\pm$ 0.11 \\ \midrule
\multicolumn{11}{l}{\textbf{Prior:} Use of prior, \textbf{Sum:} Sum-Normalization, \textbf{Softmax:} \textit{Softmax} with temperature annealing}\\
\multicolumn{11}{l}{\textbf{Mult:} \textit{Multinonmial} Sampling, \textbf{Gumbel:} \textit{Gumbel-Softmax} with \textit{TopK}}\\
\end{tabular}
}
\end{sc}
\end{table}
% \end{wrapfigure}




\clearpage
\subsection{Choosing Values for Regularizer coefficient \(\alpha_3\) and Parameter \(\lambda\)}
\label{app:gridsearch}
Recall that \sgs computes the total loss at each epoch as
\[
\gL = \alpha_1\gL_{CE}+\alpha_2\gL_\mathrm{assor}+\alpha_3\gL_\mathrm{cons},
\]
where $0 \leq \alpha_1,\alpha_2,\alpha_3 \leq 1$ are regularizer coefficients corresponding to the cross-entropy loss $\gL_{CE}$, assortativity loss $L_\mathrm{assor}$ and  consistency loss $\gL_\mathrm{cons}$ respectively. 

Also recall that, when we use a prior probability distribution, the learned distribution values of $\tilde{p}$ are weighted through $\lambda$ in
$\Tilde{p} = \lambda\Tilde{p}+(1-\lambda) p_\mathrm{prior}$

To avoid numerous combinations of values of three coefficients + the parameter $\lambda$, we have fixed $\alpha_1 = 1$, and $\alpha_2 = 1$. In the following, we investigate the performance of \sgs with different values for $\alpha_3$ and $\lambda$.

Fig.~\ref{fig:consbias} shows a grid search for different combinations of $\lambda$ and $\alpha_3$. As per our observation, the recommended values are $\lambda \in [0.3, 0.7], \alpha_3=0.5$.
%%%%%%%%%%%%%%%%%%%%%%%%%%%%%%%%%%%%%%%%%%%%
\begin{figure}[h]
\centering
\includegraphics[width=0.4\linewidth]{Figures/SGS-biasgrid.pdf}
\caption{Grid search for the parameter $\lambda$ for prior, and consistency loss, $\alpha_3$ (Cora dataset).}
\label{fig:consbias}
\end{figure}
%%%%%%%%%%%%%%%%%%%%%%%%%%%%%%%%%%%%%%%%%%%%








\end{document}
