\section{Analyzing the Effectiveness of \sgs with a Synthetic Graph}
\label{app:toymoon}
In this section, we demonstrate and analyze the effectiveness of \sgs with a synthetically generated heterophilic graph.

\paragraph{Synthetic Graph: Moon.}
The moon dataset has the following properties: number of nodes $|\gV|=150$, number of edges $|\gE|=870$, average degree $d=5.8$, node homophily $\gH_n=0.2$, edge homophily $\gH_e = 0.32$, training/test split = $30\%/70\%$, and 2D coordinates of the points representing the nodes are the node features $\mX$.
The dataset comprises two half-moons representing two communities with $68\%$ edges connecting them as bridge edges.

%n_samples=150, degree=4, train=0.3, h = 0.2


\paragraph{Explaining the Effectiveness of \sgs on Heterophilic graph.} Fig.~\ref{fig:moongraph} juxtaposes the input moon graph (Fig.~\ref{fig:moongraph}, left) and the sparsified moon graph by \sgs (Fig.~\ref{fig:moongraph}, right). \sgs removes a significant portion of bridge edges, causing an increase in edge homophily from $0.32$ to $1.0$. As a result, the accuracy of vanilla GCN increased from $80\%$ on the full graph to $100\%$ on the sparsified graph. Since heterophilous edges significantly hinder the node representation learning, \sgs identifies them during training and learns to put less probability mass on such edges for downstream node classification. 
% \sgs identifies such edges that are detrimental to a downstream task by learning to put less probability mass on them. 
Due to this learning dynamics, \sgs is more effective on heterophilic graphs such as the Moon graph.

%%%%%%%%%%%%%%%%%%%%%%%%%%%%%%%%%%%%%%%%%%%%
\begin{figure}[!htbp]
\centering
\includegraphics[width=\linewidth]{Figures/SGS-moon.png}
\caption{Toy example with two half moon demonstrates the effectiveness of \sgs. The original graph has $68\%$ edges with different node labels; in contrast, the learned sparse subgraph from \sgs contains no such bridge edges.}
\label{fig:moongraph}
\end{figure}

%%%%%%%%%%%%%%%%%%%%%%%%%%%%%%%%%%%%%%%%%%%%