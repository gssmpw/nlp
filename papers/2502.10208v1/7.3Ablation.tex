\section{Runtime Comparison}
\label{app:runtime}

\subsection{Impact of Conditional Updates on Runtime}
Table.~\ref{tab:largescaleruntime} compares the runtime of \sgs with and without conditional updates for large-scale graphs (with $|\gE| \ge 1M$). The results indicate that conditional updates are similar to our standard training algorithm in terms of computational efficiency while providing improvements in F1-score under identical conditions. The additional computational costs of evaluation with prior get compensated by fewer updates of \edgemlp.

% Please add the following required packages to your document preamble:
% \usepackage{booktabs}
\begin{table}[!htbp]
\caption{Comparison of runtime of \sgs with and without conditional updates on large-scale graphs (with $|\gE| \ge 1M$). Here, \textit{Runtime (s)} refers to the mean training time per epoch. The terms \edgemlp/\gnn represent the proportion of time the \edgemlp module is updated relative to the \gnn. The results indicate that conditional updates are not significantly slower than our standard training algorithm, yet provide performance improvements to \sgs under similar conditions.}
\label{tab:largescaleruntime}
% \begin{wrapfigure}{c}{1.0\textwidth}
\centering
\begin{sc}
\resizebox{1.0\columnwidth}{!}
{
\def\arraystretch{1.0}
\begin{tabular}{@{}crrr|cc|cc|c@{}}
\toprule
\multirow{2}{*}{\textbf{Dataset}} & \multirow{2}{*}{\textbf{Node}} & \multirow{2}{*}{\textbf{Edges}} & \multirow{2}{*}{\textbf{Degree}} & \multicolumn{2}{c|}{\textbf{\sgs Runtime (s)}} & \multicolumn{2}{c|}{\textbf{\sgs F1-Score}} & \multirow{2}{*}{\textbf{\#EdgeMLP/\#GNN}} \\
 &  &  &  & \textbf{w/o. cond} & \textbf{w. cond} & \textbf{w/o. cond} & \textbf{w. cond} &  \\\midrule
cornell5 & 18,660 & 1,581,554 & 84.76 & \textbf{0.3625} & 0.3795 & 69.02 $\pm$ 0.09 & \textbf{69.12 $\pm$ 0.20} & 0.94 \\
Tolokers & 11,758 & 1,038,000 & 88.28 & 0.1743 & \textbf{0.1630} & 78.12 $\pm$ 0.13 & \textbf{78.13 $\pm$ 0.17} & 0.42 \\
genius & 421,961 & 1,845,736 & 4.37 & \textbf{0.3884} & 0.4799 & 79.92 $\pm$ 0.08 & \textbf{80.07 $\pm$ 0.11} & 0.43 \\
pokec & 1,632,803 & 44,603,928 & 27.32 & 6.7984 & \textbf{6.4885} & 62.05 $\pm$ 0.33 & \textbf{62.20 $\pm$ 0.10} & 0.75 \\
arxiv-year & 169,343 & 2,315,598 & 13.67 & \textbf{0.4571} & 0.4580 & \textbf{36.99 $\pm$ 0.11} & 36.98 $\pm$ 0.13 & 0.23 \\
snap-patents & 2,923,922 & 27,945,092 & 9.56 & \textbf{6.3470} & 7.1236 & 34.86 $\pm$ 0.15 & \textbf{34.95 $\pm$ 0.16} & 0.84 \\
Reddit & 232,965 & 114,615,892 & 491.99 & \textbf{8.0892} & 8.2960 & \textbf{91.45 $\pm$ 0.07} & 91.43 $\pm$ 0.02 & 0.44\\\bottomrule
\end{tabular}
}
\end{sc}
\end{table}
% \end{wrapfigure}

\subsection{Comparison with Baseline GNN based Sparsifiers}
\label{app:runtimerelated}
Table~\ref{tab:runtimerelated} shows related algorithms' mean training time (s). Although \sgs is slower than the unsupervised sparsification-based GNNs, it is significantly faster than supervised sparsifiers.
% Please add the following required packages to your document preamble:
% \usepackage{booktabs}
\begin{table}[!htbp]
% \begin{wrapfigure}{c}{1.0\textwidth}
\caption{Mean training time (s) per epoch of related methods. OOM refers to out-of-memory.}
\label{tab:runtimerelated}
\centering
\begin{sc}
\resizebox{1.0\columnwidth}{!}
{
\def\arraystretch{1.0}
\begin{tabular}{@{}c|ccccccc@{}}
\toprule
\textbf{Method} & \textbf{ClusterGCN} & \textbf{GraphSAINT} & \textbf{DropEdge} & \textbf{MOG} & \textbf{SparseGAT} & \textbf{Neural Sparse} & \textbf{SGS-GNN} \\ \midrule
CS & 0.0095 & 0.0089 & 0.0146 & OOM & 0.1009 & 0.1515 & 0.0221 \\
Questions & 0.0082 & 0.0072 & 0.0290 & 0.1263 & 0.0236 & 0.1221 & 0.0261 \\
Amazon-ratings & 0.0068 & 0.0062 & 0.0169 & 0.1054 & 0.0152 & 0.0499 & 0.0178 \\
johnshopkins55 & 0.0071 & 0.0061 & 0.0207 & OOM & 0.0102 & 0.1234 & 0.0244 \\
amherst41 & 0.0062 & 0.0058 & 0.0101 & OOM & 0.0053 & 0.0368 & 0.0162 \\ \bottomrule
\end{tabular}
}
\end{sc}
\end{table}
% \end{wrapfigure}
\clearpage


\section{Ablation Studies}
\label{app:ablationstudy}

This section investigates how different components of \sgs behave and contribute to overall performance. We organize this section as follows,

\begin{enumerate}
    \item Section~\ref{subsec:ab_edgemlpgnn} investigates $\gL_\mathrm{assor}, \gL_{cons}$, \edgemlp, \gnn, and Conditional Updates mechanism. We also compare its runtime against standard \sgs training vs \sgs with conditional updates. We also show \sgs can be used with other GNNs in Sec~\ref{app:othergnn}.

    \item Section~\ref{app:parameters} explores parameter settings with/without prior, different normalization and sampling methods, and inference with/without an ensemble of subgraphs.

    \item Section~\ref{app:gridsearch} shows ideal settings for regularizer coefficients $\alpha_1, \alpha_2, \alpha_3$. We also show the impact of $\lambda$ for augmenting the learned probability distribution $p$ using $p_\mathrm{prior}$.
\end{enumerate}


\subsection{$\gL_\mathrm{assor}, \gL_{cons}$, \edgemlp, \gnn, and Conditional Updates}
\label{subsec:ab_edgemlpgnn}

Table~\ref{tab:ablationgnn} illustrates the performance of \sgs with various combinations of regularizers, embedding layers in \edgemlp, and convolutional layers in \gnn. 

\begin{enumerate}
    \item $\gL_\mathrm{assor}$: Case 1, 2 shows improvement in results when $\gL_\mathrm{assor}$ is used.

    \item $\gL_\mathrm{cons}$: From cases 4, 6, 8 shows $L_\mathrm{cons}$ improves results when $\texttt{GCN}$ module is used in the GNN.

    \item \edgemlp: In general, we found that the \texttt{GCN} layers for \edgemlp encodings performs best (cases 5-6, 11-12). 
    
    \item \gnn: Both \texttt{GCN} and  \texttt{GAT} modules yielded overall the best results (case 6, 11).
    
    \item Conditional updates: Case 3 shows that conditional updates can benefit some graphs.     
    
    We also investigated the runtime and quality of \sgs with and without conditional updates for large-scale graphs. We found both have similar runtime as the condition check expense gets compensated by fewer updates of \edgemlp. Detailed comparisons of conditional updates in large graphs ($|\gE|\ge 1M$) are included in the Table~\ref{tab:largescaleruntime}.   
\end{enumerate}







\begin{table}[!htbp]
\caption{Combination of \edgemlp, \gnn, Conditional update and $L_\mathrm{cons}.$}
\label{tab:ablationgnn}
% \begin{wrapfigure}{c}{1.0\linewidth}
\centering
\begin{sc}
\resizebox{0.9\linewidth}{!}
{
\def\arraystretch{1.0}
\begin{tabular}{cccccc|ccc}
\toprule
% \rowcolor[HTML]{B7B7B7} 
\textbf{} & {$\mathbf{L_\mathrm{assor}}$} & {$\mathbf{L_\mathrm{cons}}$} & \textbf{\edgemlp} & \textbf{\gnn} & \textbf{Cond.} & {\textbf{SmallCora}} & {\textbf{CoraFull}} & {\textbf{johnshopkin}} \\ \midrule
1 & N & N & \cellcolor[HTML]{F4CCCC}MLP & \cellcolor[HTML]{FFF2CC}GCN & N & 73.80 $\pm$ 0.67 & 61.78 $\pm$ 0.20 & 66.12 $\pm$ 1.38 \\
2 & Y & N & \cellcolor[HTML]{F4CCCC}MLP & \cellcolor[HTML]{FFF2CC}GCN & N & 74.88 $\pm$ 0.15 & 63.99 $\pm$ 0.24 & 66.18 $\pm$ 1.05 \\
3 & Y & N & \cellcolor[HTML]{F4CCCC}MLP & \cellcolor[HTML]{FFF2CC}GCN & Y & 75.82 $\pm$ 0.46 & 64.07 $\pm$ 0.31 & 66.87 $\pm$ 0.93 \\
4 & Y & Y & \cellcolor[HTML]{F4CCCC}MLP & \cellcolor[HTML]{FFF2CC}GCN & Y & 76.58 $\pm$ 0.47 & 65.33 $\pm$ 0.28 & 69.25 $\pm$ 0.76 \\
5 & Y & N & \cellcolor[HTML]{D0E0E3}GCN & \cellcolor[HTML]{FFF2CC}GCN & Y & 75.80 $\pm$ 0.77 & 65.66 $\pm$ 0.14 & 71.06 $\pm$ 0.32 \\
\rowcolor[HTML]{D9D9D9} 
6 & Y & Y & \cellcolor[HTML]{D0E0E3}GCN & \cellcolor[HTML]{FFF2CC}GCN & Y & 77.50 $\pm$ 0.62 & \textbf{66.56 $\pm$ 0.22} & 70.79 $\pm$ 0.18 \\
7 & Y & N & \cellcolor[HTML]{CFE2F3}GSAGE & \cellcolor[HTML]{FFF2CC}GCN & Y & 75.82 $\pm$ 0.44 & 63.70 $\pm$ 0.09 & 67.53 $\pm$ 0.80 \\
8 & Y & Y & \cellcolor[HTML]{CFE2F3}GSAGE & \cellcolor[HTML]{FFF2CC}GCN & Y & 77.48 $\pm$ 0.61 & 65.12 $\pm$ 0.11 & 68.63 $\pm$ 0.66 \\
9 & Y & N & \cellcolor[HTML]{F4CCCC}MLP & \cellcolor[HTML]{D9EAD3}GAT & Y & 77.72 $\pm$ 1.63 & 66.40 $\pm$ 0.08 & 67.92 $\pm$ 0.73 \\
10 & Y & Y & \cellcolor[HTML]{F4CCCC}MLP & \cellcolor[HTML]{D9EAD3}GAT & Y & 75.78 $\pm$ 3.22 & 66.46 $\pm$ 0.16 & 68.17 $\pm$ 0.33 \\
\rowcolor[HTML]{D9D9D9} 
11 & Y & N & \cellcolor[HTML]{D0E0E3}GCN & \cellcolor[HTML]{D9EAD3}GAT & Y & \textbf{78.18 $\pm$ 0.74} & 66.33 $\pm$ 0.20 & \textbf{71.97 $\pm$ 0.59} \\
12 & Y & Y & \cellcolor[HTML]{D0E0E3}GCN & \cellcolor[HTML]{D9EAD3}GAT & Y & 76.94 $\pm$ 2.76 & 66.39 $\pm$ 0.18 & 71.00 $\pm$ 0.96 \\
13 & Y & N & \cellcolor[HTML]{CFE2F3}GSAGE & \cellcolor[HTML]{D9EAD3}GAT & Y & 77.98 $\pm$ 0.79 & 66.38 $\pm$ 0.23 & 69.29 $\pm$ 1.56 \\
14 & Y & Y & \cellcolor[HTML]{CFE2F3}GSAGE & \cellcolor[HTML]{D9EAD3}GAT & Y & 75.74 $\pm$ 2.02 & 66.41 $\pm$ 0.25 & 68.82 $\pm$ 0.24\\\bottomrule
\end{tabular}
}
\end{sc}
\end{table}
% \end{wrapfigure}




\subsubsection{\sgs with other GNN modules}
\label{app:othergnn}
The sampled sparse subgraphs from \edgemlp can be fed into any downstream GNNs and demonstrate a couple of variants of \sgs. Chebnet from Chebyshev~\cite{he2022convolutional}, Graph Attention Network (GAT)~\cite{velivckovic2017graph}, Graph Isomorphic Network (GIN)~\cite{xu2018powerful}, Graph Convolutional Network (GCN)~\cite{kipf2016semi} are some of the GNNs used for demonstration. 

Fig.~\ref{fig:sparsityvsgnn} shows the performance of these GNNs on homophilic and heterophilic datasets. \texttt{SGS-GCN} and \texttt{SGS-GAT} are two best performing models.


%%%%%%%%%%%%%%%%%%%%%%%%%%%%%%%%%%%%%%%%%%%%
\begin{figure}[!htbp]
\centering
% \includegraphics[width=0.5\linewidth]{Figures/SparsityvsGNN.png}
\includegraphics[width=0.6\linewidth]{Figures/SGS-differentgnn.pdf}

\caption{Performance of \sgs with different GNN modules using $20\%$ edges.}
\label{fig:sparsityvsgnn}
\end{figure}
%%%%%%%%%%%%%%%%%%%%%%%%%%%%%%%%%%%%%%%%%%%%


% \subsection{Additional Ablation studies}
% \label{app:moreablation}

\subsection{Impact of  $p_\mathrm{prior}$, Normalization \& Sampling schemes, and Ensembling on \sgs}
\label{app:parameters}

Table~\ref{tab:ablation} highlights the impact of the following components: 

% Please add the following required packages to your document preamble:
% \usepackage{booktabs}
\begin{table}[t]
\caption{Ablation Studies different components of \sgs.}
\label{tab:ablation}
% \begin{wrapfigure}{c}{1.0\textwidth}
\centering
\begin{sc}
\resizebox{1.0\linewidth}{!}
{
\def\arraystretch{1.0}
\begin{tabular}{@{}cccccc|cccccc@{}}
\toprule
\textbf{Case} & \textbf{Prior} & \textbf{Norm.} & \textbf{Sampl.}  & \multicolumn{1}{c|}{\textbf{Ensem.}} & \textbf{SmallCora} & \textbf{Cora\_ML} & \textbf{CiteSeer} & \textbf{Squirrel} & \textbf{johnshopkins55} & \textbf{Roman-empire} \\ \midrule
1 & N  & Sum & Mult  & \multicolumn{1}{c|}{N} & 69.30 $\pm$ 1.20 & 81.05 $\pm$ 0.74 & 82.84 $\pm$ 0.47 & 48.90 $\pm$ 1.06 & 63.86 $\pm$ 0.58 & 63.27 $\pm$ 0.31 \\
2 & N  & Sum & Mult  & \multicolumn{1}{c|}{Y} & 72.84 $\pm$ 0.91 & 82.92 $\pm$ 0.73 & \textbf{87.42 $\pm$ 0.42} & 46.30 $\pm$ 1.18 & 65.14 $\pm$ 1.14 & \textbf{64.31 $\pm$ 0.13} \\
% 3 & N  & Sum & Mult & $L_\mathrm{a*}$ & \multicolumn{1}{c|}{Y} & 73.48 $\pm$ 1.11 & \textbf{85.01 $\pm$ 0.33} & 86.43 $\pm$ 0.23 & 49.68 $\pm$ 0.73 & \textbf{74.73 $\pm$ 0.51} & 63.20 $\pm$ 0.21 \\
%4 & N  & Sum & Mult & $L_\mathrm{a*}$, $L_\mathrm{c*}$ & \multicolumn{1}{c|}{Y} & 75.86 $\pm$ 0.74 & 84.82 $\pm$ 0.24 & 86.55 $\pm$ 0.33 & 48.03 $\pm$ 0.79 & 72.08 $\pm$ 1.16 & 63.16 $\pm$ 0.30 \\
% 5 & Y & 0 & Sum & Mult & $L_\mathrm{a*}$, $L_\mathrm{c*}$ & \multicolumn{1}{c|}{Y} & 74.82 $\pm$ 0.64 & 84.21 $\pm$ 0.60 & 86.55 $\pm$ 0.25 & 47.53 $\pm$ 0.32 & 68.44 $\pm$ 0.46 & 63.06 $\pm$ 0.11 \\
% 6 & Y & 1 & Sum & Mult & $L_\mathrm{a*}$, $L_\mathrm{c*}$ & \multicolumn{1}{c|}{Y} & 75.54 $\pm$ 0.23 & 84.01 $\pm$ 0.74 & 86.34 $\pm$ 0.21 & 48.63 $\pm$ 0.44 & 70.77 $\pm$ 0.40 & 63.27 $\pm$ 0.12 \\
3 & Y  & Sum & Mult & \multicolumn{1}{c|}{Y} & 75.54 $\pm$ 0.41 & \textbf{83.87 $\pm$ 0.69} & 86.31 $\pm$ 0.26 & 47.97 $\pm$ 0.60 & 72.68 $\pm$ 0.51 & 62.88 $\pm$ 0.19 \\
4 & Y & Softmax & Mult & \multicolumn{1}{c|}{Y} & 75.44 $\pm$ 0.51 & 83.81 $\pm$ 0.72 & 86.31 $\pm$ 0.26 & 47.90 $\pm$ 0.42 & \textbf{72.97 $\pm$ 0.20} & 62.98 $\pm$ 0.16 \\
5 & Y & Gumbel & TopK & \multicolumn{1}{c|}{Y} & \textbf{76.24 $\pm$ 0.43} & 83.36 $\pm$ 0.34 & 86.44 $\pm$ 0.16 & \textbf{51.49 $\pm$ 0.72} & 71.83 $\pm$ 1.00 & 63.00 $\pm$ 0.11 \\ \midrule
\multicolumn{11}{l}{\textbf{Prior:} Use of prior, \textbf{Sum:} Sum-Normalization, \textbf{Softmax:} \textit{Softmax} with temperature annealing}\\
\multicolumn{11}{l}{\textbf{Mult:} \textit{Multinonmial} Sampling, \textbf{Gumbel:} \textit{Gumbel-Softmax} with \textit{TopK}}\\
\end{tabular}
}
\end{sc}
\end{table}
% \end{wrapfigure}

\begin{enumerate}
    \item Prior $p_\mathrm{prior}$: Cases 2-3 show that augmenting the learned probability distribution $\tilde{p}$ with prior $p_\mathrm{prior}$ can benefit some datasets. We have also conducted an in-depth comparison between the distributions $\tilde{p}$ and augmented distribution $\tilde{p}_a$. Figure~\ref{fig:augment_p} shows that there $\tilde{p}_a$ is left skewed whereas $\tilde{p}$ is not. Since rare edges still get some negligible mass, it is possible for $\tilde{\gG}$ constructed from $\tilde{p}_a$ to retain some bridge edges from these tails, if there are any. 
    
\begin{figure}[!htbp]
    \centering
    %\subfigure{\includegraphics[width=0.48\linewidth]{Figures/SparsityvsHomophily.png}
    \subfigure{\includegraphics[width=0.4\linewidth]{Figures/SGS-learnedp.png}
    \label{subfig:learnedp}} 
    %\hfill
    % \subfigure{\includegraphics[width=0.48\linewidth]{Figures/SparsityvsAccuracy2.png}
    %\hfill     
     \subfigure{\includegraphics[width=0.4\linewidth]{Figures/SGS-learnedp_a.png}
     \label{subfig:priorpa}} 
     \subfigure{\includegraphics[width=0.4\linewidth]{Figures/SGS-prior.png}
     \label{subfig:priorp}}    
    \caption{The learned probability distribution $\tilde{p}$ (top-left), augmented distribution $\tilde{p}_a$(top-right) and fixed prior $p_\mathrm{prior}$ (bottom). Augmentation puts negligible mass on some rare yet critical edges in the left tail of $\tilde{p}_a$.}
    \label{fig:augment_p}
\end{figure}
    \item Normalization and Sampling: We considered three normalization and sampling techniques. i) sum-normalization with multinomial sampling, ii) softmax-normalization with temperature with multinomial sampling, and iii) Gumbel softmax normalization with Topk selection. Cases 3-5 show that each of these techniques can improve results in certain datasets, and thus, it is difficult to nominate a single one as best. However, in our experiments, we opted for multinomial sampling with softmax temperature annealing for training to encourage exploration in early iterations.

    \item Ensemble subgraphs during inference: Case 2 demonstrates that using multiple subgraphs for ensemble prediction yields better results than a single subgraph (Case 1).    
\end{enumerate}

 % The addition of assortative loss $L_\mathrm{assor}$ in Case 3 enhances performance on heterophilic graphs by promoting homophily in sampled subgraphs. %Case 4 shows that incorporating consistency loss $L_\mathrm{cons}$ benefits certain graphs. 
%While our base training method updates \edgemlp at each iteration, conditional update with a prior distribution speeds up results by reducing the search space. Cases 5-7 indicate that hop size at \edgemlp influences performance. 

% % Please add the following required packages to your document preamble:
% \usepackage{booktabs}
\begin{table}[t]
\caption{Ablation Studies different components of \sgs.}
\label{tab:ablation}
% \begin{wrapfigure}{c}{1.0\textwidth}
\centering
\begin{sc}
\resizebox{1.0\linewidth}{!}
{
\def\arraystretch{1.0}
\begin{tabular}{@{}cccccc|cccccc@{}}
\toprule
\textbf{Case} & \textbf{Prior} & \textbf{Norm.} & \textbf{Sampl.}  & \multicolumn{1}{c|}{\textbf{Ensem.}} & \textbf{SmallCora} & \textbf{Cora\_ML} & \textbf{CiteSeer} & \textbf{Squirrel} & \textbf{johnshopkins55} & \textbf{Roman-empire} \\ \midrule
1 & N  & Sum & Mult  & \multicolumn{1}{c|}{N} & 69.30 $\pm$ 1.20 & 81.05 $\pm$ 0.74 & 82.84 $\pm$ 0.47 & 48.90 $\pm$ 1.06 & 63.86 $\pm$ 0.58 & 63.27 $\pm$ 0.31 \\
2 & N  & Sum & Mult  & \multicolumn{1}{c|}{Y} & 72.84 $\pm$ 0.91 & 82.92 $\pm$ 0.73 & \textbf{87.42 $\pm$ 0.42} & 46.30 $\pm$ 1.18 & 65.14 $\pm$ 1.14 & \textbf{64.31 $\pm$ 0.13} \\
% 3 & N  & Sum & Mult & $L_\mathrm{a*}$ & \multicolumn{1}{c|}{Y} & 73.48 $\pm$ 1.11 & \textbf{85.01 $\pm$ 0.33} & 86.43 $\pm$ 0.23 & 49.68 $\pm$ 0.73 & \textbf{74.73 $\pm$ 0.51} & 63.20 $\pm$ 0.21 \\
%4 & N  & Sum & Mult & $L_\mathrm{a*}$, $L_\mathrm{c*}$ & \multicolumn{1}{c|}{Y} & 75.86 $\pm$ 0.74 & 84.82 $\pm$ 0.24 & 86.55 $\pm$ 0.33 & 48.03 $\pm$ 0.79 & 72.08 $\pm$ 1.16 & 63.16 $\pm$ 0.30 \\
% 5 & Y & 0 & Sum & Mult & $L_\mathrm{a*}$, $L_\mathrm{c*}$ & \multicolumn{1}{c|}{Y} & 74.82 $\pm$ 0.64 & 84.21 $\pm$ 0.60 & 86.55 $\pm$ 0.25 & 47.53 $\pm$ 0.32 & 68.44 $\pm$ 0.46 & 63.06 $\pm$ 0.11 \\
% 6 & Y & 1 & Sum & Mult & $L_\mathrm{a*}$, $L_\mathrm{c*}$ & \multicolumn{1}{c|}{Y} & 75.54 $\pm$ 0.23 & 84.01 $\pm$ 0.74 & 86.34 $\pm$ 0.21 & 48.63 $\pm$ 0.44 & 70.77 $\pm$ 0.40 & 63.27 $\pm$ 0.12 \\
3 & Y  & Sum & Mult & \multicolumn{1}{c|}{Y} & 75.54 $\pm$ 0.41 & \textbf{83.87 $\pm$ 0.69} & 86.31 $\pm$ 0.26 & 47.97 $\pm$ 0.60 & 72.68 $\pm$ 0.51 & 62.88 $\pm$ 0.19 \\
4 & Y & Softmax & Mult & \multicolumn{1}{c|}{Y} & 75.44 $\pm$ 0.51 & 83.81 $\pm$ 0.72 & 86.31 $\pm$ 0.26 & 47.90 $\pm$ 0.42 & \textbf{72.97 $\pm$ 0.20} & 62.98 $\pm$ 0.16 \\
5 & Y & Gumbel & TopK & \multicolumn{1}{c|}{Y} & \textbf{76.24 $\pm$ 0.43} & 83.36 $\pm$ 0.34 & 86.44 $\pm$ 0.16 & \textbf{51.49 $\pm$ 0.72} & 71.83 $\pm$ 1.00 & 63.00 $\pm$ 0.11 \\ \midrule
\multicolumn{11}{l}{\textbf{Prior:} Use of prior, \textbf{Sum:} Sum-Normalization, \textbf{Softmax:} \textit{Softmax} with temperature annealing}\\
\multicolumn{11}{l}{\textbf{Mult:} \textit{Multinonmial} Sampling, \textbf{Gumbel:} \textit{Gumbel-Softmax} with \textit{TopK}}\\
\end{tabular}
}
\end{sc}
\end{table}
% \end{wrapfigure}




\clearpage
\subsection{Choosing Values for Regularizer coefficient \(\alpha_3\) and Parameter \(\lambda\)}
\label{app:gridsearch}
Recall that \sgs computes the total loss at each epoch as
\[
\gL = \alpha_1\gL_{CE}+\alpha_2\gL_\mathrm{assor}+\alpha_3\gL_\mathrm{cons},
\]
where $0 \leq \alpha_1,\alpha_2,\alpha_3 \leq 1$ are regularizer coefficients corresponding to the cross-entropy loss $\gL_{CE}$, assortativity loss $L_\mathrm{assor}$ and  consistency loss $\gL_\mathrm{cons}$ respectively. 

Also recall that, when we use a prior probability distribution, the learned distribution values of $\tilde{p}$ are weighted through $\lambda$ in
$\Tilde{p} = \lambda\Tilde{p}+(1-\lambda) p_\mathrm{prior}$

To avoid numerous combinations of values of three coefficients + the parameter $\lambda$, we have fixed $\alpha_1 = 1$, and $\alpha_2 = 1$. In the following, we investigate the performance of \sgs with different values for $\alpha_3$ and $\lambda$.

Fig.~\ref{fig:consbias} shows a grid search for different combinations of $\lambda$ and $\alpha_3$. As per our observation, the recommended values are $\lambda \in [0.3, 0.7], \alpha_3=0.5$.
%%%%%%%%%%%%%%%%%%%%%%%%%%%%%%%%%%%%%%%%%%%%
\begin{figure}[h]
\centering
\includegraphics[width=0.4\linewidth]{Figures/SGS-biasgrid.pdf}
\caption{Grid search for the parameter $\lambda$ for prior, and consistency loss, $\alpha_3$ (Cora dataset).}
\label{fig:consbias}
\end{figure}
%%%%%%%%%%%%%%%%%%%%%%%%%%%%%%%%%%%%%%%%%%%%




