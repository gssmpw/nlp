\section{Dataset Description}
\label{app:dataset}
\begin{table}[!htbp]
\caption{Additional details of the dataset are provided. $\gH_\mathrm{adj}$ refers to adjusted homophily. $d$ corresponds to the average degree, $C$ number of classes, and $F$ is the feature dimension. \textit{Tr.} is the training label rate.}
\label{tab:datasetdescription}
% \begin{wrapfigure}{c}{1.0\textwidth}
\centering
\begin{sc}
\resizebox{1.0\linewidth}{!}
{
\def\arraystretch{1.0}
\begin{tabular}{@{}crrrccrcccl@{}}
\toprule
\textbf{Dataset} &
  $|\gV|$ &
  $|\gE|$ &
  \textbf{$d$} &
  \textbf{$\gH_\mathrm{adj}$} &
  \textbf{$C$} &
  \textbf{$F$} &
  \textbf{Tr.} &
  \textbf{Self-Loop} &
  \textbf{Isolated} &
  \textbf{Context} \\ \midrule
Cornell        & 183       & 557         & 3.04   & -0.42 & 5  & 1703 & 0.48 & TRUE  & FALSE & Web Pages           \\
Texas          & 183       & 574         & 3.14   & -0.26 & 5  & 1703 & 0.48 & TRUE  & FALSE & Web Pages           \\
Wisconsin      & 251       & 916         & 3.65   & -0.20 & 5  & 1703 & 0.48 & TRUE  & FALSE & Web Pages           \\
reed98         & 962       & 37,624      & 39.11  & -0.10 & 3  & 1001 & 0.6  & FALSE & FALSE & Social Network      \\
amherst41      & 2,235     & 181,908     & 81.39  & -0.07 & 3  & 1193 & 0.6  & FALSE & FALSE & Social Network      \\
penn94         & 41,554    & 2,724,458   & 65.56  & -0.06 & 2  & 4814 & 0.47 & FALSE & FALSE & Social Network      \\
Roman-empire   & 22,662    & 65,854      & 2.91   & -0.05 & 18 & 300  & 0.5  & FALSE & FALSE & Wikipedia           \\
cornell5       & 18,660    & 1,581,554   & 84.76  & -0.04 & 3  & 4735 & 0.6  & FALSE & FALSE & Web pages           \\
Squirrel       & 5,201     & 396,846     & 76.30  & -0.01 & 5  & 2345 & 0.48 & TRUE  & FALSE & Wikipedia           \\
johnshopkins55 & 5,180     & 373,172     & 72.04  & 0.00  & 3  & 2406 & 0.6  & FALSE & FALSE & Web Pages           \\
Actor          & 7,600     & 53,411      & 7.03   & 0.01  & 5  & 932  & 0.48 & TRUE  & FALSE & Actors in Movies    \\
Minesweeper    & 10,000    & 78,804      & 7.88   & 0.01  & 2  & 7    & 0.5  & FALSE & FALSE & Synthetic           \\
Questions      & 48,921    & 307,080     & 6.28   & 0.02  & 2  & 301  & 0.5  & FALSE & FALSE & Yandex Q            \\
Chameleon      & 2,277     & 62,792      & 27.58  & 0.03  & 5  & 2581 & 0.48 & TRUE  & FALSE & Wiki Pages          \\
Tolokers       & 11,758    & 1,038,000   & 88.28  & 0.09  & 2  & 10   & 0.5  & FALSE & FALSE & Toloka Platform     \\
Amazon-ratings & 24,492    & 186,100     & 7.60   & 0.14  & 5  & 556  & 0.5  & FALSE & FALSE & Co-purchase network \\
genius         & 421,961   & 1,845,736   & 4.37   & 0.17  & 2  & 12   & 0.6  & FALSE & TRUE  & Social Network      \\
pokec          & 1,632,803 & 44,603,928  & 27.32  & 0.42  & 3  & 65   & 0.6  & FALSE & FALSE & Social Network      \\
arxiv-year     & 169,343   & 2,315,598   & 13.67  & 0.26  & 5  & 128  & 0.6  & FALSE & FALSE & Citation            \\
snap-patents   & 2,923,922 & 27,945,092  & 9.56   & 0.21  & 5  & 269  & 0.6  & TRUE  & TRUE  & Citation            \\
ogbn-proteins  & 132,534   & 79,122,504  & 597.00 & 0.05  & 94 & 8    & 0.2  & FALSE & FALSE & Protein Network     \\\midrule \midrule
Cora           & 19,793    & 126,842     & 6.41   & 0.56  & 70 & 8710 & 0.2  & FALSE & FALSE & Citation Network    \\
DBLP           & 17,716    & 105,734     & 5.97   & 0.68  & 4  & 1639 & 0.2  & FALSE & FALSE & Citation Network    \\
Computers      & 13,752    & 491,722     & 35.76  & 0.68  & 10 & 767  & 0.6  & FALSE & TRUE  & Co-purchase Network \\
PubMed         & 19,717    & 88,648      & 4.50   & 0.69  & 3  & 500  & 0.2  & FALSE & FALSE & Social Network      \\
Cora\_ML        & 2,995     & 16,316      & 5.45   & 0.75  & 7  & 2879 & 0.2  & FALSE & FALSE & Citation Network    \\
SmallCora      & 2,708     & 10,556      & 3.90   & 0.77  & 7  & 1433 & 0.05 & FALSE & FALSE & Citation Network    \\
CS             & 18,333    & 163,788     & 8.93   & 0.78  & 15 & 6805 & 0.2  & FALSE & FALSE & Co-author Network   \\
Photo          & 7,650     & 238,162     & 31.13  & 0.79  & 8  & 745  & 0.2  & FALSE & TRUE  & Co-purchase Network \\
Physics        & 34,493    & 495,924     & 14.38  & 0.87  & 5  & 8415 & 0.2  & FALSE & FALSE & Co-author Network   \\
CiteSeer       & 4,230     & 10,674      & 2.52   & 0.94  & 6  & 602  & 0.2  & FALSE & FALSE & Citation Network    \\
wiki           & 11,701    & 431,726     & 36.90  & 0.58  & 10 & 300  & 0.99 & TRUE  & TRUE  & Wikipedia           \\
Reddit         & 232,965   & 114,615,892 & 491.99 & 0.74  & 41 & 602  & 0.66 & FALSE & FALSE & Social Network      \\ \bottomrule
\end{tabular}
}
\end{sc}
\end{table}
% \end{wrapfigure}
Table~\ref{tab:datasetdescription} shows the details of the characteristics of the graph datasets, including the splits used throughout the experimentation.

Along with synthetic dataset, for heterophily, we used, 
\textit{Cornell, Texas}, \textit{Wisconsin} from the \textit{WebKB}~\cite{pei2020geom}; \textit{Chameleon}, \textit{Squirrel} ~\cite{rozemberczki2021multi}; \textit{Actor} ~\cite{pei2020geom}; \textit{Wiki, ArXiv-year, Snap-Patents, Penn94, Pokec, Genius, reed98, amherst41, cornell5}, and \textit{Yelp}~\cite{lim2021large}. 
We also experiment on some recent benchmark datasets, \textit{Roman-empire, Amazon-ratings, Minesweeper, Tolokers}, and \textit{Questions} from~\cite{platonov2023critical}.

For homophily, we used
\textit{Cora}~\cite{sen2008collective}; \textit{Citeseer}~\cite{giles1998citeseer}; \textit{pubmed} \cite{namata2012query}; \textit{Coauthor-cs}, \textit{Coauthor-physics}~\cite{shchur2018pitfalls}; \textit{Amazon-computers},  \textit{Amazon-photo} ~\cite{shchur2018pitfalls}; \textit{Reddit}~\cite{hamilton2017inductive}; and, \textit{DBLP}~\cite{fu2020magnn}. 

\noindent\textbf{Heterophily Characterization.} The term \emph{homophily} in a graph describes the likelihood that nodes with the same labels are neighbors. Although there are several ways to measure homophily, three commonly used measures are {\em homophily of the nodes} ($\gH_{n}$), {\em homophily of the edges} ($\gH_{e}$), and {\em adjusted homophily} ($\gH_\mathrm{adj}$).
The {\em node homophily}~\cite{pei2020geom} is defined as,  
\begin{align}
\gH_{n} & = \frac{1}{|\gV|} \sum_{u\in \gV}\frac{| \{v\in \gN(u) : y_v = y_u\}|}{|\gN(u)|}.
\end{align}
The {\em edge homophily}~\cite{zhu2020beyond} of a graph is,

\begin{equation}
    \gH_{e} = \frac{|\{(u,v)\in \gE : y_u = y_v\}|}{|\gE|}. 
\end{equation}

The 
{\em adjusted homophily}~\cite{platonov2022characterizing} is defined as,
\begin{equation}
    \gH_\mathrm{adj} = \frac{\gH_{e}-\sum_{k=1}^{c} D_k^2/(2|\gE|^2)}{1-\sum_{k=1}^c D_k^2/2|\gE|^2}. 
\end{equation}

Here, $D_k = \sum_{v:y_v=k}d_v$ denote the sum of degrees of the nodes belonging to class $k$. 

The values of the node homophily and the edge homophily range from $0$ to $1$, and the adjusted homophily ranges from $-\frac{1}{3}$ to $+1$ (Proposition 1 in~\cite{platonov2022characterizing}). 
Among these measures, adjusted homophily considers the class imbalance. Thus, this work classifies graphs with adjusted homophily, $\gH_\mathrm {adj} \le 0.50$ as heterophilic.

% \paragraph{Dataset used in experiments.} 

\clearpage