%%
%% This is file `sample-manuscript.tex',
%% generated with the docstrip utility.
%%
%% The original source files were:
%%
%% samples.dtx  (with options: `manuscript')
%% 
%% IMPORTANT NOTICE:
%% 
%% For the copyright see the source file.
%% 
%% Any modified versions of this file must be renamed
%% with new filenames distinct from sample-manuscript.tex.
%% 
%% For distribution of the original source see the terms
%% for copying and modification in the file samples.dtx.
%% 
%% This generated file may be distributed as long as the
%% original source files, as listed above, are part of the
%% same distribution. (The sources need not necessarily be
%% in the same archive or directory.)
%%
%% Commands for TeXCount
%TC:macro \cite [option:text,text]
%TC:macro \citep [option:text,text]
%TC:macro \citet [option:text,text]
%TC:envir table 0 1
%TC:envir table* 0 1
%TC:envir tabular [ignore] word
%TC:envir displaymath 0 word
%TC:envir math 0 word
%TC:envir comment 0 0
%%
%%
%% The first command in your LaTeX source must be the \documentclass command.
%%%% Small single column format, used for CIE, CSUR, DTRAP, JACM, JDIQ, JEA, JERIC, JETC, PACMCGIT, TAAS, TACCESS, TACO, TALG, TALLIP (formerly TALIP), TCPS, TDSCI, TEAC, TECS, TELO, THRI, TIIS, TIOT, TISSEC, TIST, TKDD, TMIS, TOCE, TOCHI, TOCL, TOCS, TOCT, TODAES, TODS, TOIS, TOIT, TOMACS, TOMM (formerly TOMCCAP), TOMPECS, TOMS, TOPC, TOPLAS, TOPS, TOS, TOSEM, TOSN, TQC, TRETS, TSAS, TSC, TSLP, TWEB.
% \documentclass[acmsmall]{acmart}

%%%% Large single column format, used for IMWUT, JOCCH, PACMPL, POMACS, TAP, PACMHCI
% \documentclass[acmlarge,screen]{acmart}

%%%% Large double column format, used for TOG
% \documentclass[acmtog, authorversion]{acmart}

%%%% Generic manuscript mode, required for submission
%%%% and peer review
% \documentclass[manuscript,screen,review]{acmart}
%% Fonts used in the template cannot be substituted; margin 
%% adjustments are not allowed.

% \documentclass[sigplan,authordraft]{acmart}
\documentclass[letterpaper,twocolumn,10pt]{article}
\usepackage{usenix}
\usepackage{titling}
\usepackage{tikz}
\usepackage{multirow} 
\usepackage{pifont}
\usepackage{amsmath}
\usepackage{amssymb}
\usepackage{url}
\usepackage{subcaption}
\usepackage{graphicx}
\usepackage{booktabs}
\usepackage{algpseudocode}
\usepackage[linesnumbered,ruled,vlined]{algorithm2e}
\usepackage{tabularx}
\usepackage{authblk}
% \usepackage{draftwatermark}
% \usepackage{titling}
\microtypecontext{spacing=nonfrench}

% \SetWatermarkText{Unpublished working draft \\ Not for distribution}
% \SetWatermarkText{}
% \SetWatermarkScale{0.4}
% \SetWatermarkLightness{0.8}

\usepackage{enumitem} % Add this line to your preamble
\usepackage{pifont}
\usepackage{placeins}
% Define commands for tick and cross
\newcommand{\tick}{\ding{51}} % Tick symbol ✔
\newcommand{\cross}{\ding{55}} % Cross symbol ✘
\newcommand{\mixed}{\ding{51}\kern-0.62em\ding{55}} % Mixed symbol (tick and cross overlay)

\usepackage{tikz}
\newcommand*\circled[1]{%
  \scalebox{0.78}{\begin{tikzpicture}[baseline=-3pt]
    \node[draw,circle,inner sep=0.5pt, fill=black] {\textcolor{white}{\textsf{\textbf{#1}}}};
  \end{tikzpicture}}}

\usepackage{xcolor}
\newcommand{\yeqi}[1]{\textcolor{orange}{Yeqi: #1}}
\newcommand{\cj}[1]{\textcolor{blue}{Congjie: #1}}

% reduce space below title
 \usepackage{etoolbox}
 
 \makeatletter
 \patchcmd{\maketitle}
 	{\@maketitle}
 	{\vspace{-7em}\@maketitle\vspace{-5em}}% change the value as needed
 	{}
 	{}
 \makeatother

\newcommand{\todo}[1]{{\textcolor{red}{[~TODO:~#1~]}}}
\newcommand{\ncite}[1]{{\textcolor{blue}{[~Cite]}}}


% \makeatletter
% \def\UrlAlphabet{%
%       \do\a\do\b\do\c\do\d\do\e\do\f\do\g\do\h\do\i\do\j%
%       \do\k\do\l\do\m\do\n\do\o\do\p\do\q\do\r\do\s\do\t%
%       \do\u\do\v\do\w\do\x\do\y\do\z\do\A\do\B\do\C\do\D%
%       \do\E\do\F\do\G\do\H\do\I\do\J\do\K\do\L\do\M\do\N%
%       \do\O\do\P\do\Q\do\R\do\S\do\T\do\U\do\V\do\W\do\X%
%       \do\Y\do\Z}
% \def\UrlDigits{\do\1\do\2\do\3\do\4\do\5\do\6\do\7\do\8\do\9\do\0}
% \g@addto@macro{\UrlBreaks}{\UrlOrds}
% \g@addto@macro{\UrlBreaks}{\UrlAlphabet}
% \g@addto@macro{\UrlBreaks}{\UrlDigits}
% \makeatother

\algdef{SE}[LOOP]{Loop}{EndLoop}[1]{\textbf{loop} #1}{}

\setlength{\droptitle}{-2cm}

\begin{document}

\newcommand{\meshtp}{MeshTP\xspace}
\newcommand{\meshpp}{MeshPP\xspace}
\newcommand{\gemm}{MeshGEMM\xspace}
\newcommand{\gemv}{MeshGEMV\xspace}
\newcommand{\sys}{WaferLLM\xspace}


\usepackage{booktabs}
\usepackage{multirow}
\usepackage{graphicx}
\usepackage{amsmath}
\usepackage{xspace}
\DeclareMathOperator*{\argmax}{arg\,max}
\DeclareMathOperator*{\stmx}{Softmax}
\allowdisplaybreaks
 
\newcommand{\ours}{\textsc{suPreMe}\xspace}
\newcommand{\ITC}{image-text consistency\xspace}
\newcommand{\BPG}{biased prompts generation\xspace}
\usepackage{colortbl}

\usepackage{amsthm}

\newtheorem*{remark}{Remark}
\newtheorem{theorem}{Theorem}
\newtheorem{assumption}{Assumption}

\usepackage{makecell}

\usepackage{amsbsy}

%%
%% The "title" command has an optional parameter,
%% allowing the author to define a "short title" to be used in page headers.
\title{\sys: A Wafer-Scale LLM Inference System}
\date{}


% \author{\#203}
% \author[1]{Congjie He\thanks{congjie.he@ed.ac.uk}}
\author{
    Congjie He\textsuperscript{1}, 
    Yeqi Huang\textsuperscript{1}, 
    Pei Mu\textsuperscript{1}, 
    Ziming Miao\textsuperscript{2}, 
    Jilong Xue\textsuperscript{2}, 
    Lingxiao Ma\textsuperscript{2}, 
    Fan Yang\textsuperscript{2}, 
    Luo Mai\textsuperscript{1}
}

\affil{\textsuperscript{1}University of Edinburgh \hspace{1cm} \textsuperscript{2}Microsoft Research}

\maketitle

\begin{abstract}
Emerging AI accelerators increasingly adopt wafer-scale manufacturing technologies, integrating hundreds of thousands of AI cores in a mesh-based architecture with large distributed on-chip memory (tens of GB in total) and ultra-high on-chip memory bandwidth (tens of PB/s). However, current LLM inference systems, optimized for shared memory architectures like GPUs, fail to fully exploit these accelerators.

We introduce \sys, the first wafer-scale LLM inference system. \sys is guided by a novel PLMR model (pronounced as "Plummer") that captures the unique hardware characteristics of wafer-scale architectures. Leveraging this model, \sys pioneers wafer-scale LLM parallelism, optimizing the utilization of hundreds of thousands of on-chip cores. It also introduces MeshGEMM and MeshGEMV, the first GEMM and GEMV implementations designed to scale effectively on wafer-scale accelerators.

Evaluations show that \sys achieves 200$\times$ better wafer-scale accelerator utilization than state-of-the-art systems. On a commodity wafer-scale accelerator, \sys delivers 606$\times$ faster and 22$\times$ more energy-efficient GEMV compared to an advanced GPU. For LLMs, based on 16-bit data type, \sys achieves 2700 toks/sec/req decode speed on Llama3-8B model and 840 toks/sec/req decode speed on Qwen2-72B model, which enables 39$\times$ faster decoding with 1.7$\times$ better energy efficiency. We anticipate these numbers will grow significantly as wafer-scale AI models, software, and hardware continue to mature.

\end{abstract}

\section{Introduction}
\label{sec:introduction}
The business processes of organizations are experiencing ever-increasing complexity due to the large amount of data, high number of users, and high-tech devices involved \cite{martin2021pmopportunitieschallenges, beerepoot2023biggestbpmproblems}. This complexity may cause business processes to deviate from normal control flow due to unforeseen and disruptive anomalies \cite{adams2023proceddsriftdetection}. These control-flow anomalies manifest as unknown, skipped, and wrongly-ordered activities in the traces of event logs monitored from the execution of business processes \cite{ko2023adsystematicreview}. For the sake of clarity, let us consider an illustrative example of such anomalies. Figure \ref{FP_ANOMALIES} shows a so-called event log footprint, which captures the control flow relations of four activities of a hypothetical event log. In particular, this footprint captures the control-flow relations between activities \texttt{a}, \texttt{b}, \texttt{c} and \texttt{d}. These are the causal ($\rightarrow$) relation, concurrent ($\parallel$) relation, and other ($\#$) relations such as exclusivity or non-local dependency \cite{aalst2022pmhandbook}. In addition, on the right are six traces, of which five exhibit skipped, wrongly-ordered and unknown control-flow anomalies. For example, $\langle$\texttt{a b d}$\rangle$ has a skipped activity, which is \texttt{c}. Because of this skipped activity, the control-flow relation \texttt{b}$\,\#\,$\texttt{d} is violated, since \texttt{d} directly follows \texttt{b} in the anomalous trace.
\begin{figure}[!t]
\centering
\includegraphics[width=0.9\columnwidth]{images/FP_ANOMALIES.png}
\caption{An example event log footprint with six traces, of which five exhibit control-flow anomalies.}
\label{FP_ANOMALIES}
\end{figure}

\subsection{Control-flow anomaly detection}
Control-flow anomaly detection techniques aim to characterize the normal control flow from event logs and verify whether these deviations occur in new event logs \cite{ko2023adsystematicreview}. To develop control-flow anomaly detection techniques, \revision{process mining} has seen widespread adoption owing to process discovery and \revision{conformance checking}. On the one hand, process discovery is a set of algorithms that encode control-flow relations as a set of model elements and constraints according to a given modeling formalism \cite{aalst2022pmhandbook}; hereafter, we refer to the Petri net, a widespread modeling formalism. On the other hand, \revision{conformance checking} is an explainable set of algorithms that allows linking any deviations with the reference Petri net and providing the fitness measure, namely a measure of how much the Petri net fits the new event log \cite{aalst2022pmhandbook}. Many control-flow anomaly detection techniques based on \revision{conformance checking} (hereafter, \revision{conformance checking}-based techniques) use the fitness measure to determine whether an event log is anomalous \cite{bezerra2009pmad, bezerra2013adlogspais, myers2018icsadpm, pecchia2020applicationfailuresanalysispm}. 

The scientific literature also includes many \revision{conformance checking}-independent techniques for control-flow anomaly detection that combine specific types of trace encodings with machine/deep learning \cite{ko2023adsystematicreview, tavares2023pmtraceencoding}. Whereas these techniques are very effective, their explainability is challenging due to both the type of trace encoding employed and the machine/deep learning model used \cite{rawal2022trustworthyaiadvances,li2023explainablead}. Hence, in the following, we focus on the shortcomings of \revision{conformance checking}-based techniques to investigate whether it is possible to support the development of competitive control-flow anomaly detection techniques while maintaining the explainable nature of \revision{conformance checking}.
\begin{figure}[!t]
\centering
\includegraphics[width=\columnwidth]{images/HIGH_LEVEL_VIEW.png}
\caption{A high-level view of the proposed framework for combining \revision{process mining}-based feature extraction with dimensionality reduction for control-flow anomaly detection.}
\label{HIGH_LEVEL_VIEW}
\end{figure}

\subsection{Shortcomings of \revision{conformance checking}-based techniques}
Unfortunately, the detection effectiveness of \revision{conformance checking}-based techniques is affected by noisy data and low-quality Petri nets, which may be due to human errors in the modeling process or representational bias of process discovery algorithms \cite{bezerra2013adlogspais, pecchia2020applicationfailuresanalysispm, aalst2016pm}. Specifically, on the one hand, noisy data may introduce infrequent and deceptive control-flow relations that may result in inconsistent fitness measures, whereas, on the other hand, checking event logs against a low-quality Petri net could lead to an unreliable distribution of fitness measures. Nonetheless, such Petri nets can still be used as references to obtain insightful information for \revision{process mining}-based feature extraction, supporting the development of competitive and explainable \revision{conformance checking}-based techniques for control-flow anomaly detection despite the problems above. For example, a few works outline that token-based \revision{conformance checking} can be used for \revision{process mining}-based feature extraction to build tabular data and develop effective \revision{conformance checking}-based techniques for control-flow anomaly detection \cite{singh2022lapmsh, debenedictis2023dtadiiot}. However, to the best of our knowledge, the scientific literature lacks a structured proposal for \revision{process mining}-based feature extraction using the state-of-the-art \revision{conformance checking} variant, namely alignment-based \revision{conformance checking}.

\subsection{Contributions}
We propose a novel \revision{process mining}-based feature extraction approach with alignment-based \revision{conformance checking}. This variant aligns the deviating control flow with a reference Petri net; the resulting alignment can be inspected to extract additional statistics such as the number of times a given activity caused mismatches \cite{aalst2022pmhandbook}. We integrate this approach into a flexible and explainable framework for developing techniques for control-flow anomaly detection. The framework combines \revision{process mining}-based feature extraction and dimensionality reduction to handle high-dimensional feature sets, achieve detection effectiveness, and support explainability. Notably, in addition to our proposed \revision{process mining}-based feature extraction approach, the framework allows employing other approaches, enabling a fair comparison of multiple \revision{conformance checking}-based and \revision{conformance checking}-independent techniques for control-flow anomaly detection. Figure \ref{HIGH_LEVEL_VIEW} shows a high-level view of the framework. Business processes are monitored, and event logs obtained from the database of information systems. Subsequently, \revision{process mining}-based feature extraction is applied to these event logs and tabular data input to dimensionality reduction to identify control-flow anomalies. We apply several \revision{conformance checking}-based and \revision{conformance checking}-independent framework techniques to publicly available datasets, simulated data of a case study from railways, and real-world data of a case study from healthcare. We show that the framework techniques implementing our approach outperform the baseline \revision{conformance checking}-based techniques while maintaining the explainable nature of \revision{conformance checking}.

In summary, the contributions of this paper are as follows.
\begin{itemize}
    \item{
        A novel \revision{process mining}-based feature extraction approach to support the development of competitive and explainable \revision{conformance checking}-based techniques for control-flow anomaly detection.
    }
    \item{
        A flexible and explainable framework for developing techniques for control-flow anomaly detection using \revision{process mining}-based feature extraction and dimensionality reduction.
    }
    \item{
        Application to synthetic and real-world datasets of several \revision{conformance checking}-based and \revision{conformance checking}-independent framework techniques, evaluating their detection effectiveness and explainability.
    }
\end{itemize}

The rest of the paper is organized as follows.
\begin{itemize}
    \item Section \ref{sec:related_work} reviews the existing techniques for control-flow anomaly detection, categorizing them into \revision{conformance checking}-based and \revision{conformance checking}-independent techniques.
    \item Section \ref{sec:abccfe} provides the preliminaries of \revision{process mining} to establish the notation used throughout the paper, and delves into the details of the proposed \revision{process mining}-based feature extraction approach with alignment-based \revision{conformance checking}.
    \item Section \ref{sec:framework} describes the framework for developing \revision{conformance checking}-based and \revision{conformance checking}-independent techniques for control-flow anomaly detection that combine \revision{process mining}-based feature extraction and dimensionality reduction.
    \item Section \ref{sec:evaluation} presents the experiments conducted with multiple framework and baseline techniques using data from publicly available datasets and case studies.
    \item Section \ref{sec:conclusions} draws the conclusions and presents future work.
\end{itemize}

\section{Related Works}
\label{sec:background_motivation}
%\vspace{-1mm}
%3GPP employs dynamic scheduling to allocate resources to UEs in the time domain, representing the prevalent method in commercial cellular networks\,\cite{fezeu2023depth}. Our proposed solution builds upon this approach, aiming to enhance its efficacy.
%\vspace{-1mm}
%\subsection{Related Works \textcolor{red}{(to-do)}}
\textcolor{red}{to-do}Existing studies investigate 5G performance improvements through layer specific optimization solutions, such as application layer solutions [cite], RLC layer solutions [cite], MAC layer solutions [cite], and PHY layer solutions [cite]. Studies such as [cite] investigate the effect of 5G on emerging human-centric applications such as XR and propose application-aware network layer solutions such as PDU sets, but these solutions are for a specific class of applications and are confined to a specific layer of the network stack. Overall, these studies do not consider emerging machine-centric application requirements and also do not explore the effect of joint optimization across the different layers of the network stack through a learnable approach.
%\subsection{Legacy 5G network optimization}
\vspace{-1mm}
%A frame in 5G is 10\,ms long, which is broken down into 10 subframes. Depending on the numerology, the number of slots in a subframe varies. Here, we consider a subframe of 30\,KHz subcarrier spacing (SCS) spanning two slots. In TDD, the DL-UL-periodicity determines the time for which there can be a consecutive set of DL and UL slots. Each slot is further broken down into 14 symbols assuming a normal cyclic prefix (CP). 
\begin{comment}
\noindent $\bullet$ \textbf{Time domain resource allocation: }For allocating resources in the time domain in 5G, the NW informs the UE about which slots/symbols can be used for data transmission/reception through signaling of time-domain resources using DCI in PDCCH. In 5G, DCI formats [{0\_0}] and [{0\_1}] allocate time-domain resources for UL data in the physical uplink shared channel (PUSCH). DCI formats {0\_0} and {0\_1} carry a 4-bit field named ‘time domain resource assignment’ which points to one of the 16 rows of a look-up table\,\cite{snr-cqi-mcs}. Each row in the look-up table provides the following parameters:
\begin{figure}[t!]
	\centering
	\includegraphics[trim=0 0 0 0,clip,width=\linewidth]{Figures/time_domain_allocation.pdf}
	\vspace{-2mm}
	\caption{Example time domain resource assignment from NW to UE in 5G.}
 \vspace{-4mm}
	\label{fig:time_domain_alloc}
\end{figure}
\begin{itemize}
    \item \textit{Slot offset K2.} This parameter is used to derive the slot in which PUSCH transmission occurs.
    \item Jointly coded \textit{Start and Length Indicator Values (SLIV)}, or individual values for the start symbol $S$ and the allocation length $L$.
    \item \textit{PUSCH mapping type} to be applied on the PUSCH transmission.
\end{itemize}
\end{comment}
%\noindent $\bullet$ \textbf{Dynamic scheduling procedure: }
%The time domain resource allocation for UL transmission in 5G are shown in Fig.\,\ref{fig:time_domain_alloc}. 
%In TDD, the NW divides the time domain into either dedicated DL or UL or mixed/flexible (F) slots. Here, the UE first sends a scheduling request (SR) as control information in either a UL or an F slot, informing the NW of pending data to transmit. The UE informs the NW of the most recent channel estimates and the available data volume through the buffer status report (BSR) as control information\,\cite{3gppmac}. The NW after receiving these control data, informs the UE of the assigned resources via DCI, which indicates the specific slot, starting OFDM symbol, and symbol length in the following PHY frame for uplink data transmission. The UE then prepares the transport block sizes (TBS) based on the MCS and the assigned OFDM symbols and transmits the application data in the scheduled UL slots. 3GPP allows flexible scheduling in which slots are dedicated for DL vs. UL transmissions\,\cite{3gpprlc}.
%Existing 5G optimization techniques primarily focus on individual network layers. Application data flows through the network stack until it reaches the RLC buffer where it can accumulate due to wireless channel degradation, network congestion, etc. The RLC layer has FIFO queues (RLC channels) per application class [cite]. Packets wait at the RLC buffer until the MAC scheduler allocates Resource Blocks (RBs), in the network signaled PHY frame slots, to the data in the RLC buffer. This allocation process considers various factors such as channel conditions, application class KPI bounds, buffer status, etc. While commercial networks employ proprietary non-public MAC scheduling algorithms tailored to their specific needs, 3GPP provides standardized algorithms like \textit{Round Robin}, \textit{Proportional Fair}, and \textit{bestCQI} as benchmarks for evaluating new scheduling methods. Commercial 5G networks such as AT\&T, T-Mobile, and Verizon all configure the 5G PHY frame of 10\,ms duration, with the same {\em fixed} configuration that has more downlink (DL) than uplink (UL) slots\,\cite{fezeu:techreport2023}, leading to imbalanced traffic distribution. The resource allocation in the PHY layer is \textit{reactive}, where the UE triggers the dynamic scheduling process by transmitting a Scheduling Request (SR) to the NW in a UL or F slot, indicating pending UL data, and providing channel estimates via the physical uplink control channel (PUCCH)\,\cite{3gppmac}. Subsequently, the NW informs the UEs of resource allocation using downlink control information (DCI) in the downlink control channel (PDCCH). Here, among other things, the NW specifies slot assignments for UEs in the uplink service channel (PUSCH) and downlink service channel (PDSCH) for UL transmission and DL reception, respectively. 

%\noindent $\bullet$ \textbf {PHY layer:} The 5G PHY layer performance is based on the resource allocation in the PHY frames which are divided into DL and UL slots. Commercial 5G networks such as AT\&T, T-Mobile, and Verizon all configure the 5G PHY frame of 10\,ms duration, with the same {\em fixed} configuration that has more DL than UL slots\,\cite{fezeu:techreport2023}, leading to imbalanced traffic distribution.
%In TDD, the NW configures the PHY frame of 10\,ms duration into dedicated DL, UL, and mixed/flexible (F) slots in the time domain (20 slots per frame in this example), as shown in Fig.\,\ref{fig:7_2_1}. 
%The resource allocation in the PHY layer is \textit{reactive}, where the UE triggers the dynamic scheduling process by transmitting a Scheduling Request (SR) to the NW in a UL or F slot, indicating pending UL data, and providing channel estimates via the physical uplink control channel (PUCCH)\,\cite{3gppmac}. Subsequently, the NW informs the UEs of resource allocation using downlink control information (DCI) in the downlink control channel (PDCCH). Here, among other things, the NW specifies slot assignments for UEs in the uplink service channel (PUSCH) and downlink service channel (PDSCH) for UL transmission and DL reception, respectively. 

%\noindent $\bullet$ \textbf {MAC layer:} 
%The MAC scheduler at each base station (gNB) decides the UE-wise PRB allocation for every slot. In FDD, PDSCH and PUSCH allocations are output per slot, while in TDD, the appropriate (PDSCH or PUSCH) allocation is output every slot (DL or UL). The scheduler uses the information related to the SINR, CQI, MCS, number of MIMO layers, buffer status, and HARQ  as inputs for each gNB and attached UE. The scheduler also uses the number of PRBs available in the gNB. Re-transmissions are prioritized over first transmissions. The scheduler's output is UE-wise PRB allocation in the UL and DL, at every slot. MAC scheduling algorithms in commercial networks are customized for specific business needs and are unavailable in the public domain. 3GPP provides some baseline MAC scheduling algorithms (Round Robin, Proportional Fair, and bestCQI) for evaluating the performance of proposed methods in the upstream and downstream functions. The \textit{RoundRobin} method divides the available PRBs among the logical channels that have a non-empty RLC queue. The MCS for each user is calculated according to the received CQIs. The \textit{Proportional Fair} method works by scheduling a (active) user when its instantaneous channel quality is high relative to its own average channel condition over time. The PF scheme is based on the current data rate for each user and the exponentially weighted moving average (EWMA) data rate over an immediately prior predetermined interval for each user. In comparison with the round-robin (RR) scheduler in which UEs are cyclically scheduled irrespective of the channel condition, the PF scheduler maximizes the system throughput while maintaining long-term fairness in the allocation of resources between users. The \textit{bestCQI} method allocates PRBs to the active flow(s) to maximize the achievable rate. It selects the user that sees the highest CQI.
%In the 5G MAC layer, the scheduler at each base station (gNB) allocates Physical Resource Blocks (PRBs) to User Equipments (UEs) for every time slot. This allocation process considers various factors such as Signal-to-Interference-plus-Noise Ratio (SINR), Channel Quality Indicator (CQI), Modulation and Coding Scheme (MCS), number of Multiple-Input Multiple-Output (MIMO) layers, buffer status, and Hybrid Automatic Repeat reQuest (HARQ) information. 
%The scheduler prioritizes re-transmissions over first transmissions to ensure data reliability. 
%While commercial networks employ proprietary non-public MAC scheduling algorithms tailored to their specific needs, 3GPP provides standardized algorithms like \textit{Round Robin}, \textit{Proportional Fair}, and \textit{bestCQI} as benchmarks for evaluating new scheduling methods. 
%The Round Robin algorithm distributes PRBs equally among active logical channels, while the Proportional Fair algorithm dynamically allocates resources to users with favorable instantaneous channel conditions relative to their average channel quality, thereby maximizing system throughput and ensuring long-term fairness. The \textit{bestCQI} algorithm prioritizes the user with the highest CQI to maximize the achievable data rate.

%\noindent $\bullet$ \textbf {Application layer:} 
%Flexible  scheduling, sanctioned by 3GPP, assigns slots for DL and UL transmissions as needed.

%Commercial 5G networks such as AT\&T, T-Mobile, and Verizon all use the same {\em fixed} PHY frame configuration that has more DL than UL slots -- 7\,DL-2\,UL-1\,F (7\,DL 3\,UL)\,\cite{fezeu:techreport2023}, with a periodicity of 5\,ms (Fig.\,\ref{fig:7_2_1}), leading to imbalanced traffic distribution.
%Existing drive-test studies on these commercial networks using 5G smartphones reveal impressive DL speeds, surpassing 3.5\,Gbps, while UL throughput remains consistently lower\,\cite{ghoshal:imc2023}. The fixed DL-heavy PHY frame configuration may strain 5G networks, particularly with the rise of uplink-heavy applications. The use of this fixed PHY frame configuration persists due to the prevalence of existing DL-heavy applications and the 3GPP-supported {\em reactive} scheduling method based on UE-reported channel metrics. The lack of channel forecast knowledge provides little incentive for the NW to modify PHY frame configuration relying solely on UE-reported channel metrics as this process may take time to converge and can result in sub-optimal NW performance. 
\vspace{-1mm}


\vspace{-0.3cm}
\section{Device Model for Wafer-Scale Accelerators}
\vspace{-0.2cm}
% that differentiate wafer-scale accelerators from traditional shared-memory and NUMA architectures.

\subsection{The PLMR model}
\vspace{-0.1cm}
We develop the PLMR model to capture the unique hardware properties of wafer-scale accelerators and to motivate system requirements needed for utilizing this emerging hardware.

\begin{enumerate}[label=(\arabic*), leftmargin=0.5cm, noitemsep,topsep=0pt]
    \item \textbf{Massive Parallelism (P)}:
A wafer-scale accelerator can easily be equipped with millions of parallel cores, compared to thousands in GPUs. Each core features a local hardware pipeline that overlaps data ingress, egress, computation, and memory access at the cycle level.
This requires the computation to be partitioned at a massive scale and a fine-grained schedule to overlap computation, memory access, and NoC communication.

\item \textbf{Highly non-uniform memory access Latency (L)}:
Accessing memory on other cores in a mesh exhibits highly non-uniform latency. In a mesh with \(N_w \times N_h\) cores, the maximum NoC hops to a remote core is \(\max(N_w, N_h)\). For a million-core mesh, this can reach 1000 hops, causing a 1000$\times$ latency difference between local and remote memory access.
Therefore, it is crucial for the computation to minimize long-range communication whenever possible.

\item \textbf{Constrained local Memory (M)}: Each core has a small local memory (tens of KBs to several MBs), as performance and energy efficiency decline with larger capacities~\cite{sram-wiki}.
As a result, computation data must be explicitly partitioned into fine-grained chunks to fully fit within the constraints of each core's local memory.

\item \textbf{Constrained Routing resources (R)}: 
The message size in the NoC of a wafer-scale accelerator is extremely limited (e.g., a few bytes). This constraint requires message headers (e.g., address encoding) to be restricted to just a few bits, maximizing the capacity for actual data transfer. Consequently, only limited routing paths can be used, and the software system must carefully plan these paths.
\end{enumerate}

We expect these properties to remain relevant, as they are rooted in the fundamental characteristics of hardware and its manufacturing process. The PLMR model applies to both current (Cerebras WSE) and future (Tesla Dojo) wafer-scale devices. Even some non-wafer-scale devices with mesh-based NoC architectures, such as Tenstorrent Blackhole~\cite{tenstorrent}, can be represented by PLMR with adjusted parameters for parallelism (P), the size of the mesh (L), or relaxed constraints on local memory (M) and routing resources (R).


    \vspace{-3mm}
\subsection{Limitations of state-of-the-art approaches}
    \vspace{-1mm}
    
Leveraging the PLMR model, we analyze why existing AI systems fail to fully utilize wafer-scale accelerators.
To run an LLM model on a wafer-scale accelerator, we generally have two choices: (i) abstract the distributed local memory in each core as a shared memory and directly access data placed in a remote core through NoC; and (ii) explicitly partition computation into distributed cores and use message passing to exchange necessary data. 
We analyze two types of representative systems: LLM runtime or DNN compilers for shared memory architecture such as GPUs, e.g., Ladder~\cite{ladder}; and the SOTA compiler for distributed on-chip memory architectures, e.g., T10~\cite{t10} for GraphCore IPU.

\mypar{Shared-memory system} 
A shared-memory-based DNN compiler such as Ladder usually assumes a uniform memory access pattern within the underlying memory hierarchy, which cannot tolerate the 1000$\times$ latency variance in wafer-scale accelerators when accessing data from remote memory (failing in L).
Moreover, these compilers~\cite{tvm,rammer,ansor,flextensor,roller,welder,ladder} often focus primarily on partitioning computation, with less emphasis on optimizing data partitioning. This approach can easily lead to significant data duplication and violate the memory constraint requirements (failing in M).
Finally, these compilers are unaware of the communication distance of each core, poorly addressing the constraint of routing resources.

\mypar{Distributed-memory system} The T10 system~\cite{t10} is designed for AI accelerators with an on-chip crossbar which ensures a constant hop of memory access to other cores on the same chip. T10 handles small local memory and balances communication loads, addressing memory constraints (M) and routing resource limits (R). However, on a PLMR device, it fails to account for varying hop distances (failing in L) and scales to thousands, not millions, of cores (failing in P).


    \vspace{-3mm}
\section{Wafer-Scale LLM Parallelism}\label{sec:meshmp}
    \vspace{-1mm}
    
We present wafer-scale LLM parallelism, featuring new designs across prefill, decode and KV cache management.


    \vspace{-3mm}
\subsection{Prefill parallelism}
    \vspace{-1mm}

The parallelism for LLM prefill must ensure compliance with the PLMR model. Key challenges include: (i)~Handling multiple large matrices during prefill, requiring effective dimension partitioning to achieve million-core parallelism (P); (ii)~Optimizing GEMM operations, which involve further partitioning and overlapping computation and communication, to minimize long-range communication overhead (L), respect local memory constraints (M), and account for limited routing resources (R); and (iii)~Handling matrix transposes, which are costly on a NoC (L) but often required for sequential GEMM operations.

\begin{figure}[t!]
    \centering
    \includegraphics[width=0.46\textwidth]{imgs/prefill-partition.pdf}
    \vspace{-2mm}
    \caption{Prefill parallelism plan. $E_xF_y$ represents a matrix of shape $EF$, where the $E$ dimension is partitioned along the $x$-axis of cores, and $F$ along the $y$-axis of cores on a mesh.}
    \vspace{-5mm}
    \label{fig:prefill-partition}
\end{figure}

\mypar{Designing fine-grained partitioning for million-core parallelism} To achieve high chip utilization, we propose partitioning two dimensions of the input activation and weight matrices along both the $X$- and $Y$-axes of cores. This approach enables finer-grained, million-scale parallelism compared to existing methods~\cite{flashattn2, flashdecoding++, megatron, google2023efficiently}, which typically partition only the embedding dimension, resulting in insufficient parallelism on PLMR devices. 

We illustrate this partitioning using self-attention and feedforward, as shown in Figure~\ref{fig:prefill-partition}. For this discussion, we define the following annotations: the input activation $A$ and weight $W$ are multi-dimensional tensors during the prefill process. $B$ represents the batch size, $L$ the sequence dimension, $E$ the embedding dimension, $H$ the head dimension, and $F$ the hidden dimension in the feedforward block.
As shown by \myc{1}, the partitioning layout of $A$ is represented as $BL_yE_x$, where the $L$ dimension is partitioned along the $Y$-axis of cores, and the $E$ dimension along the $X$-axis of cores. Similarly, all weight matrices ($W_Q$, $W_K$, $W_V$, $W_{in}$, and $W_{out}$) are partitioned across both dimensions. 


\mypar{Designing PLMR-compliant distributed GEMM} 
We propose replacing conventional GEMM operators, designed for shared memory architectures, with a newly designed PLMR-compliant distributed GEMM during the prefill phase (as shown in \myc{2} of Figure~\ref{fig:prefill-partition}). Unlike TPU and GPU systems that primarily rely on allgather operations for GEMM, PLMR-compliant distributed GEMM algorithms achieve high NoC bandwidth utilization while respecting local memory and routing constraints, ensuring compliance with the L, M, and R properties. This PLMR-compliant distributed GEMM is fully described in Section~\ref{sec:gemm}.

\mypar{Using transposed distributed GEMM to avoid matrix transpose}
We propose a transpose-free parallelism plan for prefill to avoid matrix transpose, a common operation in LLM systems designed for shared memory architectures. The L property in PLMR highlights that matrix transposition is particularly costly on a wafer-scale device. It requires a core on one corner of the mesh to send data to the opposite diagonal corner, creating a long-range communication path.

Our transpose-free parallelism plan leverages transposed distributed GEMM (denoted as dist-GEMM-T)~\cite{summa, trans-dist} to compute $Q@K^T$ during LLM prefill, as shown by \myc{3} in Figure~\ref{fig:prefill-partition}. Specifically, the intermediate $Q$ and $K$ tensors, generated by multiplying $X$ with $W_Q$ and $W_K$, require transposing $K$ before proceeding with dist-GEMM operations due to the on-chip partition shape.


    \vspace{-3mm}
\subsection{Decode parallelism}
    \vspace{-1mm}

\begin{figure}[t!]
    \centering
    \includegraphics[width=0.46\textwidth]{imgs/decode-partition.pdf}
    \vspace{-2mm}
    \caption{Decode parallelism plan. $E^yF_x$ indicates the $E$ dimension is replicated along the $y$-axis, and $F$ is partitioned along the $x$-axis.}
    \vspace{-5mm}
    \label{fig:decode-partition}
\end{figure}

The parallelism strategy for LLM decode must address its memory-bandwidth-intensive nature, presenting several challenges: (i)~Decode uses smaller matrices than prefill due to limited input sequences and batch sizes, requiring careful parallelization when dimensions are insufficient for partitioning; (ii)~The phase heavily relies on GEMV operations, which are less compute-intensive than GEMM, resulting in short computation phases with limited overlap with communication, making GEMV vulnerable to long-range communication overhead on a NoC (L) and requiring adherence to local memory and routing constraints (M and R); and (iii)~Sequential GEMV operations introduce costly matrix transpose on a NoC, risking violation of the L property.

\mypar{Designing fine-grained replication to enable parallelism at minimal communication cost} When tensor dimensions are insufficient to achieve the high parallelism required for decode, we propose fine-grained replication of tensors in LLMs, specifically replicating the sequence dimension, where the sequence length equals the prompt length during prefill phase and equals 1 during the decode phase. This approach offers two key advantages: (i)~It improves parallelism and ensures balanced loads across all cores, and (ii)~It avoids additional communication operations such as allreduce. As shown by \circled{1} in Figure\ref{fig:decode-partition}, the $E$ dimension is partitioned along the $y$-axis, and the $L$ dimension is replicated along the $x$-axis, represented as $BE_yL^x$. Weight matrices $W$ are partitioned across both dimensions, consistent with the prefill phase.

Our fine-grained replication differs from recent work on long-context/sequence inference systems~\cite{loongserve, distserve}, which selectively replicate certain dimensions during the prefill phase rather than the decode phase.

\mypar{Designing PLMR-compliant distributed GEMV} We found that existing GEMV implementations fail to fully comply with PLMR requirements due to long-range communication and excessive routing resource consumption at each core. To address this, we propose a PLMR-compliant distributed GEMV, utilizing this new implementation throughout the decode phase (as detailed in \myc{2} of Figure~\ref{fig:decode-partition}). A comprehensive description of this GEMV design is provided in Section~\ref{sec:gemv}.

\mypar{Pre-optimizing model weight placement to avoid matrix transpose}
To avoid matrix transpose during decode, we propose pre-optimizing the model weight layout for decode, particularly for the distributed GEMV operation, to eliminate matrix transpose. While this introduces re-placement overhead between prefill and decode phases, the overhead is far smaller than that of sequential matrix transpose during token generation.

Figure~\ref{fig:decode-partition} illustrates this proposal, detailed in \myc{3}. Specifically, we optimize the placement of weights such as $W_{O}$ and $W_{out}$ for distributed GEMV in decode, differing from their layout in the prefill phase. This approach also removes the need for transpose operations in calculating $Q@K^T$ during decode self-attention.

    \vspace{-3mm}
\subsection{Shift-based KV cache management}\label{sec:parallel:kvcache}
    \vspace{-1mm}
    
\begin{figure}[t!]
    \centering
    \includegraphics[width=1\linewidth]{imgs/KVManage.pdf}
    \vspace{-9mm}
    \caption{KV cache concatenation vs. KV cache shift}
    \vspace{-5mm}
    \label{fig:kv_cache_update}
\end{figure}

KV cache management on PLMR devices is challenging as it requires storing large data across distributed cores while adhering to local memory constraints (M) and distributing KV cache computations to achieve high parallelism (P). To address these, we have the following insights:

\mypar{Existing concatenate-based management causes skewed core utilization} Current KV cache management methods primarily concatenate newly generated KV vectors to the existing cache. While efficient in shared memory architectures, this concatenate operation leads to highly skewed core utilization on PLMR devices, as shown in \circled{1} of Figure~\ref{fig:kv_cache_update}, where only core in a row is responsible for storing and computing over the newly generated KV vector. After several token generation steps, this only core quickly becomes the bottleneck, as depicted in \circled{2} of Figure~\ref{fig:kv_cache_update}, causing skewed memory usage and violating the M in PLMR. Moreover, the imbalanced KV cache distribution across cores results in inefficient parallelism, violating the P property.


\mypar{Proposing shift-based management for balanced core utilization} We propose a shift-based KV cache management strategy that evenly distributes cache data across all cores. Instead of concatenating new KV cache vectors at the end, this method performs a balancing shift operation, where each row transfers the oldest KV cache data to the row above, as shown in \circled{3} of Figure~\ref{fig:kv_cache_update}. When new KV data arrives, each core checks its local capacity against its neighbors. If equal, upward shifts are triggered, with each row receiving data from below and passing some to the row above. As illustrated in \circled{4}, this ensures even KV cache distribution across all cores.

The upward shifts utilize all NoC links in parallel, maintaining high performance and satisfying the P property. The physical placement of KV cache aligns with logical continuity, adhering to the L property. This method also fully resolves the M violation issue observed in the last row of cores with the concatenate-based approach.

    \vspace{-3mm}
\subsection{Implementation details}
    \vspace{-1mm}

We outline several implementation details below:

\mypar{Prefill and decode transition} Prefill and decode require distinct strategies. To handle the transition efficiently, we reshuffle KV cache and weights through the fast NoC which often provides 100s Pbits/s aggregated bandwidth, completing instantly without relying on slower off-chip memory.

\mypar{Parallelism configuration} We empirically determine the scalable parallelism for LLM operators. Automatic parallelism configuration is left for future work.

\mypar{Variations of self-attention} \sys supports variations of Self-Attention, including Grouped Query Attention~\cite{gqa}, Multihead Attention~\cite{mha}, and Multi-query Attention~\cite{mqa}. These differ by performing dist-GEMM, dist-GEMV and dist-GEMM-T locally after grouping by head dimensions.



    \vspace{-3mm}
\section{Wafer-Scale GEMM} \label{sec:gemm}
    \vspace{-1mm}

In this section, we introduce MeshGEMM, a scalable distributed GEMM for massive-scale, mesh architectures. 

    \vspace{-3mm}
\subsection{PLMR compliance in distributed GEMM}
    \vspace{-1mm}

\begin{figure}
    \centering
    \includegraphics[width=1\linewidth]{imgs/gemm-analysis.pdf}
    \vspace{-3mm}
    \caption{PLMR compliance in distributed GEMM}
    \vspace{-3mm}
    \label{fig:distributed-gemm-analysis}
\end{figure}

To identify an scalable distributed GEMM for PLMR devices, we define the following metrics: 
(i)~\emph{Paths per core}: The number of routing paths per core, with fewer paths ensuring compliance with the R property.
(ii)~\emph{critical path}: The longest communication path in each step to transmit submatrix (as the red lines in Figure~\ref{fig:distributed-gemm-analysis}), with fewer hops adhering to the L property.
(iii)~\emph{Memory per core}: The memory required per core, with lower usage ensuring the M property.

We analyze current distributed GEMM methods and show how MeshGEMM meets these metrics:

\begin{enumerate}[label=(\arabic*), leftmargin=0.5cm, noitemsep,topsep=0pt]
\item \textbf{GEMM via Allgather} is commonly used in GPU and TPU pods for distributed GEMM~\cite{google2023efficiently, tensorrt-llm, megatron}. Its longest communication path in each step is one core gathering data from the farthest cores, shown as the red line in Figure~\ref{fig:distributed-gemm-analysis}~\circled{1}, and $N$ steps to complete the allgather. Each core creates $N$ communication paths to neighbors in its row and column (violating R). The gather in each step spans the critical path with $O(N)$ hops (violating L), and each core uses $O(1/N)$ memory due to inflated working buffers, far exceeding the $O(1/N^2)$ for local submatrices (violating M).

\item \textbf{SUMMA} is Cerebras' default choice for distributed GEMM on its wafer-scale engine~\cite{cerebrasgemm}. Its longest communication path in each step is where one core broadcasts data to the farthest core along the column or row, shown by the red line in \circled{2} of Figure~\ref{fig:distributed-gemm-analysis}. Every core creates $N$ communication paths (violating R) and spans the critical path with $O(N)$ hops (violating L) in the longest path. While SUMMA improves memory usage compared to AllGather, requiring only a working set equal to the size of locally partitioned submatrices, it still doubles memory usage.

\item \textbf{Cannon} is mesh-optimized choice for distributed GEMM~\cite{cannon}, popular in supercomputers. 
Its longest communication path in each step is the head cores send data to the tail cores.
% It needs $N$ steps submatrices shifting to complete GEMM on cores. 
As shown in \circled{3} of Figure~\ref{fig:distributed-gemm-analysis}, each core communicates with two neighbours in a 2D torus, and only needs $O(1)$ communication paths and optimal memory usage of $O(1/N^2)$. However, it incurs the critical path with $O(N)$ hops as the red line, violating L.

\item \textbf{MeshGEMM (Ours)} is a distributed GEMM which complies with the PLMR model. Its longest communication path in each step is shown as the red line in \circled{4} of Figure~\ref{fig:distributed-gemm-analysis}. Each core communicates with two neighbors, two hops away (proven in later sections to be scalable for mesh architectures). This design achieves $O(1)$ communication paths per core needed and optimal memory usage of $O(1/N^2)$, similar to Cannon. Crucially, it bounds the critical path to 2 hops with $O(1)$ complexity, making it uniquely capable of addressing the L property.

\end{enumerate}

\vspace{-0.2cm}
\subsection{Design intuitions and scalability analysis} 

\begin{figure}[t!]
    \centering
    \includegraphics[width=1\linewidth]{imgs/gemm_intuition.pdf}
    \caption{Design intuitions and scalability analysis.}
    \label{fig:gemm_intuition}
\end{figure}

Our design involves two steps: (i)~We ensure algorithm correctness using a cyclic shifting process for GEMM, and (ii)~We prove that two-hop communication on this cycle is the minimal distance required to satisfy the L property.

\mypar{Cyclic shifting} Cyclic shifting enables \gemm to satisfy the M and R properties by limiting communication to two neighbors and minimizing memory usage. It ensures correct GEMM results, following reasoning similar to Cannon~\cite{cannon}. As illustrated in \circled{3} of Figure~\ref{fig:distributed-gemm-analysis}, a logical circle of 5 cores is flattened into the physical communication mapping, with a critical path from head core to tail core. 

\begin{algorithm}[t]
    \caption{$\textbf{INTERLEAVE}$}
    \label{alg:interleave}
    % \SetKwFunction{min}{Min}
    % \SetKwFunction{max}{Max}
    \SetKwFunction{assert}{assert}
    \SetKw{ret}{Return}
    \SetKw{int}{Int}
    \LinesNumbered
    % \KwIn{Current PE's Coordinate: $pc$}
    % \KwIn{Current Axis's PE count: $P$}
    \KwIn{index, N}
    % \KwOut{Send To Destination PE ID: sendId}
    % \KwOut{Receive From Source PE ID: recvId}
    \KwOut{send\_index, recv\_index}
    
    % \BlankLine
    % \tcp{\textbf{Assign Stage}}
    \eIf{\textnormal{index} \textnormal{\textbf{mod}} \textnormal{2 == 0}}{
        recv\_index = \texttt{Max} (index - 2, 0)\;% RId = \min {$n+2$, N-1}\;
        send\_index = \texttt{Min} (index + 2, N - 1)\;
    }{
        recv\_index = \texttt{Min} (index + 2, N - 1)\;%, RId = \max {$n-2$, 0}\;
        send\_index = \texttt{Max} (index - 2, 0)\;
    }
    % \BlankLine
    \textbf{if} \textnormal{index == 0} \textbf{then} recv\_index = 1\;
    % \BlankLine
    \If{\textnormal{index == N - 1}}{
        % \eIf{$N\;mode\;2==0$}{
        %     RId = $P-2$\;
        % }{
        %     SId = $P-2$\;
        % }
        \textbf{if} N \textbf{mod} 2 == 0 \textbf{then}
            recv\_index = N - 2\;
        \textbf{else}
            send\_index = N - 2\;
    }
    \ret send\_index, recv\_index\;
\end{algorithm}

\mypar{Interleaving} For the flatten communication plan, we would like to minimize the length of the critical path further, thus satisfying the L property. Our key intuition here is to introduce an INTERLEAVE operation to find the mapping relationship from logical to physical, defined in Algorithm~\ref{alg:interleave}. As shown by \circled{1} of Figure~\ref{fig:gemm_intuition}, MeshGEMM first insert core 1 in between core 0 and 4 and core 2 in between core 4 and 3 to form a logical mapping, and then call the INTERLEAVE operation to get the send to and receive from neighbours' index, resulting in a permutated, equivalent communication plan as shown by \circled{2} in Figure~\ref{fig:gemm_intuition}. For example, there are 5 cores total (N=5), so physical core 2 (index=2) sends data to physical core 4 (send\_index=4) and receives from physical core 0 (recv\_index=0).


\mypar{Scalability analysis}  We can prove that the two-hop distance created by INTERLEAVE cannot be further reduced. The proof relies on the fundamental properties of sequential arrangements: if we attempt to create a circular sequence where each number differs from its neighbors by exactly one hop, we encounter a mathematical impossibility. This can be understood by visualizing the numbers as points on a line - while adjacent numbers can be connected, the endpoints of the sequence cannot simultaneously maintain single-hop differences with their neighbors while forming a circle.

Note that our discussion, based on a 1D array, naturally extends to a 2D mesh, as the 1D array corresponds to the mesh's X-axis and Y-axis due to their symmetry.

\vspace{-0.3cm}
\subsection{The \gemm algorithm} \label{subsec:meshgemm} 

We outline the key steps of \gemm below:

\begin{enumerate}[label=(\arabic*), leftmargin=0.5cm, noitemsep,topsep=0pt]
\item \textbf{Initialization:} Consider $C = A \times B$. \gemm will partition $A$ and $B$ into tiles $A_{sub}$ and $B_{sub}$ along two dimensions, forming $N \times N$ tiles, which are distributed across the cores. Each core receives one tile of $A_{sub}$ and one of $B_{sub}$. \gemm will then use INTERLEAVE to initialize the neighbor's positions for each core. 

\item \textbf{Alignment:}  
Each core will then align with neighbors to align the input submatrices in a way that ensures every core in the distributed system begins with the appropriate operands for the matrix multiplication process. 

\item \textbf{Compute-shift loop:}  
Each core operates with a compute-shift loop involving $N$ steps of communication and computation. In each step, every core computes the partial sum of its corresponding $C_{sub} = A_{sub} \times B_{sub} + C_{sub}$. Meanwhile, shift $A_{sub}$ along the X-axis and $B_{sub}$ along the Y-axis to get new $A'_{sub}$ 
 and $B'_{sub}$ for the next step computation as \circled{3} we shown in Figure~\ref{fig:gemm_intuition}. After $N$ steps, the accumulated $C_{sub}$ is returned.
\end{enumerate}

\vspace{-0.3cm}
\subsection{Implementation details}

\mypar{Handling non-square mesh} For a non-square mesh $N_h \times N_w$ ($N_h \neq N_w$), the $A$ and $B$ matrices can be logically partitioned into $N_{lcm} \times N_{lcm}$ cores, where $N_{lcm}$ is the least common multiple of $N_h$ and $N_w$.

\mypar{Transposed distributed GEMM} The above algorithm key steps can be applied to the computation of $C = A \times B^T$, the dist-GEMM-T in Figure~\ref{fig:prefill-partition} to avoid transposing $B$ on mesh. It does not require alignment before computation and only necessitates $N$ steps two-hop compute-shift for the right matrix $B$ along the Y-axis. After each shift step, each core computes $C_{\text{sub}} = A_{\text{sub}} \times B_{\text{sub}}$, followed by a ReduceAdd of $C_{\text{sub}}$ along the X-axis. After $N$ steps, the final matrix $C$ is obtained.

\vspace{-0.3cm}
\section{Wafer-Scale GEMV}\label{sec:gemv}
\vspace{-0.2cm}

In this section, we describe \gemv, a scalable GEMV algorithm for PLMR devices.


\subsection{PLMR compliance in distributed GEMV}

\begin{figure}[t!]
    \centering
    \includegraphics[width=1\linewidth]{imgs/reduce-algo.pdf}
    \caption{PLMR compliance in distributed GEMV}
    \label{fig:gemv-plmr-compliance}
\end{figure}

The completion time of a distributed GEMV is primarily determined by an allreduce operation that aggregates partial results from all selected cores and broadcasts the aggregated results back to all cores. So, we define the number of add-operations (hops) in the longest aggregation path as the \emph{critical path} in GEMV. Below, we analyze common distributed GEMV implementations in LLM systems and demonstrate that \gemv is the only approach fully compliant with the PLMR model.

\begin{enumerate}[label=(\arabic*), leftmargin=0.5cm, noitemsep,topsep=0pt]

 \item \textbf{GEMV with pipeline allreduce} is commonly used in TPU pod systems~\cite{google2023efficiently} and as the default in Cerebras demo~\cite{cerebrasgemv}. As shown by \circled{1} in Figure~\ref{fig:gemv-plmr-compliance}, it bounds routing resource usage to $O(1)$ per core (meeting R in PLMR). However, its longest aggregation path is from tail to head cores, as shown in the red line, and spans the critical path at $O(N)$, violating the L property.

\item \textbf{GEMV with ring allreduce} is commonly used in GPU pod systems, where it is the default configuration. As shown by \circled{2} in Figure~\ref{fig:gemv-plmr-compliance}, it bounds routing resource usage to $O(1)$ (meeting R in PLMR). However, it spans $O(N)$ hops in the critical path, violating the L property.

\item \textbf{GEMV with two-way K-tree allreduce (Ours)}. As shown by \circled{4} in Figure~\ref{fig:gemv-plmr-compliance}, we build a balanced K-tree to reduce from two-way; its longest aggregation path is from the head or tail core to the tree root core. The critical path is $O(\sqrt[K]{N}K)$ which can address the L. The max number of communication paths at each root core is $O(K)$, and can meet the R limitation by adjusting the K.

\end{enumerate}


\subsection{The \gemv algorithm} 

We will outline the key steps of \gemm below:

\begin{enumerate}[label=(\arabic*), leftmargin=0.5cm, noitemsep,topsep=0pt]
\item \textbf{Initialization:} Consider $C = A \times B$ and $A$ is a vector. \gemv will partition $B$ into tiles $B_{sub}$ along two dimensions, forming $N \times N$ tiles and distributed across the cores. For $A$, \gemv will partition it along the vector length, forming $N$ tiles distributed on one axis and replica $A$ on another axis. Each core receives one tile of $A_{sub}$ and one of $B_{sub}$. Then we determine which cores form a group to obtain aggregated results in each phase based on the K-tree.

\item \textbf{Parallel computation:} In this stage, each core performs a local GEMV $A_{\text{sub}} \times B_{\text{sub}}$ to obtain $C_{sub}$ partial sum.

\item \textbf{Aggregation:} The aggregation step primarily involves using the two-way K-tree allreduce we design. The key steps as follows: (i)~In the 1st-phase, each group performs group reduction and obtains the partial sum of $C_{sub}$ at the root core of each group. (ii)~In the $k$th-phase, the results from the $(k-1)$~th-phase are reduced to the root cores of each group in the $k$th-phase. After K times repeating, $C$ can be obtained by concatenating the $C_{sub}$ from all K-tree root cores. (iii)~ Optionally, a broadcast operation from the root core of the $K$-tree may follow, depending on whether continuous GEMV is required.
\end{enumerate}

\mypar{Scalability Analysis}  
As shown in \circled{1} of Figure~\ref{fig:gemv-plmr-compliance}, this method scales efficiently with parallelism and meets the L property by selecting an appropriate $K$. It requires $K+1$ paths at the tree root core but allows flexible adjustment of $K$ to address R based on hardware limitations.  

However, a larger $K$ is not always better, as it depends on $N$ and R constraints. Additionally, larger $K$ increases routing complexity and overhead. Considering these factors, we have chosen $K=2$ for our current implementation evaluated in the following sections.


\definecolor{darkgreen}{rgb}{0.0, 0.5, 0.0}
\definecolor{violet}{rgb}{0.56, 0.0, 1.0}
\section{Evaluation}
We apply our methodology to derive counterfactual policies for various MDPs, addressing three main research questions: (1) how does our policy's performance compare to the Gumbel-max SCM approach; (2) how do the counterfactual stability and monotonicity assumptions impact the probability bounds; and (3) how fast is our approach compared with the Gumbel-max SCM method?

\begin{figure*}
    \centering
    %
    \resizebox{0.6\textwidth}{!}{
        \begin{tikzpicture}[scale=1.0, every node/.style={scale=1.0}]
            \draw[thick, black] (-3, -0.25) rectangle (10, 0.25);
            %
            \draw[black, line width=1pt] (-2.5, 0.0) -- (-2,0.0);
            \fill[black] (-2.25,0.0) circle (2pt); %
            \node[right] at (-2,0.0) {\small Observed Path};
            
            %
            \draw[blue, line width=1pt] (1.0,0.0) -- (1.5,0.0);
            \node[draw=blue, circle, minimum size=4pt, inner sep=0pt] at (1.25,0.0) {}; %
            \node[right] at (1.5,0.0) {\small Interval CFMDP Policy};
            
            %
            \draw[red, line width=1pt] (5.5,0) -- (6,0);
            \node[red] at (5.75,0) {$\boldsymbol{\times}$}; %
            \node[right] at (6,0) {\small Gumbel-max SCM Policy};
        \end{tikzpicture}
    }\\
    %
    \subfigure[\footnotesize Lowest cumulative reward: Interval CFMDP ($312$), Gumbel-max SCM ($312$)]{%
        \resizebox{0.76\columnwidth}{!}{
             \begin{tikzpicture}
                \begin{axis}[
                    xlabel={$t$},
                    ylabel={Mean reward at time step $t$},
                    title={Optimal Path},
                    grid=both,
                    width=20cm, height=8.5cm,
                    every axis/.style={font=\Huge},
                    %
                ]
                \addplot[
                    color=black, %
                    mark=*, %
                    line width=2pt,
                    mark size=3pt,
                    error bars/.cd,
                    y dir=both, %
                    y explicit, %
                    error bar style={line width=1pt,solid},
                    error mark options={line width=1pt,mark size=4pt,rotate=90}
                ]
                coordinates {
                    (0, 0.0)  +- (0, 0.0)
                    (1, 0.0)  +- (0, 0.0) 
                    (2, 1.0)  +- (0, 0.0) 
                    (3, 1.0)  +- (0, 0.0)
                    (4, 2.0)  +- (0, 0.0)
                    (5, 3.0) +- (0, 0.0)
                    (6, 5.0) +- (0, 0.0)
                    (7, 100.0) +- (0, 0.0)
                    (8, 100.0) +- (0, 0.0)
                    (9, 100.0) +- (0, 0.0)
                };
                %
                \addplot[
                    color=blue, %
                    mark=o, %
                    line width=2pt,
                    mark size=3pt,
                    error bars/.cd,
                    y dir=both, %
                    y explicit, %
                    error bar style={line width=1pt,solid},
                    error mark options={line width=1pt,mark size=4pt,rotate=90}
                ]
                 coordinates {
                    (0, 0.0)  +- (0, 0.0)
                    (1, 0.0)  +- (0, 0.0) 
                    (2, 1.0)  +- (0, 0.0) 
                    (3, 1.0)  +- (0, 0.0)
                    (4, 2.0)  +- (0, 0.0)
                    (5, 3.0) +- (0, 0.0)
                    (6, 5.0) +- (0, 0.0)
                    (7, 100.0) +- (0, 0.0)
                    (8, 100.0) +- (0, 0.0)
                    (9, 100.0) +- (0, 0.0)
                };
                %
                \addplot[
                    color=red, %
                    mark=x, %
                    line width=2pt,
                    mark size=6pt,
                    error bars/.cd,
                    y dir=both, %
                    y explicit, %
                    error bar style={line width=1pt,solid},
                    error mark options={line width=1pt,mark size=4pt,rotate=90}
                ]
                coordinates {
                    (0, 0.0)  +- (0, 0.0)
                    (1, 0.0)  +- (0, 0.0) 
                    (2, 1.0)  +- (0, 0.0) 
                    (3, 1.0)  +- (0, 0.0)
                    (4, 2.0)  +- (0, 0.0)
                    (5, 3.0) +- (0, 0.0)
                    (6, 5.0) +- (0, 0.0)
                    (7, 100.0) +- (0, 0.0)
                    (8, 100.0) +- (0, 0.0)
                    (9, 100.0) +- (0, 0.0)
                };
                \end{axis}
            \end{tikzpicture}
         }
    }
    \hspace{1cm}
    \subfigure[\footnotesize Lowest cumulative reward: Interval CFMDP ($19$), Gumbel-max SCM ($-88$)]{%
         \resizebox{0.76\columnwidth}{!}{
            \begin{tikzpicture}
                \begin{axis}[
                    xlabel={$t$},
                    ylabel={Mean reward at time step $t$},
                    title={Slightly Suboptimal Path},
                    grid=both,
                    width=20cm, height=8.5cm,
                    every axis/.style={font=\Huge},
                    %
                ]
                \addplot[
                    color=black, %
                    mark=*, %
                    line width=2pt,
                    mark size=3pt,
                    error bars/.cd,
                    y dir=both, %
                    y explicit, %
                    error bar style={line width=1pt,solid},
                    error mark options={line width=1pt,mark size=4pt,rotate=90}
                ]
              coordinates {
                    (0, 0.0)  +- (0, 0.0)
                    (1, 1.0)  +- (0, 0.0) 
                    (2, 1.0)  +- (0, 0.0) 
                    (3, 1.0)  +- (0, 0.0)
                    (4, 2.0)  +- (0, 0.0)
                    (5, 3.0) +- (0, 0.0)
                    (6, 3.0) +- (0, 0.0)
                    (7, 2.0) +- (0, 0.0)
                    (8, 2.0) +- (0, 0.0)
                    (9, 4.0) +- (0, 0.0)
                };
                %
                \addplot[
                    color=blue, %
                    mark=o, %
                    line width=2pt,
                    mark size=3pt,
                    error bars/.cd,
                    y dir=both, %
                    y explicit, %
                    error bar style={line width=1pt,solid},
                    error mark options={line width=1pt,mark size=4pt,rotate=90}
                ]
              coordinates {
                    (0, 0.0)  +- (0, 0.0)
                    (1, 1.0)  +- (0, 0.0) 
                    (2, 1.0)  +- (0, 0.0) 
                    (3, 1.0)  +- (0, 0.0)
                    (4, 2.0)  +- (0, 0.0)
                    (5, 3.0) +- (0, 0.0)
                    (6, 3.0) +- (0, 0.0)
                    (7, 2.0) +- (0, 0.0)
                    (8, 2.0) +- (0, 0.0)
                    (9, 4.0) +- (0, 0.0)
                };
                %
                \addplot[
                    color=red, %
                    mark=x, %
                    line width=2pt,
                    mark size=6pt,
                    error bars/.cd,
                    y dir=both, %
                    y explicit, %
                    error bar style={line width=1pt,solid},
                    error mark options={line width=1pt,mark size=4pt,rotate=90}
                ]
                coordinates {
                    (0, 0.0)  +- (0, 0.0)
                    (1, 1.0)  +- (0, 0.0) 
                    (2, 1.0)  +- (0, 0.0) 
                    (3, 1.0)  +- (0, 0.0)
                    (4, 2.0)  += (0, 0.0)
                    (5, 3.0)  += (0, 0.0)
                    (6, 3.17847) += (0, 0.62606746) -= (0, 0.62606746)
                    (7, 2.5832885) += (0, 1.04598233) -= (0, 1.04598233)
                    (8, 5.978909) += (0, 17.60137623) -= (0, 17.60137623)
                    (9, 5.297059) += (0, 27.09227512) -= (0, 27.09227512)
                };
                \end{axis}
            \end{tikzpicture}
         }
    }\\[-1.5pt]
    \subfigure[\footnotesize Lowest cumulative reward: Interval CFMDP ($14$), Gumbel-max SCM ($-598$)]{%
         \resizebox{0.76\columnwidth}{!}{
             \begin{tikzpicture}
                \begin{axis}[
                    xlabel={$t$},
                    ylabel={Mean reward at time step $t$},
                    title={Almost Catastrophic Path},
                    grid=both,
                    width=20cm, height=8.5cm,
                    every axis/.style={font=\Huge},
                    %
                ]
                \addplot[
                    color=black, %
                    mark=*, %
                    line width=2pt,
                    mark size=3pt,
                    error bars/.cd,
                    y dir=both, %
                    y explicit, %
                    error bar style={line width=1pt,solid},
                    error mark options={line width=1pt,mark size=4pt,rotate=90}
                ]
                coordinates {
                    (0, 0.0)  +- (0, 0.0)
                    (1, 1.0)  +- (0, 0.0) 
                    (2, 2.0)  +- (0, 0.0) 
                    (3, 1.0)  +- (0, 0.0)
                    (4, 0.0)  +- (0, 0.0)
                    (5, 1.0) +- (0, 0.0)
                    (6, 2.0) +- (0, 0.0)
                    (7, 2.0) +- (0, 0.0)
                    (8, 3.0) +- (0, 0.0)
                    (9, 2.0) +- (0, 0.0)
                };
                %
                \addplot[
                    color=blue, %
                    mark=o, %
                    line width=2pt,
                    mark size=3pt,
                    error bars/.cd,
                    y dir=both, %
                    y explicit, %
                    error bar style={line width=1pt,solid},
                    error mark options={line width=1pt,mark size=4pt,rotate=90}
                ]
                coordinates {
                    (0, 0.0)  +- (0, 0.0)
                    (1, 1.0)  +- (0, 0.0) 
                    (2, 2.0)  +- (0, 0.0) 
                    (3, 1.0)  +- (0, 0.0)
                    (4, 0.0)  +- (0, 0.0)
                    (5, 1.0) +- (0, 0.0)
                    (6, 2.0) +- (0, 0.0)
                    (7, 2.0) +- (0, 0.0)
                    (8, 3.0) +- (0, 0.0)
                    (9, 2.0) +- (0, 0.0)
                };
                %
                \addplot[
                    color=red, %
                    mark=x, %
                    line width=2pt,
                    mark size=6pt,
                    error bars/.cd,
                    y dir=both, %
                    y explicit, %
                    error bar style={line width=1pt,solid},
                    error mark options={line width=1pt,mark size=4pt,rotate=90}
                ]
                coordinates {
                    (0, 0.0)  +- (0, 0.0)
                    (1, 0.7065655)  +- (0, 0.4553358) 
                    (2, 1.341673)  +- (0, 0.67091621) 
                    (3, 1.122926)  +- (0, 0.61281824)
                    (4, -1.1821935)  +- (0, 13.82444042)
                    (5, -0.952399)  +- (0, 15.35195457)
                    (6, -0.72672) +- (0, 20.33508414)
                    (7, -0.268983) +- (0, 22.77861454)
                    (8, -0.1310835) +- (0, 26.31013314)
                    (9, 0.65806) +- (0, 28.50670214)
                };
                %
            %
            %
            %
            %
            %
            %
            %
            %
            %
            %
            %
            %
            %
            %
            %
            %
            %
            %
                \end{axis}
            \end{tikzpicture}
         }
    }
    \hspace{1cm}
    \subfigure[\footnotesize Lowest cumulative reward: Interval CFMDP ($-698$), Gumbel-max SCM ($-698$)]{%
         \resizebox{0.76\columnwidth}{!}{
            \begin{tikzpicture}
                \begin{axis}[
                    xlabel={$t$},
                    ylabel={Mean reward at time step $t$},
                    title={Catastrophic Path},
                    grid=both,
                    width=20cm, height=8.5cm,
                    every axis/.style={font=\Huge},
                    %
                ]
                \addplot[
                    color=black, %
                    mark=*, %
                    line width=2pt,
                    mark size=3pt,
                    error bars/.cd,
                    y dir=both, %
                    y explicit, %
                    error bar style={line width=1pt,solid},
                    error mark options={line width=1pt,mark size=4pt,rotate=90}
                ]
                coordinates {
                    (0, 1.0)  +- (0, 0.0)
                    (1, 2.0)  +- (0, 0.0) 
                    (2, -100.0)  +- (0, 0.0) 
                    (3, -100.0)  +- (0, 0.0)
                    (4, -100.0)  +- (0, 0.0)
                    (5, -100.0) +- (0, 0.0)
                    (6, -100.0) +- (0, 0.0)
                    (7, -100.0) +- (0, 0.0)
                    (8, -100.0) +- (0, 0.0)
                    (9, -100.0) +- (0, 0.0)
                };
                %
                \addplot[
                    color=blue, %
                    mark=o, %
                    line width=2pt,
                    mark size=3pt,
                    error bars/.cd,
                    y dir=both, %
                    y explicit, %
                    error bar style={line width=1pt,solid},
                    error mark options={line width=1pt,mark size=4pt,rotate=90}
                ]
                coordinates {
                    (0, 0.0)  +- (0, 0.0)
                    (1, 0.504814)  +- (0, 0.49997682) 
                    (2, 0.8439835)  +- (0, 0.76831917) 
                    (3, -8.2709165)  +- (0, 28.93656754)
                    (4, -9.981082)  +- (0, 31.66825363)
                    (5, -12.1776325) +- (0, 34.53463233)
                    (6, -13.556076) +- (0, 38.62845372)
                    (7, -14.574418) +- (0, 42.49603359)
                    (8, -15.1757075) +- (0, 46.41913968)
                    (9, -15.3900395) +- (0, 50.33563368)
                };
                %
                \addplot[
                    color=red, %
                    mark=x, %
                    line width=2pt,
                    mark size=6pt,
                    error bars/.cd,
                    y dir=both, %
                    y explicit, %
                    error bar style={line width=1pt,solid},
                    error mark options={line width=1pt,mark size=4pt,rotate=90}
                ]
                coordinates {
                    (0, 0.0)  +- (0, 0.0)
                    (1, 0.701873)  +- (0, 0.45743556) 
                    (2, 1.1227805)  +- (0, 0.73433129) 
                    (3, -8.7503255)  +- (0, 30.30257976)
                    (4, -10.722092)  +- (0, 33.17618589)
                    (5, -13.10721)  +- (0, 36.0648089)
                    (6, -13.7631645) +- (0, 40.56553451)
                    (7, -13.909043) +- (0, 45.23829402)
                    (8, -13.472517) +- (0, 49.96270296)
                    (9, -12.8278835) +- (0, 54.38618735)
                };
                %
            %
            %
            %
            %
            %
            %
            %
            %
            %
            %
            %
            %
            %
            %
            %
            %
            %
            %
                \end{axis}
            \end{tikzpicture}
         }
    }
    \caption{Average instant reward of CF paths induced by policies on GridWorld $p=0.4$.}
    \label{fig: reward p=0.4}
\end{figure*}

\subsection{Experimental Setup}
To compare policy performance, we measure the average rewards of counterfactual paths induced by our policy and the Gumbel-max policy by uniformly sampling $200$ counterfactual MDPs from the ICFMDP and generating $10,000$ counterfactual paths over each sampled CFMDP. \jl{Since the interval CFMDP depends on the observed path, we select $4$  paths of varying optimality to evaluate how the observed path impacts the performance of both policies: an optimal path, a slightly suboptimal path that could reach the optimal reward with a few changes, a catastrophic path that enters a catastrophic, terminal state with low reward, and an almost catastrophic path that was close to entering a catastrophic state.} When measuring the average probability bound widths and execution time needed to generate the ICFMDPs, we averaged over $20$ randomly generated observed paths
\footnote{Further training details are provided in Appendix \ref{app: training details}, and the code is provided at \href{https://github.com/ddv-lab/robust-cf-inference-in-MDPs}{https://github.com/ddv-lab/robust-cf-inference-in-MDPs}
%
%
.}.

\subsection{GridWorld}
\jl{The GridWorld MDP is a $4 \times 4$ grid where an agent must navigate from the top-left corner to the goal state in the bottom-right corner, avoiding a dangerous terminal state in the centre. At each time step, the agent can move up, down, left, or right, but there is a small probability (controlled by hyper-parameter $p$) of moving in an unintended direction. As the agent nears the goal, the reward for each state increases, culminating in a reward of $+100$ for reaching the goal. Entering the dangerous state results in a penalty of $-100$. We use two versions of GridWorld: a less stochastic version with $p=0.9$ (i.e., $90$\% chance of moving in the chosen direction) and a more stochastic version with $p=0.4$.}

\paragraph{GridWorld ($p=0.9$)}
When $p=0.9$, the counterfactual probability bounds are typically narrow (see Table \ref{tab:nonzero_probs} for average measurements). Consequently, as shown in Figure \ref{fig: reward p=0.9}, both policies are nearly identical and perform similarly well across the optimal, slightly suboptimal, and catastrophic paths.
%
However, for the almost catastrophic path, the interval CFMDP path is more conservative and follows the observed path more closely (as this is where the probability bounds are narrowest), which typically requires one additional step to reach the goal state than the Gumbel-max SCM policy.
%

\paragraph{GridWorld ($p=0.4$)}
\jl{When $p=0.4$, the GridWorld environment becomes more uncertain, increasing the risk of entering the dangerous state even if correct actions are chosen. Thus, as shown in Figure \ref{fig: reward p=0.4}, the interval CFMDP policy adopts a more conservative approach, avoiding deviation from the observed policy if it cannot guarantee higher counterfactual rewards (see the slightly suboptimal and almost catastrophic paths), whereas the Gumbel-max SCM is inconsistent: it can yield higher rewards, but also much lower rewards, reflected in the wide error bars.} For the catastrophic path, both policies must deviate from the observed path to achieve a higher reward and, in this case, perform similarly.
%
%
%
%
\subsection{Sepsis}
The Sepsis MDP \citep{oberst2019counterfactual} simulates trajectories of Sepsis patients. Each state consists of four vital signs (heart rate, blood pressure, oxygen concentration, and glucose levels), categorised as low, normal, or high.
and three treatments that can be toggled on/off at each time step (8 actions in total). Unlike \citet{oberst2019counterfactual}, we scale rewards based on the number of out-of-range vital signs, between $-1000$ (patient dies) and $1000$ (patient discharged). \jl{Like the GridWorld $p=0.4$ experiment, the Sepsis MDP is highly uncertain, as many states are equally likely to lead to optimal and poor outcomes. Thus, as shown in Figure \ref{fig: reward sepsis}, both policies follow the observed optimal and almost catastrophic paths to guarantee rewards are no worse than the observation.} However, improving the catastrophic path requires deviating from the observation. Here, the Gumbel-max SCM policy, on average, performs better than the interval CFMDP policy. But, since both policies have lower bounds clipped at $-1000$, neither policy reliably improves over the observation. In contrast, for the slightly suboptimal path, the interval CFMDP policy performs significantly better, shown by its higher lower bounds. 
Moreover, in these two cases, the worst-case counterfactual path generated by the interval CFMDP policy is better than that of the Gumbel-max SCM policy,
indicating its greater robustness.
%
\begin{figure*}
    \centering
     \resizebox{0.6\textwidth}{!}{
        \begin{tikzpicture}[scale=1.0, every node/.style={scale=1.0}]
            \draw[thick, black] (-3, -0.25) rectangle (10, 0.25);
            %
            \draw[black, line width=1pt] (-2.5, 0.0) -- (-2,0.0);
            \fill[black] (-2.25,0.0) circle (2pt); %
            \node[right] at (-2,0.0) {\small Observed Path};
            
            %
            \draw[blue, line width=1pt] (1.0,0.0) -- (1.5,0.0);
            \node[draw=blue, circle, minimum size=4pt, inner sep=0pt] at (1.25,0.0) {}; %
            \node[right] at (1.5,0.0) {\small Interval CFMDP Policy};
            
            %
            \draw[red, line width=1pt] (5.5,0) -- (6,0);
            \node[red] at (5.75,0) {$\boldsymbol{\times}$}; %
            \node[right] at (6,0) {\small Gumbel-max SCM Policy};
        \end{tikzpicture}
    }\\
    \subfigure[\footnotesize Lowest cumulative reward: Interval CFMDP ($8000$), Gumbel-max SCM ($8000$)]{%
         \resizebox{0.76\columnwidth}{!}{
             \begin{tikzpicture}
                \begin{axis}[
                    xlabel={$t$},
                    ylabel={Mean reward at time step $t$},
                    title={Optimal Path},
                    grid=both,
                    width=20cm, height=8.5cm,
                    every axis/.style={font=\Huge},
                    %
                ]
                \addplot[
                    color=black, %
                    mark=*, %
                    line width=2pt,
                    mark size=3pt,
                ]
                coordinates {
                    (0, -50.0)
                    (1, 50.0)
                    (2, 1000.0)
                    (3, 1000.0)
                    (4, 1000.0)
                    (5, 1000.0)
                    (6, 1000.0)
                    (7, 1000.0)
                    (8, 1000.0)
                    (9, 1000.0)
                };
                %
                \addplot[
                    color=blue, %
                    mark=o, %
                    line width=2pt,
                    mark size=3pt,
                    error bars/.cd,
                    y dir=both, %
                    y explicit, %
                    error bar style={line width=1pt,solid},
                    error mark options={line width=1pt,mark size=4pt,rotate=90}
                ]
                coordinates {
                    (0, -50.0)  +- (0, 0.0)
                    (1, 50.0)  +- (0, 0.0) 
                    (2, 1000.0)  +- (0, 0.0) 
                    (3, 1000.0)  +- (0, 0.0)
                    (4, 1000.0)  +- (0, 0.0)
                    (5, 1000.0) +- (0, 0.0)
                    (6, 1000.0) +- (0, 0.0)
                    (7, 1000.0) +- (0, 0.0)
                    (8, 1000.0) +- (0, 0.0)
                    (9, 1000.0) +- (0, 0.0)
                };
                %
                \addplot[
                    color=red, %
                    mark=x, %
                    line width=2pt,
                    mark size=6pt,
                    error bars/.cd,
                    y dir=both, %
                    y explicit, %
                    error bar style={line width=1pt,solid},
                    error mark options={line width=1pt,mark size=4pt,rotate=90}
                ]
                coordinates {
                    (0, -50.0)  +- (0, 0.0)
                    (1, 50.0)  +- (0, 0.0) 
                    (2, 1000.0)  +- (0, 0.0) 
                    (3, 1000.0)  +- (0, 0.0)
                    (4, 1000.0)  +- (0, 0.0)
                    (5, 1000.0) +- (0, 0.0)
                    (6, 1000.0) +- (0, 0.0)
                    (7, 1000.0) +- (0, 0.0)
                    (8, 1000.0) +- (0, 0.0)
                    (9, 1000.0) +- (0, 0.0)
                };
                %
                \end{axis}
            \end{tikzpicture}
         }
    }
    \hspace{1cm}
    \subfigure[\footnotesize Lowest cumulative reward: Interval CFMDP ($-5980$), Gumbel-max SCM ($-8000$)]{%
         \resizebox{0.76\columnwidth}{!}{
            \begin{tikzpicture}
                \begin{axis}[
                    xlabel={$t$},
                    ylabel={Mean reward at time step $t$},
                    title={Slightly Suboptimal Path},
                    grid=both,
                    width=20cm, height=8.5cm,
                    every axis/.style={font=\Huge},
                    %
                ]
               \addplot[
                    color=black, %
                    mark=*, %
                    line width=2pt,
                    mark size=3pt,
                ]
                coordinates {
                    (0, -50.0)
                    (1, 50.0)
                    (2, -50.0)
                    (3, -50.0)
                    (4, -1000.0)
                    (5, -1000.0)
                    (6, -1000.0)
                    (7, -1000.0)
                    (8, -1000.0)
                    (9, -1000.0)
                };
                %
                \addplot[
                    color=blue, %
                    mark=o, %
                    line width=2pt,
                    mark size=3pt,
                    error bars/.cd,
                    y dir=both, %
                    y explicit, %
                    error bar style={line width=1pt,solid},
                    error mark options={line width=1pt,mark size=4pt,rotate=90}
                ]
                coordinates {
                    (0, -50.0)  +- (0, 0.0)
                    (1, 50.0)  +- (0, 0.0) 
                    (2, -50.0)  +- (0, 0.0) 
                    (3, 20.0631)  +- (0, 49.97539413)
                    (4, 71.206585)  +- (0, 226.02033693)
                    (5, 151.60797) +- (0, 359.23292559)
                    (6, 200.40593) +- (0, 408.86185176)
                    (7, 257.77948) +- (0, 466.10372804)
                    (8, 299.237465) +- (0, 501.82579506)
                    (9, 338.9129) +- (0, 532.06124996)
                };
                %
                \addplot[
                    color=red, %
                    mark=x, %
                    line width=2pt,
                    mark size=6pt,
                    error bars/.cd,
                    y dir=both, %
                    y explicit, %
                    error bar style={line width=1pt,solid},
                    error mark options={line width=1pt,mark size=4pt,rotate=90}
                ]
                coordinates {
                    (0, -50.0)  +- (0, 0.0)
                    (1, 20.00736)  +- (0, 49.99786741) 
                    (2, -12.282865)  +- (0, 267.598755) 
                    (3, -47.125995)  +- (0, 378.41755832)
                    (4, -15.381965)  +- (0, 461.77616558)
                    (5, 41.15459) +- (0, 521.53189262)
                    (6, 87.01595) +- (0, 564.22243126 )
                    (7, 132.62376) +- (0, 607.31338037)
                    (8, 170.168145) +- (0, 641.48013693)
                    (9, 201.813135) +- (0, 667.29441777)
                };
                %
                %
                %
                %
                %
                %
                %
                %
                %
                %
                %
                %
                %
                %
                %
                %
                %
                %
                %
                \end{axis}
            \end{tikzpicture}
         }
    }\\[-1.5pt]
    \subfigure[\footnotesize Lowest cumulative reward: Interval CFMDP ($100$), Gumbel-max SCM ($100$)]{%
         \resizebox{0.76\columnwidth}{!}{
             \begin{tikzpicture}
                \begin{axis}[
                    xlabel={$t$},
                    ylabel={Mean reward at time step $t$},
                    title={Almost Catastrophic Path},
                    grid=both,
                    every axis/.style={font=\Huge},
                    width=20cm, height=8.5cm,
                    %
                ]
               \addplot[
                    color=black, %
                    mark=*, %
                    line width=2pt,
                    mark size=3pt,
                ]
                coordinates {
                    (0, -50.0)
                    (1, 50.0)
                    (2, 50.0)
                    (3, 50.0)
                    (4, -50.0)
                    (5, 50.0)
                    (6, -50.0)
                    (7, 50.0)
                    (8, -50.0)
                    (9, 50.0)
                };
                %
                %
                \addplot[
                    color=blue, %
                    mark=o, %
                    line width=2pt,
                    mark size=3pt,
                    error bars/.cd,
                    y dir=both, %
                    y explicit, %
                    error bar style={line width=1pt,solid},
                    error mark options={line width=1pt,mark size=4pt,rotate=90}
                ]
                coordinates {
                    (0, -50.0)  +- (0, 0.0)
                    (1, 50.0)  +- (0, 0.0) 
                    (2, 50.0)  +- (0, 0.0) 
                    (3, 50.0)  +- (0, 0.0)
                    (4, -50.0)  +- (0, 0.0)
                    (5, 50.0) +- (0, 0.0)
                    (6, -50.0) +- (0, 0.0)
                    (7, 50.0) +- (0, 0.0)
                    (8, -50.0) +- (0, 0.0)
                    (9, 50.0) +- (0, 0.0)
                };
                %
                \addplot[
                    color=red, %
                    mark=x, %
                    line width=2pt,
                    mark size=6pt,
                    error bars/.cd,
                    y dir=both, %
                    y explicit, %
                    error bar style={line width=1pt,solid},
                    error mark options={line width=1pt,mark size=4pt,rotate=90}
                ]
                coordinates {
                    (0, -50.0)  +- (0, 0.0)
                    (1, 50.0)  +- (0, 0.0) 
                    (2, 50.0)  +- (0, 0.0) 
                    (3, 50.0)  +- (0, 0.0)
                    (4, -50.0)  +- (0, 0.0)
                    (5, 50.0) +- (0, 0.0)
                    (6, -50.0) +- (0, 0.0)
                    (7, 50.0) +- (0, 0.0)
                    (8, -50.0) +- (0, 0.0)
                    (9, 50.0) +- (0, 0.0)
                };
                %
                %
                %
                %
                %
                %
                %
                %
                %
                %
                %
                %
                %
                %
                %
                %
                %
                %
                %
                \end{axis}
            \end{tikzpicture}
         }
    }
    \hspace{1cm}
    \subfigure[\footnotesize Lowest cumulative reward: Interval CFMDP ($-7150$), Gumbel-max SCM ($-9050$)]{%
         \resizebox{0.76\columnwidth}{!}{
            \begin{tikzpicture}
                \begin{axis}[
                    xlabel={$t$},
                    ylabel={Mean reward at time step $t$},
                    title={Catastrophic Path},
                    grid=both,
                    width=20cm, height=8.5cm,
                    every axis/.style={font=\Huge},
                    %
                ]
               \addplot[
                    color=black, %
                    mark=*, %
                    line width=2pt,
                    mark size=3pt,
                ]
                coordinates {
                    (0, -50.0)
                    (1, -50.0)
                    (2, -1000.0)
                    (3, -1000.0)
                    (4, -1000.0)
                    (5, -1000.0)
                    (6, -1000.0)
                    (7, -1000.0)
                    (8, -1000.0)
                    (9, -1000.0)
                };
                %
                %
                \addplot[
                    color=blue, %
                    mark=o, %
                    line width=2pt,
                    mark size=3pt,
                    error bars/.cd,
                    y dir=both, %
                    y explicit, %
                    error bar style={line width=1pt,solid},
                    error mark options={line width=1pt,mark size=4pt,rotate=90}
                ]
                coordinates {
                    (0, -50.0)  +- (0, 0.0)
                    (1, -50.0)  +- (0, 0.0) 
                    (2, -50.0)  +- (0, 0.0) 
                    (3, -841.440725)  += (0, 354.24605512) -= (0, 158.559275)
                    (4, -884.98225)  += (0, 315.37519669) -= (0, 115.01775)
                    (5, -894.330425) += (0, 304.88572805) -= (0, 105.669575)
                    (6, -896.696175) += (0, 301.19954514) -= (0, 103.303825)
                    (7, -897.4635) += (0, 299.61791279) -= (0, 102.5365)
                    (8, -897.77595) += (0, 298.80392585) -= (0, 102.22405)
                    (9, -897.942975) += (0, 298.32920557) -= (0, 102.057025)
                };
                %
                \addplot[
                    color=red, %
                    mark=x, %
                    line width=2pt,
                    mark size=6pt,
                    error bars/.cd,
                    y dir=both, %
                    y explicit, %
                    error bar style={line width=1pt,solid},
                    error mark options={line width=1pt,mark size=4pt,rotate=90}
                ]
            coordinates {
                    (0, -50.0)  +- (0, 0.0)
                    (1, -360.675265)  +- (0, 479.39812699) 
                    (2, -432.27629)  +- (0, 510.38620897) 
                    (3, -467.029545)  += (0, 526.36009628) -= (0, 526.36009628)
                    (4, -439.17429)  += (0, 583.96638919) -= (0, 560.82571)
                    (5, -418.82704) += (0, 618.43027478) -= (0, 581.17296)
                    (6, -397.464895) += (0, 652.67322574) -= (0, 602.535105)
                    (7, -378.49052) += (0, 682.85407033) -= (0, 621.50948)
                    (8, -362.654195) += (0, 707.01412023) -= (0, 637.345805)
                    (9, -347.737935) += (0, 729.29076479) -= (0, 652.262065)
                };
                %
                %
                %
                %
                %
                %
                %
                %
                %
                %
                %
                %
                %
                %
                %
                %
                %
                %
                %
                \end{axis}
            \end{tikzpicture}
         }
    }
    \caption{Average instant reward of CF paths induced by policies on Sepsis.}
    \label{fig: reward sepsis}
\end{figure*}

%
%
%
\subsection{Interval CFMDP Bounds}
%
%
Table \ref{tab:nonzero_probs} presents the mean counterfactual probability bound widths (excluding transitions where the upper bound is $0$) for each MDP, averaged over 20 observed paths. We compare the bounds under counterfactual stability (CS) and monotonicity (M) assumptions, CS alone, and no assumptions. This shows that the assumptions marginally reduce the bound widths, indicating the assumptions tighten the bounds without excluding too many causal models, as intended.
\renewcommand{\arraystretch}{1}

\begin{table}
\centering
\caption{Mean width of counterfactual probability bounds}
\resizebox{0.8\columnwidth}{!}{%
\begin{tabular}{|c|c|c|c|}
\hline
\multirow{2}{*}{\textbf{Environment}} & \multicolumn{3}{c|}{\textbf{Assumptions}} \\ \cline{2-4}
 & \textbf{CS + M} & \textbf{CS} & \textbf{None\tablefootnote{\jl{Equivalent to \citet{li2024probabilities}'s bounds (see Section \ref{sec: equivalence with Li}).}}} \\ \hline
\textbf{GridWorld} ($p=0.9$) & 0.0817 & 0.0977 & 0.100 \\ \hline
\textbf{GridWorld} ($p=0.4$) & 0.552  & 0.638  & 0.646 \\ \hline
\textbf{Sepsis} & 0.138 & 0.140 & 0.140 \\ \hline
\end{tabular}
}
\label{tab:nonzero_probs}
\end{table}


\subsection{Execution Times}
Table \ref{tab: times} compares the average time needed to generate the interval CFMDP vs.\ the Gumbel-max SCM CFMDP for 20 observations.
The GridWorld algorithms were run single-threaded, while the Sepsis experiments were run in parallel.
Generating the interval CFMDP is significantly faster as it uses exact analytical bounds, whereas the Gumbel-max CFMDP requires sampling from the Gumbel distribution to estimate counterfactual transition probabilities. \jl{Since constructing the counterfactual MDP models is the main bottleneck in both approaches, ours is more efficient overall and suitable for larger MDPs.}
\begin{table}
\centering
\caption{Mean execution time to generate CFMDPs}
\resizebox{0.99\columnwidth}{!}{%
\begin{tabular}{|c|c|c|}
\hline
\multirow{2}{*}{\textbf{Environment}} & \multicolumn{2}{c|}{\textbf{Mean Execution Time (s)}} \\ \cline{2-3} 
                                      & \textbf{Interval CFMDP} & \textbf{Gumbel-max CFMDP} \\ \hline
\textbf{GridWorld ($p=0.9$) }                  & 0.261                   & 56.1                      \\ \hline
\textbf{GridWorld ($p=0.4$)  }                 & 0.336                   & 54.5                      \\ \hline
\textbf{Sepsis}                                 & 688                     & 2940                      \\ \hline
\end{tabular}%
}
\label{tab: times}
\end{table}


\section{Conclusion}
In this work, we propose a simple yet effective approach, called SMILE, for graph few-shot learning with fewer tasks. Specifically, we introduce a novel dual-level mixup strategy, including within-task and across-task mixup, for enriching the diversity of nodes within each task and the diversity of tasks. Also, we incorporate the degree-based prior information to learn expressive node embeddings. Theoretically, we prove that SMILE effectively enhances the model's generalization performance. Empirically, we conduct extensive experiments on multiple benchmarks and the results suggest that SMILE significantly outperforms other baselines, including both in-domain and cross-domain few-shot settings.

\clearpage
%%
%% The next two lines define the bibliography style to be used, and
%% the bibliography file.
\bibliographystyle{plain}
\bibliography{main}

%%
%% If your work has an appendix, this is the place to put it.
% \appendix

\end{document}
\endinput
%%
%% End of file `sample-authordraft.tex'.
