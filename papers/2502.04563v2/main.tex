%%
%% This is file `sample-manuscript.tex',
%% generated with the docstrip utility.
%%
%% The original source files were:
%%
%% samples.dtx  (with options: `manuscript')
%% 
%% IMPORTANT NOTICE:
%% 
%% For the copyright see the source file.
%% 
%% Any modified versions of this file must be renamed
%% with new filenames distinct from sample-manuscript.tex.
%% 
%% For distribution of the original source see the terms
%% for copying and modification in the file samples.dtx.
%% 
%% This generated file may be distributed as long as the
%% original source files, as listed above, are part of the
%% same distribution. (The sources need not necessarily be
%% in the same archive or directory.)
%%
%% Commands for TeXCount
%TC:macro \cite [option:text,text]
%TC:macro \citep [option:text,text]
%TC:macro \citet [option:text,text]
%TC:envir table 0 1
%TC:envir table* 0 1
%TC:envir tabular [ignore] word
%TC:envir displaymath 0 word
%TC:envir math 0 word
%TC:envir comment 0 0
%%
%%
%% The first command in your LaTeX source must be the \documentclass command.
%%%% Small single column format, used for CIE, CSUR, DTRAP, JACM, JDIQ, JEA, JERIC, JETC, PACMCGIT, TAAS, TACCESS, TACO, TALG, TALLIP (formerly TALIP), TCPS, TDSCI, TEAC, TECS, TELO, THRI, TIIS, TIOT, TISSEC, TIST, TKDD, TMIS, TOCE, TOCHI, TOCL, TOCS, TOCT, TODAES, TODS, TOIS, TOIT, TOMACS, TOMM (formerly TOMCCAP), TOMPECS, TOMS, TOPC, TOPLAS, TOPS, TOS, TOSEM, TOSN, TQC, TRETS, TSAS, TSC, TSLP, TWEB.
% \documentclass[acmsmall]{acmart}

%%%% Large single column format, used for IMWUT, JOCCH, PACMPL, POMACS, TAP, PACMHCI
% \documentclass[acmlarge,screen]{acmart}

%%%% Large double column format, used for TOG
% \documentclass[acmtog, authorversion]{acmart}

%%%% Generic manuscript mode, required for submission
%%%% and peer review
% \documentclass[manuscript,screen,review]{acmart}
%% Fonts used in the template cannot be substituted; margin 
%% adjustments are not allowed.

% \documentclass[sigplan,authordraft]{acmart}
\documentclass[letterpaper,twocolumn,10pt]{article}
\usepackage{usenix}
\usepackage{titling}
\usepackage{tikz}
\usepackage{multirow} 
\usepackage{pifont}
\usepackage{amsmath}
\usepackage{amssymb}
\usepackage{url}
\usepackage{subcaption}
\usepackage{graphicx}
\usepackage{booktabs}
\usepackage{algpseudocode}
\usepackage[linesnumbered,ruled,vlined]{algorithm2e}
\usepackage{tabularx}
\usepackage{authblk}
% \usepackage{draftwatermark}
% \usepackage{titling}
\microtypecontext{spacing=nonfrench}

% \SetWatermarkText{Unpublished working draft \\ Not for distribution}
% \SetWatermarkText{}
% \SetWatermarkScale{0.4}
% \SetWatermarkLightness{0.8}

\usepackage{enumitem} % Add this line to your preamble
\usepackage{pifont}
\usepackage{placeins}
% Define commands for tick and cross
\newcommand{\tick}{\ding{51}} % Tick symbol ✔
\newcommand{\cross}{\ding{55}} % Cross symbol ✘
\newcommand{\mixed}{\ding{51}\kern-0.62em\ding{55}} % Mixed symbol (tick and cross overlay)

\usepackage{tikz}
\newcommand*\circled[1]{%
  \scalebox{0.78}{\begin{tikzpicture}[baseline=-3pt]
    \node[draw,circle,inner sep=0.5pt, fill=black] {\textcolor{white}{\textsf{\textbf{#1}}}};
  \end{tikzpicture}}}

\usepackage{xcolor}
\newcommand{\yeqi}[1]{\textcolor{orange}{Yeqi: #1}}
\newcommand{\cj}[1]{\textcolor{blue}{Congjie: #1}}

% reduce space below title
 \usepackage{etoolbox}
 
 \makeatletter
 \patchcmd{\maketitle}
 	{\@maketitle}
 	{\vspace{-7em}\@maketitle\vspace{-5em}}% change the value as needed
 	{}
 	{}
 \makeatother

\newcommand{\todo}[1]{{\textcolor{red}{[~TODO:~#1~]}}}
\newcommand{\ncite}[1]{{\textcolor{blue}{[~Cite]}}}


% \makeatletter
% \def\UrlAlphabet{%
%       \do\a\do\b\do\c\do\d\do\e\do\f\do\g\do\h\do\i\do\j%
%       \do\k\do\l\do\m\do\n\do\o\do\p\do\q\do\r\do\s\do\t%
%       \do\u\do\v\do\w\do\x\do\y\do\z\do\A\do\B\do\C\do\D%
%       \do\E\do\F\do\G\do\H\do\I\do\J\do\K\do\L\do\M\do\N%
%       \do\O\do\P\do\Q\do\R\do\S\do\T\do\U\do\V\do\W\do\X%
%       \do\Y\do\Z}
% \def\UrlDigits{\do\1\do\2\do\3\do\4\do\5\do\6\do\7\do\8\do\9\do\0}
% \g@addto@macro{\UrlBreaks}{\UrlOrds}
% \g@addto@macro{\UrlBreaks}{\UrlAlphabet}
% \g@addto@macro{\UrlBreaks}{\UrlDigits}
% \makeatother

\algdef{SE}[LOOP]{Loop}{EndLoop}[1]{\textbf{loop} #1}{}

\setlength{\droptitle}{-2cm}

\begin{document}

\newcommand{\meshtp}{MeshTP\xspace}
\newcommand{\meshpp}{MeshPP\xspace}
\newcommand{\gemm}{MeshGEMM\xspace}
\newcommand{\gemv}{MeshGEMV\xspace}
\newcommand{\sys}{WaferLLM\xspace}


% \usepackage{glossaries}
% \makeglossaries
% \usepackage{tikz}

\newcommand{\para}[1]{\noindent\textbf{#1}}

\newcommand{\tinyskip}{\vspace{3pt}}
\newcommand{\mypar}[1]{\tinyskip\noindent\textbf{#1.}\xspace}
\newcommand{\myitem}[1]{\item\textbf{#1.}\xspace}

\newcommand*\myc[1]{%
\scalebox{0.78}{\begin{tikzpicture}[baseline=-3pt]
  \node[draw,circle,inner sep=0.5pt, fill=black] {\textcolor{white}{\textsf{\textbf{#1}}}};
\end{tikzpicture}}}

% \usepackage{mathtools}
% \usepackage{xspace}

% Add a period to the end of an abbreviation unless there's one
% already, then \xspace.
\makeatletter
\DeclareRobustCommand\onedot{\futurelet\@let@token\@onedot}
\def\@onedot{\ifx\@let@token.\else.\null\fi\xspace}

\def\eg{\emph{e.g}\onedot} \def\Eg{\emph{E.g}\onedot}
\def\ie{\emph{i.e}\onedot} \def\Ie{\emph{I.e}\onedot}
\def\cf{\emph{c.f}\onedot} \def\Cf{\emph{C.f}\onedot}
\def\etc{\emph{etc}\onedot} \def\vs{\emph{vs}\onedot}
\def\wrt{w.r.t\onedot} \def\dof{d.o.f\onedot}
\def\etal{\emph{et al}\onedot}
\makeatother
\newcommand{\Sum}[3]{\sum\limits_{#1}^{#2}{#3}}
\newcommand{\lr}[3]{\left #1 {#3} \right #2}
\newcommand{\E}[2]{E[#1,#2]}

\newenvironment{tightlist}{
\begin{list}{$\bullet$}{
%    \setlength{\topsep}{0in}
    \setlength{\topsep}{.1em}
    \setlength{\partopsep}{0in}
    \setlength{\parskip}{0in}
    \setlength{\itemsep}{0in}
    \setlength{\parsep}{0in}
    % \setlength{\leftmargin}{1.5em}
    \setlength{\leftmargin}{1em}
    \setlength{\rightmargin}{0in}
    \setlength{\itemindent}{0in}
}}
{\end{list}}

% \captionsetup[subfigure]{justification=centering}


%%
%% The "title" command has an optional parameter,
%% allowing the author to define a "short title" to be used in page headers.
\title{\sys: A Wafer-Scale LLM Inference System}
\date{}


% \author{\#203}
% \author[1]{Congjie He\thanks{congjie.he@ed.ac.uk}}
\author{
    Congjie He\textsuperscript{1}, 
    Yeqi Huang\textsuperscript{1}, 
    Pei Mu\textsuperscript{1}, 
    Ziming Miao\textsuperscript{2}, 
    Jilong Xue\textsuperscript{2}, 
    Lingxiao Ma\textsuperscript{2}, 
    Fan Yang\textsuperscript{2}, 
    Luo Mai\textsuperscript{1}
}

\affil{\textsuperscript{1}University of Edinburgh \hspace{1cm} \textsuperscript{2}Microsoft Research}

\maketitle

\begin{abstract}
Emerging AI accelerators increasingly adopt wafer-scale manufacturing technologies, integrating hundreds of thousands of AI cores in a mesh-based architecture with large distributed on-chip memory (tens of GB in total) and ultra-high on-chip memory bandwidth (tens of PB/s). However, current LLM inference systems, optimized for shared memory architectures like GPUs, fail to fully exploit these accelerators.

We introduce \sys, the first wafer-scale LLM inference system. \sys is guided by a novel PLMR model (pronounced as "Plummer") that captures the unique hardware characteristics of wafer-scale architectures. Leveraging this model, \sys pioneers wafer-scale LLM parallelism, optimizing the utilization of hundreds of thousands of on-chip cores. It also introduces MeshGEMM and MeshGEMV, the first GEMM and GEMV implementations designed to scale effectively on wafer-scale accelerators.

Evaluations show that \sys achieves 200$\times$ better wafer-scale accelerator utilization than state-of-the-art systems. On a commodity wafer-scale accelerator, \sys delivers 606$\times$ faster and 22$\times$ more energy-efficient GEMV compared to an advanced GPU. For LLMs, based on 16-bit data type, \sys achieves 2700 toks/sec/req decode speed on Llama3-8B model and 840 toks/sec/req decode speed on Qwen2-72B model, which enables 39$\times$ faster decoding with 1.7$\times$ better energy efficiency. We anticipate these numbers will grow significantly as wafer-scale AI models, software, and hardware continue to mature.

\end{abstract}

\documentclass[../main.tex]{subfiles}
\graphicspath{{../images/}}
\makeatletter
\def\input@path{{../images/}}
\makeatother
\begin{document}
\section{Introduction}
\begin{figure}
\centering
\begin{tikzpicture}
\node[inner sep=0pt] (ws) at (0, 0) {
\includegraphics[height=.4\textwidth, trim={10cm 0 10cm 0},clip]{world_space.png}};
\node[inner sep=0pt] (cs) at (6,0) {\includegraphics[height=.4\textwidth, trim={10cm 1cm 10cm 4cm},clip]{conf_space.png}};
\end{tikzpicture}
\vspace{-5pt}
\label{fig:pbrm_intro}
\caption{\textbf{Left}: Shows world space obstacles as grey spheres. Robots start and goal configuration is colored red and green, respectively. Configurations along the computed path are colored transparent blue. \textbf{Right:} Mapped world space scenario to configuration space. Obstacle region is the grey mesh. Red spheres are collision-free regions computed by the neural SCDF. The optimized shortest path in the convex corridor is the blue curve.}
\vspace{-25pt}
\end{figure}
Motion planning is the problem of finding a collision-free trajectory that connects a given start and goal configuration. The planning takes place in the configuration space of the robot. For single body robots, like mobile robots or drones, the configuration space and the world space are usually the same. This simplifies the planning, since explicit obstacle representations are available which enables geometrical tools like separating hyperplanes, smallest distance to obstacles etc., to be used when designing motion planning algorithms. For multi-body robots like manipulators, the situation is completely different. The world space obstacles are usually mapped to non-convex regions, and to make the problem even harder, the mapping is usually not known. Forming explicit representations of the obstacle region in the configuration space is usually too expensive or intractable. Despite all of this, sampling based planners are used with great success, which mainly is due to their use of implicit representations of the obstacle region. The basic idea is to construct a graph in the configuration space that covers and connects the collision-free region. From this graph, a path can be extracted that connects a given start and goal configuration. The approach is computationally expensive, since the graph is constructed with the smallest geometrical building block available, points, which represents a collision-check. Furthermore, the extracted paths from the graph are non-smooth and jagged due to the stochastic nature of the approach. This adds an additional post-processing step to the process, where the paths are shortcutted and smoothened, before the path can be used for tracking. Clearly a lot of time is invested to form this graph and produce smooth paths. Thus, if the obstacles start to move, then all of this work is done in no use, since all points that make up this graph need to be re-verified, which is simply too time consuming to be done in real time.
\\\\
In this work, we want to address the existing drawbacks of the sampling based planners. Our main contribution is an improved motion planner where each vertex in the graph covers a collision-free region in the form of a sphere instead of a point and where the edges are formed with neighboring intersecting spheres. This representation has the advantage of instead of returning piecewise linear paths, returning a sequence of overlapping spheres, i.e. a convex corridor, that connects a given start and goal configuration, illustrated in Figure \ref{fig:pbrm_intro}. This convex corridor allows us to use convex optimization to produce smooth trajectories, instead of computationally expensive post-processing methods. The representation further allows us to estimate the coverage of the collision-free space, which gives us awareness and feedback in the offline roadmap construction phase. Finally, our representation is simple to adapt to moving obstacles, simply requery for the new radii and recheck for intersections. 
\\\\
The spherical collision-free regions are formed using a signed distance function (SDF), which is a function that returns the smallest distance from an arbitrary point to the boundary of an obstacle. As the name implies, the distance is signed, thus if the point is inside the obstacle it is negative otherwise positive. If the distance is positive, a sphere with radius equal to the distance is guaranteed to cover a collision-free region. Using an SDF in motion planning is not new, but what is novel about our approach is that we express the distance in the configuration space instead of the world space and by doing so allows us to form these convex collision-free regions. We refer to the resulting SDF as a signed configuration distance function (SCDF). Computing an SCDF analytically is non-trivial, our approach is therefore to parameterize the SCDF with a deep neural network and learn the mapping by supervised learning. Our resulting neural SCDF can compute distances for different parameter values of obstacle shapes and we also show how multiple distances can be combined, thus making our approach flexible.
\section{Related work}
Motion planning algorithms can roughly be divided into three families, grid-based, sampling based and optimization based methods. Grid-based methods (GBM) discretize the planning space from which a graph is then compiled. A standard search method is A$^\star$ \citep{a_star}, which is classified as an \textit{informed} search method, since it employs a heuristic function to speed up the search. A$^\star$ guarantees to return an optimal path at the level of discretization used. GBMs usually discretize the planning space by a regular lattice and this limits the GBMs to problems with low dimensionality due to the curse of dimensionality. Thus, GBMs are usually limited to single-body robots where the degrees of freedom (DOF) are low. To overcome the inherent scaling problem with the GBMs, stochastic methods are usually used for multi-body robots. These methods are termed as sampling-based methods (SBM) and core members within this family are the rapidly-exploring random trees (RRT) \citep{rrt} and the probabilistic roadmap (PRM) \citep{prm}. RRT grows a tree from the start configuration and explores the collision-free region in a rapid way until it is able to connect to the goal region. RRT is usually improved by bi-directional planning \citep{rrt_connect}, i.e. an additional tree is grown from the goal configuration and the trees are tested for connection after any tree has been expanded. RRT is a single-query method, thus it searches for a path from scratch each time it is queried. Contrary to this, PRM is a multi-query method, which solves for multiple queries without starting from scratch. PRM does this by creating a roadmap (graph) that covers the collision-free space as an offline step. The graph is then used to solve for multiple queries. PRMs are used in cases where the environment does not change since the extra offline step is too computationally costly and needs to be re-done if the environment is changed. In our work, we address this inherent issue by using a different roadmap representation. Our vertices in the graph cover a collision-free region in the form of spheres and we form the edges by checking for intersecting spheres. If something in the environment changes, we recompute the spheres radii and recheck the intersections, without relying on collision detection. We use a trained neural network to compute the sphere radius, therefore querying for the radius can be done fast, hence our representation enables the PRM for dynamic environments.
\\\\
In the recent decades, optimization based methods (OBM) \citep{chomp, schulman, itomp, stomp} have been introduced as an alternative to SBM for multi-body robots. Like the SBM, the OBMs scale well to higher dimensional problems and produce smoother motion. It is common to use a SDF in the optimization since it is a smooth function, thus enabling gradient-based methods. However, the standard way of expressing the SDF is in world space. The distance therefore needs to be mapped to the configuration space by the forward kinematics. This mapping makes the optimization problem a non-linear program (NLP), which is computationally expensive to solve. Recently, a different approach has been proposed. In \cite{mp_gcs} motion planning is formulated as a convex optimization problem by using the graph of convex sets framework \citep{gcs}. The underlying idea is to decompose the collision-free space into intersecting convex sets from which a convex optimization problem is formulated. In cases where an explicit representation of the obstacles in the configuration space exists, like for single-body robots, creating collision-free convex regions can be done fast \citep{iris}. For multi-body robots, this is non-trivial. Existing work does this successfully \citep{iris_nlp, iris_c} by an optimization based approach, but the methods are still too time consuming to be used in the presence of moving obstacles. Our approach is instead to use deep learning to learn an SDF expressed in the configuration space. With this, we can query for shortest distances to the collision boundary, which allows us to expand spherical regions which are collision-free. Our approach is fast and therefore enables our suggested roadmap planner to be used in dynamic environments.
\\\\
Recent research has focused on learning collision detection \citep{fk_kernel_distance, diffco, graphdistnet} by predicting the signed distance between the robot links and the surrounding obstacles in the world space. The learned SDF is used in trajectory optimization but since the distance is expressed in the world space, the problem becomes an NLP and therefore takes a long time to solve. We take a novel approach and suggest to instead express the signed distance in the configuration space. This allows us to improve the PRM at the same time as it enables convex optimization for trajectory optimization, which runs faster and is more reliable than NLP solvers. In \cite{cspf} a learned signed distance function in the configuration space is proposed similar to our approach. However, their approach is restricted to point cloud representations, while we propose to represent the obstacles as parameterized geometric shapes, e.g. spheres. Furthermore, we also show how to use our learned SCDF to improve an existing roadmap planner.
\section{Problem formulation}
A robot is located in the world space, $\W \subset \R^3 $. The unique location of the robot is given by its configuration $\q \in \C$, where $\C$ is the configuration space. The set of points covered by the robots bodies at a certain configuration is expressed as $\B(\q) \subset \W$. The robot is surrounded by $\NrObst$ obstacles $\O = \bigcup_{i=1}^{\NrObst} \O_i$, where  $\O_i \subset \W$. The representation of the obstacle in the configuration space is the set $\C\O_i = \{\q \in \C \: |\: \B(\q) \cap \O_i \neq \emptyset \}$. The obstacle space is formed as $\Co = \bigcup_{i=1}^{\NrObst} \C \O_i$. The complement is referred to as the free space, $\Cf = \C \setminus \Co$. The path planning problem is a tuple, ($\Cf$, $\qStart$, $\qGoal$), where we want to connect a query pair, consisting of a start, $\qStart$, and goal configuration, $\qGoal$, with a geometric path, $\q(s): [0, 1] \mapsto \Cf$, such that $\q(0)=\qStart$ and $\q(1)=\qGoal$, or report correctly when such a path does not exist.
\end{document}


\section{Background and Motivation} \label{sec:background_motivation}

\subsection{LLM Training \& Token Filtering} \label{sec:background_llm}

\begin{figure}[t]
	\centering
	\includegraphics[scale=0.55]{figures/llm_training.pdf}
	\caption{An overview of LLM training.}
	\label{fig:llm_training}
    % \vspace{+2mm}
\end{figure}

Training LLMs is a computationally intensive process that that demands substantial computational resources. Figure~\ref{fig:llm_training} shows an overview of the LLM training process. The training data is first tokenized and fed into the LLM, which consists of multiple transformer layers. The model processes the input data and generates predictions, which are compared to the ground truth labels (\ie, next tokens) to compute the loss. Finally, gradients are computed based on the loss to update the model parameters. 
Two primary factors significantly influence the computational cost: the size of the model (\eg, the number of layers) and the number of training tokens. For instance, training foundation models like LLaMA3-70B requires approximately 7 million GPU hours and involves processing more than 15 trillion tokens. Additionally, LLM-based applications necessitate extensive domain knowledge to fine-tune the model, which can also be computationally expensive—particularly for applications that require frequent updates (\eg, LLM-based recommender systems~\cite{DBLP:conf/sigir/LinWLYFWC24}).
Accelerating LLM training is crucial to enable faster application development, reducing costs, and minimizing the environmental impact of training LLMs.

Existing studies have explored various techniques to accelerate LLM training. However, many of them either leave limited room for further improvement or adversely affect model utility. Specifically, distributed training systems~\cite{MegatronLM,MegaScale} have been proposed to effectively leverage computational resources in parallel and reduce idle time in the computation pipeline by overlapping communication and data input/output (I/O) with computations. State-of-the-art LLM training systems~\cite{MegaScale} have achieved 55.2\% model FLOPs utilization (MFU) while training on more than 10,000 GPUs. Further enhancing the utilization rate of hardware remains a challenging task with limited room for improvement.
Techniques such as layer freezing~\cite{SmartFRZ,yiding-layer-freezing}, model pruning~\cite{DBLP:conf/nips/MaFW23,DBLP:conf/iclr/Sun0BK24}, and low-rank fine-tuning~\cite{LoRA} have been explored to reduce the number of trainable model parameters and improve training efficiency. However, decreasing the number of trainable parameters may negatively impact the model's utility and generalization ability~\cite{DBLP:journals/natmi/DingQYWYSHCCCYZWLZCLTLS23}.

Token filtering is a recently proposed technology that has been well recognized by the AI community. The core idea is to identify and filter out tokens that are either noisy or unlikely to contribute meaningfully to the training process, which implicitly improves the quality of training data to benefit the model utility. Moreover, by reducing the total number of tokens to be trained, token filtering also brings opportunity for efficiency improvement.

Existing token filtering works can be categorized into two types: \emph{forward token filtering} and \emph{backward token filtering}. As illustrated in \Cref{fig:token_filter_intro}, forward token filtering techniques remove training tokens during the forward pass, whereas backward token filtering methods eliminate tokens exclusively during the backward pass.

\begin{figure}[t]
	\centering
	\includegraphics[scale=0.6]{figures/token_filter_intro.pdf}
	\caption{An overview of existing token filter studies. Forward token filtering methods (a) filter hidden states during forward process, while backward token filtering methods (b) filter training loss during the backward process.}
	\label{fig:token_filter_intro}
    \vspace{+2mm}
\end{figure}

Forward token filtering methods have been extensively studied in previous works~\cite{DBLP:conf/acl/HouPZWSSZ22,DBLP:conf/acl/ZhongDL0ZDT23,DBLP:journals/corr/abs-2211-11586,DBLP:journals/corr/abs-2401-15293}. However, they typically underperform compared to backward filtering methods due to semantic losses~\cite{DBLP:conf/acl/ZhongDL0ZDT23,DBLP:journals/corr/abs-2211-11586,RHO}. As shown in \Cref{fig:token_filter_intro}, forward token filtering methods filter tokens at each layer of the forward computation, such that each layer of the model only processes partial context. However, this approach has been shown to cause semantic loss and potential harm model utility~\cite{DBLP:conf/acl/ZhongDL0ZDT23,DBLP:journals/corr/abs-2211-11586}. Evaluations in existing forward filtering studies~\cite{DBLP:conf/acl/HouPZWSSZ22,DBLP:conf/acl/ZhongDL0ZDT23,DBLP:journals/corr/abs-2211-11586,DBLP:journals/corr/abs-2401-15293} report only similar or lower model utilities and fail to achieve the improvements in utility seen with backward filtering methods~\cite{RHO}.

Backward token filtering is an effective solution for enhancing model utility and is widely accepted within the AI community. Existing work~\cite{RHO} has demonstrated that backward filtering methods can not only reduce the number of training tokens processed during the backward pass but also improve model utility by eliminating inconsequential tokens. As illustrated in \Cref{fig:token_filter_intro}, the backward filtering method maintains standard forward computation while performing selective token training in the output layer.
Existing studies leverage a reference model to assess the importance of each token. For instance, when training a target model to enhance mathematical reasoning, the reference model is trained on a small but high-quality mathematical corpus (\eg, clean datasets with clear instructions and derivations). During the training process, the loss of the target model (\ie, the model being trained) is compared to the loss of the reference model. Tokens with high excessive loss (\ie, the loss of the target model minus the loss of the reference model) are considered important, while those with lower excessive loss are filtered out during the backward pass.
Empirically, tokens with high excessive loss have larger room to be trained, and lower loss in the reference model also indicates that the tokens match the distribution of high-quality data. Mathematically, backward token filtering can be formulated as follows~\cite{RHO}:
\begin{equation}
    \mathcal{L}_{filter} = -\frac{1}{N \times k\%} \sum^N_{i=1} I_{k\%}(\mathbf{x}_i) \log P_{\theta}(\mathbf{x}_i|\mathbf{x}_{<i};\theta)
\end{equation}
\begin{equation}
	I_{k\%}(\mathbf{x}_i) = \left\{
	\begin{aligned}
		1, & \ if \ \mathbf{x}_i \ \in \ top \ k\% \ of \ (\mathcal{L}_{\theta}(\mathbf{x}_i)-\mathcal{L}_{ref}(\mathbf{x}_i)) \\
		0, & \ \text{otherwise}
	\end{aligned}
	\right.
\end{equation}
where $\mathcal{L}_{\theta}$ is the loss of the target model, $\mathcal{L}_{ref}$ is the loss of the reference model, and $\mathcal{L}_{filter}$ is the actual loss to train the target model while keeping $k\%$ of tokens.

In this paper, we mainly focus on backward token filtering due to its aforementioned advantages.

\subsection{Existing Token Filtering Fails to Improve Efficiency} \label{sec:efficiency_motivation}

Although backward token filtering has shown promising results in improving model utility, its potential of improving training efficiency remains unexplored. In principle, reducing the number of training tokens should bring significant efficiency improvement due to the reduced computation workload. However, existing studies fail to improve training efficiency due to the following two reasons: (1) insufficient sparsity after token filtering; and (2) inefficiency of sparse GEMM implementations.

\parab{Insufficient sparsity after token filtering.} 
The potential for efficiency improvements in token filtering methods arises from the sparsity achieved by filtering out unimportant tokens. The gradients of the filtered tokens become zero, allowing for a reduction in computational costs during the backward process. Essentially, backpropagation involves computations between gradients and activations, which are intermediate results specifically stored for the backward pass.
Current methods filter the loss of unimportant tokens at the output layer, resulting in sparse gradients. However, they leave all dense activations unchanged. Consequently, after being multiplied by these dense activations, the gradients are no longer sparse once they pass through the final attention block. Therefore, existing backward filtering methods~\cite{RHO} exhibit insufficient sparsity, even after filtering the loss at the output layer.

\Cref{eq:attention} shows the forward computation of the attention block in the transformer model, where $softmax(\mathbf{Q}\mathbf{K}^T/\sqrt{d})$ is stored as activation\footnote{Eager implementation of attention block in PyTorch stores the softmax as intermediate results. The FlashAttention recomputes the the softmax matrix during backward, which is mathematically equalivent to storing the matrix.}.
\begin{equation} \label{eq:attention}
	\mathbf{X} = softmax(\frac{\mathbf{Q}\mathbf{K}^T}{\sqrt{d}}) \times \mathbf{V}
\end{equation}
\begin{figure}[t]
	\centering
	\includegraphics[scale=0.45]{figures/dv.pdf}
	\vspace{+0.5mm}
	\caption{Leaving the activation (\ie, $softmax$) of filtered tokens unchanged makes the $\mathbf{V}$'s gradients computed by the attention block not sparse anymore after the backpropagation. The dense gradients $\mathbf{G}_V$ will be passed to the front layers, undermining sparsity in all the rest computations .}
	\label{fig:dv}
    \vspace{+1mm}
\end{figure}
\Cref{fig:dv} illustrates the process of computing gradients for $\mathbf{V}$ (\ie, $\mathbf{G_V}$) using sparse gradients while maintaining unchanged activations (\ie, activations of all tokens are retained). After filtering the tokens based on loss, the gradients of the corresponding tokens become zero, as depicted in \Cref{fig:dv}. However, because the activations of the filtered tokens remain unchanged, the gradients of $\mathbf{V}$ are no longer sparse. Consequently, the backward computation following the first attention block lacks sparsity, limiting efficiency improvements solely within the output layer.

Following the setting in existing work~\cite{RHO}, we can estimate the upper bound of efficiency improvement with existing token filtering schemes. Taking TinyLlama, a model with 22 layers and 1.1B parameters, as an example. Filtering 40\% tokens will only linearly improve the efficiency on backward propagation of the last layer, while no front layers can be improved. Thus, the overall backward efficiency can only be improved by 1.8\%. Given that backpropagation consumes 66\% of the whole training~\cite{MegatronLM}, the end-to-end efficiency improvement is only 1.2\%.

To unlock the full efficiency of token filtering, we propose to further filter the activations to retain the sparsity in the whole backpropagation, as we illustrate in \S\ref{sec:system:filter_activation}.

\parab{Inefficient sparse GEMM.} Existing sparse GEMM implementations are not well-suited for token filtering training. Although sparse GEMM is a hot research topic and PyTorch has provided a sparse tensor implementation (\ie, \texttt{torch.sparse}), the efficiency of existing sparse GEMM is only improved when the data has very high sparsity (\eg, 95\%). Furthermore, \texttt{torch.sparse} does not fully support model training. For instance, the commonly used Compressed Sparse Row (CSR) format only accommodates 2D tensors, whereas the data in transformer models is typically represented as 3D or 4D tensors.
% Moreover, \texttt{torch.sparse} fails to provide full support for modrts 2D tensors while the data in transformer models is typically 3D or 4D tensors.el training. For example, the frequently used CSR-sparse format only suppo

\begin{figure}[t]
	\centering
	\includegraphics[scale=0.52]{figures/sparse_gemm_eff.pdf}
	\caption{PyTorch sparse GEMM outperforms regular GEMM only when filtering more than 95\% tokens and cannot improve efficiency of token filtering training which typically drops 30\% $\sim$ 40\% tokens \cite{RHO}.}
	\label{fig:sparse_gemm_eff}
    % \vspace{+2mm}
\end{figure}

To demonstrate the problem, we perform experiments on our testbed (details in \S\ref{sec:eval:setup}).
We compare the efficiency of sparse GEMM in PyTorch and regular GEMM in the scenario of token filtering, \ie, the matrix is sparse by row or columns. \Cref{fig:sparse_gemm_eff} shows the comparison results under different ratios of token filtering and batch sizes. The sparse GEMM is more efficient only when over 95\% of all tokens are filtered, which is unrealistic for token filtering. Under the typical filtering rate of 40\%, sparse GEMM is even 10$\times$ slower than regular GEMM. 



\vspace{-0.3cm}
\section{Device Model for Wafer-Scale Accelerators}
\vspace{-0.2cm}
% that differentiate wafer-scale accelerators from traditional shared-memory and NUMA architectures.

\subsection{The PLMR model}
\vspace{-0.1cm}
We develop the PLMR model to capture the unique hardware properties of wafer-scale accelerators and to motivate system requirements needed for utilizing this emerging hardware.

\begin{enumerate}[label=(\arabic*), leftmargin=0.5cm, noitemsep,topsep=0pt]
    \item \textbf{Massive Parallelism (P)}:
A wafer-scale accelerator can easily be equipped with millions of parallel cores, compared to thousands in GPUs. Each core features a local hardware pipeline that overlaps data ingress, egress, computation, and memory access at the cycle level.
This requires the computation to be partitioned at a massive scale and a fine-grained schedule to overlap computation, memory access, and NoC communication.

\item \textbf{Highly non-uniform memory access Latency (L)}:
Accessing memory on other cores in a mesh exhibits highly non-uniform latency. In a mesh with \(N_w \times N_h\) cores, the maximum NoC hops to a remote core is \(\max(N_w, N_h)\). For a million-core mesh, this can reach 1000 hops, causing a 1000$\times$ latency difference between local and remote memory access.
Therefore, it is crucial for the computation to minimize long-range communication whenever possible.

\item \textbf{Constrained local Memory (M)}: Each core has a small local memory (tens of KBs to several MBs), as performance and energy efficiency decline with larger capacities~\cite{sram-wiki}.
As a result, computation data must be explicitly partitioned into fine-grained chunks to fully fit within the constraints of each core's local memory.

\item \textbf{Constrained Routing resources (R)}: 
The message size in the NoC of a wafer-scale accelerator is extremely limited (e.g., a few bytes). This constraint requires message headers (e.g., address encoding) to be restricted to just a few bits, maximizing the capacity for actual data transfer. Consequently, only limited routing paths can be used, and the software system must carefully plan these paths.
\end{enumerate}

We expect these properties to remain relevant, as they are rooted in the fundamental characteristics of hardware and its manufacturing process. The PLMR model applies to both current (Cerebras WSE) and future (Tesla Dojo) wafer-scale devices. Even some non-wafer-scale devices with mesh-based NoC architectures, such as Tenstorrent Blackhole~\cite{tenstorrent}, can be represented by PLMR with adjusted parameters for parallelism (P), the size of the mesh (L), or relaxed constraints on local memory (M) and routing resources (R).


    \vspace{-3mm}
\subsection{Limitations of state-of-the-art approaches}
    \vspace{-1mm}
    
Leveraging the PLMR model, we analyze why existing AI systems fail to fully utilize wafer-scale accelerators.
To run an LLM model on a wafer-scale accelerator, we generally have two choices: (i) abstract the distributed local memory in each core as a shared memory and directly access data placed in a remote core through NoC; and (ii) explicitly partition computation into distributed cores and use message passing to exchange necessary data. 
We analyze two types of representative systems: LLM runtime or DNN compilers for shared memory architecture such as GPUs, e.g., Ladder~\cite{ladder}; and the SOTA compiler for distributed on-chip memory architectures, e.g., T10~\cite{t10} for GraphCore IPU.

\mypar{Shared-memory system} 
A shared-memory-based DNN compiler such as Ladder usually assumes a uniform memory access pattern within the underlying memory hierarchy, which cannot tolerate the 1000$\times$ latency variance in wafer-scale accelerators when accessing data from remote memory (failing in L).
Moreover, these compilers~\cite{tvm,rammer,ansor,flextensor,roller,welder,ladder} often focus primarily on partitioning computation, with less emphasis on optimizing data partitioning. This approach can easily lead to significant data duplication and violate the memory constraint requirements (failing in M).
Finally, these compilers are unaware of the communication distance of each core, poorly addressing the constraint of routing resources.

\mypar{Distributed-memory system} The T10 system~\cite{t10} is designed for AI accelerators with an on-chip crossbar which ensures a constant hop of memory access to other cores on the same chip. T10 handles small local memory and balances communication loads, addressing memory constraints (M) and routing resource limits (R). However, on a PLMR device, it fails to account for varying hop distances (failing in L) and scales to thousands, not millions, of cores (failing in P).


    \vspace{-3mm}
\section{Wafer-Scale LLM Parallelism}\label{sec:meshmp}
    \vspace{-1mm}
    
We present wafer-scale LLM parallelism, featuring new designs across prefill, decode and KV cache management.


    \vspace{-3mm}
\subsection{Prefill parallelism}
    \vspace{-1mm}

The parallelism for LLM prefill must ensure compliance with the PLMR model. Key challenges include: (i)~Handling multiple large matrices during prefill, requiring effective dimension partitioning to achieve million-core parallelism (P); (ii)~Optimizing GEMM operations, which involve further partitioning and overlapping computation and communication, to minimize long-range communication overhead (L), respect local memory constraints (M), and account for limited routing resources (R); and (iii)~Handling matrix transposes, which are costly on a NoC (L) but often required for sequential GEMM operations.

\begin{figure}[t!]
    \centering
    \includegraphics[width=0.46\textwidth]{imgs/prefill-partition.pdf}
    \vspace{-2mm}
    \caption{Prefill parallelism plan. $E_xF_y$ represents a matrix of shape $EF$, where the $E$ dimension is partitioned along the $x$-axis of cores, and $F$ along the $y$-axis of cores on a mesh.}
    \vspace{-5mm}
    \label{fig:prefill-partition}
\end{figure}

\mypar{Designing fine-grained partitioning for million-core parallelism} To achieve high chip utilization, we propose partitioning two dimensions of the input activation and weight matrices along both the $X$- and $Y$-axes of cores. This approach enables finer-grained, million-scale parallelism compared to existing methods~\cite{flashattn2, flashdecoding++, megatron, google2023efficiently}, which typically partition only the embedding dimension, resulting in insufficient parallelism on PLMR devices. 

We illustrate this partitioning using self-attention and feedforward, as shown in Figure~\ref{fig:prefill-partition}. For this discussion, we define the following annotations: the input activation $A$ and weight $W$ are multi-dimensional tensors during the prefill process. $B$ represents the batch size, $L$ the sequence dimension, $E$ the embedding dimension, $H$ the head dimension, and $F$ the hidden dimension in the feedforward block.
As shown by \myc{1}, the partitioning layout of $A$ is represented as $BL_yE_x$, where the $L$ dimension is partitioned along the $Y$-axis of cores, and the $E$ dimension along the $X$-axis of cores. Similarly, all weight matrices ($W_Q$, $W_K$, $W_V$, $W_{in}$, and $W_{out}$) are partitioned across both dimensions. 


\mypar{Designing PLMR-compliant distributed GEMM} 
We propose replacing conventional GEMM operators, designed for shared memory architectures, with a newly designed PLMR-compliant distributed GEMM during the prefill phase (as shown in \myc{2} of Figure~\ref{fig:prefill-partition}). Unlike TPU and GPU systems that primarily rely on allgather operations for GEMM, PLMR-compliant distributed GEMM algorithms achieve high NoC bandwidth utilization while respecting local memory and routing constraints, ensuring compliance with the L, M, and R properties. This PLMR-compliant distributed GEMM is fully described in Section~\ref{sec:gemm}.

\mypar{Using transposed distributed GEMM to avoid matrix transpose}
We propose a transpose-free parallelism plan for prefill to avoid matrix transpose, a common operation in LLM systems designed for shared memory architectures. The L property in PLMR highlights that matrix transposition is particularly costly on a wafer-scale device. It requires a core on one corner of the mesh to send data to the opposite diagonal corner, creating a long-range communication path.

Our transpose-free parallelism plan leverages transposed distributed GEMM (denoted as dist-GEMM-T)~\cite{summa, trans-dist} to compute $Q@K^T$ during LLM prefill, as shown by \myc{3} in Figure~\ref{fig:prefill-partition}. Specifically, the intermediate $Q$ and $K$ tensors, generated by multiplying $X$ with $W_Q$ and $W_K$, require transposing $K$ before proceeding with dist-GEMM operations due to the on-chip partition shape.


    \vspace{-3mm}
\subsection{Decode parallelism}
    \vspace{-1mm}

\begin{figure}[t!]
    \centering
    \includegraphics[width=0.46\textwidth]{imgs/decode-partition.pdf}
    \vspace{-2mm}
    \caption{Decode parallelism plan. $E^yF_x$ indicates the $E$ dimension is replicated along the $y$-axis, and $F$ is partitioned along the $x$-axis.}
    \vspace{-5mm}
    \label{fig:decode-partition}
\end{figure}

The parallelism strategy for LLM decode must address its memory-bandwidth-intensive nature, presenting several challenges: (i)~Decode uses smaller matrices than prefill due to limited input sequences and batch sizes, requiring careful parallelization when dimensions are insufficient for partitioning; (ii)~The phase heavily relies on GEMV operations, which are less compute-intensive than GEMM, resulting in short computation phases with limited overlap with communication, making GEMV vulnerable to long-range communication overhead on a NoC (L) and requiring adherence to local memory and routing constraints (M and R); and (iii)~Sequential GEMV operations introduce costly matrix transpose on a NoC, risking violation of the L property.

\mypar{Designing fine-grained replication to enable parallelism at minimal communication cost} When tensor dimensions are insufficient to achieve the high parallelism required for decode, we propose fine-grained replication of tensors in LLMs, specifically replicating the sequence dimension, where the sequence length equals the prompt length during prefill phase and equals 1 during the decode phase. This approach offers two key advantages: (i)~It improves parallelism and ensures balanced loads across all cores, and (ii)~It avoids additional communication operations such as allreduce. As shown by \circled{1} in Figure\ref{fig:decode-partition}, the $E$ dimension is partitioned along the $y$-axis, and the $L$ dimension is replicated along the $x$-axis, represented as $BE_yL^x$. Weight matrices $W$ are partitioned across both dimensions, consistent with the prefill phase.

Our fine-grained replication differs from recent work on long-context/sequence inference systems~\cite{loongserve, distserve}, which selectively replicate certain dimensions during the prefill phase rather than the decode phase.

\mypar{Designing PLMR-compliant distributed GEMV} We found that existing GEMV implementations fail to fully comply with PLMR requirements due to long-range communication and excessive routing resource consumption at each core. To address this, we propose a PLMR-compliant distributed GEMV, utilizing this new implementation throughout the decode phase (as detailed in \myc{2} of Figure~\ref{fig:decode-partition}). A comprehensive description of this GEMV design is provided in Section~\ref{sec:gemv}.

\mypar{Pre-optimizing model weight placement to avoid matrix transpose}
To avoid matrix transpose during decode, we propose pre-optimizing the model weight layout for decode, particularly for the distributed GEMV operation, to eliminate matrix transpose. While this introduces re-placement overhead between prefill and decode phases, the overhead is far smaller than that of sequential matrix transpose during token generation.

Figure~\ref{fig:decode-partition} illustrates this proposal, detailed in \myc{3}. Specifically, we optimize the placement of weights such as $W_{O}$ and $W_{out}$ for distributed GEMV in decode, differing from their layout in the prefill phase. This approach also removes the need for transpose operations in calculating $Q@K^T$ during decode self-attention.

    \vspace{-3mm}
\subsection{Shift-based KV cache management}\label{sec:parallel:kvcache}
    \vspace{-1mm}
    
\begin{figure}[t!]
    \centering
    \includegraphics[width=1\linewidth]{imgs/KVManage.pdf}
    \vspace{-9mm}
    \caption{KV cache concatenation vs. KV cache shift}
    \vspace{-5mm}
    \label{fig:kv_cache_update}
\end{figure}

KV cache management on PLMR devices is challenging as it requires storing large data across distributed cores while adhering to local memory constraints (M) and distributing KV cache computations to achieve high parallelism (P). To address these, we have the following insights:

\mypar{Existing concatenate-based management causes skewed core utilization} Current KV cache management methods primarily concatenate newly generated KV vectors to the existing cache. While efficient in shared memory architectures, this concatenate operation leads to highly skewed core utilization on PLMR devices, as shown in \circled{1} of Figure~\ref{fig:kv_cache_update}, where only core in a row is responsible for storing and computing over the newly generated KV vector. After several token generation steps, this only core quickly becomes the bottleneck, as depicted in \circled{2} of Figure~\ref{fig:kv_cache_update}, causing skewed memory usage and violating the M in PLMR. Moreover, the imbalanced KV cache distribution across cores results in inefficient parallelism, violating the P property.


\mypar{Proposing shift-based management for balanced core utilization} We propose a shift-based KV cache management strategy that evenly distributes cache data across all cores. Instead of concatenating new KV cache vectors at the end, this method performs a balancing shift operation, where each row transfers the oldest KV cache data to the row above, as shown in \circled{3} of Figure~\ref{fig:kv_cache_update}. When new KV data arrives, each core checks its local capacity against its neighbors. If equal, upward shifts are triggered, with each row receiving data from below and passing some to the row above. As illustrated in \circled{4}, this ensures even KV cache distribution across all cores.

The upward shifts utilize all NoC links in parallel, maintaining high performance and satisfying the P property. The physical placement of KV cache aligns with logical continuity, adhering to the L property. This method also fully resolves the M violation issue observed in the last row of cores with the concatenate-based approach.

    \vspace{-3mm}
\subsection{Implementation details}
    \vspace{-1mm}

We outline several implementation details below:

\mypar{Prefill and decode transition} Prefill and decode require distinct strategies. To handle the transition efficiently, we reshuffle KV cache and weights through the fast NoC which often provides 100s Pbits/s aggregated bandwidth, completing instantly without relying on slower off-chip memory.

\mypar{Parallelism configuration} We empirically determine the scalable parallelism for LLM operators. Automatic parallelism configuration is left for future work.

\mypar{Variations of self-attention} \sys supports variations of Self-Attention, including Grouped Query Attention~\cite{gqa}, Multihead Attention~\cite{mha}, and Multi-query Attention~\cite{mqa}. These differ by performing dist-GEMM, dist-GEMV and dist-GEMM-T locally after grouping by head dimensions.



    \vspace{-3mm}
\section{Wafer-Scale GEMM} \label{sec:gemm}
    \vspace{-1mm}

In this section, we introduce MeshGEMM, a scalable distributed GEMM for massive-scale, mesh architectures. 

    \vspace{-3mm}
\subsection{PLMR compliance in distributed GEMM}
    \vspace{-1mm}

\begin{figure}
    \centering
    \includegraphics[width=1\linewidth]{imgs/gemm-analysis.pdf}
    \vspace{-3mm}
    \caption{PLMR compliance in distributed GEMM}
    \vspace{-3mm}
    \label{fig:distributed-gemm-analysis}
\end{figure}

To identify an scalable distributed GEMM for PLMR devices, we define the following metrics: 
(i)~\emph{Paths per core}: The number of routing paths per core, with fewer paths ensuring compliance with the R property.
(ii)~\emph{critical path}: The longest communication path in each step to transmit submatrix (as the red lines in Figure~\ref{fig:distributed-gemm-analysis}), with fewer hops adhering to the L property.
(iii)~\emph{Memory per core}: The memory required per core, with lower usage ensuring the M property.

We analyze current distributed GEMM methods and show how MeshGEMM meets these metrics:

\begin{enumerate}[label=(\arabic*), leftmargin=0.5cm, noitemsep,topsep=0pt]
\item \textbf{GEMM via Allgather} is commonly used in GPU and TPU pods for distributed GEMM~\cite{google2023efficiently, tensorrt-llm, megatron}. Its longest communication path in each step is one core gathering data from the farthest cores, shown as the red line in Figure~\ref{fig:distributed-gemm-analysis}~\circled{1}, and $N$ steps to complete the allgather. Each core creates $N$ communication paths to neighbors in its row and column (violating R). The gather in each step spans the critical path with $O(N)$ hops (violating L), and each core uses $O(1/N)$ memory due to inflated working buffers, far exceeding the $O(1/N^2)$ for local submatrices (violating M).

\item \textbf{SUMMA} is Cerebras' default choice for distributed GEMM on its wafer-scale engine~\cite{cerebrasgemm}. Its longest communication path in each step is where one core broadcasts data to the farthest core along the column or row, shown by the red line in \circled{2} of Figure~\ref{fig:distributed-gemm-analysis}. Every core creates $N$ communication paths (violating R) and spans the critical path with $O(N)$ hops (violating L) in the longest path. While SUMMA improves memory usage compared to AllGather, requiring only a working set equal to the size of locally partitioned submatrices, it still doubles memory usage.

\item \textbf{Cannon} is mesh-optimized choice for distributed GEMM~\cite{cannon}, popular in supercomputers. 
Its longest communication path in each step is the head cores send data to the tail cores.
% It needs $N$ steps submatrices shifting to complete GEMM on cores. 
As shown in \circled{3} of Figure~\ref{fig:distributed-gemm-analysis}, each core communicates with two neighbours in a 2D torus, and only needs $O(1)$ communication paths and optimal memory usage of $O(1/N^2)$. However, it incurs the critical path with $O(N)$ hops as the red line, violating L.

\item \textbf{MeshGEMM (Ours)} is a distributed GEMM which complies with the PLMR model. Its longest communication path in each step is shown as the red line in \circled{4} of Figure~\ref{fig:distributed-gemm-analysis}. Each core communicates with two neighbors, two hops away (proven in later sections to be scalable for mesh architectures). This design achieves $O(1)$ communication paths per core needed and optimal memory usage of $O(1/N^2)$, similar to Cannon. Crucially, it bounds the critical path to 2 hops with $O(1)$ complexity, making it uniquely capable of addressing the L property.

\end{enumerate}

\vspace{-0.2cm}
\subsection{Design intuitions and scalability analysis} 

\begin{figure}[t!]
    \centering
    \includegraphics[width=1\linewidth]{imgs/gemm_intuition.pdf}
    \caption{Design intuitions and scalability analysis.}
    \label{fig:gemm_intuition}
\end{figure}

Our design involves two steps: (i)~We ensure algorithm correctness using a cyclic shifting process for GEMM, and (ii)~We prove that two-hop communication on this cycle is the minimal distance required to satisfy the L property.

\mypar{Cyclic shifting} Cyclic shifting enables \gemm to satisfy the M and R properties by limiting communication to two neighbors and minimizing memory usage. It ensures correct GEMM results, following reasoning similar to Cannon~\cite{cannon}. As illustrated in \circled{3} of Figure~\ref{fig:distributed-gemm-analysis}, a logical circle of 5 cores is flattened into the physical communication mapping, with a critical path from head core to tail core. 

\begin{algorithm}[t]
    \caption{$\textbf{INTERLEAVE}$}
    \label{alg:interleave}
    % \SetKwFunction{min}{Min}
    % \SetKwFunction{max}{Max}
    \SetKwFunction{assert}{assert}
    \SetKw{ret}{Return}
    \SetKw{int}{Int}
    \LinesNumbered
    % \KwIn{Current PE's Coordinate: $pc$}
    % \KwIn{Current Axis's PE count: $P$}
    \KwIn{index, N}
    % \KwOut{Send To Destination PE ID: sendId}
    % \KwOut{Receive From Source PE ID: recvId}
    \KwOut{send\_index, recv\_index}
    
    % \BlankLine
    % \tcp{\textbf{Assign Stage}}
    \eIf{\textnormal{index} \textnormal{\textbf{mod}} \textnormal{2 == 0}}{
        recv\_index = \texttt{Max} (index - 2, 0)\;% RId = \min {$n+2$, N-1}\;
        send\_index = \texttt{Min} (index + 2, N - 1)\;
    }{
        recv\_index = \texttt{Min} (index + 2, N - 1)\;%, RId = \max {$n-2$, 0}\;
        send\_index = \texttt{Max} (index - 2, 0)\;
    }
    % \BlankLine
    \textbf{if} \textnormal{index == 0} \textbf{then} recv\_index = 1\;
    % \BlankLine
    \If{\textnormal{index == N - 1}}{
        % \eIf{$N\;mode\;2==0$}{
        %     RId = $P-2$\;
        % }{
        %     SId = $P-2$\;
        % }
        \textbf{if} N \textbf{mod} 2 == 0 \textbf{then}
            recv\_index = N - 2\;
        \textbf{else}
            send\_index = N - 2\;
    }
    \ret send\_index, recv\_index\;
\end{algorithm}

\mypar{Interleaving} For the flatten communication plan, we would like to minimize the length of the critical path further, thus satisfying the L property. Our key intuition here is to introduce an INTERLEAVE operation to find the mapping relationship from logical to physical, defined in Algorithm~\ref{alg:interleave}. As shown by \circled{1} of Figure~\ref{fig:gemm_intuition}, MeshGEMM first insert core 1 in between core 0 and 4 and core 2 in between core 4 and 3 to form a logical mapping, and then call the INTERLEAVE operation to get the send to and receive from neighbours' index, resulting in a permutated, equivalent communication plan as shown by \circled{2} in Figure~\ref{fig:gemm_intuition}. For example, there are 5 cores total (N=5), so physical core 2 (index=2) sends data to physical core 4 (send\_index=4) and receives from physical core 0 (recv\_index=0).


\mypar{Scalability analysis}  We can prove that the two-hop distance created by INTERLEAVE cannot be further reduced. The proof relies on the fundamental properties of sequential arrangements: if we attempt to create a circular sequence where each number differs from its neighbors by exactly one hop, we encounter a mathematical impossibility. This can be understood by visualizing the numbers as points on a line - while adjacent numbers can be connected, the endpoints of the sequence cannot simultaneously maintain single-hop differences with their neighbors while forming a circle.

Note that our discussion, based on a 1D array, naturally extends to a 2D mesh, as the 1D array corresponds to the mesh's X-axis and Y-axis due to their symmetry.

\vspace{-0.3cm}
\subsection{The \gemm algorithm} \label{subsec:meshgemm} 

We outline the key steps of \gemm below:

\begin{enumerate}[label=(\arabic*), leftmargin=0.5cm, noitemsep,topsep=0pt]
\item \textbf{Initialization:} Consider $C = A \times B$. \gemm will partition $A$ and $B$ into tiles $A_{sub}$ and $B_{sub}$ along two dimensions, forming $N \times N$ tiles, which are distributed across the cores. Each core receives one tile of $A_{sub}$ and one of $B_{sub}$. \gemm will then use INTERLEAVE to initialize the neighbor's positions for each core. 

\item \textbf{Alignment:}  
Each core will then align with neighbors to align the input submatrices in a way that ensures every core in the distributed system begins with the appropriate operands for the matrix multiplication process. 

\item \textbf{Compute-shift loop:}  
Each core operates with a compute-shift loop involving $N$ steps of communication and computation. In each step, every core computes the partial sum of its corresponding $C_{sub} = A_{sub} \times B_{sub} + C_{sub}$. Meanwhile, shift $A_{sub}$ along the X-axis and $B_{sub}$ along the Y-axis to get new $A'_{sub}$ 
 and $B'_{sub}$ for the next step computation as \circled{3} we shown in Figure~\ref{fig:gemm_intuition}. After $N$ steps, the accumulated $C_{sub}$ is returned.
\end{enumerate}

\vspace{-0.3cm}
\subsection{Implementation details}

\mypar{Handling non-square mesh} For a non-square mesh $N_h \times N_w$ ($N_h \neq N_w$), the $A$ and $B$ matrices can be logically partitioned into $N_{lcm} \times N_{lcm}$ cores, where $N_{lcm}$ is the least common multiple of $N_h$ and $N_w$.

\mypar{Transposed distributed GEMM} The above algorithm key steps can be applied to the computation of $C = A \times B^T$, the dist-GEMM-T in Figure~\ref{fig:prefill-partition} to avoid transposing $B$ on mesh. It does not require alignment before computation and only necessitates $N$ steps two-hop compute-shift for the right matrix $B$ along the Y-axis. After each shift step, each core computes $C_{\text{sub}} = A_{\text{sub}} \times B_{\text{sub}}$, followed by a ReduceAdd of $C_{\text{sub}}$ along the X-axis. After $N$ steps, the final matrix $C$ is obtained.

\vspace{-0.3cm}
\section{Wafer-Scale GEMV}\label{sec:gemv}
\vspace{-0.2cm}

In this section, we describe \gemv, a scalable GEMV algorithm for PLMR devices.


\subsection{PLMR compliance in distributed GEMV}

\begin{figure}[t!]
    \centering
    \includegraphics[width=1\linewidth]{imgs/reduce-algo.pdf}
    \caption{PLMR compliance in distributed GEMV}
    \label{fig:gemv-plmr-compliance}
\end{figure}

The completion time of a distributed GEMV is primarily determined by an allreduce operation that aggregates partial results from all selected cores and broadcasts the aggregated results back to all cores. So, we define the number of add-operations (hops) in the longest aggregation path as the \emph{critical path} in GEMV. Below, we analyze common distributed GEMV implementations in LLM systems and demonstrate that \gemv is the only approach fully compliant with the PLMR model.

\begin{enumerate}[label=(\arabic*), leftmargin=0.5cm, noitemsep,topsep=0pt]

 \item \textbf{GEMV with pipeline allreduce} is commonly used in TPU pod systems~\cite{google2023efficiently} and as the default in Cerebras demo~\cite{cerebrasgemv}. As shown by \circled{1} in Figure~\ref{fig:gemv-plmr-compliance}, it bounds routing resource usage to $O(1)$ per core (meeting R in PLMR). However, its longest aggregation path is from tail to head cores, as shown in the red line, and spans the critical path at $O(N)$, violating the L property.

\item \textbf{GEMV with ring allreduce} is commonly used in GPU pod systems, where it is the default configuration. As shown by \circled{2} in Figure~\ref{fig:gemv-plmr-compliance}, it bounds routing resource usage to $O(1)$ (meeting R in PLMR). However, it spans $O(N)$ hops in the critical path, violating the L property.

\item \textbf{GEMV with two-way K-tree allreduce (Ours)}. As shown by \circled{4} in Figure~\ref{fig:gemv-plmr-compliance}, we build a balanced K-tree to reduce from two-way; its longest aggregation path is from the head or tail core to the tree root core. The critical path is $O(\sqrt[K]{N}K)$ which can address the L. The max number of communication paths at each root core is $O(K)$, and can meet the R limitation by adjusting the K.

\end{enumerate}


\subsection{The \gemv algorithm} 

We will outline the key steps of \gemm below:

\begin{enumerate}[label=(\arabic*), leftmargin=0.5cm, noitemsep,topsep=0pt]
\item \textbf{Initialization:} Consider $C = A \times B$ and $A$ is a vector. \gemv will partition $B$ into tiles $B_{sub}$ along two dimensions, forming $N \times N$ tiles and distributed across the cores. For $A$, \gemv will partition it along the vector length, forming $N$ tiles distributed on one axis and replica $A$ on another axis. Each core receives one tile of $A_{sub}$ and one of $B_{sub}$. Then we determine which cores form a group to obtain aggregated results in each phase based on the K-tree.

\item \textbf{Parallel computation:} In this stage, each core performs a local GEMV $A_{\text{sub}} \times B_{\text{sub}}$ to obtain $C_{sub}$ partial sum.

\item \textbf{Aggregation:} The aggregation step primarily involves using the two-way K-tree allreduce we design. The key steps as follows: (i)~In the 1st-phase, each group performs group reduction and obtains the partial sum of $C_{sub}$ at the root core of each group. (ii)~In the $k$th-phase, the results from the $(k-1)$~th-phase are reduced to the root cores of each group in the $k$th-phase. After K times repeating, $C$ can be obtained by concatenating the $C_{sub}$ from all K-tree root cores. (iii)~ Optionally, a broadcast operation from the root core of the $K$-tree may follow, depending on whether continuous GEMV is required.
\end{enumerate}

\mypar{Scalability Analysis}  
As shown in \circled{1} of Figure~\ref{fig:gemv-plmr-compliance}, this method scales efficiently with parallelism and meets the L property by selecting an appropriate $K$. It requires $K+1$ paths at the tree root core but allows flexible adjustment of $K$ to address R based on hardware limitations.  

However, a larger $K$ is not always better, as it depends on $N$ and R constraints. Additionally, larger $K$ increases routing complexity and overhead. Considering these factors, we have chosen $K=2$ for our current implementation evaluated in the following sections.


\section{Evaluation}
We provide three sets of insights into this section, organised as \textit{findings (F*)}. We quantitatively study the effect of the adversarial and counterfactual perturbations on the performance of informal reasoners and autoformalisation methods. Then, we dive deeper into method variants. Finally, 
we analyse the nature of formalisation errors made by the models.

\subsection{Robustness Analysis}
\paragraph{\textbf{\emph{F1: Noise perturbations have a stronger effect on formalisation methods than informal \ac{LLM} reasoners.}}}
Table~\ref{tab:distraction_k4_formalisation} shows that, on average, the accuracy of both direct and \ac{CoT} informal reasoning remains between $73\%$ and $74\%$ in the face of added noise. While the autoformalisation method performs similarly to informal reasoners on the original dataset, its performance decreases between $4\%$ and $11\%$. The accuracy drops especially with logical (L) and tautological (T) distractions, whose logical language formats trick the \ac{LLM} into formalizing the noisy clauses. On the other hand, the linguistically complex and more natural sentences of encyclopedic distractions show a minor effect, suggesting that \acp{LLM} successfully avoids formalizing the more complicated sentences.

\paragraph{\textbf{\emph{F2: All \ac{LLM}-based reasoning methods suffer a drop for counterfactual perturbations.}}} % influence .}}}
Table~\ref{tab:distraction_k4_formalisation} shows that counterfactual statements cause a significant decrease in performance for both the informal reasoners and autoformalisation methods of between $12\%$ and $13\%$ on average. 
Moreover, this observation also holds for all tested models, i.e., none are robust towards counterfactual perturbations across every evaluated dimension. Even the strongest model, GPT 4o-mini, yields a performance of 63-68\%, which is relatively close to the random performance of 50\%. The high impact of counterfactual statements (the single ``not'' inserted) could be due to the inability of \acp{LLM} to overwrite prior knowledge with explicitly stated information or memorization of the answers. We study the error sources further in §\ref{subsec:errors}.  

\noindent \paragraph{\textbf{\emph{F3: Introducing multiple noise sentences has an effect only for logical distractions.}}}
We show the impact of introducing between one and four sentences for the two top-performing autoformalisation models in Figure~\ref{fig:length_distraction}. The figure shows similar trends with and without counterfactual perturbations.
As additional logical distractions are introduced, the model performance consistently decreases. Tautological (T) distractions lead to a decline in accuracy with a single disruptive sentence, yet adding more noise does not worsen the outcome. 
The tautological corpus introduces truth constants for all sentences as a persistent unseen logical construct. Given that this leads only to a decrease for a single occurrence, we can assume that a model can consistently handle the same unseen logical construct. In contrast, the logical corpus increases the chance of adding text, requiring new, previously unseen reasoning constructs for each added sentence. The impact of encyclopedic noise remains negligible, generalising F1 to $k$ sentences. Similarly, counterfactual perturbations remain much more effective for all settings, generalising F2.

\begin{table}[!t]
\small
\setlength{\modelspacing}{2pt}
\setlength{\tabcolsep}{1.7pt} % Default value: 6pt
\setlength{\belowrulesep}{4pt}
\begin{threeparttable}
    \centering
    \begin{tabular}{cc l r rrr @{\quad} rrrr}
\toprule
\multirow{2}{*}{} & \multirow{2}{*}{} & Reasoning & \multirow{2}{*}{O} & \multicolumn{3}{c}{Distraction} & \multicolumn{4}{c}{Counterfactual} \\
 & & Format & & E& L & T & $\text{O}_C$ & $\text{E}_C$& $\text{L}_C$ & $\text{T}_C$\\
\midrule
\multirow{6}{*}{\rotatebox{90}{Gemma-2}} & \multirow{3}{*}{\rotatebox{90}{9b}}
   & Informal (direct) & \textbf{0.78} & \textbf{0.80} & \textbf{0.79} & \textbf{0.77} & 0.58 & 0.52 & 0.50 & 0.59 \\
 & & Informal (CoT) & 0.72 & 0.78 & 0.73 & 0.76 & 0.61 & \textbf{0.57} & \textbf{0.60} & \textbf{0.66} \\
 & & Formal (FOL) & 0.62 & 0.58 & 0.52 & 0.53 & \textbf{0.63} & 0.52 & 0.46 & 0.46 \\[\modelspacing]
\cmidrule{2-11}
 & \multirow{3}{*}{\rotatebox{90}{27b}} 
   & Informal (direct) & 0.71 & 0.69 & \textbf{0.66} & \textbf{0.68} & 0.59 & 0.51 & 0.54 & 0.59 \\
 & & Informal (CoT) & 0.66 & 0.65 & 0.64 & 0.63 & 0.62 & 0.58 & \textbf{0.62} & \textbf{0.64} \\
 & & Formal (FOL) & \textbf{0.74} & \textbf{0.74} & 0.61 & 0.61 & \underline{\textbf{0.72}} & \underline{\textbf{0.67}} & 0.58 & 0.51 \\[\modelspacing]
\midrule
\multirow{6}{*}{\rotatebox{90}{Mistral}} & \multirow{3}{*}{\rotatebox{90}{7B}} 
   & Informal (direct) & 0.77 & \textbf{0.77} & 0.75 & \textbf{0.79} & \textbf{0.63} & \textbf{0.54} & \textbf{0.54} & \textbf{0.66} \\
 & & Informal (CoT) & \textbf{0.79} & 0.75 & \textbf{0.77} & 0.78 & 0.55 & 0.52 & \textbf{0.54} & 0.58 \\
 & & Formal (FOL) & 0.62 & 0.58 & 0.54 & 0.57 & 0.50 & \textbf{0.54} & 0.51 & 0.52 \\[\modelspacing]
\cmidrule{2-11}
 & \multirow{3}{*}{\rotatebox{90}{Small}} 
   & Informal (direct) & \textbf{0.77} & \textbf{0.76} & \textbf{0.76} & \textbf{0.75} & 0.61 & 0.51 & 0.56 & 0.59 \\
 & & Informal (CoT) & 0.72 & 0.72 & 0.72 & 0.71 & \textbf{0.62} & \textbf{0.59} & \textbf{0.62} & \textbf{0.68} \\
 & & Formal (FOL) & 0.68 & 0.59 & 0.53 & 0.64 & 0.54 & 0.55 & 0.49 & 0.51 \\[\modelspacing]
\midrule
\multirow{6}{*}{\rotatebox{90}{Llama-3.1}} & \multirow{3}{*}{\rotatebox{90}{8B}} 
   & Informal (direct) & 0.63 & 0.61 & 0.64 & 0.66 & 0.61 & \textbf{0.62} & 0.59 & 0.61 \\
 & & Informal (CoT) & 0.73 & \textbf{0.73} & \textbf{0.71} & \textbf{0.72} & \textbf{0.62} & 0.59 & \textbf{0.61} & \textbf{0.65} \\
 & & Formal (FOL) & \textbf{0.77} & 0.71 & 0.63 & 0.52 & 0.60 & 0.58 & 0.55 & 0.52 \\[\modelspacing]
\cmidrule{2-11}
 & \multirow{3}{*}{\rotatebox{90}{70B}} 
   & Informal (direct) & 0.77 & 0.74 & 0.74 & 0.73 & 0.62 & 0.53 & 0.56 & 0.64 \\
 & & Informal (CoT) & \textbf{0.78} & \textbf{0.75} & \textbf{0.76} & \textbf{0.76} & 0.64 & 0.61 & \textbf{0.66} & \underline{\textbf{0.73}} \\
 & & Formal (FOL) & 0.74 & 0.73 & 0.71 & 0.71 & \textbf{0.66} & \textbf{0.62} & 0.59 & 0.57 \\[\modelspacing]
 \midrule
\multirow{3}{*}{\rotatebox{90}{GPT}} & \multirow{3}{*}{\rotatebox{90}{4o-mini}} 
   & Informal (direct) & 0.78 & 0.77 & 0.79 & 0.79 & 0.64 & 0.61 & 0.61 & 0.63 \\
 & & Informal (CoT) & 0.80 & 0.80 & \underline{\textbf{0.81}} & \underline{\textbf{0.82}} & \textbf{0.68} & \textbf{0.63} & \underline{\textbf{0.68}} & \textbf{0.64} \\
 & & Formal (FOL) & \underline{\textbf{0.84}} & \underline{\textbf{0.82}} & 0.73 & 0.79 & 0.63 & 0.62 & 0.57 & 0.54 \\[\modelspacing]
 \midrule
\multicolumn{2}{c}{\multirow{3}{*}{\textbf{Avg}}} 
 & Informal (direct) & 0.74 & 0.73 & 0.73 & 0.73 & 0.61 & 0.55 & 0.56 & 0.62 \\
 & & Informal (CoT) & 0.74 & 0.74 & 0.73 & 0.74 & 0.62 & 0.58 & 0.62 & 0.65 \\
  & & Formal (FOL) & 0.72 & 0.68 &	0.61 & 0.62 & 0.61 & 0.59 & 0.54 & 0.52 \\
\bottomrule
\end{tabular}
\caption{Accuracies of informal and autoformalisation-based deductive reasoners. The best overall model per dataset is underlined; the best model version is marked in bold.}
\label{tab:distraction_k4_formalisation}
\end{threeparttable}
\end{table} 

\begin{figure}[!t]
    \centering
    \scriptsize
    \begin{tikzpicture}
        \begin{axis}[name=gpt,
            title={GPT-4o-mini},
            width=0.6\linewidth,
            height=0.6\linewidth,
            xlabel={\# Noise sentences},
            ylabel={Accuracy},
            xmin=-0.1, xmax=4.1,
            ymin=0.5, ymax=0.9,
            xtick={1,2,4},
            ytick={0.55, 0.6, 0.65, 0.75, 0.8, 0.85},
            title style={yshift=-0.6em},
            legend style={at={(1,-0.15)},
	           anchor=north,legend columns=-1},
            x label style={at={(axis description cs:1,-0.05)},anchor=north},
            y label style={at={(axis description cs:-0.15,0.5)},anchor=south},
            ymajorgrids=true,
            grid style=dashed,
        ]
            \addplot[color=blue, mark=square,]
                coordinates {
                (0,0.848076939582825)(1,0.823076903820038)(2,0.826923072338104)(4,0.821153819561005)
                };
            \addplot[color=red, mark=triangle,]
                coordinates {
                (0,0.848076939582825)(1,0.817307710647583)(2,0.801923096179962)(4,0.759615361690521)
                };
            \addplot[color=green, mark=diamond,] 
                coordinates {
                (0,0.848076939582825)(1,0.767307698726654)(2,0.769230782985687)(4,0.803846180438995)
                };
            \addplot[color=blue, mark=square*] 
                coordinates {
                (0,0.627777755260468)(1,0.622222244739533)(2,0.600000023841858)(4,0.633333325386047)
                };
            \addplot[color=red, mark=triangle*,] 
                coordinates {
                (0,0.627777755260468)(1,0.611111104488373)(2,0.611111104488373)(4,0.594444453716278)
                };
            \addplot[color=green, mark=diamond*,] 
                coordinates {
                (0,0.627777755260468)(1,0.572222232818604)(2,0.538888871669769)(4,0.555555582046509)
                };
                \legend{E,L,T,$\text{E}_C$, $\text{L}_C$ , $\text{T}_C$}
        \end{axis}

        \begin{axis}[name=llama, at={($(gpt.east)+(0.1cm,0)$)},anchor=west,
            title={Llama 3.1 70b},
            width=0.6\linewidth,
            height=0.6\linewidth,
            xmin=-0.1,, xmax=4.1,
            ymin=0.5, ymax=0.9,
            xtick={1,2,4},
            ytick={0.55, 0.6, 0.65, 0.75, 0.8, 0.85},
            title style={yshift=-0.6em},
            yticklabel=\empty,
            ymajorgrids=true,
            grid style=dashed,
        ]
            \addplot[color=blue, mark=square,]
                coordinates {
                (0,0.838461518287659)(1,0.817307710647583)(2,0.805769205093384)(4,0.817307710647583)
                };
            \addplot[color=red, mark=triangle,]
                coordinates {
                (0,0.838461518287659)(1,0.819230794906616)(2,0.803846180438995)(4,0.771153867244721)
                };
            \addplot[color=green, mark=diamond,]
                coordinates {
                (0,0.838461518287659)(1,0.803846180438995)(2,0.807692289352417)(4,0.805769205093384)
                };
            \addplot[color=blue, mark=square*]
                coordinates {
                (0,0.627777755260468)(1,0.622222244739533)(2,0.577777802944183)(4,0.594444453716278)
                };
            \addplot[color=red, mark=triangle*,]
                coordinates {
                (0,0.627777755260468)(1,0.583333313465118)(2,0.561111092567444)(4,0.577777802944183)
                };
            \addplot[color=green, mark=diamond*,]
                coordinates {
                (0,0.627777755260468)(1,0.627777755260468)(2,0.566666662693024)(4,0.577777802944183)
                };
        \end{axis}
    \end{tikzpicture}
    \caption{Influence of the number of noisy sentences for FOL.}
    \label{fig:length_distraction}
\end{figure}



\subsection{Impact of Method Design}
\paragraph{\textbf{\emph{F4: \ac{CoT} prompting is most impactful when both noise and counterfactual perturbations are applied.}}}
The accuracies for the individual \acp{LLM} in Table~\ref{tab:distraction_k4_formalisation} show that the impact of \ac{CoT} is negligible for noise-only datasets (first four columns). Meanwhile, the benefit from \ac{CoT} is most pronounced in the datasets that combine noise and counterfactual perturbations.
The better-performing informal prompting strategy for a model remains stable for all types of distractions. Still, the decline in performance due to counterfactuals leads to a less consistent preference for a specific prompting style.

\paragraph{\textbf{\emph{F5: The best-performing grammar differs per model and is unstable across data versions.}}}

The evaluation of different logical forms for formal \ac{LLM}-based reasoning in Table~\ref{tab:distraction_k4_logical_form} shows the preference of some models for specific syntactic formats.
Llama 3.1 70B has a considerable improvement of $12\%$ with TPTP syntax on the original set, while Llama 3.1 8B benefits from the R-FOL syntax. However, all grammars show a declining accuracy trend and increased syntax errors for noise perturbations, where the best grammar loses its advantage over the rest. 
When comparing the grammars on the counterfactual partitions, we observe that TPTP is consistently more robust than the standard first-order logic grammar. Here, GPT 4o-mini shows a reduction from $O$ to $O_C$ of $20\%$ for FOL and only $12\%$ for the TPTP grammar. Since this does not correlate with fewer syntax errors, the formalisation in TPTP prevents semantical errors for counterfactual premises. 
A positive reading of these results, especially the minor differences between FOL and R-FOL, is that autoformalisation \acp{LLM} can adapt to the grammar syntax prescribed in the prompt without further loss in performance.

\begin{table}[!t]
\small
\setlength{\modelspacing}{2pt}
\setlength{\tabcolsep}{1.7pt} % Default value: 6pt
\setlength{\belowrulesep}{4pt}
\begin{threeparttable}
    \centering
    \begin{tabular}{cc l r rrr @{\quad} rrrr}
\toprule
\multirow{2}{*}{} & \multirow{2}{*}{} & Grammar & \multirow{2}{*}{O} & \multicolumn{3}{c}{Distraction} & \multicolumn{4}{c}{Counterfactual} \\
 & & Syntax & & E& L & T & $\text{O}_C$ & $\text{E}_C$& $\text{L}_C$ & $\text{T}_C$\\
\midrule
\multirow{6}{*}{\rotatebox{90}{Llama-3.1}} & \multirow{3}{*}{\rotatebox{90}{8B}} 
   & FOL & 0.77 & \textbf{0.71} & 0.61 & \textbf{0.53} & 0.58 & \textbf{0.55} & 0.52 & \textbf{0.56} \\
 & & R-FOL & \textbf{0.78} & 0.69 & \textbf{0.62} & \textbf{0.53} & 0.58 & \textbf{0.55} & \textbf{0.54} & 0.52 \\
 & & TPTP & 0.73 & 0.67 & 0.55 & 0.51 & \textbf{0.68} & 0.54 & 0.46 & 0.51 \\[\modelspacing]
\cmidrule{2-11}
 & \multirow{3}{*}{\rotatebox{90}{70B}} 
   & FOL & 0.76 & 0.73 & 0.71 & \textbf{0.72} & 0.67 & 0.57 & 0.63 & 0.56 \\
 & & R-FOL & 0.76 & 0.73 & 0.67 & 0.71 & 0.64 & 0.57 & 0.53 & 0.64 \\
 & & TPTP & \underline{\textbf{0.88}} & \underline{\textbf{0.84}} & \underline{\textbf{0.81}} & \textbf{0.72} & \underline{\textbf{0.81}} & \underline{\textbf{0.68}} & \underline{\textbf{0.67}} & \underline{\textbf{0.68}} \\[\modelspacing]
\midrule
\multirow{3}{*}{\rotatebox{90}{GPT}} & \multirow{3}{*}{\rotatebox{90}{4o-mini}} 
   & FOL & \textbf{0.84} & \textbf{0.82} & \textbf{0.72} & \underline{\textbf{0.78}} & 0.64 & \textbf{0.63} & \textbf{0.61} & 0.51 \\
 & & R-FOL & \textbf{0.84} & 0.77 & 0.70 & \underline{\textbf{0.78}} & \textbf{0.72} & 0.56 & 0.54 & \textbf{0.63} \\
 & & TPTP & 0.83 & \textbf{0.82} & 0.71 & 0.71 & 0.69 & \textbf{0.63} & 0.57 & 0.57 \\
\bottomrule
\end{tabular}
\caption{Accuracies of different formalisation grammars for autoformalisation.}
\label{tab:distraction_k4_logical_form}
\end{threeparttable}
\end{table} 

\paragraph{\textbf{\emph{F6: Feedback does not help \acp{LLM} self-correct to mitigate robustness issues.}}}
\autoref{tab:distraction_k4_feedback} shows the results with different error recovery mechanisms. The results indicate that no feedback strategy emerges as a winner in the different datasets. 
All feedback variants reduce syntax errors for noise perturbations, but given the lack of a consistent increase in accuracy, the corrected formalisations are most likely to contain semantic errors still. 
The type of feedback message only has a minor influence on correcting syntax errors, whereas Llama 3.1 70b and GPT 4o-mini correct slightly more syntax errors with specific error messages. This finding aligns with \cite{huang2023large}, who also found that \acp{LLM} cannot consistently self-correct their reasoning after receiving relevant feedback.

\begin{table}[!ht]
\small
\setlength{\modelspacing}{2pt}
\setlength{\tabcolsep}{1.7pt} % Default value: 6pt
\setlength{\belowrulesep}{4pt}
\begin{threeparttable}
    \centering
    \begin{tabular}{cc l r rrr @{\quad} rrrr}
\toprule
\multirow{2}{*}{} & \multirow{2}{*}{} & \multirow{2}{*}{Feedback} & \multirow{2}{*}{O} & \multicolumn{3}{c}{Distraction} & \multicolumn{4}{c}{Counterfactual} \\
 & & & & E& L & T & $\text{O}_C$ & $\text{E}_C$& $\text{L}_C$ & $\text{T}_C$\\
\midrule
\multirow{8}{*}{\rotatebox{90}{Llama-3.1}} & \multirow{4}{*}{\rotatebox{90}{8B}} 
   & No recovery & 0.77 & \textbf{0.72} & 0.62 & 0.53 & 0.59 & 0.58 & 0.56 & \textbf{0.56} \\
 & & Error type & \textbf{0.79} & 0.71 & 0.63 & \textbf{0.56} & \textbf{0.66} & 0.54 & 0.52 & 0.51 \\
 & & Error message & 0.78 & 0.71 & \textbf{0.67} & 0.55 & 0.59 & 0.53 & \underline{\textbf{0.64}} & 0.49 \\
 & & Warning & 0.74 & 0.66 & 0.58 & 0.55 & 0.55 & \textbf{0.60} & 0.49 & 0.49 \\[\modelspacing]
\cmidrule{2-11}
 & \multirow{4}{*}{\rotatebox{90}{70B}} 
   & No recovery & \textbf{0.77} & \textbf{0.72} & \textbf{0.73} & 0.71 & \textbf{0.64} & 0.59 & \textbf{0.61} & 0.56 \\
 & & Error type & 0.72 & 0.70 & 0.72 & \textbf{0.73} & 0.62 & 0.56 & 0.60 & 0.58 \\
 & & Error message & 0.71 & 0.70 & \textbf{0.73} & 0.71 & \textbf{0.64} & 0.59 & 0.54 & \underline{\textbf{0.64}} \\
 & & Warning & 0.69 & \textbf{0.72} & 0.72 & 0.72 & 0.62 & \underline{\textbf{0.65}} & \textbf{0.61} & 0.63 \\[\modelspacing]
\midrule
\multirow{4}{*}{\rotatebox{90}{GPT}} & \multirow{4}{*}{\rotatebox{90}{4o-mini}} 
   & No recovery & \underline{\textbf{0.84}} & \underline{\textbf{0.82}} & 0.73 & 0.79 & 0.64 & \textbf{0.62} & 0.56 & \textbf{0.56} \\
 & & Error type & 0.83 & 0.79 & 0.74 & 0.76 & 0.67 & 0.57 & 0.56 & \textbf{0.56} \\
 & & Error message & \underline{\textbf{0.84}} & 0.78 & \underline{\textbf{0.77}} & \underline{\textbf{0.80}} & 0.62 & 0.59 & 0.56 & \textbf{0.56} \\
 & & Warning & \underline{\textbf{0.84}} & 0.75 & 0.73 & 0.76 & \underline{\textbf{0.70}} & 0.61 & \textbf{0.61} & 0.55 \\
 \bottomrule
\end{tabular}
\caption{Accuracies of error recovery strategies.}
\label{tab:distraction_k4_feedback}
\end{threeparttable}
\end{table} 

\subsection{Error Analysis}
\label{subsec:errors}
\paragraph{\textbf{\emph{F7: Autoformalisation increases syntax errors for noise perturbations.}}}
The low performance for noise perturbations correlates with more syntax errors for all models and distraction categories (cf. execution rates in Table~\ref{tab:appendix_k4_formalisation_exec}). The three worst-performing models (both Mistral models, Gemma-2 9b) generate, at best, for $37\%$  and, at worst, for only $4\%$ of the samples, a valid logical form.
Gemma-2 9b and Llama3.1 8b produce more syntax errors than the larger counterparts, suggesting that larger models are more robust towards noise perturbations. 
The accuracy of syntactically valid samples is higher than the informal reasoning methods for most distractions (Table~\ref{tab:appendix_k4_formalisation_vacc}), motivating informal reasoning as a backup strategy for formal reasoning. The error message feedback reveals two common syntax errors: 1) errors by models with an initial low execution rate exhibit issues with the template structure, including using incorrect keywords or adding conversational phrases;
2) perturbation-related errors, the most common of which is using undefined truth constants as part of tautological distractions. 

\paragraph{\textbf{\emph{F8: Autoformalisation increases semantic errors for counterfactuals.}}}
Unlike the introduced noise, counterfactual perturbations do not lead to more syntax errors. The execution rate in Table~\ref{tab:appendix_k4_formalisation_exec} is stable or improves for counterfactuals. However, we see a drop in accuracy for the counterfactual column $\text{O}_C$ in Table~\ref{tab:distraction_k4_formalisation} and can conclude that the number of logical forms with semantic errors has to increase. This suggests that the introduced negation is not correctly formalised. Looking at the warnings generated by the feedback mechanism, for GPT 4o-mini, $161$ warning messages are generated on the unperturbed data. $54$ of these were fixed with a single iteration. Not considering predicates and individuals as part of the context is the most frequent warning across all models. 

\section*{Conclusion}
This paper aims to enhance our understanding of the computational complexity of computing various Shapley value variants. We found that for various ML models --- including decision trees, regression tree ensembles, weighted automata, and linear regression --- both local and global interventional and baseline SHAP can be computed in polynomial time under HMM modeled distributions. This extends popular algorithms, such as TreeSHAP, beyond their empirical distributional scope. We also establish strict complexity gaps between the various SHAP variants (baseline, interventional, and conditional) and prove the intractability of computing SHAP for tree ensembles and neural networks in simplified scenarios. Overall, we present SHAP as a versatile framework whose complexity depends on four key factors: \begin{inparaenum}[(i)] \item model type, \item SHAP variant, \item distribution modeling approach, \item and local vs. global explanations\end{inparaenum}. We believe this perspective provides deeper insight into the computational complexity of SHAP, paving the way for future work.




%We believe that our framework provides a more intricate understanding of SHAP computation complexity across different models, distributions, and variants, paving the way for further research.

Our work opens promising directions for future research. First, expanding our computational analysis to other SHAP-related metrics, such as asymmetric SHAP~\citep{frye20} and SAGE~\citep{covert2020understanding}, would be valuable. Additionally, we aim to explore more expressive distribution classes and relaxed assumptions beyond those in Section \ref{sec:tractable} while maintaining tractable SHAP computation. Finally, when exact computation is intractable (Section \ref{sec:intractable}), investigating the approximability of SHAP metrics through approximation and parameterized complexity theory~\citep{downey2012parameterized} is an important direction.

%Our work opens several promising avenues for future research on the computational properties of explainable AI methods, with a particular focus on SHAP. First, it would be interesting to broaden the computational analysis conducted in this work to include other popular SHAP-related metrics in the literature, such as asymmetric SHAP \cite{frye20} and SAGE \cite{covert2020understanding}. Also, in the future, we aim to explore more expressive distribution classes and relaxed distributional assumptions—extending beyond those examined in Section \ref{sec:tractable} —that still yield tractable SHAP computation. Finally, when exact computation proves intractable (Section \ref{sec:intractable}), it is worthwhile to theoretically investigate the question of the approximability of computing the SHAP metrics across various configurations, through the lens of approximation and parametrized complexity theory \cite{arora2009computational}.

%This paper aims to deepen our understanding of the computational complexity involved in obtaining different Shapley value variants. We found that for a variety of ML models, including decision trees, tree ensembles for regression, weighted automata, and linear regression models — computing both local and global interventional and baseline SHAP can be done in polynomial time when distributions are modeled by HMMs. This extends the distributional scope of popular algorithms like TreeSHAP, which is limited to empirical distributions. Additionally, we demonstrate a strict complexity gap between SHAP variants, showing that interventional and baseline SHAP can be strictly easier to compute than conditional SHAP. Despite these positive results, we uncovered intractability for various SHAP variants in neural networks and tree ensembles. Finally, we provided generalized complexity relations across SHAP variants. We believe that our framework offers a deeper understanding of the complexity involved in computing SHAP across various variants, models, distributions, as well as in both local and global computations, laying the groundwork for future research.

\clearpage
%%
%% The next two lines define the bibliography style to be used, and
%% the bibliography file.
\bibliographystyle{plain}
\bibliography{main}

%%
%% If your work has an appendix, this is the place to put it.
% \appendix

\end{document}
\endinput
%%
%% End of file `sample-authordraft.tex'.
