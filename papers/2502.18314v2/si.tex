\documentclass[preprint,
 amsmath,amssymb,
 aip,
]{revtex4-2}

\usepackage{graphicx}\usepackage{dcolumn}\usepackage{bm}
\usepackage[T1]{fontenc}    \usepackage[hidelinks]{hyperref}       \usepackage{url}            \usepackage{booktabs}       \usepackage{amsfonts}       \usepackage{nicefrac}       \usepackage{microtype}      \usepackage{xcolor}         \usepackage{graphicx}
\usepackage{amsmath}

\makeatletter
\renewcommand*{\ext@figure}{toc}
\renewcommand*{\ext@table}{toc}
\makeatother

\usepackage{tocbasic}
\addtotoclist[\jobname]{toc}

\newcommand\entrywithprefix[2]{\hfill#1~#2 - }
\DeclareTOCStyleEntry[
  numwidth=6em,
  entrynumberformat=\entrywithprefix{\figurename}
]{tocline}{figure}
\DeclareTOCStyleEntry[
  numwidth=6em,
  entrynumberformat=\entrywithprefix{\tablename}
]{tocline}{table}

\renewcommand*{\thesection}{S\arabic{section}}
\renewcommand*{\thesubsection}{S\Alph{subsection}}
\renewcommand*{\theequation}{S\arabic{equation}}
\renewcommand*{\thetable}{S\arabic{table}}
\renewcommand*{\thefigure}{S\arabic{figure}}
\renewcommand*{\citenumfont}[1]{S#1}
\renewcommand*{\bibnumfmt}[1]{[S#1] }

\AtBeginDocument{\renewcommand{\natexlab}[1]{#1}} 
\begin{document}
\title{Supplementary Information for:\\
Learning atomic forces from uncertainty-calibrated adversarial attacks}
\author{Henrique Musseli Cezar}
\affiliation{Hylleraas Centre for Quantum Molecular Sciences and Department of Chemistry, University of Oslo, PO Box 1033 Blindern, 0315 Oslo, Norway}

\author{Tilmann Bodenstein}
\affiliation{Hylleraas Centre for Quantum Molecular Sciences and Department of Chemistry, University of Oslo, PO Box 1033 Blindern, 0315 Oslo, Norway}

\author{Henrik Andersen Sveinsson}
\affiliation{The Njord Centre, Department of Physics, University of Oslo, PO Box 1048 Blindern, 0316 Oslo, Norway}

\author{Morten Ledum}
\affiliation{Hylleraas Centre for Quantum Molecular Sciences and Department of Chemistry, University of Oslo, PO Box 1033 Blindern, 0315 Oslo, Norway}

\author{Simen Reine}
\affiliation{Hylleraas Centre for Quantum Molecular Sciences and Department of Chemistry, University of Oslo, PO Box 1033 Blindern, 0315 Oslo, Norway}

\author{Sigbjørn Løland Bore}
\email{s.l.bore@kjemi.uio.no}
\affiliation{Hylleraas Centre for Quantum Molecular Sciences and Department of Chemistry, University of Oslo, PO Box 1033 Blindern, 0315 Oslo, Norway}
\maketitle

{\small
\newpage
\tableofcontents
\newpage
}

\section{Additional analysis for CAGO to target error}
Training, validation, and test sets were created as specified in Table~\ref{sm:tab:train-val-test}. Twenty models were trained for 200000 steps, with an example learning curve for a single model shown in Figure~\ref{sm:fig:lcurve}. Then, picking out 40 structures from the test set, we apply the adversarial geometry optimization on those structures to benchmark different settings, such as uncertainty calibration, committee size, etc. In the final geometries, we compute the reference DNN@MB-pol forces, which we use to compute the errors for the committee.

\begin{table}[!htb]
  \centering
  \caption{The training, validation, and test set were created from DNN@MB-pol MD simulations of liquid water with 32 water molecules. The structures were gathered from 0.5~ns trajectories at various temperatures, disregarding the first 0.1~ns, for two different seeds each. For each of the 20 models, we created individual bootstrap training and validation sets from simulations with seed 1. The test set was created from simulations of seed 2, thereby being distinct from training and validation sets.}\label{sm:tab:train-val-test}
\begin{tabular}{ccc}
\hline\hline
Type&\# of structures& source\\
\hline
     Training set&500& 1~Bar, [300~K, 350~K, 400~K, 450~K, 500~K], seed 1\\
     Validation set&500&1~Bar, [300~K, 350~K, 400~K, 450~K, 500~K], seed 1\\
     Test set & 1000 &1~Bar, [300~K, 350~K, 400~K, 450~K, 500~K], seed 2\\ \hline\hline
\end{tabular}
\end{table}  
\begin{figure}[!htp]
    \centering
    \includegraphics{si-figures/si-figure1.pdf}
    \caption{Representative learning curve for forces, energies, and virials for one out of 20 MLIP models for DNN@MB-pol, with a running average over ten training steps.}
    \label{sm:fig:lcurve}
\end{figure}

Our negative log-likelihood optimization determines the uncertainty calibrations reported in Table~\ref{sm:tab:calibration-parameters}. Figure~\ref{fig:sm:calibration}a reports the distribution residuals, error divided by estimated uncertainty for forces, energies, and virials. For well-calibrated uncertainty estimates, residuals are distributed according to the normal distribution with mean zero and standard deviation equal to 1. This is the case for calibrated uncertainty for energy, forces, and virials, while the uncalibrated uncertainty underestimates the error, leading to broader distributions. The same picture is reflected in our force RMSE binned over uncertainty estimate in Figure~\ref{fig:sm:calibration}b, c, and d, where calibrated uncertainty has a near-perfect correlation with force RMSE, virial and energy RMSE. In contrast, uncalibrated uncertainty underestimates the true error.  
\begin{table}[!htb]
    \centering
        \caption{Calibration parameters for power-law correction.}\label{sm:tab:calibration-parameters}
\begin{tabular}{lccc}
\hline\hline
Property & Unit & a & b \\
\hline
forces & eV/Å & 0.609 & 0.506 \\
energy & eV & 0.8799& 0.7379 \\
virial & eV & 1.338& 0.7256 \\
\hline\hline
\end{tabular}
\end{table}

\begin{figure}[!htb]
    \centering
    \includegraphics[width=1\textwidth]{si-figures/si-figure2.pdf}
    \caption{Statistical properties of calibrated uncertainty. (a) The distribution of errors is divided by estimated uncertainty for force, energy, and virial prediction, with and without error calibration. (b,c,d)  Average RMSE binned by uncertainty estimate for calibrated and uncalibrated uncertainty for forces, energy and virials, respectively. Error bars correspond to 95~\% confidence intervals. }
    \label{fig:sm:calibration}
\end{figure}

As discussed in the main text, the size of the committees affects the calibration, as can be seen from the change of the calibration parameters with respect to committee size in Figure~\ref{fig:sm:benchmark}a. However, despite the calibration parameters being different for the different committee sizes, as shown in the main text, starting from 10 committee members, the calibration is sufficient to provide reliable results. In Figure~\ref{fig:sm:benchmark}b and c, we show an example of biasing towards zero average force magnitude. If chosen moderately, one can avoid highly steric structures, with error statistics largely unaffected. This is potentially useful when subsequently doing reference calculations, which can either fail or struggle to converge for structures dominated by steric repulsions.

\begin{figure}[!htb]
    \centering
    \includegraphics[width=1\textwidth]{si-figures/si-figure3.pdf}
    \caption{Statistical properties of calibrated uncertainty. \textbf{a.} Uncertainty calibration constants as a function of committee size. \textbf{b.} Distribution of prediction error for different force bias prefactors and corresponding mean force error. \textbf{c.} Distribution of force magnitude for different force-bias prefactors, as well as from reference molecular dynamics simulations and corresponding means in inset. }
    \label{fig:sm:benchmark}
\end{figure}

\section{Structures and training sets}
The main systems used to get the results presented in the manuscript are presented in Figure~\ref{sm-fig:systems}.
System I uses structures extracted from molecular dynamics simulations using DNN@MB-pol.\cite{bore_realistic_2023}
System II is created based on a snapshot of a TIP5P\cite{mahoney-fivesite-2000} simulation at 1~bar and 300 ~K.
System III is built starting from a crystal structure of UiO-66 optimized with PBEsol,\cite{wmd-crystal-2025} and by adding water molecules using Packmol.\cite{martinez_packmol_2009}
For the structures in systems II and III, we have geometry optimized the structures using the PBE functional\cite{perdew-generalized-1996} with Grimme dispersion corrections (DFT-D3)\cite{grimme-consistent-2010} before using them in the active learning loop. All the initial structures, as well as the CAGO optimized structures during the active learning loops, are provided in the data repository associated with this manuscript.

\begin{figure}[!htb]
    \centering
    \includegraphics[width=1\textwidth]{si-figures/si-figure4.pdf}
    \caption{Molecular structures. {\bf a.} System I: box of 32 water molecules. {\bf b.} System II: Box of 64 water molecules. {\bf c.} System III: UiO-66 with water weight percentage from 0 to 45, and a box of 64 water molecules. The color coding for the atoms is red, white, gray, and green, respectively, for oxygen, hydrogen, carbon, and zirconium.}
    \label{sm-fig:systems}
\end{figure}

\section{Additional analysis for Learning liquid water from a single structure}
In Figure~4 of the main text, we have computed the first coordination number based on the O-O radial distribution function (radial distribution function). Here, in Figure~\ref{sm:fig:rdfs-water}, the radial distribution functions considering the average between all frames and models at each iteration for the DeePMD and Allegro models are shown.
The shaded regions in each color represent the minimum and maximum at each bin distance $r$ among the 12 different models of the committee.
In the first iterations, the models do not capture the liquid structure of water, especially the DeePMD models.
Many models also show peaks at distances shorter than the typical first peak of solvation around 2.5~\AA. As more configurations are added to the training set, the radial distribution functions converge to a liquid radial distribution function,  exhibiting well-converged radial distribution functions in the last few iterations. However, these small differences are captured by the first coordination number, as shown in Figure~4 of the main text.
\begin{figure*}
    \centering
    \includegraphics[width=0.99\linewidth]{si-figures/si-figure5.pdf}
    \caption{Oxygen-oxygen radial distribution functions for the pure water models. The radial distribution functions are shifted based on the iteration of the active learning loop, meaning that the radial distribution function that has its zero at 5, for example, belongs to the fifth iteration. The radial distribution functions are computed as the average between all the committee models, and the minimum and maximum value among the models for each distance is represented as the shaded region.}
    \label{sm:fig:rdfs-water}
\end{figure*}

\bibliographystyle{apsrev4-1}
\bibliography{mybib}

\end{document}