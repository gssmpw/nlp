
\documentclass{article} % For LaTeX2e
\usepackage{iclr2025_conference,times}

% Optional math commands from https://github.com/goodfeli/dlbook_notation.
%%%%% NEW MATH DEFINITIONS %%%%%

\usepackage{amsmath,amsfonts,bm}
\usepackage{derivative}
% Mark sections of captions for referring to divisions of figures
\newcommand{\figleft}{{\em (Left)}}
\newcommand{\figcenter}{{\em (Center)}}
\newcommand{\figright}{{\em (Right)}}
\newcommand{\figtop}{{\em (Top)}}
\newcommand{\figbottom}{{\em (Bottom)}}
\newcommand{\captiona}{{\em (a)}}
\newcommand{\captionb}{{\em (b)}}
\newcommand{\captionc}{{\em (c)}}
\newcommand{\captiond}{{\em (d)}}

% Highlight a newly defined term
\newcommand{\newterm}[1]{{\bf #1}}

% Derivative d 
\newcommand{\deriv}{{\mathrm{d}}}

% Figure reference, lower-case.
\def\figref#1{figure~\ref{#1}}
% Figure reference, capital. For start of sentence
\def\Figref#1{Figure~\ref{#1}}
\def\twofigref#1#2{figures \ref{#1} and \ref{#2}}
\def\quadfigref#1#2#3#4{figures \ref{#1}, \ref{#2}, \ref{#3} and \ref{#4}}
% Section reference, lower-case.
\def\secref#1{section~\ref{#1}}
% Section reference, capital.
\def\Secref#1{Section~\ref{#1}}
% Reference to two sections.
\def\twosecrefs#1#2{sections \ref{#1} and \ref{#2}}
% Reference to three sections.
\def\secrefs#1#2#3{sections \ref{#1}, \ref{#2} and \ref{#3}}
% Reference to an equation, lower-case.
\def\eqref#1{equation~\ref{#1}}
% Reference to an equation, upper case
\def\Eqref#1{Equation~\ref{#1}}
% A raw reference to an equation---avoid using if possible
\def\plaineqref#1{\ref{#1}}
% Reference to a chapter, lower-case.
\def\chapref#1{chapter~\ref{#1}}
% Reference to an equation, upper case.
\def\Chapref#1{Chapter~\ref{#1}}
% Reference to a range of chapters
\def\rangechapref#1#2{chapters\ref{#1}--\ref{#2}}
% Reference to an algorithm, lower-case.
\def\algref#1{algorithm~\ref{#1}}
% Reference to an algorithm, upper case.
\def\Algref#1{Algorithm~\ref{#1}}
\def\twoalgref#1#2{algorithms \ref{#1} and \ref{#2}}
\def\Twoalgref#1#2{Algorithms \ref{#1} and \ref{#2}}
% Reference to a part, lower case
\def\partref#1{part~\ref{#1}}
% Reference to a part, upper case
\def\Partref#1{Part~\ref{#1}}
\def\twopartref#1#2{parts \ref{#1} and \ref{#2}}

\def\ceil#1{\lceil #1 \rceil}
\def\floor#1{\lfloor #1 \rfloor}
\def\1{\bm{1}}
\newcommand{\train}{\mathcal{D}}
\newcommand{\valid}{\mathcal{D_{\mathrm{valid}}}}
\newcommand{\test}{\mathcal{D_{\mathrm{test}}}}

\def\eps{{\epsilon}}


% Random variables
\def\reta{{\textnormal{$\eta$}}}
\def\ra{{\textnormal{a}}}
\def\rb{{\textnormal{b}}}
\def\rc{{\textnormal{c}}}
\def\rd{{\textnormal{d}}}
\def\re{{\textnormal{e}}}
\def\rf{{\textnormal{f}}}
\def\rg{{\textnormal{g}}}
\def\rh{{\textnormal{h}}}
\def\ri{{\textnormal{i}}}
\def\rj{{\textnormal{j}}}
\def\rk{{\textnormal{k}}}
\def\rl{{\textnormal{l}}}
% rm is already a command, just don't name any random variables m
\def\rn{{\textnormal{n}}}
\def\ro{{\textnormal{o}}}
\def\rp{{\textnormal{p}}}
\def\rq{{\textnormal{q}}}
\def\rr{{\textnormal{r}}}
\def\rs{{\textnormal{s}}}
\def\rt{{\textnormal{t}}}
\def\ru{{\textnormal{u}}}
\def\rv{{\textnormal{v}}}
\def\rw{{\textnormal{w}}}
\def\rx{{\textnormal{x}}}
\def\ry{{\textnormal{y}}}
\def\rz{{\textnormal{z}}}

% Random vectors
\def\rvepsilon{{\mathbf{\epsilon}}}
\def\rvphi{{\mathbf{\phi}}}
\def\rvtheta{{\mathbf{\theta}}}
\def\rva{{\mathbf{a}}}
\def\rvb{{\mathbf{b}}}
\def\rvc{{\mathbf{c}}}
\def\rvd{{\mathbf{d}}}
\def\rve{{\mathbf{e}}}
\def\rvf{{\mathbf{f}}}
\def\rvg{{\mathbf{g}}}
\def\rvh{{\mathbf{h}}}
\def\rvu{{\mathbf{i}}}
\def\rvj{{\mathbf{j}}}
\def\rvk{{\mathbf{k}}}
\def\rvl{{\mathbf{l}}}
\def\rvm{{\mathbf{m}}}
\def\rvn{{\mathbf{n}}}
\def\rvo{{\mathbf{o}}}
\def\rvp{{\mathbf{p}}}
\def\rvq{{\mathbf{q}}}
\def\rvr{{\mathbf{r}}}
\def\rvs{{\mathbf{s}}}
\def\rvt{{\mathbf{t}}}
\def\rvu{{\mathbf{u}}}
\def\rvv{{\mathbf{v}}}
\def\rvw{{\mathbf{w}}}
\def\rvx{{\mathbf{x}}}
\def\rvy{{\mathbf{y}}}
\def\rvz{{\mathbf{z}}}

% Elements of random vectors
\def\erva{{\textnormal{a}}}
\def\ervb{{\textnormal{b}}}
\def\ervc{{\textnormal{c}}}
\def\ervd{{\textnormal{d}}}
\def\erve{{\textnormal{e}}}
\def\ervf{{\textnormal{f}}}
\def\ervg{{\textnormal{g}}}
\def\ervh{{\textnormal{h}}}
\def\ervi{{\textnormal{i}}}
\def\ervj{{\textnormal{j}}}
\def\ervk{{\textnormal{k}}}
\def\ervl{{\textnormal{l}}}
\def\ervm{{\textnormal{m}}}
\def\ervn{{\textnormal{n}}}
\def\ervo{{\textnormal{o}}}
\def\ervp{{\textnormal{p}}}
\def\ervq{{\textnormal{q}}}
\def\ervr{{\textnormal{r}}}
\def\ervs{{\textnormal{s}}}
\def\ervt{{\textnormal{t}}}
\def\ervu{{\textnormal{u}}}
\def\ervv{{\textnormal{v}}}
\def\ervw{{\textnormal{w}}}
\def\ervx{{\textnormal{x}}}
\def\ervy{{\textnormal{y}}}
\def\ervz{{\textnormal{z}}}

% Random matrices
\def\rmA{{\mathbf{A}}}
\def\rmB{{\mathbf{B}}}
\def\rmC{{\mathbf{C}}}
\def\rmD{{\mathbf{D}}}
\def\rmE{{\mathbf{E}}}
\def\rmF{{\mathbf{F}}}
\def\rmG{{\mathbf{G}}}
\def\rmH{{\mathbf{H}}}
\def\rmI{{\mathbf{I}}}
\def\rmJ{{\mathbf{J}}}
\def\rmK{{\mathbf{K}}}
\def\rmL{{\mathbf{L}}}
\def\rmM{{\mathbf{M}}}
\def\rmN{{\mathbf{N}}}
\def\rmO{{\mathbf{O}}}
\def\rmP{{\mathbf{P}}}
\def\rmQ{{\mathbf{Q}}}
\def\rmR{{\mathbf{R}}}
\def\rmS{{\mathbf{S}}}
\def\rmT{{\mathbf{T}}}
\def\rmU{{\mathbf{U}}}
\def\rmV{{\mathbf{V}}}
\def\rmW{{\mathbf{W}}}
\def\rmX{{\mathbf{X}}}
\def\rmY{{\mathbf{Y}}}
\def\rmZ{{\mathbf{Z}}}

% Elements of random matrices
\def\ermA{{\textnormal{A}}}
\def\ermB{{\textnormal{B}}}
\def\ermC{{\textnormal{C}}}
\def\ermD{{\textnormal{D}}}
\def\ermE{{\textnormal{E}}}
\def\ermF{{\textnormal{F}}}
\def\ermG{{\textnormal{G}}}
\def\ermH{{\textnormal{H}}}
\def\ermI{{\textnormal{I}}}
\def\ermJ{{\textnormal{J}}}
\def\ermK{{\textnormal{K}}}
\def\ermL{{\textnormal{L}}}
\def\ermM{{\textnormal{M}}}
\def\ermN{{\textnormal{N}}}
\def\ermO{{\textnormal{O}}}
\def\ermP{{\textnormal{P}}}
\def\ermQ{{\textnormal{Q}}}
\def\ermR{{\textnormal{R}}}
\def\ermS{{\textnormal{S}}}
\def\ermT{{\textnormal{T}}}
\def\ermU{{\textnormal{U}}}
\def\ermV{{\textnormal{V}}}
\def\ermW{{\textnormal{W}}}
\def\ermX{{\textnormal{X}}}
\def\ermY{{\textnormal{Y}}}
\def\ermZ{{\textnormal{Z}}}

% Vectors
\def\vzero{{\bm{0}}}
\def\vone{{\bm{1}}}
\def\vmu{{\bm{\mu}}}
\def\vtheta{{\bm{\theta}}}
\def\vphi{{\bm{\phi}}}
\def\va{{\bm{a}}}
\def\vb{{\bm{b}}}
\def\vc{{\bm{c}}}
\def\vd{{\bm{d}}}
\def\ve{{\bm{e}}}
\def\vf{{\bm{f}}}
\def\vg{{\bm{g}}}
\def\vh{{\bm{h}}}
\def\vi{{\bm{i}}}
\def\vj{{\bm{j}}}
\def\vk{{\bm{k}}}
\def\vl{{\bm{l}}}
\def\vm{{\bm{m}}}
\def\vn{{\bm{n}}}
\def\vo{{\bm{o}}}
\def\vp{{\bm{p}}}
\def\vq{{\bm{q}}}
\def\vr{{\bm{r}}}
\def\vs{{\bm{s}}}
\def\vt{{\bm{t}}}
\def\vu{{\bm{u}}}
\def\vv{{\bm{v}}}
\def\vw{{\bm{w}}}
\def\vx{{\bm{x}}}
\def\vy{{\bm{y}}}
\def\vz{{\bm{z}}}

% Elements of vectors
\def\evalpha{{\alpha}}
\def\evbeta{{\beta}}
\def\evepsilon{{\epsilon}}
\def\evlambda{{\lambda}}
\def\evomega{{\omega}}
\def\evmu{{\mu}}
\def\evpsi{{\psi}}
\def\evsigma{{\sigma}}
\def\evtheta{{\theta}}
\def\eva{{a}}
\def\evb{{b}}
\def\evc{{c}}
\def\evd{{d}}
\def\eve{{e}}
\def\evf{{f}}
\def\evg{{g}}
\def\evh{{h}}
\def\evi{{i}}
\def\evj{{j}}
\def\evk{{k}}
\def\evl{{l}}
\def\evm{{m}}
\def\evn{{n}}
\def\evo{{o}}
\def\evp{{p}}
\def\evq{{q}}
\def\evr{{r}}
\def\evs{{s}}
\def\evt{{t}}
\def\evu{{u}}
\def\evv{{v}}
\def\evw{{w}}
\def\evx{{x}}
\def\evy{{y}}
\def\evz{{z}}

% Matrix
\def\mA{{\bm{A}}}
\def\mB{{\bm{B}}}
\def\mC{{\bm{C}}}
\def\mD{{\bm{D}}}
\def\mE{{\bm{E}}}
\def\mF{{\bm{F}}}
\def\mG{{\bm{G}}}
\def\mH{{\bm{H}}}
\def\mI{{\bm{I}}}
\def\mJ{{\bm{J}}}
\def\mK{{\bm{K}}}
\def\mL{{\bm{L}}}
\def\mM{{\bm{M}}}
\def\mN{{\bm{N}}}
\def\mO{{\bm{O}}}
\def\mP{{\bm{P}}}
\def\mQ{{\bm{Q}}}
\def\mR{{\bm{R}}}
\def\mS{{\bm{S}}}
\def\mT{{\bm{T}}}
\def\mU{{\bm{U}}}
\def\mV{{\bm{V}}}
\def\mW{{\bm{W}}}
\def\mX{{\bm{X}}}
\def\mY{{\bm{Y}}}
\def\mZ{{\bm{Z}}}
\def\mBeta{{\bm{\beta}}}
\def\mPhi{{\bm{\Phi}}}
\def\mLambda{{\bm{\Lambda}}}
\def\mSigma{{\bm{\Sigma}}}

% Tensor
\DeclareMathAlphabet{\mathsfit}{\encodingdefault}{\sfdefault}{m}{sl}
\SetMathAlphabet{\mathsfit}{bold}{\encodingdefault}{\sfdefault}{bx}{n}
\newcommand{\tens}[1]{\bm{\mathsfit{#1}}}
\def\tA{{\tens{A}}}
\def\tB{{\tens{B}}}
\def\tC{{\tens{C}}}
\def\tD{{\tens{D}}}
\def\tE{{\tens{E}}}
\def\tF{{\tens{F}}}
\def\tG{{\tens{G}}}
\def\tH{{\tens{H}}}
\def\tI{{\tens{I}}}
\def\tJ{{\tens{J}}}
\def\tK{{\tens{K}}}
\def\tL{{\tens{L}}}
\def\tM{{\tens{M}}}
\def\tN{{\tens{N}}}
\def\tO{{\tens{O}}}
\def\tP{{\tens{P}}}
\def\tQ{{\tens{Q}}}
\def\tR{{\tens{R}}}
\def\tS{{\tens{S}}}
\def\tT{{\tens{T}}}
\def\tU{{\tens{U}}}
\def\tV{{\tens{V}}}
\def\tW{{\tens{W}}}
\def\tX{{\tens{X}}}
\def\tY{{\tens{Y}}}
\def\tZ{{\tens{Z}}}


% Graph
\def\gA{{\mathcal{A}}}
\def\gB{{\mathcal{B}}}
\def\gC{{\mathcal{C}}}
\def\gD{{\mathcal{D}}}
\def\gE{{\mathcal{E}}}
\def\gF{{\mathcal{F}}}
\def\gG{{\mathcal{G}}}
\def\gH{{\mathcal{H}}}
\def\gI{{\mathcal{I}}}
\def\gJ{{\mathcal{J}}}
\def\gK{{\mathcal{K}}}
\def\gL{{\mathcal{L}}}
\def\gM{{\mathcal{M}}}
\def\gN{{\mathcal{N}}}
\def\gO{{\mathcal{O}}}
\def\gP{{\mathcal{P}}}
\def\gQ{{\mathcal{Q}}}
\def\gR{{\mathcal{R}}}
\def\gS{{\mathcal{S}}}
\def\gT{{\mathcal{T}}}
\def\gU{{\mathcal{U}}}
\def\gV{{\mathcal{V}}}
\def\gW{{\mathcal{W}}}
\def\gX{{\mathcal{X}}}
\def\gY{{\mathcal{Y}}}
\def\gZ{{\mathcal{Z}}}

% Sets
\def\sA{{\mathbb{A}}}
\def\sB{{\mathbb{B}}}
\def\sC{{\mathbb{C}}}
\def\sD{{\mathbb{D}}}
% Don't use a set called E, because this would be the same as our symbol
% for expectation.
\def\sF{{\mathbb{F}}}
\def\sG{{\mathbb{G}}}
\def\sH{{\mathbb{H}}}
\def\sI{{\mathbb{I}}}
\def\sJ{{\mathbb{J}}}
\def\sK{{\mathbb{K}}}
\def\sL{{\mathbb{L}}}
\def\sM{{\mathbb{M}}}
\def\sN{{\mathbb{N}}}
\def\sO{{\mathbb{O}}}
\def\sP{{\mathbb{P}}}
\def\sQ{{\mathbb{Q}}}
\def\sR{{\mathbb{R}}}
\def\sS{{\mathbb{S}}}
\def\sT{{\mathbb{T}}}
\def\sU{{\mathbb{U}}}
\def\sV{{\mathbb{V}}}
\def\sW{{\mathbb{W}}}
\def\sX{{\mathbb{X}}}
\def\sY{{\mathbb{Y}}}
\def\sZ{{\mathbb{Z}}}

% Entries of a matrix
\def\emLambda{{\Lambda}}
\def\emA{{A}}
\def\emB{{B}}
\def\emC{{C}}
\def\emD{{D}}
\def\emE{{E}}
\def\emF{{F}}
\def\emG{{G}}
\def\emH{{H}}
\def\emI{{I}}
\def\emJ{{J}}
\def\emK{{K}}
\def\emL{{L}}
\def\emM{{M}}
\def\emN{{N}}
\def\emO{{O}}
\def\emP{{P}}
\def\emQ{{Q}}
\def\emR{{R}}
\def\emS{{S}}
\def\emT{{T}}
\def\emU{{U}}
\def\emV{{V}}
\def\emW{{W}}
\def\emX{{X}}
\def\emY{{Y}}
\def\emZ{{Z}}
\def\emSigma{{\Sigma}}

% entries of a tensor
% Same font as tensor, without \bm wrapper
\newcommand{\etens}[1]{\mathsfit{#1}}
\def\etLambda{{\etens{\Lambda}}}
\def\etA{{\etens{A}}}
\def\etB{{\etens{B}}}
\def\etC{{\etens{C}}}
\def\etD{{\etens{D}}}
\def\etE{{\etens{E}}}
\def\etF{{\etens{F}}}
\def\etG{{\etens{G}}}
\def\etH{{\etens{H}}}
\def\etI{{\etens{I}}}
\def\etJ{{\etens{J}}}
\def\etK{{\etens{K}}}
\def\etL{{\etens{L}}}
\def\etM{{\etens{M}}}
\def\etN{{\etens{N}}}
\def\etO{{\etens{O}}}
\def\etP{{\etens{P}}}
\def\etQ{{\etens{Q}}}
\def\etR{{\etens{R}}}
\def\etS{{\etens{S}}}
\def\etT{{\etens{T}}}
\def\etU{{\etens{U}}}
\def\etV{{\etens{V}}}
\def\etW{{\etens{W}}}
\def\etX{{\etens{X}}}
\def\etY{{\etens{Y}}}
\def\etZ{{\etens{Z}}}

% The true underlying data generating distribution
\newcommand{\pdata}{p_{\rm{data}}}
\newcommand{\ptarget}{p_{\rm{target}}}
\newcommand{\pprior}{p_{\rm{prior}}}
\newcommand{\pbase}{p_{\rm{base}}}
\newcommand{\pref}{p_{\rm{ref}}}

% The empirical distribution defined by the training set
\newcommand{\ptrain}{\hat{p}_{\rm{data}}}
\newcommand{\Ptrain}{\hat{P}_{\rm{data}}}
% The model distribution
\newcommand{\pmodel}{p_{\rm{model}}}
\newcommand{\Pmodel}{P_{\rm{model}}}
\newcommand{\ptildemodel}{\tilde{p}_{\rm{model}}}
% Stochastic autoencoder distributions
\newcommand{\pencode}{p_{\rm{encoder}}}
\newcommand{\pdecode}{p_{\rm{decoder}}}
\newcommand{\precons}{p_{\rm{reconstruct}}}

\newcommand{\laplace}{\mathrm{Laplace}} % Laplace distribution

\newcommand{\E}{\mathbb{E}}
\newcommand{\Ls}{\mathcal{L}}
\newcommand{\R}{\mathbb{R}}
\newcommand{\emp}{\tilde{p}}
\newcommand{\lr}{\alpha}
\newcommand{\reg}{\lambda}
\newcommand{\rect}{\mathrm{rectifier}}
\newcommand{\softmax}{\mathrm{softmax}}
\newcommand{\sigmoid}{\sigma}
\newcommand{\softplus}{\zeta}
\newcommand{\KL}{D_{\mathrm{KL}}}
\newcommand{\Var}{\mathrm{Var}}
\newcommand{\standarderror}{\mathrm{SE}}
\newcommand{\Cov}{\mathrm{Cov}}
% Wolfram Mathworld says $L^2$ is for function spaces and $\ell^2$ is for vectors
% But then they seem to use $L^2$ for vectors throughout the site, and so does
% wikipedia.
\newcommand{\normlzero}{L^0}
\newcommand{\normlone}{L^1}
\newcommand{\normltwo}{L^2}
\newcommand{\normlp}{L^p}
\newcommand{\normmax}{L^\infty}

\newcommand{\parents}{Pa} % See usage in notation.tex. Chosen to match Daphne's book.

\DeclareMathOperator*{\argmax}{arg\,max}
\DeclareMathOperator*{\argmin}{arg\,min}

\DeclareMathOperator{\sign}{sign}
\DeclareMathOperator{\Tr}{Tr}
\let\ab\allowbreak


\usepackage{hyperref}
\usepackage{url}
\usepackage{graphicx}
\usepackage{amsmath}
\usepackage{booktabs}
\usepackage{multirow}
\usepackage{subcaption}
\usepackage{caption}
\usepackage{algorithm}
\usepackage{algpseudocode}
\usepackage{wrapfig}
\usepackage{amssymb}% http://ctan.org/pkg/amssymb
\usepackage{pifont}% http://ctan.org/pkg/pifont
\newcommand{\cmark}{\ding{51}}%
\newcommand{\xmark}{\ding{55}}%
\usepackage{colortbl}

\title{Think-Then-React: Towards Unconstrained \\Human Action-to-Reaction Generation}


\author{Wenhui Tan, Boyuan Li, Chuhao Jin, Wenbing Huang, Xiting Wang \& Ruihua Song \\
Gaoling School of Artificial Intelligence\\
Renmin University of China\\
Beijing, China\\
\texttt{\{tanwenhui404,liboyuan,jinchuhao,hwenbing,xitingwang,rsong\}@ruc.edu.cn}
}


\newcommand{\fix}{\marginpar{FIX}}
\newcommand{\new}{\marginpar{NEW}}


% macros:
\newcommand{\ModelName}{Think-Then-React}
\newcommand{\ModelAbbr}{TTR}

\iclrfinalcopy % Uncomment for camera-ready version, but NOT for submission.
\begin{document}


\maketitle


\begin{figure}[h]
    \centering
    \includegraphics[width=\textwidth]{figs/teaser.pdf}
    \caption{Given a human action as input, our Think-Then-React model first \textbf{thinks} by generating an action description and reasons out a reaction prompt. It then \textbf{reacts} to the action based on the results of this thinking process. TTR reacts in a real-time manner at every timestep and periodically re-thinks at specific interval (every two timesteps in the illustration) to mitigate accumulated errors.}
    \label{fig:teaser}
\end{figure}


\begin{abstract}
Modeling human-like action-to-reaction generation has significant real-world applications, like human-robot interaction and games.
Despite recent advancements in single-person motion generation, it is still challenging to well handle action-to-reaction generation, due to the difficulty of directly predicting reaction from action sequence without prompts, and the absence of a unified representation that effectively encodes multi-person motion.
To address these challenges, we introduce Think-Then-React (TTR), a large language-model-based framework designed to generate human-like reactions.
First, with our fine-grained multimodal training strategy, TTR is capable to unify two processes during inference: a \textbf{thinking} process that explicitly infers action intentions and reasons corresponding reaction description, which serve as semantic prompts, and a \textbf{reacting} process that predicts reactions based on input action and the inferred semantic prompts.
Second, to effectively represent multi-person motion in language models, we propose a unified motion tokenizer by decoupling egocentric pose and absolute space features, which effectively represents action and reaction motion with same encoding.
Extensive experiments demonstrate that TTR outperforms existing baselines, achieving significant improvements in evaluation metrics, such as reducing FID from 3.988 to 1.942.
\end{abstract}

\section{Introduction}

Deep Reinforcement Learning (DRL) has emerged as a transformative paradigm for solving complex sequential decision-making problems. By enabling autonomous agents to interact with an environment, receive feedback in the form of rewards, and iteratively refine their policies, DRL has demonstrated remarkable success across a diverse range of domains including games (\eg Atari~\citep{mnih2013playing,kaiser2020model}, Go~\citep{silver2018general,silver2017mastering}, and StarCraft II~\citep{vinyals2019grandmaster,vinyals2017starcraft}), robotics~\citep{kalashnikov2018scalable}, communication networks~\citep{feriani2021single}, and finance~\citep{liu2024dynamic}. These successes underscore DRL's capability to surpass traditional rule-based systems, particularly in high-dimensional and dynamically evolving environments.

Despite these advances, a fundamental challenge remains: DRL agents typically rely on deep neural networks, which operate as black-box models, obscuring the rationale behind their decision-making processes. This opacity poses significant barriers to adoption in safety-critical and high-stakes applications, where interpretability is crucial for trust, compliance, and debugging. The lack of transparency in DRL can lead to unreliable decision-making, rendering it unsuitable for domains where explainability is a prerequisite, such as healthcare, autonomous driving, and financial risk assessment.

To address these concerns, the field of Explainable Deep Reinforcement Learning (XRL) has emerged, aiming to develop techniques that enhance the interpretability of DRL policies. XRL seeks to provide insights into an agent’s decision-making process, enabling researchers, practitioners, and end-users to understand, validate, and refine learned policies. By facilitating greater transparency, XRL contributes to the development of safer, more robust, and ethically aligned AI systems.

Furthermore, the increasing integration of Reinforcement Learning (RL) with Large Language Models (LLMs) has placed RL at the forefront of natural language processing (NLP) advancements. Methods such as Reinforcement Learning from Human Feedback (RLHF)~\citep{bai2022training,ouyang2022training} have become essential for aligning LLM outputs with human preferences and ethical guidelines. By treating language generation as a sequential decision-making process, RL-based fine-tuning enables LLMs to optimize for attributes such as factual accuracy, coherence, and user satisfaction, surpassing conventional supervised learning techniques. However, the application of RL in LLM alignment further amplifies the explainability challenge, as the complex interactions between RL updates and neural representations remain poorly understood.

This survey provides a systematic review of explainability methods in DRL, with a particular focus on their integration with LLMs and human-in-the-loop systems. We first introduce fundamental RL concepts and highlight key advances in DRL. We then categorize and analyze existing explanation techniques, encompassing feature-level, state-level, dataset-level, and model-level approaches. Additionally, we discuss methods for evaluating XRL techniques, considering both qualitative and quantitative assessment criteria. Finally, we explore real-world applications of XRL, including policy refinement, adversarial attack mitigation, and emerging challenges in ensuring interpretability in modern AI systems. Through this survey, we aim to provide a comprehensive perspective on the current state of XRL and outline future research directions to advance the development of interpretable and trustworthy DRL models.
\section{Related Work}

\subsection{Large Language Models in Biosciences}
Large language models (LLMs) have emerged as powerful tools for natural language comprehension and generation~\cite{llms-survey}. Beyond their application in traditional natural language tasks, there is a growing interest in leveraging LLMs to accelerate scientific research. Early studies revealed that general-purpose LLMs, owing to their rich pre-training data, exhibit promise across various research domains~\cite{ai4science}. Subsequent efforts have focused on directly training LLMs using domain-specific data, aiming to extend the transfer learning paradigm from natural language processing (NLP) to biosciences. This body of work primarily falls into three categories: molecular LLMs, protein LLMs, and genomic LLMs.

For molecular modeling, extensive work has been conducted on training with various molecular string representations, such as SMILES~\cite{Smiles-bert,space-of-chemical,large-scale-chemical}, SELFIES~\cite{SELFIES,chemberta,chemberta2}, and InChI~\cite{inchi}. Additionally, several studies address the modeling of molecular 2D~\cite{mol-2d} and 3D structures~\cite{uni-mol} to capture more detailed molecular characteristics. In the realm of protein LLMs, related work~\cite{msa-transformer,esm2,Prottrans} mainly concentrates on modeling the primary structure of proteins (amino acid sequences), providing a solid foundation for protein structure prediction~\cite{AlphaFold2,AlphaFold3}. For genomic sequences, numerous studies have attempted to leverage the power of LLMs for improved genomic analysis and understanding. These efforts predominantly involve training models on DNA~\cite{BPNet,DNABERT,enformer,nucleotide-transformer,DNABERT-2,GROVER,gena-lm,Caduceus,dnagpt,megaDNA,HyenaDNA,Evo} and RNA~\cite{RNAErnie,uni-rna,Rinalmo} sequences. In the following section, we delve deeper into genomic LLMs specifically designed for DNA sequence modeling.

\subsection{DNA Language Models}
In the early stages, \citeauthor{BPNet} introduced the BPNet convolutional architecture to learn transcription factor binding patterns and their syntax in a supervised manner. Prior to the emergence of large-scale pre-training, BPNet was widely used in genomics for supervised learning on relatively small datasets. With the advent of BERT~\cite{BERT}, DNABERT~\cite{DNABERT} pioneered the application of pre-training on the human genome using K-mer tokenizers. To effectively capture long-range interactions, Enformer~\cite{enformer} advanced human genome modeling by incorporating convolutional downsampling into transformer architectures.

Following these foundational works, numerous models based on the transformer encoder architecture have emerged. A notable example is the Nucleotide Transformer (NT)~\cite{nucleotide-transformer}, which scales model parameters from 100 million to 2.5 billion and includes a diverse set of multispecies genomes. Recent studies, DNABERT-2~\cite{DNABERT-2} and GROVER~\cite{GROVER}, have investigated optimal tokenizer settings for masked language modeling, concluding that Byte Pair Encoding (BPE) is better suited for masked DNA LLMs. The majority of these models face the limitation of insufficient context length, primarily due to the high computational cost associated with extending the context length in the transformer architecture. To address this limitation, GENA-LM~\cite{gena-lm} employs sparse attention, and Caduceus~\cite{Caduceus} uses the more lightweight BiMamba architecture~\cite{Mamba}, both trained on the human genome.

Although these masked DNA LLMs effectively understand and predict DNA sequences, they lack generative capabilities, and generative DNA LLMs remain in the early stages of development. An early preprint~\cite{dnagpt} introduced DNAGPT, which learns mammalian genomic structures through three pre-training tasks, including next token prediction. Recent works, such as HyenaDNA~\cite{HyenaDNA} and megaDNA~\cite{megaDNA}, achieve longer context lengths by employing the Hyena~\cite{Hyena} and multiscale transformer architectures respectively, though they are significantly limited by their data and model scales. A more recent influential study, Evo~\cite{Evo}, trained on an extensive dataset of prokaryotic and viral genomes, has garnered widespread attention for its success in designing CRISPR-Cas molecular complexes, thus demonstrating the practical utility of generative DNA LLMs in the genomic field.


\section{\underline{V}ision \underline{L}anguage \underline{D}isinformation Detection \underline{Bench}mark}  
\label{method}  
\textsf{\textbf{\textsc{VLDBench}}} (Figure~\ref{fig:vlbias}) is a comprehensive classification multimodal benchmark for disinformation detection in news articles. It comprises 31,339 articles and visual samples curated from 58 news sources ranging from the Financial Times, CNN, and New York Times to Axios and Wall Street Journal as shown in Figure \ref{fig:news_sources_distribution}. \textsf{\textbf{\textsc{VLDBench}}} spans 13 unique categories (Figure \ref{fig:news_categories}) : \textit{National, Business and Finance, International, Entertainment, Local/Regional, Opinion/Editorial, Health, Sports, Politics, Weather and Environment, Technology, Science, and Other} —adding depth to the disinformation domains. We present the further statistical details in Appendix \ref{app:data-analysis}.
 
\subsection{Task Definition}
\textbf{Disinformation Detection:}  
The core task is to determine whether a text–image news article contains disinformation. We adopt the following definition:  

\emph{“False, misleading, or manipulated information—textual or visual—intentionally created or disseminated to deceive, harm, or influence individuals, groups, or public opinion.”}  

This definition aligns with established social science research \cite{benkler2018network} and governance frameworks \cite{unesco2023journalism}. We specifically focus on the `intent' behind disinformation, which remains relevant over time but has broader effects beyond just being factually incorrect.
\begin{figure*}
    \centering
    \includegraphics[width=0.95\textwidth]{figures/qualitative_figure.pdf}
    \vspace{-1em}
\caption{Disinformation Trends Across News Categories: We analyze the likelihood of disinformation across different categories, based on disinformation narratives and confidence levels generated by GPT-4o.}
    \vspace{-1em}
    \label{fig:disinfo-analysis}
\end{figure*}
\subsection{Data Pipeline}  
\textbf{Dataset Collection:}  
From May 6, 2023, to September 6, 2023, we aggregated data via Google RSS feeds from diverse news sources (Table~\ref{tab:sources}), adhering to Google’s terms of service \cite{google_tos}. We carefully curated  high-quality visual samples from these news sources to ensure a diverse representation of topics. All data collection complied with ethical guidelines \cite{uwaterloo_ethics_review}, regarding intellectual property and privacy protection. 

\textbf{Quality Assurance.}  
Collected articles underwent a rigorous human review and pre-processing phase. First, we removed entries with incomplete text, low-resolution or missing images, duplicates, and media-focused URLs (e.g., \texttt{/video}, \texttt{/gallery}). Articles with fewer than 20 sentences were discarded to ensure textual depth. For each article, the first image was selected to represent the visual context. We periodically reviewed the quality of the curated data to ensure the API returned valid and consistent results. These steps yielded over 31k curated text-image news articles that are moved to the annotation pipeline.

\subsection{Annotation Pipeline}  
To the best of our knowledge, \textsf{\textbf{\textsc{VLDBench}}} is the largest and most comprehensive humanly verified disinformation detection benchmark with over 300 hours of human verification. Figure \ref{fig:vlbias} shows our semi-annotated, data collection and annotation pipeline. After quality assurance, each article was prompted and categorized by GPT-4o as either \texttt{Likely} or \texttt{Unlikely} to contain disinformation, a binary choice designed to balance nuance with manageability. To ensure reliability, GPT-4o assessed text-image alignment three times per sample—first, to minimize random variance in its responses, and second, to resolve potential ties in classification, with an odd number of evaluations ensuring a definitive outcome. GPT-4o was chosen for this task due to its demonstrated effectiveness in both textual \cite{kim2024meganno+} and visual reasoning tasks \cite{shahriar2024putting}. An example is shown in Figure \ref{fig:disinfo-analysis}.

We categorized our data into 13 unique news categories (Figure \ref{fig:news_categories}) by providing image-text pairs to GPT-4o, drawing inspiration from AllSlides \cite{allsides_mediabiaschart}  and frameworks like Media Cloud \cite{media_cloud}. The dataset statistics are given in Table~\ref{tab:dataset_statistics}.

\begin{figure}[ht]
    \centering
    \includegraphics[width=0.48\textwidth]{figures/news_categories_distribution.pdf}
    \caption{Category distribution with overlaps. Total unique articles = 31,339. Percentages sum to \(>100\%\) due to multi-category articles.}
    \label{fig:news_categories}
    \vspace{-1em}
\end{figure}

To ensure high-quality benchmarking, a team of 22 domain experts (Appendix~\ref{app:team}) systematically reviewed the GPT-4o labels and rationales, assessing their accuracy, consistency, and alignment with human judgment. This process included a rigorous structured reconciliation phase, refining annotation guidelines and finalizing the labels. The evaluation resulted in a Cohen’s $\kappa$ of 0.78, indicating strong inter-annotator agreement.


\paragraph{Stability of Automatic Annotations:} To assess the reliability of automated annotations, we conducted a controlled experiment comparing GPT-4o labels with those of human annotators. We randomly selected 1,000 GPT-4o-annotated samples from the previous step, provided annotation guidelines, and asked domain experts (without showing the GPT-4o labels) to manually annotate them. Comparing both sets of labels, GPT-4o achieved an F1 score of 0.89 and an MCC of 0.77, while human annotators scored F1 = 0.92 and MCC = 0.81 (Figure~\ref{fig:alignment_metrics}). These results demonstrate the effectiveness of our semi-annotated pipeline, aligning well with human judgment and ensuring reliable automated labeling.

\begin{table*}[!t]
    \centering
    \resizebox{0.8\textwidth}{!}{
    \begin{tabular}{@{}l|c|c|c|c|c||c@{}}
        \toprule
        & \makecell{MATRES} & \makecell{TB-Dense} & \makecell{TCR} & \makecell{TDD-Manual} & \makecell{NarrativeTime} & \makecell{\textbf{\App{}}} \\
        \midrule
        \multicolumn{7}{c}{\textbf{Datasets Statistics}} \\
        \midrule
        Documents & 275 & 36 & 25 & 34 & 36 & 30 \\
        Events & 6,099 & 1,498 & 1,134 & 1,101 & 1,715 & 470 \\
        \midrule
        \textit{before} & 6,852 (50) & 1,361 (21) & 1,780 (67) & 1,561 (25) & 17,011 (22) & 1,540 (44) \\
        \textit{after} & 4,752 (35) & 1,182 (19) & 862 (33) & 1,054 (17) & 18,366 (23) & 1,347 (39) \\
        \textit{equal} & 448 (4) & 237 (4) & 4 (0) & 140 (2) & 5,298 (7) & 150 (4) \\
        \textit{vague} & 1,525 (11) & 2,837 (45) & -- & -- & 25,679 (33) & 446 (13) \\
        \textit{includes} & -- & 305 (5) & -- & 2,008 (33) & 5,781 (7) & -- \\
        \textit{is-included} & -- & 383 (6) & -- & 1,387 (23) & 6,639 (8) & -- \\
        \textit{overlaps} & -- & -- & -- & -- & 227 (0) & -- \\
        \midrule
        Total Relations & 13,577 & 6,305 & 2,646 & 6,150 & 79,001 & 3,483 \\
        \midrule
        \multicolumn{7}{c}{\textbf{Per Document Average Annotation Sparsity}} \\
        \midrule
        Events & 22.2 & 41.6 & 45.4 & 32.4 & 47.6 & 15.6 \\
        Actual Relations & 49.4 & 183.7 & 105.8 & 180.9 & 1,110.1 & 114.9 \\
        Expected Relations & 234.8 & 844.5 & 1,006.1 & 508.1 & 1,110.1 & 114.9 \\
        \midrule
        Missing Relations & 79\% & 78.3\% & 89.5\% & 64.4\% & 0\% & 0\% \\
        \bottomrule
    \end{tabular}}
    \caption{The upper part of the table presents the statistics of notable datasets for the temporal relation extraction task alongside \App{}. In parentheses, the values indicate the percentage of each relation type relative to the total relations in the dataset. The bottom part of the table summarizes the average percentage of missing relations per document, calculated as the ratio of actual annotated relations to a complete relation coverage, referred to as \textit{Expected Relations}.}
    \label{tab:stats_all}
\end{table*}


% \begin{table*}[!t]
%     \centering
%     \resizebox{0.8\textwidth}{!}{
%     \begin{tabular}{@{}l|c|c|c|c|c|c@{}}
%         \toprule
%         & \makecell{MATRES} & \makecell{TBD} & \makecell{TCR} & \makecell{TDD-Man} & \makecell{NarrativeTime} & \makecell{\App{}} \\
%         \midrule
%         Docs & 275 & 36 & 25 & 34 & 36 & 30 \\
%         Events & 6,099 & 1,498 & 1,134 & 1,101 & 1,715 & 470 \\
%         \midrule
%         Before (\%) & 6,852 (50) & 1,361 (21) & 1,780 (67) & 1,561 (25) & 17,011 (22) & 1,540 (44) \\
%         After (\%) & 4,752 (35) & 1,182 (19) & 862 (33) & 1,054 (17) & 18,366 (23) & 1,347 (39) \\
%         Equal (\%) & 448 (4) & 237 (4) & 4 (0) & 140 (2) & 5,298 (7) & 150 (4) \\
%         Vague (\%) & 1,525 (11) & 2,837 (45) & -- & -- & 25,679 (33) & 446 (13) \\
%         Includes (\%) & -- & 305 (5) & -- & 2,008 (33) & 5,781 (7) & -- \\
%         IsIncluded (\%) & -- & 383 (6) & -- & 1,387 (23) & 6,639 (8) & -- \\
%         Overlaps (\%) & -- & -- & -- & -- & 227 (0) & -- \\
%         \midrule
%         Total Rels & 13,577 & 6,305 & 2,646 & 6,150 & 79,001 & 3,483 \\
%         \bottomrule
%     \end{tabular}}
%     \caption{Statistics of notable datasets for the temporal relation extraction task.}
%     \label{tab:stats}
% \end{table*}



\section{Experiment}
We evaluate our proposed method with strong baselines and further analyze contributions of different components, and the impact of key parameters.

\subsection{Experiment Setup}
\textbf{Dataset.}
We evaluate all the methods on Inter-X dataset, which consists about 9K training samples and 1,708 test samples. Each sample is an action-reaction sequence and three corresponding textual description.
As supplementation, we mix our pre-training data with single person motion-text dataset HumanML3D~\citep{humanml3d}, which consists more than 23K annotated motion sequences.
We uniformly sample frames for both datasets to 30 FPS. 

\textbf{Evaluation Metrics.}
Following single-person motion generation~\citep{t2mgpt}, we adopt the these metrics to quantitatively evaluate the generated motion: R-Precision measures the ranking of Euclidean distances between motion and text features. Accuracy (Acc.) assesses how likely a generated motion could be successfully recognized as its interaction label, like ``high-five''. Frechet Inception Distance~\citep{fid} (FID) evaluates the similarity in feature space between predicted and ground-truth motion. Multimodal Distance (MMDist.) calculates the average Euclidean distance between generated motion and the corresponding text description. Diversity (Div.) measures the feature diversity within generated motions. All the metrics reported are calculated with batch size set to 32, and accumulated across the test dataset, and we evaluate each method for 20 times with different seeds to calculate the final results at 95\% confidence interval.

\textbf{Evaluation Model.} \label{sec:eval}
Every metric mentioned above requires an encoder $\mathcal{M}$ to extract motion feature.
For single person text-to-motion generation tasks, a motion-text matching model are commonly trained as human motion feature extractor.
A simple way to transfer this method to interaction domain is to directly train an interaction-to-text matching model $\mathcal{M}(\mathbf{a}, \hat{\mathbf{b}}, text)$, where action sequence $\mathbf{a}$ and predicted reaction sequence $\hat{\mathbf{b}}$ together is regarded as a generated interaction sequence, or a reaction-to-text match model $\mathcal{M}(\hat{\mathbf{b}}, text)$.
However, the former one may focus too much on the ground-truth action input, leading insufficient discriminative power of $\hat{\mathbf{b}}$'s quality, while the latter one lacks semantics provided by action, thus leading to subpar matching capability.

To address the issue, we simply uniformly mask off a large portion of $\mathbf{a}$, obtaining down-sampled action motion sequence $\mathbf{a}'$ (downsampled to 1 FPS in our setting), which serves as a semantic hint for the matching process while not introducing too much emphasis on input action sequence.
The final evaluation model consists of an masked interaction encoder and a text encoder.
We use contrastive loss following CLIP~\citep{clip}, which encourages paired motion and text features to be close geometrically.
In addition, we add a classification head after the predicted motion features, to simultaneously predict interaction labels, such as ``high-five''.

\textbf{Baselines.} To evaluate the performance of our method \ModelAbbr~on online and unconstrained setting, we compare \ModelAbbr~with the following baselines:
1) \textbf{InterFormer}~\citep{interformer} is a transformer based action-to-reaction generation model that leverages human skeleton as prior knowledge for efficient attention process.
2) \textbf{MotionGPT}~\citep{motiongpt} is a motion-language model that leverages an LLM for motion and text generation. We extend the motion tokenizer of MotionGPT to encode multi-person motion, while keeping other settings unchanged.
3) \textbf{InterGen}~\citep{intergen} proposes a mutual attention mechanism within diffusion process for human interaction generation, we reproduce and adapt IngerGen to action-to-reaction generation.
4) \textbf{ReGenNet}~\citep{regennet} is latest state-of-the-art model on action-to-reaction generation. It adopts a transformer decoder based diffusion model, which directly predicts human reaction given action input in unconstrained and online manner as ours.


\textbf{Implementation Details.}
For the LLM, we adopt Flan-T5-base~\citep{flan,t5} as our base model, with extended vocabulary. We warm up the learning rate for 1,000 steps, peaking at 1e-4 for the pre-training phase, and use the same learning rate for fine-tuning.
Both the pre-training and fine-tuning phases are trained on a single machine with 8 Tesla V100 GPUs. The training batch size is set to 32 for the LLM and we monitor the validation loss and reaction generation metrics for early-stopping, resulting about 100K pre-training steps and 40K fine-tuning steps.
We set the re-thinking interval $N_r$ to 4 tokens and divide each space signal into $N_b=10$ bins.

\begin{table}[t]
\centering
\tiny
\caption{Comparison to state-of-the-art baselines and ablation studies of our method on Inter-X dataset. $\uparrow$ or $\downarrow$ denotes a higher or lower value is better, and $\rightarrow$ means that the value closer to real is better. We use $\pm$ to represent 95\% confidence interval and highlight the best results in \textbf{bold}. For ablation methods (in grey), PT, M, P, S, and SP are abbreviations for pre-training, motion, pose, space, and single-person data, respectively.}
\label{tab:main}
\begin{tabular}{l|ccccccc}
\toprule
\multirow{2}{*}{Methods} &  \multicolumn{3}{c}{R-Precision$\uparrow$} & \multirow{2}{*}{Acc.$\uparrow$}& \multirow{2}{*}{FID$\downarrow$}         & \multirow{2}{*}{MMDist$\downarrow$} & \multirow{2}{*}{Div.$\rightarrow$} \\
               & Top-1       & Top-2       & Top-3     &     &         &                               &                       \\ \midrule
Real                    & $0.511^{\pm.003}$ & $0.682^{\pm.002}$ & $0.776^{\pm.002}$ & $0.463^{\pm.000}$  & $0.000^{\pm.000}$         & $5.348^{\pm.002}$         & $2.498^{\pm.005}$           \\ \midrule
InterFormer             & $0.172^{\pm.012}$ & $0.292^{\pm.013}$ & $0.343^{\pm.012}$ & $0.171^{\pm.009}$ & $10.468^{\pm.021}$        &  $7.831^{\pm.018}$         & $3.505^{\pm.023}$           \\
MotionGPT &            $0.238^{\pm.003}$        &     $0.354^{\pm.004}$        &     $0.441^{\pm.003}$   &    $0.186^{\pm.002}$            &     $5.823^{\pm.048}$               &      $6.211^{\pm.005}$           &      $2.615^{\pm.007}$ \\
InterGen                & $0.326^{\pm.036}$ & $0.423^{\pm.063}$ & $0.525^{\pm.053}$ & $0.254^{\pm.019}$  & $5.506^{\pm.257}$         & $6.182^{\pm.038}$         & $2.284^{\pm.009}$           \\
ReGenNet                & $0.384^{\pm.005}$ & $0.483^{\pm.002}$ & $0.572^{\pm.003}$ & $0.297^{\pm.004}$  & $3.988^{\pm.048}$         & $5.867^{\pm.009}$         & $\mathbf{2.502^{\pm.001}}$           \\ \midrule
% \rowcolor[HTML]{EFEFEF}
\ModelAbbr~(Ours)       & $\mathbf{0.423^{\pm.005}}$ & $\mathbf{0.599^{\pm.003}}$ & $\mathbf{0.693^{\pm.003}}$ & $\mathbf{0.318^{\pm.003}}$  & $\mathbf{1.942^{\pm.017}}$         & $\mathbf{5.643^{\pm.003}}$         & $2.629^{\pm.006}$           \\
\rowcolor[HTML]{EFEFEF}
w/o Think           & $0.367^{\pm.003}$ & $0.491^{\pm.027}$ & $0.584^{\pm.008}$ & $0.230^{\pm.036}$ & $3.828^{\pm.016}$         & $6.186^{\pm.055}$         & $2.609^{\pm.006}$           \\
\rowcolor[HTML]{EFEFEF}
w/o All PT.         & $0.398^{\pm.007}$ & $0.531^{\pm.002}$ & $0.628^{\pm.003}$ & $0.288^{\pm.002}$ & $3.467^{\pm.113}$         & $5.822^{\pm.003}$         & $2.909^{\pm.053}$           \\
\rowcolor[HTML]{EFEFEF}
w/o M-M PT. & $0.408^{\pm.005}$ & $0.563^{\pm.004}$ & $0.646^{\pm.005}$ & $0.293^{\pm.002}$ & $2.874^{\pm.020}$         & $5.736^{\pm.003}$         & $2.553^{\pm.006}$           \\
\rowcolor[HTML]{EFEFEF}
w/o P-S PT. & $0.417^{\pm.004}$ & $0.582^{\pm.004}$ & $0.664^{\pm.004}$ & $0.308^{\pm.003}$ & $2.685^{\pm.024}$         & $5.699^{\pm.004}$         & $2.859^{\pm.007}$           \\
\rowcolor[HTML]{EFEFEF}
w/o M-T PT. & $0.406^{\pm.003}$ & $0.557^{\pm.004}$ & $0.637^{\pm.004}$ & $0.304^{\pm.003}$ & $2.580^{\pm.021}$         & $5.822 ^{\pm.003}$         & $2.889^{\pm.005}$           \\
\rowcolor[HTML]{EFEFEF}
w/o SP Data     & $0.414^{\pm.004}$ & $0.592^{\pm.005}$ & $0.685^{\pm.003}$ & $0.315^{\pm.004}$ & $2.007^{\pm.015}$         & $5.667^{\pm.003}$         & $2.611^{\pm.005}$           \\
\bottomrule
\end{tabular}
\end{table}



\begin{figure}
    \centering
    \includegraphics[width=\linewidth]{figs/tsne.pdf}
    \caption{Visualization of a person's motion sequences in Inter-X dataset and HumanML3D dataset.}
    \label{fig:tsne}
\end{figure}

\subsection{Comparison to Baselines}\label{sec:sota}
As shown in the upper side of Table~\ref{tab:main}, our method \ModelAbbr~significantly outperforms baseline methods in terms of ranking, accuracy, FID and multimodal distance, showing superior human reaction generation quality.
Compared to MotionGPT, which adopts a similar motion-language architecture, \ModelAbbr~expresses stronger performance, which we attribute to our unified representation of motion via space and pose tokenizers, enabling effective individual pose and inter-person spatial relationship representation.
\ModelAbbr~also surpasses the diffusion-based methods, InterGen and ReGenNet, with our think-then-react architecture, improving generated motions by describing observed action and reasoning what reaction is expected on semantic level. In addition, ReGenNet and MotionGPT get closer diversity to the real than our model. We mainly attribute to that, \ModelAbbr~may conduct multiple re-thinking processes during inference, and the inferred semantics may bring a higher diversity.


\subsection{Ablation Study of Key Components}
To evaluate the effectiveness of our proposed key designs, we conduct detailed ablation studies by removing each of them to observe how much drop compared to the full version of our \ModelAbbr~method. The larger drop indicates more contribution. The results are shown in gray lines of Table~\ref{tab:main}. According to the drops in FID, all designs, including thinking, pre-training tasks and using single person data in pre-training, have positive contributions to the final performance, and thinking contributes the most. Some detailed findings and analyses are as follows.

First, we skip \textbf{thinking} stage during inference, and find the performance drops significantly in FID from 1.9 to 3.8. This supports the necessity of our proposed thinking process before reacting. We also notice decreasing diversity of generated samples, as the model relies solely on input action, and cannot explicitly capture and infer action's intent, thus leading to more rigid motion in some cases.

Second, to evaluate the effectiveness of \textbf{pre-training}, we omit the pre-training stage, and directly train our model \ModelAbbr~for thinking and reacting tasks. As shown in Table~\ref{tab:main}, our model's performance deteriorates without a fine-grained pre-training phase from 1.9 to 3.4 in FID. This indicates that pre-training can effectively adapt a language model (Flan-T5-base) into a motion and language model. We further removing three kinds of pre-training tasks: motion-motion (M-M PT.), pose-space (P-S PT.), and motion-text (M-T PT.). The results show that the without any task, the performance obviously gets worse, from 1.9 to 2.5 - 2.8 in FID, indicating their positive contribution to the final performance and complementary values to each other.

Third, to see how much \textbf{single-person data} helps reaction generation, we remove single person motion-text data, i.e., the data from HumanML3D dataset, from our training set. The result (w/o SP Data) shows that the model performs worse without training on HumanML3D, which proves that our unified motion encoder and motion-language architecture can leverage both single- and multi-person data, alleviating the insufficiency of training data. However, the benefit from single-person data is not as large as we expect. 

% What's more, we evaluate the necessity of \textbf{decoupled space-motion tokenizer}, and the results are shown in Table~\ref{tab:vqvae}. We design a plain motion VQ-VAE with unnormalized action and reaction as input, maintaining absolute space and pose features. With the trained motion VQ-VAE, we encode action/reaction into tokens, which are then fed into TTR for reaction prediction task. First, without normalized motion as input, the reconstruction FID significantly rises from 0.262 to 0.983, showing deteriorated reconstruction performance due to insufficient utilization of codebook. Second, in the reaction generation phase, TTR's performance drops dramatically, as the badly constructed codebook leads to inaccurate action understanding and reaction prediction, highlighting the necessity of decoupling token representation of space and pose features in multi-person scenario.

\begin{figure}
    \centering
    \includegraphics[width=\linewidth]{figs/case_study.pdf}
    \caption{Visualized cases of our predicted reactions (in green) to input action (in blue) and corresponding thinking results. We also provide a failure case in figure (d), where TTR misunderstands the input action as ``wrestling'', which should be ``embracing''.}
    \label{fig:case_study}
\end{figure}


\subsection{Analysis on Overlapping between Single- and Multi-Person Motions}
To investigate the reason of small contribution from single-person data, we further visualize motion sequences of single-person motion (HumanML3D), two-person action (Inter-X Action) and reaction (Inter-X Reaction) in the same space, as presented in Figure~\ref{fig:tsne}. Specifically, we use t-SNE tool~\cite{tsne} to project motion token sequence features into two-dimension. As shown in Figure~\ref{fig:tsne}, the single- and two-person motion sequences have little overlap. When doing case studies, we find that most two-person motion are unique, e.g., massage and being pulled, and will never be used in single-person motion. Similarly, most single-person motions are unique too, e.g., T-pose, and seldom appear in multi-person interaction. There are only a few overlapped motions, e.g., standing still. In addition, when comparing action and reaction sequences in multi-person interaction, we have some interesting findings. When reactions are close to actions, the motion usually belongs to symmetrical interactions, e.g., pulling or being pulled; whereas, when actions are far from reactions, the motion usually belongs to asymmetrical interaction, e.g., massage.


\subsection{Impact of Down-Sampling Parameter in Matching Model for Evaluation}

As described in Section~\ref{sec:eval}, we propose downsampling action motion sequence to avoid matching models for evaluation pay too much attention to input action rather than output reaction. We conduct an experiment to change the downsampling parameter frame rate and calculate the difference between taking ground-truth action and random action as the input of $\mathcal{M}$, in terms of summed ranking scores (Top-1, Top-2, Top-3 and Acc.). As presented in Figure~\ref{fig:discriminative}, 
difference is lowest when FPS equals to 0, which meaning we only match generated reaction motion with text. It goes up to the peak when FPS equals 1 and quickly goes down to low values, even close to the lowest when FPS is about 15. This indicates that it is necessary to concatenate input action with generated reaction to compose a meaningful interaction in evaluation, otherwise the motion-text matching model cannot well recognize the interaction. However, only 1 FPS is enough. With larger FPS, the matching models will be disturbed by input action rather than the generated reaction. Thus, we choose 1 FPS, corresponding to the largest difference, as our final setting.

\subsection{Impact of Re-thinking Interval}
% Our aim is to generate real-time reaction online, and thus time interval is an important parameter to generation quality. 
We change the re-thinking interval $N_r$ from about 1 to 100 timesteps (about 0.1 to 10 seconds) and observe how it impacts generative quality measure FID. As shown in Figure~\ref{fig:latency}, FID falls down first until $N_r=4$ (about 0.5 second) and then continues rising up. This indicate that the best time interval is about 0.5 second. When the time interval is too short, our \ModelAbbr~model cannot get enough information to re-think what the input action means and will bring some randomness into predicting appropriate reaction. When the time interval gets too long, our \ModelAbbr~model give slow responses to the input action sequences and generates coarse-grained reaction.

We also evaluate the average inference time per step (AITS) with respect to the re-thinking interval. As shown in Figure~\ref{fig:latency}, the inference time significantly decreases as the re-thinking interval increases, eventually converging to approximately 10 milliseconds per step (100 FPS). In our setup, we opt to re-think every four steps, resulting in an inference time of less than 50 milliseconds, which meets the requirements for a real-time system.

\subsection{User Study}
To further evaluate our model qualitatively, we conduct a user study on TTR vs. the latest SOTA method ReGenNet, and the results are shown in Figure~\ref{fig:user_study}. We randomly sample 100 action sequences from Inter-X dataset, which are fed into TTR and ReGenNet to predict reactions, and ask four real human to choose the better ones. It can be seen that TTR surpasses ReGenNet on all the duration range, and the winning rate rises significantly when motion duration is longer. We mainly contribute this to our explicit thinking and re-thinking procedure, which ensures semantics matching and alleviates accumulated errors. 

\begin{figure}[t]
    \centering
    \begin{minipage}[b]{0.3\textwidth}
        \centering
        \includegraphics[width=\textwidth]{figs/discriminative_power.pdf}
        \caption{Impact of input action FPS to summed ranking score differences.}
        \label{fig:discriminative}
    \end{minipage}
    \hfill
    \begin{minipage}[b]{0.35\textwidth}
        \centering
        \includegraphics[width=\textwidth]{figs/latency.pdf}
        \caption{Impact of re-thinking interval to FID and average inference time per step (AITS).}
        \label{fig:latency}
    \end{minipage}
    \hfill
    \begin{minipage}[b]{0.3\textwidth}
        \centering
        \includegraphics[width=\linewidth]{figs/user_study.pdf}
        \caption{User preference between TTR and ReGenNet on different motion duration.}
        \label{fig:user_study}
    \end{minipage}
\end{figure}

\section{Conclusion}

We justify that a flow matching generative model can produce dense and reliable rewards for training LLMs to explain the decisions of RL agents and other LLMs. 
Looking into the future, we envision extending this method to a general LLM training approach, automatically generating high-quality dense rewards, and ultimately reducing the reliance on human feedback. 

% Our method has the potential to facilitate human-AI collaboration applications, such as transportation, education, and security defense.

\newpage
\section{Impact Statements}
This paper presents work whose goal is to advance the field of machine learning by developing a model-agnostic explanation generator for intelligent agents, enhancing transparency and interpretability in agent decision prediction. The ability to generate effective and interpretable explanations has the potential to foster trust in AI systems, improving effectiveness in high-stakes applications such as healthcare, finance, and autonomous systems. Overall, we believe our work contributes positively to the broader AI ecosystem by promoting more explainable and trustworthy AI.

\textbf{Acknowledgment} This work is supported by the National Natural Science Foundation of China (No. 62276268) and Kuaishou Technology.

\bibliography{iclr2025_conference}
\bibliographystyle{iclr2025_conference}

\newpage

\appendix
\section{Supplementary Experiment Results}

\subsection{Gener Tasks}
The confusion matrices for the gene classification task are shown in \textit{Fig.} \ref{fig:gene_classification}, whereas \textit{Fig.} \ref{fig:species_classification} illustrates those for the taxonomic classification task. For both tasks, the \textbf{Gener}\textit{ator} not only excels in average performance but also maintains top performance across different categories of genes and taxonomic groups.

\subsection{Next K-mer Prediction}
The evaluations of next K-mer prediction are provided in \textit{Fig.} \ref{fig:kmer_tokenizer} and \textit{Fig.} \ref{fig:kmer_model}. Specifically, \textit{Fig.} \ref{fig:kmer_tokenizer} presents the evaluations for tokenizer comparison, where the 6-mer tokenizer outperforms others across different taxonomic groups. \textit{Fig.} \ref{fig:kmer_model} presents the evaluations for model comparison, where the \textbf{Gener}\textit{ator} consistently outperforms others across various taxonomic groups.

\section{Details of Pre-training}
In the pre-training phase, we employed the AdamW~\cite{adamw} optimizer with $\beta_1 = 0.9$, $\beta_2 = 0.95$, and $\text{weight decay} = 0.1$. The learning rate schedule incorporated a linear warm-up followed by cosine decay: the learning rate increased linearly from 0 to its peak value over the first 2,000 steps, then followed a cosine decay, decreasing until it reached 10\% of the peak value by the end of the training process. The peak learning rate was established at $4e^{-4}$, and gradient clipping was applied with a norm threshold of $1.0$. We adhered to standard practices in pre-training LLMs, using a batch size that encompasses 2 million tokens. Given the maximum sequence length of 16,384 tokens, this configuration resulted in each batch comprising 128 samples, where each sample was initialized with a random starting point between 0 and 5 at each epoch. The complete pre-training of the \textbf{Gener}\textit{ator} spanned 6 epochs, amounting to approximately 185,000 steps. To enhance computational efficiency, we employed optimization techniques like Flash Attention~\cite{flashattention,flashattention2} and the Zero Redundancy Optimizer~\cite{deepspeed,fsdp}. The pre-training process, utilizing 32 NVIDIA A100 GPUs, was completed in 368 hours. This configuration achieved a model FLOPs utilization (MFU) of approximately 46\%.

The summary statistics of the pre-training data in terms of the number of nucleotides and number of genes are provided in Table \ref{tab:pretrain_data_statistics_species} and Table \ref{tab:pretrain_data_statistics_gene}. The pre-training losses for the \textbf{Gener}\textit{ator} and \textbf{Gener}\textit{ator}-All are presented in \textit{Fig.} \ref{fig:pretrain_loss}. Although the \textbf{Gener}\textit{ator}-All exhibits a lower pre-training loss compared to the \textbf{Gener}\textit{ator}, it consistently underperforms in almost all benchmark tests, as discussed in previous sections. This discrepancy is likely due to the inclusion of non-gene regions, which often contain highly repetitive and simple segments. Additionally, the pre-training losses for different tokenizers and the Mamba model are shown in \textit{Fig.} \ref{fig:pretrain_loss_tokenizer}. The significant variation in vocabulary sizes among the tokenizers results in considerable differences in pre-training losses, making direct comparisons challenging. This issue inspires the proposition of next K-mer prediction for a more effective model comparison.

\section{Details of Benchmark Experiments}

\subsection{Benchmarks}
In this paper, we conducted benchmark evaluations on two widely used benchmarks, Nucleotide Transformer tasks~\cite{nucleotide-transformer} and Genomic Benchmarks~\cite{genomic-benchmarks}, as well as on the Gener tasks we propose. For Nucleotide Transformer tasks, there are two different versions available: the original version can be accessed \href{https://huggingface.co/datasets/InstaDeepAI/nucleotide_transformer_downstream_tasks}{\textcolor{blue}{here}}, and the revised version is available \href{https://huggingface.co/datasets/InstaDeepAI/nucleotide_transformer_downstream_tasks_revised}{\textcolor{blue}{here}}. For Genomic Benchmarks, the dataset can be downloaded from \href{https://huggingface.co/katarinagresova}{\textcolor{blue}{here}}.

\subsection{Metrics}
Regarding evaluation metrics, we follow the original settings of the benchmarks. For Nucleotide Transformer tasks, we employ the Matthews correlation coefficient (MCC):
\begin{equation}
\text{MCC} = \frac{\text{TP} \times \text{TN} - \text{FP} \times \text{FN}}{\sqrt{(\text{TP} + \text{FP})(\text{TP} + \text{FN})(\text{TN} + \text{FP})(\text{TN} + \text{FN})}}, \nonumber
\end{equation} 
where TP, TN, FP, FN represent true positives, true negatives, false positives, and false negatives, respectively. For Genomic Benchmarks, accuracy is used as the evaluation metric. For Gener tasks, we utilize the weighted F1 score for multi-classification, calculated as follows:
\begin{equation}
\text{Precision}_i = \frac{\text{TP}_i}{\text{TP}_i + \text{FP}_i},
~~~
\text{Recall}_i = \frac{\text{TP}_i}{\text{TP}_i + \text{FN}_i}, \nonumber
\end{equation}

\begin{equation}
\text{F1}_i = 2 \times \frac{\text{Precision}_i \times \text{Recall}_i}{\text{Precision}_i + \text{Recall}_i},
~~~
\text{F1}_{\text{weighted}} = \sum_{i=1}^{n} w_i \times \text{F1}_i, \nonumber
\end{equation}
where $w_i = {n_i}/{N}$ denotes the weight for class $i$, $n_i$ is the number of samples in class $i$, and $N$ is the total number of samples.

\subsection{Baselines}
To comprehensively assess the capability of the \textbf{Gener}\textit{ator} in sequence comprehension, we include several state-of-the-art models as baselines. These models differ in terms of training data, model size, model architecture, and other aspects. Table~\ref{tab:baseline_introductuon} provides a summary of these baselines. Where applicable, evaluation results for the original and revised NT tasks are sourced directly from the NT paper~\cite{nucleotide-transformer}. For the remaining baseline models and datasets, we conduct consistent and fair evaluation tasks ourselves to obtain the results. These self-evaluated baseline models can be accessed through the following Huggingface\footnote{\url{https://huggingface.co}} repositories:
\begin{itemize}
    \item DNABERT-2: \texttt{zhihan1996/DNABERT-2-117M}
    \item HyenaDNA: \texttt{LongSafari/hyenadna-large-1m-seqlen}
    % \item NT-multi: \texttt{InstaDeepAI/nucleotide-transformer-2.5b-multi-species}
    \item NT-v2: \texttt{InstaDeepAI/nucleotide-transformer-v2-500m-multi-species}
    \item Caduceus-Ph: \texttt{kuleshov-group/caduceus-ph\_seqlen-131k\_d\_model-256\_n\_layer-16}
    \item Caduceus-PS: \texttt{kuleshov-group/caduceus-ps\_seqlen-131k\_d\_model-256\_n\_layer-16}
    \item GROVER: \texttt{PoetschLab/GROVER}
\end{itemize}

\subsection{Experimental Setups}
In our benchmark experiments, we retained the optimizer configuration from the pre-training phase, with $\beta_1=0.9$, $\beta_2=0.95$ and $\text{weight decay} = 0.1$. We adopted the `reduce on plateau' strategy for the learning rate scheduler and implemented early stopping based on validation metrics, with a patience of 5. Optimal learning rates and batch sizes for various models and datasets were determined through hyperparameter search, as detailed in Tables~\ref{tab:benchmark_hyperparam1} to \ref{tab:benchmark_hyperparam4}. For Nucleotide Transformer tasks and Genomic Benchmarks, all models had sufficient context length, allowing them to utilize the entire input sequences. In contrast, for Gener tasks, input sequences exceeding the available context length were truncated to fit within the model constraints. For all causal language models, we performed prediction through an additional linear layer using the embedding of the \texttt{<EOS>} token, while for masked language models, we used the \texttt{<BOS>} token or \texttt{<CLS>} token instead. Notably, for Caduceus, we followed its original configuration using mean pooling and applied the recommended reverse complement data augmentation for Caduceus-Ph. All models obtained embeddings from their final layer and underwent full fine-tuning. All evaluation metrics were obtained through 10-fold cross-validation.

\section{Details of Central Dogma}
\label{sec:central_dogma_supp}
We sourced the protein sequences and the corresponding protein-coding DNA fragments for the Histone and Cytochrome P450 families from the UniProt~\cite{UniProt} database. Each family had approximately 400,000 entries. The \textbf{Gener}\textit{ator} model was then fine-tuned on these datasets using the protein-coding sequences, with a learning rate of $5e^{-5}$ and a batch size of 1,024. For the fine-tuned models, we generated 100,000 sequences for each combination of temperature $T \in \{0.5, 0.6, 0.7, 0.8, 0.9, 1.0\}$ and nucleus sampling $P \in \{0.5, 0.6, 0.7, 0.8, 0.9, 1.0\}$. Following translation into protein sequences, duplicates were removed by comparing the generated sequences with the training set, resulting in a unique selection of 10,000 samples per family. The hyperparameters were set to $T=0.6$, $P=0.6$ for the Cytochrome P450 family (\textit{Fig.}~\ref{fig:cytochrome_generation}), and $T=1.0$, $P=0.8$ for the Histone family (\textit{Fig.}~\ref{fig:histone_generation}). 

\section{Details of Enhancer Design}
For the enhancer activity prediction, we adhered to the data partitioning of DeepSTARR~\cite{DeepSTARR}. We conducted the same hyperparameter search strategy as in our benchmark experiments and selected a learning rate of $2e^{-5}$ and a batch size of 64 for training the activity predictor. For enhancer generation, we selected enhancer sequences from the top and bottom quartiles of activity values within the DeepSTARR training set. These sequences were used to perform supervised fine-tuning (SFT) on the \textbf{Gener}\textit{ator}, labeled with prompts \texttt{<high>} and \texttt{<low>} for high and low activity sequences, respectively. The hyperparameters for SFT were set to a learning rate of $1e^{-5}$ and a batch size of 2,048. During the generation process, we applied the same hyperparameter search strategy detailed in \textit{Sec.}~\ref{sec:central_dogma_supp}. The hyperparameters used for visualization in \textit{Fig.}~\ref{fig:enhancer_design} were set to $T=0.5$, $P=0.7$ for the developmental activity, and $T=0.8$, $P=0.6$ for the housekeeping activity.

\section{Computational Resources}
In this study, the primary computational resources utilized for model training and inference included NVIDIA V100 and A100 GPUs. Prior to pre-training, we trained 14 models for one epoch each to evaluate the DNA sequence modeling capabilities of various tokenizers. Our pre-training phase involved two models that varied based on the training data used. In the downstream tasks, we executed over 10,000 runs, which entailed an extensive hyperparameter search across 36 combinations and utilized 10-fold cross-validation for each benchmark task per model. Detailed statistics on the usage of main computational resources are presented in Table~\ref{tab:computational_resources}.

\section{Supplementary Figures \& Tables}

\begin{figure}[ht]
    \centering
    \includegraphics[width=0.8\textwidth]{figures/pdf/gene_classification_full.pdf}
    \caption{Evaluation of model performance on gene classification.}
    \label{fig:gene_classification}
\end{figure}
\begin{figure}[ht]
    \centering
    \includegraphics[width=0.8\textwidth]{figures/pdf/species_classification_full.pdf}
    \caption{Evaluation of model performance on taxonomic classification.}
    \label{fig:species_classification}
\end{figure}
\begin{figure}[ht]
    \centering
    \includegraphics[width=0.8\textwidth]{figures/pdf/kmer_tokenizer_full.pdf}
    \caption{Evaluation of next K-mer prediction for tokenizer comparison.}
    \label{fig:kmer_tokenizer}
\end{figure}
\begin{figure}[ht]
    \centering
    \includegraphics[width=0.8\textwidth]{figures/pdf/kmer_model_full.pdf}
    \caption{Evaluation of next K-mer prediction for model comparison.}
    \label{fig:kmer_model}
\end{figure}
% \begin{figure}[!htb]
    \centering
    \includegraphics[width=0.6\textwidth]{figures/pdf/histone_generation.pdf}
    \caption{Histone generation. (A) Distribution densities of the protein sequence lengths for generated and natural samples. (B) Distribution densities of Progen2 PPL for generated and natural samples, along with randomly shuffled sequences. (C) Scatter plot of TM-score and AlphaFold3 prediction confidence (pLDDT) with marginal distributions. (D) Two folded structures of generated samples displaying structural congruence with natural samples.}
    \label{fig:histone_generation}
\end{figure}
\begin{figure}[ht]
    \centering
    \includegraphics[width=0.5\textwidth]{figures/pdf/pretrain_loss.pdf}
    \caption{Pre-training losses of the \textbf{Gener}\textit{ator} and \textbf{Gener}\textit{ator}-All.}
    \label{fig:pretrain_loss}
\end{figure}
\begin{figure}[ht]
    \centering
    \includegraphics[width=0.6\textwidth]{figures/pdf/pretrain_loss_tokenizer.pdf}
    \caption{Pre-training losses of different tokenizers.}
    \label{fig:pretrain_loss_tokenizer}
\end{figure}

% \begin{table*}[!htb]
\small
\renewcommand{\arraystretch}{1.2}
\centering
\caption{Evaluation of the original Nucleotide Transformer tasks. The reported values represent the Matthews correlation coefficient (MCC) averaged over 10-fold cross-validation, with the standard error in parentheses.}
\resizebox{\textwidth}{!}{%
\begin{tabular}{lcccccccccc}
\toprule
& Enformer & DNABERT-2 & HyenaDNA & NT-multi & NT-v2 & Caduceus-Ph & Caduceus-PS & GROVER & \textbf{Gener}\textit{ator} & \textbf{Gener}\textit{ator}-All \\
& (252M) & (117M) & (55M) & (2.5B) & (500M) & (8M) & (8M) & (87M) & (1.2B) & (1.2B) \\
\midrule
H3 & 0.724 (0.018) & 0.785 (0.012) & 0.781 (0.015) & 0.793 (0.013) & 0.788 (0.010) & 0.794 (0.012) & 0.772 (0.022) & 0.768 (0.008) & \textbf{0.806 (0.005)} & \underline{0.803 (0.007)} \\
H3K14ac & 0.284 (0.024) & 0.515 (0.009) & \textbf{0.608 (0.020)} & 0.538 (0.009) & 0.538 (0.015) & 0.564 (0.033) & 0.596 (0.038) & 0.548 (0.020) & \underline{0.605 (0.008)} & 0.580 (0.038) \\
H3K36me3 & 0.345 (0.019) & 0.591 (0.005) & 0.614 (0.014) & 0.618 (0.011) & 0.618 (0.015) & 0.590 (0.018) & 0.611 (0.048) & 0.563 (0.017) & \textbf{0.657 (0.007)} & \underline{0.631 (0.013)} \\
H3K4me1 & 0.291 (0.016) & 0.512 (0.008) & 0.512 (0.008) & 0.541 (0.005) & 0.544 (0.009) & 0.468 (0.015) & 0.487 (0.029) & 0.461 (0.018) & \textbf{0.553 (0.009)} & \underline{0.549 (0.018)} \\
H3K4me2 & 0.207 (0.021) & 0.333 (0.013) & \textbf{0.455 (0.028)} & 0.324 (0.014) & 0.302 (0.020) & 0.332 (0.034) & \underline{0.431 (0.016)} & 0.403 (0.042) & 0.424 (0.013) & 0.400 (0.015) \\
H3K4me3 & 0.156 (0.022) & 0.353 (0.021) & \textbf{0.550 (0.015)} & 0.408 (0.011) & 0.437 (0.028) & 0.490 (0.042) & \underline{0.528 (0.033)} & 0.458 (0.022) & 0.512 (0.009) & 0.473 (0.047) \\
H3K79me3 & 0.498 (0.013) & 0.615 (0.010) & 0.669 (0.014) & 0.623 (0.010) & 0.621 (0.012) & 0.641 (0.028) & \textbf{0.682 (0.018)} & 0.626 (0.026) & \underline{0.670 (0.011)} & 0.631 (0.021) \\
H3K9ac & 0.415 (0.020) & 0.545 (0.009) & 0.586 (0.021) & 0.547 (0.011) & 0.567 (0.020) & 0.575 (0.024) & 0.564 (0.018) & 0.581 (0.015) & \textbf{0.612 (0.006)} & \underline{0.603 (0.019)} \\
H4 & 0.735 (0.023) & 0.797 (0.008) & 0.763 (0.012) & \underline{0.808 (0.007)} & 0.795 (0.008) & 0.788 (0.010) & 0.799 (0.010) & 0.769 (0.017) & \textbf{0.815 (0.008)} & \underline{0.808 (0.010)} \\
H4ac & 0.275 (0.022) & 0.465 (0.013) & 0.564 (0.011) & 0.492 (0.014) & 0.502 (0.025) & 0.548 (0.027) & \underline{0.585 (0.018)} & 0.530 (0.017) & \textbf{0.592 (0.015)} & 0.565 (0.035) \\
Enhancer & 0.454 (0.029) & 0.525 (0.026) & 0.520 (0.031) & 0.545 (0.028) & \underline{0.561 (0.029)} & 0.522 (0.024) & 0.511 (0.026) & 0.516 (0.018) & \textbf{0.580 (0.015)} & 0.540 (0.026) \\
Enhancer type & 0.312 (0.043) & 0.423 (0.018) & 0.403 (0.056) & 0.444 (0.022) & 0.444 (0.036) & 0.403 (0.028) & 0.410 (0.026) & 0.433 (0.029) & \textbf{0.477 (0.017)} & \underline{0.463 (0.023)} \\
Promoter all & 0.910 (0.004) & 0.945 (0.003) & 0.919 (0.003) & 0.951 (0.004) & 0.952 (0.002) & 0.937 (0.002) & 0.941 (0.003) & 0.926 (0.004) & \textbf{0.962 (0.002)} & \underline{0.955 (0.002)} \\
Promoter non-TATA & 0.910 (0.006) & 0.944 (0.003) & 0.919 (0.004) & \underline{0.955 (0.003)} & 0.952 (0.003) & 0.935 (0.007) & 0.940 (0.002) & 0.925 (0.006) & \textbf{0.962 (0.001)} & \underline{0.955 (0.002)} \\
Promoter TATA & 0.920 (0.012) & 0.911 (0.011) & 0.881 (0.020) & 0.919 (0.008) & \underline{0.933 (0.009)} & 0.895 (0.010) & 0.903 (0.010) & 0.891 (0.009) & \textbf{0.948 (0.008)} & 0.931 (0.007) \\
Splice acceptor & 0.772 (0.007) & 0.909 (0.004) & 0.935 (0.005) & \underline{0.973 (0.002)} & \underline{0.973 (0.004)} & 0.918 (0.017) & 0.907 (0.015) & 0.912 (0.010) & \textbf{0.981 (0.002)} & 0.957 (0.009) \\
Splice site all & 0.831 (0.012) & 0.950 (0.003) & 0.917 (0.006) & 0.974 (0.004) & \underline{0.975 (0.002)} & 0.935 (0.011) & 0.953 (0.005) & 0.919 (0.005) & \textbf{0.978 (0.001)} & 0.973 (0.002) \\
Splice donor & 0.813 (0.015) & 0.927 (0.003) & 0.894 (0.013) & 0.974 (0.002) & \underline{0.977 (0.007)} & 0.912 (0.009) & 0.930 (0.010) & 0.888 (0.012) & \textbf{0.978 (0.002)} & 0.967 (0.005) \\
\bottomrule
\end{tabular}
}
\label{tab:nucleotide_transformer_tasks}
\end{table*}
% \begin{table*}[!htb]
\small
\renewcommand{\arraystretch}{1.2}
\centering
\caption{Evaluation of the Genomic Benchmarks. The reported values represent the accuracy averaged over 10-fold cross-validation, with the standard error in parentheses.}
\resizebox{\textwidth}{!}{
\begin{tabular}{lcccccccc}
\toprule
& DNABERT-2 & HyenaDNA & NT-v2 & Caduceus-Ph & Caduceus-PS & GROVER & \textbf{Gener}\textit{ator} & \textbf{Gener}\textit{ator}-All \\
& (117M) & (55M) & (500M) & (8M) & (8M) & (87M) & (1.2B) & (1.2B) \\
\midrule
Coding vs. Intergenomic & 0.951 (0.002) & 0.902 (0.004) & 0.955 (0.001) & 0.933 (0.001) & 0.944 (0.002) & 0.919 (0.002) & \textbf{0.963 (0.000)} & \underline{0.959 (0.001)} \\
Drosophila Enhancers Stark & 0.774 (0.011) & 0.770 (0.016) & 0.797 (0.009) & \textbf{0.827 (0.010)} & 0.816 (0.015) & 0.761 (0.011) & \underline{0.821 (0.005)} & 0.768 (0.015) \\
Human Enhancers Cohn & \underline{0.758 (0.005)} & 0.725 (0.009) & 0.756 (0.006) & 0.747 (0.003) & 0.749 (0.003) & 0.738 (0.003) & \textbf{0.763 (0.002)} & 0.754 (0.006) \\
Human Enhancers Ensembl & 0.918 (0.003) & 0.901 (0.003) & 0.921 (0.004) & \textbf{0.924 (0.002)} & \underline{0.923 (0.002)} & 0.911 (0.004) & 0.917 (0.002) & 0.912 (0.002) \\
Human Ensembl Regulatory & 0.874 (0.007) & 0.932 (0.001) & \textbf{0.941 (0.001)} & \underline{0.938 (0.004)} & \textbf{0.941 (0.002)} & 0.897 (0.001) & 0.928 (0.001) & 0.926 (0.001) \\
Human non-TATA Promoters & 0.957 (0.008) & 0.894 (0.023) & 0.932 (0.006) & \textbf{0.961 (0.003)} & \textbf{0.961 (0.002)} & 0.950 (0.005) & \underline{0.958 (0.001)} & 0.955 (0.005) \\
Human OCR Ensembl & 0.806 (0.003) & 0.774 (0.004) & 0.813 (0.001) & \underline{0.825 (0.004)} & \textbf{0.826 (0.003)} & 0.789 (0.002) & 0.823 (0.002) & 0.812 (0.003) \\
Human vs. Worm & 0.977 (0.001) & 0.958 (0.004) & 0.976 (0.001) & 0.975 (0.001) & 0.976 (0.001) & 0.966 (0.001) & \textbf{0.980 (0.000)} & \underline{0.978 (0.001)} \\
Mouse Enhancers Ensembl & \underline{0.865 (0.014)} & 0.756 (0.030) & 0.855 (0.018) & 0.788 (0.028) & 0.826 (0.021) & 0.742 (0.025) & \textbf{0.871 (0.015)} & 0.784 (0.027) \\
\bottomrule
\end{tabular}
}
\label{tab:genomic_benchmarks}
\end{table*}
\begin{table*}[!htb]
\small
\renewcommand{\arraystretch}{1.2}
\centering
\caption{Pre-training data statistics (taxonomic group).}
\begin{tabular}{lccc}
\toprule
Taxonomic group & Number of genes & Number of nucleotides (bp) \\
\midrule
Protozoa & 1,107,499 & 2,034,179,213 \\
Fungi & 6,299,990 & 10,593,031,563 \\
Plant & 7,279,591 & 30,548,451,231 \\
Invertebrate & 7,602,349 & 71,979,748,208 \\
Vertebrate (other) & 10,915,710 & 161,686,774,317 \\
Mammalian & 6,837,553 & 109,406,723,311 \\
\bottomrule
\end{tabular}
\label{tab:pretrain_data_statistics_species}
\end{table*}

\begin{table*}[!htb]
\small
\renewcommand{\arraystretch}{1.2}
\centering
\caption{Pre-training data statistics (gene type).}
\begin{tabular}{lccc}
\toprule
Gene type & Number of genes & Number of nucleotides (bp) \\
\midrule
Protein Coding & 30,883,451 & 340,466,000,000 \\
miscRNA & 342,761 & 6,448,121,770 \\
ncRNA & 4,151,851 & 28,538,620,565 \\
pseudo & 2,322,278 & 10,307,467,170 \\
rRNA & 654,807 & 360,329,916 \\
tRNA & 1,687,523 & 128,397,507 \\
tmRNA & 21 & 9,793 \\
\bottomrule
\end{tabular}
\label{tab:pretrain_data_statistics_gene}
\end{table*}
\begin{table*}[!htb]
\small
\renewcommand{\arraystretch}{1.2}
\centering
\caption{Summary of baseline models.}
\resizebox{\textwidth}{!}{
\begin{tabular}{lccccccc}
\toprule
Model Name & Pre-train Task & Architecture & Tokenizer & Pre-train Data Scope & Pre-train Data Volume (bp) & Context Length (bp) & Parameter Size \\
\midrule
Enformer & Supervised & Transformer & Single Nucleotide & Human, Mouse & 64B & 200K & 252M \\
DNABERT-2 & MLM & Transformer & BPE & Multispecies & 32.5B & 3K & 117M \\
HyenaDNA & NTP & SSM & Single Nucleotide & Human & 3B & 1M & 55M \\
NT-multi & MLM & Transformer & 6-mer & Multispecies & 174B & 6K & 2.5B \\
NT-v2 & MLM & Transformer & 6-mer & Multispecies & 174B & 12K & 500M \\
Caduceus & MLM & SSM & Single Nucleotide & Human & 35B & 131K & 8M \\
GROVER & MLM & Transformer & BPE & Human & 3B & 3K & 87M \\
\textbf{Gener}\textit{ator} & NTP & Transformer & 6-mer & Multispecies (Gene Only) & 386B & 98K & 1.2B \\
\textbf{Gener}\textit{ator}-All & NTP & Transformer & 6-mer & Multispecies & 1.9T & 98K & 1.2B \\
\bottomrule
\end{tabular}
}
\label{tab:baseline_introductuon}
\end{table*}
\begin{table*}[!htb]
\centering
\small
\caption{Hyperparameter settings for the original Nucleotide Transformer tasks.}
\begin{tabular}{@{}l*{20}{c}@{}}
\toprule
 & 
\multicolumn{2}{c}{\shortstack{NT-v2}} & 
\multicolumn{2}{c}{\shortstack{Caduceus-Ph}} & 
\multicolumn{2}{c}{\shortstack{Caduceus-PS}} & 
\multicolumn{2}{c}{\shortstack{GROVER}} & 
\multicolumn{2}{c}{\shortstack{\textbf{Gener}\textit{ator}}} & 
\multicolumn{2}{c}{\shortstack{\textbf{Gener}\textit{ator}-All}} \\ 
\cmidrule(l){2-3} 
\cmidrule(l){4-5} 
\cmidrule(l){6-7} 
\cmidrule(l){8-9} 
\cmidrule(l){10-11} 
\cmidrule(l){12-13}
 & LR & BS & LR & BS & LR & BS & LR & BS & LR & BS & LR & BS \\ 
\midrule
H3 & $2e^{-5}$ & 256 & $5e^{-4}$ & 64 & $1e^{-3}$ & 128 & $1e^{-4}$ & 256 & $1e^{-4}$ & 64 & $2e^{-5}$ & 64 \\
H3K14ac & $1e^{-5}$ & 64 & $2e^{-4}$ & 64 & $5e^{-4}$ & 64 & $5e^{-4}$ & 512 & $2e^{-4}$ & 512 & $5e^{-5}$ & 128 \\
H3K36me3 & $2e^{-5}$ & 64 & $5e^{-4}$ & 128 & $1e^{-3}$ & 256 & $5e^{-5}$ & 64 & $2e^{-4}$ & 512 & $1e^{-5}$ & 64 \\
H3K4me1 & $5e^{-5}$ & 64 & $2e^{-4}$ & 64 & $1e^{-3}$ & 512 & $5e^{-5}$ & 128 & $1e^{-4}$ & 128 & $2e^{-5}$ & 256 \\
H3K4me2 & $5e^{-5}$ & 256 & $2e^{-4}$ & 128 & $5e^{-4}$ & 512 & $1e^{-4}$ & 256 & $5e^{-5}$ & 256 & $1e^{-4}$ & 512 \\
H3K4me3 & $5e^{-5}$ & 128 & $5e^{-4}$ & 256 & $5e^{-4}$ & 256 & $1e^{-4}$ & 128 & $5e^{-5}$ & 64 & $5e^{-5}$ & 64 \\
H3K79me3 & $5e^{-5}$ & 128 & $2e^{-4}$ & 64 & $5e^{-4}$ & 256 & $1e^{-4}$ & 512 & $2e^{-4}$ & 128 & $5e^{-5}$ & 128 \\
H3K9ac & $5e^{-5}$ & 256 & $5e^{-4}$ & 256 & $5e^{-4}$ & 64 & $1e^{-4}$ & 64 & $1e^{-4}$ & 256 & $1e^{-4}$ & 256 \\
H4 & $2e^{-5}$ & 128 & $1e^{-4}$ & 128 & $5e^{-4}$ & 128 & $5e^{-5}$ & 512 & $5e^{-5}$ & 512 & $2e^{-5}$ & 128 \\
H4ac & $5e^{-5}$ & 64 & $5e^{-4}$ & 256 & $2e^{-4}$ & 64 & $5e^{-5}$ & 64 & $1e^{-4}$ & 128 & $5e^{-5}$ & 64 \\
Enhancers & $2e^{-5}$ & 128 & $2e^{-4}$ & 128 & $5e^{-4}$ & 512 & $5e^{-5}$ & 512 & $1e^{-4}$ & 512 & $2e^{-5}$ & 128 \\
Enhancers types & $1e^{-4}$ & 128 & $2e^{-4}$ & 256 & $2e^{-4}$ & 64 & $1e^{-4}$ & 512 & $2e^{-4}$ & 512 & $1e^{-4}$ & 128 \\
Promoter all & $1e^{-5}$ & 64 & $1e^{-4}$ & 64 & $2e^{-4}$ & 64 & $1e^{-4}$ & 64 & $1e^{-5}$ & 64 & $1e^{-5}$ & 64 \\
Promoter non-TATA & $5e^{-5}$ & 256 & $1e^{-3}$ & 512 & $5e^{-4}$ & 512 & $1e^{-4}$ & 256 & $2e^{-5}$ & 64 & $5e^{-5}$ & 256 \\
Promoter TATA & $5e^{-5}$ & 128 & $2e^{-3}$ & 512 & $1e^{-3}$ & 256 & $2e^{-4}$ & 512 & $5e^{-5}$ & 512 & $5e^{-5}$ & 128 \\
Splice sites acceptors & $2e^{-5}$ & 64 & $1e^{-3}$ & 64 & $2e^{-3}$ & 512 & $1e^{-4}$ & 64 & $2e^{-5}$ & 128 & $5e^{-5}$ & 64 \\
Splice sites all & $5e^{-5}$ & 128 & $2e^{-3}$ & 256 & $1e^{-3}$ & 64 & $1e^{-4}$ & 64 & $2e^{-5}$ & 256 & $5e^{-5}$ & 128 \\
Splice sites donors & $5e^{-5}$ & 128 & $1e^{-3}$ & 64 & $1e^{-3}$ & 64 & $5e^{-5}$ & 64 & $1e^{-4}$ & 64 & $5e^{-5}$ & 128 \\ 
\bottomrule
\end{tabular}
\label{tab:benchmark_hyperparam1}
\end{table*}

\begin{table*}[!htb]
\centering
\small
\caption{Hyperparameter settings for the revised Nucleotide Transformer tasks.}
\begin{tabular}{@{}l*{12}{c}@{}}
\toprule
 & 
\multicolumn{2}{c}{\shortstack{NT-v2}} & 
\multicolumn{2}{c}{\shortstack{Caduceus-Ph}} & 
\multicolumn{2}{c}{\shortstack{Caduceus-PS}} & 
\multicolumn{2}{c}{\shortstack{GROVER}} & 
\multicolumn{2}{c}{\shortstack{\textbf{Gener}\textit{ator}}} & 
\multicolumn{2}{c}{\shortstack{\textbf{Gener}\textit{ator}-All}} \\ 
\cmidrule(l){2-3} 
\cmidrule(l){4-5} 
\cmidrule(l){6-7} 
\cmidrule(l){8-9} 
\cmidrule(l){10-11} 
\cmidrule(l){12-13}
 & LR & BS & LR & BS & LR & BS & LR & BS & LR & BS & LR & BS \\ 
\midrule
H2AFZ & $2e^{-5}$ & 64 & $1e^{-5}$ & 64 & $1e^{-3}$ & 256 & $1e^{-5}$ & 128 & $2e^{-5}$ & 128 & $1e^{-4}$ & 128 \\
H3K27ac & $2e^{-5}$ & 128 & $5e^{-4}$ & 64 & $5e^{-4}$ & 64 & $2e^{-5}$ & 128 & $2e^{-5}$ & 256 & $2e^{-5}$ & 128 \\
H3K27me3 & $5e^{-5}$ & 256 & $1e^{-3}$ & 256 & $1e^{-3}$ & 128 & $1e^{-5}$ & 64 & $5e^{-5}$ & 256 & $5e^{-5}$ & 256 \\
H3K36me3 & $5e^{-5}$ & 128 & $2e^{-4}$ & 64 & $1e^{-3}$ & 256 & $5e^{-5}$ & 256 & $1e^{-5}$ & 256 & $5e^{-5}$ & 128 \\
H3K4me1 & $5e^{-5}$ & 512 & $5e^{-4}$ & 256 & $1e^{-3}$ & 256 & $5e^{-5}$ & 128 & $5e^{-5}$ & 64 & $1e^{-5}$ & 64 \\
H3K4me2 & $2e^{-5}$ & 256 & $1e^{-3}$ & 256 & $1e^{-3}$ & 256 & $2e^{-5}$ & 128 & $2e^{-5}$ & 64 & $5e^{-5}$ & 512 \\
H3K4me3 & $2e^{-5}$ & 64 & $1e^{-3}$ & 128 & $5e^{-4}$ & 64 & $1e^{-5}$ & 128 & $2e^{-5}$ & 64 & $5e^{-5}$ & 64 \\
H3K9ac & $5e^{-5}$ & 512 & $5e^{-4}$ & 128 & $1e^{-4}$ & 64 & $1e^{-5}$ & 64 & $2e^{-5}$ & 128 & $5e^{-5}$ & 256 \\
H3K9me3 & $1e^{-4}$ & 128 & $2e^{-4}$ & 128 & $1e^{-3}$ & 512 & $1e^{-5}$ & 64 & $1e^{-4}$ & 64 & $1e^{-4}$ & 128 \\
H4K20me1 & $2e^{-5}$ & 128 & $1e^{-3}$ & 256 & $2e^{-4}$ & 64 & $2e^{-5}$ & 128 & $5e^{-5}$ & 256 & $1e^{-5}$ & 64 \\
Enhancer & $2e^{-5}$ & 256 & $2e^{-4}$ & 128 & $5e^{-4}$ & 512 & $5e^{-5}$ & 256 & $5e^{-5}$ & 64 & $2e^{-5}$ & 64 \\
Enhancer types & $5e^{-5}$ & 512 & $1e^{-3}$ & 64 & $2e^{-4}$ & 128 & $2e^{-5}$ & 128 & $2e^{-5}$ & 64 & $5e^{-5}$ & 64 \\
Promoter all & $1e^{-4}$ & 64 & $5e^{-4}$ & 128 & $5e^{-4}$ & 128 & $1e^{-5}$ & 128 & $1e^{-4}$ & 128 & $5e^{-5}$ & 512 \\
Promoter non-TATA & $2e^{-5}$ & 128 & $5e^{-5}$ & 64 & $5e^{-5}$ & 128 & $2e^{-5}$ & 256 & $2e^{-5}$ & 64 & $1e^{-4}$ & 128 \\
Promoter TATA & $2e^{-4}$ & 128 & $5e^{-4}$ & 64 & $1e^{-3}$ & 128 & $5e^{-5}$ & 64 & $5e^{-5}$ & 128 & $1e^{-4}$ & 64 \\
Splice acceptor & $5e^{-5}$ & 256 & $1e^{-3}$ & 128 & $5e^{-3}$ & 64 & $1e^{-5}$ & 256 & $5e^{-5}$ & 128 & $5e^{-5}$ & 64 \\
Splice site all & $5e^{-5}$ & 256 & $1e^{-3}$ & 64 & $2e^{-3}$ & 64 & $1e^{-5}$ & 64 & $1e^{-4}$ & 256 & $1e^{-4}$ & 128 \\
Splice site donors & $2e^{-5}$ & 128 & $2e^{-3}$ & 64 & $5e^{-3}$ & 64 & $2e^{-5}$ & 128 & $5e^{-5}$ & 128 & $2e^{-5}$ & 64 \\
\bottomrule
\end{tabular}
\label{tab:benchmark_hyperparam2}
\end{table*}


\begin{table*}[!htb]
\centering
\small
\caption{Hyperparameter settings for the Genomic Benchmarks.}
\resizebox{\textwidth}{!}{%
\begin{tabular}{@{}l*{16}{c}@{}}
\toprule
 & 
\multicolumn{2}{c}{\shortstack{DNABERT-2}} & 
\multicolumn{2}{c}{\shortstack{HyenaDNA}} & 
\multicolumn{2}{c}{\shortstack{NT-v2}} & 
\multicolumn{2}{c}{\shortstack{Caduceus-Ph}} & 
\multicolumn{2}{c}{\shortstack{Caduceus-PS}} & 
\multicolumn{2}{c}{\shortstack{GROVER}} & 
\multicolumn{2}{c}{\shortstack{\textbf{Gener}\textit{ator}}} & 
\multicolumn{2}{c}{\shortstack{\textbf{Gener}\textit{ator}-All}} \\ 
\cmidrule(l){2-3} 
\cmidrule(l){4-5} 
\cmidrule(l){6-7} 
\cmidrule(l){8-9} 
\cmidrule(l){10-11} 
\cmidrule(l){12-13} 
\cmidrule(l){14-15} 
\cmidrule(l){16-17}
 & LR & BS & LR & BS & LR & BS & LR & BS & LR & BS & LR & BS & LR & BS & LR & BS \\ 
\midrule
Coding vs. Intergenomic & $1e^{-5}$ & 64 & $1e^{-4}$ & 64 & $5e^{-5}$ & 512 & $2e^{-4}$ & 128 & $2e^{-4}$ & 128 & $5e^{-5}$ & 256 & $2e^{-5}$ & 256 & $1e^{-5}$ & 128 \\
Drosophila Enhancers Stark & $5e^{-5}$ & 512 & $1e^{-3}$ & 512 & $5e^{-5}$ & 64 & $5e^{-4}$ & 64 & $1e^{-3}$ & 64 & $1e^{-4}$ & 64 & $2e^{-4}$ & 256 & $1e^{-5}$ & 128 \\
Human Enhancers Cohn & $5e^{-5}$ & 128 & $5e^{-5}$ & 128 & $5e^{-5}$ & 256 & $2e^{-4}$ & 64 & $2e^{-4}$ & 256 & $2e^{-5}$ & 256 & $1e^{-5}$ & 128 & $2e^{-5}$ & 64 \\
Human Enhancers Ensembl & $1e^{-4}$ & 256 & $5e^{-4}$ & 512 & $5e^{-5}$ & 128 & $5e^{-4}$ & 256 & $2e^{-4}$ & 64 & $5e^{-5}$ & 512 & $5e^{-5}$ & 128 & $1e^{-4}$ & 512 \\
Human Ensembl Regulatory & $5e^{-5}$ & 128 & $1e^{-4}$ & 128 & $2e^{-5}$ & 128 & $5e^{-4}$ & 64 & $5e^{-4}$ & 64 & $1e^{-5}$ & 128 & $1e^{-5}$ & 128 & $2e^{-5}$ & 512 \\
Human NonTATA Promoters & $2e^{-4}$ & 256 & $1e^{-4}$ & 128 & $5e^{-5}$ & 64 & $5e^{-4}$ & 64 & $5e^{-4}$ & 128 & $5e^{-5}$ & 64 & $5e^{-5}$ & 128 & $1e^{-4}$ & 64 \\
Human OCR Ensembl & $1e^{-4}$ & 256 & $5e^{-5}$ & 256 & $5e^{-5}$ & 128 & $1e^{-3}$ & 256 & $5e^{-4}$ & 128 & $5e^{-5}$ & 64 & $1e^{-5}$ & 64 & $5e^{-5}$ & 128 \\
Human vs. Worm & $2e^{-5}$ & 64 & $1e^{-4}$ & 128 & $2e^{-5}$ & 64 & $2e^{-4}$ & 64 & $2e^{-4}$ & 128 & $2e^{-5}$ & 128 & $2e^{-5}$ & 512 & $2e^{-5}$ & 64 \\
Mouse Enhancers Ensembl & $1e^{-5}$ & 64 & $2e^{-4}$ & 128 & $5e^{-5}$ & 64 & $1e^{-3}$ & 64 & $2e^{-4}$ & 128 & $1e^{-5}$ & 128 & $5e^{-4}$ & 512 & $5e^{-5}$ & 64 \\
\bottomrule
\end{tabular}
}
\label{tab:benchmark_hyperparam3}
\end{table*}


\begin{table*}[!htb]
\centering
\small
\caption{Hyperparameter settings for the Gener tasks.}
\resizebox{\textwidth}{!}{%
\begin{tabular}{@{}l*{16}{c}@{}}
\toprule
 & 
\multicolumn{2}{c}{\shortstack{DNABERT-2}} & 
\multicolumn{2}{c}{\shortstack{HyenaDNA}} & 
\multicolumn{2}{c}{\shortstack{NT-v2}} & 
\multicolumn{2}{c}{\shortstack{Caduceus-Ph}} & 
\multicolumn{2}{c}{\shortstack{Caduceus-PS}} & 
\multicolumn{2}{c}{\shortstack{GROVER}} & 
\multicolumn{2}{c}{\shortstack{\textbf{Gener}\textit{ator}}} & 
\multicolumn{2}{c}{\shortstack{\textbf{Gener}\textit{ator}-All}} \\ 
\cmidrule(l){2-3} 
\cmidrule(l){4-5} 
\cmidrule(l){6-7} 
\cmidrule(l){8-9} 
\cmidrule(l){10-11} 
\cmidrule(l){12-13} 
\cmidrule(l){14-15} 
\cmidrule(l){16-17}
 & LR & BS & LR & BS & LR & BS & LR & BS & LR & BS & LR & BS & LR & BS & LR & BS\\ 
\midrule
Gene & $5e^{-5}$ & 64 & $2e^{-4}$ & 128 & $5e^{-5}$ & 128 & $5e^{-4}$ & 128 & $2e^{-4}$ & 64 & $5e^{-5}$ & 128 & $1e^{-5}$ & 64 & $1e^{-4}$ & 64 \\
Taxonomic & $5e^{-5}$ & 64 & $2e^{-4}$ & 64 & $1e^{-4}$ & 512 & $2e^{-4}$ & 128 & $5e^{-4}$ & 256 & $1e^{-4}$ & 128 & $1e^{-5}$ & 128 & $5e^{-6}$ & 64 \\
\bottomrule
\end{tabular}
}
\label{tab:benchmark_hyperparam4}
\end{table*}
\begin{table}[!htb]
\small
\renewcommand{\arraystretch}{1.2}
\centering
\caption{Computational resource usage ({\color{gray}gray color denotes that the model was trained for less than 1 epoch}).}
\begin{tabular}{llrc}
\hline
 & Model & Computational Resources & Data-parallel Size \\ \hline
\multirow{15}{*}{Development} & BPE-512 & 11,551 V100 hours & 32 \\
& BPE-1024 & 10,562 V100 hours & 32 \\
& BPE-2048 & 8,431 V100 hours & 32 \\
& BPE-4096 & 11,155 V100 hours & 64 \\
& BPE-8192 & 8,118 V100 hours & 32 \\
& 2-mer & 28,380 V100 hours & 64 \\
& 3-mer & 15,270 V100 hours & 32 \\
& 4-mer & 11,635 V100 hours & 32 \\
& 5-mer & 9,202 V100 hours & 32 \\
& 6-mer & 9,494 V100 hours & 64 \\
& 7-mer & 8,133 V100 hours & 64 \\
& 8-mer & 7,883 V100 hours & 64 \\
& {\color{gray}Single Nucleotide (llama)} & 1,350 A100 hours & 8 \\
& {\color{gray}Single Nucleotide (mamba)} & 1,465 A100 hours & 8 \\
\hline
\multirow{2}{*}{Pre-train}
& \textbf{Gener}\textit{ator} & 11,793 A100 hours & 32 \\
& \textbf{Gener}\textit{ator}-All & 9,494 A100 hours & 32 \\ \hline
Downstream & 46 runs per task per model & $>$ 64,000 V100 hours & 8 \\ \hline
\end{tabular}
\label{tab:computational_resources}
\end{table}


\end{document}
