
\documentclass{article} % For LaTeX2e
\usepackage{iclr2025_conference,times}

% Optional math commands from https://github.com/goodfeli/dlbook_notation.
%%%%% NEW MATH DEFINITIONS %%%%%

% \usepackage{amsmath,amsfonts,bm}
\usepackage{amsmath,amsfonts}

\usepackage{pifont}


\newcommand{\R}{\mathbb{R}}


\def\va{{\mathbf{a}}}
\def\vg{{\mathbf{g}}}

% Sets
\def\sR{\mathbb{R}}
\def\sC{\mathbb{C}}
\def\sZ{\mathbb{Z}}
\def\sN{\mathbb{N}}
\def\sQ{\mathbb{Q}}

\def\sS{\mathcal{S}}



% Vectors
\def\vzero{{\mathbf{0}}}
\def\vone{{\mathbf{1}}}
\def\vmu{{\mathbf{\mu}}}
\def\vtheta{{\mathbf{\theta}}}
\def\va{{\mathbf{a}}}
\def\vb{{\mathbf{b}}}
\def\vc{{\mathbf{c}}}
\def\vd{{\mathbf{d}}}
\def\ve{{\mathbf{e}}}
\def\vf{{\mathbf{f}}}
\def\vg{{\mathbf{g}}}
\def\vh{{\mathbf{h}}}
\def\vi{{\mathbf{i}}}
\def\vj{{\mathbf{j}}}
\def\vk{{\mathbf{k}}}
\def\vl{{\mathbf{l}}}
\def\vm{{\mathbf{m}}}
\def\vn{{\mathbf{n}}}
\def\vo{{\mathbf{o}}}
\def\vp{{\mathbf{p}}}
\def\vq{{\mathbf{q}}}
\def\vr{{\mathbf{r}}}
\def\vs{{\mathbf{s}}}
\def\vt{{\mathbf{t}}}
\def\vu{{\mathbf{u}}}
\def\vv{{\mathbf{v}}}
\def\vw{{\mathbf{w}}}
\def\vx{{\mathbf{x}}}
\def\vy{{\mathbf{y}}}
\def\vz{{\mathbf{z}}}
\def\vzeta{{\mathbf{\zeta}}}

% Matrix
\def\mA{{\mathbf{A}}}
\def\mB{{\mathbf{B}}}
\def\mC{{\mathbf{C}}}
\def\mD{{\mathbf{D}}}
\def\mE{{\mathbf{E}}}
\def\mF{{\mathbf{F}}}
\def\mG{{\mathbf{G}}}
\def\mH{{\mathbf{H}}}
\def\mI{{\mathbf{I}}}
\def\mJ{{\mathbf{J}}}
\def\mK{{\mathbf{K}}}
\def\mL{{\mathbf{L}}}
\def\mM{{\mathbf{M}}}
\def\mN{{\mathbf{N}}}
\def\mO{{\mathbf{O}}}
\def\mP{{\mathbf{P}}}
\def\mQ{{\mathbf{Q}}}
\def\mR{{\mathbf{R}}}
\def\mS{{\mathbf{S}}}
\def\mT{{\mathbf{T}}}
\def\mU{{\mathbf{U}}}
\def\mV{{\mathbf{V}}}
\def\mW{{\mathbf{W}}}
\def\mX{{\mathbf{X}}}
\def\mY{{\mathbf{Y}}}
\def\mZ{{\mathbf{Z}}}
\def\mBeta{{\mathbf{\beta}}}
\def\mPhi{{\mathbf{\Phi}}}
\def\mLambda{{\mathbf{\Lambda}}}
\def\mSigma{{\mathbf{\Sigma}}}


% Expectation
% \def\eE{\mathop{\mathbb{E}}\limits}
\def\eE{\mathbb{E}}

% Probability
\def\pP{\mathbb{P}}

% Tilde
\def\tf{\tilde{f}}
\def\tS{\tilde{S}}
\def\wtF{\widetilde{\mathcal{F}}}
\def\whR{\widehat{R}}
\def\tvx{\tilde{\mathbf{x}}}
\def\ty{\tilde{y}}


\def\defeq{\overset{\textup{def}}{=}}
% \def\defeq{\overset{.}{=}}
\def\defone{\overset{\text{\ding{172}}}{=}}
\def\deftwo{\overset{\text{\ding{173}}}{=}}
\def\leqone{\overset{\text{\ding{172}}}{\leq}}
\def\leqtwo{\overset{\text{\ding{173}}}{\leq}}
\def\leqthree{\overset{\text{\ding{174}}}{\leq}}
\def\leqfour{\overset{\text{\ding{175}}}{\leq}}
\def\eqone{\overset{\text{\ding{172}}}{=}}
\def\eqtwo{\overset{\text{\ding{173}}}{=}}
\def\eqthree{\overset{\text{\ding{174}}}{=}}
\def\eqfour{\overset{\text{\ding{175}}}{=}}
\def\geqfive{\overset{\text{\ding{176}}}{\geq}}

\usepackage{hyperref}
\usepackage{url}
\usepackage{graphicx}
\usepackage{amsmath}
\usepackage{booktabs}
\usepackage{multirow}
\usepackage{subcaption}
\usepackage{caption}
\usepackage{algorithm}
\usepackage{algpseudocode}
\usepackage{wrapfig}
\usepackage{amssymb}% http://ctan.org/pkg/amssymb
\usepackage{pifont}% http://ctan.org/pkg/pifont
\newcommand{\cmark}{\ding{51}}%
\newcommand{\xmark}{\ding{55}}%
\usepackage{colortbl}

\title{Think-Then-React: Towards Unconstrained \\Human Action-to-Reaction Generation}


\author{Wenhui Tan, Boyuan Li, Chuhao Jin, Wenbing Huang, Xiting Wang \& Ruihua Song \\
Gaoling School of Artificial Intelligence\\
Renmin University of China\\
Beijing, China\\
\texttt{\{tanwenhui404,liboyuan,jinchuhao,hwenbing,xitingwang,rsong\}@ruc.edu.cn}
}


\newcommand{\fix}{\marginpar{FIX}}
\newcommand{\new}{\marginpar{NEW}}


% macros:
\newcommand{\ModelName}{Think-Then-React}
\newcommand{\ModelAbbr}{TTR}

\iclrfinalcopy % Uncomment for camera-ready version, but NOT for submission.
\begin{document}


\maketitle


\begin{figure}[h]
    \centering
    \includegraphics[width=\textwidth]{figs/teaser.pdf}
    \caption{Given a human action as input, our Think-Then-React model first \textbf{thinks} by generating an action description and reasons out a reaction prompt. It then \textbf{reacts} to the action based on the results of this thinking process. TTR reacts in a real-time manner at every timestep and periodically re-thinks at specific interval (every two timesteps in the illustration) to mitigate accumulated errors.}
    \label{fig:teaser}
\end{figure}


\begin{abstract}
Modeling human-like action-to-reaction generation has significant real-world applications, like human-robot interaction and games.
Despite recent advancements in single-person motion generation, it is still challenging to well handle action-to-reaction generation, due to the difficulty of directly predicting reaction from action sequence without prompts, and the absence of a unified representation that effectively encodes multi-person motion.
To address these challenges, we introduce Think-Then-React (TTR), a large language-model-based framework designed to generate human-like reactions.
First, with our fine-grained multimodal training strategy, TTR is capable to unify two processes during inference: a \textbf{thinking} process that explicitly infers action intentions and reasons corresponding reaction description, which serve as semantic prompts, and a \textbf{reacting} process that predicts reactions based on input action and the inferred semantic prompts.
Second, to effectively represent multi-person motion in language models, we propose a unified motion tokenizer by decoupling egocentric pose and absolute space features, which effectively represents action and reaction motion with same encoding.
Extensive experiments demonstrate that TTR outperforms existing baselines, achieving significant improvements in evaluation metrics, such as reducing FID from 3.988 to 1.942.
\end{abstract}

\section{Introduction}

Chain-of-Thought (CoT) prompting~\cite{Nye:2021, cot, Kojima:2022cotzero} has emerged as a cornerstone strategy for enhancing Large Language Models (LLMs) in complex reasoning tasks. By eliciting step-by-step inference, CoT enables LLMs to decompose intricate problems into manageable subtasks, thereby improving their problem-solving performance~\cite{Yao:2023tot, Wang:2023self-consistency, Zhou:2023least, Shinn:2023Reflexion}. Recent advancements, such as OpenAI's o1~\cite{o1} and DeepSeek-R1~\cite{deepseekr1}, further demonstrate that scaling up CoT lengths from hundreds to thousands of reasoning steps could continuously improve LLM reasoning. These breakthroughs have underscored CoT’s potential to advance LLM capabilities, expanding the boundaries of AI-driven problem-solving.

\begin{figure}[t]
\centering
    \includegraphics[width=0.95\columnwidth]{fig/intro.pdf}
    \caption{In contrast to vanilla CoT that generates all reasoning tokens sequentially, \method enables LLMs to \textit{skip} tokens with less semantic importance (\textit{e.g.,} \includegraphics[width=7pt]{fig/token.pdf}~) and learn shortcuts between critical reasoning tokens, facilitating controllable CoT compression.}
    \label{fig:intro}
\end{figure}

Despite its effectiveness, the increased length of CoT sequences introduces substantial computational overhead. Due to the autoregressive nature of LLM decoding, longer CoT outputs lead to proportional increases in both inference latency and memory footprints of key-value cache. Additionally, the quadratic computational cost of attention layers further exacerbates this burden. These issues become particularly pronounced when CoT sequences extend into thousands of reasoning steps, resulting in significant computational costs and prolonged response times. While prior research has explored methods for selectively skipping reasoning steps~\cite{Ding:2024cotshortcut, liu2024skipstep}, recent findings~\cite{jin:2024cotlength, Merrill:2024cotlength} suggest that such reductions may conflict with test-time scaling~\cite{o1-blog, snell2025scaling}, ultimately impairing LLM reasoning performance. Therefore, striking an optimal balance between CoT efficiency and reasoning accuracy remains a critical open challenge.

In this work, we delve into CoT efficiency and seek the answer to an important question: \textit{``Does every token in the CoT output contribute equally to deriving the answer?''} We empirically analyze the semantic importance of tokens within CoT outputs and reveal that their contributions to the reasoning performance vary, as depicted in Figure 2. Building on this insight, we introduce \method, a simple yet effective approach that enables LLMs to \textit{skip} less important tokens within CoT sequences and learn shortcuts between critical reasoning tokens, thereby allowing for controllable CoT compression with adjustable ratios. Specifically, as shown in Figure~\ref{fig:intro}, \method constructs compressed CoT training data with various compression ratios, by pruning unimportance tokens from original LLM CoT trajectories. Then, it conducts a general supervised fine-tuning process on target LLMs with this training data, facilitating LLMs to automatically trim redundant tokens during reasoning.

We conduct extensive experiments across various models, including LLaMA-3.1-8B-Instruct and the Qwen2.5-Instruct series, using two widely recognized math reasoning benchmarks: GSM8K and MATH-500. The results validate the effectiveness of \method in compressing CoT outputs while maintaining robust reasoning performance. Notably, Qwen2.5-14B-Instruct exhibits almost \textbf{NO} performance drop (less than $0.4\%$) with a $\bm{40\%}$ reduction in token usage on GSM8K. On the challenging MATH-500 dataset, LLaMA-3.1-8B-Instruct effectively reduces CoT token usage by $\bm{30}\%$ with a performance decline of less than $4\%$, resulting in a $\bm{1.4}\times$ inference speedup. Further analysis underscores the coherence of \method in specified compression ratios and its potential scalability with stronger compression techniques.

\method is distinguished by its low training cost. For Qwen2.5-14B-Instruct, \method fine-tunes only 0.2\% of the model's parameters using LoRA. The size of the compressed CoT training data is no larger than that of the original training set, with 7,473 examples in GSM8K and 7,500 in MATH. The training is completed in approximately 2 hours for the 7B model and 2.5 hours for the 14B model on two 3090 GPUs. These characteristics make \method an efficient and reproducible approach, suitable for use in efficient and cost-effective LLM deployment.

To sum up, our key contributions are:
\begin{enumerate}
    \item To the best of our knowledge, this work is the \textit{first} to investigate the potential of enhancing CoT efficiency through \textit{token skipping}, inspired by the varying semantic importance of tokens in CoT trajectories of LLMs.
    \item We introduce \method, a simple yet effective approach that enables LLMs to skip redundant tokens within CoTs and learn shortcuts between critical tokens, facilitating CoT compression with adjustable ratios.
    \item Our experiments validate the effectiveness of \method. When applied to Qwen2.5-14B-Instruct, \method reduces reasoning tokens by $40\%$ (from 313 to 181) on GSM8K, with less than a $0.4\%$ performance drop.
\end{enumerate}

\section{Related Work}
Researchers have been leveraging eye tracking methodologies from human perception research to model how people perceive images~\cite{shanmuga2015eye, bonhage2015combined, conklin2016using}.
These models help assess the appearance and salience of visual representations, enabling eye movement tracking to understand the perceptual and cognitive mechanisms of scene perception~\cite{itti1998model} and object detection~\cite{borji2015salient}.
The existing saliency models perform well in naturalistic scenes
%and real-world object detection
; however, there are unique perception rules and cognitive biases in the artificial world of data visualization 
%does not always follow the rules of perception in the natural world
~\cite{franconeri2021science, correll2012comparing, polatsek2018exploring, knittel2024gridlines}, and, thus, these models do not accurately predict where people would look in visualizations. 
Visualization researchers have been building visual saliency models geared to visualizations~\cite{DVSaliencyModel2017Matzen, bylinskii2016should}. %and adopting them for predicting eye gaze on visualizations. %enabling the prediction of visual saliency across design styles~\cite{fosco2020predicting}.
However, these models rely on handcrafted features, making it difficult to generalize to complex visualizations. Additionally, these models cannot incorporate textual information to generate task-specific saliency maps since the prediction is solely based on visual inputs.

With the advent of deep learning, gaze data were used as the ground truth of saliency models~\cite{fosco2020predicting, scannerDeeply, scanpath}, leading to higher performance in saliency prediction while enabling task-specific saliency~\cite{salchartQA}. 
These models usually need large-scale datasets to learn complex patterns. However, gathering precise gaze data is 
%challenging and requires specialized eye-tracking devices. While these devices provide accurate results, they tend to be 
costly and cumbersome, which limits large-scale data collection efforts. 
Many researchers, therefore, proposed several proxies for eye gaze. WebGaze~\cite{webgaze} uses a webcam for cheap and easy deployment in online studies yet suffers from data quality issues due to low-resolution cameras and uncontrolled calibration.
Therefore, mouse-(cursor-)based annotation tools~\cite{jiang2015salicon,bubbleView,importAnnot} were proposed to improve data quality. Among these methods, BubbleView~\cite{bubbleView} was the most used tool for capturing visual saliency and importance~\cite{graphicDesignImportance, salchartQA}.
However, BubbleView is primarily designed for exploring images and gathering information, which differs slightly from the goal of capturing perceived importance. As a result, while BubbleView is well-suited for measuring visual saliency, it may not be the best tool for capturing %instruction-tuned \yao{I would keep it consistent saying task-specific}
task-specific importance~\cite{turkeyes}. Built upon these prior approaches' limitations, our Grid Labeling aims to collect responses that cover all essential areas of the visualization with minimum noise, leading to more efficient data collection.



% One key motivation to our grid-based approach is to help people 
% We also demonstrate that the grid-based approaches can minimize biases in annotation to disproportionally emphasize text elements~\cite{DVSaliencyModel2017Matzen}
% % blurring the visualization can disproportionately emphasize text elements~\cite{DVSaliencyModel2017Matzen}, potentially misrepresenting a user's true areas of interest.
% More recently, 
% % \ms{Changed a bit using Yao's work (task-dependent saliency), but not sure whether it looks ok}
% Yao et al.~\cite{salchartQA} collect task-dependent saliency using the BubbleView method, % but their approach had some limitations. 
% and made a significant improvement on existing saliency models.
% First, the blurred visualization allowed users to perceive the overall structure of the chart, which prevented the system from capturing the specific action of identifying the maximum value. However, increasing the blur to address this issue introduced another challenge. As the structure became less visible, users had to explore the entire image, leading to the consideration of irrelevant regions as salient.
\section{\name: Modeling Task-driven Eye Movement on Charts}
\label{sec:model}

This section introduces the problem formulation and presents the computational model of eye movement control on charts in settings of analytical tasks.

\subsection{Problem Formulation}

Given a chart image $C$ and an associated analytical task $x$ stated as text, the model is expected to generate a sequence of fixation positions $\{ p_1, p_2, \dots , p_t\}$.
The objective of the output sequence is to closely match the scanpath from humans reading the chart. 
Specifically, the sequence of fixations represents the visual reasoning process, and the information in the patches of pixels fixated upon should be able to support $x$.
We consider general analytical tasks in information visualization~\cite{amar2005low}, and
select three of them used in a human eye-tracking data collection~\cite{polatsek2018exploring}: % \textit{RV}, \textit{F}, and \textit{FE} tasks
\rv{
\begin{itemize}
    \item[1)] \textit{Retrieve value (\textit{RV})}: Given a specific target, find the data value of the target (e.g., what is the value for a certain category?)
    \item[2)] \textit{Filter (\textit{F})}: Given a concrete condition, find which data point satisfies it (e.g., which category has the specific value stated?)
    \item[3)] \textit{Find extreme (\textit{FE})}: Find the data point showing an extreme value for a given attribute within the set of data (e.g., which category shows the highest/lowest value?)
\end{itemize}
}

\subsection{Modeling Overview}

Our goal was to develop the model \name to handle tasks articulated as free-form text and be able to perform gaze movement at a detailed pixel level.
We conceptualize the design of the hierarchical gaze control model in Figure~\ref{fig:model}, where the high-level (cognitive) controller is responsible for reasoning while the low-level (oculomotor) controller determines details of gaze movement. 
The idea behind this is hierarchical supervisory control~\cite{eppe2022intelligent}, which refers to a tiered control system in which the superior controller set goals for its subordinates. The actions from subordinates are integrated into an overall pattern for high-level control~\cite{pew1966acquisition}.
The concept also follows the modeling principle of computational rationality, where we assume that the controllers optimize their policy to maximize expected utility within relevant cognitive bounds~\cite{oulasvirta2022computational,chandramouli2024workflow}.
Specifically, the high-level controller handles abstract information processing, comprehension, and memory storage. 
It sets subtasks to the low-level controller, which then moves the gaze to gather information for task completion. Subsequently, the high-level controller utilizes the amassed information to answer the question.

\begin{figure*}[!t]
\centering
  \includegraphics[width=\textwidth]{Images/h-gaze-control.png}
  \caption{\textbf{An overview of the hierarchical eye-movement control architecture.} When presented with a chart and a task, a cognitive controller, powered by large language models, makes decisions on what to look at next and judges whether it is confident enough to provide an answer to the task's question. It relies on internal memory, which summarizes the information gathered from the chart through eye movements. Once cognitive control has determined the next action, the oculomotor controller is responsible for moving the gaze and observing the chart through a limited vision field. The model's objective is to accurately address the task as quickly as possible within set cognitive and physical constraints.}
  \Description{An overview of the hierarchical eye movement control architecture.}
  \label{fig:model}
\end{figure*}

\subsection{Cognitive Control}

The high-level controller provides cognitive control over the mental processes for a chart, control that performs reasoning in working memory~\cite{liu2010mental}. When performing vision tasks, one observes and analyzes visual information interactively~\cite{chen2020air}. Throughout this process, people analyze the information in their memory and try to gather more useful information to reduce uncertainty in solving the task.
To represent this decision problem accurately, we formulate it as a bounded optimality problem in a partially observable Markov decision process (POMDP). Instead of having access to a full state ($\mathcal{S}$) with pixels of the chart associated with the given task, the POMDP expresses a subset of ($\mathcal{S}$) as the observation of the model:
\begin{itemize}
    \item Observation $O$ refers to the information in memory that is captured from eye movements over the chart.
    \item Action $A$ includes subtasks that the model gives to oculomotor control for performing eye movements.
    \item Reward $R$ is the correctness of the answer for the task from the chart question answering.
\end{itemize}
To solve this POMDP, our model uses LLMs for the policy. The rationale behind this choice is that LLMs are well suited to processing higher-level information, as they have been pre-trained on human text data encompassing a wealth of logic related to planning, reasoning, and interaction~\cite{huang2022language, vemprala2024chatgpt, li2023interactive}.
Although LLMs are limited in their ability to control low-level motor functions in a precise manner~\cite{dalal2024psl}, they are proficient at planning and reasoning, with LLaMA~\cite{touvron2023llama} and GPT~\cite{achiam2023gpt} showing impressive language interpretation and reasoning capabilities.
Also, recent work has shown that utilizing LLMs in the high-level controllers in hierarchical architecture can produce promising results~\cite{huang2022language, brohan2023can, liang2023code}.
For our setting, we used GPT-4o~\cite{achiam2023gpt} for the policy, which takes the information accumulated in the memory as the observation and sets subtasks to guide eye movements in order to obtain information needed for solving the task efficiently.

We consider two human limitations when constructing the model's observation: a limited field of vision~\cite{duchowski2018gaze} and memory capacity~\cite{loftus2019human}. 
The model gets information from the gaze position purely by mimicking the human vision system. 
An optical character recognition technique~\cite{singh2010optical} is used to extract text from the pixels of the chart, and the text in the gaze area, with the position, is passed to the memory.
As a result, the observation consists of image patches (in a limited number) from the full set of chart pixels. The reliability of items in memory is determined by their visit history~\cite{li2023modeling}, with overall memory capacity being restricted too. When new information is added to the memory, a previously added item is removed on the basis of a forgetting probability. The probability of forgetting an item in the memory is calculated by means of the formula $\text{Softmax}(\rho \cdot (t-t_i)) $, where $t$ is the current fixation index, $t_i$ is the index of the $i$th item in the memory, and $\rho$ is the weight parameter (set to 0.1 here). The observation is designed as a prompt that summarizes the memory in line with the memory model and explains the model's goal. 

Given the summary of the memory information, the LLM policy selects predefined operations for task solving~\cite{brohan2023can, liang2023code}. The operations here are based on a sequence of cognitive stages for charts~\cite{goldberg2011eye} -- 1) \textit{search for text label}: visually searching for a text label or value label related to the task, 2) \textit{find associated mark}: visually searching for a graphical mark of the data point when given a reference label, 3) \textit{read associated value}: visually searching to read the given mark's associated value or textual label.
All these actions are allowed to be reused in the process, which enables the model to revisit previous positions for confirmation of the information.
Ultimately, if the information in the memory is sufficient to address the task, the gaze movement can stop and an answer can be given. Operations other than answering the question will be performed by the oculomotor controller for detailed gaze movement. 

The examples in Figure~\ref{fig:memory} demonstrate how utilizing memory information and predefined operations aids in scanpath prediction. Model memory uses the summarization capability of LLMs to convert the text and positions gathered to a paragraph as the observation (as shown in the green boxes). The LLM policy then makes decisions and issues subtasks as actions (in red boxes) for the oculomotor control, which performs pixel-level gaze movements.

\begin{figure*}[!t]
\centering
  \includegraphics[width=\textwidth]{Images/scanpath-memory.png}
  \caption{The figure gives examples of how the internal memory helps the cognitive controller to remember what has been read and then select actions for detailed gaze movement. A green box indicates the information held in memory, a red box represents the action selected by cognitive control, and the blue lines in the images reflect the eye movement scanpaths.}
  \label{fig:memory}
\end{figure*}

\subsection{Oculomotor Control}

The oculomotor controller acts as the interface between the cognitive controller and the actual chart-pixel images. Its main function is to control the movement of the gaze over the pixels in order to gather information related to the task at hand.
Generating oculomotor behavior at pixel level is another sequential decision-making problem that can be formulated as a POMDP:
\begin{itemize}
    \item Observation $o$ comprises vision information obtained from the external environment, which is jointly represented by the human vision system and visual short-term memory (VSTM).
    \item Action $a$  involves specifying the coordinates $(x, y)$ of a particular position to move to.
    \item Reward $r$ is designed to encourage the gaze to reach the target with less cost. It takes into account the number of target hits as well as the cost associated with the distance of the gaze movement.
\end{itemize}

Our modeling of a chart reader's observation follows an idea similar to that in visual search~\cite{yang2020predicting}. Utilizing a representation for accumulating information through fixations, this employs four components: 
1) The foveal and peripheral view come from the human vision system, which receives high-resolution visual input only from the region of the image around the fixation location. It includes two pixel-based modules to read the chart: foveal and peripheral vision~\cite{duchowski2018gaze}). 
2) Visual saliency provides a bottom-up signal to a chart reader for the given task. The saliency of the chart affects gaze behavior. We use a task-driven saliency model to represent this feature ~\cite{wang2024salchartqa}.
3)  Visit history represents VSTM, which stores visual information for a few seconds, thereby allowing its use in ongoing cognitive tasks~\cite{alvarez2004capacity}. We represent this history through a matrix where each point is marked as visited or not.
4) A goal-related reference position serves as the initial starting point of gaze movement. For example, the reader might begin at the position of a text label for locating the associated graphical mark, where the position of the text label serves as the reference for the sub-goal. 
\rv{We use a one-hot matrix to represent the reference, in which all cell values are 0 apart from the single 1 that identifies the target.}
All these components are encoded together via the deep convolutional neural network, followed by a fully connected network.

We train reinforcement learning policies to solve the POMDP for the oculomotor control, because it has been proven to effectively address decision-making challenges in prediction of details of gaze movement~\cite{yang2020predicting, jiang2024eyeformer, shi2024crtypist, bai2024heads}.
In our detail-level implementation, we resize the input chart images to be $320 \times 320$ and discretize the fixation position into a $20 \times 20$ map. Consequently, each fixation becomes a $16 \times 16$ image patch, and the gaze position is randomly sampled from within that patch. In this setup, the maximum approximation error resulting from this discretization process is less than one degree of the visual angle~\cite{yang2020predicting}.
\rv{
Ultimately, both the scanpath and the image will be converted back to the original chart size from $320 \times 320$ pixels.
}

\subsection{\rv{Workflow}}
\label{sec:workflow}

\rv{
Our implementation of \name is trained and tested on a collection of tasks and charts. There are four steps, illustrated in Figure~\ref{fig:pipeline}.
In Step 1, real-world charts are manually collected and labeled for areas of interest (AOIs), while synthetic charts are automatically generated and labeled in a manner powered by Vega-Lite~\cite{satyanarayan2016vega}. The inclusion of synthetic charts helps increase the diversity of the chart collection and addresses the challenge of obtaining numerous annotated charts.
In Step 2, tasks are automatically generated in line with specific rules for the \textit{RV}, \textit{F}, and \textit{FE} tasks. These tasks and labeled charts constitute a data collection for the training environment.t
With Step 3, the policies for oculomotor control are trained through reinforcement learning (using proximal policy pptimization, PPO~\cite{schulman2017proximal}) to optimize gaze movements, enabling the system to reach task-relevant positions as quickly as possible while adhering to vision constraints. Importantly, no eye tracking data are required for PPO training.
In the last phase, prediction, the hierarchical architecture combines pre-trained LLMs (GPT-4o) for cognitive control with RL policies for oculomotor control to generate the scanpath prediction.
}

\begin{figure*}[!h]
\centering
  \includegraphics[width=\textwidth]{Images/pipeline.png}
  \caption{\rv{An overview of the training workflow: 1) chart collection and labeling, wherein diverse real-world and synthetic charts are gathered, involving manual and automatic annotation of AOIs; 2) task generation, utilizing a rule-based approach to create tasks based on labeled charts to construct a data collection for training; 3) policy training, in which policy models are trained via RL from chart images with tasks; and 4) scanpath prediction, wherein pre-trained LLMs and RL policies are coordinated hierarchically to predict task-driven gaze movements over charts.}}
  \label{fig:pipeline}
  % \vspace{-10mm}
\end{figure*}
\section{Experiment}
We evaluate our proposed method with strong baselines and further analyze contributions of different components, and the impact of key parameters.

\subsection{Experiment Setup}
\textbf{Dataset.}
We evaluate all the methods on Inter-X dataset, which consists about 9K training samples and 1,708 test samples. Each sample is an action-reaction sequence and three corresponding textual description.
As supplementation, we mix our pre-training data with single person motion-text dataset HumanML3D~\citep{humanml3d}, which consists more than 23K annotated motion sequences.
We uniformly sample frames for both datasets to 30 FPS. 

\textbf{Evaluation Metrics.}
Following single-person motion generation~\citep{t2mgpt}, we adopt the these metrics to quantitatively evaluate the generated motion: R-Precision measures the ranking of Euclidean distances between motion and text features. Accuracy (Acc.) assesses how likely a generated motion could be successfully recognized as its interaction label, like ``high-five''. Frechet Inception Distance~\citep{fid} (FID) evaluates the similarity in feature space between predicted and ground-truth motion. Multimodal Distance (MMDist.) calculates the average Euclidean distance between generated motion and the corresponding text description. Diversity (Div.) measures the feature diversity within generated motions. All the metrics reported are calculated with batch size set to 32, and accumulated across the test dataset, and we evaluate each method for 20 times with different seeds to calculate the final results at 95\% confidence interval.

\textbf{Evaluation Model.} \label{sec:eval}
Every metric mentioned above requires an encoder $\mathcal{M}$ to extract motion feature.
For single person text-to-motion generation tasks, a motion-text matching model are commonly trained as human motion feature extractor.
A simple way to transfer this method to interaction domain is to directly train an interaction-to-text matching model $\mathcal{M}(\mathbf{a}, \hat{\mathbf{b}}, text)$, where action sequence $\mathbf{a}$ and predicted reaction sequence $\hat{\mathbf{b}}$ together is regarded as a generated interaction sequence, or a reaction-to-text match model $\mathcal{M}(\hat{\mathbf{b}}, text)$.
However, the former one may focus too much on the ground-truth action input, leading insufficient discriminative power of $\hat{\mathbf{b}}$'s quality, while the latter one lacks semantics provided by action, thus leading to subpar matching capability.

To address the issue, we simply uniformly mask off a large portion of $\mathbf{a}$, obtaining down-sampled action motion sequence $\mathbf{a}'$ (downsampled to 1 FPS in our setting), which serves as a semantic hint for the matching process while not introducing too much emphasis on input action sequence.
The final evaluation model consists of an masked interaction encoder and a text encoder.
We use contrastive loss following CLIP~\citep{clip}, which encourages paired motion and text features to be close geometrically.
In addition, we add a classification head after the predicted motion features, to simultaneously predict interaction labels, such as ``high-five''.

\textbf{Baselines.} To evaluate the performance of our method \ModelAbbr~on online and unconstrained setting, we compare \ModelAbbr~with the following baselines:
1) \textbf{InterFormer}~\citep{interformer} is a transformer based action-to-reaction generation model that leverages human skeleton as prior knowledge for efficient attention process.
2) \textbf{MotionGPT}~\citep{motiongpt} is a motion-language model that leverages an LLM for motion and text generation. We extend the motion tokenizer of MotionGPT to encode multi-person motion, while keeping other settings unchanged.
3) \textbf{InterGen}~\citep{intergen} proposes a mutual attention mechanism within diffusion process for human interaction generation, we reproduce and adapt IngerGen to action-to-reaction generation.
4) \textbf{ReGenNet}~\citep{regennet} is latest state-of-the-art model on action-to-reaction generation. It adopts a transformer decoder based diffusion model, which directly predicts human reaction given action input in unconstrained and online manner as ours.


\textbf{Implementation Details.}
For the LLM, we adopt Flan-T5-base~\citep{flan,t5} as our base model, with extended vocabulary. We warm up the learning rate for 1,000 steps, peaking at 1e-4 for the pre-training phase, and use the same learning rate for fine-tuning.
Both the pre-training and fine-tuning phases are trained on a single machine with 8 Tesla V100 GPUs. The training batch size is set to 32 for the LLM and we monitor the validation loss and reaction generation metrics for early-stopping, resulting about 100K pre-training steps and 40K fine-tuning steps.
We set the re-thinking interval $N_r$ to 4 tokens and divide each space signal into $N_b=10$ bins.

\begin{table}[t]
\centering
\tiny
\caption{Comparison to state-of-the-art baselines and ablation studies of our method on Inter-X dataset. $\uparrow$ or $\downarrow$ denotes a higher or lower value is better, and $\rightarrow$ means that the value closer to real is better. We use $\pm$ to represent 95\% confidence interval and highlight the best results in \textbf{bold}. For ablation methods (in grey), PT, M, P, S, and SP are abbreviations for pre-training, motion, pose, space, and single-person data, respectively.}
\label{tab:main}
\begin{tabular}{l|ccccccc}
\toprule
\multirow{2}{*}{Methods} &  \multicolumn{3}{c}{R-Precision$\uparrow$} & \multirow{2}{*}{Acc.$\uparrow$}& \multirow{2}{*}{FID$\downarrow$}         & \multirow{2}{*}{MMDist$\downarrow$} & \multirow{2}{*}{Div.$\rightarrow$} \\
               & Top-1       & Top-2       & Top-3     &     &         &                               &                       \\ \midrule
Real                    & $0.511^{\pm.003}$ & $0.682^{\pm.002}$ & $0.776^{\pm.002}$ & $0.463^{\pm.000}$  & $0.000^{\pm.000}$         & $5.348^{\pm.002}$         & $2.498^{\pm.005}$           \\ \midrule
InterFormer             & $0.172^{\pm.012}$ & $0.292^{\pm.013}$ & $0.343^{\pm.012}$ & $0.171^{\pm.009}$ & $10.468^{\pm.021}$        &  $7.831^{\pm.018}$         & $3.505^{\pm.023}$           \\
MotionGPT &            $0.238^{\pm.003}$        &     $0.354^{\pm.004}$        &     $0.441^{\pm.003}$   &    $0.186^{\pm.002}$            &     $5.823^{\pm.048}$               &      $6.211^{\pm.005}$           &      $2.615^{\pm.007}$ \\
InterGen                & $0.326^{\pm.036}$ & $0.423^{\pm.063}$ & $0.525^{\pm.053}$ & $0.254^{\pm.019}$  & $5.506^{\pm.257}$         & $6.182^{\pm.038}$         & $2.284^{\pm.009}$           \\
ReGenNet                & $0.384^{\pm.005}$ & $0.483^{\pm.002}$ & $0.572^{\pm.003}$ & $0.297^{\pm.004}$  & $3.988^{\pm.048}$         & $5.867^{\pm.009}$         & $\mathbf{2.502^{\pm.001}}$           \\ \midrule
% \rowcolor[HTML]{EFEFEF}
\ModelAbbr~(Ours)       & $\mathbf{0.423^{\pm.005}}$ & $\mathbf{0.599^{\pm.003}}$ & $\mathbf{0.693^{\pm.003}}$ & $\mathbf{0.318^{\pm.003}}$  & $\mathbf{1.942^{\pm.017}}$         & $\mathbf{5.643^{\pm.003}}$         & $2.629^{\pm.006}$           \\
\rowcolor[HTML]{EFEFEF}
w/o Think           & $0.367^{\pm.003}$ & $0.491^{\pm.027}$ & $0.584^{\pm.008}$ & $0.230^{\pm.036}$ & $3.828^{\pm.016}$         & $6.186^{\pm.055}$         & $2.609^{\pm.006}$           \\
\rowcolor[HTML]{EFEFEF}
w/o All PT.         & $0.398^{\pm.007}$ & $0.531^{\pm.002}$ & $0.628^{\pm.003}$ & $0.288^{\pm.002}$ & $3.467^{\pm.113}$         & $5.822^{\pm.003}$         & $2.909^{\pm.053}$           \\
\rowcolor[HTML]{EFEFEF}
w/o M-M PT. & $0.408^{\pm.005}$ & $0.563^{\pm.004}$ & $0.646^{\pm.005}$ & $0.293^{\pm.002}$ & $2.874^{\pm.020}$         & $5.736^{\pm.003}$         & $2.553^{\pm.006}$           \\
\rowcolor[HTML]{EFEFEF}
w/o P-S PT. & $0.417^{\pm.004}$ & $0.582^{\pm.004}$ & $0.664^{\pm.004}$ & $0.308^{\pm.003}$ & $2.685^{\pm.024}$         & $5.699^{\pm.004}$         & $2.859^{\pm.007}$           \\
\rowcolor[HTML]{EFEFEF}
w/o M-T PT. & $0.406^{\pm.003}$ & $0.557^{\pm.004}$ & $0.637^{\pm.004}$ & $0.304^{\pm.003}$ & $2.580^{\pm.021}$         & $5.822 ^{\pm.003}$         & $2.889^{\pm.005}$           \\
\rowcolor[HTML]{EFEFEF}
w/o SP Data     & $0.414^{\pm.004}$ & $0.592^{\pm.005}$ & $0.685^{\pm.003}$ & $0.315^{\pm.004}$ & $2.007^{\pm.015}$         & $5.667^{\pm.003}$         & $2.611^{\pm.005}$           \\
\bottomrule
\end{tabular}
\end{table}



\begin{figure}
    \centering
    \includegraphics[width=\linewidth]{figs/tsne.pdf}
    \caption{Visualization of a person's motion sequences in Inter-X dataset and HumanML3D dataset.}
    \label{fig:tsne}
\end{figure}

\subsection{Comparison to Baselines}\label{sec:sota}
As shown in the upper side of Table~\ref{tab:main}, our method \ModelAbbr~significantly outperforms baseline methods in terms of ranking, accuracy, FID and multimodal distance, showing superior human reaction generation quality.
Compared to MotionGPT, which adopts a similar motion-language architecture, \ModelAbbr~expresses stronger performance, which we attribute to our unified representation of motion via space and pose tokenizers, enabling effective individual pose and inter-person spatial relationship representation.
\ModelAbbr~also surpasses the diffusion-based methods, InterGen and ReGenNet, with our think-then-react architecture, improving generated motions by describing observed action and reasoning what reaction is expected on semantic level. In addition, ReGenNet and MotionGPT get closer diversity to the real than our model. We mainly attribute to that, \ModelAbbr~may conduct multiple re-thinking processes during inference, and the inferred semantics may bring a higher diversity.


\subsection{Ablation Study of Key Components}
To evaluate the effectiveness of our proposed key designs, we conduct detailed ablation studies by removing each of them to observe how much drop compared to the full version of our \ModelAbbr~method. The larger drop indicates more contribution. The results are shown in gray lines of Table~\ref{tab:main}. According to the drops in FID, all designs, including thinking, pre-training tasks and using single person data in pre-training, have positive contributions to the final performance, and thinking contributes the most. Some detailed findings and analyses are as follows.

First, we skip \textbf{thinking} stage during inference, and find the performance drops significantly in FID from 1.9 to 3.8. This supports the necessity of our proposed thinking process before reacting. We also notice decreasing diversity of generated samples, as the model relies solely on input action, and cannot explicitly capture and infer action's intent, thus leading to more rigid motion in some cases.

Second, to evaluate the effectiveness of \textbf{pre-training}, we omit the pre-training stage, and directly train our model \ModelAbbr~for thinking and reacting tasks. As shown in Table~\ref{tab:main}, our model's performance deteriorates without a fine-grained pre-training phase from 1.9 to 3.4 in FID. This indicates that pre-training can effectively adapt a language model (Flan-T5-base) into a motion and language model. We further removing three kinds of pre-training tasks: motion-motion (M-M PT.), pose-space (P-S PT.), and motion-text (M-T PT.). The results show that the without any task, the performance obviously gets worse, from 1.9 to 2.5 - 2.8 in FID, indicating their positive contribution to the final performance and complementary values to each other.

Third, to see how much \textbf{single-person data} helps reaction generation, we remove single person motion-text data, i.e., the data from HumanML3D dataset, from our training set. The result (w/o SP Data) shows that the model performs worse without training on HumanML3D, which proves that our unified motion encoder and motion-language architecture can leverage both single- and multi-person data, alleviating the insufficiency of training data. However, the benefit from single-person data is not as large as we expect. 

% What's more, we evaluate the necessity of \textbf{decoupled space-motion tokenizer}, and the results are shown in Table~\ref{tab:vqvae}. We design a plain motion VQ-VAE with unnormalized action and reaction as input, maintaining absolute space and pose features. With the trained motion VQ-VAE, we encode action/reaction into tokens, which are then fed into TTR for reaction prediction task. First, without normalized motion as input, the reconstruction FID significantly rises from 0.262 to 0.983, showing deteriorated reconstruction performance due to insufficient utilization of codebook. Second, in the reaction generation phase, TTR's performance drops dramatically, as the badly constructed codebook leads to inaccurate action understanding and reaction prediction, highlighting the necessity of decoupling token representation of space and pose features in multi-person scenario.

\begin{figure}
    \centering
    \includegraphics[width=\linewidth]{figs/case_study.pdf}
    \caption{Visualized cases of our predicted reactions (in green) to input action (in blue) and corresponding thinking results. We also provide a failure case in figure (d), where TTR misunderstands the input action as ``wrestling'', which should be ``embracing''.}
    \label{fig:case_study}
\end{figure}


\subsection{Analysis on Overlapping between Single- and Multi-Person Motions}
To investigate the reason of small contribution from single-person data, we further visualize motion sequences of single-person motion (HumanML3D), two-person action (Inter-X Action) and reaction (Inter-X Reaction) in the same space, as presented in Figure~\ref{fig:tsne}. Specifically, we use t-SNE tool~\cite{tsne} to project motion token sequence features into two-dimension. As shown in Figure~\ref{fig:tsne}, the single- and two-person motion sequences have little overlap. When doing case studies, we find that most two-person motion are unique, e.g., massage and being pulled, and will never be used in single-person motion. Similarly, most single-person motions are unique too, e.g., T-pose, and seldom appear in multi-person interaction. There are only a few overlapped motions, e.g., standing still. In addition, when comparing action and reaction sequences in multi-person interaction, we have some interesting findings. When reactions are close to actions, the motion usually belongs to symmetrical interactions, e.g., pulling or being pulled; whereas, when actions are far from reactions, the motion usually belongs to asymmetrical interaction, e.g., massage.


\subsection{Impact of Down-Sampling Parameter in Matching Model for Evaluation}

As described in Section~\ref{sec:eval}, we propose downsampling action motion sequence to avoid matching models for evaluation pay too much attention to input action rather than output reaction. We conduct an experiment to change the downsampling parameter frame rate and calculate the difference between taking ground-truth action and random action as the input of $\mathcal{M}$, in terms of summed ranking scores (Top-1, Top-2, Top-3 and Acc.). As presented in Figure~\ref{fig:discriminative}, 
difference is lowest when FPS equals to 0, which meaning we only match generated reaction motion with text. It goes up to the peak when FPS equals 1 and quickly goes down to low values, even close to the lowest when FPS is about 15. This indicates that it is necessary to concatenate input action with generated reaction to compose a meaningful interaction in evaluation, otherwise the motion-text matching model cannot well recognize the interaction. However, only 1 FPS is enough. With larger FPS, the matching models will be disturbed by input action rather than the generated reaction. Thus, we choose 1 FPS, corresponding to the largest difference, as our final setting.

\subsection{Impact of Re-thinking Interval}
% Our aim is to generate real-time reaction online, and thus time interval is an important parameter to generation quality. 
We change the re-thinking interval $N_r$ from about 1 to 100 timesteps (about 0.1 to 10 seconds) and observe how it impacts generative quality measure FID. As shown in Figure~\ref{fig:latency}, FID falls down first until $N_r=4$ (about 0.5 second) and then continues rising up. This indicate that the best time interval is about 0.5 second. When the time interval is too short, our \ModelAbbr~model cannot get enough information to re-think what the input action means and will bring some randomness into predicting appropriate reaction. When the time interval gets too long, our \ModelAbbr~model give slow responses to the input action sequences and generates coarse-grained reaction.

We also evaluate the average inference time per step (AITS) with respect to the re-thinking interval. As shown in Figure~\ref{fig:latency}, the inference time significantly decreases as the re-thinking interval increases, eventually converging to approximately 10 milliseconds per step (100 FPS). In our setup, we opt to re-think every four steps, resulting in an inference time of less than 50 milliseconds, which meets the requirements for a real-time system.

\subsection{User Study}
To further evaluate our model qualitatively, we conduct a user study on TTR vs. the latest SOTA method ReGenNet, and the results are shown in Figure~\ref{fig:user_study}. We randomly sample 100 action sequences from Inter-X dataset, which are fed into TTR and ReGenNet to predict reactions, and ask four real human to choose the better ones. It can be seen that TTR surpasses ReGenNet on all the duration range, and the winning rate rises significantly when motion duration is longer. We mainly contribute this to our explicit thinking and re-thinking procedure, which ensures semantics matching and alleviates accumulated errors. 

\begin{figure}[t]
    \centering
    \begin{minipage}[b]{0.3\textwidth}
        \centering
        \includegraphics[width=\textwidth]{figs/discriminative_power.pdf}
        \caption{Impact of input action FPS to summed ranking score differences.}
        \label{fig:discriminative}
    \end{minipage}
    \hfill
    \begin{minipage}[b]{0.35\textwidth}
        \centering
        \includegraphics[width=\textwidth]{figs/latency.pdf}
        \caption{Impact of re-thinking interval to FID and average inference time per step (AITS).}
        \label{fig:latency}
    \end{minipage}
    \hfill
    \begin{minipage}[b]{0.3\textwidth}
        \centering
        \includegraphics[width=\linewidth]{figs/user_study.pdf}
        \caption{User preference between TTR and ReGenNet on different motion duration.}
        \label{fig:user_study}
    \end{minipage}
\end{figure}

\section{Conclusion and Discussion}
In this paper, we introduce 3DMolFormer for structure-based drug discovery, a dual-channel transformer-based framework designed to process parallel sequences of tokens and numerical values representing pocket-ligand complexes. Through self-supervised large-scale pre-training and supervised fine-tuning, 3DMolFormer can accurately and efficiently predict the binding poses of ligands to protein pockets. Furthermore, through reinforcement learning fine-tuning, 3DMolFormer can generate drug candidates that exhibit high binding affinities for a given protein target, along with favorable drug-likeness and synthesizability. Above all, 3DMolFormer is the first machine learning framework that can simultaneously address both protein-ligand docking and pocket-aware 3D drug design, and it outperforms previous baselines in both tasks.

It is noteworthy that many recent deep learning models for 3D molecules, such as Uni-Mol, Pocket2Mol, TargetDiff, and DecompDiff, which serve as baselines in our experiments, adhere to the concept of "equivariance" introduced by geometric deep learning~\citep{Equivariance,Equivariance2}. However, the 3DMolFormer model does not explicitly enforce SE(3)-symmetry. It appears that through the normalization of 3D coordinates and random rotations during data augmentation, 3DMolFormer has acquired the SE(3)-equivariance by training on a sufficiently large and diverse dataset. This approach aligns with recent successful methods in the field, including AlphaFold3~\citep{AlphaFold3}, which also does not rely on SE(3)-equivariant architectures.

Admittedly, our approach still has some limitations. First, 3DMolFormer does not account for the flexibility of proteins during ligand binding, which may affect the accuracy of subsequent binding affinity prediction. Second, protein-ligand binding is a dynamic process, but 3DMolFormer struggles to capture this dynamism effectively. Finally, 3DMolFormer does not consider environmental factors such as temperature and pH, which can significantly influence the 3D conformation of the binding complex. These issues represent core challenges in current computational methods for structure-based drug discovery, and we look forward to future work addressing these limitations. Furthermore, the implementation details in 3DMolFormer have the potential to be further optimized, for example, advanced methods of multi-objective reinforcement learning~\citep{MORL} may be introduced into the drug design process.

\textbf{Acknowledgment} This work is supported by the National Natural Science Foundation of China (No. 62276268) and Kuaishou Technology.

\bibliography{iclr2025_conference}
\bibliographystyle{iclr2025_conference}

\newpage

\appendix
\section{Supplementary Experiment Results}

\subsection{Gener Tasks}
The confusion matrices for the gene classification task are shown in \textit{Fig.} \ref{fig:gene_classification}, whereas \textit{Fig.} \ref{fig:species_classification} illustrates those for the taxonomic classification task. For both tasks, the \textbf{Gener}\textit{ator} not only excels in average performance but also maintains top performance across different categories of genes and taxonomic groups.

\subsection{Next K-mer Prediction}
The evaluations of next K-mer prediction are provided in \textit{Fig.} \ref{fig:kmer_tokenizer} and \textit{Fig.} \ref{fig:kmer_model}. Specifically, \textit{Fig.} \ref{fig:kmer_tokenizer} presents the evaluations for tokenizer comparison, where the 6-mer tokenizer outperforms others across different taxonomic groups. \textit{Fig.} \ref{fig:kmer_model} presents the evaluations for model comparison, where the \textbf{Gener}\textit{ator} consistently outperforms others across various taxonomic groups.

\section{Details of Pre-training}
In the pre-training phase, we employed the AdamW~\cite{adamw} optimizer with $\beta_1 = 0.9$, $\beta_2 = 0.95$, and $\text{weight decay} = 0.1$. The learning rate schedule incorporated a linear warm-up followed by cosine decay: the learning rate increased linearly from 0 to its peak value over the first 2,000 steps, then followed a cosine decay, decreasing until it reached 10\% of the peak value by the end of the training process. The peak learning rate was established at $4e^{-4}$, and gradient clipping was applied with a norm threshold of $1.0$. We adhered to standard practices in pre-training LLMs, using a batch size that encompasses 2 million tokens. Given the maximum sequence length of 16,384 tokens, this configuration resulted in each batch comprising 128 samples, where each sample was initialized with a random starting point between 0 and 5 at each epoch. The complete pre-training of the \textbf{Gener}\textit{ator} spanned 6 epochs, amounting to approximately 185,000 steps. To enhance computational efficiency, we employed optimization techniques like Flash Attention~\cite{flashattention,flashattention2} and the Zero Redundancy Optimizer~\cite{deepspeed,fsdp}. The pre-training process, utilizing 32 NVIDIA A100 GPUs, was completed in 368 hours. This configuration achieved a model FLOPs utilization (MFU) of approximately 46\%.

The summary statistics of the pre-training data in terms of the number of nucleotides and number of genes are provided in Table \ref{tab:pretrain_data_statistics_species} and Table \ref{tab:pretrain_data_statistics_gene}. The pre-training losses for the \textbf{Gener}\textit{ator} and \textbf{Gener}\textit{ator}-All are presented in \textit{Fig.} \ref{fig:pretrain_loss}. Although the \textbf{Gener}\textit{ator}-All exhibits a lower pre-training loss compared to the \textbf{Gener}\textit{ator}, it consistently underperforms in almost all benchmark tests, as discussed in previous sections. This discrepancy is likely due to the inclusion of non-gene regions, which often contain highly repetitive and simple segments. Additionally, the pre-training losses for different tokenizers and the Mamba model are shown in \textit{Fig.} \ref{fig:pretrain_loss_tokenizer}. The significant variation in vocabulary sizes among the tokenizers results in considerable differences in pre-training losses, making direct comparisons challenging. This issue inspires the proposition of next K-mer prediction for a more effective model comparison.

\section{Details of Benchmark Experiments}

\subsection{Benchmarks}
In this paper, we conducted benchmark evaluations on two widely used benchmarks, Nucleotide Transformer tasks~\cite{nucleotide-transformer} and Genomic Benchmarks~\cite{genomic-benchmarks}, as well as on the Gener tasks we propose. For Nucleotide Transformer tasks, there are two different versions available: the original version can be accessed \href{https://huggingface.co/datasets/InstaDeepAI/nucleotide_transformer_downstream_tasks}{\textcolor{blue}{here}}, and the revised version is available \href{https://huggingface.co/datasets/InstaDeepAI/nucleotide_transformer_downstream_tasks_revised}{\textcolor{blue}{here}}. For Genomic Benchmarks, the dataset can be downloaded from \href{https://huggingface.co/katarinagresova}{\textcolor{blue}{here}}.

\subsection{Metrics}
Regarding evaluation metrics, we follow the original settings of the benchmarks. For Nucleotide Transformer tasks, we employ the Matthews correlation coefficient (MCC):
\begin{equation}
\text{MCC} = \frac{\text{TP} \times \text{TN} - \text{FP} \times \text{FN}}{\sqrt{(\text{TP} + \text{FP})(\text{TP} + \text{FN})(\text{TN} + \text{FP})(\text{TN} + \text{FN})}}, \nonumber
\end{equation} 
where TP, TN, FP, FN represent true positives, true negatives, false positives, and false negatives, respectively. For Genomic Benchmarks, accuracy is used as the evaluation metric. For Gener tasks, we utilize the weighted F1 score for multi-classification, calculated as follows:
\begin{equation}
\text{Precision}_i = \frac{\text{TP}_i}{\text{TP}_i + \text{FP}_i},
~~~
\text{Recall}_i = \frac{\text{TP}_i}{\text{TP}_i + \text{FN}_i}, \nonumber
\end{equation}

\begin{equation}
\text{F1}_i = 2 \times \frac{\text{Precision}_i \times \text{Recall}_i}{\text{Precision}_i + \text{Recall}_i},
~~~
\text{F1}_{\text{weighted}} = \sum_{i=1}^{n} w_i \times \text{F1}_i, \nonumber
\end{equation}
where $w_i = {n_i}/{N}$ denotes the weight for class $i$, $n_i$ is the number of samples in class $i$, and $N$ is the total number of samples.

\subsection{Baselines}
To comprehensively assess the capability of the \textbf{Gener}\textit{ator} in sequence comprehension, we include several state-of-the-art models as baselines. These models differ in terms of training data, model size, model architecture, and other aspects. Table~\ref{tab:baseline_introductuon} provides a summary of these baselines. Where applicable, evaluation results for the original and revised NT tasks are sourced directly from the NT paper~\cite{nucleotide-transformer}. For the remaining baseline models and datasets, we conduct consistent and fair evaluation tasks ourselves to obtain the results. These self-evaluated baseline models can be accessed through the following Huggingface\footnote{\url{https://huggingface.co}} repositories:
\begin{itemize}
    \item DNABERT-2: \texttt{zhihan1996/DNABERT-2-117M}
    \item HyenaDNA: \texttt{LongSafari/hyenadna-large-1m-seqlen}
    % \item NT-multi: \texttt{InstaDeepAI/nucleotide-transformer-2.5b-multi-species}
    \item NT-v2: \texttt{InstaDeepAI/nucleotide-transformer-v2-500m-multi-species}
    \item Caduceus-Ph: \texttt{kuleshov-group/caduceus-ph\_seqlen-131k\_d\_model-256\_n\_layer-16}
    \item Caduceus-PS: \texttt{kuleshov-group/caduceus-ps\_seqlen-131k\_d\_model-256\_n\_layer-16}
    \item GROVER: \texttt{PoetschLab/GROVER}
\end{itemize}

\subsection{Experimental Setups}
In our benchmark experiments, we retained the optimizer configuration from the pre-training phase, with $\beta_1=0.9$, $\beta_2=0.95$ and $\text{weight decay} = 0.1$. We adopted the `reduce on plateau' strategy for the learning rate scheduler and implemented early stopping based on validation metrics, with a patience of 5. Optimal learning rates and batch sizes for various models and datasets were determined through hyperparameter search, as detailed in Tables~\ref{tab:benchmark_hyperparam1} to \ref{tab:benchmark_hyperparam4}. For Nucleotide Transformer tasks and Genomic Benchmarks, all models had sufficient context length, allowing them to utilize the entire input sequences. In contrast, for Gener tasks, input sequences exceeding the available context length were truncated to fit within the model constraints. For all causal language models, we performed prediction through an additional linear layer using the embedding of the \texttt{<EOS>} token, while for masked language models, we used the \texttt{<BOS>} token or \texttt{<CLS>} token instead. Notably, for Caduceus, we followed its original configuration using mean pooling and applied the recommended reverse complement data augmentation for Caduceus-Ph. All models obtained embeddings from their final layer and underwent full fine-tuning. All evaluation metrics were obtained through 10-fold cross-validation.

\section{Details of Central Dogma}
\label{sec:central_dogma_supp}
We sourced the protein sequences and the corresponding protein-coding DNA fragments for the Histone and Cytochrome P450 families from the UniProt~\cite{UniProt} database. Each family had approximately 400,000 entries. The \textbf{Gener}\textit{ator} model was then fine-tuned on these datasets using the protein-coding sequences, with a learning rate of $5e^{-5}$ and a batch size of 1,024. For the fine-tuned models, we generated 100,000 sequences for each combination of temperature $T \in \{0.5, 0.6, 0.7, 0.8, 0.9, 1.0\}$ and nucleus sampling $P \in \{0.5, 0.6, 0.7, 0.8, 0.9, 1.0\}$. Following translation into protein sequences, duplicates were removed by comparing the generated sequences with the training set, resulting in a unique selection of 10,000 samples per family. The hyperparameters were set to $T=0.6$, $P=0.6$ for the Cytochrome P450 family (\textit{Fig.}~\ref{fig:cytochrome_generation}), and $T=1.0$, $P=0.8$ for the Histone family (\textit{Fig.}~\ref{fig:histone_generation}). 

\section{Details of Enhancer Design}
For the enhancer activity prediction, we adhered to the data partitioning of DeepSTARR~\cite{DeepSTARR}. We conducted the same hyperparameter search strategy as in our benchmark experiments and selected a learning rate of $2e^{-5}$ and a batch size of 64 for training the activity predictor. For enhancer generation, we selected enhancer sequences from the top and bottom quartiles of activity values within the DeepSTARR training set. These sequences were used to perform supervised fine-tuning (SFT) on the \textbf{Gener}\textit{ator}, labeled with prompts \texttt{<high>} and \texttt{<low>} for high and low activity sequences, respectively. The hyperparameters for SFT were set to a learning rate of $1e^{-5}$ and a batch size of 2,048. During the generation process, we applied the same hyperparameter search strategy detailed in \textit{Sec.}~\ref{sec:central_dogma_supp}. The hyperparameters used for visualization in \textit{Fig.}~\ref{fig:enhancer_design} were set to $T=0.5$, $P=0.7$ for the developmental activity, and $T=0.8$, $P=0.6$ for the housekeeping activity.

\section{Computational Resources}
In this study, the primary computational resources utilized for model training and inference included NVIDIA V100 and A100 GPUs. Prior to pre-training, we trained 14 models for one epoch each to evaluate the DNA sequence modeling capabilities of various tokenizers. Our pre-training phase involved two models that varied based on the training data used. In the downstream tasks, we executed over 10,000 runs, which entailed an extensive hyperparameter search across 36 combinations and utilized 10-fold cross-validation for each benchmark task per model. Detailed statistics on the usage of main computational resources are presented in Table~\ref{tab:computational_resources}.

\section{Supplementary Figures \& Tables}

\begin{figure}[ht]
    \centering
    \includegraphics[width=0.8\textwidth]{figures/pdf/gene_classification_full.pdf}
    \caption{Evaluation of model performance on gene classification.}
    \label{fig:gene_classification}
\end{figure}
\begin{figure}[ht]
    \centering
    \includegraphics[width=0.8\textwidth]{figures/pdf/species_classification_full.pdf}
    \caption{Evaluation of model performance on taxonomic classification.}
    \label{fig:species_classification}
\end{figure}
\begin{figure}[ht]
    \centering
    \includegraphics[width=0.8\textwidth]{figures/pdf/kmer_tokenizer_full.pdf}
    \caption{Evaluation of next K-mer prediction for tokenizer comparison.}
    \label{fig:kmer_tokenizer}
\end{figure}
\begin{figure}[ht]
    \centering
    \includegraphics[width=0.8\textwidth]{figures/pdf/kmer_model_full.pdf}
    \caption{Evaluation of next K-mer prediction for model comparison.}
    \label{fig:kmer_model}
\end{figure}
% \input{figures/tex/histone_generation}
\begin{figure}[ht]
    \centering
    \includegraphics[width=0.5\textwidth]{figures/pdf/pretrain_loss.pdf}
    \caption{Pre-training losses of the \textbf{Gener}\textit{ator} and \textbf{Gener}\textit{ator}-All.}
    \label{fig:pretrain_loss}
\end{figure}
\begin{figure}[ht]
    \centering
    \includegraphics[width=0.6\textwidth]{figures/pdf/pretrain_loss_tokenizer.pdf}
    \caption{Pre-training losses of different tokenizers.}
    \label{fig:pretrain_loss_tokenizer}
\end{figure}

% \begin{table*}[!htb]
\small
\renewcommand{\arraystretch}{1.2}
\centering
\caption{Evaluation of the original Nucleotide Transformer tasks. The reported values represent the Matthews correlation coefficient (MCC) averaged over 10-fold cross-validation, with the standard error in parentheses.}
\resizebox{\textwidth}{!}{%
\begin{tabular}{lcccccccccc}
\toprule
& Enformer & DNABERT-2 & HyenaDNA & NT-multi & NT-v2 & Caduceus-Ph & Caduceus-PS & GROVER & \textbf{Gener}\textit{ator} & \textbf{Gener}\textit{ator}-All \\
& (252M) & (117M) & (55M) & (2.5B) & (500M) & (8M) & (8M) & (87M) & (1.2B) & (1.2B) \\
\midrule
H3 & 0.724 (0.018) & 0.785 (0.012) & 0.781 (0.015) & 0.793 (0.013) & 0.788 (0.010) & 0.794 (0.012) & 0.772 (0.022) & 0.768 (0.008) & \textbf{0.806 (0.005)} & \underline{0.803 (0.007)} \\
H3K14ac & 0.284 (0.024) & 0.515 (0.009) & \textbf{0.608 (0.020)} & 0.538 (0.009) & 0.538 (0.015) & 0.564 (0.033) & 0.596 (0.038) & 0.548 (0.020) & \underline{0.605 (0.008)} & 0.580 (0.038) \\
H3K36me3 & 0.345 (0.019) & 0.591 (0.005) & 0.614 (0.014) & 0.618 (0.011) & 0.618 (0.015) & 0.590 (0.018) & 0.611 (0.048) & 0.563 (0.017) & \textbf{0.657 (0.007)} & \underline{0.631 (0.013)} \\
H3K4me1 & 0.291 (0.016) & 0.512 (0.008) & 0.512 (0.008) & 0.541 (0.005) & 0.544 (0.009) & 0.468 (0.015) & 0.487 (0.029) & 0.461 (0.018) & \textbf{0.553 (0.009)} & \underline{0.549 (0.018)} \\
H3K4me2 & 0.207 (0.021) & 0.333 (0.013) & \textbf{0.455 (0.028)} & 0.324 (0.014) & 0.302 (0.020) & 0.332 (0.034) & \underline{0.431 (0.016)} & 0.403 (0.042) & 0.424 (0.013) & 0.400 (0.015) \\
H3K4me3 & 0.156 (0.022) & 0.353 (0.021) & \textbf{0.550 (0.015)} & 0.408 (0.011) & 0.437 (0.028) & 0.490 (0.042) & \underline{0.528 (0.033)} & 0.458 (0.022) & 0.512 (0.009) & 0.473 (0.047) \\
H3K79me3 & 0.498 (0.013) & 0.615 (0.010) & 0.669 (0.014) & 0.623 (0.010) & 0.621 (0.012) & 0.641 (0.028) & \textbf{0.682 (0.018)} & 0.626 (0.026) & \underline{0.670 (0.011)} & 0.631 (0.021) \\
H3K9ac & 0.415 (0.020) & 0.545 (0.009) & 0.586 (0.021) & 0.547 (0.011) & 0.567 (0.020) & 0.575 (0.024) & 0.564 (0.018) & 0.581 (0.015) & \textbf{0.612 (0.006)} & \underline{0.603 (0.019)} \\
H4 & 0.735 (0.023) & 0.797 (0.008) & 0.763 (0.012) & \underline{0.808 (0.007)} & 0.795 (0.008) & 0.788 (0.010) & 0.799 (0.010) & 0.769 (0.017) & \textbf{0.815 (0.008)} & \underline{0.808 (0.010)} \\
H4ac & 0.275 (0.022) & 0.465 (0.013) & 0.564 (0.011) & 0.492 (0.014) & 0.502 (0.025) & 0.548 (0.027) & \underline{0.585 (0.018)} & 0.530 (0.017) & \textbf{0.592 (0.015)} & 0.565 (0.035) \\
Enhancer & 0.454 (0.029) & 0.525 (0.026) & 0.520 (0.031) & 0.545 (0.028) & \underline{0.561 (0.029)} & 0.522 (0.024) & 0.511 (0.026) & 0.516 (0.018) & \textbf{0.580 (0.015)} & 0.540 (0.026) \\
Enhancer type & 0.312 (0.043) & 0.423 (0.018) & 0.403 (0.056) & 0.444 (0.022) & 0.444 (0.036) & 0.403 (0.028) & 0.410 (0.026) & 0.433 (0.029) & \textbf{0.477 (0.017)} & \underline{0.463 (0.023)} \\
Promoter all & 0.910 (0.004) & 0.945 (0.003) & 0.919 (0.003) & 0.951 (0.004) & 0.952 (0.002) & 0.937 (0.002) & 0.941 (0.003) & 0.926 (0.004) & \textbf{0.962 (0.002)} & \underline{0.955 (0.002)} \\
Promoter non-TATA & 0.910 (0.006) & 0.944 (0.003) & 0.919 (0.004) & \underline{0.955 (0.003)} & 0.952 (0.003) & 0.935 (0.007) & 0.940 (0.002) & 0.925 (0.006) & \textbf{0.962 (0.001)} & \underline{0.955 (0.002)} \\
Promoter TATA & 0.920 (0.012) & 0.911 (0.011) & 0.881 (0.020) & 0.919 (0.008) & \underline{0.933 (0.009)} & 0.895 (0.010) & 0.903 (0.010) & 0.891 (0.009) & \textbf{0.948 (0.008)} & 0.931 (0.007) \\
Splice acceptor & 0.772 (0.007) & 0.909 (0.004) & 0.935 (0.005) & \underline{0.973 (0.002)} & \underline{0.973 (0.004)} & 0.918 (0.017) & 0.907 (0.015) & 0.912 (0.010) & \textbf{0.981 (0.002)} & 0.957 (0.009) \\
Splice site all & 0.831 (0.012) & 0.950 (0.003) & 0.917 (0.006) & 0.974 (0.004) & \underline{0.975 (0.002)} & 0.935 (0.011) & 0.953 (0.005) & 0.919 (0.005) & \textbf{0.978 (0.001)} & 0.973 (0.002) \\
Splice donor & 0.813 (0.015) & 0.927 (0.003) & 0.894 (0.013) & 0.974 (0.002) & \underline{0.977 (0.007)} & 0.912 (0.009) & 0.930 (0.010) & 0.888 (0.012) & \textbf{0.978 (0.002)} & 0.967 (0.005) \\
\bottomrule
\end{tabular}
}
\label{tab:nucleotide_transformer_tasks}
\end{table*}
% \begin{table*}[!htb]
\small
\renewcommand{\arraystretch}{1.2}
\centering
\caption{Evaluation of the Genomic Benchmarks. The reported values represent the accuracy averaged over 10-fold cross-validation, with the standard error in parentheses.}
\resizebox{\textwidth}{!}{
\begin{tabular}{lcccccccc}
\toprule
& DNABERT-2 & HyenaDNA & NT-v2 & Caduceus-Ph & Caduceus-PS & GROVER & \textbf{Gener}\textit{ator} & \textbf{Gener}\textit{ator}-All \\
& (117M) & (55M) & (500M) & (8M) & (8M) & (87M) & (1.2B) & (1.2B) \\
\midrule
Coding vs. Intergenomic & 0.951 (0.002) & 0.902 (0.004) & 0.955 (0.001) & 0.933 (0.001) & 0.944 (0.002) & 0.919 (0.002) & \textbf{0.963 (0.000)} & \underline{0.959 (0.001)} \\
Drosophila Enhancers Stark & 0.774 (0.011) & 0.770 (0.016) & 0.797 (0.009) & \textbf{0.827 (0.010)} & 0.816 (0.015) & 0.761 (0.011) & \underline{0.821 (0.005)} & 0.768 (0.015) \\
Human Enhancers Cohn & \underline{0.758 (0.005)} & 0.725 (0.009) & 0.756 (0.006) & 0.747 (0.003) & 0.749 (0.003) & 0.738 (0.003) & \textbf{0.763 (0.002)} & 0.754 (0.006) \\
Human Enhancers Ensembl & 0.918 (0.003) & 0.901 (0.003) & 0.921 (0.004) & \textbf{0.924 (0.002)} & \underline{0.923 (0.002)} & 0.911 (0.004) & 0.917 (0.002) & 0.912 (0.002) \\
Human Ensembl Regulatory & 0.874 (0.007) & 0.932 (0.001) & \textbf{0.941 (0.001)} & \underline{0.938 (0.004)} & \textbf{0.941 (0.002)} & 0.897 (0.001) & 0.928 (0.001) & 0.926 (0.001) \\
Human non-TATA Promoters & 0.957 (0.008) & 0.894 (0.023) & 0.932 (0.006) & \textbf{0.961 (0.003)} & \textbf{0.961 (0.002)} & 0.950 (0.005) & \underline{0.958 (0.001)} & 0.955 (0.005) \\
Human OCR Ensembl & 0.806 (0.003) & 0.774 (0.004) & 0.813 (0.001) & \underline{0.825 (0.004)} & \textbf{0.826 (0.003)} & 0.789 (0.002) & 0.823 (0.002) & 0.812 (0.003) \\
Human vs. Worm & 0.977 (0.001) & 0.958 (0.004) & 0.976 (0.001) & 0.975 (0.001) & 0.976 (0.001) & 0.966 (0.001) & \textbf{0.980 (0.000)} & \underline{0.978 (0.001)} \\
Mouse Enhancers Ensembl & \underline{0.865 (0.014)} & 0.756 (0.030) & 0.855 (0.018) & 0.788 (0.028) & 0.826 (0.021) & 0.742 (0.025) & \textbf{0.871 (0.015)} & 0.784 (0.027) \\
\bottomrule
\end{tabular}
}
\label{tab:genomic_benchmarks}
\end{table*}
\begin{table*}[!htb]
\small
\renewcommand{\arraystretch}{1.2}
\centering
\caption{Pre-training data statistics (taxonomic group).}
\begin{tabular}{lccc}
\toprule
Taxonomic group & Number of genes & Number of nucleotides (bp) \\
\midrule
Protozoa & 1,107,499 & 2,034,179,213 \\
Fungi & 6,299,990 & 10,593,031,563 \\
Plant & 7,279,591 & 30,548,451,231 \\
Invertebrate & 7,602,349 & 71,979,748,208 \\
Vertebrate (other) & 10,915,710 & 161,686,774,317 \\
Mammalian & 6,837,553 & 109,406,723,311 \\
\bottomrule
\end{tabular}
\label{tab:pretrain_data_statistics_species}
\end{table*}

\begin{table*}[!htb]
\small
\renewcommand{\arraystretch}{1.2}
\centering
\caption{Pre-training data statistics (gene type).}
\begin{tabular}{lccc}
\toprule
Gene type & Number of genes & Number of nucleotides (bp) \\
\midrule
Protein Coding & 30,883,451 & 340,466,000,000 \\
miscRNA & 342,761 & 6,448,121,770 \\
ncRNA & 4,151,851 & 28,538,620,565 \\
pseudo & 2,322,278 & 10,307,467,170 \\
rRNA & 654,807 & 360,329,916 \\
tRNA & 1,687,523 & 128,397,507 \\
tmRNA & 21 & 9,793 \\
\bottomrule
\end{tabular}
\label{tab:pretrain_data_statistics_gene}
\end{table*}
\begin{table*}[!htb]
\small
\renewcommand{\arraystretch}{1.2}
\centering
\caption{Summary of baseline models.}
\resizebox{\textwidth}{!}{
\begin{tabular}{lccccccc}
\toprule
Model Name & Pre-train Task & Architecture & Tokenizer & Pre-train Data Scope & Pre-train Data Volume (bp) & Context Length (bp) & Parameter Size \\
\midrule
Enformer & Supervised & Transformer & Single Nucleotide & Human, Mouse & 64B & 200K & 252M \\
DNABERT-2 & MLM & Transformer & BPE & Multispecies & 32.5B & 3K & 117M \\
HyenaDNA & NTP & SSM & Single Nucleotide & Human & 3B & 1M & 55M \\
NT-multi & MLM & Transformer & 6-mer & Multispecies & 174B & 6K & 2.5B \\
NT-v2 & MLM & Transformer & 6-mer & Multispecies & 174B & 12K & 500M \\
Caduceus & MLM & SSM & Single Nucleotide & Human & 35B & 131K & 8M \\
GROVER & MLM & Transformer & BPE & Human & 3B & 3K & 87M \\
\textbf{Gener}\textit{ator} & NTP & Transformer & 6-mer & Multispecies (Gene Only) & 386B & 98K & 1.2B \\
\textbf{Gener}\textit{ator}-All & NTP & Transformer & 6-mer & Multispecies & 1.9T & 98K & 1.2B \\
\bottomrule
\end{tabular}
}
\label{tab:baseline_introductuon}
\end{table*}
\begin{table*}[!htb]
\centering
\small
\caption{Hyperparameter settings for the original Nucleotide Transformer tasks.}
\begin{tabular}{@{}l*{20}{c}@{}}
\toprule
 & 
\multicolumn{2}{c}{\shortstack{NT-v2}} & 
\multicolumn{2}{c}{\shortstack{Caduceus-Ph}} & 
\multicolumn{2}{c}{\shortstack{Caduceus-PS}} & 
\multicolumn{2}{c}{\shortstack{GROVER}} & 
\multicolumn{2}{c}{\shortstack{\textbf{Gener}\textit{ator}}} & 
\multicolumn{2}{c}{\shortstack{\textbf{Gener}\textit{ator}-All}} \\ 
\cmidrule(l){2-3} 
\cmidrule(l){4-5} 
\cmidrule(l){6-7} 
\cmidrule(l){8-9} 
\cmidrule(l){10-11} 
\cmidrule(l){12-13}
 & LR & BS & LR & BS & LR & BS & LR & BS & LR & BS & LR & BS \\ 
\midrule
H3 & $2e^{-5}$ & 256 & $5e^{-4}$ & 64 & $1e^{-3}$ & 128 & $1e^{-4}$ & 256 & $1e^{-4}$ & 64 & $2e^{-5}$ & 64 \\
H3K14ac & $1e^{-5}$ & 64 & $2e^{-4}$ & 64 & $5e^{-4}$ & 64 & $5e^{-4}$ & 512 & $2e^{-4}$ & 512 & $5e^{-5}$ & 128 \\
H3K36me3 & $2e^{-5}$ & 64 & $5e^{-4}$ & 128 & $1e^{-3}$ & 256 & $5e^{-5}$ & 64 & $2e^{-4}$ & 512 & $1e^{-5}$ & 64 \\
H3K4me1 & $5e^{-5}$ & 64 & $2e^{-4}$ & 64 & $1e^{-3}$ & 512 & $5e^{-5}$ & 128 & $1e^{-4}$ & 128 & $2e^{-5}$ & 256 \\
H3K4me2 & $5e^{-5}$ & 256 & $2e^{-4}$ & 128 & $5e^{-4}$ & 512 & $1e^{-4}$ & 256 & $5e^{-5}$ & 256 & $1e^{-4}$ & 512 \\
H3K4me3 & $5e^{-5}$ & 128 & $5e^{-4}$ & 256 & $5e^{-4}$ & 256 & $1e^{-4}$ & 128 & $5e^{-5}$ & 64 & $5e^{-5}$ & 64 \\
H3K79me3 & $5e^{-5}$ & 128 & $2e^{-4}$ & 64 & $5e^{-4}$ & 256 & $1e^{-4}$ & 512 & $2e^{-4}$ & 128 & $5e^{-5}$ & 128 \\
H3K9ac & $5e^{-5}$ & 256 & $5e^{-4}$ & 256 & $5e^{-4}$ & 64 & $1e^{-4}$ & 64 & $1e^{-4}$ & 256 & $1e^{-4}$ & 256 \\
H4 & $2e^{-5}$ & 128 & $1e^{-4}$ & 128 & $5e^{-4}$ & 128 & $5e^{-5}$ & 512 & $5e^{-5}$ & 512 & $2e^{-5}$ & 128 \\
H4ac & $5e^{-5}$ & 64 & $5e^{-4}$ & 256 & $2e^{-4}$ & 64 & $5e^{-5}$ & 64 & $1e^{-4}$ & 128 & $5e^{-5}$ & 64 \\
Enhancers & $2e^{-5}$ & 128 & $2e^{-4}$ & 128 & $5e^{-4}$ & 512 & $5e^{-5}$ & 512 & $1e^{-4}$ & 512 & $2e^{-5}$ & 128 \\
Enhancers types & $1e^{-4}$ & 128 & $2e^{-4}$ & 256 & $2e^{-4}$ & 64 & $1e^{-4}$ & 512 & $2e^{-4}$ & 512 & $1e^{-4}$ & 128 \\
Promoter all & $1e^{-5}$ & 64 & $1e^{-4}$ & 64 & $2e^{-4}$ & 64 & $1e^{-4}$ & 64 & $1e^{-5}$ & 64 & $1e^{-5}$ & 64 \\
Promoter non-TATA & $5e^{-5}$ & 256 & $1e^{-3}$ & 512 & $5e^{-4}$ & 512 & $1e^{-4}$ & 256 & $2e^{-5}$ & 64 & $5e^{-5}$ & 256 \\
Promoter TATA & $5e^{-5}$ & 128 & $2e^{-3}$ & 512 & $1e^{-3}$ & 256 & $2e^{-4}$ & 512 & $5e^{-5}$ & 512 & $5e^{-5}$ & 128 \\
Splice sites acceptors & $2e^{-5}$ & 64 & $1e^{-3}$ & 64 & $2e^{-3}$ & 512 & $1e^{-4}$ & 64 & $2e^{-5}$ & 128 & $5e^{-5}$ & 64 \\
Splice sites all & $5e^{-5}$ & 128 & $2e^{-3}$ & 256 & $1e^{-3}$ & 64 & $1e^{-4}$ & 64 & $2e^{-5}$ & 256 & $5e^{-5}$ & 128 \\
Splice sites donors & $5e^{-5}$ & 128 & $1e^{-3}$ & 64 & $1e^{-3}$ & 64 & $5e^{-5}$ & 64 & $1e^{-4}$ & 64 & $5e^{-5}$ & 128 \\ 
\bottomrule
\end{tabular}
\label{tab:benchmark_hyperparam1}
\end{table*}

\begin{table*}[!htb]
\centering
\small
\caption{Hyperparameter settings for the revised Nucleotide Transformer tasks.}
\begin{tabular}{@{}l*{12}{c}@{}}
\toprule
 & 
\multicolumn{2}{c}{\shortstack{NT-v2}} & 
\multicolumn{2}{c}{\shortstack{Caduceus-Ph}} & 
\multicolumn{2}{c}{\shortstack{Caduceus-PS}} & 
\multicolumn{2}{c}{\shortstack{GROVER}} & 
\multicolumn{2}{c}{\shortstack{\textbf{Gener}\textit{ator}}} & 
\multicolumn{2}{c}{\shortstack{\textbf{Gener}\textit{ator}-All}} \\ 
\cmidrule(l){2-3} 
\cmidrule(l){4-5} 
\cmidrule(l){6-7} 
\cmidrule(l){8-9} 
\cmidrule(l){10-11} 
\cmidrule(l){12-13}
 & LR & BS & LR & BS & LR & BS & LR & BS & LR & BS & LR & BS \\ 
\midrule
H2AFZ & $2e^{-5}$ & 64 & $1e^{-5}$ & 64 & $1e^{-3}$ & 256 & $1e^{-5}$ & 128 & $2e^{-5}$ & 128 & $1e^{-4}$ & 128 \\
H3K27ac & $2e^{-5}$ & 128 & $5e^{-4}$ & 64 & $5e^{-4}$ & 64 & $2e^{-5}$ & 128 & $2e^{-5}$ & 256 & $2e^{-5}$ & 128 \\
H3K27me3 & $5e^{-5}$ & 256 & $1e^{-3}$ & 256 & $1e^{-3}$ & 128 & $1e^{-5}$ & 64 & $5e^{-5}$ & 256 & $5e^{-5}$ & 256 \\
H3K36me3 & $5e^{-5}$ & 128 & $2e^{-4}$ & 64 & $1e^{-3}$ & 256 & $5e^{-5}$ & 256 & $1e^{-5}$ & 256 & $5e^{-5}$ & 128 \\
H3K4me1 & $5e^{-5}$ & 512 & $5e^{-4}$ & 256 & $1e^{-3}$ & 256 & $5e^{-5}$ & 128 & $5e^{-5}$ & 64 & $1e^{-5}$ & 64 \\
H3K4me2 & $2e^{-5}$ & 256 & $1e^{-3}$ & 256 & $1e^{-3}$ & 256 & $2e^{-5}$ & 128 & $2e^{-5}$ & 64 & $5e^{-5}$ & 512 \\
H3K4me3 & $2e^{-5}$ & 64 & $1e^{-3}$ & 128 & $5e^{-4}$ & 64 & $1e^{-5}$ & 128 & $2e^{-5}$ & 64 & $5e^{-5}$ & 64 \\
H3K9ac & $5e^{-5}$ & 512 & $5e^{-4}$ & 128 & $1e^{-4}$ & 64 & $1e^{-5}$ & 64 & $2e^{-5}$ & 128 & $5e^{-5}$ & 256 \\
H3K9me3 & $1e^{-4}$ & 128 & $2e^{-4}$ & 128 & $1e^{-3}$ & 512 & $1e^{-5}$ & 64 & $1e^{-4}$ & 64 & $1e^{-4}$ & 128 \\
H4K20me1 & $2e^{-5}$ & 128 & $1e^{-3}$ & 256 & $2e^{-4}$ & 64 & $2e^{-5}$ & 128 & $5e^{-5}$ & 256 & $1e^{-5}$ & 64 \\
Enhancer & $2e^{-5}$ & 256 & $2e^{-4}$ & 128 & $5e^{-4}$ & 512 & $5e^{-5}$ & 256 & $5e^{-5}$ & 64 & $2e^{-5}$ & 64 \\
Enhancer types & $5e^{-5}$ & 512 & $1e^{-3}$ & 64 & $2e^{-4}$ & 128 & $2e^{-5}$ & 128 & $2e^{-5}$ & 64 & $5e^{-5}$ & 64 \\
Promoter all & $1e^{-4}$ & 64 & $5e^{-4}$ & 128 & $5e^{-4}$ & 128 & $1e^{-5}$ & 128 & $1e^{-4}$ & 128 & $5e^{-5}$ & 512 \\
Promoter non-TATA & $2e^{-5}$ & 128 & $5e^{-5}$ & 64 & $5e^{-5}$ & 128 & $2e^{-5}$ & 256 & $2e^{-5}$ & 64 & $1e^{-4}$ & 128 \\
Promoter TATA & $2e^{-4}$ & 128 & $5e^{-4}$ & 64 & $1e^{-3}$ & 128 & $5e^{-5}$ & 64 & $5e^{-5}$ & 128 & $1e^{-4}$ & 64 \\
Splice acceptor & $5e^{-5}$ & 256 & $1e^{-3}$ & 128 & $5e^{-3}$ & 64 & $1e^{-5}$ & 256 & $5e^{-5}$ & 128 & $5e^{-5}$ & 64 \\
Splice site all & $5e^{-5}$ & 256 & $1e^{-3}$ & 64 & $2e^{-3}$ & 64 & $1e^{-5}$ & 64 & $1e^{-4}$ & 256 & $1e^{-4}$ & 128 \\
Splice site donors & $2e^{-5}$ & 128 & $2e^{-3}$ & 64 & $5e^{-3}$ & 64 & $2e^{-5}$ & 128 & $5e^{-5}$ & 128 & $2e^{-5}$ & 64 \\
\bottomrule
\end{tabular}
\label{tab:benchmark_hyperparam2}
\end{table*}


\begin{table*}[!htb]
\centering
\small
\caption{Hyperparameter settings for the Genomic Benchmarks.}
\resizebox{\textwidth}{!}{%
\begin{tabular}{@{}l*{16}{c}@{}}
\toprule
 & 
\multicolumn{2}{c}{\shortstack{DNABERT-2}} & 
\multicolumn{2}{c}{\shortstack{HyenaDNA}} & 
\multicolumn{2}{c}{\shortstack{NT-v2}} & 
\multicolumn{2}{c}{\shortstack{Caduceus-Ph}} & 
\multicolumn{2}{c}{\shortstack{Caduceus-PS}} & 
\multicolumn{2}{c}{\shortstack{GROVER}} & 
\multicolumn{2}{c}{\shortstack{\textbf{Gener}\textit{ator}}} & 
\multicolumn{2}{c}{\shortstack{\textbf{Gener}\textit{ator}-All}} \\ 
\cmidrule(l){2-3} 
\cmidrule(l){4-5} 
\cmidrule(l){6-7} 
\cmidrule(l){8-9} 
\cmidrule(l){10-11} 
\cmidrule(l){12-13} 
\cmidrule(l){14-15} 
\cmidrule(l){16-17}
 & LR & BS & LR & BS & LR & BS & LR & BS & LR & BS & LR & BS & LR & BS & LR & BS \\ 
\midrule
Coding vs. Intergenomic & $1e^{-5}$ & 64 & $1e^{-4}$ & 64 & $5e^{-5}$ & 512 & $2e^{-4}$ & 128 & $2e^{-4}$ & 128 & $5e^{-5}$ & 256 & $2e^{-5}$ & 256 & $1e^{-5}$ & 128 \\
Drosophila Enhancers Stark & $5e^{-5}$ & 512 & $1e^{-3}$ & 512 & $5e^{-5}$ & 64 & $5e^{-4}$ & 64 & $1e^{-3}$ & 64 & $1e^{-4}$ & 64 & $2e^{-4}$ & 256 & $1e^{-5}$ & 128 \\
Human Enhancers Cohn & $5e^{-5}$ & 128 & $5e^{-5}$ & 128 & $5e^{-5}$ & 256 & $2e^{-4}$ & 64 & $2e^{-4}$ & 256 & $2e^{-5}$ & 256 & $1e^{-5}$ & 128 & $2e^{-5}$ & 64 \\
Human Enhancers Ensembl & $1e^{-4}$ & 256 & $5e^{-4}$ & 512 & $5e^{-5}$ & 128 & $5e^{-4}$ & 256 & $2e^{-4}$ & 64 & $5e^{-5}$ & 512 & $5e^{-5}$ & 128 & $1e^{-4}$ & 512 \\
Human Ensembl Regulatory & $5e^{-5}$ & 128 & $1e^{-4}$ & 128 & $2e^{-5}$ & 128 & $5e^{-4}$ & 64 & $5e^{-4}$ & 64 & $1e^{-5}$ & 128 & $1e^{-5}$ & 128 & $2e^{-5}$ & 512 \\
Human NonTATA Promoters & $2e^{-4}$ & 256 & $1e^{-4}$ & 128 & $5e^{-5}$ & 64 & $5e^{-4}$ & 64 & $5e^{-4}$ & 128 & $5e^{-5}$ & 64 & $5e^{-5}$ & 128 & $1e^{-4}$ & 64 \\
Human OCR Ensembl & $1e^{-4}$ & 256 & $5e^{-5}$ & 256 & $5e^{-5}$ & 128 & $1e^{-3}$ & 256 & $5e^{-4}$ & 128 & $5e^{-5}$ & 64 & $1e^{-5}$ & 64 & $5e^{-5}$ & 128 \\
Human vs. Worm & $2e^{-5}$ & 64 & $1e^{-4}$ & 128 & $2e^{-5}$ & 64 & $2e^{-4}$ & 64 & $2e^{-4}$ & 128 & $2e^{-5}$ & 128 & $2e^{-5}$ & 512 & $2e^{-5}$ & 64 \\
Mouse Enhancers Ensembl & $1e^{-5}$ & 64 & $2e^{-4}$ & 128 & $5e^{-5}$ & 64 & $1e^{-3}$ & 64 & $2e^{-4}$ & 128 & $1e^{-5}$ & 128 & $5e^{-4}$ & 512 & $5e^{-5}$ & 64 \\
\bottomrule
\end{tabular}
}
\label{tab:benchmark_hyperparam3}
\end{table*}


\begin{table*}[!htb]
\centering
\small
\caption{Hyperparameter settings for the Gener tasks.}
\resizebox{\textwidth}{!}{%
\begin{tabular}{@{}l*{16}{c}@{}}
\toprule
 & 
\multicolumn{2}{c}{\shortstack{DNABERT-2}} & 
\multicolumn{2}{c}{\shortstack{HyenaDNA}} & 
\multicolumn{2}{c}{\shortstack{NT-v2}} & 
\multicolumn{2}{c}{\shortstack{Caduceus-Ph}} & 
\multicolumn{2}{c}{\shortstack{Caduceus-PS}} & 
\multicolumn{2}{c}{\shortstack{GROVER}} & 
\multicolumn{2}{c}{\shortstack{\textbf{Gener}\textit{ator}}} & 
\multicolumn{2}{c}{\shortstack{\textbf{Gener}\textit{ator}-All}} \\ 
\cmidrule(l){2-3} 
\cmidrule(l){4-5} 
\cmidrule(l){6-7} 
\cmidrule(l){8-9} 
\cmidrule(l){10-11} 
\cmidrule(l){12-13} 
\cmidrule(l){14-15} 
\cmidrule(l){16-17}
 & LR & BS & LR & BS & LR & BS & LR & BS & LR & BS & LR & BS & LR & BS & LR & BS\\ 
\midrule
Gene & $5e^{-5}$ & 64 & $2e^{-4}$ & 128 & $5e^{-5}$ & 128 & $5e^{-4}$ & 128 & $2e^{-4}$ & 64 & $5e^{-5}$ & 128 & $1e^{-5}$ & 64 & $1e^{-4}$ & 64 \\
Taxonomic & $5e^{-5}$ & 64 & $2e^{-4}$ & 64 & $1e^{-4}$ & 512 & $2e^{-4}$ & 128 & $5e^{-4}$ & 256 & $1e^{-4}$ & 128 & $1e^{-5}$ & 128 & $5e^{-6}$ & 64 \\
\bottomrule
\end{tabular}
}
\label{tab:benchmark_hyperparam4}
\end{table*}
\begin{table}[!htb]
\small
\renewcommand{\arraystretch}{1.2}
\centering
\caption{Computational resource usage ({\color{gray}gray color denotes that the model was trained for less than 1 epoch}).}
\begin{tabular}{llrc}
\hline
 & Model & Computational Resources & Data-parallel Size \\ \hline
\multirow{15}{*}{Development} & BPE-512 & 11,551 V100 hours & 32 \\
& BPE-1024 & 10,562 V100 hours & 32 \\
& BPE-2048 & 8,431 V100 hours & 32 \\
& BPE-4096 & 11,155 V100 hours & 64 \\
& BPE-8192 & 8,118 V100 hours & 32 \\
& 2-mer & 28,380 V100 hours & 64 \\
& 3-mer & 15,270 V100 hours & 32 \\
& 4-mer & 11,635 V100 hours & 32 \\
& 5-mer & 9,202 V100 hours & 32 \\
& 6-mer & 9,494 V100 hours & 64 \\
& 7-mer & 8,133 V100 hours & 64 \\
& 8-mer & 7,883 V100 hours & 64 \\
& {\color{gray}Single Nucleotide (llama)} & 1,350 A100 hours & 8 \\
& {\color{gray}Single Nucleotide (mamba)} & 1,465 A100 hours & 8 \\
\hline
\multirow{2}{*}{Pre-train}
& \textbf{Gener}\textit{ator} & 11,793 A100 hours & 32 \\
& \textbf{Gener}\textit{ator}-All & 9,494 A100 hours & 32 \\ \hline
Downstream & 46 runs per task per model & $>$ 64,000 V100 hours & 8 \\ \hline
\end{tabular}
\label{tab:computational_resources}
\end{table}


\end{document}
