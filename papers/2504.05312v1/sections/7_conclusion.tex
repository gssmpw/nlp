% \begin{table}[t]
  \centering
      % \renewcommand{\arraystretch}{1.1}
    \setlength{\tabcolsep}{6pt}
    \fontsize{7 }{9}\selectfont
    \begin{tabular}{ccccccc}
    \toprule
    \multirow{2}[1]{*}{\textbf{top-$\boldsymbol{k}$}} & \multicolumn{2}{c}{2WikiMQA} & \multicolumn{2}{c}{HotpotQA} & \multicolumn{2}{c}{ASQA}\\
       & (acc) & (f1) & (acc) & (f1) & (str-em) & (str-hit) \\
    \midrule
    \rowcolor{gray!20}
    \multicolumn{7}{c}
    {\textbf{Vanilla}} \\
    top-3 & 36.3  & 36.82  & 35.9  & 37.8 & 42.5  & 17.5 \\
    top-5 & \textbf{37.0}  & \textbf{37.45} & 37.6  & 38.16  & 42.78  & 18.14 \\
    top-7 & 35.6 & 35.48 & \textbf{39.8} & \textbf{38.4} & \textbf{43.53} & \textbf{17.93} \\
    \midrule
    \rowcolor{gray!20}
    \multicolumn{7}{c}{\textbf{Ours}} \\
    top-3 & 53.2  & 40.3  & 50.9  & 49.4 & 48.3  & 23.5 \\
    top-5 & \textbf{56.0} & \textbf{42.73} & 52.6  & \textbf{51.13} & 49.7  & 25.2  \\
    top-7 & 55.2  & 41.7  & \textbf{52.8} & 51.02 & \textbf{50.1} & \textbf{25.6} \\
    \bottomrule
    \end{tabular}%
    \vspace{-0.2cm}
      \caption{\textbf{\Ours with different top-$\boldsymbol{k}$ with Qwen2-7b}.}
      % \vspace{-0.1cm}
  \label{tab:Different Top-k}%
\end{table}%
% \begin{table}[t]
  \centering
      % \renewcommand{\arraystretch}{1.1}
    \setlength{\tabcolsep}{6pt}
    \fontsize{7}{9}\selectfont
    \begin{tabular}{ccccccc}
    \toprule
    \multirow{2}[2]{*}{\textbf{max-iter}} & \multicolumn{2}{c}{2WikiMQA} & \multicolumn{2}{c}{HotpotQA} & \multicolumn{2}{c}{ASQA} \\
     & (acc) & (f1) & (acc) & (f1) & (str-em) & (str-hit) \\
    \midrule
    1 & 53.2 & 38.95 & 50.4 & 49.73 & 45.3 & 20.9 \\
    2 & 55.3 & 40.72 & 51.8 & 50.08 & 47.4 & 23.2 \\
    3 & \textbf{56.0} & \textbf{42.73} & \textbf{52.6} & \textbf{51.13} & \textbf{49.7} & \textbf{25.2} \\
 
    \bottomrule
    \end{tabular}%
    \vspace{-0.2cm}
      \caption{\textbf{\Ours with varying max iterations in AIC.}}
      \vspace{-0.4cm}
  \label{tab:Adaptive parameter}%
\end{table}%


\section{Conclusion}
In this paper, we propose a novel RAG method, \Ours, based on memory-adaptive updates. Our approach introduces a collaborative multi-agent memory updating mechanism, combined with an adaptive retrieval feedback iteration and a multi-granular filtering strategy. This design enables efficient information gathering and adaptive updates, significantly improving answer accuracy while reducing hallucinations. We validated \Ours and its core components across several open-domain QA datasets. Extensive experiments
prove the superiority and effectiveness of \Ours  .


% Limitation 可以在 conclusion 后面超过 8 页。
% \noindent\textbf{Limitation.}  
\paragraph{Limitation.}
Although \Ours has made significant progress in open-domain question answering with RAG, there are still some limitations. First, the framework requires multiple fine-tuning steps to train the Multi-granular Content Filter, which necessitates the collection of a substantial amount of data, as well as considerable computational resources and time. Second, since \Ours requires multiple accesses to the LLM, answering each question takes more time compared to the vanilla approach. In future work, we plan to design a more time-efficient and generalizable fine-tuning strategy to improve the quality of open-domain question answering, thereby enhancing the overall effectiveness of \Ours.