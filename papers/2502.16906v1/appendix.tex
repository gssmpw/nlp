\definecolor{bblue}{HTML}{4F81BD}
\definecolor{rred}{HTML}{C0504D}
\definecolor{ggreen}{HTML}{9BBB59}
\definecolor{ppurple}{HTML}{9F4C7C}
\definecolor{darkGreen}{rgb}{0.2,0.5,0.2}
\definecolor{mydarkblue}{rgb}{0,0.08,0.45}
\newcommand{\checkme}[1]{\textcolor{violet}{#1}}
\newcommand{\cellcheck}[0]{\makecell[c]{\checkmark}}


\section{Appendix A}
\label{sec:appendix}

\subsection{Difficulty Control through Data Augmentation}
\label{appendix:en}
The experimental results on the English subset are presented in Figure~\ref{fig:varying-difficulty-en}. Our analysis reveals a clear negative correlation between model performance and two key difficulty metrics: the number of constraints and the solution space size. Specifically, model accuracy demonstrates a consistent decline as the number of constraints increases and as the solution space becomes more constrained. These findings align closely with our previous observations on the Chinese dataset, reinforcing the language-agnostic nature of these performance patterns.

The augmented dataset achieves a more balanced distribution across these difficulty indicators, enabling a more nuanced and comprehensive evaluation of model capabilities. Through systematic manipulation of constraint density and solution space, AutoLogi demonstrates its effectiveness in generating targeted evaluation scenarios with controlled difficulty levels.

\begin{figure*}[!htb]
\centering
\includegraphics[width=1.0\linewidth]{pic/across-varying-difficulties_final_en-cropped.pdf}
\caption{Distribution of questions in AutoLogi's English subset before and after data augmentation, alongside the performance analysis of eight models across varying constraints and solution space proportions.}
\label{fig:varying-difficulty-en}
\end{figure*}

\subsection{Error Analysis}
\label{sec:appendix-error-analysis}
To gain insights into improving model performance in logical reasoning tasks, we conducted a systematic analysis of incorrect responses. We examined 90 samples, one from each background category in AutoLogi's Chinese dataset, focusing on Claude 3.5 Sonnet's responses. Among these samples, the model failed on 37 cases. We categorized these errors into four distinct types:

\begin{enumerate}
\item \textbf{Unanalyzed Condition}: Cases where the model completely overlooked critical conditions in the problem statement.

\item \textbf{Incorrect Logical Inference}: Instances where the model made erroneous deductions when processing specific logical conditions.

\item \textbf{Contradictory Conclusion}: Cases where despite appearing to follow a sound reasoning process, the model's final conclusion contradicted its own analysis.

\item \textbf{Inconsistent Format}: Responses that deviated from the required output format, including issues with naming conventions and data structure consistency.
\end{enumerate}

Detailed examples for each error category are provided in Appendix~\ref{sec:appendix-case}. The distribution of these errors across the 37 failed cases is illustrated in Figure~\ref{fig:erro-types}. Our analysis reveals that Claude 3.5 Sonnet demonstrates strong instruction-following capabilities and processes constraints systematically. Consequently, Inconsistent Format and Unanalyzed Condition errors were relatively rare. The most prevalent issue was \textbf{Incorrect Logical Inference}.

\begin{figure}[!htb]
\centering
\includegraphics[width=0.9\linewidth]{pic/pie_v4png.png}
\caption{Distribution of error types in model responses, analyzing 37 incorrect answers from Claude across 90 test questions.}
\label{fig:erro-types}
\end{figure}

Further investigation into the Incorrect Logical Inference cases revealed two primary challenges:

\begin{enumerate}
\item \textbf{Complex Logic}: Errors emerged when handling logical conditions that required multiple interconnected judgments.

\item \textbf{Misinterpretation of Logical Negations}: Common mistakes included misunderstanding logical operators, particularly in cases involving negations (e.g., confusing "not less than" with "less than").
\end{enumerate}


\subsection{Error Cases}
\label{sec:appendix-case}
Following the analysis of four distinct error types and their distribution in Section~\ref{sec:appendix-error-analysis}, here we present one representative example for each type from the AutoLogi Chinese dataset. These examples also serve as sample problems demonstrating how the system processes logical reasoning questions. While the AutoLogi dataset contains both Chinese and English problems, we focus on Chinese examples here since English problems typically have much longer problem descriptions. For each example, we provide both the original Chinese text and its English translation.

The following examples illustrate: Unanalyzed Condition (Table~\ref{tab:logical-reasoning-example1}), Incorrect Logical Inference (Table~\ref{tab:logical-reasoning-example2}), Contradictory Conclusion (Table~\ref{tab:logical-reasoning-example3}), and Inconsistent Format (Table~\ref{tab:logical-reasoning-example4}).

\begingroup
\begin{table*}[!ht]
\centering
\small
\begin{tabular}{p{\linewidth}}
\toprule
\textbf{\textsc{Prompt (Chinese):}} \begin{CJK*}{UTF8}{gbsn}有7名运动员参加男子5千米的决赛,他们是:S、T、U、W、X、Y和Z。运动员穿的服装不是红色,就是绿色,没有运动员同时到达终点。请你生成一个安排方案满足以下约束:\
(1) 相继到达终点的运动员,他们的服装不全是红色的。\
(2) Y在T和W之前的某一时刻到达了终点。\
(3) 在Y之前到达终点的运动员,恰好有两位穿的是红色服装。\
(4) S是第六个到达终点的运动员。\
(5) Z在U之前的某一时刻到达终点。\end{CJK*}\\
\begin{CJK*}{UTF8}{gbsn}
请你一步一步思考,你的安排方案必须按照为以下输入格式要求来进行回答:
在回答的最后,你需要给出一个输入,以表示你的最终安排方案, 其中 inputs
是一个字典,包含两个键: "order" 和 "colors"。
- inputs["order"] 是一个包含7个元素的 list,元素为运动员的名字(字符串),
表示运动员的到达顺序。所有可能的元素为["S", "T", "U", "W", "X", "Y", "Z"]。
- inputs["colors"] 是一个字典,键为运动员的名字(字符串),值为 "red" 或
"green"(字符串),表示运动员的衣服颜色。
下面提供一个输入样例(注意这只是一个合法输入的例子,不一定是正确方
案):\end{CJK*}\\
\{
"order": ["T", "Y", "Z", "U", "W", "S", "X"],
"colors": \{S": "red", "T": "green", "U": "red", "W": "green", "X": "red",
"Y": "green", "Z": "green"\}
\}
\\
\midrule
\textbf{\textsc{Prompt (English,translated):}} There are 7 athletes participating in the men's 5-kilometer final, and they are: S, T, U, W, X, Y, and Z. The athletes' outfits are either red or green, and no athletes reached the finish line simultaneously. Please generate an arrangement that meets the following constraints:\\
(1) The athletes who reached the finish line consecutively do not all wear red outfits.\
(2) Y reached the finish line at some point before T and W.\
(3) Among the athletes who reached the finish line before Y, exactly two wore red outfits.\
(4) S is the sixth athlete to reach the finish line.\
(5) Z reached the finish line at some point before U.\\
Please think step by step, and your arrangement must be answered according to
the following input format requirements:
At the end of your answer, you need to provide an input to represent your final
arrangement. The inputs should be a dictionary containing two keys: "order" and
"colors".
- `inputs["order"]` is a list containing 7 elements, with each element being the
name of an athlete (string) representing the order in which the athletes reached
the finish line. Possible elements are `["S", "T", "U", "W", "X", "Y", "Z"]`.
- `inputs["colors"]` is a dictionary where the key is the name of an athlete (string)
and the value is "red" or "green" (string), representing the color of the athlete's
outfit.
Below is an example input (note this is just an example of valid input, not
necessarily the correct arrangement):\\
\{
"order": ["T", "Y", "Z", "U", "W", "S", "X"],
"colors": \{S": "red", "T": "green", "U": "red", "W": "green", "X": "red",
"Y": "green", "Z": "green"\}
\}\\
\midrule
\textbf{\textsc{Model Response (Chinese):}} \begin{CJK*}{UTF8}{gbsn}好的,让我们一步步思考来满足所有约束条件:\
1. 首先,我们知道S是第六个到达终点的,所以我们可以先把S放在第六位。\
{\color{darkGreen}2. Y在T和W之前到达终点。\ {\color{blue}(这里进行了简单重复没有分析)}}
3. Z在U之前到达终点,所以Z必须在U前面。\
4. 在Y之前到达终点的运动员,恰好有两位穿的是红色服装。这意味着Y之前的运动员中,必须有两个红色,一个绿色。\
5. 相继到达终点的运动员,他们的服装不全是红色的。这意味着不能有两个连续的红色。\

    根据这些条件,我们可以尝试以下安排:\
    1. X (红色)\
    2. T (绿色)\
    3. Y (绿色)\
    4. Z (红色)\
    5. W (绿色)\
    6. S (红色)\
    7. U (绿色)\
    
    这个安排满足所有条件:\
    - S在第六位\
    - Y在T和W之前\
    - Z在U之前\
    - Y之前有两个红色(X和T)\
    - 没有连续的红色\
    - 总共有3个红色,4个绿色。
    因此,最终的输入格式如下:\end{CJK*} \\
\begin{verbatim}
{"order": ["X", "T", "Y", "Z", "W", "S", "U"],
"colors": {S": "red","T": "green","U": "green","W": "green","X": "red","Y":"green","Z": "red"}}
\end{verbatim}\\
    \midrule
\textbf{\textsc{Model Response(English,translated):}} Let's think step by step to meet all the constraints:\
1. First, we know S is the sixth to reach the finish line\
{\color{darkGreen}2. Y reached the finish line before T and W\ {\color{blue}(This was merely repeated without proper analysis)}}
3. Z reached the finish line before U\
4. Among the athletes before Y, exactly two wore red outfits\
5. No consecutive athletes can wear red\\
Based on these conditions, we can arrange as follows:\
1. X (red)\
2. T (green)\
3. Y (green)\
4. Z (red)\
5. W (green)\
6. S (red)\
7. U (green)\

Final solution in required format:\\
\begin{verbatim}
{"order": ["X", "T", "Y", "Z", "W", "S", "U"],
"colors": {"S": "red", "T": "green", "U": "green","W": "green", "X": "red", "Y": "green","Z": "red"} }
\end{verbatim}\\
\midrule
\textbf{Error Type:} \textbf{Unanalyzed Condition} \\
{\color{blue}There was no analysis of the constraint "(2) Y reached the finish line at some point before T and W".}\\
\bottomrule
\end{tabular}
\caption{Error Case: Unanalyzed Condition}
\label{tab:logical-reasoning-example1}
\end{table*}
\endgroup


\begingroup
\begin{table*}[!ht]
\centering
\small
\begin{tabular}{p{\linewidth}}
\toprule
\textbf{\textsc{Prompt (Chinese):}} \begin{CJK*}{UTF8}{gbsn}某国东部沿海有5个火山岛E、F、G、H、I,它们由北至南排成一条直线。\
请你生成一个安排方案满足以下约束:\
(1) F与H相邻并且在H的北边\
(2) I和E相邻\
(3) G在F的北边某个位置\
(4) G和E相邻\end{CJK*}\\
\begin{CJK*}{UTF8}{gbsn}
请你一步一步思考,你的安排方案必须按照为以下输入格式要求来进行回答:
在回答的最后,你需要给出一个输入,以表示你的最终安排方案, 其中 inputs
是一个包含5个元素的list,元素为字符串,取值范围是["E","F","G","H","I"]。
下面提供一个输入样例(注意这只是一个合法输入的例子,不一定是正确方
案):
['F', 'H', 'I', 'E', 'G']
\end{CJK*}\\
\midrule
\textbf{\textsc{Prompt (English,translated):}} There are 5 volcanic islands, E, F, G, H, and I, located along the eastern coast of a country, arranged in a straight line from north to south.\
Please generate an arrangement that satisfies the following constraints:\
1. F is adjacent to H and is located north of H.\
2. I is adjacent to E.\
3. G is located somewhere north of F.\
4. G is adjacent to E.\\
Please think step by step, and your arrangement must follow the input format
requirements below:
At the end of your answer, you need to provide an input to represent your final
arrangement. The input should be a list containing 5 elements, with each element
being a string from the range `["E", "F", "G", "H", "I"]`.
Below is an example input (note that this is just an example of valid input and may
not be the correct arrangement):
['F', 'H', 'I', 'E', 'G']\\
\midrule
\textbf{\textsc{Model Response (Chinese):}} \begin{CJK*}{UTF8}{gbsn}让我们一步步思考这个问题:\
1. 根据约束(1),F与H相邻并且在H的北边。所以我们知道F和H必须相邻,且F在H的北边。\
2. 根据约束(3),G在F的北边某个位置。这意味着G必须在F的北边,但不一定紧邻F。\
{\color{darkGreen}3. 根据约束(4),G和E相邻。结合约束(3),我们可以推断E必须在G的北边,因为G已经在F的北边了。\color{blue}{(推理错误:G不一定在F北边)}}\
4. 根据约束(2),I和E相邻。考虑到前面的推论,I只能在E的北边或者H的南边。\
5. 综合以上分析,我们可以得出一个可能的排列:G-E-I-F-H 或 I-G-E-F-H。\
6. 考虑到题目要求从北到南排列,I-G-E-F-H 是符合所有约束的一个有效排列。因此,最终的安排方案如下:\end{CJK*}\\
\begin{verbatim}
['I', 'G', 'E', 'F', 'H']
\end{verbatim}\\
\begin{CJK*}{UTF8}{gbsn}
这个排列满足所有给定的约束条件:
- F与H相邻并且在H的北边
- I和E相邻
- G在F的北边
- G和E相邻\end{CJK*}\\

\midrule
\textbf{\textsc{Model Response (English,translated):}} Let's think through this problem step by step:\
1. According to constraint (1), F is adjacent to H and is north of H. So, we know F and H must be adjacent, with F to the north of H.\
2. According to constraint (3), G is located somewhere north of F. This means G must be north of F, but not necessarily immediately adjacent.\
{\color{darkGreen}3. According to constraint (4), G is adjacent to E. Combining this with constraint (3), we can infer that E must be north of G because G is already north of F.\color{blue}{(Logical error: G is not necessarily north of F)}}\
4. According to constraint (2), I is adjacent to E. Considering the previous inferences, I can only be either north of E or south of H.\
5. Combining all the above analysis, one possible arrangement is G-E-I-F-H or I-G-E-F-H.\
6. Considering the requirement that the arrangement is from north to south, I-G-E-F-H is a valid arrangement that satisfies all constraints. 
Therefore, the final arrangement is as follows:
\begin{verbatim}
['I', 'G', 'E', 'F', 'H']
\end{verbatim}
This arrangement satisfies all the given constraints:
- F is adjacent to H and is north of H.
- I is adjacent to E.
- G is north of F.
- G is adjacent to E\\
\midrule
\textbf{Error Type:} \textbf{Incorrect Logical Inference} \\
{\color{blue}Error in step 3 of the reasoning: The conclusion that "E must be north of G" cannot be derived from the premises "G is north of F" and "G is adjacent to E"}\\
\bottomrule
\end{tabular}
\caption{Error Case: Incorrect Logical Inference}
\label{tab:logical-reasoning-example2}
\end{table*}
\endgroup



\begingroup
\begin{table*}[!ht]
\centering
\small
\begin{tabular}{p{\linewidth}}
\toprule
\textbf{\textsc{Prompt (Chinese):}} \begin{CJK*}{UTF8}{gbsn}G、H、J、K、L、M和O七人结伴出游。在他们4天的露营生活中,每天要安排两个人值班为大家做饭。要求这4天的人员安排不能重复,且恰好有一人值班两天。\
请你生成一个安排方案满足以下约束:\
(1) J要排在H值班的次日。\
(2) 值班两天的那个人必须排在第四天,不能排在第三天。\
(3) G必须与J和O中的某一个人排在同一天。\
(4) 第一天或第二天至少有一天安排K。\
(5) O必须排在第三天。\end{CJK*}\\
\begin{CJK*}{UTF8}{gbsn}
请你一步一步思考,你的安排方案必须按照为以下输入格式要求来进行回答:
在回答的最后,你需要给出一个输入,以表示你的最终安排方案,
其中 inputs是一个字典,包含四个键:`"day1"`, `"day2"`, `"day3"`和`"day4"`。
`inputs["day1"]`, `inputs["day2"]`, `inputs["day3"]`, `inputs["day4"]`都是一个包含
两个元素的 set,包含值班人员名字(字符型)。取值范围是['G', 'H', 'J', 'K', 'L',
'M', 'O']。
下面提供一个输入样例(注意这只是一个合法输入的例子,不一定是正确方
案):
\{'day1': \{'K', 'G'\}, 'day2': \{'K', 'H'\}, 'day3': \{'O', 'L'\}, 'day4': \{'J', 'M'\}\}
\end{CJK*}\\
\midrule
\textbf{\textsc{Prompt (English,translated):}} Seven people, G, H, J, K, L, M, and O, go on a trip together. During their 4-day camping, two people are assigned each day to cook for the group. The assignment must not be repeated over the 4 days, and exactly one person must be on duty for two days.\
Please generate an arrangement that satisfies the following constraints:\
1. J must be on duty the day after H.\
2. The person who is on duty for two days must be assigned on the fourth day and cannot be assigned on the third day.\
3. G must be on the same day as either J or O.\
4. K must be scheduled on either the first or the second day.\
5. O must be scheduled on the third day.\\
Please think step by step, and your arrangement must follow the input format
requirements below:
At the end of your answer, you need to provide an input to represent your final
arrangement. The input should be a dictionary containing four keys: "day1", "day2",
"day3", and "day4".
`inputs["day1"]`, `inputs["day2"]`, `inputs["day3"]`, and `inputs["day4"]` are each
*sets* containing two elements representing the names of the people on duty
(*strings*). The range of values is `['G', 'H', 'J', 'K', 'L', 'M', 'O']`.
Below is an example input (note that this is just an example of valid input and may
not be the correct arrangement):
\{'day1': \{'K', 'G'\}, 'day2': \{'K', 'H'\}, 'day3': \{'O', 'L'\}, 'day4': \{'J', 'M'\}\} \\
\midrule
\textbf{\textsc{Model Response (Chinese):}} \begin{CJK*}{UTF8}{gbsn}好的,我会一步步思考来制定一个满足所有约束条件的安排方案。\
1. O必须排在第三天,所以先确定:\
- inputs["day3"] = {'O', 某人}\
{\color{darkGreen}2. J要排在H值班的次日,所以H必须在第1或第2天,J在第2或第3天。但J不能和O同天,所以:\
- inputs["day2"] = {'H', 某人}\
- inputs["day3"] = {'O', 'J'}  \color{blue}{(这里要求'J'安排在第三天,与最终给出的结论是相互矛盾的)}}\
3. 值班两天的人必须在第四天,不能在第三天。G必须和J或O同一天,但G不能是值班两天的人(因为G不在第四天),所以G必须和O同一天:\
- inputs["day3"] = {'O', 'G'}\
- inputs["day4"] = {'J', 值班两天的人}\
4. 第一天或第二天至少有一天安排K。我们可以把K安排在第一天:\
- inputs["day1"] = {'K', 某人}\
5. 剩下L和M还没安排。L可以安排在第一天,M可以安排在第二天:\
- inputs["day1"] = {'K', 'L'}\
- inputs["day2"] = {'H', 'M'}\
6. 现在,值班两天的人只能是M,因为其他人都只值班了一次。
综上所述,最终的安排方案是:
\begin{verbatim}
{'day1': {'K', 'L'}, 'day2': {'H', 'M'}, 'day3': {'O', 'G'}, 'day4': {'J', 'M'}}
\end{verbatim}
这个方案满足了所有给定的约束条件。\end{CJK*}\\
\midrule
\textbf{\textsc{Model Response (English,translated):}} Let's think step by step to create an arrangement that satisfies all the constraints.\
1. O must be scheduled on the third day, so first we set:\
- inputs["day3"] = {'O', someone}\
{\color{darkGreen}2. J must be on duty the day after H, so H must be on the 1st or 2nd day, and J on the 2nd or 3rd day. But J cannot be on the same day as O, so:\
- inputs["day2"] = {'H', someone}\
- inputs["day3"] = {'O', 'J'} \color{blue}{(There's a contradiction between requiring 'J' to be on day 3 here and the final arrangement presented)}}\
3. The person on duty for two days must be on the fourth day, and not on the third day. G must be on the same day as either J or O, but G cannot be the person on duty for two days (because G is not on the fourth day). So G must be with O:\
- inputs["day3"] = {'O', 'G'}\
- inputs["day4"] = {'J', the person on duty for two days}\
4. K must be scheduled on either the first or second day. We can place K on the first day:\
- inputs["day1"] = {'K', someone}\
5. L and M are left. L can be scheduled on the first day, and M on the second day:\
- inputs["day1"] = {'K', 'L'}\
- inputs["day2"] = {'H', 'M'}\
6. Now, the person on duty for two days can only be M, because everyone else is only on duty once.
Therefore, the final arrangement is:
\begin{verbatim}
{'day1': {'K', 'L'}, 'day2': {'H', 'M'}, 'day3': {'O', 'G'}, 'day4': {'J', 'M'}}
\end{verbatim}
This arrangement satisfies all the given constraints.
\\
\midrule
\textbf{Error Type:} \textbf{Contradictory Conclusion} \\
{\color{blue}There's a contradiction: While step 2 of the reasoning process states that 'J' must be on day 3, the final solution puts it on day 4!}\\
\bottomrule
\end{tabular}
\caption{Error Case: Contradictory Conclusion}
\label{tab:logical-reasoning-example3}
\end{table*}
\endgroup





\begingroup
\begin{table*}[!ht]
\centering
\small
\begin{tabular}{p{\linewidth}}
\toprule
\textbf{\textsc{Prompt (Chinese):}} \begin{CJK*}{UTF8}{gbsn}一个委员会工作两年,每年都由 4 人组成,其中 2 名成员来自下面 4 位法官:F、G、H 和 I,另外 2 名成员来自下面 3 位科学家:V、Y 和 Z。每一年,该委员会有 1 名成员做主席。\
请你生成一个安排方案满足以下约束:\
1. 在第一年做主席的成员在第二年必须退出该委员会。在第二年做主席的人在第一年必须是该委员会的成员。\
2. G 和 V 不能在同一年成为该委员会的成员。\
3. H 和 Y 不能在同一年成为该委员会的成员。\
4. 每一年,I 和 V 中有且只有一位做该委员会的成员。\end{CJK*}\\
\begin{CJK*}{UTF8}{gbsn}
请你一步一步思考,你的安排方案必须按照为以下输入格式要求来进行回答:
输入是一个字典,包含四个键:"committee\_year1"、"chairman\_year1"、
"committee\_year2" 和 "chairman\_year2"。
- 在回答的最后,你需要给出一个输入,以表示你的最终安排方案, 其中
`inputs["committee\_year1"]` 是一个包含 4 个元素的 set,包含第一年的委员会
成员的名字。取值范围是`["F","G","H","I","V","Y","Z"]`。
- `inputs["chairman\_year1"]` 是一个字符串,是第一年的主席名字,必须是第
一年的委员。
- `inputs["committee\_year2"]` 是一个包含 4 个元素的 set,包含第二年的委员
会成员的名字。取值范围是 `["F","G","H","I","V","Y","Z"]`。
- `inputs["chairman\_year2"]` 是一个字符串,第二年的主席名字必须是第二年
的委员,并且是第一年的委员。
下面提供一个输入样例(注意这只是一个合法输入的例子,不一定是正确方
案):
\{'committee\_year1': \{'Z', 'V', 'F', 'G'\}, 'chairman\_year1': 'G', 'committee\_year2': \{'Y',
'V', 'F', 'H'\}, 'chairman\_year2': 'F'\}
\end{CJK*}\\
\midrule
\textbf{\textsc{Prompt (English):}} A committee works for two years, and each year it consists of 4 members, including 2 judges from the following: F, G, H, and I, and 2 scientists from the following: V, Y, and Z. Each year, the committee has one member serve as chairman.\
Please generate an arrangement that satisfies the following constraints:\
1. The member who serves as chairman in the first year must leave the committee in the second year. The member who serves as chairman in the second year must be a member of the committee in the first year.\
2. G and V cannot be members of the committee in the same year.\
3. H and Y cannot be members of the committee in the same year.\
4. Each year, either I or V, but not both, must be a member of the committee.\\
Please think step by step, and your arrangement must follow the input format
requirements below:
The input is a dictionary containing four keys: "committee\_year1",
"chairman\_year1", "committee\_year2", and "chairman\_year2".
- At the end of your answer, you need to provide an input representing your final
arrangement, where inputs["committee\_year1"] is a set containing 4 elements,
representing the members of the committee in the first year. The range of values is
["F","G","H","I","V","Y","Z"].
- inputs["chairman\_year1"] is a string representing the chairman of the first year,
and this person must be a committee member in the first year.
- inputs["committee\_year2"] is a set containing 4 elements, representing the
members of the committee in the second year. The range of values is
["F","G","H","I","V","Y","Z"].
- inputs["chairman\_year2"] is a string representing the chairman of the second
year, and this person must be a committee member in both the current and
previous year.
Below is an example input (note that this is just an example of valid input and may
not be the correct arrangement):
\{'committee\_year1': \{'Z', 'V', 'F', 'G'\}, 'chairman\_year1': 'G', 'committee\_year2': \{'Y',
'V', 'F', 'H'\}, 'chairman\_year2': 'F'\} \\
\midrule
\textbf{\textsc{Model Response (Chinese):}} \begin{CJK*}{UTF8}{gbsn}好的, 我会一步步思考来制定一个满足所有约束条件的安排方案。\
第一步:考虑第一年的委员会成员\
- 需要 2 名法官和 2 名科学家\
- G 和 V 不能同时在委员会中\
- H 和 Y 不能同时在委员会中\
- I 和 V 中必须有且只有一位\
考虑到这些约束,我们可以选择:\
- F, I (法官)\
- V, Z (科学家)\
第二步:选择第一年的主席\
从 F, I, V, Z 中选择一位作为主席。我们选择 F 作为第一年的主席。\
第三步:考虑第二年的委员会成员\
- 第一年的主席 F 必须退出\
- 需要包含第二年主席的人选\
- 仍然需要遵守其他约束条件\
考虑到这些, 我们可以选择:\
- G, H (法官)\
- Y, Z (科学家)\
第四步:选择第二年的主席\
从 G, H, Y, Z 中选择一位作为主席,且这个人必须是第一年的委员。唯一符合条件的是 Z。最终的安排方案如下:\end{CJK*}\\
{\color{darkGreen}
\begin{verbatim}
```python
committee_year1 = {"F", "I", "V", "Z"}
chairman_year1 = "F"
committee_year2 = {"G", "H", "Y", "Z"}
chairman_year2 = "Z"
```
\end{verbatim} \color{blue}{(\begin{CJK*}{UTF8}{gbsn}不符合格式要求\end{CJK*})}}\\
\midrule
\textbf{\textsc{Model Response (English):}} Let's think step by step to create an arrangement that satisfies all the constraints.\

Step 1: Consider the members of the committee for the first year\
- We need 2 judges and 2 scientists\
- G and V cannot both be in the committee\
- H and Y cannot both be in the committee\
- One of I or V must be in the committee, but not both\
Considering these constraints, we can select:\
- Judges: F, I\
- Scientists: V, Z\

Step 2: Select the chairman for the first year\
From F, I, V, Z, select one as chairman. We select F as the chairman for the first year.\

Step 3: Consider the members of the committee for the second year\
- The first year's chairman F must leave the committee\
- The second year's chairman must be included\
- Still need to adhere to the other constraints\
Considering these, we can select:\
- Judges: G, H\
- Scientists: Y, Z\

Step 4: Select the chairman for the second year\
From G, H, Y, Z, select one as chairman, and this person must have been a member in the first year.\
The only eligible person is Z.\
The final arrangement is as follows:
{\color{darkGreen}
\begin{verbatim}
```python
committee_year1 = {"F", "I", "V", "Z"}
chairman_year1 = "F"
committee_year2 = {"G", "H", "Y", "Z"}
chairman_year2 = "Z"
```
\end{verbatim} \color{blue}{(Does not meet the format requirements)}}\\
\midrule
\textbf{Error Type:} \textbf{Inconsistent Format} \\
{\color{blue}The answer uses Python assignment syntax, but a dictionary definition is required.}\\
\bottomrule
\end{tabular}
\caption{Error Case: Inconsistent Format}
\label{tab:logical-reasoning-example4}
\end{table*}
\endgroup



\subsection{Prompt Design}
\label{sec:appendix-prompt}
\subsubsection{Stage 1: Puzzle Formulation}
The first stage focuses on extracting two essential components from the source corpus for open-ended logic puzzle generation: Background information and Logical Constraints. To automate this extraction process, we employ GPT-4o with a one-shot learning approach, providing a single comprehensive example. Table~\ref{tab:appendix-stage1-prompt} details the prompt instructions in both Chinese and English.

\begingroup
\begin{table*}[!ht]
    \centering
    \small
    \begin{tabular}{p{\linewidth}}
        \toprule
        \vspace{-2mm}
        \textbf{\textsc{Instruction\color{darkGreen}{(English)}:}} Provide a description of a problem, which contains some background and definition of the problem (usually the first one or two sentences), as well as some logical constraints. You need to divide these two parts, you can reorganize the language, but do not change the original meaning of the problem. \\
        \vspace{-1mm}
        \textbf{\textsc{Input:}} "A supermarket is neatly arranged with 7 rows of shelves from front to back, displaying 7 categories of goods: stationery, snacks, condiments, daily necessities, alcohol, grains and oils, and beverages, each occupying one row. It is known that: \textbackslash{}nThe alcohol is placed before the condiments; \textbackslash{}nThere are 3 rows between the stationery and the condiments; \textbackslash{}nThe grains and oils are after the snacks, with 2 rows in between; \textbackslash{}nThe daily necessities are either in the row before or after the stationery; \textbackslash{}nThe snacks are in the first row." \\
        \vspace{-2mm}
        \textbf{\textsc{Output:}} 
        \begin{verbatim}
```json
{
    "background": "A supermarket is neatly arranged with 7 rows of shelves from front to back, displaying 7 
    categories of goods: stationery, snacks, condiments, daily necessities, alcohol, grains and oils, 
    and beverages, each occupying one row.",
    "logic_constraints": "The alcohol is placed before the condiments; There are 3 rows between the 
    stationery and the condiments; The grains and oils are after the snacks, with 2 rows in between; The 
    daily necessities are either in the row before or after the stationery; The snacks are in the first row."
}
``` \end{verbatim}\\
    \vspace{-1mm}
    \textbf{\textsc{Input:}} {\color{blue} [[[question]]]}  \\
    \vspace{-2mm}
    \textbf{\textsc{Output:}}  \\
    \midrule
    
    \vspace{-1mm}
    \textbf{\textsc{Instruction\color{darkGreen}{(Chinese)}:}} \begin{CJK*}{UTF8}{gbsn}提供一个问题的描述,这个描述中有一些是问题的背景以及定义(一般就是开头一两句),也有一些是逻辑的限制条件,你需要把这两部分划分出来,可以自己重新组织语言,但不要改变原题意。\end{CJK*}\\
    \vspace{-1mm}
    \textbf{\textsc{Input:}} \begin{CJK*}{UTF8}{gbsn}"某超市从前到后整齐排列着7排货架,放置着文具、零食、调料、日用品、酒、粮油和饮料7类商品,每类商品\textbackslash{}n占据一排。已知:\textbackslash{}n酒类排在调料类之前;\textbackslash{}n文具类和调料类中间隔着3排;\textbackslash{}n粮油类在零食类之后,中间隔着2排;\textbackslash{}n日用品类紧挨在文具类前一排或者后一排。\textbackslash{}n零食类排在第1排"\end{CJK*}\\
    \vspace{-2mm}
    \textbf{\textsc{Output:}} 
\begin{verbatim}```json 
{
\end{verbatim}
\begin{CJK*}{UTF8}{gbsn}
\qquad"background":"某超市从前到后整齐排列着7排货架,放置着文具、零食、调料、日用品、酒、粮油和饮料7类商品,每类商品占据一排。", \end{CJK*}\\
    \begin{CJK*}{UTF8}{gbsn}
\qquad"logic\_constraints": "酒类排在调料类之前;文具类和调料类中间隔着3排;粮油类在零食类之后,中间隔着2排;日用品类紧挨在文具类前一排或者后一排;零食类排在第1排。"\end{CJK*}
\begin{verbatim}
}
```
\end{verbatim}
\\
    \vspace{-1mm}
    \textbf{\textsc{Input:}} {\color{blue} [[[question]]]}  \\
    \vspace{-2mm}
    \textbf{\textsc{Output:}}  \\ 
    \bottomrule
    \end{tabular}
    \caption{Instructions for Puzzle Formulation}
    \label{tab:appendix-stage1-prompt}
\end{table*}
\endgroup

\subsection{Stage 2: Generation of Format and Verifiers}
This stage utilizes prompts to instruct GPT-4o to generate two key components based on the previously extracted Background information and Logical Constraints. The first component is the synthesized puzzle, consisting of the Arrangement Format and an Arrangement Example. The second component comprises two code-based verifiers: a Format Verifier to ensure structural correctness and a Constraint Verifier to validate logical consistency. The detailed prompts in both Chinese and English are presented in Figures~\ref{fig:2-en1}, \ref{fig:2-en2}, \ref{fig:2-en3}, \ref{fig:2-cn1}, \ref{fig:2-cn2}, and \ref{fig:2-cn3}.

\subsection{Stage 2: Validation and Regeneration}
A crucial verification step involves the implementation of a Traversal Function to validate the correctness of the generated puzzles. This function employs an enumeration approach to verify the existence of solutions that satisfy both verifiers established in the previous stage. By exhaustively exploring the value ranges of puzzle parameters, the Traversal Function helps ensure the quality and solvability of the synthesized puzzles. We utilize a two-shot prompting approach to guide GPT-4o in generating this function based on the value domains specified in the puzzle. The detailed prompts in both English and Chinese can be found in Figures~\ref{fig:t-en1}, \ref{fig:t-en2}, \ref{fig:t-en3}, \ref{fig:t-cn1}, \ref{fig:t-cn2}, and \ref{fig:t-cn3}.


\subsection{Stage 3: Data Augmentation}
The third stage implements data augmentation through two approaches: Reduction and Expansion. In the Expansion process, we leverage GPT-4o to generate additional constraints by learning from existing ones and their known solution sets. The model analyzes the patterns in current constraints and their corresponding solutions to synthesize new, compatible constraints. To ensure the validity of the augmented puzzles, we employ the Traversal Function for verification, confirming that each expanded puzzle maintains solvability. The complete prompts for this stage are presented in Figures~\ref{fig:3-en1}, \ref{fig:3-en2}, \ref{fig:3-en3}, \ref{fig:3-cn1}, \ref{fig:3-cn2}, and \ref{fig:3-cn3}.

\begin{figure*}[ht!]
    \centering
    \includegraphics[width=\linewidth]{pic/appendix/stage2-en-1-3.pdf}
    \caption{Stage 2 prompt for English Data(1/3). This prompt is designed to simultaneously generate both puzzle requirements (Arrangement Format + Arrangement Example) and program-based verifiers for answer validation, using a one-shot demonstration approach.}
    \label{fig:2-en1}
\end{figure*}
\begin{figure*}[ht!]
    \centering
    \includegraphics[width=\linewidth]{pic/appendix/stage2-en-2-3.pdf}
    \caption{Stage 2 prompt for English Data(2/3). This prompt is designed to simultaneously generate both puzzle requirements (Arrangement Format + Arrangement Example) and program-based verifiers for answer validation, using a one-shot demonstration approach.}
    \label{fig:2-en2}
\end{figure*}
\begin{figure*}[ht!]
    \centering
    \includegraphics[width=\linewidth]{pic/appendix/stage2-en-3-3.pdf}
    \caption{Stage 2 prompt for English Data(3/3). This prompt is designed to simultaneously generate both puzzle requirements (Arrangement Format + Arrangement Example) and program-based verifiers for answer validation, using a one-shot demonstration approach.}
    \label{fig:2-en3}
\end{figure*}
\begin{figure*}[ht!]
    \centering
    \includegraphics[width=\linewidth]{pic/appendix/stage2-cn-1-3.pdf}
    \caption{Stage 2 prompt for Chinese Data(1/3). This prompt is designed to simultaneously generate both puzzle requirements (Arrangement Format + Arrangement Example) and program-based verifiers for answer validation, using a one-shot demonstration approach.}
    \label{fig:2-cn1}
\end{figure*}
\begin{figure*}[ht!]
    \centering
    \includegraphics[width=\linewidth]{pic/appendix/stage2-cn-2-3.pdf}
    \caption{Stage 2 prompt for Chinese Data(2/3). This prompt is designed to simultaneously generate both puzzle requirements (Arrangement Format + Arrangement Example) and program-based verifiers for answer validation, using a one-shot demonstration approach.}
    \label{fig:2-cn2}
\end{figure*}
\begin{figure*}[ht!]
    \centering
    \includegraphics[width=\linewidth]{pic/appendix/stage2-cn-3-3.pdf}
    \caption{Stage 2 prompt for Chinese Data(3/3).This prompt is designed to simultaneously generate both puzzle requirements (Arrangement Format + Arrangement Example) and program-based verifiers for answer validation, using a one-shot demonstration approach.}
    \label{fig:2-cn3}
\end{figure*}


\begin{figure*}[ht!]
    \centering
    \includegraphics[width=\linewidth]{pic/appendix/traverse-en-1-3.pdf}
    \caption{Traversal Function prompt for English Data(1/3). This prompt is designed to generate traversal functions that enumerate all possible solutions, using a two-shots demonstration approach.}
    \label{fig:t-en1}
\end{figure*}
\begin{figure*}[ht!]
    \centering
    \includegraphics[width=\linewidth]{pic/appendix/traverse-en-2-3.pdf}
    \caption{Traversal Function prompt for English Data(2/3). This prompt is designed to generate traversal functions that enumerate all possible solutions, using a two-shots demonstration approach.}
    \label{fig:t-en2}
\end{figure*}
\begin{figure*}[ht!]
    \centering
    \includegraphics[width=\linewidth]{pic/appendix/traverse-en-3-3.pdf}
    \caption{Traversal Function prompt for English Data(3/3). This prompt is designed to generate traversal functions that enumerate all possible solutions, using a two-shots demonstration approach.}
    \label{fig:t-en3}
\end{figure*}
\begin{figure*}[ht!]
    \centering
    \includegraphics[width=\linewidth]{pic/appendix/traverse-cn-1-3.pdf}
    \caption{Traversal Function prompt for Chinese Data(1/3). This prompt is designed to generate traversal functions that enumerate all possible solutions, using a two-shots demonstration approach.}
    \label{fig:t-cn1}
\end{figure*}
\begin{figure*}[ht!]
    \centering
    \includegraphics[width=\linewidth]{pic/appendix/traverse-cn-2-3.pdf}
    \caption{Traversal Function prompt for Chinese Data(2/3). This prompt is designed to generate traversal functions that enumerate all possible solutions, using a two-shots demonstration approach.}
    \label{fig:t-cn2}
\end{figure*}
\begin{figure*}[ht!]
    \centering
    \includegraphics[width=\linewidth]{pic/appendix/traverse-cn-3-3.pdf}
    \caption{Traversal Function prompt for Chinese Data(3/3). This prompt is designed to generate traversal functions that enumerate all possible solutions, using a two-shots demonstration approach.}
    \label{fig:t-cn3}
\end{figure*}

\begin{figure*}[ht!]
    \centering
    \includegraphics[width=\linewidth]{pic/appendix/stage3-en-1-3.pdf}
    \caption{Stage 3 prompt for English Data(1/3). This prompt is designed to augment logical constraints by generating both textual descriptions and corresponding program-based verifiers, using a one-shot demonstration approach.}
    \label{fig:3-en1}
\end{figure*}
\begin{figure*}[ht!]
    \centering
    \includegraphics[width=\linewidth]{pic/appendix/stage3-en-2-3.pdf}
    \caption{Stage 3 prompt for English Data(2/3). This prompt is designed to augment logical constraints by generating both textual descriptions and corresponding program-based verifiers, using a one-shot demonstration approach.}
    \label{fig:3-en2}
\end{figure*}
\begin{figure*}[ht!]
    \centering
    \includegraphics[width=\linewidth]{pic/appendix/stage3-en-3-3.pdf}
    \caption{Stage 3 prompt for English Data(3/3). This prompt is designed to augment logical constraints by generating both textual descriptions and corresponding program-based verifiers, using a one-shot demonstration approach.}
    \label{fig:3-en3}
\end{figure*}
\begin{figure*}[ht!]
    \centering
    \includegraphics[width=\linewidth]{pic/appendix/stage3-cn-1-3.pdf}
    \caption{Stage 3 prompt for Chinese Data(1/3). This prompt is designed to augment logical constraints by generating both textual descriptions and corresponding program-based verifiers, using a one-shot demonstration approach.}
    \label{fig:3-cn1}
\end{figure*}
\begin{figure*}[ht!]
    \centering
    \includegraphics[width=\linewidth]{pic/appendix/stage3-cn-2-3.pdf}
    \caption{Stage 3 prompt for Chinese Data(2/3). This prompt is designed to augment logical constraints by generating both textual descriptions and corresponding program-based verifiers, using a one-shot demonstration approach.}
    \label{fig:3-cn2}
\end{figure*}
\begin{figure*}[ht!]
    \centering
    \includegraphics[width=\linewidth]{pic/appendix/stage3-cn-3-3.pdf}
    \caption{Stage 3 prompt for Chinese Data(3/3). This prompt is designed to augment logical constraints by generating both textual descriptions and corresponding program-based verifiers, using a one-shot demonstration approach.}
    \label{fig:3-cn3}
\end{figure*}



\subsection{Detailed Training Results}
\label{sec:appendix-training-details}
The experimental results presented in Table~\ref{tab:training-set-results-appendix} report the accuracy scores with their corresponding standard deviations (std), calculated across five independent trials.




\begin{table*}[!ht]
\centering
\small
\begin{tabular}{lcccccc}
\toprule
\multirow{2}{*}{\textbf{Model}} & \multicolumn{2}{c}{\textbf{AutoLogi}} & \multirow{2}{*}{\textbf{AR-LSAT}} & \multirow{2}{*}{\textbf{LogiQA}} & \multirow{2}{*}{\textbf{MUSR}} & \multirow{2}{*}{\textbf{LiveBench}} \\ \cmidrule(lr){2-3}
& \textbf{EN} & \textbf{CN} & & & & \\
\midrule
% Qwen2.5-7b-instruct & & & & & & \\
\textit{Baseline} Qwen2.5-7b-instruct & $\text{43.64}{\textcolor{deepgreen}{\pm\text{1.25}}}$ & $\text{42.08}{\textcolor{deepgreen}{\pm\text{1.50}}}$ & $\text{22.70}{\textcolor{deepgreen}{\pm\text{0.84}}}$ & $\text{34.42}{\textcolor{deepgreen}{\pm\text{1.38}}}$ & $\text{47.14}{\textcolor{deepgreen}{\pm\text{0.73}}}$ & $\text{30.67}^{\dagger}$ \\
% \midrule
% \textit{Self-Alignment} & & & & & & \\
+\textit{Self-Alignment}  $DPO$ & $\text{48.80}{\textcolor{deepgreen}{\pm\text{0.91}}}$ & $\text{45.39}{\textcolor{deepgreen}{\pm\text{0.41}}}$ & $\text{26.09}{\textcolor{deepgreen}{\pm\text{2.37}}}$ & $\text{38.05}{\textcolor{deepgreen}{\pm\text{4.00}}}$ & $\text{47.57}{\textcolor{deepgreen}{\pm\text{0.30}}}$ & $\text{35.73}{\textcolor{deepgreen}{\pm\text{2.94}}}$ \\
% \midrule
% \textit{Strong-to-Weak} & & & & & & \\
+\textit{Strong-to-Weak} $RFT$ & $\text{48.33}{\textcolor{deepgreen}{\pm\text{0.63}}}$ & $\text{47.54}{\textcolor{deepgreen}{\pm\text{0.71}}}$ & $\text{27.30}{\textcolor{deepgreen}{\pm\text{0.97}}}$ & $\text{35.93}{\textcolor{deepgreen}{\pm\text{2.59}}}$ & $\text{49.15}{\textcolor{deepgreen}{\pm\text{0.55}}}$ & $\text{30.07}{\textcolor{deepgreen}{\pm\text{2.00}}}$ \\
\midrule
% \textit{Baseline} & & & & & & \\
\textit{Baseline} Qwen2.5-72b-instruct & $\text{68.18}{\textcolor{deepgreen}{\pm\text{0.77}}}$ & $\text{63.92}{\textcolor{deepgreen}{\pm\text{0.56}}}$ & $\text{31.65}{\textcolor{deepgreen}{\pm\text{1.89}}}$ & $\text{47.92}{\textcolor{deepgreen}{\pm\text{1.37}}}$ & $\text{54.21}{\textcolor{deepgreen}{\pm\text{0.41}}}$ & $\text{46.00}^{\dagger}$ \\
% \midrule
% \textit{Self-Alignment} & & & & & & \\
+\textit{Self-Alignment} $DPO$ & $\text{74.79}{\textcolor{deepgreen}{\pm\text{0.41}}}$ & $\text{69.54}{\textcolor{deepgreen}{\pm\text{0.87}}}$ & $\text{38.70}{\textcolor{deepgreen}{\pm\text{1.98}}}$ & $\text{48.14}{\textcolor{deepgreen}{\pm\text{3.98}}}$ & $\text{56.48}{\textcolor{deepgreen}{\pm\text{0.47}}}$ & $\text{52.13}{\textcolor{deepgreen}{\pm\text{1.48}}}$ \\
\bottomrule
\end{tabular}
\caption{Results on various benchmarks. Numbers in parentheses indicate standard deviation over 5 trials. $^{\dagger}$Results reported from LiveBench leaderboard(2024-08-31).}
\label{tab:training-set-results-appendix}
\end{table*}