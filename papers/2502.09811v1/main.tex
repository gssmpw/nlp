%%
%% This is file `sample-manuscript.tex',
%% generated with the docstrip utility.
%%
%% The original source files were:
%%
%% samples.dtx  (with options: `manuscript')
%% 
%% IMPORTANT NOTICE:
%% 
%% For the copyright see the source file.
%% 
%% Any modified versions of this file must be renamed
%% with new filenames distinct from sample-manuscript.tex.
%% 
%% For distribution of the original source see the terms
%% for copying and modification in the file samples.dtx.
%% 
%% This generated file may be distributed as long as the
%% original source files, as listed above, are part of the
%% same distribution. (The sources need not necessarily be
%% in the same archive or directory.)
%%
%% Commands for TeXCount
%TC:macro \cite [option:text,text]
%TC:macro \citep [option:text,text]
%TC:macro \citet [option:text,text]
%TC:envir table 0 1
%TC:envir table* 0 1
%TC:envir tabular [ignore] word
%TC:envir displaymath 0 word
%TC:envir math 0 word
%TC:envir comment 0 0
%%
%%
%% The first command in your LaTeX source must be the \documentclass command.
%%%% Small single column format, used for CIE, CSUR, DTRAP, JACM, JDIQ, JEA, JERIC, JETC, PACMCGIT, TAAS, TACCESS, TACO, TALG, TALLIP (formerly TALIP), TCPS, TDSCI, TEAC, TECS, TELO, THRI, TIIS, TIOT, TISSEC, TIST, TKDD, TMIS, TOCE, TOCHI, TOCL, TOCS, TOCT, TODAES, TODS, TOIS, TOIT, TOMACS, TOMM (formerly TOMCCAP), TOMPECS, TOMS, TOPC, TOPLAS, TOPS, TOS, TOSEM, TOSN, TQC, TRETS, TSAS, TSC, TSLP, TWEB.
% \documentclass[acmsmall]{acmart}

%%%% Large single column format, used for IMWUT, JOCCH, PACMPL, POMACS, TAP, PACMHCI
% \documentclass[acmlarge,screen]{acmart}

%%%% Large double column format, used for TOG
% \documentclass[acmtog, authorversion]{acmart}

%%%% Generic manuscript mode, required for submission
%%%% and peer review
%\documentclass[manuscript, anonymous, screen, review]{acmart}

\documentclass[sigconf]{acmart}

%% Fonts used in the template cannot be substituted; margin 
%% adjustments are not allowed.
%%
%% \BibTeX command to typeset BibTeX logo in the docs
\AtBeginDocument{%
  \providecommand\BibTeX{{%
    Bib\TeX}}}
    
%% Rights management information.  This information is sent to you
%% when you complete the rights form.  These commands have SAMPLE
%% values in them; it is your responsibility as an author to replace
%% the commands and values with those provided to you when you
%% complete the rights form.
% \setcopyright{acmlicensed}
% \copyrightyear{2018}
% \acmYear{2018}
% \acmDOI{XXXXXXX.XXXXXXX}
\copyrightyear{2025}
\acmYear{2025}
\setcopyright{cc}
\setcctype{by-nd}
\acmConference[CHI '25]{CHI Conference on Human Factors in Computing Systems}{April 26-May 1, 2025}{Yokohama, Japan}
\acmBooktitle{CHI Conference on Human Factors in Computing Systems (CHI '25), April 26-May 1, 2025, Yokohama, Japan}\acmDOI{10.1145/3706598.3714230}
\acmISBN{979-8-4007-1394-1/25/04}



%% These commands are for a PROCEEDINGS abstract or paper.
%\acmConference[Conference acronym 'XX]{Make sure to enter the correct conference title from your rights confirmation emai}{June 03--05, 2018}{Woodstock, NY}
%
%  Uncomment \acmBooktitle if th title of the proceedings is different
%  from ``Proceedings of ...''!
%
% \acmBooktitle{Woodstock '18: ACM Symposium on Neural Gaze Detection,
%  June 03--05, 2018, Woodstock, NY} 
% \acmPrice{15.00}
%\acmISBN{978-1-4503-XXXX-X/18/06}


% add commands as needed
%\newcommand{\todo}[1]{{\color{black} \textbf{(Change: #1)}}}
\newcommand{\kexin}[1]{{\color{magenta} (Kexin: #1)}}
\newcommand{\yuhang}[1]{{\color{red} \textbf{(YZ: #1)}}}
\newcommand{\yaxing}[1]{{\color{orange} \textbf{(Yaxing: #1)}}}
\newcommand{\change}[1]{{\color{black} #1}}
\newcommand{\cameraready}[1]{{\color{black} #1}}

%\newcommand{\remove}[1]{{\st{#1}}}
\newcommand{\remove}[1]{}

\graphicspath{{figures/}}

%\acmSubmissionID{4344}
%%
%% Submission ID.
%% Use this when submitting an article to a sponsored event. You'll
%% receive a unique submission ID from the organizers
%% of the event, and this ID should be used as the parameter to this command.
%%\acmSubmissionID{123-A56-BU3}

%%
%% For managing citations, it is recommended to use bibliography
%% files in BibTeX format.
%%
%% You can then either use BibTeX with the ACM-Reference-Format style,
%% or BibLaTeX with the acmnumeric or acmauthoryear sytles, that include
%% support for advanced citation of software artefact from the
%% biblatex-software package, also separately available on CTAN.
%%
%% Look at the sample-*-biblatex.tex files for templates showcasing
%% the biblatex styles.
%%

%%
%% The majority of ACM publications use numbered citations and
%% references.  The command \citestyle{authoryear} switches to the
%% "author year" style.
%%
%% If you are preparing content for an event
%% sponsored by ACM SIGGRAPH, you must use the "author year" style of
%% citations and references.
%% Uncommenting
%% the next command will enable that style.
%%\citestyle{acmauthoryear}


% used package
\usepackage{longtable}
\usepackage{graphicx}
\usepackage{subcaption}
\usepackage{multirow}
\usepackage{array}
\usepackage{booktabs}
\usepackage{geometry}
\usepackage{makecell}
\usepackage{soul}
\usepackage{float}
\renewcommand{\thetable}{\arabic{table}}
%\usepackage[normalem]{ulem} 
\usepackage{afterpage}

%%
%% end of the preamble, start of the body of the document source.
\begin{document}

% \setstcolor{red}
% \setul{}{0.2ex}

%%
%% The "title" command has an optional parameter,
%% allowing the author to define a "short title" to be used in page headers.
\title[Inclusive Avatar Guidelines]{Inclusive Avatar Guidelines for People with Disabilities: Supporting Disability Representation in Social Virtual Reality}

%%
%% The "author" command and its associated commands are used to define
%% the authors and their affiliations.
%% Of note is the shared affiliation of the first two authors, and the
%% "authornote" and "authornotemark" commands
%% used to denote shared contribution to the research.

\author{Kexin Zhang}
\affiliation{
    \institution{University of Wisconsin-Madison}
    \city{Madison}
    \state{Wisconsin}
    \country{USA}
}
\email{kzhang284@wisc.edu}

\author{Edward Glenn Scott Spencer}
\affiliation{
    \institution{Virginia Tech}
    \city{Blacksburg}
    \state{Virginia}
    \country{USA}
}
\email{scottspencer@vt.edu}

\author{Abijith Manikandan}
\affiliation{
    \institution{Virginia Tech}
    \city{Blacksburg}
    \state{Virginia}
    \country{USA}
}
\email{abijith@vt.edu}

\author{Andric Li}
\affiliation{
    \institution{University of California San Diego}
    \city{La Jolla}
    \state{California}
    \country{USA}
}
\email{arl009@ucsd.edu}


\author{Ang Li}
\affiliation{
    \institution{University of Wisconsin-Madison}
    \city{Madison}
    \state{Wisconsin}
    \country{USA}
}
\email{ali253@wisc.edu}


\author{Yaxing Yao}
\affiliation{
    \institution{Virginia Tech}
    \city{Blacksburg}
    \state{Virginia}
    \country{USA}
}
\email{yaxing@vt.edu}

\author{Yuhang Zhao}
\affiliation{
    \institution{University of Wisconsin-Madison}
    \city{Madison}
    \state{Wisconsin}
    \country{USA}
}
\email{yzhao469@wisc.edu}


% \author{Ben Trovato}
% \authornote{Both authors contributed equally to this research.}
% \email{trovato@corporation.com}
% \orcid{1234-5678-9012}
% \author{G.K.M. Tobin}
% \authornotemark[1]
% \email{webmaster@marysville-ohio.com}
% \affiliation{%
%   \institution{Institute for Clarity in Documentation}
%   \streetaddress{P.O. Box 1212}
%   \city{Dublin}
%   \state{Ohio}
%   \country{USA}
%   \postcode{43017-6221}
% }

% \author{Lars Th{\o}rv{\"a}ld}
% \affiliation{%
%   \institution{The Th{\o}rv{\"a}ld Group}
%   \streetaddress{1 Th{\o}rv{\"a}ld Circle}
%   \city{Hekla}
%   \country{Iceland}}
% \email{larst@affiliation.org}

% \author{Valerie B\'eranger}
% \affiliation{%
%   \institution{Inria Paris-Rocquencourt}
%   \city{Rocquencourt}
%   \country{France}
% }

% \author{Aparna Patel}
% \affiliation{%
%  \institution{Rajiv Gandhi University}
%  \streetaddress{Rono-Hills}
%  \city{Doimukh}
%  \state{Arunachal Pradesh}
%  \country{India}}

% \author{Huifen Chan}
% \affiliation{%
%   \institution{Tsinghua University}
%   \streetaddress{30 Shuangqing Rd}
%   \city{Haidian Qu}
%   \state{Beijing Shi}
%   \country{China}}

% \author{Charles Palmer}
% \affiliation{%
%   \institution{Palmer Research Laboratories}
%   \streetaddress{8600 Datapoint Drive}
%   \city{San Antonio}
%   \state{Texas}
%   \country{USA}
%   \postcode{78229}}
% \email{cpalmer@prl.com}

% \author{John Smith}
% \affiliation{%
%   \institution{The Th{\o}rv{\"a}ld Group}
%   \streetaddress{1 Th{\o}rv{\"a}ld Circle}
%   \city{Hekla}
%   \country{Iceland}}
% \email{jsmith@affiliation.org}

% \author{Julius P. Kumquat}
% \affiliation{%
%   \institution{The Kumquat Consortium}
%   \city{New York}
%   \country{USA}}
% \email{jpkumquat@consortium.net}

%%
%% By default, the full list of authors will be used in the page
%% headers. Often, this list is too long, and will overlap
%% other information printed in the page headers. This command allows
%% the author to define a more concise list
%% of authors' names for this purpose.
\renewcommand{\shortauthors}{Zhang et al.}




%%
%% The abstract is a short summary of the work to be presented in the
%% article.
\begin{abstract}
Avatar is a critical medium for identity representation in social virtual reality (VR). However, options for disability expression are highly limited on current avatar interfaces. Improperly designed disability features may even perpetuate misconceptions about people with disabilities (PWD). As more PWD use social VR, there is an emerging need for comprehensive design standards that guide developers and designers to create inclusive avatars. Our work aim to advance the avatar design practices by delivering a set of centralized, comprehensive, and validated design guidelines that are easy to adopt, disseminate, and update. Through a systematic literature review and interview with 60 participants with various disabilities, we derived 20 initial design guidelines that cover diverse disability expression methods through five aspects, including avatar appearance, body dynamics, assistive technology design, peripherals around avatars, and customization control. We further evaluated the guidelines via a heuristic evaluation study with 10 VR practitioners, validating the guideline coverage, applicability, and actionability. Our evaluation resulted in a final set of 17 design guidelines with recommendation levels.%The evaluation confirmed that our guidelines are comprehensive, applicable, and actionable. 
%\change{Based on experts' suggestions, we further refined the guidelines to incorporate practitioners' feedback, resulting in a finalized set of 17 design guidelines.}
%\yuhang{refine the abstract based on intro}

%Avatar is a critical medium for identity representation in the embodied social virtual reality (VR). With the growing presence of people with disabilities (PWD) in social VR, the demand for avatars that can represent disability identity is increasing significantly. However, options for avatar-based disability representations are almost completely missing on mainstream social VR platforms, and it is unclear how to design and develop inclusive avatars that can properly represent PWD. To fill this gap, we presented 20 design guidelines to support disability representations in social VR. Our study had two phases. We first created the guidelines by interviewing 60 PWD with various disabilities to understand their self-representation preferences. We then conducted heuristic evaluations with 10 VR experts to validate the guidelines’ applicability and actionability. This resulted in a comprehensive set of guidelines, covering a wide range of disability representations through multiple aspects of embodied avatars (e.g., appearance, body motion, posture, and interaction). Experts verified the guidelines were highly actionable with concrete avatar examples and comprehensive in including diverse disabilities and avatar designs. Based on the evaluation results, we believe the guidelines can serve as a source to industry practitioners working on the avatar development and design. 

\end{abstract}

%%
%% The code below is generated by the tool at http://dl.acm.org/ccs.cfm.
%% Please copy and paste the code instead of the example below.
%%
\begin{CCSXML}
<ccs2012>
   <concept>
       <concept_id>10003120.10011738</concept_id>
       <concept_desc>Human-centered computing~Accessibility</concept_desc>
       <concept_significance>500</concept_significance>
       </concept>
   <concept>
       <concept_id>10003120.10003121.10003124.10010866</concept_id>
       <concept_desc>Human-centered computing~Virtual reality</concept_desc>
       <concept_significance>500</concept_significance>
       </concept>
 </ccs2012>
\end{CCSXML}
\ccsdesc[500]{Human-centered computing~Accessibility}
\ccsdesc[500]{Human-centered computing~Virtual reality}

%%
%% the work being presented. Separate the keywords with commas.
\keywords{Social virtual realities, avatars, design guidelines, disability representation, diversity and inclusion}


% \received{20 February 2007}
% \received[revised]{12 March 2009}
% \received[accepted]{5 June 2009}

%% This command processes the author and affiliation and title
%% information and builds the first part of the formatted document.
\maketitle

%\section{Introduction}
\section{Introduction}

Despite the remarkable capabilities of large language models (LLMs)~\cite{DBLP:conf/emnlp/QinZ0CYY23,DBLP:journals/corr/abs-2307-09288}, they often inevitably exhibit hallucinations due to incorrect or outdated knowledge embedded in their parameters~\cite{DBLP:journals/corr/abs-2309-01219, DBLP:journals/corr/abs-2302-12813, DBLP:journals/csur/JiLFYSXIBMF23}.
Given the significant time and expense required to retrain LLMs, there has been growing interest in \emph{model editing} (a.k.a., \emph{knowledge editing})~\cite{DBLP:conf/iclr/SinitsinPPPB20, DBLP:journals/corr/abs-2012-00363, DBLP:conf/acl/DaiDHSCW22, DBLP:conf/icml/MitchellLBMF22, DBLP:conf/nips/MengBAB22, DBLP:conf/iclr/MengSABB23, DBLP:conf/emnlp/YaoWT0LDC023, DBLP:conf/emnlp/ZhongWMPC23, DBLP:conf/icml/MaL0G24, DBLP:journals/corr/abs-2401-04700}, 
which aims to update the knowledge of LLMs cost-effectively.
Some existing methods of model editing achieve this by modifying model parameters, which can be generally divided into two categories~\cite{DBLP:journals/corr/abs-2308-07269, DBLP:conf/emnlp/YaoWT0LDC023}.
Specifically, one type is based on \emph{Meta-Learning}~\cite{DBLP:conf/emnlp/CaoAT21, DBLP:conf/acl/DaiDHSCW22}, while the other is based on \emph{Locate-then-Edit}~\cite{DBLP:conf/acl/DaiDHSCW22, DBLP:conf/nips/MengBAB22, DBLP:conf/iclr/MengSABB23}. This paper primarily focuses on the latter.

\begin{figure}[t]
  \centering
  \includegraphics[width=0.48\textwidth]{figures/demonstration.pdf}
  \vspace{-4mm}
  \caption{(a) Comparison of regular model editing and EAC. EAC compresses the editing information into the dimensions where the editing anchors are located. Here, we utilize the gradients generated during training and the magnitude of the updated knowledge vector to identify anchors. (b) Comparison of general downstream task performance before editing, after regular editing, and after constrained editing by EAC.}
  \vspace{-3mm}
  \label{demo}
\end{figure}

\emph{Sequential} model editing~\cite{DBLP:conf/emnlp/YaoWT0LDC023} can expedite the continual learning of LLMs where a series of consecutive edits are conducted.
This is very important in real-world scenarios because new knowledge continually appears, requiring the model to retain previous knowledge while conducting new edits. 
Some studies have experimentally revealed that in sequential editing, existing methods lead to a decrease in the general abilities of the model across downstream tasks~\cite{DBLP:journals/corr/abs-2401-04700, DBLP:conf/acl/GuptaRA24, DBLP:conf/acl/Yang0MLYC24, DBLP:conf/acl/HuC00024}. 
Besides, \citet{ma2024perturbation} have performed a theoretical analysis to elucidate the bottleneck of the general abilities during sequential editing.
However, previous work has not introduced an effective method that maintains editing performance while preserving general abilities in sequential editing.
This impacts model scalability and presents major challenges for continuous learning in LLMs.

In this paper, a statistical analysis is first conducted to help understand how the model is affected during sequential editing using two popular editing methods, including ROME~\cite{DBLP:conf/nips/MengBAB22} and MEMIT~\cite{DBLP:conf/iclr/MengSABB23}.
Matrix norms, particularly the L1 norm, have been shown to be effective indicators of matrix properties such as sparsity, stability, and conditioning, as evidenced by several theoretical works~\cite{kahan2013tutorial}. In our analysis of matrix norms, we observe significant deviations in the parameter matrix after sequential editing.
Besides, the semantic differences between the facts before and after editing are also visualized, and we find that the differences become larger as the deviation of the parameter matrix after editing increases.
Therefore, we assume that each edit during sequential editing not only updates the editing fact as expected but also unintentionally introduces non-trivial noise that can cause the edited model to deviate from its original semantics space.
Furthermore, the accumulation of non-trivial noise can amplify the negative impact on the general abilities of LLMs.

Inspired by these findings, a framework termed \textbf{E}diting \textbf{A}nchor \textbf{C}ompression (EAC) is proposed to constrain the deviation of the parameter matrix during sequential editing by reducing the norm of the update matrix at each step. 
As shown in Figure~\ref{demo}, EAC first selects a subset of dimension with a high product of gradient and magnitude values, namely editing anchors, that are considered crucial for encoding the new relation through a weighted gradient saliency map.
Retraining is then performed on the dimensions where these important editing anchors are located, effectively compressing the editing information.
By compressing information only in certain dimensions and leaving other dimensions unmodified, the deviation of the parameter matrix after editing is constrained. 
To further regulate changes in the L1 norm of the edited matrix to constrain the deviation, we incorporate a scored elastic net ~\cite{zou2005regularization} into the retraining process, optimizing the previously selected editing anchors.

To validate the effectiveness of the proposed EAC, experiments of applying EAC to \textbf{two popular editing methods} including ROME and MEMIT are conducted.
In addition, \textbf{three LLMs of varying sizes} including GPT2-XL~\cite{radford2019language}, LLaMA-3 (8B)~\cite{llama3} and LLaMA-2 (13B)~\cite{DBLP:journals/corr/abs-2307-09288} and \textbf{four representative tasks} including 
natural language inference~\cite{DBLP:conf/mlcw/DaganGM05}, 
summarization~\cite{gliwa-etal-2019-samsum},
open-domain question-answering~\cite{DBLP:journals/tacl/KwiatkowskiPRCP19},  
and sentiment analysis~\cite{DBLP:conf/emnlp/SocherPWCMNP13} are selected to extensively demonstrate the impact of model editing on the general abilities of LLMs. 
Experimental results demonstrate that in sequential editing, EAC can effectively preserve over 70\% of the general abilities of the model across downstream tasks and better retain the edited knowledge.

In summary, our contributions to this paper are three-fold:
(1) This paper statistically elucidates how deviations in the parameter matrix after editing are responsible for the decreased general abilities of the model across downstream tasks after sequential editing.
(2) A framework termed EAC is proposed, which ultimately aims to constrain the deviation of the parameter matrix after editing by compressing the editing information into editing anchors. 
(3) It is discovered that on models like GPT2-XL and LLaMA-3 (8B), EAC significantly preserves over 70\% of the general abilities across downstream tasks and retains the edited knowledge better.
%\section{Related Work}
\section{Related Work and Background}
\subsection{Related Work}
% Few-shot continual relation extraction (FCRE) is a specialized area of relation extraction that focuses on identifying semantic relationships between entity pairs in sentences while addressing the challenge of continuously learning new relations from limited data. A key challenge in FCRE is avoiding \textit{catastrophic forgetting} of previously learned knowledge \citep{THRUN199525, DBLP:journals/neco/FrenchC02} and \textit{overfitting} \citep{hawkins2004problem} as training on limited dataset. 
Most existing FCRE methods \citep{DBLP:conf/acl/WangWH23, hu-etal-2022-improving, DBLP:conf/coling/MaHL024, tran-etal-2024-preserving} have utilized contrastive learning and memory replay techniques to significantly mitigate catastrophic forgetting. However, these approaches largely overlook the present of undetermined relations — relations that are unseen or nonexistent, which remains a critical gap in real-world applications. On the other hand, several methods \citep{WANG2023151, zhao-etal-2025-dynamic, zhao-etal-2023-open, meng-etal-2023-rapl} have considered unknown labels, but their training only relies on available information, including provided entities and relations from the training set, and poorly considers a NOTA (None Of The Above) label for all possible relations that are uncovered. 

Historically, relation extraction research has explored various types of undetermined relations. For example, prior work has defined “no relation (NA)” \citep{xie-etal-2021-revisiting} as sentences with no meaningful relationship between entities, “out-of-scope (OOS)” \citep{liu-etal-2023-novel} as relations outside predefined sets, and “none of the above (NOTA)” \citep{zhao-etal-2023-open} as relations that do not match any known type. While these studies address specific aspects of undetermined relations, their approaches are often simplistic and unrealistic, focusing on single labeled entity pairs rather than considering multiple possible relations within sentences.

Moreover, Open Information Extraction (OIE) has emerged as a powerful tool for open entity and relation extraction, particularly for knowledge graph construction, due to its ability to operate without predefined schemas. Recent studies \citep{li2023evaluating} highlight the strong performance of large language models (LLMs) in OIE tasks. For instance, EDC \citep{zhang-soh-2024-extract} propose an end-to-end pipeline that extracts, defines, and canonicalizes triplets to build knowledge graphs more efficiently. This pipeline includes three phases: (1) Open Information Extraction, where entity-relation triplets are freely extracted from text; (2) Schema Definition, where entity and relation types are defined based on extracted triplets; and (3) Schema Canonicalization, which standardizes relations to fit a target schema. This approach is particularly promising for handling undetermined relations, as it enables the extraction of relations beyond predefined sets.
% Few-shot continual relation extraction is a branch of relation extraction that not only aims to extract semantic relationships between pairs of entities in a sentence but also face a challenge setting that has continuously capture semantic information of new emerging relations from \textit{a small and limited amount data}, while avoiding forgetting knowledge of previously learned ones \textit{catastrophic forgetting} \citep{THRUN199525, DBLP:journals/neco/FrenchC02} and \textit{overfitting} of FCRE models. Recent advancements in few-shot continual relation extraction (FCRE) \citep{DBLP:conf/acl/WangWH23, hu-etal-2022-improving, DBLP:conf/coling/MaHL024, tran-etal-2024-preserving} that utilze constrative learning for presenting protype and memmory replay, that gaim significantly mproved the mitigation of catastrophic forgetting. While these methods contribute to improving continual relation extraction, they largely overlook the challenge of extracting undetermined relations, which remains a crucial gap in real-world applications where numerous relations remain unseen or unlearned. 

% Additionally, look back the history of relation extraction many work already research on handling relation extraction. They defined multiple type of \textbf{undetermined relation}. For instance, prior studies define “no relation (NA)” \cite{xie-etal-2021-revisiting} as sentences that contain no meaningful relation between entities (CITE), “out-of-scope (OOS)” \citep{liu-etal-2023-novel}, “none of the above (NOTA)”\citep{zhao-etal-2023-open} for relations that fall outside the predefined set ,  do not match any known relation type, and   for relations that (CITE). However, these approaches primarily focus on one aspects of \textbf{undetermined relations} is NA or NOTA. They also construct and present method for present these relation is to naive and not realistic, have limit quantity, used only one labeled pair of entities, where we should consider many relation from possible entities in sentences. 

% Besides, Open Information Extraction (OIE) has gained significant attention in entities, relation extraction then knowledge graph construction, due to its ability to leverage large language models (LLMs) without requiring a predefined schema or relation set. Recent studies \citep{li2023evaluating} have demonstrated that LLMs achieve strong performance in OIE tasks, with \citet{zhang-soh-2024-extract} proposing an end-to-end pipeline that extracts, defines, and canonicalizes triplets to construct knowledge graphs more efficiently and with reduced redundancy. This pipeline typically consists of three phases: Open Information Extraction, where entity-relation triplets are extracted freely from text; Schema Definition, where definitions for entity and relation types are generated based on extracted triplets; and Schema Canonicalization, which standardizes relation to relation in given target schema. This approach presents a promising direction for extracting relations beyond predefined schemas, which is particularly relevant for handling undetermined relations in continual relation extraction. By integrating OIE techniques, we can potentially improve FCRE by recognizing triplets that contain relations and give relation that capture semantic align to original sample. Therefore, we consider OIE as a valuable component in our work, both for training data creation and for enhancing relation extraction in scenarios where a large number of undetermined relations emerge dynamically.

% have significantly improved the mitigation of catastrophic forgetting. SCKD  employs a systematic knowledge distillation strategy to preserve prior knowledge while utilizing contrastive learning with pseudo samples to enhance relation differentiation. ConPL integrates a prototype-based classification module, memory-enhanced learning, and distribution-consistent learning to mitigate forgetting, further leveraging prompt learning and focal loss to improve representation learning and reduce class confusion. CPLintroduces a Contrastive Prompt Learning framework, which enhances generalization through prompts and applies margin-based contrastive learning to handle difficult samples. Additionally, it employs memory augmentation with ChatGPT-generated samples to combat overfitting in low-resource settings. MI  takes a novel approach by preserving prior knowledge through often-discarded language model heads, aligning the classification head with backbone knowledge via mutual information maximization. While these methods contribute to improving continual relation extraction, they largely overlook the challenge of extracting undetermined relations, which remains a crucial gap in real-world applications where numerous relations remain unseen or unlearned.

% Some works in traditional relation extraction have addressed the challenge of handling unseen relations. For instance, prior studies define “no relation (NA)” as sentences that contain no meaningful relation between entities (CITE), “out-of-scope (OOS)” for relations that fall outside the predefined set (CITE), and “none of the above (NOTA)” for relations that do not match any known relation type (CITE). However, these approaches primarily focus on some aspects of undetermined relations in standard relation extraction settings and do not adequately consider continual relation extraction, where the dynamic nature of real-world data introduces many unseen relations that remain unlearned.



\subsection{Background}
\subsubsection{Problem Definition}
Few-Shot Continual Relation Extraction (FCRE) requires a model to sequentially acquire new relational knowledge while retaining previously learned information. At each task $t$, the model is trained on a dataset $D^t = \{(x_i^t, y_i^t)\}_{i=1}^{N \times K}$, where $N$ denotes the number of labels provided in the set of relations $R^t$, and $K$ represents the limited number of training instances per relation (i.e., "$N$-way-$K$-shot" paradigm \citet{chen-etal-2023-consistent}). Each training example $(x, y)$ consists of a sentence $x$, which is originally given two entities $(e_h, e_t)$ and the associated relation labels $y \in R^t$. After completing task $t$, previously observed datasets $D^t$ are not extensively reused. The model's final evaluation is conducted on a test set comprising all encountered relations $\tilde{R}^T = \bigcup_{t=1}^{T} R^t$.

Beyond the standard setting and requirements of FCRE, in terms of mitigating forgetting and overfitting, our work aims at designing advanced models, which are capable of continuously capturing and recognizing new relational knowledge, which is not available in the training set.

\subsubsection{Latent Representation Encoding}
One of the fundamental challenges in relation extraction lies in effectively {encoding the latent representation} of input sentences, particularly given that Transformer-based models \citep{vaswani2017attention} produce structured matrix representations. In this study, we adopt an approach inspired by \citet{ma-etal-2024-making}. Given an input sentence $x$ that contains a head entity $e_h$ and a tail entity $e_t$, we transform it into a Cloze-style template $T(x)$ by inserting a \texttt{[MASK]} token to represent the missing relation. The structured template is defined as:

\begin{align}
\begin{aligned}
  T({x}) = \; &x \left[v_{0:n_0-1}\right] e_h \left[v_{n_0:n_1-1}\right] [\texttt{MASK}] \\
  &\left[v_{n_1:n_2-1}\right] e_t \left[v_{n_2:n_3-1}\right].
\label{eq:template}
\end{aligned}
\end{align}

where $[v_i]$ represents learnable continuous tokens, and $n_i$ denotes the respective token positions in the sentence. In our specific implementation, BERT’s \texttt{[UNUSED]} tokens are used for $[v]$. We set the soft prompt length to 3 tokens, with $n_0, n_1, n_2$, and $n_3$ assigned values of 3, 6, 9, and 12, respectively. The transformed input $T(x)$ is then processed through a pre-trained BERT model, encoding it into a sequence of continuous vectors. The hidden representation $z$ of the input is extracted at the position of the \texttt{[MASK]} token:

\begin{equation}
    z = \mathcal{M} \circ T(x)[\text{position}(\texttt{[MASK]})],
\end{equation}

where $\mathcal{M}$ represents the backbone language model. The extracted latent representation is subsequently passed through a multi-layer perceptron (MLP), allowing the model to infer the most appropriate relation for the \texttt{[MASK]} token.
% \subsection{Learning Latent Representation}
% In conventional Relation Extraction scenarios, a basic framework typically employs a backbone PLM followed by an MLP classifier to directly map the input space to the label space using Cross Entropy Loss. However, this approach faces inefficacy in data-scarce settings \cite{snell2017, swersky2017}. Consequently, training paradigms which directly target the latent space, such as contrastive learning, emerge as more suitable approaches. To enhance the semantics-richness of the information extracted from the training samples, two popular losses are often utilized: \textit{Supervised Contrastive Loss} and \textit{Hard Soft Margin Triplet Loss}.

% \subsubsection{Supervised Contrastive Loss}
% To enhance the model’s discriminative capability, we employ the Supervised Contrastive Loss (SCL) \cite{khosla2020}. This loss function is designed to bring positive pairs of samples, which share the same class label, closer together in the latent space. Simultaneously, it pushes negative pairs, belonging to different classes, further apart. Let $z_x$ represent the hidden vector output of sample $x$, the positive pairs $(z_x, z_p)$ are those who share a class, while the negative pairs $(z_x, z_n)$ correspond to different labels. The SCL is computed as follows:

% \begin{equation}
%     \mathcal{L}_{SC}(x) = -\sum_{p \in P(x)} \log \frac{f(z_x, z_p)}{\sum_{u \in D(x)} f(z_x, z_u)}
% \end{equation}

% where $f(x, y) = \exp\left(\frac{\gamma(x,y)}{\tau}\right)$, $\gamma(\cdot, \cdot)$ denotes the cosine similarity function, and $\tau$ is the temperature scaling hyperparameter. $P(x)$ and $D$ denote the sets of positive samples with respect to sample $x$ and the training set, respectively.

% \subsubsection{Hard Soft Margin Triplet Loss}
% To achieve a balance between flexibility and discrimination, the Hard Soft Margin Triplet Loss (HSMT) integrates both hard and soft margin triplet loss concepts \cite{hermans2017, beyeler2017}. This loss function is designed to maximize the separation between the most challenging positive and negative samples, while preserving a soft margin for improved flexibility. Formally, the loss is defined as:

% \begin{equation}
%     \mathcal{L}_{ST}(x) = -\log \left(1 + \max_{p \in P(x)} e^{\xi(x, z_p)} - \min_{n \in N(x)} e^{\xi(x, z_n)} \right),
% \end{equation}

% where $\xi(\cdot, \cdot)$ denotes the Euclidean distance function. The objective of this loss is to ensure that the hardest positive sample is as distant as possible from the hardest negative sample, thereby enforcing a flexible yet effective margin.

% During training, these two losses are aggregated and referred to as the \textit{Sample-based learning loss}:

% \begin{equation}
%     \mathcal{L}_{samp} = \beta_{SC} \cdot \mathcal{L}_{SC} + \beta_{ST} \cdot \mathcal{L}_{ST}
% \end{equation}

% where $\beta_{SC}$ and $\beta_{ST}$ are weighting coefficients.

% \subsection{Undetermined Relation Data Construction}
% In this work, we consider to extract any relation it can be undetermined relation (not any relation or 
% In this work, we create the dataset that contains undetermined relation as real world. 
%\section{Guideline Generation}
\vspace{-5pt}
\section{Method}
\label{sec:method}
\section{Overview}

\revision{In this section, we first explain the foundational concept of Hausdorff distance-based penetration depth algorithms, which are essential for understanding our method (Sec.~\ref{sec:preliminary}).
We then provide a brief overview of our proposed RT-based penetration depth algorithm (Sec.~\ref{subsec:algo_overview}).}



\section{Preliminaries }
\label{sec:Preliminaries}

% Before we introduce our method, we first overview the important basics of 3D dynamic human modeling with Gaussian splatting. Then, we discuss the diffusion-based 3d generation techniques, and how they can be applied to human modeling.
% \ZY{I stopp here. TBC.}
% \subsection{Dynamic human modeling with Gaussian splatting}
\subsection{3D Gaussian Splatting}
3D Gaussian splatting~\cite{kerbl3Dgaussians} is an explicit scene representation that allows high-quality real-time rendering. The given scene is represented by a set of static 3D Gaussians, which are parameterized as follows: Gaussian center $x\in {\mathbb{R}^3}$, color $c\in {\mathbb{R}^3}$, opacity $\alpha\in {\mathbb{R}}$, spatial rotation in the form of quaternion $q\in {\mathbb{R}^4}$, and scaling factor $s\in {\mathbb{R}^3}$. Given these properties, the rendering process is represented as:
\begin{equation}
  I = Splatting(x, c, s, \alpha, q, r),
  \label{eq:splattingGA}
\end{equation}
where $I$ is the rendered image, $r$ is a set of query rays crossing the scene, and $Splatting(\cdot)$ is a differentiable rendering process. We refer readers to Kerbl et al.'s paper~\cite{kerbl3Dgaussians} for the details of Gaussian splatting. 



% \ZY{I would suggest move this part to the method part.}
% GaissianAvatar is a dynamic human generation model based on Gaussian splitting. Given a sequence of RGB images, this method utilizes fitted SMPLs and sampled points on its surface to obtain a pose-dependent feature map by a pose encoder. The pose-dependent features and a geometry feature are fed in a Gaussian decoder, which is employed to establish a functional mapping from the underlying geometry of the human form to diverse attributes of 3D Gaussians on the canonical surfaces. The parameter prediction process is articulated as follows:
% \begin{equation}
%   (\Delta x,c,s)=G_{\theta}(S+P),
%   \label{eq:gaussiandecoder}
% \end{equation}
%  where $G_{\theta}$ represents the Gaussian decoder, and $(S+P)$ is the multiplication of geometry feature S and pose feature P. Instead of optimizing all attributes of Gaussian, this decoder predicts 3D positional offset $\Delta{x} \in {\mathbb{R}^3}$, color $c\in\mathbb{R}^3$, and 3D scaling factor $ s\in\mathbb{R}^3$. To enhance geometry reconstruction accuracy, the opacity $\alpha$ and 3D rotation $q$ are set to fixed values of $1$ and $(1,0,0,0)$ respectively.
 
%  To render the canonical avatar in observation space, we seamlessly combine the Linear Blend Skinning function with the Gaussian Splatting~\cite{kerbl3Dgaussians} rendering process: 
% \begin{equation}
%   I_{\theta}=Splatting(x_o,Q,d),
%   \label{eq:splatting}
% \end{equation}
% \begin{equation}
%   x_o = T_{lbs}(x_c,p,w),
%   \label{eq:LBS}
% \end{equation}
% where $I_{\theta}$ represents the final rendered image, and the canonical Gaussian position $x_c$ is the sum of the initial position $x$ and the predicted offset $\Delta x$. The LBS function $T_{lbs}$ applies the SMPL skeleton pose $p$ and blending weights $w$ to deform $x_c$ into observation space as $x_o$. $Q$ denotes the remaining attributes of the Gaussians. With the rendering process, they can now reposition these canonical 3D Gaussians into the observation space.



\subsection{Score Distillation Sampling}
Score Distillation Sampling (SDS)~\cite{poole2022dreamfusion} builds a bridge between diffusion models and 3D representations. In SDS, the noised input is denoised in one time-step, and the difference between added noise and predicted noise is considered SDS loss, expressed as:

% \begin{equation}
%   \mathcal{L}_{SDS}(I_{\Phi}) \triangleq E_{t,\epsilon}[w(t)(\epsilon_{\phi}(z_t,y,t)-\epsilon)\frac{\partial I_{\Phi}}{\partial\Phi}],
%   \label{eq:SDSObserv}
% \end{equation}
\begin{equation}
    \mathcal{L}_{\text{SDS}}(I_{\Phi}) \triangleq \mathbb{E}_{t,\epsilon} \left[ w(t) \left( \epsilon_{\phi}(z_t, y, t) - \epsilon \right) \frac{\partial I_{\Phi}}{\partial \Phi} \right],
  \label{eq:SDSObservGA}
\end{equation}
where the input $I_{\Phi}$ represents a rendered image from a 3D representation, such as 3D Gaussians, with optimizable parameters $\Phi$. $\epsilon_{\phi}$ corresponds to the predicted noise of diffusion networks, which is produced by incorporating the noise image $z_t$ as input and conditioning it with a text or image $y$ at timestep $t$. The noise image $z_t$ is derived by introducing noise $\epsilon$ into $I_{\Phi}$ at timestep $t$. The loss is weighted by the diffusion scheduler $w(t)$. 
% \vspace{-3mm}

\subsection{Overview of the RTPD Algorithm}\label{subsec:algo_overview}
Fig.~\ref{fig:Overview} presents an overview of our RTPD algorithm.
It is grounded in the Hausdorff distance-based penetration depth calculation method (Sec.~\ref{sec:preliminary}).
%, similar to that of Tang et al.~\shortcite{SIG09HIST}.
The process consists of two primary phases: penetration surface extraction and Hausdorff distance calculation.
We leverage the RTX platform's capabilities to accelerate both of these steps.

\begin{figure*}[t]
    \centering
    \includegraphics[width=0.8\textwidth]{Image/overview.pdf}
    \caption{The overview of RT-based penetration depth calculation algorithm overview}
    \label{fig:Overview}
\end{figure*}

The penetration surface extraction phase focuses on identifying the overlapped region between two objects.
\revision{The penetration surface is defined as a set of polygons from one object, where at least one of its vertices lies within the other object. 
Note that in our work, we focus on triangles rather than general polygons, as they are processed most efficiently on the RTX platform.}
To facilitate this extraction, we introduce a ray-tracing-based \revision{Point-in-Polyhedron} test (RT-PIP), significantly accelerated through the use of RT cores (Sec.~\ref{sec:RT-PIP}).
This test capitalizes on the ray-surface intersection capabilities of the RTX platform.
%
Initially, a Geometry Acceleration Structure (GAS) is generated for each object, as required by the RTX platform.
The RT-PIP module takes the GAS of one object (e.g., $GAS_{A}$) and the point set of the other object (e.g., $P_{B}$).
It outputs a set of points (e.g., $P_{\partial B}$) representing the penetration region, indicating their location inside the opposing object.
Subsequently, a penetration surface (e.g., $\partial B$) is constructed using this point set (e.g., $P_{\partial B}$) (Sec.~\ref{subsec:surfaceGen}).
%
The generated penetration surfaces (e.g., $\partial A$ and $\partial B$) are then forwarded to the next step. 

The Hausdorff distance calculation phase utilizes the ray-surface intersection test of the RTX platform (Sec.~\ref{sec:RT-Hausdorff}) to compute the Hausdorff distance between two objects.
We introduce a novel Ray-Tracing-based Hausdorff DISTance algorithm, RT-HDIST.
It begins by generating GAS for the two penetration surfaces, $P_{\partial A}$ and $P_{\partial B}$, derived from the preceding step.
RT-HDIST processes the GAS of a penetration surface (e.g., $GAS_{\partial A}$) alongside the point set of the other penetration surface (e.g., $P_{\partial B}$) to compute the penetration depth between them.
The algorithm operates bidirectionally, considering both directions ($\partial A \to \partial B$ and $\partial B \to \partial A$).
The final penetration depth between the two objects, A and B, is determined by selecting the larger value from these two directional computations.

%In the Hausdorff distance calculation step, we compute the Hausdorff distance between given two objects using a ray-surface-intersection test. (Sec.~\ref{sec:RT-Hausdorff}) Initially, we construct the GAS for both $\partial A$ and $\partial B$ to utilize the RT-core effectively. The RT-based Hausdorff distance algorithms then determine the Hausdorff distance by processing the GAS of one object (e.g. $GAS_{\partial A}$) and set of the vertices of the other (e.g. $P_{\partial B}$). Following the Hausdorff distance definition (Eq.~\ref{equation:hausdorff_definition}), we compute the Hausdorff distance to both directions ($\partial A \to \partial B$) and ($\partial B \to \partial A$). As a result, the bigger one is the final Hausdorff distance, and also it is the penetration depth between input object $A$ and $B$.


%the proposed RT-based penetration depth calculation pipeline.
%Our proposed methods adopt Tang's Hausdorff-based penetration depth methods~\cite{SIG09HIST}. The pipeline is divided into the penetration surface extraction step and the Hausdorff distance calculation between the penetration surface steps. However, since Tang's approach is not suitable for the RT platform in detail, we modified and applied it with appropriate methods.

%The penetration surface extraction step is extracting overlapped surfaces on other objects. To utilize the RT core, we use the ray-intersection-based PIP(Point-In-Polygon) algorithms instead of collision detection between two objects which Tang et al.~\cite{SIG09HIST} used. (Sec.~\ref{sec:RT-PIP})
%RT core-based PIP test uses a ray-surface intersection test. For purpose this, we generate the GAS(Geometry Acceleration Structure) for each object. RT core-based PIP test takes the GAS of one object (e.g. $GAS_{A}$) and a set of vertex of another one (e.g. $P_{B}$). Then this computes the penetrated vertex set of another one (e.g. $P_{\partial B}$). To calculate the Hausdorff distance, these vertex sets change to objects constructed by penetrated surface (e.g. $\partial B$). Finally, the two generated overlapped surface objects $\partial A$ and $\partial B$ are used in the Hausdorff distance calculation step.

Our goal is to increase the robustness of T2I models, particularly with rare or unseen concepts, which they struggle to generate. To do so, we investigate a retrieval-augmented generation approach, through which we dynamically select images that can provide the model with missing visual cues. Importantly, we focus on models that were not trained for RAG, and show that existing image conditioning tools can be leveraged to support RAG post-hoc.
As depicted in \cref{fig:overview}, given a text prompt and a T2I generative model, we start by generating an image with the given prompt. Then, we query a VLM with the image, and ask it to decide if the image matches the prompt. If it does not, we aim to retrieve images representing the concepts that are missing from the image, and provide them as additional context to the model to guide it toward better alignment with the prompt.
In the following sections, we describe our method by answering key questions:
(1) How do we know which images to retrieve? 
(2) How can we retrieve the required images? 
and (3) How can we use the retrieved images for unknown concept generation?
By answering these questions, we achieve our goal of generating new concepts that the model struggles to generate on its own.

\vspace{-3pt}
\subsection{Which images to retrieve?}
The amount of images we can pass to a model is limited, hence we need to decide which images to pass as references to guide the generation of a base model. As T2I models are already capable of generating many concepts successfully, an efficient strategy would be passing only concepts they struggle to generate as references, and not all the concepts in a prompt.
To find the challenging concepts,
we utilize a VLM and apply a step-by-step method, as depicted in the bottom part of \cref{fig:overview}. First, we generate an initial image with a T2I model. Then, we provide the VLM with the initial prompt and image, and ask it if they match. If not, we ask the VLM to identify missing concepts and
focus on content and style, since these are easy to convey through visual cues.
As demonstrated in \cref{tab:ablations}, empirical experiments show that image retrieval from detailed image captions yields better results than retrieval from brief, generic concept descriptions.
Therefore, after identifying the missing concepts, we ask the VLM to suggest detailed image captions for images that describe each of the concepts. 

\vspace{-4pt}
\subsubsection{Error Handling}
\label{subsec:err_hand}

The VLM may sometimes fail to identify the missing concepts in an image, and will respond that it is ``unable to respond''. In these rare cases, we allow up to 3 query repetitions, while increasing the query temperature in each repetition. Increasing the temperature allows for more diverse responses by encouraging the model to sample less probable words.
In most cases, using our suggested step-by-step method yields better results than retrieving images directly from the given prompt (see 
\cref{subsec:ablations}).
However, if the VLM still fails to identify the missing concepts after multiple attempts, we fall back to retrieving images directly from the prompt, as it usually means the VLM does not know what is the meaning of the prompt.

The used prompts can be found in \cref{app:prompts}.
Next, we turn to retrieve images based on the acquired image captions.

\vspace{-3pt}
\subsection{How to retrieve the required images?}

Given $n$ image captions, our goal is to retrieve the images that are most similar to these captions from a dataset. 
To retrieve images matching a given image caption, we compare the caption to all the images in the dataset using a text-image similarity metric and retrieve the top $k$ most similar images.
Text-to-image retrieval is an active research field~\cite{radford2021learning, zhai2023sigmoid, ray2024cola, vendrowinquire}, where no single method is perfect.
Retrieval is especially hard when the dataset does not contain an exact match to the query \cite{biswas2024efficient} or when the task is fine-grained retrieval, that depends on subtle details~\cite{wei2022fine}.
Hence, a common retrieval workflow is to first retrieve image candidates using pre-computed embeddings, and then re-rank the retrieved candidates using a different, often more expensive but accurate, method \cite{vendrowinquire}.
Following this workflow, we experimented with cosine similarity over different embeddings, and with multiple re-ranking methods of reference candidates.
Although re-ranking sometimes yields better results compared to simply using cosine similarity between CLIP~\cite{radford2021learning} embeddings, the difference was not significant in most of our experiments. Therefore, for simplicity, we use cosine similarity between CLIP embeddings as our similarity metric (see \cref{tab:sim_metrics}, \cref{subsec:ablations} for more details about our experiments with different similarity metrics).

\vspace{-3pt}
\subsection{How to use the retrieved images?}
Putting it all together, after retrieving relevant images, all that is left to do is to use them as context so they are beneficial for the model.
We experimented with two types of models; models that are trained to receive images as input in addition to text and have ICL capabilities (e.g., OmniGen~\cite{xiao2024omnigen}), and T2I models augmented with an image encoder in post-training (e.g., SDXL~\cite{podellsdxl} with IP-adapter~\cite{ye2023ip}).
As the first model type has ICL capabilities, we can supply the retrieved images as examples that it can learn from, by adjusting the original prompt.
Although the second model type lacks true ICL capabilities, it offers image-based control functionalities, which we can leverage for applying RAG over it with our method.
Hence, for both model types, we augment the input prompt to contain a reference of the retrieved images as examples.
Formally, given a prompt $p$, $n$ concepts, and $k$ compatible images for each concept, we use the following template to create a new prompt:
``According to these examples of 
$\mathord{<}c_1\mathord{>:<}img_{1,1}\mathord{>}, ... , \mathord{<}img_{1,k}\mathord{>}, ... , \mathord{<}c_n\mathord{>:<}img_{n,1}\mathord{>}, ... , $
$\mathord{<}img_{n,k}\mathord{>}$,
generate $\mathord{<}p\mathord{>}$'', 
where $c_i$ for $i\in{[1,n]}$ is a compatible image caption of the image $\mathord{<}img_{i,j}\mathord{>},  j\in{[1,k]}$. 

This prompt allows models to learn missing concepts from the images, guiding them to generate the required result. 

\textbf{Personalized Generation}: 
For models that support multiple input images, we can apply our method for personalized generation as well, to generate rare concept combinations with personal concepts. In this case, we use one image for personal content, and 1+ other reference images for missing concepts. For example, given an image of a specific cat, we can generate diverse images of it, ranging from a mug featuring the cat to a lego of it or atypical situations like the cat writing code or teaching a classroom of dogs (\cref{fig:personalization}).
\vspace{-2pt}
\begin{figure}[htp]
  \centering
   \includegraphics[width=\linewidth]{Assets/personalization.pdf}
   \caption{\textbf{Personalized generation example.}
   \emph{ImageRAG} can work in parallel with personalization methods and enhance their capabilities. For example, although OmniGen can generate images of a subject based on an image, it struggles to generate some concepts. Using references retrieved by our method, it can generate the required result.
}
   \label{fig:personalization}\vspace{-10pt}
\end{figure}
%\section{Description of Initial Guidelines}
\section{Description of Initial Guidelines}

Building upon the knowledge from our literature review and interview study, 
%from prior works, %%literature and application review, %yuhang{did we do the literature review?} \kexin{we did the literature review to inform the structure of interview protocol; I added this detail in method.}
we derive 20 design guidelines for inclusive avatar design for PWD. Our guidelines cover a broad range of disability expression methods across five aspects, including avatar appearance (G1), body dynamics (G2), assistive technology design (G3), peripherals around avatars (G4), and customization controls in the avatar interface (G5). 

To ensure actionability, each guideline has three components: a detailed description, quote examples from PWD, and concrete avatar feature examples to demonstrate the guideline implementation. %To ensure each guideline is actionable and applicable}, we provide a detailed description, quote examples from PWD, and concrete avatar feature examples to demonstrate its implementation. 
Appendix Table \ref{tab:overview_original} %\yuhang{fix}
presents a summarized version of our initial guidelines. %(%full version of the initial guidelines are in Appendix Table \ref{tab:full_original}; 
%revised guidelines after evaluation can be found in Table \ref{tab:overview_revised} and Appendix Table \ref{tab:full_revised}; Appendix Table \ref{tab:changes} demonstrated the changes of guidelines).} %\yuhang{fix})}. 
In this section, we elaborate on each guideline and the rationales, \change{grounded in both prior literature and findings from our interview study}. % and the rationale behind each guideline. 

% remember to compare and contrast; sharpen the thoughts
% first sentence of each sections/subsections should be hitting to the point -> what's the most interesting thing in this section 

\subsection{Avatar Body Appearance (G1)}
\change{Customizing avatar body appearance is the most common way to express disabilities. But deciding how to represent disabilities via avatar body is challenging, as it involves multiple design considerations (e.g., body compositions, customization of each body part). We derive guidelines to inform suitable avatar body design for PWD.
}
%\st{Our findings identified participants' needs in expressing disability identities on avatars, with the majority of them preferred authentic, realistic self-representation, echoing the prior works. Moving beyond prior work, we systematically uncovered the design dimensions of how to represent disabilities authentically through avatar's body appearance, as discussed below.} 

\textbf{\textit{G1.1. Support disability representation in social VR avatars.}} 
All participants desired the options to represent their disabilities via avatars, \change{echoing the findings from prior works \cite{kelly2023, zhang2022, zhang2023}}. As P52 described: \textit{``I think the biggest thing for me is the flexibility and the freedom to choose [how I can represent myself.]}

%\remove{However, such options are largely missing on existing avatar interfaces, blocking PWD from representing their identities through avatars. For instance, P52 wanted to add her prosthesis on avatar but could not find an option in the interface: \textit{``I think the biggest thing for me is the flexibility and the freedom to choose [how I can represent myself.] I would like to represent myself [in social VR] as an accurate reflection of what I look like. For what I've seen online, there is no standard option in the avatar creation [to represent my prosthesis]. I think that would be great to see if such a thing existed.''}}
%\yaxing{I think in this part of the results, we should focus primarily on the describing the guidelines and no need to touch on what developers and design should do - that's more of implication. This is also for consistency reason. Check with Yuhang on this one @yuhang}

\change{\textbf{\textit{Guideline}}: Avatar interfaces should allow all users to flexibly express their identities and present their disabilities ~\cite{zhang2022, kelly2023, assets_24}. The disability features should not be blocked behind the paywall ~\cite{kelly2023}.} %To ensure PWD to express their disability identities, avatar interfaces should provide more avatar body options to represent diverse disabilities. %This could be achieved through a variety of methods, such as providing diverse assistive devices options in the interface for users to easily add on (P18, P32) or enabling self-upload avatar for personalized representation (P4). 

%Emerging technology such as AI-generated avatars form photos can also be incorporated into VR technology \cite{Scorzin2023}, empowering users to have accurate avatar representations wit h minimum customization efforts (P7, P12, P40).
%. For example, several participants (e.g., P18, P32) wanted the interface to provide a variety of assistive device options so that they could easily add on to their avatars, or the option for users to upload their own custom avatars for more personalized representation (P4). Six participants (e.g., P7, P12, P40) were interested in AI-generated avatars from selfies, enabling them to have accurate avatar representations with minimum customization efforts.


\textbf{\textit{G1.2 Default to full-body avatars to enable diverse disability representation across different body parts.}}
Almost all participants preferred a full-body avatar, \change{echoing prior insights from Mack et al. \cite{kelly2023}}. In our study, more than half of the participants had a disability that affected the below-waist body area, which could only be reflected via full-body avatars. P9 indicated: \textit{``people can only get my whole disability identity with the full body [avatar].''}
%Almost all participants preferred full-body avatar, which provided them space to reflect disabilities that influenced different body parts. Specifically, more than half of our participants managed a disability that affected them from the waist downward, and only full-body avatars could express such disabilities authentically. P9 indicated: \textit{``people can only get my whole disability identity with the full body.''} 

In addition, multiple participants (e.g., P2, P34, P52) wanted to express their lived experiences as PWD, which could only be achieved \change{through the behaviors of full-body avatars}. As P34 described: \textit{``I prefer a full-body avatar, because [it shows] how [people with visual impairments] navigate the pathway, how they move the hands and the fingers, and how they read braille.''} %\st{Everything can be oriented to the people.}''} 

\change{\textbf{\textit{Guideline}}:} Avatar interfaces should offer full-body avatar options \cite{kelly2023}. Given the dominant preferences for full-body avatars over others (e.g., upper-body only, or head and hands only), we recommend making it the default or the starting avatar template, giving users the maximum flexibility to further customize their avatars as they prefer. 

\textbf{\textit{G1.3 Enable flexible customization of body parts as opposed to using non-adjustable avatar templates.}}
\change{Similar to prior work ~\cite{zhang2022, kelly2023}}, a third of participants (e.g., P3, P37, P52) preferred matching the avatar body with their authentic self, requesting full customization of the avatar body. \change{Moreover, our participants highlighted the need for asymmetric designs of avatar body parts (P4, P9). For example, P9 wanted avatars with different eye appearance to reflect her amblyopia (i.e., lazy eyes): \textit{``[I want] the opportunity to move or articulate the eyeballs to represent my disability, because my right eye is litter lazier than the other one.''}}
%For example, P3 desired to customize the avatar limbs to represent her amputation: st{\textit{``I have a limb difference, [and I am] a left forearm amputation. So I would really like to see characters where they don't have two long arms, but they have one long standard arm and one shorter [arm], like to the elbow. Or even if they do have two long arms, but one of them doesn't have a hand, or just one round-off around the wrist instead of extending completely with a five fingered hand.''}} \st{However, the existing avatar platforms mostly have ``standard'' body appearances without any adjustable features, and multiple participants (e.g., P9, P50) complained that they could hardly find any options to represent themselves accurately.} 

\change{\textbf{\textit{Guideline}}: Avatar interfaces should provide PWD sufficient flexibility to customize each avatar body part \cite{kelly2023}. While the customization spans a wide range, the most commonly mentioned body parts to customize include (1) avatar height, (2) body shape, (3) limbs (i.e., number of limbs, length and strength of each limb), and (4) facial features (e.g., mouth shape, eye size). Asymmetrical design options of body parts (e.g., eyes, ears) should also be available, %To address these needs, developers and designers should include features that enable users to customize each body parts of their avatar, such as options to customize the presence, length, and strength of each limb (P3, P12, P14, P56). Asymmetrical design options of body parts (e.g., eyes, ears) should also be available to fully represent body, 
such as changing size and direction of each eyeball to reflect disabilities like strabismus.}

%\yuhang{follow the same structure across all guidelines. In G1.2, one paragraph for concrete evidence and one paragraph for in-depth summary of the guideline. Currently, the second paragraph is commonly missing in most guidelines. Refer to the formal description of each guide since the language is quite dedicated.}

%For example, avatar platforms should offer options to customize the presence, length, and strength of each limb (P3, P12, P14, P56). Asymmetrical design options of body parts (e.g., eyes, ears) should also be available to fully represent body, such as allowing users to select the size and detailed look of each eyeball to reflect disabilities like strabismus (P4, P9). 


\textbf{\textit{G1.4 Prioritize human avatars to emphasize the ``humanity'' rather than the ``disability'' aspect of identity.}}
Approximately half of participants chose \change{\textit{human}} avatars to stress disability as an inherent part of personal identity. Multiple participants (e.g., P9, P31, P46) reported real-life experiences of being degraded and wanted to use human avatars to express that they should be seen as ``a whole person, not the disability'' (P46). Additionally, the human avatars also let users to represent multiple and intersectional identities (e.g., age, gender, and race) all together as an integral human being. As P9 emphasized: \textit{``Humanoid avatars show me as a whole person. I identify as a black woman with a disability, and that's really important when discussing a personal identity, because when describing somebody, you wouldn't just say, ‘Oh, they have a disability,’ instead you would say, ‘Oh, they're non-binary or female, and they're African American or Caucasian, and they have a disability.’ They all go together.''} %\st{In this sense, the human form of avatars set up the fundamental base for authentic representation.}
%, with disabilities as an important part of it. Many participants (e.g., P5, P33, P57) also reported feeling more connected to humanoid avatars when representing personal identities in social VR. 

\change{\textbf{\textit{Guideline:}} Social VR applications should offer human avatar options whenever the application theme allows. %Even in certain VR applications where human avatars do not fit (e.g., \textit{Among Us}), practitioners should considering adding human design options should be reflect certain  do not fit  It is well adapted to various social VR contexts (e.g., multi-player games, collaboration platforms, education) and compatible with diverse avatar styles (e.g., the abstract style in Roblox, the cartoony avatars in Horizon Worlds).
} %\kexin{HOLD - use cases: Developers think this guideline dependents on use cases (e.g., Among us may not be suitable), so I want to expand on that a bit more. I wonder how do we incorporate developers' feedback when revising guidelines - do we mention them as evidence with P-ID? It sounds a bit off and break the flow if we just mention the use cases directly.}

\textbf{\textit{G1.5 Provide non-human avatar options to free users from social stigma in real life.}} %\kexin{re comment: this guideline applies to all disabilities, not only limited to the invisible ones.}
\remove{Unlike those who authentically indicated disabilities with accurate details,} Eleven participants (e.g., P17, P43, P49) preferred non-human avatars, such as robots or animal characters, to avoid disclosing disabilities and protect themselves from the judgment they often faced in real life. \remove{They perceived social VR as a utopia where they can escape from real life, and avatars in non-humanoid forms, such as robotics or animal avatars, allowed them to avoid the social judgments and norms commonly tied to disabilities.}  
For example, \remove{P45 said: \textit{``[Non-human avatars] hide more of my real disabilities, being the opposite of showing off my disabilities.''}} P49, who identified as neurodivergent, chose a robotic avatar to lower social expectations: \textit{``It feels like people's expectations of how neurotypical I'm gonna seem are lowered if I'm a robot or just a non-human avatar.''} 
%Meanwhile, non-humanoid avatars still allow PWD to express themselves and represent their disability identities in a symbolic way. For example, P17 used animal avatars as spiritual animal to demonstrate her autistic traits: \textit{``Well, I just feel like [the animal avatars of lions] relates to what I'm passing through.''} 

\change{\textbf{\textit{Guideline:}} Besides human avatars, avatar interfaces should also provide diverse forms of non-human avatars, empowering PWD to choose the one they relate with flexibly.}


\subsection{Avatar Dynamics: Facial Expressions, Posture, and Body Motion (G2)}
Unlike 2D interfaces, the embodied and multi-modal nature of avatars in social VR enabled PWD to express themselves through diverse approaches beyond static appearance. \change{We derive guidelines based on how PWD leveraged avatar dynamics, such as facial expressions, posture, and body motions, to represent disabilities.} % through facial expressions, posture, and body motion.
 
\textbf{\textit{G2.1 Allow simulation or tracking of disability-related behaviors but only based on user preference.}}
Nine participants (e.g., P7, P47) wanted their avatars to reflect the realistic behaviors caused by disability for a stronger connection (e.g., limp by P18, stumbling by P4). %\yuhang{example behaviors that are suitable to present with participant number, limps?}). 
However, \change{similar to Gaulano et al. \cite{chronic_pain_gualano_2024}}, eight participants (e.g., P6, P14, P49) were concerned that showing disability-related behaviors would reinforce stigma (e.g., involuntary behaviors like nervous tics). \remove{Participants preferred avatars to reflect behaviors with their controls. For example, P6 didn't want her avatar to reflect disability-related movements at all to avoid misconception, as she explained: \textit{``The way I move authentically is kind of jaggy, and I swerve. People asked me if I'm drunk all the time. So I'd like to go as quickly as I can in a smooth way [...] even though that's not authentic.''}} \change{Participants highlighted the need for controlling what behaviors to track or simulate (e.g., P14, P16, P49).} As P14 indicated: \textit{``I think it would be cool if you could choose to have [the movement to be reflected]. But I also think there is a fine line between inclusion and offensive imitation.''} 

\change{\textbf{\textit{Guideline:}} Users should be able to control the extent of behavior tracking in social VR. With the advance of motion tracking techniques, avatar platforms may disable subtle behavior tracking by default to avoid disrespectful simulation, but allow users to easily adjust the tracking granularity for potential disability expression.}

%should be careful not to create simulations that may cause misunderstanding or reinforce stereotypes, and they should only do so if PWD prefer it. % empowering PWD to decide how they prefer to express the behavioral characteristics. 

%For example, P49 worried showing her nervous tics would cause confusions thus preferred smoother reflections instead of full simulation: \textit{``I have nervous tics that are kind of full body shutters. When I do those in real life, the VR avatar does often follow those, which makes it hard for people to figure out if it's glitching out or something. So finding ways to make those smoother and more reflective of reality, rather than like, `is this internet thing? or what's happening?' ''} Other participants, like P6, didn't want her avatar to reflect disability-related movements at all, as she explained: \textit{``The way I move authentically is kind of jaggy, and I swerve. People asked me if I'm drunk all the time. So I'd like to go as quickly as I can in a smooth way [...] even though that's not authentic.''} Developers and designers should be careful not to create simulations that may cause misunderstanding or reinforce stereotypes, and they should only do so if users prefer it, empowering PWD to decide how they prefer to express the behavioral characteristics. 


\textbf{\textit{G2.2 Enable expressive facial animations to deliver invisible status.}}
A third of participants (e.g., P1, P4, P40) desired to express disabilities through avatar facial expressions. This is particularly important for people with invisible disabilities, whose conditions mostly surface through emotions and subtle non-verbal cues. %\st{For instance, participants (e.g., P7, P44, P47) with autism used the direction and focus of the avatar's eyes looking away as a way to express their autistic identity, as P47 described: \textit{``a big thing [that] a lot of people on the autism spectrum struggled with [is making] eye contacts.''}}
Three participants (P4, P40, P51) noted that their disabilities involved rapid fluctuation or contradicting feelings (e.g., bipolar disorder, ADHD), thus preferring avatars to show a spectrum of facial expressions. For example, \change{P51 experienced multiple invisible disabilities (i.e., depression and ADHD) and used different facial expressions to represent different aspects of their disabilities}: \textit{``When representing depression, the facial expression is more sad or in thought. When having ADHD moments, [the avatar] being more excited or manic.''} 

\change{\textbf{\textit{Guideline:}} Avatar platforms should enable diverse facial expressions, allowing PWD to express emotion, portray mental status, and indicate fluctuation of invisible disabilities \cite{assets_24}.  %For example, the five basic emotions (i.e., anger, fear, sadness, disgust, enjoyment) \cite{ccp_basic_emotions} could be a starting point. We encourage practitioners to expand and diversify based on their unique use scenarios.
}

\textbf{\textit{G2.3 Prioritize equitable capability and performance over authentic simulation.}}
Four participants (P7, P14, P15, P52) highlighted that the avatar performance should demonstrate equitable capabilities to other users, not being limited by their disabilities or direct motion tracking. %Although some participants (e.g., P7, P14, P15, P52) wanted their avatars to authentically reflect how they move or behave in real life, they particularly mentioned that their avatar's actual performance and capabilities should not be limited by the behavior's characteristics.  
For example, P52 mentioned that the moving speed of her avatar walking with limps should not be slower than other avatars: \textit{``I walk with a slight limp, [but] I don't think I need the actual movement [on avatars] to reflect how [fast] I walk. \remove{[Because] when I use games, I see the movement aspect more of a practicality than part of the game [...]} 
I think having a limp would be cool, but I wouldn't want to be slower than [other avatars]. \remove{I wouldn't want to have a maximum speed, because I chose to have a limp earlier in the avatar making process [...]} 
Being able to just keep up with peers’ [avatars], pace-wise, would be the most important thing.''} 

\change{\textbf{\textit{Guideline:}} PWD value equitable and fair interaction experiences more than the authentic disability expression. Therefore, avatar platforms should ensure the same level of capabilities and performance for all avatars, regardless of whether disability features or behaviors are involved.}


\textbf{\textit{G2.4 Leverage avatar posture to demonstrate PWD's lived experiences.}}
In addition to the facial expressions and body movements, five participants (P4, P9, P31, P33, P34) preferred leveraging avatar postures to demonstrate their lived experiences. For example, as a person with low vision, P4 described his unique posture when interacting with others: \textit{``[My] vision is directed at one angle. So my head is turned lightly, because I'm not looking at people directly all the time.''} Representing postures and mannerisms on avatars also help increase awareness and resolve misunderstandings about disabilities, as P34 shared: \textit{``Instead of looking at the person who is speaking, people with visual impairments take their ears nearby to the place where the sound is coming from. This gives some wrong impressions to the people that the visually impaired people have not given attention to the speakers. That is not the real story.''}

\change{\textbf{\textit{Guideline:}} Disabilities can be expressed via avatar posture. Avatar platforms should enable certain posture tracking or simulation (e.g., unique facing directions of individuals with low vision during conversation) to enable authentic disability representation.}

%This informs practitioners the opportunities to use avatar posture as a design medium for disability expressions. With the benefits of facilitating social interactions for PWD, we see that the posture representation could be particularly helpful for life-like avatars and platforms that centered on interactions and communications.

\subsection{Assistive Technology Design (G3)}
Adding assistive technologies to avatars is a key method adopted by PWD for disability disclosure ~\cite{zhang2022, kelly2023}. \change{Beyond prior literature, we revealed key aspects of assistive technologies, such as types, appearances, and relationship to avatars, to guide proper designs.} 
%how to design assistive technologies properly to avoid misconception and misuse in the social VR setting. In the following, we outline key design aspects of assistive technologies that developers and designers should consider. 

\textbf{\textit{G3.1 Offer various types of assistive technology to cover a wide range of disabilities.}} 
 Like prior work has indicated \cite{zhang2022,kelly2023}, we found that multiple participants (e.g., P18, P33, P39) viewed assistive technologies as part of their body. P39 described the meaning of wheelchairs to wheelchair users: \textit{``\remove{For people in wheelchairs, our wheelchair is an extension of our body.} We view it emotionally as an extension of ourselves, and it gives us our independence.''} \remove{P18 and P33 also reflected that being able to have avatar with assistive technologies they used in daily life made them feel empowering and being included in the social VR.}

\change{\textbf{\textit{Guideline:}}} Avatar interfaces should offer assistive technologies that are commonly used by PWD \cite{zhang2022}. \change{The most desired types of assistive technologies include: (1) mobility aids (e.g., wheelchair, cane, and crutches); (2) prosthetic limbs; (3) visual aids (e.g., white cane, glasses, and guide dog); (4) hearing aids and cochlear implants; and (5) health monitoring devices (e.g., insulin pumps, ventilator, smart watches). % \yuhang{narrow down}. 
Practitioners should consider including at least these five categories of assistive technologies in avatar interfaces.
In addition, due to PWD's different technology preferences \cite{kelly_AI24}, we encourage practitioners to offer more than one assistive technology option in each category, for example, including guide dog, white cane, and glasses for visual aids.} 

\textbf{\textit{G3.2 Allow detail customization of assistive technology for personalized disability representation.}}
Eleven participants (e.g., P15, P18, P32) desired to better convey their personalities through assistive technology customization. \change{Echoing prior research ~\cite{zhang2022, kelly2023}, changing the colors of assistive technologies and attaching personalized decorations (e.g., add stickers on wheelchair, P18) are two most preferred customization options.}
%PWD viewed assistive technologies as part of their body \cite{zhang2022, kelly2023}, and many of them customize it to convey their personalities (e.g., P38, P45, P46).  
%customize the design of assistive technology to represent their disability in more diverse and personalized way. 
\remove{With assistive technology being an extension of the user’s body, being able to have diverse customization options of assistive technologies is as important as customizing the avatar’s appearance, as P39 said: \textit{``Making some more customization in the wheelchair [is] in the same way that you make customization for eye color, nose shape, [and] all those things.''}} However, some participants (e.g., P20, P50) %\yuhang{add more example}
\change{desired to see more various styles of assistive technologies, such as a futuristic styled hoverchair (P20).}
%emphasized the need for a wider range of customization options, such as xx \yuhang{such as???}. % so that they can choose the one they feel connected with. 
\remove{as current avatar platforms offer only limited default choices.}  
\remove{As P52 expressed frustration over the lack of w heelchair variations in social VR avatars: \textit{``[Now] you either have a wheelchair or no wheelchair, but you can't customize the type, shape, or any various add-ons. Like is it motorized [wheelchair]? Is it like a manual one? So I think having the ability to choose what additional features you'd like to add would be really nice.''}} 

\change{\textbf{\textit{Guideline:}} Avatar platforms should allow customizations for assistive technology \cite{kelly2023,zhang2022}. Basic customization options should include adjusting the colors of different assistive technology components and adding decorations (e.g., stickers, logos) to them. More customization could be added based on specific use cases.} 

%should be customizable Instead of only having one default choice, practitioners should ensure that assistive technologies features are customizable. Based on findings from  prior works \cite{zhang2022, kelly2023} and our large-scale interview, changing the color of assistive technologies and attaching personalized decorations to them (e.g., add stickers on wheelchair, P18) are two of the most preferred customization options among PWD. We suggest practitioners to incorporate these two as the base customization level and add on based on specific use cases.}
%such as adjusting the color (e.g., P18, P39, P44) or adding personalized decoration such as stickers on wheelchairs (e.g., P9, P45, P49). 


\textbf{\textit{G3.3 Provide high-quality, authentic simulation of assistive technology to present disability respectfully and avoid misuse.}}
Four participants (P4, P6, P34, P35) preferred high-quality assistive technology simulation with authentic details similar to those in real life. They were concerned that inaccurate assistive technology designs in social VR may depict misleading figures of PWD and lead to misuse, echoing Zhang et al. \cite{zhang2023}. As P35 described: 
%\st{wanted the assistive technologies to have high-fidelity looking with realistic details. As an emerging social platforms, participants found it was not uncommon to see some avatars with disability features, such as avatars on wheelchairs. However, these avatars were usually poorly designed with low-quality or stereotypical manner, leading to the misuse of assistive technology and perpetuation towards PWD. For instance, P35 recalled seeing poor wheelchair representation in social VR where people treated it as trolling or memeing:}
\textit{``[I’ve seen] really poor representation [of wheelchairs]. They're usually joke avatars or meme avatars that have wheelchairs.''}

\change{\textbf{\textit{Guideline:}} To avoid misunderstandings or misuse, the assistive technolgy simulation should convey standardized, authentic details of the real-world assistive devices \cite{zhang2023}, regardless the overall avatar style. For example, the design of a white cane should show the details of tip and follow its standardized color selection, no matter the design style is photorealistic or cartoon. 
%(P34). 
We recommend practitioners to model assistive technologies by following their established design standards, such as design guidelines for white canes \cite{who_white_canes}, wheelchairs \cite{russotti_ansi_wheelchairs}, and hearing devices \cite{ecfr_800_30}.}
\remove{Their designs should be high in quality and contain sufficient details, so that users can tell these avatars were invested with great efforts, aiming for identity representation instead of trolling (P6, P34). For example, the design of a white cane for people with low vision should show the details of tip and follow the standardized color selection for such walking aids (P34).} %: \textit{``While walking, the tip of the white cane should move like the pendulum motion, [moving] forth and back in that way''} (P34).
%\kexin{consider changing the visual examples for this guideline? I think the key of this guideline is to follow the conventional design standards of AT that is true-to-life, instead of how realistic/high-fidelity the AT features are (in which some developers think this guideline is limited to styles and only apply to those with realistic avatars). A better example might be that both pixel-style white cane and realistic-style white cane can follow the guidelines, as long as they have authentic details of how does a white cane look like in real life (i.e., red tip, black handle, a straight thin tube). The style (e.g., realistic or abstract) does not matter, but the correct looking of AT does -> HOLD: how to convey this / cite evidence} \yaxing{this is a good point. I think you should add the point about "instead of how realistic the AT are" in the description.}

\textbf{\textit{G3.4 Focus on simulating assistive technology that empower PWD rather than highlighting their challenges.}} %\kexin{I think this guideline can be merged to G3.3., as we only have one wheelchair example for it, and I have a hard time to generalize it to other AT designs (many developers also asked to diversify examples for this one). If we agree to merge, this one can be an example of true-to-life design by not having medicalized design of AT.}
%Although participants preferred diverse types of assistive technologies, 
Eight participants (e.g., P18, P39) only wanted to add assistive technology features that can demonstrate their independence instead of challenges, (e.g., hospital wheelchair vs. power wheelchair). %Participants felt frustrated to be misportrayed as being dependent or incapable when using assistive technologies. 
For example, P18 found media often misrepresented PWD by showing them sitting in a hospital-style wheelchair that requires others' assistance to move: \textit{``Most of the representations we see in fiction, video games and TV, they always use hospital chairs, which are not practical. No actual disabled person uses a hospital chair in real life, which has armrests and big push handles, because it's built for somebody to push you. However, a manual wheelchair is designed for you to push yourself.''}
\remove{Participants wanted to correct the media misrepresentation by showing how they can achieve independence through the use of assistive technology. Taking the wheelchairs as examples, P6 noted the power chair should have a joystick to show the user can move independently; and P39 strongly preferred manual wheelchair without any pushable handles to demonstrate the self-independence: \textit{``It's important to me that it doesn't look like I'm ready to be pushed by someone else. I'm stating that independence [achieved through wheelchair]. I'm solidly myself, and I don't need another person. This is a big deal in our community [...] we’re not going to want push handles.''}}

\change{\textbf{\textit{Guideline:}} When determining what assistive technology features to offer, practitioners should only select assistive or medical devices that can be easily controlled by PWD to demonstrate their capability (e.g., manual wheelchair, cane) and leave out the ones that PWD cannot independently use or the ones that highlight their challenges (e.g., hospital-style wheelchair, bedridden avatars).}

\textbf{\textit{G3.5 Demonstrate the liveliness of PWD through dynamic interactions with assistive technology.}}
In addition to the visual details, five participants (P4, P6, P18, P39, P44) \change{wanted their avatars to actively interact with the assistive technologies, such as rolling their manual wheelchair (P18) or sweeping their cane (P34) when moving, to demonstrate their capability and liveliness}. %demonstrated independence by showing how they actively control their assistive technologies. To achieve that, participants noted that the interaction with assistive technology should authentically simulate what they look like in real life. 
%For example, P18 \change{wanted the \yuhang{his?her?} avatar to display a circular arm movement while pushing the manual wheelchair, and P34 preferred avatar to control the white cane in pendulum motion, 
As P34 mentioned: \textit{``While my avatar is walking, the tip of white cane should be moving back and forth like a pendulum motion.''}
\remove{\textit{``I think having the option to roll [wheelchair] would be good. I’ve seen some 3D models of wheelchair users in video games, and their arms don't move while they're rolling, which is really weird to me. Because I push myself with my hands.''}} %P39 noted that proper postures for avatars using assistive technologies, like sitting up tall in a wheelchair, can also reflect the liveliness and capability of people with disabilities.

\change{\textbf{\textit{Guideline:}} Beyond providing assistive technology options, social VR platforms should enable suitable interactions between avatars and assistive technology. The interactions should authentically reflect PWD's real-world usage of their assistive technology, such as how a blind user sweeps their cane, or how a wheelchair user moves their arms to control their wheelchair.  %developing assistive technologies, practitioners should ensure the avatar could demonstrate how PWD actively control and interact with the assistive technologies in real life.
}

\textbf{\textit{G3.6 Avoid overshadowing the avatar body with assistive technology.}}
Seven participants (e.g., P9, P18, P57) demanded to flexibly adjust the size of assistive technologies to fit their avatar body. \change{They emphasized that, while expressing disabilities, the image curation should focus on the whole avatar rather than just the assistive technology. As P39 highlighted: \textit{``The wheelchair is not the focus of the image; [rather,] the focus is on the avatar having a good time.''}} \change{P18 recalled their %\yuhang{her?his?} \kexin{P18 is non-binary and use they/their}
prior experience of being overshadowed by the assistive technology}: \textit{``
\remove{I think that having the option to actually make the chair larger or smaller, depending on how large or small your avatar is, is a good detail. Because sometimes wheelchairs don't fit you.}I have encountered 3d models where the wheelchair is so big and the person sitting in it is so small, and it just doesn't look right.''}

\change{\textbf{\textit{Guideline:}} The size of assistive technology should not dominate the avatar body but rather fit the body size. Avatar platforms should automatically match the assistive technology model to different avatar body sizes, and allow users to adjust the size of assistive technology to achieve the preferred avatar-aid ratio. The combination of avatar and assistive technology should also be seamless without affecting the quality and aesthetics of the original avatar ~\cite{kelly2023}.}


\subsection{Peripherals around Avatars (G4)}
Beyond the design of avatars, the peripheral space around them can also be leveraged for disability expression. We explored this new design space and identified design guidelines.

\textbf{\textit{G4.1 Provide suitable icons, logos, and slogans that represent disability communities.}}
Sixteen participants (e.g., P5, P37, P58) desired %represented disabilities symbolically by incorporating 
representative icons, logos, or slogans of disability communities for identity expressions, and they wanted to creatively attach these symbols to a variety of places, such as on avatar's clothing (P53, P56), accessories (P13, P54), or even the space surrounding the avatars (P14, P47). \change{This confirm previous insights that disability-related symbols can help PWD educate other users and raise awareness in the social VR space \cite{zhang2022, assets_24}.} 
%By wearing the community icons or logos, participants not only showed support to the disability community they connected with but also raised disability awareness in the social VR space. 
\remove{For example, multiple participants mentioned using the rainbow infinity icon to represent the autism community (e.g., P1, P46, P47), zebra printing for rare disease (P56), and sunflowers that symbolize interac tion invisible disabilities community (P5, P14). For example, P14 planned to attach a sunflower yard in the background of her avatar to symbolize her invisible disabilities. %, as she pictured: \textit{``It’s about hidden disabilities, and you can get sunflower lanyards, which is a way of saying ‘I'm disabled, but you can't tell.’ So if the avatar looks like they're walking and they've got sparkly, flowy sunflowers behind them, [that] would be cool.''} 
Another participant, P3, would like to have disability community icons on the avatar's T-shirt to show community pride.} %: \textit{``Now we have avatars who can wear T-shirts with the LGBTQ plus pride flag on it, or they can wear T-shirts that have ‘Black Lives Matter.’ So having equivalent things for disability would be awesome.''} These examples suggested that avatar platforms should include some widely recognized icons, logos, and slogans representing diverse disability communities in the avatar interface.}

\change{\textbf{\textit{Guideline:}} Awareness-building items (e.g., logos, slogans) should be provided, allowing users to attach them to various areas on or around the avatars \cite{assets_24, zhang2022}. %, such as the apparel, accessories, and assistive technologies. 
Some widely recognized and preferred symbols that represent different disabilities for practitioners to refer to include (1) the rainbow infinity symbol that represents the autism community \cite{assets_24, rainbow_infinity_symbol}, (2) the sunflower that represents hidden disabilities \cite{isit_assets24, hidden_disability_sunflower}, (3) the disability pride flag \cite{disability_pride_flag}, (4) the spoons, symbolizing spoon theory for people with chronic illness \cite{kelly2023, assets_24}, and (5) the zebra symbols for rare diseases \cite{assets_24, Gualano_2023}. %We recommend practitioners to include these symbols in their avatar interface and allow PWD to flexibly attach them to multiple avatar parts.
}

\textbf{\textit{G4.2 Leverage spaces beyond the avatar body to present disabilities.}}
%While the symbols provided participants a standard and easy way to represent their disabilities, 
Eight participants (e.g., P37, P43, P46) wanted to express disabilities more creatively and flexibly through the space behind the avatar body. \change{This is especially favored by people with invisible disabilities, as it helps visualize PWD's mental conditions. For example, P43 wanted a visual indicator of a cloudy and rainy background to symbolize her depression and anti-social mode at the moment. This echoes prior implications that an avatar's background can provide contexts into PWD's experiences \cite{kelly2023, assets_24}.}
%People with invisible and fluctuating disabilities particularly preferred this approach, as it provided indicators to visualize their frequently changing conditions in social scenarios, keeping others informed. For examples, 
%P43 wanted to add a visual indicator of a cloudy and rainy background to symbolize she felt depressed at the moment and was in anti-social mode: \textit{``I’d imagine there were things around me, like a dark gray cloud or it's raining in the background and being right above you. And everywhere you go, it's right there.''} 
\remove{P47 would like to add a variation of battery symbols over the avatar's head, which would change levels based on her energy: \textit{``My energy levels can fluctuate just a lot. Someday, I may have a little bit of energy, and the next day I may have a lot of energy, and that could actually change within a matter of hours. So the idea that I have is a battery symbol that I could adjust the battery level shown on that to show you how much energy that I have to spend. It's a signal to my friends that ‘hey, my battery's low, I may sound really tired right? I'm okay, I just have low energy.’ [Other times] I could turn my battery all the way up and be like, ‘Hey, let's see, we can do something a little bit more active.’''}}

\change{\textbf{\textit{Guideline:}} When designing avatars, practitioners should consider leveraging avatar's peripheral space to enable users to better express their status, especially for individuals with invisible disabilities. Some design examples include a weather background to indicate mood and a battery sign to indicate energy level \cite{assets_24}. %The space beyond the avatar body not only provided a novel medium for PWD to express their disability status but also cultivate pro-social behaviors in social VR.
}

\subsection{Design of Avatar Customization and Control Interface (G5)}
\change{The usability and accessibility of avatar interfaces can significantly impact PWD's avatar customization experiences. We thus identified guidelines for avatar customization and control interfaces to enable smooth avatar curation for PWD.}

%that influenced PWD's engagement during avatar customization process, including interface layout, input controls, and mechanisms of displaying disability-related features.

\textbf{\textit{G5.1 Distribute disability features across the entire avatar interface rather than gathering them in a specialized category.}}
Five participants (P18, P32, P44, P49, P57) strongly preferred embedding the disability-related features naturally into different categories of the avatar interfaces \change{(e.g., asymmetrical eyes under the eye category, amputation under the body category), as opposed to collecting them in a specialized category for PWD, which marginalized them by ``setting PWD apart from other users'' (P32)}. % exclude PWD and make them feel they are using features intentionally designed as ``for disabilities'', which sets them apart from other users (P32). 
\remove{P49 suggested developers and designers to treat the disability-related features in the same way as any other avatar features in the interface: \textit{``Just treating them as neutral instead of either a burden to have to design or something you get to feel really special for designing''}.}
%They reported that seeing all features related to disabilities in a separate category made them feel they were using features intentionally designed as `for disabled people', which further isolated them from other users. %P32 emphasized the importance of not separating disability-related features from others:
% \begin{quote}
%     \textit{``Have those [disability representation] options in a variety of places, not like to create a disabled avatar, [you need to] go 13 levels down to the left, and [there’s a] sub-menu for that. Just make it integral to what you're designing, instead of making it like, ‘you gotta go on the short bus to get to the avatars for people with disabilities. Make it a part of everything else. Don't isolate it.''} -- P32, a blind person. 
% \end{quote}

\change{\textbf{\textit{Guideline:}} Avatar features for disability expression should be treated in the same way as other avatar features. In avatar interfaces, disability-related features should be properly distributed in their corresponding categories. There should not be a specialized category for PWD. 
 For example, assistive technologies should be included in the accessory category rather than a separate assistive technology category. }
%\kexin{do we have a more inclusive term for 'disability-related features'...R1 was criticizing some language are not inclusive.}
%\textit{``You can include a cane with the accessories tab instead of having a disabled tab over there…that can be kind of ostracizing. Just treating them as neutral instead of either a burden to have to design or something you get to feel really special for designing.''} 


\textbf{\textit{G5.2 Use continuous controls for high-granularity customization.}}
Eight participants (e.g., P16, P37, P49) believed that the control components in avatar interfaces can largely affect their customization flexibility. To accurately represent their disabilities, participants preferred continuous control methods (e.g., a slider) over discrete options (e.g., binary switches, drop-down menu with limited options). As P47 said: \textit{``[I prefer] the sliding scale. You can really change [the length of the limb] to a very particular level.''}
\remove{\textit{``It's better to have a spectrum of choices, or even a slider-like for people to change your nuanced level.''} Since disability representation was a spectrum (P16, P48), input controls that offered a continuous range of options, such as sliders and knobs, were preferred in the avatar customization interface. As P47 said: \textit{``[I prefer] the sliding scale. You can really change it on a very particular level.''}; P48 also agreed that \textit{``slider is better than binary options.''}}

\change{\textbf{\textit{Guideline:}} Avatar interfaces should adopt input controls that offer a continuous range of options to enable flexible customization. This could be widely applied to a variety of design attributes, such as the size and shape of multiple avatar body parts.}

\textbf{\textit{G5.3 Offer an easy control to turn on/off or switch between disability features.}} \label{g5.3}
Twelve participants (e.g., P11, P32, P57) noted that they didn't want to always disclose their disability identities in social VR. Instead, disability representations were often context-dependent \change{\cite{zhang2022, kelly2023, assets_24}}. %, and participants used avatar with disability features when they felt comfortable in a social environment. 
%For example, P41 didn't want to disclose her autistic identity when surrounded by strangers or unfriendly users: \textit{``Disability representation is very dependent on the social environment of the space. There are times where I am in virtual spaces that feel very hostile to disabled and autistic people. In those spaces, I would be less likely to openly present [my disabilities].''} 
For example, P39 didn't want to use avatars on wheelchair when surrounded by strangers or in unfamiliar VR worlds. \change{Moreover, people with multiple disabilities or fluctuated status  also need a fast and easy control to switch between different avatars (e.g., avatars with different facial expression in G2.2) or adjust status indicators (e.g., the weather background in G4.2) to flexibly update their disability expression based on contexts.}  
\remove{\textit{``Although I have a disability and I'm comfortable with it, it is not the most important thing to me. Sometimes I might not want to lead with [my disability], especially when you have physical disabilities that people can see [but] you have no control over how people see you right away.''}}

% \begin{quote}
%     %\textit{``If I'm in a very comfortable setting, and [people are] accepting, I'm going to come in there [as an avatar with disability-related features]. [At the same time,] 
%     \textit{It's important to have that [avatar with disability-related features] to be changed, where I can still have how my body shows up in the world but not necessarily with the wheelchair. That’s important, because although I have a disability and I'm comfortable with it, it is not the most important thing to me. So sometimes I might not want to lead with that, especially when you have physical disabilities that people can see, where you encounter a lot in the world where you have no control over how people see you right away.''} -- P39, a person with mobility disability
% \end{quote}

\change{\textbf{\textit{Guideline:}} Interface should provide easy-to-access shortcut control that enable users to conduct \textit{ad-hoc} avatar updates. Important control functions include: (1) toggling on and off the disability-related features \cite{assets_24}; (2) switching between different saved avatars \cite{kelly2023}; and (3) updating status for fluctuating conditions ~\cite{assets_24}. 
%as needed. For examples, developers can implement a shortcut to instantly remove disability features from an avatar (e.g., P6, P57), or allow users to save multiple versions avatars so that they can easily switched to those without disability representations when needed \change{\cite{kelly2023}}. Easy control method should be provided, allowing users to adjust the peripheral design \textit{ad-hoc} to reflect their condition fluctuation. Control interfaces should be designed to enable easy switching among facial expressions during social interactions \cite{kelly2023}.
}


%=======================

% \subsubsection{Highlight the empowerment of AT instead of stigma} (xx \% of participants).

% Using AT is often associated with the stereotype of being dependent. To combat such stereotype, the AT design should demonstrate how PWD could achieve independence with the use of AT, as P39 emphasized: \textit{``It's important to me that it doesn't look like I'm ready to be pushed by someone else.''} For example, adding a joystick design on wheelchair, which shows that wheelchair users could move independently (P6). 

% % [mannerism-wise] show capabilities: 
%     % P52: "so I feel like as far as walking is concerned, the feature I would most be interested in is being able to go at a pace that would keep up with my friends' avatars. So I don't think there's, for me, at least the the visual appearance of an avatar is more sort of how other people will, you know, perceive you in that online setting. And being able to just keep up with peers, pace-wise, would be the most important thing, as far as specific mannerisms are concerning."

% \subsubsection{Avoid overshadowing the person with AT simulation (G2.3.)} (xx \% of participants). 

% The AT design should prioritize showing the individuality of PWD, instead of making the AT more prominent than the user. For example, when designing the wheelchair, minimizing its appearance by having the back height lower than the user's shoulders could help (P39). The size of AT should also streamline with avatar's size, allowing users to customize AT size based on their avatars. [add example image of wheelchair design and interface of size customization]
%     % P52: "But I think definitely proportional. I wouldn't want anything exaggerated just because I feel like that would fool me at least what are on it sort of demeaning stereotype. And saying, when I look at this person, this is what I see first, which is not not a thing I aim to go for. Not a thing that I would like to believe that myself as other than, yeah."

% % make the guideline actionable by doing: 
% --> AT size should be proportionate to avatar body size + 
% AT design style should match with avatar style + 
%     % [AT should match with avatar style, don't break the integrity]: "So just because I feel like at the end of the day, it's, it's mostly a visual thing, how? Yeah, how are you being perceived? And is it an amount of that's roughly physically realistic. But not, not to the extent of oversimplification, or hyper realism when, oh, I guess that's the other thing. The greeter which the animation development is, I would like to see that be roughly equivalent to that of the other able bodied avatars, just so it's not like this is something additional and special and so unique. And look how hyper realistic it is. But with with the same level of level of animation refinement as the other avatars."
    
% change AT color to show personality + 
% add accessories and decoration to AT

% \subsubsection{Show realistic movement of AT for authentic disability representation (G2.4.)} (xx \% of participants). 

% The AT should exhibit realistic movement that accurately reflects how the technology functions in real life. This not only provides an authentic and respectful representation of PWD but also helps to enhance understanding and acceptance of their experiences with AT in everyday life. For example, the wheelchair's wheels should roll when in motion (P6), and avatars should have the pendulum movements of people with visual impairments using a white cane (P34).



%-----------------------
% write the interesting findings when we generate the guidelines.
% present a nice table like AI guideline paper (i.e., our system)
% the heuristic evaluation is a summative study, write like human-AI paper

    
% Objectifying disability is a common form of ableism experienced by PWD. To combat objectification, avatars should have humanoid model to represent and signify PWD as real human-being in social VR. As P46 elaborated: \textit{``I want people to see that people with mental illness and depression are real people and not just their disability.''}

% \subsubsection{Default to full-body avatars to cover a broad range of disabilities across different body parts (G1.2)} (xx \% of participants). 

% Many aspects of disabilities are mostly visible through the full body. If not full body, no way to show the disability. For example, the use of assistive technology (e.g., wheelchair, leg braces-P14) and disability representation through movement often require adaptations that are visible in the avatar's lower body.

% \subsubsection{Enable customization to the presence, length, and strength of limbs.}
% % merge -> all customization, as one guideline; further specific to different body parts as subsubsubguidelines

% \subsubsection{Have asymmetric design of eyes.} % stand alone
% - e.g., size differences (P4)
% - eyeball directions, cross eyes (P9. P31)

% \subsubsection{Allow customization of body characteristics for chronic health condition.}
% - body shape fluctuates due to medication
% - skin conditions change 
% % - scar thing -> add on of the skin, instead of part of the skin (not customization) / accessories?

% % be careful with the "customization" use


% \subsection{Design Guidelines for Facial Expressions: }

% \subsubsection{Leverage facial expressions to show realistic mannerism related to disability.}
% an important way to represent invisible disabilities like ADHD and autism...
% avoid/hard to make eye contact to rep. autism (P7, P47)

% \subsubsection{Design expressive facial cues to visualize emotion.}
% ...


% \subsection{Guidelines for Avatar Customization Process: }

% \subsubsection{Avoid highlighting disability representation features as an individual category, instead integrating them to general customization interface.}

% e.g., AT is part of accessories, rather than "assistive technology" stand alone

% \subsubsection{Allow multiple avatars to be saved for dynamic representation}
% easy to change through different avatars for fluctuating conditions

% \subsubsection{Easy interaction mechanism to turn on and off disability features of an avatar}

%)--------------------
% novelty of body part customization -> a bit dry
    % disability specific -> how people with different disabilities may present it differently
    % anything new that developer could learn 
    % distinguish from generic guidelines
    % be a bit more sharp and concise about 
% the rationale goes to explaination
    % should not conflict with each one 
    % use game a11y as a template -> publishable list of guideline that we should follow as the format; guideline + explaination + examples (-> avatar library): https://gameaccessibilityguidelines.com/basic/ 
    % -> set a expected final outcome (each code should be aligned with the research goal) | build the sense of good and interesting findings -> e.g., what people already know vs. novelty 
    % how impactful
    % what to be presented to the experts -> formal list: also ask = what do we want to evaluate with (not overleaf version, follow a11y game guidelines)
    % show guideline to Daniel and Ru, discuss with Scott -> get some fresh eyes

%\section{Guideline Evaluation}
\begin{table*}[tb]
\centering
\caption{Demographics of Participant Clients: Previous Art Therapy Sessions indicates the number of times the client has previously participated in art therapy; Familiarity with Traditional Drawing reflects the client's level of experience with traditional drawing techniques (0-not familiar; 1-very familiar); Familiarity with Digital Drawing reflects the client's level of experience with digital drawing techniques (0-not familiar; 1-very familiar); Participation Purposes reflects the reasons clients choose to engage in the activity.}
\vspace{-3mm}
\label{tab:clients}
\small
\resizebox{1\linewidth}{!}{
\begin{tabular}{cccccccccc}
\toprule
\textbf{ID} & \textbf{Gender} & \textbf{Age} & \textbf{Education} & \textbf{Region} & \parbox[t]{2.5cm}{\centering\textbf{Previous Art Therapy Sessions}} & \parbox[t]{3cm}{\centering\textbf{Familiarity with Traditional Drawing}} & \parbox[t]{2cm}{\centering\textbf{Familiarity with Digital Drawing}} & \parbox[t]{2cm}{\centering\textbf{Therapist Assignment}} & \parbox[t]{2.5cm}{\centering\textbf{Participation Purposes}} \\
\midrule
C1  & Female & 37  & Bachelor's & China/Shanghai & 0                            & 1                                   & 0.25  &T3 & Personal Growth                   \\
C2  & Female & 35  & Bachelor's & China/Shenzhen & 3                            & 0.5                                   & 0.5   &T3 & Career Development and Family                 \\
C3  & Female & 28  & Master's   & China/Hebei    & 2                            & 0.75                                  & 0.75   &T3  & Family and Emotional Management                \\
C4  & Female & 36  & Bachelor's & China/Beijing  & 10                           & 0.75                                   & 0   &T3  &Career Development                \\
C5  & Male   & 28  & Master's   & Germany       & 0                            & 1                                   & 0.75   &T3   &  Emotional Management and Personal Growth                       \\
C6  & Other  & 26  & Associate's & China/Heilongjiang & 1                            & 0.5                                   & 0.25  &T5  & Emotional Exploration and Intimate Relationships                           \\
C7  & Female & 23  & Master's   & China/Shanghai & 0                            & 1                                   & 1     &T5     &  Intimate Relationships                    \\
C8  & Female & 20  & Bachelor's & China/Shenzhen & 0                            & 0.5                                   & 0.5    &T5   &  Emotional Management and Intimate Relationships                       \\
C9  & Female & 25  & Bachelor's & China/Guangxi  & 4                            & 0                                   & 0.5    &T5    &  Self-Expression and Emotional Exploration                      \\
C10 & Male   & 23  & Master's   & China/Shenzhen & 0                            & 0.75                                   & 0.5   &T5   &             Self-Expression and Social Skills             \\
C11 & Female & 26  & Master's   & China/Hangzhou & 0                            & 0.5                                   & 0.25    &T4  &        Emotional Management, Social Skills and Intimate Relationships                 \\
C12 & Female & 26  & Master's   & China/Shanghai & 2                            & 0.75                                   & 0.5    &T4   &                   Stress Relieving and Intimate Relationships  \\
C13 & Female & 30  & Master's   & China/Dalian   & 0                            & 0.5                                   & 0.25   &T4    &             Family and Emotional Management            \\
C14 & Female & 19  & Bachelor's & China/Chongqing & 0                            & 0.25                                   & 0.25   &T4  &                Personal Growth and Self-Exploration           \\
C15 & Male   & 27  & Bachelor's & China/Beijing  & 0                            & 0.25                                  & 0.25   &T4    &                 Stress Relieving and Personal Growth        \\
C16 & Female & 22  & Bachelor's & China/Shandong & 0                            & 0.5                                   & 0.25   &T1     &              Emotional Management and Social Skills       \\
C17 & Male   & 38  & Master's   & China/Sichuan  & 0                            & 0.75                                   & 0.75   &T1     &                    Personal Growth      \\
C18 & Female & 40  & Master's   & China/Beijing  & 20                           & 1                                   & 0.75    &T1      &               Stress Relieving and Emotional Management          \\
C19 & Female & 28  & Bachelor's & China/Guangzhou & 0                            & 0.5                                   & 0   &T1       &                 Future Career Planning and Personal Growth      \\
C20 & Male   & 25  & Master's   & China/Guangzhou & 0                            & 1                                   & 1   &T1        &                    Academic Pressure Relieving   \\
C21 & Male   & 24  & Master's   & China/Hubei    & 0                            & 0                                   & 0   &T2        &                Childhood Family and Dreams Exploration  \\
C22 & Female & 24  & Master's   & China/Shenzhen & 0                            & 0.25                                   & 0.25    &T2  &                Emotional Management and Personal Growth     \\
C23 & Male   & 25  & Master's   & China/Zhejiang & 10                           & 0.5                                   & 0.5    &T2   &                  Emotional Development and Self-Expression        \\
C24 & Male & 55  & Bachelor's & Dubai& 0 & 0.5& 0.5&T2 &                           Emotional Management \\
\bottomrule

\end{tabular}}
\Description{The table 2 describes 24 participants in art therapy sessions. The participants are from diverse locations, including China (Shanghai, Shenzhen, Hebei, Beijing, Heilongjiang, Guangxi, Hangzhou, Chongqing, Shandong, Sichuan, Hubei, and Zhejiang), Germany, and Dubai. The ages range from 19 to 55 years old, with varying levels of education from associate degrees to master's degrees and bachelor's degrees. Their familiarity with traditional drawing techniques ranges from no familiarity to very familiar, while their familiarity with digital drawing techniques also varies across the spectrum. The participants have attended between 0 and 20 previous art therapy sessions and are assigned to different therapists identified by codes T1 to T5.Participation Purposes reflects the reasons clients choose to engage in the activity}
\end{table*}

\section{Field study}
Using \name{} as both a novel system to study and a research tool to study with, we aim to explore how a human-AI system support clients' art therapy homework in their daily settings (\textbf{RQ1}) and how such a system could mediate therapist-client collaboration surrounding art therapy homework (\textbf{RQ2}). To this end, we conducted a field deployment involving 24 recruited clients and five therapists over the course of one month.



%参与者与实验的setup
    %参与者招募
        % 我们招募的途径:To recruit our clients, we distributed digital recruitment flyers through social media platforms.
        % 海报上描述了什么:The recruitment flyer described the art therapy activities as "promoting self-exploration using a digital software".
        % 我首先要求参与者填写pre-问卷,这个问卷主要包括descriptions of the art therapy activities, demographic information, the number of art therapy sessions they attended, familiarity with digital drawing, and specific needs for the art therapy activities.
        % Participants were included in this study with the aim of reducing stress and anxiety, fostering personal growth, improving emotional regulation, and strengthening social skills.
        % 此外,we tried to selection of participants based on their regions, occupations, the types of devices they used, and the number of times they participated in art therapy.
        % finally, 有27名参与者开始使用这个系统,其中有3名参与者drop out因为缺乏时间
\subsection{Participants and Study Procedure}
\subsubsection{Participants}

The five therapists who participated in the field evaluation were the same ones from our contextual study (see \autoref{tab:expert}). Each therapist was compensated at their regular hourly rate.
For client recruitment, we distributed digital flyers through social media platforms, describing the art therapy activities as an "online art therapy experience promoting self-exploration using a digital software." This aligns with the common goal of art therapy sessions, which are widely used to promote self-exploration for all clients, beyond treating mental illness~\cite{kahn1999art, riley2003family}.

Participants first completed a pre-questionnaire, which provided an overview of the activities and collected demographics, and prior experiences with art therapy experience and with digital drawing---to ensure that we include both novices and experienced user---and their personal goals for participation. 
The therapists guided the recruitment and screening of the the clients, and included individuals seeking for reducing stress, fostering personal growth, enhancing emotional regulation, and strengthening social skills. The therapists excluded individuals with serious mental health conditions to minimize ethical risks.
%Based on the therapists' advice, clients with goals such as reducing stress and anxiety, fostering personal growth, enhancing emotional regulation, and strengthening social skills were included, avoiding ethical concerns related to clinically diagnosed mental health conditions. 
%We also considered participants' regions, device types, drawing familiarity, and prior art therapy experience to create a balanced selection.

In total, 27 clients began using \name{}, but 3 withdrew early due to scheduling conflicts. The final group of 24 clients (C1-C24; 8 self-identified males, 15 self-identified females, 1 identifying as other; aged 19-55) completed the study (client demographics are detailed in the~\autoref{tab:clients}). Clients who completed the full process were compensated with \$37, others were compensated with a prorated fee.
Our study protocol was approved by the institutional research ethics board, and all participant names in this paper have been changed to pseudonyms. Participants reviewed and signed informed consent forms before taking part, acknowledging their understanding of the study.

% The five therapists participated in the field evaluation were the ones who also participated in our contextual study (see \autoref{tab:expert}).
% Five art therapists were compensated with their regular hourly rate.
% For the clients recruitment, we distributed digital recruitment flyers through social media platforms. 
% The recruitment flyer described the art therapy activities as ``online art therapy experience promoting self-exploration using a digital software''.
% This is due to that this is a common goal for art therapy sessions, since Art therapy activities are not only effective in treating mental illness but also widely promote self-exploration for every clients, as commonly integrated into practice~\cite{kahn1999art,riley2003family}.
% First, participants completed a pre-questionnaire that provided an overview of the art therapy activities and gathered details such as their demographics, the number of art therapy sessions they've attended, familiarity with digital drawing, and any specific needs they hoped to address.
% Following that, based on the advices from the therapists, clients were included with the goal of reducing stress and anxiety, fostering personal growth, enhancing emotional regulation, and strengthening social skills.
% The therapists suggest so since they agree that these therapeutic goals would be beneficial for eavery day therapy clients and would could It might avoid the potential ethical and safety risks associated with clinically diagnosed mental health issue.
% Further, we selected participants based on a balance of their regions, the types of devices they used, the familiarity with drawing and their prior experience with art therapy. 

% In total, 27 clients began using \name{}, but 3 withdrew from the study at the early stage due to scheduling conflicts.
% Finally, 24 clients (C1-C24; 8 self-identified males, 15 self-identified females, 1 identifying as other; aged 19-55) completed our field study. 
% APPENDIX shows the specific client demographics.
% We compensated clients based on their level of involvement, with those who completed the full one-month study receiving 200 RMB as a bonus, and clients who dropped out receiving a prorated fee according to the duration of their participation.

% Our protocol was approved by the institutional research ethics board, and all names in this paper have been changed to pseudonyms.
% Also, before participating in the activity, participants carefully reviewed and signed the informed consent form, acknowledging their understanding.

%在与治疗师协商讨论下,这些用户被分到5位治疗师(see Table),其中T2有4位来访者,其余治疗师有5位来访者。
%这个研究. .
%在活动开始前,我们邀请每位参与者开展了一场介绍session. 主要是目的是介绍活动目的与流程,并且演示如何使用\name{},并且为每位来访者可以接触到系统的URL的链接;
%介绍活动结束后,来访者被鼓励有规律地去自行探索使用\name{};
%每隔一周,我们会安排治疗师与来访者进行线上一对一的session。我们会鼓励治疗师在线上一对一session之前提前review来访者的使用数据,并通过即时通讯软件与我们交流review之后的洞见与想法。
%在线上一对一session时,在不干扰治疗师艺术治疗实践的基础上,我们鼓励治疗师在线上一对一session时利用这些数据。在艺术创作阶段,来访者可以通过分享屏幕的方式使用系统的第一个阶段进行创作并与治疗师进行讨论交流,在session快结束前治疗师会给来访者推荐家庭作业。
%在session结束后,治疗师会在治疗师系统上安排家庭作业并给予来访者的个人赠言。此外,来访者在结束线上session后可以按照治疗师的推荐完成家庭作业或者自行探索使用系统。
\subsubsection{Procedures}

Clients were distributed in coordination with the five therapists, as shown in \autoref{tab:expert}. T2 was assigned four clients, while the other therapists each had five clients. The field study consisted of two main activities: (1) three online in-session activities, where clients had one-on-one conversations and collaborated with the therapist, and (2) unstructured between-session activities, where clients practiced therapy homework using \name{} following the therapist’s recommendations.
Before the study, we held online introductory sessions to familiarize the clients with \name{}, and provided both demonstrations and hands-on exploration on their preferred devices. Similarly, we offered online training for therapists on customizing and reviewing homework, while allowing them to explore both the therapist-facing and client-facing applications. After the session, clients were encouraged to regularly explore \name{}.
Two weeks into the study, we scheduled weekly one-on-one online sessions between therapists and clients, each lasting approximately 60 minutes. Therapists were encouraged to review the clients' homework history using \autoref{fig:ui}(c) before each session. During the online session, therapists used this data to inform their practices without interrupting the flow of therapy. We encouraged clients in advance, to create artworks during the Art-making Phase~(\autoref{fig:qual_results}(a)), sharing screens and discussing their creations with the therapist, but did not interfere with the therapeutic process.

%Clients also used \autoref{fig:qual_results}(a) to create artwork, sharing their screens and discussing their creations with the therapist.

At the end of each session, therapists recommended homework tasks based on insights gained during the conversation. After the session, therapists might customize homework agents, including customizing conversational principles, assigning homework tasks, and providing personal messages through \autoref{fig:ui}~(d). Clients could then either complete the assigned homework or engage in self-exploration using \name{} between sessions.

% Clients were distributed In coordination with the five therapists, as shown by \autoref{tab:clients}: T2 was assigned with four clients, while each of the other therapists was assigned with five clients.
% The procedure for the field study consisted of two activities: (1) three online in-session activities where they have one-on-one conversation and collaboration with the therapist and (2) unstructured between-session activities where they perform therapy homework practices either upon recommendations of usage from the therapist or volunteerily use it in their daily lives.
% Before the study, we conducted an introductory session for each client to explain the activities, demonstrate how to use \name{}, and provide access to \name{} via a URL on their preferred devices.
% After the introductory session, the clients were encouraged to explore the use of \name{} on a regular basis.

% After two weeks of self-exploration, we started scheduling weekly one one-on-one online sessions between the therapists and the clients.
% Therapists were encouraged to review clients' homework history using \autoref{fig:ui}~(c) before the online session.
% During the online one-on-one session, we encouraged therapists to use these history data without interfering with their art therapy practices. 
% Also, they would utilize \autoref{fig:ui}~(a) to create their artwork by sharing their screens and discussing their artworks with therapists. 
% Before the end of the session, the therapist would recommend the homework tasks for the client based on the insights gained from the one-on-one session.
% After the online session ends, therapists would customize homework agents, including modifying or updating the conversational agent principles, assigning homework tasks and providing therapist's messages to the client through \autoref{fig:system}~(d). 
% Correspondingly, clients could either complete the homework or engage in self-exploration using \name{} between sessions.

% 对于异步session场景数据收集下,所有来访者使用系统的图像以及对话记录等日志数据以及治疗师在治疗师系统中使用定制功能的日志数据在保存在数据库中。
% 此外,我们鼓励来访者和治疗师通过即时通讯软件发送给我们images以及comments关于使用系统的实践以及感受。
% 对于线上session的场景数据收集,首先,online sessions were audio- and video-recorded.
% 此外,at the end of each online session, we conducted a 5-minute interview with therapists, mainly to collect their practices and experiences about the session.
% Upon concluding all the sessions,我们与治疗师以及来访者开展了约为30分钟的semi-structured interview to 探索ai agents如何支持艺术治疗场景的家庭作业(RQ1)以及AI agents如何mediate 治疗师与来访者合作(RQ2). We used 治疗师与来访者在 the trial period使用系统的log 数据以及他们的反馈作为stimuli 去问特定的使用实践的问题。
% With participants' consent, we recorded the interviews and transcribed them for thematic analysis.
% First, two researchers conducted collaborative inductive coding. They initially annotated the transcript to identify relevant quotes, key concepts, and recurring patterns in the data. These findings were further developed through regular discussions, leading to a detailed coding scheme aligned with the research questions. Quotes were then coded and clustered into a hierarchy of emerging themes, continually reviewed, and refined in recurrent meetings, where exemplar quotes were also selected for presenting each theme and sub-theme. 
% Also, we collected the log data from 治疗师和来访者 作为证据以及examples for the thematic analysis results.

\subsection{Data Gathering Methods} 

For between-sessions, we stored all homework-related data in a database, including artwork, dialogue, usage logs, as well as information on homework customization such as conversational principles, tasks, and personal messages.
We encouraged participants to use personal messaging (WeChat) to share pictures and comments about on-the-spot experience and feelings after homework with \name{} to compensate for semi-structured interviews.
During online sessions, we recorded audio and video. 
The researchers did not observe the therapy session in live, but reviewed post hoc, as the therapists believed a third party's presence could affect a client's emotional expression and the therapist-client dynamic.
After each session, we conducted a brief 5-minute interview with the therapists to gather their insights and feelings.

Upon the completion of the final one-on-one sessions, we conducted 30-minute semi-structured interviews with both therapists and clients. These interviews aimed to explore how \name{} supported art therapy homework in clients' daily lives (\textbf{RQ1}) and how therapists and clients collaborated surrounding art therapy homework (\textbf{RQ2}). We used feedback and homework outcomes from the trial period to ask targeted questions about their practices.
With participants' consent, we recorded and transcribed the brief 5-minute interviews and the 30-minute interviews for thematic analysis~\cite{braun2006using}. This analysis also included the personal messages shared by the participants about their on-the-spot experiences.
%we recorded and transcribed the interviews for thematic analysis. 
Two researchers then engaged in inductive coding, annotating transcripts to identify relevant quotes, key concepts, and patterns. They developed a detailed coding scheme through regular discussions, grouping quotes into a hierarchical structure of themes and sub-themes. Exemplar quotes were selected to represent each theme. We also used homework history (e.g., images or conversation data) and customization data (e.g., homework dialogue principle data) as evidences or examples to back up the findings in our thematic analysis.



% In between sessions, all homework history data~(e.g., artwork, creative process data and dialogue data) and history data on homework customization~(e.g., principles of conversational agents, homework tasks and personal messages) were stored in the database.
% In addition, we encouraged clients and therapists to send us images and comments about their experiences and feelings when using \name{} via an instant messaging app.
% For online in-sessions, the sessions were first audio- and video-recorded.
% At the end of each in-session, we conducted a brief 5-minute interview with the therapists to gather insights into their practices and feelings during the session.
% Upon concluding all the sessions, we conducted approximately 30-minute semi-structured interviews with both the therapists and the clients to explore how \name{} support art therapy homework in clients' daily settings~(\textbf{RQ1}), and how therapists tailored the homework and tracked the homework history surrounding art therapy homework~(\textbf{RQ2}). 
% Further, we employed the homework outcomes and feedback from both therapists and clients during the trial period as stimuli to ask specific questions about their practices. 

% With participants' consent, we recorded the interviews and transcribed them for thematic analysis~\cite{braun2006using}.
% Initially, two researchers engaged in collaborative inductive coding. They began by annotating the transcript to highlight relevant quotes, key concepts, and recurring patterns in the data. Through regular discussions, they expanded these insights into a detailed coding scheme that aligned with their research questions. The quotes were then systematically coded and grouped into a hierarchical structure of emerging themes, which were continuously reviewed and refined during recurring meetings. During these discussions, exemplar quotes were also chosen to represent each theme and sub-theme.
% We also gathered homework history and customization data, including artworks and conversation records from both therapists and clients, as evidence and examples to support the results of the thematic analysis.

\begin{figure*}[tb]
  \centering
  \includegraphics[width=\linewidth]{images/findings_1.png}
  \vspace{-7mm}
  \caption{Overview of The Homework Engagement of Clients with \name{}: (a) Homework Activity Date Distribution; (b) Accumulated Homework Activity Hourly Distribution of the Day; (c) Usage of AI Brushes in Artworks; 
  }
  \Description{Figure 5 contains three sub-figures. Figure 5a shows the Homework Activity Date Distribution for 24 clients over a four-week period, using seven different shades of purple to represent varying levels of participation in the homework sessions. Figure 5b illustrates the frequency of AI brush usage during clients' homework art-making, with the top 20 most frequently used brushes highlighted in larger font. Figure 5c depicts the distribution of homework sessions across different times of the day, revealing that clients tend to engage in homework sessions more frequently in the afternoon and evening.}
  \label{fig:quan_results}
\end{figure*}




%\section{Guideline Revision}
\section{Guideline Revision}


\change{Based on the feedback from VR practitioners, we refined the guidelines and made the following changes: 
(1) Convert the original G1.1 to an overarching statement G0 to contextualize and motivate practitioners as well as specifying the application scope of our guidelines (D1);
(2) Assigned recommendation levels to each guideline to distinguish their priority based on development resources, size of user groups, and use cases (D1, D2, S3);
(3) Added statistics about served user groups, informing practitioners of the potential impact of implementing the guideline (D2);
(4) Specified the customization scope in G1.3, G3.1, G3.2, and G4.1 (S1); 
(5) Diversified implementation examples in G2.2, G2.3, and G3.3 (S2);
(6) Suggested potential use cases for G1.4 and G2.2 (D1, S3);
(7) Merged G2.4 into G2.1 to remove redundant information, and merged G3.1 and G3.4 to clarify the characteristics of simulated assistive technologies in social VR (S4);
\cameraready{(8) Refined the wording in G2.2, G3.3, and G5.2 to improve clarity.
Appendix Table \ref{tab:changes} demonstrated the changes we made to the initial guidelines.}

After revision, we ended up with 17 design guidelines upon experts' feedback. We present an overview table of the revised guidelines in Table ~\ref{tab:overview_revised} and a full version in Appendix Table ~\ref{tab:full_revised}.} 


%(1) adding an overarching guideline G0 to fullfill the suggestion  (Sx); (2) Adding recommendation levels to all guidelines to distinguish  priority (3) adding stats info to each guideline to highlight the disability coverage and impact of the guideline to motivate developers (Sx) 
%(2) merging G 3.1 and G3.4 to clarify the scope of assistive tech to be simulated in social VR (Sy); xxx 
%(3) Specify the suggested customization scope in G1.2, e.g., xxx (Sz); }

%\change{A consensus achieved from prior literature \cite{zhang2022, assets_24, chronic_pain_gualano_2024, kelly2023} and our study is to \textbf{\textit{support disability representation in social VR avatars} (G0).} Our findings from experts' evaluations further highlighted the widely applicable use cases of this guideline: as long as the platform involved avatar-based interactions, there are design space to support disability representation. In other words, it can be applied to a variety of social VR platforms with different (1) avatar types (e.g., humanoid avatars in Rec Room \cite{recroom} vs. robotic-type avatars in Among Us \cite{amonguscharacters}), (2) aesthetic styles (e.g., life-like avatars in Horizon Worlds \cite{metaavatars} vs. abstract avatars in Roblox \cite{robloxwiki})), and (3) content focus (e.g., communication-heavy type in VRChat \cite{vrchat} vs. game-centric in Rec Room). We encourage practitioners to adopt this guideline (G0) as a fundamental mindset when developing and designing avatars, considering it in the early stages, and consistently exploring opportunities to support disability representation. 

%\yuhang{this needs to be updated to fit the context; you only need to describe how you made the change, e.g., "To fulfill the suggestion of xxx, we added an overarching guideline G0 to xxx". Also, G0 is not a guidelin to apply, but rather an overall statement for awareness and clarify suitable VR scenarios to use our guidelines. The detailed content in G0 needs to be added to the table. }}

%\change{
% The revised 

% Based on experts' feedback, we revised and iterated on the guidelines to make them more applicable and actionable in practice. We describe the revision below and present the finalized guidelines in Table \ref{}.

% \textbf{\textit{Added the Recommendation Levels.}} For each guidelines, we 

% \textbf{\textit{Merging.}}

% \textbf{\textit{Cross-referencing.}}
% e.g., G3.3 add standard requirement of assistive technologies

% \textbf{\textit{Specify a scope / bare minimum standards to follow.}}
% e.g., G1.2 body parts that at least should be customizable, G3.1 five types of AT that should be included at the minimum, 

% \begin{itemize}
%     \item Highly Recommended (HR): Highly recommended guidelines are easy to implement, apply to almost all use cases, and are considered as the bare minimum for avatar design and development. 
%     \item Recommend (R): Recommended guidelines may require planning and effort to implement. They are more tailored to specific user groups or use case scenarios, but they lead to good user experiences and help expand user groups.
% \end{itemize}
%}


%\section{Discussion}
\section{Discussion}
The development of foundation models has increasingly relied on accessible data support to address complex tasks~\cite{zhang2024data}. Yet major challenges remain in collecting scalable clinical data in the healthcare system, such as data silos and privacy concerns. To overcome these challenges, MedForge integrates multi-center clinical knowledge sources into a cohesive medical foundation model via a collaborative scheme. MedForge offers a collaborative path to asynchronously integrate multi-center knowledge while maintaining strong flexibility for individual contributors.
This key design allows a cost-effective collaboration among clinical centers to build comprehensive medical models, enhancing private resource utilization across healthcare systems.

Inspired by collaborative open-source software development~\cite{raffel2023building, github}, our study allows individual clinical institutions to independently develop branch modules with their data locally. These branch modules are asynchronously integrated into a comprehensive model without the need to share original data, avoiding potential patient raw data leakage. Conceptually similar to the open-source collaborative system, iterative module merging development ensures the aggregation of model knowledge over time while incorporating diverse data insights from distributed institutions. In particular, this asynchronous scheme alleviates the demand for all users to synchronize module updates as required by conventional methods (e.g., LoRAHub~\cite{huang2023lorahub}).


MedForge's framework addresses multiple data challenges in the cycle of medical foundation model development, including data storage, transmission, and leakage. As the data collection process requires a large amount of distributed data, we show that dataset distillation contributes greatly to reducing data storage capacity. In MedForge, individual contributors can simply upload a lightweight version of the dataset to the central model developer. As a result, the remarkable reduction in data volume (e.g., 175 times less in LC25000) alleviates the burden of data transfer among multiple medical centers. For example, we distilled a 10,500 image training set into 60 representative distilled data while maintaining a strong model performance. We choose to use a lightweight dataset as a transformed representation of raw data to avoid the leakage of sensitive raw information.
Second, the asynchronous collaboration mode in MedForge allows flexible model merging, particularly for users from various local medical centers to participate in model knowledge integration. 
Third, MedForge reformulates the conventional top-down workflow of building foundational models by adopting a bottom-up approach. Instead of solely relying on upstream builders to predefine model functionalities, MedForge allows medical centers to actively contribute to model knowledge integration by providing plugin modules (i.e., LoRA) and distilled datasets. This approach supports flexible knowledge integration and allows models to be applicable to wide-ranging clinical tasks, addressing the key limitation of fixed functionalities in traditional workflows.

We demonstrate the strong capacity of MedForge via the asynchronous merging of three image classification tasks. MedForge offered an incremental merging strategy that is highly flexible compared to plain parameter average~\cite{wortsman2022model} and LoRAHub~\cite{huang2023lorahub}. Specifically, plain parameter averaging merges module parameters directly and ignores the contribution differences of each module. Although LoRAHub allows for flexible distribution of coefficients among modules, it lacks the ability to continuously update, limiting its capacity to incorporate new knowledge during the merging process. In contrast, MedForge shows its strong flexibility of continuous updates while considering the contribution differences among center modules. The robustness of MedForge has been demonstrated by shuffling merging order (Tab~\ref{tab:order}), which shows that merging new-coming modules will not hurt the model ability of previous tasks in various orders, mitigating the model catastrophic forgetting. 
MedForge also reveals a strong generality on various choices of component modules. Our experiments on dataset distillation settings (such as DC and without DSA technique) and PEFT techniques (such as DoRA) emphasize the extensible ability of MedForge's module settings. 

To fully exploit multi-scale clinical data, it will be necessary to include broader data modalities (e.g., electronic health records and radiological images). Managing these diverse data formats and standards among numerous contributors can be challenging due to the potential conflict between collaborators. 
Moreover, since MedForge integrates multiple clinical tasks that involve varying numbers of classification categories, conventional classifier heads with fixed class sizes are not applicable. However, the projection head of the CLIP model, designed to calculate similarities between image and text, is well-suited for this scenario. It allows MedForge to flexibly handle medical datasets with different category numbers, thus overcoming the challenge of multi-task classification. That said, this design choice also limits the variety of model architectures that can be utilized, as it depends specifically on the CLIP framework. Future investigations will explore extensive solutions to make the overall architecture more flexible. Additionally, incorporating more sophisticated data anonymization, such as synthetic data generation~\cite{ding2023large}, and encryption methods can also be considerable. To improve data privacy protection, test-time adaptation technique~\cite{wang2020tent, liang2024comprehensive} without substantial training data can be considered to alleviate the burden of data sharing in the healthcare system.



             

%\section{Conclusion}
\section{Conclusion}
We reveal a tradeoff in robust watermarks: Improved redundancy of watermark information enhances robustness, but increased redundancy raises the risk of watermark leakage. We propose DAPAO attack, a framework that requires only one image for watermark extraction, effectively achieving both watermark removal and spoofing attacks against cutting-edge robust watermarking methods. Our attack reaches an average success rate of 87\% in detection evasion (about 60\% higher than existing evasion attacks) and an average success rate of 85\% in forgery (approximately 51\% higher than current forgery studies). 

\begin{acks}
We thank all anonymous participants for their efforts and valuable feedback. This work was supported in part by the National Science Foundation under Grant No. IIS-2328182 \& No. IIS-2328183 and a Meta Research Award.
\end{acks}


%%
%% The next two lines define the bibliography style to be used, and
%% the bibliography file.
\bibliographystyle{ACM-Reference-Format}
%TC:ignore
\bibliography{reference}
%TC:endignore

%%
%% If your work has an appendix, this is the place to put it.

\appendix
%TC:ignore
\section{Appendix}
 \newpage
\appendix
\section{Appendix}

\subsection{Conversational agent prompts for generating stable diffusion prompts in art-making phase}

\textbf{Role:} You will be able to capture the essence of the sessions and drawings in the recordings based on the art therapy session recordings I have given you and summarize them into a short sentence that will be used to guide the PROMPT for the Stable Diffusion model.

\vspace{0.5em} % 添加一些垂直间距

\textbf{Example input:}

\begin{itemize}[leftmargin=*]
    \item \textbf{USER:} [user-drawn] I drew the ocean. [canvas content] There is nothing on the canvas right now.
    \item \textbf{ASSISTANT:} What kind of ocean is this?
    \item \textbf{USER:} [user-drawn] I drew grass. [canvas content] Now there is an ocean on the canvas.
    \item \textbf{ASSISTANT:} What kind of grass is this?
    \item \textbf{USER:} [user-drawn] I drew the sky. [canvas content] Now there is grass and ocean on the canvas.
    \item \textbf{ASSISTANT:} What kind of sky is this?
    \item \textbf{USER:} [user-drawn] I drew mountains. [canvas content] Now there is sky, grass, and ocean on the canvas.
    \item \textbf{ASSISTANT:} What kind of mountain is this?
    \item \textbf{USER:} [user-drawn] I drew clouds. [canvas content] Now there is sky, mountain, grass, and ocean on the canvas.
    \item \textbf{ASSISTANT:} What kind of cloud is this?
    \item \textbf{USER:} [user dialogue] Colorful clouds, emerald green mountains and grass, choppy ocean
\end{itemize}

\vspace{0.5em} % 添加一些垂直间距

\textbf{Task:}

\begin{enumerate}[label=\textbf{Step \arabic*:}]
    \item \textbf{[Step 0]} Read the given transcript of the art therapy session, focusing on the content of \texttt{user: [user drawing]} and \texttt{user: [user dialog]}: Go to \textbf{[Step 1]}.
    \item \textbf{[Step 1]} Based on the input, find the last entry of user's input with \texttt{[canvas content]}, find the keywords of the screen elements that the canvas now contains (in the example input above, it is: sky, grass, sea), separate the keywords of each element with a comma, and add them to the generated result. Examples: [keyword1], [keyword2], [keyword3], \dots, [keyword n].
    \item \textbf{[Step 2]} Find whether there are more specific descriptions of the keywords of the painting elements in \texttt{[Step 1]} in \texttt{[User Dialog]} according to the input. If not, this step ends into \textbf{[Step 3]}; if there are, combine these descriptions and the keywords corresponding to the descriptions into a new descriptive phrase, and replace the previous keywords with the new phrases. Examples: [description of keyword 1] [keyword 1], [keyword 2 description of keyword 2], [description of keyword 3], \dots. Based on the above example input, the output is: rough sea, lush grass, blue sky.
    \item \textbf{[Step 3]} Based on the input, find out if there is a description of the painting style in the \texttt{[User Dialog]} in the dialog record, and if there is, add the style of the picture as a separate phrase after the corresponding phrase generated in \texttt{[Step 2]}, separated by commas. For example: [description of keyword 1] [keyword 1], [description of keyword 2] [keyword 2], \dots, [screen style phrase 1], [screen style phrase 2], [screen style phrase 3], \dots, [Picture Style Phrase n].
\end{enumerate}

\vspace{0.5em} % 添加一些垂直间距

\textbf{Output:} 

Only need to output the generated result of \textbf{[Step 3]}.

\vspace{0.5em} % 添加一些垂直间距

\textbf{Example output:} 

\emph{Rough sea, lush grass}

\subsection{Conversational agent prompts for discussion phase}

\textbf{Role:} <therapist\_name>, Professional Art Therapist

\textbf{Characteristics:} Flexible, empathetic, honest, respectful, trustworthy, non-judgmental.

\vspace{0.5em} % 添加垂直间距

\textbf{Task:} Based on the user's dialogic input, start sequentially from step [A], then step [B], to step [C], step [D], step [E] \dots Step [N] will be asked in a dialogical order, and after step [N], you can go to \textbf{Concluding Remarks}. You can select only one question to be asked at a time from the sample output display of step [N]! You have the flexibility to ask up to one round of extended dialog questions at step [N] based on the user's answers. Lead the user to deeper self-exploration and emotional expression, rather than simply asking questions.

\vspace{0.5em} % 添加垂直间距

\textbf{Operational Guidelines:}

\begin{enumerate}
    \item You must start with the first question and proceed sequentially through the steps in the conversational process (step [A], step [B], step [C], step [D], step [E], \dots, step [N]).
    \item Do not include references like step '[A]', step '[B]' directly in your reply text.
    \item You may include one round of extended dialog questions at any step [N] depending on the user's responses and situation. After that, move on to the next step.
    \item Always ensure empathy and respect are present in your responses, e.g., re-telling or summarizing the user's previous answer to show empathy and attention.
\end{enumerate}

\vspace{0.5em} % 添加垂直间距

\textbf{Therapist’s Configuration:}

\textbf{Principle 1:}  
\textit{Sample question:} How are you feeling about what you are creating in this moment?

\vspace{0.5em}

\textbf{Principle 2:}  
\textit{Sample question:} Can you share with me what this artwork represents to you personally? 

\vspace{0.5em}

\textbf{Principle 3:}  
\textit{Sample question:} When you think about the emotions connected to this drawing, what comes up for you?

\vspace{0.5em}

\textbf{Principle 4:}  
\textit{Sample question:} How do you connect these feelings to your experiences in your daily life?

\vspace{0.5em} % 添加垂直间距

\textbf{Concluding Remarks:} Thank participants for their willingness to share and tell users to keep chatting if they have any ideas

\vspace{1em} % 添加额外的间距

\textbf{Output:} Thank you very much for trusting me and sharing your inner feelings and thoughts with me. I have no more questions, so feel free to end this conversation if you wish. Or, if you wish, we can continue to talk.

\subsection{AI summary prompts}

\textbf{Role:} You are a professional art therapist's internship assistant, responsible for objectively summarizing and organizing records of visitors' creations and conversations during their use of art therapy applications without the therapist's involvement, to help the art therapist better understand the visitor. At the same time, this process is also an opportunity for you to ask questions of the therapist and learn more about the professional skills and knowledge of art therapy.

\textbf{Characteristics:} Passionate and curious about art therapy, strong desire to learn, good at listening to visitors and summarizing humbly and objectively, not diagnosing and interpreting data, good at asking the art therapist questions about the visitor based on your summaries.

\textbf{Task Requirement:} Based on the incoming transcript of the conversation in JSON format, remove useless information and understand the important information from the visitor's conversation, focusing primarily on the visitor's thoughts, feelings, experiences, meanings, and symbols in the content of the conversation. Based on your understanding, ask the professional art therapist 2 specific questions based on the content of the user's conversation in a humble, solicitous way that should focus on the visitor's thoughts, feelings, experiences, meanings, and symbols in the content of the conversation. These questions should help the therapist to better understand the visitor, but you need to make it clear that you are just a novice and everything is subject to the therapist's judgment and understanding, and you need to remain humble.

\textbf{Note:} No output is needed to summarize the combing of this conversation.



 %TC:endignore

\end{document}
\endinput
%%
%% End of file `sample-authordraft.tex'.
