\begin{table*}[!t]
\centering
\onecolumn
%\renewcommand{\arraystretch}{1.1} % Adjust vertical spacing
\small 
\begin{tabular}{|p{0.5cm}|p{0.4cm}|p{6.8cm}|p{8.5cm}|}
\hline
\textbf{} & \textbf{} & \textbf{Design Guidelines} & \textbf{Examples} \\
\hline
\multirow{5}{*}{\rotatebox[origin=c]{90}{\hspace{1em} \textbf{G1. Body Appearance} \hspace{1em}}} 
& G1.1
& Support disability representation in social VR avatars.
& Avatar system provides hearing devices for disability representation. \\ \cline{2-4}

& G1.2
&  Default to full-body avatars to enable diverse disability representation across different body parts.
& A full-body avatar can show a prosthetic left foreleg.  \\ \cline{2-4}

& G1.3
& Enable flexible customization of body parts as opposed to using non-adjustable avatar templates. 
& Avatar platforms should offer options to customize the size of each eye. \\ \cline{2-4}

& G1.4
& Prioritize human avatars to emphasize the ``humanity'' rather than the ``disability'' aspect of identity.
& VALID validated avatar library (left) and Ready Player Me avatar (right) present human avatars that can show intersecting identities of disabilities, race, and gender.  \\ \cline{2-4}

& G1.5
& Provide non-human avatar options to free users from social stigma in real life 
& A robotic avatar can represent left forearm amputation.  \\
\hline

\multirow{4}{*}{\rotatebox[origin=c]{90}{\hspace{0.1em} \textbf{G2. Avatar Dynamics} \hspace{0.1em}}} 
& G2.1
& Allow simulation or tracking of disability-related behaviors but only based on user preference.
& Avatar can show motor tics or not based on the user's preference. \\ \cline{2-4}

& G2.2
& Enable expressive facial animations that deliver a spectrum of emotions.
& Avatar can show diverse emotions, including anxiety, sadness, and happiness. \\ \cline{2-4}

& G2.3
& Prioritize equitable capability and performance over authentic simulation.
& The avatar with the wheelchair can move at the same speed as the avatar without the wheelchair.\\ \cline{2-4}

& G2.4
& Leverage avatar posture and motion to demonstrate the lived experiences of people with disabilities.
& The avatar representing a low vision person can show a different posture than the avatar representing a sighted person in a conversation.   \\
\hline

\multirow{6}{*}{\rotatebox[origin=c]{90}{\hspace{1em} \textbf{G3. Assistive Technology Design} \hspace{1em}}} 
& G3.1
& Offer various types of assistive technology to cover a wide range of disabilities.
& Users can add multiple types of mobility aids, such as crutches and prosthetic limb, to their avatars. \\ \cline{2-4}

& G3.2
& Allow detail customization of assistive technology for personalized disability representation.
& Users can adjust the color of the power wheelchair, like the cushions, wheels, and chassis cover. \\ \cline{2-4}

& G3.3
& Provide high-quality, realistic simulation of assistive technology to present disability respectfully and avoid misuse.
& The design of a white cane for people with low vision should follow the standardized color selection for such walking aids.\\ \cline{2-4}

& G3.4
& Focus on simulating assistive technologies that empower people with disabilities, rather than those that highlight their challenges.
& The manual wheelchair should not have handles, demonstrating that it’s designed for users who want to navigate independently instead of being pushed. \\ \cline{2-4}

& G3.5
& Demonstrate the liveliness of PWD through dynamic interactions with assistive technology.
& Avatar controls the manual wheelchair through pushing. \\ \cline{2-4}

& G3.6
& Avoid overshadowing the avatar body with the assistive technology. 
& Users can change the size of assistive technology to match with their avatar size.  \\
\hline

\multirow{2}{*}{\rotatebox[origin=c]{90}{\parbox{1.4cm}{\centering \textbf{G4. Peri-}\\\textbf{pherals}}}} 
& G4.1
& Provide suitable icons, logos, and slogans that represent disability communities.
& An avatar wearing a T-shirt with a rainbow and infinity symbol to represent the autism spectrum disorder community. \\ \cline{2-4}

& G4.2
& Leverage spaces beyond the avatar body to present disabilities.
& An avatar with a floating bubble overhead, showing a level of social energy. \\
\hline

\multirow{3}{*}{\rotatebox[origin=c]{90}{\parbox{2.5cm}{\centering \textbf{G5. Interface}}}} 
& G5.1
& Distribute disability features across the entire avatar interface rather than gathering them in a specialized category.
& Walking canes and wheelchairs are included under the accessory category along with items like glasses, hats and bags. \\ \cline{2-4}

& G5.2
& Use input controls that offer precise adjustments whenever possible.
& Avatar should offer a slider to change the length of limbs. \\ \cline{2-4}

& G5.3
& Offer an easy control to turn on/off or switch between disability features.
& Users should be able to turn disability-related features on and off with a single click. \\
\hline

\end{tabular}
\caption{Our \change{\textit{initial}} 20 design guidelines for inclusive avatars. }
\Description{The table titled "Table 5. Our original 20 design guidelines for inclusive avatars" presents 20 guidelines grouped into five categories: Avatar Body Appearance, Avatar Dynamics, Assistive Technology Design, Peripherals, and Interface. Each guideline labeled with identifiers (e.g., G1.1, G2.2). The "Examples" column illustrates each guideline with practical implementations, such as using full-body avatars to represent prosthetic limbs or allowing customization of assistive technology components. Each row corresponds to a specific guideline, presenting a structured layout that aligns guidelines with illustrative use cases for creating inclusive avatars.}
\label{tab:overview_original}
\end{table*}