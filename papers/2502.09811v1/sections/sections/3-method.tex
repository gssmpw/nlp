%TC:ignore
\begin{table*}
    \centering
    \begin{tabular}{
    >{\raggedright\arraybackslash}p{1.55cm} c|
    >{\raggedright\arraybackslash}p{0.7cm} c|
    >{\raggedright\arraybackslash}p{3.4cm} c|
    >{\raggedright\arraybackslash}p{3.3cm} c|
    >{\raggedright\arraybackslash}p{1.56cm} c}
        \toprule
        \multicolumn{2}{c}{\textbf{Gender}} & \multicolumn{2}{c}{\textbf{Age}} & \multicolumn{2}{c}{\textbf{Race}} & \multicolumn{2}{c}{\textbf{Disability}} & \multicolumn{2}{c}{\textbf{Avatar Experience}} \\ \toprule
        Female & 53.3\% & 18-24 & 28.3\% & White & 53.3\% & Mobility disability & 28.3\% & Meta Avatar & 53.3\% \\ 
        Male & 38.3\% & 25-34 & 33.3\% & Black or African American & 20.0\% & Mental health conditions & 20.0\% & VRChat & 31.7\% \\ 
        Non-binary & 8.30\% & 35-44 & 16.7\% & Asian & 11.7\% & Sensory disability & 11.7\% & Minecraft & 15.0\% \\ 
        & & 45-54 & 8.30\% & Hispanic or Latino & 6.70\% & Neurodivergence & 11.6\%  & Snapchat & 15.0\% \\ 
        & & 55+ & 13.3\% & Mixed race & 6.70\% & Chronic illness & 1.70 \% & AltspaceVR & 10.0\% \\ 
        & & & & American Indian or Alaska Native & 1.70\% & Multiple disabilities & 26.7\% & Rec Room & 6.70\% \\
        \bottomrule
    \end{tabular}
    \Description{The table is titled "Table 1. Participants’ demographics distribution (in percentage)." It presents the demographic breakdown of study participants, including categories for gender, age, race, disabilities, and avatar experiences. The percentages for each category are provided.}
    \caption{Participants' demographics distribution (in percentage)%in Phase 1 \yuhang{what is Phase 1? this has never been mentioned...}
    , including gender, age, race and ethnicity, self-reported disabilities, and avatar experiences.} %\yaxing{explain AIAN} \yuhang{need to pay attention to the ordering... i noticed that you order race, disability using an alphabetical order, which is not common. Let's order these two columns based on distribution from high to low. Table is another type of visualization to enable people to see the data. need to present things in a way that best highlight your idea. so what you want people to see first in the table?---use that to guide your table design decision.}} \kexin{DONE}
\label{tab:pwd_demographics}
\end{table*}
%TC:endignore

\section{Guideline Generation}

\change{To derive a comprehensive set of design guidelines, we integrated prior design implications for inclusive avatars \cite{zhang2022, zhang2023, assets_24, chronic_pain_gualano_2024} %\yuhang{cite} 
with findings from a supplementary interview study with 60 PWD.} %To ensure the interview thoroughly cover various design aspects, we developed the study protocol based on a literature review on inclusive avatar research, so that we can summarize existing design implications and identify the missing pieces in prior literature. Given the review insights, we conducted a semi-structured interview to refine and expand upon current knowledge on designing inclusive avatars for PWD.} 
The interview study was approved by the Institutional Review Board (IRB) at our university. 

%\yuhang{need to mention that we also summarized implications from prior work---is our interview structure also based on prior implications? If so, we should mention that. Highlight the relationship between the literature review and the interview study---the interview study is to supplement and add to existing implications to ensure a comprehensive set of guidelines covering all various aspects.}

\subsection{\cameraready{Systematic Literature Review} of Inclusive Avatar Designs} \label{literature_review}

\change{We first conducted a \cameraready{systematic literature review} on avatar design for disability representation to (1) summarize prior design implications to integrate into our guidelines and (2) inform our interview study, thus covering various perspectives of avatar design to derive comprehensive guidelines. %Unlike prior works that openly explore PWD's self-presentation preferences \cite{zhang2022, kelly2023, assets_24}, our study has a strategic goal of deriving a set of comprehensive design guidelines. The protocol informed by the literature review enabled us to supplement prior design implications and cover a wide range of avatar design aspects.
}

%\yuhang{this is a good place to highlight the major differences between this study and prior work; they all use interview, then how does our interview script/method make a difference? You sort of getting there but not clear enough. Try something like "Unlike prior work that focus on open-ended questions to understand people's experiences and preferences, we adopt a well-structured interview by dividing our question into xxx. This allowed us to xxx, thus enabling us to derive comprehensive guidelines."---then talk about what exact you do. You also want to talk about how we come up with these design aspects---from prior work?} \yuhang{i think you also need to be clear about the method, we asked people to image an ideal avatar they want to create, right? and then go through each aspect of the avatar design decision and ask how they are relevant to disability.}   
%\yuhang{Then you need to highlight why we design a well-structured interview... to supplement prior implications and cover a wide range of aspects around avatar design to ensure the comprehensiveness of the guidelines?}

To collect prior work thoroughly, we searched on Google Scholar and representative HCI and AR/VR venues (i.e., ACM CHI, CSCW, IEEE VR, ISMAR), using keywords: ``inclusive avatars,'' ``avatar design,'' ``avatar representation,'' combined with ``people with disabilities,'' ``disability representation,'' and ``social virtual realities (or VR).'' We filtered the searched results by manually examining the full text of each paper and removing irrelevant and repetitive papers. \change{In the filtering process, we also identified papers discussing avatar designs for other minoritized identities, such as race \cite{Lee2014} and gender \cite{Morris_2023_women}. We included these papers in our review to ensure our interview protocol covers the intersectionality of different identities.}

As a result, we collected a list of \change{19} papers that focused on inclusive avatar designs and identified five key design aspects, including: (1) visual representation of avatars' physical appearance ~\cite{zhang2022, kelly2023, Freeman2020, DavisStanovsek2021, Lee2014, lee_2021, McArthur_2015, McArthur2014, do2023valid, Weidner_2023, Morris_2023_women};
(2) facial expressions and body movements \change{~\cite{assets_24, Gualano_2023, HerreraBailenson2021, Weidner_2023, kopf_2023, chronic_pain_gualano_2024}};
(3) design of assistive technology and their interactions with avatars ~\cite{zhang2022, ParkKim2022, zhang2023, isit_assets24};
(4) avatar's outfits, accessories, and peripheral design elements ~\cite{zhang2022, zhang2023, DavisStanovsek2021};
and (5) design of avatars representing multiple and intersecting identities ~\cite{kelly2023}.
%\yuhang{this aspect is unique and you ask very specific questions, but this is not elaborated in the next paragraph.}.
\change{We thus structured our interview protocol based on these five design aspects but encouraged participants to expand to other aspects if they can think of any.}

\change{Among the 19 papers, four papers \cite{zhang2022, kelly2023, assets_24, zhang2023} derived design implications on inclusive avatars for PWD, including seven implications from Mack et al. \cite{kelly2023}, four from Gualano et al. \cite{assets_24}, three from Zhang et al. \cite{zhang2022}, and one from Zhang et al. \cite{zhang2023}. We thoroughly collected these design implications and cross-referenced them with our interview findings, ensuring the comprehensive coverage of derived guidelines.} %\yuhang{add a paragraph about that, there are xx papers focus on avatar design for PWD, specify which paper have how many guidelines. We also thorough collect the design implications from these papers, and generate guidelines by combining them with our interview findings.}

\subsection{Participants}
\change{We further conducted a semi-structured interview with 60 participants with diverse disabilities to supplement prior insights.} To ensure a wide coverage, %\yuhang{what is the rationale for broad recruitment? never jumping to your decision without motivation.} 
we broadly recruited people who identified as PWD without restricting the disability types. We recruited via multiple channels, including the mailing lists of non-profit disability organizations (e.g., the United Spinal Association), disability and VR communities on mainstream social media platforms (e.g., DisabiliTEA on Discord, r/amputee on Reddit), referrals from recruited participants, and our university’s research forum. Interested participants completed a screening survey with their age, disability conditions, and general social VR and avatar experiences. Eligible participants must (1) be over 18 years old, (2) identify as having disabilities, and (3) have experience using avatars in social virtual worlds and VR platforms. We limited our recruitment to individuals who spoke English. If selected, participants were asked to complete an oral consent at the beginning of the study.

We recruited 60 participants, with ages ranging from 18 to 71 ($mean$ = 34, $SD$ = 13). Our participants covered a comprehensive set of disabilities, with a total of 26 unique disabilities in six categories, including mobility disabilities (e.g., amputee, cerebral palsy), mental health conditions (e.g., anxiety, depression), sensory disabilities (e.g., blind or low vision), neurodivergence (e.g., autism, ADHD), chronic illness conditions (e.g., Epidermal Nevus Syndromes), and multiple disabilities. %(e.g., Friedrich’s ataxia that affected both speech and mobility abilities). 
These categories encompassed both visible and invisible disabilities, \change{presenting a wide coverage on disability characteristics and social implications}. \change{Notably, sixteen participants experienced more than one type of disability, offering insights into managing multiple disabilities in avatar design.} 

Moreover, \change{as disability may intersect with other identities} (e.g., gender, race, and ethnicity) and \change{result in unique identity representation strategies} \cite{Crenshaw1990, kelly2023}, we recruited people with diverse gender and racial identities \change{to discuss potential conflicts and trade-off in presenting multiple identities via avatar design}. As a result, our participants included 32 female, 23 male, and five non-binary. Regarding race, 32 participants identified as White, 12 as Black or African American, seven as Asian, four as Hispanic or Latino, four as mixed race, and one as American Indian or Alaska Native. %\yuhang{start with disabilities first, and race... also need to motivate why we care about race diversity. and whether we have done any recruitment strategy to achieve that.} 
%\yuhang{i believe i've given general tips about how to write number in paper. make sure that you follow that.} \kexin{as this sentence contains both big and small number, should i use all numerals for consistency instead?}
%Our participants had a wide range of disabilities, with a total of 26 unique disabilities identified in the data. This included mobility disabilities (e.g., amputee, cerebral palsy, muscular dystrophy), D/deaf or Hard of Hearing (DHH), blind or low vision (BLV), chronic illness (e.g., epidermal nevus syndrome, asthma), autism, attention deficit, mental health conditions (e.g., depression, anxiety), and neurological disability (e.g., seizures, epilepsy, tics) 

All participants had created and used avatars in social virtual environments. Their commonly used platforms included \textit{Meta Avatars} \cite{metaavatars}, \textit{VRChat} \cite{vrchat}, \textit{Minecraft} \cite{minecraft}, \textit{Snapchat} \cite{snapchat}, \textit{AltspaceVR} \cite{altspacevr}, and \textit{Rec Room} \cite{recroom}.
Table \ref{tab:pwd_demographics} presents participants' demographic breakdowns from different aspects.

\subsection{Procedure}
%\yuhang{the study is NOT semi-structured... this is not an open-ended study similar to prior work, instead, we thoroughly designed the script to cover all possible aspects of avatar design.} \yaxing{This is a structured interview} %via a video conferencing software (approved by our university) \yuhang{Zoom? no need to complicate things if not necessary} \yaxing{yep, I was going to suggest that}. 

The interview was a single-session study conducted in English, lasting approximately 60 minutes. We conducted interviews via Zoom, and participants received \$25/hour compensation upon completion. \change{Our interview included three phases: an introduction phase, an avatar design phase, and a reflection phase.} 

\change{In the introduction phase, we first went through the consent form with participants and obtained their oral consent. We then asked about participants' background information, including their}
%The interview included three phases. The first phase asked participants’ background, including 
demographics (i.e., age, gender, race, and ethnicity), self-reported disability, and experiences with avatars and social VR platforms. 

In \change{the avatar design phase}, we focused on understanding participants' design decisions and disability expression preferences. \change{To better contextualize participants and encourage them to consider avatar design details}, we employed an example-oriented approach by first asking participants to imagine and describe their ideal avatar designs. \change{We then dived into the five design aspects informed by our \cameraready{systematic literature review} (Section \ref{literature_review})---physical appearance, facial expressions and body movements, assistive technology and interactions with it, outfits and accessories, and multiple and intersecting identities---and asked about participants' design choices for that ideal avatar along each aspect, respectively.} 

\change{Specifically for each design aspect,}  
%Unlike prior works that used open-ended questions to explore PWD's self-representation practices \cite{zhang2022, kelly2023, gualano2024try}, we developed a well-structured interview by asking questions that were categorized into five key aspects of avatar designs, which we identified from a systematic literature review on inclusive avatar design. This approach allowed us to thoroughly consider and include the avatar designs discussed in prior works, thus enabling us to generate comprehensive guidelines.
%We then structured our interview protocol based on these five design aspects.
%To generate actionable guidelines with concrete avatar examples, we designed our protocol to be example-oriented by first asking participants to imagine and describe their ideal avatar design. %We dived into each aspect of the avatar details and investigated their design choices. 
we broke it down into the smallest possible design elements to fully explore the avatar's potential for disability expression. For example, when asking about the physical appearance of avatar, we began from a higher level of general avatar types to the visual details of each component of avatar body, including physical composition, \change{skin color,} body size and shape, the proportion of different body parts, facial features (\change{e.g., appearances of eyes, cheekbones, nose, mouth, forehead, and jawline}), and hairstyles. 
For each design element, we asked participants about how they wanted to design it, followed up with if and how their choices may reflect their disabilities and the rationales. %encouraging them to think critically and thoroughly. 
\change{For the multiple and intersecting disability aspect, we included a few dedicated questions to ask about how participants would like to represent fluctuating disabilities and multiple identities if they have any, whether they have encountered any conflicts in representing multiple identities, and their coping strategies.}

\change{Lastly, in the reflection phase, we asked participants to discuss other design aspects if they could think of any and whether and how they may express their disability along that aspect. Moreover, we asked about their concerns of expressing disabilities on avatars and preferences for avatar interface design. %We also summarized the discussed design aspects and checked if there were any other uncovered aspects that could represent their disabilities. 
We concluded the study by discussing the general insights on designing avatars for PWD, including what to do and to avoid.}

%\yuhang{In xx phase,} we concluded the study by asking participants' concerns in using avatars with disability representation features and their desired way to interact with avatar customization interface. We provide the full interview protocol in Appendix \ref{protocol_pwd}. % to demonstrate each design elements we asked and how we asked them. 

% To prompt participants to think thoroughly, we delved into each aspect of the avatar design and asked how their choices may reflect their disabilities, if applicable. For example, we asked, ``What body parts do you want your avatar to have?'', followed by ``How does the choice of body parts reflect your disabilities, if at all?'' 

%We began from a higher level of general avatar types to the visual details of each component of an avatar (e.g., physical composition, size and proportion of different body parts, facial appearance). We also explored movement-based representation, a unique attribute of social VR avatars, by asking participants' preferences about avatar's facial animations and movement characteristics (e.g., moving speed and patterns, range of motion). If participants used any assistive technologies in daily life, we further asked whether and how they desire to represent the assistive technologies, including the details of appearances, compatibility with avatar's motion, and desired interactions with avatars. Additionally, we asked participants' preferences for outfit, accessories, and the surrounding spaces of avatars. We encouraged participants to be creative through brainstorming design ideas and fully leveraging the potentials of social VR avatars. \yuhang{this is a linear list of what you did... but i'd like you to describe your method and decisions with rationales from a higher-level first. It should never be a list going on and on.}

%\yuhang{rationale first...this is an important and unique part of the study, but i don't think this should be a separate phase in your study... it's part of the avatar design aspect. And we decide to ask these also because of insights from prior work, correct? Think clearly about what you involve in your study and why you involve them.} 
%The third section of the interview explored the complexity of representing fluctuating disabilities and intersectional identities through social VR avatars. For example, we asked whether participants have disability conditions that fluctuated day by day, and how would they like to convey the fluctuations through avatars if applicable. If participants have multiple disabilities, we also asked their choices in presenting those. Moreover, since each person's identity is multifaceted (e.g., gender, race, age, disability) and often closely intersect with each other, we investigated preferences, challenges, and conflicts \cite{kelly2023} when representing multiple identities through avatar designs. 


\subsection{Analysis \& Guideline Generation} \label{analysis_phase1}
Upon participants' consent, we audio-recorded and transcribed all interviews using an automatic transcription service. Two researchers reviewed the transcripts and manually corrected all transcription errors. We conducted thematic analysis \cite{Braun2006Thematic, Clarke2015Thematic} to identify repetitive patterns and themes in the interview data. First, two researchers open-coded ten identical sample transcripts (more than 15\% of the data) independently at the sentence level. We created an initial codebook by discussing and reconciling codes to resolve any differences. Next, the two researchers split up the remaining transcripts and coded them independently based on the initial codebook. Throughout this process, they regularly checked each other's codes to iterate upon and refine the codebook until a complete agreement. Meanwhile, a third researcher oversaw all coding activities to ensure a high-level agreement. As we reached complete agreement for our coding, inter-coder agreement was not necessary~\cite{mcdonald2019reliability}. \change{The final codebook contains 239 codes.}

\change{Our theme generation adopted a mixed analysis method, where we used design recommendations from prior literature ~\cite{zhang2022, zhang2023, kelly2023, assets_24} to inform deductive themes, along with an inductive method to generate new themes and guidelines from the interview data via an affinity diagram \cite{clarke2017thematic}.} % \yuhang{ref}.}
%We further categorized the codes into high-level themes and sub-themes using axial coding and affinity diagram. 
After the initial themes were identified, researchers cross-referenced the original data, the codebook, and the themes, to make final adjustments, ensuring that all codes fell in the correct themes. %Our analysis resulted in xx codes and xx themes, and the details could be found in theme table in Appendix \ref{}. 
%\yuhang{there is a gap between themes from your study and the guidelines. How did you combine the themes and literature review into guidelines?---refer to Section 3 of the HAI guideline paper}