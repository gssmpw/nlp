\section{Introduction}

Social Virtual Reality (VR) has gained increasing popularity in the online social ecosystem, where multiple users can meet, interact, and socialize in the form of avatars \cite{Freeman2020}. Unlike 2D social media, social VR is equipped with body-tracking avatars, synchronous voice chat, and kinesthetic interactions. These features enabled rich communication modalities through the avatar's body gestures, facial expressions, and even gaze interactions \cite{maloney_nonverbal}. Many users thus view the avatar as an extension of themselves and a major medium for self-expression \cite{Manninen2007, Freeman2021}.

People with disabilities (PWD), a prevalent yet marginalized community \cite{WHO2023}, have shown increasing presence in social VR \cite{zhang2022, meta_horizon_accessibility}. 
Like any other user group, PWD curate social images by customizing avatars, and many prefer disclosing their disabilities via avatars as a core part of their identities \cite{zhang2022, kelly2023}. Previous research has identified PWD's needs and strategies to express disabilities in virtual worlds and social VR spaces \cite{zhang2022, kelly2023, Gualano_2023, zhang2023}. For example, Zhang et al. \cite{zhang2022} interviewed PWD and revealed that they preferred to disclose disabilities through various strategies, such as showing disability authentically or presenting a certain aspect of their disability selectively. Some other works \cite{kelly2023, Gualano_2023} focused on people with invisible disabilities (e.g., ADHD) and identified their preferences and needs for disability expression in various social VR contexts. 

Despite the tremendous needs, enabling PWD to flexibly and properly express their disabilities remains challenging \change{in practice}. \change{Many social VR platforms do not support disability representation ~\cite{zhang2022}}, and some avatar designs for PWD can be misleading (e.g., avatar on a virtual hospital wheelchair \cite{kelly_AI24, zhang2023}), further perpetuating misconception and bias towards disability. \change{While prior research has derived various design recommendations for disability representation \cite{zhang2022, kelly2023, assets_24}, they were dispersed across multiple venues, making it challenging for practitioners to access and apply comprehensively}. \change{Moreover, no research has systematically validated these design recommendations.} 
%they tend to be general and lack practical guidance of how developers and designers can exactly implement them. They are also scattered throughout different venues, making them to difficult to find and use in practice. 
\change{There is a pressing need for a centralized, comprehensive, and validated set of design guidelines that are easy to adopt, disseminate, and update \cite{human_ai_guidelines}, facilitating the practical integration of inclusive avatar design into social VR development.}


% making it challenging for VR practitioners to access and apply comprehensively. Centralized, comprehensive, systematically-validated guidelines to enable easy adoption, integration, and dissemination by the VR practitioner community.}   

%\yuhang{i agree with Yaxing's feedback... need to highlight that, while prior work have started investigated these needs, they focus on a subgroup of disabilities and no concrete, actionable guidelines are derived and evaluated to enable designers and developers to generate suitable support to enable disability expression. As a results, even with the  notion of supporting inclusive avatars, some design options could be misleading and bring misconception, e.g., xxx }

%Although some prior works \cite{zhang2022, zhang2023, kelly2023, Gualano_2023} have explored how PWD prefer to disclose and represent disabilities, they present a few limitations that need to be addressed. First, most of the prior works has studied a narrow range of disability types and included only a small number of participants. This has left many user groups within the disability community remain underrepresented. Second, previous research primarily focused on the static, visual aspects of avatars representation, such as adding assistive technologies to avatars \cite{zhang2022, zhang2023} or putting community icons to avatar's cloth \cite{kelly2023, Gualano_2023}. But they had not fully explored avatars' potentials in the embodied social environment, where disability representation could extend beyond visuals to include motions, postures, facial expressions, and even mannerisms. Last but not least, although a myriad of works %\yuhang{i don't think there are myriad of works...this field is still very nascent...} 
%have identified the growing interests of representing disabilities on avatars, options for doing so are very limited or even completely missing on mainstream social VR platforms today \cite{zhang2022, ribeiro2024towards}. \kexin{further revise to emphasize the importance of proper design: some disability features may perpetuate PWD and lead to unwanted experiences, e.g., ...}
%This highlights a pressing need for developers and designers to create inclusive avatars, ensuring that PWD can express themselves equally. However, a few important questions remain unaddressed: How can we effectively make developers and designers aware of users' needs so that they are motivated to address them? And with this awareness, how can they properly design disability representation features that meet the needs and preferences of PWD? What are something that they should do and avoid when developing avatars for PWD? %\yaxing{This paragraph needs to be revised - it didn't show the lack of design guideline in how to design inclusive avatars, and it also needs to justify why we focus on "guidelines", rather than something else, e.g., design a set of avatars or build a new platform, etc.}

%Developers and designers in the VR industry urgently need to create inclusive avatar options to ensure PWD have equal access to express themselves like other user groups. \kexin{further revise the last point -> how to connect to developer and designers} Yet, without a thorough understanding of what PWD want for avatar representation and how to represent diverse types of disabilities, developers and designers are unable to effectively address PWD's self-representation needs.

To fill this gap, we generated and evaluated a comprehensive set of avatar guidelines to provide actionable guidance and examples for VR developers and designers, enabling them to integrate avatar features for suitable disability representation without bringing bias and misconceptions. To achieve this goal, we conducted an interview with 60 PWD with diverse types of disabilities to deeply understand their preferred disability expression methods and the potentially biased designs they want to avoid. \change{Combining our interview findings and a systematic literature review of prior implications on inclusive avatar design (e.g., \cite{zhang2022, zhang2023, kelly2023, assets_24})}, we derived 20 \change{initial} %\yuhang{number} 
guidelines that cover five %\yuhang{number} 
aspects of avatar design, including avatar body appearance (G1), body dynamics (G2), assistive technology design (G3), peripherals around avatars (G4), and customization controls in the avatar interface (G5). %\yuhang{list the five guideline themes}. 
% Our participants cover a wide range of disabilities, race, and sexuality to be comprehensive about PWD's self-representation preferences. 
%We thoroughly explore the potential of avatars to represent disabilities through various modes, including appearance, posture, motion, facial animations, and interactions with the environment. 
We then validated the guidelines by conducting a heuristic evaluation with 10 VR developers and designers. %Drawing on their expert knowledge and practical experiences with avatar development, we collect feedback on the guideline's applicability, clarity, actionability, and importance. 
Our evaluation results confirmed that the guidelines comprehensively cover a broad range of disability expression methods through diverse avatar design aspects, are applicable to various avatar platforms, and are actionable to implement with concrete avatar examples. \change{Based on developers and designers' feedback, we further refined the guidelines to make them more accessible and actionable to industry practitioners (e.g., adding applicable use cases, defined the customization scope). As a result, we ended up with a set of 17 finalized guidelines upon practitioners' feedback.} %\yuhang{may need update after finalizing the guidelines}
%xxx \yuhang{summarize the findings from the evaluation study.}

Our paper presents two major contributions. First, \change{we contribute a centralized, comprehensive set of guidelines with concrete evidence and implementation examples to support VR practitioners in creating appropriate avatars for PWD.} %Our 20 design guidelines thoroughly cover diverse disability expression methods through five design aspects of avatars. 
Second, we validated \change{and refined} our guidelines with VR practitioners and confirmed the coverage, applicability, and actionability of the guidelines. 
\cameraready{To ensure practitioners can easily adopt, share, and update the guidelines, we have open-sourced the full version of the guidelines on GitHub\footnote{Inclusive Avatar Guidelines and Library.} \url{https://github.com/MadisonAbilityLab/Inclusive-Avatar-Guidelines-and-Library}.}. 
%\kexin{moved to footnote}

%Our 20 design guidelines covere five aspects of avatars, including avatar's body appearance, dynamics (i.e., facial expression, posture, and body motion), assistive technologies design, pehripherals around avatars, and interface design and customization 
%Second, we conducted heuristic evaluation ......
%Third, we suggested ....





%For example, Zhang et al. \cite{zhang2022} interviewed people with visual and hearing impairments to study their self-representation strategies in social VR. They found that PWD used a spectrum of strategies to shape their self-images in social VR, ranging from accurately showing one’s disability as in the real life to hiding disabilities for a different self. Mack et al. \cite{kelly2023} and Gualano et al. \cite{Gualano_2023} focused on the representation of disabilities that do not have immediately visible characteristics, highlighting that the disability disclosure was dynamic and highly context-dependent in social VR. \todo{highlight PWD's need to self-represent through avatars -> connect to our paper}

%However, more research is needed to understand disability representation through social VR avatars for several reasons. 


% strength of our guideliens: 
% - comprehensive - not limited to static visual appearance, but we covered 

% Even though some instances of disability representations have been seen in the social virtual spaces, they often depict PWD in a degrading, patronizing, or stereotypical way \cite{zhang2023, heung2022}.


% Contribution: 
% 1. involve PWD in the design process
% 2. cover multiple and diverse types of disabilities -> very comprehensive
% 3. focus on how -> propose actionable guidelines with concrete examples
% 4. seek experts' feedback in improve the guidelines, test the usability of it through heuristic evaluation 


% our guideline is generated from first-person perspectives from users with disabilities


% justify the need of guideline -> what is gudieline purpose (don't mix that with the avatar library)


% Despite that instances of disability representations have been increasingly seen in social media, they often depict PWD in a degrading, patronizing, or stereotypical way \cite{zhang2023, heung2022}. 

%Zhang et al. specifically studied how presenting disabilities on avatars affected PWD's experiences in social VR \cite{zhang2023}. The findings revealed that avatars with disability representations triggered more abelist harassment, largely due to people's misconceptions that such avatars are designed for trolling or memes instead of identity representation. One key way to correct the misconceptions is to improve the avatar design by having high-quality, polished disability representation features. Yet, no research has comprehensively examined how PWD would like to represent themselves through avatars, nor is there a standard design framework for practitioners to follow for inclusive and respectful disability representations. 

% Promise: What is a technology that promises us something good?
% Problem: What is the problem that prevents this technology from realizing its promise?
% Prior work: What is the most representative prior work that has attempted to solve this problem and why did it fail?
% Proposed solution: What is your proposed solution given where prior work fell short?
% Proof of contribution: What is your proof to validate that your proposed solution contributes to solving that problem?


%tell developers/HCI practitioners why making avatar inclusive is so important | inspiration: https://developer.apple.com/videos/play/wwdc2021/10275