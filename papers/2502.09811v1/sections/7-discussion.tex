\section{Discussion}

\change{We derived a centralized, comprehensive, and validated set of guidelines with concrete evidence and actionable implementation examples. Through a systematic literature review and interview with 60 PWD,}
%We are the first that derived a set of 20 design guidelines to support inclusive avatar representation. Through a well-structured interview with 60 PWD, 
our guidelines thoroughly covered diverse disability expression strategies through five design aspects, including avatar appearance (G1), body dynamics (G2), assistive technology design (G3), peripherals around avatars (G4), and avatar interface design and customization controls (G5). %We equipped each guideline with concrete avatar examples to demonstrate the implementation cases. 
Moreover, we evaluated and iterated the guidelines with 10 AR/VR experts. The evaluation results suggested that: (1) the guidelines comprehensively covered a broad range of disability expression methods through avatar designs; (2) practitioners in the industry can use guidelines to identify real problems in existing avatar platforms, proving that the guidelines are applicable; (3) the guidelines were easy to understand and actionable for avatar development in practice. 

\change{Based on experts' feedback, we further revised the guidelines to make them more concise and actionable for industry practitioners. The finalized version contains 17 revised design guidelines with detailed descriptions, concrete evidence, recommendation levels, and implementation examples.} We believe the guidelines can serve as a valuable resource that supports industry practitioners in designing inclusive avatars for PWD. 

In the following sections, we discuss the guidelines that spurred interesting conversations among practitioners, delving into their implementation considerations and  applicable use cases, and envisioning future directions.

\change{\textbf{Centralized Guidelines with Actionable Implementation Details.}
Building upon the valuable design implications derived by prior works \cite{zhang2022, zhang2023, kelly2023, assets_24}, we further advance the field of inclusive avatar designs by 1) centralizing relevant design implications, 2) strengthening them with implementation details, and 3) adding new design guidelines.

When generating the guidelines, we cross-referenced the prior implications with our interview findings, aiming for a comprehensive and centralized set of guidelines that are easy to apply, share, and update. Such centralized guidelines are pressingly needed for industry practitioners, as they often have limited access to, and find it hard to locate such resources, which are the two main challenges in translating academic research into industry practices~\cite{Colusso2017}. To better integrate inclusion in VR industry, more translational efforts are needed in future research, such as website collecting various resources \cite{xraccess} and open-source toolkits \cite{zhao2019seeingvr}, to enable easy access to high-quality resources for the practitioners. 

Among the finalized 17 guidelines, seven (G1.3, G1.4, G2.1, G2.3, G3.4, G5.1, G5.2) %\yuhang{label number Gx.x for new guidelines} 
are newly generated from interview findings, and 11 are expanded from prior literature with concrete evidence (e.g., applicable use cases, served user groups), scope, and implementation examples. For example, both Zhang et al. \cite{zhang2022} and Mack et al. \cite{kelly2023} suggested offering diverse types of assistive technologies (G3.1). We concretize this implication by specifying a list of the five most preferred types of assistive technologies, making it more actionable with a concrete implementation scope. %, as suggested by Colusso et al.~\cite{Colusso2017}. 
With the collective effort of prior works and our supplemental large-scale interview, we believe this set of guidelines can serve as a valuable resource for industry practitioners to practice upon.}

\change{\textbf{Flexible Applications based on Use Cases.}} Our guidelines aim to optimize for the preferred identity representations of PWD in social VR contexts, and we do not expect all platforms to implement every guideline. We \change{recognize the compounding factors (e.g., cost-effectiveness, development effort, user size) that may affect practitioners' decisions of following a guideline.} We also anticipate situations where the suggestions in the guidelines may not align well with the overall theme or aesthetic style of a social VR platform. For example, prioritizing human avatars as suggested in G1.4 may conflict with platforms featuring fantastical themes. In these cases, developers and designers may choose to follow the guidelines that apply to their use cases. \change{In the revised guidelines, we attempted to resolve this tension by assigning recommendation levels (i.e., highly recommended and recommended) to each guideline, clarifying their priority to better inform practitioners' decision-making.}


\change{\textbf{Balance Safety and Diverse Representation}.} Our guidelines briefly touch on the safety issues in expressing disabilities on avatars. The avatar-based disability representation could trigger explicit and embodied harassment targeting at PWD \cite{zhang2023}. Our guidelines proposed some reactive ways to mitigate the negative experiences, such as G5.3 (i.e., offering easy control to turn on and off disability features). %\yuhang{ref}. 
Future work should explore more proactive strategies to prevent harassment from happening when using avatars with disability-related features. It is imperative to ensure PWD represent themselves safely in social VR.

\change{\textbf{Multimodal Disability Representations.}} Lastly, we note that our work primarily focus on deriving guidelines through the visual aspects of avatar designs. Beyond visual features, avatar voice is another critical channel that can reveal and represent a user's identity in real life \cite{Povinelli_voice_2024}. \change{However, using voice for disability representation could lead to possible tensions and conflicts. For example, people with speech disabilities (e.g., stuttering) may not want to disclose their disabilities, which often comes with substantial social penalties in real life, such as negative listener reactions and teasing \cite{wu2023stuttering}. Moreover, avatar voice could unintentionally reveal a person's voluntary identities, leading to negative user experiences \cite{Maloney2020Anonymity, freeman2022disturbing}. In these cases, some of our guidelines may not be fully applicable, such as G2.1 (i.e., allow simulation or tracking of disability-related behaviors based on user preferences).} 
We encourage future work to further study identity construction through voice in social VR. We hope the methodologies and outcomes presented in this work could inform and inspire future research to develop guidelines for inclusive social VR through multi-modal perspectives.  




% \subsection{Safe Use of Disability Representation Features}

% user violation (misuse of the options)
% developer violation (not providing such options)

% (too much freedom encourage user violations)

% a few guidelines shadow this concern, such as 
% G2.1 (allow the simulation of behavior only based on user's preference)
% G3.6. 

% developer need more knowledge: 
% G3.4 (focus on simulating assitive technology that empower PWD)

% \subsubsection{potential solutions} When and How to Use Guidelines
% Improvement:
% mentioned the avatar library 

% AI-generated avatars 


% tradeoff of comprehensiveness and actionability 


% subsections: 
% - guidelines
% - a section about actionability of guidelines (how flexible they should follow the guidelines, etc.)
% - a section about guidance of how to use guidelines?