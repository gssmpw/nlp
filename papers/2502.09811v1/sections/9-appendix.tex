\subsection{Study I: Interview with PWD} \label{protocol_pwd}
\subsubsection{Introduction Phase}
\begin{enumerate}
    \item What’s your age?
    \item How do you identify your gender? 
    \item What’s your ethnicity? 
    \item What disability do you have?
    \item Do you have any episodic disabilities, where the type or severity of symptoms fluctuate and cause a variation of abilities?
        \begin{itemize}
            \item If so, in what ways?
        \end{itemize}
    \item What social VR platforms have you used?
    \item In general, what aspects of your identities would you like to convey through avatar? (prompt: Do you want to reveal gender through avatars? Disability? Race? Age?)
        \begin{itemize}
            \item (Go through each of the mentioned identities:) Why? 
            \item How do you want to represent the disability? 
            \item (for disability, if not:) why not?
        \end{itemize}
    
\end{enumerate}

\subsubsection{Avatar Design Phase}
In the following, we will focus on understanding your preferences for presenting disability identities on avatars. We will delve into each aspect of avatars and ask how you would like to design your ideal avatars. We will go through from a higher level of general avatar types to the details of each component of an avatar. Please think through with me, imagine me as a design agent, and we will discuss the avatar aspects step by step. Let’s start with the general types of avatars.

\textbf{Body Appearance:}
\begin{enumerate}
    \item What types of avatar do you want?
        \begin{itemize}
            \item Why?
            \item How does it relate to your disability identity representation, if at all? 
        \end{itemize}
    \item What body parts do you want your avatar to have? 
        \begin{itemize}
            \item Why?
            \item How does this physical composition of your avatar reflect your disability, if at all?
        \end{itemize}
    \item What’s the desired skin color of your avatar? 
        \begin{itemize}
            \item Does the skin color relate to your disability identity or its intersection with other identities? If so, how do you think it could relate? 
        \end{itemize}
    \item How tall do you want your avatar to be? 
        \begin{itemize}
            \item Why? 
            \item How does avatar's height reflect your disability, if at all?
            \item What’s the preferred body shape of your avatar?
            \item What are your preferences for the size and proportion of your avatar’s different body parts?
                \begin{itemize}
                    \item Do you have any body parts that you want to highlight/distinguish in comparison to other parts? OR: Do you want to customize any body parts? 
                    \item Why?
                \end{itemize}
            \item How does the choice of body shape, body size, and body proportion reflect your disability, if at all?
        \end{itemize}
    \item Considering the facial features of your avatar, how do you want to design the eyes? How about cheekbones, nose, mouth-lips, forehead, jawline of your avatar? (go through each of them with the following question; if not, just skipped)
        \begin{itemize}
            \item Why? 
            \item How does the appearance of eyes, cheekbones, mouth, forehead, and jaw reflect your disability, if at all? 
        \end{itemize}
    \item What’s the preferred hairstyle for your avatar?
        \begin{itemize}
            \item How does the hairstyle reflect your disability, if at all?
        \end{itemize}
\end{enumerate}

\textbf{Body Motion:}
\begin{enumerate}
    \item What specific movements and expressions would you like to see in the avatar's facial features, such as the eyes, mouth, and eyebrows? 
        \begin{itemize}
            \item How do the facial animations of the avatar reflect your disability, if at all?
        \end{itemize}
    \item What do you want your avatar's movement to be like? 
        \begin{itemize}
            \item How do you feel about the movement of the avatar's limbs? (prompt: ease of movement? The range of motion?)
            \item How does the movement of avatars reflect your disability, if at all?
        \end{itemize}
\end{enumerate}

\textbf{Assistive Technology:}
\begin{enumerate}
    \item Do you use any assistive technologies (AT) in real life?
    \item Do you want to add any AT to avatars?
        \begin{itemize}
            \item If yes, what are the AT you want to add?
            \item Why do you want to add the AT?
            \item How would you like the AT to look like? 
                \begin{itemize}
                    \item What do you think would be the proper size of the AT to your avatar body? What’s the ratio?
                    \item Do you have any specific aesthetic preferences for the AT? (e.g., color, brand)
                    \item How would you like the AT to be positioned?
                    \item Do you expect the AT to be interactive?
                        \begin{itemize}
                            \item If so, how do you imagine to interact with it?
                            \item Do you want others to interact with your AT? If so, how do you expect other people to interact with it? 
                        \end{itemize}
                \end{itemize}
                \begin{itemize}
                    \item If not, why?
                \end{itemize}
        \end{itemize}
    \item Do you have any other ideas about the AT design that you would like to share? 
\end{enumerate}

\textbf{Outfit, Accessories, and Avatar Surroundings:}
\begin{enumerate}
    \item How would you like to design the outfit of avatars? 
        \begin{itemize}
            \item Why do you want to design the outfit in this way? 
            \item Do you think the outfit design could reflect your disability? 
            \item If so, how? 
        \end{itemize}
    \item Would you like to have any accessories on avatars?
        \begin{itemize}
            \item If so, what are they? 
            \item Why do you want to have these accessories? 
            \item How do they represent your disability, if at all?
        \end{itemize}
    \item Do you think anything surrounding the avatar, instead of directly on/attached to the avatar, would help represent your disabilities? 
        \begin{itemize}
            \item If so, how? What do they look like?
        \end{itemize}
    \item Is there anything else you would like to have on your avatar for disability representation?
\end{enumerate}

\textbf{Multiple and Intersectional Identities:}
\begin{enumerate}
    \item If your disability condition fluctuates, how do you want to present the fluctuating conditions via avatar design, if at all?
    \item If you have multiple disabilities, how would you like to present them, if at all? 
    \item As each person’s identity is multifaceted (such as gender, race, disability, etc.), have you encountered or do you envision encountering any conflicts in representing multiple identities via avatar design?
        \begin{itemize}
            \item If so, what are the conflicts? 
            \item How do you deal with them through avatars?
            \item What features do you think can mitigate such conflicts?
        \end{itemize}
\end{enumerate}

\subsubsection{Reflection Phase}
\begin{enumerate}
    \item What factors would affect your willingness/decision to disclose disability via avatars?
    \item How do you want to interact with the avatar customization interface? 
    \item Are there any other design aspects that you think can represent your disabilities that we haven’t mentioned? 
    \item Do you have any other suggestions about how to design inclusive avatar for people with disabilities? (Prompt: What should we do? What should we not do?)
\end{enumerate}



\subsection{Study II: Protocol with VR Experts}

\subsubsection{Heuristic Evaluation Survey}
\label{heuristic_survey}
[Survey page starts with the presentation of a guideline.]
\begin{enumerate}
    \item How do you interpret this guideline?
    \item Do you have any questions or anything you found confusing about this guideline? Please share your thoughts below.
    \item Does this guideline apply (e.g., is it relevant) to the social VR avatars you reviewed? 
        \begin{itemize}
            \item Yes
            \item No (i.e., irrelevant or our of scope)
        \end{itemize}
    \item If not, can you explain why do you think this guideline does not apply to the avatars you reviewed? 
    \item Based on the social VR app you reviewed, how well does its avatars apply (i.e., follow or implement) this guideline? Please rate it on a scale of 1 to 5. %, with 1 being ``clearly violated'' and 5 being ``clearly applied.''

% \begin{tabular}{>{\raggedright\arraybackslash}p{9.5cm}p{0.5cm}p{0.5cm}p{0.5cm}p{0.5cm}p{0.5cm}}
% \hline
% \textbf{} & \textbf{1} & \textbf{2} & \textbf{3} & \textbf{4} & \textbf{5} \\
% \hline
% Level of Application - How well does the avatar apply this guideline? (1 = clearly violate, 5 = clearly apply) & $\circ$ & $\circ$ & $\circ$ & $\circ$ & $\circ$ \\
% \hline
% \end{tabular}

    \item Please provide an example to describe how it applies. (We encourage you to revisit the application and take screenshots to demonstrate the example. Please also feel free to annotate on the screenshots.)
    \item Please provide an example to describe how it violates. (We encourage you to revisit the application and take screenshots to demonstrate the example. Please also feel free to annotate on the screenshots.)
    \item As a developer/designer, please rate the following aspects of this guideline based on your experiences in developing and designing avatars for AR/VR applications. Please use the scale from 1 (worst) to 5 (best) for each dimension: 
        \begin{itemize}
            \item Level of Clarity - How clear and understandable is the guideline? (1 = very confusing, and 5 = very clear)
            \item Level of Actionability - How actionable do you find the guideline? (1 = very unactionable, and 5 = very actionable)
            \item Level of Importance - How important do you think it is to follow this guideline? (1 = very unimportant, and 5 = very important)
        \end{itemize}

% \begin{tabular}{>{\raggedright\arraybackslash}p{9.5cm}p{0.5cm}p{0.5cm}p{0.5cm}p{0.5cm}p{0.5cm}}
% \hline
% \textbf{} & \textbf{1} & \textbf{2} & \textbf{3} & \textbf{4} & \textbf{5} \\
% \hline
% 1. Level of Clarity - How clear and understandable is the guideline? (1 = very confusing, and 5 = very clear) & $\circ$ & $\circ$ & $\circ$ & $\circ$ & $\circ$ \\
% \hline
% 2. Level of Actionability - How actionable do you find the guideline? (1 = very unactionable, and 5 = very actionable) & $\circ$ & $\circ$ & $\circ$ & $\circ$ & $\circ$ \\
% \hline
% 3. Level of Importance - How important do you think it is to follow this guideline? (1 = very unimportant, and 5 = very important) & $\circ$ & $\circ$ & $\circ$ & $\circ$ & $\circ$ \\
% \hline

%     
% \end{tabular}

    \item Could you explain your rating of the clarity, actionability, and importance of this guideline?

\end{enumerate}


\subsubsection{Interview Protocol} \label{protocol_experts}

\begin{enumerate}
    \item How old are you? 
    \item What’s your ethnicity?
    \item In total, how long have you worked as a full-time AR/VR developer/designer?
    \item Is your current job centered on AR/VR development?
        \begin{itemize}
            \item If yes, 
            \begin{itemize}
                \item What company do you work for?
                \item What is your position in this organization?
                \item Can you briefly describe what you do for this position?
                \item How long have you been working at this company?
            \end{itemize}
            \item If not, 
            \begin{itemize}
                \item What’s your most recent AR/VR related job?
                \item What company did you work for?
                \item What was your position?
                \item How long did you work at this company? 
            \end{itemize}
        \end{itemize}
    \item Do you have any experience with avatar modeling, animation, or design? 
        \begin{itemize}
            \item If yes, could you briefly describe your experience?
            \item What development platforms have you used?
            \item Have you ever designed avatars for people with disabilities? or for any other underrepresented groups? Can you briefly describe it if so?
        \end{itemize}
\end{enumerate}

\textbf{Guidelines Evaluation and Survey Follow-up}
[Ask the following questions for each guideline:]
\begin{enumerate}
    \item In [social vr app], do you think its avatar applies (or follows) this guideline? 
        \begin{itemize}
            \item If yes: could you revisit [social vr app] and show us an example of how it applies? 
            \item If not:  could you explain your answer? 
        \end{itemize}
    \item In [social vr app], do you think its avatar violates (or breaks) this guideline?
        \begin{itemize}
            \item If yes: could you revisit [social vr app] and show us an example of how it violates?
            \item If not: could you explain your answer? 
        \end{itemize}
    \item How clear and understandable is this guideline? Please rate it on a scale of 1 to 5, with 1 being very unclear and 5 being very clear. 
        \begin{itemize}
            \item Could you explain your choice?
            \item If unclear, what confused you specifically? [prob: wording/phrasing? Or need further explanation?] 
            \item How can we improve this guideline to make it more clear? 
        \end{itemize}
    \item As a developer/designer, how actionable do you find this guideline is? Please rate it on a scale of 1 to 5, with 1 being very not actionable and 5 being very actionable. 
        \begin{itemize}
            \item Can you explain your choice? 
            \item If actionable, can you describe how you want to apply this guideline to the avatar systems in practice?
            \item If not actionable, what information do you need to implement this guideline?
        \end{itemize}
    \item How important do you think it is to follow this guideline when you develop/design avatars for your AR/VR applications? Please rate it on a scale of 1 to 5, with 1 being very unimportant and 5 being very important.
        \begin{itemize}
            \item Can you explain your choice? Why do you think it’s important/not important?
        \end{itemize}
\end{enumerate}

\textbf{Exit Interview}

\begin{enumerate}
    \item Is there anything that you think is important but we haven’t included here?
        \begin{itemize}
            \item If so, what are they?
        \end{itemize}
    \item Have you noticed any redundancies in the guidelines?
        \begin{itemize}
            \item If yes, what are they? Which parts make you feel they are redundant? 
            \item Why do you think they are redundant?
            \item How would you like to improve them?
        \end{itemize}
    \item Have you noticed any guidelines conflicting with each other? 
        \begin{itemize}
            \item If so, what are they? 
            \item Why do you think they are conflicting with each other?
            \item How would you like to improve them?
        \end{itemize}
    \item In general, do you think the language (wording and phrasing) is easy to understand? 
        \begin{itemize}
            \item Any suggestions to further improve the language?
        \end{itemize}
    \item Currently each guideline has four components: guideline-explanation-examples-visual demos. What do you think of this guideline structure?
        \begin{itemize}
            \item How would you like to further improve the structure? (prob: add/remove any components?)
        \end{itemize}
    \item Anything you like about this guideline?
    \item How should we further improve the guidelines to make it easier for you to understand and implement it in your project? 
\end{enumerate}




\subsection{Tables and Figures}

\newpage
\begin{table}[h!]
\centering
\small
\onecolumn
\begin{tabular}{p{2cm}p{2.3cm}p{2.1cm}p{5cm}p{4.5cm}}
\toprule
\textbf{Social VR Apps} & \textbf{Avatar Types} & \textbf{Components} & \textbf{Customization Options} & \textbf{Disability Representation} \\
\toprule
VRChat & Human/animals/ furry/robotic avatar, etc.  & Full body & Selection from default avatar interface; self-upload avatars & No default disability representations; allow self-upload avatars with disability features \\
\hline
Horizon Worlds & Cartoonish human avatar & Full body & Selection from default avatar interface; offer body appearance and facial details customization & Cochlear implants and hearing aids with color customization \\
\hline
Rec Room & Cartoonish human avatar & Upper body with floating hands & Selection from default avatar interface; offer body appearance and facial details customization with paywall & No default disability representations \\
\hline
Roblox & Lego-character-like avatar  /animal avatar & Full body & Selection from default avatar interface; offer body appearance and facial details customization with paywall & Wheelchair options; crutches; emotes with diverse emotions expressions; animation bundles \\
\hline
Multiverse & Robotic avatar & Upper body with floating hands & No avatar customization & No disability representation \\
\bottomrule
\end{tabular}
\Description{The table titled "Table 4. Review of Selected Social VR Platforms and Avatar Features" provides a comparative analysis of various social VR applications, focusing on avatar types, customization options, components, and disability representation.}
\caption{Review of Selected Social VR Apps and Avatar Features. Five mainstream social VR apps are reviewed for their avatar types, components, customization options, and offered disability representations.}
\label{tab:vr_platforms}
\end{table}



\begin{table*}[!t]
\centering
\onecolumn
%\renewcommand{\arraystretch}{1.1} % Adjust vertical spacing
\small 
\begin{tabular}{|p{0.5cm}|p{0.4cm}|p{6.8cm}|p{8.5cm}|}
\hline
\textbf{} & \textbf{} & \textbf{Design Guidelines} & \textbf{Examples} \\
\hline
\multirow{5}{*}{\rotatebox[origin=c]{90}{\hspace{1em} \textbf{G1. Body Appearance} \hspace{1em}}} 
& G1.1
& Support disability representation in social VR avatars.
& Avatar system provides hearing devices for disability representation. \\ \cline{2-4}

& G1.2
&  Default to full-body avatars to enable diverse disability representation across different body parts.
& A full-body avatar can show a prosthetic left foreleg.  \\ \cline{2-4}

& G1.3
& Enable flexible customization of body parts as opposed to using non-adjustable avatar templates. 
& Avatar platforms should offer options to customize the size of each eye. \\ \cline{2-4}

& G1.4
& Prioritize human avatars to emphasize the ``humanity'' rather than the ``disability'' aspect of identity.
& VALID validated avatar library (left) and Ready Player Me avatar (right) present human avatars that can show intersecting identities of disabilities, race, and gender.  \\ \cline{2-4}

& G1.5
& Provide non-human avatar options to free users from social stigma in real life 
& A robotic avatar can represent left forearm amputation.  \\
\hline

\multirow{4}{*}{\rotatebox[origin=c]{90}{\hspace{0.1em} \textbf{G2. Avatar Dynamics} \hspace{0.1em}}} 
& G2.1
& Allow simulation or tracking of disability-related behaviors but only based on user preference.
& Avatar can show motor tics or not based on the user's preference. \\ \cline{2-4}

& G2.2
& Enable expressive facial animations that deliver a spectrum of emotions.
& Avatar can show diverse emotions, including anxiety, sadness, and happiness. \\ \cline{2-4}

& G2.3
& Prioritize equitable capability and performance over authentic simulation.
& The avatar with the wheelchair can move at the same speed as the avatar without the wheelchair.\\ \cline{2-4}

& G2.4
& Leverage avatar posture and motion to demonstrate the lived experiences of people with disabilities.
& The avatar representing a low vision person can show a different posture than the avatar representing a sighted person in a conversation.   \\
\hline

\multirow{6}{*}{\rotatebox[origin=c]{90}{\hspace{1em} \textbf{G3. Assistive Technology Design} \hspace{1em}}} 
& G3.1
& Offer various types of assistive technology to cover a wide range of disabilities.
& Users can add multiple types of mobility aids, such as crutches and prosthetic limb, to their avatars. \\ \cline{2-4}

& G3.2
& Allow detail customization of assistive technology for personalized disability representation.
& Users can adjust the color of the power wheelchair, like the cushions, wheels, and chassis cover. \\ \cline{2-4}

& G3.3
& Provide high-quality, realistic simulation of assistive technology to present disability respectfully and avoid misuse.
& The design of a white cane for people with low vision should follow the standardized color selection for such walking aids.\\ \cline{2-4}

& G3.4
& Focus on simulating assistive technologies that empower people with disabilities, rather than those that highlight their challenges.
& The manual wheelchair should not have handles, demonstrating that it’s designed for users who want to navigate independently instead of being pushed. \\ \cline{2-4}

& G3.5
& Demonstrate the liveliness of PWD through dynamic interactions with assistive technology.
& Avatar controls the manual wheelchair through pushing. \\ \cline{2-4}

& G3.6
& Avoid overshadowing the avatar body with the assistive technology. 
& Users can change the size of assistive technology to match with their avatar size.  \\
\hline

\multirow{2}{*}{\rotatebox[origin=c]{90}{\parbox{1.4cm}{\centering \textbf{G4. Peri-}\\\textbf{pherals}}}} 
& G4.1
& Provide suitable icons, logos, and slogans that represent disability communities.
& An avatar wearing a T-shirt with a rainbow and infinity symbol to represent the autism spectrum disorder community. \\ \cline{2-4}

& G4.2
& Leverage spaces beyond the avatar body to present disabilities.
& An avatar with a floating bubble overhead, showing a level of social energy. \\
\hline

\multirow{3}{*}{\rotatebox[origin=c]{90}{\parbox{2.5cm}{\centering \textbf{G5. Interface}}}} 
& G5.1
& Distribute disability features across the entire avatar interface rather than gathering them in a specialized category.
& Walking canes and wheelchairs are included under the accessory category along with items like glasses, hats and bags. \\ \cline{2-4}

& G5.2
& Use input controls that offer precise adjustments whenever possible.
& Avatar should offer a slider to change the length of limbs. \\ \cline{2-4}

& G5.3
& Offer an easy control to turn on/off or switch between disability features.
& Users should be able to turn disability-related features on and off with a single click. \\
\hline

\end{tabular}
\caption{Our \change{\textit{initial}} 20 design guidelines for inclusive avatars. }
\Description{The table titled "Table 5. Our original 20 design guidelines for inclusive avatars" presents 20 guidelines grouped into five categories: Avatar Body Appearance, Avatar Dynamics, Assistive Technology Design, Peripherals, and Interface. Each guideline labeled with identifiers (e.g., G1.1, G2.2). The "Examples" column illustrates each guideline with practical implementations, such as using full-body avatars to represent prosthetic limbs or allowing customization of assistive technology components. Each row corresponds to a specific guideline, presenting a structured layout that aligns guidelines with illustrative use cases for creating inclusive avatars.}
\label{tab:overview_original}
\end{table*}





\begin{figure}[!ht]
    \centering
    \begin{subfigure}[b]{0.495\textwidth}
        \centering
        \includegraphics[width=\textwidth]{sections/images/application.png}
        \caption{Level of Application}
        \label{fig:application}
    \end{subfigure}
    \hfill % add some horizontal spacing
    \begin{subfigure}[b]{0.495\textwidth}
        \centering
        \includegraphics[width=\textwidth]{sections/images/clarity.png}
        \caption{Level of Clarity}
        \label{fig:clarity}
    \end{subfigure}
    \newline % to move the next pair of sub-figures to the next line
    \begin{subfigure}[b]{0.495\textwidth}
        \centering
        \includegraphics[width=\textwidth]{sections/images/actionability.png}
        \caption{Level of Actionability}
        \label{fig:actionability}
    \end{subfigure}
    \hfill
    \begin{subfigure}[b]{0.495\textwidth}
        \centering
        \includegraphics[width=\textwidth]{sections/images/importance.png}
        \caption{Level of Importance}
        \label{fig:importance}
    \end{subfigure}
    \caption{\change{Mean rating of applications (top left), clarity (top right), actionability (bottom left), and importance (bottom right) for each guideline. X-axis shows the ID of each guideline, and y-axis shows the mean value of 5-point Likert scale.}}
    \Description{This figure presents four bar charts evaluating various design guidelines based on four metrics: Level of Application, Level of Clarity, Level of Actionability, and Level of Importance. Each chart uses a Likert scale (mean values) with error bars representing variability.}
    \label{fig:complete}
\end{figure} 


\begin{table*}
\centering
\onecolumn
%\renewcommand{\arraystretch}{1.1} % Adjust vertical spacing
\small 
\begin{tabular}{|p{0.5cm}|p{5.9cm}|p{5.9cm}|p{3.9cm}|}
\hline
\textbf{} & \textbf{Initial Guidelines} & \textbf{Revised Guidelines} &\textbf{Changes}\\
\hline
\multirow{5}{*}{\rotatebox[origin=c]{90}{\hspace{1em} \textbf{G1. Body Appearance} \hspace{1em}}} 
& G1.1. Support disability representation in social VR avatars.
& G0. Support disability representation in social VR avatars. 
& Changed the original G1.1 to an overarching statement G0 to contextualize and motivate practitioners as well as specifying the application scope of the guidelines. \\ \cline{2-4}

& G1.2. Default to full-body avatars to enable diverse disability representation across different body parts. 
 & G1.1. Default to full-body avatars to enable diverse disability representation across different body parts. 
&  \\ \cline{2-4}

& G1.3. Enable flexible customization of body parts as opposed to using non-adjustable avatar templates.
& G1.2. Enable flexible customization of body parts as opposed to using non-adjustable avatar templates.
&  \\ \cline{2-4}

& G1.4. Prioritize human avatars to emphasize the ``humanity'' rather than the ``disability'' aspect of identity.
& G1.3. Prioritize human avatars to emphasize the ``humanity'' rather than the ``disability'' aspect of identity.
&  \\ \cline{2-4}

& G1.5. Provide non-human avatar options to free users from social stigma in real life.
& G1.4. Provide non-human avatar options to free users from social stigma in real life.
&  \\
\hline

\multirow{4}{*}{\rotatebox[origin=c]{90}{\hspace{0.5em} \textbf{G2. Avatar Dynamics} \hspace{0.5em}}} 
& G2.1. Allow simulation or tracking of disability-related behaviors but only based on user preference.
& G2.1. Allow simulation or tracking of disability-related behaviors but only based on user preference.
&  \\ \cline{2-4}

& G2.2. Enable expressive facial animations that deliver a spectrum of emotions.
& G2.2. Enable expressive facial animations to deliver invisible status.
& Changed the phrase ``a spectrum of emotions'' to ``invisible status'' to describe PWD's preferences more accurately. \\ \cline{2-4}

& G2.3. Prioritize equitable capability and performance over authentic simulation.
& G2.3. Prioritize equitable capability and performance over authentic simulation. 
&  \\ \cline{2-4}

& G2.4. Leverage avatar posture and motion to demonstrate the lived experiences of PWD.
& \st{G2.4. Leverage avatar posture and motion to demonstrate the lived experiences of PWD.}
& Merged the original G2.4 into G2.1. to remove redundant information. \\
\hline

\multirow{6}{*}{\rotatebox[origin=c]{90}{\hspace{1em} \textbf{G3. Assistive Technology Design} \hspace{1em}}} 
& G3.1. Offer various types of assistive technology to cover a wide range of disabilities.
& G3.1. Offer various types of assistive technology to cover a wide range of disabilities.
&  \\ \cline{2-4}

& G3.2. Allow detail customization of assistive technology for personalized disability representation.
& G3.2. Allow detail customization of assistive technology for personalized disability representation.
&  \\ \cline{2-4}

& G3.3. Provide high-quality, realistic simulation of assistive technology to present disability respectfully and avoid misuse.
& G3.3. Provide high-quality, authentic simulation of assistive technology to present disability respectfully and avoid misuse.
& Changed the word ``realistic'' to ``authentic'' to better describe the preferred characteristics of assistive technology designs. \\ \cline{2-4}

& G3.4. Focus on simulating assistive technologies that empower PWD, rather than those that highlight their challenges.
& \st{G3.4. Focus on simulating assistive technologies that empower PWD, rather than those that highlight their challenges.}
& Merged the original G3.4 to G3.1 to remove redundant information. \\ \cline{2-4}

& G3.5. Demonstrate the liveliness of PWD through dynamic interactions with assistive technology. 
& G3.4. Demonstrate the liveliness of PWD through dynamic interactions with assistive technology.
& %\yuhang{wording differences} 
\\ \cline{2-4}

& G3.6. Avoid overshadowing the avatar body with the assistive technology. 
& G3.5. Avoid overshadowing the avatar body with the assistive technology.
&  \\
\hline

\multirow{2}{*}{\rotatebox[origin=c]{90}{\parbox{1.4cm}{\centering \textbf{G4. Peri-}\\\textbf{pherals}}}} 
& G4.1. Provide suitable icons, logos, and slogans that represent disability communities. 
& G4.1. Provide suitable icons, logos, and slogans that represent disability communities.
&  \\ \cline{2-4}

& G4.2. Leverage spaces beyond the avatar body to present disabilities.
& G4.2. Leverage spaces beyond the avatar body to present disabilities.
&  \\
\hline

\multirow{3}{*}{\rotatebox[origin=c]{90}{\parbox{2.5cm}{\centering \textbf{G5. Interface}}}} 
& G5.1. Distribute disability features across the entire avatar interface rather than gathering them in a specialized category.
& G5.1. Distribute disability features across the entire avatar interface rather than gathering them in a specialized category. 
&  \\ \cline{2-4}

& G5.2. Use input controls that offer precise adjustments whenever possible. 
& G5.2. Use continuous controls for high-granularity customization. 
& Refined the wording to better describe PWD's customization control preferences. \\ \cline{2-4}

& G5.3. Offer an easy control to turn on/off or switch between disability features.
& G5.3. Offer an easy control to turn on/off or switch between disability features.
&  \\
\hline

\end{tabular}
\caption{The table compares the initial and revised guidelines to highlight changes. After revision, the initial set of 20 guidelines was refined and trimmed to 17 finalized design guidelines.}
\Description{}
\label{tab:changes}
\end{table*}


%\renewcommand{\arraystretch}{1.1} % Adjust vertical spacing

\afterpage{
\clearpage
\small
%\footnotesize

\begin{longtable}{p{0.4cm}p{4cm}p{6.5cm}p{5cm}}
\toprule
\textbf{} & \textbf{Design Guideline} & \textbf{Description} & \textbf{Examples and Demos} \\
\toprule
\endfirsthead

\toprule
\textbf{} & \textbf{Design Guideline} & \textbf{Description} & \textbf{Examples and Demos} \\
\toprule
\endhead

% G1. Avatar Body Appearance
%\multirow{5}{*}{\rotatebox[origin=c]{90}{\parbox{6cm}{\centering \textbf{G1. Avatar Body Appearance}}}} 

\multicolumn{4}{p{17cm}}{\change{\textbf{G0. Support disability representation in social VR avatars \cite{zhang2022, assets_24, chronic_pain_gualano_2024, kelly2023}.} Approximately 1.3 billion people experience significant disability, representing about 16\% of the global population \cite{WHO2023}. It's important to ensure PWD are included and represented equally in emerging technology such as social VR. As long as the platform involves avatar-based interactions, there is design space to support disability representation. The following set of guidelines can be flexibly applied to a variety of social VR platforms with different (1) avatar types (e.g., humanoid avatars in Rec Room \cite{recroom} vs. robotic-type avatars in Among Us \cite{amonguscharacters}), (2) aesthetic styles (e.g., life-like avatars in Horizon Worlds \cite{metaavatars} vs. abstract avatars in Roblox \cite{robloxwiki})), and (3) content focus (e.g., communication-heavy type in VRChat \cite{vrchat} vs. game-centric in Rec Room). We encourage practitioners to adopt G0 as a fundamental mindset when developing and designing avatars, considering it in the early stages, and consistently exploring opportunities to support disability representation.}}
\\ \midrule

% G1. Avatar Body Appearance
\textbf{} & 
& \multicolumn{1}{c}{\textbf{G1. Avatar Body Appearance}} & 
\\ \midrule
\textbf{G1.1 \change{(HR)}}
& \textbf{Default to full-body avatars to enable diverse disability
representation across different body parts.}
& Avatar interfaces should offer full-body avatar options \cite{kelly2023}. \change{About 296,000 people in the U.S. live with paralysis of the lower half of the body, with around 17,900 new cases each year \cite{NSCISC2021}.} Given the \change{large affected user size and} dominant preferences for full-body avatars over others (e.g., upper-body only, or head and hands only), we recommend making it the default or the starting avatar template, giving users the maximum flexibility to further customize their avatars as they prefer. 
& \begin{minipage}[t]{\linewidth}
    E.g., A full-body avatar can show a prosthetic left foreleg. \\
    \includegraphics[width=3cm, height=2.8cm]{sections/images/guidelines_image/G1.2.png}
  \end{minipage}
\\ \midrule

\textbf{G1.2 \change{(HR)}}
& \textbf{Enable flexible customization of body parts as opposed to using non-adjustable avatar templates.}
& Avatar interfaces should provide PWD sufficient flexibility to customize each avatar body part \cite{kelly2023}. \change{While the customization spans a wide range, the most commonly mentioned body parts to customize include (1) avatar height, (2) body shape, (3) limbs (i.e., number of limbs, length and strength of each limb), and (4) facial features (e.g., mouth shape, eye size).} Asymmetrical design options of body parts (e.g., eyes, ears) should also be available, such as changing size and direction of each eyeball to reflect disabilities like strabismus.
& \begin{minipage}[t]{\linewidth}
    E.g., Options that can customize the size of each eye. \\
    \includegraphics[width=3cm, height=2.7cm]{sections/images/guidelines_image/G1.3.png}
  \end{minipage}
\\ \midrule

\textbf{G1.3 \change{(HR)}}
& \textbf{Prioritize human avatars to emphasize the ``humanity'' rather than the ``disability'' aspect of identity.}
& Social VR applications should offer human avatar options whenever the application theme allows.
& \begin{minipage}[t]{\linewidth}
    E.g., Human avatars that can show intersectional identities. \\
    \includegraphics[width=4cm, height=2cm]{sections/images/guidelines_image/G1.4.png}
  \end{minipage}
\\ \midrule

\textbf{G1.4 \change{(R)}}
& \textbf{Provide non-human avatar options to free users from social stigma in real life.}
& Besides human avatars, avatar interfaces should also provide diverse forms of non-human avatars, empowering PWD to choose the one they relate with flexibly.
& \begin{minipage}[t]{\linewidth}
    E.g., A robotic avatar representing left forearm amputation.\\
    \includegraphics[width=3cm, height=2.6cm]{sections/images/guidelines_image/G1.5.png}
  \end{minipage}
\\ \midrule

% G2. Avatar Body Dynamics
%\multirow{4}{*}{\rotatebox[origin=c]{90}{\parbox{10cm}{\centering \textbf{G2. Avatar Body Dynamics: facial expression, posture, and body motion.}}}}
\textbf{} & 
& \multicolumn{1}{c}{\textbf{G2. Avatar Body Dynamics}} & 
\\ \midrule

\textbf{G2.1 \change{(HR)}}
& \textbf{Allow simulation or tracking of disability-related behaviors but only based on user preference.}
& Users should be able to control the extent of behavior tracking in social VR. With the advance of motion tracking techniques, avatar platforms may disable subtle behavior tracking by default to avoid disrespectful simulation, but allow users to easily adjust the tracking granularity for potential disability expression.
& \begin{minipage}[t]{\linewidth}
    E.g., Avatar can show motor tics (left) or not (right) based on the user's preference.\\
    \includegraphics[width=4cm, height=2cm]{sections/images/guidelines_image/G2.1.png}
  \end{minipage}
\\ \midrule

\textbf{G2.2 \change{(R)}}
& \textbf{Enable expressive facial animations to deliver invisible status.}
& Avatar platforms should enable diverse facial expressions, allowing PWD to express emotion, portray mental status, and indicate fluctuation of invisible disabilities \cite{assets_24}. \change{Including the five basic emotions is a good starting point, and practitioners should expand and diversify based on their own use scenarios. This guideline is particularly helpful for communication-heavy or mental therapy platforms, where expressing a range of emotions helps PWD convey their feelings more accurately.}
& \begin{minipage}[t]{\linewidth}
    E.g., Avatar shows diverse emotions, including anxiety (left), sadness (middle), and happiness (right).\\
    \includegraphics[width=4cm, height=1.9cm]{sections/images/guidelines_image/G2.2.png}
  \end{minipage}
\\ \midrule

\textbf{G2.3 \change{(HR)}}
& \textbf{Prioritize equitable capability and performance over authentic simulation.}
& PWD value equitable and fair interaction experiences more than the authentic disability expression. Therefore, social VR platforms should ensure same level of capabilities and performance for all avatars no matter whether disability features or behaviors are involved. 
& \begin{minipage}[t]{\linewidth}
    E.g., The avatar with the wheelchair moves at the same speed as the avatar without the wheelchair.\\
    \includegraphics[width=3cm, height=2.2cm]{sections/images/guidelines_image/G2.3.png}
  \end{minipage}
\\ \midrule

% G3. Assistive Technology Design
%\multirow{6}{*}{\rotatebox[origin=c]{90}{\parbox{6cm}{\centering \textbf{G3. Assistive Technology Design}}}}
\textbf{} & 
& \multicolumn{1}{c}{\textbf{G3. Assistive Technology Design}} & 
\\ \midrule

\textbf{G3.1 \change{(HR)}}
& \textbf{Offer various types of assistive technology to cover a wide range of disabilities.}
& Avatar interfaces should offer assistive technologies that are commonly used by PWD \cite{zhang2022}. \change{Approximately 2.5 million people around the world are assistive technologies users \cite{LevelAccess2024}. According to our large-scale interview data, the most desired types of assistive technologies include: (1) mobility aids (e.g., wheelchair, cane, and crutches); (2) prosthetic limbs; (3) visual aids (e.g., white cane, glasses, and guide dog); (4) hearing aids and cochlear implants; and (5) health monitoring devices (e.g., insulin pumps, ventilator, smart watches). % \yuhang{narrow down}. 
Practitioners should consider including at least these five categories of assistive technologies in avatar interfaces.
In addition, due to PWD's different technology preferences \cite{kelly_AI24}, we encourage practitioners to offer more than one assistive technology option within each category, for example, including guide dog, white cane, and glasses for visual aids.} 
& \begin{minipage}[t]{\linewidth}
    E.g., Offer multiple types of mobility aids, such as walking cane, crutches, and forearm crutches. \\
    \includegraphics[width=3cm, height=2.2cm]{sections/images/guidelines_image/G3.1.png}
  \end{minipage}
\\ \midrule

\textbf{G3.2 \change{(HR)}}
& \textbf{Allow detail customization of assistive technology for personalized disability representation.}
& Avatar platforms should allow customizations for assistive technology \change{\cite{kelly2023,zhang2022}. Basic customization options should include adjusting the colors of different assistive tech components and adding decorations (e.g., stickers, logos) to the assistive technologies. More customization could be added based on specific use cases.} 
& \begin{minipage}[t]{\linewidth}
    E.g., Users can adjust the color of the power wheelchair, like the cushions, wheels, and chassis cover.\\
    \includegraphics[width=4cm, height=2cm]{sections/images/guidelines_image/G3.2.png}
  \end{minipage}
\\ \midrule

\textbf{G3.3 (HR)}
& \textbf{Provide high-quality, authentic simulation of assistive technology to present disability respectfully and avoid misuse.}
& To avoid misunderstandings or misuse, the assistive tech simulation should convey standardized, \change{authentic details of the real-world assistive devices \cite{zhang2023}, regardless the overall avatar style. For example, the design of a white cane should show the details of tip and follow its standardized color selection, no matter the design style is photorealistic or cartoon. We recommend practitioners to model assistive technologies by following their established design standards, such as design guidelines for white canes \cite{who_white_canes}, wheelchairs \cite{russotti_ansi_wheelchairs}, and hearing devices \cite{ecfr_800_30}.}
& \begin{minipage}[t]{\linewidth}
    E.g., Thh wheelchair should be designed with high-fidelity and realistic details to represent disability respectfully.\\
    \includegraphics[width=3cm, height=2.4cm]{sections/images/guidelines_image/G3.3.png}
  \end{minipage}
\\ \midrule

\textbf{G3.4 \change{(R)}}
& \textbf{Demonstrate the liveliness of PWD through dynamic interactions with assistive technology.}
& Beyond providing assistive tech options, social VR platforms should enable suitable interactions between avatars and assistive tech. The interactions should authentically reflect PWD's real-world usage of their assistive tech, such as how a blind user sweeps their cane, or how a wheelchair user moves their arms to control their wheelchair.
& \begin{minipage}[t]{\linewidth}
    E.g., Avatar controls the manual wheelchair through pushing. \\
    \includegraphics[width=3cm, height=2.8cm]{sections/images/guidelines_image/G3.5.png}
  \end{minipage}
\\ \midrule

\textbf{G3.5 \change{(HR)}}
& \textbf{Avoid overshadowing the avatar body with assistive technology.}
& The size of assistive technology should not dominate the avatar body but rather fit the body size. \change{Avatar platforms should automatically match the assistive tech model to different avatar body sizes by default, and allow users to adjust the size of assistive technology to achieve their preferred avatar-aid ratio. Moreover, the combination of avatar and assistive tech should be seamless without affecting the quality and aesthetics of the original avatar \cite{kelly2023}.}
& \begin{minipage}[t]{\linewidth}
    E.g., Users can change the size of assistive technology to match with their avatar. \\
    \includegraphics[width=3cm, height=3cm]{sections/images/guidelines_image/G3.6.png}
  \end{minipage}
\\ \midrule

% G4. Peripherals around Avatars
%\multirow{2}{*}{\rotatebox[origin=c]{90}{\parbox{6cm}{\centering \textbf{G4. Peripherals around Avatars}}}}
\textbf{} & 
& \multicolumn{1}{c}{\textbf{G4. Peripherals around Avatars}} & 
\\ \midrule

\textbf{G4.1 \change{(HR)}}
& \textbf{Provide suitable icons, logos, and slogans that represent disability communities.}
& Awareness-building items or presets (e.g., logos, slogans) should be provided, allowing users to attach them to various areas on or around the avatars, such as the apparel, accessories, and assistive tech \cite{assets_24, zhang2022}. \change{We compiled a list of widely recognized and preferred symbols that represent different disabilities for practitioners to refer to, including (1) the rainbow infinity symbol the represents the autism community \cite{assets_24, rainbow_infinity_symbol}, (2) the sunflower that represents hidden disabilities \cite{isit_assets24, hidden_disability_sunflower}, (3) the disability pride flag \cite{disability_pride_flag}, (4) the spoons, symbolizing spoon theory for people with chronic illness \cite{kelly2023, assets_24}, and (5) the zebra symbols for rare diseases \cite{assets_24, Gualano_2023}.} 
& \begin{minipage}[t]{\linewidth}
    E.g., An avatar wearing a T-shirt with a rainbow and infinity symbol to represent the autism spectrum disorder community. \\
    \includegraphics[width=3cm, height=2.8cm]{sections/images/guidelines_image/G4.1.png}
  \end{minipage}
\\ \midrule

\textbf{G4.2 \change{(R)}}
& \textbf{Leverage spaces beyond the avatar body to present disabilities.}
& When designing avatars, developers and designers should consider leveraging avatar's peripheral space to enable users to better express their status, especially for individuals with invisible disabilities. \change{Some design examples include a weather background to indicate mood and a battery sign to indicate energy level \cite{assets_24}.} 
& \begin{minipage}[t]{\linewidth}
    E.g., An avatar with a floating bubble overhead, showing a level of social energy. \\
    \includegraphics[width=4cm, height=2.4cm]{sections/images/guidelines_image/G4.2.png}
  \end{minipage}
\\ \midrule

% G5. Design of avatar customization and control interface
%\multirow{3}{*}{\rotatebox[origin=c]{90}{\parbox{8cm}{\centering \textbf{G5. Design of Avatar Customization and Control Interface}}}} 
\textbf{} & 
& \multicolumn{1}{c}{\textbf{G5. Interface and Control}} & 
\\ \midrule

\textbf{G5.1 \change{(HR)}}
& \textbf{Distribute disability features across the entire avatar interface rather than gathering them in a specialized category.}
& Avatar features for disability expression should be treated in the same way as other avatar features. In avatar interfaces, disability-related features should be properly distributed in their corresponding categories. \change{There should not be a specialized category for PWD. For example, assistive technologies should be included in the accessory category rather than an assistive tech category.} 
& \begin{minipage}[t]{\linewidth}
    E.g., Walking canes and wheelchairs are included under the accessory category along with items like glasses, hats and bags. \\
    \includegraphics[width=3cm, height=3cm]{sections/images/guidelines_image/G5.1.png}
  \end{minipage}
\\ \midrule

\textbf{G5.2 \change{(HR)}}
& \textbf{Use continuous controls for high-granularity customization.}
& Avatar interfaces should adopt input controls that offer a continuous range of options to enable flexible customization. \change{This could be widely applied to a variety of design attributes, such as the size and shape of multiple avatar body parts.}
& \begin{minipage}[t]{\linewidth}
    E.g., Offer a slider to change the length of limbs. \\
    \includegraphics[width=3cm, height=2.7cm]{sections/images/guidelines_image/G5.2.png}
  \end{minipage}
\\ \midrule

\textbf{G5.3 \change{(HR)}}
& \textbf{Offer an easy control to turn on/off or switch between disability features.}
& Social VR platforms should provide easy-to-access shortcut control enable users to conduct \change{\textit{ad-hoc} avatar updates on the go. Important control functions include: (1) toggling on and off the disability-related features \cite{assets_24}; (2) switching between different saved avatars \cite{kelly2023}; and (3) updating status for fluctuating conditions \cite{assets_24}.} 
& \begin{minipage}[t]{\linewidth}
    E.g., Users should be able to turn disability-related features on and off with a single click. \\
    \includegraphics[width=4cm, height=1.5cm]{sections/images/guidelines_image/G5.3.png}
  \end{minipage}
\\  
\bottomrule
\caption{\change{A full version of our \textit{revised} guidelines that incorporated VR experts' feedback. We present a set of 17 inclusive avatar guidelines for PWD that covered five avatar design aspects. Each guideline includes: the guideline statement with a recommendation level (Highly Recommended (HR) or Recommended (R)), a detailed description of guideline, and an actionable implementation example.}}
\Description{This table, titled "Table 7," provides a detailed overview of 17 revised guidelines for inclusive avatar design. It includes three columns: "Design Guideline," "Description," and "Examples and Demos," organized into five specific guideline sections. Each guideline has a recommendation level denoted as ``HR'' for Highly Recommended or ``R'' for Recommended.}
\label{tab:full_revised}
\end{longtable}
}
