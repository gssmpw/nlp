\section{Guideline Revision}


\change{Based on the feedback from VR practitioners, we refined the guidelines and made the following changes: 
(1) Convert the original G1.1 to an overarching statement G0 to contextualize and motivate practitioners as well as specifying the application scope of our guidelines (D1);
(2) Assigned recommendation levels to each guideline to distinguish their priority based on development resources, size of user groups, and use cases (D1, D2, S3);
(3) Added statistics about served user groups, informing practitioners of the potential impact of implementing the guideline (D2);
(4) Specified the customization scope in G1.3, G3.1, G3.2, and G4.1 (S1); 
(5) Diversified implementation examples in G2.2, G2.3, and G3.3 (S2);
(6) Suggested potential use cases for G1.4 and G2.2 (D1, S3);
(7) Merged G2.4 into G2.1 to remove redundant information, and merged G3.1 and G3.4 to clarify the characteristics of simulated assistive technologies in social VR (S4);
\cameraready{(8) Refined the wording in G2.2, G3.3, and G5.2 to improve clarity.
Appendix Table \ref{tab:changes} demonstrated the changes we made to the initial guidelines.}

After revision, we ended up with 17 design guidelines upon experts' feedback. We present an overview table of the revised guidelines in Table ~\ref{tab:overview_revised} and a full version in Appendix Table ~\ref{tab:full_revised}.} 


%(1) adding an overarching guideline G0 to fullfill the suggestion  (Sx); (2) Adding recommendation levels to all guidelines to distinguish  priority (3) adding stats info to each guideline to highlight the disability coverage and impact of the guideline to motivate developers (Sx) 
%(2) merging G 3.1 and G3.4 to clarify the scope of assistive tech to be simulated in social VR (Sy); xxx 
%(3) Specify the suggested customization scope in G1.2, e.g., xxx (Sz); }

%\change{A consensus achieved from prior literature \cite{zhang2022, assets_24, chronic_pain_gualano_2024, kelly2023} and our study is to \textbf{\textit{support disability representation in social VR avatars} (G0).} Our findings from experts' evaluations further highlighted the widely applicable use cases of this guideline: as long as the platform involved avatar-based interactions, there are design space to support disability representation. In other words, it can be applied to a variety of social VR platforms with different (1) avatar types (e.g., humanoid avatars in Rec Room \cite{recroom} vs. robotic-type avatars in Among Us \cite{amonguscharacters}), (2) aesthetic styles (e.g., life-like avatars in Horizon Worlds \cite{metaavatars} vs. abstract avatars in Roblox \cite{robloxwiki})), and (3) content focus (e.g., communication-heavy type in VRChat \cite{vrchat} vs. game-centric in Rec Room). We encourage practitioners to adopt this guideline (G0) as a fundamental mindset when developing and designing avatars, considering it in the early stages, and consistently exploring opportunities to support disability representation. 

%\yuhang{this needs to be updated to fit the context; you only need to describe how you made the change, e.g., "To fulfill the suggestion of xxx, we added an overarching guideline G0 to xxx". Also, G0 is not a guidelin to apply, but rather an overall statement for awareness and clarify suitable VR scenarios to use our guidelines. The detailed content in G0 needs to be added to the table. }}

%\change{
% The revised 

% Based on experts' feedback, we revised and iterated on the guidelines to make them more applicable and actionable in practice. We describe the revision below and present the finalized guidelines in Table \ref{}.

% \textbf{\textit{Added the Recommendation Levels.}} For each guidelines, we 

% \textbf{\textit{Merging.}}

% \textbf{\textit{Cross-referencing.}}
% e.g., G3.3 add standard requirement of assistive technologies

% \textbf{\textit{Specify a scope / bare minimum standards to follow.}}
% e.g., G1.2 body parts that at least should be customizable, G3.1 five types of AT that should be included at the minimum, 

% \begin{itemize}
%     \item Highly Recommended (HR): Highly recommended guidelines are easy to implement, apply to almost all use cases, and are considered as the bare minimum for avatar design and development. 
%     \item Recommend (R): Recommended guidelines may require planning and effort to implement. They are more tailored to specific user groups or use case scenarios, but they lead to good user experiences and help expand user groups.
% \end{itemize}
%}

