\section{Related Work}

In recent years, social VR has emerged as a promising digital social space, where users could meet and interact via embodied avatars \cite{Tham_2018_understanding, maloney_2020_sleep}. Unlike 2D virtual worlds where user can only control avatars by mouse and keyboard, %media where users can only control avatars by mouse and keyboard, 
social VR offers more immersive experiences through full-body tracking avatars \cite{Freeman2020, Freeman2021, Zamanifard_2019_togetherness, Caserman_2019_real_time}. Prior work has summarized the unique characteristics of social VR avatars, including rendering full-body movements and gestures in real time \cite{Zamanifard_2019_togetherness, Sra_2018_your_place}, affording vivid spatial and temporal experiences in high-fidelity 3D virtual spaces \cite{McVeigh_2019_shaping, McVeigh_2018_what}, and mediating both verbal and non-verbal communications through facial expressions and body language \cite{maloney_nonverbal, Moustafa_2018_longitudinal}.
\change{These characteristics create a strong sense of embodiment. As a result, many users see avatars as their proxy in the virtual social space and thus curate the avatar design for self-representation \cite{Freeman2021, Morris_2023_women, freeman_2022_noncis}. In this section, we summarize prior literature on avatar-based self-representation, misrepresentation of disability in social media, and  guideline generation and evaluation methods to contextualize our research.} 

\subsection{Avatars and Self-representation in Social VR}
A growing body of HCI work has explored the avatar-based self-representation for underrepresented user groups, such as racial minorities ~\cite{VRharass2019, freeman2022disturbing, Freeman2021}, women~\cite{Morris_2023_women, Sadeh-Sharvit_2021_sexual}, LGBTQ ~\cite{freeman2022disturbing, queeractingout2022, freeman_2022_noncis}, and older adults ~\cite{Baker2021elder}. For example, Freeman et al. ~\cite{freeman_2022_noncis} interviewed 59 non-cisgender users to explore their avatar choices in social VR, revealing that they used avatars with different genders to signify t heir flexible and fluid gender identities. Morris et al. ~\cite{Morris_2023_women} studied avatar representations at the intersection of age and gender. By interviewing ten women in midlife, they found participants desired to use nuanced avatar options with a gradation colors, textures, and body types for self-expression, avoiding hyper-sexualized or stereotypical portray of midlife women. %While the representation of minority user groups has attracted increasing attention, \change{avatar research about PWD is in its infancy.} %PWD has been left out in avatar research. 

\change{Recently, more research attention has been drawn to PWD in social VR, investigating their avatar representation preferences ~\cite{zhang2022, zhang2023, isit_assets24, kelly2023, assets_24, chronic_pain_gualano_2024, ribeiro2024towards}, with four papers ~\cite{zhang2022, kelly2023, assets_24, zhang2023} deriving design implications for inclusive avatar design.} For example, Zhang et al. ~\cite{zhang2022} interviewed 19 \change{people with sensory disabilities, including people who are blind or low vision people and people who are d/Deaf or hard of hearing, to understand their avatar preferences} and revealed a spectrum of disability expression strategies, such as accurately reflecting all disabilities, or disclosing only selective disabilities.
\change{Three design implications were derived, including offering diverse assistive technology options, supporting invisible disability representation through awareness-building accessories, and avoiding misuse of disability features.} %\yuhang{summarize the implications, not randomly pick one...}

Moreover, Gualano at al. \cite{assets_24} interviewed 15 people with invisible disabilities (e.g., ADHD, dyslexia), finding that they wished to leverage the embodied characteristics of avatars (e.g., facial expressions and body language) to dynamically present their disability conditions. \change{The research therefore discussed four design implications, such as creating more options for multi-modal communication and enabling dynamic presentation of fluctuating disability conditions.} %xxx \yuhang{again, summarize} that platforms should allow users to customize movement and facial expressions for easier and more authentic representation.} 
Beyond focusing merely on disability identity, Mack et al. ~\cite{kelly2023} considered the intersectional identities, such as disability intersecting with race and ethnicity. They revealed that participants with intersecting identities may face conflict in presenting these identities and had to make trade-offs when deciding which identities to present. \change{They derived seven design implications, such as allowing customization of body sizes and enabling users to save and switch between multiple versions of avatars.} 

While prior research has explored the self-representation preferences of PWD and derived corresponding implications, \change{these work focused on different aspects and the design implications were dispersed across multiple publications and venues \cite{zhang2022, assets_24, kelly2023}, being difficult for practitioners to collect and apply comprehensively}. \change{There is also a lack of systematic evaluation on their usability and actionability.}  
%they tend to be general and lack practical guidance of how developers and designers can exactly implement them. They are also scattered throughout different venues, making them to difficult to find and use in practice. 
\change{Centralized and systematically-validated design guidelines are needed to enable easy adoption and dissemination.} %, thus supporting practical integration of inclusive avatar design into social VR development.}

%\yuhang{merge above two paragraphs, since you are now repeating the same group of papers... At the beginning, you should say that there has been prior work on understanding PWD's preferences and some design implications have been derived. Then when talking about each work, you can mention both---key findings and key implications. By then end, you can comment that while the understanding is thorough, the guidelines or implications are distributed across different papers. We need a systematically generated, well-organized, and validated guidelines that are easy to be adopted and shared by practitioners.}


\subsection{Stigma and Misrepresentation of PWD in the Media}
\change{Improper disability representation can lead to misconception and perpetuate stigma} 
%Disability could be particularly challenging to represent on avatars, given the prevalent media misrepresentation of PWD 
\cite{kelly_AI24, ippolito2020misrepresentation, hall1997representation, young2014imnot}. PWD are seldom represented in the media \cite{darke2004changing}, and when they do appear, they are often misportrayed as objects of pity, superheroes who have accomplished great feats, or people in need of charity and help \cite{garlandthomson2002politics, quayson2007aesthetic, young2014imnot}. Prior work has studied the misrepresentation of disability in various forms of digital media \cite{garlandthomson2002politics, elliott1982media, kelly_AI24}. For example, Garland-Thomson ~\cite{garlandthomson2002politics} analyzed the visual rhetoric of disability in popular photography and found many of them propagated PWD as being pitiful. Leonard \cite{elliott1982media} revealed that television stigmatized PWD by portraying them as being predominately from lower social classes and unemployed. With the advent of AI technology, Mack et al. \cite{kelly_AI24} explored the disability representation in text-to-image generative AI systems, highlighting that the AI-generated images perpetuated the misconception of PWD as primarily using wheelchairs, being sad and lonely, incapable, and inactive.

Compared to conventional 2D media like images or videos, misrepresentation of disability via avatars in the 3D social VR space can be more harmful to PWD due to its highly embodied and interactive nature \cite{blackwell2019harassment, VRharass2019, freeman2022disturbing}. \change{Researchers started to investigate the experiences of using avatars with disability representation features in social VR \cite{zhang2023, isit_assets24}.} For example, Zhang et al. \cite{zhang2023} conducted a two-week diary study with 10 PWD to compare their experiences when using avatars with and without %regular avatars and avatars with 
disability signifiers (e.g., an avatar on a virtual wheelchair), \change{revealing different types of harassment targeting PWD (e.g., physical harassment, mimicking disability). They further highlighted the misunderstanding between PWD and other social VR users---while PWD used avatars to present their disability identity, others misinterpreted these avatars as trolling or meme avatars, leading to offensive social behaviors.}
To avoid misinterpretation and harassment, they suggested improving the quality and aesthetics of disability features on avatars. %Yet, to our knowledge, no work has effectively addressed how to better design the disability representation features. 
\change{Similarly, Angerbauer et al. \cite{isit_assets24} conducted a one-week diary study to understand the experiences of 26 PWD when using avatars with disability signifiers, including both visible and invisible ones (e.g., avatar with a sunflower patch). They found that participants experienced both positive and negative social interactions regardless the visibility of their disability signifiers.}
These works indicated that, even with the goal of designing signifiers for positive disability representation, many design choices can still result in negative user experiences. \change{This emphasized the needs for inclusive avatar design guidelines generated with the PWD community to effectively assist VR practitioners in creating avatars that represent disabilities properly.} % based disability representations in social VR. 


\subsection{Guideline Generation and Evaluation} 
Design guidelines are important resources that support designers and developers to create positive user experiences. %Standard design guidelines have been constructed for multiple disciplines in HCI, such as interaction design guidelines \cite{benyon2014designing, sjostrom2002non, human_ai_guidelines} and accessibility guideline \cite{w3c_wcag21_2018, w3c_mobile_accessibility_2015} for various software platforms. 
\change{Mature fields usually have well-established guidelines for practitioners to follow, such as 2D interface design guidelines~\cite{benyon2014designing}, haptic interaction guidelines \cite{sjostrom2002non}, human-AI interaction guidelines \cite{human_ai_guidelines}, and accessibility guidelines for different platforms \cite{w3c_wcag21_2018, w3c_mobile_accessibility_2015}. These guidelines are often derived from systematic literature review \cite{human_ai_guidelines, wei_2018, Mueller_2014, webdyslexia2012}.} %In the mature fields, many design guidelines have been derived from the systematization of knowledge from existing literature and experts review \cite{friedman2008web, Mueller_2014, webdyslexia2012, human_ai_guidelines}, 
For example, Amershi et al. \cite{human_ai_guidelines} consolidated 18 human-AI interaction guidelines from %a collection of 150 AI-related design recommendations collected from a review of
both academia publications and industry sources. Muller et al. \cite{Mueller_2014} developed a set of movement-based game design guidelines by reviewing existing literature and incorporating insights from 20 years of working experiences in the gaming industry. Santana et al. \cite{webdyslexia2012} surveyed the state of the art on dyslexia and web accessibility and compiled 41 guidelines that support people with dyslexia in using website. \change{Following a similar method, we seek to derive inclusive avatar design guidelines by reviewing prior literature~\cite{zhang2022, zhang2023, isit_assets24, kelly2023, assets_24, chronic_pain_gualano_2024, ribeiro2024towards}. However, as research on avatar representation for PWD is still nascent, with publications emerging only in recent years (mainly between 2022 and 2024), we further interviewed 60 people with diverse disabilities to supplement prior literature, ensuring comprehensive coverage of our guidelines.}

%social VR and avatar design is inclusive avatars for PWD is an emerging field with no extensive prior literature. \change{To our best knowledge, seven papers so far \cite{zhang2022, zhang2023, isit_assets24, kelly2023, assets_24, chronic_pain_gualano_2024, ribeiro2024towards} have thoroughly understood PWD's self-presentation preferences via avatars, but none of them focused on deriving comprehensive design guidelines that are readily usable.} %\yuhang{count the number of papers so far... 6? and mention that explicitly}} %nor widely-agreed design standards. Some guidelines tried to standardized accessible design for VR interactions \cite{heilemann2021accessibility, nng2021usability, oculus2022accessible}, but no prior works focus on deriving inclusive avatar design guidelines. 
%\change{To fill this research gap, we first did a literature review to summarize key design insights from existing literature. Then we interviewed 60 PWD to supplement any missing pieces in prior works, ensuring our guidelines are comprehensive.}
%Therefore, instead of soliciting guidelines from prior literature, we adopted a user-centered method to derive design guidelines through a well-structured interview with 60 PWD, gaining first-hand design insights from the target users.  

%\yuhang{the next paragraph sort of indicating that most of the work above does not include an evaluation for the guideilnes, is it true? If so, you should say that explictly, otherwise you need to reframe these two paragraphs.}
%While many guidelines are generated to support the design process, not many of them are thoroughly evaluated or validated, resulting in unpractical or in-actionable guidelines \cite{lotfi2022taxonomy}. 

\change{Guideline evaluation is critical to ensure practitioners to properly act upon and apply the guidelines in development~\cite{lotfi2022taxonomy}.} A commonly used guideline evaluation method is \textit{heuristic evaluation}~\cite{nielsen1990heuristic}, where evaluators examine an interface for the application and violations of a given set of usability heuristics \cite{human_ai_guidelines, wei_2018}. For example, to improve and validate their speech interface guidelines, Wei et al. \cite{wei_2018} asked eight experts to follow heuristic evaluation and examine three speech-based smart devices with the guidelines. Amershi et al. \cite{human_ai_guidelines} adopted similar method to evaluate their human-AI interaction guidelines. \change{They modified the heuristic evaluation method to focus on evaluating the guidelines rather than an interface.} %With the primary goal to evaluate the design guidelines rather than to evaluate an interface, they coined the term \textit{modified heuristic evaluation}. 
Specifically, participants first evaluated an AI-based interface with the proposed guidelines and then reflected on the guideline usability. 
We adopted the similar modified heuristic evaluation method to evaluate and refine our design guidelines for inclusive avatars.


%\kexin{focus on their *methods* of generating guidelines and how they evaluate/validate the guidelines if there are any}

% generating the guideline 

% mostly driven through piror knowledge systematization / prior work ;


% heuristic evaluation






% These unique features not only facilitate a greater sense of embodiment but also introduce new requirement for avatar designs \cite{hepperle2021aspects, kolesnichenko2019understanding, waltemate2018impact, McVeigh_2019_shaping}. For example, Kolesnichenko et al. \cite{kolesnichenko2019understanding} found that VR practitioners must consider additional design factors, such as avatar aesthetics, locomotion, and the avatar's relation to one's virtual identity. McVeigh-Schultz et al. \cite{McVeigh_2019_shaping} studied VR design choices that shaped positive user experiences, and their findings showed that avatar's facial expressions and body language can mediate pro-social interactions (e.g., conveying friendliness through various emotional expressions), thus should be carefully designed.

% disability is a unique and vlunerable identities -> must be carefully designs 


% design guidelines for people with disabilities -> accessibility
% diversity and inclusion guidelins in 2D web space; mainly accessibility?
% design guidelines in VR space (in its infancy)
% design guidelines for avatars in 2D space? 

% design guidelins for what, what they are about


% - design guidelines in HCI -> definition and purpose
% - design guidelines for accessibility and inclusion
% - missing -> our work












% *** below is Scott's writting ***
% \subsection{Self Expression in Virtual Reality [update]}
% %I think these would be a valuable addition to related works:

% %G. Freeman and D. Maloney, “Body, avatar, and me: The presentation and perception of self in social virtual reality,” in Proc. Assoc. Comput. Machinery Human-Comput. Interact., 2021, vol. 4, pp. 2391–23927, doi: 10.1145/3432938.

% %Guo Freeman, Samaneh Zamanifard, Divine Maloney, and Alexandra Adkins. 2020. My body, my avatar: How people perceive their avatars in social virtual reality. In Conference on Human Factors in Computing Systems - Proceedings. Association for Computing Machinery. https://doi.org/10.1145/3334480.3382923
% %url: https://dl.acm.org/doi/abs/10.1145/3334480.3382923?casa_token=dqWAtj--JHcAAAAA:ktz4p9OTmTuPCrg_bMkv3TMMHl_6qbx12fO3_6Ev5glLTjMEAso2sEFR2BAIXEp8zQ3ubAZuAMcV


%     Since the advent of Social Virtual Reality (do we need to define social virtual reality), attention to the player-character and representation of the user in the virtual space has been a point of discussion for many developers. Some applications, like \textit{VR Chat}, allow users to import their own custom-made player character which meets basic standards defined by the system. Other platforms offer customization through a built-in customization interface, allowing users to personalize different parts of the avatar and produce a unique character to represent themselves in the Social VR space. These avatars can then be used in face-to-face communication consisting of facial animation, body movement, and verbalization. \par 
    
%     According to Freeman et Al, The VR avatars in combination with one's physical body offer a completely new way of representing oneself[G. Freeman]. They found that while some participants engaged in social VR with a creative avatar which differed from themselves in real life, most participants were interested in "consistent" self-presentation in avatar customization systems[G. Freeman]. This refers to the capability of constructing an avatar similar to one's physical self outside of VR[G. Freeman]. For people interested in representing themselves this way, being able to create an avatar similar to themselves offered another level of immersion and engagement with the system. \par
      
%         Acknowledging the idea that a majority of people want to present themselves in the virtual world similarly to how they look in the real world was an important consideration in the design of our study. Having previously determined that people with disabilities did indeed want to disclose the disability part of themselves [prior work], we looked to further understanding of what those avatars looked like in the eyes of our participants.

% \subsection{Disability Disclosure Online}

% Additional references: 

% %Bowker, N., \& Tuffin, K. (2002). Disability Discourses for Online Identities. \textit{Disability \& Society}, \textit{17}(3), 327–344. https://doi.org/10.1080/09687590220139883 

% Fairness/ethnicity in Representing Disability in Media

% \subsection{Disability Inclusivity in Virtual Reality}
% Virtual Reality (VR) provides a place to socialize, play, and take part in creative experiences. For some, VR is a medium through which the user can escape reality and embrace something otherwise unavailable. For others, VR offers a social space where users can be themselves and enjoy activities with people from across the world. Both of these experiences revolve around the immersion of the user in the virtual space through the environment, activities, and most importantly, the user's character. In order to increase the inclusive aspects of these social experiences and forums, we strive to understand the features and visual aspects of characters expected by people with disabilities (PWD) for self-representation.\par

% First, we pursued an understanding of whether improvements were justified by PWD's. Was the need for inclusive avatars there, or were PWD's happy with the options available to them? We came to the conclusion that PWD's did indeed require accurate representation in the Social VR landscape through other means than what was currently available in typical Social VR applications[]. This foundation of belief is validated by our previous research, published in 2022 []. Through this study, we found that PWD felt left out based on the lack of assistive technology  and other disability identifier when presented with a selection of VR character customization interfaces. We- also found that many participants wanted to control the physical attributes of the avatar which denote disability in social context and to pursue representation of their disability through these identifiers. These attributes could be anything from the representation of a prosthetic arm, to a difference in proportion of certain limbs. 