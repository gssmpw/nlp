\section{Description of Initial Guidelines}

Building upon the knowledge from our literature review and interview study, 
%from prior works, %%literature and application review, %yuhang{did we do the literature review?} \kexin{we did the literature review to inform the structure of interview protocol; I added this detail in method.}
we derive 20 design guidelines for inclusive avatar design for PWD. Our guidelines cover a broad range of disability expression methods across five aspects, including avatar appearance (G1), body dynamics (G2), assistive technology design (G3), peripherals around avatars (G4), and customization controls in the avatar interface (G5). 

To ensure actionability, each guideline has three components: a detailed description, quote examples from PWD, and concrete avatar feature examples to demonstrate the guideline implementation. %To ensure each guideline is actionable and applicable}, we provide a detailed description, quote examples from PWD, and concrete avatar feature examples to demonstrate its implementation. 
Appendix Table \ref{tab:overview_original} %\yuhang{fix}
presents a summarized version of our initial guidelines. %(%full version of the initial guidelines are in Appendix Table \ref{tab:full_original}; 
%revised guidelines after evaluation can be found in Table \ref{tab:overview_revised} and Appendix Table \ref{tab:full_revised}; Appendix Table \ref{tab:changes} demonstrated the changes of guidelines).} %\yuhang{fix})}. 
In this section, we elaborate on each guideline and the rationales, \change{grounded in both prior literature and findings from our interview study}. % and the rationale behind each guideline. 

% remember to compare and contrast; sharpen the thoughts
% first sentence of each sections/subsections should be hitting to the point -> what's the most interesting thing in this section 

\subsection{Avatar Body Appearance (G1)}
\change{Customizing avatar body appearance is the most common way to express disabilities. But deciding how to represent disabilities via avatar body is challenging, as it involves multiple design considerations (e.g., body compositions, customization of each body part). We derive guidelines to inform suitable avatar body design for PWD.
}
%\st{Our findings identified participants' needs in expressing disability identities on avatars, with the majority of them preferred authentic, realistic self-representation, echoing the prior works. Moving beyond prior work, we systematically uncovered the design dimensions of how to represent disabilities authentically through avatar's body appearance, as discussed below.} 

\textbf{\textit{G1.1. Support disability representation in social VR avatars.}} 
All participants desired the options to represent their disabilities via avatars, \change{echoing the findings from prior works \cite{kelly2023, zhang2022, zhang2023}}. As P52 described: \textit{``I think the biggest thing for me is the flexibility and the freedom to choose [how I can represent myself.]}

%\remove{However, such options are largely missing on existing avatar interfaces, blocking PWD from representing their identities through avatars. For instance, P52 wanted to add her prosthesis on avatar but could not find an option in the interface: \textit{``I think the biggest thing for me is the flexibility and the freedom to choose [how I can represent myself.] I would like to represent myself [in social VR] as an accurate reflection of what I look like. For what I've seen online, there is no standard option in the avatar creation [to represent my prosthesis]. I think that would be great to see if such a thing existed.''}}
%\yaxing{I think in this part of the results, we should focus primarily on the describing the guidelines and no need to touch on what developers and design should do - that's more of implication. This is also for consistency reason. Check with Yuhang on this one @yuhang}

\change{\textbf{\textit{Guideline}}: Avatar interfaces should allow all users to flexibly express their identities and present their disabilities ~\cite{zhang2022, kelly2023, assets_24}. The disability features should not be blocked behind the paywall ~\cite{kelly2023}.} %To ensure PWD to express their disability identities, avatar interfaces should provide more avatar body options to represent diverse disabilities. %This could be achieved through a variety of methods, such as providing diverse assistive devices options in the interface for users to easily add on (P18, P32) or enabling self-upload avatar for personalized representation (P4). 

%Emerging technology such as AI-generated avatars form photos can also be incorporated into VR technology \cite{Scorzin2023}, empowering users to have accurate avatar representations wit h minimum customization efforts (P7, P12, P40).
%. For example, several participants (e.g., P18, P32) wanted the interface to provide a variety of assistive device options so that they could easily add on to their avatars, or the option for users to upload their own custom avatars for more personalized representation (P4). Six participants (e.g., P7, P12, P40) were interested in AI-generated avatars from selfies, enabling them to have accurate avatar representations with minimum customization efforts.


\textbf{\textit{G1.2 Default to full-body avatars to enable diverse disability representation across different body parts.}}
Almost all participants preferred a full-body avatar, \change{echoing prior insights from Mack et al. \cite{kelly2023}}. In our study, more than half of the participants had a disability that affected the below-waist body area, which could only be reflected via full-body avatars. P9 indicated: \textit{``people can only get my whole disability identity with the full body [avatar].''}
%Almost all participants preferred full-body avatar, which provided them space to reflect disabilities that influenced different body parts. Specifically, more than half of our participants managed a disability that affected them from the waist downward, and only full-body avatars could express such disabilities authentically. P9 indicated: \textit{``people can only get my whole disability identity with the full body.''} 

In addition, multiple participants (e.g., P2, P34, P52) wanted to express their lived experiences as PWD, which could only be achieved \change{through the behaviors of full-body avatars}. As P34 described: \textit{``I prefer a full-body avatar, because [it shows] how [people with visual impairments] navigate the pathway, how they move the hands and the fingers, and how they read braille.''} %\st{Everything can be oriented to the people.}''} 

\change{\textbf{\textit{Guideline}}:} Avatar interfaces should offer full-body avatar options \cite{kelly2023}. Given the dominant preferences for full-body avatars over others (e.g., upper-body only, or head and hands only), we recommend making it the default or the starting avatar template, giving users the maximum flexibility to further customize their avatars as they prefer. 

\textbf{\textit{G1.3 Enable flexible customization of body parts as opposed to using non-adjustable avatar templates.}}
\change{Similar to prior work ~\cite{zhang2022, kelly2023}}, a third of participants (e.g., P3, P37, P52) preferred matching the avatar body with their authentic self, requesting full customization of the avatar body. \change{Moreover, our participants highlighted the need for asymmetric designs of avatar body parts (P4, P9). For example, P9 wanted avatars with different eye appearance to reflect her amblyopia (i.e., lazy eyes): \textit{``[I want] the opportunity to move or articulate the eyeballs to represent my disability, because my right eye is litter lazier than the other one.''}}
%For example, P3 desired to customize the avatar limbs to represent her amputation: st{\textit{``I have a limb difference, [and I am] a left forearm amputation. So I would really like to see characters where they don't have two long arms, but they have one long standard arm and one shorter [arm], like to the elbow. Or even if they do have two long arms, but one of them doesn't have a hand, or just one round-off around the wrist instead of extending completely with a five fingered hand.''}} \st{However, the existing avatar platforms mostly have ``standard'' body appearances without any adjustable features, and multiple participants (e.g., P9, P50) complained that they could hardly find any options to represent themselves accurately.} 

\change{\textbf{\textit{Guideline}}: Avatar interfaces should provide PWD sufficient flexibility to customize each avatar body part \cite{kelly2023}. While the customization spans a wide range, the most commonly mentioned body parts to customize include (1) avatar height, (2) body shape, (3) limbs (i.e., number of limbs, length and strength of each limb), and (4) facial features (e.g., mouth shape, eye size). Asymmetrical design options of body parts (e.g., eyes, ears) should also be available, %To address these needs, developers and designers should include features that enable users to customize each body parts of their avatar, such as options to customize the presence, length, and strength of each limb (P3, P12, P14, P56). Asymmetrical design options of body parts (e.g., eyes, ears) should also be available to fully represent body, 
such as changing size and direction of each eyeball to reflect disabilities like strabismus.}

%\yuhang{follow the same structure across all guidelines. In G1.2, one paragraph for concrete evidence and one paragraph for in-depth summary of the guideline. Currently, the second paragraph is commonly missing in most guidelines. Refer to the formal description of each guide since the language is quite dedicated.}

%For example, avatar platforms should offer options to customize the presence, length, and strength of each limb (P3, P12, P14, P56). Asymmetrical design options of body parts (e.g., eyes, ears) should also be available to fully represent body, such as allowing users to select the size and detailed look of each eyeball to reflect disabilities like strabismus (P4, P9). 


\textbf{\textit{G1.4 Prioritize human avatars to emphasize the ``humanity'' rather than the ``disability'' aspect of identity.}}
Approximately half of participants chose \change{\textit{human}} avatars to stress disability as an inherent part of personal identity. Multiple participants (e.g., P9, P31, P46) reported real-life experiences of being degraded and wanted to use human avatars to express that they should be seen as ``a whole person, not the disability'' (P46). Additionally, the human avatars also let users to represent multiple and intersectional identities (e.g., age, gender, and race) all together as an integral human being. As P9 emphasized: \textit{``Humanoid avatars show me as a whole person. I identify as a black woman with a disability, and that's really important when discussing a personal identity, because when describing somebody, you wouldn't just say, ‘Oh, they have a disability,’ instead you would say, ‘Oh, they're non-binary or female, and they're African American or Caucasian, and they have a disability.’ They all go together.''} %\st{In this sense, the human form of avatars set up the fundamental base for authentic representation.}
%, with disabilities as an important part of it. Many participants (e.g., P5, P33, P57) also reported feeling more connected to humanoid avatars when representing personal identities in social VR. 

\change{\textbf{\textit{Guideline:}} Social VR applications should offer human avatar options whenever the application theme allows. %Even in certain VR applications where human avatars do not fit (e.g., \textit{Among Us}), practitioners should considering adding human design options should be reflect certain  do not fit  It is well adapted to various social VR contexts (e.g., multi-player games, collaboration platforms, education) and compatible with diverse avatar styles (e.g., the abstract style in Roblox, the cartoony avatars in Horizon Worlds).
} %\kexin{HOLD - use cases: Developers think this guideline dependents on use cases (e.g., Among us may not be suitable), so I want to expand on that a bit more. I wonder how do we incorporate developers' feedback when revising guidelines - do we mention them as evidence with P-ID? It sounds a bit off and break the flow if we just mention the use cases directly.}

\textbf{\textit{G1.5 Provide non-human avatar options to free users from social stigma in real life.}} %\kexin{re comment: this guideline applies to all disabilities, not only limited to the invisible ones.}
\remove{Unlike those who authentically indicated disabilities with accurate details,} Eleven participants (e.g., P17, P43, P49) preferred non-human avatars, such as robots or animal characters, to avoid disclosing disabilities and protect themselves from the judgment they often faced in real life. \remove{They perceived social VR as a utopia where they can escape from real life, and avatars in non-humanoid forms, such as robotics or animal avatars, allowed them to avoid the social judgments and norms commonly tied to disabilities.}  
For example, \remove{P45 said: \textit{``[Non-human avatars] hide more of my real disabilities, being the opposite of showing off my disabilities.''}} P49, who identified as neurodivergent, chose a robotic avatar to lower social expectations: \textit{``It feels like people's expectations of how neurotypical I'm gonna seem are lowered if I'm a robot or just a non-human avatar.''} 
%Meanwhile, non-humanoid avatars still allow PWD to express themselves and represent their disability identities in a symbolic way. For example, P17 used animal avatars as spiritual animal to demonstrate her autistic traits: \textit{``Well, I just feel like [the animal avatars of lions] relates to what I'm passing through.''} 

\change{\textbf{\textit{Guideline:}} Besides human avatars, avatar interfaces should also provide diverse forms of non-human avatars, empowering PWD to choose the one they relate with flexibly.}


\subsection{Avatar Dynamics: Facial Expressions, Posture, and Body Motion (G2)}
Unlike 2D interfaces, the embodied and multi-modal nature of avatars in social VR enabled PWD to express themselves through diverse approaches beyond static appearance. \change{We derive guidelines based on how PWD leveraged avatar dynamics, such as facial expressions, posture, and body motions, to represent disabilities.} % through facial expressions, posture, and body motion.
 
\textbf{\textit{G2.1 Allow simulation or tracking of disability-related behaviors but only based on user preference.}}
Nine participants (e.g., P7, P47) wanted their avatars to reflect the realistic behaviors caused by disability for a stronger connection (e.g., limp by P18, stumbling by P4). %\yuhang{example behaviors that are suitable to present with participant number, limps?}). 
However, \change{similar to Gaulano et al. \cite{chronic_pain_gualano_2024}}, eight participants (e.g., P6, P14, P49) were concerned that showing disability-related behaviors would reinforce stigma (e.g., involuntary behaviors like nervous tics). \remove{Participants preferred avatars to reflect behaviors with their controls. For example, P6 didn't want her avatar to reflect disability-related movements at all to avoid misconception, as she explained: \textit{``The way I move authentically is kind of jaggy, and I swerve. People asked me if I'm drunk all the time. So I'd like to go as quickly as I can in a smooth way [...] even though that's not authentic.''}} \change{Participants highlighted the need for controlling what behaviors to track or simulate (e.g., P14, P16, P49).} As P14 indicated: \textit{``I think it would be cool if you could choose to have [the movement to be reflected]. But I also think there is a fine line between inclusion and offensive imitation.''} 

\change{\textbf{\textit{Guideline:}} Users should be able to control the extent of behavior tracking in social VR. With the advance of motion tracking techniques, avatar platforms may disable subtle behavior tracking by default to avoid disrespectful simulation, but allow users to easily adjust the tracking granularity for potential disability expression.}

%should be careful not to create simulations that may cause misunderstanding or reinforce stereotypes, and they should only do so if PWD prefer it. % empowering PWD to decide how they prefer to express the behavioral characteristics. 

%For example, P49 worried showing her nervous tics would cause confusions thus preferred smoother reflections instead of full simulation: \textit{``I have nervous tics that are kind of full body shutters. When I do those in real life, the VR avatar does often follow those, which makes it hard for people to figure out if it's glitching out or something. So finding ways to make those smoother and more reflective of reality, rather than like, `is this internet thing? or what's happening?' ''} Other participants, like P6, didn't want her avatar to reflect disability-related movements at all, as she explained: \textit{``The way I move authentically is kind of jaggy, and I swerve. People asked me if I'm drunk all the time. So I'd like to go as quickly as I can in a smooth way [...] even though that's not authentic.''} Developers and designers should be careful not to create simulations that may cause misunderstanding or reinforce stereotypes, and they should only do so if users prefer it, empowering PWD to decide how they prefer to express the behavioral characteristics. 


\textbf{\textit{G2.2 Enable expressive facial animations to deliver invisible status.}}
A third of participants (e.g., P1, P4, P40) desired to express disabilities through avatar facial expressions. This is particularly important for people with invisible disabilities, whose conditions mostly surface through emotions and subtle non-verbal cues. %\st{For instance, participants (e.g., P7, P44, P47) with autism used the direction and focus of the avatar's eyes looking away as a way to express their autistic identity, as P47 described: \textit{``a big thing [that] a lot of people on the autism spectrum struggled with [is making] eye contacts.''}}
Three participants (P4, P40, P51) noted that their disabilities involved rapid fluctuation or contradicting feelings (e.g., bipolar disorder, ADHD), thus preferring avatars to show a spectrum of facial expressions. For example, \change{P51 experienced multiple invisible disabilities (i.e., depression and ADHD) and used different facial expressions to represent different aspects of their disabilities}: \textit{``When representing depression, the facial expression is more sad or in thought. When having ADHD moments, [the avatar] being more excited or manic.''} 

\change{\textbf{\textit{Guideline:}} Avatar platforms should enable diverse facial expressions, allowing PWD to express emotion, portray mental status, and indicate fluctuation of invisible disabilities \cite{assets_24}.  %For example, the five basic emotions (i.e., anger, fear, sadness, disgust, enjoyment) \cite{ccp_basic_emotions} could be a starting point. We encourage practitioners to expand and diversify based on their unique use scenarios.
}

\textbf{\textit{G2.3 Prioritize equitable capability and performance over authentic simulation.}}
Four participants (P7, P14, P15, P52) highlighted that the avatar performance should demonstrate equitable capabilities to other users, not being limited by their disabilities or direct motion tracking. %Although some participants (e.g., P7, P14, P15, P52) wanted their avatars to authentically reflect how they move or behave in real life, they particularly mentioned that their avatar's actual performance and capabilities should not be limited by the behavior's characteristics.  
For example, P52 mentioned that the moving speed of her avatar walking with limps should not be slower than other avatars: \textit{``I walk with a slight limp, [but] I don't think I need the actual movement [on avatars] to reflect how [fast] I walk. \remove{[Because] when I use games, I see the movement aspect more of a practicality than part of the game [...]} 
I think having a limp would be cool, but I wouldn't want to be slower than [other avatars]. \remove{I wouldn't want to have a maximum speed, because I chose to have a limp earlier in the avatar making process [...]} 
Being able to just keep up with peers’ [avatars], pace-wise, would be the most important thing.''} 

\change{\textbf{\textit{Guideline:}} PWD value equitable and fair interaction experiences more than the authentic disability expression. Therefore, avatar platforms should ensure the same level of capabilities and performance for all avatars, regardless of whether disability features or behaviors are involved.}


\textbf{\textit{G2.4 Leverage avatar posture to demonstrate PWD's lived experiences.}}
In addition to the facial expressions and body movements, five participants (P4, P9, P31, P33, P34) preferred leveraging avatar postures to demonstrate their lived experiences. For example, as a person with low vision, P4 described his unique posture when interacting with others: \textit{``[My] vision is directed at one angle. So my head is turned lightly, because I'm not looking at people directly all the time.''} Representing postures and mannerisms on avatars also help increase awareness and resolve misunderstandings about disabilities, as P34 shared: \textit{``Instead of looking at the person who is speaking, people with visual impairments take their ears nearby to the place where the sound is coming from. This gives some wrong impressions to the people that the visually impaired people have not given attention to the speakers. That is not the real story.''}

\change{\textbf{\textit{Guideline:}} Disabilities can be expressed via avatar posture. Avatar platforms should enable certain posture tracking or simulation (e.g., unique facing directions of individuals with low vision during conversation) to enable authentic disability representation.}

%This informs practitioners the opportunities to use avatar posture as a design medium for disability expressions. With the benefits of facilitating social interactions for PWD, we see that the posture representation could be particularly helpful for life-like avatars and platforms that centered on interactions and communications.

\subsection{Assistive Technology Design (G3)}
Adding assistive technologies to avatars is a key method adopted by PWD for disability disclosure ~\cite{zhang2022, kelly2023}. \change{Beyond prior literature, we revealed key aspects of assistive technologies, such as types, appearances, and relationship to avatars, to guide proper designs.} 
%how to design assistive technologies properly to avoid misconception and misuse in the social VR setting. In the following, we outline key design aspects of assistive technologies that developers and designers should consider. 

\textbf{\textit{G3.1 Offer various types of assistive technology to cover a wide range of disabilities.}} 
 Like prior work has indicated \cite{zhang2022,kelly2023}, we found that multiple participants (e.g., P18, P33, P39) viewed assistive technologies as part of their body. P39 described the meaning of wheelchairs to wheelchair users: \textit{``\remove{For people in wheelchairs, our wheelchair is an extension of our body.} We view it emotionally as an extension of ourselves, and it gives us our independence.''} \remove{P18 and P33 also reflected that being able to have avatar with assistive technologies they used in daily life made them feel empowering and being included in the social VR.}

\change{\textbf{\textit{Guideline:}}} Avatar interfaces should offer assistive technologies that are commonly used by PWD \cite{zhang2022}. \change{The most desired types of assistive technologies include: (1) mobility aids (e.g., wheelchair, cane, and crutches); (2) prosthetic limbs; (3) visual aids (e.g., white cane, glasses, and guide dog); (4) hearing aids and cochlear implants; and (5) health monitoring devices (e.g., insulin pumps, ventilator, smart watches). % \yuhang{narrow down}. 
Practitioners should consider including at least these five categories of assistive technologies in avatar interfaces.
In addition, due to PWD's different technology preferences \cite{kelly_AI24}, we encourage practitioners to offer more than one assistive technology option in each category, for example, including guide dog, white cane, and glasses for visual aids.} 

\textbf{\textit{G3.2 Allow detail customization of assistive technology for personalized disability representation.}}
Eleven participants (e.g., P15, P18, P32) desired to better convey their personalities through assistive technology customization. \change{Echoing prior research ~\cite{zhang2022, kelly2023}, changing the colors of assistive technologies and attaching personalized decorations (e.g., add stickers on wheelchair, P18) are two most preferred customization options.}
%PWD viewed assistive technologies as part of their body \cite{zhang2022, kelly2023}, and many of them customize it to convey their personalities (e.g., P38, P45, P46).  
%customize the design of assistive technology to represent their disability in more diverse and personalized way. 
\remove{With assistive technology being an extension of the user’s body, being able to have diverse customization options of assistive technologies is as important as customizing the avatar’s appearance, as P39 said: \textit{``Making some more customization in the wheelchair [is] in the same way that you make customization for eye color, nose shape, [and] all those things.''}} However, some participants (e.g., P20, P50) %\yuhang{add more example}
\change{desired to see more various styles of assistive technologies, such as a futuristic styled hoverchair (P20).}
%emphasized the need for a wider range of customization options, such as xx \yuhang{such as???}. % so that they can choose the one they feel connected with. 
\remove{as current avatar platforms offer only limited default choices.}  
\remove{As P52 expressed frustration over the lack of w heelchair variations in social VR avatars: \textit{``[Now] you either have a wheelchair or no wheelchair, but you can't customize the type, shape, or any various add-ons. Like is it motorized [wheelchair]? Is it like a manual one? So I think having the ability to choose what additional features you'd like to add would be really nice.''}} 

\change{\textbf{\textit{Guideline:}} Avatar platforms should allow customizations for assistive technology \cite{kelly2023,zhang2022}. Basic customization options should include adjusting the colors of different assistive technology components and adding decorations (e.g., stickers, logos) to them. More customization could be added based on specific use cases.} 

%should be customizable Instead of only having one default choice, practitioners should ensure that assistive technologies features are customizable. Based on findings from  prior works \cite{zhang2022, kelly2023} and our large-scale interview, changing the color of assistive technologies and attaching personalized decorations to them (e.g., add stickers on wheelchair, P18) are two of the most preferred customization options among PWD. We suggest practitioners to incorporate these two as the base customization level and add on based on specific use cases.}
%such as adjusting the color (e.g., P18, P39, P44) or adding personalized decoration such as stickers on wheelchairs (e.g., P9, P45, P49). 


\textbf{\textit{G3.3 Provide high-quality, authentic simulation of assistive technology to present disability respectfully and avoid misuse.}}
Four participants (P4, P6, P34, P35) preferred high-quality assistive technology simulation with authentic details similar to those in real life. They were concerned that inaccurate assistive technology designs in social VR may depict misleading figures of PWD and lead to misuse, echoing Zhang et al. \cite{zhang2023}. As P35 described: 
%\st{wanted the assistive technologies to have high-fidelity looking with realistic details. As an emerging social platforms, participants found it was not uncommon to see some avatars with disability features, such as avatars on wheelchairs. However, these avatars were usually poorly designed with low-quality or stereotypical manner, leading to the misuse of assistive technology and perpetuation towards PWD. For instance, P35 recalled seeing poor wheelchair representation in social VR where people treated it as trolling or memeing:}
\textit{``[I’ve seen] really poor representation [of wheelchairs]. They're usually joke avatars or meme avatars that have wheelchairs.''}

\change{\textbf{\textit{Guideline:}} To avoid misunderstandings or misuse, the assistive technolgy simulation should convey standardized, authentic details of the real-world assistive devices \cite{zhang2023}, regardless the overall avatar style. For example, the design of a white cane should show the details of tip and follow its standardized color selection, no matter the design style is photorealistic or cartoon. 
%(P34). 
We recommend practitioners to model assistive technologies by following their established design standards, such as design guidelines for white canes \cite{who_white_canes}, wheelchairs \cite{russotti_ansi_wheelchairs}, and hearing devices \cite{ecfr_800_30}.}
\remove{Their designs should be high in quality and contain sufficient details, so that users can tell these avatars were invested with great efforts, aiming for identity representation instead of trolling (P6, P34). For example, the design of a white cane for people with low vision should show the details of tip and follow the standardized color selection for such walking aids (P34).} %: \textit{``While walking, the tip of the white cane should move like the pendulum motion, [moving] forth and back in that way''} (P34).
%\kexin{consider changing the visual examples for this guideline? I think the key of this guideline is to follow the conventional design standards of AT that is true-to-life, instead of how realistic/high-fidelity the AT features are (in which some developers think this guideline is limited to styles and only apply to those with realistic avatars). A better example might be that both pixel-style white cane and realistic-style white cane can follow the guidelines, as long as they have authentic details of how does a white cane look like in real life (i.e., red tip, black handle, a straight thin tube). The style (e.g., realistic or abstract) does not matter, but the correct looking of AT does -> HOLD: how to convey this / cite evidence} \yaxing{this is a good point. I think you should add the point about "instead of how realistic the AT are" in the description.}

\textbf{\textit{G3.4 Focus on simulating assistive technology that empower PWD rather than highlighting their challenges.}} %\kexin{I think this guideline can be merged to G3.3., as we only have one wheelchair example for it, and I have a hard time to generalize it to other AT designs (many developers also asked to diversify examples for this one). If we agree to merge, this one can be an example of true-to-life design by not having medicalized design of AT.}
%Although participants preferred diverse types of assistive technologies, 
Eight participants (e.g., P18, P39) only wanted to add assistive technology features that can demonstrate their independence instead of challenges, (e.g., hospital wheelchair vs. power wheelchair). %Participants felt frustrated to be misportrayed as being dependent or incapable when using assistive technologies. 
For example, P18 found media often misrepresented PWD by showing them sitting in a hospital-style wheelchair that requires others' assistance to move: \textit{``Most of the representations we see in fiction, video games and TV, they always use hospital chairs, which are not practical. No actual disabled person uses a hospital chair in real life, which has armrests and big push handles, because it's built for somebody to push you. However, a manual wheelchair is designed for you to push yourself.''}
\remove{Participants wanted to correct the media misrepresentation by showing how they can achieve independence through the use of assistive technology. Taking the wheelchairs as examples, P6 noted the power chair should have a joystick to show the user can move independently; and P39 strongly preferred manual wheelchair without any pushable handles to demonstrate the self-independence: \textit{``It's important to me that it doesn't look like I'm ready to be pushed by someone else. I'm stating that independence [achieved through wheelchair]. I'm solidly myself, and I don't need another person. This is a big deal in our community [...] we’re not going to want push handles.''}}

\change{\textbf{\textit{Guideline:}} When determining what assistive technology features to offer, practitioners should only select assistive or medical devices that can be easily controlled by PWD to demonstrate their capability (e.g., manual wheelchair, cane) and leave out the ones that PWD cannot independently use or the ones that highlight their challenges (e.g., hospital-style wheelchair, bedridden avatars).}

\textbf{\textit{G3.5 Demonstrate the liveliness of PWD through dynamic interactions with assistive technology.}}
In addition to the visual details, five participants (P4, P6, P18, P39, P44) \change{wanted their avatars to actively interact with the assistive technologies, such as rolling their manual wheelchair (P18) or sweeping their cane (P34) when moving, to demonstrate their capability and liveliness}. %demonstrated independence by showing how they actively control their assistive technologies. To achieve that, participants noted that the interaction with assistive technology should authentically simulate what they look like in real life. 
%For example, P18 \change{wanted the \yuhang{his?her?} avatar to display a circular arm movement while pushing the manual wheelchair, and P34 preferred avatar to control the white cane in pendulum motion, 
As P34 mentioned: \textit{``While my avatar is walking, the tip of white cane should be moving back and forth like a pendulum motion.''}
\remove{\textit{``I think having the option to roll [wheelchair] would be good. I’ve seen some 3D models of wheelchair users in video games, and their arms don't move while they're rolling, which is really weird to me. Because I push myself with my hands.''}} %P39 noted that proper postures for avatars using assistive technologies, like sitting up tall in a wheelchair, can also reflect the liveliness and capability of people with disabilities.

\change{\textbf{\textit{Guideline:}} Beyond providing assistive technology options, social VR platforms should enable suitable interactions between avatars and assistive technology. The interactions should authentically reflect PWD's real-world usage of their assistive technology, such as how a blind user sweeps their cane, or how a wheelchair user moves their arms to control their wheelchair.  %developing assistive technologies, practitioners should ensure the avatar could demonstrate how PWD actively control and interact with the assistive technologies in real life.
}

\textbf{\textit{G3.6 Avoid overshadowing the avatar body with assistive technology.}}
Seven participants (e.g., P9, P18, P57) demanded to flexibly adjust the size of assistive technologies to fit their avatar body. \change{They emphasized that, while expressing disabilities, the image curation should focus on the whole avatar rather than just the assistive technology. As P39 highlighted: \textit{``The wheelchair is not the focus of the image; [rather,] the focus is on the avatar having a good time.''}} \change{P18 recalled their %\yuhang{her?his?} \kexin{P18 is non-binary and use they/their}
prior experience of being overshadowed by the assistive technology}: \textit{``
\remove{I think that having the option to actually make the chair larger or smaller, depending on how large or small your avatar is, is a good detail. Because sometimes wheelchairs don't fit you.}I have encountered 3d models where the wheelchair is so big and the person sitting in it is so small, and it just doesn't look right.''}

\change{\textbf{\textit{Guideline:}} The size of assistive technology should not dominate the avatar body but rather fit the body size. Avatar platforms should automatically match the assistive technology model to different avatar body sizes, and allow users to adjust the size of assistive technology to achieve the preferred avatar-aid ratio. The combination of avatar and assistive technology should also be seamless without affecting the quality and aesthetics of the original avatar ~\cite{kelly2023}.}


\subsection{Peripherals around Avatars (G4)}
Beyond the design of avatars, the peripheral space around them can also be leveraged for disability expression. We explored this new design space and identified design guidelines.

\textbf{\textit{G4.1 Provide suitable icons, logos, and slogans that represent disability communities.}}
Sixteen participants (e.g., P5, P37, P58) desired %represented disabilities symbolically by incorporating 
representative icons, logos, or slogans of disability communities for identity expressions, and they wanted to creatively attach these symbols to a variety of places, such as on avatar's clothing (P53, P56), accessories (P13, P54), or even the space surrounding the avatars (P14, P47). \change{This confirm previous insights that disability-related symbols can help PWD educate other users and raise awareness in the social VR space \cite{zhang2022, assets_24}.} 
%By wearing the community icons or logos, participants not only showed support to the disability community they connected with but also raised disability awareness in the social VR space. 
\remove{For example, multiple participants mentioned using the rainbow infinity icon to represent the autism community (e.g., P1, P46, P47), zebra printing for rare disease (P56), and sunflowers that symbolize interac tion invisible disabilities community (P5, P14). For example, P14 planned to attach a sunflower yard in the background of her avatar to symbolize her invisible disabilities. %, as she pictured: \textit{``It’s about hidden disabilities, and you can get sunflower lanyards, which is a way of saying ‘I'm disabled, but you can't tell.’ So if the avatar looks like they're walking and they've got sparkly, flowy sunflowers behind them, [that] would be cool.''} 
Another participant, P3, would like to have disability community icons on the avatar's T-shirt to show community pride.} %: \textit{``Now we have avatars who can wear T-shirts with the LGBTQ plus pride flag on it, or they can wear T-shirts that have ‘Black Lives Matter.’ So having equivalent things for disability would be awesome.''} These examples suggested that avatar platforms should include some widely recognized icons, logos, and slogans representing diverse disability communities in the avatar interface.}

\change{\textbf{\textit{Guideline:}} Awareness-building items (e.g., logos, slogans) should be provided, allowing users to attach them to various areas on or around the avatars \cite{assets_24, zhang2022}. %, such as the apparel, accessories, and assistive technologies. 
Some widely recognized and preferred symbols that represent different disabilities for practitioners to refer to include (1) the rainbow infinity symbol that represents the autism community \cite{assets_24, rainbow_infinity_symbol}, (2) the sunflower that represents hidden disabilities \cite{isit_assets24, hidden_disability_sunflower}, (3) the disability pride flag \cite{disability_pride_flag}, (4) the spoons, symbolizing spoon theory for people with chronic illness \cite{kelly2023, assets_24}, and (5) the zebra symbols for rare diseases \cite{assets_24, Gualano_2023}. %We recommend practitioners to include these symbols in their avatar interface and allow PWD to flexibly attach them to multiple avatar parts.
}

\textbf{\textit{G4.2 Leverage spaces beyond the avatar body to present disabilities.}}
%While the symbols provided participants a standard and easy way to represent their disabilities, 
Eight participants (e.g., P37, P43, P46) wanted to express disabilities more creatively and flexibly through the space behind the avatar body. \change{This is especially favored by people with invisible disabilities, as it helps visualize PWD's mental conditions. For example, P43 wanted a visual indicator of a cloudy and rainy background to symbolize her depression and anti-social mode at the moment. This echoes prior implications that an avatar's background can provide contexts into PWD's experiences \cite{kelly2023, assets_24}.}
%People with invisible and fluctuating disabilities particularly preferred this approach, as it provided indicators to visualize their frequently changing conditions in social scenarios, keeping others informed. For examples, 
%P43 wanted to add a visual indicator of a cloudy and rainy background to symbolize she felt depressed at the moment and was in anti-social mode: \textit{``I’d imagine there were things around me, like a dark gray cloud or it's raining in the background and being right above you. And everywhere you go, it's right there.''} 
\remove{P47 would like to add a variation of battery symbols over the avatar's head, which would change levels based on her energy: \textit{``My energy levels can fluctuate just a lot. Someday, I may have a little bit of energy, and the next day I may have a lot of energy, and that could actually change within a matter of hours. So the idea that I have is a battery symbol that I could adjust the battery level shown on that to show you how much energy that I have to spend. It's a signal to my friends that ‘hey, my battery's low, I may sound really tired right? I'm okay, I just have low energy.’ [Other times] I could turn my battery all the way up and be like, ‘Hey, let's see, we can do something a little bit more active.’''}}

\change{\textbf{\textit{Guideline:}} When designing avatars, practitioners should consider leveraging avatar's peripheral space to enable users to better express their status, especially for individuals with invisible disabilities. Some design examples include a weather background to indicate mood and a battery sign to indicate energy level \cite{assets_24}. %The space beyond the avatar body not only provided a novel medium for PWD to express their disability status but also cultivate pro-social behaviors in social VR.
}

\subsection{Design of Avatar Customization and Control Interface (G5)}
\change{The usability and accessibility of avatar interfaces can significantly impact PWD's avatar customization experiences. We thus identified guidelines for avatar customization and control interfaces to enable smooth avatar curation for PWD.}

%that influenced PWD's engagement during avatar customization process, including interface layout, input controls, and mechanisms of displaying disability-related features.

\textbf{\textit{G5.1 Distribute disability features across the entire avatar interface rather than gathering them in a specialized category.}}
Five participants (P18, P32, P44, P49, P57) strongly preferred embedding the disability-related features naturally into different categories of the avatar interfaces \change{(e.g., asymmetrical eyes under the eye category, amputation under the body category), as opposed to collecting them in a specialized category for PWD, which marginalized them by ``setting PWD apart from other users'' (P32)}. % exclude PWD and make them feel they are using features intentionally designed as ``for disabilities'', which sets them apart from other users (P32). 
\remove{P49 suggested developers and designers to treat the disability-related features in the same way as any other avatar features in the interface: \textit{``Just treating them as neutral instead of either a burden to have to design or something you get to feel really special for designing''}.}
%They reported that seeing all features related to disabilities in a separate category made them feel they were using features intentionally designed as `for disabled people', which further isolated them from other users. %P32 emphasized the importance of not separating disability-related features from others:
% \begin{quote}
%     \textit{``Have those [disability representation] options in a variety of places, not like to create a disabled avatar, [you need to] go 13 levels down to the left, and [there’s a] sub-menu for that. Just make it integral to what you're designing, instead of making it like, ‘you gotta go on the short bus to get to the avatars for people with disabilities. Make it a part of everything else. Don't isolate it.''} -- P32, a blind person. 
% \end{quote}

\change{\textbf{\textit{Guideline:}} Avatar features for disability expression should be treated in the same way as other avatar features. In avatar interfaces, disability-related features should be properly distributed in their corresponding categories. There should not be a specialized category for PWD. 
 For example, assistive technologies should be included in the accessory category rather than a separate assistive technology category. }
%\kexin{do we have a more inclusive term for 'disability-related features'...R1 was criticizing some language are not inclusive.}
%\textit{``You can include a cane with the accessories tab instead of having a disabled tab over there…that can be kind of ostracizing. Just treating them as neutral instead of either a burden to have to design or something you get to feel really special for designing.''} 


\textbf{\textit{G5.2 Use continuous controls for high-granularity customization.}}
Eight participants (e.g., P16, P37, P49) believed that the control components in avatar interfaces can largely affect their customization flexibility. To accurately represent their disabilities, participants preferred continuous control methods (e.g., a slider) over discrete options (e.g., binary switches, drop-down menu with limited options). As P47 said: \textit{``[I prefer] the sliding scale. You can really change [the length of the limb] to a very particular level.''}
\remove{\textit{``It's better to have a spectrum of choices, or even a slider-like for people to change your nuanced level.''} Since disability representation was a spectrum (P16, P48), input controls that offered a continuous range of options, such as sliders and knobs, were preferred in the avatar customization interface. As P47 said: \textit{``[I prefer] the sliding scale. You can really change it on a very particular level.''}; P48 also agreed that \textit{``slider is better than binary options.''}}

\change{\textbf{\textit{Guideline:}} Avatar interfaces should adopt input controls that offer a continuous range of options to enable flexible customization. This could be widely applied to a variety of design attributes, such as the size and shape of multiple avatar body parts.}

\textbf{\textit{G5.3 Offer an easy control to turn on/off or switch between disability features.}} \label{g5.3}
Twelve participants (e.g., P11, P32, P57) noted that they didn't want to always disclose their disability identities in social VR. Instead, disability representations were often context-dependent \change{\cite{zhang2022, kelly2023, assets_24}}. %, and participants used avatar with disability features when they felt comfortable in a social environment. 
%For example, P41 didn't want to disclose her autistic identity when surrounded by strangers or unfriendly users: \textit{``Disability representation is very dependent on the social environment of the space. There are times where I am in virtual spaces that feel very hostile to disabled and autistic people. In those spaces, I would be less likely to openly present [my disabilities].''} 
For example, P39 didn't want to use avatars on wheelchair when surrounded by strangers or in unfamiliar VR worlds. \change{Moreover, people with multiple disabilities or fluctuated status  also need a fast and easy control to switch between different avatars (e.g., avatars with different facial expression in G2.2) or adjust status indicators (e.g., the weather background in G4.2) to flexibly update their disability expression based on contexts.}  
\remove{\textit{``Although I have a disability and I'm comfortable with it, it is not the most important thing to me. Sometimes I might not want to lead with [my disability], especially when you have physical disabilities that people can see [but] you have no control over how people see you right away.''}}

% \begin{quote}
%     %\textit{``If I'm in a very comfortable setting, and [people are] accepting, I'm going to come in there [as an avatar with disability-related features]. [At the same time,] 
%     \textit{It's important to have that [avatar with disability-related features] to be changed, where I can still have how my body shows up in the world but not necessarily with the wheelchair. That’s important, because although I have a disability and I'm comfortable with it, it is not the most important thing to me. So sometimes I might not want to lead with that, especially when you have physical disabilities that people can see, where you encounter a lot in the world where you have no control over how people see you right away.''} -- P39, a person with mobility disability
% \end{quote}

\change{\textbf{\textit{Guideline:}} Interface should provide easy-to-access shortcut control that enable users to conduct \textit{ad-hoc} avatar updates. Important control functions include: (1) toggling on and off the disability-related features \cite{assets_24}; (2) switching between different saved avatars \cite{kelly2023}; and (3) updating status for fluctuating conditions ~\cite{assets_24}. 
%as needed. For examples, developers can implement a shortcut to instantly remove disability features from an avatar (e.g., P6, P57), or allow users to save multiple versions avatars so that they can easily switched to those without disability representations when needed \change{\cite{kelly2023}}. Easy control method should be provided, allowing users to adjust the peripheral design \textit{ad-hoc} to reflect their condition fluctuation. Control interfaces should be designed to enable easy switching among facial expressions during social interactions \cite{kelly2023}.
}


%=======================

% \subsubsection{Highlight the empowerment of AT instead of stigma} (xx \% of participants).

% Using AT is often associated with the stereotype of being dependent. To combat such stereotype, the AT design should demonstrate how PWD could achieve independence with the use of AT, as P39 emphasized: \textit{``It's important to me that it doesn't look like I'm ready to be pushed by someone else.''} For example, adding a joystick design on wheelchair, which shows that wheelchair users could move independently (P6). 

% % [mannerism-wise] show capabilities: 
%     % P52: "so I feel like as far as walking is concerned, the feature I would most be interested in is being able to go at a pace that would keep up with my friends' avatars. So I don't think there's, for me, at least the the visual appearance of an avatar is more sort of how other people will, you know, perceive you in that online setting. And being able to just keep up with peers, pace-wise, would be the most important thing, as far as specific mannerisms are concerning."

% \subsubsection{Avoid overshadowing the person with AT simulation (G2.3.)} (xx \% of participants). 

% The AT design should prioritize showing the individuality of PWD, instead of making the AT more prominent than the user. For example, when designing the wheelchair, minimizing its appearance by having the back height lower than the user's shoulders could help (P39). The size of AT should also streamline with avatar's size, allowing users to customize AT size based on their avatars. [add example image of wheelchair design and interface of size customization]
%     % P52: "But I think definitely proportional. I wouldn't want anything exaggerated just because I feel like that would fool me at least what are on it sort of demeaning stereotype. And saying, when I look at this person, this is what I see first, which is not not a thing I aim to go for. Not a thing that I would like to believe that myself as other than, yeah."

% % make the guideline actionable by doing: 
% --> AT size should be proportionate to avatar body size + 
% AT design style should match with avatar style + 
%     % [AT should match with avatar style, don't break the integrity]: "So just because I feel like at the end of the day, it's, it's mostly a visual thing, how? Yeah, how are you being perceived? And is it an amount of that's roughly physically realistic. But not, not to the extent of oversimplification, or hyper realism when, oh, I guess that's the other thing. The greeter which the animation development is, I would like to see that be roughly equivalent to that of the other able bodied avatars, just so it's not like this is something additional and special and so unique. And look how hyper realistic it is. But with with the same level of level of animation refinement as the other avatars."
    
% change AT color to show personality + 
% add accessories and decoration to AT

% \subsubsection{Show realistic movement of AT for authentic disability representation (G2.4.)} (xx \% of participants). 

% The AT should exhibit realistic movement that accurately reflects how the technology functions in real life. This not only provides an authentic and respectful representation of PWD but also helps to enhance understanding and acceptance of their experiences with AT in everyday life. For example, the wheelchair's wheels should roll when in motion (P6), and avatars should have the pendulum movements of people with visual impairments using a white cane (P34).



%-----------------------
% write the interesting findings when we generate the guidelines.
% present a nice table like AI guideline paper (i.e., our system)
% the heuristic evaluation is a summative study, write like human-AI paper

    
% Objectifying disability is a common form of ableism experienced by PWD. To combat objectification, avatars should have humanoid model to represent and signify PWD as real human-being in social VR. As P46 elaborated: \textit{``I want people to see that people with mental illness and depression are real people and not just their disability.''}

% \subsubsection{Default to full-body avatars to cover a broad range of disabilities across different body parts (G1.2)} (xx \% of participants). 

% Many aspects of disabilities are mostly visible through the full body. If not full body, no way to show the disability. For example, the use of assistive technology (e.g., wheelchair, leg braces-P14) and disability representation through movement often require adaptations that are visible in the avatar's lower body.

% \subsubsection{Enable customization to the presence, length, and strength of limbs.}
% % merge -> all customization, as one guideline; further specific to different body parts as subsubsubguidelines

% \subsubsection{Have asymmetric design of eyes.} % stand alone
% - e.g., size differences (P4)
% - eyeball directions, cross eyes (P9. P31)

% \subsubsection{Allow customization of body characteristics for chronic health condition.}
% - body shape fluctuates due to medication
% - skin conditions change 
% % - scar thing -> add on of the skin, instead of part of the skin (not customization) / accessories?

% % be careful with the "customization" use


% \subsection{Design Guidelines for Facial Expressions: }

% \subsubsection{Leverage facial expressions to show realistic mannerism related to disability.}
% an important way to represent invisible disabilities like ADHD and autism...
% avoid/hard to make eye contact to rep. autism (P7, P47)

% \subsubsection{Design expressive facial cues to visualize emotion.}
% ...


% \subsection{Guidelines for Avatar Customization Process: }

% \subsubsection{Avoid highlighting disability representation features as an individual category, instead integrating them to general customization interface.}

% e.g., AT is part of accessories, rather than "assistive technology" stand alone

% \subsubsection{Allow multiple avatars to be saved for dynamic representation}
% easy to change through different avatars for fluctuating conditions

% \subsubsection{Easy interaction mechanism to turn on and off disability features of an avatar}

%)--------------------
% novelty of body part customization -> a bit dry
    % disability specific -> how people with different disabilities may present it differently
    % anything new that developer could learn 
    % distinguish from generic guidelines
    % be a bit more sharp and concise about 
% the rationale goes to explaination
    % should not conflict with each one 
    % use game a11y as a template -> publishable list of guideline that we should follow as the format; guideline + explaination + examples (-> avatar library): https://gameaccessibilityguidelines.com/basic/ 
    % -> set a expected final outcome (each code should be aligned with the research goal) | build the sense of good and interesting findings -> e.g., what people already know vs. novelty 
    % how impactful
    % what to be presented to the experts -> formal list: also ask = what do we want to evaluate with (not overleaf version, follow a11y game guidelines)
    % show guideline to Daniel and Ru, discuss with Scott -> get some fresh eyes
