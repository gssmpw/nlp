% {\newpage
% %\renewcommand{\arraystretch}{1.1} % Adjust vertical spacing
% \small
% %\footnotesize
% \onecolumn

% \begin{longtable}{p{0.4cm}p{4cm}p{6.5cm}p{5.5cm}}
% \toprule
% \textbf{} & \textbf{Design Guideline} & \textbf{Description} & \textbf{Examples and Demos} \\
% \toprule
% \endfirsthead

% \toprule
% \textbf{} & \textbf{Design Guideline} & \textbf{Description} & \textbf{Examples and Demos} \\
% \toprule
% \endhead

% % G1. Avatar Body Appearance
% %\multirow{5}{*}{\rotatebox[origin=c]{90}{\parbox{6cm}{\centering \textbf{G1. Avatar Body Appearance}}}} 

% \textbf{G1.1}
% & \textbf{Support disability representation in social VR avatars.}
% &   Avatar interfaces should offer all users the agency to flexibly express their identities and present their disabilities [27, 37, 81]. The disability features should not be blocked behind the paywall [37].
% & \begin{minipage}[t]{\linewidth}
%     E.g., Avatar system provides hearing devices. \\
%     \includegraphics[width=3cm, height=2.7cm]{sections/images/guidelines_image/G1.1.jpg}
%   \end{minipage} 
% \\ \midrule

% \textbf{G1.2}
% & \textbf{Default to full-body avatar to enable diverse disability representation across different body parts. }
% & Avatar interfaces should offer full-body avatar options \cite{kelly2023}. Given the dominant preferences for full-body avatars over others, we recommend making it the default or the starting avatar template, giving users the maximum flexibility to further customize their avatars as they prefer.
% & \begin{minipage}[t]{\linewidth}
%     E.g., A full-body avatar can show a prosthetic left foreleg. \\
%     \includegraphics[width=3cm, height=2.8cm]{sections/images/guidelines_image/G1.2.png}
%   \end{minipage}
% \\ \midrule

% \textbf{G1.3}
% & \textbf{Enable flexible customization of body parts as opposed to using non-adjustable avatar templates.}
% & Avatar interfaces should provide PWD sufficient flexibility to customize each avatar body part \cite{kelly2023}. While the customization spans a wide range, the most commonly mentioned body parts to customize include (1) avatar height, (2) body shape, (3) limbs (i.e., number of limbs, length and strength of each limb), and (4) facial features (e.g., mouth shape, eye size). Asymmetrical design options of body parts should also be available.
% & \begin{minipage}[t]{\linewidth}
%     E.g., Options that can customize the size of each eye. \\
%     \includegraphics[width=3cm, height=2.7cm]{sections/images/guidelines_image/G1.3.png}
%   \end{minipage}
% \\ \midrule

% \textbf{G1.4}
% & \textbf{Prioritize human avatars to emphasize the ``humanity'' rather than the ``disability'' aspect of identity.}
% & Social VR applications should offer human avatar options whenever the application theme allows.
% & \begin{minipage}[t]{\linewidth}
%     E.g., Human avatars that can show intersectional identities. \\
%     \includegraphics[width=4cm, height=2cm]{sections/images/guidelines_image/G1.4.png}
%   \end{minipage}
% \\ \midrule

% \textbf{G1.5}
% & \textbf{Provide non-human avatar options to free users from social stigma in real life.}
% & Besides human avatars, avatar interfaces should also provide diverse forms of non-human avatars, empowering PWD to choose the one they relate with flexibly.
% & \begin{minipage}[t]{\linewidth}
%     E.g., A robotic avatar representing left forearm amputation.\\
%     \includegraphics[width=3cm, height=2.6cm]{sections/images/guidelines_image/G1.5.png}
%   \end{minipage}
% \\ \midrule

% % G2. Avatar Body Dynamics
% %\multirow{4}{*}{\rotatebox[origin=c]{90}{\parbox{10cm}{\centering \textbf{G2. Avatar Body Dynamics: facial expression, posture, and body motion.}}}}
% \textbf{G2.1}
% & \textbf{Allow simulation or tracking of disability-related behaviors but only based on user preference.}
% & Users should be able to control the extent of behavior tracking in social VR. With the advance of motion tracking techniques, avatar platforms may disable subtle behavior tracking by default to avoid disrespectful simulation, but allow users to easily adjust the tracking granularity for potential disability expression. 
% & \begin{minipage}[t]{\linewidth}
%     E.g., Avatar can show motor tics (left) or not (right) based on the user's preference.\\
%     \includegraphics[width=4cm, height=2cm]{sections/images/guidelines_image/G2.1.png}
%   \end{minipage}
% \\ \midrule

% \textbf{G2.2}
% & \textbf{Enable expressive facial animations to deliver invisible status}
% & Avatar platforms should enable diverse facial expressions, allowing PWD to express emotion, portray mental status, and indicate fluctuation of invisible disabilities \cite{assets_24}.
% & \begin{minipage}[t]{\linewidth}
%     E.g., Avatar shows diverse emotions, including anxiety (left), sadness (middle), and happiness (right).\\
%     \includegraphics[width=4cm, height=1.9cm]{sections/images/guidelines_image/G2.2.png}
%   \end{minipage}
% \\ \midrule

% \textbf{G2.3}
% & \textbf{Prioritize equitable capability and performance over authentic simulation.}
% & PWD value equitable and fair interaction experiences more than the authentic disability expression. Therefore, social VR platforms should ensure same level of capabilities and performance for all avatars no matter whether disability features or behaviors are involved.
% & \begin{minipage}[t]{\linewidth}
%     E.g., The avatar with the wheelchair moves at the same speed as the avatar without the wheelchair.\\
%     \includegraphics[width=3cm, height=2.2cm]{sections/images/guidelines_image/G2.3.png}
%   \end{minipage}
% \\ \midrule

% \textbf{G2.4}
% & \textbf{Leverage avatar posture to demonstrate PWD's lived experiences.}
% & Disabilities can be expressed via avatar posture. Avatar platforms should enable certain posture tracking or simulation (e.g., unique facing directions of individuals with low vision during conversation) to enable authentic disability representation. 
% & \begin{minipage}[t]{\linewidth}
%     E.g., The avatar representing a low vision person (left) shows a different posture than the avatar representing a sighted person (right) ®in a conversation. \\
%     \includegraphics[width=4cm, height=1.6cm]{sections/images/guidelines_image/G2.4.png}
%   \end{minipage}
% \\ \midrule

% % G3. Assistive Technology Design
% %\multirow{6}{*}{\rotatebox[origin=c]{90}{\parbox{6cm}{\centering \textbf{G3. Assistive Technology Design}}}}
% \textbf{G3.1}
% & \textbf{Offer various types of assistive technology to cover a wide range of disabilities.}
% &  Avatar interfaces should offer assistive technologies that are commonly used by PWD \cite{zhang2022}. The most desired types of assistive technologies include: (1) mobility aids (e.g., wheelchair, cane, and crutches); (2) prosthetic limbs; (3) visual aids (e.g., white cane, glasses, and guide dog); (4) hearing aids and cochlear implants; and (5) health monitoring and administration devices (e.g., insulin pumps, ventilator). Practitioners should consider including at least these five categories of assistive technologies in avatar interfaces.
% & \begin{minipage}[t]{\linewidth}
%     E.g., Offer multiple types of mobility aids, such as crutches (left) and prosthetic limb (right). \\
%     \includegraphics[width=4cm, height=1.6cm]{sections/images/guidelines_image/G3.1.png}
%   \end{minipage}
% \\ \midrule

% \textbf{G3.2}
% & \textbf{Allow detail customization of assistive technology for personalized disability representation.}
% & Avatar platforms should allow customizations for assistive technology \cite{kelly2023,zhang2022}. Basic customization options should include adjusting the colors of different assistive tech components and adding decorations (e.g., stickers, logos) to the assistive tech. More customization could be added based on specific use cases.
% & \begin{minipage}[t]{\linewidth}
%     E.g., Users can adjust the color of the power wheelchair, like the cushions, wheels, and chassis cover.\\
%     \includegraphics[width=4cm, height=2cm]{sections/images/guidelines_image/G3.2.png}
%   \end{minipage}
% \\ \midrule

% \textbf{G3.3}
% & \textbf{Provide high-quality, authentic simulation of assistive technology to present disability respectfully and avoid misuse.}
% & To avoid misunderstandings or misuse, the assistive tech simulation should convey standardized, authentic details of the real-world assistive devices \cite{zhang2023}, regardless the overall avatar style. We recommend practitioners to model assistive technologies by following their established design standards, such as design guidelines for white canes \cite{who_white_canes}, wheelchairs \cite{russotti_ansi_wheelchairs}, and hearing devices \cite{ecfr_800_30}.
% & \begin{minipage}[t]{\linewidth}
%     E.g., The low-fidelity wheelchair (left) might be misperceived as mocking. Instead, the wheelchair should be designed with high-fidelity and realistic details (right) to represent disability respectfully.\\
%     \includegraphics[width=4cm, height=2cm]{sections/images/guidelines_image/G3.3.png}
%   \end{minipage}
% \\ \midrule

% \textbf{G3.4}
% & \textbf{Focus on simulating assistive technologies that empower PWD rather than highlighting their challenges.}
% & When determining what assistive tech features to offer, practitioners should only select assistive or medical devices that can be easily controlled by PWD to demonstrate their capability (e.g., manual wheelchair, cane) and leave out the ones that PWD cannot independently use or the ones that highlight their challenges (e.g., hospital-style wheelchair, bedridden avatars). 
% & \begin{minipage}[t]{\linewidth}
%     E.g., The manual wheelchair has no handles, demonstrating that it’s designed for users who want to navigate independently instead of being pushed. \\
%     \includegraphics[width=3cm, height=2.5cm]{sections/images/guidelines_image/G3.4.png}
%   \end{minipage}
% \\ \midrule

% \textbf{G3.5}
% & \textbf{Demonstrate the liveliness of PWD through dynamic interactions with assistive technology.}
% & Beyond providing assistive tech options, social VR platforms should enable suitable interactions between avatar and the assistive tech. The interactions should authentically reflect PWD's real-world usage of their assistive tech, such as how a blind user sweeps their cane, or how a wheelchair user moves their arms to control their wheelchair.
% & \begin{minipage}[t]{\linewidth}
%     E.g., Avatar controls the manual wheelchair through pushing. \\
%     \includegraphics[width=3cm, height=2.8cm]{sections/images/guidelines_image/G3.5.png}
%   \end{minipage}
% \\ \midrule

% \textbf{G3.6}
% & \textbf{Avoid overshadowing the avatar body with assistive technology.}
% & The size of assistive technology should not dominate the avatar body but rather fit the body size. Avatar platforms should automatically match the assistive tech model to different avatar body sizes by default, and allow users to adjust the size of assistive technology to achieve their preferred avatar-aid ratio. Moreover, the combination of avatar and assistive tech should be seamless without affecting the quality and aesthetics of the original avatar \cite{kelly2023}.
% & \begin{minipage}[t]{\linewidth}
%     E.g., Users can change the size of assistive technology to match with their avatar. \\
%     \includegraphics[width=3cm, height=3cm]{sections/images/guidelines_image/G3.6.png}
%   \end{minipage}
% \\ \midrule

% % G4. Peripherals around Avatars
% %\multirow{2}{*}{\rotatebox[origin=c]{90}{\parbox{6cm}{\centering \textbf{G4. Peripherals around Avatars}}}}
% \textbf{G4.1}
% & \textbf{Provide suitable icons, logos, and slogans that represent disability communities.}
% & Awareness-building items or presets (e.g., logos, slogans) should be provided, allowing users to attach them to various areas on or around the avatars, such as the apparel, accessories, and assistive tech \cite{assets_24, zhang2022}. We complied a list of widely recognized and preferred symbols that represent different disabilities for practitioners to refer to, including (1) the rainbow infinity symbol the represents the autism community \cite{assets_24, rainbow_infinity_symbol}, (2) the sunflower that represents hidden disabilities \cite{isit_assets24, hidden_disability_sunflower}, (3) the disability pride flag \cite{disability_pride_flag}, (4) the spoons, symbolizing spoon theory for people with chronic illness \cite{kelly2023, assets_24}, and (5) the zebra symbols for rare diseases \cite{assets_24, Gualano_2023}.
% & \begin{minipage}[t]{\linewidth}
%     E.g., An avatar wearing a T-shirt with a rainbow and infinity symbol to represent the autism spectrum disorder community. \\
%     \includegraphics[width=3cm, height=2.8cm]{sections/images/guidelines_image/G4.1.png}
%   \end{minipage}
% \\ \midrule

% \textbf{G4.2}
% & \textbf{Leverage spaces beyond the avatar body to present disabilities.}
% & When designing avatars, developers and designers should consider leveraging avatar's peripheral space to enable users to better express their status, especially for individuals with invisible disabilities.
% & \begin{minipage}[t]{\linewidth}
%     E.g., An avatar with a floating bubble overhead, showing a level of social energy. \\
%     \includegraphics[width=4cm, height=2.4cm]{sections/images/guidelines_image/G4.2.png}
%   \end{minipage}
% \\ \midrule

% % G5. Design of avatar customization and control interface
% %\multirow{3}{*}{\rotatebox[origin=c]{90}{\parbox{8cm}{\centering \textbf{G5. Design of Avatar Customization and Control Interface}}}} 
% \textbf{G5.1}
% & \textbf{Distribute disability features across the entire avatar interface rather than gathering them in a specialized category.}
% & Avatar features for disability expression should be treated in the same way to other avatar features. In avatar interfaces, disability-related features should be properly distributed in their corresponding categories. There should not be a specialized category for PWD.
% & \begin{minipage}[t]{\linewidth}
%     E.g., Walking canes and wheelchairs are included under the accessory category along with items like glasses, hats and bags. \\
%     \includegraphics[width=3cm, height=3cm]{sections/images/guidelines_image/G5.1.png}
%   \end{minipage}
% \\ \midrule

% \textbf{G5.2}
% & \textbf{Use continuous controls for high-granularity customization.}
% & Avatar interfaces should adopt input controls that offer a continuous range of options to enable flexible customization. This could be widely applied to a variety of design attributes, such as the size and shape of multiple avatar body parts.
% & \begin{minipage}[t]{\linewidth}
%     E.g., Offer a slider to change the length of limbs. \\
%     \includegraphics[width=3cm, height=2.7cm]{sections/images/guidelines_image/G5.2.png}
%   \end{minipage}
% \\ \midrule

% \textbf{G5.3}
% & \textbf {Offer an easy control to turn on/off or switch between disability features.}
% & Social VR platforms should provide easy-to-access shortcut control enable users to conduct \textit{ad-hoc} avatar updates on the go. Important control functions in include: (1) toggling on and off the disability-related features \cite{assets_24}; (2) switching between different saved avatars \cite{kelly2023}; and (3) updating status for fluctuating conditions \cite{assets_24}.
% & \begin{minipage}[t]{\linewidth}
%     E.g., Users should be able to turn disability-related features on and off with a single click. \\
%     \includegraphics[width=4cm, height=1.5cm]{sections/images/guidelines_image/G5.3.png}
%   \end{minipage}
% \\  
% \bottomrule
% \caption{Our \change{\textit{original} 20 guidelines for inclusive avatars.}}
% \label{tab:full_original}
% \end{longtable}}
