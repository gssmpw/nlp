%\renewcommand{\arraystretch}{1.1} % Adjust vertical spacing

\afterpage{
\clearpage
\small
%\footnotesize

\begin{longtable}{p{0.4cm}p{4cm}p{6.5cm}p{5cm}}
\toprule
\textbf{} & \textbf{Design Guideline} & \textbf{Description} & \textbf{Examples and Demos} \\
\toprule
\endfirsthead

\toprule
\textbf{} & \textbf{Design Guideline} & \textbf{Description} & \textbf{Examples and Demos} \\
\toprule
\endhead

% G1. Avatar Body Appearance
%\multirow{5}{*}{\rotatebox[origin=c]{90}{\parbox{6cm}{\centering \textbf{G1. Avatar Body Appearance}}}} 

\multicolumn{4}{p{17cm}}{\change{\textbf{G0. Support disability representation in social VR avatars \cite{zhang2022, assets_24, chronic_pain_gualano_2024, kelly2023}.} Approximately 1.3 billion people experience significant disability, representing about 16\% of the global population \cite{WHO2023}. It's important to ensure PWD are included and represented equally in emerging technology such as social VR. As long as the platform involves avatar-based interactions, there is design space to support disability representation. The following set of guidelines can be flexibly applied to a variety of social VR platforms with different (1) avatar types (e.g., humanoid avatars in Rec Room \cite{recroom} vs. robotic-type avatars in Among Us \cite{amonguscharacters}), (2) aesthetic styles (e.g., life-like avatars in Horizon Worlds \cite{metaavatars} vs. abstract avatars in Roblox \cite{robloxwiki})), and (3) content focus (e.g., communication-heavy type in VRChat \cite{vrchat} vs. game-centric in Rec Room). We encourage practitioners to adopt G0 as a fundamental mindset when developing and designing avatars, considering it in the early stages, and consistently exploring opportunities to support disability representation.}}
\\ \midrule

% G1. Avatar Body Appearance
\textbf{} & 
& \multicolumn{1}{c}{\textbf{G1. Avatar Body Appearance}} & 
\\ \midrule
\textbf{G1.1 \change{(HR)}}
& \textbf{Default to full-body avatars to enable diverse disability
representation across different body parts.}
& Avatar interfaces should offer full-body avatar options \cite{kelly2023}. \change{About 296,000 people in the U.S. live with paralysis of the lower half of the body, with around 17,900 new cases each year \cite{NSCISC2021}.} Given the \change{large affected user size and} dominant preferences for full-body avatars over others (e.g., upper-body only, or head and hands only), we recommend making it the default or the starting avatar template, giving users the maximum flexibility to further customize their avatars as they prefer. 
& \begin{minipage}[t]{\linewidth}
    E.g., A full-body avatar can show a prosthetic left foreleg. \\
    \includegraphics[width=3cm, height=2.8cm]{sections/images/guidelines_image/G1.2.png}
  \end{minipage}
\\ \midrule

\textbf{G1.2 \change{(HR)}}
& \textbf{Enable flexible customization of body parts as opposed to using non-adjustable avatar templates.}
& Avatar interfaces should provide PWD sufficient flexibility to customize each avatar body part \cite{kelly2023}. \change{While the customization spans a wide range, the most commonly mentioned body parts to customize include (1) avatar height, (2) body shape, (3) limbs (i.e., number of limbs, length and strength of each limb), and (4) facial features (e.g., mouth shape, eye size).} Asymmetrical design options of body parts (e.g., eyes, ears) should also be available, such as changing size and direction of each eyeball to reflect disabilities like strabismus.
& \begin{minipage}[t]{\linewidth}
    E.g., Options that can customize the size of each eye. \\
    \includegraphics[width=3cm, height=2.7cm]{sections/images/guidelines_image/G1.3.png}
  \end{minipage}
\\ \midrule

\textbf{G1.3 \change{(HR)}}
& \textbf{Prioritize human avatars to emphasize the ``humanity'' rather than the ``disability'' aspect of identity.}
& Social VR applications should offer human avatar options whenever the application theme allows.
& \begin{minipage}[t]{\linewidth}
    E.g., Human avatars that can show intersectional identities. \\
    \includegraphics[width=4cm, height=2cm]{sections/images/guidelines_image/G1.4.png}
  \end{minipage}
\\ \midrule

\textbf{G1.4 \change{(R)}}
& \textbf{Provide non-human avatar options to free users from social stigma in real life.}
& Besides human avatars, avatar interfaces should also provide diverse forms of non-human avatars, empowering PWD to choose the one they relate with flexibly.
& \begin{minipage}[t]{\linewidth}
    E.g., A robotic avatar representing left forearm amputation.\\
    \includegraphics[width=3cm, height=2.6cm]{sections/images/guidelines_image/G1.5.png}
  \end{minipage}
\\ \midrule

% G2. Avatar Body Dynamics
%\multirow{4}{*}{\rotatebox[origin=c]{90}{\parbox{10cm}{\centering \textbf{G2. Avatar Body Dynamics: facial expression, posture, and body motion.}}}}
\textbf{} & 
& \multicolumn{1}{c}{\textbf{G2. Avatar Body Dynamics}} & 
\\ \midrule

\textbf{G2.1 \change{(HR)}}
& \textbf{Allow simulation or tracking of disability-related behaviors but only based on user preference.}
& Users should be able to control the extent of behavior tracking in social VR. With the advance of motion tracking techniques, avatar platforms may disable subtle behavior tracking by default to avoid disrespectful simulation, but allow users to easily adjust the tracking granularity for potential disability expression.
& \begin{minipage}[t]{\linewidth}
    E.g., Avatar can show motor tics (left) or not (right) based on the user's preference.\\
    \includegraphics[width=4cm, height=2cm]{sections/images/guidelines_image/G2.1.png}
  \end{minipage}
\\ \midrule

\textbf{G2.2 \change{(R)}}
& \textbf{Enable expressive facial animations to deliver invisible status.}
& Avatar platforms should enable diverse facial expressions, allowing PWD to express emotion, portray mental status, and indicate fluctuation of invisible disabilities \cite{assets_24}. \change{Including the five basic emotions is a good starting point, and practitioners should expand and diversify based on their own use scenarios. This guideline is particularly helpful for communication-heavy or mental therapy platforms, where expressing a range of emotions helps PWD convey their feelings more accurately.}
& \begin{minipage}[t]{\linewidth}
    E.g., Avatar shows diverse emotions, including anxiety (left), sadness (middle), and happiness (right).\\
    \includegraphics[width=4cm, height=1.9cm]{sections/images/guidelines_image/G2.2.png}
  \end{minipage}
\\ \midrule

\textbf{G2.3 \change{(HR)}}
& \textbf{Prioritize equitable capability and performance over authentic simulation.}
& PWD value equitable and fair interaction experiences more than the authentic disability expression. Therefore, social VR platforms should ensure same level of capabilities and performance for all avatars no matter whether disability features or behaviors are involved. 
& \begin{minipage}[t]{\linewidth}
    E.g., The avatar with the wheelchair moves at the same speed as the avatar without the wheelchair.\\
    \includegraphics[width=3cm, height=2.2cm]{sections/images/guidelines_image/G2.3.png}
  \end{minipage}
\\ \midrule

% G3. Assistive Technology Design
%\multirow{6}{*}{\rotatebox[origin=c]{90}{\parbox{6cm}{\centering \textbf{G3. Assistive Technology Design}}}}
\textbf{} & 
& \multicolumn{1}{c}{\textbf{G3. Assistive Technology Design}} & 
\\ \midrule

\textbf{G3.1 \change{(HR)}}
& \textbf{Offer various types of assistive technology to cover a wide range of disabilities.}
& Avatar interfaces should offer assistive technologies that are commonly used by PWD \cite{zhang2022}. \change{Approximately 2.5 million people around the world are assistive technologies users \cite{LevelAccess2024}. According to our large-scale interview data, the most desired types of assistive technologies include: (1) mobility aids (e.g., wheelchair, cane, and crutches); (2) prosthetic limbs; (3) visual aids (e.g., white cane, glasses, and guide dog); (4) hearing aids and cochlear implants; and (5) health monitoring devices (e.g., insulin pumps, ventilator, smart watches). % \yuhang{narrow down}. 
Practitioners should consider including at least these five categories of assistive technologies in avatar interfaces.
In addition, due to PWD's different technology preferences \cite{kelly_AI24}, we encourage practitioners to offer more than one assistive technology option within each category, for example, including guide dog, white cane, and glasses for visual aids.} 
& \begin{minipage}[t]{\linewidth}
    E.g., Offer multiple types of mobility aids, such as walking cane, crutches, and forearm crutches. \\
    \includegraphics[width=3cm, height=2.2cm]{sections/images/guidelines_image/G3.1.png}
  \end{minipage}
\\ \midrule

\textbf{G3.2 \change{(HR)}}
& \textbf{Allow detail customization of assistive technology for personalized disability representation.}
& Avatar platforms should allow customizations for assistive technology \change{\cite{kelly2023,zhang2022}. Basic customization options should include adjusting the colors of different assistive tech components and adding decorations (e.g., stickers, logos) to the assistive technologies. More customization could be added based on specific use cases.} 
& \begin{minipage}[t]{\linewidth}
    E.g., Users can adjust the color of the power wheelchair, like the cushions, wheels, and chassis cover.\\
    \includegraphics[width=4cm, height=2cm]{sections/images/guidelines_image/G3.2.png}
  \end{minipage}
\\ \midrule

\textbf{G3.3 (HR)}
& \textbf{Provide high-quality, authentic simulation of assistive technology to present disability respectfully and avoid misuse.}
& To avoid misunderstandings or misuse, the assistive tech simulation should convey standardized, \change{authentic details of the real-world assistive devices \cite{zhang2023}, regardless the overall avatar style. For example, the design of a white cane should show the details of tip and follow its standardized color selection, no matter the design style is photorealistic or cartoon. We recommend practitioners to model assistive technologies by following their established design standards, such as design guidelines for white canes \cite{who_white_canes}, wheelchairs \cite{russotti_ansi_wheelchairs}, and hearing devices \cite{ecfr_800_30}.}
& \begin{minipage}[t]{\linewidth}
    E.g., Thh wheelchair should be designed with high-fidelity and realistic details to represent disability respectfully.\\
    \includegraphics[width=3cm, height=2.4cm]{sections/images/guidelines_image/G3.3.png}
  \end{minipage}
\\ \midrule

\textbf{G3.4 \change{(R)}}
& \textbf{Demonstrate the liveliness of PWD through dynamic interactions with assistive technology.}
& Beyond providing assistive tech options, social VR platforms should enable suitable interactions between avatars and assistive tech. The interactions should authentically reflect PWD's real-world usage of their assistive tech, such as how a blind user sweeps their cane, or how a wheelchair user moves their arms to control their wheelchair.
& \begin{minipage}[t]{\linewidth}
    E.g., Avatar controls the manual wheelchair through pushing. \\
    \includegraphics[width=3cm, height=2.8cm]{sections/images/guidelines_image/G3.5.png}
  \end{minipage}
\\ \midrule

\textbf{G3.5 \change{(HR)}}
& \textbf{Avoid overshadowing the avatar body with assistive technology.}
& The size of assistive technology should not dominate the avatar body but rather fit the body size. \change{Avatar platforms should automatically match the assistive tech model to different avatar body sizes by default, and allow users to adjust the size of assistive technology to achieve their preferred avatar-aid ratio. Moreover, the combination of avatar and assistive tech should be seamless without affecting the quality and aesthetics of the original avatar \cite{kelly2023}.}
& \begin{minipage}[t]{\linewidth}
    E.g., Users can change the size of assistive technology to match with their avatar. \\
    \includegraphics[width=3cm, height=3cm]{sections/images/guidelines_image/G3.6.png}
  \end{minipage}
\\ \midrule

% G4. Peripherals around Avatars
%\multirow{2}{*}{\rotatebox[origin=c]{90}{\parbox{6cm}{\centering \textbf{G4. Peripherals around Avatars}}}}
\textbf{} & 
& \multicolumn{1}{c}{\textbf{G4. Peripherals around Avatars}} & 
\\ \midrule

\textbf{G4.1 \change{(HR)}}
& \textbf{Provide suitable icons, logos, and slogans that represent disability communities.}
& Awareness-building items or presets (e.g., logos, slogans) should be provided, allowing users to attach them to various areas on or around the avatars, such as the apparel, accessories, and assistive tech \cite{assets_24, zhang2022}. \change{We compiled a list of widely recognized and preferred symbols that represent different disabilities for practitioners to refer to, including (1) the rainbow infinity symbol the represents the autism community \cite{assets_24, rainbow_infinity_symbol}, (2) the sunflower that represents hidden disabilities \cite{isit_assets24, hidden_disability_sunflower}, (3) the disability pride flag \cite{disability_pride_flag}, (4) the spoons, symbolizing spoon theory for people with chronic illness \cite{kelly2023, assets_24}, and (5) the zebra symbols for rare diseases \cite{assets_24, Gualano_2023}.} 
& \begin{minipage}[t]{\linewidth}
    E.g., An avatar wearing a T-shirt with a rainbow and infinity symbol to represent the autism spectrum disorder community. \\
    \includegraphics[width=3cm, height=2.8cm]{sections/images/guidelines_image/G4.1.png}
  \end{minipage}
\\ \midrule

\textbf{G4.2 \change{(R)}}
& \textbf{Leverage spaces beyond the avatar body to present disabilities.}
& When designing avatars, developers and designers should consider leveraging avatar's peripheral space to enable users to better express their status, especially for individuals with invisible disabilities. \change{Some design examples include a weather background to indicate mood and a battery sign to indicate energy level \cite{assets_24}.} 
& \begin{minipage}[t]{\linewidth}
    E.g., An avatar with a floating bubble overhead, showing a level of social energy. \\
    \includegraphics[width=4cm, height=2.4cm]{sections/images/guidelines_image/G4.2.png}
  \end{minipage}
\\ \midrule

% G5. Design of avatar customization and control interface
%\multirow{3}{*}{\rotatebox[origin=c]{90}{\parbox{8cm}{\centering \textbf{G5. Design of Avatar Customization and Control Interface}}}} 
\textbf{} & 
& \multicolumn{1}{c}{\textbf{G5. Interface and Control}} & 
\\ \midrule

\textbf{G5.1 \change{(HR)}}
& \textbf{Distribute disability features across the entire avatar interface rather than gathering them in a specialized category.}
& Avatar features for disability expression should be treated in the same way as other avatar features. In avatar interfaces, disability-related features should be properly distributed in their corresponding categories. \change{There should not be a specialized category for PWD. For example, assistive technologies should be included in the accessory category rather than an assistive tech category.} 
& \begin{minipage}[t]{\linewidth}
    E.g., Walking canes and wheelchairs are included under the accessory category along with items like glasses, hats and bags. \\
    \includegraphics[width=3cm, height=3cm]{sections/images/guidelines_image/G5.1.png}
  \end{minipage}
\\ \midrule

\textbf{G5.2 \change{(HR)}}
& \textbf{Use continuous controls for high-granularity customization.}
& Avatar interfaces should adopt input controls that offer a continuous range of options to enable flexible customization. \change{This could be widely applied to a variety of design attributes, such as the size and shape of multiple avatar body parts.}
& \begin{minipage}[t]{\linewidth}
    E.g., Offer a slider to change the length of limbs. \\
    \includegraphics[width=3cm, height=2.7cm]{sections/images/guidelines_image/G5.2.png}
  \end{minipage}
\\ \midrule

\textbf{G5.3 \change{(HR)}}
& \textbf{Offer an easy control to turn on/off or switch between disability features.}
& Social VR platforms should provide easy-to-access shortcut control enable users to conduct \change{\textit{ad-hoc} avatar updates on the go. Important control functions include: (1) toggling on and off the disability-related features \cite{assets_24}; (2) switching between different saved avatars \cite{kelly2023}; and (3) updating status for fluctuating conditions \cite{assets_24}.} 
& \begin{minipage}[t]{\linewidth}
    E.g., Users should be able to turn disability-related features on and off with a single click. \\
    \includegraphics[width=4cm, height=1.5cm]{sections/images/guidelines_image/G5.3.png}
  \end{minipage}
\\  
\bottomrule
\caption{\change{A full version of our \textit{revised} guidelines that incorporated VR experts' feedback. We present a set of 17 inclusive avatar guidelines for PWD that covered five avatar design aspects. Each guideline includes: the guideline statement with a recommendation level (Highly Recommended (HR) or Recommended (R)), a detailed description of guideline, and an actionable implementation example.}}
\Description{This table, titled "Table 7," provides a detailed overview of 17 revised guidelines for inclusive avatar design. It includes three columns: "Design Guideline," "Description," and "Examples and Demos," organized into five specific guideline sections. Each guideline has a recommendation level denoted as ``HR'' for Highly Recommended or ``R'' for Recommended.}
\label{tab:full_revised}
\end{longtable}
}