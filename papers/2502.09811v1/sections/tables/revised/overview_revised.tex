%TC:ignore
\begin{table*}
\centering
%\renewcommand{\arraystretch}{1.2} % Adjust vertical spacing
%\footnotesize 
\small
\begin{tabular}{|p{0.2cm}|p{0.5cm}|p{0.3cm}|p{6.3cm}|p{8.5cm}|}
\hline
\multicolumn{3}{|p{1.0cm}|}{} & \textbf{Design Guidelines} & \textbf{Description} \\
\hline

\multicolumn{5}{|>{\raggedright\arraybackslash}p{17.3cm}|}{\vspace{-2mm}\change{G0. Support Disability Representation in Social VR Avatars.
Approximately 1.3 billion people experience significant disability, representing about 16\% of the global population \cite{WHO2023}. It's important to ensure PWD are included and represented equally in emerging technology such as social VR. As long as the platform involves avatar-based interactions, there is design space to support disability representation.
}} \\ \hline


\multirow{4}{*}{\rotatebox[origin=c]{90}{\hspace{1em} \textbf{G1. Avatar Body Appearance} \hspace{1em}}} 

% & G1.1
% & HR
% & Support disability representation in social VR avatars.
% & Avatar system provides hearing devices for disability representation. \\ \cline{2-5}

& G1.1
& \raisebox{0.2ex}{\change{HR}} 
& Default to full-body avatars to enable diverse disability representation across different body parts. 
&  The avatar interfaces should at least offer a full-body avatar option \cite{kelly2023}. We recommend making it the default or the starting avatar template, giving users the maximum flexibility to further customize their avatars as they prefer. 
\\ \cline{2-5}

& G1.2
& \raisebox{0.2ex}{\change{HR}}
& Enable flexible customization of body parts as opposed to using non-adjustable avatar templates. 
& Avatar interfaces should provide PWD sufficient flexibility to customize each avatar body part \cite{kelly2023}. \change{While the customization spans a wide range, the most commonly mentioned body parts to customize include (1) avatar height, (2) body shape, (3) limbs, and (4) facial features. %Asymmetrical design options of body parts should also be available.
} 
\\ \cline{2-5}

& G1.3
& \raisebox{0.2ex}{\change{HR}}
& Prioritize human avatars to emphasize the ``humanity'' rather than the ``disability'' aspect of identity. 
& Social VR applications \change{should offer human avatar options whenever the application theme allows.} 
\\ \cline{2-5}

& G1.4
& \raisebox{0.2ex}{\change{R}}
& Provide non-human avatar options to free users from social stigma in real life. 
& Besides human avatars, avatar interfaces should also provide diverse forms of non-human avatars, empowering PWD to choose the one they relate with flexibly. 
\\
\hline

\multirow{3}{*}{\rotatebox[origin=c]{90}{\hspace{0.1em} \textbf{G2. Dynamics} \hspace{1em}}} 
& G2.1
& \raisebox{0.2ex}{\change{HR}}
& Allow simulation or tracking of disability-related behaviors but only based on user preference.
& Users should be able to control the extent of behavior tracking in social VR. %With the advance of motion tracking techniques, avatar platforms may disable subtle behavior tracking by default to avoid disrespectful simulation, but allow users to easily adjust the tracking granularity for potential disability expression. 
\\ \cline{2-5}

& G2.2
& \raisebox{0.2ex}{\change{R}}
& Enable expressive facial animations to deliver invisible status.
& Avatar platforms should enable diverse facial expressions, allowing PWD to express emotion, portray mental status, and indicate fluctuation of invisible disabilities \change{\cite{assets_24}}.%\change{Including the five basic emotions is a good starting point, and practitioners should expand and diversify based on their own use scenarios. This guideline is particularly helpful for communication-heavy or mental therapy platforms, where expressing a range of emotions helps PWD convey their feelings more accurately.}
\\ \cline{2-5}

& G2.3
& \raisebox{0.2ex}{\change{HR}}
& Prioritize equitable capability and performance over authentic simulation. 
& %PWD value equitable and fair interaction experiences more than the authentic disability expression. Therefore, 
Social VR platforms should ensure same level of capabilities and performance for all avatars no matter whether disability features or behaviors are involved.
\\ 
\hline

\multirow{5}{*}{\rotatebox[origin=c]{90}{\hspace{3em} \textbf{G3. Assistive Technology Design} \hspace{3em}}} 
& G3.1
& \raisebox{0.2ex}{\change{HR}}
& Offer various types of assistive technology to cover a wide range of disabilities.
& Avatar interfaces should offer assistive technologies that are commonly used by PWD \cite{zhang2022}. \change{The most desired types of assistive technologies should be included: (1) mobility aids; (2) prosthetic limbs; (3) visual aids; (4) hearing aids and cochlear implants; and (5) health monitoring devices.}
\\ \cline{2-5}

& G3.2
& \raisebox{0.2ex}{\change{HR}}
& Allow detail customization of assistive technology for personalized disability representation.
& Avatar platforms should allow customizations for assistive technology \change{\cite{kelly2023,zhang2022}. Basic customization options should include adjusting the colors of different assistive tech components and adding decorations to the assistive technologies} 
\\ \cline{2-5}

& G3.3
& \raisebox{0.2ex}{\change{HR}}
& Provide high-quality, authentic simulation of assistive technology to present disability respectfully and avoid misuse.
& To avoid misunderstandings or misuse, the assistive tech simulation should convey standardized, authentic details of the real-world assistive devices \change{\cite{zhang2023}, regardless the overall avatar style.} 
\\ \cline{2-5}

& G3.4
& \raisebox{0.2ex}{\change{R}}
& Demonstrate the liveliness of PWD through dynamic interactions with assistive technology.
& Beyond providing assistive tech options, social VR platforms should enable suitable interactions between avatar and the assistive technologies. The interactions should authentically reflect PWD's real-world usage of their assistive technologies.
\\ \cline{2-5}

& G3.5
& \raisebox{0.2ex}{\change{HR}}
& Avoid overshadowing the avatar body with assistive technology. 
& The size of assistive technology should not dominate the avatar body but rather fit the body size. 
\\
\hline

\multirow{5}{*}{\rotatebox[origin=c]{90}{\hspace{1em} \textbf{G4. Peripherals} \hspace{1em}}} 
& G4.1
& \raisebox{0.2ex}{\change{HR}}
& Provide suitable icons, logos, and slogans that represent disability communities. 
& Awareness-building items or presets should be provided, allowing users to attach them to various areas on or around the avatars. \change{A list of widely recognized and preferred symbols are: (1) the rainbow infinity symbol the represents the autism community \cite{assets_24, rainbow_infinity_symbol}, (2) the sunflower tha   t represents hidden disabilities \cite{isit_assets24, hidden_disability_sunflower}, (3) the disability pride flag \cite{disability_pride_flag}, (4) the spoons, symbolizing spoon theory for people with chronic illness \cite{kelly2023, assets_24}, and (5) the zebra symbols for rare diseases \cite{assets_24, Gualano_2023}.}
\\ \cline{2-5}

& G4.2
& \raisebox{0.2ex}{\change{R}}
& Leverage spaces beyond the avatar body to present disabilities.
& Developers and designers should consider leveraging avatar's peripheral space to enable users to better express their status, \change{especially for individuals with invisible disabilities.}
\\
\hline

\multirow{5}{*}{\rotatebox[origin=c]{90}{\hspace{1em} \textbf{G5. Interface} \hspace{1em}}} 
& G5.1
& \raisebox{0.2ex}{\change{HR}}
& Distribute disability features across the entire avatar interface rather than gathering them in a specialized category.
& Avatar features for disability expression should be treated in the same way to other avatar features. 
\\ \cline{2-5}

& G5.2
& \raisebox{0.2ex}{\change{HR}}
& Use continuous controls for high-granularity customization. 
&  Avatar interfaces should adopt input controls that offer a continuous range of options to enable flexible customization. 
\\ \cline{2-5}

& G5.3
& \raisebox{0.2ex}{\change{HR}}
& Offer an easy control to turn on/off or switch between disability features.
& Social VR platforms should provide easy-to-access shortcut control enable users to \change{conduct \textit{ad-hoc} avatar updates on the go.}
\\
\hline

\end{tabular}
\caption{\change{An overview of our \textit{revised} guidelines that incorporated VR practitioners' feedback. We present a set of 17 inclusive avatar guidelines for PWD that covered five avatar design aspects. Each guideline includes three elements: a recommendation level (Highly Recommended (HR) or Recommended (R)), guideline statement, and a detailed description. %\yuhang{add G0}
}}
\Description{The table is titled "An overview of our revised guidelines that incorporated VR practitioners' feedback." It presents a set of 17 inclusive avatar guidelines for PWD covering five avatar design aspects. Each guideline includes a recommendation level (Highly Recommended - HR, or Recommended - R), a guideline statement, and a detailed description.}
\label{tab:overview_revised}
\end{table*}
%TC:endignore


%     \item Highly Recommended (HR): Highly recommended guidelines are easy to implement, apply to almost all use cases, and are considered as the bare minimum for avatar design and development. 
%     \item Recommend (R): Recommended guidelines may require planning and effort to implement. They are more tailored to specific user groups or use case scenarios, but they lead to good user experiences and help expand user groups.