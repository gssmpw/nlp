\documentclass{article}

% Recommended, but optional, packages for figures and better typesetting:
\usepackage{microtype}
\usepackage{graphicx}
\usepackage{subfigure}
\usepackage{booktabs} % for professional tables

% hyperref makes hyperlinks in the resulting PDF.
\usepackage{hyperref}

% Attempt to make hyperref and algorithmic work together better:
% \usepackage{algorithm}
% \usepackage{algpseudocode}
% \usepackage{algorithmic}
\newcommand{\theHalgorithm}{\arabic{algorithm}}

% For theorems and such
\usepackage{amsmath}
\usepackage{amssymb}
\usepackage{mathtools}
\usepackage{amsthm}
\usepackage[dvipsnames]{xcolor}
\usepackage{colortbl}
\usepackage{array}


% Clever referencing
\usepackage[capitalize,noabbrev]{cleveref}

% ICML Style
\usepackage[accepted]{icml2024}

% Theorem environments
\theoremstyle{plain}
\newtheorem{theorem}{Theorem}[section]
\newtheorem{proposition}[theorem]{Proposition}
\newtheorem{lemma}[theorem]{Lemma}
\newtheorem{corollary}[theorem]{Corollary}
\theoremstyle{definition}
\newtheorem{definition}[theorem]{Definition}
\theoremstyle{remark}
\newtheorem{remark}[theorem]{Remark}
\setlength{\fboxsep}{0pt} % Padding around text
\setlength{\fboxrule}{0pt} % Border thickness (set to 0 to remove border)
\newcommand{\cbt}[2][]{%
  \colorbox[RGB]{255,#2,#2}{\texttt{#1}}%
}
\newcommand{\strongreject}{\textsc{StrongReject} }

% Add near the beginning of the document:
\usepackage{natbib}

% Add near the beginning with other packages:
\usepackage{multirow}
\usepackage{makecell}  % Also needed for \makecell commands
\usepackage{wrapfig}

% Add for appendix
\usepackage[most]{tcolorbox}
\usepackage{caption}    % \captionof
\usepackage{tabularx}   % table with auto width
\usepackage{multirow}   % multi-row header
\usepackage{booktabs}   % professional table lines

\icmltitlerunning{The Hidden Dimensions of LLM Alignment}

\begin{document}

\twocolumn[
\icmltitle{The Hidden Dimensions of LLM Alignment: \\ A Multi-Dimensional Safety Analysis}


% \icmlsetsymbol{equal}{*}

\begin{icmlauthorlist}
\icmlauthor{Wenbo Pan}{cityu}
\icmlauthor{Zhichao Liu}{hit}
\icmlauthor{Qiguang Chen}{scir}
\icmlauthor{Xiangyang Zhou}{microsoft}
\icmlauthor{Haining Yu}{hit}
\icmlauthor{Xiaohua Jia}{cityu}
\end{icmlauthorlist}

\icmlaffiliation{cityu}{Department of Computer Science, City University of Hong Kong, Hong Kong}
\icmlaffiliation{hit}{School of Cyberspace Science, Harbin Institute of Technology, China}
\icmlaffiliation{scir}{Research Center for Social Computing and Information Retrieval, Harbin Institute of Technology, China}
\icmlaffiliation{microsoft}{Microsoft}

\icmlcorrespondingauthor{Wenbo Pan}{wenbo.pan@my.cityu.edu.hk}

\icmlkeywords{Machine Learning, Causal Attribution, Jailbreaks, LLMs}

\vskip 0.3in
]

\printAffiliationsAndNotice{}

\begin{abstract}

Large Language Models' safety-aligned behaviors, such as refusing harmful queries, can be represented by linear directions in activation space. Previous research modeled safety behavior with a single direction, limiting mechanistic understanding to an isolated safety feature. In this work, we discover that safety-aligned behavior is jointly controlled by multi-dimensional directions. Namely, we study the vector space of representation shifts during safety fine-tuning on Llama 3 8B for refusing jailbreaks. By studying orthogonal directions in the space, we first find that a dominant direction governs the model's refusal behavior, while multiple smaller directions represent distinct and interpretable features like hypothetical narrative and role-playing. We then measure how different directions promote or suppress the dominant direction, showing the important role of secondary directions in shaping the model's refusal representation. Finally, we demonstrate that removing certain trigger tokens in harmful queries can mitigate these directions to bypass the learned safety capability, providing new insights on understanding safety alignment vulnerability from a multi-dimensional perspective. 
Code and artifacts are available at \url{https://github.com/BMPixel/safety-residual-space}.

\end{abstract}

\section{Introduction}
\label{sec:introduction}
The business processes of organizations are experiencing ever-increasing complexity due to the large amount of data, high number of users, and high-tech devices involved \cite{martin2021pmopportunitieschallenges, beerepoot2023biggestbpmproblems}. This complexity may cause business processes to deviate from normal control flow due to unforeseen and disruptive anomalies \cite{adams2023proceddsriftdetection}. These control-flow anomalies manifest as unknown, skipped, and wrongly-ordered activities in the traces of event logs monitored from the execution of business processes \cite{ko2023adsystematicreview}. For the sake of clarity, let us consider an illustrative example of such anomalies. Figure \ref{FP_ANOMALIES} shows a so-called event log footprint, which captures the control flow relations of four activities of a hypothetical event log. In particular, this footprint captures the control-flow relations between activities \texttt{a}, \texttt{b}, \texttt{c} and \texttt{d}. These are the causal ($\rightarrow$) relation, concurrent ($\parallel$) relation, and other ($\#$) relations such as exclusivity or non-local dependency \cite{aalst2022pmhandbook}. In addition, on the right are six traces, of which five exhibit skipped, wrongly-ordered and unknown control-flow anomalies. For example, $\langle$\texttt{a b d}$\rangle$ has a skipped activity, which is \texttt{c}. Because of this skipped activity, the control-flow relation \texttt{b}$\,\#\,$\texttt{d} is violated, since \texttt{d} directly follows \texttt{b} in the anomalous trace.
\begin{figure}[!t]
\centering
\includegraphics[width=0.9\columnwidth]{images/FP_ANOMALIES.png}
\caption{An example event log footprint with six traces, of which five exhibit control-flow anomalies.}
\label{FP_ANOMALIES}
\end{figure}

\subsection{Control-flow anomaly detection}
Control-flow anomaly detection techniques aim to characterize the normal control flow from event logs and verify whether these deviations occur in new event logs \cite{ko2023adsystematicreview}. To develop control-flow anomaly detection techniques, \revision{process mining} has seen widespread adoption owing to process discovery and \revision{conformance checking}. On the one hand, process discovery is a set of algorithms that encode control-flow relations as a set of model elements and constraints according to a given modeling formalism \cite{aalst2022pmhandbook}; hereafter, we refer to the Petri net, a widespread modeling formalism. On the other hand, \revision{conformance checking} is an explainable set of algorithms that allows linking any deviations with the reference Petri net and providing the fitness measure, namely a measure of how much the Petri net fits the new event log \cite{aalst2022pmhandbook}. Many control-flow anomaly detection techniques based on \revision{conformance checking} (hereafter, \revision{conformance checking}-based techniques) use the fitness measure to determine whether an event log is anomalous \cite{bezerra2009pmad, bezerra2013adlogspais, myers2018icsadpm, pecchia2020applicationfailuresanalysispm}. 

The scientific literature also includes many \revision{conformance checking}-independent techniques for control-flow anomaly detection that combine specific types of trace encodings with machine/deep learning \cite{ko2023adsystematicreview, tavares2023pmtraceencoding}. Whereas these techniques are very effective, their explainability is challenging due to both the type of trace encoding employed and the machine/deep learning model used \cite{rawal2022trustworthyaiadvances,li2023explainablead}. Hence, in the following, we focus on the shortcomings of \revision{conformance checking}-based techniques to investigate whether it is possible to support the development of competitive control-flow anomaly detection techniques while maintaining the explainable nature of \revision{conformance checking}.
\begin{figure}[!t]
\centering
\includegraphics[width=\columnwidth]{images/HIGH_LEVEL_VIEW.png}
\caption{A high-level view of the proposed framework for combining \revision{process mining}-based feature extraction with dimensionality reduction for control-flow anomaly detection.}
\label{HIGH_LEVEL_VIEW}
\end{figure}

\subsection{Shortcomings of \revision{conformance checking}-based techniques}
Unfortunately, the detection effectiveness of \revision{conformance checking}-based techniques is affected by noisy data and low-quality Petri nets, which may be due to human errors in the modeling process or representational bias of process discovery algorithms \cite{bezerra2013adlogspais, pecchia2020applicationfailuresanalysispm, aalst2016pm}. Specifically, on the one hand, noisy data may introduce infrequent and deceptive control-flow relations that may result in inconsistent fitness measures, whereas, on the other hand, checking event logs against a low-quality Petri net could lead to an unreliable distribution of fitness measures. Nonetheless, such Petri nets can still be used as references to obtain insightful information for \revision{process mining}-based feature extraction, supporting the development of competitive and explainable \revision{conformance checking}-based techniques for control-flow anomaly detection despite the problems above. For example, a few works outline that token-based \revision{conformance checking} can be used for \revision{process mining}-based feature extraction to build tabular data and develop effective \revision{conformance checking}-based techniques for control-flow anomaly detection \cite{singh2022lapmsh, debenedictis2023dtadiiot}. However, to the best of our knowledge, the scientific literature lacks a structured proposal for \revision{process mining}-based feature extraction using the state-of-the-art \revision{conformance checking} variant, namely alignment-based \revision{conformance checking}.

\subsection{Contributions}
We propose a novel \revision{process mining}-based feature extraction approach with alignment-based \revision{conformance checking}. This variant aligns the deviating control flow with a reference Petri net; the resulting alignment can be inspected to extract additional statistics such as the number of times a given activity caused mismatches \cite{aalst2022pmhandbook}. We integrate this approach into a flexible and explainable framework for developing techniques for control-flow anomaly detection. The framework combines \revision{process mining}-based feature extraction and dimensionality reduction to handle high-dimensional feature sets, achieve detection effectiveness, and support explainability. Notably, in addition to our proposed \revision{process mining}-based feature extraction approach, the framework allows employing other approaches, enabling a fair comparison of multiple \revision{conformance checking}-based and \revision{conformance checking}-independent techniques for control-flow anomaly detection. Figure \ref{HIGH_LEVEL_VIEW} shows a high-level view of the framework. Business processes are monitored, and event logs obtained from the database of information systems. Subsequently, \revision{process mining}-based feature extraction is applied to these event logs and tabular data input to dimensionality reduction to identify control-flow anomalies. We apply several \revision{conformance checking}-based and \revision{conformance checking}-independent framework techniques to publicly available datasets, simulated data of a case study from railways, and real-world data of a case study from healthcare. We show that the framework techniques implementing our approach outperform the baseline \revision{conformance checking}-based techniques while maintaining the explainable nature of \revision{conformance checking}.

In summary, the contributions of this paper are as follows.
\begin{itemize}
    \item{
        A novel \revision{process mining}-based feature extraction approach to support the development of competitive and explainable \revision{conformance checking}-based techniques for control-flow anomaly detection.
    }
    \item{
        A flexible and explainable framework for developing techniques for control-flow anomaly detection using \revision{process mining}-based feature extraction and dimensionality reduction.
    }
    \item{
        Application to synthetic and real-world datasets of several \revision{conformance checking}-based and \revision{conformance checking}-independent framework techniques, evaluating their detection effectiveness and explainability.
    }
\end{itemize}

The rest of the paper is organized as follows.
\begin{itemize}
    \item Section \ref{sec:related_work} reviews the existing techniques for control-flow anomaly detection, categorizing them into \revision{conformance checking}-based and \revision{conformance checking}-independent techniques.
    \item Section \ref{sec:abccfe} provides the preliminaries of \revision{process mining} to establish the notation used throughout the paper, and delves into the details of the proposed \revision{process mining}-based feature extraction approach with alignment-based \revision{conformance checking}.
    \item Section \ref{sec:framework} describes the framework for developing \revision{conformance checking}-based and \revision{conformance checking}-independent techniques for control-flow anomaly detection that combine \revision{process mining}-based feature extraction and dimensionality reduction.
    \item Section \ref{sec:evaluation} presents the experiments conducted with multiple framework and baseline techniques using data from publicly available datasets and case studies.
    \item Section \ref{sec:conclusions} draws the conclusions and presents future work.
\end{itemize}
% !TEX root =  ../main.tex
\section{Background on causality and abstraction}\label{sec:preliminaries}

This section provides the notation and key concepts related to causal modeling and abstraction theory.

\spara{Notation.} The set of integers from $1$ to $n$ is $[n]$.
The vectors of zeros and ones of size $n$ are $\zeros_n$ and $\ones_n$.
The identity matrix of size $n \times n$ is $\identity_n$. The Frobenius norm is $\frob{\mathbf{A}}$.
The set of positive definite matrices over $\reall^{n\times n}$ is $\pd^n$. The Hadamard product is $\odot$.
Function composition is $\circ$.
The domain of a function is $\dom{\cdot}$ and its kernel $\ker$.
Let $\mathcal{M}(\mathcal{X}^n)$ be the set of Borel measures over $\mathcal{X}^n \subseteq \reall^n$. Given a measure $\mu^n \in \mathcal{M}(\mathcal{X}^n)$ and a measurable map $\varphi^{\V}$, $\mathcal{X}^n \ni \mathbf{x} \overset{\varphi^{\V}}{\longmapsto} \V^\top \mathbf{x} \in \mathcal{X}^m$, we denote by $\varphi^{\V}_{\#}(\mu^n) \coloneqq \mu^n(\varphi^{\V^{-1}}(\mathbf{x}))$ the pushforward measure $\mu^m \in \mathcal{M}(\mathcal{X}^m)$. 


We now present the standard definition of SCM.

\begin{definition}[SCM, \citealp{pearl2009causality}]\label{def:SCM}
A (Markovian) structural causal model (SCM) $\scm^n$ is a tuple $\langle \myendogenous, \myexogenous, \myfunctional, \zeta^\myexogenous \rangle$, where \emph{(i)} $\myendogenous = \{X_1, \ldots, X_n\}$ is a set of $n$ endogenous random variables; \emph{(ii)} $\myexogenous =\{Z_1,\ldots,Z_n\}$ is a set of $n$ exogenous variables; \emph{(iii)} $\myfunctional$ is a set of $n$ functional assignments such that $X_i=f_i(\parents_i, Z_i)$, $\forall \; i \in [n]$, with $ \parents_i \subseteq \myendogenous \setminus \{ X_i\}$; \emph{(iv)} $\zeta^\myexogenous$ is a product probability measure over independent exogenous variables $\zeta^\myexogenous=\prod_{i \in [n]} \zeta^i$, where $\zeta^i=P(Z_i)$. 
\end{definition}
A Markovian SCM induces a directed acyclic graph (DAG) $\mathcal{G}_{\scm^n}$ where the nodes represent the variables $\myendogenous$ and the edges are determined by the structural functions $\myfunctional$; $ \parents_i$ constitutes then the parent set for $X_i$. Furthermore, we can recursively rewrite the set of structural function $\myfunctional$ as a set of mixing functions $\mymixing$ dependent only on the exogenous variables (cf. \cref{app:CA}). A key feature for studying causality is the possibility of defining interventions on the model:
\begin{definition}[Hard intervention, \citealp{pearl2009causality}]\label{def:intervention}
Given SCM $\scm^n = \langle \myendogenous, \myexogenous, \myfunctional, \zeta^\myexogenous \rangle$, a (hard) intervention $\iota = \operatorname{do}(\myendogenous^{\iota} = \mathbf{x}^{\iota})$, $\myendogenous^{\iota}\subseteq \myendogenous$,
is an operator that generates a new post-intervention SCM $\scm^n_\iota = \langle \myendogenous, \myexogenous, \myfunctional_\iota, \zeta^\myexogenous \rangle$ by replacing each function $f_i$ for $X_i\in\myendogenous^{\iota}$ with the constant $x_i^\iota\in \mathbf{x}^\iota$. 
Graphically, an intervention mutilates $\mathcal{G}_{\mathsf{M}^n}$ by removing all the incoming edges of the variables in $\myendogenous^{\iota}$.
\end{definition}

Given multiple SCMs describing the same system at different levels of granularity, CA provides the definition of an $\alpha$-abstraction map to relate these SCMs:
\begin{definition}[$\abst$-abstraction, \citealp{rischel2020category}]\label{def:abstraction}
Given low-level $\mathsf{M}^\ell$ and high-level $\mathsf{M}^h$ SCMs, an $\abst$-abstraction is a triple $\abst = \langle \Rset, \amap, \alphamap{} \rangle$, where \emph{(i)} $\Rset \subseteq \datalow$ is a subset of relevant variables in $\mathsf{M}^\ell$; \emph{(ii)} $\amap: \Rset \rightarrow \datahigh$ is a surjective function between the relevant variables of $\mathsf{M}^\ell$ and the endogenous variables of $\mathsf{M}^h$; \emph{(iii)} $\alphamap{}: \dom{\Rset} \rightarrow \dom{\datahigh}$ is a modular function $\alphamap{} = \bigotimes_{i\in[n]} \alphamap{X^h_i}$ made up by surjective functions $\alphamap{X^h_i}: \dom{\amap^{-1}(X^h_i)} \rightarrow \dom{X^h_i}$ from the outcome of low-level variables $\amap^{-1}(X^h_i) \in \datalow$ onto outcomes of the high-level variables $X^h_i \in \datahigh$.
\end{definition}
Notice that an $\abst$-abstraction simultaneously maps variables via the function $\amap$ and values through the function $\alphamap{}$. The definition itself does not place any constraint on these functions, although a common requirement in the literature is for the abstraction to satisfy \emph{interventional consistency} \cite{rubenstein2017causal,rischel2020category,beckers2019abstracting}. An important class of such well-behaved abstractions is \emph{constructive linear abstraction}, for which the following properties hold. By constructivity, \emph{(i)} $\abst$ is interventionally consistent; \emph{(ii)} all low-level variables are relevant $\Rset=\datalow$; \emph{(iii)} in addition to the map $\alphamap{}$ between endogenous variables, there exists a map ${\alphamap{}}_U$ between exogenous variables satisfying interventional consistency \cite{beckers2019abstracting,schooltink2024aligning}. By linearity, $\alphamap{} = \V^\top \in \reall^{h \times \ell}$ \cite{massidda2024learningcausalabstractionslinear}. \cref{app:CA} provides formal definitions for interventional consistency, linear and constructive abstraction.
\section{Safety Residual Space}
\label{sec:safety_residual}

We first define our framework of safety residual space. Our approach is motivated by recent research on training dynamics of alignment algorithms \cite{jain2024makes,lee2024mechanistic}, which shows that representation shifts during training are both meaningful and interpretable. We focus specifically on the effects of safety fine-tuning by comparing representation dynamics before and after the fine-tuning process, limiting our scope to a single forward pass. Let $\mathbf{x}$ denote vectors in the representation space $\mathcal{X} \subset \mathbb{R}^d$. Then $\mathcal{T}: \mathcal{X} \to \mathcal{X}$ describes the representation shift from unaligned (before fine-tuning) to aligned (after fine-tuning) states. We define the safety residual space as the linear span of representation shifts during safety fine-tuning. Formally:

\begin{definition}[Safety Residual Space]
    Consider $\mathcal{T}$ from training on unaligned samples whose representation is $\mathbf{x} \sim \mathbb{P}(\mathcal{X}_u)$.
    The \emph{safety residual space} $V_\mathcal{T}$ is defined as the optimal affine transformation parameterized as $V_\mathcal{T}(\mathbf{x}) = \mathbf{W} \mathbf{x} + \mathbf{b}$ that minimizes:
    \[
    V_\mathcal{T}
    \;=\;
    \underset{\mathbf{\hat{W}},\,\mathbf{\hat{b}}}{\mathrm{argmin}}
    \;\; 
    \mathbb{E}_{\mathbf{x} \sim \mathcal{X}_u}
    \Bigl\|\mathcal{T}(\mathbf{x}) \;-\; \bigl(\mathbf{\hat{W}}\mathbf{x} + \mathbf{\hat{b}}\bigr)\Bigr\|^2.
    \]
    \label{def:residual}
    \end{definition}
    

Intuitively, this definition captures the linear activation shifts that the model learns from the training. We consider the safety feature directions as linear and ignore non-linear error between $V_\mathcal{T}$ and $\mathcal{T}$. We use activations from transformer block outputs at the position of the first generated token from each layer. We compute activations from the training data as an approximation of the unaligned distribution $\mathcal{X}_u$.

\paragraph{Extracting Principal Components}
 To identify important directions in the residual space, we apply Singular Value Decomposition (SVD) to $\mathbf{W} - \mathbf{I}$ and take the first $k$ right vectors (components) $V^{:k}_\mathcal{T}$. This describes the span of the largest $k$ orthogonal representation shifts from the input space (i.e., the model before training). 

\paragraph{Notation}
We denote different components as \texttt{LN-CK}, where \texttt{LN} is the layer number and \texttt{K} is the \texttt{K}th largest right vector from SVD. Specifically, we refer to the \texttt{LN-C1} as the \emph{dominant component}, while others are \emph{non-dominant components}. We provide an illustration in \autoref{fig:illustration}. We use \textit{component} and \textit{direction} interchangeably in this paper.

\subsection{Component as Feature Direction}

A key question is whether the components in the residual space contain interpretable features, similar to probe vectors. We first establish the connection between feature directions and residual space. We show that residual direction contributes to the training goal under convergence assumption:

\begin{theorem}[Convergence Ensures Utility]
    \label{thm:convergence-utility}
    Suppose training converges to parameters $\mathbf{W}^*$ with
    \[
    \begin{aligned}
    J(\mathbf{W}^*) &\geq \sup\nolimits_{\mathbf{W}\in\mathcal{M}} J(\mathbf{W}) - \epsilon, \\
    J(\mathbf{W}) &:= \mathbb{E}_{(x,y)\sim\mathcal{D}}[\log P_W(y|x)]
    \end{aligned}
    \]
    For any residual direction $v \in V_{\mathcal{T}}$ from alignment shift $\mathcal{T}$, there exist contexts $w_0, w_1 := \mathcal{G}(C)$ (where $C=1$ iff $f_{\mathbf{W}^*}(X)=Y$) such that
    \[
    \exists \alpha>0,\quad \lambda(w_1) - \lambda(w_0) = \alpha v
    \]
    with the difference satisfying \autoref{eq:pop}.
    \end{theorem}
We provide a detailed derivation in the Appendix~\ref{appd:proof}. Intuitively, this theorem shows that when training maximizes the likelihood of expected outputs, all components in the residual space contribute positively to at least the overall training objective. Otherwise, these shift directions would not have been learned. While this does not guarantee human-interpretable features, it suggests a promising approach for automatically discovering feature directions without requiring probing data pairs. In \autoref{sec:interpretation}, we empirically demonstrate that some components have clear semantic roles in generating safety refusals.

\paragraph{Not All Features are in Residuals}

On the other hand, can all features be captured in the residual space? We posit that it primarily reflects features developed during safety training. Features might be absent for two reasons: (1) they are irrelevant to the training objective (e.g., unrelated syntactic patterns), as optimization naturally excludes non-contributing directions; or (2) they already existed in the pre-trained model (e.g., recognizing toxic content), requiring no parameter updates. This implies the residual space spans directions learned during safety fine-tuning.

\begin{corollary}
    The safety residual space is the span of feature directions developed during safety training.
\end{corollary}

\subsection{Experimental Setup}
Now, we describe the experiment setup, focusing on how models learn to recognize and handle unaligned harmful queries through safety fine-tuning.

\paragraph{Dataset} 
We construct a comprehensive preference dataset incorporating various challenging jailbreak methods and alignment blindspots from recent research \cite{ding2023wolf,yu2023gptfuzzer,zou2023universal,chao2023pair,liu2024flipattack}. To generate harmful examples, we apply these jailbreak methods to toxic samples from STRONG REJECT~\cite{souly2024strongreject}. We further incorporate 50\% samples from or-bench~\cite{cui2024or} as harmless samples to balance the dataset. Detailed dataset specifications are provided in the Appendix~\ref{appd:dataset_construction}.

\paragraph{Evaluation Metrics}
Following established practices in jailbreak research~\cite{zou2023universal,souly2024strongreject}, we evaluate model responses along two dimensions: refusal accuracy and response harmfulness. We measure refusal accuracy across both harmless and harmful test samples, while quantifying response harmfulness using STRONG REJECT scores~\cite{souly2024strongreject}.

\paragraph{Safety Fine-tuning}
We perform safety fine-tuning on Llama 3.1 8B Instruct using both SSFT and DPO approaches for one epoch. For SSFT, we follow \citet{inan2023llama} to optimize the model to generate refusal for harmful queries with instruction fine-tuning. For DPO, we additionally create a preference dataset with prefered helpful responses for harmless queries and refusals with disclaimers for harmful queries. We use Llama 3.1 405B Instruct~\cite{dubey2024llama} to generate reference responses for the preference dataset. The effectiveness of our fine-tuning process is demonstrated by two key metrics: the average \strongreject score~\cite{souly2024strongreject} across all jailbreak attempts decreased significantly from 0.65 to 0.05, while the refusal accuracy improved to 90\%. We provide more details and results on different models in the Appendix~\ref{appd:training_procedure}.
\begin{figure}
	\centering

	\subfloat[Grover]{\includegraphics[width=0.5\columnwidth]{plots/linearitygrover0.01.png}}
	\subfloat[VQE]{\includegraphics[width=0.5\columnwidth]{plots/linearityvqe0.01.png}}

	\subfloat[VPE]{\includegraphics[width=0.5\columnwidth]{plots/linearityshor0.01.png}}
	\subfloat[QFT]{\includegraphics[width=0.5\columnwidth]{plots/linearityqft0.01.png}}

	\caption{
	\textbf{X-axis: circuit size in thousands.}
	\textbf{Y-axis: time in seconds.}
	The figure plots the optimization times for four quantum algorithms
	against the corresponding circuit sizes.
	The figure confirms that our algorithm scales linearly with the size of the circuit.
	}
	\label{fig:linear}

\end{figure}


% \begin{figure}[t]
% 	\centering
% 	\includegraphics[width=0.9\columnwidth]{plots/linearity0.01.png}\quad
% 	\scriptsize
% \caption{
% 	The figure plots the optimizer times for five quantum algorithms
% 	as their size varies.
% 	%
% 	The X-axis shows the relative size of a circuit for each algorithm.
% 	%
% 	The Y-axis shows the time taken to optimize the circuit.
% 	%
% 	The figure confirms that our algorithm scales linearly with the size of the circuit.
% }
% \end{figure}
\section{Feature Directions in Safety Residual Space}
\label{sec:interpretation}


So far, we have focused on examining the dominant direction in the safety residual space, which predicts the model's aligned behavior. In this section, we will investigate how \textit{non-dominant} directions represent different features.

\paragraph{Problem} 
Unlike probe vectors, arbitrary directions lack pre-defined semantic meanings~\cite{bricken2023monosemanticity}, making it challenging to observe outcome changes through intervention experiments. While previous works~\cite{Ball2024UnderstandingJS,lee2024mechanistic} have used Logit Lens~\cite{nostalgebraist2020interpreting} to map representations to the projection layer in transformers, the faithfulness of this approach relies on vector similarity to the vocabulary space, which does not apply to residual directions.

\paragraph{Our Approach}

To determine features represented by directions, we introduce a theoretically grounded method within the LRP framework. We refer it as Partial Layer-wise Relevance Propagation (PLRP): given a set of directions $\{v_i\}$ and representations $X^l$, we first project $X^l$ onto the span of $\{v_i\}$. We then decompose its Euclidean norm into relevance scores $R$ and back-propagate the relevance scores. To ensure relevance conservation, we apply the epsilon rule~\cite{bach2015pixel} for handling projections. Formally we have:

\begin{align*}
    P_V(X^l
    ) = \sum_{v \in V} \|v^T X^l\|_2^2 \propto R_l
\end{align*}

The relevance score $R_l$ is then back-propagated to either (1) input tokens in training data or (2) projections on directions of activation in earlier layers. For input tokens $t$, we follow \citet{achtibat2024attnlrp} and sum up relevance scores of all elements in the token embedding, i.e., $R^{<t>} = \sum_{i=1}^{d} R_i^{<t>}$. To compute relevance scores of directions $v_i$ in $X^{l'}$ of earlier layers, we first compose an linear reconstruction term with first $k$ SVD components $V_{:k} \in \mathbb{R}^{d \times k}$: $\hat{X}^{l'} = V_{:k}W + \epsilon$, where $W \in \mathbb{R}^k$ minimizes the reconstruction error $\epsilon$. We then calculate the relevance scores $R^W_i$ on elements of $W$ and re-normalize to remove relevance scores absorbed by $\epsilon$. The relevance scores of $v_i$ is then given by average $R^W_i$ across all training samples.

\begin{figure}
    \vskip 0.2in
    \begin{center}
    \centerline{\includegraphics[width=\columnwidth]{figures/fig_supression.pdf}}
    \caption{Intervention results after removing the direction of 6th component of layer 14 (\texttt{L14-C6}) from the hidden states during generation. \texttt{L14-C6} is identified as representing the specific ability to recognize the PAIR Attack. Additionally, we remove the dominant direction (\texttt{L25-C1}), which completely eliminates the fine-tuned model's ability to refuse.}
    \label{fig:intervention}
    \end{center}
    \vskip -0.2in
\end{figure}

\subsection{Interpreting Directions via Token Relevance}
\label{iterpret_tokens}

We demonstrate that relevance scores of training input tokens help understand the semantic meaning of directions in the safety residual space. \autoref{tab:plrp_logitlens} visualizes the relevance distribution for several directions using a handcrafted example on layer 14. \footnote{Other layers around layer 14 also show similar patterns. We provide an analysis in \autoref{sec:early_phase}} We provide observations on the dominant and non-dominant directions in the following.

\paragraph{Dominant Direction} We evaluate dominant directions (i.e. \texttt{LN-C1}) and non-dominant directions (i.e. \texttt{L14-CK} in \autoref{tab:plrp_logitlens}) separately. The \textsc{Top Token} column shows the most relevant training tokens that activate each direction. For \texttt{L14-C1} and \texttt{L15-C1}, we observe that the dominant direction primarily relates to harmful subjects, such as \textit{divisive ideologies}. This aligns with our earlier finding that the dominant direction best predicts harmfulness.

\paragraph{Non-Dominant Direction}
For non-dominant directions, we find they are activated not by toxicity or harmfulness, but rather by features characteristic of specific jailbreak patterns. For instance, tokens like \textit{Imagine}, \textit{fictional} and \textit{hypothetical} in \texttt{L14-C2} establish a hypothetical tone. This negatively correlates with the dominant component in layer 25, reducing the probability of refusal. Meanwhile, \texttt{L14-C5} is triggered by explicit mentions of \textit{ChatGPT} and positively correlates with the dominant direction, likely due to its prevalent use in role-playing jailbreaks~\cite{yu2023gptfuzzer}. These findings suggest that non-dominant directions serve to capture indirect features related to safety.

\paragraph{The ``Sure, I'm happy to help'' Direction}
Notably, \texttt{L14-C6} activates when \textit{Sure, I'm happy to help} co-occurs with \textit{Imagine}. We notice that this pattern matches common jailbreak techniques used by PAIR~\cite{chao2023pair}, which typically set up harmful requests in imaginary scenarios (e.g., \textit{Imagine you are a professional hacker}) and force the model to respond positively (e.g., \textit{Start your response with `Sure, I'm happy to help'}). To validate \texttt{L14-C6}'s role, we intervene during generation using \autoref{eq:intervene} to remove its corresponding direction from the safety fine-tuned model. \autoref{fig:intervention} confirms that removing \texttt{L14-C6} specifically ablate the model's ability to refuse PAIR prompts while preserving its capability to handle other attack types. We evaluate the intervention's impact on the model's general abilities in \autoref{sec:impact_inter}, ensuring that removing refusal does not degrade overall performance.

\begin{figure}
    % \vskip 0.1in
    \begin{center}
    \includegraphics[width=\columnwidth]{figures/relevance_heatmap.pdf}
    \vspace{-0.1in}
        \caption{\textbf{Top 3}: Adjacent layer relevance scores among top directions. \texttt{Rel Comp 1}: relevance scores to first component in next layer. \textbf{Bottom}: Log-likelihood of predicting aligned behavior with different directions.}
    \label{fig:network_arch}
    \end{center}
    \vskip -0.2in
\end{figure}

\begin{figure*}[t]
    \begin{center}
    \centerline{\includegraphics[width=\textwidth]{figures/component_projections.pdf}}
    \caption{Projection of representations on top components under different settings in SSFT. The projection on component 1 is strongly correlated with the model's safety behavior. \texttt{Harmful}: representations are from harmful samples. \texttt{Benign}: representations are from benign samples. \texttt{Non-Dominance}: \texttt{Harmful} setting with most non-dominant components removed by intervention. \texttt{Removal}: harmful samples with trigger tokens removed. We evaluate the intervention's impact on the model's general abilities in \autoref{sec:impact_inter}, ensuring that intervention does not degrade overall performance.}
    \label{fig:component_projections}
    \end{center}
    \vskip -0.2in
\end{figure*}

\subsection{Layer-Wise Dynamics of Safety Residual Space}

We now examine the evolution of safety feature directions in the space. Using PLRP, we can measure how one direction influences another by attributing feature directions to directions in earlier layers. \autoref{fig:network_arch} visualizes the  relevance score of different components between adjacent layers.

\paragraph{Early Phase: Development of Safety Features}
\label{sec:early_phase}
We analyze how feature directions evolve across layers using PLRP to trace relevance scores through the transformer network. Our analysis reveals two distinct patterns of propagation. In most layers, directions primarily retain information from their counterparts in the previous layer. For instance, as shown in \texttt{Rel Comp 1} of \autoref{fig:network_arch}, \texttt{L20-C1} inherits most of its relevance from \texttt{L19-C1}. In contrast, during early layers, directions exhibit a more dynamic pattern, receiving contributions from multiple directions in the previous layer.

\paragraph{Late Phase: Uncertainty Reduction for Safety Behavior}
After layer 15, we observe that major directions exhibit a stronger retention pattern, but their corresponding eigenvalues continue to increase. This creates an interesting dynamic: although the model's harmfulness prediction accuracy plateaus after layer 15 (as shown in \autoref{fig:classification_accuracy}), the log-likelihood of these predictions continues to grow across subsequent layers (\autoref{fig:network_arch}, Bottom).


In summary, our analysis reveals that \emph{feature directions develop gradually through the network, stabilizing their safety semantic meanings in the early layers. Subsequently, the dominant direction responsible for safety behavior continues to strengthen, reducing uncertainty in the model's aligned outputs.}

% These findings indicate that feature directions develop gradually through the network rather than emerging suddenly at specific layers.
% This pattern suggests that the late phase of forward propagation primarily serves to reduce uncertainty in the model's existing predictions rather than developing new features.


% \begin{figure}
%     % \vskip 0.1in
%     \begin{center}
%     \begin{tabular}{c}
%         \includegraphics[width=\columnwidth]{figures/network_0.pdf} \\[-0.2in]
%         \includegraphics[width=\columnwidth]{figures/network_1.pdf} \\[-0.2in]
%         \includegraphics[width=\columnwidth]{figures/network_2.pdf} \\[-0.2in]
%         \includegraphics[width=\columnwidth]{figures/network_3.pdf}
%     \end{tabular}
%     \caption{Relevance propagation visualization across different components.}
%     \label{fig:network_arch}
%     \end{center}
%     % \vskip -0.2in
% \end{figure}
\section{Application: Harnessing the Linearity}
\label{sec:application}
Leveraging the \emph{linearity} of DMD operator, as well as the intuition of bases exposed by the spectral decomposition, we have developed several novel applications that extend the capabilities of our Koopman-based reduced-order simulation pipeline. In this section, we explore these applications, demonstrating that our method's unique strengths translate into practical tools for graphics and simulation.

\subsection{Direct Editing Temporal Dynamics}
\label{sec:editing}
\begin{figure}[!ht]
    \centering
    \includegraphics[width=1\columnwidth]{figure/karman_vortex_street_editing.pdf}
    \caption{\textbf{Editing temporal dynamics of K\'arm\'an Vortex Street with the Koopman Operator Approximation}. The modifications are applied to the DMD basis coefficients: (a) Scaling the modulus of the DMD basis by factors of 0.5, 1.0, and 1.5, affecting overall amplitude; (b) Adjusting the real part of $\bm{\Omega}$, influencing growth and decay rates of modal contributions; (c) Modifying the imaginary part, altering phase dynamics and wave propagation characteristics. }
    \label{fig:karman_editing}
    \Description{}
\end{figure}


\begin{figure*}[!ht]
    \centering
    \includegraphics[width=1\linewidth]{figure/reversibility.pdf}
    \caption{\textbf{Reversibility of Flows with Inversed DMD Operator}. We compare the reconstruction of two distinct fluid flows using Dynamic Mode Decomposition (DMD). The top row in each panel shows the velocity L2-norm of the field used to train the DMD, while the second and third rows depict the temporal evolution of the reconstructed flow fields as applied to an initial density field. The forward-time training phase is followed by a backward-time testing phase to assess predictive accuracy when advecting backward in time. The bottom plots show the evolution of kinetic energy over time. From the buoyant case, we observe the inverted DMD operator $\bm{A^{-1}}$ can still reasonably trace backward in time without compromising much visual quality. The vortical case exhibits a more challenging example where the symmetry should be reconstructed backwards in time. We see that the inverse operator indeed recovers this symmetry, with some acceptable levels of incurred noise. Bottom plots show the evolution of the total kinetic energy over time, demonstrating that our inverse operator actually correctly reverses the arrow of time, reversing the dissipation-related entropy increase over time. Decreasing kinetic energy also validates the \emph{physical plausibility} of our result.}
    \label{fig:reverse_simulation}
    \Description{}
\end{figure*}


Since our method approximates \refeq{eqn:euler_equations} with a linear operator in the full space, this allows us to transform the operator acting on the velocity field into the evolution of different modes under a linear operator. Therefore, we can directly edit the temporal dynamics of the fluid system by modifying the modes of the reduced \koopman{} $\bm{\hat{K}}$:
we set $t_0$ to be the initial time, $\bm{\Omega} = \nicefrac{\log(\bm\Lambda)}{\Delta t}$, where $\Delta t$ is the time step of the dataset. With this, we can rewrite \refeq{eqn:reduced_koopman_simulation} in the following form:
\begin{equation}
    \begin{aligned}
    \bm{u}(t_0 + k\Delta t) &= \bm{\Phi}\exp(\bm{\Omega} t) \bm{z}(t_0) \\
    &= \bm{\Phi}\exp{\left(k(\log(r) + i\theta)\right)} \bm{z}(t_0) \\
    &= \sum_{i = 1}^{n} {w_i} \bm{\Phi_i} r_i^k \left(\cos(k\theta_i) + \sin(k\theta_i)\right) \bm{z_i}(t_0)\\
    \end{aligned}
    \label{eqn:edit_temporal}
\end{equation}
% explanation for the formula
where ${w_i}$ is a user-defined scalar weight, $r_i = \sqrt{\Re(\lambda_i)^2 + \Im(\lambda_i)^2}$ is the \emph{modulus} and $\theta_i = \arctan\left(\Im(\lambda_i), \Re(\lambda_i)\right)$ is the \emph{phase} of the $i$-th eigenvalue $\lambda_i$ in the diagonal \emph{complex} eigenvalue matrix $\bm{\Lambda}$. Notice that this implies that the modes of the spectral decomposition represent different scales of vorticity, completing the physical intuition of the reduced space modes.
% show the benefits of our method for artist to edit

As shown in \refeq{eqn:edit_temporal}, our method decomposes a simulation sequence into modes with different growth/decay rates and frequencies.
The growth/decay rate of a mode is reflected in $r_i$, where a larger $r_i$ indicates a higher growth rate (or a lower decay rate), and vice versa.
The frequency of a mode is represented by the absolute value of $\theta_i$, with a larger absolute value corresponding to a higher frequency mode, and vice versa.
Furthermore, the different modes are decoupled, allowing for the adjustment of the relative proportions between modes.
As a result, these properties provide the artist with powerful tools to edit the simulation playback. The artist can modify the overall velocity field by adjusting the proportion ($w_i$), growth/decay rate ($r_i$), and frequency ($\theta_i$) of specific modes.
% explanation for what we actually do in code
In the experiments, we directly adjust the real part of $\bm{\Omega_i}$ to control $r_i$, modify the imaginary part of $\bm{\Omega_i}$ to control $\theta_i$, and vary the modulus of $\bm{\Phi_i}$ to control $w_i$.
% explanation for what we did to edit in karman vortex street scene
\paragraph{Editing the K\'arm\'an Vortex Street}
The first example is editing on the classic K\'arm\'an vortex street. We filter the imaginary part of $\bm{\Omega}$ and cluster modes with an absolute value smaller than $0.01$ as \emph{low-frequency cluster}, and the rest as \emph{high-frequency cluster}.
The low-frequency mode manifests as a laminar flow, with its phase changing very slowly over time. The high-frequency mode is represented by vortical structures distributed on both sides of the cylinder, where the phase of this mode changes relatively quickly over time.
As seen in \reffig{fig:karman_editing}, when we adjust the modulus of the high-frequency cluster from $0.5$ to $1.5$, the intensity of the vortices increases, which is as we expected. When we set the real part of $\bm{\Omega}$ to $0.5$, it can be observed that the high-frequency motion decays faster than user input. When we set the real part of $\bm{\Omega}$ to $1.5$, it can be observed that the high-frequency motion decays slower than user input. Similarly, when we tune the imaginary part of $\bm{\Omega}$ from $0.5$ to $1.5$, we could observe the oscillation frequency of the fluid trail transitions from slow to fast compared to user input.
% explanation for what we did to edit in 3D plume scene
\paragraph{Editing the Plume with Bunny}
To evaluate the editing capability of our method, we scale our editing scenario to 3D. With the same filtering procedure as in the K\'arm\'an vortex street example, we set the low-frequency cluster to high-frequency cluster ratio to $4:1$, $2:1$, $1:2$, and $1:4$, and compared the results with the user input. From the results, we observe that when the proportion of low-frequency cluster is increased, with a ratio of $4:1$, the top of the plume lacks "wrinkles" and appears more "fluffy". This is because the velocity field is dominated by smoother, lower-frequency modes than the original user input. Conversely, when the proportion of high-frequency cluster is increased, with ratios of $1:4$, the plume developes more detailed plume structure around the top, as the velocity field now emphasizes more high-frequency details compared to the user input.

\subsection{Reversibility of the Reduced Simulation}
Although physically-based fluid simulations have the capability to generate stunning visuals, when artists aim to direct the fluid's evolution toward a predefined target shape, challenges arise. It is a long standing problem in the community that people aim to enable users with \emph{spatial control}. In this example, we aim to enable users to do \emph{temporal control}, motived by a prior work \citet{oborn2018time}. Compared to previous work \shortcite{oborn2018time} where the authors employ a self-attraction force to replace the arbitrary external forces, providing a stable, physics-motivated, but time-consuming approach, we propose a data-driven, fast, and easy to implement method to address the same problem.

\label{sec:reversibility}

We observe that that given $\bm{\tilde{K}} = \bm{\Phi} \bm{\Lambda} \bm{\Phi}^+$, we could easily compute the \emph{inverse} of the truncated \koopman{} $\bm{\tilde{K}}^{-1} = (\bm{\Phi} \bm{\Lambda} \bm{\Phi}^+)^{-1} = \bm{\Phi} \bm{\Lambda}^{-1} \bm{\Phi}^+$, which is essentially the approximate inverse time evolution $\bm{f}^{-1}(\bm u)$ of the fluid system. This allows us to reverse the simulation by applying the inverse truncated \koopman{} to the current state of the fluid system:
\begin{equation}
    \label{eqn:reverse_simulation}
    \begin{aligned}
        \bm{u}(t) &= \bm{A}^{-1} \bm{u}(t + \Delta t), \\
        \bm{u}(t) &= \bm{\Phi} \bm{\Lambda}^{-1}\bm{\Phi}^+ \bm{u}(t + \Delta t), \\
        \bm{u}(t) &= \bm{\Phi} \bm{\Lambda}^{-1} \bm{z}(t + \Delta t).
    \end{aligned}
\end{equation}

Similar to \refeq{eqn:reduced_koopman_projection}, we could train the reduced \koopman{} on the forward simulation data and then apply the inverse reduced \koopman{} to reverse the simulation, given a state of the fluid system.


\begin{figure*}[!ht]
    \centering
    \includegraphics[width=1\linewidth]{figure/upsample.pdf}
    \caption{\textbf{Upsampling and Generalization to Unseen Sequences with Trained DMD Operator}. Two different input low-resolution fluid simulations (bunny and strawberry) are upscaled using the same DMD operator trained on a different velocity field. Initial velocity fields are seeded as moving down based on the input density field.    
    Naive application of DMD shown in each middle column, and our \emph{augmented DMD upresolution} method shown on the right columns. 
    Schematic of our method presented on the far right. At each frame, we project the low-resolution artist-directed input into the low-order bases of our reduced representation, using these to replace the low-order terms of the DMD field. Notice that naive application of DMD simply moves towards the known input training data, while our augmented field matches the low-resolution input more closely, with extra high-order detail gained from the DMD operator.}
    \label{fig:upsample}
    \Description{}
\end{figure*}


% first explanation for buoyant reversibility
\paragraph{Reversibility of Buoyant Flow}
We experiment our approach on a simple buoyant flow setup (\reffig{fig:reverse_simulation}, left). Our dataset was initialized with a \textit{qian}, a density field shaped like a round coin with a square hole, with the density value set to $1$. A density value of $1$ density field was driven by a velocity field where an upwards velocity of $0.3$ is set within the qian and downwards elsewhere. We run the simulation for $300$ frames to construct the dataset, and trained the DMD operator on this dataset. The inverse operator $\bm{\tilde{K}}^{-1}$ was then applied to the initial velocity field of the dataset at $t=0$ (frame $0$). By iteratively applying the inverse operator, we obtained the velocity fields for the preceding frames, starting from frame $-1$, frame $-2$, and all the way back to frame $-300$.
% stability
When examining the evolution of the density field from frame -300 to frame 300, it is evident that the velocity field remains consistently upward and smooth, indicating that our method is both reasonable and effective.
% energy
Further analysis of the energy of the velocity field obtained through the inverse process and the velocity field from the dataset reveals a downward trend in energy, with a smooth and reasonable curve, consistent with fluids with dissipative properties. This demonstrates that our inverse operator has the ability to predict a \emph{physically-plausible} velocity field prior to the dataset.

% second explanation for vortical reversibility
\paragraph{Reversibility of Vortical Flow}
To challenge the method with a scene of nontrivial vortical structure, we initialized a vortex sheet by placing four vortices at the corners of the domain (\reffig{fig:reverse_simulation}, right). We generated the dataset using the same procedure as in the previous experiment, resulting in a collection of $500$ frames. Subsequently, we constructed the inverse operator to recover the velocity fields preceding the dataset.
% stability
The results show that the density field (counterclockwise) and the dataset (clockwise) rotate in the opposite direction, which indicates that the velocity field predicted by the inverse operator is correct. This is because the vortex sheet velocity field continuously rotates in a clockwise direction, and by examining the density field from frame -500 to frame 500, we observe that the field indeed undergoes continuous clockwise rotation.
% energy
From the energy field analysis, the results show that, except for the significant energy fluctuation between frames -500 and -450, the energy consistently decreases in the remaining frames, with a consistent slope. This further demonstrates the robustness of our method.

\subsection{Upsampling with Reduced Koopman Operator}

The scale of the imaginary part of eigenvalues in $\bm{\Lambda}$ encode different scales of turbulent modes, enabling us to use a trained DMD operator to add in secondary motion to an existing fluid simulation. This is particularly useful for \emph{upscaling} a low-resolution fluid, simulated using stable fluid for example, leveraging the DMD basis to add in turbulent modes that were too small for the low-res sim to capture. This upscaling problem has been explored in prior work \cite{kim2008wavelet, nielsen2009guiding}, but we show that due to the linearity of the Koopman operator, and the physical intuition on each of its reduced bases, this upscaling is essentially attained for \emph{free}, amounting to nothing more than a linear combination of two matrix multiplications. 

\subsubsection{Evolution} \label{sec:upres_direct}

Suppose we have frames of a low-res input velocity field $\{\bm{L}_0, \bm{L}_1, \bm{L}_2, \dots, \bm{L}_T\}$, a high-res initial condition $H_0$. Additionally, we have some DMD basis $\bm{\Phi}$ trained on some high-res simulation distinct from the low-res simulation, with corresponding eigenvalues $\bm{\Lambda}$, sorted by the length of their imaginary parts in increasing order. At the first frame, we can generate the reduced-space initial condition by simply using our basis mapping $R_0 = \bm{\Phi}^TH_0$.

Now, for every subsequent frame $t$, we generate $R_t$ by first applying the DMD evolution on the previous reduced space frame to produce an intermediate state $R^*_t=\bm{\Lambda}R_{t-1}$. We also produce a representation of the current frame of the low-res input in reduced space $P_t = \bm{\Phi}^TL_t$. Now, we have a representation of the \emph{current} frame of the low-res input, and the DMD \emph{time evolution} of the \emph{previous} reduced space frame. We want to keep the low-order bulk flow of the low-res input, and augment it with the high-order turbulent flow learned by the DMD basis. To that end, we split each reduced-space vector into a low-order and high-order part: $R^*_t = \left[R_t^{*L}\ R_t^{*H}\right]$, $P_t=\left[P_t^L\ P_t^H\right]$. Now, we take only the low-order modes of the input flow, and the high-order modes of the DMD-evolved flow, to produce our new reduced space velocity field $R_t=\left[P_t^L\ R_t^{*H}\right]$. From here, we can just apply the basis to return to high-resolution full-space $H_t=\bm{\Phi}R_t$.

We note that the composition operators here are linear. We can simply represent them with selection matrices $S^H$, $S_L$, for the high- and low-order bases respectively, such that $R_t=S^LP_t + S^HR_t^*$. Since the DMD operator is also linear, we note that this entire upscaling method is linear by construction.

Results are shown on Figure \ref{fig:upsample}. We see that even if the initial velocity field is significantly different from the input field, the low-order basis is able to capture the bulk flow of the low-resolution input, and modify the DMD-produced field accordingly. In particular, we note that naively applying the DMD operator, without passing the low-resolution input field into the low-order bases, ends up reconstructing the original training set, rather than a velocity field directed by our input. This is demonstrated by the results for the two initial conditions being very similar, whereas our augmented field matches the input much closer.

\subsubsection{Projection}

The above governs the time evolution of the velocity field. In some cases, where the input velocity field differs significantly from the training data used for the DMD basis, the above as written will still produce velocity fields that are unacceptably different from the input velocity field. This is largely representation error, fields that are far away from the training data are less representable by the reduced space. In these cases, we can again leverage our input low-res field, this time as a constraint. 

Essentially, we would like to project our velocity field $\bm{H}_t$ onto the space of velocity fields that are identical to the input low-res field when downsampled to that resolution. This can be represented as an equality-constrained quadratic problem,
\begin{align}
    &\argmin_x \frac{1}{2}(\bm{x}-\bm{H}_t)^T(\bm{x}-\bm{H}_t) \\
    &\text{subject to } \bm{Ax} = \bm{L}_t,
\end{align}
where $\bm{A}$ is a downsampling operator that converts from high-res to low-res. 
Notice that because the downsampling operator does not change for the duration of the simulation. Thus, the KKT (Karush-Kuhn-Tucker) matrix can be precomputed making the projection a single matrix multiply during runtime.

\begin{wrapfigure}{r}{0.5\columnwidth}
    \vspace{-2pt}
    \includegraphics[width=0.5\columnwidth]{figure/qr.pdf}
    \hspace{5pt}
    \label{fig:qp_project}
\end{wrapfigure}

As a sanity check, we show the effect of this projection here: it is apparent with the projection step,we can recover fields that are much closer to the input, yet retaining extra high-order detail. And of course, because these are all linear, linear combinations of the direct and projected fields can be taken. In particular, because the basis functions of reduced space are orthogonal, a diagonal matrix of linear weights can be taken, preferring projected for low-order modes and direct for high-order modes for example.

\begin{comment}
Given a high-resolution DMD matrix $A \in \mathbb{R}^{N_{hi} \times N_{hi}}$ trained on high-dimensional data, we reconstruct a high-resolution sequence $\bm{x}_{hi}(t) \in \mathbb{R}^{N_{hi}}$ using an initial high-resolution frame $\bm{x}_{hi}(0)$ and subsequent low-resolution frames $\bm{x}_{lo}(t) \in \mathbb{R}^{N_{lo}}$, where $N_{lo} < N_{hi}$. The matrix $A$ is structured as
\begin{equation}
    \label{eqn:slice_A}
    A = \begin{bmatrix} A_{ll} & A_{lh} \\ A_{hl} & A_{hh} \end{bmatrix}
\end{equation}, with $A_{ll} \in \mathbb{R}^{N_{lo} \times N_{lo}}$ map low frequency component to, $A_{lh} \in \mathbb{R}^{N_{lo} \times (N_{hi} - N_{lo})}$, $A_{hl} \in \mathbb{R}^{(N_{hi} - N_{lo}) \times N_{lo}}$, and $A_{hh} \in \mathbb{R}^{(N_{hi} - N_{lo}) \times (N_{hi} - N_{lo})}$.

Starting from the initial condition $\bm{x}_{hi}(0)$, the high-resolution state at time $t + \Delta t$ is updated using:
\begin{equation}
    \label{eqn:upsampling_advect}
    \bm{x}_{hi}(t + \Delta t) = A \bm{x}_{hi}(t) + \begin{bmatrix} \bm{x}_{lo}(t + \Delta t) - A_{ll} \bm{x}_{lo}(t) \\ A_{hl} \left( \bm{x}_{lo}(t + \Delta t) - A_{ll} \bm{x}_{lo}(t) \right) \end{bmatrix}.
\end{equation}

In this equation, $A \bm{x}_{hi}(t)$ evolves the high-resolution dynamics. The term $\bm{x}_{lo}(t + \Delta t) - A_{ll} \bm{x}_{lo}(t)$ represents the correction to the low-frequency component, and $A_{hl} \left( \bm{x}_{lo}(t + \Delta t) - A_{ll} \bm{x}_{lo}(t) \right)$ in-paints the missing high-frequency details. This process ensures the reconstructed high-resolution sequence remains consistent with the initial frame and the low-resolution input while leveraging the full dynamics encoded in $A$.

\end{comment}
\section{Discussion of Assumptions}\label{sec:discussion}
In this paper, we have made several assumptions for the sake of clarity and simplicity. In this section, we discuss the rationale behind these assumptions, the extent to which these assumptions hold in practice, and the consequences for our protocol when these assumptions hold.

\subsection{Assumptions on the Demand}

There are two simplifying assumptions we make about the demand. First, we assume the demand at any time is relatively small compared to the channel capacities. Second, we take the demand to be constant over time. We elaborate upon both these points below.

\paragraph{Small demands} The assumption that demands are small relative to channel capacities is made precise in \eqref{eq:large_capacity_assumption}. This assumption simplifies two major aspects of our protocol. First, it largely removes congestion from consideration. In \eqref{eq:primal_problem}, there is no constraint ensuring that total flow in both directions stays below capacity--this is always met. Consequently, there is no Lagrange multiplier for congestion and no congestion pricing; only imbalance penalties apply. In contrast, protocols in \cite{sivaraman2020high, varma2021throughput, wang2024fence} include congestion fees due to explicit congestion constraints. Second, the bound \eqref{eq:large_capacity_assumption} ensures that as long as channels remain balanced, the network can always meet demand, no matter how the demand is routed. Since channels can rebalance when necessary, they never drop transactions. This allows prices and flows to adjust as per the equations in \eqref{eq:algorithm}, which makes it easier to prove the protocol's convergence guarantees. This also preserves the key property that a channel's price remains proportional to net money flow through it.

In practice, payment channel networks are used most often for micro-payments, for which on-chain transactions are prohibitively expensive; large transactions typically take place directly on the blockchain. For example, according to \cite{river2023lightning}, the average channel capacity is roughly $0.1$ BTC ($5,000$ BTC distributed over $50,000$ channels), while the average transaction amount is less than $0.0004$ BTC ($44.7k$ satoshis). Thus, the small demand assumption is not too unrealistic. Additionally, the occasional large transaction can be treated as a sequence of smaller transactions by breaking it into packets and executing each packet serially (as done by \cite{sivaraman2020high}).
Lastly, a good path discovery process that favors large capacity channels over small capacity ones can help ensure that the bound in \eqref{eq:large_capacity_assumption} holds.

\paragraph{Constant demands} 
In this work, we assume that any transacting pair of nodes have a steady transaction demand between them (see Section \ref{sec:transaction_requests}). Making this assumption is necessary to obtain the kind of guarantees that we have presented in this paper. Unless the demand is steady, it is unreasonable to expect that the flows converge to a steady value. Weaker assumptions on the demand lead to weaker guarantees. For example, with the more general setting of stochastic, but i.i.d. demand between any two nodes, \cite{varma2021throughput} shows that the channel queue lengths are bounded in expectation. If the demand can be arbitrary, then it is very hard to get any meaningful performance guarantees; \cite{wang2024fence} shows that even for a single bidirectional channel, the competitive ratio is infinite. Indeed, because a PCN is a decentralized system and decisions must be made based on local information alone, it is difficult for the network to find the optimal detailed balance flow at every time step with a time-varying demand.  With a steady demand, the network can discover the optimal flows in a reasonably short time, as our work shows.

We view the constant demand assumption as an approximation for a more general demand process that could be piece-wise constant, stochastic, or both (see simulations in Figure \ref{fig:five_nodes_variable_demand}).
We believe it should be possible to merge ideas from our work and \cite{varma2021throughput} to provide guarantees in a setting with random demands with arbitrary means. We leave this for future work. In addition, our work suggests that a reasonable method of handling stochastic demands is to queue the transaction requests \textit{at the source node} itself. This queuing action should be viewed in conjunction with flow-control. Indeed, a temporarily high unidirectional demand would raise prices for the sender, incentivizing the sender to stop sending the transactions. If the sender queues the transactions, they can send them later when prices drop. This form of queuing does not require any overhaul of the basic PCN infrastructure and is therefore simpler to implement than per-channel queues as suggested by \cite{sivaraman2020high} and \cite{varma2021throughput}.

\subsection{The Incentive of Channels}
The actions of the channels as prescribed by the DEBT control protocol can be summarized as follows. Channels adjust their prices in proportion to the net flow through them. They rebalance themselves whenever necessary and execute any transaction request that has been made of them. We discuss both these aspects below.

\paragraph{On Prices}
In this work, the exclusive role of channel prices is to ensure that the flows through each channel remains balanced. In practice, it would be important to include other components in a channel's price/fee as well: a congestion price  and an incentive price. The congestion price, as suggested by \cite{varma2021throughput}, would depend on the total flow of transactions through the channel, and would incentivize nodes to balance the load over different paths. The incentive price, which is commonly used in practice \cite{river2023lightning}, is necessary to provide channels with an incentive to serve as an intermediary for different channels. In practice, we expect both these components to be smaller than the imbalance price. Consequently, we expect the behavior of our protocol to be similar to our theoretical results even with these additional prices.

A key aspect of our protocol is that channel fees are allowed to be negative. Although the original Lightning network whitepaper \cite{poon2016bitcoin} suggests that negative channel prices may be a good solution to promote rebalancing, the idea of negative prices in not very popular in the literature. To our knowledge, the only prior work with this feature is \cite{varma2021throughput}. Indeed, in papers such as \cite{van2021merchant} and \cite{wang2024fence}, the price function is explicitly modified such that the channel price is never negative. The results of our paper show the benefits of negative prices. For one, in steady state, equal flows in both directions ensure that a channel doesn't loose any money (the other price components mentioned above ensure that the channel will only gain money). More importantly, negative prices are important to ensure that the protocol selectively stifles acyclic flows while allowing circulations to flow. Indeed, in the example of Section \ref{sec:flow_control_example}, the flows between nodes $A$ and $C$ are left on only because the large positive price over one channel is canceled by the corresponding negative price over the other channel, leading to a net zero price.

Lastly, observe that in the DEBT control protocol, the price charged by a channel does not depend on its capacity. This is a natural consequence of the price being the Lagrange multiplier for the net-zero flow constraint, which also does not depend on the channel capacity. In contrast, in many other works, the imbalance price is normalized by the channel capacity \cite{ren2018optimal, lin2020funds, wang2024fence}; this is shown to work well in practice. The rationale for such a price structure is explained well in \cite{wang2024fence}, where this fee is derived with the aim of always maintaining some balance (liquidity) at each end of every channel. This is a reasonable aim if a channel is to never rebalance itself; the experiments of the aforementioned papers are conducted in such a regime. In this work, however, we allow the channels to rebalance themselves a few times in order to settle on a detailed balance flow. This is because our focus is on the long-term steady state performance of the protocol. This difference in perspective also shows up in how the price depends on the channel imbalance. \cite{lin2020funds} and \cite{wang2024fence} advocate for strictly convex prices whereas this work and \cite{varma2021throughput} propose linear prices.

\paragraph{On Rebalancing} 
Recall that the DEBT control protocol ensures that the flows in the network converge to a detailed balance flow, which can be sustained perpetually without any rebalancing. However, during the transient phase (before convergence), channels may have to perform on-chain rebalancing a few times. Since rebalancing is an expensive operation, it is worthwhile discussing methods by which channels can reduce the extent of rebalancing. One option for the channels to reduce the extent of rebalancing is to increase their capacity; however, this comes at the cost of locking in more capital. Each channel can decide for itself the optimum amount of capital to lock in. Another option, which we discuss in Section \ref{sec:five_node}, is for channels to increase the rate $\gamma$ at which they adjust prices. 

Ultimately, whether or not it is beneficial for a channel to rebalance depends on the time-horizon under consideration. Our protocol is based on the assumption that the demand remains steady for a long period of time. If this is indeed the case, it would be worthwhile for a channel to rebalance itself as it can make up this cost through the incentive fees gained from the flow of transactions through it in steady state. If a channel chooses not to rebalance itself, however, there is a risk of being trapped in a deadlock, which is suboptimal for not only the nodes but also the channel.

\section{Conclusion}
This work presents DEBT control: a protocol for payment channel networks that uses source routing and flow control based on channel prices. The protocol is derived by posing a network utility maximization problem and analyzing its dual minimization. It is shown that under steady demands, the protocol guides the network to an optimal, sustainable point. Simulations show its robustness to demand variations. The work demonstrates that simple protocols with strong theoretical guarantees are possible for PCNs and we hope it inspires further theoretical research in this direction.
\section{Related Works}
\label{sec:related_works}


\noindent\textbf{Diffusion-based Video Generation. }
The advancement of diffusion models \cite{rombach2022high, ramesh2022hierarchical, zheng2022entropy} has led to significant progress in video generation. Due to the scarcity of high-quality video-text datasets \cite{Blattmann2023, Blattmann2023a}, researchers have adapted existing text-to-image (T2I) models to facilitate text-to-video (T2V) generation. Notable examples include AnimateDiff \cite{Guo2023}, Align your Latents \cite{Blattmann2023a}, PYoCo \cite{ge2023preserve}, and Emu Video \cite{girdhar2023emu}. Further advancements, such as LVDM \cite{he2022latent}, VideoCrafter \cite{chen2023videocrafter1, chen2024videocrafter2}, ModelScope \cite{wang2023modelscope}, LAVIE \cite{wang2023lavie}, and VideoFactory \cite{wang2024videofactory}, have refined these approaches by fine-tuning both spatial and temporal blocks, leveraging T2I models for initialization to improve video quality.
Recently, Sora \cite{brooks2024video} and CogVideoX \cite{yang2024cogvideox} enhance video generation by introducing Transformer-based diffusion backbones \cite{Peebles2023, Ma2024, Yu2024} and utilizing 3D-VAE, unlocking the potential for realistic world simulators. Additionally, SVD \cite{Blattmann2023}, SEINE \cite{chen2023seine}, PixelDance \cite{zeng2024make} and PIA \cite{zhang2024pia} have made significant strides in image-to-video generation, achieving notable improvements in quality and flexibility.
Further, I2VGen-XL \cite{zhang2023i2vgen}, DynamicCrafter \cite{Xing2023}, and Moonshot \cite{zhang2024moonshot} incorporate additional cross-attention layers to strengthen conditional signals during generation.



\noindent\textbf{Controllable Generation.}
Controllable generation has become a central focus in both image \citep{Zhang2023,jiang2024survey, Mou2024, Zheng2023, peng2024controlnext, ye2023ip, wu2024spherediffusion, song2024moma, wu2024ifadapter} and video \citep{gong2024atomovideo, zhang2024moonshot, guo2025sparsectrl, jiang2024videobooth} generation, enabling users to direct the output through various types of control. A wide range of controllable inputs has been explored, including text descriptions, pose \citep{ma2024follow,wang2023disco,hu2024animate,xu2024magicanimate}, audio \citep{tang2023anytoany,tian2024emo,he2024co}, identity representations \citep{chefer2024still,wang2024customvideo,wu2024customcrafter}, trajectory \citep{yin2023dragnuwa,chen2024motion,li2024generative,wu2024motionbooth, namekata2024sg}.


\noindent\textbf{Text-based Camera Control.}
Text-based camera control methods use natural language descriptions to guide camera motion in video generation. AnimateDiff \cite{Guo2023} and SVD \cite{Blattmann2023} fine-tune LoRAs \cite{hu2021lora} for specific camera movements based on text input. 
Image conductor\cite{li2024image} proposed to separate different camera and object motions through camera LoRA weight and object LoRA weight to achieve more precise motion control.
In contrast, MotionMaster \cite{hu2024motionmaster} and Peekaboo \cite{jain2024peekaboo} offer training-free approaches for generating coarse-grained camera motions, though with limited precision. VideoComposer \cite{wang2024videocomposer} adjusts pixel-level motion vectors to provide finer control, but challenges remain in achieving precise camera control.

\noindent\textbf{Trajectory-based Camera Control.}
MotionCtrl \cite{Wang2024Motionctrl}, CameraCtrl \cite{He2024Cameractrl}, and Direct-a-Video \cite{yang2024direct} use camera pose as input to enhance control, while CVD \cite{kuang2024collaborative} extends CameraCtrl for multi-view generation, though still limited by motion complexity. To improve geometric consistency, Pose-guided diffusion \cite{tseng2023consistent}, CamCo \cite{Xu2024}, and CamI2V \cite{zheng2024cami2v} apply epipolar constraints for consistent viewpoints. VD3D \cite{bahmani2024vd3d} introduces a ControlNet\cite{Zhang2023}-like conditioning mechanism with spatiotemporal camera embeddings, enabling more precise control.
CamTrol \cite{hou2024training} offers a training-free approach that renders static point clouds into multi-view frames for video generation. Cavia \cite{xu2024cavia} introduces view-integrated attention mechanisms to improve viewpoint and temporal consistency, while I2VControl-Camera \cite{feng2024i2vcontrol} refines camera movement by employing point trajectories in the camera coordinate system. Despite these advancements, challenges in maintaining camera control and scene-scale consistency remain, which our method seeks to address. It is noted that 4Dim~\cite{watson2024controlling} introduces absolute scale but in  4D novel view synthesis (NVS) of scenes.




\section{Conclusion}
In this work, we provide a multi-dimensional mechanistic understanding of \emph{what LLMs learn from safety fine-tuning}. We identify multiple feature directions that jointly control safety behavior---a hidden dimension previously invisible to probing or static methods. We characterize the residual space and uncover key roles of non-dominant directions in affecting the model's safety behavior, linking them to specific trigger tokens. These insights into the underlying safety mechanisms shed new light on robust alignment research. One promising direction is preventing models from learning spurious correlations through targeted interventions in activation space or data augmentation for balanced training. We leave these possibilities for future work.

\section*{Impact Statement}
Our research shows methods for analyzing and bypassing LLM safety mechanisms, which could enable harmful content generation. We acknowledge these risks and emphasize the need for careful use of our methods. However, since multiple effective jailbreaks for the studied models are already public, our work does not create new safety concerns. We therefore believe sharing our code and methods openly benefits the research community by supporting reproducibility and future safety research.

\section*{Acknowledgments}
We thank Jianfei He, Xiang Li and Yulong Ming for their valuable feedback and suggestions that helped improve this paper.

\bibliography{custom}
\bibliographystyle{icml2024}


\appendix
\subsection{Lloyd-Max Algorithm}
\label{subsec:Lloyd-Max}
For a given quantization bitwidth $B$ and an operand $\bm{X}$, the Lloyd-Max algorithm finds $2^B$ quantization levels $\{\hat{x}_i\}_{i=1}^{2^B}$ such that quantizing $\bm{X}$ by rounding each scalar in $\bm{X}$ to the nearest quantization level minimizes the quantization MSE. 

The algorithm starts with an initial guess of quantization levels and then iteratively computes quantization thresholds $\{\tau_i\}_{i=1}^{2^B-1}$ and updates quantization levels $\{\hat{x}_i\}_{i=1}^{2^B}$. Specifically, at iteration $n$, thresholds are set to the midpoints of the previous iteration's levels:
\begin{align*}
    \tau_i^{(n)}=\frac{\hat{x}_i^{(n-1)}+\hat{x}_{i+1}^{(n-1)}}2 \text{ for } i=1\ldots 2^B-1
\end{align*}
Subsequently, the quantization levels are re-computed as conditional means of the data regions defined by the new thresholds:
\begin{align*}
    \hat{x}_i^{(n)}=\mathbb{E}\left[ \bm{X} \big| \bm{X}\in [\tau_{i-1}^{(n)},\tau_i^{(n)}] \right] \text{ for } i=1\ldots 2^B
\end{align*}
where to satisfy boundary conditions we have $\tau_0=-\infty$ and $\tau_{2^B}=\infty$. The algorithm iterates the above steps until convergence.

Figure \ref{fig:lm_quant} compares the quantization levels of a $7$-bit floating point (E3M3) quantizer (left) to a $7$-bit Lloyd-Max quantizer (right) when quantizing a layer of weights from the GPT3-126M model at a per-tensor granularity. As shown, the Lloyd-Max quantizer achieves substantially lower quantization MSE. Further, Table \ref{tab:FP7_vs_LM7} shows the superior perplexity achieved by Lloyd-Max quantizers for bitwidths of $7$, $6$ and $5$. The difference between the quantizers is clear at 5 bits, where per-tensor FP quantization incurs a drastic and unacceptable increase in perplexity, while Lloyd-Max quantization incurs a much smaller increase. Nevertheless, we note that even the optimal Lloyd-Max quantizer incurs a notable ($\sim 1.5$) increase in perplexity due to the coarse granularity of quantization. 

\begin{figure}[h]
  \centering
  \includegraphics[width=0.7\linewidth]{sections/figures/LM7_FP7.pdf}
  \caption{\small Quantization levels and the corresponding quantization MSE of Floating Point (left) vs Lloyd-Max (right) Quantizers for a layer of weights in the GPT3-126M model.}
  \label{fig:lm_quant}
\end{figure}

\begin{table}[h]\scriptsize
\begin{center}
\caption{\label{tab:FP7_vs_LM7} \small Comparing perplexity (lower is better) achieved by floating point quantizers and Lloyd-Max quantizers on a GPT3-126M model for the Wikitext-103 dataset.}
\begin{tabular}{c|cc|c}
\hline
 \multirow{2}{*}{\textbf{Bitwidth}} & \multicolumn{2}{|c|}{\textbf{Floating-Point Quantizer}} & \textbf{Lloyd-Max Quantizer} \\
 & Best Format & Wikitext-103 Perplexity & Wikitext-103 Perplexity \\
\hline
7 & E3M3 & 18.32 & 18.27 \\
6 & E3M2 & 19.07 & 18.51 \\
5 & E4M0 & 43.89 & 19.71 \\
\hline
\end{tabular}
\end{center}
\end{table}

\subsection{Proof of Local Optimality of LO-BCQ}
\label{subsec:lobcq_opt_proof}
For a given block $\bm{b}_j$, the quantization MSE during LO-BCQ can be empirically evaluated as $\frac{1}{L_b}\lVert \bm{b}_j- \bm{\hat{b}}_j\rVert^2_2$ where $\bm{\hat{b}}_j$ is computed from equation (\ref{eq:clustered_quantization_definition}) as $C_{f(\bm{b}_j)}(\bm{b}_j)$. Further, for a given block cluster $\mathcal{B}_i$, we compute the quantization MSE as $\frac{1}{|\mathcal{B}_{i}|}\sum_{\bm{b} \in \mathcal{B}_{i}} \frac{1}{L_b}\lVert \bm{b}- C_i^{(n)}(\bm{b})\rVert^2_2$. Therefore, at the end of iteration $n$, we evaluate the overall quantization MSE $J^{(n)}$ for a given operand $\bm{X}$ composed of $N_c$ block clusters as:
\begin{align*}
    \label{eq:mse_iter_n}
    J^{(n)} = \frac{1}{N_c} \sum_{i=1}^{N_c} \frac{1}{|\mathcal{B}_{i}^{(n)}|}\sum_{\bm{v} \in \mathcal{B}_{i}^{(n)}} \frac{1}{L_b}\lVert \bm{b}- B_i^{(n)}(\bm{b})\rVert^2_2
\end{align*}

At the end of iteration $n$, the codebooks are updated from $\mathcal{C}^{(n-1)}$ to $\mathcal{C}^{(n)}$. However, the mapping of a given vector $\bm{b}_j$ to quantizers $\mathcal{C}^{(n)}$ remains as  $f^{(n)}(\bm{b}_j)$. At the next iteration, during the vector clustering step, $f^{(n+1)}(\bm{b}_j)$ finds new mapping of $\bm{b}_j$ to updated codebooks $\mathcal{C}^{(n)}$ such that the quantization MSE over the candidate codebooks is minimized. Therefore, we obtain the following result for $\bm{b}_j$:
\begin{align*}
\frac{1}{L_b}\lVert \bm{b}_j - C_{f^{(n+1)}(\bm{b}_j)}^{(n)}(\bm{b}_j)\rVert^2_2 \le \frac{1}{L_b}\lVert \bm{b}_j - C_{f^{(n)}(\bm{b}_j)}^{(n)}(\bm{b}_j)\rVert^2_2
\end{align*}

That is, quantizing $\bm{b}_j$ at the end of the block clustering step of iteration $n+1$ results in lower quantization MSE compared to quantizing at the end of iteration $n$. Since this is true for all $\bm{b} \in \bm{X}$, we assert the following:
\begin{equation}
\begin{split}
\label{eq:mse_ineq_1}
    \tilde{J}^{(n+1)} &= \frac{1}{N_c} \sum_{i=1}^{N_c} \frac{1}{|\mathcal{B}_{i}^{(n+1)}|}\sum_{\bm{b} \in \mathcal{B}_{i}^{(n+1)}} \frac{1}{L_b}\lVert \bm{b} - C_i^{(n)}(b)\rVert^2_2 \le J^{(n)}
\end{split}
\end{equation}
where $\tilde{J}^{(n+1)}$ is the the quantization MSE after the vector clustering step at iteration $n+1$.

Next, during the codebook update step (\ref{eq:quantizers_update}) at iteration $n+1$, the per-cluster codebooks $\mathcal{C}^{(n)}$ are updated to $\mathcal{C}^{(n+1)}$ by invoking the Lloyd-Max algorithm \citep{Lloyd}. We know that for any given value distribution, the Lloyd-Max algorithm minimizes the quantization MSE. Therefore, for a given vector cluster $\mathcal{B}_i$ we obtain the following result:

\begin{equation}
    \frac{1}{|\mathcal{B}_{i}^{(n+1)}|}\sum_{\bm{b} \in \mathcal{B}_{i}^{(n+1)}} \frac{1}{L_b}\lVert \bm{b}- C_i^{(n+1)}(\bm{b})\rVert^2_2 \le \frac{1}{|\mathcal{B}_{i}^{(n+1)}|}\sum_{\bm{b} \in \mathcal{B}_{i}^{(n+1)}} \frac{1}{L_b}\lVert \bm{b}- C_i^{(n)}(\bm{b})\rVert^2_2
\end{equation}

The above equation states that quantizing the given block cluster $\mathcal{B}_i$ after updating the associated codebook from $C_i^{(n)}$ to $C_i^{(n+1)}$ results in lower quantization MSE. Since this is true for all the block clusters, we derive the following result: 
\begin{equation}
\begin{split}
\label{eq:mse_ineq_2}
     J^{(n+1)} &= \frac{1}{N_c} \sum_{i=1}^{N_c} \frac{1}{|\mathcal{B}_{i}^{(n+1)}|}\sum_{\bm{b} \in \mathcal{B}_{i}^{(n+1)}} \frac{1}{L_b}\lVert \bm{b}- C_i^{(n+1)}(\bm{b})\rVert^2_2  \le \tilde{J}^{(n+1)}   
\end{split}
\end{equation}

Following (\ref{eq:mse_ineq_1}) and (\ref{eq:mse_ineq_2}), we find that the quantization MSE is non-increasing for each iteration, that is, $J^{(1)} \ge J^{(2)} \ge J^{(3)} \ge \ldots \ge J^{(M)}$ where $M$ is the maximum number of iterations. 
%Therefore, we can say that if the algorithm converges, then it must be that it has converged to a local minimum. 
\hfill $\blacksquare$


\begin{figure}
    \begin{center}
    \includegraphics[width=0.5\textwidth]{sections//figures/mse_vs_iter.pdf}
    \end{center}
    \caption{\small NMSE vs iterations during LO-BCQ compared to other block quantization proposals}
    \label{fig:nmse_vs_iter}
\end{figure}

Figure \ref{fig:nmse_vs_iter} shows the empirical convergence of LO-BCQ across several block lengths and number of codebooks. Also, the MSE achieved by LO-BCQ is compared to baselines such as MXFP and VSQ. As shown, LO-BCQ converges to a lower MSE than the baselines. Further, we achieve better convergence for larger number of codebooks ($N_c$) and for a smaller block length ($L_b$), both of which increase the bitwidth of BCQ (see Eq \ref{eq:bitwidth_bcq}).


\subsection{Additional Accuracy Results}
%Table \ref{tab:lobcq_config} lists the various LOBCQ configurations and their corresponding bitwidths.
\begin{table}
\setlength{\tabcolsep}{4.75pt}
\begin{center}
\caption{\label{tab:lobcq_config} Various LO-BCQ configurations and their bitwidths.}
\begin{tabular}{|c||c|c|c|c||c|c||c|} 
\hline
 & \multicolumn{4}{|c||}{$L_b=8$} & \multicolumn{2}{|c||}{$L_b=4$} & $L_b=2$ \\
 \hline
 \backslashbox{$L_A$\kern-1em}{\kern-1em$N_c$} & 2 & 4 & 8 & 16 & 2 & 4 & 2 \\
 \hline
 64 & 4.25 & 4.375 & 4.5 & 4.625 & 4.375 & 4.625 & 4.625\\
 \hline
 32 & 4.375 & 4.5 & 4.625& 4.75 & 4.5 & 4.75 & 4.75 \\
 \hline
 16 & 4.625 & 4.75& 4.875 & 5 & 4.75 & 5 & 5 \\
 \hline
\end{tabular}
\end{center}
\end{table}

%\subsection{Perplexity achieved by various LO-BCQ configurations on Wikitext-103 dataset}

\begin{table} \centering
\begin{tabular}{|c||c|c|c|c||c|c||c|} 
\hline
 $L_b \rightarrow$& \multicolumn{4}{c||}{8} & \multicolumn{2}{c||}{4} & 2\\
 \hline
 \backslashbox{$L_A$\kern-1em}{\kern-1em$N_c$} & 2 & 4 & 8 & 16 & 2 & 4 & 2  \\
 %$N_c \rightarrow$ & 2 & 4 & 8 & 16 & 2 & 4 & 2 \\
 \hline
 \hline
 \multicolumn{8}{c}{GPT3-1.3B (FP32 PPL = 9.98)} \\ 
 \hline
 \hline
 64 & 10.40 & 10.23 & 10.17 & 10.15 &  10.28 & 10.18 & 10.19 \\
 \hline
 32 & 10.25 & 10.20 & 10.15 & 10.12 &  10.23 & 10.17 & 10.17 \\
 \hline
 16 & 10.22 & 10.16 & 10.10 & 10.09 &  10.21 & 10.14 & 10.16 \\
 \hline
  \hline
 \multicolumn{8}{c}{GPT3-8B (FP32 PPL = 7.38)} \\ 
 \hline
 \hline
 64 & 7.61 & 7.52 & 7.48 &  7.47 &  7.55 &  7.49 & 7.50 \\
 \hline
 32 & 7.52 & 7.50 & 7.46 &  7.45 &  7.52 &  7.48 & 7.48  \\
 \hline
 16 & 7.51 & 7.48 & 7.44 &  7.44 &  7.51 &  7.49 & 7.47  \\
 \hline
\end{tabular}
\caption{\label{tab:ppl_gpt3_abalation} Wikitext-103 perplexity across GPT3-1.3B and 8B models.}
\end{table}

\begin{table} \centering
\begin{tabular}{|c||c|c|c|c||} 
\hline
 $L_b \rightarrow$& \multicolumn{4}{c||}{8}\\
 \hline
 \backslashbox{$L_A$\kern-1em}{\kern-1em$N_c$} & 2 & 4 & 8 & 16 \\
 %$N_c \rightarrow$ & 2 & 4 & 8 & 16 & 2 & 4 & 2 \\
 \hline
 \hline
 \multicolumn{5}{|c|}{Llama2-7B (FP32 PPL = 5.06)} \\ 
 \hline
 \hline
 64 & 5.31 & 5.26 & 5.19 & 5.18  \\
 \hline
 32 & 5.23 & 5.25 & 5.18 & 5.15  \\
 \hline
 16 & 5.23 & 5.19 & 5.16 & 5.14  \\
 \hline
 \multicolumn{5}{|c|}{Nemotron4-15B (FP32 PPL = 5.87)} \\ 
 \hline
 \hline
 64  & 6.3 & 6.20 & 6.13 & 6.08  \\
 \hline
 32  & 6.24 & 6.12 & 6.07 & 6.03  \\
 \hline
 16  & 6.12 & 6.14 & 6.04 & 6.02  \\
 \hline
 \multicolumn{5}{|c|}{Nemotron4-340B (FP32 PPL = 3.48)} \\ 
 \hline
 \hline
 64 & 3.67 & 3.62 & 3.60 & 3.59 \\
 \hline
 32 & 3.63 & 3.61 & 3.59 & 3.56 \\
 \hline
 16 & 3.61 & 3.58 & 3.57 & 3.55 \\
 \hline
\end{tabular}
\caption{\label{tab:ppl_llama7B_nemo15B} Wikitext-103 perplexity compared to FP32 baseline in Llama2-7B and Nemotron4-15B, 340B models}
\end{table}

%\subsection{Perplexity achieved by various LO-BCQ configurations on MMLU dataset}


\begin{table} \centering
\begin{tabular}{|c||c|c|c|c||c|c|c|c|} 
\hline
 $L_b \rightarrow$& \multicolumn{4}{c||}{8} & \multicolumn{4}{c||}{8}\\
 \hline
 \backslashbox{$L_A$\kern-1em}{\kern-1em$N_c$} & 2 & 4 & 8 & 16 & 2 & 4 & 8 & 16  \\
 %$N_c \rightarrow$ & 2 & 4 & 8 & 16 & 2 & 4 & 2 \\
 \hline
 \hline
 \multicolumn{5}{|c|}{Llama2-7B (FP32 Accuracy = 45.8\%)} & \multicolumn{4}{|c|}{Llama2-70B (FP32 Accuracy = 69.12\%)} \\ 
 \hline
 \hline
 64 & 43.9 & 43.4 & 43.9 & 44.9 & 68.07 & 68.27 & 68.17 & 68.75 \\
 \hline
 32 & 44.5 & 43.8 & 44.9 & 44.5 & 68.37 & 68.51 & 68.35 & 68.27  \\
 \hline
 16 & 43.9 & 42.7 & 44.9 & 45 & 68.12 & 68.77 & 68.31 & 68.59  \\
 \hline
 \hline
 \multicolumn{5}{|c|}{GPT3-22B (FP32 Accuracy = 38.75\%)} & \multicolumn{4}{|c|}{Nemotron4-15B (FP32 Accuracy = 64.3\%)} \\ 
 \hline
 \hline
 64 & 36.71 & 38.85 & 38.13 & 38.92 & 63.17 & 62.36 & 63.72 & 64.09 \\
 \hline
 32 & 37.95 & 38.69 & 39.45 & 38.34 & 64.05 & 62.30 & 63.8 & 64.33  \\
 \hline
 16 & 38.88 & 38.80 & 38.31 & 38.92 & 63.22 & 63.51 & 63.93 & 64.43  \\
 \hline
\end{tabular}
\caption{\label{tab:mmlu_abalation} Accuracy on MMLU dataset across GPT3-22B, Llama2-7B, 70B and Nemotron4-15B models.}
\end{table}


%\subsection{Perplexity achieved by various LO-BCQ configurations on LM evaluation harness}

\begin{table} \centering
\begin{tabular}{|c||c|c|c|c||c|c|c|c|} 
\hline
 $L_b \rightarrow$& \multicolumn{4}{c||}{8} & \multicolumn{4}{c||}{8}\\
 \hline
 \backslashbox{$L_A$\kern-1em}{\kern-1em$N_c$} & 2 & 4 & 8 & 16 & 2 & 4 & 8 & 16  \\
 %$N_c \rightarrow$ & 2 & 4 & 8 & 16 & 2 & 4 & 2 \\
 \hline
 \hline
 \multicolumn{5}{|c|}{Race (FP32 Accuracy = 37.51\%)} & \multicolumn{4}{|c|}{Boolq (FP32 Accuracy = 64.62\%)} \\ 
 \hline
 \hline
 64 & 36.94 & 37.13 & 36.27 & 37.13 & 63.73 & 62.26 & 63.49 & 63.36 \\
 \hline
 32 & 37.03 & 36.36 & 36.08 & 37.03 & 62.54 & 63.51 & 63.49 & 63.55  \\
 \hline
 16 & 37.03 & 37.03 & 36.46 & 37.03 & 61.1 & 63.79 & 63.58 & 63.33  \\
 \hline
 \hline
 \multicolumn{5}{|c|}{Winogrande (FP32 Accuracy = 58.01\%)} & \multicolumn{4}{|c|}{Piqa (FP32 Accuracy = 74.21\%)} \\ 
 \hline
 \hline
 64 & 58.17 & 57.22 & 57.85 & 58.33 & 73.01 & 73.07 & 73.07 & 72.80 \\
 \hline
 32 & 59.12 & 58.09 & 57.85 & 58.41 & 73.01 & 73.94 & 72.74 & 73.18  \\
 \hline
 16 & 57.93 & 58.88 & 57.93 & 58.56 & 73.94 & 72.80 & 73.01 & 73.94  \\
 \hline
\end{tabular}
\caption{\label{tab:mmlu_abalation} Accuracy on LM evaluation harness tasks on GPT3-1.3B model.}
\end{table}

\begin{table} \centering
\begin{tabular}{|c||c|c|c|c||c|c|c|c|} 
\hline
 $L_b \rightarrow$& \multicolumn{4}{c||}{8} & \multicolumn{4}{c||}{8}\\
 \hline
 \backslashbox{$L_A$\kern-1em}{\kern-1em$N_c$} & 2 & 4 & 8 & 16 & 2 & 4 & 8 & 16  \\
 %$N_c \rightarrow$ & 2 & 4 & 8 & 16 & 2 & 4 & 2 \\
 \hline
 \hline
 \multicolumn{5}{|c|}{Race (FP32 Accuracy = 41.34\%)} & \multicolumn{4}{|c|}{Boolq (FP32 Accuracy = 68.32\%)} \\ 
 \hline
 \hline
 64 & 40.48 & 40.10 & 39.43 & 39.90 & 69.20 & 68.41 & 69.45 & 68.56 \\
 \hline
 32 & 39.52 & 39.52 & 40.77 & 39.62 & 68.32 & 67.43 & 68.17 & 69.30  \\
 \hline
 16 & 39.81 & 39.71 & 39.90 & 40.38 & 68.10 & 66.33 & 69.51 & 69.42  \\
 \hline
 \hline
 \multicolumn{5}{|c|}{Winogrande (FP32 Accuracy = 67.88\%)} & \multicolumn{4}{|c|}{Piqa (FP32 Accuracy = 78.78\%)} \\ 
 \hline
 \hline
 64 & 66.85 & 66.61 & 67.72 & 67.88 & 77.31 & 77.42 & 77.75 & 77.64 \\
 \hline
 32 & 67.25 & 67.72 & 67.72 & 67.00 & 77.31 & 77.04 & 77.80 & 77.37  \\
 \hline
 16 & 68.11 & 68.90 & 67.88 & 67.48 & 77.37 & 78.13 & 78.13 & 77.69  \\
 \hline
\end{tabular}
\caption{\label{tab:mmlu_abalation} Accuracy on LM evaluation harness tasks on GPT3-8B model.}
\end{table}

\begin{table} \centering
\begin{tabular}{|c||c|c|c|c||c|c|c|c|} 
\hline
 $L_b \rightarrow$& \multicolumn{4}{c||}{8} & \multicolumn{4}{c||}{8}\\
 \hline
 \backslashbox{$L_A$\kern-1em}{\kern-1em$N_c$} & 2 & 4 & 8 & 16 & 2 & 4 & 8 & 16  \\
 %$N_c \rightarrow$ & 2 & 4 & 8 & 16 & 2 & 4 & 2 \\
 \hline
 \hline
 \multicolumn{5}{|c|}{Race (FP32 Accuracy = 40.67\%)} & \multicolumn{4}{|c|}{Boolq (FP32 Accuracy = 76.54\%)} \\ 
 \hline
 \hline
 64 & 40.48 & 40.10 & 39.43 & 39.90 & 75.41 & 75.11 & 77.09 & 75.66 \\
 \hline
 32 & 39.52 & 39.52 & 40.77 & 39.62 & 76.02 & 76.02 & 75.96 & 75.35  \\
 \hline
 16 & 39.81 & 39.71 & 39.90 & 40.38 & 75.05 & 73.82 & 75.72 & 76.09  \\
 \hline
 \hline
 \multicolumn{5}{|c|}{Winogrande (FP32 Accuracy = 70.64\%)} & \multicolumn{4}{|c|}{Piqa (FP32 Accuracy = 79.16\%)} \\ 
 \hline
 \hline
 64 & 69.14 & 70.17 & 70.17 & 70.56 & 78.24 & 79.00 & 78.62 & 78.73 \\
 \hline
 32 & 70.96 & 69.69 & 71.27 & 69.30 & 78.56 & 79.49 & 79.16 & 78.89  \\
 \hline
 16 & 71.03 & 69.53 & 69.69 & 70.40 & 78.13 & 79.16 & 79.00 & 79.00  \\
 \hline
\end{tabular}
\caption{\label{tab:mmlu_abalation} Accuracy on LM evaluation harness tasks on GPT3-22B model.}
\end{table}

\begin{table} \centering
\begin{tabular}{|c||c|c|c|c||c|c|c|c|} 
\hline
 $L_b \rightarrow$& \multicolumn{4}{c||}{8} & \multicolumn{4}{c||}{8}\\
 \hline
 \backslashbox{$L_A$\kern-1em}{\kern-1em$N_c$} & 2 & 4 & 8 & 16 & 2 & 4 & 8 & 16  \\
 %$N_c \rightarrow$ & 2 & 4 & 8 & 16 & 2 & 4 & 2 \\
 \hline
 \hline
 \multicolumn{5}{|c|}{Race (FP32 Accuracy = 44.4\%)} & \multicolumn{4}{|c|}{Boolq (FP32 Accuracy = 79.29\%)} \\ 
 \hline
 \hline
 64 & 42.49 & 42.51 & 42.58 & 43.45 & 77.58 & 77.37 & 77.43 & 78.1 \\
 \hline
 32 & 43.35 & 42.49 & 43.64 & 43.73 & 77.86 & 75.32 & 77.28 & 77.86  \\
 \hline
 16 & 44.21 & 44.21 & 43.64 & 42.97 & 78.65 & 77 & 76.94 & 77.98  \\
 \hline
 \hline
 \multicolumn{5}{|c|}{Winogrande (FP32 Accuracy = 69.38\%)} & \multicolumn{4}{|c|}{Piqa (FP32 Accuracy = 78.07\%)} \\ 
 \hline
 \hline
 64 & 68.9 & 68.43 & 69.77 & 68.19 & 77.09 & 76.82 & 77.09 & 77.86 \\
 \hline
 32 & 69.38 & 68.51 & 68.82 & 68.90 & 78.07 & 76.71 & 78.07 & 77.86  \\
 \hline
 16 & 69.53 & 67.09 & 69.38 & 68.90 & 77.37 & 77.8 & 77.91 & 77.69  \\
 \hline
\end{tabular}
\caption{\label{tab:mmlu_abalation} Accuracy on LM evaluation harness tasks on Llama2-7B model.}
\end{table}

\begin{table} \centering
\begin{tabular}{|c||c|c|c|c||c|c|c|c|} 
\hline
 $L_b \rightarrow$& \multicolumn{4}{c||}{8} & \multicolumn{4}{c||}{8}\\
 \hline
 \backslashbox{$L_A$\kern-1em}{\kern-1em$N_c$} & 2 & 4 & 8 & 16 & 2 & 4 & 8 & 16  \\
 %$N_c \rightarrow$ & 2 & 4 & 8 & 16 & 2 & 4 & 2 \\
 \hline
 \hline
 \multicolumn{5}{|c|}{Race (FP32 Accuracy = 48.8\%)} & \multicolumn{4}{|c|}{Boolq (FP32 Accuracy = 85.23\%)} \\ 
 \hline
 \hline
 64 & 49.00 & 49.00 & 49.28 & 48.71 & 82.82 & 84.28 & 84.03 & 84.25 \\
 \hline
 32 & 49.57 & 48.52 & 48.33 & 49.28 & 83.85 & 84.46 & 84.31 & 84.93  \\
 \hline
 16 & 49.85 & 49.09 & 49.28 & 48.99 & 85.11 & 84.46 & 84.61 & 83.94  \\
 \hline
 \hline
 \multicolumn{5}{|c|}{Winogrande (FP32 Accuracy = 79.95\%)} & \multicolumn{4}{|c|}{Piqa (FP32 Accuracy = 81.56\%)} \\ 
 \hline
 \hline
 64 & 78.77 & 78.45 & 78.37 & 79.16 & 81.45 & 80.69 & 81.45 & 81.5 \\
 \hline
 32 & 78.45 & 79.01 & 78.69 & 80.66 & 81.56 & 80.58 & 81.18 & 81.34  \\
 \hline
 16 & 79.95 & 79.56 & 79.79 & 79.72 & 81.28 & 81.66 & 81.28 & 80.96  \\
 \hline
\end{tabular}
\caption{\label{tab:mmlu_abalation} Accuracy on LM evaluation harness tasks on Llama2-70B model.}
\end{table}

%\section{MSE Studies}
%\textcolor{red}{TODO}


\subsection{Number Formats and Quantization Method}
\label{subsec:numFormats_quantMethod}
\subsubsection{Integer Format}
An $n$-bit signed integer (INT) is typically represented with a 2s-complement format \citep{yao2022zeroquant,xiao2023smoothquant,dai2021vsq}, where the most significant bit denotes the sign.

\subsubsection{Floating Point Format}
An $n$-bit signed floating point (FP) number $x$ comprises of a 1-bit sign ($x_{\mathrm{sign}}$), $B_m$-bit mantissa ($x_{\mathrm{mant}}$) and $B_e$-bit exponent ($x_{\mathrm{exp}}$) such that $B_m+B_e=n-1$. The associated constant exponent bias ($E_{\mathrm{bias}}$) is computed as $(2^{{B_e}-1}-1)$. We denote this format as $E_{B_e}M_{B_m}$.  

\subsubsection{Quantization Scheme}
\label{subsec:quant_method}
A quantization scheme dictates how a given unquantized tensor is converted to its quantized representation. We consider FP formats for the purpose of illustration. Given an unquantized tensor $\bm{X}$ and an FP format $E_{B_e}M_{B_m}$, we first, we compute the quantization scale factor $s_X$ that maps the maximum absolute value of $\bm{X}$ to the maximum quantization level of the $E_{B_e}M_{B_m}$ format as follows:
\begin{align}
\label{eq:sf}
    s_X = \frac{\mathrm{max}(|\bm{X}|)}{\mathrm{max}(E_{B_e}M_{B_m})}
\end{align}
In the above equation, $|\cdot|$ denotes the absolute value function.

Next, we scale $\bm{X}$ by $s_X$ and quantize it to $\hat{\bm{X}}$ by rounding it to the nearest quantization level of $E_{B_e}M_{B_m}$ as:

\begin{align}
\label{eq:tensor_quant}
    \hat{\bm{X}} = \text{round-to-nearest}\left(\frac{\bm{X}}{s_X}, E_{B_e}M_{B_m}\right)
\end{align}

We perform dynamic max-scaled quantization \citep{wu2020integer}, where the scale factor $s$ for activations is dynamically computed during runtime.

\subsection{Vector Scaled Quantization}
\begin{wrapfigure}{r}{0.35\linewidth}
  \centering
  \includegraphics[width=\linewidth]{sections/figures/vsquant.jpg}
  \caption{\small Vectorwise decomposition for per-vector scaled quantization (VSQ \citep{dai2021vsq}).}
  \label{fig:vsquant}
\end{wrapfigure}
During VSQ \citep{dai2021vsq}, the operand tensors are decomposed into 1D vectors in a hardware friendly manner as shown in Figure \ref{fig:vsquant}. Since the decomposed tensors are used as operands in matrix multiplications during inference, it is beneficial to perform this decomposition along the reduction dimension of the multiplication. The vectorwise quantization is performed similar to tensorwise quantization described in Equations \ref{eq:sf} and \ref{eq:tensor_quant}, where a scale factor $s_v$ is required for each vector $\bm{v}$ that maps the maximum absolute value of that vector to the maximum quantization level. While smaller vector lengths can lead to larger accuracy gains, the associated memory and computational overheads due to the per-vector scale factors increases. To alleviate these overheads, VSQ \citep{dai2021vsq} proposed a second level quantization of the per-vector scale factors to unsigned integers, while MX \citep{rouhani2023shared} quantizes them to integer powers of 2 (denoted as $2^{INT}$).

\subsubsection{MX Format}
The MX format proposed in \citep{rouhani2023microscaling} introduces the concept of sub-block shifting. For every two scalar elements of $b$-bits each, there is a shared exponent bit. The value of this exponent bit is determined through an empirical analysis that targets minimizing quantization MSE. We note that the FP format $E_{1}M_{b}$ is strictly better than MX from an accuracy perspective since it allocates a dedicated exponent bit to each scalar as opposed to sharing it across two scalars. Therefore, we conservatively bound the accuracy of a $b+2$-bit signed MX format with that of a $E_{1}M_{b}$ format in our comparisons. For instance, we use E1M2 format as a proxy for MX4.

\begin{figure}
    \centering
    \includegraphics[width=1\linewidth]{sections//figures/BlockFormats.pdf}
    \caption{\small Comparing LO-BCQ to MX format.}
    \label{fig:block_formats}
\end{figure}

Figure \ref{fig:block_formats} compares our $4$-bit LO-BCQ block format to MX \citep{rouhani2023microscaling}. As shown, both LO-BCQ and MX decompose a given operand tensor into block arrays and each block array into blocks. Similar to MX, we find that per-block quantization ($L_b < L_A$) leads to better accuracy due to increased flexibility. While MX achieves this through per-block $1$-bit micro-scales, we associate a dedicated codebook to each block through a per-block codebook selector. Further, MX quantizes the per-block array scale-factor to E8M0 format without per-tensor scaling. In contrast during LO-BCQ, we find that per-tensor scaling combined with quantization of per-block array scale-factor to E4M3 format results in superior inference accuracy across models. 


\end{document}