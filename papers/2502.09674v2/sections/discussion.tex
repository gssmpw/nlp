\section{Discussion}

% \subsection{Why Residual Space?}
% sdsdsd
\paragraph{Connection with Linear Representation Hypothesis}
Our work builds upon the Linear Representation Hypothesis, which posits that studied features can be expressed through linear projections. Recent works have shown that not all feature directions are linear \cite{engels2024not}. We observe that some directions occasionally flip between different layers, and feature directions cannot be extended indefinitely without degrading generation quality. Neverthless, we identify several linear feature directions in the safety residual space and verify their linearity.

\paragraph{Practical Considerations for Data Complexity}
In this paper, we constructed a dataset consisting of harmful misaligned prompts. However, practical safety alignment data may contain more diverse samples, and the desired behavior is not limited to refusal responses. As data complexity and model size increase, we expect the effective rank of the residual space will also increase, introducing more potential feature directions. While our framework's methodology remains applicable, interpreting these directions becomes more challenging. Future work could address this by analyzing the fine-tuning process in smaller intervals or grouping samples by domain.
