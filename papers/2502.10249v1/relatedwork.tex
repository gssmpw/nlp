\section{Related Work}
Several studies in Software Engineering have explored \textit{burnout} and its implications.

Sonnetag and Brodbeck \cite{sonnentag1994stressor} found a negative correlation between the time spent learning on the job and depersonalization, as well as a negative correlation between communication levels and depersonalization. Additionally, they identified several relationships between environmental factors (e.g., job complexity and work control) and \textit{burnout} in Software Engineering.

Moore \cite{moore2000one} investigated the impact of emotional exhaustion on turnover intention in Software Engineering. The study found a positive relationship between these two variables. Furthermore, Moore observed that autonomy and rewards were negatively correlated with emotional exhaustion, while work overload and role conflict were positively associated with emotional exhaustion.

Singh and Suar \cite{singh2013health} examined the consequences of \textit{burnout} on software engineers’ health, identifying a positive correlation between \textit{burnout} and conditions such as anxiety, depression, and social dysfunction. Their findings underscore the negative impact of \textit{burnout} and the importance of addressing it.

Cook \cite{cook2015job} analyzed the prevalence of \textit{burnout} dimensions in Software Engineering. The study found that cynicism was the most prevalent dimension (43\%), followed by emotional exhaustion (32%). The professional efficacy dimension was measured inversely, meaning higher values indicated lower \textit{burnout}. Notably, all respondents scored 100\% in this dimension, suggesting that software engineers generally perceive themselves as capable in their roles. This finding raises the hypothesis that, in Software Engineering, low efficacy is less problematic than cynicism and emotional exhaustion.

Mellblom and Arason \cite{mellblom2019connection} explored the relationship between personality traits and \textit{burnout}, finding that individuals with higher levels of neuroticism were more prone to \textit{burnout}. However, the study did not specify which \textit{burnout} dimensions were linked to personality traits.

While this research shares similarities with previous studies on \textit{burnout} in Software Engineering, it extends beyond them by investigating the relationship between \textit{burnout} and a characteristic inherent to software projects: change.