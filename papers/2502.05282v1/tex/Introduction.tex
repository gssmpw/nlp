\section{Introduction}
\label{sec:intro}
\begin{figure}
  \centering
  \includegraphics[width=\linewidth]{./picture/intro_CDRL.pdf}
  \caption{The DCRL with the large-scale FP\&N problem. A) The DCRL pulls and pushes the positive and negative feature pairs for consistent or distinct representation. However, B) medical images' properties cause unreliable correspondence discovery, resulting in the open problem of large-scale FP\&N features pairs in DCRL and extremely limiting the representation learning ability.}\label{intro:dcrl}
\end{figure}
\IEEEPARstart{D}{ense} contrastive representation learning (DCRL) \cite{li2021dense,o2020unsupervised,wang2022densecl,wang2022exploring,xie2021propagate,bengio2013representation,you2022simcvd} is crucial for medical image dense prediction (MIDP) tasks, e.g., segmentation \cite{he2022learning,he2021meta,you2022class}. With the increasing demand for deep learning in medical image applications \cite{piccialli2021survey}, the extremely high cost of medical image collection and dense annotation are becoming a large bottleneck \cite{cheplygina2019not}. The DCRL discovers the corresponding pixel\footnote{pixel for 2D and voxel for 3D images, we call "pixel" uniformly}-wise features \cite{milbich2021visual,zhang2022attributable,roth2020pads,you2022momentum} to drive the learning of consistent or distinct representation for them (Fig.\ref{intro:dcrl} A)) which will effectively capture the dense posterior distribution of the underlying explanatory factors for the input. Therefore, it will make models easier to extract useful information \cite{bengio2013representation} when learning MIDP tasks, thus pushing the label and data efficiency to soaring heights in the learning process and coping with the large challenge in the data collection and dense annotation \cite{litjens2017survey}.
\begin{figure}
  \centering
  \includegraphics[width=\linewidth]{./picture/motivation.pdf}
  \caption{The homeomorphism prior enables the pixel-wise correspondence discovery under the condition of medical images' inherent topology, promoting its reliability. A) In topologie, the homeomorphic objects are able to align their topologies via a homeomorphism mapping for point-to-point correspondence with topological preservation. B) Due to the consistency of human body, the medical images are homeomorphic in image space. This provides prior knowledge to construct a deformable mapping for the pixels' correspondence under the condition of their inherent topology, which will effectively reduce the searching space of pairing. C) This gives a potential to enable a reliable pixel-wise correspondence discovery in the medical image DCRL.}\label{intro:1}
\end{figure}

Although some DCRL works \cite{li2021dense,o2020unsupervised,wang2022densecl,wang2022exploring,xie2021propagate} on natural images have been reported, medical images' properties will cause extremely unreliable correspondence discovery (Fig.\ref{intro:dcrl} B)), leading to an open problem of \emph{large-scale false positive and negative} (FP\&N) \emph{feature pairs} \cite{chuang2020debiased,chuang2022robust} in DCRL:

\emph{1) Large-scale false positive (FP) pairs caused by the semantic dependence:} (Fig.\ref{intro:dcrl} B) a)) Some existing works \cite{wang2022densecl,li2021dense} measure the similarity between the pixel-wise features to discover the positive pairs. However, due to the low- and noisy-contrast acquisitions \cite{zhou2019high} of medical images, numerous semantic regions in these images are insignificant (e.g., the soft tissues in CT images). These insignificant regions have large dependence with each other making it a challenge to distinguish them. Therefore, if directly measuring the similarity for all features, the features will be mispaired easily resulting in large-scale FP pairs. Although some other works \cite{o2020unsupervised,xie2021propagate,wang2022exploring} utilize the correspondence from the manual transformation of the same images weakening the FP problem, the diversity of positive pairs is limited (only paired from the same images) and they are still limited by large-scale FN problem.

\emph{2) Large-scale false negative (FN) pairs caused by semantic continuity and semantic overlap:} \textbf{a.} Different from the image-wise contrastive learning \cite{he2020momentum,chen2020simple,chen2020improved,grill2020bootstrap} whose unit is ``instance", the DCRL utilizes the ``pixel" as the unit which continuously arranges on images and constructs semantics via numerous pixels due to the continuity of image signal \cite{he2020momentum} (Fig.\ref{intro:dcrl} B) b)). This makes the semantics continuously distributed on images, so it is unreliable to absolutely divide the pixel-wise features as different semantics on images as negative pairs. \textbf{b.} Some natural DCRL methods \cite{wang2022densecl,li2021dense,wang2022exploring} utilize the features from different images as negative pairs, like momentum contrast mechanism \cite{He2020CVPR,wang2022densecl,li2021dense} which utilizes a memory bank to save previous features as negative samples (Fig.\ref{intro:dcrl} B) c)). However, due to the consistency of human body, medical images have similar global content which shares numerous same anatomies. This makes numerous overlapped semantics between images, so that it will construct numerous negative pairs with the same semantics resulting in FN pairs.

During correspondence discovery, these properties will enlarge the risk of the FP\&N feature pairs \cite{chuang2020debiased,chuang2022robust} resulting in large-scale FP\&N problem. It will train the network's representation to deviate from reality, making the pre-trained network even worse than the randomly initialized network. Therefore, we seek to answer the following question: \emph{How to cope with the FP\&N problem for reliable pixel-wise correspondence in the medical image DCRL?}

\emph{Motivation:} Inspired by topologie \cite{alexandroff2013topologie} (Fig.\ref{intro:1}), homeomorphism \cite{i1996new} between medical images \cite{heimann2009statistical,bazin2008homeomorphic,miller2001group}, e.g., CT, MR, X-ray, provides a prior knowledge for reliable pixel-wise correspondence. An often-repeated mathematical joke is that ``topologists cannot tell the difference between a coffee cup and a donut" \cite{hubbard1991differential} (Fig.\ref{intro:1} A)), because the coffee and donut are homeomorphic (have the same topology) and they are able to transform to each other via a topology-preserved mapping (homeomorphism, a bijective and continuous transformation). The consistency of human genes determines that the human bodies have similar anatomies \cite{netter2014atlas}, for example, human hearts have four chambers with a fixed spatial relationship, and the human brain has a fixed functional regions distribution \cite{bazin2008homeomorphic}. This makes the medical images scanned from the same body ranges have stable similar anatomies \cite{netter2014atlas} with consistent context topology \cite{He_2023_CVPR}, showing the homeomorphic topology (Fig.\ref{intro:1} B)). Therefore, based on the topologie principle and the intrinsic homeomorphic topology of medical images, it will be easy to align the semantic regions via a deformation (a homeomorphism mapping function, we name it deformable mapping in this paper). This makes an effective prior knowledge that enables a reliable pixel-wise correspondence discovery inter images in DCRL under the condition of medical images' inherent topology (Fig.\ref{intro:1} C)), reducing the searching space of constructing the correspondence. Therefore, we hypothesize that \emph{``Embedding this homeomorphism prior knowledge to the medical images DCRL will prompt the pixel-wise correspondence discovery to improve the dense representation.}

However, it is challenging to directly utilize this homeomorphism prior knowledge in DCRL due to: \textbf{1)} Lack of negative pairs: Although the homeomorphism prior enables a reliable pixel-wise correspondence for positive pairs, the non-corresponding positions are unable to be directly divided as negative pairs due to the semantic continuity, limiting the contraction of negative pairs. \textbf{2)} Weak positive pairs: The estimation of the deformation requires a measurement of the alignment degree between images. However, due to the limitation in the insignificant and varied appearance of medical images, the widely used visual similarity \cite{7987758,ba2018un,haskins2020deep,dalca2019unsupervised,He_2023_CVPR}, which utilizes the pixel intensity of the images to measure the alignment degree, will be interrupted. It will limit the deformation accuracy and cause numerous false positive pairs on misalignment regions, resulting in weak learning once very poor alignment occurs.

In this paper, we advance the geometric visual similarity learning (GVSL, our CVPR 2023 work \cite{He_2023_CVPR} discussed in Sec.\ref{sec:adv}) in DCRL, model the homeomorphism behind the GVSL, and propose the \textbf{GE}o\textbf{M}etric v\textbf{I}sual de\textbf{N}se s\textbf{I}milarity (\textbf{GEMINI}) Learning. Its objective is to model the homeomorphism prior to DCRL to cope with the large-scale FP\&N problem in DCRL, thus advancing the effectiveness of dense representation learning for medical images with reliable pixel-wise correspondence discovery. It has two aspects:

\textbf{1) Deformable Homeomorphism Learning (DHL)} for soft learning of feature pairs. Based on the homeomorphism prior, it promotes the GVSL, models that two images have homeomorphic topology and learns to estimate a deformable mapping to align them for pixel-wise correspondence. It consists of two share-weighted representation networks (backbone) and one deformation network (deformer). The deformer is trained on the represented dense features from the backbones to predict a displacement vector flow (DVF, a deformable mapping) which deforms one image to align the other image via moving the pixels. Instead of  directly dividing negative pairs \cite{li2021dense,o2020unsupervised,wang2022densecl,wang2022exploring,xie2021propagate}, the deformer learns to discover the corresponding feature pairs from numerous pixel-wise features in the receptive field. This gradient-driven approach encourages the backbones to extract distinct features for non-corresponding (negative) pairs and consistent features for corresponding (positive) pairs between the image with overlapped semantic regions. Therefore, it will implicitly learn the feature pairs and also avoid the hard division of pixel-wise features as negative pairs, making soft learning for the continuous image signal.

\textbf{2) Geometric Semantic Similarity (GSS)} for reliable learning of positive pairs. Our GSS fuses semantic similarity into the measurement of alignment degree to promote the learning of correspondence for accurate alignment, thus constructing and learning the positive pairs reliably. Due to the representability of the backbones, the extracted features will represent significant semantic information inner their corresponding receptive fields. The same and different semantic regions will have consistent and distinct features, which will bring a more efficient measurement than the original geometric visual similarity (GVS) \cite{He_2023_CVPR}. Therefore, our GSS calculates the distance of the features between images, contributing to the DCRL in two aspects: a) This measurement will consider the significant semantic information and avoid the interference of appearance's limitation, driving an effective learning of correspondence in DHL as a novel loss function. Therefore, it will improve the soft learning of feature pairs in DHL. b) It reliably discovers pixel-wise correspondence under the condition of medical images' inherent topology, reliability enhancing the effectiveness of consistency learning from these positive features for powerful representation.

Finally, our GEMINI trains the backbones to extract consistent and distinct features for the same and different semantic regions, achieving powerful dense representation ability. Based on our GEMINI learning, we implement two practical variant frameworks on two typical representation learning tasks (semi-supervised learning, representation pre-training) \cite{bengio2013representation} in our experiments, proving the powerful ability of this novel and effective learning paradigm. This paper is an extension of the CVPR 2023 conference vision (GVSL), we have detailed discussed the advancement of the preliminary work in Sec.\ref{sec:adv}. This paper makes four significant contributions:
\begin{enumerate}[leftmargin=*]
  \item For the first time, we propose the GEMINI learning which is the first framework for the open problem of large-scale FP\&N pairs in the medical image DCRL. It embeds the homeomorphism prior and constructs a reliable correspondence discovery under the condition of topological preservation, thus building an effective learning of pixel-wise features, and promoting the representation for MIDP tasks.
  \item Our proposed \emph{Deformable Homeomorphism Learning} (DHL) models the homeomorphism prior as the prediction of a deformable mapping which trains the network a distinct representation for non-corresponding regions via gradient, achieving soft learning of negative pairs.
  \item Our proposed \emph{Geometric Semantic Similarity} (GSS) utilizes the semantic information represented in features to efficiently measure the alignment degree and promote the correspondence learning, reliably learning positive pairs.
  \item Based on our GEMINI learning, we implement two practical variant frameworks on two typical representation learning tasks, and our complete experiments on these two tasks on seven datasets demonstrate our powerful representation ability and application potential.
\end{enumerate}

Overall, our GEMINI learning has three key advantages: \textbf{a) Great efficiency:} Our GEMINI captures the posterior distribution of the underlying explanatory factors for the observed input from unlabeled images \cite{bengio2013representation}, improving label and data efficiency to soaring heights in the MIDP learning process. \textbf{b) Higher reliability:} Compared with other DCRL methods \cite{li2021dense,o2020unsupervised,wang2022densecl,wang2022exploring,xie2021propagate}, our homeomorphism prior of medical images significantly promote the correspondence discovery, bringing reliable learning of positive feature pairs and soft learning of negative feature pairs for powerful dense representation learning. \textbf{c) Powerful flexibility:} As a general DCRL paradigm, our GEMINI only needs to make some simple adjustments, e.g., adding the learning loss of MIDP tasks, changing the dimensions of the backbone and deformer networks, etc., to achieve variant frameworks that adapt to different settings and dimensions for efficient learning. In this paper, we provide two variant frameworks on both 2D and 3D experiment settings only via very simple adjustments showing our GEMINI's powerful flexibility.
