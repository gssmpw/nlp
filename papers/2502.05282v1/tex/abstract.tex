
\begin{abstract}
Dense contrastive representation learning (DCRL) has greatly improved the learning efficiency for image dense prediction tasks, showing its great potential to reduce the large costs of medical image collection and dense annotation. However, the properties of medical images make unreliable correspondence discovery, bringing an open problem of \emph{large-scale false positive and negative} (FP\&N) \emph{pairs} in DCRL. In this paper, we propose \textbf{GE}o\textbf{M}etric v\textbf{I}sual de\textbf{N}se s\textbf{I}milarity (\textbf{GEMINI}) learning which embeds the homeomorphism prior to DCRL and enables a reliable correspondence discovery for effective dense contrast. We proposes a deformable homeomorphism learning (DHL) which models the homeomorphism of medical images and learns to estimate a deformable mapping to predict the pixels' correspondence under the condition of topological preservation. It effectively reduces the searching space of pairing and drives an implicit and soft learning of negative pairs via gradient. We also proposes a geometric semantic similarity (GSS) which extracts semantic information in features to measure the alignment degree for the correspondence learning. It will promote the learning efficiency and performance of deformation, constructing positive pairs reliably. We implement two practical variants on two typical representation learning tasks in our experiments. Our promising results on seven datasets which outperform the existing methods show our great superiority. We will release our code at a companion \href{https://github.com/YutingHe-list/GEMINI}{\color{magenta}\emph{website}}. %All of our code will be made publicly available online\footnote{Code: \url{https://github.com/YutingHe-list/GVSL/tree/main/GEMINI}}. %We believe that this novel and effective framework will provide a powerful benchmark for the field of medical image and efficiently reduce the costs of medical image research.
\end{abstract}

\begin{IEEEkeywords}
Medical image analysis, Dense contrastive representation learning, False positive and negative pairs problem, Homeomorphism prior, Correspondence problem
\end{IEEEkeywords} 