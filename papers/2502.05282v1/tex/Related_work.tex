\section{Related Work}
\label{sec:related}
\textbf{1) Correspondence Problem:} Broadly speaking, the correspondence problem is one of the basic problems in cognitive science \cite{nehaniv2002correspondence,brass2005imitation}, machine learning \cite{scholkopf2005object}, and computer vision \cite{he2021few}. It studies the notion of correspondence between two autonomous agents in human cognition \cite{nehaniv2002correspondence}, thus further exploring the social learning, imitation, copying, or mimicry in human activation. These human cognition studies for correspondence have been influential in machine learning in the past several decades \cite{lake2015human,scholkopf2005object}. Numerous machine learning tasks, such as the protein 3D structure prediction from amino acid sequence \cite{jumper2021highly}, the machine translation \cite{wu2016google}, the position alignment between images \cite{he2021few}, etc., are able to be modeled as correspondence problems. In this paper, we limit our scope to visual contrastive representation learning and review the below topics that are relevant to the applications considered in the sequel.

\textbf{2) Dense Contrastive Representation Learning:} DCRL is a typical correspondence problem that learns consistent or distinct representation for pixel-wise features of positive or negative pairs via constructing pixel-wise correspondence \cite{milbich2021visual,zhang2022attributable,roth2020pads}. It will effectively capture the posterior distribution of the underlying explanatory factors and make models easier to extract useful information \cite{bengio2013representation} when learning MIDP tasks. Therefore, the label and data efficiency will be effectively improved to cope with the large challenge of the extremely high cost of medical image collection and pixel-wise annotation. It has three kinds to construct positive and negative pairs. \textbf{a.} The pixel similarity-based methods \cite{wang2022densecl,li2021dense,xie2021propagate,chen2021unsupervised,gao2022unsupervised} measure the Mahalanobis \cite{de2000mahalanobis} or Euclidean \cite{wang2005euclidean} distance between pixel-wise features for the correspondence. However, the low-contrast medical images limit the discrimination of features making the measurement unreliable and constructing FP\&N pairs. \textbf{b.} The location-based methods take the shared part of two cropped patches from one image\cite{o2020unsupervised} or the same position between two medical images \cite{chaitanya2020contrastive} as the positive pairs and the features from different images or different positions as the negative pairs, avoiding the limitation of mismeasurement. However, due to the consistency of image content, the same semantic pixel regions are widely existing in different images making a serious false negative problem, and interfering with the representation learning process. \textbf{c.} The attention-based method \cite{wang2022exploring} utilizes the attention maps to extract positive pairs and their negative pairs are still directly paired from different images. It relies on a large dataset to train an attention prediction model, once the dataset is not large enough, the inaccurate attention will bring numerous mis-correspondence. The unreliable negative pairs also bring large limitations.

\textbf{3) False Positive and Negative Pairs Problem:} FP\&N Problem \cite{chuang2020debiased,chuang2022robust} is one of the key open problems in contrastive representation learning \cite{Chen2021CVPR,grill2020bootstrap,chen2020simple,He2020CVPR,caron2018deep} and the existing works focus on FP\&N problem in image-wise. FP\&N problem is caused by the mis-correspondence of feature pairs which makes the networks learn distinct representations for the same-semantic pairs and consistent representations for the different-semantic pairs. The network will learn the inaccurate posterior distribution which is contrary to the underlying explanatory factors, extremely limiting the learning of practical tasks. Some methods \cite{grill2020bootstrap,xie2021propagate} remove the construction of negative pairs, and only learn the positive pairs constructed by different views of one image to avoid the FP\&N problem. However, it will bring the risk of dimensional collapse \cite{jing2021understanding} which makes the network unable to represent the information in images. Some other methods \cite{chuang2020debiased,chuang2022robust,huynh2022boosting} construct FP\&N-robust losses or FP\&N-discovery mechanisms to reduce the interference of inaccurate feature pairs, and have achieved promising results. However, these methods are sensitive to their additional hyper-parameters, and these hyper-parameters have to be carefully adjusted for their effectiveness. There is no success reported to cope with the FP\&N problem in the medical image DCRL whose special properties (continuous image signal, low contrast, varied appearance, consistent image content) bring more challenges in correspondence.

\textbf{4) Homeomorphism in Medical Images Analysis:} Homeomorphism is a powerful prior for medical image learning. As introduced in Sec.\ref{sec:intro}, this property comes from the consistency of human anatomy \cite{netter2014atlas}, and numerous medical image categories, e.g., CT, MR, X-ray, etc., inherit it. Therefore, using homeomorphism as prior knowledge, some medical image works have been studied. One of the classic applications is the registration \cite{he2021few,7987758,ba2018un,haskins2020deep,dalca2019unsupervised}. According to this prior, it aligns the anatomical structures in medical images to the same spatial coordinate system, so that the images' consistent anatomies will be aligned in image space. Based on the registration, some works \cite{bazin2008homeomorphic,6226425,8014481,wang2020lt,BalakrishnanVoxelMorph(u)} further study the atlas-based segmentation methods. They align the labeled images to unlabeled images, thus indirectly mapping the labels to unlabeled images for segmentation results. These methods have very small label requirements and large robustness due to homeomorphism. With the development of deep learning \cite{lecun2015deep,shen2017deep}, the homeomorphism prior is further promoting the medical image learning, e.g., few-shot segmentation \cite{he2022learning,he2020deep,ding2021modeling,zhao2019data,xu2019deepatlas}. They have effectively improved the learning efficiency and robustness in these tasks, owing to the large contribution from the homeomorphism prior. \textbf{However}, due to the limitations illustrated in Sec.\ref{sec:intro}, it is still challenging to embed this great prior into the DCRL tasks and there are no preliminary studies reported. 