\section{Conclusion}
\label{sec:conclusion}
In this paper, we have advanced the homeomorphism prior in the open problem of large-scale FP\&N pairs in the medical image DCRL, and proposed the \textbf{GE}o\textbf{M}etric v\textbf{I}sual de\textbf{N}se s\textbf{I}milarity (\textbf{GEMINI}) Learning for a reliable dense correspondence discovery and learning. Based on our GEMINI, dense contrastive representation for medical images is learned, effectively reducing the data and annotation costs in medical image dense prediction tasks. Its unique properties of learning implicit negative pairs in our DHL and positive pairs in our GSS have bright powerful performance in few-shot semi-supervised medical image segmentation tasks and self-supervised medical image pre-training tasks. We believe that our GEMINI in DCRL will promote the research of efficient learning in medical image analysis, and coping with the large challenge in data collection and dense annotation. For intuition progress, the objects with the homeomorphic property are all able to construct a reliable point-to-point correspondence discovery via a homeomorphism mapping. This effectively promotes a new representation learning paradigm based on topological consistency, and will inspire future researchers for more powerful innovations.

%\textbf{Conclusion for limitation} There are still some limitations in our GEMINI. 1) The additional calculation in the Z-Matching head and the fundamental self-restoration learning makes larger GPU memory requirement and more computing costs. 2) The inter-scene transferring is still a large challenge for medical image pre-training models due to the large gap between the source and target scenes. Fortunately, these limitations are gradually being solved due to the development of GPU and enlarging of medical datasets (provides more possibilities for inner-scene transferring). 