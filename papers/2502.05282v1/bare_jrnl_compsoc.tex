

\documentclass[10pt,journal,compsoc]{IEEEtran}

\ifCLASSINFOpdf
  % \usepackage[pdftex]{graphicx}
  % declare the path(s) where your graphic files are
  % \graphicspath{{../pdf/}{../jpeg/}}
  % and their extensions so you won't have to specify these with
  % every instance of \includegraphics
  % \DeclareGraphicsExtensions{.pdf,.jpeg,.png}
\else
  % or other class option (dvipsone, dvipdf, if not using dvips). graphicx
  % will default to the driver specified in the system graphics.cfg if no
  % driver is specified.
  % \usepackage[dvips]{graphicx}
  % declare the path(s) where your graphic files are
  % \graphicspath{{../eps/}}
  % and their extensions so you won't have to specify these with
  % every instance of \includegraphics
  % \DeclareGraphicsExtensions{.eps}
\fi

\hyphenation{op-tical net-works semi-conduc-tor}
\usepackage{booktabs}
\usepackage{multirow}
\usepackage{arydshln}
\usepackage{makecell}
\usepackage{enumitem}
\usepackage{diagbox}
%\usepackage{wasysym}
\usepackage{amssymb}
\usepackage{graphicx}
\usepackage{soul}
\usepackage{url}
\usepackage{amsmath}
\usepackage{cite}
\usepackage{amsthm}
\usepackage{hyperref}

\usepackage{algorithm}
\usepackage{algorithmic}
\usepackage{multirow}
\usepackage{marvosym}
\usepackage{threeparttable}
\usepackage{array,subfig}

\usepackage{mathrsfs}
\usepackage{color,xcolor}
\usepackage{colortbl}
\begin{document}
\title{Homeomorphism Prior for False Positive and Negative Problem in Medical Image Dense Contrastive Representation Learning}

\author{Yuting He,
        Boyu Wang,
        Rongjun Ge,
        Yang Chen,
        Guanyu Yang\IEEEauthorrefmark{1},
        Shuo Li
\thanks{\vspace*{-1\baselineskip} \newline {\emph{\IEEEauthorrefmark{1}Corresponding authors: G. Yang. (e-mail: yang.list@seu.edu.cn)}}}
\thanks{This research was supported by the National Natural Science Foundation of China (Grant No. 82441021), the Natural Science Foundation of Jiangsu Province (Grant No. BK20210291), the National Natural Science Foundation of China (Grant No. 62101249, T2225025), the Jiangsu Shuangchuang Talent Program (Grant No. JSSCBS20220202).}
\IEEEcompsocitemizethanks{
\IEEEcompsocthanksitem Y. He, Y. Chen and G. Yang\IEEEauthorrefmark{1} are with the Key Laboratory of New Generation Artificial Intelligence Technology and Its Interdisciplinary Applications (Southeast University), Ministry of Education, Nanjing, China, the Centre de Recherche en Information Biomédicale Sino-Français (CRIBs), and Jiangsu Provincial Joint International Research Laboratory of Medical Information Processing, Nanjing, China. (e-mail: yang.list@seu.edu.cn)
\IEEEcompsocthanksitem R. Ge is with the School of Instrument Science and Engineering, Southeast University, Nanjing, China (e-mail: rongjun\_ge@seu.edu.cn)
\IEEEcompsocthanksitem B. Wang is with the Department of Computer Science, Western University, London, ON N6A 3K7, Canada. (e-mail: bwang@csd.uwo.ca)
\IEEEcompsocthanksitem S. Li and Y. He are with the Department of Biomedical Engineering and the Department of Computer and Data Science, Case Western Reserve University, Cleveland, OH 44106 USA (e-mail: shuo.li11@case.edu).
}% <-this % stops an unwanted space
\thanks{Manuscript received April 19, 2005; revised August 26, 2015.}}

% The paper headers
\markboth{Journal of \LaTeX\ Class Files,~Vol.~14, No.~8, August~2015}%
{Shell \MakeLowercase{\textit{et al.}}: Bare Demo of IEEEtran.cls for Computer Society Journals}

\IEEEtitleabstractindextext{%

\begin{abstract}
Retrieval-Augmented Generation (RAG) is often used with Large Language Models (LLMs) to infuse domain knowledge or user-specific information. In RAG, given a user query, a retriever extracts chunks of relevant text from a knowledge base. These chunks are sent to an LLM as part of the input prompt. Typically, any given chunk is repeatedly retrieved across user questions. However, currently, for every question, attention-layers in LLMs fully compute the key values (KVs) repeatedly for the input chunks, as state-of-the-art methods cannot reuse KV-caches when chunks appear at arbitrary locations with arbitrary contexts. Naive reuse leads to output quality degradation.  This leads to potentially redundant computations on expensive GPUs and increases latency. In this work, we propose \sys, a system for managing and reusing precomputed KVs corresponding to the text chunks (we call \textit{chunk-caches}) in RAG-based systems. We present how to identify \hl{\textit{chunk-caches} that are reusable}, how to efficiently perform a small fraction of recomputation to \textit{fix} the cache to maintain output quality, and how to efficiently store and evict \textit{chunk-caches} in the hardware for maximizing reuse while masking any overheads. With real production workloads as well as synthetic datasets, we show that \sys reduces redundant computation by \textbf{51\%} over SOTA prefix-caching and \textbf{75\%} over full recomputation.
\hl{Additionally, with continuous batching on a real production workload, we get a \textbf{1.6$\times$} speedup in throughput and a \textbf{2$\times$} reduction in end-to-end response latency over prefix-caching while maintaining quality, for both the \llama-3-8B and \llama-3-70B models. 
}
\end{abstract}





}

\maketitle

\IEEEdisplaynontitleabstractindextext
\IEEEpeerreviewmaketitle


% 
% 
The widespread integration of communication networks and smart devices in modern control systems has increased the vulnerability of industrial systems to online cyber-attacks, e.g., Industroyer, Blackenergy, etc \citep{osti_1505628}.
% Modern control systems have seen a large push to include communication networks and smart devices to increase performance, made possible by improvements in communication device cost and energy consumption. This trend has been coupled with the usage of open-standard communication protocols among industrial control systems, making them vulnerable to online cyber-attacks such as Industroyer, Blackenergy, etc \citep{osti_1505628}. 
To counter this, methods have been developed to improve security by achieving attack detection, mitigation, and monitoring, among others \citep{sandberg2022secure}. This paper focuses on active attack diagnosis to mitigate stealthy attacks. 
%
%\subsection{Literature review}

Active diagnosis techniques rely on the inclusion of additional moduli to control systems
% inclusion within the control system of additional moduli 
to alter the behavior of the system compared to information known by the attacker. 
For instance, the concept of additive watermarking was introduced in \cite{mo2015physical}, where noise signals of known mean and variance are added at the plant and compensated for it at the controller. 
This compensation, however, is not exact, causing some performance degradation. Thus, trade-offs between performance and detectability  are necessary \citep{zhu2023detection}.
% A later work \citep{zhu2023detection} designs the watermark signal by trading performance for detection. Thus, although additive watermarking serves as a good detection scheme, they endure performance losses even in the nominal case. 

In encrypted control \citep{darup2021encrypted}, the sensor data is encrypted, sent to the controller, and then operated on directly. Encrypted input signals are sent back to the plant for decryption. Although encryption is widespread in IT security, in control systems it presents some concerns, such as the introduction of time delays \citep{stabile2024verifiable}, while it may present inherent weaknesses \citep{alisic2023model}.
% they are not preferred as they introduce time delays \citep{stabile2024verifiable} which can cause instability, and some encryption schemes can be very weak  \citep{alisic2023model}. 

In moving target defense \citep{griffioen2020moving}, the plant is augmented with fictitious dynamics, known to the controller. The plant output is transmitted to the controller along with the fictitious states over a network under attack. 
The additional measurements then aide in the detection of attacks. 
This comes at the cost of higher communication bandwidth needs, which increases rapidly with the dimension of the augmented systems.
% Since the dynamics of the fictitious dynamics are exactly known to the controller, the attack is detected easily. However, when the scale of the system increases, the communication bandwidth used by moving the target defense approach increases rapidly. 

Other recently proposed works include two-way coding \citep{fang2019two}, a weak encryuption technique, and dynamic masking \citep{abdalmoaty2023privacy}, which enhances privacy as well as security, have been shown to be effective against zero-dynamics attacks.
% Two-way coding \citep{fang2019two} and dynamic masking \citep{abdalmoaty2023privacy} are other recently proposed approaches. Two-way coding is another form of weak encryption technique whilst dynamic masking proposes an architecture that enhances both privacy and security. These schemes are shown to be effective against zero dynamics attacks but remain to be studied for other classes of attacks. 
% Recent extensions include \citep{mukherjee2021secure,ramos2024privacy}.
% Some other works which are related are \citep{mukherjee2021secure}, an extension of \cite{fang2019two}. The work \citep{ramos2024privacy} is an extension of moving target defense for multi-agent systems. 
Furthermore, filtering techniques for attack detection are proposed by \cite{murguia2020security,hashemi2022codesign,escudero2023safety}, while not focusing on stealthy attacks.
% The works \citep{murguia2020security,hashemi2022codesign,escudero2023safety} develop filtering techniques to guarantee safety, without being focused on stealthy covert attacks.

Multiplicative watermarking (mWM) has been proposed by the authors as a diagnosis technique \citep{ferrari2020switching}. mWM consists of a pair of filters on each communication channel between the plant and its controller; the scheme is affine to weak encryption, whereby ``encoding'' and ``decoding'' are done by changing signals' dynamic characteristics through inverse pairs of filters. This enables original signals to be recovered exactly, and thus does not lead to performance degradation.
% A multiplicative watermark is an affine to a weak encryption technique, through which the signal is ``encoded'' by a filter, changing its dynamic behavior. The use of inverse pairs means that the original signal can be recovered, through ``decoding'' via an inverse filter. As such, differently to techniques based on additive watermarking, no performance is lost due to the injection of noise, and there are no bandwidth limitations.

%\subsection{Contributions}
One of the critical features of multiplicative watermarking is that to detect stealthy attacks, the mWM filter parameters must be switched over time. In this paper, an algorithm to optimally design the mWM parameters after a switching event is presented, enhancing detection performance, without changing the switching time.
% This is done without changing the switching time, which is taken as given.

\textcolor{black}{
To formalize the filter design problem, we suppose the defender is interested in optimal performance against adversaries injecting covert attacks with matched system parameters \citep{smith2015covert}, including the mWM parameters prior to the switch. This scenario represents a worst case where malicious agents can take full control of the system while remaining undetected.
Thus, the attack strategy is explicitly included within the formulation of the closed-loop system, and the mWM filters are chosen by solving an optimization problem minimizing the attack-energy-constrained output-to-output gain (AEC-OOG) \citep{anand2023risk}, a variation of the output-to-output gain proposed in  \cite{teixeira2015strategic}.
}
The main contributions of this paper are:
% We consider an adversary injecting a covert attack with matched system parameters \citep{smith2015covert}, i.e., an attacker with full knowledge of the control system parameters, including those of the mWM filters before the switch. This scenario is taken as a worst case, as it has been shown that this class of attacks can be made stealthy. To quantitatively define a cost, the output-to-output gain (OOG) \citep{teixeira2015strategic} is leveraged,
% a metric introduced to evaluate the impact of an additive attack in a control system. %Specifically, OOG evaluates the worst-case performance loss that an attacker injecting an undetectable attack can obtain. 
% Here, the maximum performance loss caused by a stealthy adversary with limited energy is taken, the attack-energy-constrained OOG (AEC-OOG) \citep{anand2023risk}. The main contributions of this paper are:
\begin{enumerate}
%[label=\alph*.]
\item The problem of optimally designing the switching mWM filters is formulated as an optimization problem, with the AEC-OOG is taken as the objective;%where the AEC-OOG is taken as the impact metric; 
\item The worst-case scenario of a covert attack with exact knowledge of plant and mWM filter parameters is embedded within the design problem;
% The optimization problem is defined to incorporate the worst-case scenario of a covert attack with exact knowledge of plant and mWM filter parameters;
\item The feasibility of the optimization problem is shown to be dependent only on stability conditions; 
\item A solution scheme is proposed to promote randomization of the mWM filter parameters such that an eavesdropping adversary cannot remain stealthy.
\end{enumerate} 

This builds on the results of \cite{ferrari2020switching}, where the focus was on the design of the switching protocols, rather than the parameters themselves.
Compared to previous work \citep{gallo2021design}, this paper introduces an optimization problem which is always feasible (thanks to the use of AEC-OOG in the objective), while also considering a more sophisticated class of covert attacks, where the presence of watermark is known to the adversary. 
Moreover, this paper poses a different objective than \citep{zhang2023hybrid}; indeed, while \citep{zhang2023hybrid} provided a design strategy to ensure certain privacy properties, in this paper we address the problem of optimal parameter design following a switching event.


%\subsection{Organization}
The rest of the paper is organized as follows. 
After formulating the problem in Section~\ref{sec:PF}, we propose our design algorithm in Section~\ref{sec:main}, and analyze its properties. It is then evaluated through a numerical example in Section~\ref{sec:NE}, and concluding remarks are given Section~\ref{sec:Con}.
% We provide the problem background in Section~\ref{sec:PF}. We formulate the design problem in Section~\ref{sec:main}, together with an analysis of its properties. The proposed algorithm is evaluated through a numerical example in Section \ref{sec:NE}. Concluding remarks are offered in Section \ref{sec:Con}.
\section{Related Work}
In this section, we provide a broad overview of self-supervised learning research that has inspired our work, along with recent trends in image clustering using pre-trained models.


\subsection{Self-Supervised Learning}
Self-supervised learning learns representations from data without explicit labels. The objective is to create a representation space where positive pairs are closer together, while negative pairs are pushed farther apart \cite{geiping2023cookbook}.

SimCLR \cite{chen2020simple} uses data augmentations, such as flipping and colour jittering, to create positive and negative pairs for optimizing objectives. It also introduces a projection head that maps embeddings into a space where contrastive loss is applied. BYOL \cite{grill2020bootstrap} shows that high-quality representations can be learned by simply maximizing agreement between two augmented views of the same input, without requiring negative pairs. Building on these advancements, SimSiam \cite{chen2020exploringsimplesiameserepresentation} eliminates the need for both negative pairs and momentum encoders by introducing a stop-gradient operation, which effectively prevents representational collapse. Inspired by these methods, we adopt similar ideas to develop a simple and effective self-supervised framework for image clustering.

\subsection{Pre-trained Models in Vision} 
Building on advances in self-supervised learning, CLIP \cite{radford2021learning} introduced a paradigm of contrastive pre-training that aligns images with corresponding textual descriptions. This approach enables broad task generalization without task-specific fine-tuning. DINO \cite{caron2021emerging}, which stands for self-distillation with no labels, demonstrates a self-supervised method for optimizing a student network from a teacher network based on vision input data only.

One of the key advantages of pre-trained models like CLIP is their ability to eliminate the need for training models from scratch for downstream tasks, significantly reducing computational costs and time. Instead of training a self-supervised neural network from the ground up, pre-trained models provide high-quality feature representations out of the box, leading to faster experimentation and improved performance on a variety of tasks. The scalability of CLIP has been further validated by openCLIP \cite{Cherti_2023}, which extended CLIP using the larger Vision Transformer models \cite{dosovitskiy2020image}. Similarly, models such as DINO~\cite{9709990} and DINOv2~\cite{oquab2024dinov2learningrobustvisual} are capable of processing visual data and mapping it to high-quality latent representations.

\subsection{Image Clustering via Pre-trained Models}
To address the challenges of scaling to modern image datasets, methods such as NMCE \cite{li2022neural} and MLC \cite{deng2023acp} have integrated deep learning with manifold clustering using the minimum coding rate principle \cite{Arthur_Vassilvitskii_2007}. Building on this idea, CPP \cite{chu2024image} further refines CLIP features and estimates the optimal number of clusters when unknown. TEMI \cite{adaloglou2023exploring} improves clustering by leveraging associations between image features, introducing a variant of pointwise mutual information with instance weighting. Unlike our approach, TEMI utilizes a nearest-neighbors set and an exponential moving average for parameter optimization.

SIC \cite{cai2023semantic} leverages multi-modality by mapping images to a semantic space and generating pseudo-labels based on image-semantic relationships. More recently, TAC~\cite{li2023image} utilizes the textual semantics of WordNet~\cite{miller1995wordnet} to enhance image clustering by selecting and retrieving nouns that best distinguish the images, facilitating collaboration between text and image modalities through mutual cross-modal neighborhood distillation.

Current pre-trained approaches often rely on heavy or complex architectures to ensure consistency, motivating us to develop a simple yet effective pipeline for image clustering. Our method requires only a simple clustering head and basic data augmentations, demonstrating strong competitiveness among recent models.








\section{Methodology}
In this section, we outline the key research questions driving this study, followed by a detailed description of the methodology used to design and conduct the survey.
\subsection{Research Questions}
\begin{enumerate}
    \item[\textbf{RQ1:}] How do developers allocate their time during a typical workweek, and how does this compare to their perception of an \textbf{ideal workweek?}
    \item[\textbf{RQ2:}] How are developer's satisfaction and productivity affected by \textbf{deviations} from their ideal workweek?
     \item[\textbf{RQ3:}] For which tasks do developers prefer using \textbf{AI tools}, and how does the frequency of AI tool usage \textbf{influence} their satisfaction and productivity?
\end{enumerate}

\subsection{Survey Design}
% Describe how the survey was conducted, survey structure, sample size, which activities were selected and how, incentives, etc. 

To gain insights into the types of activities developers engage in during a typical work week, we conducted a series of exploratory interviews with 12 randomly selected participants. These semi-structured interviews provided a qualitative foundation, allowing us to iteratively develop a comprehensive list of higher-level activities that reflect both ideal and actual workweek allocations. The findings from these interviews were instrumental in refining our survey questions and design.

% - When was it distributed
% - How many people were invited
% - how was the survey advertised
% - incentive provided to participants
% - how many responses received (with response rates)
% - Board of ethics description \& instruments
% - Describe the main questions asked in the survey

The survey was distributed in \textcolor{blue}{May 2024} to software engineers working in Microsoft teams across India and the United States. A total of 6000 developers were invited to participate via email. Framed as a study aimed at boosting developer productivity by understanding how they allocate their time in a workday, the survey received 510 complete responses (responses rate of 8.5\%). After finishing the survey, the participants could enter a sweepstake to win one out of ten \$50 Amazon.com Gift Cards.
\textcolor{blue}{description of ethics}.

The main questions in the survey were as follows:
\begin{enumerate}
    \item Their roles and years of experience in the industry/team
    \item The hours spent on various activities in their typical workweek
    \item Ideally, the percentage of time they would want to allocate to each activity in a workweek
    \item How productive and satisfied were they by their past workweek
    \item Activities they find most cognitively challenging
    \item How often do they use AI tools to assist in their daily activities
    \item Two open-ended questions about the activities they would want to automate using AI tools, and advice for new hires to boost their productivity and satisfaction levels 
\end{enumerate}



\subsection{Data Analysis \& Exploration}
% Here, we could start with discussing the survey group:
% - demographic observations
% - distribution of participants (based on the years experience in the industry/team), 

From the exploratory interviews, we identified sixteen key activities, which were subsequently used to quantify the developers' time allocation across their work week. 

\subsection{Limitations}

\section{Experiment 1: Few-shot Semi-supervised Medical Image Segmentation (FS-Semi)}
\label{sec:task2}
We implement our GEMINI learning on few-shot semi-supervised (FS-Semi) medical image segmentation (GEMINI-Semi) providing a variant on the situation that labels are very few. Three public-available tasks are enrolled in our experiments for a very complete evaluation.
\subsection{Experiments configurations}
\label{sec:configurations2}
\subsubsection{Variant design} The variant of our GEMINI-Semi learns a segmentation head $Seg_{\kappa}$ on the extracted dense features $f^{A},f^{B}$. Therefore, except the optimization for deformable homeomorphism learning $\mathcal{L}_{DHL}$, the GEMINI-Semi also has an additional optimization for segmentation $\mathcal{L}_{Seg}$:
\begin{equation}\label{equ:variant2}
\underset{\xi,\theta,\kappa}{\arg\min}\ (\mathcal{L}_{DHL}(\theta,\xi,\mathcal{S}_{ul})+\mathcal{L}_{Seg}(\theta,\kappa,\mathcal{S}_{l})),
\end{equation}
where the $\mathcal{S}_{ul}$ and the $\mathcal{S}_{l}$ are the unlabeled dataset and the labeled dataset. In our experiment, we utilize the sum of Dice loss and cross-entropy loss \cite{ma2021loss} to train segmentation objective $\mathcal{L}_{Seg}$. The other compared DCRL methods (Sec.\ref{sec:comparison2}) also use the same setting as this variant which adds the $\mathcal{L}_{Seg}$ in the training to learn segmentation.
\begin{table}
  \centering
  \caption{Total seven publicly available datasets are involved in this paper for the experiments of our GEMINI's variants, achieving great reproducibility.}\label{dataset}
\resizebox{\linewidth}{!}{
  \begin{tabular}{lccccccccc}
  \toprule
  \textbf{Dataset}                       &\textbf{Type}    &\textbf{Num}  &\textbf{FS-Semi} &\textbf{SS-MIP}\\
  \midrule
  %\midrule
  ASOCA \cite{gharleghi2022automated}    &3D cardiac CT    &60            &$\surd$          &\\
  CAT08 \cite{schaap2009standardized}    &3D cardiac CT    &32            &$\surd$          &\\
  WHS-CT \cite{zhuang2019evaluation}     &3D cardiac CT    &60            &$\surd$          &\\
  CANDI \cite{kennedy2012candishare}     &3D brain MRI     &103           &$\surd$          &$\surd$\\
  SCR \cite{van2006segmentation}         &2D chest X-ray   &247           &$\surd$          &$\surd$\\
  KiPA22 \cite{he2021meta}               &3D kidney CT     &130           &                 &$\surd$\\
  %CARDIAC               &3D cardiac CT              &302                 &                 &$\surd$\\
  ChestX-ray8 \cite{wang2017chestx}      &2D chest X-ray   &112,120       &                 &$\surd$\\
  \bottomrule
  \end{tabular}}
\end{table}

\subsubsection{Datasets} We evaluate GEMINI on three public tasks in 2D and 3D dimensions, showcasing its powerful representation ability in semi-supervised tasks \cite{you2024mine,you2024rethinking} with minimal labels (Tab.\ref{dataset}). \textbf{Task 1: FS-Semi cardiac structure segmentation (3D)} targets seven cardiac structures on 3D CT images, combining WHS-CT \cite{zhuang2019evaluation} (20 labeled, 40 unlabeled), ASOCA \cite{gharleghi2022automated} (60 unlabeled), and CAT08 \cite{schaap2009standardized} (32 labeled from\footnote{\url{http://www.sdspeople.fudan.edu.cn/zhuangxiahai/0/mmwhs/}}). Images are cropped and resampled to $144\times144\times128$, with a five-shot evaluation (5, 100, and 47 images as labeled training, unlabeled training, and testing sets). \textbf{Task 2: FS-Semi brain tissue segmentation (3D)} involves 27 brain tissues on 3D T1 MR images from the CANDI dataset \cite{kennedy2012candishare} (103 labeled). Cropped volumes of $160\times160\times128$ undergo five-shot evaluation (5, 78, and 20 images as labeled training, unlabeled training, and testing sets). \textbf{Task 3: FS-Semi chest structure segmentation (2D)} focuses on three chest-related structures in 2D chest X-rays using the SCR dataset \cite{van2006segmentation} (247 labeled) whose images are from the JSRT database \cite{shiraishi2000development}, split into 5 labeled, 142 unlabeled, and 100 testing images for five-shot evaluation. All tasks use rotation [$-20^\circ$, $20^\circ$] and scaling [0.75, 1.25] for data augmentation.

\subsubsection{Comparison setting} \label{sec:comparison2}
We compare GEMINI-Semi with 19 widely-used methods and our GVSL \cite{He_2023_CVPR} (CVPR 2023) to demonstrate its superiority. \textbf{1)} We train a U-Net \cite{ronneberger2015u} to establish upper and lower bounds using 5 labeled images (Five) and all labeled training data (Full). \textbf{2) Semi-supervised methods} without homeomorphism prior (UA-MT \cite{yu2019uncertainty}, MASSL \cite{chen2019multi}, DPA-DBN \cite{he2020dense}, CPS \cite{chen2021semi}) highlight the significance of prior knowledge for semi-supervised learning with limited labels. \textbf{3) Atlas-based methods} with homeomorphism prior (VM \cite{ba2018un}, LC-VM \cite{BalakrishnanVoxelMorph(u)}, LT-Net \cite{wang2020lt}) illustrate the limitation caused by the inefficient correspondence learning. \textbf{4) Learning registration to learn segmentation methods} with homeomorphism prior (DeepAtlas \cite{xu2019deepatlas}, DataAug \cite{zhao2019data}, DeepRS \cite{he2020deep}, PC-Reg-RT \cite{he2021few}, BRBS \cite{he2022learning}) show gains from improved correspondence but are limited by pseudo-labels from unreliable GVS. \textbf{5) Dense contrastive representation learning methods} without homeomorphism prior (VADeR \cite{o2020unsupervised}, GLCL \cite{chaitanya2020contrastive}, DSC-PM \cite{li2021dense}, PixPro \cite{xie2021propagate}, DenseCL \cite{wang2022densecl}, SetSim \cite{wang2022exploring}) reveal FP\&N problem in DCRL. For fairness, all methods use 2D/3D U-Net \cite{ronneberger2015u} with group normalization \cite{wu2018group} as the backbone.

\subsubsection{Implementation and evaluation metrics} In this task, our GEMINI-Semi is implemented by PyTorch \cite{paszke2019pytorch} on NVIDIA GeForce RTX 3090 GPUs with 24 GB memory. We take Adam whose learning rate is $1\times10^{-4}$ to optimize our framework for fast convergence. For task 1 and task 2, we sample two unlabeled images and one labeled image randomly in each iteration to save the memory for large 3D images, and for task 3, we sample 10 unlabeled images and 5 labeled images randomly in each iteration for 2D images. Following \cite{he2022learning}, we perform an affine transformation on these images via AntsPy\footnote{\url{https://github.com/ANTsX/ANTsPy}} to normalize the spatial system. We utilize the DSC [\%] to evaluate the area-based overlap index and the average Hausdorf distances (AVD) to evaluate the coincidence of the surface \cite{taha2015metrics}.

\subsection{Results and Analysis}
\label{sec:results2}
\begin{table*}
\centering
\caption{The quantitative evaluation demonstrates our powerful representation ability in FS-Semi tasks. Our GEMINI-Semi achieves the best performance on CT, MR, and X-ray images compared with 19 popular methods and the GVSL. The ``unable" means that the extremely poor results make the AVD unable to be calculated. The ``-" means that the setting is unable to be implemented. The ``HP" means these methods have or do not have homeomorphism prior. ``T1", ``T2", ``T3" are the task 1, task 2, task 3. The red and blue values are the highest and the second-highest values in the columns.}
\resizebox{\textwidth}{!}{
\begin{tabular}{clccccccccccccccc}
  \toprule
  \multirow{2}{*}{\textbf{Type}}
  &\multirow{2}{*}{\textbf{Method}}
  &\multirow{2}{*}{\textbf{HP}}
  &\multicolumn{2}{c}{\textbf{T1: 3D cardiac structures}}
  &\multicolumn{2}{c}{\textbf{T2: 3D brain tissues}}
  &\multicolumn{2}{c}{\textbf{T3: 2D chest structures}}
  &\textbf{AVG}\\ \cmidrule(r){4-5}\cmidrule(r){6-7}\cmidrule(r){8-9}\cmidrule(r){10-10}
  &
  &
  &DSC$_{\pm std}\uparrow$
  &AVD$_{\pm std}\downarrow$
  &DSC$_{\pm std}\uparrow$
  &AVD$_{\pm std}\downarrow$
  &DSC$_{\pm std}\uparrow$
  &AVD$_{\pm std}\downarrow$
  &DSC$_{\pm std}\uparrow$
  \\
  \midrule
  Full
  &U-Net \cite{ronneberger2015u}
  &$\times$
  &-
  &-
  &88.7$_{\pm1.2}$
  &0.31$_{\pm0.04}$
  &96.1$_{\pm1.4}$
  &2.28$_{\pm1.00}$
  &-
  \\
  Five
  &U-Net \cite{ronneberger2015u}
  &$\times$
  &84.3$_{\pm9.6}$
  &2.43$_{\pm2.14}$
  &69.5$_{\pm8.8}$
  &1.59$_{\pm0.84}$
  &83.4$_{\pm6.9}$
  &10.34$_{\pm4.80}$
  &79.1$_{\pm8.4}$
  \\
  \cdashline{1-10}[0.8pt/2pt]
  Semi
  &UA-MT \cite{yu2019uncertainty}
  &$\times$
  &66.4$_{\pm16.2}$
  &4.69$_{\pm2.27}$
  &75.5$_{\pm3.4}$
  &1.31$_{\pm0.95}$
  &83.9$_{\pm6.2}$
  &9.52$_{\pm4.03}$
  &75.3$_{\pm8.6}$
  \\
  &CPS \cite{chen2021semi}
  &$\times$
  &87.4$_{\pm5.4}$
  &1.40$_{\pm0.76}$
  &37.1$_{\pm1.8}$
  &unable
  &63.2$_{\pm1.4}$
  &19.57$_{\pm5.67}$
  &62.6$_{\pm2.9}$
  \\
  &MASSL \cite{chen2019multi}
  &$\times$
  &77.4$_{\pm8.7}$
  &9.07$_{\pm3.11}$
  &80.5$_{\pm3.1}$
  &0.92$_{\pm0.43}$
  &81.9$_{\pm7.0}$
  &10.99$_{\pm4.58}$
  &79.9$_{\pm6.3}$
  \\
  &DPA-DBN \cite{he2020dense}
  &$\times$
  &68.0$_{\pm14.5}$
  &5.75$_{\pm3.89}$
  &68.7$_{\pm8.2}$
  &3.90$_{\pm2.39}$
  &67.4$_{\pm8.7}$
  &24.05$_{\pm6.75}$
  &68.0$_{\pm10.5}$
  \\
  %\midrule
  Atlas
  &VM \cite{ba2018un}
  &$\surd$
  &81.0$_{\pm6.1}$
  &2.13$_{\pm0.78}$
  &83.1$_{\pm1.8}$
  &0.56$_{\pm0.08}$
  &59.9$_{\pm5.0}$
  &15.36$_{\pm4.34}$
  &74.7$_{\pm4.3}$
  \\
  &LC-VM \cite{BalakrishnanVoxelMorph(u)}
  &$\surd$
  &81.7$_{\pm6.0}$
  &2.04$_{\pm0.77}$
  &83.0$_{\pm1.8}$
  &0.56$_{\pm0.07}$
  &60.2$_{\pm7.4}$
  &14.72$_{\pm4.89}$
  &74.9$_{\pm5.1}$
  \\
  &LT-Net \cite{wang2020lt}
  &$\surd$
  &77.8$_{\pm7.8}$
  &2.25$_{\pm0.95}$
  &82.6$_{\pm1.2}$
  &0.57$_{\pm0.05}$
  &60.4$_{\pm7.4}$
  &14.62$_{\pm4.84}$
  &73.6$_{\pm5.5}$
  \\
  %\hline
  LRLS
  &DeepAtlas \cite{xu2019deepatlas}
  &$\surd$
  &87.9$_{\pm4.3}$
  &1.30$_{\pm0.57}$
  &79.3$_{\pm2.6}$
  &0.74$_{\pm0.12}$
  &64.8$_{\pm9.6}$
  &12.87$_{\pm3.56}$
  &77.3$_{\pm5.5}$
  \\
  &DataAug \cite{zhao2019data}
  &$\surd$
  &82.2$_{\pm5.2}$
  &2.04$_{\pm0.73}$
  &83.9$_{\pm1.2}$
  &0.55$_{\pm0.06}$
  &22.2$_{\pm2.8}$
  &unable
  &62.8$_{\pm3.1}$
  \\
  &DeepRS \cite{he2020deep}
  &$\surd$
  &87.0$_{\pm5.0}$
  &1.60$_{\pm0.90}$
  &73.0$_{\pm5.9}$
  &0.93$_{\pm0.25}$
  &86.0$_{\pm5.6}$
  &8.55$_{\pm3.98}$
  &82.0$_{\pm5.5}$
  \\
  &PC-Reg-RT \cite{he2021few}
  &$\surd$
  &88.5$_{\pm4.9}$
  &1.23$_{\pm0.72}$
  &73.1$_{\pm3.1}$
  &1.09$_{\pm0.17}$
  &59.1$_{\pm3.6}$
  &20.71$_{\pm5.21}$
  &73.6$_{\pm3.9}$
  \\
  &BRBS \cite{he2022learning}
  &$\surd$
  &\color{blue}91.1$_{\pm3.9}$
  &\color{red}\textbf{0.93$_{\pm0.57}$}
  &\color{blue}87.2$_{\pm1.0}$
  &0.43$_{\pm0.05}$
  &71.5$_{\pm6.4}$
  &10.85$_{\pm2.99}$
  &83.3$_{\pm3.8}$
  \\
  %\hline
  DCRL
  &VADeR \cite{o2020unsupervised}
  &$\times$
  &85.4$_{\pm4.7}$
  &1.69$_{\pm0.77}$
  &81.2$_{\pm3.2}$
  &0.59$_{\pm0.13}$
  &79.9$_{\pm5.8}$
  &8.95$_{\pm3.37}$
  &82.2$_{\pm4.6}$
  \\
  &DenseCL \cite{wang2022densecl}
  &$\times$
  &87.3$_{\pm4.3}$
  &1.52$_{\pm0.79}$
  &83.9$_{\pm1.9}$
  &0.48$_{\pm0.09}$
  &77.1$_{\pm8.8}$
  &12.11$_{\pm6.51}$
  &82.8$_{\pm5.0}$
  \\
  &SetSim \cite{wang2022exploring}
  &$\times$
  &87.0$_{\pm4.5}$
  &1.60$_{\pm0.84}$
  &81.2$_{\pm3.0}$
  &0.58$_{\pm0.13}$
  &79.0$_{\pm7.3}$
  &11.72$_{\pm5.03}$
  &82.4$_{\pm4.9}$
  \\
  &DSC-PM \cite{li2021dense}
  &$\times$
  &87.0$_{\pm4.6}$
  &1.60$_{\pm0.86}$
  &82.6$_{\pm2.1}$
  &0.53$_{\pm0.09}$
  &85.7$_{\pm6.2}$
  &7.33$_{\pm3.32}$
  &85.1$_{\pm4.3}$
  \\
  &PixPro \cite{xie2021propagate}
  &$\times$
  &89.5$_{\pm3.9}$
  &1.31$_{\pm0.75}$
  &86.3$_{\pm1.2}$
  &\color{blue}0.38$_{\pm0.04}$
  &83.3$_{\pm8.7}$
  &8.73$_{\pm4.55}$
  &\color{blue}86.4$_{\pm4.6}$
  \\
  &GLCL\cite{chaitanya2020contrastive}
  &$\times$
  &84.5$_{\pm7.0}$
  &1.82$_{\pm1.09}$
  &83.0$_{\pm2.7}$
  &0.52$_{\pm0.11}$
  &85.5$_{\pm8.9}$
  &8.65$_{\pm5.18}$
  &84.3$_{\pm6.2}$
  \\
  %\hline
  \cdashline{1-10}[0.8pt/2pt]
  \textbf{DCRL}
  &\textbf{GVSL-Semi (CVPR)} \cite{He_2023_CVPR}
  &$\surd$
  &90.0$_{\pm3.7}$
  &1.21$_{\pm0.81}$
  &82.3$_{\pm5.9}$
  &0.55$_{\pm0.27}$
  &\color{blue}86.3$_{\pm5.5}$
  &\color{blue}7.18$_{\pm4.01}$
  &86.2$_{\pm5.0}$
  \\
  \textbf{(Ours)}
  &\textbf{GEMINI-Semi}
  &$\surd$
  &\color{red}\textbf{91.2$_{\pm3.6}$}
  &\color{blue}0.97$_{\pm0.56}$
  &\color{red}\textbf{87.3$_{\pm1.0}$}
  &\color{red}\textbf{0.35$_{\pm0.03}$}
  &\color{red}\textbf{87.7$_{\pm5.2}$}
  &\color{red}\textbf{7.14$_{\pm3.63}$}
  &\color{red}\textbf{88.7$_{\pm3.3}$}
  \\
  \bottomrule
\end{tabular}
}
\label{tab:metrics2}
\end{table*}
\begin{figure}
  \centering
  \includegraphics[width=\linewidth]{./picture/results2.pdf}
  \caption{Our GEMINI-Semi has significant visual superiority on three FS-Semi medical image segmentation tasks.}\label{Fig:results2}
\end{figure}
\subsubsection{Quantitative evaluation shows metric superiority}
As shown in Tab.\ref{tab:metrics2}, 19 compared methods demonstrate that the DCRL will greatly improve the representability, and the homeomorphism prior (``HP") further improves the reliability of the representation learning. There are three interesting observations in Tab.\ref{tab:metrics2}: \textbf{1)} The semi-supervised methods are limited by the extremely few labels. They utilize the pseudo-label generation (UA-MT, CPS) or multi-task learning (MASSL, DPA-DBN) to improve the representation, but the extremely few labels have no enough ability to give them reliable optimization directions to overcome the noise in pseudo labels or multiple tasks. As a result, the UA-MT, MASSL, and DPA-DBN have worse performance than U-Net on task 1, and the CPS is worse on task 2 and 3. \textbf{2)} With the ``HP", the Atlas and LRLS methods achieve robust performance in task 1 and task 2, but are limited in task 3. The ``HP" brings an alignment between labeled and unlabeled images for numerous reliable pseudo labels. Therefore, they have achieved significant improvement on task 1 and task 2 compared with the semi-supervised methods. However, the X-ray images in task 3 have low contrast and their appearances are varied caused by the 2D projection of 3D human body, this makes inefficient GVS brings large misalignment between images, thus interfering with the learning and reducing the performance. \textbf{3)} The DCRL methods have robust performance in all three tasks compared with the LRLS methods, although the VADeR, DenseCL, SetSim, DSC-PM, PixPro and GLCL have no homeomorphism prior. Because their feature-level learning reduce the direct interference caused by misalignment in LRLS's pseudo labels and the supervision from the few labels bring basic representability which will promote their correspondence discovery. However, the FP\&N problem is still a problem in the learning and their performance on task 3 is poor without ``HP" like the semi-supervised methods.

Compared with the LRLS, other DCRL methods, and our previous GVSL-Semi, our GEMINI-Semi achieves the best performance on three tasks with four observations: \textbf{1)} Compared with the LRLS methods which have ``HP", our method has better performance on all tasks. Although the BRBS has similar performance as our GEMINI-Semi on task 1 and task 2, our method achieves 16.2\% DSC and 3.71 AVD higher and lower than it on task 3. This is because our GEMINI-Semi utilizes our GSS for alignment measurement and shares the representation between the segmentation and deformation learning, bringing more efficient and robust learning for alignment. It has a great ability to construct positive feature pairs even with varied appearances. The gradient from our DHL also trains the soft negative feature pairs to drive the learning of distinct representations for potentially different semantics in shared backbones, bringing a regularization for potential mispaired positive pairs. \textbf{2)} Compared with the other DCRL methods which have no ``HP", our GEMINI-Semi shows great improvements in all three tasks. It achieves more than 1.7\%, 1.0\%, and 2.0\% DSC improvements on task 1, 2, and 3 compared with the best DCRL models without ``HP" (PixPro in task 1 and 2, DSC-PM in task 3). Because the ``HP" in our GEMINI-Semi constructs a more reliable correspondence discovery process which reduces the production risk of the FP\&N pairs, bringing comprehensive improvement for the DCRL. \textbf{3)} Compared to our CVPR vision (GVSL-Semi), we find even though the GVSL utilizes the visual similarity like the BRBS, it also achieves great performance in task 3, demonstrating the superiority of the DCRL paradigm. The GVSL-semi avoids the interference of pseudo labels like BRBS reducing the noisy information from the extremely mis-alignment, so that it takes the advantage of DCRL and our homeomorphism prior and achieves good performance in all three tasks. Our GEMINI-Semi promotes the GVSL and utilizes the GSS for a more powerful dense representation learning, thus achieving the highest 88.7\% average DSC in this experiment. \textbf{4)} Compared with the fully supervised setting (``Full") in task 2 (83 labeled images), our GEMINI-Semi achieves a similar performance only with 5 labeled images demonstrating our great potential in reducing of annotation costs. In the task 3, our framework is lower than the upper bound (96.1\%) only with five annotations, but it still achieves significant improvement (4.3\%) compared with the model directly trained on five labeled images.

\subsubsection{Qualitative evaluation shows visual superiority}
As shown in Fig.\ref{Fig:results2}, we show typical cases on the three tasks in this experiment and our framework has higher accuracy on thin regions and fewer outliers. In the task 1, the segmentation result of our method has better integrity, and the different semantic structures have good adjacency without outliers. However, the other four DCRL methods have discontinuous mis-segmentation which destroys the heart topology. This is because the pairing strategies in the DCRL methods are unable to make the pairs under the condition of topology consistency, so the large-scale mispaired features interrupt the learning and make numerous outliers. The same as the task 3, there are also serious outlier problems in the four typical DCRL methods and the GVSL, and our GEMINI-Semi has fine segmentation. In the task 2, our GEMINI and GVSL show finer segmentation on the thin brain structures which is sensitive and will be interrupted by the noise in the semi-supervised learning process. In some prominent and gully regions of task 2 (enlarged part), the compared four DCRL methods have numerous distortions due to their unreliable correspondence discovery, showing their fragility.



Our first study sought to understand the key factors underlying human expert evaluation of the creativity of solutions to design problems (DPT) items. A participant in this task is given a scientific or engineering problem (e.g., increasing the use of renewable energy) and is instructed to come up with as many novel solutions to the problem as they can think of. Similar to expert-level science, the best solutions are both original and feasible, though unlike other STEM assessments the DPT benefits from but is not contingent on expertise to come up with creative ideas. The greater complexity of DPT responses compared to those from other creativity tests and its relationship to scientific creativity more broadly makes it a strong choice for our analysis. Unlike prior studies, which often have experts rate only the originality or quality of products, we instead ask our raters to provide fine-grained assessments of cleverness (whether the solution is insightful or witty), remoteness (whether the solution is ``far'' from everyday ideas), and uncommonness (whether the solution is rare, given by few people) in addition to originality, each of which is thought to influence ratings of creativity \citep{silvia2008assessing}. These assessments are performed both with and without the presence of example creativity ratings to DPT items, enabling us to examine how added context affects the evaluation process. Finally, we ask experts to briefly explain their originality scores, enabling us to employ methods from computational text analysis to probe the cognitive processes experts employ when rating and how such processes may be modulated by added context.

\subsection{Methods}
We use the data from \citet{Patterson2025}, who obtained more than 7000 responses to DPT items from undergraduate STEM majors. Each response was rated for originality using a five-point Likert scale by at least three expert raters with formal training in engineering. We drop items that did not obtain at least one rating from every point of the scale (certain items never had a response that received a five). We convert Likert scores into factor scores, as this has been shown to provide more accurate creativity ratings \citep{silvia2008another}, and we treat these factor scores as the true originality scores of each response.

We recruit 80 participants on Prolific to provide finegrained creativity ratings to DPT responses, requiring that they have a bachelor's degree or higher in a STEM field and are fluent in English. We split participants into two conditions: a \textit{no example} condition where participants are given responses to rate without any additional context, and an \textit{example} condition where participants are first shown example solutions with originality scores for responses to the same prompt being rated. We pull three example solutions from the same dataset while ensuring that participants never rate them. We include a solution with a score of one, one with a score of three, and one with a score of five, to avoid biasing participants towards either end of the scale. We first have each participant rate for originality following the same procedure, instructions, and facet definitions as \citet{Patterson2025}. After rating originality, participants in both groups then provide 1-2 sentences explaining their rating process \citep{orwig2024creative}, and they finish by rating the uncommonness, remoteness, and cleverness of the response using a five-point Likert scale for each. We instruct participants to be specific in their explanations, to draw on their domain expertise as holders of a STEM degree, and to avoid overly simplistic explanations (e.g., ``it's not original'' or ``it's an obvious answer''). We define a good explanation as being at least one sentence long and including specific details from the participant's prior experience, the response, or the examples (if applicable). We also provide definitions of uncommonness, remoteness, and cleverness for the final rating task, emphasizing that each facet is related while being distinct from originality. We include educational background and AI use checks at the end of the survey.

We administer each participant 15 DPT responses at random. To encourage high-quality explanations, we offer \$20 per hour to complete a 30-minute study. We exclude participants with an approval rating of less than 90\%, who report using AI to complete the task, or who report an education level lower than the minimum specified on Prolific. We also exclude participants who were exceptionally slow or fast (with a completion time further than three standard deviations from the mean), who gave the same rating for every response, or who did not follow our instructions for formatting explanations (as checked by a research assistant). This resulted in a final sample size of 37 participants and 481 ratings in the example condition and 35 participants and 455 responses for the no example.

% out of the full archival set

When examining the participants' explanations, we employ an analysis plan similar to \citet{orwig2024creative}, who used LIWC to analyze explanations of originality scores for AUTs. However, recent work has found that LLMs can predict psycholinguistic features of text more strongly than LIWC, even zero shot \citep{rathje2024gpt}. Therefore, we use LLMs to automatically rate linguistic markers in the explanations. We instruct LLMs to rate for the following variables: 

\begin{itemize}
    \item \textit{Past/future expressions}: Is the explanation past-focused or future-focused in its evaluation of the response?
    \item \textit{Perceptual details}: Does the explanation focus on the process of perceiving (``observe'', ``seen'', ``heard'', ``feel'', etc.)?
    \item \textit{Causal/analytical}: Does the explanation involve a structured evaluation of the response, evidencing an analytical process, or is the explanation more intuitive in its justifications?
    \item \textit{Comparative}: Does the explanation make explicit references to standards or examples or compare the response to other ideas?
    \item \textit{Cleverness}: Does the explanation refer to the cleverness, wittiness, shrewdness, or ingenuity (or lack thereof) of the response?
\end{itemize}

Both past/future language use and perceptual details have been explored to assess cognitive strategies employed on other creativity tests \citep{orwig2024creative}. We elect to use causal/analytical, comparative, and cleverness linguistic markers to aid in assessing whether participants employed a more structured process --- which might be evidenced by causal/analytical or comparative language use --- or a more intuitive process, as evidenced by language indicating sensory experiences or other ``gut reactions'' (e.g, ``it feels like a clever idea''). These linguistic markers also map onto the finegrained facets participants were asked to rate, with cleverness language mapping onto cleverness and comparative language mapping onto remoteness and uncommonness (as both remoteness and uncommonness often require making references to prior solutions). We use both \textsc{claude-3.5-sonnet}\footnote{https://www.anthropic.com/news/claude-3-5-sonnet} and \textsc{gpt-4o}\footnote{https://openai.com/index/hello-gpt-4o/} to check for reliability in ratings and avoid biases specific to a single LLM, though due to space constraints we mainly report results from \textsc{gpt-4o} as this is the model \citet{rathje2024gpt} validated. To encourage deterministic output, we set the temperature for both models to $0$ and top P to $1$. We instruct LLMs to rate each facet and provide a binary evaluation of whether the explanation does or does not contain the feature. Prompts are provided in the supplementary materials.

% We drop any explanations for which the model fails to follow this instruction.



% Specifically, we have each participant first rate for originality following the same procedure as \TODO{cite the paper that obtained the gold scores}, after which they are instructed to provide 1-2 sentences explaining their originality rating. We instruct participants to draw on their domain expertise while responding, asking them to consider if they have observed similar solutions to the problem in the past, to help ensure that the explanation is grounded in the originality of the solution and not merely its quality. Finally, participants rate the uncommonness, remoteness, and cleverness of the solution using separate five-point Likert scales, to disentangle how each facets contributes to the final creativity score. Figure \TODO{make it} shows the experimental interface, we collect data using Qualtrics and give 30 minutes to complete the task. Participants are split into two conditions: a \textit{no oracle} conditions where participants are shown DPT solutions without any additional context (equivalent to how creativity ratings are typically solicited), and an \textit{oracle} condition where participants are first shown example solutions with originality scores for responses to the same prompt being rated. We pull these solutions from the same dataset, while ensuring that participants never rate them. At the end, participants complete a demographic questionnaire and a check for use of generative AI during the study.

% The design problems task (DPT) is a test of domain-general creative problem solving and divergent thinking in science and engineering \TODO{cite}.   Further, creativity evaluation also hinges on weighing multiple competing factors: a highly uncommon solution may still receive a poor creativity score if it is not especially clever or could not be feasibly put into practice. This makes the DPT a strong testbed for our experiments: evaluation is more complex than purely theoretical creativity tests like the alternative uses task \TODO{cite}, yet it is not so challenging as to require domain experts to obtain meaningful creativity scores, enabling us to recruit a larger pool of participants.

% The task targets STEM undergraduate students; a general understanding of science and engineering is beneficial but not necessary for the task.

% Our goal is to obtain \textit{finegrained} creativity assessments for these responses, to better understand how each facet of creativity influences a raters final score, and to provide explanations for why responses are assigned a particular rating. 

% \TODO{define the research questions and null hypotheses}

\subsection{Results}
We begin by examining inter-correlations among all facets (cleverness, remoteness, uncommonness) and correlations between each facet and originality for both conditions. Results are in Figure \ref{fig:experiment_1_correlations}. As expected, each facet is moderately correlated with originality as well as each other, with Pearson r in the range 0.45--0.67 (all correlations are significant).\footnote{Results from all correlational analysis in both studies were similar using Spearman $\rho$.} Comparing the example to no example conditions, we see an increase in correlation between originality and cleverness and a decrease in correlation between originality and both remoteness and uncommonness. Changes in correlation across conditions were significant for cleverness-remoteness (Fisher's z = 2.83, p $<$ 0.01), remoteness-uncommonness (z = -4.61, p $<$ 0.001), and remoteness-originality (z = -2.96, p $<$ 0.01), but were insignificant for all other comparisons. Notably, the presence of the examples did not make experts significantly more accurate in their evaluations of originality, with correlations in the moderate range for both conditions (no example r = 0.44, example r = 0.47).

% We report descriptive statistics for all Likert evaluations in Table \TODO{make it}, broken down by condition.

% Examining the distribution of cleverness more closely (the only facet to become more strongly related to originality in the oracle condition), we plot the distribution of cleverness scores for both conditions in Figure \ref{fig:experiment_1_cleverness}. 

% Participants in the oracle condition appear to be stricter judges of cleverness, giving more 1 or 2 rating than their no oracle counterparts, though this difference was only marginally significant (Mann Whitney U test = 102948.5, p $<$ 0.1).

\begin{figure}[htb]
    \centering
    \footnotesize
    \includegraphics[width=0.8\linewidth]{Figures/Correlations_human.eps}
    % \includesvg[width=0.85\linewidth]{Figures/Correlations_human.svg}
    \caption{Pearson correlations among pairwise Likert ratings for both conditions. o = originality, c = cleverness, u = uncommonness, r = remoteness.}
    \label{fig:experiment_1_correlations}
\end{figure}

% \begin{figure}[htb]
%     \centering
%     \includegraphics[width=1\linewidth]{Figures/cleverness.png}
%     \caption{Distribution of cleverness ratings for both conditions.}
%     \label{fig:experiment_1_cleverness}
% \end{figure}

Turning to participant explanations, \textsc{gpt-4o}'s ratings did not reveal significant differences per condition for perceptual details, past/future language use, or cleverness, but differences are significant for both causal/analytical language (Mann-Whitney U = 78039.5, p $<$ 0.05) and comparative language (U = 75627.5, p $<$ 0.01) with the example condition using less comparative and causal/analytical language than the no examples. Distributions for linguistic markers are shown in Figure \ref{fig:liwc_analysis}. \textsc{claude-3.5-sonnet}'s ratings generally agreed with \textsc{gpt-4o} (Cramer's V in the range 0.549--0.798) with the only notable departure being that \textsc{claude-3.5-sonnet} found no significant difference in causal/analytical language between the conditions (U = 75076.5, p $<$ 0.5). We report additional linguistic marker analysis in the supplementary materials.

% We report model agreement statistics and linguistic marker analyses in the supplementary files.

\subsection{Discussion}
As expected, the facet ratings for cleverness, remoteness, and uncommonness did not perfectly correlate with each other nor with originality, implying that participants do not weigh each facet equally when assessing originality. Further, correlations changed by a significant degree when including example ratings, with both remoteness and uncommonness becoming weaker predictors of originality and cleverness becoming a stronger one. Given that participants in the no example condition needed to actively retrieve example solutions from memory when evaluating, a possible explanation is that this retrieval process biased them towards placing stronger emphasis on the remoteness and uncommonness of the response in relation to solutions they had seen in the past, while example participants would not need to focus as much effort on thinking of prior solutions and could instead focus on the cleverness of the idea. Notably, participants in both groups did not differ significantly in terms of education, making it unlikely this effect could be explained as a skill confound. The idea that participants in the example condition were biased toward cleverness rather than the other facets was also partially supported by their explanations, as no example participants used significantly more comparative language than example participants. Given that assessing remoteness or uncommonness often requires making direct comparisons to prior solutions, it makes sense that an evaluation rooted around these facets would contain more comparisons than an evaluation rooted around cleverness, which is more readily evaluated in isolation (e.g., whether the idea is resource efficient, not immediately obvious, etc.).


\section{Discussion and Future Work}\label{sec:discussion}
This paper pioneers the novel approach of selective response, showing that withholding responses can be a powerful tool for GenAI systems. By opting not to answer every query as accurately as it can---particularly when new or complex topics emerge---GenAI can encourage user participation on community-driven platforms and thereby generate more high-quality data for future training. This mechanism ultimately enhances GenAI's long-term performance and revenue. From a welfare perspective, our results indicate that such selective engagement can also benefit users, leading to better solutions and increased overall satisfaction. Since this work is the first to address selective response strategies for GenAI, numerous promising directions remain for future research; we highlight some of them below. 

First, from a technical standpoint, all of the results in this paper rely on Assumption~\ref{assumption: data lip}, involving the lipshitz condition of the accuracy function and the sensitivity parameter $\beta$. Future work could seek to relax this assumption. Furthermore, our constrained optimization approach in Subsection~\ref{sec: welfare constrained revenue maximization} could be extended to approximate the optimal (continuous) strategy instead of the optimal discrete strategy.

Second, our stylized model adopts the simplifying---though unrealistic---assumption that only a single GenAI platform exists. Admittedly, this makes it easier to focus on the idea of selective responses, and indeed, this assumption is pivotal in keeping our analysis tractable. Future research could explore scenarios with multiple GenAI platforms and human-centered forums. In such settings, one platform's selective response might redirect users not only to forums but also to competing GenAI platforms, leading to the tragedy of the commons \cite{hardin1968tragedy}: Although all GenAI platforms benefit from fresh data generation, none may choose to respond selectively if it means losing users to competitors. 

Third, we assumed Forum behaves non-strategically. In reality, human-centered platforms often monetize their data by selling it to GenAI platforms, adding a further layer of strategic interaction for GenAI. Moreover, data transfer between the platforms can form the basis for collaboration: GenAI could employ selective response to bolster Forum content creation, and Forum could, in turn, attribute that content to GenAI for subsequent use in retraining.


%Third, we make the (again) simplifying assumption that Forum is non-strategic. However, in practice, human-centered platforms can sell their data to GenAI platforms. This adds additional considerations for GenAI. Furthermore, data transmission between the platforms can also become the basis for collaboration: GenAI can use selective response to ensure enough content is generated in Forum, and Forum could provide the data attributed to this mechanism back to GenAI. 


%Second, this paper makes the simplifying yet unrealistic assumption of the existence of one GenAI platform. Indeed, this simplifies many aspects and allows us to analyze selective responses. Future work could address the data generation process with more than one GenAI platform and possibly several human-centered forums. In such a case, selective response of one GenAI platform can either drive users to forums or to other GenAI platforms; thus, we might face a tragedy of the commons situation~\ref{hardin1968tragedy}, where all GenAI platforms are interested in fresh data generation but none volunteer to selectively respond and lose users. 

%This paper examines the competition between a generative AI platform and human-based platforms, challenging the assumption that always providing answers is optimal. We analyzed the impact of withholding answers on GenAI's revenue and developed an efficient approximately optimal algorithm for this purpose. We further explored how withholding affects users, showing that it can lead to better outcomes compared to always answering. Specifically, we demonstrated that withholding can Pareto-dominate this strategy and derived the necessary and sufficient conditions for that. Finally, we proposed a second approximately optimal algorithm that maximizes GenAI's revenue while ensuring users are better off than when GenAI answers all queries.

%On a more conceptual level, our model assumes that GenAI’s data comes solely from the competing platform (Forum). Future research could explore a scenario where GenAI can purchase additional data from a third party. This extension could provide valuable insights into the interplay between withholding answers and data purchasing, and whether these two strategies can complement each other or must be traded off.
Software development is increasingly conceived as a collaboration activity between developers and AIs. Indeed, IDEs already implement features to enable interactive development, with AI suggesting implementations that are reused by developers.

Although multiple studies show this interaction can be successful, there is still limited understanding of how the models must be configured and used in the context of code generation tasks. This study addresses this gap, systematically investigating the impact of several key parameters, including the repeated submission of a prompt to accommodate for the non-deterministic nature of the models.

Our study reveals several key findings about the usage of ChatGPT. In particular, we discovered how creativity, although up to a limited extent, is useful to increase the range of methods whose code can be generated correctly. A major role is played by parameter top-p, which is commonly underrated, and instead has a major impact on the correctness of the results, with lower values producing better results. Finally, prompts should be submitted multiple times, with $5$ repetitions combined with a temperature of $1.2$ resulting in an effective configuration in our experiments.  

Future work concerns two main research directions. One is about replicating this experiment with other AI assistants, to validate our findings in multiple contexts. The second research direction concerns finding strategies to deal with the need to submit the same prompt multiple times to obtain a useful result, and thus developing approaches able to select or merge multiple responses automatically. 

\appendix



\begin{table*}[tb]
\centering
\caption{The fine-tuning evaluations demonstrate our great transferring ability on SS-MIP tasks which pre-trained on PPMI dataset. Our GEMINI-MIP achieves the best performance compared with 18 methods on two downstream tasks.}
\begin{tabular}{clcccccccccc}
\toprule
\multirow{2}{*}{\textbf{Type}}
&\multirow{2}{*}{\textbf{Pre-training}}
&\multicolumn{2}{c}{\textbf{T2: KiPA22} \emph{Inter-scene}}
&\multicolumn{2}{c}{\textbf{T3: CANDI} \emph{Inner-scene}}
%&\multicolumn{2}{c}{\textbf{T3: CANDI-mini} \emph{Inner-scene}}
&\textbf{AVG}
\\
\cmidrule(r){3-4}
\cmidrule(r){5-6}
%\cmidrule(r){7-8}
\cmidrule(r){7-7}
&
&DSC$_{\pm std}\uparrow$
&AVD$_{\pm std}\downarrow$
&DSC$_{\pm std}\uparrow$
&AVD$_{\pm std}\downarrow$
%&DSC$_{\pm std}\uparrow$
%&AVD$_{\pm std}\downarrow$
&DSC$_{\pm std}\uparrow$
\\
\midrule
-
&Scratch (3D U-Net)
&72.4$_{\pm16.3}$
&6.11$_{\pm5.91}$
&84.0$_{\pm3.2}$
&0.52$_{\pm0.14}$
%&62.1$_{\pm10.1}$
%&1.59$_{\pm0.83}$
&78.2$_{\pm9.8}$
\\
\cdashline{1-7}[0.8pt/2pt]
Sup
&Med3D \cite{chen2019med3d}
&81.7$_{\pm12.0}$
&\color{blue}2.61$_{\pm2.77}$
&72.7$_{\pm19.0}$
&1.57$_{\pm2.56}$
%&$_{\pm}$
%&$_{\pm}$
&77.2$_{\pm15.5}$
\\
GRL
&Denosing \cite{vincent2010stacked}
&70.0$_{\pm15.4}$
&7.60$_{\pm5.03}$
&\cellcolor[gray]{0.9}83.7$_{\pm3.3}$
&\cellcolor[gray]{0.9}1.71$_{\pm0.20}$
%&\cellcolor[gray]{0.9}unable%$_{\pm}$
%&\cellcolor[gray]{0.9}unable%$_{\pm}$
&76.9$_{\pm9.4}$
\\
&In-painting \cite{pathak2016context}
&69.7$_{\pm17.1}$
&7.57$_{\pm5.93}$
&\cellcolor[gray]{0.9}88.5$_{\pm3.1}$
&\cellcolor[gray]{0.9}0.32$_{\pm0.11}$
%&\cellcolor[gray]{0.9}$_{\pm}$
%&\cellcolor[gray]{0.9}$_{\pm}$
&79.1$_{\pm10.1}$
\\
&Models Genesis \cite{zhou2019models}
&75.8$_{\pm13.7}$
&4.64$_{\pm4.49}$
&\cellcolor[gray]{0.9}88.7$_{\pm3.1}$
&\cellcolor[gray]{0.9}0.31$_{\pm0.10}$
%&\cellcolor[gray]{0.9}$_{\pm}$
%&\cellcolor[gray]{0.9}$_{\pm}$
&82.3$_{\pm8.4}$
\\
&Rotation \cite{komodakis2018unsupervised}
&77.4$_{\pm14.3}$
&4.82$_{\pm6.29}$
&\cellcolor[gray]{0.9}89.4$_{\pm2.6}$
&\cellcolor[gray]{0.9}0.28$_{\pm0.08}$
%&\cellcolor[gray]{0.9}$_{\pm}$
%&\cellcolor[gray]{0.9}$_{\pm}$
&83.4$_{\pm8.5}$
\\
CRL
&SimSiam \cite{Chen2021CVPR}
&83.8$_{\pm11.9}$
&3.69$_{\pm7.47}$
&\cellcolor[gray]{0.9}87.3$_{\pm3.1}$
&\cellcolor[gray]{0.9}0.36$_{\pm0.10}$
%&\cellcolor[gray]{0.9}$_{\pm}$
%&\cellcolor[gray]{0.9}$_{\pm}$
&85.6$_{\pm7.5}$
\\
&BYOL \cite{grill2020bootstrap}
&83.6$_{\pm11.2}$
&2.78$_{\pm5.42}$
&\cellcolor[gray]{0.9}89.7$_{\pm2.4}$
&\cellcolor[gray]{0.9}\color{blue}0.27$_{\pm0.08}$
%&\cellcolor[gray]{0.9}$_{\pm}$
%&\cellcolor[gray]{0.9}$_{\pm}$
&\color{blue}86.7$_{\pm6.8}$
\\
&SimCLR \cite{chen2020simple}
&78.9$_{\pm13.9}$
&4.49$_{\pm5.15}$
&\cellcolor[gray]{0.9}89.2$_{\pm3.0}$
&\cellcolor[gray]{0.9}0.30$_{\pm0.14}$
%&\cellcolor[gray]{0.9}$_{\pm}$
%&\cellcolor[gray]{0.9}$_{\pm}$
&84.1$_{\pm8.5}$
\\
&MoCov2 \cite{chen2020improved}
&78.0$_{\pm15.3}$
&4.42$_{\pm5.67}$
&\cellcolor[gray]{0.9}89.7$_{\pm2.4}$
&\cellcolor[gray]{0.9}0.28$_{\pm0.11}$
%&\cellcolor[gray]{0.9}$_{\pm}$
%&\cellcolor[gray]{0.9}$_{\pm}$
&83.9$_{\pm8.9}$
\\
&DeepCluster \cite{caron2018deep}
&79.7$_{\pm13.7}$
&4.28$_{\pm5.76}$
&\cellcolor[gray]{0.9}89.8$_{\pm2.4}$
&\cellcolor[gray]{0.9}\color{blue}0.27$_{\pm0.08}$
%&\cellcolor[gray]{0.9}$_{\pm}$
%&\cellcolor[gray]{0.9}$_{\pm}$
&84.8$_{\pm8.1}$
\\
DCRL
&VADeR \cite{o2020unsupervised}
&72.1$_{\pm13.8}$
&6.56$_{\pm5.89}$
&\cellcolor[gray]{0.9}87.4$_{\pm3.6}$
&\cellcolor[gray]{0.9}0.35$_{\pm0.11}$
%&\cellcolor[gray]{0.9}$_{\pm}$
%&\cellcolor[gray]{0.9}$_{\pm}$
&79.8$_{\pm8.7}$
\\
&DenseCL \cite{wang2022densecl}
&74.0$_{\pm15.8}$
&6.42$_{\pm8.21}$
&\cellcolor[gray]{0.9}87.7$_{\pm3.8}$
&\cellcolor[gray]{0.9}0.34$_{\pm0.13}$
%&\cellcolor[gray]{0.9}$_{\pm}$
%&\cellcolor[gray]{0.9}$_{\pm}$
&80.9$_{\pm9.8}$
\\
&SetSim \cite{wang2022exploring}
&73.5$_{\pm15.9}$
&6.34$_{\pm6.68}$
&\cellcolor[gray]{0.9}88.4$_{\pm3.1}$
&\cellcolor[gray]{0.9}0.32$_{\pm0.10}$
%&\cellcolor[gray]{0.9}$_{\pm}$
%&\cellcolor[gray]{0.9}$_{\pm}$
&81.0$_{\pm9.5}$
\\
&DSC-PM \cite{li2021dense}
&79.0$_{\pm14.6}$
&4.90$_{\pm6.05}$
&\cellcolor[gray]{0.9}88.5$_{\pm3.4}$
&\cellcolor[gray]{0.9}0.32$_{\pm0.13}$
%&\cellcolor[gray]{0.9}$_{\pm}$
%&\cellcolor[gray]{0.9}$_{\pm}$
&83.8$_{\pm9.0}$
\\
&PixPro \cite{xie2021propagate}
&80.0$_{\pm14.4}$
&4.60$_{\pm6.25}$
&\cellcolor[gray]{0.9}\color{blue}89.9$_{\pm2.4}$
&\cellcolor[gray]{0.9}\color{blue}0.27$_{\pm0.07}$
%&\cellcolor[gray]{0.9}$_{\pm}$
%&\cellcolor[gray]{0.9}$_{\pm}$
&85.0$_{\pm8.4}$
\\
&GLCL \cite{chaitanya2020contrastive}
&70.7$_{\pm16.9}$
&7.33$_{\pm7.05}$
&\cellcolor[gray]{0.9}87.4$_{\pm3.2}$
&\cellcolor[gray]{0.9}0.34$_{\pm0.09}$
%&\cellcolor[gray]{0.9}$_{\pm}$
%&\cellcolor[gray]{0.9}$_{\pm}$
&79.1$_{\pm10.1}$
\\
\cdashline{1-7}[0.8pt/2pt]
\textbf{DCRL}
&\textbf{GVSL-MIP (CVPR)}\cite{He_2023_CVPR}
&\color{blue}84.3$_{\pm10.3}$
&2.85$_{\pm5.12}$
&\cellcolor[gray]{0.9}89.1$_{\pm2.8}$
&\cellcolor[gray]{0.9}0.31$_{\pm0.11}$
%&\cellcolor[gray]{0.9}$_{\pm}$
%&\cellcolor[gray]{0.9}$_{\pm}$
&\color{blue}86.7$_{\pm6.6}$
\\
\textbf{(Ours)}
&\textbf{GEMINI-MIP}
&\color{red}\textbf{85.0$_{\pm10.2}$}
&\color{red}\textbf{2.55$_{\pm5.71}$}
&\cellcolor[gray]{0.9}\color{red}\textbf{90.0$_{\pm2.4}$}
&\cellcolor[gray]{0.9}\color{red}\textbf{0.26$_{\pm0.07}$}
%&\cellcolor[gray]{0.9}76.3$_{\pm6.0}$
%&\cellcolor[gray]{0.9}0.78$_{\pm0.26}$
&\color{red}\textbf{87.5$_{\pm6.3}$}
\\
\bottomrule
\end{tabular}
\label{supp:tab:metrics}
\end{table*}
\section*{A SS-MIP on more datasets}
\label{aupp:sec:task1}
\subsection*{A.1 Self-supervised pre-training on PPMI dataset}
We further evaluate the SS-MIP task on another pretext dataset for pre-training to demonstrate our representation ability. We extracted 837 3D brain T1 MR images with Parkinson’s disease from the PPMI database\footnote{PPMI database: \url{https://www.ppmi-info.org/}} as our pretext dataset. In our experiment, we extract the brain regions via HD-BET \cite{isensee2019automated}, crop and resize the images to $160\times160\times128$, and finally normalize them via the zero-score. Due to the consistency of the human brain regions, we randomly pair these brain images to pre-train the frameworks. Following the Experiment 2 (Sec.5) in our manuscript, we take the Task 2: KiPA22 dataset and Task 3: CANDA as the downstream tasks to evaluate the inter-scene and inner-scene transferring abilities. (Because the Task 1: SCR$_{25}$ dataset is 2D and the pre-trained models are 3D, we exclude this task in this experiment.) We utilize the same implementation and evaluation metrics as the Sec.5 in this experiment.

As shown in Tab.\ref{supp:tab:metrics}, it achieves similar observations as the SS-MIP experiment in Sec.5. For most of the methods, the pre-training on the PPMI dataset will bring better performance than random initialization (“Scratch”) both in the T2: KiPA22 and T3: CANDI tasks. Especially in the T3 (inner-scene), most of the pre-training methods achieve more than 4.0\% DSC improvement compared with the “Scratch”. Even though the other CRL and DCRL methods have FP\&N problems in this experiment, they are still able to learn the representation of some domain features and promote their final performance to the upper limit of the task 3 (near 90\%). When transferring the pre-trained models to the T2 (inter-scene), the SimSiam, BYOL, our GVSL-MIP, and our GEMINI-MIP all still have significant improvement (more than 10\% DSC). This is because these methods learn the consistency of features and avoid the FP\&N problems. The other methods’ performance improvement is obviously decreased owing to the FP or FN problem which interrupts their representation learning of high-level semantics and makes their representations deviate from reality. On both two tasks, our GEMINI-MIP achieves the highest performance showing our superiority.
\begin{table}[tb]
\centering
\caption{The gap coefficient $G^{i}$ quantifies the gap between “pre-trained on chest X-ray images \& fine-tuning on brain T1 MR images” (inter-scene) and “pre-trained on brain T1 MR images \& fine-tuning on brain T1 MR images” (inner-scene).}
\resizebox{\linewidth}{!}{
\begin{tabular}{ccccc}
\toprule
\textbf{Index}
&\multirow{3}{*}{\textbf{Method}}
&\textbf{Chest X-ray}
&\textbf{Brain T1 MR}
%&\multicolumn{2}{c}{\textbf{T3: CANDI-mini} \emph{Inner-scene}}
&\textbf{Gap}
\\
\cmidrule(r){1-1}
\cmidrule(r){3-3}
\cmidrule(r){4-4}
%\cmidrule(r){7-8}
\cmidrule(r){5-5}
\multirow{2}{*}{$i$}
&
&2D U-Net
&3D U-Net
&\multirow{2}{*}{$G^{i}$}
\\
&
&\emph{Inter-scene}
&\emph{Inner-scene}
&
\\
\midrule
0
&Scratch
&65.0$_{\pm4.4}$
&84.0$_{\pm3.2}$
&1
\\
\cdashline{1-5}[0.8pt/2pt]
1
&BYOL
&70.5$_{\pm2.1}$
&\cellcolor[gray]{0.9}89.7$_{\pm2.4}$
&1.01
\\
2
&DeepCluster
&60.0$_{\pm2.2}$
&\cellcolor[gray]{0.9}89.8$_{\pm2.4}$
&1.57
\\
3
&Model Genesis
&88.1$_{\pm3.1}$
&\cellcolor[gray]{0.9}88.7$_{\pm3.1}$
&0.03
\\
4
&DenseCL
&76.8$_{\pm2.9}$
&\cellcolor[gray]{0.9}87.7$_{\pm3.8}$
&0.57
\\
\cdashline{1-5}[0.8pt/2pt]
\textbf{5}
&\textbf{Our GEMINI-MIP}
&\textbf{89.8$_{\pm2.6}$}
&\cellcolor[gray]{0.9}\textbf{90.0$_{\pm2.4}$}
&\textbf{0.01}
\\
\bottomrule
\end{tabular}
}
\label{supp:tab:gap}
\end{table}
\subsection*{A.2 Analysis of the gap between the inner-scene and inter-scene transferring}
As shown in Tab.\ref{supp:tab:gap}, the quantitative evaluation of the gap between the inner-scene and inter-scene transferring show our great transferring ability both inner scene and inter scene. Here, we formulate a gap coefficient $G$ to quantify this gap:
\begin{equation}\label{equ:gap}
G^{i}=\frac{S^{i}_{inner}-S^{i}_{inter}}{S^{0}_{inner}-S^{0}_{inter}},
\end{equation}
where the $i$ is the index of the method, $S$ is the score of the method (here we take the DSC). The $S_{inner}^{0}-S_{inter}^{0}$ is the gap of the “Scratch” between the two settings which means the difference caused by the initial situation, such as network structure and dimension. The $S_{inner}^{i}-S_{inter}^{i}$ is the gap of the $i_{th}$ method between the two settings. Therefore, the $\frac{S_{inner}^{i}-S_{inter}^{i}}{S_{inner}^{0}-S_{inter}^{0}}$ means the gap of the model in two settings excluding the gap caused by the initial network. If the $G^{i}$ is larger than 1, it means that the pre-trained model has weaker inter-scene transferring ability than inner-scene transferring. If it is smaller than 1, it means that the model has great inter-scene transferring ability.

Most self-supervised learning methods have large gap between inner- and inter-scene transferring, and our GEMINI has great universal representation for different scenes. The BYOL and DeepCluster are limited in the inter-scene transferring ($G>1$) because they only take the image-level contrast which will represent the high-level semantic features and this representation is very different between scenes. The DenseCL has 0.57 gap coefficient which is better than the BYOL and DenseCL. Because it takes dense contrastive learning which also represent low-level detail features and this representation is shared in different scene. Our GEMINI and the Model Genesis all have good inter-scene transferring ability with very low gap coefficient (0.01 and 0.03), showing their great universal representation ability and demonstrating their potential as an initialization for more scenes.

\section*{B Discussion of the research problem and method}
\subsection*{B.1 Discussion of FP\&N problem}
\begin{figure}[tb]
  \centering
  \includegraphics[width=\linewidth]{./picture/FPproblem.pdf}
  \caption{The evaluation of the large-scale FP problem. The true positive (TP) pairs constructed by the features’ similarity (used in DenseCL) only occupy the 5.79\% of the foreground region, and our GEMINI is able to bring 60.74\% TP pairs. }\label{supp:fig:fp}
\end{figure}
As analyzed in the Introduction section, medical images' semantic dependence property will make large-scale FP problem, and their semantic continuity and semantic overlap properties will make large-scale FN problem. In this section, we make an experiment to quantitatively count the percentage of FP and FN pairs in the pairing process.

For FP pairs, we utilize two cardiac CT images (image A and B), and extract their pixel-wise features via a random initialized 3D U-Net. Then, we utilize the pixel-wise feature similarity measurement method in the DenseCL \cite{wang2022densecl} to extract the positive pairs. Because the semantics of the background region are unclear, we count the accuracy of the feature pairs in the foreground regions. As shown in Fig.A, only 5.79\% of the positive pairs in the foreground region are accurate. Therefore, if we directly pair the features only according to their similarity, most of the contrasts (94.21\%) for positive pairs are inaccurate in the medical images and will interrupt the whole contrastive learning process. This is because medical images have very weak contrast due to their special imaging way, making the directly extracted features lack discrimination. Therefore, it makes the “Semantic dependence” one of the inherent properties in medical images constructing large-scale FP pairs.

For FN pairs, we further evaluate the percentage of FN pairs caused by the semantic continuity and semantic overlap properties, and the results show large potential limitations in the DCRL. a) For the PN pairs caused by the “Semantic continuity”, we follow the SimCLR \cite{chen2020simple} which pairs the negative features for each feature. We pair the features in different positions of image A’s foreground regions as negative pairs. The result shows that 17.79\% of the negative pairs are FN pairs which have the same semantics. Although the existing DCRL methods utilize attention \cite{wang2022exploring} or clustering \cite{li2021dense} to avoid directly dividing adjacent pixel-wise features as negative pairs, the FN caused by “semantic continuity” is still an open and challenging problem. b) For the FN pairs caused by the “Semantic overlap”, we follow the DenseCL \cite{wang2022densecl} which pairs the current features and the memory bank features as the negative pairs. We make the features of image B in the foreground as the memory bank features and the features of image A in the foreground as the current features. Then, we pair the current and memory bank features as negative pairs and calculate the accuracy. Finally, 17.53\% of the negative pairs are FN pairs which have the same semantics. The “Semantic overlap” property of the medical images makes it inevitable that there will be numerous consistent semantic regions between medical images. Therefore, it will produce 17.53\% FN pairs in the training process making the model learn in an unreliable direction.


\begin{figure}[tb]
  \centering
  \includegraphics[width=\linewidth]{./picture/fitting.pdf}
  \caption{The FP and FN pairs have a serious impact on learning. a) The fitting process with FP and FN pairs on a cardiac CT image. b) The models' learned segmentation ability on the fitted case and their generalization ability on another testing case.}\label{supp:fig:fitting}
\end{figure}

According to the above probability of FP and FN pairs, we simulated the number of these FP and FN pairs in a supervised heart segmentation learning task. Specifically, we train a U-Net on the cardiac structures segmentation task with a cardiac CT image (Image A in Fig.\ref{supp:fig:fp}) to evaluate the fitting ability of the model with or without FP\&N pairs. a) In the non-FP\&N pairs setting, we utilize the contrastive segmentation learning like Wang et al. \cite{wang2019panet}. b) In the FP\&N pairs setting, we randomly generate FP (94\%) and FN (17\%) pairs in the contrastive segmentation learning. c) We further reduce the probabilities of FP and FN pairs to one-third of original (31\% and 6\%) to give an ablation of the false pairs’ degree. We take 1000 iterations, and draw the loss values of the learning process on a line chart to visualize the fitting process. We also evaluate the segmentation of the fitted case and another testing case (Image B in Fig.\ref{supp:fig:fp}) to evaluate the model learned representation with false pairs.

As shown in Fig.\ref{supp:fig:fitting}, the FP and FN pairs have a serious impact on learning. Without the FP\&N pairs, the model is able to be fitted to the target cardiac images, and learn the representation ability of the semantic regions. However, when learning with large-scale FP\&N pairs (94\%, 17\%), the model is unable to be fitted to the targets owing to the interference of the noisy optimization targets. When reducing the FP\&N degree to one-third, the model is able to be gradually fitted to the target image and has a certain generalization, but its performance is weaker than the ``no FP\&N" situation. Therefore, we can draw the following two conclusions in DCRL: a) the large-scale FP\&N problem will make the model unable to learn representation; b) alleviating the FP\&N degree, the model will be able to learn the representation ability of data with generalization ability. Therefore, our GEMINI embeds the homeomorphism prior to the DCRL for the large-scale FP\&N problem, enhancing the learning of true feature pairs. Although it is challenging to remove FP\&N pairs without annotation, reducing the FP\&N degree via our GEMINI is still able to guides the model to learn a generalizable representation.

\subsection*{B.2 Discussion of the novelty in GEMINI}
The proposed GEMINI is a novel dense contrastive representation learning paradigm in medical image analysis. Not only in the innovations, i.e., our DHL and GSS, it also achieved great novelty in principle.

\emph{In principle}, our GEMINI has advanced the theoretical foundation of homeomorphism for the dense contrastive representation learning, providing a principle inspiration to the community. It modeled the human consistent anatomy in medical images based on the principle of topologie \cite{hubbard1991differential}, proposed a new principal concept, homeomorphism prior, and formulated it in the DCRL task as a new paradigm. Therefore, the community will be further inspired by our principle of homeomorphism and make new scientific and technological progress in other tasks and fields.

\emph{In methodology}, our work has proposed a novel dense contrastive representation learning framework that enables the contrast of feature pairs under the condition of human inherent topology, thus promoting the DCRL in medical images. It modeled the consistency of human inherent topology (i.e., homeomorphism prior) as a learning for deformable mapping to overcome the reliability issue in DCRL’s feature correspondence process, giving one potential answer to the long-standing question of “how to achieve a reliable dense feature correspondence for unlabeled data?” Based on the modeling, the proposed DHL and GSS bring soft learning of feature pairs and reliable learning of positive pairs, promoting the contrast of features in DCRL. Finally, our work has achieved a new ability to learn reliable semi-supervised medical image segmentation and pre-training models.

\section*{C More framework analysis and experiment discussion}
\subsection*{C.1 Discussion of the receptive field $r$ in the Deformer network}
\begin{figure}[tb]
  \centering
  \includegraphics[width=\linewidth]{./picture/receptfield.pdf}
  \caption{The ablation study of the receptive field size $r$ and the network parameter amounts. a) The segmentation performance on the T1 of FS-Semi setting with the increasing of the receptive field size $r$. b) The fine-tuning performance with the enlarging of the parameter amount (million, $M$) in the pre-trained networks.}\label{supp:fig:rece}
\end{figure}

The performance is robust for the receptive field $r$. As shown in Fig.\ref{supp:fig:rece} a), we enlarge the receptive field $r$ via adding the depth and down-sampling stages of the “Deformer” network and evaluate the model's performance on T1 of FS-Semi setting. With the enlarging of the receptive field, the models’ performance is stalely around 90\% DSC. Because the backbone network and “Deformer” network together constitute a whole network to learn the feature representation, and the features from the backbone network have been extracted from a large receptive field. Therefore, even the receptive field of the “Deformer” network is small, the final DVF is still calculated from a large receptive field. The layers inner the backbone is still optimized by the gradient with a big reception, so that our GEMINI keeps stable performance with the enlarging of $r$. Owing to the soft learning of feature pairs in our DHL, once added this module, the framework achieves more than 5\% DSC improvement.

\subsection*{C.2 Discussion of the parameter amount}
As shown in Fig.\ref{supp:fig:rece} b), we have evaluated our GEMINI on different settings of model parameters. We pre-trained our GEMINI-MIP on the ChestX-ray8 dataset for the networks with 0.12$M$, 0.49$M$, 1.97$M$, 7.85$M$, 31.39$M$, and 125.52$M$ ($M$ is million) parameters, and fine-tuned them on the T1: SCR25 task. With the enlarging of the network, the model performance is improving quickly. This is because the network capacity increases with the enlarging of the networks so that it will be able to learn the representation of more features in the pre-training process. When the parameter amount is 1.97M, the speed of performance improving is reduced, illustrating that the increase of network capacity has approached the upper bound of this task. Therefore, when the network is further enlarged to 125.52$M$ (more than 50 times compared with 1.97$M$), the performance is only improved 1.4\% DSC. %Considering the cost of model training, we all utilize the networks with 1.97$M$ parameters as the backbone network for all experiments.

\subsection*{C.3 Discussion of the feature distribution}
\begin{figure}[tb]
  \centering
  \includegraphics[width=\linewidth]{./picture/points.pdf}
  \caption{The t-SNE visualization of the learned pixel representations. We provide the coordinates of pixels in a zoomed view, indicating their spatial relationship.}\label{supp:fig:distri}
\end{figure}

As shown in Fig.\ref{supp:fig:distri}, we visualize the learned representation by our framework to demonstrate its effectiveness in distinguishing different semantic regions. In the three tasks of the SS-MIP experiment, we randomly select the slices or patches from the test datasets and extract their pixel-wise features via the backbone network initialized from scratch (a) and our GEMINI-MIP (b). Then, these features are zoomed by t-SNE \cite{van2008visualizing} to two dimensions. As demonstrated in the enlarged region, the features from the ``Scratch" model is mixed owing to its initial weak representation. The pixel-wise features from our framework are clustered into several meaningful groups. Most of the pixels in each group are spatially close and in different groups are also spatially separated (indicated by their coordinates (c)) in the original image. Because our GEMINI discovers the correspondence of pixel-wise features based on the homeomorphism of human body and learns the representation according to the consistent context topology, the same semantic features which are spatially close will be clustered.

\subsection*{C.4 Discussion of the cross-architecture compatibility}
\begin{table*}[tb]
\centering
\caption{The FS-Semi evaluations on U-Net \cite{ronneberger2015u}, TransUNet \cite{chen2021transunet}, and SwinUNet \cite{cao2022swin} demonstrate the cross-architecture compatibility of our GEMINI. The ``-" means that the setting is unable to be implemented.}
%\resizebox{\textwidth}{!}
{
\begin{tabular}{clccccccccccccccc}
  \toprule
  \multirow{2}{*}{\textbf{Type}}
  &\multirow{2}{*}{\textbf{Method}}
  &\multicolumn{2}{c}{\textbf{T1: 3D cardiac structures}}
  &\multicolumn{2}{c}{\textbf{T2: 3D brain tissues}}
  &\multicolumn{2}{c}{\textbf{T3: 2D chest structures}}
  &\textbf{AVG}\\ \cmidrule(r){3-4}\cmidrule(r){5-6}\cmidrule(r){7-8}\cmidrule(r){9-9}
  &
  &DSC$_{\pm std}\uparrow$
  &AVD$_{\pm std}\downarrow$
  &DSC$_{\pm std}\uparrow$
  &AVD$_{\pm std}\downarrow$
  &DSC$_{\pm std}\uparrow$
  &AVD$_{\pm std}\downarrow$
  &DSC$_{\pm std}\uparrow$
  \\
  \midrule
  \textbf{Five}
  &U-Net \cite{ronneberger2015u}
  &84.3$_{\pm9.6}$
  &2.43$_{\pm2.14}$
  &69.5$_{\pm8.8}$
  &1.59$_{\pm0.84}$
  &83.4$_{\pm6.9}$%{\color{purple}$^{*}$}
  &10.34$_{\pm4.80}$
  &79.1$_{\pm8.4}$
  \\
  \textbf{(Lower)}
  & TransUNet \cite{chen2021transunet}
  & 74.5$_{\pm8.3}$
  & 4.41$_{\pm1.39}$
  & 67.4$_{\pm5.4}$
  & 2.02$_{\pm0.46}$
  & 76.5$_{\pm8.2}$
  & 16.59$_{\pm6.53}$
  & 72.8$_{\pm7.3}$
  \\
  & SwinUNet \cite{cao2022swin}
  & 40.8$_{\pm8.0}$
  & 11.59$_{\pm1.32}$
  & 67.8$_{\pm5.3}$
  & 4.04$_{\pm0.39}$
  & 63.9$_{\pm11.5}$
  & 14.26$_{\pm8.91}$
  & 57.5$_{\pm8.3}$
  \\
  \cdashline{1-9}[0.8pt/2pt]
  \textbf{Full}
  &U-Net \cite{ronneberger2015u}
  &-
  &-
  &88.7$_{\pm1.2}$
  &0.31$_{\pm0.04}$
  &96.1$_{\pm1.4}$%{\color{purple}$^{*}$}
  &2.28$_{\pm1.00}$
  &-
  \\
   \textbf{(Upper)}
  & TransUNet \cite{chen2021transunet}
  & -
  & -
  & 85.7$_{\pm1.2}$
  & 0.43$_{\pm0.05}$
  & 95.2$_{\pm2.1}$
  & 2.78$_{\pm1.35}$
  & -
  \\
  & SwinUNet \cite{cao2022swin}
  & -
  & -
  & 82.8$_{\pm2.7}$
  & 0.54$_{\pm0.15}$
  & 95.3$_{\pm1.2}$
  & 2.17$_{\pm0.65}$
  & -
  \\
  \cdashline{1-9}[0.8pt/2pt]
  \textbf{Semi}
  &\textbf{GEMINI+U-Net}
  &91.2$_{\pm3.6}$
  &0.97$_{\pm0.56}$
  &87.3$_{\pm1.0}$
  &0.35$_{\pm0.03}$
  &87.7$_{\pm5.2}$
  &7.14$_{\pm3.63}$
  &88.7$_{\pm3.3}$
  \\
   \textbf{(Ours)}
  &\textbf{GEMINI+TransUNet}
  &90.8$_{\pm3.4}$
  &0.94$_{\pm0.51}$
  &84.4$_{\pm1.3}$
  &0.45$_{\pm0.05}$
  &88.4$_{\pm5.7}$
  &8.63$_{\pm4.68}$
  &87.9$_{\pm3.5}$
  \\
  &\textbf{GEMINI+SwinUNet}
  &88.6$_{\pm4.2}$
  &1.28$_{\pm0.64}$
  &79.9$_{\pm5.0}$
  &0.62$_{\pm0.20}$
  &86.2$_{\pm7.8}$
  &6.34$_{\pm4.34}$
  &84.9$_{\pm5.7}$
  \\
  \bottomrule
\end{tabular}
}
\label{tab:arch}
\end{table*}
As shown in the Tab.\ref{tab:arch}, we perform the TransUNet \cite{chen2021transunet} (CNN+transformer), SwinUNet \cite{cao2022swin} (Transformer), and U-Net \cite{ronneberger2015u} (CNN) on the FS-Semi tasks with three datasets, and our GEMINI demonstrates a great compatibility across these model architectures. There are two observations: \textbf{a)} Our GEMINI has achieved significant improvement on both TransUNet \cite{chen2021transunet}, SwinUNet \cite{cao2022swin}, and U-Net \cite{ronneberger2015u}. Compared with the lower bound of the architectures that are trained only with five labeled images, our GEMINI has improved them more than 9\% DSC on AVG owing to our learning of the  homeomorphism mapping between medical images. Especially, for the 3D brain tissues segmentation, GEMINI achieves a similar performance compared with the FULL setting (83 labels) only with 5 labels in all architectures, demonstrating our great potential in reducing of annotation costs. \textbf{b)} Our GEMINI has great architecture compatibility across CNN-based (U-Net [3]), transformer-based (SwinUNet [2]), and CNN-transformer-based (TransUNet) networks. For U-Net and TransUNet that utilizes CNN to encode and decode features, our GEMINI has similar significant improvement that achieves 88.7\% and 87.9\% on AVG DSC. SwinUNet takes patch-embedding and four-times down sampling at the beginning, and utilizes the shifted window to learn global features. Therefore, it is challenging to represent fine-grained dense features and makes the whole network easy to overfit to the global features when the amount of training cases is small. As a result, SwinUNet has very poor performance on “FIVE” setting. When adding GEMINI, it learns the inter-image consistency for unlabeled images and effectively reduces the over-fitting, thus also achieving more than 20\% DSC improvement.

\subsection*{C.5 Discussion of the computing costs}
\begin{table}[tb]
\centering
\caption{Owing to the additional deformer networks, our GEMINI has relatively higher computing costs in the pre-training stage, but it has same computing costs in the fine-tuning for downstream tasks as other methods and achieves much higher performance.}
\resizebox{\linewidth}{!}
{
\begin{tabular}{clccccccccccccccc}
  \toprule
  \multirow{2}{*}{\textbf{Type}}
  &\multirow{2}{*}{\textbf{Method}}
  &\textbf{Pre-training}
  &\textbf{Downstream}
  &\textbf{T1: SCR$_{25}$}
  \\
  \cmidrule(r){3-3}\cmidrule(r){4-4}\cmidrule(r){5-5}
  &
  & FLOPs
  & FLOPs
  & DSC$_{\pm std}$
  \\
  \midrule
  1$\times$Encoder
  &Rotation \cite{komodakis2018unsupervised}
  &5.99G
  &20.15G
  &80.5$_{\pm7.7}$
  \\
  2$\times$Encoder
  &BYOL \cite{grill2020bootstrap}
  &11.98G
  &20.15G
  &89.4$_{\pm4.9}$
  \\
  1$\times$Encoder-decoder
  &Model Genesis \cite{zhou2019models}
  &19.74G
  &20.15G
  &86.1$_{\pm4.6}$
  \\
  2$\times$Encoder-decoder
  &VADeR \cite{o2020unsupervised}
  &39.67G
  &20.15G
  &85.2$_{\pm5.1}$
  \\
  \cdashline{1-5}[0.8pt/2pt]
  2$\times$Encoder-decoder
  &\textbf{Our GEMINI}
  &\textbf{52.59G}
  &20.15G
  &\textbf{92.1$_{\pm2.8}$}
  \\
  \bottomrule
\end{tabular}
}
\label{tab:computing}
\end{table}
As shown in Tab.\ref{tab:computing}, we compare the methods’ number of floating-point operations (FLOPs) in four architecture types in pre-training stage, i.e., ``1$\times$Encoder" that only runs an encoder in the pre-training, here, we take the Rotation method \cite{komodakis2018unsupervised}; ``2$\times$Encoder" that runs two encoders for contrastive representation learning in the pre-training, here we take the BYOL \cite{grill2020bootstrap}; ``1$\times$Encoder-decoder" that runs an encoder-decoder network, like the U-Net, in the pre-training, here we take the Model Genesis \cite{zhou2019models}; ``2$\times$Encoder-decoder" that runs two encoder-decoder networks for dense contrastive representation learning in the pre-training, here we take the VADeR \cite{o2020unsupervised}. Our GEMINI is also a ``2$\times$Encoder-decoder" method. In the pre-training stage, all methods take U-Net (for Encoder-decoder) or the encoder part of the U-Net (for Encoder) as their backbone. In the downstream adaptation stage, all methods’ pre-trained parameters are used to initialize the U-Net to learn segmentation task (T1: SCR$_{25}$) and the part without pre-training is initialized randomly. All methods utilize same input sizes with [300$\times$300] in pre-training stage and [512$\times$512] in downstream stage. Owing to two additional deformer networks to learning the homeomorphism mapping, our GEMINI has the highest FLOPs in the pre-training stage, but it greatly contributes to the pre-training performance. In the downstream stages, owing to all methods take the U-Net with same parameter amount, our GEMINI has same FLOPs as other methods. As a results, our GEMINI has very significant performance improvement on the SCR$_{25}$ task owing our reliable learning for positive and negative feature pairs. The large-scale FP\&N problem in VADeR makes it has worse performance than the BYOL even it has larger FLOPs in pre-training.

The additional the one-time cost of our GEMINI in the pre-training stage bring obtain better representation, effectively reducing the long-time computing costs in the downstream tasks. Because once the pre-training is completed, stronger representation will accelerate the convergence speed of the model on downstream tasks, thus reducing the long-time computing cost in the training of numerous downstream tasks. As illustrated in the Fig.12 of our paper, compared with the BYOL \cite{grill2020bootstrap}, DenseCL \cite{wang2022densecl}, Model Genesis \cite{zhou2019models}, our GEMINI achieved better performance with fewer iterations, illustrating its potential in reducing the computing costs in downstream tasks.

\subsection*{C.6 Discussion of the second-best models}
Compared with the second-best methods (BRBS \cite{he2022learning} in Tab.2 and our CVPR version, GVSL \cite{He_2023_CVPR}, in Tab.3), we can find these methods also fused the homeomorphism prior into their framework, and their great performance demonstrates the great potential of this prior knowledge in medical images. In our Experiment 1: FS-Semi (Sec.4), the BRBS is a “learning registration to learn segmentation” method whose registration part is based on the homeomorphism prior. Therefore, its powerful performance in the T1: 3D cardiac structures and T2: 3D brain tissues illustrate the advantages. However, the BRBS’s visual similarity make it unable to generalize to the chest X-ray (T3: 2D chest structures) that has relatively low contrast as illustrated in Fig.8. Our GEMINI utilize the semantic similarity based on features and achieves significant improvement on this task, demonstrating our superiority. In our Experiment 2: SS-MIP (Sec.4), our CVPR version, GVSL, benefits from our homeomorphism prior, achieving second-best performance on the T2: KiPA22 and T3: CANDI. However, it also utilizes the visual similarity which is limited on the low-contrast images, i.e., the chest x-ray images in the pre-training dataset. Therefore, its pre-trained representation for chest x-ray is relatively weaker and limits its performance in the inner-scene transferring. Our GSS improves the measurement of the correspondence degree, and drive the representation learning for low-contrast targets during pre-training. Therefore, our GEMINI has significantly improved the GVSL’s performance on the T1: SCR$_{25}$ task.

\subsection*{C.7 Discussion of the reliability}
\begin{table*}[tb]
\centering
\caption{The evaluation of our GEMINI's reliability on the tasks in Experiment 1. The \emph{Cor} is the Pearson correlation coefficient \cite{cohen2009pearson}, and the \emph{p} is the p-value.}
%\resizebox{\linewidth}{!}
{
\begin{tabular}{lccccccccccccccc}
  \toprule
  \diagbox{Evaluations}{Tasks}
  &\multicolumn{2}{c}{\textbf{T1: 3D cardiac structures}}
  &\multicolumn{2}{c}{\textbf{T2: 3D brain tissues}}
  &\multicolumn{2}{c}{\textbf{T3: 2D chest structures}}
  &\multicolumn{2}{c}{\textbf{AVG}}
  \\
  \midrule
  \multirow{2}{*}{\textbf{a) Reliability across samples}}
  &\textbf{DSC} $\uparrow$
  &\textbf{std} $\downarrow$
  &\textbf{DSC} $\uparrow$
  &\textbf{std} $\downarrow$
  &\textbf{DSC} $\uparrow$
  &\textbf{std} $\downarrow$
  &\textbf{DSC} $\uparrow$
  &\textbf{std} $\downarrow$
  \\
  &91.2
  &3.6
  &87.3
  &1.0
  &87.7
  &5.2
  &88.7
  &3.3
  \\
  \midrule
  \multirow{2}{*}{\textbf{b) Reliability across training}}
  &\textbf{\emph{Cor}} $\uparrow$
  &\textbf{\emph{p}} $\downarrow$
  &\textbf{\emph{Cor}} $\uparrow$
  &\textbf{\emph{p}} $\downarrow$
  &\textbf{\emph{Cor}} $\uparrow$
  &\textbf{\emph{p}} $\downarrow$
  &\textbf{\emph{Cor}} $\uparrow$
  &\textbf{\emph{p}} $\downarrow$
  \\
  &0.989
  & $<$0.001
  &0.999
  & $<$0.001
  &0.968
  & $<$0.001
  &0.985
  & $<$0.001
  \\
  \bottomrule
\end{tabular}
}
\label{tab:reliability}
\end{table*}
As shown in Tab.\ref{tab:reliability}, in the three tasks of our Experiment 1, we calculated the standard deviations (std) and the inter-training Pearson correlation coefficients (Cor) \cite{cohen2009pearson}. The results indicate that our GEMINI demonstrates strong reliability across different tested samples and training initializations. \emph{a) Reliability across samples:} We evaluated the DSC and std of the performance across the tested samples. Our GEMINI-Semi achieved an average of 88.7\% DSC with a 3.3 std, indicating high performance with robustness across diverse samples, which supports its reliability in real-world applications. \emph{b) Reliability across training:} We conducted a test-retest reliability analysis \cite{guttman1945basis} and reported the Cor for the performance when our GEMINI-Semi was trained twice from different initialization states. The Cors for all three tasks exceeded 0.95 demonstrating very high consistency between the two training sessions. Additionally, all p-values were below 0.001, indicating significant consistency. Thus, our model shows excellent reliability across different initialization states, which supports its reliability in model implementation.

\section*{D More details in experiments}

\subsection*{D.1 Details of the training diagram}
\begin{figure}[tb]
  \centering
  \includegraphics[width=0.9\linewidth]{./picture/training.pdf}
  \caption{The overall training diagram of our GEMINI. a) The inference process of the whole framework. The gray path in the last line is the additional learning part in the variants of our GEMINI in self-supervised pre-training (GEMINI-MIP) and semi-supervised segmentation (GEMINI-Semi). b) The loss calculation to optimize the whole framework.}\label{supp:fig:diagram}
\end{figure}
\begin{figure*}[!]
  \centering
  \includegraphics[width=\linewidth]{./picture/details_framework.pdf}
  \caption{The detailed architecture of our GEMINI. a) The backbone architecture utilizes the 3D U-Net in 3D image tasks and 2D U-Net in 2D image tasks. b) The deformer network architecture utilized a lightweight U-Net. c-d) The segmentation head in the variant of GEMINI-Semi and the self-restoration head in the variant of GEMINI-MIP.}\label{supp:fig:architecture}
\end{figure*}
As shown in Fig.\ref{supp:fig:diagram}, the training diagram introduces the details of our GEMINI's variants in SSP and Semi experiments. In the forward inference, as described in the ``Methodology" section of our paper, two images $x^{A}, x^{B}$ are put into two shared-weight backbones $\mathcal{N}_{\theta}$ separately to extract the features $f^A, f^B$. The features are further put into two shared-weight deformers together to predict two DVFs $\psi^{AB}, \psi^{BA}$ that are bidirectional. For the variant of GEMINI-Semi, a labeled image $x^C$ is put into the shared-weight backbone $\mathcal{N}_{\theta}$, and then put into an additional segmentation head $Seg_{\kappa}$ to predict the segmentation results $\hat{y}^{C}_{Seg}$. For the variant of GEMINI-MIP, an appearance transformed image $x^C$ (described in Sec.5.1.1) is put into the shared-weight backbones $\mathcal{N}_{\theta}$, and then put into an additional self-restoration head $Res_{\tau}$ to predict the restored image $\hat{y}^{C}_{Res}$. In the loss calculation, the smooth loss (Equ.3) is calculated on the DVFs $\psi^{AB}, \psi^{BA}$ to learn the continuity of the deformable mapping, the GVS loss ($\mathcal{L}_{GVS}$, Equ.6) and GSS loss ($\mathcal{L}_{GSS}$, Equ.7) are calculated on the deformed images $x^{AB}=\psi^{AB}(x^A), x^{BA}=\psi^{BA}(x^B)$ and deformed features $f^{AB}=\psi^{AB}(f^A), f^{BA}=\psi^{BA}(f^B)$ (described in Sec.3.3) to learn the correspondence. For the variant of GEMINI-Semi, a segmentation loss $\mathcal{L}_{Seg}$ is calculated on segmentation result $\hat{y}^{C}_{Seg}$ and the groundtruth $y^{C}_{Seg}$. For the variant of GEMINI-MIP, a self-restoration loss $\mathcal{L}_{Res}$ is calculated on the restored image $\hat{y}^{C}_{Res}$ and the original image $y^{C}_{Res}$. The learning of self-restoration in SSP is a fundamental task for a warm-up of our GSS, due to the initial weak representation in the pretext task.

\subsection*{D.2 Details of the architectures and implementation}
As shown in Fig.\ref{supp:fig:architecture}, we utilize the U-Net \cite{ronneberger2015u} architecture (3D U-Net for 3D images and 2D U-Net for 2D images) as our backbone architecture for great basic dense representation in our experiment. In the encoding path, it takes max pooling layers to reduce the feature maps' resolution and in the decoding path, it takes up-sampling layers (bilinear for 2D images and trilinear for 3D images) to restore the features' resolution. Skip connections are used to transmit features from the encoding path to the decoding path in each resolution stage. There are five resolution stages in the network and each stage utilizes Conv-GN-LeckyReLU\footnote{Conv is a convolution layer and GN is a group normalization layer \cite{wu2018group}.} modules to extract features. The deformer network also takes a lightweight U-Net architecture with very shallow depth to estimate the DVF. It only has three resolution stages and each stage has half of the Conv-GN-LeckyReLU module amount compared with the backbone. Both the segmentation head and self-restoration head take one Conv-GN-LeckyReLU module to project the input features and follow a convolution layer to predict the targets. The detailed hyper-parameters inner these architectures are marked in Fig.\ref{supp:fig:architecture}.






\bibliographystyle{IEEEtran}
\bibliography{mybib}
%\begin{IEEEbiography}[{\includegraphics[width=0.8in,height=1in,clip,keepaspectratio]{bio_images/xinweiliu.jpg}}]
{Xinwei Liu} 
is a Ph.D. student in the Institute of Information Engineering, Chinese Academy of Sciences and the School of Cyber Security, University of Chinese Academy of Sciences, Beijing. His research interests include computer vision, deep learning and adversarial machine learning.
\end{IEEEbiography}

% \begin{IEEEbiography}
% [{\includegraphics[width=0.8in,height=1in,clip,keepaspectratio]{bio_images/Siyuan Liang}}]
% {Siyuan Liang} 
% is 
% \end{IEEEbiography}

\begin{IEEEbiography}[{\includegraphics[width=0.8in,height=1in,clip,keepaspectratio]{bio_images/xiaojunjia.jpg}}]
{Xiaojun Jia} 
received his Ph.D. degree in  State Key Laboratory of Information Security, Institute
of Information Engineering, Chinese Academy of Sciences and School of
Cyber Security, University of Chinese Academy of Sciences, Beijing. He is now a Research Fellow in Cyber Security Research Centre @ NTU, Nanyang Technological University, Singapore. His research interests include computer vision, deep learning and adversarial machine learning.
\end{IEEEbiography}

\begin{IEEEbiography}
[{\includegraphics[width=0.8in,height=1in,clip,keepaspectratio]{bio_images/yuanxun.jpg}}]
{Yuan Xun} 
is a Ph.D. student in the Institute of Information Engineering, Chinese Academy of Sciences and the School of Cyber Security, University of Chinese Academy of Sciences, Beijing. Her research interests include computer vision, deep learning and adversarial machine learning.
\end{IEEEbiography}



\begin{IEEEbiography}[{\includegraphics[width=0.8in,height=1in,clip,keepaspectratio]{bio_images/zhanghua.jpg}}]{Hua Zhang} is a professor with the Institute of Information Engineering, Chinese Academy of Sciences. He received the Ph.D. degree in computer science from the School of Computer Science and Technology, Tianjin University, Tianjin, China in 2015. His research interests include computer vision, multimedia, and machine learning.
\end{IEEEbiography}



\begin{IEEEbiography}
[{\includegraphics[width=0.8in,height=1in,clip,keepaspectratio]{bio_images/xiaochuncao.jpg}}]
{Xiaochun Cao}(SM'14)
received the B.S. and M.S. degrees in computer science from Beihang University, Beijing, China, and the Ph.D. degree in computer science from the University of Central Florida, Orlando, FL, USA. After graduation, he spent about three years at ObjectVideo Inc. as a Research Scientist. He is with the School of Cyber Science and Technology, Shenzhen Campus, Sun Yat-sen University, Shenzhen 518107, P.R. China. He has authored and co-authored more than 100 journal and conference papers.
Prof. Cao is a Fellow of the IET. He is on the Editorial Boards of the IEEE Transactions on Image Processing, IEEE Transactions on Multimedia, IEEE Transactions on Circuits and Systems for Video Technology. His dissertation was nominated for the University of Central Florida's university-level Outstanding Dissertation Award. In 2004 and 2010, he was the recipient of the Piero Zamperoni Best Student Paper Award at the International Conference on Pattern Recognition.
\end{IEEEbiography}


\end{document}


