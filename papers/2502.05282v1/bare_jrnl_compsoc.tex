

\documentclass[10pt,journal,compsoc]{IEEEtran}

\ifCLASSINFOpdf
  % \usepackage[pdftex]{graphicx}
  % declare the path(s) where your graphic files are
  % \graphicspath{{../pdf/}{../jpeg/}}
  % and their extensions so you won't have to specify these with
  % every instance of \includegraphics
  % \DeclareGraphicsExtensions{.pdf,.jpeg,.png}
\else
  % or other class option (dvipsone, dvipdf, if not using dvips). graphicx
  % will default to the driver specified in the system graphics.cfg if no
  % driver is specified.
  % \usepackage[dvips]{graphicx}
  % declare the path(s) where your graphic files are
  % \graphicspath{{../eps/}}
  % and their extensions so you won't have to specify these with
  % every instance of \includegraphics
  % \DeclareGraphicsExtensions{.eps}
\fi

\hyphenation{op-tical net-works semi-conduc-tor}
\usepackage{booktabs}
\usepackage{multirow}
\usepackage{arydshln}
\usepackage{makecell}
\usepackage{enumitem}
\usepackage{diagbox}
%\usepackage{wasysym}
\usepackage{amssymb}
\usepackage{graphicx}
\usepackage{soul}
\usepackage{url}
\usepackage{amsmath}
\usepackage{cite}
\usepackage{amsthm}
\usepackage{hyperref}

\usepackage{algorithm}
\usepackage{algorithmic}
\usepackage{multirow}
\usepackage{marvosym}
\usepackage{threeparttable}
\usepackage{array,subfig}

\usepackage{mathrsfs}
\usepackage{color,xcolor}
\usepackage{colortbl}
\begin{document}
\title{Homeomorphism Prior for False Positive and Negative Problem in Medical Image Dense Contrastive Representation Learning}

\author{Yuting He,
        Boyu Wang,
        Rongjun Ge,
        Yang Chen,
        Guanyu Yang\IEEEauthorrefmark{1},
        Shuo Li
\thanks{\vspace*{-1\baselineskip} \newline {\emph{\IEEEauthorrefmark{1}Corresponding authors: G. Yang. (e-mail: yang.list@seu.edu.cn)}}}
\thanks{This research was supported by the National Natural Science Foundation of China (Grant No. 82441021), the Natural Science Foundation of Jiangsu Province (Grant No. BK20210291), the National Natural Science Foundation of China (Grant No. 62101249, T2225025), the Jiangsu Shuangchuang Talent Program (Grant No. JSSCBS20220202).}
\IEEEcompsocitemizethanks{
\IEEEcompsocthanksitem Y. He, Y. Chen and G. Yang\IEEEauthorrefmark{1} are with the Key Laboratory of New Generation Artificial Intelligence Technology and Its Interdisciplinary Applications (Southeast University), Ministry of Education, Nanjing, China, the Centre de Recherche en Information Biomédicale Sino-Français (CRIBs), and Jiangsu Provincial Joint International Research Laboratory of Medical Information Processing, Nanjing, China. (e-mail: yang.list@seu.edu.cn)
\IEEEcompsocthanksitem R. Ge is with the School of Instrument Science and Engineering, Southeast University, Nanjing, China (e-mail: rongjun\_ge@seu.edu.cn)
\IEEEcompsocthanksitem B. Wang is with the Department of Computer Science, Western University, London, ON N6A 3K7, Canada. (e-mail: bwang@csd.uwo.ca)
\IEEEcompsocthanksitem S. Li and Y. He are with the Department of Biomedical Engineering and the Department of Computer and Data Science, Case Western Reserve University, Cleveland, OH 44106 USA (e-mail: shuo.li11@case.edu).
}% <-this % stops an unwanted space
\thanks{Manuscript received April 19, 2005; revised August 26, 2015.}}

% The paper headers
\markboth{Journal of \LaTeX\ Class Files,~Vol.~14, No.~8, August~2015}%
{Shell \MakeLowercase{\textit{et al.}}: Bare Demo of IEEEtran.cls for Computer Society Journals}

\IEEEtitleabstractindextext{%

\begin{abstract}  
Test time scaling is currently one of the most active research areas that shows promise after training time scaling has reached its limits.
Deep-thinking (DT) models are a class of recurrent models that can perform easy-to-hard generalization by assigning more compute to harder test samples.
However, due to their inability to determine the complexity of a test sample, DT models have to use a large amount of computation for both easy and hard test samples.
Excessive test time computation is wasteful and can cause the ``overthinking'' problem where more test time computation leads to worse results.
In this paper, we introduce a test time training method for determining the optimal amount of computation needed for each sample during test time.
We also propose Conv-LiGRU, a novel recurrent architecture for efficient and robust visual reasoning. 
Extensive experiments demonstrate that Conv-LiGRU is more stable than DT, effectively mitigates the ``overthinking'' phenomenon, and achieves superior accuracy.
\end{abstract}  
}

\maketitle

\IEEEdisplaynontitleabstractindextext
\IEEEpeerreviewmaketitle


\section{Introduction}

Large language models (LLMs) have achieved remarkable success in automated math problem solving, particularly through code-generation capabilities integrated with proof assistants~\citep{lean,isabelle,POT,autoformalization,MATH}. Although LLMs excel at generating solution steps and correct answers in algebra and calculus~\citep{math_solving}, their unimodal nature limits performance in plane geometry, where solution depends on both diagram and text~\citep{math_solving}. 

Specialized vision-language models (VLMs) have accordingly been developed for plane geometry problem solving (PGPS)~\citep{geoqa,unigeo,intergps,pgps,GOLD,LANS,geox}. Yet, it remains unclear whether these models genuinely leverage diagrams or rely almost exclusively on textual features. This ambiguity arises because existing PGPS datasets typically embed sufficient geometric details within problem statements, potentially making the vision encoder unnecessary~\citep{GOLD}. \cref{fig:pgps_examples} illustrates example questions from GeoQA and PGPS9K, where solutions can be derived without referencing the diagrams.

\begin{figure}
    \centering
    \begin{subfigure}[t]{.49\linewidth}
        \centering
        \includegraphics[width=\linewidth]{latex/figures/images/geoqa_example.pdf}
        \caption{GeoQA}
        \label{fig:geoqa_example}
    \end{subfigure}
    \begin{subfigure}[t]{.48\linewidth}
        \centering
        \includegraphics[width=\linewidth]{latex/figures/images/pgps_example.pdf}
        \caption{PGPS9K}
        \label{fig:pgps9k_example}
    \end{subfigure}
    \caption{
    Examples of diagram-caption pairs and their solution steps written in formal languages from GeoQA and PGPS9k datasets. In the problem description, the visual geometric premises and numerical variables are highlighted in green and red, respectively. A significant difference in the style of the diagram and formal language can be observable. %, along with the differences in formal languages supported by the corresponding datasets.
    \label{fig:pgps_examples}
    }
\end{figure}



We propose a new benchmark created via a synthetic data engine, which systematically evaluates the ability of VLM vision encoders to recognize geometric premises. Our empirical findings reveal that previously suggested self-supervised learning (SSL) approaches, e.g., vector quantized variataional auto-encoder (VQ-VAE)~\citep{unimath} and masked auto-encoder (MAE)~\citep{scagps,geox}, and widely adopted encoders, e.g., OpenCLIP~\citep{clip} and DinoV2~\citep{dinov2}, struggle to detect geometric features such as perpendicularity and degrees. 

To this end, we propose \geoclip{}, a model pre-trained on a large corpus of synthetic diagram–caption pairs. By varying diagram styles (e.g., color, font size, resolution, line width), \geoclip{} learns robust geometric representations and outperforms prior SSL-based methods on our benchmark. Building on \geoclip{}, we introduce a few-shot domain adaptation technique that efficiently transfers the recognition ability to real-world diagrams. We further combine this domain-adapted GeoCLIP with an LLM, forming a domain-agnostic VLM for solving PGPS tasks in MathVerse~\citep{mathverse}. 
%To accommodate diverse diagram styles and solution formats, we unify the solution program languages across multiple PGPS datasets, ensuring comprehensive evaluation. 

In our experiments on MathVerse~\citep{mathverse}, which encompasses diverse plane geometry tasks and diagram styles, our VLM with a domain-adapted \geoclip{} consistently outperforms both task-specific PGPS models and generalist VLMs. 
% In particular, it achieves higher accuracy on tasks requiring geometric-feature recognition, even when critical numerical measurements are moved from text to diagrams. 
Ablation studies confirm the effectiveness of our domain adaptation strategy, showing improvements in optical character recognition (OCR)-based tasks and robust diagram embeddings across different styles. 
% By unifying the solution program languages of existing datasets and incorporating OCR capability, we enable a single VLM, named \geovlm{}, to handle a broad class of plane geometry problems.

% Contributions
We summarize the contributions as follows:
We propose a novel benchmark for systematically assessing how well vision encoders recognize geometric premises in plane geometry diagrams~(\cref{sec:visual_feature}); We introduce \geoclip{}, a vision encoder capable of accurately detecting visual geometric premises~(\cref{sec:geoclip}), and a few-shot domain adaptation technique that efficiently transfers this capability across different diagram styles (\cref{sec:domain_adaptation});
We show that our VLM, incorporating domain-adapted GeoCLIP, surpasses existing specialized PGPS VLMs and generalist VLMs on the MathVerse benchmark~(\cref{sec:experiments}) and effectively interprets diverse diagram styles~(\cref{sec:abl}).

\iffalse
\begin{itemize}
    \item We propose a novel benchmark for systematically assessing how well vision encoders recognize geometric premises, e.g., perpendicularity and angle measures, in plane geometry diagrams.
	\item We introduce \geoclip{}, a vision encoder capable of accurately detecting visual geometric premises, and a few-shot domain adaptation technique that efficiently transfers this capability across different diagram styles.
	\item We show that our final VLM, incorporating GeoCLIP-DA, effectively interprets diverse diagram styles and achieves state-of-the-art performance on the MathVerse benchmark, surpassing existing specialized PGPS models and generalist VLM models.
\end{itemize}
\fi

\iffalse

Large language models (LLMs) have made significant strides in automated math word problem solving. In particular, their code-generation capabilities combined with proof assistants~\citep{lean,isabelle} help minimize computational errors~\citep{POT}, improve solution precision~\citep{autoformalization}, and offer rigorous feedback and evaluation~\citep{MATH}. Although LLMs excel in generating solution steps and correct answers for algebra and calculus~\citep{math_solving}, their uni-modal nature limits performance in domains like plane geometry, where both diagrams and text are vital.

Plane geometry problem solving (PGPS) tasks typically include diagrams and textual descriptions, requiring solvers to interpret premises from both sources. To facilitate automated solutions for these problems, several studies have introduced formal languages tailored for plane geometry to represent solution steps as a program with training datasets composed of diagrams, textual descriptions, and solution programs~\citep{geoqa,unigeo,intergps,pgps}. Building on these datasets, a number of PGPS specialized vision-language models (VLMs) have been developed so far~\citep{GOLD, LANS, geox}.

Most existing VLMs, however, fail to use diagrams when solving geometry problems. Well-known PGPS datasets such as GeoQA~\citep{geoqa}, UniGeo~\citep{unigeo}, and PGPS9K~\citep{pgps}, can be solved without accessing diagrams, as their problem descriptions often contain all geometric information. \cref{fig:pgps_examples} shows an example from GeoQA and PGPS9K datasets, where one can deduce the solution steps without knowing the diagrams. 
As a result, models trained on these datasets rely almost exclusively on textual information, leaving the vision encoder under-utilized~\citep{GOLD}. 
Consequently, the VLMs trained on these datasets cannot solve the plane geometry problem when necessary geometric properties or relations are excluded from the problem statement.

Some studies seek to enhance the recognition of geometric premises from a diagram by directly predicting the premises from the diagram~\citep{GOLD, intergps} or as an auxiliary task for vision encoders~\citep{geoqa,geoqa-plus}. However, these approaches remain highly domain-specific because the labels for training are difficult to obtain, thus limiting generalization across different domains. While self-supervised learning (SSL) methods that depend exclusively on geometric diagrams, e.g., vector quantized variational auto-encoder (VQ-VAE)~\citep{unimath} and masked auto-encoder (MAE)~\citep{scagps,geox}, have also been explored, the effectiveness of the SSL approaches on recognizing geometric features has not been thoroughly investigated.

We introduce a benchmark constructed with a synthetic data engine to evaluate the effectiveness of SSL approaches in recognizing geometric premises from diagrams. Our empirical results with the proposed benchmark show that the vision encoders trained with SSL methods fail to capture visual \geofeat{}s such as perpendicularity between two lines and angle measure.
Furthermore, we find that the pre-trained vision encoders often used in general-purpose VLMs, e.g., OpenCLIP~\citep{clip} and DinoV2~\citep{dinov2}, fail to recognize geometric premises from diagrams.

To improve the vision encoder for PGPS, we propose \geoclip{}, a model trained with a massive amount of diagram-caption pairs.
Since the amount of diagram-caption pairs in existing benchmarks is often limited, we develop a plane diagram generator that can randomly sample plane geometry problems with the help of existing proof assistant~\citep{alphageometry}.
To make \geoclip{} robust against different styles, we vary the visual properties of diagrams, such as color, font size, resolution, and line width.
We show that \geoclip{} performs better than the other SSL approaches and commonly used vision encoders on the newly proposed benchmark.

Another major challenge in PGPS is developing a domain-agnostic VLM capable of handling multiple PGPS benchmarks. As shown in \cref{fig:pgps_examples}, the main difficulties arise from variations in diagram styles. 
To address the issue, we propose a few-shot domain adaptation technique for \geoclip{} which transfers its visual \geofeat{} perception from the synthetic diagrams to the real-world diagrams efficiently. 

We study the efficacy of the domain adapted \geoclip{} on PGPS when equipped with the language model. To be specific, we compare the VLM with the previous PGPS models on MathVerse~\citep{mathverse}, which is designed to evaluate both the PGPS and visual \geofeat{} perception performance on various domains.
While previous PGPS models are inapplicable to certain types of MathVerse problems, we modify the prediction target and unify the solution program languages of the existing PGPS training data to make our VLM applicable to all types of MathVerse problems.
Results on MathVerse demonstrate that our VLM more effectively integrates diagrammatic information and remains robust under conditions of various diagram styles.

\begin{itemize}
    \item We propose a benchmark to measure the visual \geofeat{} recognition performance of different vision encoders.
    % \item \sh{We introduce geometric CLIP (\geoclip{} and train the VLM equipped with \geoclip{} to predict both solution steps and the numerical measurements of the problem.}
    \item We introduce \geoclip{}, a vision encoder which can accurately recognize visual \geofeat{}s and a few-shot domain adaptation technique which can transfer such ability to different domains efficiently. 
    % \item \sh{We develop our final PGPS model, \geovlm{}, by adapting \geoclip{} to different domains and training with unified languages of solution program data.}
    % We develop a domain-agnostic VLM, namely \geovlm{}, by applying a simple yet effective domain adaptation method to \geoclip{} and training on the refined training data.
    \item We demonstrate our VLM equipped with GeoCLIP-DA effectively interprets diverse diagram styles, achieving superior performance on MathVerse compared to the existing PGPS models.
\end{itemize}

\fi 

\section{Related Work}

\noindent\textbf{Diffusion Efficiency Improvements:} 
\citet{das2023image} utilized the shortest path between two Gaussians and \citet{song2020denoising} generalized DDPMs via a class of non-Markovian diffusion processes to reduce the number of diffusion steps. \citet{nichol2021improved} introduced a few simple modifications to improve the log-likelihood. \citet{pandey2022diffusevae, pandey2021vaes} used DDPMs to refine VAE-generated samples. \citet{rombach2022high} performed the diffusion process in the lower dimensional latent space of an autoencoder to achieve high-resolution image synthesis, and \citet{liu2023audioldm} studied using such latent diffusion models for audio. \citet{popov2021grad} explored using a text encoder to extract better representations for continuous-time diffusion-based text-to-speech generation. More recently, \citet{nielsendiffenc} explored using a time-dependent image encoder to parameterize the mean of the diffusion process. Orthogonal to the above, PriorGrad \citep{lee2021priorgrad} and follow-up work \citep{koizumi22_interspeech} studied utilizing informative prior extracted from the conditioner data for improving learning efficiency. \textit{However, they become sub-optimal when the conditioner are degraded versions of the target data, posing challenges in applications like signal restoration tasks.}

\noindent\textbf{Diffusion-Based Signal Restoration:}
Built on top of the diffusion models for audio generation, e.g., \citet{kong2020diffwave,chen2020wavegrad,leng2022binauralgrad}, many SE models have been proposed. The pioneering work of \citet{lu2022conditional} introduced conditional DDPMs to the SE task and demonstrated the potential. Other works \citep{serra2022universal,welker2022speech,richter2023speech,yen2023cold,lemercier2023storm,tai2024dose} have also attempted to improve SE by exploiting diffusion models. In the vision domain, diffusion models have demonstrated impressive performance for IR tasks \citep{li2023diffusion,zhu2023denoising,huang2024wavedm,luo2023refusion,xia2023diffir,fei2023generative,hurault2022gradient,liu20232,chung2024direct,chungdiffusion,zhoudenoising,xiaodreamclean,zheng2024diffusion}. A notable IR work is \cite{ozdenizci2023restoring} that achieved impressive performance on several benchmark datasets for restoring vision in adverse weather conditions. \textit{Despite showing promising results, existing works have not fully exploited prior information about the data as they mostly settle on standard Gaussian priors.} 
\section{Methodology}
\subsection{Preliminary}
\label{sec:preliminary}
\mypara{Architecture of MLLM.}
% The MLLM architectures generally consist of three components: a visual encoder, a modality projector, and a LLM. The visual encoder, typically a pre-trained image encoder like CLIP's vision model, converts input images into visual tokens. The projector module aligns these visual tokens with the LLM's word embedding space, enabling the LLM to process visual data effectively. The LLM then integrates the aligned visual and textual information to generate responses.
The architecture of Multimodal Large Language Models (MLLMs) typically comprises three core components: a visual encoder, a modality projector, and a language model (LLM). Given an image $I$, the visual encoder and a subsequent learnable MLP are used to encode $I$ into a set of visual tokens $e_v$. These visual tokens $e_v$ are then concatenated with text tokens $e_t$ encoded from text prompt $p_t$, forming the input for the LLM. The LLM decodes the output tokens $y$ sequentially, which can be formulated as:
\begin{equation}
\label{eq1}
    y_i = f(I, p_t, y_0, y_1, \cdots, y_{i-1}).
\end{equation}

\mypara{Computational Complexity.}  
To evaluate the computational complexity of MLLMs, it is essential to analyze their core components, including the self-attention mechanism and the feed-forward network (FFN). The total floating-point operations (FLOPs) required can be expressed as:  
\begin{equation}
\text{Total FLOPs} = T \times (4nd^2 + 2n^2d + 2ndm),
\end{equation}  
where $T$ denotes the number of transformer layers, $n$ is the sequence length, $d$ represents the hidden dimension size, and $m$ is the intermediate size of the FFN.  
This equation highlights the significant impact of sequence length $n$ on computational complexity. In typical MLLM tasks, the sequence length is defined as: 
\begin{equation}
    n = n_S + n_I + n_Q, 
\end{equation}
where $n_I$, the tokenized image representation, often dominates, sometimes exceeding other components by an order of magnitude or more.  
As a result, minimizing $n_I$ becomes a critical strategy for enhancing the efficiency of MLLMs.

\subsection{Beyond Token Importance: Questioning the Status Quo}
Given the computational burden associated with the length of visual tokens in MLLMs, numerous studies have embraced a paradigm that utilizes attention scores to evaluate the significance of visual tokens, thereby facilitating token reduction.
Specifically, in transformer-based MLLMs, each layer performs attention computation as illustrated below:
\begin{equation}
   \text{Attention}(\mathbf{Q}, \mathbf{K}, \mathbf{V}) = \text{softmax}\left(\frac{\mathbf{Q} \cdot \mathbf{K}^\mathbf{T}}{\sqrt{d_k}}\right)\cdot \mathbf{V},
\end{equation}
where $d_k$ is the dimension of $\mathbf{K}$. The result of $\text{Softmax}(\mathbf{Q}\cdot \mathbf{K}^\mathbf{T}/\sqrt{d_k})$ is a square matrix known as the attention map.
Existing methods extract the corresponding attention maps from one or multiple layers and compute the average attention score for each visual token based on these attention maps:
\begin{equation}
    \phi_{\text{attn}}(x_i) = \frac{1}{N} \sum_{j=1}^{N} \text{Attention}(x_i, x_j),
\end{equation}
where $\text{Attention}(x_i, x_j)$ denotes the attention score between token $x_i$ and token $x_j$, $\phi_{\text{attn}}(x_i)$ is regarded as the importance score of the token $x_i$, $N$ represents the number of visual tokens.
Finally, based on the importance score of each token and the predefined reduction ratio, the most significant tokens are selectively retained:
\begin{equation}
    \mathcal{R} = \{ x_i \mid (\phi_{\text{attn}}(x_i) \geq \tau) \},
\end{equation}
where $\mathcal{R}$ represents the set of retained tokens, and $\tau$ is a threshold determined by the predefined reduction ratio.

\noindent{\textbf{Problems:}} Although this paradigm has demonstrated initial success in enhancing the efficiency of MLLMs, it is accompanied by several inherent limitations that are challenging to overcome.

First, when it comes to leveraging attention scores to derive token importance, it inherently lacks full compatibility with Flash Attention, resulting in limited hardware acceleration affinity and diminished acceleration benefits.

Second, does the paradigm of using attention scores to evaluate token importance truly ensure the effective retention of crucial visual tokens? Our empirical investigations reveal that it is not the optimal approach.

% As illustrated in Figure~\ref{fig:random_vs_others}, performance evaluations on certain benchmarks show that methods meticulously designed based on this paradigm sometimes underperform compared to randomly retaining the same number of visual tokens.
Performance evaluations on certain benchmarks, as illustrated in Figure~\ref{fig:random_vs_others}, demonstrate that methods meticulously designed based on this paradigm sometimes underperform compared to randomly retaining the same number of visual tokens.

% As depicted in Figure~\ref{fig:teaser_curry}, which visualizes the results of token reduction, the selection of visual tokens based on attention scores exhibits a noticeable bias, favoring tokens located in the lower-right region of the image—those positioned later in the visual token sequence. However, it is evident that the lower-right region is not always the most significant in every image.
% Furthermore, in Figure~\ref{fig:teaser_curry}, we present the outputs of the original LLaVA-1.5-7B, FastV, and our proposed \algname. Notably, FastV introduces more hallucinations compared to the vanilla model, while \algname demonstrates a noticeable trend of reducing hallucinations.
% We suppose that this phenomenon arises because the important-based method, which relies on attention scores, tends to retain visual tokens that are concentrated in specific regions of the image due to the inherent bias in attention scores. As a result, relying on only a portion of the image often leads to outputs that are inconsistent with the overall image content. In contrast, \algname primarily removes highly duplication tokens and retains tokens that are more evenly distributed across the entire image, enabling it to make more accurate and consistent judgments.
%--------------- shorter version ---------------------
Figure~\ref{fig:teaser_curry} visualizes the results of token reduction, revealing that selecting visual tokens based on attention scores introduces a noticeable bias toward tokens in the lower-right region of the image—those appearing later in the visual token sequence. However, this region is not always the most significant in every image. Additionally, we present the outputs of the original LLaVA-1.5-7B, FastV, and our proposed \algname. Notably, FastV generates more hallucinations compared to the vanilla model, while \algname effectively reduces them. 
We attribute this to the inherent bias of attention-based methods, which tend to retain tokens concentrated in specific regions, often neglecting the broader context of the image. In contrast, \algname removes highly duplication tokens and preserves a more balanced distribution across the image, enabling more accurate and consistent outputs.

\subsection{Token Duplication: Rethinking Reduction}
Given the numerous drawbacks associated with the paradigm of using attention scores to evaluate token importance for token reduction, \textit{what additional factors should we consider beyond token importance in the process of token reduction?}
Inspired by the intuitive ideas mentioned in \secref{sec:introduction} and the phenomenon of tokens in transformers tending toward uniformity (i.e., over-smoothing)~\citep{nguyen2023mitigating, gong2021vision}, we propose that token duplication should be a critical focus.

Due to the prohibitively high computational cost of directly measuring duplication among all tokens, we adopt a paradigm that involves selecting a minimal number of pivot tokens. 
\begin{equation}
    \mathcal{P} = \{p_1, p_2, \dots, p_k\}, \quad k \ll n,
\end{equation}
where $p_i$ denotes pivot token, $\mathcal{P}$ represents the set of pivot tokens and $n$ means the length of tokens.

Subsequently, we compute the cosine similarity between these pivot tokens and the remaining visual tokens:
\begin{equation}
    dup (p_i, x_j) = \frac{p_i \cdot x_j}{\|p_i\| \cdot \|x_j\|}, \quad p_i \in \mathcal{P}, \, x_j \in \mathcal{X},
\end{equation}
where $dup (p_i, x_j)$ represents the token duplication score between $i$-th pivot token $p_i$ and $j$-th visual token $x_j$,
ultimately retaining those tokens that exhibit the lowest duplication with the pivot tokens.
\begin{equation}
    \mathcal{R} = \{ x_j \mid \min_{p_i \in \mathcal{P}} dup (p_i, x_j) \leq \epsilon \}.
\end{equation}
Here, $\mathcal{R}$ denotes the set of retained tokens, and $\epsilon$ is a threshold determined by the reduction ratio.

Our method is orthogonal to the paradigm of using attention scores to measure token importance, meaning it is compatible with existing approaches. Specifically, we can leverage attention scores to select pivot tokens, and subsequently incorporate token duplication into the process.

However, this approach still does not fully achieve compatibility with Flash Attention. To this end, we explored alternative strategies for selecting pivot tokens, such as using K-norm, V-norm\footnote{Here, the K-norm and V-norm refer to the L1-norm of K matrix and V matrix in attention computing, respectively.}, or even random selection. Surprisingly, we found that all these methods achieve competitive performance across multiple benchmarks. This indicates that our token reduction paradigm based on token duplication is not highly sensitive to the choice of pivot tokens. Furthermore, it suggests that removing duplicate tokens may be more critical than identifying ``important tokens'', highlighting token duplication as a potentially more significant factor to consider in token reduction.
The selection of pivot tokens is discussed in greater detail in \secref{pivot_token_selection}.
% 加个总结


\section{Experiment 1: Few-shot Semi-supervised Medical Image Segmentation (FS-Semi)}
\label{sec:task2}
We implement our GEMINI learning on few-shot semi-supervised (FS-Semi) medical image segmentation (GEMINI-Semi) providing a variant on the situation that labels are very few. Three public-available tasks are enrolled in our experiments for a very complete evaluation.
\subsection{Experiments configurations}
\label{sec:configurations2}
\subsubsection{Variant design} The variant of our GEMINI-Semi learns a segmentation head $Seg_{\kappa}$ on the extracted dense features $f^{A},f^{B}$. Therefore, except the optimization for deformable homeomorphism learning $\mathcal{L}_{DHL}$, the GEMINI-Semi also has an additional optimization for segmentation $\mathcal{L}_{Seg}$:
\begin{equation}\label{equ:variant2}
\underset{\xi,\theta,\kappa}{\arg\min}\ (\mathcal{L}_{DHL}(\theta,\xi,\mathcal{S}_{ul})+\mathcal{L}_{Seg}(\theta,\kappa,\mathcal{S}_{l})),
\end{equation}
where the $\mathcal{S}_{ul}$ and the $\mathcal{S}_{l}$ are the unlabeled dataset and the labeled dataset. In our experiment, we utilize the sum of Dice loss and cross-entropy loss \cite{ma2021loss} to train segmentation objective $\mathcal{L}_{Seg}$. The other compared DCRL methods (Sec.\ref{sec:comparison2}) also use the same setting as this variant which adds the $\mathcal{L}_{Seg}$ in the training to learn segmentation.
\begin{table}
  \centering
  \caption{Total seven publicly available datasets are involved in this paper for the experiments of our GEMINI's variants, achieving great reproducibility.}\label{dataset}
\resizebox{\linewidth}{!}{
  \begin{tabular}{lccccccccc}
  \toprule
  \textbf{Dataset}                       &\textbf{Type}    &\textbf{Num}  &\textbf{FS-Semi} &\textbf{SS-MIP}\\
  \midrule
  %\midrule
  ASOCA \cite{gharleghi2022automated}    &3D cardiac CT    &60            &$\surd$          &\\
  CAT08 \cite{schaap2009standardized}    &3D cardiac CT    &32            &$\surd$          &\\
  WHS-CT \cite{zhuang2019evaluation}     &3D cardiac CT    &60            &$\surd$          &\\
  CANDI \cite{kennedy2012candishare}     &3D brain MRI     &103           &$\surd$          &$\surd$\\
  SCR \cite{van2006segmentation}         &2D chest X-ray   &247           &$\surd$          &$\surd$\\
  KiPA22 \cite{he2021meta}               &3D kidney CT     &130           &                 &$\surd$\\
  %CARDIAC               &3D cardiac CT              &302                 &                 &$\surd$\\
  ChestX-ray8 \cite{wang2017chestx}      &2D chest X-ray   &112,120       &                 &$\surd$\\
  \bottomrule
  \end{tabular}}
\end{table}

\subsubsection{Datasets} We evaluate GEMINI on three public tasks in 2D and 3D dimensions, showcasing its powerful representation ability in semi-supervised tasks \cite{you2024mine,you2024rethinking} with minimal labels (Tab.\ref{dataset}). \textbf{Task 1: FS-Semi cardiac structure segmentation (3D)} targets seven cardiac structures on 3D CT images, combining WHS-CT \cite{zhuang2019evaluation} (20 labeled, 40 unlabeled), ASOCA \cite{gharleghi2022automated} (60 unlabeled), and CAT08 \cite{schaap2009standardized} (32 labeled from\footnote{\url{http://www.sdspeople.fudan.edu.cn/zhuangxiahai/0/mmwhs/}}). Images are cropped and resampled to $144\times144\times128$, with a five-shot evaluation (5, 100, and 47 images as labeled training, unlabeled training, and testing sets). \textbf{Task 2: FS-Semi brain tissue segmentation (3D)} involves 27 brain tissues on 3D T1 MR images from the CANDI dataset \cite{kennedy2012candishare} (103 labeled). Cropped volumes of $160\times160\times128$ undergo five-shot evaluation (5, 78, and 20 images as labeled training, unlabeled training, and testing sets). \textbf{Task 3: FS-Semi chest structure segmentation (2D)} focuses on three chest-related structures in 2D chest X-rays using the SCR dataset \cite{van2006segmentation} (247 labeled) whose images are from the JSRT database \cite{shiraishi2000development}, split into 5 labeled, 142 unlabeled, and 100 testing images for five-shot evaluation. All tasks use rotation [$-20^\circ$, $20^\circ$] and scaling [0.75, 1.25] for data augmentation.

\subsubsection{Comparison setting} \label{sec:comparison2}
We compare GEMINI-Semi with 19 widely-used methods and our GVSL \cite{He_2023_CVPR} (CVPR 2023) to demonstrate its superiority. \textbf{1)} We train a U-Net \cite{ronneberger2015u} to establish upper and lower bounds using 5 labeled images (Five) and all labeled training data (Full). \textbf{2) Semi-supervised methods} without homeomorphism prior (UA-MT \cite{yu2019uncertainty}, MASSL \cite{chen2019multi}, DPA-DBN \cite{he2020dense}, CPS \cite{chen2021semi}) highlight the significance of prior knowledge for semi-supervised learning with limited labels. \textbf{3) Atlas-based methods} with homeomorphism prior (VM \cite{ba2018un}, LC-VM \cite{BalakrishnanVoxelMorph(u)}, LT-Net \cite{wang2020lt}) illustrate the limitation caused by the inefficient correspondence learning. \textbf{4) Learning registration to learn segmentation methods} with homeomorphism prior (DeepAtlas \cite{xu2019deepatlas}, DataAug \cite{zhao2019data}, DeepRS \cite{he2020deep}, PC-Reg-RT \cite{he2021few}, BRBS \cite{he2022learning}) show gains from improved correspondence but are limited by pseudo-labels from unreliable GVS. \textbf{5) Dense contrastive representation learning methods} without homeomorphism prior (VADeR \cite{o2020unsupervised}, GLCL \cite{chaitanya2020contrastive}, DSC-PM \cite{li2021dense}, PixPro \cite{xie2021propagate}, DenseCL \cite{wang2022densecl}, SetSim \cite{wang2022exploring}) reveal FP\&N problem in DCRL. For fairness, all methods use 2D/3D U-Net \cite{ronneberger2015u} with group normalization \cite{wu2018group} as the backbone.

\subsubsection{Implementation and evaluation metrics} In this task, our GEMINI-Semi is implemented by PyTorch \cite{paszke2019pytorch} on NVIDIA GeForce RTX 3090 GPUs with 24 GB memory. We take Adam whose learning rate is $1\times10^{-4}$ to optimize our framework for fast convergence. For task 1 and task 2, we sample two unlabeled images and one labeled image randomly in each iteration to save the memory for large 3D images, and for task 3, we sample 10 unlabeled images and 5 labeled images randomly in each iteration for 2D images. Following \cite{he2022learning}, we perform an affine transformation on these images via AntsPy\footnote{\url{https://github.com/ANTsX/ANTsPy}} to normalize the spatial system. We utilize the DSC [\%] to evaluate the area-based overlap index and the average Hausdorf distances (AVD) to evaluate the coincidence of the surface \cite{taha2015metrics}.

\subsection{Results and Analysis}
\label{sec:results2}
\begin{table*}
\centering
\caption{The quantitative evaluation demonstrates our powerful representation ability in FS-Semi tasks. Our GEMINI-Semi achieves the best performance on CT, MR, and X-ray images compared with 19 popular methods and the GVSL. The ``unable" means that the extremely poor results make the AVD unable to be calculated. The ``-" means that the setting is unable to be implemented. The ``HP" means these methods have or do not have homeomorphism prior. ``T1", ``T2", ``T3" are the task 1, task 2, task 3. The red and blue values are the highest and the second-highest values in the columns.}
\resizebox{\textwidth}{!}{
\begin{tabular}{clccccccccccccccc}
  \toprule
  \multirow{2}{*}{\textbf{Type}}
  &\multirow{2}{*}{\textbf{Method}}
  &\multirow{2}{*}{\textbf{HP}}
  &\multicolumn{2}{c}{\textbf{T1: 3D cardiac structures}}
  &\multicolumn{2}{c}{\textbf{T2: 3D brain tissues}}
  &\multicolumn{2}{c}{\textbf{T3: 2D chest structures}}
  &\textbf{AVG}\\ \cmidrule(r){4-5}\cmidrule(r){6-7}\cmidrule(r){8-9}\cmidrule(r){10-10}
  &
  &
  &DSC$_{\pm std}\uparrow$
  &AVD$_{\pm std}\downarrow$
  &DSC$_{\pm std}\uparrow$
  &AVD$_{\pm std}\downarrow$
  &DSC$_{\pm std}\uparrow$
  &AVD$_{\pm std}\downarrow$
  &DSC$_{\pm std}\uparrow$
  \\
  \midrule
  Full
  &U-Net \cite{ronneberger2015u}
  &$\times$
  &-
  &-
  &88.7$_{\pm1.2}$
  &0.31$_{\pm0.04}$
  &96.1$_{\pm1.4}$
  &2.28$_{\pm1.00}$
  &-
  \\
  Five
  &U-Net \cite{ronneberger2015u}
  &$\times$
  &84.3$_{\pm9.6}$
  &2.43$_{\pm2.14}$
  &69.5$_{\pm8.8}$
  &1.59$_{\pm0.84}$
  &83.4$_{\pm6.9}$
  &10.34$_{\pm4.80}$
  &79.1$_{\pm8.4}$
  \\
  \cdashline{1-10}[0.8pt/2pt]
  Semi
  &UA-MT \cite{yu2019uncertainty}
  &$\times$
  &66.4$_{\pm16.2}$
  &4.69$_{\pm2.27}$
  &75.5$_{\pm3.4}$
  &1.31$_{\pm0.95}$
  &83.9$_{\pm6.2}$
  &9.52$_{\pm4.03}$
  &75.3$_{\pm8.6}$
  \\
  &CPS \cite{chen2021semi}
  &$\times$
  &87.4$_{\pm5.4}$
  &1.40$_{\pm0.76}$
  &37.1$_{\pm1.8}$
  &unable
  &63.2$_{\pm1.4}$
  &19.57$_{\pm5.67}$
  &62.6$_{\pm2.9}$
  \\
  &MASSL \cite{chen2019multi}
  &$\times$
  &77.4$_{\pm8.7}$
  &9.07$_{\pm3.11}$
  &80.5$_{\pm3.1}$
  &0.92$_{\pm0.43}$
  &81.9$_{\pm7.0}$
  &10.99$_{\pm4.58}$
  &79.9$_{\pm6.3}$
  \\
  &DPA-DBN \cite{he2020dense}
  &$\times$
  &68.0$_{\pm14.5}$
  &5.75$_{\pm3.89}$
  &68.7$_{\pm8.2}$
  &3.90$_{\pm2.39}$
  &67.4$_{\pm8.7}$
  &24.05$_{\pm6.75}$
  &68.0$_{\pm10.5}$
  \\
  %\midrule
  Atlas
  &VM \cite{ba2018un}
  &$\surd$
  &81.0$_{\pm6.1}$
  &2.13$_{\pm0.78}$
  &83.1$_{\pm1.8}$
  &0.56$_{\pm0.08}$
  &59.9$_{\pm5.0}$
  &15.36$_{\pm4.34}$
  &74.7$_{\pm4.3}$
  \\
  &LC-VM \cite{BalakrishnanVoxelMorph(u)}
  &$\surd$
  &81.7$_{\pm6.0}$
  &2.04$_{\pm0.77}$
  &83.0$_{\pm1.8}$
  &0.56$_{\pm0.07}$
  &60.2$_{\pm7.4}$
  &14.72$_{\pm4.89}$
  &74.9$_{\pm5.1}$
  \\
  &LT-Net \cite{wang2020lt}
  &$\surd$
  &77.8$_{\pm7.8}$
  &2.25$_{\pm0.95}$
  &82.6$_{\pm1.2}$
  &0.57$_{\pm0.05}$
  &60.4$_{\pm7.4}$
  &14.62$_{\pm4.84}$
  &73.6$_{\pm5.5}$
  \\
  %\hline
  LRLS
  &DeepAtlas \cite{xu2019deepatlas}
  &$\surd$
  &87.9$_{\pm4.3}$
  &1.30$_{\pm0.57}$
  &79.3$_{\pm2.6}$
  &0.74$_{\pm0.12}$
  &64.8$_{\pm9.6}$
  &12.87$_{\pm3.56}$
  &77.3$_{\pm5.5}$
  \\
  &DataAug \cite{zhao2019data}
  &$\surd$
  &82.2$_{\pm5.2}$
  &2.04$_{\pm0.73}$
  &83.9$_{\pm1.2}$
  &0.55$_{\pm0.06}$
  &22.2$_{\pm2.8}$
  &unable
  &62.8$_{\pm3.1}$
  \\
  &DeepRS \cite{he2020deep}
  &$\surd$
  &87.0$_{\pm5.0}$
  &1.60$_{\pm0.90}$
  &73.0$_{\pm5.9}$
  &0.93$_{\pm0.25}$
  &86.0$_{\pm5.6}$
  &8.55$_{\pm3.98}$
  &82.0$_{\pm5.5}$
  \\
  &PC-Reg-RT \cite{he2021few}
  &$\surd$
  &88.5$_{\pm4.9}$
  &1.23$_{\pm0.72}$
  &73.1$_{\pm3.1}$
  &1.09$_{\pm0.17}$
  &59.1$_{\pm3.6}$
  &20.71$_{\pm5.21}$
  &73.6$_{\pm3.9}$
  \\
  &BRBS \cite{he2022learning}
  &$\surd$
  &\color{blue}91.1$_{\pm3.9}$
  &\color{red}\textbf{0.93$_{\pm0.57}$}
  &\color{blue}87.2$_{\pm1.0}$
  &0.43$_{\pm0.05}$
  &71.5$_{\pm6.4}$
  &10.85$_{\pm2.99}$
  &83.3$_{\pm3.8}$
  \\
  %\hline
  DCRL
  &VADeR \cite{o2020unsupervised}
  &$\times$
  &85.4$_{\pm4.7}$
  &1.69$_{\pm0.77}$
  &81.2$_{\pm3.2}$
  &0.59$_{\pm0.13}$
  &79.9$_{\pm5.8}$
  &8.95$_{\pm3.37}$
  &82.2$_{\pm4.6}$
  \\
  &DenseCL \cite{wang2022densecl}
  &$\times$
  &87.3$_{\pm4.3}$
  &1.52$_{\pm0.79}$
  &83.9$_{\pm1.9}$
  &0.48$_{\pm0.09}$
  &77.1$_{\pm8.8}$
  &12.11$_{\pm6.51}$
  &82.8$_{\pm5.0}$
  \\
  &SetSim \cite{wang2022exploring}
  &$\times$
  &87.0$_{\pm4.5}$
  &1.60$_{\pm0.84}$
  &81.2$_{\pm3.0}$
  &0.58$_{\pm0.13}$
  &79.0$_{\pm7.3}$
  &11.72$_{\pm5.03}$
  &82.4$_{\pm4.9}$
  \\
  &DSC-PM \cite{li2021dense}
  &$\times$
  &87.0$_{\pm4.6}$
  &1.60$_{\pm0.86}$
  &82.6$_{\pm2.1}$
  &0.53$_{\pm0.09}$
  &85.7$_{\pm6.2}$
  &7.33$_{\pm3.32}$
  &85.1$_{\pm4.3}$
  \\
  &PixPro \cite{xie2021propagate}
  &$\times$
  &89.5$_{\pm3.9}$
  &1.31$_{\pm0.75}$
  &86.3$_{\pm1.2}$
  &\color{blue}0.38$_{\pm0.04}$
  &83.3$_{\pm8.7}$
  &8.73$_{\pm4.55}$
  &\color{blue}86.4$_{\pm4.6}$
  \\
  &GLCL\cite{chaitanya2020contrastive}
  &$\times$
  &84.5$_{\pm7.0}$
  &1.82$_{\pm1.09}$
  &83.0$_{\pm2.7}$
  &0.52$_{\pm0.11}$
  &85.5$_{\pm8.9}$
  &8.65$_{\pm5.18}$
  &84.3$_{\pm6.2}$
  \\
  %\hline
  \cdashline{1-10}[0.8pt/2pt]
  \textbf{DCRL}
  &\textbf{GVSL-Semi (CVPR)} \cite{He_2023_CVPR}
  &$\surd$
  &90.0$_{\pm3.7}$
  &1.21$_{\pm0.81}$
  &82.3$_{\pm5.9}$
  &0.55$_{\pm0.27}$
  &\color{blue}86.3$_{\pm5.5}$
  &\color{blue}7.18$_{\pm4.01}$
  &86.2$_{\pm5.0}$
  \\
  \textbf{(Ours)}
  &\textbf{GEMINI-Semi}
  &$\surd$
  &\color{red}\textbf{91.2$_{\pm3.6}$}
  &\color{blue}0.97$_{\pm0.56}$
  &\color{red}\textbf{87.3$_{\pm1.0}$}
  &\color{red}\textbf{0.35$_{\pm0.03}$}
  &\color{red}\textbf{87.7$_{\pm5.2}$}
  &\color{red}\textbf{7.14$_{\pm3.63}$}
  &\color{red}\textbf{88.7$_{\pm3.3}$}
  \\
  \bottomrule
\end{tabular}
}
\label{tab:metrics2}
\end{table*}
\begin{figure}
  \centering
  \includegraphics[width=\linewidth]{./picture/results2.pdf}
  \caption{Our GEMINI-Semi has significant visual superiority on three FS-Semi medical image segmentation tasks.}\label{Fig:results2}
\end{figure}
\subsubsection{Quantitative evaluation shows metric superiority}
As shown in Tab.\ref{tab:metrics2}, 19 compared methods demonstrate that the DCRL will greatly improve the representability, and the homeomorphism prior (``HP") further improves the reliability of the representation learning. There are three interesting observations in Tab.\ref{tab:metrics2}: \textbf{1)} The semi-supervised methods are limited by the extremely few labels. They utilize the pseudo-label generation (UA-MT, CPS) or multi-task learning (MASSL, DPA-DBN) to improve the representation, but the extremely few labels have no enough ability to give them reliable optimization directions to overcome the noise in pseudo labels or multiple tasks. As a result, the UA-MT, MASSL, and DPA-DBN have worse performance than U-Net on task 1, and the CPS is worse on task 2 and 3. \textbf{2)} With the ``HP", the Atlas and LRLS methods achieve robust performance in task 1 and task 2, but are limited in task 3. The ``HP" brings an alignment between labeled and unlabeled images for numerous reliable pseudo labels. Therefore, they have achieved significant improvement on task 1 and task 2 compared with the semi-supervised methods. However, the X-ray images in task 3 have low contrast and their appearances are varied caused by the 2D projection of 3D human body, this makes inefficient GVS brings large misalignment between images, thus interfering with the learning and reducing the performance. \textbf{3)} The DCRL methods have robust performance in all three tasks compared with the LRLS methods, although the VADeR, DenseCL, SetSim, DSC-PM, PixPro and GLCL have no homeomorphism prior. Because their feature-level learning reduce the direct interference caused by misalignment in LRLS's pseudo labels and the supervision from the few labels bring basic representability which will promote their correspondence discovery. However, the FP\&N problem is still a problem in the learning and their performance on task 3 is poor without ``HP" like the semi-supervised methods.

Compared with the LRLS, other DCRL methods, and our previous GVSL-Semi, our GEMINI-Semi achieves the best performance on three tasks with four observations: \textbf{1)} Compared with the LRLS methods which have ``HP", our method has better performance on all tasks. Although the BRBS has similar performance as our GEMINI-Semi on task 1 and task 2, our method achieves 16.2\% DSC and 3.71 AVD higher and lower than it on task 3. This is because our GEMINI-Semi utilizes our GSS for alignment measurement and shares the representation between the segmentation and deformation learning, bringing more efficient and robust learning for alignment. It has a great ability to construct positive feature pairs even with varied appearances. The gradient from our DHL also trains the soft negative feature pairs to drive the learning of distinct representations for potentially different semantics in shared backbones, bringing a regularization for potential mispaired positive pairs. \textbf{2)} Compared with the other DCRL methods which have no ``HP", our GEMINI-Semi shows great improvements in all three tasks. It achieves more than 1.7\%, 1.0\%, and 2.0\% DSC improvements on task 1, 2, and 3 compared with the best DCRL models without ``HP" (PixPro in task 1 and 2, DSC-PM in task 3). Because the ``HP" in our GEMINI-Semi constructs a more reliable correspondence discovery process which reduces the production risk of the FP\&N pairs, bringing comprehensive improvement for the DCRL. \textbf{3)} Compared to our CVPR vision (GVSL-Semi), we find even though the GVSL utilizes the visual similarity like the BRBS, it also achieves great performance in task 3, demonstrating the superiority of the DCRL paradigm. The GVSL-semi avoids the interference of pseudo labels like BRBS reducing the noisy information from the extremely mis-alignment, so that it takes the advantage of DCRL and our homeomorphism prior and achieves good performance in all three tasks. Our GEMINI-Semi promotes the GVSL and utilizes the GSS for a more powerful dense representation learning, thus achieving the highest 88.7\% average DSC in this experiment. \textbf{4)} Compared with the fully supervised setting (``Full") in task 2 (83 labeled images), our GEMINI-Semi achieves a similar performance only with 5 labeled images demonstrating our great potential in reducing of annotation costs. In the task 3, our framework is lower than the upper bound (96.1\%) only with five annotations, but it still achieves significant improvement (4.3\%) compared with the model directly trained on five labeled images.

\subsubsection{Qualitative evaluation shows visual superiority}
As shown in Fig.\ref{Fig:results2}, we show typical cases on the three tasks in this experiment and our framework has higher accuracy on thin regions and fewer outliers. In the task 1, the segmentation result of our method has better integrity, and the different semantic structures have good adjacency without outliers. However, the other four DCRL methods have discontinuous mis-segmentation which destroys the heart topology. This is because the pairing strategies in the DCRL methods are unable to make the pairs under the condition of topology consistency, so the large-scale mispaired features interrupt the learning and make numerous outliers. The same as the task 3, there are also serious outlier problems in the four typical DCRL methods and the GVSL, and our GEMINI-Semi has fine segmentation. In the task 2, our GEMINI and GVSL show finer segmentation on the thin brain structures which is sensitive and will be interrupted by the noise in the semi-supervised learning process. In some prominent and gully regions of task 2 (enlarged part), the compared four DCRL methods have numerous distortions due to their unreliable correspondence discovery, showing their fragility.



Our first study sought to understand the key factors underlying human expert evaluation of the creativity of solutions to design problems (DPT) items. A participant in this task is given a scientific or engineering problem (e.g., increasing the use of renewable energy) and is instructed to come up with as many novel solutions to the problem as they can think of. Similar to expert-level science, the best solutions are both original and feasible, though unlike other STEM assessments the DPT benefits from but is not contingent on expertise to come up with creative ideas. The greater complexity of DPT responses compared to those from other creativity tests and its relationship to scientific creativity more broadly makes it a strong choice for our analysis. Unlike prior studies, which often have experts rate only the originality or quality of products, we instead ask our raters to provide fine-grained assessments of cleverness (whether the solution is insightful or witty), remoteness (whether the solution is ``far'' from everyday ideas), and uncommonness (whether the solution is rare, given by few people) in addition to originality, each of which is thought to influence ratings of creativity \citep{silvia2008assessing}. These assessments are performed both with and without the presence of example creativity ratings to DPT items, enabling us to examine how added context affects the evaluation process. Finally, we ask experts to briefly explain their originality scores, enabling us to employ methods from computational text analysis to probe the cognitive processes experts employ when rating and how such processes may be modulated by added context.

\subsection{Methods}
We use the data from \citet{Patterson2025}, who obtained more than 7000 responses to DPT items from undergraduate STEM majors. Each response was rated for originality using a five-point Likert scale by at least three expert raters with formal training in engineering. We drop items that did not obtain at least one rating from every point of the scale (certain items never had a response that received a five). We convert Likert scores into factor scores, as this has been shown to provide more accurate creativity ratings \citep{silvia2008another}, and we treat these factor scores as the true originality scores of each response.

We recruit 80 participants on Prolific to provide finegrained creativity ratings to DPT responses, requiring that they have a bachelor's degree or higher in a STEM field and are fluent in English. We split participants into two conditions: a \textit{no example} condition where participants are given responses to rate without any additional context, and an \textit{example} condition where participants are first shown example solutions with originality scores for responses to the same prompt being rated. We pull three example solutions from the same dataset while ensuring that participants never rate them. We include a solution with a score of one, one with a score of three, and one with a score of five, to avoid biasing participants towards either end of the scale. We first have each participant rate for originality following the same procedure, instructions, and facet definitions as \citet{Patterson2025}. After rating originality, participants in both groups then provide 1-2 sentences explaining their rating process \citep{orwig2024creative}, and they finish by rating the uncommonness, remoteness, and cleverness of the response using a five-point Likert scale for each. We instruct participants to be specific in their explanations, to draw on their domain expertise as holders of a STEM degree, and to avoid overly simplistic explanations (e.g., ``it's not original'' or ``it's an obvious answer''). We define a good explanation as being at least one sentence long and including specific details from the participant's prior experience, the response, or the examples (if applicable). We also provide definitions of uncommonness, remoteness, and cleverness for the final rating task, emphasizing that each facet is related while being distinct from originality. We include educational background and AI use checks at the end of the survey.

We administer each participant 15 DPT responses at random. To encourage high-quality explanations, we offer \$20 per hour to complete a 30-minute study. We exclude participants with an approval rating of less than 90\%, who report using AI to complete the task, or who report an education level lower than the minimum specified on Prolific. We also exclude participants who were exceptionally slow or fast (with a completion time further than three standard deviations from the mean), who gave the same rating for every response, or who did not follow our instructions for formatting explanations (as checked by a research assistant). This resulted in a final sample size of 37 participants and 481 ratings in the example condition and 35 participants and 455 responses for the no example.

% out of the full archival set

When examining the participants' explanations, we employ an analysis plan similar to \citet{orwig2024creative}, who used LIWC to analyze explanations of originality scores for AUTs. However, recent work has found that LLMs can predict psycholinguistic features of text more strongly than LIWC, even zero shot \citep{rathje2024gpt}. Therefore, we use LLMs to automatically rate linguistic markers in the explanations. We instruct LLMs to rate for the following variables: 

\begin{itemize}
    \item \textit{Past/future expressions}: Is the explanation past-focused or future-focused in its evaluation of the response?
    \item \textit{Perceptual details}: Does the explanation focus on the process of perceiving (``observe'', ``seen'', ``heard'', ``feel'', etc.)?
    \item \textit{Causal/analytical}: Does the explanation involve a structured evaluation of the response, evidencing an analytical process, or is the explanation more intuitive in its justifications?
    \item \textit{Comparative}: Does the explanation make explicit references to standards or examples or compare the response to other ideas?
    \item \textit{Cleverness}: Does the explanation refer to the cleverness, wittiness, shrewdness, or ingenuity (or lack thereof) of the response?
\end{itemize}

Both past/future language use and perceptual details have been explored to assess cognitive strategies employed on other creativity tests \citep{orwig2024creative}. We elect to use causal/analytical, comparative, and cleverness linguistic markers to aid in assessing whether participants employed a more structured process --- which might be evidenced by causal/analytical or comparative language use --- or a more intuitive process, as evidenced by language indicating sensory experiences or other ``gut reactions'' (e.g, ``it feels like a clever idea''). These linguistic markers also map onto the finegrained facets participants were asked to rate, with cleverness language mapping onto cleverness and comparative language mapping onto remoteness and uncommonness (as both remoteness and uncommonness often require making references to prior solutions). We use both \textsc{claude-3.5-sonnet}\footnote{https://www.anthropic.com/news/claude-3-5-sonnet} and \textsc{gpt-4o}\footnote{https://openai.com/index/hello-gpt-4o/} to check for reliability in ratings and avoid biases specific to a single LLM, though due to space constraints we mainly report results from \textsc{gpt-4o} as this is the model \citet{rathje2024gpt} validated. To encourage deterministic output, we set the temperature for both models to $0$ and top P to $1$. We instruct LLMs to rate each facet and provide a binary evaluation of whether the explanation does or does not contain the feature. Prompts are provided in the supplementary materials.

% We drop any explanations for which the model fails to follow this instruction.



% Specifically, we have each participant first rate for originality following the same procedure as \TODO{cite the paper that obtained the gold scores}, after which they are instructed to provide 1-2 sentences explaining their originality rating. We instruct participants to draw on their domain expertise while responding, asking them to consider if they have observed similar solutions to the problem in the past, to help ensure that the explanation is grounded in the originality of the solution and not merely its quality. Finally, participants rate the uncommonness, remoteness, and cleverness of the solution using separate five-point Likert scales, to disentangle how each facets contributes to the final creativity score. Figure \TODO{make it} shows the experimental interface, we collect data using Qualtrics and give 30 minutes to complete the task. Participants are split into two conditions: a \textit{no oracle} conditions where participants are shown DPT solutions without any additional context (equivalent to how creativity ratings are typically solicited), and an \textit{oracle} condition where participants are first shown example solutions with originality scores for responses to the same prompt being rated. We pull these solutions from the same dataset, while ensuring that participants never rate them. At the end, participants complete a demographic questionnaire and a check for use of generative AI during the study.

% The design problems task (DPT) is a test of domain-general creative problem solving and divergent thinking in science and engineering \TODO{cite}.   Further, creativity evaluation also hinges on weighing multiple competing factors: a highly uncommon solution may still receive a poor creativity score if it is not especially clever or could not be feasibly put into practice. This makes the DPT a strong testbed for our experiments: evaluation is more complex than purely theoretical creativity tests like the alternative uses task \TODO{cite}, yet it is not so challenging as to require domain experts to obtain meaningful creativity scores, enabling us to recruit a larger pool of participants.

% The task targets STEM undergraduate students; a general understanding of science and engineering is beneficial but not necessary for the task.

% Our goal is to obtain \textit{finegrained} creativity assessments for these responses, to better understand how each facet of creativity influences a raters final score, and to provide explanations for why responses are assigned a particular rating. 

% \TODO{define the research questions and null hypotheses}

\subsection{Results}
We begin by examining inter-correlations among all facets (cleverness, remoteness, uncommonness) and correlations between each facet and originality for both conditions. Results are in Figure \ref{fig:experiment_1_correlations}. As expected, each facet is moderately correlated with originality as well as each other, with Pearson r in the range 0.45--0.67 (all correlations are significant).\footnote{Results from all correlational analysis in both studies were similar using Spearman $\rho$.} Comparing the example to no example conditions, we see an increase in correlation between originality and cleverness and a decrease in correlation between originality and both remoteness and uncommonness. Changes in correlation across conditions were significant for cleverness-remoteness (Fisher's z = 2.83, p $<$ 0.01), remoteness-uncommonness (z = -4.61, p $<$ 0.001), and remoteness-originality (z = -2.96, p $<$ 0.01), but were insignificant for all other comparisons. Notably, the presence of the examples did not make experts significantly more accurate in their evaluations of originality, with correlations in the moderate range for both conditions (no example r = 0.44, example r = 0.47).

% We report descriptive statistics for all Likert evaluations in Table \TODO{make it}, broken down by condition.

% Examining the distribution of cleverness more closely (the only facet to become more strongly related to originality in the oracle condition), we plot the distribution of cleverness scores for both conditions in Figure \ref{fig:experiment_1_cleverness}. 

% Participants in the oracle condition appear to be stricter judges of cleverness, giving more 1 or 2 rating than their no oracle counterparts, though this difference was only marginally significant (Mann Whitney U test = 102948.5, p $<$ 0.1).

\begin{figure}[htb]
    \centering
    \footnotesize
    \includegraphics[width=0.8\linewidth]{Figures/Correlations_human.eps}
    % \includesvg[width=0.85\linewidth]{Figures/Correlations_human.svg}
    \caption{Pearson correlations among pairwise Likert ratings for both conditions. o = originality, c = cleverness, u = uncommonness, r = remoteness.}
    \label{fig:experiment_1_correlations}
\end{figure}

% \begin{figure}[htb]
%     \centering
%     \includegraphics[width=1\linewidth]{Figures/cleverness.png}
%     \caption{Distribution of cleverness ratings for both conditions.}
%     \label{fig:experiment_1_cleverness}
% \end{figure}

Turning to participant explanations, \textsc{gpt-4o}'s ratings did not reveal significant differences per condition for perceptual details, past/future language use, or cleverness, but differences are significant for both causal/analytical language (Mann-Whitney U = 78039.5, p $<$ 0.05) and comparative language (U = 75627.5, p $<$ 0.01) with the example condition using less comparative and causal/analytical language than the no examples. Distributions for linguistic markers are shown in Figure \ref{fig:liwc_analysis}. \textsc{claude-3.5-sonnet}'s ratings generally agreed with \textsc{gpt-4o} (Cramer's V in the range 0.549--0.798) with the only notable departure being that \textsc{claude-3.5-sonnet} found no significant difference in causal/analytical language between the conditions (U = 75076.5, p $<$ 0.5). We report additional linguistic marker analysis in the supplementary materials.

% We report model agreement statistics and linguistic marker analyses in the supplementary files.

\subsection{Discussion}
As expected, the facet ratings for cleverness, remoteness, and uncommonness did not perfectly correlate with each other nor with originality, implying that participants do not weigh each facet equally when assessing originality. Further, correlations changed by a significant degree when including example ratings, with both remoteness and uncommonness becoming weaker predictors of originality and cleverness becoming a stronger one. Given that participants in the no example condition needed to actively retrieve example solutions from memory when evaluating, a possible explanation is that this retrieval process biased them towards placing stronger emphasis on the remoteness and uncommonness of the response in relation to solutions they had seen in the past, while example participants would not need to focus as much effort on thinking of prior solutions and could instead focus on the cleverness of the idea. Notably, participants in both groups did not differ significantly in terms of education, making it unlikely this effect could be explained as a skill confound. The idea that participants in the example condition were biased toward cleverness rather than the other facets was also partially supported by their explanations, as no example participants used significantly more comparative language than example participants. Given that assessing remoteness or uncommonness often requires making direct comparisons to prior solutions, it makes sense that an evaluation rooted around these facets would contain more comparisons than an evaluation rooted around cleverness, which is more readily evaluated in isolation (e.g., whether the idea is resource efficient, not immediately obvious, etc.).


\section{ Task Generalization Beyond i.i.d. Sampling and Parity Functions
}\label{sec:Discussion}
% Discussion: From Theory to Beyond
% \misha{what is beyond?}
% \amir{we mean two things: in the first subsection beyond i.i.d subsampling of parity tasks and in the second subsection beyond parity task}
% \misha{it has to be beyond something, otherwise it is not clear what it is about} \hz{this is suggested by GPT..., maybe can be interpreted as from theory to beyond theory. We can do explicit like Discussion: Beyond i.i.d. task sampling and the Parity Task}
% \misha{ why is "discussion" in the title?}\amir{Because it is a discussion, it is not like separate concrete explnation about why these thing happens or when they happen, they just discuss some interesting scenraios how it relates to our theory.   } \misha{it is not really a discussion -- there is a bunch of experiments}

In this section, we extend our experiments beyond i.i.d. task sampling and parity functions. We show an adversarial example where biased task selection substantially hinders task generalization for sparse parity problem. In addition, we demonstrate that exponential task scaling extends to a non-parity tasks including arithmetic and multi-step language translation.

% In this section, we extend our experiments beyond i.i.d. task sampling and parity functions. On the one hand, we find that biased task selection can significantly degrade task generalization; on the other hand, we show that exponential task scaling generalizes to broader scenarios.
% \misha{we should add a sentence or two giving more detail}


% 1. beyond i.i.d tasks sampling
% 2. beyond parity -> language, arithmetic -> task dependency + implicit bias of transformer (cannot implement this algorithm for arithmatic)



% In this section, we emphasize the challenge of quantifying the level of out-of-distribution (OOD) differences between training tasks and testing tasks, even for a simple parity task. To illustrate this, we present two scenarios where tasks differ between training and testing. For each scenario, we invite the reader to assess, before examining the experimental results, which cases might appear “more” OOD. All scenarios consider \( d = 10 \). \kaiyue{this sentence should be put into 5.1}






% for parity problem




% \begin{table*}[th!]
%     \centering
%     \caption{Generalization Results for Scenarios 1 and 2 for $d=10$.}
%     \begin{tabular}{|c|c|c|c|}
%         \hline
%         \textbf{Scenario} & \textbf{Type/Variation} & \textbf{Coordinates} & \textbf{Generalization accuracy} \\
%         \hline
%         \multirow{3}{*}{Generalization with Missing Pair} & Type 1 & \( c_1 = 4, c_2 = 6 \) & 47.8\%\\ 
%         & Type 2 & \( c_1 = 4, c_2 = 6 \) & 96.1\%\\ 
%         & Type 3 & \( c_1 = 4, c_2 = 6 \) & 99.5\%\\ 
%         \hline
%         \multirow{3}{*}{Generalization with Missing Pair} & Type 1 &  \( c_1 = 8, c_2 = 9 \) & 40.4\%\\ 
%         & Type 2 & \( c_1 = 8, c_2 = 9 \) & 84.6\% \\ 
%         & Type 3 & \( c_1 = 8, c_2 = 9 \) & 99.1\%\\ 
%         \hline
%         \multirow{1}{*}{Generalization with Missing Coordinate} & --- & \( c_1 = 5 \) & 45.6\% \\ 
%         \hline
%     \end{tabular}
%     \label{tab:generalization_results}
% \end{table*}

\subsection{Task Generalization Beyond i.i.d. Task Sampling }\label{sec: Experiment beyond iid sampling}

% \begin{table*}[ht!]
%     \centering
%     \caption{Generalization Results for Scenarios 1 and 2 for $d=10, k=3$.}
%     \begin{tabular}{|c|c|c|}
%         \hline
%         \textbf{Scenario}  & \textbf{Tasks excluded from training} & \textbf{Generalization accuracy} \\
%         \hline
%         \multirow{1}{*}{Generalization with Missing Pair}
%         & $\{4,6\} \subseteq \{s_1, s_2, s_3\}$ & 96.2\%\\ 
%         \hline
%         \multirow{1}{*}{Generalization with Missing Coordinate}
%         & \( s_2 = 5 \) & 45.6\% \\ 
%         \hline
%     \end{tabular}
%     \label{tab:generalization_results}
% \end{table*}




In previous sections, we focused on \textit{i.i.d. settings}, where the set of training tasks $\mathcal{F}_{train}$ were sampled uniformly at random from the entire class $\mathcal{F}$. Here, we explore scenarios that deliberately break this uniformity to examine the effect of task selection on out-of-distribution (OOD) generalization.\\

\textit{How does the selection of training tasks influence a model’s ability to generalize to unseen tasks? Can we predict which setups are more prone to failure?}\\

\noindent To investigate this, we consider two cases parity problems with \( d = 10 \) and \( k = 3 \), where each task is represented by its tuple of secret indices \( (s_1, s_2, s_3) \):

\begin{enumerate}[leftmargin=0.4 cm]
    \item \textbf{Generalization with a Missing Coordinate.} In this setup, we exclude all training tasks where the second coordinate takes the value \( s_2 = 5 \), such as \( (1,5,7) \). At test time, we evaluate whether the model can generalize to unseen tasks where \( s_2 = 5 \) appears.
    \item \textbf{Generalization with Missing Pair.} Here, we remove all training tasks that contain both \( 4 \) \textit{and} \( 6 \) in the tuple \( (s_1, s_2, s_3) \), such as \( (2,4,6) \) and \( (4,5,6) \). At test time, we assess whether the model can generalize to tasks where both \( 4 \) and \( 6 \) appear together.
\end{enumerate}

% \textbf{Before proceeding, consider the following question:} 
\noindent \textbf{If you had to guess.} Which scenario is more challenging for generalization to unseen tasks? We provide the experimental result in Table~\ref{tab:generalization_results}.

 % while the model struggles for one of them while as it generalizes almost perfectly in the other one. 

% in the first scenario, it generalizes almost perfectly in the second. This highlights how exposure to partial task structures can enhance generalization, even when certain combinations are entirely absent from the training set. 

In the first scenario, despite being trained on all tasks except those where \( s_2 = 5 \), which is of size $O(\d^T)$, the model struggles to generalize to these excluded cases, with prediction at chance level. This is intriguing as one may expect model to generalize across position. The failure  suggests that positional diversity plays a crucial role in the task generalization of Transformers. 

In contrast, in the second scenario, though the model has never seen tasks with both \( 4 \) \textit{and} \( 6 \) together, it has encountered individual instances where \( 4 \) appears in the second position (e.g., \( (1,4,5) \)) or where \( 6 \) appears in the third position (e.g., \( (2,3,6) \)). This exposure appears to facilitate generalization to test cases where both \( 4 \) \textit{and} \( 6 \) are present. 



\begin{table*}[t!]
    \centering
    \caption{Generalization Results for Scenarios 1 and 2 for $d=10, k=3$.}
    \resizebox{\textwidth}{!}{  % Scale to full width
        \begin{tabular}{|c|c|c|}
            \hline
            \textbf{Scenario}  & \textbf{Tasks excluded from training} & \textbf{Generalization accuracy} \\
            \hline
            Generalization with Missing Pair & $\{4,6\} \subseteq \{s_1, s_2, s_3\}$ & 96.2\%\\ 
            \hline
            Generalization with Missing Coordinate & \( s_2 = 5 \) & 45.6\% \\ 
            \hline
        \end{tabular}
    }
    \label{tab:generalization_results}
\end{table*}

As a result, when the training tasks are not i.i.d, an adversarial selection such as exclusion of specific positional configurations may lead to failure to unseen task generalization even though the size of $\mathcal{F}_{train}$ is exponentially large. 


% \paragraph{\textbf{Key Takeaways}}
% \begin{itemize}
%     \item Out-of-distribution generalization in the parity problem is highly sensitive to the diversity and positional coverage of training tasks.
%     \item Adversarial exclusion of specific pairs or positional configurations can lead to systematic failures, even when most tasks are observed during training.
% \end{itemize}




%################ previous veriosn down
% \textit{How does the choice of training tasks affect the ability of a model to generalize to unseen tasks? Can we predict which setups are likely to lead to failure?}

% To explore these questions, we crafted specific training and test task splits to investigate what makes one setup appear “more” OOD than another.

% \paragraph{Generalization with Missing Pair.}

% Imagine we have tasks constructed from subsets of \(k=3\) elements out of a larger set of \(d\) coordinates. What happens if certain pairs of coordinates are adversarially excluded during training? For example, suppose \(d=5\) and two specific coordinates, \(c_1 = 1\) and \(c_2 = 2\), are excluded. The remaining tasks are formed from subsets of the other coordinates. How would a model perform when tested on tasks involving the excluded pair \( (c_1, c_2) \)? 

% To probe this, we devised three variations in how training tasks are constructed:
%     \begin{enumerate}
%         \item \textbf{Type 1:} The training set includes all tasks except those containing both \( c_1 = 1 \) and \( c_2 = 2 \). 
%         For this example, the training set includes only $\{(3,4,5)\}$. The test set consists of all tasks containing the rest of tuples.

%         \item \textbf{Type 2:} Similar to Type 1, but the training set additionally includes half of the tasks containing either \( c_1 = 1 \) \textit{or} \( c_2 = 2 \) (but not both). 
%         For the example, the training set includes all tasks from Type 1 and adds tasks like \(\{(1, 3, 4), (2, 3, 5)\}\) (half of those containing \( c_1 = 1 \) or \( c_2 = 2 \)).

%         \item \textbf{Type 3:} Similar to Type 2, but the training set also includes half of the tasks containing both \( c_1 = 1 \) \textit{and} \( c_2 = 2 \). 
%         For the example, the training set includes all tasks from Type 2 and adds, for instance, \(\{(1, 2, 5)\}\) (half of the tasks containing both \( c_1 \) and \( c_2 \)).
%     \end{enumerate}

% By systematically increasing the diversity of training tasks in a controlled way, while ensuring no overlap between training and test configurations, we observe an improvement in OOD generalization. 

% % \textit{However, the question is this improvement similar across all coordinate pairs, or does it depend on the specific choices of \(c_1\) and \(c_2\) in the tasks?} 

% \textbf{Before proceeding, consider the following question:} Is the observed improvement consistent across all coordinate pairs, or does it depend on the specific choices of \(c_1\) and \(c_2\) in the tasks? 

% For instance, consider two cases for \(d = 10, k = 3\): (i) \(c_1 = 4, c_2 = 6\) and (ii) \(c_1 = 8, c_2 = 9\). Would you expect similar OOD generalization behavior for these two cases across the three training setups we discussed?



% \paragraph{Answer to the Question.} for both cases of \( c_1, c_2 \), we observe that generalization fails in Type 1, suggesting that the position of the tasks the model has been trained on significantly impacts its generalization capability. For Type 2, we find that \( c_1 = 4, c_2 = 6 \) performs significantly better than \( c_1 = 8, c_2 = 9 \). 

% Upon examining the tasks where the transformer fails for \( c_1 = 8, c_2 = 9 \), we see that the model only fails at tasks of the form \((*, 8, 9)\) while perfectly generalizing to the rest. This indicates that the model has never encountered the value \( 8 \) in the second position during training, which likely explains its failure to generalize. In contrast, for \( c_1 = 4, c_2 = 6 \), while the model has not seen tasks of the form \((*, 4, 6)\), it has encountered tasks where \( 4 \) appears in the second position, such as \((1, 4, 5)\), and tasks where \( 6 \) appears in the third position, such as \((2, 3, 6)\). This difference may explain why the model generalizes almost perfectly in Type 2 for \( c_1 = 4, c_2 = 6 \), but not for \( c_1 = 8, c_2 = 9 \).



% \paragraph{Generalization with Missing Coordinates.}
% Next, we investigate whether a model can generalize to tasks where a specific coordinate appears in an unseen position during training. For instance, consider \( c_1 = 5 \), and exclude all tasks where \( c_1 \) appears in the second position. Despite being trained on all other tasks, the model fails to generalize to these excluded cases, highlighting the importance of positional diversity in training tasks.



% \paragraph{Key Takeaways.}
% \begin{itemize}
%     \item OOD generalization depends heavily on the diversity and positional coverage of training tasks for the parity problem.
%     \item adversarial exclusion of specific pairs or positional configurations in the parity problem can lead to failure, even when the majority of tasks are observed during training.
% \end{itemize}


%################ previous veriosn up

% \paragraph{Key Takeaways} These findings highlight the complexity of OOD generalization, even in seemingly simple tasks like parity. They also underscore the importance of task design: the diversity of training tasks can significantly influence a model’s ability to generalize to unseen tasks. By better understanding these dynamics, we can design more robust training regimes that foster generalization across a wider range of scenarios.


% #############


% Upon examining the tasks where the transformer fails for \( c_1 = 8, c_2 = 9 \), we see that the model only fails at tasks of the form \((*, 8, 9)\) while perfectly generalizing to the rest. This indicates that the model has never encountered the value \( 8 \) in the second position during training, which likely explains its failure to generalize. In contrast, for \( c_1 = 4, c_2 = 6 \), while the model has not seen tasks of the form \((*, 4, 6)\), it has encountered tasks where \( 4 \) appears in the second position, such as \((1, 4, 5)\), and tasks where \( 6 \) appears in the third position, such as \((2, 3, 6)\). This difference may explain why the model generalizes almost perfectly in Type 2 for \( c_1 = 4, c_2 = 6 \), but not for \( c_1 = 8, c_2 = 9 \).

% we observe a striking pattern: generalization fails entirely in Type 1, regardless of the coordinate pair (\(c_1, c_2\)). However, in Type 2, generalization varies: \(c_1 = 4, c_2 = 6\) achieves 96\% accuracy, while \(c_1 = 8, c_2 = 9\) lags behind at 70\%. Why? Upon closer inspection, the model struggles specifically with tasks like \((*, 8, 9)\), where the combination \(c_1 = 8\) and \(c_2 = 9\) is entirely novel. In contrast, for \(c_1 = 4, c_2 = 6\), the model benefits from having seen tasks where \(4\) appears in the second position or \(6\) in the third. This suggests that positional exposure during training plays a key role in generalization.

% To test whether task structure influences generalization, we consider two variations:
% \begin{enumerate}
%     \item \textbf{Sorted Tuples:} Tasks are always sorted in ascending order.
%     \item \textbf{Unsorted Tuples:} Tasks can appear in any order.
% \end{enumerate}

% If the model struggles with generalizing to the excluded position, does introducing variability through unsorted tuples help mitigate this limitation?

% \paragraph{Discussion of Results}

% In \textbf{Generalization with Missing Pairs}, we observe a striking pattern: generalization fails entirely in Type 1, regardless of the coordinate pair (\(c_1, c_2\)). However, in Type 2, generalization varies: \(c_1 = 4, c_2 = 6\) achieves 96\% accuracy, while \(c_1 = 8, c_2 = 9\) lags behind at 70\%. Why? Upon closer inspection, the model struggles specifically with tasks like \((*, 8, 9)\), where the combination \(c_1 = 8\) and \(c_2 = 9\) is entirely novel. In contrast, for \(c_1 = 4, c_2 = 6\), the model benefits from having seen tasks where \(4\) appears in the second position or \(6\) in the third. This suggests that positional exposure during training plays a key role in generalization.

% In \textbf{Generalization with Missing Coordinates}, the results confirm this hypothesis. When \(c_1 = 5\) is excluded from the second position during training, the model fails to generalize to such tasks in the sorted case. However, allowing unsorted tuples introduces positional diversity, leading to near-perfect generalization. This raises an intriguing question: does the model inherently overfit to positional patterns, and can task variability help break this tendency?




% In this subsection, we show that the selection of training tasks can affect the quality of the unseen task generalization significantly in practice. To illustrate this, we present two scenarios where tasks differ between training and testing. For each scenario, we invite the reader to assess, before examining the experimental results, which cases might appear “more” OOD. 

% % \amir{add examples, }

% \kaiyue{I think the name of scenarios here are not very clear}
% \begin{itemize}
%     \item \textbf{Scenario 1:  Generalization Across Excluded Coordinate Pairs (\( k = 3 \))} \\
%     In this scenario, we select two coordinates \( c_1 \) and \( c_2 \) out of \( d \) and construct three types of training sets. 

%     Suppose \( d = 5 \), \( c_1 = 1 \), and \( c_2 = 2 \). The tuples are all possible subsets of \( \{1, 2, 3, 4, 5\} \) with \( k = 3 \):
%     \[
%     \begin{aligned}
%     \big\{ & (1, 2, 3), (1, 2, 4), (1, 2, 5), (1, 3, 4), (1, 3, 5), \\
%            & (1, 4, 5), (2, 3, 4), (2, 3, 5), (2, 4, 5), (3, 4, 5) \big\}.
%     \end{aligned}
%     \]

%     \begin{enumerate}
%         \item \textbf{Type 1:} The training set includes all tuples except those containing both \( c_1 = 1 \) and \( c_2 = 2 \). 
%         For this example, the training set includes only $\{(3,4,5)\}$ tuple. The test set consists of tuples containing the rest of tuples.

%         \item \textbf{Type 2:} Similar to Type 1, but the training set additionally includes half of the tuples containing either \( c_1 = 1 \) \textit{or} \( c_2 = 2 \) (but not both). 
%         For the example, the training set includes all tuples from Type 1 and adds tuples like \(\{(1, 3, 4), (2, 3, 5)\}\) (half of those containing \( c_1 = 1 \) or \( c_2 = 2 \)).

%         \item \textbf{Type 3:} Similar to Type 2, but the training set also includes half of the tuples containing both \( c_1 = 1 \) \textit{and} \( c_2 = 2 \). 
%         For the example, the training set includes all tuples from Type 2 and adds, for instance, \(\{(1, 2, 5)\}\) (half of the tuples containing both \( c_1 \) and \( c_2 \)).
%     \end{enumerate}

% % \begin{itemize}
% %     \item \textbf{Type 1:} The training set includes tuples \(\{1, 3, 4\}, \{2, 3, 4\}\) (excluding tuples with both \( c_1 \) and \( c_2 \): \(\{1, 2, 3\}, \{1, 2, 4\}\)). The test set contains the excluded tuples.
% %     \item \textbf{Type 2:} The training set includes all tuples in Type 1 plus half of the tuples containing either \( c_1 = 1 \) or \( c_2 = 2 \) (e.g., \(\{1, 2, 3\}\)).
% %     \item \textbf{Type 3:} The training set includes all tuples in Type 2 plus half of the tuples containing both \( c_1 = 1 \) and \( c_2 = 2 \) (e.g., \(\{1, 2, 4\}\)).
% % \end{itemize}
    
%     \item \textbf{Scenario 2: Scenario 2: Generalization Across a Fixed Coordinate (\( k = 3 \))} \\
%     In this scenario, we select one coordinate \( c_1 \) out of \( d \) (\( c_1 = 5 \)). The training set includes all task tuples except those where \( c_1 \) is the second coordinate of the tuple. For this scenario, we examine two variations:
%     \begin{enumerate}
%         \item \textbf{Sorted Tuples:} Task tuples are always sorted (e.g., \( (x_1, x_2, x_3) \) with \( x_1 \leq x_2 \leq x_3 \)).
%         \item \textbf{Unsorted Tuples:} Task tuples can appear in any order.
%     \end{enumerate}
% \end{itemize}




% \paragraph{Discussion of Results.} In the first scenario, for both cases of \( c_1, c_2 \), we observe that generalization fails in Type 1, suggesting that the position of the tasks the model has been trained on significantly impacts its generalization capability. For Type 2, we find that \( c_1 = 4, c_2 = 6 \) performs significantly better than \( c_1 = 8, c_2 = 9 \). 

% Upon examining the tasks where the transformer fails for \( c_1 = 8, c_2 = 9 \), we see that the model only fails at tasks of the form \((*, 8, 9)\) while perfectly generalizing to the rest. This indicates that the model has never encountered the value \( 8 \) in the second position during training, which likely explains its failure to generalize. In contrast, for \( c_1 = 4, c_2 = 6 \), while the model has not seen tasks of the form \((*, 4, 6)\), it has encountered tasks where \( 4 \) appears in the second position, such as \((1, 4, 5)\), and tasks where \( 6 \) appears in the third position, such as \((2, 3, 6)\). This difference may explain why the model generalizes almost perfectly in Type 2 for \( c_1 = 4, c_2 = 6 \), but not for \( c_1 = 8, c_2 = 9 \).

% This position-based explanation appears compelling, so in the second scenario, we focus on a single position to investigate further. Here, we find that the transformer fails to generalize to tasks where \( 5 \) appears in the second position, provided it has never seen any such tasks during training. However, when we allow for more task diversity in the unsorted case, the model achieves near-perfect generalization. 

% This raises an important question: does the transformer have a tendency to overfit to positional patterns, and does introducing more task variability, as in the unsorted case, prevent this overfitting and enable generalization to unseen positional configurations?

% These findings highlight that even in a simple task like parity, it is remarkably challenging to understand and quantify the sources and levels of OOD behavior. This motivates further investigation into the nuances of task design and its impact on model generalization.


\subsection{Task Generalization Beyond Parity Problems}

% \begin{figure}[t!]
%     \centering
%     \includegraphics[width=0.45\textwidth]{Figures/arithmetic_v1.png}
%     \vspace{-0.3cm}
%     \caption{Task generalization for arithmetic task with CoT, it has $\d =2$ and $T = d-1$ as the ambient dimension, hence $D\ln(DT) = 2\ln(2T)$. We show that the empirical scaling closely follows the theoretical scaling.}
%     \label{fig:arithmetic}
% \end{figure}



% \begin{wrapfigure}{r}{0.4\textwidth}  % 'r' for right, 'l' for left
%     \centering
%     \includegraphics[width=0.4\textwidth]{Figures/arithmetic_v1.png}
%     \vspace{-0.3cm}
%     \caption{Task generalization for the arithmetic task with CoT. It has $d =2$ and $T = d-1$ as the ambient dimension, hence $D\ln(DT) = 2\ln(2T)$. We show that the empirical scaling closely follows the theoretical scaling.}
%     \label{fig:arithmetic}
% \end{wrapfigure}

\subsubsection{Arithmetic Task}\label{subsec:arithmetic}











We introduce the family of \textit{Arithmetic} task that, like the sparse parity problem, operates on 
\( d \) binary inputs \( b_1, b_2, \dots, b_d \). The task involves computing a structured arithmetic expression over these inputs using a sequence of addition and multiplication operations.
\newcommand{\op}{\textrm{op}}

Formally, we define the function:
\[
\text{Arithmetic}_{S} \colon \{0,1\}^d \to \{0,1,\dots,d\},
\]
where \( S = (\op_1, \op_2, \dots, \op_{d-1}) \) is a sequence of \( d-1 \) operations, each \( \op_k \) chosen from \( \{+, \times\} \). The function evaluates the expression by applying the operations sequentially from left-to-right order: for example, if \( S = (+, \times, +) \), then the arithmetic function would compute:
\[
\text{Arithmetic}_{S}(b_1, b_2, b_3, b_4) = ((b_1 + b_2) \times b_3) + b_4.
\]
% Thus, the sequence of operations \( S \) defines how the binary inputs are combined to produce an integer output between \( 0 \) and \( d \).
% \[
% \text{Arithmetic}_{S} 
% (b_1,\,b_2,\,\dots,b_d)
% =
% \Bigl(\dots\bigl(\,(b_1 \;\op_1\; b_2)\;\op_2\; b_3\bigr)\,\dots\Bigr) 
% \;\op_{d-1}\; b_d.
% \]
% We now introduce an \emph{Arithmetic} task that, like the sparse parity problem, operates on $d$ binary inputs $b_1, b_2, \dots, b_d$. Specifically, we define an arithmetic function
% \[
% \text{Arithmetic}_{S}\colon \{0,1\}^d \;\to\; \{0,1,\dots,d\},
% \]
% where $S = (i_1, i_2, \dots, i_{d-1})$ is a sequence of $d-1$ operations, each $i_k \in \{+,\,\times\}$. The value of $\text{Arithmetic}_{S}$ is obtained by applying the prescribed addition and multiplication operations in order, namely:
% \[
% \text{Arithmetic}_{S}(b_1,\,b_2,\,\dots,b_d)
% \;=\;
% \Bigl(\dots\bigl(\,(b_1 \;i_1\; b_2)\;i_2\; b_3\bigr)\,\dots\Bigr) 
% \;i_{d-1}\; b_d.
% \]

% This is an example of our framework where $T = d-1$ and $|\Theta_t| = 2$ with total $2^d$ possible tasks. 




By introducing a step-by-step CoT, arithmetic class belongs to $ARC(2, d-1)$: this is because at every step, there is only $\d = |\Theta_t| = 2$ choices (either $+$ or $\times$) while the length is  $T = d-1$, resulting a total number of $2^{d-1}$ tasks. 


\begin{minipage}{0.5\textwidth}  % Left: Text
    Task generalization for the arithmetic task with CoT. It has $d =2$ and $T = d-1$ as the ambient dimension, hence $D\ln(DT) = 2\ln(2T)$. We show that the empirical scaling closely follows the theoretical scaling.
\end{minipage}
\hfill
\begin{minipage}{0.4\textwidth}  % Right: Image
    \centering
    \includegraphics[width=\textwidth]{Figures/arithmetic_v1.png}
    \refstepcounter{figure}  % Manually advances the figure counter
    \label{fig:arithmetic}  % Now this label correctly refers to the figure
\end{minipage}

Notably, when scaling with \( T \), we observe in the figure above that the task scaling closely follow the theoretical $O(D\log(DT))$ dependency. Given that the function class grows exponentially as \( 2^T \), it is truly remarkable that training on only a few hundred tasks enables generalization to an exponentially larger space—on the order of \( 2^{25} > 33 \) Million tasks. This exponential scaling highlights the efficiency of structured learning, where a modest number of training examples can yield vast generalization capability.





% Our theory suggests that only $\Tilde{O}(\ln(T))$ i.i.d training tasks is enough to generalize to the rest of unseen tasks. However, we show in Figure \ref{fig:arithmetic} that transformer is not able to match  that. The transformer out-of distribution generalization behavior is not consistent across different dimensions when we scale the number of training tasks with $\ln(T)$. \hongzhou{implicit bias, optimization, etc}
 






% \subsection{Task generalization Beyond parity problem}

% \subsection{Arithmetic} In this setting, we still use the set-up we introduced in \ref{subsec:parity_exmaple}, the input is still a set of $d$ binary variable, $b_1, b_2,\dots,b_d$ and ${Arithmatic_{S}}:\{0,1\}\rightarrow \{0, 1, \dots, d\}$, where $S = (i_1,i_2,\dots,i_{d-1})$ is a tuple of size $d-1$ where each coordinate is either add($+
% $) or multiplication ($\times$). The function is as following,

% \begin{align*}
%     Arithmatic_{S}(b_1, b_2,\dots,b_d) = (\dots(b1(i1)b2)(i3)b3\dots)(i{d-1})
% \end{align*}
    


\subsubsection{Multi-Step Language Translation Task}

 \begin{figure*}[h!]
    \centering
    \includegraphics[width=0.9\textwidth]{Figures/combined_plot_horiz.png}
    \vspace{-0.2cm}
    \caption{Task generalization for language translation task: $\d$ is the number of languages and $T$ is the length of steps.}
    \vspace{-2mm}
    \label{fig:language}
\end{figure*}
% \vspace{-2mm}

In this task, we study a sequential translation process across multiple languages~\cite{garg2022can}. Given a set of \( D \) languages, we construct a translation chain by randomly sampling a sequence of \( T \) languages \textbf{with replacement}:  \(L_1, L_2, \dots, L_T,\)
where each \( L_t \) is a sampled language. Starting with a word, we iteratively translate it through the sequence:
\vspace{-2mm}
\[
L_1 \to L_2 \to L_3 \to \dots \to L_T.
\]
For example, if the sampled sequence is EN → FR → DE → FR, translating the word "butterfly" follows:
\vspace{-1mm}
\[
\text{butterfly} \to \text{papillon} \to \text{schmetterling} \to \text{papillon}.
\]
This task follows an \textit{AutoRegressive Compositional} structure by itself, specifically \( ARC(D, T-1) \), where at each step, the conditional generation only depends on the target language, making \( D \) as the number of languages and the total number of possible tasks is \( D^{T-1} \). This example illustrates that autoregressive compositional structures naturally arise in real-world languages, even without explicit CoT. 

We examine task scaling along \( D \) (number of languages) and \( T \) (sequence length). As shown in Figure~\ref{fig:language}, empirical  \( D \)-scaling closely follows the theoretical \( O(D \ln D T) \). However, in the \( T \)-scaling case, we observe a linear dependency on \( T \) rather than the logarithmic dependency \(O(\ln T) \). A possible explanation is error accumulation across sequential steps—longer sequences require higher precision in intermediate steps to maintain accuracy. This contrasts with our theoretical analysis, which focuses on asymptotic scaling and does not explicitly account for compounding errors in finite-sample settings.

% We examine task scaling along \( D \) (number of languages) and \( T \) (sequence length). As shown in Figure~\ref{fig:language}, empirical scaling closely follows the theoretical \( O(D \ln D T) \) trend, with slight exceptions at $ T=10 \text{ and } 3$ in Panel B. One possible explanation for this deviation could be error accumulation across sequential steps—longer sequences require each intermediate translation to be approximated with higher precision to maintain test accuracy. This contrasts with our theoretical analysis, which primarily focuses on asymptotic scaling and does not explicitly account for compounding errors in finite-sample settings.

Despite this, the task scaling is still remarkable — training on a few hundred tasks enables generalization to   $4^{10} \approx 10^6$ tasks!






% , this case, we are in a regime where \( D \ll T \), we observe  that the task complexity empirically scales as \( T \log T \) rather than \( D \log T \). 


% the model generalizes to an exponentially larger space of \( 2^T \) unseen tasks. In case $T=25$, this is $2^{25} > 33$ Million tasks. This remarkable exponential generalization demonstrates the power of structured task composition in enabling efficient generalization.


% In the case of parity tasks, introducing CoT effectively decomposes the problem from \( ARC(D^T, 1) \) to \( ARC(D, T) \), significantly improving task generalization.

% Again, in the regime scaling $T$, we again observe a $T\log T$ dependency. Knowing that the function class is scaling as $D^T$, it is remarkable that training on a few hundreds tasks can generalize to $4^{10} \approx 1M$ tasks. 





% We further performed a preliminary investigation on a semi-synthetic word-level translation task to show that (1) task generalization via composition structure is feasible beyond parity and (2) understanding the fine-grained mechanism leading to this generalization is still challenging. 
% \noindent
% \noindent
% \begin{minipage}[t]{\columnwidth}
%     \centering
%     \textbf{\scriptsize In-context examples:}
%     \[
%     \begin{array}{rl}
%         \textbf{Input} & \hspace{1.5em} \textbf{Output} \\
%         \hline
%         \textcolor{blue}{car}   & \hspace{1.5em} \textcolor{red}{voiture \;,\; coche} \\
%         \textcolor{blue}{house} & \hspace{1.5em} \textcolor{red}{maison \;,\; casa} \\
%         \textcolor{blue}{dog}   & \hspace{1.5em} \textcolor{red}{chien \;,\; perro} 
%     \end{array}
%     \]
%     \textbf{\scriptsize Query:}
%     \[
%     \begin{array}{rl}
%         \textbf{Input} & \textbf{Output} \\
%         \hline
%         \textcolor{blue}{cat} & \hspace{1.5em} \textcolor{red}{?} \\
%     \end{array}
%     \]
% \end{minipage}



% \begin{figure}[h!]
%     \centering
%     \includegraphics[width=0.45\textwidth]{Figures/translation_scale_d.png}
%     \vspace{-0.2cm}
%     \caption{Task generalization behavior for word translation task.}
%     \label{fig:arithmetic}
% \end{figure}


\vspace{-1mm}
\section{Conclusions}
% \misha{is it conclusion of the section or of the whole paper?}    
% \amir{The whole paper. It is very short, do we need a separate section?}
% \misha{it should not be a subsection if it is the conclusion the whole thing. We can just remove it , it does not look informative} \hz{let's do it in a whole section, just to conclude and end the paper, even though it is not informative}
%     \kaiyue{Proposal: Talk about the implication of this result on theory development. For example, it calls for more fine-grained theoretical study in this space.  }

% \huaqing{Please feel free to edit it if you have better wording or suggestions.}

% In this work, we propose a theoretical framework to quantitatively investigate task generalization with compositional autoregressive tasks. We show that task to $D^T$ task is theoretically achievable by training on only $O (D\log DT)$ tasks, and empirically observe that transformers trained on parity problem indeed achieves such task generalization. However, for other tasks beyond parity, transformers seem to fail to achieve this bond. This calls for more fine-grained theoretical study the phenomenon of task generalization specific to transformer model. It may also be interesting to study task generalization beyond the setting of in-context learning. 
% \misha{what does this add?} \amir{It does not, i dont have any particular opinion to keep it. @Hongzhou if you want to add here?}\hz{While it may not introduce anything new, we are following a good practice to have a short conclusion. It provides a clear closing statement, reinforces key takeaways, and helps the reader leave with a well-framed understanding of our contributions. }
% In this work, we quantitatively investigate task generalization under autoregressive compositional structure. We demonstrate that task generalization to $D^T$ tasks is theoretically achievable by training on only $\tilde O(D)$ tasks. Empirically, we observe that transformers trained indeed achieve such exponential task generalization on problems such as parity, arithmetic and multi-step language translation. We believe our analysis opens up a new angle to understand the remarkable generalization ability of Transformer in practice. 

% However, for tasks beyond the parity problem, transformers appear to fail to reach this bound. This highlights the need for a more fine-grained theoretical exploration of task generalization, especially for transformer models. Additionally, it may be valuable to investigate task generalization beyond the scope of in-context learning.



In this work, we quantitatively investigated task generalization under the autoregressive compositional structure, demonstrating both theoretically and empirically that exponential task generalization to $D^T$ tasks can be achieved with training on only $\tilde{O}(D)$ tasks. %Our theoretical results establish a fundamental scaling law for task generalization, while our experiments validate these insights across problems such as parity, arithmetic, and multi-step language translation. The remarkable ability of transformers to generalize exponentially highlights the power of structured learning and provides a new perspective on how large language models extend their capabilities beyond seen tasks. 
We recap our key contributions  as follows:
\begin{itemize}
    \item \textbf{Theoretical Framework for Task Generalization.} We introduced the \emph{AutoRegressive Compositional} (ARC) framework to model structured task learning, demonstrating that a model trained on only $\tilde{O}(D)$ tasks can generalize to an exponentially large space of $D^T$ tasks.
    
    \item \textbf{Formal Sample Complexity Bound.} We established a fundamental scaling law that quantifies the number of tasks required for generalization, proving that exponential generalization is theoretically achievable with only a logarithmic increase in training samples.
    
    \item \textbf{Empirical Validation on Parity Functions.} We showed that Transformers struggle with standard in-context learning (ICL) on parity tasks but achieve exponential generalization when Chain-of-Thought (CoT) reasoning is introduced. Our results provide the first empirical demonstration of structured learning enabling efficient generalization in this setting.
    
    \item \textbf{Scaling Laws in Arithmetic and Language Translation.} Extending beyond parity functions, we demonstrated that the same compositional principles hold for arithmetic operations and multi-step language translation, confirming that structured learning significantly reduces the task complexity required for generalization.
    
    \item \textbf{Impact of Training Task Selection.} We analyzed how different task selection strategies affect generalization, showing that adversarially chosen training tasks can hinder generalization, while diverse training distributions promote robust learning across unseen tasks.
\end{itemize}



We introduce a framework for studying the role of compositionality in learning tasks and how this structure can significantly enhance generalization to unseen tasks. Additionally, we provide empirical evidence on learning tasks, such as the parity problem, demonstrating that transformers follow the scaling behavior predicted by our compositionality-based theory. Future research will  explore how these principles extend to real-world applications such as program synthesis, mathematical reasoning, and decision-making tasks. 


By establishing a principled framework for task generalization, our work advances the understanding of how models can learn efficiently beyond supervised training and adapt to new task distributions. We hope these insights will inspire further research into the mechanisms underlying task generalization and compositional generalization.

\section*{Acknowledgements}
We acknowledge support from the National Science Foundation (NSF) and the Simons Foundation for the Collaboration on the Theoretical Foundations of Deep Learning through awards DMS-2031883 and \#814639 as well as the  TILOS institute (NSF CCF-2112665) and the Office of Naval Research (ONR N000142412631). 
This work used the programs (1) XSEDE (Extreme science and engineering discovery environment)  which is supported by NSF grant numbers ACI-1548562, and (2) ACCESS (Advanced cyberinfrastructure coordination ecosystem: services \& support) which is supported by NSF grants numbers \#2138259, \#2138286, \#2138307, \#2137603, and \#2138296. Specifically, we used the resources from SDSC Expanse GPU compute nodes, and NCSA Delta system, via allocations TG-CIS220009. 

We present RiskHarvester, a risk-based tool to compute a security risk score based on the value of the asset and ease of attack on a database. We calculated the value of asset by identifying the sensitive data categories present in a database from the database keywords. We utilized data flow analysis, SQL, and Object Relational Mapper (ORM) parsing to identify the database keywords. To calculate the ease of attack, we utilized passive network analysis to retrieve the database host information. To evaluate RiskHarvester, we curated RiskBench, a benchmark of 1,791 database secret-asset pairs with sensitive data categories and host information manually retrieved from 188 GitHub repositories. RiskHarvester demonstrates precision of (95\%) and recall (90\%) in detecting database keywords for the value of asset and precision of (96\%) and recall (94\%) in detecting valid hosts for ease of attack. Finally, we conducted an online survey to understand whether developers prioritize secret removal based on security risk score. We found that 86\% of the developers prioritized the secrets for removal with descending security risk scores.

\appendix



\begin{table*}[tb]
\centering
\caption{The fine-tuning evaluations demonstrate our great transferring ability on SS-MIP tasks which pre-trained on PPMI dataset. Our GEMINI-MIP achieves the best performance compared with 18 methods on two downstream tasks.}
\begin{tabular}{clcccccccccc}
\toprule
\multirow{2}{*}{\textbf{Type}}
&\multirow{2}{*}{\textbf{Pre-training}}
&\multicolumn{2}{c}{\textbf{T2: KiPA22} \emph{Inter-scene}}
&\multicolumn{2}{c}{\textbf{T3: CANDI} \emph{Inner-scene}}
%&\multicolumn{2}{c}{\textbf{T3: CANDI-mini} \emph{Inner-scene}}
&\textbf{AVG}
\\
\cmidrule(r){3-4}
\cmidrule(r){5-6}
%\cmidrule(r){7-8}
\cmidrule(r){7-7}
&
&DSC$_{\pm std}\uparrow$
&AVD$_{\pm std}\downarrow$
&DSC$_{\pm std}\uparrow$
&AVD$_{\pm std}\downarrow$
%&DSC$_{\pm std}\uparrow$
%&AVD$_{\pm std}\downarrow$
&DSC$_{\pm std}\uparrow$
\\
\midrule
-
&Scratch (3D U-Net)
&72.4$_{\pm16.3}$
&6.11$_{\pm5.91}$
&84.0$_{\pm3.2}$
&0.52$_{\pm0.14}$
%&62.1$_{\pm10.1}$
%&1.59$_{\pm0.83}$
&78.2$_{\pm9.8}$
\\
\cdashline{1-7}[0.8pt/2pt]
Sup
&Med3D \cite{chen2019med3d}
&81.7$_{\pm12.0}$
&\color{blue}2.61$_{\pm2.77}$
&72.7$_{\pm19.0}$
&1.57$_{\pm2.56}$
%&$_{\pm}$
%&$_{\pm}$
&77.2$_{\pm15.5}$
\\
GRL
&Denosing \cite{vincent2010stacked}
&70.0$_{\pm15.4}$
&7.60$_{\pm5.03}$
&\cellcolor[gray]{0.9}83.7$_{\pm3.3}$
&\cellcolor[gray]{0.9}1.71$_{\pm0.20}$
%&\cellcolor[gray]{0.9}unable%$_{\pm}$
%&\cellcolor[gray]{0.9}unable%$_{\pm}$
&76.9$_{\pm9.4}$
\\
&In-painting \cite{pathak2016context}
&69.7$_{\pm17.1}$
&7.57$_{\pm5.93}$
&\cellcolor[gray]{0.9}88.5$_{\pm3.1}$
&\cellcolor[gray]{0.9}0.32$_{\pm0.11}$
%&\cellcolor[gray]{0.9}$_{\pm}$
%&\cellcolor[gray]{0.9}$_{\pm}$
&79.1$_{\pm10.1}$
\\
&Models Genesis \cite{zhou2019models}
&75.8$_{\pm13.7}$
&4.64$_{\pm4.49}$
&\cellcolor[gray]{0.9}88.7$_{\pm3.1}$
&\cellcolor[gray]{0.9}0.31$_{\pm0.10}$
%&\cellcolor[gray]{0.9}$_{\pm}$
%&\cellcolor[gray]{0.9}$_{\pm}$
&82.3$_{\pm8.4}$
\\
&Rotation \cite{komodakis2018unsupervised}
&77.4$_{\pm14.3}$
&4.82$_{\pm6.29}$
&\cellcolor[gray]{0.9}89.4$_{\pm2.6}$
&\cellcolor[gray]{0.9}0.28$_{\pm0.08}$
%&\cellcolor[gray]{0.9}$_{\pm}$
%&\cellcolor[gray]{0.9}$_{\pm}$
&83.4$_{\pm8.5}$
\\
CRL
&SimSiam \cite{Chen2021CVPR}
&83.8$_{\pm11.9}$
&3.69$_{\pm7.47}$
&\cellcolor[gray]{0.9}87.3$_{\pm3.1}$
&\cellcolor[gray]{0.9}0.36$_{\pm0.10}$
%&\cellcolor[gray]{0.9}$_{\pm}$
%&\cellcolor[gray]{0.9}$_{\pm}$
&85.6$_{\pm7.5}$
\\
&BYOL \cite{grill2020bootstrap}
&83.6$_{\pm11.2}$
&2.78$_{\pm5.42}$
&\cellcolor[gray]{0.9}89.7$_{\pm2.4}$
&\cellcolor[gray]{0.9}\color{blue}0.27$_{\pm0.08}$
%&\cellcolor[gray]{0.9}$_{\pm}$
%&\cellcolor[gray]{0.9}$_{\pm}$
&\color{blue}86.7$_{\pm6.8}$
\\
&SimCLR \cite{chen2020simple}
&78.9$_{\pm13.9}$
&4.49$_{\pm5.15}$
&\cellcolor[gray]{0.9}89.2$_{\pm3.0}$
&\cellcolor[gray]{0.9}0.30$_{\pm0.14}$
%&\cellcolor[gray]{0.9}$_{\pm}$
%&\cellcolor[gray]{0.9}$_{\pm}$
&84.1$_{\pm8.5}$
\\
&MoCov2 \cite{chen2020improved}
&78.0$_{\pm15.3}$
&4.42$_{\pm5.67}$
&\cellcolor[gray]{0.9}89.7$_{\pm2.4}$
&\cellcolor[gray]{0.9}0.28$_{\pm0.11}$
%&\cellcolor[gray]{0.9}$_{\pm}$
%&\cellcolor[gray]{0.9}$_{\pm}$
&83.9$_{\pm8.9}$
\\
&DeepCluster \cite{caron2018deep}
&79.7$_{\pm13.7}$
&4.28$_{\pm5.76}$
&\cellcolor[gray]{0.9}89.8$_{\pm2.4}$
&\cellcolor[gray]{0.9}\color{blue}0.27$_{\pm0.08}$
%&\cellcolor[gray]{0.9}$_{\pm}$
%&\cellcolor[gray]{0.9}$_{\pm}$
&84.8$_{\pm8.1}$
\\
DCRL
&VADeR \cite{o2020unsupervised}
&72.1$_{\pm13.8}$
&6.56$_{\pm5.89}$
&\cellcolor[gray]{0.9}87.4$_{\pm3.6}$
&\cellcolor[gray]{0.9}0.35$_{\pm0.11}$
%&\cellcolor[gray]{0.9}$_{\pm}$
%&\cellcolor[gray]{0.9}$_{\pm}$
&79.8$_{\pm8.7}$
\\
&DenseCL \cite{wang2022densecl}
&74.0$_{\pm15.8}$
&6.42$_{\pm8.21}$
&\cellcolor[gray]{0.9}87.7$_{\pm3.8}$
&\cellcolor[gray]{0.9}0.34$_{\pm0.13}$
%&\cellcolor[gray]{0.9}$_{\pm}$
%&\cellcolor[gray]{0.9}$_{\pm}$
&80.9$_{\pm9.8}$
\\
&SetSim \cite{wang2022exploring}
&73.5$_{\pm15.9}$
&6.34$_{\pm6.68}$
&\cellcolor[gray]{0.9}88.4$_{\pm3.1}$
&\cellcolor[gray]{0.9}0.32$_{\pm0.10}$
%&\cellcolor[gray]{0.9}$_{\pm}$
%&\cellcolor[gray]{0.9}$_{\pm}$
&81.0$_{\pm9.5}$
\\
&DSC-PM \cite{li2021dense}
&79.0$_{\pm14.6}$
&4.90$_{\pm6.05}$
&\cellcolor[gray]{0.9}88.5$_{\pm3.4}$
&\cellcolor[gray]{0.9}0.32$_{\pm0.13}$
%&\cellcolor[gray]{0.9}$_{\pm}$
%&\cellcolor[gray]{0.9}$_{\pm}$
&83.8$_{\pm9.0}$
\\
&PixPro \cite{xie2021propagate}
&80.0$_{\pm14.4}$
&4.60$_{\pm6.25}$
&\cellcolor[gray]{0.9}\color{blue}89.9$_{\pm2.4}$
&\cellcolor[gray]{0.9}\color{blue}0.27$_{\pm0.07}$
%&\cellcolor[gray]{0.9}$_{\pm}$
%&\cellcolor[gray]{0.9}$_{\pm}$
&85.0$_{\pm8.4}$
\\
&GLCL \cite{chaitanya2020contrastive}
&70.7$_{\pm16.9}$
&7.33$_{\pm7.05}$
&\cellcolor[gray]{0.9}87.4$_{\pm3.2}$
&\cellcolor[gray]{0.9}0.34$_{\pm0.09}$
%&\cellcolor[gray]{0.9}$_{\pm}$
%&\cellcolor[gray]{0.9}$_{\pm}$
&79.1$_{\pm10.1}$
\\
\cdashline{1-7}[0.8pt/2pt]
\textbf{DCRL}
&\textbf{GVSL-MIP (CVPR)}\cite{He_2023_CVPR}
&\color{blue}84.3$_{\pm10.3}$
&2.85$_{\pm5.12}$
&\cellcolor[gray]{0.9}89.1$_{\pm2.8}$
&\cellcolor[gray]{0.9}0.31$_{\pm0.11}$
%&\cellcolor[gray]{0.9}$_{\pm}$
%&\cellcolor[gray]{0.9}$_{\pm}$
&\color{blue}86.7$_{\pm6.6}$
\\
\textbf{(Ours)}
&\textbf{GEMINI-MIP}
&\color{red}\textbf{85.0$_{\pm10.2}$}
&\color{red}\textbf{2.55$_{\pm5.71}$}
&\cellcolor[gray]{0.9}\color{red}\textbf{90.0$_{\pm2.4}$}
&\cellcolor[gray]{0.9}\color{red}\textbf{0.26$_{\pm0.07}$}
%&\cellcolor[gray]{0.9}76.3$_{\pm6.0}$
%&\cellcolor[gray]{0.9}0.78$_{\pm0.26}$
&\color{red}\textbf{87.5$_{\pm6.3}$}
\\
\bottomrule
\end{tabular}
\label{supp:tab:metrics}
\end{table*}
\section*{A SS-MIP on more datasets}
\label{aupp:sec:task1}
\subsection*{A.1 Self-supervised pre-training on PPMI dataset}
We further evaluate the SS-MIP task on another pretext dataset for pre-training to demonstrate our representation ability. We extracted 837 3D brain T1 MR images with Parkinson’s disease from the PPMI database\footnote{PPMI database: \url{https://www.ppmi-info.org/}} as our pretext dataset. In our experiment, we extract the brain regions via HD-BET \cite{isensee2019automated}, crop and resize the images to $160\times160\times128$, and finally normalize them via the zero-score. Due to the consistency of the human brain regions, we randomly pair these brain images to pre-train the frameworks. Following the Experiment 2 (Sec.5) in our manuscript, we take the Task 2: KiPA22 dataset and Task 3: CANDA as the downstream tasks to evaluate the inter-scene and inner-scene transferring abilities. (Because the Task 1: SCR$_{25}$ dataset is 2D and the pre-trained models are 3D, we exclude this task in this experiment.) We utilize the same implementation and evaluation metrics as the Sec.5 in this experiment.

As shown in Tab.\ref{supp:tab:metrics}, it achieves similar observations as the SS-MIP experiment in Sec.5. For most of the methods, the pre-training on the PPMI dataset will bring better performance than random initialization (“Scratch”) both in the T2: KiPA22 and T3: CANDI tasks. Especially in the T3 (inner-scene), most of the pre-training methods achieve more than 4.0\% DSC improvement compared with the “Scratch”. Even though the other CRL and DCRL methods have FP\&N problems in this experiment, they are still able to learn the representation of some domain features and promote their final performance to the upper limit of the task 3 (near 90\%). When transferring the pre-trained models to the T2 (inter-scene), the SimSiam, BYOL, our GVSL-MIP, and our GEMINI-MIP all still have significant improvement (more than 10\% DSC). This is because these methods learn the consistency of features and avoid the FP\&N problems. The other methods’ performance improvement is obviously decreased owing to the FP or FN problem which interrupts their representation learning of high-level semantics and makes their representations deviate from reality. On both two tasks, our GEMINI-MIP achieves the highest performance showing our superiority.
\begin{table}[tb]
\centering
\caption{The gap coefficient $G^{i}$ quantifies the gap between “pre-trained on chest X-ray images \& fine-tuning on brain T1 MR images” (inter-scene) and “pre-trained on brain T1 MR images \& fine-tuning on brain T1 MR images” (inner-scene).}
\resizebox{\linewidth}{!}{
\begin{tabular}{ccccc}
\toprule
\textbf{Index}
&\multirow{3}{*}{\textbf{Method}}
&\textbf{Chest X-ray}
&\textbf{Brain T1 MR}
%&\multicolumn{2}{c}{\textbf{T3: CANDI-mini} \emph{Inner-scene}}
&\textbf{Gap}
\\
\cmidrule(r){1-1}
\cmidrule(r){3-3}
\cmidrule(r){4-4}
%\cmidrule(r){7-8}
\cmidrule(r){5-5}
\multirow{2}{*}{$i$}
&
&2D U-Net
&3D U-Net
&\multirow{2}{*}{$G^{i}$}
\\
&
&\emph{Inter-scene}
&\emph{Inner-scene}
&
\\
\midrule
0
&Scratch
&65.0$_{\pm4.4}$
&84.0$_{\pm3.2}$
&1
\\
\cdashline{1-5}[0.8pt/2pt]
1
&BYOL
&70.5$_{\pm2.1}$
&\cellcolor[gray]{0.9}89.7$_{\pm2.4}$
&1.01
\\
2
&DeepCluster
&60.0$_{\pm2.2}$
&\cellcolor[gray]{0.9}89.8$_{\pm2.4}$
&1.57
\\
3
&Model Genesis
&88.1$_{\pm3.1}$
&\cellcolor[gray]{0.9}88.7$_{\pm3.1}$
&0.03
\\
4
&DenseCL
&76.8$_{\pm2.9}$
&\cellcolor[gray]{0.9}87.7$_{\pm3.8}$
&0.57
\\
\cdashline{1-5}[0.8pt/2pt]
\textbf{5}
&\textbf{Our GEMINI-MIP}
&\textbf{89.8$_{\pm2.6}$}
&\cellcolor[gray]{0.9}\textbf{90.0$_{\pm2.4}$}
&\textbf{0.01}
\\
\bottomrule
\end{tabular}
}
\label{supp:tab:gap}
\end{table}
\subsection*{A.2 Analysis of the gap between the inner-scene and inter-scene transferring}
As shown in Tab.\ref{supp:tab:gap}, the quantitative evaluation of the gap between the inner-scene and inter-scene transferring show our great transferring ability both inner scene and inter scene. Here, we formulate a gap coefficient $G$ to quantify this gap:
\begin{equation}\label{equ:gap}
G^{i}=\frac{S^{i}_{inner}-S^{i}_{inter}}{S^{0}_{inner}-S^{0}_{inter}},
\end{equation}
where the $i$ is the index of the method, $S$ is the score of the method (here we take the DSC). The $S_{inner}^{0}-S_{inter}^{0}$ is the gap of the “Scratch” between the two settings which means the difference caused by the initial situation, such as network structure and dimension. The $S_{inner}^{i}-S_{inter}^{i}$ is the gap of the $i_{th}$ method between the two settings. Therefore, the $\frac{S_{inner}^{i}-S_{inter}^{i}}{S_{inner}^{0}-S_{inter}^{0}}$ means the gap of the model in two settings excluding the gap caused by the initial network. If the $G^{i}$ is larger than 1, it means that the pre-trained model has weaker inter-scene transferring ability than inner-scene transferring. If it is smaller than 1, it means that the model has great inter-scene transferring ability.

Most self-supervised learning methods have large gap between inner- and inter-scene transferring, and our GEMINI has great universal representation for different scenes. The BYOL and DeepCluster are limited in the inter-scene transferring ($G>1$) because they only take the image-level contrast which will represent the high-level semantic features and this representation is very different between scenes. The DenseCL has 0.57 gap coefficient which is better than the BYOL and DenseCL. Because it takes dense contrastive learning which also represent low-level detail features and this representation is shared in different scene. Our GEMINI and the Model Genesis all have good inter-scene transferring ability with very low gap coefficient (0.01 and 0.03), showing their great universal representation ability and demonstrating their potential as an initialization for more scenes.

\section*{B Discussion of the research problem and method}
\subsection*{B.1 Discussion of FP\&N problem}
\begin{figure}[tb]
  \centering
  \includegraphics[width=\linewidth]{./picture/FPproblem.pdf}
  \caption{The evaluation of the large-scale FP problem. The true positive (TP) pairs constructed by the features’ similarity (used in DenseCL) only occupy the 5.79\% of the foreground region, and our GEMINI is able to bring 60.74\% TP pairs. }\label{supp:fig:fp}
\end{figure}
As analyzed in the Introduction section, medical images' semantic dependence property will make large-scale FP problem, and their semantic continuity and semantic overlap properties will make large-scale FN problem. In this section, we make an experiment to quantitatively count the percentage of FP and FN pairs in the pairing process.

For FP pairs, we utilize two cardiac CT images (image A and B), and extract their pixel-wise features via a random initialized 3D U-Net. Then, we utilize the pixel-wise feature similarity measurement method in the DenseCL \cite{wang2022densecl} to extract the positive pairs. Because the semantics of the background region are unclear, we count the accuracy of the feature pairs in the foreground regions. As shown in Fig.A, only 5.79\% of the positive pairs in the foreground region are accurate. Therefore, if we directly pair the features only according to their similarity, most of the contrasts (94.21\%) for positive pairs are inaccurate in the medical images and will interrupt the whole contrastive learning process. This is because medical images have very weak contrast due to their special imaging way, making the directly extracted features lack discrimination. Therefore, it makes the “Semantic dependence” one of the inherent properties in medical images constructing large-scale FP pairs.

For FN pairs, we further evaluate the percentage of FN pairs caused by the semantic continuity and semantic overlap properties, and the results show large potential limitations in the DCRL. a) For the PN pairs caused by the “Semantic continuity”, we follow the SimCLR \cite{chen2020simple} which pairs the negative features for each feature. We pair the features in different positions of image A’s foreground regions as negative pairs. The result shows that 17.79\% of the negative pairs are FN pairs which have the same semantics. Although the existing DCRL methods utilize attention \cite{wang2022exploring} or clustering \cite{li2021dense} to avoid directly dividing adjacent pixel-wise features as negative pairs, the FN caused by “semantic continuity” is still an open and challenging problem. b) For the FN pairs caused by the “Semantic overlap”, we follow the DenseCL \cite{wang2022densecl} which pairs the current features and the memory bank features as the negative pairs. We make the features of image B in the foreground as the memory bank features and the features of image A in the foreground as the current features. Then, we pair the current and memory bank features as negative pairs and calculate the accuracy. Finally, 17.53\% of the negative pairs are FN pairs which have the same semantics. The “Semantic overlap” property of the medical images makes it inevitable that there will be numerous consistent semantic regions between medical images. Therefore, it will produce 17.53\% FN pairs in the training process making the model learn in an unreliable direction.


\begin{figure}[tb]
  \centering
  \includegraphics[width=\linewidth]{./picture/fitting.pdf}
  \caption{The FP and FN pairs have a serious impact on learning. a) The fitting process with FP and FN pairs on a cardiac CT image. b) The models' learned segmentation ability on the fitted case and their generalization ability on another testing case.}\label{supp:fig:fitting}
\end{figure}

According to the above probability of FP and FN pairs, we simulated the number of these FP and FN pairs in a supervised heart segmentation learning task. Specifically, we train a U-Net on the cardiac structures segmentation task with a cardiac CT image (Image A in Fig.\ref{supp:fig:fp}) to evaluate the fitting ability of the model with or without FP\&N pairs. a) In the non-FP\&N pairs setting, we utilize the contrastive segmentation learning like Wang et al. \cite{wang2019panet}. b) In the FP\&N pairs setting, we randomly generate FP (94\%) and FN (17\%) pairs in the contrastive segmentation learning. c) We further reduce the probabilities of FP and FN pairs to one-third of original (31\% and 6\%) to give an ablation of the false pairs’ degree. We take 1000 iterations, and draw the loss values of the learning process on a line chart to visualize the fitting process. We also evaluate the segmentation of the fitted case and another testing case (Image B in Fig.\ref{supp:fig:fp}) to evaluate the model learned representation with false pairs.

As shown in Fig.\ref{supp:fig:fitting}, the FP and FN pairs have a serious impact on learning. Without the FP\&N pairs, the model is able to be fitted to the target cardiac images, and learn the representation ability of the semantic regions. However, when learning with large-scale FP\&N pairs (94\%, 17\%), the model is unable to be fitted to the targets owing to the interference of the noisy optimization targets. When reducing the FP\&N degree to one-third, the model is able to be gradually fitted to the target image and has a certain generalization, but its performance is weaker than the ``no FP\&N" situation. Therefore, we can draw the following two conclusions in DCRL: a) the large-scale FP\&N problem will make the model unable to learn representation; b) alleviating the FP\&N degree, the model will be able to learn the representation ability of data with generalization ability. Therefore, our GEMINI embeds the homeomorphism prior to the DCRL for the large-scale FP\&N problem, enhancing the learning of true feature pairs. Although it is challenging to remove FP\&N pairs without annotation, reducing the FP\&N degree via our GEMINI is still able to guides the model to learn a generalizable representation.

\subsection*{B.2 Discussion of the novelty in GEMINI}
The proposed GEMINI is a novel dense contrastive representation learning paradigm in medical image analysis. Not only in the innovations, i.e., our DHL and GSS, it also achieved great novelty in principle.

\emph{In principle}, our GEMINI has advanced the theoretical foundation of homeomorphism for the dense contrastive representation learning, providing a principle inspiration to the community. It modeled the human consistent anatomy in medical images based on the principle of topologie \cite{hubbard1991differential}, proposed a new principal concept, homeomorphism prior, and formulated it in the DCRL task as a new paradigm. Therefore, the community will be further inspired by our principle of homeomorphism and make new scientific and technological progress in other tasks and fields.

\emph{In methodology}, our work has proposed a novel dense contrastive representation learning framework that enables the contrast of feature pairs under the condition of human inherent topology, thus promoting the DCRL in medical images. It modeled the consistency of human inherent topology (i.e., homeomorphism prior) as a learning for deformable mapping to overcome the reliability issue in DCRL’s feature correspondence process, giving one potential answer to the long-standing question of “how to achieve a reliable dense feature correspondence for unlabeled data?” Based on the modeling, the proposed DHL and GSS bring soft learning of feature pairs and reliable learning of positive pairs, promoting the contrast of features in DCRL. Finally, our work has achieved a new ability to learn reliable semi-supervised medical image segmentation and pre-training models.

\section*{C More framework analysis and experiment discussion}
\subsection*{C.1 Discussion of the receptive field $r$ in the Deformer network}
\begin{figure}[tb]
  \centering
  \includegraphics[width=\linewidth]{./picture/receptfield.pdf}
  \caption{The ablation study of the receptive field size $r$ and the network parameter amounts. a) The segmentation performance on the T1 of FS-Semi setting with the increasing of the receptive field size $r$. b) The fine-tuning performance with the enlarging of the parameter amount (million, $M$) in the pre-trained networks.}\label{supp:fig:rece}
\end{figure}

The performance is robust for the receptive field $r$. As shown in Fig.\ref{supp:fig:rece} a), we enlarge the receptive field $r$ via adding the depth and down-sampling stages of the “Deformer” network and evaluate the model's performance on T1 of FS-Semi setting. With the enlarging of the receptive field, the models’ performance is stalely around 90\% DSC. Because the backbone network and “Deformer” network together constitute a whole network to learn the feature representation, and the features from the backbone network have been extracted from a large receptive field. Therefore, even the receptive field of the “Deformer” network is small, the final DVF is still calculated from a large receptive field. The layers inner the backbone is still optimized by the gradient with a big reception, so that our GEMINI keeps stable performance with the enlarging of $r$. Owing to the soft learning of feature pairs in our DHL, once added this module, the framework achieves more than 5\% DSC improvement.

\subsection*{C.2 Discussion of the parameter amount}
As shown in Fig.\ref{supp:fig:rece} b), we have evaluated our GEMINI on different settings of model parameters. We pre-trained our GEMINI-MIP on the ChestX-ray8 dataset for the networks with 0.12$M$, 0.49$M$, 1.97$M$, 7.85$M$, 31.39$M$, and 125.52$M$ ($M$ is million) parameters, and fine-tuned them on the T1: SCR25 task. With the enlarging of the network, the model performance is improving quickly. This is because the network capacity increases with the enlarging of the networks so that it will be able to learn the representation of more features in the pre-training process. When the parameter amount is 1.97M, the speed of performance improving is reduced, illustrating that the increase of network capacity has approached the upper bound of this task. Therefore, when the network is further enlarged to 125.52$M$ (more than 50 times compared with 1.97$M$), the performance is only improved 1.4\% DSC. %Considering the cost of model training, we all utilize the networks with 1.97$M$ parameters as the backbone network for all experiments.

\subsection*{C.3 Discussion of the feature distribution}
\begin{figure}[tb]
  \centering
  \includegraphics[width=\linewidth]{./picture/points.pdf}
  \caption{The t-SNE visualization of the learned pixel representations. We provide the coordinates of pixels in a zoomed view, indicating their spatial relationship.}\label{supp:fig:distri}
\end{figure}

As shown in Fig.\ref{supp:fig:distri}, we visualize the learned representation by our framework to demonstrate its effectiveness in distinguishing different semantic regions. In the three tasks of the SS-MIP experiment, we randomly select the slices or patches from the test datasets and extract their pixel-wise features via the backbone network initialized from scratch (a) and our GEMINI-MIP (b). Then, these features are zoomed by t-SNE \cite{van2008visualizing} to two dimensions. As demonstrated in the enlarged region, the features from the ``Scratch" model is mixed owing to its initial weak representation. The pixel-wise features from our framework are clustered into several meaningful groups. Most of the pixels in each group are spatially close and in different groups are also spatially separated (indicated by their coordinates (c)) in the original image. Because our GEMINI discovers the correspondence of pixel-wise features based on the homeomorphism of human body and learns the representation according to the consistent context topology, the same semantic features which are spatially close will be clustered.

\subsection*{C.4 Discussion of the cross-architecture compatibility}
\begin{table*}[tb]
\centering
\caption{The FS-Semi evaluations on U-Net \cite{ronneberger2015u}, TransUNet \cite{chen2021transunet}, and SwinUNet \cite{cao2022swin} demonstrate the cross-architecture compatibility of our GEMINI. The ``-" means that the setting is unable to be implemented.}
%\resizebox{\textwidth}{!}
{
\begin{tabular}{clccccccccccccccc}
  \toprule
  \multirow{2}{*}{\textbf{Type}}
  &\multirow{2}{*}{\textbf{Method}}
  &\multicolumn{2}{c}{\textbf{T1: 3D cardiac structures}}
  &\multicolumn{2}{c}{\textbf{T2: 3D brain tissues}}
  &\multicolumn{2}{c}{\textbf{T3: 2D chest structures}}
  &\textbf{AVG}\\ \cmidrule(r){3-4}\cmidrule(r){5-6}\cmidrule(r){7-8}\cmidrule(r){9-9}
  &
  &DSC$_{\pm std}\uparrow$
  &AVD$_{\pm std}\downarrow$
  &DSC$_{\pm std}\uparrow$
  &AVD$_{\pm std}\downarrow$
  &DSC$_{\pm std}\uparrow$
  &AVD$_{\pm std}\downarrow$
  &DSC$_{\pm std}\uparrow$
  \\
  \midrule
  \textbf{Five}
  &U-Net \cite{ronneberger2015u}
  &84.3$_{\pm9.6}$
  &2.43$_{\pm2.14}$
  &69.5$_{\pm8.8}$
  &1.59$_{\pm0.84}$
  &83.4$_{\pm6.9}$%{\color{purple}$^{*}$}
  &10.34$_{\pm4.80}$
  &79.1$_{\pm8.4}$
  \\
  \textbf{(Lower)}
  & TransUNet \cite{chen2021transunet}
  & 74.5$_{\pm8.3}$
  & 4.41$_{\pm1.39}$
  & 67.4$_{\pm5.4}$
  & 2.02$_{\pm0.46}$
  & 76.5$_{\pm8.2}$
  & 16.59$_{\pm6.53}$
  & 72.8$_{\pm7.3}$
  \\
  & SwinUNet \cite{cao2022swin}
  & 40.8$_{\pm8.0}$
  & 11.59$_{\pm1.32}$
  & 67.8$_{\pm5.3}$
  & 4.04$_{\pm0.39}$
  & 63.9$_{\pm11.5}$
  & 14.26$_{\pm8.91}$
  & 57.5$_{\pm8.3}$
  \\
  \cdashline{1-9}[0.8pt/2pt]
  \textbf{Full}
  &U-Net \cite{ronneberger2015u}
  &-
  &-
  &88.7$_{\pm1.2}$
  &0.31$_{\pm0.04}$
  &96.1$_{\pm1.4}$%{\color{purple}$^{*}$}
  &2.28$_{\pm1.00}$
  &-
  \\
   \textbf{(Upper)}
  & TransUNet \cite{chen2021transunet}
  & -
  & -
  & 85.7$_{\pm1.2}$
  & 0.43$_{\pm0.05}$
  & 95.2$_{\pm2.1}$
  & 2.78$_{\pm1.35}$
  & -
  \\
  & SwinUNet \cite{cao2022swin}
  & -
  & -
  & 82.8$_{\pm2.7}$
  & 0.54$_{\pm0.15}$
  & 95.3$_{\pm1.2}$
  & 2.17$_{\pm0.65}$
  & -
  \\
  \cdashline{1-9}[0.8pt/2pt]
  \textbf{Semi}
  &\textbf{GEMINI+U-Net}
  &91.2$_{\pm3.6}$
  &0.97$_{\pm0.56}$
  &87.3$_{\pm1.0}$
  &0.35$_{\pm0.03}$
  &87.7$_{\pm5.2}$
  &7.14$_{\pm3.63}$
  &88.7$_{\pm3.3}$
  \\
   \textbf{(Ours)}
  &\textbf{GEMINI+TransUNet}
  &90.8$_{\pm3.4}$
  &0.94$_{\pm0.51}$
  &84.4$_{\pm1.3}$
  &0.45$_{\pm0.05}$
  &88.4$_{\pm5.7}$
  &8.63$_{\pm4.68}$
  &87.9$_{\pm3.5}$
  \\
  &\textbf{GEMINI+SwinUNet}
  &88.6$_{\pm4.2}$
  &1.28$_{\pm0.64}$
  &79.9$_{\pm5.0}$
  &0.62$_{\pm0.20}$
  &86.2$_{\pm7.8}$
  &6.34$_{\pm4.34}$
  &84.9$_{\pm5.7}$
  \\
  \bottomrule
\end{tabular}
}
\label{tab:arch}
\end{table*}
As shown in the Tab.\ref{tab:arch}, we perform the TransUNet \cite{chen2021transunet} (CNN+transformer), SwinUNet \cite{cao2022swin} (Transformer), and U-Net \cite{ronneberger2015u} (CNN) on the FS-Semi tasks with three datasets, and our GEMINI demonstrates a great compatibility across these model architectures. There are two observations: \textbf{a)} Our GEMINI has achieved significant improvement on both TransUNet \cite{chen2021transunet}, SwinUNet \cite{cao2022swin}, and U-Net \cite{ronneberger2015u}. Compared with the lower bound of the architectures that are trained only with five labeled images, our GEMINI has improved them more than 9\% DSC on AVG owing to our learning of the  homeomorphism mapping between medical images. Especially, for the 3D brain tissues segmentation, GEMINI achieves a similar performance compared with the FULL setting (83 labels) only with 5 labels in all architectures, demonstrating our great potential in reducing of annotation costs. \textbf{b)} Our GEMINI has great architecture compatibility across CNN-based (U-Net [3]), transformer-based (SwinUNet [2]), and CNN-transformer-based (TransUNet) networks. For U-Net and TransUNet that utilizes CNN to encode and decode features, our GEMINI has similar significant improvement that achieves 88.7\% and 87.9\% on AVG DSC. SwinUNet takes patch-embedding and four-times down sampling at the beginning, and utilizes the shifted window to learn global features. Therefore, it is challenging to represent fine-grained dense features and makes the whole network easy to overfit to the global features when the amount of training cases is small. As a result, SwinUNet has very poor performance on “FIVE” setting. When adding GEMINI, it learns the inter-image consistency for unlabeled images and effectively reduces the over-fitting, thus also achieving more than 20\% DSC improvement.

\subsection*{C.5 Discussion of the computing costs}
\begin{table}[tb]
\centering
\caption{Owing to the additional deformer networks, our GEMINI has relatively higher computing costs in the pre-training stage, but it has same computing costs in the fine-tuning for downstream tasks as other methods and achieves much higher performance.}
\resizebox{\linewidth}{!}
{
\begin{tabular}{clccccccccccccccc}
  \toprule
  \multirow{2}{*}{\textbf{Type}}
  &\multirow{2}{*}{\textbf{Method}}
  &\textbf{Pre-training}
  &\textbf{Downstream}
  &\textbf{T1: SCR$_{25}$}
  \\
  \cmidrule(r){3-3}\cmidrule(r){4-4}\cmidrule(r){5-5}
  &
  & FLOPs
  & FLOPs
  & DSC$_{\pm std}$
  \\
  \midrule
  1$\times$Encoder
  &Rotation \cite{komodakis2018unsupervised}
  &5.99G
  &20.15G
  &80.5$_{\pm7.7}$
  \\
  2$\times$Encoder
  &BYOL \cite{grill2020bootstrap}
  &11.98G
  &20.15G
  &89.4$_{\pm4.9}$
  \\
  1$\times$Encoder-decoder
  &Model Genesis \cite{zhou2019models}
  &19.74G
  &20.15G
  &86.1$_{\pm4.6}$
  \\
  2$\times$Encoder-decoder
  &VADeR \cite{o2020unsupervised}
  &39.67G
  &20.15G
  &85.2$_{\pm5.1}$
  \\
  \cdashline{1-5}[0.8pt/2pt]
  2$\times$Encoder-decoder
  &\textbf{Our GEMINI}
  &\textbf{52.59G}
  &20.15G
  &\textbf{92.1$_{\pm2.8}$}
  \\
  \bottomrule
\end{tabular}
}
\label{tab:computing}
\end{table}
As shown in Tab.\ref{tab:computing}, we compare the methods’ number of floating-point operations (FLOPs) in four architecture types in pre-training stage, i.e., ``1$\times$Encoder" that only runs an encoder in the pre-training, here, we take the Rotation method \cite{komodakis2018unsupervised}; ``2$\times$Encoder" that runs two encoders for contrastive representation learning in the pre-training, here we take the BYOL \cite{grill2020bootstrap}; ``1$\times$Encoder-decoder" that runs an encoder-decoder network, like the U-Net, in the pre-training, here we take the Model Genesis \cite{zhou2019models}; ``2$\times$Encoder-decoder" that runs two encoder-decoder networks for dense contrastive representation learning in the pre-training, here we take the VADeR \cite{o2020unsupervised}. Our GEMINI is also a ``2$\times$Encoder-decoder" method. In the pre-training stage, all methods take U-Net (for Encoder-decoder) or the encoder part of the U-Net (for Encoder) as their backbone. In the downstream adaptation stage, all methods’ pre-trained parameters are used to initialize the U-Net to learn segmentation task (T1: SCR$_{25}$) and the part without pre-training is initialized randomly. All methods utilize same input sizes with [300$\times$300] in pre-training stage and [512$\times$512] in downstream stage. Owing to two additional deformer networks to learning the homeomorphism mapping, our GEMINI has the highest FLOPs in the pre-training stage, but it greatly contributes to the pre-training performance. In the downstream stages, owing to all methods take the U-Net with same parameter amount, our GEMINI has same FLOPs as other methods. As a results, our GEMINI has very significant performance improvement on the SCR$_{25}$ task owing our reliable learning for positive and negative feature pairs. The large-scale FP\&N problem in VADeR makes it has worse performance than the BYOL even it has larger FLOPs in pre-training.

The additional the one-time cost of our GEMINI in the pre-training stage bring obtain better representation, effectively reducing the long-time computing costs in the downstream tasks. Because once the pre-training is completed, stronger representation will accelerate the convergence speed of the model on downstream tasks, thus reducing the long-time computing cost in the training of numerous downstream tasks. As illustrated in the Fig.12 of our paper, compared with the BYOL \cite{grill2020bootstrap}, DenseCL \cite{wang2022densecl}, Model Genesis \cite{zhou2019models}, our GEMINI achieved better performance with fewer iterations, illustrating its potential in reducing the computing costs in downstream tasks.

\subsection*{C.6 Discussion of the second-best models}
Compared with the second-best methods (BRBS \cite{he2022learning} in Tab.2 and our CVPR version, GVSL \cite{He_2023_CVPR}, in Tab.3), we can find these methods also fused the homeomorphism prior into their framework, and their great performance demonstrates the great potential of this prior knowledge in medical images. In our Experiment 1: FS-Semi (Sec.4), the BRBS is a “learning registration to learn segmentation” method whose registration part is based on the homeomorphism prior. Therefore, its powerful performance in the T1: 3D cardiac structures and T2: 3D brain tissues illustrate the advantages. However, the BRBS’s visual similarity make it unable to generalize to the chest X-ray (T3: 2D chest structures) that has relatively low contrast as illustrated in Fig.8. Our GEMINI utilize the semantic similarity based on features and achieves significant improvement on this task, demonstrating our superiority. In our Experiment 2: SS-MIP (Sec.4), our CVPR version, GVSL, benefits from our homeomorphism prior, achieving second-best performance on the T2: KiPA22 and T3: CANDI. However, it also utilizes the visual similarity which is limited on the low-contrast images, i.e., the chest x-ray images in the pre-training dataset. Therefore, its pre-trained representation for chest x-ray is relatively weaker and limits its performance in the inner-scene transferring. Our GSS improves the measurement of the correspondence degree, and drive the representation learning for low-contrast targets during pre-training. Therefore, our GEMINI has significantly improved the GVSL’s performance on the T1: SCR$_{25}$ task.

\subsection*{C.7 Discussion of the reliability}
\begin{table*}[tb]
\centering
\caption{The evaluation of our GEMINI's reliability on the tasks in Experiment 1. The \emph{Cor} is the Pearson correlation coefficient \cite{cohen2009pearson}, and the \emph{p} is the p-value.}
%\resizebox{\linewidth}{!}
{
\begin{tabular}{lccccccccccccccc}
  \toprule
  \diagbox{Evaluations}{Tasks}
  &\multicolumn{2}{c}{\textbf{T1: 3D cardiac structures}}
  &\multicolumn{2}{c}{\textbf{T2: 3D brain tissues}}
  &\multicolumn{2}{c}{\textbf{T3: 2D chest structures}}
  &\multicolumn{2}{c}{\textbf{AVG}}
  \\
  \midrule
  \multirow{2}{*}{\textbf{a) Reliability across samples}}
  &\textbf{DSC} $\uparrow$
  &\textbf{std} $\downarrow$
  &\textbf{DSC} $\uparrow$
  &\textbf{std} $\downarrow$
  &\textbf{DSC} $\uparrow$
  &\textbf{std} $\downarrow$
  &\textbf{DSC} $\uparrow$
  &\textbf{std} $\downarrow$
  \\
  &91.2
  &3.6
  &87.3
  &1.0
  &87.7
  &5.2
  &88.7
  &3.3
  \\
  \midrule
  \multirow{2}{*}{\textbf{b) Reliability across training}}
  &\textbf{\emph{Cor}} $\uparrow$
  &\textbf{\emph{p}} $\downarrow$
  &\textbf{\emph{Cor}} $\uparrow$
  &\textbf{\emph{p}} $\downarrow$
  &\textbf{\emph{Cor}} $\uparrow$
  &\textbf{\emph{p}} $\downarrow$
  &\textbf{\emph{Cor}} $\uparrow$
  &\textbf{\emph{p}} $\downarrow$
  \\
  &0.989
  & $<$0.001
  &0.999
  & $<$0.001
  &0.968
  & $<$0.001
  &0.985
  & $<$0.001
  \\
  \bottomrule
\end{tabular}
}
\label{tab:reliability}
\end{table*}
As shown in Tab.\ref{tab:reliability}, in the three tasks of our Experiment 1, we calculated the standard deviations (std) and the inter-training Pearson correlation coefficients (Cor) \cite{cohen2009pearson}. The results indicate that our GEMINI demonstrates strong reliability across different tested samples and training initializations. \emph{a) Reliability across samples:} We evaluated the DSC and std of the performance across the tested samples. Our GEMINI-Semi achieved an average of 88.7\% DSC with a 3.3 std, indicating high performance with robustness across diverse samples, which supports its reliability in real-world applications. \emph{b) Reliability across training:} We conducted a test-retest reliability analysis \cite{guttman1945basis} and reported the Cor for the performance when our GEMINI-Semi was trained twice from different initialization states. The Cors for all three tasks exceeded 0.95 demonstrating very high consistency between the two training sessions. Additionally, all p-values were below 0.001, indicating significant consistency. Thus, our model shows excellent reliability across different initialization states, which supports its reliability in model implementation.

\section*{D More details in experiments}

\subsection*{D.1 Details of the training diagram}
\begin{figure}[tb]
  \centering
  \includegraphics[width=0.9\linewidth]{./picture/training.pdf}
  \caption{The overall training diagram of our GEMINI. a) The inference process of the whole framework. The gray path in the last line is the additional learning part in the variants of our GEMINI in self-supervised pre-training (GEMINI-MIP) and semi-supervised segmentation (GEMINI-Semi). b) The loss calculation to optimize the whole framework.}\label{supp:fig:diagram}
\end{figure}
\begin{figure*}[!]
  \centering
  \includegraphics[width=\linewidth]{./picture/details_framework.pdf}
  \caption{The detailed architecture of our GEMINI. a) The backbone architecture utilizes the 3D U-Net in 3D image tasks and 2D U-Net in 2D image tasks. b) The deformer network architecture utilized a lightweight U-Net. c-d) The segmentation head in the variant of GEMINI-Semi and the self-restoration head in the variant of GEMINI-MIP.}\label{supp:fig:architecture}
\end{figure*}
As shown in Fig.\ref{supp:fig:diagram}, the training diagram introduces the details of our GEMINI's variants in SSP and Semi experiments. In the forward inference, as described in the ``Methodology" section of our paper, two images $x^{A}, x^{B}$ are put into two shared-weight backbones $\mathcal{N}_{\theta}$ separately to extract the features $f^A, f^B$. The features are further put into two shared-weight deformers together to predict two DVFs $\psi^{AB}, \psi^{BA}$ that are bidirectional. For the variant of GEMINI-Semi, a labeled image $x^C$ is put into the shared-weight backbone $\mathcal{N}_{\theta}$, and then put into an additional segmentation head $Seg_{\kappa}$ to predict the segmentation results $\hat{y}^{C}_{Seg}$. For the variant of GEMINI-MIP, an appearance transformed image $x^C$ (described in Sec.5.1.1) is put into the shared-weight backbones $\mathcal{N}_{\theta}$, and then put into an additional self-restoration head $Res_{\tau}$ to predict the restored image $\hat{y}^{C}_{Res}$. In the loss calculation, the smooth loss (Equ.3) is calculated on the DVFs $\psi^{AB}, \psi^{BA}$ to learn the continuity of the deformable mapping, the GVS loss ($\mathcal{L}_{GVS}$, Equ.6) and GSS loss ($\mathcal{L}_{GSS}$, Equ.7) are calculated on the deformed images $x^{AB}=\psi^{AB}(x^A), x^{BA}=\psi^{BA}(x^B)$ and deformed features $f^{AB}=\psi^{AB}(f^A), f^{BA}=\psi^{BA}(f^B)$ (described in Sec.3.3) to learn the correspondence. For the variant of GEMINI-Semi, a segmentation loss $\mathcal{L}_{Seg}$ is calculated on segmentation result $\hat{y}^{C}_{Seg}$ and the groundtruth $y^{C}_{Seg}$. For the variant of GEMINI-MIP, a self-restoration loss $\mathcal{L}_{Res}$ is calculated on the restored image $\hat{y}^{C}_{Res}$ and the original image $y^{C}_{Res}$. The learning of self-restoration in SSP is a fundamental task for a warm-up of our GSS, due to the initial weak representation in the pretext task.

\subsection*{D.2 Details of the architectures and implementation}
As shown in Fig.\ref{supp:fig:architecture}, we utilize the U-Net \cite{ronneberger2015u} architecture (3D U-Net for 3D images and 2D U-Net for 2D images) as our backbone architecture for great basic dense representation in our experiment. In the encoding path, it takes max pooling layers to reduce the feature maps' resolution and in the decoding path, it takes up-sampling layers (bilinear for 2D images and trilinear for 3D images) to restore the features' resolution. Skip connections are used to transmit features from the encoding path to the decoding path in each resolution stage. There are five resolution stages in the network and each stage utilizes Conv-GN-LeckyReLU\footnote{Conv is a convolution layer and GN is a group normalization layer \cite{wu2018group}.} modules to extract features. The deformer network also takes a lightweight U-Net architecture with very shallow depth to estimate the DVF. It only has three resolution stages and each stage has half of the Conv-GN-LeckyReLU module amount compared with the backbone. Both the segmentation head and self-restoration head take one Conv-GN-LeckyReLU module to project the input features and follow a convolution layer to predict the targets. The detailed hyper-parameters inner these architectures are marked in Fig.\ref{supp:fig:architecture}.






\bibliographystyle{IEEEtran}
\bibliography{mybib}
%\begin{IEEEbiography}[{\includegraphics[width=0.8in,height=1in,clip,keepaspectratio]{bio_images/xinweiliu.jpg}}]
{Xinwei Liu} 
is a Ph.D. student in the Institute of Information Engineering, Chinese Academy of Sciences and the School of Cyber Security, University of Chinese Academy of Sciences, Beijing. His research interests include computer vision, deep learning and adversarial machine learning.
\end{IEEEbiography}

% \begin{IEEEbiography}
% [{\includegraphics[width=0.8in,height=1in,clip,keepaspectratio]{bio_images/Siyuan Liang}}]
% {Siyuan Liang} 
% is 
% \end{IEEEbiography}

\begin{IEEEbiography}[{\includegraphics[width=0.8in,height=1in,clip,keepaspectratio]{bio_images/xiaojunjia.jpg}}]
{Xiaojun Jia} 
received his Ph.D. degree in  State Key Laboratory of Information Security, Institute
of Information Engineering, Chinese Academy of Sciences and School of
Cyber Security, University of Chinese Academy of Sciences, Beijing. He is now a Research Fellow in Cyber Security Research Centre @ NTU, Nanyang Technological University, Singapore. His research interests include computer vision, deep learning and adversarial machine learning.
\end{IEEEbiography}

\begin{IEEEbiography}
[{\includegraphics[width=0.8in,height=1in,clip,keepaspectratio]{bio_images/yuanxun.jpg}}]
{Yuan Xun} 
is a Ph.D. student in the Institute of Information Engineering, Chinese Academy of Sciences and the School of Cyber Security, University of Chinese Academy of Sciences, Beijing. Her research interests include computer vision, deep learning and adversarial machine learning.
\end{IEEEbiography}



\begin{IEEEbiography}[{\includegraphics[width=0.8in,height=1in,clip,keepaspectratio]{bio_images/zhanghua.jpg}}]{Hua Zhang} is a professor with the Institute of Information Engineering, Chinese Academy of Sciences. He received the Ph.D. degree in computer science from the School of Computer Science and Technology, Tianjin University, Tianjin, China in 2015. His research interests include computer vision, multimedia, and machine learning.
\end{IEEEbiography}



\begin{IEEEbiography}
[{\includegraphics[width=0.8in,height=1in,clip,keepaspectratio]{bio_images/xiaochuncao.jpg}}]
{Xiaochun Cao}(SM'14)
received the B.S. and M.S. degrees in computer science from Beihang University, Beijing, China, and the Ph.D. degree in computer science from the University of Central Florida, Orlando, FL, USA. After graduation, he spent about three years at ObjectVideo Inc. as a Research Scientist. He is with the School of Cyber Science and Technology, Shenzhen Campus, Sun Yat-sen University, Shenzhen 518107, P.R. China. He has authored and co-authored more than 100 journal and conference papers.
Prof. Cao is a Fellow of the IET. He is on the Editorial Boards of the IEEE Transactions on Image Processing, IEEE Transactions on Multimedia, IEEE Transactions on Circuits and Systems for Video Technology. His dissertation was nominated for the University of Central Florida's university-level Outstanding Dissertation Award. In 2004 and 2010, he was the recipient of the Piero Zamperoni Best Student Paper Award at the International Conference on Pattern Recognition.
\end{IEEEbiography}


\end{document}


