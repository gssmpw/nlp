\documentclass[tightpage]{standalone}

% import style library
\usepackage{myStyleIEEE}
%--- TikZ shapes and definitions ---%
\usepackage{myPlotStyle}


\pgfdeclarelayer{bg}    % declare background layer
\pgfsetlayers{bg,main}  % set the order of the layers (main is the standard layer)
\usetikzlibrary{intersections}

\begin{document}

\begin{tikzpicture}[node distance = 3cm]
\sffamily
\footnotesize

% Open Loop
\node[block](NN) {\begin{tabular}{c} PIML Algorithm \\ PINN or PI-KAN \\ [0.5em]
\scalebox{0.1}{\includegraphics[]{Plots/NNs/NeuralNetwork.png}}\end{tabular}};

\node[block, left of = NN, text width = 2cm, align = center, yshift = 0.8cm, xshift = -1cm](x_in){State at $t_n$ \\ $\mathbf{u_n} = [\theta_n, \omega_n]$};
\node[block, below of = x_in, text width = 2cm, minimum width = 0.5cm, yshift = 0.5cm, align = center](t_in){Final time of $n$-th step \\ $t_{n+1}$};

\node[block, right of = NN, text width = 3cm, align = center, xshift = 1.5cm](x_out){State at time $t_{n+1}$ \\
$\mathbf{u}_{n+1} = [\theta_{n+1}, \omega_{n+1}]$};

\draw[-latex] (x_in.east) --(x_in.east-|NN.west);
\draw[-latex] (t_in.east) --(t_in.east-|NN.west);
\draw[-latex] (NN.east) --(x_out);

\draw[-latex] (x_out.north) -- +(0,1) -| (x_in.north);

% % background
\begin{pgfonlayer}{bg}

    \path (x_in) -- +(-1.25,0.8) coordinate (r1);
    \path (x_out) -- +(1.75,-1.5) coordinate (r2);
    \draw[sBlue, dashed,fill = sBlue!10] (r1) rectangle(r2);
    % \node [left of = r2, anchor = south east, node distance = 0.1, sBlue, yshift = 0cm] {\scriptsize \textbf{Recursively repeated to reconstruct entire trajectory} };
        
\end{pgfonlayer}

\end{tikzpicture}
\end{document}