\section{Discussion}

Regarding the research question raised in Section~\ref{sec:methodology}, we have systematically examined \textbf{RQ1-3} in the preceding sections, and the key findings are summarized as follows:

\begin{itemize}
    \item \textbf{RQ1:} We have shown the distribution of smart ring papers in terms of publication years (Figure~\ref{fig:method_trend}) and proposed applications (Figure~\ref{fig:taxo_main}). Furthermore, we explore the distribution of papers across various subsets of applications, including gesture types (Table~\ref{tab:gesture_distribution}), commercial rings used (Table~\ref{tab:in-body_commercial}), and methods for establishing ground truth (Table~\ref{tab:health_gt_description}).
    
    \item \textbf{RQ2:} Our review demonstrates that approximately 87\% of the articles in our set of literature require a sensing solution for their proposed applications. We have comprehensively summarized the different sensor configurations in Table~\ref{tab:taxo_sensor} and described the phenomena that these sensors can detect in Table~\ref{tab:taxo_phenomena}.
    \item \textbf{RQ3:} To better understand the applications enabled by the identified sensing methods, we developed a four-layer taxonomy structured as ``$Application \rightarrow Phenomena \rightarrow Fundamental\,Phenomena (FPs) \rightarrow Sensors$'' as described in Section~\ref{sec:taxonomy}. We categorized applications into four primary groups using a ``binary decision tree'' approach: interaction - input, interaction - output, passive sensing - in-body feature, and passive sensing - out-body feature, as detailed in Section~\ref{sec:taxo_application}. Each study's focused application is further detailed in four corresponding sections.
\end{itemize}

We will now discuss the challenges and future directions for \textbf{RQ4} in this section.

% This section answers the research questions, and proposes the challenges, and future directions. 

\textit{Bridging the Gap: From Smart Ring Research to Real-World Adoption.}
Wearable devices have seen significant commercial success in recent years, particularly smartwatches, which have become mainstream in the consumer electronics market~\cite{Pangarkar2025smartwatch}. While smart rings have gained many traction, their adoption remains limited compared to smartwatches, despite their promising potential. This section explores the gaps between current smart ring research and their commercial viability, focusing on our categories of applications, along with associated challenges and future directions.

\subsection{Interaction - Input}
The unique placement of smart rings enables precise gesture-based input by detecting subtle finger movements, expanding the design space for gesture recognition. This allows for quick, intuitive interactions, such as controlling smart home devices~\cite{liu2023understanding}. Gesture input is especially useful in scenarios where traditional methods fail, like cooking with wet hands~\cite{han2016exploring} or holding objects~\cite{wolf2016microgesture}. However, seamless integration with other devices is essential, requiring a robust smart device ecosystem to fully unlock the potential of smart ring gestures, which limits the potential of smart ring gestures in daily activities.

Another promising application of smart ring inputs is their ability to mimic traditional input devices, such as keyboards (Section~\ref{sec:input_text_keyboard}), mice (Section~\ref{sec:input_traj_2d}), and VR controllers (Section~\ref{sec:input_traj_3d}). The smart ring's key advantage lies in constant availability and portability. However, further improvements are needed to enhance their efficiency and accuracy to match conventional input devices. For example, the most advanced smart ring virtual keyboard~\cite{gu2020qwertyring} currently achieves 20.35 words per minute (WPM), still lower than a standard computer keyboard. Additionally, existing smart ring input techniques impose certain operational constraints. Virtual keyboards often require fixed wrist or finger positions, such as QwertyRing~\cite{gu2020qwertyring} and RingVKB~\cite{li2023ringvkb}. Similarly, trajectory-based input methods limit motion space to improve precision, as seen in SoundTrak~\cite{zhang2017soundtrak} and MouseRing~\cite{shen2024mousering}. To ensure comfort and an always-available experience, factors such as ring size, the number of sensors, and power consumption are limited, which can be key reasons why some research cannot be practically implemented. Future advancements in adaptive algorithms and sensing techniques are crucial to overcoming these challenges, enabling more flexible, natural, and efficient smart ring inputs.

%The unique placement of smart rings on the finger makes them highly effective for gesture-based inputs, as they can detect subtle finger movements. This capability significantly broadens the design space for gesture recognition, allowing for more intricate and varied gesture commands. Such gestures offer a viable alternative for quick interaction demand, such as controlling smart home furniture~\cite{liu2023understanding}. Additionally, gesture inputs are particularly advantageous in situations where conventional input methods are impractical, including scenarios with wet hands during cooking~\cite{han2016exploring}, or when hands are occupied holding other objects~\cite{wolf2016microgesture}. However, these use cases depend heavily on the integration with other devices that accept gesture inputs, requiring a robust and seamless network connection between smart devices and smart rings. This interconnected ecosystem is crucial for fully leveraging the potential of smart ring gestures in daily activities.

%Another popular application for smart ring inputs is to mimic traditional input devices, including keyboards (Section~\ref{sec:input_text_keyboard}), mice (Section~\ref{sec:input_traj_2d}), and VR controllers (Section~\ref{sec:input_traj_3d}). The primary advantage of a smart ring over traditional input devices is its constant availability and portability. However, the efficiency and accuracy of smart ring inputs generally do not match those of conventional devices. For instance, the most advanced smart ring virtual keyboard technology to date~\cite{gu2020qwertyring} achieves  20.35 words per minute (WPM), which is significantly lower than that of a standard computer keyboard. Another common limitation is the operational constraints imposed on users when employing these input techniques. Smart ring keyboard usually restricts user's input poses, such as QwertyRing~\cite{gu2020qwertyring} requires users to maintain a fixed wrist position during typing, while the RingVKB~\cite{li2023ringvkb} necessitates keeping the finger's angle constant. Meanwhile, trajectory-based applications often limit the user's motion space to enhance accuracy. For instance, SoundTrak~\cite{zhang2017soundtrak} confines the user's input space to a $20\times16\times11(cm)$ box, and MouseRing~\cite{shen2024mousering} restricts finger movements to within a 5 cm diameter circle. These constraints highlight the gap in achieving the ease and flexibility of traditional input methods, underscoring the need for advancements in algorithms that could bridge this divide and enhance user experience.

\subsection{Interaction - Output}

The majority of smart ring interaction outputs focus on haptic feedback (26 out of 29), as discussed in Section~\ref{sec:output_haptic}. One key challenge in smart ring haptics is integrating actuators into a compact form factor, particularly for skin-dragging~\cite{je2017designing, je2017tactoring} and poke-based haptics~\cite{je2018pokering}. Future advancements in miniaturization and actuation technologies could enhance the effectiveness and comfort of haptic feedback in smart rings. 

Additionally, there is limited comparative research on haptic feedback across wearable devices, such as smart rings and smartwatches, particularly in the context of notifications. Similarly, no studies have explored how smart rings compare to haptic gloves in VR/AR scenarios where portability is less critical. Future research could address these gaps by systematically evaluating the unique advantages of haptic feedback in smart rings. 

Another promising yet underexplored direction is leveraging smart rings to augment human perception, as discussed in Section~\ref{sec:output_enhance}. By integrating high-precision sensors and providing real-time perceptual feedback, smart rings could assist users in executing fine motor tasks~\cite{obushi2019magnifinger} and improving task efficiency~\cite{wang2012portable}.

%Beyond traditional interaction, smart rings hold potential for enhancing emotional communication due to their symbolic significance in personal interactions. 


%The majority of smart ring interaction outputs are focused on haptic feedback (26 out of 29), as discussed in Section~\ref{sec:output_haptic}. A significant limitation of smart ring haptics is the size of the actuators, which are challenging to incorporate into an appropriately sized smart ring, particularly for skin-dragging haptic~\cite{je2017designing, je2017tactoring} and poke haptic~\cite{je2018pokering}. 
%Furthermore, there is a lack of research comparing the haptic feedback between smart rings and other wearable devices, such as smartwatches, specifically regarding notifications. In the realm of perception, no studies have yet compared smart rings with haptic gloves, which are frequently used in VR/AR scenarios where device portability is not a primary concern. Therefore, there is a need to address these gaps in comparative studies and rethink the advantages of haptic feedback in smart rings. In contrast, within the communication domain, smart rings hold a unique symbolic significance that potentially enhances emotional communication between individuals more effectively than other wearable devices.
% 没有工作进行戒指触觉反馈和手表反馈的对比(notification),在perception方面,没有工作进行戒指和手套的对比,而perception多用于VR/AR等场景,对设备的portability要求并不是很高。因此,一方面需要补齐这类对比工作的缺口,另一方面需要rethink戒指触觉反馈的优势。
% 对communication类,戒指的独特之处在于它的寓意,比起其他可穿戴设备更能增进人与人的感情交流。

%We would like to highlight an intriguing yet underexplored area of research within smart ring applications that involves using these devices to enhance human perception, as detailed in Section~\ref{sec:output_enhance}. Human hands are instrumental in executing precise movements~\cite{obushi2019magnifinger} and accomplishing complex tasks~\cite{wang2012portable}. By integrating high-precision sensors into smart rings, and subsequently presenting the gathered information to the user, these devices can amplify specific perceptual modalities or even aid users in completing tasks more efficiently.

\subsection{Passive Sensing}
\label{sec:discuss_open_platform}
Smart rings, known for their lightweight design, are widely used for health tracking, offering features such as physiological signal sensing, sleep monitoring, and fitness tracking. However, many passive sensing applications remain underexplored, particularly in disease diagnosis, treatment, authentication, and environmental sensing. These areas could greatly benefit from a smart ring platform that provides access to raw sensor data.

Currently, most smart rings offer physiological measurements like heart rate and SpO2 to developers, but these rely on proprietary inertial algorithms on the embedded system. The lack of access to raw sensor data and algorithmic details introduces potential biases, limiting researchers' ability to validate and refine measurements. Consequently, significant effort is required to ensure data reliability. Future advancements should focus on enabling controlled access to raw sensor data while maintaining privacy and security, fostering more robust and customizable health monitoring applications.

The research and development platforms for wearable technology play a crucial role in the effective utilization and integration of these devices in various applications. However, currently, there is a notable lack of robust, open platforms for the development of smart ring applications~\cite{zhou2023one}. On one hand, interaction rings offer unique possibilities for user interaction by interpreting gestures and motions as input commands. To harness these capabilities, it is essential that the development platform provides robust API support for integration with popular development environments. For instance, integrating interaction rings with Unity would allow developers to create more immersive and intuitive VR experiences. On the other hand, passive sensing rings are designed to collect data unobtrusively from the wearer's environment or physiological state. To fully exploit this capability, researchers need comprehensive APIs that allow for the extraction, processing, and interpretation of raw sensor data. This access allows developers to implement custom algorithms and applications that can provide more personalized and detailed insights into the data collected.

Further, future research should explore scalable deployment strategies to leverage smart rings in large-scale longitudinal studies~\cite{keeler2024biometrics}, addressing challenges such as data consistency, user adherence, and integration with existing healthcare systems. 

\subsection{Self-Containment}
Self-containment describes the ability to operate and compute without external supply. As rings are limited in size, they are incapable of incorporating heavy computing resources. Most previous work opted for external computational resources to function, while some research proposed alternatives for self-computing platforms \cite{magno2019self, gummeson2014energy}. However, as these platforms can be self-contained and support real-world applications, they are limited in sensor type and require energy harvesting methods, which restrain the usability of these prototypes. Generally speaking, higher self-containment requires trade-offs on sensing ability and operational time. To support in-the-wild usage of rings, the most common approach was to leverage wireless connection for data transition and use external devices for computation (e.g. laptops or smartphones). In these cases, rings are acting like a carrier of multiple sensors. As technology advances, smart rings could become more self-sufficient, featuring enhanced computational capabilities and extended battery life.


\subsection{Power Consumption and Battery Endurance.}
Due to the size and weight constraints of rings, making it difficult to accommodate large batteries, most ring prototypes in previous research relied on an external power supply. However, it is crucial to balance the ring's power consumption with battery endurance to enable extended wireless use in real-world conditions. Since sensor integration and cross-device communication are the primary contributors to power consumption, previous research \cite{boukhayma2021ring, chen2014mobiring, xu2022washring} has focused on the careful design of sensors (e.g. accelerator, low power PPG), reading circuits, and the choice of energy-efficient communication protocols (e.g. BLE). Moreover, multiple approaches were proposed to further reduce the power consumption of ring itself, including using low-power consumption sensors as triggers for functions, and passive reading \cite{takahashi2020telemetring, takahashi2024picoring}. On the other hand, battery endurance is another critical factor for achieving longer operational times. In addition to integrating larger batteries, various energy-harvesting methods have been proposed, such as using phones as wireless chargers or harnessing solar power \cite{nguyen2021smart, magno2019self, gummeson2014energy}. Given that rings are closely linked to users' touch interactions, there is potential to leverage wireless connectivity techniques to implicitly harvest power from nearby devices.

% \subsubsection{Algorithm Complexity and Computing Power on Ring}  % 算力

% \subsubsection{algorithm performance and user demand}
% 分application讨论
% 可以和别的wearable device比
% 用户需要的精度是什么
% cross-validation and participant number.


\subsection{User variability and culture}
% 可以分section讲
% 从手势角度,user preferred gesture 自定义手势,用户对同一个手势的理解不同 做出来的形态不同
% PPG skin color, electric field sensing skin moisture 
% activity recognition
In the development and evaluation of smart ring technologies, user variability and cultural differences play crucial roles in almost all applications summarized in this survey paper. The effectiveness of gesture recognition can vary significantly across users due to individual preferences~\cite{gheran2018gestures}, physical differences~\cite{altakrouri2016insights}, and cultural backgrounds~\cite{kwon2018cultural}. A gesture that is intuitive in one culture may be unfamiliar or even offensive in another, leading to inconsistencies in user interaction with the device. For physiological sensing technologies, photoplethysmography (PPG) for monitoring heart rate and blood oxygen levels can also be affected by user variability. Skin tone, for example, can influence the accuracy of PPG sensors, as darker skin may absorb more light, potentially leading to less accurate readings~\cite{scardulla2023photoplethysmograhic}. Activity recognition is another area where cultural differences are pronounced. Smart rings that aim to recognize and interpret specific activities must consider the diversity in how these activities are performed across different cultures. For example, eating habits vary widely: people might eat with their hands~\cite{moschetti2016recognition}, chopsticks, forks and knives~\cite{liu2017novel} in different culture.




% For passive sensing rings, access to raw sensor data. For interaction rings, api to popular development platform such as Unity, use interaction rings' function with other devices, such as gesture + VR.

% \subsection{Limitations}


\section{Conclusion}
In conclusion, this survey has comprehensively reviewed 206 pieces of literature pertaining to smart rings, categorizing them into four distinct application categories. Each category has been meticulously detailed in the corresponding sections of this paper. Furthermore, we introduced a phenomena-based taxonomy, which systematically breaks down the elements of smart ring technology into a hierarchy of applications, phenomena, fundamental phenomena, and sensors. Additionally, we have discussed several challenges that persist in the field, as well as potential directions for future research.