
\section{Introduction}
Over the past years, wearable computing devices have gained increasing popularity such as smartwatches~\cite{king2018survey}, smart earables~\cite{roddiger2022sensing}, smart glasses~\cite{wang2024g} and smart rings~\cite{rissanen2013subtle}. 
Among these, smart rings are lightweight, socially
acceptable across cultures, and seamlessly integrate into daily life with minimal obtrusiveness~\cite{bilius2023could, rissanen2013subtle}.
% For example, recent research has demonstrated the subtleness and social acceptability of smart rings across different ages.
However, despite their potential, there remains a gap in comprehensive understanding and systematic documentation of the capabilities and advancements in smart ring technology. This raises an important question: \textit{What are smart rings capable of, and how can they enable interaction, well-being, and overall lifestyle management?}


Depending on the wearing position, smart rings can capitalize on the unique properties of the human finger. Fingers are the termini of our limbs, equipped with rich sensory innervations and capable of intricate movements. 
Smart rings are positioned in a way that allows them to sense a wide range of motion-related phenomena, making them exceptionally good at detecting gestures~\cite{vatavu2021gesturing} and fine motor activities~\cite{rissanen2013subtle}. 
The flexibility of fingers enable a high degree of freedom (DOF) spatial interactions, which can be harnessed for complex command inputs in various applications, substituting numerous input devices, including mouse~\cite{chen2014mobiring, waghmare2023zring, shen2024mousering}, keyboard~\cite{liang2023drg, gu2020qwertyring, li2023ringvkb, nirjon2015typingring}, or controller~\cite{yau2020subtle, yeo2019wrist}. 
Moreover, due to the fingers' constant blood flow and exposure, smart rings can effectively monitor physiological signals that are typically checked in clinical settings, such as pulse rate~\cite{boukhayma2021ring, haddad2021ear, dong2021cloud, mahmud2018sensoring} or oxygen saturation~\cite{magno2019self}, thus enabling health tracking applications~\cite{rajput2023assessment, usman2019analyzing, halkola2019towards}.
Additionally, the strategic placement on the finger enables continuous health monitoring~\cite{wongtaweesup2023using, joseph2022integrated} without the need for more invasive or cumbersome equipment. 
This dual capability of interaction and passive sensing underscores the potential of smart rings as both input/output devices and health monitoring systems.

Prior to this work, several surveys related to smart ring technologies have been published. 
In 2013, \citet{rissanen2013subtle} conducted an early survey of ring-shaped user interfaces based on 16 research articles to illustrate the ``ringterface'' concept. 
In 2015, \citet{shilkrot2015digital} conducted a comprehensive survey on finger augmentation devices (FADs), covering sensing and feedback modalities on smart rings. Their work also included non-ring form factors, such as fingernail addendums and finger sleeves. 
However, given the significant advancements and increasing commercialization of smart rings over the past ten years, there is a pressing need for an updated survey that reflects the current state and emerging trends in smart ring technology. 
In the most recent work from 2021, \citet{vatavu2021gesturing} conducted a systematic literature review on ring-based gesture input and contributed a searchable gesture-to-function dictionary. 
However, there are numerous applications that can be enabled by a ring form-factor that go beyond just gesture recognition.  
Therefore, in comparison to previous works, our survey consolidates the full spectrum of capabilities enabled by smart rings, encompassing a broader range of functionalities to provide a comprehensive overview of the field's current state and future directions.


In this paper, we follow the FADs form factors proposed by~\citet{shilkrot2015digital} and solely focus on the ``rings'' category. 
We explore the research field of smart rings by conducting a systematic literature review of 206 smart ring-related publications. 
We present a taxonomy that categorizes the sensing and feedback modalities, applications, and the phenomena these devices can detect and interact with. Our review organizes the existing literature into four main areas: (1) interaction - input, where we explore how smart rings serve as command interfaces; (2) interaction - output, examining the feedback mechanisms rings can provide; (3) passive sensing - in-body features, detailing how physiological and biometric data are captured for health monitoring; and (4) passive sensing - out-body activity, discussing how rings monitor external physical activities and environmental cues. This comprehensive review highlights the advancements within the field and also identifies potential areas for future research.