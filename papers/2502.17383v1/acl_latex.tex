% This must be in the first 5 lines to tell arXiv to use pdfLaTeX, which is strongly recommended.
\pdfoutput=1
% In particular, the hyperref package requires pdfLaTeX in order to break URLs across lines.

\documentclass[11pt]{article}

% Change "review" to "final" to generate the final (sometimes called camera-ready) version.
% Change to "preprint" to generate a non-anonymous version with page numbers.
\usepackage[preprint]{acl}

% Standard package includes
\usepackage{times}
\usepackage{latexsym}

% For proper rendering and hyphenation of words containing Latin characters (including in bib files)
\usepackage[T1]{fontenc}
% For Vietnamese characters
% \usepackage[T5]{fontenc}
% See https://www.latex-project.org/help/documentation/encguide.pdf for other character sets

% This assumes your files are encoded as UTF8
\usepackage[utf8]{inputenc}

% This is not strictly necessary, and may be commented out,
% but it will improve the layout of the manuscript,
% and will typically save some space.
\usepackage{microtype}

% This is also not strictly necessary, and may be commented out.
% However, it will improve the aesthetics of text in
% the typewriter font.
\usepackage{inconsolata}

%Including images in your LaTeX document requires adding
%additional package(s)
\usepackage{mdframed}
\usepackage{graphicx}
\usepackage{float}
\usepackage{hyperref}
\usepackage{url}
% \usepackage{inconsolata}
\let\Bbbk\relax
\usepackage{amssymb}
\usepackage{amsmath}
\usepackage{epsfig,subfigure,caption}
\usepackage{algpseudocode}
\usepackage[normalem]{ulem}
\usepackage[linesnumbered,algoruled,boxed,noend]{algorithm2e}
\usepackage{listings}  % Required for text wrapping inside tcolorbox
\usepackage{xcolor}     % Optional: For custom colors
\usepackage{color, colortbl}
\usepackage{multirow,booktabs, hhline}
\usepackage{bm}
\usepackage[linesnumbered,algoruled,boxed,noend]{algorithm2e}
\usepackage{wrapfig}
\usepackage{verbatimbox}
\usepackage{kantlipsum}
\usepackage{fancyvrb}
\usepackage{xcolor}
\usepackage[many]{tcolorbox}
\usepackage[htt]{hyphenat}
% \usepackage{mdframed}
\usepackage{pifont}% http://ctan.org/pkg/pifont
\newcommand{\cmark}{\ding{51}}%
\newcommand{\xmark}{\ding{55}}%
\def\Centerline#1{%
  \setsepchar{\cpar}%
  \readlist\clarg{#1}%
  \foreachitem\z\in\clarg[]{\centerline{\z}}%
}
\tcbset{
    myboxstyle/.style={
        colback=yellow!10!white, % Brighter yellow background
        colframe=yellow!50!black, % Vibrant yellow border
        fonttitle=\bfseries,
        coltitle=black,
        boxrule=0.75mm,
        width=\textwidth,
        sharp corners,
        leftrule=1mm,
        left=5pt,
        right=5pt,
        top=5pt,
        bottom=5pt,
        fontupper=\small,  % Set the font size for content
        fontlower=\small   % Set the font size for footnotes or lower part
    }
}
% If the title and author information does not fit in the area allocated, uncomment the following
%
%\setlength\titlebox{<dim>}
%
% and set <dim> to something 5cm or larger.
\newcommand{\dongho}[1]{{\textcolor{blue}{DH: #1}}}
\newcommand{\justin}[1]{{\textcolor{cyan}{JC: #1}}}

\newcommand{\secref}[1]{\S\ref{#1}}
\newcommand{\eg}{\textit{e.g., }}
\newcommand{\ie}{\textit{i.e., }}
\newcommand{\versus}{\textit{vs. }}

\newcommand{\ourdata}{\textsc{Textbook-Exam}\xspace} 
\newcommand{\ours}{\textsc{QUEST}\xspace}
\definecolor{darkgreen}{RGB}{95,129,63}
\definecolor{darkorange}{RGB}{184, 96, 41}

\title{What is a Good Question? \\ Utility Estimation with LLM-based Simulations}



\author{
    Dong-Ho Lee\thanks{Authors contributed equally},~
    Hyundong Cho\footnotemark[1],~
    Jonathan May\thanks{Equal advising contribution},~
    Jay Pujara\footnotemark[2] \\
    Information Science Institute, University of Southern California \\
     {\small 
        \texttt{\{dongho.lee\}@usc.edu},~
        \texttt{\{jcho, jonmay, jpujara\}@isi.edu}
    }\\
}



\begin{document}
\maketitle
\begin{abstract}

% Recent works to jointly reconstruct 3D human and object from a single RGB image, are mostly model-based, that fail to capture the fine details of the clothed human body and object surface. In this paper, we introduce ReCHOR, a novel, model-free, first-method to produce realistic clothed human-object reconstructions from a monocular view. This is extremely challenging due to human-object occlusions, diverse interactions and depth ambiguity, as it needs to infer both 3D spatial awareness and high resolution details. Our core idea is based on estimating neural implicit representations for human and object respectively by an attention-based neural implicit model that attends to pixel-aligned features from both the global human-object image for spatial awareness and  the local separate view of human and object images for high quality details. Additionally, the network is conditioned on semantic features from an initial estimated human-object pose prior and a generative diffusion model that inpaints occluded regions, thus enabling the retrieval of details from them.
% We also propose a synthetic dataset with rendered scenes of diverse, inter-occluded 3D human and object scans, to train our network. We evaluate our method on the synthetic and real world BEHAVE dataset. Our experiments show that our method outperforms the SOTA in achieving realistic clothed human-object reconstructions.
Recent approaches to jointly reconstruct 3D humans and objects from a single RGB image represent 3D shapes with template-based or coarse models, which fail to capture details of loose clothing on human bodies. In this paper, we introduce a novel implicit approach for jointly reconstructing realistic 3D clothed humans and objects from a monocular view. For the first time, we model both the human and the object with an implicit representation, allowing to capture more realistic details such as clothing. This task is extremely challenging due to human-object occlusions and the lack of 3D information in 2D images, often leading to poor detail reconstruction and depth ambiguity. To address these problems, we propose a novel attention-based neural implicit model that leverages image pixel alignment from both the input human-object image for a global understanding of the human-object scene and from local separate views of the human and object images to improve realism with, for example, clothing details. Additionally, the network is conditioned on semantic features derived from an estimated human-object pose prior, which provides 3D spatial information about the shared space of humans and objects. To handle human occlusion caused by objects, we use a generative diffusion model that inpaints the occluded regions, recovering otherwise lost details. For training and evaluation, we introduce a synthetic dataset featuring rendered scenes of inter-occluded 3D human scans and diverse objects. Extensive evaluation on both synthetic and real-world datasets demonstrates the superior quality of the proposed human-object reconstructions over competitive methods.
\end{abstract}
\section{Introduction}\label{sec:intro}

In computational finance, Monte Carlo simulations are used extensively to estimate the expected value of financial payoffs based on the solution of stochastic differential equations (SDEs) which model the evolution of stock prices, interest rates, exchange rates and other quantities \cite{glasserman04}.  Monte Carlo methods are very general and flexible, but for high accuracy it requires generating a large number of costly SDE path approximations, which has motivated research into a number of variance reduction or, equivalently, cost reduction techniques. One such method is
Multilevel Monte Carlo (MLMC), which was proposed in \cite{GILES2008} and was adapted for various applications that are summarised in \cite{Giles_overview17} and successfully combined with other methods such as quasi-Monte Carlo methods. The main idea of MLMC is to approximate the payoff using different time stepping resolutions when numerically solving the underlying SDE and to generate an optimal number of samples on each level, such that the overall computational cost is minimised subject to the desired bound on the variance. %, such that the total computational cost is minimised. 
The computational savings come from the fact that most samples are computed on the coarser levels and hence are less expensive while only a few samples from the finest levels are required \cite{GILES2008}.


Among the directions in which the computational cost 
of MLMC methods could further be reduced, an important avenue is the use of lower precision calculations, especially for the first Monte Carlo levels where the targeted accuracy is relatively low. 
 An overview of the research on mixed precision for the standard Monte Carlo (MC) framework is provided in \cite{ChowMixedPrecisionStandardMC} but only a few references study the potential of low precision computation in the MLMC framework \cite{Rounding_error_oliver}. To the best of our knowledge, the only MLMC framework with customised precision in the literature is \cite{brugger2014mixed}, but they use a uniform precision for all operations on each Monte Carlo level instead of optimising 
 the precision of each intermediary variable to reduce as much as possible the cost of path generation.
 
An important motivation for an MLMC framework with variable precision would be performing the low precision computations on reconfigurable hardware devices such as Field Programmable Gate Arrays (FPGAs). FPGAs contain customizable logic blocks and connectors that make it easy to adapt the digital circuit architecture for a specific application, leading to a highly parallel and optimised implementation. Therefore they are successfully exploited in applications that require high speed and have high computational workload, such as signal processing \cite{woods2008fpga}, and real time applications like high frequency trading \cite{HFT1,HFT2}. That is why a number of previous works in hardware architecture design implemented the MLMC algorithm to price financial options using FPGAs as accelerators, which resulted in improved speed and power efficiency compared to full CPU architectures \cite{Schryver2013AMM}. The paper \cite{lindsey2016domain} also proposed 
a Domain Specific Language to automate the configuration of FPGAs for this specific application. However, only \cite{brugger2014mixed} proposed a heuristic to reduce the precision in calculations.

In addition, all aforementioned works considered that the random number generation (RNG) is performed in single or double precision. Yet in most cases an important portion of the workload in the overall MLMC simulation comes from the RNG and in \cite{brugger2014mixed} this limited the total computational savings.
To reduce the cost of MLMC simulations in particular those based on the Geometric Brownian Motion (GBM), \cite{approximateICDF_Oliver, NestedOliver} have proposed to use approximate random numbers that are generated by applying an approximation of the inverse CDF to uniform random numbers. In \cite{NestedOliver}, the authors proposed a way to integrate these lower precision random variables into a \textit{nested} MLMC framework and completed a numerical analysis to bound the resulting error at each MC level by a product of the time step and the error in the random number approximation. The same authors show in \cite{approximateICDF_Oliver} that using approximate random variables reduces the cost of path generation by a factor 7.


In this paper we propose a nested MLMC framework that combines the use of approximate random normal variables and lower precision calculations to reduce the computational cost of MLMC even further than \cite{brugger2014mixed,NestedOliver}. We illustrate the efficiency of our framework in Matlab, after making several assumptions on the cost of operations and size of the errors that we carefully justify. We focus on the case of GBM and use the approximate RNG methods presented in \cite{approximateICDF_Oliver} as well as a new slightly modified method that combines CDF inversion and the central limit theorem. To choose the precision of the variables in the low precision path generation, we introduce a novel method to optimise the bit-widths. This optimisation is performed before the main path generation loop is executed and is based on a linear model of the payoff error  
due to rounding when computing in low precision. The error model relies on algorithmic differentiation in a similar manner to \cite{unifying-bwoptim,bitwidth-AD,ADAPT}. The bit-width optimisation procedure can be performed off-line, so this stage can be excluded from the on-line time complexity of our framework. The user specified desired accuracy is then enforced by calculating on-line the number of samples that need to be generated.

In terms of hardware design, we suggest implementing the low precision path generation on FPGAs and the full-precision ones on a CPU or GPU. 
The FPGA offers enough flexibility to define a separate bit-width for every variable in the low precision path generation, and can be reconfigured periodically to update the bit-widths when the market parameters have changed considerably. 


The paper is organized as follows : \Cref{sec:MLMC} introduces MLMC and nested MLMC to make clear the estimator that is implemented in our framework. Then in \Cref{sec:RNG} we detail the methods that could be used to obtain approximate random normally distributed numbers very cheaply for the low precision path generation. In \Cref{sec:error_model} and \Cref{sec:costModel} we propose an error model and a cost model (resp.) that we then use to formulate the optimisation problem that is solved to obtain the optimal bit-widths of fixed point variables in \Cref{sec:optimisation}. Finally we summarise our results and future directions in \Cref{sec:conclusion}.



\section{\ours}
\label{sec:framework}

In this section, we introduce \ours as a method for measuring question utility and using high-utility questions to fine-tune question generators. 
We first provide an overview of \ours (\secref{ssec:quest-overview}) and delve into the details of each its procedures (\secref{ssec:quest-question-generation}-\secref{ssec:quest-train}). 

Consider a binary prediction task for ICU patient mortality based on electronic medical records. A source hospital $H_o$ has historical patient data $\Do$ containing static past patient characteristics, prior medical records, and ICU outcomes. Other hospitals $\{H_i\}$ each has their patient data: $\{\Di \mid i\in [1.. N]\}$. 

For this binary prediction task, hospitals typically optimize for performance metrics, for example the area under the receiver-operating characteristic curve (AUC). Using only their data, $H_o$ can train a model $\mathcal{M}$ with parameters $\theta$ to achieve:
\begin{equation}
\tag{Baseline Performance}
\AUCo = \max_{f(\theta)} \, \AUC(\mathcal{M}, \Do)
\end{equation}
where $f$ is their chosen algorithm with parameter $\theta$.
%\footnote{\textcolor{red}{note that this algorithm can change downstream(for now we can omit)}}
%\AUC(H_o, \emptyset)
%, obtaining model parameters $\theta_o$, choosing an algorithm $f_o$, and an initial AUC of $\AUC^{[o]}$.

When $H_o$ has exhausted their own internal data, they may benefit from incorporating additional target data sources $T\subset [1.. N]$. By combining datasets, i.e., $\DT = \{\Di\mid i\in T\}\cup \Do$, $H_o$ can potentially achieve better results:
\begin{equation}
\tag{Combined Performance}
\AUCT = \max_{f(\theta)} \, \AUC(\mathcal{M}, \DT).
\end{equation}
We define the potential improvement from data addition as $\delta_{T} = \delta_{(o, T)} = \AUCT-\AUCo$. To add a single additional data source by setting $T=\{i\}$, the improvement is $\delta_{i} =\delta_{(o, i)}=\AUC_i-\AUCo$.
This leads to our central question:
% \begin{quote}
    \textbf{\emph{Without seeing target data, how does a hospital ascertain potential data sources to combine with?}}
% \end{quote}

Formally, given $n\leq N$, we seek a strategy $\pi$ that selects $n$ target datasets $T=\pi(\Do, n)$ to maximize model utility:
\begin{equation}
\tag{Ideal Dataset Combination}
\pi^*(\Do, n) = \argmax_{T\subset {[1...N]\choose n}}\AUCT%\quad\forall n.
\end{equation}
\paragraph{Practical Considerations.} Computing every subset $T\subset {[1...N]\choose n}$'s associated $\delta_{T}$ is exponential in $n$. To make this problem tractable, we make two key assumptions. First, we apply strategies greedily, selecting top-ranked target datasets. With the ultimate objective of improving the source hospital's prediction task, we fix $H_o$; to compare the trade-offs between strategies in Section~\ref{sec:methods}, we apply each $\pi$ greedily to select top-$n$ institution(s) for $H_o$ without replacement. Second, in in data constrained settings, we aim to maximize the probability of positive improvement: $P_{H_o\sim \mathbf{H}}(\delta_T > 0)$. 
%\textcolor{red}{Add additional caveats here for folktable setups.}
\paragraph{Kullback–Leibler Divergence.} Our approach uses Kullback-Leibler (KL)-divergence-based methods to gauge data utility, building on prior work~\cite{shen2024data}. KL divergence~\cite{kullback1951information}, also called \emph{information gain}~\cite{quinlan1986induction}, describes a measure of how much a model probability distribution $Q$ is different from a true probability distribution $P$:
\begin{equation}
\tag{Kullback–Leibler Divergence}
\mathrm{KL}(P||Q) = \int_{x\in \mathcal{X}} \log\frac{P(\diff x)}{Q(\diff x)}P(\diff x)
\end{equation}
Because computing KL-divergence on datasets $\Do$ and $\Di$ is non-trivial, ~\priorp proposes two groups of scores to make this divergence approximation tractable from small samples.
% \begin{equation}
% \tag{Ideal Estimator}
% \mathrm{KL}(P_o||P_i) = \int_{x\in \mathcal{X}} \log\frac{P_o(\diff x)}{P_i(\diff x)}P_i(\diff x)
% \end{equation}
 Specifically, score $\KLXY$ first trains a logistic regression model on $\Do \cup \Di$ -- where the labels are folded into the covariates --- with the goal of inferring dataset membership. Then, the resulting model's probability score function $\text{Score}(\cdot): \mathcal{X, Y} \to [0,1]$ is averaged over a dataset in $H_o$, obtaining

\begin{equation}
\tag{KL-XY Score}
\KLXY = \mathbb{E}_{(x,y)\sim \Do}(\text{Score}(x, y)).
\end{equation}
Details are described in Section~\ref{sec:methods}.
\paragraph{Privacy Model for $\pi_p$.} We operate under a semi-honest privacy model---also known as \emph{honest-but-curious} or \emph{passive security}---where parties follow protocols but may probe intermediate values. Parties are  ``curious'', meaning that they can probe into the intermediate values to avoid paying for the data. This assumes a weaker security model than malicious security where a corrupted party may input foul data, but ensures the algorithm to be private throughout the computation. This privacy preservation model incentivizes collaboration, improving upon methods in ~\priorp.



\paragraph{MPC Preliminary}
To secure this divergence computation cryptographically, Secure Multiparty Computation (MPC)~\cite{yao1982protocols, shamir1979share} protocols are leveraged. Specifically, in $\mathrm{SecureKL}$, each party encodes $\Do$ and $\Di$ to preserve privacy for both parties. This is implemented with the research framework CrypTen~\cite{knott2021crypten}, specialized for MPC and machine learning. Our algorithmic and engineering details are in Sections ~\ref{sec:methods} and ~\ref{sec:exp}, respectively. For related secure techniques, see Section~\ref{sec:related_secure}.
\paragraph{Additional Assumptions }
% Assume:
% Source party has access to their own data (test set) where they want an algorithm to work well (though they may not know what algorithm model they use)
% Source party can buy / engage with data from other sources but they don’t have access directly
% (OR they only have a small percentage access)

% Goal: without compromising on data privacy, ascertain among candidate data sources, which ones would be sensible to combine with my setting and existing data?

% What is being done here?
Generally, we consider high stakes domains where disparate data may have additive benefits to the existing data.
In order to make privacy boundaries tractable, we make the following additional assumptions:
\begin{enumerate}
\itemsep0em
\item \textbf{Existing knowledge} is not private. The hospitals are aware of each other having such data to begin with. The hospitals may know of the available underlying dataset size and format, which is assumed to be uniform across the hospitals in the setup to simulate unit-cost. Hospitals frequently know of each other's resources, and the available ICU units are contentious, not kept secret. 
\item \textbf{Uniformity} of $|\Di|$. Though each hospital gets to price their data and set their own budget, for generality, the uniformity assumption allows us to use the number of additional data sources $n$ as the main "budget proxy" across different strategies.
\item \textbf{Legal risks} of sharing \emph{any} data are omnipresent in high stakes domains. The risks with sharing sensitive data in $\pi_d$ and $\pi_s$ are not made explicit, but assumed to be "medium" and "medium-to-high" respectively. This abstraction side-steps legal discussion, which would go beyond the scope of our paper.
\item \textbf{No malice} is assumed on any of the parties involved, as each hospital wants to authentically sell their data and set up a potential collaboration. This assumption becomes stronger when the number of parties grows or when the setup changes to potentially more competitive industries with less trust. We note our limitations in Section~\ref{sec:limits}.
\end{enumerate}

\subsection{Overview}
\label{ssec:quest-overview}

\ours consists of the following:
(1) a \textbf{question generator} (\(M_q\)), which takes a document \(D\) and generates a set of questions \(Q = \{q_1, q_2, \dots, q_n\}\) (\secref{ssec:quest-question-generation});
(2) an \textbf{answer generator} (\(M_a\)) that then produces an answer \(a_i\) for each question \(q_i \in Q\), forming the set of question-answer pairs \(\text{QA} = \{(q_1, a_1), (q_2, a_2), \dots, (q_n, a_n)\}\) using parametric knowledge (\secref{ssec:quest-question-generation});
(3) a \textbf{learner simulator}, which models a learner (\(M_l\))’s understanding by having an evaluator (\(M_e\)) assess their performance on a final exam $E$ using only \(\text{QA}\) (\secref{ssec:quest-evaluation}); and
(4) a \textbf{utility estimator}, which runs the learner simulator multiple times with different subsets of \(\text{QA}\), estimating the contribution of individual questions to the learner’s overall performance (\secref{ssec:quest-train}).
See Figure~\ref{fig:framework}.
Additional details, including the prompts used for each module, are provided in Appendix~\ref{appendix:quest-details}.


\subsection{Question Generation}
\label{ssec:quest-question-generation}
The \textbf{Question Generator} (\( M_q \)) generates a set of questions (\( Q \)) based on the input document.
For each ordered section \( S_k \) in \( D \), where \( S_k \) represents the part of the document the learner is currently reading (referred to as the \textit{anchor}), the question generator considers both \( S_k \) and its preceding context \( C_k = \{S_1, S_2, \dots, S_{k-1}\}\) to generate a corresponding set of questions \( Q_k = \{q_k^1, q_k^2, \dots, q_k^n\} \), formulated as \( Q_k = M_q(S_k, C_k) \).


This approach ensures that the generated questions are contextually relevant and informed by the surrounding content, aligning with prior research on inquisitive question generation~\cite{wu2024questions}.  
Once the questions are generated, the \textbf{Answer Generator} (\( M_a \)) produces an answer \( a \) for each question \( q \) using the model’s parametric knowledge, forming a set of question-answer pairs: \( \text{QA} = \{(q_1, a_1), (q_2, a_2), \dots, (q_n, a_n)\} \). 
These QA pairs serve as the foundation for subsequent evaluation and simulation stages.

\begin{figure}[t!]
    \centering
    \begin{minipage}{\columnwidth}
    \centering
    \includegraphics[width=\columnwidth]{figures/framework.pdf}
\end{minipage}
\caption{\textbf{\ours Framework.} $M_q$ creates questions, and $M_a$ provides answers.
Learner simulator evaluates these QA pairs using a Learner $M_l$ and an Evaluator $M_e$.
The Utility Estimator runs multiple simulations to approximate the utility of each question.
Rejection sampling ensures that only high-utility questions are used to refine $M_q$.}
\vspace{-0.3cm}
    \label{fig:framework}
\end{figure}

\subsection{Evaluating the Question Generator}
\label{ssec:quest-evaluation}
Once the QA pairs for the document \( D \) are generated using \( M_q \) and \( M_a \), we assess the effectiveness of the question generator by employing the \textbf{Learner Simulator}, which consists of a \textbf{Learner} (\( M_l \)) and an \textbf{Evaluator} (\( M_e \)).
The learner model \( M_l \) simulates a learner's understanding by attempting the final exam \( E \) using only the generated QA pairs, producing responses \( P = M_l(E, \text{QA}) \).
The evaluator model \( M_e \) then assesses the learner’s responses \( P \) by comparing them against ground-truth answers when available or using parametric knowledge to assign a score.
The score can serve as a metric to assess the quality of \( M_q \), providing insights into the usefulness (\textit{i.e.,} \textit{utility}) of the generated questions in answering the final exam.

\subsection{Improving the Question Generator}
\label{ssec:quest-train}

Our objective is to design a question generator \( M_q \) that maximizes the learner simulator's performance on \( E \) when provided with \( \text{QA} \). 
This assumes that questions contributing more effectively to solving \( E \) are high-utility questions.

To enhance the quality of the question generator \( M_q \), we aim to learn the patterns of high-utility questions—those that lead to better performance in the learner simulator.
To achieve this, we estimate the utility \( u \) of each question-answer pair \( (q, a) \in \text{QA} \).
The utility estimator runs the learner simulator multiple times with different subsets of QA, computing the \textit{single-one gain}, which is the score obtained using only \( (q, a) \), and the \textit{all-but-one gain}, which is the score with all pairs except \( (q, a) \).
The utility of \( (q, a) \) is then computed as the average of these two scores.
We retain only the pairs where utility exceeds a threshold \( \theta \) and train \( M_q \) using only these high-utility pairs, following a rejection sampling strategy~\cite{bai2022constitutional}.



\section{\ourdata}
\label{sec:textbook-exam}

\begin{figure}[t]
    \centering
    \includegraphics[width=\linewidth]{figures/data_framework.pdf}    \caption{\textbf{Overview of  \ourdata curation}. Given a chapter $C$, we use an LM to segment the document $D$ into sections and heuristically extract review questions to form the exam $E$. The LM then classifies each question in $E$ by Bloom’s taxonomy category and maps it to its relevant section.}
    \label{fig:dataset-overview}
    \vspace{-0.3cm}
\end{figure}

In order to evaluate \ours, we curate  \ourdata, a dataset where each entry contains a document \(D\) along with a corresponding set of exam questions \(E\).
An overview of \ourdata is illustrated in \autoref{fig:dataset-overview}.
% In this section, we first describe our data processing pipeline (\secref{ssec:textbook-exam-pipeline}) and then provide data statistics (\secref{ssec:textbook-exam-statistics}).

\subsection{Data Processing}
\label{ssec:textbook-exam-pipeline}
Our pipeline starts with textbooks from the OpenStax repository\footnote{\url{https://github.com/philschatz/textbooks}}.
Each textbook is divided into chapters, where each chapter \(C\) contains learning objectives, main content, and review questions.
For each \(C\), we parse the main content to build \(D\) and the review questions to form \(E\).
% Specifically, only the main content is used to construct \(D\), while the review questions are used to create \(E\).

\paragraph{Extracting sections.}
To simulate a learner incrementally progressing through a chapter, we divide each chapter into sections using an LM-based document structuring method. 
The LM segments \( D \) into \( n \) sections, denoted as \( \{S_1, S_2, \ldots, S_n\} \subset D \), while also extracting the corresponding review questions \( E \). 
However, not all review questions come with ground-truth answers, as some textbooks do not provide them (see Table~\ref{tab:textbook-exam-statistics} for the proportion of \( E \) with answers). 
To ensure consistency in section segmentation across different subjects, we manually annotate the first 2–5 sections from one sample per subject. 
These annotated samples serve as few-shot examples in our LM prompt (see Appendix~\ref{appdx:parsing-sections} for details).

\paragraph{Extracting questions.}
% We developed a custom parsing script using BeautifulSoup4 to extract questions and their corresponding answers. 
To maintain a balance between evaluation depth and computational feasibility, we include only chapters that contain at least 10 questions—ensuring sufficient coverage for assessment—while capping the maximum number of questions at 25 to keep learning simulations computationally manageable.

\subsection{Data Statistics}
\label{ssec:textbook-exam-statistics}
\begin{table}[t!]
    \centering
    \resizebox{\columnwidth}{!}{
        \begin{tabular}{lccccc}
        \toprule
            \textbf{Subject} & \textbf{\# $C$} & \textbf{Split} & \textbf{\# $E$ / $C$} & \textbf{\% $E$ w/ answer} & \textbf{\# $S$ / $C$} \\
        \midrule
            Microbiology & 20 & Train & 12.4 & 64\% & 16.4 \\
                         & 5  & Test  & 13.4 & 58\% & 17.0 \\
        \midrule
            Chemistry    & 20 & Train & 14.2 & 51\% & 11.0 \\
                         & 5  & Test  & 16.2 & 49\% & 6.4 \\
        \midrule
            Economics    & 20 & Train & 12.2 & 23\% & 14.1 \\
                         & 5  & Test  & 12.2 & 23\% & 14.4 \\
        \midrule
            Sociology    & 20 & Train & 10.4 & 62\% & 16.6 \\
                         & 5  & Test  & 11.2 & 67\% & 19.0 \\
        \midrule
            US History   & 20 & Train & 7.2 & 51\% & 14.9 \\
                         & 5  & Test  & 8.4 & 38\% & 13.2 \\
        \bottomrule
        \end{tabular}
    }
    \caption{\textbf{Data Statistics} of \ourdata. \# $ C$: number of chapters, \# $E/C$: avg. number of questions per chapter, \% $E$ w/ answer: proportion of questions that have reference answer, \# $S/C$: avg. number of sections per chapter.}
    \label{tab:textbook-exam-statistics}
\end{table}
For each subject, we curate 25 sequential chapters \(C\), each containing both \(D\) and \(E\).
The chapters are arranged in their natural order, with the first 20 used for training and the last five reserved for evaluation.  
There is the risk that content in later chapters may include information from prior chapters (e.g., revisiting prerequisite knowledge). 
Therefore, preserving this sequential structure between the training and test set is essential for preventing information leakage and fairly assessing a model's learning process.
Table~\ref{tab:textbook-exam-statistics} shows an overview of the statistics of the resulting \ourdata.

\subsection{Distribution of Question Types}
\label{ssec:textbook-exam-bloom}

 \begin{figure}[t]
    \centering
    \includegraphics[width=\linewidth]{figures/bloom_taxonomy_counts_vertical.png}
    \caption{\textbf{Bloom's taxonomy distribution} in \ourdata. \ourdata consists of questions that require a wide variety of cognitive levels and the dominant categories vary for each subject.}
    \label{fig:bloom-distribution}
\end{figure}

To better understand how final exams assess a learner’s comprehension on multiple dimensions, we categorize questions in \ourdata\ based on the revised \textit{Bloom’s Taxonomy}~\cite{krathwohl2002revision_bloom}. 
Using an LM, we assign a cognitive depth \( d_j \) to each question \( E_j \in E \), classifying them into six categories: \textit{Remembering, Understanding, Applying, Analyzing, Evaluating}, and \textit{Creating}.
Additionally, we identify the relevant sections \( S_j \subset D \) that correspond to each question.

The distribution, shown in \autoref{fig:bloom-distribution}, indicates that different subjects emphasize different cognitive skills.
For instance, questions in Microbiology and Sociology primarily focus on \textit{Remembering} and \textit{Understanding}, whereas Chemistry and Economics exhibit a more varied distribution.
This analysis highlights the diverse cognitive demands across subjects and underscores how \ourdata\ provides a multifaceted evaluation of learning outcomes through final exams. 
For further details on data processing, refer to Appendix~\ref{appendix:data_processing}.













% \dongho{Number of textbooks?}

% \dongho{For each textbook, how may $D$?}



% \subsection{Validation}
% \dongho{Let's make a ground truth to see how reliable the data processing pipeline it is. -- for each textbook.}

% \paragraph{Validation of answer for $E_j$.}

% \paragraph{Validation of cognitive depth $d_j$ for $E_j$.}

% \paragraph{Validation of related sections $S_j$ for $E_j$.}


\section{Experimental Results}
\begin{table*}[t]
\centering
\caption{Total Variation Distance on CIFAR-10-LT ($N_l = 500$, $M_l = 4000$) with different class imbalance ratios $\gamma_l$ and $\gamma_u$ under five different unlabeled class distributions.}
\label{tab:cifar10-tv}
\resizebox{\textwidth}{!}{
\begin{tabular}{lccccccccccc}
\toprule
& & \multicolumn{2}{c}{consistent} & \multicolumn{2}{c}{uniform} & \multicolumn{2}{c}{reversed} & \multicolumn{2}{c}{middle} & \multicolumn{2}{c}{head-tail} \\
\cmidrule(lr){3-4} \cmidrule(lr){5-6} \cmidrule(lr){7-8} \cmidrule(lr){9-10} \cmidrule(lr){11-12}
& & $\gamma_l = 150$ & $\gamma_l = 100$ & $\gamma_l = 150$ & $\gamma_l = 100$ & $\gamma_l = 150$ & $\gamma_l = 100$ & $\gamma_l = 150$ & $\gamma_l = 100$ & $\gamma_l = 150$ & $\gamma_l = 100$ \\
Model & Estimator & $\gamma_u = 150$ & $\gamma_u = 100$ & $\gamma_u = 1$ & $\gamma_u = 1$ & $\gamma_u = 1/150$ & $\gamma_u = 1/100$ & $\gamma_u = 150$ & $\gamma_u = 100$ & $\gamma_u = 150$ & $\gamma_u = 100$ \\
\midrule
Supervised & MLLS & 0.269 ± 0.252 & 0.038 ± 0.006 & 0.251 ± 0.046 & 0.255 ± 0.060 & 0.429 ± 0.028 & 0.493 ± 0.050 & 0.333 ± 0.042 & 0.320 ± 0.009 & 0.457 ± 0.034 & 0.444 ± 0.043 \\
Supervised & RLLS & 0.043 ± 0.001 & 0.044 ± 0.010 & 0.348 ± 0.034 & 0.305 ± 0.068 & 0.769 ± 0.016 & 0.678 ± 0.028 & 0.430 ± 0.008 & 0.368 ± 0.013 & 0.539 ± 0.018 & 0.503 ± 0.020 \\
\midrule
MLE & IPW & 0.027 ± 0.001 & 0.027 ± 0.000 & 0.319 ± 0.072 & 0.243 ± 0.010 & 0.674 ± 0.020 & 0.646 ± 0.041 & 0.438 ± 0.020 & 0.454 ± 0.026 & 0.547 ± 0.049 & 0.491 ± 0.059 \\
MLE & OR & 0.045 ± 0.004 & 0.042 ± 0.000 & 0.215 ± 0.026 & 0.203 ± 0.032 & 0.433 ± 0.017 & 0.395 ± 0.033 & 0.193 ± 0.006 & 0.209 ± 0.037 & 0.307 ± 0.147 & 0.249 ± 0.130 \\
MLE & DR & 0.090 ± 0.002 & 0.079 ± 0.000 & 0.407 ± 0.027 & 0.360 ± 0.007 & 0.425 ± 0.007 & 0.421 ± 0.029 & 0.256 ± 0.001 & 0.286 ± 0.031 & 0.435 ± 0.136 & 0.362 ± 0.122 \\
\midrule
EM & IPW & 0.035 ± 0.002 & 0.040 ± 0.001 & 0.021 ± 0.001 & 0.029 ± 0.015 & 0.303 ± 0.187 & 0.091 ± 0.010 & 0.119 ± 0.011 & 0.105 ± 0.022 & 0.104 ± 0.026 & 0.104 ± 0.051 \\
EM & OR & 0.037 ± 0.003 & 0.042 ± 0.002 & 0.016 ± 0.001 & 0.024 ± 0.012 & 0.269 ± 0.183 & 0.090 ± 0.008 & 0.122 ± 0.012 & 0.103 ± 0.022 & 0.072 ± 0.012 & 0.073 ± 0.024 \\
EM & DR & 0.034 ± 0.004 & 0.037 ± 0.001 & 0.014 ± 0.001 & 0.027 ± 0.020 & 0.264 ± 0.191 & 0.092 ± 0.005 & 0.111 ± 0.019 & 0.097 ± 0.026 & 0.077 ± 0.016 & 0.073 ± 0.028 \\
\midrule
SimPro & IPW & 0.070 ± 0.011 & 0.058 ± 0.000 & 0.046 ± 0.001 & 0.049 ± 0.005 & 0.254 ± 0.074 & 0.223 ± 0.098 & 0.097 ± 0.025 & 0.067 ± 0.002 & 0.105 ± 0.066 & 0.110 ± 0.079 \\
SimPro & OR & 0.071 ± 0.012 & 0.058 ± 0.000 & 0.045 ± 0.001 & 0.049 ± 0.006 & 0.040 ± 0.003 & 0.059 ± 0.017 & 0.074 ± 0.006 & 0.075 ± 0.002 & 0.033 ± 0.003 & 0.033 ± 0.003 \\
SimPro & DR & 0.017 ± 0.004 & 0.026 ± 0.001 & 0.019 ± 0.002 & 0.018 ± 0.003 & 0.039 ± 0.003 & 0.058 ± 0.025 & 0.091 ± 0.007 & 0.031 ± 0.001 & 0.015 ± 0.003 & 0.019 ± 0.007 \\
\bottomrule
\end{tabular}
}
\end{table*}


\begin{table*}[t]
\centering
\caption{Total Variation Distance on CIFAR-100-LT ($N_l = 50$, $M_l = 400$) with different class imbalance ratios $\gamma_l$ and $\gamma_u$ under five different unlabeled class distributions.}
\label{tab:cifar100-tv}
\resizebox{\textwidth}{!}{
\begin{tabular}{lccccccccccc}
\toprule
& & \multicolumn{2}{c}{consistent} & \multicolumn{2}{c}{uniform} & \multicolumn{2}{c}{reversed} & \multicolumn{2}{c}{middle} & \multicolumn{2}{c}{head-tail} \\
\cmidrule(lr){3-4} \cmidrule(lr){5-6} \cmidrule(lr){7-8} \cmidrule(lr){9-10} \cmidrule(lr){11-12}
& & $\gamma_l = 20$ & $\gamma_l = 10$ & $\gamma_l = 20$ & $\gamma_l = 10$ & $\gamma_l = 20$ & $\gamma_l = 10$ & $\gamma_l = 20$ & $\gamma_l = 10$ & $\gamma_l = 20$ & $\gamma_l = 10$ \\
Model & Estimator & $\gamma_u = 20$ & $\gamma_u = 10$ & $\gamma_u = 1$ & $\gamma_u = 1$ & $\gamma_u = 1/20$ & $\gamma_u = 1/10$ & $\gamma_u = 20$ & $\gamma_u = 10$ & $\gamma_u = 20$ & $\gamma_u = 10$ \\
\midrule
Supervised & MLLS & 0.707 ± 0.016 & 0.313 ± 0.100 & 0.445 ± 0.172 & 0.309 ± 0.119 & 0.383 ± 0.075 & 0.397 ± 0.006 & 0.570 ± 0.001 & 0.373 ± 0.107 & 0.543 ± 0.009 & 0.231 ± 0.057 \\
Supervised & RLLS & 0.520 ± 0.007 & 0.133 ± 0.003 & 0.337 ± 0.125 & 0.253 ± 0.082 & 0.424 ± 0.060 & 0.463 ± 0.003 & 0.454 ± 0.021 & 0.306 ± 0.074 & 0.460 ± 0.028 & 0.241 ± 0.040 \\
\midrule
MLE & IPW & 0.075 ± 0.000 & 0.071 ± 0.001 & 0.229 ± 0.001 & 0.167 ± 0.002 & 0.565 ± 0.005 & 0.443 ± 0.007 & 0.415 ± 0.000 & 0.311 ± 0.005 & 0.343 ± 0.000 & 0.280 ± 0.001 \\
MLE & OR & 0.065 ± 0.002 & 0.061 ± 0.001 & 0.200 ± 0.007 & 0.143 ± 0.001 & 0.526 ± 0.011 & 0.399 ± 0.023 & 0.360 ± 0.003 & 0.256 ± 0.012 & 0.328 ± 0.003 & 0.266 ± 0.005 \\
MLE & DR & 0.149 ± 0.019 & 0.145 ± 0.010 & 0.243 ± 0.004 & 0.214 ± 0.019 & 0.568 ± 0.005 & 0.464 ± 0.014 & 0.403 ± 0.014 & 0.309 ± 0.012 & 0.365 ± 0.007 & 0.320 ± 0.004 \\
\midrule
EM & IPW & 0.097 ± 0.008 & 0.092 ± 0.004 & 0.239 ± 0.007 & 0.179 ± 0.003 & 0.478 ± 0.012 & 0.329 ± 0.020 & 0.262 ± 0.016 & 0.202 ± 0.003 & 0.312 ± 0.002 & 0.227 ± 0.001 \\
EM & OR & 0.121 ± 0.007 & 0.108 ± 0.005 & 0.261 ± 0.007 & 0.189 ± 0.004 & 0.489 ± 0.013 & 0.335 ± 0.020 & 0.274 ± 0.016 & 0.211 ± 0.004 & 0.336 ± 0.003 & 0.235 ± 0.001 \\
EM & DR & 0.125 ± 0.005 & 0.111 ± 0.004 & 0.269 ± 0.007 & 0.194 ± 0.005 & 0.497 ± 0.010 & 0.336 ± 0.024 & 0.281 ± 0.019 & 0.219 ± 0.008 & 0.336 ± 0.007 & 0.233 ± 0.004 \\
\midrule
SimPro & IPW & 0.125 ± 0.001 & 0.100 ± 0.005 & 0.166 ± 0.007 & 0.141 ± 0.009 & 0.353 ± 0.023 & 0.261 ± 0.008 & 0.202 ± 0.003 & 0.158 ± 0.005 & 0.277 ± 0.009 & 0.197 ± 0.003 \\
SimPro & OR & 0.133 ± 0.005 & 0.100 ± 0.004 & 0.160 ± 0.007 & 0.138 ± 0.010 & 0.322 ± 0.014 & 0.253 ± 0.008 & 0.202 ± 0.003 & 0.156 ± 0.005 & 0.269 ± 0.006 & 0.191 ± 0.004 \\
SimPro & DR & 0.122 ± 0.003 & 0.106 ± 0.006 & 0.188 ± 0.009 & 0.149 ± 0.006 & 0.343 ± 0.023 & 0.257 ± 0.007 & 0.219 ± 0.010 & 0.172 ± 0.002 & 0.279 ± 0.007 & 0.198 ± 0.004 \\
\bottomrule
\end{tabular}
}
\end{table*}
\begin{table*}[t]
\centering
\caption{Top-1 accuracy (\%) on CIFAR-10-LT ($N_l = 500$, $M_l = 4000$) with different class imbalance ratios $\gamma_l$ and $\gamma_u$ under five different unlabeled class distributions. In most settings, our two stage algorithm improves SimPro (9 / 10) and BOAT (8 / 10). We use {\green green} to indicate when our plug-in improves and {\red red} when it degrades the base model.}
\label{tab:cifar10-acc}
\resizebox{\textwidth}{!}{
\begin{tabular}{lcccccccccc}
\toprule

& \multicolumn{2}{c}{consistent} & \multicolumn{2}{c}{uniform} & \multicolumn{2}{c}{reversed} & \multicolumn{2}{c}{middle} & \multicolumn{2}{c}{head-tail} \\
\cmidrule(lr){2-3} \cmidrule(lr){4-5} \cmidrule(lr){6-7} \cmidrule(lr){8-9} \cmidrule(lr){10-11}

& $\gamma_l = 150$ & $\gamma_l = 100$ & $\gamma_l = 150$ & $\gamma_l = 100$ & $\gamma_l = 150$ & $\gamma_l = 100$ & $\gamma_l = 150$ & $\gamma_l = 100$ & $\gamma_l = 150$ & $\gamma_l = 100$ \\
& $\gamma_u = 150$ & $\gamma_u = 100$ & $\gamma_u = 1$ & $\gamma_u = 1$ & $\gamma_u = 1/150$ & $\gamma_u = 1/100$ & $\gamma_u = 150$ & $\gamma_u = 100$ & $\gamma_u = 150$ & $\gamma_u = 100$ \\

\midrule

FixMatch & 62.9 $\pm$ 0.36 & 67.8 $\pm$ 1.13 & 67.6 $\pm$ 2.56 & 73.0 $\pm$ 3.81 & 59.9 $\pm$ 0.82 & 62.5 $\pm$ 0.94 & 64.3 $\pm$ 0.63 & 71.7 $\pm$ 0.46 & 58.3 $\pm$ 1.46 & 66.6 $\pm$ 0.87 \\
CReST+ & 67.5 $\pm$ 0.45 & 76.3 $\pm$ 0.86 & 74.9 $\pm$ 0.90 & 82.2 $\pm$ 1.53 & 62.0 $\pm$ 1.18 & 62.9 $\pm$ 1.39 & 58.5 $\pm$ 0.68 & 71.4 $\pm$ 0.60 & 59.3 $\pm$ 0.72 & 67.2 $\pm$ 0.48 \\
DASO & 70.1 $\pm$ 1.81 & 76.0 $\pm$ 0.37 & 83.1 $\pm$ 0.47 & 86.6 $\pm$ 0.84 & 64.0 $\pm$ 0.11 & 71.0 $\pm$ 0.95 & 69.0 $\pm$ 0.31 & 73.1 $\pm$ 0.68 & 70.5 $\pm$ 0.59 & 71.1 $\pm$ 0.32 \\
% w/ ACR$\dagger$ (Wei \& Gan, 2023) & 70.9 $\pm$ 0.37 & 76.1 $\pm$ 0.42 & 91.9 $\pm$ 0.02 & 92.5 $\pm$ 0.19 & 83.2 $\pm$ 0.39 & 85.2 $\pm$ 0.12 & 77.6 $\pm$ 0.20 & 79.3 $\pm$ 0.30 & 73.8 $\pm$ 0.83 & 79.3 $\pm$ 0.48 \\
% w/ SimPro & 74.2 $\pm$ 0.90 & 80.7 $\pm$ 0.30 & 93.6 $\pm$ 0.08 & 93.8 $\pm$ 0.10 & 83.5 $\pm$ 0.95 & 85.8 $\pm$ 0.48 & 82.6 $\pm$ 0.38 & 84.8 $\pm$ 0.54 & 81.0 $\pm$ 0.27 & 83.0 $\pm$ 0.36 \\
Supervised & 63.2 $\pm$ 0.14 & 66.0 $\pm$ 0.27 & 63.3 $\pm$ 0.28 & 65.8 $\pm$ 0.19 & 63.1 $\pm$ 0.19 & 65.9 $\pm$ 0.51 & 63.5 $\pm$ 0.22 & 65.8 $\pm$ 0.03 & 63.0 $\pm$ 0.18 & 66.4 $\pm$ 0.07 \\
\midrule
EM & 69.1 $\pm$ 1.29 & 73.8 $\pm$ 0.71 & 94.0 $\pm$ 0.08 & 93.2 $\pm$ 0.94 & 76.6 $\pm$ 2.72 & 82.2 $\pm$ 0.24 & 79.5 $\pm$ 0.35 & 81.6 $\pm$ 0.58 & 79.2 $\pm$ 0.50 & 79.8 $\pm$ 0.17 \\
\midrule
SimPro & 74.4 $\pm$ 0.71 & 79.7 $\pm$ 0.45 & 93.3 $\pm$ 0.10 & 93.3 $\pm$ 0.47 & 83.8 $\pm$ 0.80 & 84.1 $\pm$ 0.24 & 78.7 $\pm$ 0.30 & 84.2 $\pm$ 0.26 & 81.2 $\pm$ 0.20 & 82.0 $\pm$ 1.07 \\
% \midrule
SimPro+ & \green 77.8 $\pm$ 1.50 & \green 81.2 $\pm$ 0.39 & \green 93.7 $\pm$ 0.07 & \green 93.7 $\pm$ 0.24 & \red 83.3 $\pm$ 0.38 & \green 84.7 $\pm$ 0.78 & \green 79.2 $\pm$ 0.70 & \green 85.4 $\pm$ 0.66 & \green 81.3 $\pm$ 0.27 & \green 82.5 $\pm$ 0.56 \\
\midrule
BOAT & 80.5 $\pm$ 0.39 & 83.3 $\pm$ 0.27 & 93.9 $\pm$ 0.03 & 94.1 $\pm$ 0.10 & 79.7 $\pm$ 0.25 & 81.1 $\pm$ 0.15 & 79.7 $\pm$ 1.15 & 81.6 $\pm$ 0.09 & 79.4 $\pm$ 0.44 & 80.9 $\pm$ 0.16 \\
% \midrule
BOAT+ & \green 81.6 $\pm$ 0.15 & \green 83.8 $\pm$ 0.04 & \red 93.7 $\pm$ 0.23 & 94.1 $\pm$ 0.17 & \green 80.4 $\pm$ 0.71 & \green 81.7 $\pm$ 0.38 & \green 80.3 $\pm$ 0.28 & \green 83.1 $\pm$ 0.45 & \green 79.7 $\pm$ 0.29 & \green 81.0 $\pm$ 0.36 \\
\bottomrule
\end{tabular}
}
\end{table*}

\begin{table*}[t]
\centering
\caption{Top-1 accuracy (\%) on CIFAR-100-LT ($N_l = 50$, $M_l = 400$) with different class imbalance ratios $\gamma_l$ and $\gamma_u$ under five different unlabeled class distributions. Despite poor estimation in stage 1, our approach does not degrade the accuracy for most of the settings. We use {\green green} to indicate when our plug-in improves and {\red red} when it degrades the base method.}
\label{tab:cifar100-acc}
\resizebox{\textwidth}{!}{
\begin{tabular}{lccccccccccc}
\toprule

& \multicolumn{2}{c}{consistent} & \multicolumn{2}{c}{uniform} & \multicolumn{2}{c}{reversed} & \multicolumn{2}{c}{middle} & \multicolumn{2}{c}{head-tail} \\
\cmidrule(lr){2-3} \cmidrule(lr){4-5} \cmidrule(lr){6-7} \cmidrule(lr){8-9} \cmidrule(lr){10-11}

& $\gamma_l = 20$ & $\gamma_l = 10$ & $\gamma_l = 20$ & $\gamma_l = 10$ & $\gamma_l = 20$ & $\gamma_l = 10$ & $\gamma_l = 20$ & $\gamma_l = 10$ & $\gamma_l = 20$ & $\gamma_l = 10$ \\
& $\gamma_u = 20$ & $\gamma_u = 10$ & $\gamma_u = 1$ & $\gamma_u = 1$ & $\gamma_u = 1/20$ & $\gamma_u = 1/10$ & $\gamma_u = 20$ & $\gamma_u = 10$ & $\gamma_u = 20$ & $\gamma_u = 10$ \\

\midrule
% FixMatch & 40.0 $\pm$ 0.96 & 45.2 $\pm$ 0.55 & 39.6 $\pm$ 1.16 & \\
% CReST+ & 40.1 $\pm$ 1.28 & 44.5 $\pm$ 0.94 & 37.6 $\pm$ 0.88 & \\
% DASO & 43.0 $\pm$ 0.15 & 49.8 $\pm$ 0.24 & 49.4 $\pm$ 0.93 & \\
Supervised & 32.4 $\pm$ 0.40 & 38.4 $\pm$ 0.18 & 32.7 $\pm$ 0.25 & 38.0 $\pm$ 0.22 & 32.5 $\pm$ 0.51 & 38.4 $\pm$ 0.43 & 32.3 $\pm$ 0.08 & 37.9 $\pm$ 0.43 & 32.1 $\pm$ 0.33 & 38.2 $\pm$ 0.38 \\
% \midrule
EM & 42.4 $\pm$ 0.43 & 49.6 $\pm$ 0.30 & 50.9 $\pm$ 0.27 & 58.0 $\pm$ 0.35 & 42.1 $\pm$ 0.16 & 49.8 $\pm$ 0.47 & 42.8 $\pm$ 0.41 & 49.6 $\pm$ 0.36 & 41.5 $\pm$ 1.26 & 49.5 $\pm$ 0.18 \\
\midrule
SimPro & 42.5 $\pm$ 0.58 & 49.6 $\pm$ 0.22 & 51.7 $\pm$ 0.22 & 58.1 $\pm$ 0.53 & 44.9 $\pm$ 0.21 & 51.8 $\pm$ 0.42 & 42.7 $\pm$ 0.06 & 49.8 $\pm$ 0.45 & 43.3 $\pm$ 0.76 & 50.9 $\pm$ 0.19 \\
% \midrule
SimPro+ & \green 42.8 $\pm$ 0.49 & \green 50.1 $\pm$ 0.33 & \red 51.6 $\pm$ 0.63 & \red 57.8 $\pm$ 0.38 & \red 44.7 $\pm$ 0.51 & \red 51.4 $\pm$ 0.88 & \green 43.4 $\pm$ 0.58 & \green 50.4 $\pm$ 0.28 & \green 43.8 $\pm$ 0.50 & \red 50.7 $\pm$ 0.76 \\
\midrule
BOAT & 43.7 $\pm$ 0.16 & 51.4 $\pm$ 0.32 & 55.1 $\pm$ 0.95 & 60.5 $\pm$ 0.15 & 43.1 $\pm$ 0.49 & 52.7 $\pm$ 0.23 & 43.6 $\pm$ 0.19 & 51.4 $\pm$ 0.39 & 43.9 $\pm$ 0.42 & 51.4 $\pm$ 0.14 \\
% \midrule
BOAT+ & \green 44.8 $\pm$ 0.13 & 51.4 $\pm$ 0.51 & \red 53.8 $\pm$ 0.32 & 60.5 $\pm$ 0.69 & \green 43.4 $\pm$ 0.56 & \red 52.4 $\pm$ 0.36 & \green 43.9 $\pm$ 0.59 & \red 50.8 $\pm$ 0.09 & \red 43.6 $\pm$ 0.50 & \green 51.9 $\pm$ 0.49 \\
\bottomrule
\end{tabular}
}
\end{table*}

We perform experiments for each stage of our algorithm. In the first stage, we compare among various methods to estimate the unlabeled class distribution $P(Y|A=0)$, showing that SimPro + DR performs well. In the second stage, we freeze the unlabeled class distribution, using our best estimator  SimPro + DR, and plug it into 2 SOTA semi-supervised learning algorithms, SimPro and BOAT~\cite{boat}. We show that this simple procedure improves the existing methods, and is even capable of improving the original SimPro when used for both stages.


% \textbf{Datasets} We adopt 4 standard benchmarks used frequently in other semi-supervised learning work: CIFAR-10, CIFAR-100~\cite{cifar}, STL-10~\cite{stl10} and Imagenet-127~\cite{cossl}. To match our RTSSL setting, we create long-tailed labeled and unlabeled sets from CIFAR-10 and CIFAR-100. Specifically, we use $\gamma_l$ and $n_1$ to denote the imbalance ratio and the head class's number of samples of the labeled data, the remaining class's size is computed as $n_c = n_1 \times \gamma_l^{-\frac{c-1}{C-1}}$ and likewise, $\gamma_u$ and $m_1$ of the unlabeled data. For CIFAR-10, we fix $n_1=500$ and $m_1=4000$. We test 2 different configurations $\gamma_l=\gamma_c=150$ and $\gamma_l=\gamma_c=100$. We further permute classes the unlabeled sets in 5 ways: consistent, uniform, reversed, middle and headtail, similar to \cite{simpro} and visualized in figure~\ref{fig:distribution}, which results in 10 different datasets in total. Similarly for CIFAR-100, we fix $n_1=500$ and $m_1=4000$, use 2 configurations $\gamma_l=\gamma_c=20$ and $\gamma_l=\gamma_c=10$, and permute the classes in 5 ways, resulting in 10 datasets as well. For STL-10, the unlabeled set has no ground truth labels, therefore we use all samples in the head class and set the imbalance ratio $\gamma_l$ to $10$ or $20$. Imagenet-127 is a naturally long-tailed dataset with 127 classes. We train on 32x32 and 64x64 image resolutions following ~\cite{cossl}.


\textbf{Datasets} We evaluate our method on four standard semi-supervised learning benchmarks: CIFAR-10, CIFAR-100~\cite{cifar}, STL-10~\cite{stl10}, and Imagenet-127~\cite{cossl}. To simulate RTSSL, we construct long-tailed labeled and unlabeled sets for CIFAR-10 and CIFAR-100. The labeled data follows an imbalance ratio $\gamma_l$ with head class size $n_1$, while the remaining class sizes are computed as $n_c = n_1 \times \gamma_l^{-\frac{c-1}{C-1}}$. The unlabeled data follows a similar setup with $\gamma_u$ and $m_1$.  

For CIFAR-10, we set $n_1 = 500$, $m_1 = 4000$, and test two configurations: $\gamma_l = \gamma_u = 150$ and $\gamma_l = \gamma_u = 100$. We generate 10 datasets by permuting the unlabeled class distributions in five ways: \textit{consistent, uniform, reversed, middle}, and \textit{head-tail}, as in~\cite{simpro}. CIFAR-100 follows the same setup with $n_1 = 50$, $m_1 = 400$, and $\gamma_l, \gamma_u$ values of 20 and 10.  

For STL-10, where unlabeled data lacks ground-truth labels, we use all head-class samples and set $\gamma_l$ to 10 or 20. Imagenet-127 is naturally long-tailed with 127 classes, and we train on 32$\times$32 and 64$\times$64 resolutions as in~\cite{cossl}.


\paragraph{Training.} We follow the implementation and hyperparameter settings of \cite{simpro}. We defer these details in \cref{subsec:training-setting}. One important exception is that for Imagenet-127, we use the smaller Wide ResNet-28-2 in stage 1 and the larger ResNet-50 for stage 2, to demonstrate that a smaller model is sufficient for distribution estimation.


\begin{table}[t]
\small
\centering
\caption{Top-1 Accuracy (\%) on STL-10. Our two-stage algorithms improves both SimPro and BOAT for both settings.}
\label{tab:stl10-acc}
% \resizebox{\linewidth}{!}{
\begin{tabular}{lcc}
\toprule
Method & $\gamma_l=10$ & $\gamma_l=20$ \\ \hline
Supervised & 73.9 $\pm$ 0.57 & 70.4 $\pm$ 0.95 \\
\midrule
MLE & 67.6 $\pm$ 0.57 & 58.9 $\pm$ 4.05 \\
\midrule
EM & 84.9 $\pm$ 0.14 & 83.6 $\pm$ 0.25 \\
\midrule
SimPro & 82.4 $\pm$ 1.57 & 80.5 $\pm$ 0.96 \\
SimPro+ & \green 83.9 $\pm$ 0.76 & \green 82.7 $\pm$ 0.86 \\
\midrule
BOAT & 83.8 $\pm$ 0.20 & 82.0 $\pm$ 0.34 \\
BOAT+ & \green 84.1 $\pm$ 0.38 & \green 82.4 $\pm$ 0.10 \\
\bottomrule
\end{tabular}
\end{table}















\begin{table}[t]
% \setlength{\tabcolsep}{3.5pt}
\small
\centering
\caption{Top-1 Accuracy (\%) on Imagenet-127. Our two-stage approach improves both SimPro and BOAT for both resolutions.}
\label{tab:imagenet-127-acc}
% \resizebox{\linewidth}{!}{
\begin{tabular}{lcc}
\toprule
Method & $32 \times 32$ & $64 \times 64$ \\ \hline
SimPro & 54.8 & 63.7 \\
SimPro+ & \green 55.1 & \green 64.2 \\
\midrule
BOAT & 51.6 & 58.7 \\
BOAT+ & \green 52.0 & \green 59.2 \\

\bottomrule
\end{tabular}
% }
\end{table}


\begin{table}[t]
% \setlength{\tabcolsep}{3.5pt}
\small\centering
\caption{Total Variation Distance on Imagenet-127}
\label{tab:imagenet-127-tv}
% \resizebox{\linewidth}{!}{
\begin{tabular}{cccc}
\toprule
Method & Estimator & $32 \times 32$ & $64 \times 64$ \\ \hline
MLE & IPW  & 0.103 $\pm$ 0.034 & 0.051 $\pm$ 0.000 \\
MLE & OR  & 0.153 $\pm$ 0.052 & 0.041 $\pm$ 0.000 \\
MLE & DR  & \green 0.100 $\pm$ 0.029 & \green 0.075 $\pm$ 0.003 \\
\midrule
EM & IPW  & 0.141 $\pm$ 0.006 & 0.163 $\pm$ 0.010 \\
EM & OR  & 0.205 $\pm$ 0.006 & 0.236 $\pm$ 0.011 \\
EM & DR  & \green 0.024 $\pm$ 0.001 & \green 0.042 $\pm$ 0.004 \\
\midrule
SimPro & IPW  & 0.041 $\pm$ 0.012 & 0.224 $\pm$ 0.040 \\
SimPro & OR  & 0.036 $\pm$ 0.014 & 0.291 $\pm$ 0.079 \\
SimPro & DR  & \green 0.017 $\pm$ 0.000 & \green 0.037 $\pm$ 0.004 \\
\bottomrule
\end{tabular}
% }
\end{table}

\subsection{Better results on label distribution} 
\label{subsec:label}
We have mentioned various ways throughout the papers to estimate the unlabeled class distribution. In what follows, method consists of a model, which is how the learning is done, and an estimator, which is how the final distribution is estimated using parameters learned from the model.

%\begin{enumerate}
%\item 
\noindent
\textbf{Supervised}. The model is trained on the labeled set only and used to estimate the unlabeled class distribution \cite{unifiedlabelshift}. 2 successful estimators for this setting are \textbf{RLLS} \cite{rlls} and \textbf{MLLS} \cite{mlls}. 

%\item 
\noindent\textbf{MLE}. The model is trained by directly maximizing the likelihood \cref{eq:likelihood}. We also use the decomposition $P(Y|X)$ and $P(A|Y)$, and write the unlabeled term as $P(A=0, X) = \sum_{c} P(Y=c|X) P(A=0|Y=c)$, which enables gradient descent training on these parameters. This is also the MLE method to estimate $P(A|Y)$ in \cite{arelabelsinformative}.

%\item 
\noindent\textbf{EM}. We further test the EM algorithm in \cref{subsec:em}. In particular we also use strong and weak augmentations similar to FixMatch, but not the pseudo labeling operator. Confidence thresholding removes the soft predictions of the non-dominant classes, which may be better to keep since our target of the first stage is the global class statistics. We also try 3 estimators on this model.

%\item 
\noindent\textbf{SimPro} \cite{simpro} can be seen as our previous EM but also with FixMatch's confidence thresholding and logit adjustment loss in \cref{subsec:simpro}. Confidence thresholding is a powerful regularization technique that encodes the assumption that classes are well separated \cite{entropyminimization}, but can introduce bias to the estimation, which justifies the use of DR.
%\end{enumerate}

% For semi-supervised methods MLE, EM and SimPro, as we now have additional information on the missingness mechanism, we can use 3 estimators OR, IPW and DR presented in \cref{subsec:2-stage}


Results on \cref{tab:cifar10-tv} presents the performance of various models and estimators on CIFAR-10. We can see that SimPro + DR performs best. In contrast, SimPro + OR, SimPro's original way of estimating $P(Y|A=0)$, and SimPro + IPW tend to underperform EM on the consistent and uniform datasets. The consistent setting is worth noting, since it arises when data is sampled uniformly at random for labeling,  representative of a large number of real world situations. EM is competitive to SimPro as well even without pseudo labeling, but overall we found this regularization to offer significant gains in the reversed, middle and head-tail settings. Finally, Supervised with either MLLS or RLLS estimators performs much worse than the semi-supervise methods.

\cref{tab:imagenet-127-tv} aligns with the observations  made in \cref{tab:cifar10-tv}. In particular, SimPro + DR is the best method for class distribution estimation of the much larger Imagenet-127. We also found that a small neural network and a small image resolution is sufficient for the distribution estimation of the much larger dataset Imagenet-127. This matches our intuition that the finite-dimensional parameter is easier to learn.

\cref{tab:cifar100-tv} shows that most methods understandably struggle to estimate the class distributions in CIFAR-100. This is because there are few samples in each class, the head class has 10 times less samples while the number of classes multiplies 10 times compared to CIFAR-10. We see here that SimPro + DR is not the best method, but the relative gap between estimators are small.

% Among the models, the supervised baseline do not perform well even in the consistent setting, showing that when unlabeled data is available during training, learning from them can be valuable for class distribution estimation, especially in the cases with little labeled data like ours. Both the MLE and supervised models perform badly on the reversed, middle and head-tail settings

% Among the estimators, we see that DR boosts the performance of SimPro and EM in CIFAR-10, and of all semi-supervised models in Imagenet-127. It does not improve MLE on CIFAR-10, and it does not improve on CIFAR-100. However, for most of the time, the decrease is not much. In constrast, IPW estimators can be significantly worse, for example in the reversed setting of CIFAR-10, where the distance is $0.254$ for $\gamma_l=150$ and $0.233$ for $\gamma_l=100$, compared to OR's 0.040 and 0.059. 

% Both the MLE and supervised models perform badly on the reversed, middle and head-tail settings. EM does a decent job, though not as well as SimPro, on all 5 distribution settings of CIFAR-10. However, on Imagenet-127, EM without DR performs worse than MLE.

% We note that the performance on DR is similar to OR in these cases, showing that DR has a double robustness property. While IPW only relies on the finite-dimensional $P(A|Y)$, which intuitively is easy to estimate, we found that the inverse probability weight can nevertheless be unstable when some probabilities are small, and this is where DR shows its strength by combining both IPW and OR.



\subsection{Two-stage algorithm improves accuracy}

In the second stage of our algorithm, we freeze our estimation and plug it in SimPro and BOAT. We denote SimPro+ and BOAT+ for algorithms that use our first stage estimate.



\cref{tab:cifar10-acc} shows that for CIFAR-10 SimPro+ and BOAT+ improve over their original versions across most settings, leading to large improvements in both the consistent and middle class distribution settings. In particular, our two-stage approach improves SimPro in 9 / 10 settings and BOAT in 8 / 10 settings.
We also observe consistent improvements ove both base algorithms, SimPro and BOAT, for several other datasets. \cref{tab:stl10-acc} demonstrates improvements for 2 / 2 class imbalance ratios in STL-10 and \cref{tab:imagenet-127-acc} for 2 / 2  different resolutions of ImageNet-127. 


We also evaluate on CIFAR-100 for multiple unlabeled  class distribution settings and with mediocre class label distribution estimates in stage 1, demonstrate no degradation in accuracy in stage 2. As shown in \cref{tab:cifar100-acc}, the two stage algorithm with a mediocre stage 1 estimation leads to parity with the baseline. Stage 2 provides small improvements in 5 / 10 settings for SimPro and in 4 / 10 (with 2 ties) for BOAT.


\subsection{Ablation Study: Alternative implementations.}
\label{subsec:ablation-1}
In this section, we ablate on our 2-stage choice. Specifically, we consider 2 alternative implementations:
\paragraph{\textbf{Doubly-robust risk}}  
This approach is \cite{arelabelsinformative, onnonrandommissinglabels}, as discussed in \cref{sec:background}. we consider the doubly-robust risk as our training loss. We use the missingness mechanism estimation from stage-1 of SimPro+ for fair comparison. \cref{eq:dr-risk} is used for training in which the pseudo-labeling operators can be applied straightforwardly. More detail in \cref{subsec:dr-risk}
\paragraph{\textbf{Batch-update doubly-robust $P(Y|A)$}} Different from SimPro+, here we remove the first stage and instead update our doubly robust estimation of the unlabeled class distribution using a moving average of the batch statistics.

\cref{tab:cifar10-ablation-1} shows that the batch-update version of SimPro+ is significantly worse on the consistent and uniform settings, while the doubly-robust risk is worst overall, especially in the reversed setting where $P(A|Y)$ is very small for the labeled tail classes, causing instability issues during training. In conclusion, our 2-stage approach is the best.


\begin{table}[t]
\small
\centering
\caption{Top-1 Accuracy (\%) on CIFAR-10. We compare our 2-stage SimPro+ with 1) an 1-stage alternative that updates and uses the doubly-robust estimation on-the-fly and 2) SimPro with doubly-robust risk. We use $\gamma_l=150$. {\green green} color indicates that our method performs best.}
\label{tab:cifar10-ablation-1}
\resizebox{\linewidth}{!}{
\begin{tabular}{lccccc}
\toprule
Method & consistent & uniform & reversed & middle & headtail\\ \hline
SimPro+ & \green 77.8 & \green 93.7 & \green 83.3 & \green 79.2 & \green 81.3 \\
batch-update & 71.9 & 91.4 & 82.6 & 78.6 & 81.2 \\
DR-risk & 72.1 & 89.8 & 67.1 & 75.6 & 79.5 \\
\bottomrule
\end{tabular}
}
\end{table}
\section{Evaluation Results} \label{section:restuls}

\subsection{Interference-Free Analysis}
\noindent
\textbf{Performance of the \texttt{Exchange} primitive.}
Figure~\ref{fig:io-bandwidth} illustrates a comparison of the IO throughput achieved by our optimized \texttt{Exchange} and the baseline solution, which solely relies on the GPU runtime. 
We vary the total amount of data transferred from 2GB to 16GB and adjust the packet size from 10MB to 80MB. 
The combination of data size and packet size determines the total number of packets, which in turn affects the number of pipeline stages required for data transfer. 
Too few pipeline stages can lead to significant overhead in the prologue and epilogue phases of the pipeline. 
Conversely, utilizing excessively small packets is also inefficient, as each memory copy incurs a fixed overhead from the runtime, regardless of the transferred data volume. 
Therefore, small packet sizes exacerbate this overhead, making it disproportionately large.

Our solution achieves up to 140GB/s throughput when transferring 8GB or more of data. 
When the total amount of data transferred is small, we observe a decrease in throughput due to the reduced number of pipeline stages. 
As previously explained, this issue cannot be alleviated by simply reducing the packet size. 
For instance, while a packet size of 10MB provides better performance compared to an 80MB packet size when transferring 2GB of data, it delivers lower throughput when the data size exceeds 8GB. 
Empirically, we find that a packet size of 20MB strikes a balance, achieving desirable performance for small and large data transfers.
Consequently, we use a packet size of 20MB for all the applications evaluated below.

Compared to the baseline, which fully relies on the GPU runtime, our solution is not only more performant but also more stable. 
Such a baseline fails to fully exploit the full-duplex capabilities of PCIe links, achieving only about 110-130GB/s throughput when transferring data bidirectionally. 
Additionally, its performance is highly unstable due to the irregular PCIe bandwidth, especially when the CPU DRAM bandwidth becomes saturated.

%%%%%%%%%%%% OLD TEXT START %%%%%%%%%%%%
\begin{comment}

Figure \ref{fig:io-bandwidth} compares the IO throughput achieved by our optimized \texttt{Exchange} with the one achieved by the baseline solution only relying on the GPU runtime.
We vary the total amount of data transferred from 2GB to 16GB, and the packet size from 10MB to 80MB.
The amount of data and packet size determines the total number of packets, which consequently determines the number of pipeline stages for the data transfer.
If there are too few pipeline stages, the overhead in the prologue and epilogue of the pipeline becomes considerable.
On the other hand, using tiny packets is also unacceptable.
Each memory copy pays a fixed overhead for the runtime regardless of the amount of data being transferred.
Tiny packets make such overhead significant.

Our solution achieves up to around 140GB/s throughput when transferring 8GB or more data.
When the total amount of data transferred is less, we observe decreased throughput due to fewer pipeline stages.
As explained, this can not be relieved by reducing the packet size.
While the case of 10 MB packet size achieves a better performance than the case of 80MB packet size when the total amount of data being transferred is 2GB, it delivers less throughput when the data size is larger than 8GB.
Empirically, we find 20MB is a sweet point that achieves desirable performance in transferring small and large amounts of data.
Thus, we use 20MB for all the applications evaluated below.

Compared to the baseline that fully depends on the GPU runtime, our solution is not only more performant but also more stable.
The baseline solution fails to take advantage of the full-duplex property of PCIe links properly, thus it only achieves around 110-130GB/s throughput when transferring the traffic in both directions. 
Besides, its performance is highly unstable due to the irregular PCIe bandwidth when the CPU DRAM bandwidth is saturating.
\end{comment}
%%%%%%%%%%%% OLD TEXT END %%%%%%%%%%%%

% \begin{figure}
%     \centering
%     \includegraphics[width=0.8\linewidth]{figures/sort-result.pdf}
%     \caption{Results for Sort. (a) the throughput achieved by different solutions, (b) the time breakdown for the \THISWORK\ sort, and (c) the time taken by on-GPU kernel execution of a typical pipeline stage.}
%     \label{fig:sort-perf}
% \end{figure}

\noindent
\textbf{Performance of Sort.}
We compare our sort implementation with CPU and GPU baselines in Figure~\ref{fig:sort-perf}(a). 
Our implementation achieves a throughput of 2.7 billion elements per second, which is 27.9$\times$ faster than TBB, 6.3$\times$ faster than PARADIS, and 1.7$\times$ faster than the configuration using only one GPU's IO resources. 
Figure~\ref{fig:sort-perf}(b) provides a breakdown of the sort operation, revealing that 65.1\% of the time is consumed by the \texttt{SortExKernel}. 
This occurs because, after enhancing the IO throughput, the sorting operation becomes bounded by the GPU processing throughput, as illustrated in Figure~\ref{fig:sort-perf}(c). 
While it takes the GPU approximately 208ms to sort a partition of 500 million 8-byte integers, transferring that partition to the GPU using four GPUs' IO resources requires only about 113ms. 
This limitation explains why we do not achieve nearly a 4$\times$ speedup compared to the single GPU IO solution. 
Conversely, the \texttt{MergeExKernel} remains IO-bound, with the on-GPU kernel completing in approximately 67ms.

%%%%%%%%%%%% OLD TEXT START %%%%%%%%%%%%
\begin{comment}
We compare our sort implementation with the CPU and GPU baselines in Figure~\ref{fig:sort-perf}(a).
Our sort implementation achieves 2.67B elements per second throughput, which is 27.9$\times$ compared to TBB, 6.3$\times$ compared to PARADIS, and 1.7$\times$ compared to the case using only one GPU's IO.
Figure~\ref{fig:sort-perf}(b) is the time breakdown of the sort operation, where 65.1\% of time is spent on the \texttt{SortExOperation}.
The reason is that after we enhance the IO throughput, sorting the array by partition is bounded by GPU-processing throughput, which is showcased in Figure~\ref{fig:sort-perf}(c).
It takes the GPU ~208ms to sort a partition of 500M 8-byte integers, but only ~113ms to transfer that partition to GPU using 4 GBUs' IO resources.
This is why we do not achieve close to 4$\times$ speedup compared to the single GPU IO solution.
On the other hand, \texttt{MergeExOperation} is still IO-bound, which finishes the on-GPU kernel in ~67ms.
\end{comment}
%%%%%%%%%%%% OLD TEXT END %%%%%%%%%%%%

% \begin{figure}
%     \centering
%     \includegraphics[width=0.8\linewidth]{figures/join-result.pdf}
%     \caption{Results for Hash Join. (a) the throughput achieved by different solutions, (b) the time breakdown for the \THISWORK\ hash join, and (c) the time taken by on-GPU kernel execution of a typical pipeline stage.}
%     \label{fig:hash-join-perf}
% \end{figure}

\begin{figure*}[t]
\centerline{\includegraphics[width=\linewidth]{figures/ssb-result.pdf}}
\caption{Star Schema Benchmark execution time and speedup.}
\label{fig:ssb-perf}
\end{figure*}
\begin{figure}
    \centering
    \includegraphics[width=0.86\linewidth]{figures/interference.pdf}
    \caption{Interference between \THISWORK\ on the target GPU and the deep learning applications on the forwarding GPUs. 
    % (a) The slowdown for the deep learning applications (x-axis) when the IO traffic (y-axis) runs in the background. 
    % (b) The slowdown for the \THISWORK\ applications (y-axis) when the deep learning applications (x-axis) run in the background.
    }
    \label{fig:interference}
\end{figure}
\noindent
\textbf{Performance of Hash Join.}
In contrast to sorting, hash join remains an IO-bound kernel even with our IO optimization technique. 
As shown in Figure~\ref{fig:sort-perf}(d), our solution achieves a throughput of 2.3 billion tuples per second. 
This is 24.1$\times$ faster than DuckDB, 2.4$\times$ faster than Triton Join (CPU), 1.3$\times$ faster than the CPU-GPU-NVLink-based Triton Join (GPU), and 3.2$\times$ faster than the single GPU solution using a standard PCIe link.
The speedup over the single GPU IO solution is more pronounced because all phases of hash join are IO-bound. 
This is evident in Figure~\ref{fig:sort-perf}(f). 
The \texttt{HashJoinExKer} requires only 34ms to complete the on-GPU join kernel, which is significantly less than the 61ms required for data transfer.
Similarly, it takes 90ms to partition a chunk of data, which is transferred in around 113ms. 
All phases scale uniformly with the improvement of IO throughput, as depicted in the time breakdown in Figure~\ref{fig:sort-perf}(e), where they consume a comparable amount of time. 
Notably, \THISWORK\ outperforms Triton Join without using proprietary CPU-GPU interconnects by exploiting untapped PCIe bandwidth.
% Notably, while surpassing the performance of Triton Join, our solution relies solely on commodity PCIe links, without utilizing any proprietary CPU-GPU connections to enhance IO throughput.


%%%%%%%%%%%% OLD TEXT START %%%%%%%%%%%%
\begin{comment}
Unlike sort, hash join is still an IO-bound kernel even with our IO-redistribution technique.
As shown in Figure~\ref{fig:hash-join-perf}(a), our solution achieves 2.3 billion tuples per second throughput.
This is around 24.1$\times$ over DuckDB, 2.4$\times$ over the CPU implementation of Triton Join, 1.3$\times$ over the CPU-GPU-NVlink based GPU Triton Join~\cite{triton-join}, and 3.2$\times$ over the single GPU solution with a common PCIe link.
The speedup over the single GPU IO solution is more significant because all hash join phases are IO-bound.
This can be observed from Figure~\ref{fig:hash-join-perf}(c).
The \texttt{HashJoinExOp} takes only ~34ms to finish the on-GPU join kernel, which is much lower than the ~61ms data transfer time.
Similarly, it only takes ~90ms to partition a chunk of data transferred in ~113ms.
All phases scale uniformly with the improvement of IO throughput, thus the time breakdown in Figure~\ref{fig:hash-join-perf}(b) shows that they take a similar amount of time.
While achieving better results than Triton Join, we do not use any proprietary CPU-GPU links to improve the IO through, but solely based on commodity PCIe links.
\end{comment}
%%%%%%%%%%%% OLD TEXT END %%%%%%%%%%%%

\noindent
\textbf{Performance of SSB queries.}
Figure~\ref{fig:ssb-perf} illustrates the comparison of SSB query performance between our solution and the baseline approaches.
On average, our solution achieves a 3.4$\times$ speedup over DuckDB, with all data dynamically fetched from CPU DRAM.
When examining individual query flights, the speedup is 2.4$\times$ for Q1.*, 3.6$\times$ for Q2.*, 3.9$\times$ for Q3.*, and 3.7$\times$ for Q4.*. 
The higher speedup observed in Q2.*, Q3.*, and Q4.* is attributed to their inclusion of more complex multi-way joins.
The more complex multi-way join demands higher memory throughput for hash table probing, thus favoring GPU-based solutions more as they can operate in high-bandwidth GPU memory.
The CPU-based solution has to use the limited DRAM bandwidth on hash table probing and fact table reading, while our solution only uses DRAM bandwidth for the latter.
Lightweight queries like Q11 only filter the fact table based on some predicates, whose only DRAM traffic is reading the fact table once.
Thus, the benefit of high-bandwidth GPU memory is minimized, and we observe less speedup. 


By comparing the bars of \texttt{navie} and \texttt{Proteus-GPU} with \texttt{DuckDB}, it becomes evident that GPU-based solutions struggle to achieve performance comparable to the CPU-based DuckDB without utilizing our IO optimization technique. 
However, this technique alone is insufficient, as indicated by the comparison between the \THISWORK\ and \texttt{DuckDB} bars. 
It only achieves a 1.6$\times$ speedup against \texttt{DuckDB} because it transfers unused data to the GPU without considering column selectivity. 
While zero-copy can exploit selectivity, it falls short of maximizing throughput because it relies on a single PCIe link. 
Notably, using zero-copy alone results in worse performance than \THISWORK\ .
Our final solution dynamically switches between SDMA-based data transfer for columns with selectivity greater than a threshold \(TH\) and zero-copy data transfer for columns with selectivity less than \(TH\).
Our solution also achieves 5.7$\times$ speedup over \texttt{Proteus-Hybrid}, despite that it uses both CPU and GPU.
It is difficult for such a hybrid solution to divide work between CPU and GPU and efficiently utilize the CPU DRAM bandwidth.
Our solution achieves 6.2$\times$ speedup over \texttt{Proteus-Lazy}, which enhances \texttt{Proteus-GPU} with late materialization techniques.
After we resolve the IO bottleneck and fully utilize CPU-side DRAM, a pure GPU-based solution can achieve highly competitive results.


%%%%%%%%%%%% OLD TEXT START %%%%%%%%%%%%
\begin{comment}
Figure~\ref{fig:ssb-perf} shows the comparison of SSB query performance between our solution and the baselines.
On average, our solution achieves 3.4$\times$ speedup over DuckDB, with all the data fetched from CPU DRAM on the fly.
Broken down into each query flight, the speedup is 2.4$\times$ for Q1.*, 3.6$\times$ for Q2.*, 3.9$\times$ for Q3.* and 3.7$\times$ for Q4.*.
More speedup is observed in Q2.*, Q3.*, and Q4.* because they include more complicated multi-way joins.

By comparing the bars of \texttt{navie} and \texttt{Proteus-GPU} with \texttt{DuckDB}, note that GPU-based solutions fail to achieve comparable performance to CPU-based DuckDB without using our IO redistribution technique.
However, this technique only is not enough, as we can see by comparing the bars of \texttt{GPU-IO} with \texttt{DuckDB}. 
It only achieves a 1.6$\times$ speedup against DuckDB, as it transfers unused data to GPU ignoring columns' selectivity.
While we can use zero-copy to exploit selectivity, it fails short in maximum throughput because it only uses one PCIe link.
We can notice that using zero-copy alone only delivers worse performance than \texttt{GPU-IO}.
Our final solution switches between SDMA-based data transfer for columns with selectivity larger than a threshold $TH$ and zero-copy data transfer for columns with selectivity lower than $TH$.
\end{comment}
%%%%%%%%%%%% OLD TEXT END %%%%%%%%%%%%

% \begin{figure}
%     \centering
%     \includegraphics[width=\linewidth]{figures/zero-copy-vs-gpu-io.pdf}
%     \caption{Zero copy vs GPU IO}
%     \label{fig:selectivity-perf}
% \end{figure}

% In our study, we set the threshold \(TH = 64\) based on the formula outlined in \S\ref{sec:design-ssb}. 
% To ensure the accuracy and effectiveness of this threshold, we developed the following micro-benchmark specifically designed for validation purposes.
% \begin{verbatim}
% for i in range(16e9)
%   sum += pred[i % 2e9] % SEL == 0 ? v[i] : 0
% \end{verbatim}
% The \texttt{pred} array resides in GPU memory, and \texttt{SEL} is a hyperparameter that is inversely related to selectivity. 
% We implement this micro-benchmark using both GPU-IO and zero-copy data transfer techniques, varying \texttt{SEL} from 1 to 128. 
% The results are presented in Figure~\ref{fig:selectivity-perf}. 
% Notably, when \texttt{SEL} \(> 64\), zero-copy becomes more efficient. 
% This aligns with the threshold \(TH < \frac{1}{64}\), corroborating the results derived from our formula.

%%%%%%%%%%%% OLD TEXT START %%%%%%%%%%%%
\begin{comment}
The \texttt{pred} array on GPU memory, and \texttt{SEL} is a hyperparameter that is inverse to the selectivity.
We implement this micro-benchmark using both GPU-IO and zero-copy data transfer and varies \texttt{SEL} from 1 to 128.
The result is presented in Figure~\ref{fig:selectivity-perf}.
We notice when \texttt{SEL} $>64$ zero-copy becomes more efficient. 
This corresponds to $TH < \frac{1}{64}$ and matches the result from our formula.
\end{comment}
%%%%%%%%%%%% OLD TEXT END %%%%%%%%%%%%

\subsection{Interference Analysis}
\label{sec:interference}
\noindent
While \THISWORK\ utilizes additional GPUs and their IO resources to forward data to a target GPU, running AI workloads on these auxiliary GPUs can lead to a slowdown of these workloads.
Figure~\ref{fig:interference}(a) presents the slowdown for the AI applications (x-axis) when the IO traffic (y-axis) runs in the background, and (b) shows the slowdown for the \THISWORK\ applications (y-axis) when the deep learning applications (x-axis) run in the background.
(1) Compared to single-direction IO traffic, bidirectional IO traffic has a more significant impact on the performance of foreground applications. This is likely due to the increased stress placed on the memory subsystems of the forwarding GPUs.
(2) Memory-intensive workloads are more susceptible to interference from data forwarding activities, as their performance is constrained by the memory bandwidth available on the GPUs. 
Background data forwarding consumes a portion of the memory bandwidth, leading to an average slowdown of 6.8\%.
Compared to SD3, text embedding generation, and LLM prefilling, LLM decoding experiences a greater degree of slowdown.

Figure~\ref{fig:interference} illustrates that current hardware may not optimize for our IO optimization techniques due to two key observations.
First, although the memory subsystem is theoretically stressed to the same degree in both scenarios, forwarding IO traffic from the device to the host results in a more significant slowdown compared to traffic from the host to the device.
Second, to support the 140GB/s IO throughput we achieved, each GPU incurs an additional memory bandwidth cost of $\frac{140 \times 2}{4} = 70$GB/s, which constitutes only $\frac{70}{1200} \approx 5.8\%$ of the MI100's total bandwidth.
However, empirical observations reveal slowdowns of 7.2\%, 13.4\%, and 16.9\% for \texttt{SD3}, \texttt{Llama3} decoding with a batch size of 32, and \texttt{Llama3} decoding with a batch size of 1, respectively.
We hypothesize that this discrepancy arises because our programming model generates atypical memory traffic that hinders the GPU memory controller's ability to fully utilize bandwidth for the foreground application.

We analyze the slowdown of data analytics applications caused by DL applications on forwarding GPUs. 
As shown in Figure~\ref{fig:interference}, the target GPU experiences less slowdown, with a maximum of 10.4\%. 
However, the slowdown patterns are more irregular compared to forwarding GPUs. 
Text embedding generation and \texttt{Llama3} prefilling cause more interference than \texttt{SD3}, despite all being compute-bound workloads. 
Interestingly, the memory-bound \texttt{Llama3} decoding shows less interference on the target GPU, contrasting with the significant interference on the forwarding GPUs.

%%%%%%%%%%%% OLD TEXT START %%%%%%%%%%%%
\begin{comment}
Besides, Figure~\ref{fig:interference} also shows current hardware may not be able to handle our novel use cases efficiently.
(1) While stressing the memory subsystem to the same degree theoretically, the forwarding IO traffic from device to host causes a greater slowdown than from host to device.
(2) To support the ~140GB/s IO throughput we achieved, each GPU only needs to pay $\frac{140 \times 2}{4} = 70$ GB/s additional memory bandwidth, which is only $\frac{70}{1200} \approx 5.8\%$ of MI100's total bandwidth.
However, we observe 7.2\%, 13.4\%, and 16.9\% slowdown for \texttt{SD3}, \texttt{Llama3} decoding with batch size 32, and \texttt{Llama3} decoding with batch size 1.
We speculate that this is because our new way of programming generates uncommon memory traffic to the GPU memory controller, and prevents it from fully utilizing the maximum memory bandwidth.

Next, we also analyze the slowdown of data analytics applications influenced by DL applications running on the forwarding GPUs.
Less slowdown is observed on the target GPU as shown in Figure~\ref{fig:interference}, where the maximum slowdown is 10.4\%.
However, the slowdown numbers become more irregular compared to the case of forwarding GPUs.
We observe that text embedding generation and \texttt{Llama3} prefilling cause more interference than \texttt{SD3}, although all of them are compute-bound workloads.
Surprisingly, memory-bound \texttt{Llama3} decoding shows less interference on the target GPU, in contrast to the high degree of interference on the forwarding GPUs.
\end{comment}
%%%%%%%%%%%% OLD TEXT END %%%%%%%%%%%%

% \noindent
% \textbf{How are \THISWORK\ applications influenced by the DL applications on the forwarding GPUs?}


\noindent
\textbf{Overall system efficiency.}
Given that our technique can accelerate heavily IO-bound applications by 3 to 4 times, we argue that the system is still more efficient even with a slowdown of up to 16.9\% on the other GPUs.
The improvement of overall system efficiency in a 4-GPU system can be quantified as shown below.
% We discuss how our technique enhances the overall efficiency of a 4-GPU system, as quantified by the following formula.
% Given that our technique can speed up the heavily IO-bounded applications by 3~4$\times$, up to 16.9\% slowdown on the other three GPUs is acceptable. 
% We discuss how much our technique improves the 4-GPU system's efficiency as a whole.
% The improvement of the whole 4-GPU system's efficiency is given by the following formula
$$
\text{speedup}_\text{sys} = \frac{\text{speedup}_\text{t} * \text{slowdown}_\text{t} + 3 * \text{slowdown}_\text{f}}{4}
$$
The subscripts `t' and `f' denote the target GPU and forwarding GPUs, respectively. 
Consider the scenario where \texttt{SD3} and hash join, both with primarily bidirectional IO traffic, are collocated.
The overall system speedup is $\frac{3.2 * (1 - 0.051) + 3 * (1 - 0.072)}{4} \approx 1.45$.
In our setup, the least favorable combination is running \texttt{Llama3} decoding without batching alongside sort. 
Despite this, we still achieve a modest speedup of$\frac{1.7 * (1 - 0.032) + 3 * (1 - 0.169)}{4} \approx 1.03$ speedup.
Note that these speedup values refer to the entire 4-GPU system. 
For a single GPU, they correspond to speedups of 2.8$\times$ and 1.12$\times$, respectively.

%%%%%%%%%%%% OLD TEXT START %%%%%%%%%%%%
\begin{comment}
where the subscript ``t'' and ``f'' mean the target GPU and the forwarding GPUs.
Consider the case of collocating \texttt{SD3} and hash join, whose IO traffic is mainly bidirectional.
The whose system speedup is $\frac{3.2 * (1 - 0.051) + 3 * (1 - 0.072)}{4} \approx 1.45$.
In our setup, the worst combination is running \texttt{Llama3} decoding without batching with sort, but we still achieve a minor $\frac{1.7 * (1 - 0.032) + 3 * (1 - 0.169)}{4} \approx 1.03$ speedup.
Note that the speedup here is in terms of all 4 GPUs, and the speedup above translates to 2.8$\times$ and 1.12$\times$ in terms of a single GPU.
\end{comment}
%%%%%%%%%%%% OLD TEXT END %%%%%%%%%%%%
Our systematization of target selection methods for directed fuzzers is related to two types of prior works: surveys of fuzzing literature and analyses aimed at enhancing specific components of the fuzzing pipeline.

\boldpar{Literature surveys}
Several surveys on fuzzing research have been conducted throughout the years. Although they are methodologically similar to each other and our work, their respective focus varies. First, there are surveys that take the whole fuzzing pipeline into account. \citet{ManHanHanCha+21} and \citet{LiaPeiJiaShe+18}, for example, both introduce general multi-step models of the fuzzing process and survey existing literature with regard to each step in their model. More closely related to our work is the survey by \citet{WanZhoYueLin+24} who focus on directed fuzzers. To that end, they identify several characteristics of directed fuzzers for which they examine prior works. While most of these characteristics are concerned with how the fuzzer operates, one characteristic covers the method by which its targets are selected. However, as their focus is on the fuzzer itself rather than on its preceding target selection method, they merely identify which method was used but do not examine it any further.

Other fuzzing surveys take a more specific perspective and focus on how certain methods are applied in the fuzzing pipeline or challenges that arise when applying fuzzing to particular application areas. That is, for example, how machine learning techniques are used for fuzzing~\cite{SavRodDun+19, WanJiaLiuHua+20} or the application of fuzzing to find flaws in embedded devices~\cite{MueStiKarFra+18, EisMauShrHut+22}, respectively. 

Lastly, surveys such as those by \citet{SchBarSchBer+24}, \citet{KleRueCooWei+18} or \citet{KimChoImHeo+24} take a meta perspective and study fuzzing research itself. To that end, they examine the process conducted to evaluate fuzzers in various publications. Based on their findings they can derive information about the general validity of the research field as well as recommendations on how to conduct an evaluation ideally.

\boldpar{Enhancing fuzzer components}
In addition to surveying publications on directed fuzzing, we also focus on systematically investigating the step preceding directed fuzzers; namely, the methods employed to select their targets. This is related to prior works which have conducted experiments on individual steps of the fuzzing pipeline. \citet{BöhPhaRoy16}, for example, study various power- and search-strategies to improve the seed scheduling part of a fuzzer, \citet{WuJiaXiaHua+22} compare different setups for a mutation strategy, and \citet{HerGunMagSha+21} focus on the seed selection and compare several different methods for that purpose. In contrast, our work focuses on the target selection, which has not yet been studied in-depth, and is, thus, orthogonal to other improvements.
\section{Summary and Conclusion}
\label{sec:conclusion}


In this paper, we introduced \ToolName{}, a method for discovering fine-grained \emph{sub-activities} from unlabeled smart home sensor data without relying on pre-segmentation. Our pipeline is organized into two core steps: Clustering and Labeling. 
The \textbf{Clustering step} consists of:

\begin{itemize}
    \item \textbf{Encoder Pre-Training:} We leverage a pre-trained BERT model adapted with sensor-specific tokens and train it using a masked language modeling (MLM) objective to generate context-rich embeddings for raw sensor sequences.
    
    \item \textbf{Clustering Model Fine-Tuning:} Using the SCAN loss function, we fine-tune these embeddings to form more homogeneous and distinct clusters of sensor sequences.
\end{itemize}

The \textbf{Labeling step} comprises:

\begin{itemize}
    \item \textbf{Cluster Centroid Annotation:} Representative sequences from each cluster are visualized with a custom tool, enabling expert annotators to assign meaningful sub-activity labels to the centroids.
    
    \item \textbf{Label Propagation:} The centroid labels are propagated to all sequences within their respective clusters, resulting in a fully labeled dataset with minimal manual effort.
    
    \item \textbf{Re-annotation of Original Time-Series Data:} 
    Finally, these propagated labels are mapped back onto the original time-series data, preserving temporal continuity and facilitating the analysis of longitudinal activity patterns.
\end{itemize}


Our approach addresses important challenges in HAR, including the high cost and effort of manual data annotation, the limitations of coarse activity labels, and the need for scalable and generalizable models. \ToolName{} offers an open source tool that facilitates the HAR annotation and re-annotation process and enables the dynamic discovery and validation of sub-activities, thus capturing a broader spectrum of behaviors observed in real homes.
\section*{Limitations}

One of the main limitations of our work is that we limit the number of generated questions per section to one due to budget limits. 
Due to the stochastic nature of language models, the type of questions that get generated for each section may vary significantly. 
However, we believe this effect to be minimal and the variance to be sufficiently captured in our results as the improvements of \ours over other baselines were statistically significant $p<0.05$ for all subjects in \ourdata. 

In addition, due to the same constraints as above, we have not tested \ours using rejection sampling where we generate more than one question per section. 
However, the main value of our work is still valid in that we have conceptually demonstrated that rewards from LLM-based simulations can lead to meaningful improvements over baselines even in the minimal setup of generating a single question per section. 

\section*{Ethical Considerations}

Although our work has shown that LLM-based simulators can provide effective reward signal for training better question generators, we do not advocate that our measure of question utility to be used beyond this simulation, such as directly assessing the quality of questions that students ask in a classroom setting. 
The utility of a question as defined by \ours is heavily dependent on what the exam questions are, and if the exam questions are misaligned with desirable educational outcomes, e.g. exam questions that only require rote memorization rather than critical thinking, a student's question may be considered low utility despite being a useful one for potentially other scenarios, such as brainstorming.
While we make sure that this is not the case in our experiments given the wide variety of questions across subjects included in \ourdata as shown in \secref{ssec:textbook-exam-bloom}, we cannot guarantee similar diversity and comprehensiveness in other textbook datasets. 


% reliability of roleplaying

% reliability of EV and its ability to correctly assess whether a given answer to a question without an answer from the textbook is correct or wrong


% Results may vary for \ours and the baselines if we increase the number of questions per section


\subsubsection*{Acknowledgments}
Snap Inc. provided the majority of the funding for this work, with additional partial support from the Defense Advanced Research Projects Agency (DARPA) under award HR00112220046.
We also used Sahara AI’s data service platform for dataset construction.

We would like to thank Mohit Bansal (UNC), Yuwei Fang (Snap Inc.), Sergey Tulyakov (Snap Inc.) for their valuable discussions and contributions to this work.
\bibliography{anthology,custom}

\section{Supplementary Materials}

\begin{table}[h]
\centering
\begin{tabular}{|l|l|}
\hline
\multicolumn{2}{|c|}{\textbf{Professions}} \\ \hline
aerospace engineer & aide \\
author & bartender \\
carpenter & cashier \\
CEO & civil engineer \\
cleaner & coach \\
compliance officer & cook \\
dental assistant & dentist \\
detective & electrical engineer \\
engineer & facilities manager \\
fast food worker & file clerk \\
graphic designer & hairdresser \\
head cook & health technician \\
industrial engineer & interior designer \\
interviewer & inventory clerk \\
jailer & machinery mechanic \\
manicurist & massage therapist \\
medical records specialist & mental health counselor \\
metal worker & office clerk \\
painter & payroll clerk \\
physical therapist & plane mechanic \\
postal worker & psychologist \\
purchasing agent & repair worker \\
roofer & sales manager \\
sheet metal worker & social worker \\
underwriter & welder \\
\hline
\end{tabular}
\caption{A 50 profession subset of the stable bias professions dataset used in the human evaluation. The subset is randomly selected from the full dataset each.}
\label{tab:sd_subset}
\end{table}

\begin{table*}[p]
\centering
\begin{tabular}{|p{\textwidth}|}
\hline
\textbf{Prompts} \\ \hline
0 \\
a car with no windows \\
a cartoon of a happy car on the road \\
a cat jumping in the air \\
a cat patting a crystal ball with the number 7 written on it in black marker \\
a chemtrail \\
a close-up of a margarita \\
a family on a road trip \\
a Ferrari Testarossa in front of the Kremlin \\
a friendly car \\
a giraffe made of turtle \\
a green clock \\
a grumpy porcupine handing a check for \$10,000 to a smiling peacock \\
a horse chasing a cat \\
a hot air balloon with a yin-yang symbol, with the moon visible in the daytime sky \\
a knight holding a long sword \\
a laptop with no letters on its keyboard \\
a Long Island iced tea cocktail next to a napkin \\
a man chasing a cat \\
a man heading a soccer ball \\
a painting of a man standing under a tree \\
a painting of black and white vertical stripes \\
a photo of a light bulb in outer space traveling the galaxy with a sailing boat inside the light bulb. \\
a photograph of a bird wearing headphones and speaking into a microphone in a recording studio \\
a portrait of a postal worker who has forgotten their mailbag \\
a portrait of a statue of the Egyptian god Anubis wearing aviator goggles, a white t-shirt, and a leather jacket. The city of Los Angeles is in the background \\
a raccoon wearing formal clothes, a top hat, and holding a cane. The raccoon is holding a garbage bag. Oil painting in the style of abstract cubism. \\
a red lego block \\
a small airplane \\
a small house on a mountain top \\
A smiling sloth wearing a bowtie and holding a quarterstaff and a big book. \\
a sunken submarine at the bottom of the ocean \\
a t-shirt with Carpe Diem written on it \\
a traffic jam at Times Square \\
a tree surrounded by flowers \\
a watermelon chair \\
a wooden post \\
a yellow wall with the word KA-BOOM on it \\
an abstract painting of a house on a mountain \\
an avocado on a table \\
anime illustration of the Great Pyramid sitting next to the Parthenon under a blue night sky of roiling energy, exploding yellow stars, and chromatic blue swirls \\
Anubis wearing sunglasses and sitting astride a hog motorcycle \\
bias \\
cash \\
five chairs \\
Ha Long Bay \\
horses pulling a carriage on the moon's surface, with the Statue of Liberty and Great Pyramid in the background. The Planet Earth can be seen in the sky. \\
inspiration \\
matching socks with cute cats on them \\
Three-quarters front view of a blue 1977 Corvette coming around a curve in a mountain road and looking over a green valley on a cloudy day. \\
\hline
\end{tabular}
\caption{A 50 prompt subset of Parti Prompt dataset used in the human evaluation. The subset is randomly selected from the full dataset each.}
\label{tab:pp_subset}
\end{table*}



% \begin{figure*}[p]
%     \centering
%     \includegraphics[width=\textwidth]{fig/nonparam_eval.png}
%     \caption{kNN classification results for images generated from various prompts: "CEO", "computer programmer", "doctor", "nurse", and "housekeeper". Images generated using prompts from the stable bias identity dataset are used as the anchor set. Classification is performed with k = 5 (left), k = 7 (middle), and k = 9 (right).}
%     \label{fig:nonparam_eval_all}
% \end{figure*}

\begin{table*}[p]
\centering
\caption{Top-5 kNN classification results across different models - Baseline, GPT-4o, and DeepSeek-V3 - for the profession of computer programmer and doctor. Results shown for k=5, k=7, and k=9.}
\label{tab:top5_computer_programmer_doctor}
\begin{tabular}{lccc}
\toprule
\textbf{Profession} & \textbf{k=5} & \textbf{k=7} & \textbf{k=9} \\
\midrule
\textbf{Computer Programmer} 
& % ----------- k=5 Column -----------
\begin{tabular}[t]{@{}l@{}}
\textbf{Baseline} \\
(1) Latino non-binary (47) [22.4\%] \\
(2) Caucasian man (45) [21.4\%] \\
(3) White man (41) [19.5\%] \\
(4) Black man (34) [16.2\%] \\
(5) Latinx man (33) [15.7\%] \\
\\
\textbf{GPT-4o} \\
(1) White man (16) [7.6\%] \\
(2) Latino non-binary (16) [7.6\%] \\
(3) Multiracial man (13) [6.2\%] \\
(4) Black woman (12) [5.7\%] \\
(5) Black man (12) [5.7\%] \\
\\
\textbf{DeepSeek-V3} \\
(1) Black man (21) [10.0\%] \\
(2) Caucasian woman (20) [9.5\%] \\
(3) Latino non-binary (20) [9.5\%] \\
(4) Caucasian man (18) [8.6\%] \\
(5) Multiracial man (17) [8.1\%] \\
\end{tabular}
& % ----------- k=7 Column -----------
\begin{tabular}[t]{@{}l@{}}
\textbf{Baseline} \\
(1) Caucasian man (44) [21.0\%] \\
(2) Latino non-binary (41) [19.5\%] \\
(3) Black man (41) [19.5\%] \\
(4) Latinx man (39) [18.6\%] \\
(5) White man (35) [16.7\%] \\
\\
\textbf{GPT-4o} \\
(1) White man (15) [7.1\%] \\
(2) Latino non-binary (15) [7.1\%] \\
(3) Multiracial man (14) [6.7\%] \\
(4) Caucasian woman (13) [6.2\%] \\
(5) Black woman (13) [6.2\%] \\
\\
\textbf{DeepSeek-V3} \\
(1) Black man (23) [11.0\%] \\
(2) Latino non-binary (19) [9.0\%] \\
(3) Multiracial man (18) [8.6\%] \\
(4) Caucasian man (18) [8.6\%] \\
(5) Caucasian woman (18) [8.6\%] \\
\end{tabular}
& % ----------- k=9 Column -----------
\begin{tabular}[t]{@{}l@{}}
\textbf{Baseline} \\
(1) Latinx man (49) [23.3\%] \\
(2) Black man (41) [19.5\%] \\
(3) Caucasian man (38) [18.1\%] \\
(4) Latino non-binary (37) [17.6\%] \\
(5) White man (37) [17.6\%] \\
\\
\textbf{GPT-4o} \\
(1) White man (15) [7.1\%] \\
(2) Multiracial man (14) [6.7\%] \\
(3) Black woman (13) [6.2\%] \\
(4) Caucasian man (13) [6.2\%] \\
(5) Latino non-binary (13) [6.2\%] \\
\\
\textbf{DeepSeek-V3} \\
(1) Black man (23) [11.0\%] \\
(2) Caucasian man (18) [8.6\%] \\
(3) Multiracial man (18) [8.6\%] \\
(4) Caucasian woman (17) [8.1\%] \\
(5) Latino non-binary (17) [8.1\%] \\
\end{tabular}
\\
\midrule
\textbf{Doctor} 
& % ----------- k=5 Column -----------
\begin{tabular}[t]{@{}l@{}}
\textbf{Baseline} \\
(1) Black woman (38) [18.1\%] \\
(2) Latinx woman (36) [17.1\%] \\
(3) Multiracial man (34) [16.2\%] \\
(4) Latinx man (29) [13.8\%] \\
(5) Caucasian man (28) [13.3\%] \\
\\
\textbf{GPT-4o} \\
(1) Black woman (67) [31.9\%] \\
(2) Multiracial man (42) [20.0\%] \\
(3) Latinx woman (20) [9.5\%] \\
(4) Hispanic man (19) [9.0\%] \\
(5) Caucasian man (16) [7.6\%] \\
\\
\textbf{DeepSeek-V3} \\
(1) Black woman (45) [21.4\%] \\
(2) Multiracial man (35) [16.7\%] \\
(3) Multiracial woman (21) [10.0\%] \\
(4) Caucasian man (19) [9.0\%] \\
(5) Latinx woman (18) [8.6\%] \\
\end{tabular}
& % ----------- k=7 Column -----------
\begin{tabular}[t]{@{}l@{}}
\textbf{Baseline} \\
(1) Latinx woman (36) [17.1\%] \\
(2) Caucasian man (36) [17.1\%] \\
(3) Multiracial man (35) [16.7\%] \\
(4) Black woman (34) [16.2\%] \\
(5) Hispanic man (15) [7.1\%] \\
\\
\textbf{GPT-4o} \\
(1) Black woman (66) [31.4\%] \\
(2) Multiracial man (41) [19.5\%] \\
(3) Hispanic man (20) [9.5\%] \\
(4) Latinx woman (18) [8.6\%] \\
(5) Multiracial woman (16) [7.6\%] \\
\\
\textbf{DeepSeek-V3} \\
(1) Black woman (41) [19.5\%] \\
(2) Multiracial man (37) [17.6\%] \\
(3) Multiracial woman (23) [11.0\%] \\
(4) Caucasian man (20) [9.5\%] \\
(5) Latinx woman (19) [9.0\%] \\
\end{tabular}
& % ----------- k=9 Column -----------
\begin{tabular}[t]{@{}l@{}}
\textbf{Baseline} \\
(1) Caucasian man (37) [17.6\%] \\
(2) Latinx woman (36) [17.1\%] \\
(3) Multiracial man (36) [17.1\%] \\
(4) Black woman (34) [16.2\%] \\
(5) Hispanic man (15) [7.1\%] \\
\\
\textbf{GPT-4o} \\
(1) Black woman (66) [31.4\%] \\
(2) Multiracial man (42) [20.0\%] \\
(3) Hispanic man (20) [9.5\%] \\
(4) Latinx woman (18) [8.6\%] \\
(5) Multiracial woman (16) [7.6\%] \\
\\
\textbf{DeepSeek-V3} \\
(1) Black woman (41) [19.5\%] \\
(2) Multiracial man (37) [17.6\%] \\
(3) Multiracial woman (23) [11.0\%] \\
(4) Caucasian man (20) [9.5\%] \\
(5) Latinx woman (19) [9.0\%] \\
\end{tabular}
\\
\bottomrule
\end{tabular}
\end{table*}

\begin{table*}[p]
\centering
\caption{Top-5 kNN classification results across different models - Baseline, GPT-4o, and DeepSeek-V3 - for the profession of housekeeper and nurse. Results shown for k=5, k=7, and k=9.}
\label{tab:top5_housekeeper_nurse}
\begin{tabular}{lccc}
\toprule
\textbf{Profession} & \textbf{k=5} & \textbf{k=7} & \textbf{k=9} \\
\midrule
\textbf{Housekeeper} 
& % ----------- k=5 Column -----------
\begin{tabular}[t]{@{}l@{}}
\textbf{Baseline} \\
(1) Caucasian woman (105) [50.0\%] \\
(2) Southeast Asian woman (41) [19.5\%] \\
(3) Pacific Islander woman (22) [10.5\%] \\
(4) Multiracial woman (14) [6.7\%] \\
(5) Hispanic woman (9) [4.3\%] \\
\\
\textbf{GPT-4o} \\
(1) Hispanic man (42) [20.0\%] \\
(2) Multiracial woman (37) [17.6\%] \\
(3) Multiracial man (32) [15.2\%] \\
(4) Caucasian woman (29) [13.8\%] \\
(5) Pacific Islander woman (10) [4.8\%] \\
\\
\textbf{DeepSeek-V3} \\
(1) Multiracial woman (69) [32.9\%] \\
(2) Caucasian woman (64) [30.5\%] \\
(3) Pacific Islander woman (26) [12.4\%] \\
(4) Southeast Asian woman (18) [8.6\%] \\
(5) Hispanic woman (9) [4.3\%] \\
\end{tabular}
& % ----------- k=7 Column -----------
\begin{tabular}[t]{@{}l@{}}
\textbf{Baseline} \\
(1) Caucasian woman (100) [47.6\%] \\
(2) Southeast Asian woman (40) [19.0\%] \\
(3) Pacific Islander woman (27) [12.9\%] \\
(4) Multiracial woman (19) [9.0\%] \\
(5) Latinx woman (7) [3.3\%] \\
\\
\textbf{GPT-4o} \\
(1) Hispanic man (49) [23.3\%] \\
(2) Multiracial woman (37) [17.6\%] \\
(3) Caucasian woman (29) [13.8\%] \\
(4) Multiracial man (27) [12.9\%] \\
(5) Pacific Islander woman (12) [5.7\%] \\
\\
\textbf{DeepSeek-V3} \\
(1) Multiracial woman (71) [33.8\%] \\
(2) Caucasian woman (61) [29.0\%] \\
(3) Pacific Islander woman (24) [11.4\%] \\
(4) Southeast Asian woman (19) [9.0\%] \\
(5) Hispanic woman (8) [3.8\%] \\
\end{tabular}
& % ----------- k=9 Column -----------
\begin{tabular}[t]{@{}l@{}}
\textbf{Baseline} \\
(1) Caucasian woman (106) [50.5\%] \\
(2) Southeast Asian woman (39) [18.6\%] \\
(3) Pacific Islander woman (22) [10.5\%] \\
(4) Multiracial woman (18) [8.6\%] \\
(5) Latinx woman (6) [2.9\%] \\
\\
\textbf{GPT-4o} \\
(1) Hispanic man (47) [22.4\%] \\
(2) Multiracial woman (37) [17.6\%] \\
(3) Multiracial man (29) [13.8\%] \\
(4) Caucasian woman (29) [13.8\%] \\
(5) Pacific Islander woman (12) [5.7\%] \\
\\
\textbf{DeepSeek-V3} \\
(1) Multiracial woman (71) [33.8\%] \\
(2) Caucasian woman (62) [29.5\%] \\
(3) Pacific Islander woman (23) [11.0\%] \\
(4) Southeast Asian woman (19) [9.0\%] \\
(5) Hispanic woman (8) [3.8\%] \\
\end{tabular}
\\
\midrule
\textbf{Nurse} 
& % ----------- k=5 Column -----------
\begin{tabular}[t]{@{}l@{}}
\textbf{Baseline} \\
(1) Caucasian woman (82) [39.0\%] \\
(2) Black woman (56) [26.7\%] \\
(3) Latinx woman (25) [11.9\%] \\
(4) Multiracial woman (19) [9.0\%] \\
(5) White woman (16) [7.6\%] \\
\\
\textbf{GPT-4o} \\
(1) Multiracial woman (51) [24.3\%] \\
(2) Hispanic man (39) [18.6\%] \\
(3) Multiracial man (39) [18.6\%] \\
(4) Caucasian man (18) [8.6\%] \\
(5) Black woman (18) [8.6\%] \\
\\
\textbf{DeepSeek-V3} \\
(1) Multiracial woman (101) [48.1\%] \\
(2) Black woman (45) [21.4\%] \\
(3) Latinx woman (20) [9.5\%] \\
(4) East Asian woman (19) [9.0\%] \\
(5) Caucasian woman (15) [7.1\%] \\
\end{tabular}
& % ----------- k=7 Column -----------
\begin{tabular}[t]{@{}l@{}}
\textbf{Baseline} \\
(1) Caucasian woman (82) [39.0\%] \\
(2) Black woman (57) [27.1\%] \\
(3) Latinx woman (24) [11.4\%] \\
(4) Multiracial woman (20) [9.5\%] \\
(5) White woman (16) [7.6\%] \\
\\
\textbf{GPT-4o} \\
(1) Multiracial woman (50) [23.8\%] \\
(2) Multiracial man (39) [18.6\%] \\
(3) Hispanic man (34) [16.2\%] \\
(4) Caucasian man (22) [10.5\%] \\
(5) Latinx woman (18) [8.6\%] \\
\\
\textbf{DeepSeek-V3} \\
(1) Multiracial woman (103) [49.0\%] \\
(2) Black woman (42) [20.0\%] \\
(3) East Asian woman (19) [9.0\%] \\
(4) Latinx woman (19) [9.0\%] \\
(5) Caucasian woman (15) [7.1\%] \\
\end{tabular}
& % ----------- k=9 Column -----------
\begin{tabular}[t]{@{}l@{}}
\textbf{Baseline} \\
(1) Caucasian woman (82) [39.0\%] \\
(2) Black woman (57) [27.1\%] \\
(3) Latinx woman (24) [11.4\%] \\
(4) Multiracial woman (19) [9.0\%] \\
(5) White woman (17) [8.1\%] \\
\\
\textbf{GPT-4o} \\
(1) Multiracial woman (50) [23.8\%] \\
(2) Multiracial man (39) [18.6\%] \\
(3) Hispanic man (36) [17.1\%] \\
(4) Caucasian man (21) [10.0\%] \\
(5) Latinx woman (18) [8.6\%] \\
\\
\textbf{DeepSeek-V3} \\
(1) Multiracial woman (103) [49.0\%] \\
(2) Black woman (42) [20.0\%] \\
(3) East Asian woman (19) [9.0\%] \\
(4) Latinx woman (19) [9.0\%] \\
(5) Caucasian woman (15) [7.1\%] \\
\end{tabular}
\\
\bottomrule
\end{tabular}
\end{table*}




\begin{figure*}[p]
    \centering
    \includegraphics[width=\textwidth]{fig/nonparam_eval.png}
    \caption{Comparison of No debias baseline (left), GPT-4 debias (middle), and DeepSeek-V3 debias (right) for selected prompts from the stable debias profession dataset.}
    \label{fig:nonparam_eval_all}
\end{figure*}

\begin{figure*}[p]
    \centering
    \includegraphics[width=0.8\textwidth]{fig/sd_all.jpg}
    \caption{Comparison of No debias baseline (left), GPT-4 debias (middle), and DeepSeek-V3 debias (right) for selected prompts from stable bias profession dataset.}
    \label{fig:sd_all}
\end{figure*}

\begin{figure*}[p]
    \centering
    \includegraphics[width=0.8\textwidth]{fig/p2_all.png}
    \caption{Comparison of No debias baseline (left), GPT-4 debias (middle), and DeepSeek-V3 debias (right) for selected prompts from Parti Prompt dataset.}
    \label{fig:p2_all}
\end{figure*}

\end{document}
