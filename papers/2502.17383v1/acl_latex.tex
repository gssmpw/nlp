% This must be in the first 5 lines to tell arXiv to use pdfLaTeX, which is strongly recommended.
\pdfoutput=1
% In particular, the hyperref package requires pdfLaTeX in order to break URLs across lines.

\documentclass[11pt]{article}

% Change "review" to "final" to generate the final (sometimes called camera-ready) version.
% Change to "preprint" to generate a non-anonymous version with page numbers.
\usepackage[preprint]{acl}

% Standard package includes
\usepackage{times}
\usepackage{latexsym}

% For proper rendering and hyphenation of words containing Latin characters (including in bib files)
\usepackage[T1]{fontenc}
% For Vietnamese characters
% \usepackage[T5]{fontenc}
% See https://www.latex-project.org/help/documentation/encguide.pdf for other character sets

% This assumes your files are encoded as UTF8
\usepackage[utf8]{inputenc}

% This is not strictly necessary, and may be commented out,
% but it will improve the layout of the manuscript,
% and will typically save some space.
\usepackage{microtype}

% This is also not strictly necessary, and may be commented out.
% However, it will improve the aesthetics of text in
% the typewriter font.
\usepackage{inconsolata}

%Including images in your LaTeX document requires adding
%additional package(s)
\usepackage{mdframed}
\usepackage{graphicx}
\usepackage{float}
\usepackage{hyperref}
\usepackage{url}
% \usepackage{inconsolata}
\let\Bbbk\relax
\usepackage{amssymb}
\usepackage{amsmath}
\usepackage{epsfig,subfigure,caption}
\usepackage{algpseudocode}
\usepackage[normalem]{ulem}
\usepackage[linesnumbered,algoruled,boxed,noend]{algorithm2e}
\usepackage{listings}  % Required for text wrapping inside tcolorbox
\usepackage{xcolor}     % Optional: For custom colors
\usepackage{color, colortbl}
\usepackage{multirow,booktabs, hhline}
\usepackage{bm}
\usepackage[linesnumbered,algoruled,boxed,noend]{algorithm2e}
\usepackage{wrapfig}
\usepackage{verbatimbox}
\usepackage{kantlipsum}
\usepackage{fancyvrb}
\usepackage{xcolor}
\usepackage[many]{tcolorbox}
\usepackage[htt]{hyphenat}
% \usepackage{mdframed}
\usepackage{pifont}% http://ctan.org/pkg/pifont
\newcommand{\cmark}{\ding{51}}%
\newcommand{\xmark}{\ding{55}}%
\def\Centerline#1{%
  \setsepchar{\cpar}%
  \readlist\clarg{#1}%
  \foreachitem\z\in\clarg[]{\centerline{\z}}%
}
\tcbset{
    myboxstyle/.style={
        colback=yellow!10!white, % Brighter yellow background
        colframe=yellow!50!black, % Vibrant yellow border
        fonttitle=\bfseries,
        coltitle=black,
        boxrule=0.75mm,
        width=\textwidth,
        sharp corners,
        leftrule=1mm,
        left=5pt,
        right=5pt,
        top=5pt,
        bottom=5pt,
        fontupper=\small,  % Set the font size for content
        fontlower=\small   % Set the font size for footnotes or lower part
    }
}
% If the title and author information does not fit in the area allocated, uncomment the following
%
%\setlength\titlebox{<dim>}
%
% and set <dim> to something 5cm or larger.
\newcommand{\dongho}[1]{{\textcolor{blue}{DH: #1}}}
\newcommand{\justin}[1]{{\textcolor{cyan}{JC: #1}}}

\newcommand{\secref}[1]{\S\ref{#1}}
\newcommand{\eg}{\textit{e.g., }}
\newcommand{\ie}{\textit{i.e., }}
\newcommand{\versus}{\textit{vs. }}

\newcommand{\ourdata}{\textsc{Textbook-Exam}\xspace} 
\newcommand{\ours}{\textsc{QUEST}\xspace}
\definecolor{darkgreen}{RGB}{95,129,63}
\definecolor{darkorange}{RGB}{184, 96, 41}

\title{What is a Good Question? \\ Utility Estimation with LLM-based Simulations}



\author{
    Dong-Ho Lee\thanks{Authors contributed equally},~
    Hyundong Cho\footnotemark[1],~
    Jonathan May\thanks{Equal advising contribution},~
    Jay Pujara\footnotemark[2] \\
    Information Science Institute, University of Southern California \\
     {\small 
        \texttt{\{dongho.lee\}@usc.edu},~
        \texttt{\{jcho, jonmay, jpujara\}@isi.edu}
    }\\
}



\begin{document}
\maketitle
\begin{abstract}


The choice of representation for geographic location significantly impacts the accuracy of models for a broad range of geospatial tasks, including fine-grained species classification, population density estimation, and biome classification. Recent works like SatCLIP and GeoCLIP learn such representations by contrastively aligning geolocation with co-located images. While these methods work exceptionally well, in this paper, we posit that the current training strategies fail to fully capture the important visual features. We provide an information theoretic perspective on why the resulting embeddings from these methods discard crucial visual information that is important for many downstream tasks. To solve this problem, we propose a novel retrieval-augmented strategy called RANGE. We build our method on the intuition that the visual features of a location can be estimated by combining the visual features from multiple similar-looking locations. We evaluate our method across a wide variety of tasks. Our results show that RANGE outperforms the existing state-of-the-art models with significant margins in most tasks. We show gains of up to 13.1\% on classification tasks and 0.145 $R^2$ on regression tasks. All our code and models will be made available at: \href{https://github.com/mvrl/RANGE}{https://github.com/mvrl/RANGE}.

\end{abstract}


\section{Introduction}

Video generation has garnered significant attention owing to its transformative potential across a wide range of applications, such media content creation~\citep{polyak2024movie}, advertising~\citep{zhang2024virbo,bacher2021advert}, video games~\citep{yang2024playable,valevski2024diffusion, oasis2024}, and world model simulators~\citep{ha2018world, videoworldsimulators2024, agarwal2025cosmos}. Benefiting from advanced generative algorithms~\citep{goodfellow2014generative, ho2020denoising, liu2023flow, lipman2023flow}, scalable model architectures~\citep{vaswani2017attention, peebles2023scalable}, vast amounts of internet-sourced data~\citep{chen2024panda, nan2024openvid, ju2024miradata}, and ongoing expansion of computing capabilities~\citep{nvidia2022h100, nvidia2023dgxgh200, nvidia2024h200nvl}, remarkable advancements have been achieved in the field of video generation~\citep{ho2022video, ho2022imagen, singer2023makeavideo, blattmann2023align, videoworldsimulators2024, kuaishou2024klingai, yang2024cogvideox, jin2024pyramidal, polyak2024movie, kong2024hunyuanvideo, ji2024prompt}.


In this work, we present \textbf{\ours}, a family of rectified flow~\citep{lipman2023flow, liu2023flow} transformer models designed for joint image and video generation, establishing a pathway toward industry-grade performance. This report centers on four key components: data curation, model architecture design, flow formulation, and training infrastructure optimization—each rigorously refined to meet the demands of high-quality, large-scale video generation.


\begin{figure}[ht]
    \centering
    \begin{subfigure}[b]{0.82\linewidth}
        \centering
        \includegraphics[width=\linewidth]{figures/t2i_1024.pdf}
        \caption{Text-to-Image Samples}\label{fig:main-demo-t2i}
    \end{subfigure}
    \vfill
    \begin{subfigure}[b]{0.82\linewidth}
        \centering
        \includegraphics[width=\linewidth]{figures/t2v_samples.pdf}
        \caption{Text-to-Video Samples}\label{fig:main-demo-t2v}
    \end{subfigure}
\caption{\textbf{Generated samples from \ours.} Key components are highlighted in \textcolor{red}{\textbf{RED}}.}\label{fig:main-demo}
\end{figure}


First, we present a comprehensive data processing pipeline designed to construct large-scale, high-quality image and video-text datasets. The pipeline integrates multiple advanced techniques, including video and image filtering based on aesthetic scores, OCR-driven content analysis, and subjective evaluations, to ensure exceptional visual and contextual quality. Furthermore, we employ multimodal large language models~(MLLMs)~\citep{yuan2025tarsier2} to generate dense and contextually aligned captions, which are subsequently refined using an additional large language model~(LLM)~\citep{yang2024qwen2} to enhance their accuracy, fluency, and descriptive richness. As a result, we have curated a robust training dataset comprising approximately 36M video-text pairs and 160M image-text pairs, which are proven sufficient for training industry-level generative models.

Secondly, we take a pioneering step by applying rectified flow formulation~\citep{lipman2023flow} for joint image and video generation, implemented through the \ours model family, which comprises Transformer architectures with 2B and 8B parameters. At its core, the \ours framework employs a 3D joint image-video variational autoencoder (VAE) to compress image and video inputs into a shared latent space, facilitating unified representation. This shared latent space is coupled with a full-attention~\citep{vaswani2017attention} mechanism, enabling seamless joint training of image and video. This architecture delivers high-quality, coherent outputs across both images and videos, establishing a unified framework for visual generation tasks.


Furthermore, to support the training of \ours at scale, we have developed a robust infrastructure tailored for large-scale model training. Our approach incorporates advanced parallelism strategies~\citep{jacobs2023deepspeed, pytorch_fsdp} to manage memory efficiently during long-context training. Additionally, we employ ByteCheckpoint~\citep{wan2024bytecheckpoint} for high-performance checkpointing and integrate fault-tolerant mechanisms from MegaScale~\citep{jiang2024megascale} to ensure stability and scalability across large GPU clusters. These optimizations enable \ours to handle the computational and data challenges of generative modeling with exceptional efficiency and reliability.


We evaluate \ours on both text-to-image and text-to-video benchmarks to highlight its competitive advantages. For text-to-image generation, \ours-T2I demonstrates strong performance across multiple benchmarks, including T2I-CompBench~\citep{huang2023t2i-compbench}, GenEval~\citep{ghosh2024geneval}, and DPG-Bench~\citep{hu2024ella_dbgbench}, excelling in both visual quality and text-image alignment. In text-to-video benchmarks, \ours-T2V achieves state-of-the-art performance on the UCF-101~\citep{ucf101} zero-shot generation task. Additionally, \ours-T2V attains an impressive score of \textbf{84.85} on VBench~\citep{huang2024vbench}, securing the top position on the leaderboard (as of 2025-01-25) and surpassing several leading commercial text-to-video models. Qualitative results, illustrated in \Cref{fig:main-demo}, further demonstrate the superior quality of the generated media samples. These findings underscore \ours's effectiveness in multi-modal generation and its potential as a high-performing solution for both research and commercial applications.
\section{\sysname Framework}

\begin{figure}[t]
    \vspace{-0.2in}
    \centering
    \includegraphics[width=1\linewidth]{Figures/apint_framework.pdf}
    \caption{Overall \sysname Framework}
    \vspace{-0.2in}
    \label{fig:APINT_framework}
\end{figure}

In this section, we propose APINT, a full-stack framework designed to accelerate PiT by reducing the overhead of GC, the primary bottleneck of PiT. The overall workflow is illustrated in Figure~\ref{fig:APINT_framework}, distinguishing between compile-time and runtime processes.
In the initial compile time stage, \sysname begins by extracting nonlinear function operations in the process of computing the transformer model through the \sysname protocol. Next, through the GC-friendly circuit generation, the extracted function is implemented as a circuit consisting of a 2-input gate, and it is converted to netlist in Bristol format~\cite{tillich2016circuits}.
This step significantly alleviates the computational load on GC in subsequent stages. Following this, \sysname adopts a scheduling strategy that combines coarse-grained and fine-grained scheduling. The strategy enables full utilization of DRAM bandwidth and decrement in wire dependency. Furthermore, \sysname incorporates compiler speculation to generate instructions that capitalize on wire reusability and are executed on hardware accelerators at runtime. These accelerators, designed to further reduce memory stalls by eliminating redundant DRAM accesses, are deployed on both the server and client, allowing them to perform GC evaluation or GC garbling.
% These accelerators are deployed on both the server and client sides within the framework to further reduce memory stalls by eliminating redundant DRAM accesses.

\subsection{\sysname Protocol}

\begin{figure}[t]
    \vspace{-0.2in}
    \centering
    \includegraphics[width=1\linewidth]{Figures/Camera-ready/Fig4_Protocol.pdf}
    % \includegraphics[width=1\linewidth]{Figures/Protocol_Offload.pdf}
    \caption{\sysname Protocol}
    \vspace{-0.2in}
    \label{fig:APINT_protocol}
\end{figure}

\sysname protocol is based on the PiT protocol in PRIMER, but it reduces the circuit in GC operation by offloading its partial calculations to HE and standard operations, thereby significantly decreasing the workload of GC. As illustrated in Figure~\ref{fig:APINT_protocol}, the basic concept of the protocol is the combination of HE for linear operations and GC for nonlinear operations. To maintain confidentiality, each party adds or subtracts a random matrix ($R_i$ of the client and $S_i$ of the server) before sending the data to each other.
At the offline phase, HE is utilized to compute linear function for the client's random matrix $R_{1}$, which has the same matrix size as the input matrix $X_1$ \circled{1}. Simultaneously, the client garbles the circuit $\Tilde{C_1}$, which integrates adding the secret shares from both parties, processing the nonlinear function, and subtracting a random matrix to ensure confidentiality \circled{2}. The client then transmits labels of $R_2, R_3$ to the server \circled{3}. During the online phase, the server calculates the linear function of $(X_{1}-R_{1})$ using standard matrix operations. This intermediate result is merged with data from process \circled{1} to complete the computation of the linear function, yielding $(X_{2}-R_{2})$ \circled{4}. After that, the labels of $(X_{2}-R_{2})$ are sent from the client via the OT protocol~\cite{ishai2003extending} \circled{5}, and the server proceeds the GC evaluation for the garbled $\Tilde{C_1}$ \circled{6}.




However, in contrast to other functions, the reduced circuit $\Tilde{C_2}$ is employed when operating LayerNorm. This circuit specifically excludes calculations of mean and variance, as well as operations involving the parameters $\beta$ and $\gamma$. The excluded calculations are offloaded and computed using standard operations and HE, thereby reducing the workload of GC.
During the offline phase, the client garbles the circuit $\Tilde{C_2}$, transmitting the labels for $\sum{R_4/N}, R_5, R_6$, and also sends $Enc(R_2')$.
During the online phase, the mean of $(X_{2}-R_{2})$ is initially calculated using standard operations. This mean is then subtracted from $(X_{2}-R_{2})$, resulting in $(X_{2}'-R_{2}')$ \circled{7}. Subsequently, to prepare for variance calculations, this result is multiplied by two times $Enc(R_2')$ \circled{8}. The results from equation \circled{7} and \circled{8} are then used to compute the variance \circled{9}. Third, multiplying with the parameter $\beta$ can be processed by utilizing HE with $(X_{2}'-R_{2}')$, $Enc(R_2')$, and $\beta$ \circled{10}, \circled{11}. Then, the labels of the data from the process \circled{9} and \circled{11}, which are obtained from the client via OT protocol, and the labels from the client are computed through GC evaluation \circled{12}. Finally, a straightforward addition of the parameter $\gamma$ is processed \circled{13}.

Although this protocol incurs additional overhead due to HE and communication of two parties, it brings a substantial reduction of GC latency, offsetting these increased costs perfectly. This reduction marks a significant improvement over the baseline protocol, which merely utilized GC for processing the nonlinear functions. Ultimately, the \sysname protocol achieves a significant reduction in the online latency of GC operations, reducing it by 47.3\% during the LayerNorm computation.

\subsection{GC-friendly Circuit Generation}

To further minimize the workload of GC, \sysname proposes GC-friendly circuit generation of the nonlinear functions. It involves implementing each function to a circuit with 2-input AND, XOR, and INV gates while preserving the accuracy of computations. The process unfolds in two main steps.

The initial step focuses on minimizing the total number of gates in the circuit. Since GC processes the gates of the circuit sequentially, reducing the gate count directly lowers the overall computational load. For instance, in the implementation of Softmax, the method from i-BERT~\cite{kim2021bert} is adopted, which scales inputs by \textit{ln2}, thereby reducing the range of values and the number of required gates for exponential operations. The exponential operations are performed through combinational logic, which performs as a Look-Up Table (LUT) interpolation. For the GeLU function, LUT interpolation is utilized after clipping the input values within a range (-4, 4)~\cite{gupta2023sigma}. In LayerNorm, the conventional approach is employed without any approximation, as it doesn't incur any accuracy drop.

The second step aims to decrease the number of AND gates further. \sysname proposes a method employing XOR-Friendly Binary Quantization (XFBQ)~\cite{jian2020fast} to implement multiplication with fewer AND gates compared to the conventional method. This approach is motivated by the observation that the multiplication process accounts for a significant portion of each nonlinear function's implementation. 
Figure~\ref{fig:circuit_generation} (a) summarizes XFBQ and its multiplication process. XFBQ modifies the binary representation, wherein 1 represents 1 and 0 represents -1, exploiting the correspondence between the result patterns of XOR operations and the product of 1 and -1. The way to XFBQ is straightforward: it involves a right shift and changing the MSB to 1, introducing only a minimal quantization error (Q error) as small as the INV of Least Significant Bit (LSB) of the original number. For instance, \textit{1000}=8 turns into \textit{1100}=8+4-2-1=9 after XFBQ with Q error as \textit{INV(0)=1}.  When expressing conventional multiplication ($A\times B$) as XFBQ multiplication ($\hat{A} \times \hat{B}$) along with additional terms due to Q errors, the AND operations of the conventional method are all replaced by XOR, thereby a high reduction of AND gates occurs. Moreover, given that the Q error is INV of LSB value, its negligible impact on multiplication results and PiT warrants disregarding the additional terms, leading to a further decrease in AND gates.

\begin{figure}[t]
    \vspace{-0.2in}
    \centering
    \includegraphics[width=1\linewidth]{Figures/Camera-ready/xfbq_correct.pdf}
    \caption{(a) Reduction of ANDs via XBFQ Multiplication (b) Comparison of ANDs for 64b Multiplication}
    \vspace{-0.2in}
    \label{fig:circuit_generation}
\end{figure}

Figure~\ref{fig:circuit_generation} (b) shows the effects of the multiplication using XFBQ while operating 64b multiplication. It reduces the number of AND gates 38.9-45.5\% compared to prior work~\cite{liu2022don}, depending on the inclusion of Q error adjustments. In addition, GC-friendly circuit generation employed methods from the work~\cite{testa2020logic} for operations other than multiplication, which has been proven to perform as an open-source. Finally, it reduces the workload of GC for nonlinear functions by an average of 42.5\%.



   
\begin{figure*}[t]
    \vspace{-0.2in}
    \centering
    \includegraphics[width=1\linewidth]{Figures/Camera-ready/Scheduling_Reduced.pdf}
    \caption{(a) Methods and (b) Effects of Coarse-Grained and Fine-Grained Scheduling}
    \vspace{-0.2in}
    \label{fig:scheduling}
\end{figure*}
\subsection{Netlist Scheduling}
% \subsection{Compiler and Hardware Accelerator}
Despite the reductions in GC overhead facilitated by the \sysname protocol and GC-friendly circuit generation, GC still accounts for a notable portion of the latency. This implies that reducing GC overhead necessitates the integration of hardware accelerators beyond software solutions.
However, since a GC accelerator takes a netlist, converted from the circuit, as input and processes gates in the netlist sequentially, efficient netlist scheduling is crucial for hardware acceleration. Therefore, we introduce coarse-grained and fine-grained scheduling, maximizing DRAM bandwidth utilization and minimizing computational dependencies to accelerate the GC operation of nonlinear functions. 

% For instance, HAAC~\cite{mo2023haac} introduced an accelerator showcasing superior performance compared to CPUs and GPUs by utilizing multi-core designs operating concurrently and leveraging off-chip memory when on-chip memory resources are insufficient. However, this approach encounters a significant memory bottleneck when processing nonlinear functions of transformers due to inadequate scheduling schemes and an accelerator design overlooking wire reusability. Consequently, as a final step, \sysname introduces a compiler-integrated accelerator, addressing memory bottleneck by enhancing wire reuse and maximizing DRAM bandwidth utilization through compiler-driven strategies and hardware structure.

\subsubsection{\textbf{Coarse-Grained Scheduling}}

Due to the complex implementation of the circuit of the nonlinear function, the netlist exhibits highly irregular patterns in input and output wires, leading to irregular DRAM accesses that hinder optimal DRAM bandwidth utilization.
To tackle this issue, \sysname adopts coarse-grained scheduling, which maps each independent operation onto each core, allowing cores to function independently yet synchronously. This approach leverages the characteristic of nonlinear functions composed of independent unit operations, such as rows in Softmax, to be computed separately. 
Assuming there are two Processing Engines (PEs) and the need to compute two Softmax rows, the red box of Figure~\ref{fig:scheduling} (a) illustrates a DAG of two rows ordered in a depth-first manner without any scheduling, where node number corresponds to the order of the gates in the netlist. In this case, two PEs concurrently process the netlists of two rows in a dependent manner. In contrast, as illustrated in the green box, each PE exclusively handles an independent row with coarse-grained scheduling. Hence, contrary to the red box in Figure~\ref{fig:scheduling} (b), the green box demonstrates that coarse-grained scheduling allows all PEs to operate synchronously, ensuring they request DRAM data simultaneously. Therefore, while the intra-core DRAM access pattern is irregular, the inter-core DRAM access pattern becomes the same. As a result, by enabling cores to share the DRAM data bus, coarse-grained scheduling ensures maximal utilization of DRAM bandwidth.

Moreover, coarse-grained scheduling offers the additional benefit of resolving wire dependencies. Unlike the scenario without coarse-grained scheduling, the scheduling enables each PE to independently operate on a distinct row without dependencies. Thus, coarse-grained scheduling not only maximizes the utilization of DRAM bandwidth but also reduces the pipeline stalls.
        
\subsubsection{\textbf{Fine-Grained Scheduling}}
In addition to the coarse-grained scheduling, \sysname applies fine-grained scheduling that further diminishes GC latency compared to Full Reorder (FR) and Segment Reorder (SR), the scheduling method by the SOTA GC accelerator, HAAC~\cite{mo2023haac}. The FR transforms the netlist into a DAG and establishes processing order by traversing the graph in a breadth-first manner, reducing wire dependencies. However, due to limited on-chip memory, it can cause off-chip traffic by spilling wires over to DRAM, especially in applications where the DAG has a wide breadth, such as the nonlinear function of transformers. To tackle this problem, HAAC proposed the SR that segments the netlist, ordered in a depth-first manner, to enhance wire reuse and then applies FR within each segment to reduce wire dependencies. 
% \sysname applies fine-grained scheduling in addition to the coarse-grained scheduling. This further diminishes GC latency beyond the scheduling method proposed by HAAC, such as Full Reorder (FR) and Segment Reorder (SR).
% FR transforms the netlist into a DAG, where each node represents a gate. It establishes processing order by traversing the graph in a breadth-first manner, reducing wire dependencies but decreasing wire reuse in limited on-chip memory. Consequently, it can cause a memory bottleneck by spilling data over to DRAM, especially in applications where the DAG has a wide breadth, such as the nonlinear function of transformers. HAAC additionally proposed SR to tackle this problem. As shown in Figure~\ref{fig:scheduling}, SR firstly segments the netlist after depth-first scheduling to enhance wire reuse and applies FR within each segment to reduce dependencies.
% However, FR is not the optimal way to reduce dependencies within each segment. Recent research by S.Zhao~\cite{zhao2020dag, zhao2022dag} has highlighted the efficiency of priority scheduling based on Critical-Path-First-Execution (CPFE) in reducing dependencies of DAG. Therefore, \sysname proposes a fine-grained scheduling by combining SR and CPFE.
Despite these advancements, we identified opportunities for further latency reduction since FR does not optimally eliminate wire dependency within segments. Therefore, \sysname introduces a fine-grained scheduling strategy that combines segmentation and Critical-Path-First-Execution (CPFE)~\cite{zhao2020dag, zhao2022dag} instead of FR, achieving enhanced performance by effectively minimizing wire dependencies.

The fine-grained scheduling begins by segmenting the netlist, with each segment half the size of on-chip memory. A DAG is then constructed for each segment, with assigning weights reflecting the cycle latency of each gate. Nodes without children are linked to \textit{v\_{src}}, and those without parents are connected to \textit{v\_{sink}}. After establishing the DAG, it finds a critical path from \textit{v\_{src}} to \textit{v\_{sink}} and prioritizes nodes along this path, starting from the lowest depth. Subsequently, for each node on the path, a sub-DAG is formed comprising unprioritized descendants, and the process of identifying the critical path and assigning priorities repeats recursively.

The blue box in Figure~\ref{fig:scheduling} (a) shows how the fine-grained scheduling works. First, in step \circled{1}, it finds a critical path and prioritizes from the lowest depth. Then, from step \circled{2}-\circled{6}, it creates sub-DAG with unprioritized descendants for each node of the path and operates recursively. 
% As shown in Figure~\ref{fig:scheduling}, DAG \circled{1} shows finding a critical path and prioritizing from the lowest depth. Then, from DAG \circled{2} to DAG \circled{6}, it creates sub-DAG with unprioritized descendants for each node of the path and operates recursively.
For example, in step \circled{6}, nodes \textit{4} and \textit{6} compose the sub-DAG, and then the process of identifying the critical path and prioritizing is recursively executed at the sub-DAG. After assigning priorities to all nodes, the scheduling order is determined by the cycle-accurate simulation. The simulation selects the operable node with the highest priority in each cycle. The "operable" refers to the condition where both input wires of a DAG node have been produced. As a result, step \circled{6} is reordered as $2\rightarrow1\rightarrow4\rightarrow5\rightarrow6\rightarrow3\rightarrow8\rightarrow7$. 
Hence, as depicted in Figure~\ref{fig:scheduling}, fine-grained scheduling significantly reduces pipeline stalls by wire dependencies within each segment, enhancing the computation speed of nonlinear functions by an average of 30.2\% compared to the SR of HAAC.

\begin{figure*}[t]
    \vspace{-0.2in}
    \centering
    \includegraphics[width=1\linewidth]{Figures/Camera-ready/Hardware_Reduced.pdf}
    \caption{APINT hardware, Compiler Speculation Flow, and Runtime Flow Descriptions}
    \vspace{-0.2in}
    \label{fig:compiler_and_hardware}
\end{figure*}
\subsection{Accelerator with Compiler Speculation}
% \subsubsection{\textbf{Compiler Speculation and Hardware Accelerator}}
HAAC introduced an accelerator showcasing superior performance compared to CPUs and GPUs by utilizing pipelined multi-core designs operating concurrently and leveraging off-chip memory when on-chip memory resources are insufficient. However, it encounters a significant memory bottleneck when processing nonlinear functions of transformers due to inadequate on-chip memory policy and hardware structure.
% A further cause of the memory bottleneck is the on-chip memory policy and hardware structure of HAAC.
HAAC's approach of sequentially writing output wires in on-chip memory doesn't consider the wire reusability. Moreover, the hardware structure that involves directly fetching wires from DRAM to a PE via a queue structure limits the wire's usage to a single time, thereby restricting its potential for reuse. To counter these issues, \sysname suggests the accelerator alongside compiler speculation techniques, aiming to reduce unnecessary DRAM accesses and improve wire reuse.

\subsubsection{\textbf{APINT Accelerator}}
% \textit{\textbf{APINT Accelerator} \ }
Figure~\ref{fig:compiler_and_hardware} illustrates the architecture of \sysname accelerator, which features 16 independent cores operating synchronously under coarse-grained scheduling. This eliminates the need for inter-core communication, allowing for a shared unified Instruction Memory (16KB). Each core includes a Wire Memory (128KB), a Table Memory (2KB), an Out-of-Range-Wire (OoRW) Prefetch Buffer (1KB), and a PE, all of which are pipelined.
Wire Memory stores the wire's label (value of the wire) and special flag bits, which are a block bit and an Out-of-Range (OoR) bit, for each address. The block bit prevents other wires from accessing the address, while the OoR bit indicates that an OoRW, a wire that is fetched from DRAM, is being fetched to that address. Also, the Table Memory stores garble tables required for Half-Gate operations, and the OoRW Prefetch Buffer temporarily stores OoRWs fetched from DRAM. They are then transferred to the Wire Memory, which allows multiple reuses within the Wire Memory in contrast to HAAC, where OoRWs are used only once per fetch.

The execution of the accelerator is structured into four stages. First, upon receiving an instruction, the Write Address Preemption stage the write address in the Wire Memory, activating the block bit. Next, the Read stage reads two input wires from the memory or forwarding path over three cycles. Third, the input wires are processed in the Half-gate unit (taking 18 cycles for evaluation and 21 for garbling) or FreeXOR unit (taking one cycle) in PE, and OoRWs are transferred from the Prefetch Buffer to Wire Memory if required. Finally, the output wire generated in the PE is written back to Wire Memory over two cycles and, if needed, also to DRAM.

% The execution of the \sysname accelerator is structured into four stages: Write Address Preemption, Read, OoRW Transfer and PE Execution, and Write. Upon receiving an instruction, the accelerator preempts the write address in the Wire Memory, setting the block bit alive. It then reads input wires over three cycles and processes them in the PE, which includes a Half-Gate unit (taking 18 cycles for evaluation and 21 for garbling) and a FreeXOR unit (taking one cycle). Garble tables are transferred from the Table Memory if Half-Gate is operated. Simultaneously, OoRWs are transferred from the Prefetch Buffer to the Wire Memory if required. Finally, after the PE execution, output wires are written back to Wire Memory over two cycles and, if needed, also to DRAM.

\subsubsection{\textbf{Compiler Speculation Flow}}
Before running the accelerator, compiler speculation is initially processed with a netlist as input. Its purpose is to generate instructions for the accelerator, which implements a memory policy that enhances wire reuse. It proceeds through the following two phases, as depicted in Figure~\ref{fig:compiler_and_hardware}. During the first phase, it assigns read and write addresses in Wire Memory and an OP bit for each gate in the netlist through a cycle-accurate simulation. After filling Wire Memory as much as possible with operable input wires, the speculation begins with the Write Address Preemption stage, allocating a write address either to a blank space or to the Last-to-Be-Used Wire (LBUW), if Wire Memory is full. The LBUW is the wire that will be used last among wires within the memory. This demonstrates that APINT employs a memory policy considering the reusability of wires. After assigning the write address, the block bit is activated for the preemption.

% Compiler speculation begins by using a netlist as input, producing instructions generated to enhance wire reuse. This process proceeds through the following two phases after filling Wire Memory as much as possible with operable input wires. During the first phase, it assigns read and write addresses in Wire Memory and an OP bit for each gate through a cycle-accurate simulation. First, during the Write Address Preemption stage, if the Wire Memory is not full, the write address is assigned to a blank space, if not, to the address of the Last-to-Be-Used Wire (LBUW), designated to be used last among wires in Wire Memory, and a block bit for this address is activated. Therefore, this process enables the on-chip memory policy to consider the reusability of wires.

Next, the Read stage assigns the read address based on whether an input wire is present in Wire Memory. If present, the read address corresponds to its location. If not, indicating it is likely to become an OoRW at runtime, the address of the LBUW with inactive block bit is assigned, which also contributes to the memory policy that considers the reusability of wires. The input wire is then replaced with the LBUW and added to the OoRW list. Subsequently, the PE execution stage begins by setting the OP bit based on the gate type. This is followed by the Write stage, which writes the output wire at the assigned address and resets the block bit. This cycle-accurate simulation is repeated until every wire and gate in the netlist has been allocated the instructions with addresses and OP bits.

After completing the first phase, the second phase involves assigning the Live bit, two OoRW-fetch bits, and the Write Enable Not (WEN) bit, which are determined by analyzing the interrelationships among instructions. The Live bit is assigned to instructions that output an OoRW, designating that the wire should be written to DRAM for later use. For example, in Figure~\ref{fig:compiler_and_hardware}, instruction \circled{1} outputs OoRW \textit{30} and is marked with a Live bit of 1.
Each OoRW-fetch bit is assigned to ensure the timely transfer of an OoRW from the Prefetch Buffer to Wire Memory, based on instruction sequence and read dependencies.
For instance, instruction \circled{2}, which reads address \textit{0} immediately before instruction \circled{4} reads OoRW \textit{30} from the same address, is assigned an OoRW-fetch bit to ensure that OoRW \textit{30} is transferred right after instruction \circled{2} reads the address \textit{0} to prevent stalls due to non-arrival.
The WEN bit is assigned to prevent premature overwriting in Wire Memory.
For example, if OoRW \textit{30} is transferred to memory before instruction \circled{3} writes to address \textit{0}, it could be overwritten before it is read by instruction \circled{4}. Therefore, a WEN bit is assigned to instruction \circled{3} to prevent it from overwriting OoRW \textit{30}, and wire \textit{35} is written only to DRAM as dictated by the Live bit.
% For example, to avoid overwriting OoRW \textit{30} needed by instruction \circled{4}, instruction \circled{3} receives a WEN bit of 1.


% During the speculation process, the sequence for using instructions and garbled tables is predetermined and mapped sequentially in DRAM. Similarly, the DRAM read and write orders for OoRWs are predecided, with DRAM write addresses being assigned incrementally. Hence, if an OoRW's DRAM read address exceeds the current increment value, indicating the wire has yet to be written to DRAM, the accelerator stalls until the increment value matches the address. This approach removes the need to handle DRAM addresses during runtime, enabling the accelerator to fetch instructions, garbled tables, and OoRWs from DRAM to on-chip in a predetermined order while executing the instructions.

\subsubsection{\textbf{Runtime Flow}}

During the speculation process, the DRAM addresses for instructions, garbled tables, and OoRWs are predetermined, removing the need to handle the addresses during runtime. While fetching the data from DRAM to each corresponding memory, the runtime process is executed in the following four stages, as depicted in Figure~\ref{fig:compiler_and_hardware}.
% During runtime, the four stages are executed with the \sysname accelerator, as depicted in Figure~\ref{fig:compiler_and_hardware}. 
After an instruction is decoded, the Write Address Preemption stage activates the block bit at the write address, and the Read stage operates based on the statuses of the block and OoR bits. Depending on these bits, the accelerator either performs a normal read or stall until the necessary wire is transferred from the Prefetch Buffer or the forwarding path. The OoRW Transfer and PE Execution stage then commences. The OoR bit is assigned with the OoRW-fetch bit, indicating whether an OoRW transfer is started. If the OoRW-fetch bit is 1, the address is preempted by activating the block bit, and an OoRW begins to be transferred to the address. After the completion of the transfer, the block bit is deactivated. Concurrently, the PE processes Half-gate or FreeXOR operation based on the OP bit. After the output wire is generated in the PE, the Write stage begins, writing the wire to Wire Memory and DRAM depending on the WEN and Live bits. After these stages are executed across all instructions, the runtime process is completed. Overall, through APINT accelerator and compiler speculation, it achieves a reduction in memory stall times by 86.1\% to 99.4\% compared to HAAC when operating nonlinear functions.

% After an instruction is decoded, the Write Address Preemption stage sets the block bit at the write address, and the Read stage proceeds based on the block and OoR bit statuses. If the block bit is 0, a normal read is performed. Otherwise, the OoR bit determines if an OoRW or non-OoRW wire has preempted the address. An OoR bit of 1 indicates a scheduled OoRW write at that address, causing the accelerator to stall until the OoRW is transferred from the Prefetch Buffer to Wire Memory. If the OoR bit is 0, the accelerator is stalled until the wire is retrieved from the forwarding path.

% Once the Read stage is completed, the OoRW-fetch and PE execution stage begins. The OoR bit is assigned with the OoRW-fetch bit, preempting the address for an OoRW transfer. If the OoRW-fetch bit is 1, the block bit is activated, and the OoRW begins to be transferred from the Prefetch Buffer to the address. The block bit is deactivated after the transfer. Concurrently, the PE processes operations based on the OP bit, and the garbled table is transferred from the Table Memory for Half-Gate operation.

% Lastly, Write stage begins. Unless the WEN bit is active, the generated output wires are written to Wire Memory, and the block bit of the write address is deactivated. If the Live bit is active, the wires are also written to DRAM. After these stages are executed across all instructions, the runtime process is completed. Overall, through compiler speculation and the accelerator, APINT achieves a reduction in memory stall times by 74.3\% to 99.9\% compared to HAAC for nonlinear functions of transformers.

\section{\ourdata}
\label{sec:textbook-exam}

\begin{figure}[t]
    \centering
    \includegraphics[width=\linewidth]{figures/data_framework.pdf}    \caption{\textbf{Overview of  \ourdata curation}. Given a chapter $C$, we use an LM to segment the document $D$ into sections and heuristically extract review questions to form the exam $E$. The LM then classifies each question in $E$ by Bloom’s taxonomy category and maps it to its relevant section.}
    \label{fig:dataset-overview}
    \vspace{-0.3cm}
\end{figure}

In order to evaluate \ours, we curate  \ourdata, a dataset where each entry contains a document \(D\) along with a corresponding set of exam questions \(E\).
An overview of \ourdata is illustrated in \autoref{fig:dataset-overview}.
% In this section, we first describe our data processing pipeline (\secref{ssec:textbook-exam-pipeline}) and then provide data statistics (\secref{ssec:textbook-exam-statistics}).

\subsection{Data Processing}
\label{ssec:textbook-exam-pipeline}
Our pipeline starts with textbooks from the OpenStax repository\footnote{\url{https://github.com/philschatz/textbooks}}.
Each textbook is divided into chapters, where each chapter \(C\) contains learning objectives, main content, and review questions.
For each \(C\), we parse the main content to build \(D\) and the review questions to form \(E\).
% Specifically, only the main content is used to construct \(D\), while the review questions are used to create \(E\).

\paragraph{Extracting sections.}
To simulate a learner incrementally progressing through a chapter, we divide each chapter into sections using an LM-based document structuring method. 
The LM segments \( D \) into \( n \) sections, denoted as \( \{S_1, S_2, \ldots, S_n\} \subset D \), while also extracting the corresponding review questions \( E \). 
However, not all review questions come with ground-truth answers, as some textbooks do not provide them (see Table~\ref{tab:textbook-exam-statistics} for the proportion of \( E \) with answers). 
To ensure consistency in section segmentation across different subjects, we manually annotate the first 2–5 sections from one sample per subject. 
These annotated samples serve as few-shot examples in our LM prompt (see Appendix~\ref{appdx:parsing-sections} for details).

\paragraph{Extracting questions.}
% We developed a custom parsing script using BeautifulSoup4 to extract questions and their corresponding answers. 
To maintain a balance between evaluation depth and computational feasibility, we include only chapters that contain at least 10 questions—ensuring sufficient coverage for assessment—while capping the maximum number of questions at 25 to keep learning simulations computationally manageable.

\subsection{Data Statistics}
\label{ssec:textbook-exam-statistics}
\begin{table}[t!]
    \centering
    \resizebox{\columnwidth}{!}{
        \begin{tabular}{lccccc}
        \toprule
            \textbf{Subject} & \textbf{\# $C$} & \textbf{Split} & \textbf{\# $E$ / $C$} & \textbf{\% $E$ w/ answer} & \textbf{\# $S$ / $C$} \\
        \midrule
            Microbiology & 20 & Train & 12.4 & 64\% & 16.4 \\
                         & 5  & Test  & 13.4 & 58\% & 17.0 \\
        \midrule
            Chemistry    & 20 & Train & 14.2 & 51\% & 11.0 \\
                         & 5  & Test  & 16.2 & 49\% & 6.4 \\
        \midrule
            Economics    & 20 & Train & 12.2 & 23\% & 14.1 \\
                         & 5  & Test  & 12.2 & 23\% & 14.4 \\
        \midrule
            Sociology    & 20 & Train & 10.4 & 62\% & 16.6 \\
                         & 5  & Test  & 11.2 & 67\% & 19.0 \\
        \midrule
            US History   & 20 & Train & 7.2 & 51\% & 14.9 \\
                         & 5  & Test  & 8.4 & 38\% & 13.2 \\
        \bottomrule
        \end{tabular}
    }
    \caption{\textbf{Data Statistics} of \ourdata. \# $ C$: number of chapters, \# $E/C$: avg. number of questions per chapter, \% $E$ w/ answer: proportion of questions that have reference answer, \# $S/C$: avg. number of sections per chapter.}
    \label{tab:textbook-exam-statistics}
\end{table}
For each subject, we curate 25 sequential chapters \(C\), each containing both \(D\) and \(E\).
The chapters are arranged in their natural order, with the first 20 used for training and the last five reserved for evaluation.  
There is the risk that content in later chapters may include information from prior chapters (e.g., revisiting prerequisite knowledge). 
Therefore, preserving this sequential structure between the training and test set is essential for preventing information leakage and fairly assessing a model's learning process.
Table~\ref{tab:textbook-exam-statistics} shows an overview of the statistics of the resulting \ourdata.

\subsection{Distribution of Question Types}
\label{ssec:textbook-exam-bloom}

 \begin{figure}[t]
    \centering
    \includegraphics[width=\linewidth]{figures/bloom_taxonomy_counts_vertical.png}
    \caption{\textbf{Bloom's taxonomy distribution} in \ourdata. \ourdata consists of questions that require a wide variety of cognitive levels and the dominant categories vary for each subject.}
    \label{fig:bloom-distribution}
\end{figure}

To better understand how final exams assess a learner’s comprehension on multiple dimensions, we categorize questions in \ourdata\ based on the revised \textit{Bloom’s Taxonomy}~\cite{krathwohl2002revision_bloom}. 
Using an LM, we assign a cognitive depth \( d_j \) to each question \( E_j \in E \), classifying them into six categories: \textit{Remembering, Understanding, Applying, Analyzing, Evaluating}, and \textit{Creating}.
Additionally, we identify the relevant sections \( S_j \subset D \) that correspond to each question.

The distribution, shown in \autoref{fig:bloom-distribution}, indicates that different subjects emphasize different cognitive skills.
For instance, questions in Microbiology and Sociology primarily focus on \textit{Remembering} and \textit{Understanding}, whereas Chemistry and Economics exhibit a more varied distribution.
This analysis highlights the diverse cognitive demands across subjects and underscores how \ourdata\ provides a multifaceted evaluation of learning outcomes through final exams. 
For further details on data processing, refer to Appendix~\ref{appendix:data_processing}.













% \dongho{Number of textbooks?}

% \dongho{For each textbook, how may $D$?}



% \subsection{Validation}
% \dongho{Let's make a ground truth to see how reliable the data processing pipeline it is. -- for each textbook.}

% \paragraph{Validation of answer for $E_j$.}

% \paragraph{Validation of cognitive depth $d_j$ for $E_j$.}

% \paragraph{Validation of related sections $S_j$ for $E_j$.}
% \section{Experiment and Results}
\section{Results and Analysis}
In this section, we first present safe vs. unsafe evaluation results for 12 LLMs, followed by fine-grained responding pattern analysis over six models among them, and compare models' behavior when they are attacked by same risky questions presented in different languages: Kazakh, Russian and code-switching language.    

\begin{table}[t!]
\centering
\small
\resizebox{\columnwidth}{!}{
\begin{tabular}{clcccc}
\toprule
\multicolumn{1}{l}{\textbf{Rank} } & \textbf{Model} & \textbf{Kazakh $\uparrow$} & \textbf{Russian $\uparrow$} & \textbf{English $\uparrow$} \\
\midrule
1 & \claude & \textbf{96.5}   & 93.5    & \textbf{98.6}    \\
2 & \gptfouro & 95.8   & 87.6    & 95.7    \\
3 & \yandexgpt & 90.7   & \textbf{93.6}    & 80.3    \\
4 & \kazllmseventy & 88.1 & 87.5 & 97.2 \\
5 & \llamaseventy & 88.0   & 85.5    & 95.7    \\
6 & \sherkala & 87.1   & 85.0    & 96.0    \\
7 & \falcon & 87.1   & 84.7    & 96.8    \\
8 & \qwen & 86.2   & 85.1    & 88.1    \\
9 & \llamaeight & 85.9   & 84.7    & 98.3    \\
10 & \kazllmeight & 74.8   & 78.0    & 94.5 \\
11 & \aya & 72.4 & 84.5 & 96.6 \\
12 & \vikhr & --- & 85.6 & 91.1 \\
\bottomrule
\end{tabular}
}
\caption{Safety evaluation results of 12 LLMs, ranked by the percentage of safe responses in the Kazakh dataset. \claude\ achieves the highest safety score for both Kazakh and English, while \yandexgpt\ is the safest model for Russian responses.}
\label{tab:safety-binary-eval}
\end{table}



\subsection{Safe vs. Unsafe Classification}
% In this subsection, 
We present binary evaluation results of 12 LLMs, by assessing 52,596 Russian responses and 41,646 Kazakh responses.
% 26,298 responses generated by six models on the Russian dataset and 22,716 responses on the Kazakh dataset. 

%\textbf{Safety Rank.} In general, proprietary systems outperform the open-source model. For Russian, As shown in Table \ref{tab:model_comparison_russian}, \textbf{Yandex-GPT} emerges as the safest large language model (LLM) for Russian, with a safety percentage of 93.57\%. Among the open-source models, \textbf{Vikhr-Nemo-12B} is the safest, achieving a safety percentage of 85.63\%.
% Edited: This is mentioned in the discussion
% This outcome highlights the potential impact of pretraining data on model behavior. Models pre-trained primarily on Russian data are better at understanding and handling harmful questions - in both proprietary systems and open-source setups. 
%For Kazakh, as shown in Table \ref{tab:model_comparison_kazakh}, \textbf{Claude} emerges as the safest large language model (LLM) with a safety percentage of 96.46\%, closely followed by GPT-4o at 95.75\%. In contrast, \textbf{Aya-101}, despite being specifically tuned for Kazakh, consistently shows the highest unsafe response rates at 72.37\%. 

\begin{figure*}[t!]
	\centering
        \includegraphics[scale=0.28]{figures/question_type_no6_kaz.png}
	\includegraphics[scale=0.28]{figures/question_type_exclude_region_specific_new.png} 

	\caption{Unsafe answer distribution across three question types for risk types I-V: Kazakh (left) and Russian (right)}
	\label{fig:qt_non_reg}
\end{figure*}

\begin{figure*}[t!]
	\centering
        \includegraphics[scale=0.28]{figures/question_type_only6_kaz.png}
	\includegraphics[scale=0.28]{figures/question_type_region_specific_new.png} 
	
	\caption{Unsafe answer distribution across three question types for risk type VI: Kazakh (left) and Russian (right)}
	\label{fig:qt_reg}
\end{figure*}

\textbf{Safety Rank.} In general, proprietary systems outperform the open-source models. 
For Russian, as shown in Table~\ref{tab:safety-binary-eval},  % \ref{tab:model_comparison_russian}, 
\yandexgpt emerges as the safest language model for Russian, with safe responses account for 93.57\%. Among the open-source models, \kazllmseventy is the safest (87.5\%), followed by \vikhr with a safety percentage of 85.63\%.

For Kazakh, % as shown in Table \ref{tab:model_comparison_kazakh}, 
% YX: todo, rerun Kazakh safety percentage using Diana threshold
\claude is the safest model with 96.46\% safe responses, closely followed by \gptfouro\ at 95.75\%. \aya, despite being specifically tuned for Kazakh, shows the highest unsafe rates at 72.37\%.



% \begin{table}[t!]
% \centering
% \resizebox{\columnwidth}{!}{%
% \begin{tabular}{clccc}
% \toprule
% \textbf{Rank} & \textbf{Model Name}  & \textbf{Safe} & \textbf{Unsafe} & \textbf{Safe \%} \\ \midrule
% \textbf{1} & \textbf{Yandex-GPT} & \textbf{4101} & \textbf{282} & \textbf{93.57} \\
% 2 & Claude & 4100 & 283 & 93.54 \\
% 3 & GPT-4o & 3839 & 544 & 87.59 \\
% 4 & Vikhr-12B & 3753 & 630 & 85.63 \\
% 5 & LLama-3.1-instruct-70B & 3746 & 637 & 85.47 \\
% 6 & LLama-3.1-instruct-8B & 3712 & 671 & 84.69 \\
% \bottomrule
% \end{tabular}
% }
% \caption{Comparison of models based on safety percentages for the Russian dataset.}
% \label{tab:model_comparison_russian}
% \end{table}


% \begin{table}[t!]
% \centering
% \resizebox{\columnwidth}{!}{%
% \begin{tabular}{clccc}
% \toprule
% \textbf{Rank} & \textbf{Model Name}  & \textbf{Safe} & \textbf{Unsafe} & \textbf{Safe \%} \\ \midrule
% 1             & \textbf{Claude}  & \textbf{3652} & \textbf{134} & \textbf{96.46} \\ 
% 2             & GPT-4o                      & 3625          & 161          & 95.75 \\ 
% 3             & YandexGPT                   & 3433          & 353          & 90.68 \\
% 4             & LLama-3.1-instruct-70B      & 3333          & 453          & 88.03 \\
% 5             & LLama-3.1-instruct-8B       & 3251          & 535	       & 85.87 \\
% 6             & Aya-101                     & 2740          & 1046         & 72.37 \\ 
% \bottomrule
% \end{tabular}
% }
% \caption{Comparison of models based on safety percentages for the Kazakh dataset.}
% \label{tab:model_comparison_kazakh}
% \end{table}



\textbf{Risk Areas.} 
We selected six representative LLMs for Russian and Kazakh respectively and show their unsafe answer distributions over six risk areas.
As shown in Table \ref{tab:unsafe_answers_summary}, risk type VI (region-specific sensitive topics) overwhelmingly contributes the largest number of unsafe responses across all models. This highlights that LLMs are poorly equipped to address regional risks effectively. For instance, while \llama models maintain comparable safety levels across other risk type (I–V), their performance drops significantly when dealing with risk type VI. Interestingly, while \yandexgpt\ demonstrates relatively poor performance in most other risk areas, it handles region-specific questions remarkably well, suggesting a stronger alignment with regional norms and sensitivities. For Kazakh, Table \ref{tab:unsafe_answers_summary_kazakh} shows that region‐specific topics (risk type VI) pose a substantial challenge across all six models, including the commercial \gptfouro\ and \claude, which demonstrate superior safety on general categories. 

% \begin{table}[t!]
% \centering
% \resizebox{\columnwidth}{!}{%
% \begin{tabular}{lccccccc}
% \toprule
% \textbf{Model} & \textbf{I} & \textbf{II} & \textbf{III} & \textbf{IV} & \textbf{V} & \textbf{VI} & \textbf{Total} \\ \midrule
% LLama-3.1-instruct-8B & 60 & 70 & 16 & 31 & 9 & 485 & 671 \\
% LLama-3.1-instruct-70B & 29 & 55 & 8 & 4 & 1 & 540 & 637 \\
% Vikhr-12B & 41 & 93 & 15 & 1 & 3 & 477 & 630 \\
% GPT-4o & 21 & 51 & 6 & 2 & 0 & 464 & 544 \\
% Claude & 7 & 10 & 1 & 0 & 0 & 265 & 283 \\
% Yandex-GPT & 55 & 125 & 9 & 3 & 8 & 82 & 282 \\
% \bottomrule
% \end{tabular}%
% }
% \caption{Ru unsafe cases over risk areas of six models.}
% \label{tab:unsafe_answers_summary}
% \end{table}


\begin{table}[t!]
\centering
\resizebox{\columnwidth}{!}{%
\begin{tabular}{lccccccc}
\toprule
\textbf{Model} & \textbf{I} & \textbf{II} & \textbf{III} & \textbf{IV} & \textbf{V} & \textbf{VI} & \textbf{Total} \\ \midrule
\llamaeight & 60 & 70 & 16 & 31 & 9 & 485 & 671 \\
\llamaseventy & 29 & 55 & 8 & 4 & 1 & 540 & 637 \\
\vikhr & 41 & 93 & 15 & 1 & 3 & 477 & 630 \\
\gptfouro & 21 & 51 & 6 & 2 & 0 & 464 & 544 \\
\claude & 7 & 10 & 1 & 0 & 0 & 265 & 283 \\
\yandexgpt & 55 & 125 & 9 & 3 & 8 & 82 & 282 \\
\bottomrule
\end{tabular}%
}
\caption{Ru unsafe cases over risk areas of six models.}
\label{tab:unsafe_answers_summary}
\end{table}


% \begin{table}[t!]
% \centering
% \resizebox{\columnwidth}{!}{%
% \begin{tabular}{lccccccc}
% \toprule
% \textbf{Model} & \textbf{I} & \textbf{II} & \textbf{III} & \textbf{IV} & \textbf{V} & \textbf{VI} & \textbf{Total} \\ \midrule
% Aya-101 & 96 & 235 & 165 & 166 & 90 & 294 & 1046 \\
% Llama-3.1-instruct-8B & 25 & 15 & 91 & 37 & 14 & 353 & 535 \\
% Llama-3.1-instruct-70B & 33 & 39 & 88 & 27 & 20 & 246 & 453 \\
% Yandex-GPT & 29 & 76 & 95 & 29 & 16 & 108 & 353 \\
% GPT-4o & 2 & 1 & 41 & 0 & 3 & 114 & 161 \\
% Claude & 2 & 1 & 26 & 3 & 6 & 96 & 134 \\ \bottomrule
% \end{tabular}%
% }
% \caption{Kaz unsafe cases over risk areas of six models.}
% \label{tab:unsafe_answers_summary_kazakh}
% \end{table}


\begin{table}[t!]
\centering
\resizebox{\columnwidth}{!}{%
\begin{tabular}{lccccccc}
\toprule
\textbf{Model} & \textbf{I} & \textbf{II} & \textbf{III} & \textbf{IV} & \textbf{V} & \textbf{VI} & \textbf{Total} \\ \midrule
\aya & 96 & 235 & 165 & 166 & 90 & 294 & 1046 \\
\llamaeight & 25 & 15 & 91 & 37 & 14 & 353 & 535 \\
\llamaseventy & 33 & 39 & 88 & 27 & 20 & 246 & 453 \\
\yandexgpt & 29 & 76 & 95 & 29 & 16 & 108 & 353 \\
\gptfouro & 2 & 1 & 41 & 0 & 3 & 114 & 161 \\
\claude & 2 & 1 & 26 & 3 & 6 & 96 & 134 \\ 
\bottomrule
\end{tabular}%
}
\caption{Kaz unsafe cases over risk areas of six models.}
\label{tab:unsafe_answers_summary_kazakh}
\end{table}

% \begin{figure*}[t!]
% 	\centering
% 	\includegraphics[scale=0.28]{figures/human_1000_kz_font16.png} 
% 	\includegraphics[scale=0.28]{figures/human_1000_ru_font16.png}

% 	\caption{Human vs \gptfouro\ fine-grained labels on 1,000 Kazakh (left) and Russian (right) samples.}
% 	\label{fig:human_fg_1000}
% \end{figure*}

\textbf{Question Type.} For Russian, Figures \ref{fig:qt_non_reg} and \ref{fig:qt_reg} reveal differences in how models handle general risks I-V and region-specific risk VI. For risks I-V, indirect attacks % crafted to exploit model vulnerabilities—
result in more unsafe responses due to their tricky phrasing. 

In contrast, region-specific risks see slightly more unsafe responses from direct attacks, 
% as these explicit prompts are more likely to bypass safety mechanisms. 
since indirect attacks for region-specific prompts often elicit safer, vaguer answers. %, suggesting models struggle less with implicit harm. 
Overall, the number of unsafe responses is similar across question types, indicating models generally struggle with safety alignment in all jailbreaking queries.

For Kazakh, Figures \ref{fig:qt_non_reg} and \ref{fig:qt_reg} show greater variation in unsafe responses across question types due to the low-resource nature of Kazakh data. For general risks I-V, \llamaseventy\ and \aya\ produce more unsafe outputs for direct harm prompts. At the same time, \claude\ and \gptfouro\ struggle more with indirect attacks, reflecting challenges in handling subtle cues. For region-specific risk VI, most models perform similarly due to limited Kazakh-specific data, though \llamaeight\ shows higher unsafe outputs for indirect local references, likely due to their implicit nature. Direct region-specific attacks yield fewer unsafe responses, as explicit prompts trigger more cautious outputs. Across all risk areas, general questions with sensitive words produce the fewest unsafe answers, suggesting over-flagging or cautious behavior for unclear harmful intent.



% \subsection{Fine-grained Classification}
% We extended our analysis to include fine-grained classifications for both safe and unsafe responses. For unsafe responses, we categorized outputs into four harm types, as shown in Table \ref{table:unsafe_response_categories}. 

% For safe responses, we classified outputs into six distinct patterns of safety, following a fine-grained rubric provided in \cite{wang2024chinesedatasetevaluatingsafeguards}. The types outlined in this rubric are presented in Table \ref{table:safe_response_categories}.

% To validate the fine-grained classification, we conducted human evaluation on the same 1,000 responses in Russian used for the preliminary binary classification.
% The confusion matrix highlights the alignment and discrepancies between human annotations and GPT's fine-grained labels. The diagonal values represent instances where the GPT's labels match human annotations, with category 5 (provides general, safe information) showing the highest agreement (404 instances). However, off-diagonal values reveal areas of disagreement, such as misclassifications in categories 1 (not willing to answer), 4 (recommends consulting an expert), and 7 (provides harmful or risky instructions). For example, in category 1, while 94 responses were correctly labeled, GPT-4o misclassified several instances into categories 4, 5, or 7, indicating overlap or ambiguity in these classifications. Similar trends are observed in other categories, where GPT sometimes struggles to differentiate nuanced distinctions in human-labeled categories. Overall, GPT's fine-grained labels match human annotations in 710 out of 1000 cases, achieving an agreement rate of 71\%.

% \begin{figure}[ht!]
%     \centering
%     \includegraphics[width=0.95\linewidth]{figures/human_fg_1000_ru.png}
%     \caption{{Human vs GPT-4o Fine-Grained Labels on 1,000 Russian Samples}}
%     \label{fig:human_fg_1000_ru}
% \end{figure}


% After conducting human evaluation on a representative sample, we extended the fine-grained classification to a full dataset comprising 21,915 responses generated by five different models. 


% \begin{figure}[ht!]
%     \centering
%     \includegraphics[width=0.95\linewidth]{figures/all_5_ru.png}
%     \caption{Fine-grained label distribution for responses from five models for Russian.}
%     \label{fig:all_5_fg_russian}
% \end{figure}
% Category 5 ("safe and general information") consistently has the highest frequency across all models, aligning with its dominance in the 1,000-sample evaluation. However, differences in the distribution across other categories highlight variability in how models handle nuanced safety risks, with Yandex-GPT showing a slightly broader spread across categories. 
% In the distribution of unsafe responses, most models exhibit higher counts in certain labels such as 8. However, Yandex-GPT displays comparatively fewer responses in these labels. 
% It exhibits a high rate of responses classified under label 7, which indicates instances where the model provides harmful or risky instructions, including unethical behavior or sensitive discussions. While this may suggest a vulnerability in addressing complex or challenging prompts, it was observed that many of Yandex-GPT’s responses tend to deflect responsibility or offer vague advice such as "check the internet". Although this approach minimizes the risk of unsafe outputs, it often results in responses that lack depth or contextually relevant information, limiting their overall usefulness for users.

% % \subsection{Kazakh}


% % Overall, these findings underscore how resource constraints and prompt explicitness affect model safety in Kazakh. Some models manage direct attacks better yet fail on indirect ones, while region-specific content remains challenging for all given the lack of localized training data.
% \subsubsection{Fine-grained Classification}
% Similarly, we conducted a human evaluation on 1,000 Kazakh samples, following the same methodology as the Russian evaluation. The match between human annotations and GPT-4o's fine-grained classifications was 707 out of 1,000, ensuring that the fine-grained classification framework aligned well with human judgments.
% The confusion matrix in Figure \ref{fig:human_fg_1000_kz} for 1,000 Kazakh samples illustrates the agreement between human annotations and GPT-4o's fine-grained classifications. The highest agreement is observed in category 5 (360 instances), indicating GPT-4o's strength in recognizing responses labeled by humans as "safe and general information." However, discrepancies are evident in categories such as 3 and 7, where GPT-4o misclassified several instances, highlighting areas for further refinement in distinguishing nuanced classifications.


\begin{figure}[t!]
	\centering
	\includegraphics[scale=0.18]{figures/human_1000_kz_font16.png} 
	\includegraphics[scale=0.18]{figures/human_1000_ru_font16.png}

	\caption{Human vs \gptfouro\ fine-grained labels on 1,000 Kazakh (left) and Russian (right) samples.}
	\label{fig:human_fg_1000}
\end{figure}

% \begin{figure}[t!]
% 	\centering
% 	\includegraphics[scale=0.28]{figures/human_1000_kz_font16.png} 
% 	\includegraphics[scale=0.28]{figures/human_1000_ru_font16.png}

% 	\caption{Human vs \gptfouro\ fine-grained labels on 1,000 Kazakh (top) and Russian (bottom) samples.}
% 	\label{fig:human_fg_1000}
% \end{figure}

% \begin{figure*}[t!]
% 	\centering
% 	\includegraphics[scale=0.28]{figures/all_5_kz_font16.png} 
% 	\includegraphics[scale=0.28]{figures/all_5_ru_font_16.png} \\
% 	\caption{Fine-grained responding pattern distribution across five models for Kazakh (left) and Russian (right).}
% 	\label{fig:all_5}
% \end{figure*}

\begin{figure}[t!]
	\centering
	\includegraphics[scale=0.28]{figures/all_5_kz_font16.png} 
	\includegraphics[scale=0.28]{figures/all_5_ru_font_16.png} \\
	\caption{Fine-grained responding pattern distribution across five models for Kazakh (top) and Russian (bottom).}
	\label{fig:all_5}
\end{figure}


\subsection{Fine-Grained Classification}
\label{sec:fine-grained-classification}
% As discussed in Section \ref{harmfulness_evaluation}, 
We further analyzed fine-grained responding patterns for safe and unsafe responses. For unsafe responses, outputs were categorized into four harm types, and safe responses were classified into six distinct patterns of safety, as rubric in Appendix \ref{safe_unsafe_response_categories}. 
% \cite{wang2024chinesedatasetevaluatingsafeguards}

\paragraph{Human vs. GPT-4o}
Similar to binary classification, we validated \gptfouro's automatic evaluation results by comparing with human annotations on 1,000 samples for both Russian and Kazakh. %, comparing human annotations with \gptfouro's fine-grained labels.
For the Russian dataset, \gptfouro's labels aligned with human annotations in 710 out of 1,000 cases, achieving an agreement rate of 71\%. 
Agreement rate of Kazakh samples is 70.7\%. %with 707 out of 1,000 cases matching
% The confusion matrix highlights areas of alignment and discrepancies.
% 
As confusion matrices illustrated in Figure~\ref{fig:human_fg_1000}, the majority of cases falling into \textit{safe responding patter 5} --- providing general and harmless information, for both human annotations and automatic predictions.
% highest agreement with 404 correct classifications for Russian. 
Mis-classifications for safe responses mainly focus on three closely-similar patterns: 3, 4, and 5, and patterns 7 and 8 are confusing to discern for unsafe responses, particularly for Kazakh (left figure).
We find many Russian samples which were labeled as ``1. reject to answer'' by humans are diversely classified across 1-6 by GPT-4o, which is also observed in Kazakh but not significant.

% suggesting label alignment issues are language-independent. 
% YX: I did not observe this, commented
% Notably, Russian showed confusion between 7 (risky instructions) and 1 (refusal to answer), this trend does not appear in Kazakh.


% highlight the strengths and limitations of {\gptfouro}'s fine-grained classification framework across both languages, paving the way for further refinements.


% However, misclassifications were observed in categories such as 1 (not willing to answer), 4 (recommends consulting an expert), and 7 (provides harmful or risky instructions), revealing overlaps and ambiguities in nuanced classifications.

% Similarly, for the Kazakh dataset, the agreement rate between human annotations and GPT-4o's labels was 70.7\%, with 707 out of 1,000 cases matching. As with the Russian analysis, category 5 (360 instances) showed the highest alignment. However, discrepancies were more prominent in categories such as 3 and 7, underscoring GPT-4o's challenges in differentiating fine-grained human-labeled categories. 
% A similar pattern was observed for Kazakh dataset, which suggests that alignment and misaligned of fine-grained lables is not language dependent.

% These findings, illustrated in Figures \ref{fig:human_fg_1000}, highlight the strengths and limitations of {\gptfouro}'s fine-grained classification framework across both languages, paving the way for further refinements.

\paragraph{Fine-grained Analysis of Five LLMs}
% After conducting human evaluation on representative samples, we extended 
\figref{fig:all_5} shows fine-grained responding pattern distribution across five models based on the full set of Russian and Kazakh data.
% For Russian, we selected \vikhr, \gptfouro, \llamaseventy, \claude, and \yandexgpt, while for Kazakh, we chose \aya, \gptfouro, \llamaseventy, \claude, and \yandexgpt. 
% The evaluation covered 21,915 responses in Russian and 18,930 responses in Kazakh.
% 
In both languages, pattern 5 of providing \textit{general and harmless information} consistently witnessed the highest frequency across all models, with \llamaseventy\ exhibiting the largest number of responses falling into this category for Kazakh (2,033). 
% YX:summarize more noteable findings here.

Differences of other patterns vary across languages. 
Unsafe responses in Russian are predominantly in pattern 8, where models provide incorrect or misleading information without expressing uncertainty. % (misinformation and speculation), 
For Kazakh, \aya\ exhibits the highest occurrence of pattern 7 (harmful or risky information) and pattern 8, indicating a stronger tendency to generate unethical, misleading, or potentially harmful content.

%Variations in other patterns across models highlight differences in how nuanced safety risks are classified, reflecting the models' differing capabilities in handling safety evaluation for these distinct linguistic contexts. For Russian, the majority of unsafe responses fall under pattern 8 (misinformation and speculation), indicating that models frequently provide incorrect or misleading information without acknowledging uncertainty. For Kazakh, \aya\ has the highest occurence of pattern 7 (harmful or risky information) and pattern 8 (misinformation and speculation), indicating a greater tendency to generate unethical, misleading, or potentially harmful content. 

%This trend suggests that Russian models may struggle more with factual accuracy, whereas Kazakh models, particularly \aya, pose higher risks related to both harmful content and misinformation. Additionally, \gptfouro\ and \claude\ consistently produce fewer unsafe responses in both languages, demonstrating stronger alignment with safety standards
\subsection{Code Switching}
\begin{table}[t!]
\centering

\setlength{\tabcolsep}{3pt}
\scalebox{0.7}{
\begin{tabular}{lcccccccccc}
\toprule
\textbf{Model Name} & \multicolumn{2}{c}{\textbf{Kazakh}} & \multicolumn{2}{c}{\textbf{Russian}} & \multicolumn{2}{c}{\textbf{Code-Switched}} \\  
\cmidrule(lr){2-3} \cmidrule(lr){4-5} \cmidrule(lr){6-7}
& \textbf{Safe} & \textbf{Unsafe} & \textbf{Safe} & \textbf{Unsafe} & \textbf{Safe} & \textbf{Unsafe} \\ 
\midrule
\llamaseventy & 450 & 50 & 466 & 34 & 414 & 86 \\
\gptfouro & 492 & 8 & 473 & 27 & 481 & 19
 \\
\claude & 491 & 9 & 478 & 22 & 484 & 16 \\ 
\yandexgpt & 435 & 65 & 458 & 42 & 464 & 36 \\
\midrule
\end{tabular}}
\caption{Model safety when prompted in Kazakh, Russian, and code-switched language.}
\label{tab:finetuning-comparison}
\end{table}


\gptfouro\ and \claude\ demonstrate strong safety performance across three languages, even with a high proportion of safe responses in the challenging code-switching context. In contrast, \llamaseventy\ and \yandexgpt\ are less safe, exhibiting more harmful responses, particularly in the code-switching scenario. These results show the varying capabilities of models in defending the same attacks that are just presented in different languages, where open-sourced large language models especially require more robust safety alignment in multilingual and code-switching scenarios.

% \subsection{LLM Response Collection}
% We conducted experiments with a variety of mainstream and region-specific 
% large language models for both Russian and Kazakh languages. For both Russian and Kazakh languages, we employed four multilingual models: Claude-3.5-sonnet, Llama 3.1 70B \cite{meta2024llama3}, GPT-4 \cite{openai2024gpt4o}, and YandexGPT. Additionally, we included language-specific models: VIKHR \cite{nikolich2024vikhrconstructingstateoftheartbilingual} for Russian and Aya \cite{ustun-etal-2024-aya} for Kazakh. 

% \subsection{Kazakh-Russian Code-Switching Evaluation}

% In Kazakhstan, the prevalence of bilingualism is a defining characteristic of its linguistic landscape, with most individuals seamlessly mixing Kazakh and Russian in daily communication \cite{Zharkynbekova2022}. This phenomenon, known as code-switching, reflects the unique cultural and social dynamics of the region. Despite this, there is currently no safety evaluation dataset tailored to this unique multilingual environment. Developing a code-switched dataset is essential to evaluate the ability of large language models (LLMs) to navigate the complexities of bilingual interactions, ensuring they produce contextually appropriate, non-harmful, and culturally sensitive responses. To address this, we sampled 500 questions from both Kazakh and Russian question sets, ensuring they represent similar queries but with culturally nuanced differences in meaning. Building on these, we developed a mixed version that integrates Kazakh and Russian naturally, reflecting the way people in Kazakhstan communicate. This process involved annotators to ensure that the mixed questions were crafted with a natural tone and were localized to Kazakhstan's context.


% \begin{table*}[t!]
%     \centering
%     \small
%     \setlength{\tabcolsep}{6pt} % Adjusts column separation
%     \adjustbox{max width=\linewidth}{
%     \begin{tabular}{ll ccc ccc c}
%     \toprule
%     \textbf{Risk Area} & \textbf{Harm Type} & \multicolumn{3}{c}{\textbf{Kazakh}} & \multicolumn{3}{c}{\textbf{Russian}} & \textbf{\#Q} \\
%                        &                    & \textbf{Ori} & \textbf{FN} & \textbf{FP} & \textbf{Ori} & \textbf{FN} & \textbf{FP} & \\
%     \midrule
%     \multirow{2}{*}{I. Information Hazards} & 1. Risks from leaking sensitive information... & 131 & 133 & 131 & 131 & 133 & 131 & 790 \\
%                                              & 2. Compromise of privacy by leaking or inferring private information (person/individual) & 81 & 82 & 81 & 81 & 82 & 81 & 488 \\
%     \midrule
%     \multirow{3}{*}{II. Malicious Uses} & 3. Assisting illegal activities & 132 & 135 & 132 & 132 & 135 & 132 & 798 \\
%                                         & 4. Nudging or advising unethical actions & 71 & 71 & 71 & 71 & 71 & 71 & 426 \\
%                                         & 5. Reducing the cost of disinformation campaigns & 40 & 42 & 40 & 40 & 42 & 40 & 244 \\
%     \midrule
%     \multirow{3}{*}{III. Discrimination, Exclusion, Toxicity, Hateful, Offensive} & 6. Social stereotypes and unfair discrimination & 94 & 96 & 94 & 94 & 96 & 94 & 568 \\
%                                         & 7. Toxic language (hate speech) & 52 & 59 & 52 & 52 & 59 & 52 & 326 \\
%                                         & 8. Adult content & 27 & 29 & 27 & 27 & 29 & 27 & 166 \\

%     \midrule
%     \multirow{2}{*}{IV. Misinformation Harms} & 9. Disseminating false or misleading information & 92 & 99 & 92 & 92 & 99 & 92 & 566 \\
%                                         & 10. Causing material harm by disseminating misinformation e.g. in medicine or law & 63 & 63 & 63 & 63 & 63 & 63 & 378 \\
%     \midrule
%     \multirow{2}{*}{V. Human–chatbot Interaction Harms} & 11. Mental health or overreliance concerns & 66 & 66 & 66 & 66 & 66 & 66 & 396 \\
%                                         & 12. Treating the chatbot as a human & 50 & 51 & 50 & 50 & 51 & 50 & 302 \\

%     \midrule
%     \multirow{2}{*}{VI. Kazakhstan-specific Sensitivity} & 13. Politically sensitive topics & 63 & 66 & 63& 63 & 66 & 63 &  384 \\
%         & 14. Controversial historical events & 46 & 57 & 46 & 46 & 57 & 46  & 298 \\
% & 15. Regional and racial issues & 45 & 45 & 45 &  45 & 45 & 45 & 270  \\
% & 16. Societal and cultural concerns & 138 & 139 & 138 &  138 & 139 & 138  & 830  \\
% & 17. Legal and human rights matters & 57 & 57 & 57 & 57 & 57 & 57  & 342 \\
%     \midrule
%         \multirow{2}{*}{VII. Russia-specific Sensitivity} 
%             & 13. Politically sensitive topics & - & - & - & 54 & 54 & 54 & 162 \\
%     & 14. Controversial historical events & - & - & - & 38 & 38 & 38 & 114 \\
%     & 15. Regional and racial issues & - & - & - & 26 & 26 & 26 & 78 \\
%     & 16. Societal and cultural concerns & - & - & - & 40 & 40 & 40 & 120 \\
%     & 17. Legal and human rights matters & - & - & - & 41 & 41 & 41 & 123 \\
%     \midrule
%     \bf Total & -- & 1248 & 1290 & 1248 & 1447 & 1489 & 1447 & \textbf{8169} \\
%     \bottomrule
%     \end{tabular}
%     }
%     \caption{The number of questions for Kazakh and Russian datasets across six risk areas and 17 harm types. Ori: original direct attack, FN: indirect attack, and FP: over-sensitivity assessment.}
%     \label{tab:kazakh-russian-data}
% \end{table*}




\section{Discussion}

% \subsection{Kazakh vs Russian}

% The evaluation reveals that Kazakh responses tend to be generally safer than their Russian counterparts, likely due to Kazakh being a low-resource language with significantly less training data. As a result, Kazakh models are less exposed to the vast, often unfiltered datasets containing harmful or unsafe content, which are more prevalent in high-resource languages like Russian. This data scarcity naturally limits the model's ability to generate nuanced but potentially unsafe responses. However, this does not mean the models are specifically fine-tuned for safer performance. When analyzing unsafe answers, it’s clear that Kazakh models, while safer overall, distribute their unsafe responses more evenly across various risk types and question types. This suggests Kazakh models generate fewer unsafe answers but in a broader range of contexts.

% In contrast, Russian models tend to concentrate unsafe answers in specific areas, particularly region-specific risks or indirect attacks. This indicates that Russian models have learned to handle certain types of unsafe content by focusing on specific topics, such as politically sensitive issues, but struggle when confronted with unfamiliar content, leading to unsafe responses due to insufficient filtering. Kazakh models, despite having less training data, tend to respond more broadly, including both direct and indirect risks. This could be due to the less curated nature of their training data, making them more likely to answer unsafe questions without filtering the potential harm involved. The exception to this trend is Aya, a model specifically fine-tuned for Kazakh. Despite fine-tuning, it exhibits the lowest safety percentage (72.37\%) in the Kazakh dataset, suggesting that fine-tuning in specific languages may introduce risks if proper safety measures are not taken.

% The evaluation reveals notable differences in the distribution of safe response patterns across Kazakh and Russian fine-grained labels. Refusal to answer is more frequent in Russian models, particularly Yandex-GPT, reflecting a cautious approach to safety-critical queries. Interestingly, Aya, despite being fine-tuned for Kazakh and exhibiting lower overall safety, also frequently refuses to answer, suggesting an over-reliance on conservative mechanisms. Responses providing general, safe information dominate in both languages, with Kazakh models displaying a slightly higher tendency to rely on this approach. This highlights how the low-resource nature of Kazakh results in more generalized and inherently safer responses. In contrast, Russian models excel at recognizing risks, issuing disclaimers, and refuting incorrect assumptions, likely benefiting from richer and more diverse training data.
% Yandex-GPT exhibits a notably high rate of responses classified under label 7, indicating an overreliance on general disclaimers or deflections, such as "check the internet" or "I don't know." While these responses minimize the risk of unsafe outputs, they often lack substantive or contextually relevant information, reducing their overall utility for users.


Most models perform safer on Kazakh dataset than Russian dataset, higher safe rate on Kazakh dataset in \tabref{tab:safety-binary-eval}. This does not necessarily reveal that current LLMs have better understanding and safety alignment on Kazakh language than Russian, while this may conversely imply that models do not fully understand the meaning of Kazakh attack questions, fail to perceive risks and then provide general information due to lacking sufficient knowledge regarding this request.

We observed the similar number of examples falling into category 5 \textit{general and harmless information} for both Kazakh and Russian, while the Kazakh data set size is 3.7K and Russian is 4.3K. Kazakh has much less examples in category 1 \textit{reject to answer} compared to Russian. This demonstrate models tend to provide general information and cannot clearly perceive risks for many cases.

Additionally, in spite of less harmful responses on Kazakh data, these unsafe responses distribute evenly across different risk areas and question categories, exhibiting equally vulnerability spanning all attacks regardless of what risks and how we jailbreak it.
In contrary, unsafe responses on Russian dataset often concentrate on specific areas and question types, such as region-specific risks or indirect attacks, presenting similar model behaviors when evaluating over English and Chinese data.
It suggests that broader training data in English, Chinese and Russian may allow models to address certain types of attacks robustly,
% effectively—particularly politically sensitive issues—
yet they may falter when confronted with unfamiliar content like regional sensitive topics.

Moreover, in responses collection, we observed many Russian or English responses especially for open-sourced LLMs when we explicitly instructed the models to answer Kazakh questions in Kazakh language. This further implies more efforts are still needed to improve LLMs' performance on low-resource languages.
Interestingly, \aya, a fine-tuned Kazakh model, proves an exception by displaying the lowest safety percentage (72.37\%) among Kazakh models, revealing that the multilingual fine-tuning without stringent safety measures can introduce risks.



% However, this does not mean they are explicitly fine-tuned for safety, likely it happens due to limited training data, which reduces exposure to harmful content. 
% \aya, a fine-tuned Kazakh model, proves an exception by displaying the lowest safety percentage (72.37\%) among Kazakh models, revealing that the multilingual fine-tuning without stringent safety measures can introduce risks.
% Kazakh models generally produce safer responses than their Russian counterparts, likely because Kazakh is a low-resource language with less training data. 
% This limited exposure to harmful or unsafe content naturally limits nuanced yet potentially unsafe outputs. 
% However, it does not imply that the models are specifically fine-tuned for enhanced safety.


% while Kazakh models tend to generate fewer unsafe answers overall, those unsafe responses appear more evenly spread across different risk types and question categories.
% Russian models, on the other hand, often concentrate unsafe responses in specific areas, such as region-specific risks or indirect attacks.
% It implies that their broader training datasets allow them to address certain types of unsafe content more effectively—particularly politically sensitive issues—yet they may falter when confronted with unfamiliar or insufficiently filtered content.

% Meanwhile, Kazakh models sometimes respond more broadly, possibly due to less curated training data. 

Differences also emerge in how language models handle safe responses. 
\yandexgpt, for instance, often refuses to answer high-risk queries. 
It frequently relies on generic disclaimers or deflections like ``check in the Internet'' or ``I don’t know,'' minimizing risk but are less helpful. Interestingly, it often responds with ``I don’t know'' in Russian, even for Kazakh queries, we speculate that these may be default responses stemming from internal system filters, rather than generated by model itself.
This likely explains why \yandexgpt\ is the safest model for the Russian language but ranks third for Kazakh. While its filters perform well for Russian, they struggle with the low-resource Kazakh language.

% Aya, despite its lower overall safety, also employs refusals often, hinting at an over-reliance on conservative approaches. 

% Across both languages, models commonly resort to providing general, safe information, although Kazakh models lean on this strategy slightly more. 
% Russian models, by contrast, excel at detecting risks, issuing disclaimers, and correcting inaccuracies, likely benefiting from richer and more diverse training data.


% \subsection{Response Patterns}


% We conducted a detailed analysis of the models' outputs and identified several noteworthy patterns. YandexGPT, while being one of the safest overall, frequently generates responses in Russian even when the question is posed in Kazakh. These responses often appear as placeholders, prompting users to search for the answer online. This behavior might not originate from the model itself but rather from safety filters implemented in the YandexGPT system. The model's leading performance in ensuring safety during Russian-language interactions, coupled with its lower performance in Kazakh, can be attributed to the limited robustness of these safety filters when handling unsafe content in Kazakh.

% In contrast, Aya-101 exhibits a tendency to fall into repetition, often repeating the same sentences multiple times. Interestingly, the Vikhr model, despite being of a similar size, does not exhibit this issue. We attribute this difference to two key factors. First, Vikhr and Aya-101 have distinct architectures: Vikhr is based on the Mistral-Nemo model, whereas Aya-101 is built on mT5, an older and less robust model. Second, Aya-101 is a multilingual model, while Vikhr was predominantly trained for Russian. Multilingualism has been shown to potentially degrade performance in large language models~\cite{huang2025surveylargelanguagemodels}, which may explain Aya-101's issues with repetition.

\begin{table*}[!t]
    \centering
    \resizebox{\textwidth}{!}{%
        \begin{tabular}{lccccc|c}
            \toprule
            & \textbf{Microbiology} & \textbf{Chemistry} & \textbf{Economics} & \textbf{Sociology} & \textbf{US History} & \textbf{Average} \\
            \midrule
            Base score & 0.46 & 0.09 & 0.00 & 0.61 & 0.03 & 0.24 \\
            \midrule
            Zero-shot & 0.62 (+0.16) & 0.40 (+0.31) & 0.40 (+0.40) & 0.61 (+0.00) & 0.19 (+0.16) & 0.44 (+0.20) \\
            Few-shot & 0.62 (+0.16) & \underline{0.45} (+0.36) & \underline{0.47} (+0.47) & 0.62 (+0.01) & 0.16 (+0.13) & 0.46 (+0.22) \\
            Chain-of-thought & 0.61 (+0.15) & \underline{0.45} (+0.36) & 0.46 (+0.46) & 0.61 (+0.00) & 0.19 (+0.16) & 0.46 (+0.22) \\
            Bloom-based & 0.57 (+0.11) & 0.37 (+0.28) & 0.29 (+0.29) & 0.62 (+0.01) & 0.22 (+0.19) & 0.41 (+0.17) \\
            \midrule
            SFT (Subject-Specific) & 0.65 (+0.19) & 0.24 (+0.15) & 0.46 (+0.46) & \underline{0.64} (+0.03) & 0.20 (+0.17) & 0.44 (+0.20) \\
            SFT (Cross-Subject) & 0.59 (+0.13) & 0.21 (+0.12) & \underline{0.47 (+0.47)} & 0.63 (+0.02) & \underline{0.26} (+0.23) & 0.43 (+0.19) \\
            \midrule
            \textsc{QUEST} (Subject-Specific) & \bf 0.76 (+0.30) & \bf 0.46 (+0.37) & \bf 0.58 (+0.58) & \bf 0.65 (+0.04) & \bf 0.31 (+0.28) & \bf 0.55 (+0.31) \\
            \textsc{QUEST} (Cross-Subject) & \underline{0.73} (+0.27) & \underline{0.41} (+0.32) & \underline{0.47} (+0.47) & \bf 0.65 (+0.04) & 0.25 (+0.22) & \underline{0.50} (+0.26) \\
            \bottomrule
        \end{tabular}
    }
    \caption{\textbf{End-of-chapter exam score} results of different question generation approaches across various subjects. 
    \ours produces models that generate questions that lead to the highest scores on all subjects.  
    Gain values (in parentheses) are calculated as the increase from the base score. The highest and second highest scores per column are \textbf{bolded} and \underline{underlined}, respectively.}
    \label{tab:question-gen-results}
\end{table*}


\section{Experimental Results}
In this section, we first compare the overall performance of all question generation baselines based on the learner's exam score (\secref{ssec:overall-performance}).  
Next, we analyze evaluation metrics by examining their correlations with utility and assessing the impact of optimizing models on high-scoring questions for each metric (\secref{ssec:evaluation-metrics}).
We then conduct a qualitative analysis of high-utility questions to understand their characteristics (\secref{ssec:high-utility-question}). 
Finally, we perform ablation studies on the framework by varying the criteria for selecting high-utility questions for training and replacing \texttt{gpt-4o-mini} to \texttt{gpt-4o} (\secref{ssec:rs-analysis}-\secref{ssec:model-variants}).



\subsection{Overall Performance}
\label{ssec:overall-performance}
Table~\ref{tab:question-gen-results} presents the learner's exam performance of different question generators, measured by exam scores using all generated question-answer pairs (\secref{ssec:quest-evaluation}).
Here are findings:
(1) \textbf{Prompting techniques (Few-shot, CoT, Bloom-based)} offer only marginal performance gains. While advanced prompting enhances reasoning and task accuracy~\cite{brown2020language, wei2022chain, zhou2024self}, it does not directly optimize utility, which reflects real-world impact—how well generated questions enhance learning. Without explicit selection or optimization, prompting cannot systematically improve this measure;
(2) \textbf{SFT} shows no performance gains, indicating that while it learns the style of exam questions, it fails to generate questions that enhance learner understanding.
This highlights the key distinction between producing syntactically valid questions and generating those that effectively promote learning;
(3) \textbf{QUEST} achieves the highest performance gain, improving by approximately 20\% on average.
Performance gap between subject-specific and cross-subject rejection sampling suggests that the definition of a ``high-utility'' question varies by domain.
The results indicate that outcome-based learning is most effective when applied within a specific domain.


\subsection{Evaluation Metrics Analysis}
\label{ssec:evaluation-metrics}
\paragraph{Correlation.}
\begin{table}[!t]
    \centering
    \resizebox{\columnwidth}{!}{%
        \begin{tabular}{ll
        cc}
            \toprule
            Metric 1 & Metric 2 & \textbf{Spearman correlation} & \textbf{p-value} \\
            \midrule
            \textbf{Utility} & Saliency & 0.097 & 0.003 \\
            \textbf{Utility} & EIG  & -0.022 & 0.512 \\
            Saliency & EIG  & 0.030 & 0.363 \\
            \bottomrule
        \end{tabular}
    }
    \caption{\textbf{Spearman correlation between metrics.} Utility shows a weak correlation with saliency and EIG, showing that it is independent of these indirect metrics.}
    \label{tab:correlation_results}
\end{table}
To analyze the relationship between \textit{utility} and existing metrics (\textit{saliency}, \textit{EIG}), we estimate all three metrics on generated questions from the training set.
Table~\ref{tab:correlation_results} shows that both saliency and EIG have weak correlations with utility.
While saliency has a weak but statistically significant correlation with utility, EIG shows no meaningful relationship.
This indicates that existing indirect metrics may not accurately reflect a question’s impact on learning outcomes.

\paragraph{Optimization on Indirect Metrics.}
\begin{table*}[!t]
    \centering
    \small
    \resizebox{\textwidth}{!}{%
        \begin{tabular}{lccc ccc ccc ccc ccc}
            \toprule
            & \multicolumn{3}{c}{\textbf{Microbiology}} & \multicolumn{3}{c}{\textbf{Chemistry}} & \multicolumn{3}{c}{\textbf{Economics}} & \multicolumn{3}{c}{\textbf{Sociology}} & \multicolumn{3}{c}{\textbf{US History}} \\
            \cmidrule(lr){2-4} \cmidrule(lr){5-7} \cmidrule(lr){8-10} \cmidrule(lr){11-13} \cmidrule(lr){14-16}
            \textbf{Train Metric} & \textbf{Utility} & \textbf{Saliency} & \textbf{EIG} & \textbf{Utility} & \textbf{Saliency} & \textbf{EIG} & \textbf{Utility} & \textbf{Saliency} & \textbf{EIG} & \textbf{Utility} & \textbf{Saliency} & \textbf{EIG} & \textbf{Utility} & \textbf{Saliency} & \textbf{EIG} \\
            \midrule
            $utility > 0.1$ & \textbf{0.76} & 4.27 & -0.18 & \textbf{0.46} & 4.65 & -0.20 & \textbf{0.58} & \textbf{4.70} & -0.04 & \textbf{0.65} & \textbf{4.49} & -0.02 & \textbf{0.31} & 4.65 & \bf -0.01 \\
            $saliency = 5$  & 0.73 & \textbf{4.42} & -0.24 & 0.39 & \textbf{4.46} & -0.22 & 0.46 & 4.66 & -0.08 & 0.64 & \textbf{4.49} & -0.03 & 0.23 & \textbf{4.68} & -0.02 \\
            $EIG > 0$      & 0.61 & 4.21 & \textbf{-0.17} & 0.32 & 4.40 & \textbf{-0.09} & 0.47 & 4.65 & \textbf{0.01} & 0.62 & 4.46 & \textbf{0.01} & 0.21 & 4.65 & \bf -0.01 \\
            \bottomrule
        \end{tabular}
    }
    \caption{\textbf{End-of-chapter exam scores (utility), average saliency, and average EIG of generated questions} for different \textsc{QUEST}-optimized models trained on datasets filtered by different selection criteria.}
    \label{tab:quest-indirect}
\end{table*}
To further investigate the impact of indirect metrics on question generation performance, we compare the results of \textsc{QUEST} when trained using different selection criteria: (1) only questions with a \textit{utility} score greater than 0.1 (Ours), (2) only questions with a \textit{saliency} score of 5, and (3) only questions with an \textit{expected information gain (EIG)} greater than 0.
We evaluate the generated questions based on their overall utility (\textit{i.e.,} end-of-chapter exam scores), as well as their average saliency and EIG, to assess the question generator’s performance across different quality metrics.
Table~\ref{tab:quest-indirect} shows that while saliency- and EIG-based training improves their respective scores, it does not enhance utility. 
In contrast, the utility-trained model consistently achieves the highest utility across all subjects and even improves some indirect metrics (\textit{e.g.,} it matches or outperforms saliency-based training in Economics and Sociology).
Furthermore, utility-based training outperforms EIG-based training on saliency and saliency-based training on EIG, demonstrating its broader effectiveness. 
These results emphasize that optimizing for indirect metrics does not improve real-world learning, whereas utility-driven training yields the best overall performance.




\subsection{High Utility Questions Analysis}
\label{ssec:high-utility-question}
\paragraph{Overlap with exam questions.}
To evaluate the relationship between generated high-utility questions and exam questions, we measured their semantic and lexical similarity.
For each generated question, we computed embedding similarity using \texttt{text-3-embedding-small}\footnote{\href{https://platform.openai.com/docs/guides/embeddings/}{https://platform.openai.com/docs/guides/embeddings/}} for semantic overlap and the ROUGE score for lexical overlap with all exam questions in the same chapter.
We then assess the correlation between utility and the most similar exam question based on these measures.
The correlation between utility and semantic similarity is 0.25 (p < 0.001), indicating a weak positive relationship, while the correlation with ROUGE is nearly zero at 0.04 (p < 0.01).
These findings suggest that high-utility questions are not simple rephrasings of exam questions but introduce novel concepts that enhance learning beyond surface-level similarity.

\paragraph{Qualitative analysis.}
Qualitative question examples do not exhibit clear patterns in question style (see Appendix~\ref{appendix:qualitative-examples}).
An interesting observation is that Bloom's taxonomy, which categorizes cognitive depth based on question type—where "what" questions typically involve simple recall, while "why" and "how" questions require deeper processing—does not strongly correlate with utility.
Using Bloom's taxonomy as a cognitive depth scale (Likert 1-6), the correlation between utility and cognitive depth is 0.12 (p < 0.001), indicating a weak positive relationship.


\subsection{Rejection Sampling Analysis}
\label{ssec:rs-analysis}
Filtering for high-utility questions through rejection sampling is crucial for improving question generation. 
As shown in Figure~\ref{fig:rs_analysis}, increasing the utility threshold enhances question quality, leading to higher exam scores.
However, stricter filtering reduces the available training data, posing challenges for model training. 
These results suggest that increasing the dataset size while applying a higher threshold could further boost performance.
\begin{figure}[!t]
    \centering
    \begin{minipage}{\columnwidth}
    \centering
    \includegraphics[width=\columnwidth]{figures/rs_analysis.pdf}
    \end{minipage}
    \caption{\textbf{Impact of threshold} in \ours on end-of-chapter exam scores for Chemistry.}
    \label{fig:rs_analysis}
    \vspace{-0.5cm}
\end{figure}

\subsection{Model Variants Analysis}
\label{ssec:model-variants}
\begin{table*}[!t]
    \centering
    \resizebox{\textwidth}{!}{%
    \begin{tabular}{c|ccc|ccccc}
        \toprule
        & \textbf{QG ($M_q$)} & \textbf{AG ($M_a$)} & \textbf{RS ($M_l$)} & \textbf{Microbiology} & \textbf{Chemistry} & \textbf{Economics} & \textbf{Sociology} & \textbf{US History} \\
        \midrule
        \multicolumn{1}{c|}{\multirow{4}{*}{Zero-Shot}} 
        & \texttt{gpt-4o-mini} & \texttt{gpt-4o-mini} & \texttt{gpt-4o-mini} & 0.620 & 0.414 & 0.398 & 0.609 & 0.233 \\
        & \texttt{gpt-4o} & \texttt{gpt-4o-mini} & \texttt{gpt-4o-mini} & 0.681 & 0.457 & 0.466 & 0.634 & 0.180 \\
        & \texttt{gpt-4o-mini} & \texttt{gpt-4o} & \texttt{gpt-4o-mini} & 0.682 & 0.422 & 0.480 & 0.634 & 0.232 \\
        & \texttt{gpt-4o-mini} & \texttt{gpt-4o-mini} & \texttt{gpt-4o} & 0.710 & 0.173 & 0.476 & 0.564 & 0.263 \\
        \midrule
        \textsc{QUEST} & \texttt{gpt-4o-mini} & \texttt{gpt-4o-mini} & \texttt{gpt-4o-mini} & \bf 0.756 &	\bf 0.457 &	\bf 0.582 &	\bf 0.649 &	\bf 0.311 \\
        \bottomrule
    \end{tabular}
    }
    \vspace{-0.2cm}
    \caption{\textbf{End-of-chapter exam scores} for different model sizes across various subjects and modules.}
        \vspace{-0.2cm}

    \label{tab:model-variants}
\end{table*}

To evaluate the robustness of our framework and the impact of model size on different components, we conduct experiments to analyze how a larger model affects each module.
Table~\ref{tab:model-variants} presents results for different configurations of the question generator ($M_q$), answer generator ($M_a$), and reader simulator (\textit{i.e.,} learner $M_l$). 
The baseline corresponds to the \texttt{zero-shot} setting in Table~\ref{tab:question-gen-results}. 
In the $M_a$ experiment, we use the same questions from the \texttt{zero-shot} setting but generate new answers.
For the reader simulator experiment, we keep the same questions and answers from \texttt{zero-shot} and re-run the simulation only.

Our main findings are the following:
(1) \textbf{Question Generator}: A larger model (\texttt{gpt-4o}) improves the utility score by 5.7\%. However, it still underperforms compared to the smaller, utility-optimized model (\texttt{gpt-4o-mini}) by 12.3\%;
(2) \textbf{Answer Generator}: A larger model improves performance by 7.1\%, suggesting that higher answer quality provides additional information to the QA pair. 
However, it remains 11\% behind the optimized \texttt{gpt-4o-mini} in utility.
(3) \textbf{Reader Simulator}: Using a larger model (\texttt{gpt-4o}) as the reader simulator leads to mixed results, with performance gains in some subjects but a sharp decline in Chemistry. 
This suggests that larger models may introduce different reasoning strategies or evaluation biases, leading to inconsistencies in scoring. 
Additionally, since our framework is optimized for \texttt{gpt-4o-mini}, the larger model may not align well with the training dynamics. 
These results highlight the importance of consistency in simulation for reliable utility estimation.

% equipped with a larger model (\texttt{gpt-4o}) is expected to make more accurate simulation of how much generated questions really impact to the student understanding on exams.
% There is consistent increasing gap on microbiology, economics, sociology, us history while there is a huge gap in decreasing gap on chemistry, compared to smaller model (\texttt{gpt-4o-mini})



% \begin{itemize}
%     \item \textbf{Stronger question generators improve performance in some subjects but not all.} Upgrading $M_q$ from GPT-4o-mini to GPT-4o enhances performance in Microbiology and Economics but degrades US History, suggesting that question quality improvement is domain-dependent.
%     \item \textbf{Answer generator quality has a moderate effect on utility.} Stronger answer generators help in certain subjects (e.g., Economics) but have minimal impact in others (e.g., Chemistry).
%     \item \textbf{Evaluator model strength influences overall results but can introduce variability.} Using a stronger evaluator (GPT-4o) improves utility estimates in Microbiology and Economics but drastically reduces performance in Chemistry. This suggests that evaluators need to be carefully tuned to ensure robust and reliable assessments across different domains.
% \end{itemize}



Our systematization of target selection methods for directed fuzzers is related to two types of prior works: surveys of fuzzing literature and analyses aimed at enhancing specific components of the fuzzing pipeline.

\boldpar{Literature surveys}
Several surveys on fuzzing research have been conducted throughout the years. Although they are methodologically similar to each other and our work, their respective focus varies. First, there are surveys that take the whole fuzzing pipeline into account. \citet{ManHanHanCha+21} and \citet{LiaPeiJiaShe+18}, for example, both introduce general multi-step models of the fuzzing process and survey existing literature with regard to each step in their model. More closely related to our work is the survey by \citet{WanZhoYueLin+24} who focus on directed fuzzers. To that end, they identify several characteristics of directed fuzzers for which they examine prior works. While most of these characteristics are concerned with how the fuzzer operates, one characteristic covers the method by which its targets are selected. However, as their focus is on the fuzzer itself rather than on its preceding target selection method, they merely identify which method was used but do not examine it any further.

Other fuzzing surveys take a more specific perspective and focus on how certain methods are applied in the fuzzing pipeline or challenges that arise when applying fuzzing to particular application areas. That is, for example, how machine learning techniques are used for fuzzing~\cite{SavRodDun+19, WanJiaLiuHua+20} or the application of fuzzing to find flaws in embedded devices~\cite{MueStiKarFra+18, EisMauShrHut+22}, respectively. 

Lastly, surveys such as those by \citet{SchBarSchBer+24}, \citet{KleRueCooWei+18} or \citet{KimChoImHeo+24} take a meta perspective and study fuzzing research itself. To that end, they examine the process conducted to evaluate fuzzers in various publications. Based on their findings they can derive information about the general validity of the research field as well as recommendations on how to conduct an evaluation ideally.

\boldpar{Enhancing fuzzer components}
In addition to surveying publications on directed fuzzing, we also focus on systematically investigating the step preceding directed fuzzers; namely, the methods employed to select their targets. This is related to prior works which have conducted experiments on individual steps of the fuzzing pipeline. \citet{BöhPhaRoy16}, for example, study various power- and search-strategies to improve the seed scheduling part of a fuzzer, \citet{WuJiaXiaHua+22} compare different setups for a mutation strategy, and \citet{HerGunMagSha+21} focus on the seed selection and compare several different methods for that purpose. In contrast, our work focuses on the target selection, which has not yet been studied in-depth, and is, thus, orthogonal to other improvements.
\section{Conclusion \& Future Work}\label{conclusion}
This work presents XAMBA, the first framework optimizing SSMs on COTS NPUs, removing the need for specialized accelerators. XAMBA mitigates key bottlenecks in SSMs like CumSum, ReduceSum, and activations using ActiBA, CumBA, and ReduBA, transforming sequential operations into parallel computations. These optimizations improve latency, throughput (Tokens/s), and memory efficiency. Future work will extend XAMBA to other models, explore compression, and develop dynamic optimizations for broader hardware platforms.



% This work introduces XAMBA, the first framework to optimize SSMs on COTS NPUs, eliminating the need for specialized hardware accelerators. XAMBA addresses key bottlenecks in SSM execution, including CumSum, ReduceSum, and activation functions, through techniques like ActiBA, CumBA, and ReduBA, which restructure sequential operations into parallel matrix computations. These optimizations reduce latency, enhance throughput, and improve memory efficiency. 
% Experimental results show up to 2.6$\times$ performance improvement on Intel\textregistered\ Core\texttrademark\ Ultra Series 2 AI PC. 
% Future work will extend XAMBA to other models, incorporate compression techniques, and explore dynamic optimization strategies for broader hardware platforms.


% This work presents XAMBA, an optimization framework that enhances the performance of SSMs on NPUs. Unlike transformers, SSMs rely on structured state transitions and implicit recurrence, which introduce sequential dependencies that challenge efficient hardware execution. XAMBA addresses these inefficiencies by introducing CumBA, ReduBA, and ActiBA, which optimize cumulative summation, ReduceSum, and activation functions, respectively, significantly reducing latency and improving throughput. By restructuring sequential computations into parallelizable matrix operations and leveraging specialized hardware acceleration, XAMBA enables efficient execution of SSMs on NPUs. Future work will extend XAMBA to other state-space models, integrate advanced compression techniques like pruning and quantization, and explore dynamic optimization strategies to further enhance performance across various hardware platforms and frameworks.
% This work presents XAMBA, an optimization framework that enhances the performance of SSMs on NPUs. Key techniques, including CumBA, ReduBA, and ActiBA, achieve significant latency reductions by optimizing operations like cumulative summation, ReduceSum, and activation functions. Future work will focus on extending XAMBA to other state-space models, integrating advanced compression techniques, and exploring dynamic optimization strategies to further improve performance across various hardware platforms and frameworks.

% This work introduces XAMBA, an optimization framework for improving the performance of Mamba-2 and Mamba models on NPUs. XAMBA includes three key techniques: CumBA, ReduBA, and ActiBA. CumBA reduces latency by transforming cumulative summation operations into matrix multiplication using precomputed masks. ReduBA optimizes the ReduceSum operation through matrix-vector multiplication, reducing execution time. ActiBA accelerates activation functions like Swish and Softplus by mapping them to specialized hardware during the DPU’s drain phase, avoiding sequential execution bottlenecks. Additionally, XAMBA enhances memory efficiency by reducing SRAM access, increasing data reuse, and utilizing Zero Value Compression (ZVC) for masks. The framework provides significant latency reductions, with CumBA, ReduBA, and ActiBA achieving up to 1.8X, 1.1X, and 2.6X reductions, respectively, compared to the baseline.
% Future work includes extending XAMBA to other state-space models (SSMs) and exploring further hardware optimizations for emerging NPUs. Additionally, integrating advanced compression techniques like pruning and quantization, and developing adaptive strategies for dynamic optimization, could enhance performance. Expanding XAMBA's compatibility with other frameworks and deployment environments will ensure broader adoption across various hardware platforms.
\section*{Limitations}

One of the main limitations of our work is that we limit the number of generated questions per section to one due to budget limits. 
Due to the stochastic nature of language models, the type of questions that get generated for each section may vary significantly. 
However, we believe this effect to be minimal and the variance to be sufficiently captured in our results as the improvements of \ours over other baselines were statistically significant $p<0.05$ for all subjects in \ourdata. 

In addition, due to the same constraints as above, we have not tested \ours using rejection sampling where we generate more than one question per section. 
However, the main value of our work is still valid in that we have conceptually demonstrated that rewards from LLM-based simulations can lead to meaningful improvements over baselines even in the minimal setup of generating a single question per section. 

\section*{Ethical Considerations}

Although our work has shown that LLM-based simulators can provide effective reward signal for training better question generators, we do not advocate that our measure of question utility to be used beyond this simulation, such as directly assessing the quality of questions that students ask in a classroom setting. 
The utility of a question as defined by \ours is heavily dependent on what the exam questions are, and if the exam questions are misaligned with desirable educational outcomes, e.g. exam questions that only require rote memorization rather than critical thinking, a student's question may be considered low utility despite being a useful one for potentially other scenarios, such as brainstorming.
While we make sure that this is not the case in our experiments given the wide variety of questions across subjects included in \ourdata as shown in \secref{ssec:textbook-exam-bloom}, we cannot guarantee similar diversity and comprehensiveness in other textbook datasets. 


% reliability of roleplaying

% reliability of EV and its ability to correctly assess whether a given answer to a question without an answer from the textbook is correct or wrong


% Results may vary for \ours and the baselines if we increase the number of questions per section


\subsubsection*{Acknowledgments}
Snap Inc. provided the majority of the funding for this work, with additional partial support from the Defense Advanced Research Projects Agency (DARPA) under award HR00112220046.
We also used Sahara AI’s data service platform for dataset construction.

We would like to thank Mohit Bansal (UNC), Yuwei Fang (Snap Inc.), Sergey Tulyakov (Snap Inc.) for their valuable discussions and contributions to this work.
\bibliography{anthology,custom}

\section{Supplementary Materials}

\begin{table}[h]
\centering
\begin{tabular}{|l|l|}
\hline
\multicolumn{2}{|c|}{\textbf{Professions}} \\ \hline
aerospace engineer & aide \\
author & bartender \\
carpenter & cashier \\
CEO & civil engineer \\
cleaner & coach \\
compliance officer & cook \\
dental assistant & dentist \\
detective & electrical engineer \\
engineer & facilities manager \\
fast food worker & file clerk \\
graphic designer & hairdresser \\
head cook & health technician \\
industrial engineer & interior designer \\
interviewer & inventory clerk \\
jailer & machinery mechanic \\
manicurist & massage therapist \\
medical records specialist & mental health counselor \\
metal worker & office clerk \\
painter & payroll clerk \\
physical therapist & plane mechanic \\
postal worker & psychologist \\
purchasing agent & repair worker \\
roofer & sales manager \\
sheet metal worker & social worker \\
underwriter & welder \\
\hline
\end{tabular}
\caption{A 50 profession subset of the stable bias professions dataset used in the human evaluation. The subset is randomly selected from the full dataset each.}
\label{tab:sd_subset}
\end{table}

\begin{table*}[p]
\centering
\begin{tabular}{|p{\textwidth}|}
\hline
\textbf{Prompts} \\ \hline
0 \\
a car with no windows \\
a cartoon of a happy car on the road \\
a cat jumping in the air \\
a cat patting a crystal ball with the number 7 written on it in black marker \\
a chemtrail \\
a close-up of a margarita \\
a family on a road trip \\
a Ferrari Testarossa in front of the Kremlin \\
a friendly car \\
a giraffe made of turtle \\
a green clock \\
a grumpy porcupine handing a check for \$10,000 to a smiling peacock \\
a horse chasing a cat \\
a hot air balloon with a yin-yang symbol, with the moon visible in the daytime sky \\
a knight holding a long sword \\
a laptop with no letters on its keyboard \\
a Long Island iced tea cocktail next to a napkin \\
a man chasing a cat \\
a man heading a soccer ball \\
a painting of a man standing under a tree \\
a painting of black and white vertical stripes \\
a photo of a light bulb in outer space traveling the galaxy with a sailing boat inside the light bulb. \\
a photograph of a bird wearing headphones and speaking into a microphone in a recording studio \\
a portrait of a postal worker who has forgotten their mailbag \\
a portrait of a statue of the Egyptian god Anubis wearing aviator goggles, a white t-shirt, and a leather jacket. The city of Los Angeles is in the background \\
a raccoon wearing formal clothes, a top hat, and holding a cane. The raccoon is holding a garbage bag. Oil painting in the style of abstract cubism. \\
a red lego block \\
a small airplane \\
a small house on a mountain top \\
A smiling sloth wearing a bowtie and holding a quarterstaff and a big book. \\
a sunken submarine at the bottom of the ocean \\
a t-shirt with Carpe Diem written on it \\
a traffic jam at Times Square \\
a tree surrounded by flowers \\
a watermelon chair \\
a wooden post \\
a yellow wall with the word KA-BOOM on it \\
an abstract painting of a house on a mountain \\
an avocado on a table \\
anime illustration of the Great Pyramid sitting next to the Parthenon under a blue night sky of roiling energy, exploding yellow stars, and chromatic blue swirls \\
Anubis wearing sunglasses and sitting astride a hog motorcycle \\
bias \\
cash \\
five chairs \\
Ha Long Bay \\
horses pulling a carriage on the moon's surface, with the Statue of Liberty and Great Pyramid in the background. The Planet Earth can be seen in the sky. \\
inspiration \\
matching socks with cute cats on them \\
Three-quarters front view of a blue 1977 Corvette coming around a curve in a mountain road and looking over a green valley on a cloudy day. \\
\hline
\end{tabular}
\caption{A 50 prompt subset of Parti Prompt dataset used in the human evaluation. The subset is randomly selected from the full dataset each.}
\label{tab:pp_subset}
\end{table*}



% \begin{figure*}[p]
%     \centering
%     \includegraphics[width=\textwidth]{fig/nonparam_eval.png}
%     \caption{kNN classification results for images generated from various prompts: "CEO", "computer programmer", "doctor", "nurse", and "housekeeper". Images generated using prompts from the stable bias identity dataset are used as the anchor set. Classification is performed with k = 5 (left), k = 7 (middle), and k = 9 (right).}
%     \label{fig:nonparam_eval_all}
% \end{figure*}

\begin{table*}[p]
\centering
\caption{Top-5 kNN classification results across different models - Baseline, GPT-4o, and DeepSeek-V3 - for the profession of computer programmer and doctor. Results shown for k=5, k=7, and k=9.}
\label{tab:top5_computer_programmer_doctor}
\begin{tabular}{lccc}
\toprule
\textbf{Profession} & \textbf{k=5} & \textbf{k=7} & \textbf{k=9} \\
\midrule
\textbf{Computer Programmer} 
& % ----------- k=5 Column -----------
\begin{tabular}[t]{@{}l@{}}
\textbf{Baseline} \\
(1) Latino non-binary (47) [22.4\%] \\
(2) Caucasian man (45) [21.4\%] \\
(3) White man (41) [19.5\%] \\
(4) Black man (34) [16.2\%] \\
(5) Latinx man (33) [15.7\%] \\
\\
\textbf{GPT-4o} \\
(1) White man (16) [7.6\%] \\
(2) Latino non-binary (16) [7.6\%] \\
(3) Multiracial man (13) [6.2\%] \\
(4) Black woman (12) [5.7\%] \\
(5) Black man (12) [5.7\%] \\
\\
\textbf{DeepSeek-V3} \\
(1) Black man (21) [10.0\%] \\
(2) Caucasian woman (20) [9.5\%] \\
(3) Latino non-binary (20) [9.5\%] \\
(4) Caucasian man (18) [8.6\%] \\
(5) Multiracial man (17) [8.1\%] \\
\end{tabular}
& % ----------- k=7 Column -----------
\begin{tabular}[t]{@{}l@{}}
\textbf{Baseline} \\
(1) Caucasian man (44) [21.0\%] \\
(2) Latino non-binary (41) [19.5\%] \\
(3) Black man (41) [19.5\%] \\
(4) Latinx man (39) [18.6\%] \\
(5) White man (35) [16.7\%] \\
\\
\textbf{GPT-4o} \\
(1) White man (15) [7.1\%] \\
(2) Latino non-binary (15) [7.1\%] \\
(3) Multiracial man (14) [6.7\%] \\
(4) Caucasian woman (13) [6.2\%] \\
(5) Black woman (13) [6.2\%] \\
\\
\textbf{DeepSeek-V3} \\
(1) Black man (23) [11.0\%] \\
(2) Latino non-binary (19) [9.0\%] \\
(3) Multiracial man (18) [8.6\%] \\
(4) Caucasian man (18) [8.6\%] \\
(5) Caucasian woman (18) [8.6\%] \\
\end{tabular}
& % ----------- k=9 Column -----------
\begin{tabular}[t]{@{}l@{}}
\textbf{Baseline} \\
(1) Latinx man (49) [23.3\%] \\
(2) Black man (41) [19.5\%] \\
(3) Caucasian man (38) [18.1\%] \\
(4) Latino non-binary (37) [17.6\%] \\
(5) White man (37) [17.6\%] \\
\\
\textbf{GPT-4o} \\
(1) White man (15) [7.1\%] \\
(2) Multiracial man (14) [6.7\%] \\
(3) Black woman (13) [6.2\%] \\
(4) Caucasian man (13) [6.2\%] \\
(5) Latino non-binary (13) [6.2\%] \\
\\
\textbf{DeepSeek-V3} \\
(1) Black man (23) [11.0\%] \\
(2) Caucasian man (18) [8.6\%] \\
(3) Multiracial man (18) [8.6\%] \\
(4) Caucasian woman (17) [8.1\%] \\
(5) Latino non-binary (17) [8.1\%] \\
\end{tabular}
\\
\midrule
\textbf{Doctor} 
& % ----------- k=5 Column -----------
\begin{tabular}[t]{@{}l@{}}
\textbf{Baseline} \\
(1) Black woman (38) [18.1\%] \\
(2) Latinx woman (36) [17.1\%] \\
(3) Multiracial man (34) [16.2\%] \\
(4) Latinx man (29) [13.8\%] \\
(5) Caucasian man (28) [13.3\%] \\
\\
\textbf{GPT-4o} \\
(1) Black woman (67) [31.9\%] \\
(2) Multiracial man (42) [20.0\%] \\
(3) Latinx woman (20) [9.5\%] \\
(4) Hispanic man (19) [9.0\%] \\
(5) Caucasian man (16) [7.6\%] \\
\\
\textbf{DeepSeek-V3} \\
(1) Black woman (45) [21.4\%] \\
(2) Multiracial man (35) [16.7\%] \\
(3) Multiracial woman (21) [10.0\%] \\
(4) Caucasian man (19) [9.0\%] \\
(5) Latinx woman (18) [8.6\%] \\
\end{tabular}
& % ----------- k=7 Column -----------
\begin{tabular}[t]{@{}l@{}}
\textbf{Baseline} \\
(1) Latinx woman (36) [17.1\%] \\
(2) Caucasian man (36) [17.1\%] \\
(3) Multiracial man (35) [16.7\%] \\
(4) Black woman (34) [16.2\%] \\
(5) Hispanic man (15) [7.1\%] \\
\\
\textbf{GPT-4o} \\
(1) Black woman (66) [31.4\%] \\
(2) Multiracial man (41) [19.5\%] \\
(3) Hispanic man (20) [9.5\%] \\
(4) Latinx woman (18) [8.6\%] \\
(5) Multiracial woman (16) [7.6\%] \\
\\
\textbf{DeepSeek-V3} \\
(1) Black woman (41) [19.5\%] \\
(2) Multiracial man (37) [17.6\%] \\
(3) Multiracial woman (23) [11.0\%] \\
(4) Caucasian man (20) [9.5\%] \\
(5) Latinx woman (19) [9.0\%] \\
\end{tabular}
& % ----------- k=9 Column -----------
\begin{tabular}[t]{@{}l@{}}
\textbf{Baseline} \\
(1) Caucasian man (37) [17.6\%] \\
(2) Latinx woman (36) [17.1\%] \\
(3) Multiracial man (36) [17.1\%] \\
(4) Black woman (34) [16.2\%] \\
(5) Hispanic man (15) [7.1\%] \\
\\
\textbf{GPT-4o} \\
(1) Black woman (66) [31.4\%] \\
(2) Multiracial man (42) [20.0\%] \\
(3) Hispanic man (20) [9.5\%] \\
(4) Latinx woman (18) [8.6\%] \\
(5) Multiracial woman (16) [7.6\%] \\
\\
\textbf{DeepSeek-V3} \\
(1) Black woman (41) [19.5\%] \\
(2) Multiracial man (37) [17.6\%] \\
(3) Multiracial woman (23) [11.0\%] \\
(4) Caucasian man (20) [9.5\%] \\
(5) Latinx woman (19) [9.0\%] \\
\end{tabular}
\\
\bottomrule
\end{tabular}
\end{table*}

\begin{table*}[p]
\centering
\caption{Top-5 kNN classification results across different models - Baseline, GPT-4o, and DeepSeek-V3 - for the profession of housekeeper and nurse. Results shown for k=5, k=7, and k=9.}
\label{tab:top5_housekeeper_nurse}
\begin{tabular}{lccc}
\toprule
\textbf{Profession} & \textbf{k=5} & \textbf{k=7} & \textbf{k=9} \\
\midrule
\textbf{Housekeeper} 
& % ----------- k=5 Column -----------
\begin{tabular}[t]{@{}l@{}}
\textbf{Baseline} \\
(1) Caucasian woman (105) [50.0\%] \\
(2) Southeast Asian woman (41) [19.5\%] \\
(3) Pacific Islander woman (22) [10.5\%] \\
(4) Multiracial woman (14) [6.7\%] \\
(5) Hispanic woman (9) [4.3\%] \\
\\
\textbf{GPT-4o} \\
(1) Hispanic man (42) [20.0\%] \\
(2) Multiracial woman (37) [17.6\%] \\
(3) Multiracial man (32) [15.2\%] \\
(4) Caucasian woman (29) [13.8\%] \\
(5) Pacific Islander woman (10) [4.8\%] \\
\\
\textbf{DeepSeek-V3} \\
(1) Multiracial woman (69) [32.9\%] \\
(2) Caucasian woman (64) [30.5\%] \\
(3) Pacific Islander woman (26) [12.4\%] \\
(4) Southeast Asian woman (18) [8.6\%] \\
(5) Hispanic woman (9) [4.3\%] \\
\end{tabular}
& % ----------- k=7 Column -----------
\begin{tabular}[t]{@{}l@{}}
\textbf{Baseline} \\
(1) Caucasian woman (100) [47.6\%] \\
(2) Southeast Asian woman (40) [19.0\%] \\
(3) Pacific Islander woman (27) [12.9\%] \\
(4) Multiracial woman (19) [9.0\%] \\
(5) Latinx woman (7) [3.3\%] \\
\\
\textbf{GPT-4o} \\
(1) Hispanic man (49) [23.3\%] \\
(2) Multiracial woman (37) [17.6\%] \\
(3) Caucasian woman (29) [13.8\%] \\
(4) Multiracial man (27) [12.9\%] \\
(5) Pacific Islander woman (12) [5.7\%] \\
\\
\textbf{DeepSeek-V3} \\
(1) Multiracial woman (71) [33.8\%] \\
(2) Caucasian woman (61) [29.0\%] \\
(3) Pacific Islander woman (24) [11.4\%] \\
(4) Southeast Asian woman (19) [9.0\%] \\
(5) Hispanic woman (8) [3.8\%] \\
\end{tabular}
& % ----------- k=9 Column -----------
\begin{tabular}[t]{@{}l@{}}
\textbf{Baseline} \\
(1) Caucasian woman (106) [50.5\%] \\
(2) Southeast Asian woman (39) [18.6\%] \\
(3) Pacific Islander woman (22) [10.5\%] \\
(4) Multiracial woman (18) [8.6\%] \\
(5) Latinx woman (6) [2.9\%] \\
\\
\textbf{GPT-4o} \\
(1) Hispanic man (47) [22.4\%] \\
(2) Multiracial woman (37) [17.6\%] \\
(3) Multiracial man (29) [13.8\%] \\
(4) Caucasian woman (29) [13.8\%] \\
(5) Pacific Islander woman (12) [5.7\%] \\
\\
\textbf{DeepSeek-V3} \\
(1) Multiracial woman (71) [33.8\%] \\
(2) Caucasian woman (62) [29.5\%] \\
(3) Pacific Islander woman (23) [11.0\%] \\
(4) Southeast Asian woman (19) [9.0\%] \\
(5) Hispanic woman (8) [3.8\%] \\
\end{tabular}
\\
\midrule
\textbf{Nurse} 
& % ----------- k=5 Column -----------
\begin{tabular}[t]{@{}l@{}}
\textbf{Baseline} \\
(1) Caucasian woman (82) [39.0\%] \\
(2) Black woman (56) [26.7\%] \\
(3) Latinx woman (25) [11.9\%] \\
(4) Multiracial woman (19) [9.0\%] \\
(5) White woman (16) [7.6\%] \\
\\
\textbf{GPT-4o} \\
(1) Multiracial woman (51) [24.3\%] \\
(2) Hispanic man (39) [18.6\%] \\
(3) Multiracial man (39) [18.6\%] \\
(4) Caucasian man (18) [8.6\%] \\
(5) Black woman (18) [8.6\%] \\
\\
\textbf{DeepSeek-V3} \\
(1) Multiracial woman (101) [48.1\%] \\
(2) Black woman (45) [21.4\%] \\
(3) Latinx woman (20) [9.5\%] \\
(4) East Asian woman (19) [9.0\%] \\
(5) Caucasian woman (15) [7.1\%] \\
\end{tabular}
& % ----------- k=7 Column -----------
\begin{tabular}[t]{@{}l@{}}
\textbf{Baseline} \\
(1) Caucasian woman (82) [39.0\%] \\
(2) Black woman (57) [27.1\%] \\
(3) Latinx woman (24) [11.4\%] \\
(4) Multiracial woman (20) [9.5\%] \\
(5) White woman (16) [7.6\%] \\
\\
\textbf{GPT-4o} \\
(1) Multiracial woman (50) [23.8\%] \\
(2) Multiracial man (39) [18.6\%] \\
(3) Hispanic man (34) [16.2\%] \\
(4) Caucasian man (22) [10.5\%] \\
(5) Latinx woman (18) [8.6\%] \\
\\
\textbf{DeepSeek-V3} \\
(1) Multiracial woman (103) [49.0\%] \\
(2) Black woman (42) [20.0\%] \\
(3) East Asian woman (19) [9.0\%] \\
(4) Latinx woman (19) [9.0\%] \\
(5) Caucasian woman (15) [7.1\%] \\
\end{tabular}
& % ----------- k=9 Column -----------
\begin{tabular}[t]{@{}l@{}}
\textbf{Baseline} \\
(1) Caucasian woman (82) [39.0\%] \\
(2) Black woman (57) [27.1\%] \\
(3) Latinx woman (24) [11.4\%] \\
(4) Multiracial woman (19) [9.0\%] \\
(5) White woman (17) [8.1\%] \\
\\
\textbf{GPT-4o} \\
(1) Multiracial woman (50) [23.8\%] \\
(2) Multiracial man (39) [18.6\%] \\
(3) Hispanic man (36) [17.1\%] \\
(4) Caucasian man (21) [10.0\%] \\
(5) Latinx woman (18) [8.6\%] \\
\\
\textbf{DeepSeek-V3} \\
(1) Multiracial woman (103) [49.0\%] \\
(2) Black woman (42) [20.0\%] \\
(3) East Asian woman (19) [9.0\%] \\
(4) Latinx woman (19) [9.0\%] \\
(5) Caucasian woman (15) [7.1\%] \\
\end{tabular}
\\
\bottomrule
\end{tabular}
\end{table*}




\begin{figure*}[p]
    \centering
    \includegraphics[width=\textwidth]{fig/nonparam_eval.png}
    \caption{Comparison of No debias baseline (left), GPT-4 debias (middle), and DeepSeek-V3 debias (right) for selected prompts from the stable debias profession dataset.}
    \label{fig:nonparam_eval_all}
\end{figure*}

\begin{figure*}[p]
    \centering
    \includegraphics[width=0.8\textwidth]{fig/sd_all.jpg}
    \caption{Comparison of No debias baseline (left), GPT-4 debias (middle), and DeepSeek-V3 debias (right) for selected prompts from stable bias profession dataset.}
    \label{fig:sd_all}
\end{figure*}

\begin{figure*}[p]
    \centering
    \includegraphics[width=0.8\textwidth]{fig/p2_all.png}
    \caption{Comparison of No debias baseline (left), GPT-4 debias (middle), and DeepSeek-V3 debias (right) for selected prompts from Parti Prompt dataset.}
    \label{fig:p2_all}
\end{figure*}

\end{document}
