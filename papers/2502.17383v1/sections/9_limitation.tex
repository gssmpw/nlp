\section*{Limitations}

One of the main limitations of our work is that we limit the number of generated questions per section to one due to budget limits. 
Due to the stochastic nature of language models, the type of questions that get generated for each section may vary significantly. 
However, we believe this effect to be minimal and the variance to be sufficiently captured in our results as the improvements of \ours over other baselines were statistically significant $p<0.05$ for all subjects in \ourdata. 

In addition, due to the same constraints as above, we have not tested \ours using rejection sampling where we generate more than one question per section. 
However, the main value of our work is still valid in that we have conceptually demonstrated that rewards from LLM-based simulations can lead to meaningful improvements over baselines even in the minimal setup of generating a single question per section. 

\section*{Ethical Considerations}

Although our work has shown that LLM-based simulators can provide effective reward signal for training better question generators, we do not advocate that our measure of question utility to be used beyond this simulation, such as directly assessing the quality of questions that students ask in a classroom setting. 
The utility of a question as defined by \ours is heavily dependent on what the exam questions are, and if the exam questions are misaligned with desirable educational outcomes, e.g. exam questions that only require rote memorization rather than critical thinking, a student's question may be considered low utility despite being a useful one for potentially other scenarios, such as brainstorming.
While we make sure that this is not the case in our experiments given the wide variety of questions across subjects included in \ourdata as shown in \secref{ssec:textbook-exam-bloom}, we cannot guarantee similar diversity and comprehensiveness in other textbook datasets. 


% reliability of roleplaying

% reliability of EV and its ability to correctly assess whether a given answer to a question without an answer from the textbook is correct or wrong


% Results may vary for \ours and the baselines if we increase the number of questions per section

