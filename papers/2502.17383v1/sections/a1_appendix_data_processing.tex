\section{Data Processing Details}
\label{appendix:data_processing}

\subsection{Parsing Sections}
\label{appdx:parsing-sections}

Our prompt for parsing sections as part of creating \ourdata is shown below. 
In order to ensure consistency across subjects in how sections are divided, we provide manually annotated few-shot examples. 
In addition, we use GPT-4o for this part as it is a one-time expense and data quality has important implications for all the experiments that we conduct in this work.  

\begin{tcolorbox}[title=Extracting Sections, myboxstyle, breakable]
\begin{verbatim}
Instructions for extracting sections from the given textbook content: 

1. Transform markdown for equations into LaTeX and remove all other markdown 
   formatting to only keep the raw content.  
2. Split the content into sections of uniform length and number each section.  
3. Skip the learning objectives, key concepts, and summary content.  
4. Ensure that all content, except skipped parts, is covered verbatim in at 
   least one of the resulting sections.  

# EXAMPLE  
## INPUT:  
{{ example_input }}  

## OUTPUT:  
{{ example_output }}  

Produce only valid JSON with the following format:
{
    "section": {
        "1": {
            "content": "Verbatim section 1 content from chapter"
        },
        ...
    }
}

# TARGET TEXTBOOK CONTENT: 
{{ target_content }}
\end{verbatim}
\end{tcolorbox}
\label{appdx:section-prompt}


\subsection{Parsing Exam Questions}
\label{appdx:parsing-exam-questions}

We parse the end-of-chapter review questions in OpenStax textbooks based on the markdown headers and formatting with BeautifulSoup.\footnote{\url{https://www.crummy.com/software/BeautifulSoup/}}
There are occasionally ill-formatted questions that we throw out, since the main goal here is to find a subset of chapters that are suitable for testing \ours. 
We avoid chapters with too few ($<10$) and too many questions ($>25$) so that our simulations are conducted with enough questions for measuring question utility while also completing within a reasonable timeframe.  

\subsection{Bloom's Taxonomy Level Distribution}
\label{appdx:bloom-taxonomy-level-distribution}
% To analyze model performance on how it affects learning outcomes at different cognitive levels according to the revised Bloom's Taxonomy, we classify exam questions with our Bloom classification prompt~(\secref{appdx:bloom-classification-prompt}). 
% The distribution in \ourdata's train and test sets are shown in \autoref{fig:bloom-distribution}.
We share our prompt for categorizing questions based on the revised Bloom's taxonomy below: 
\begin{tcolorbox}[title=Bloom Classification, myboxstyle, breakable]
\texttt{
\noindent Classify the questions into one of the six main categories of Bloom's 
Taxonomy based on the cognitive processes required for answering it correctly.  
\\
\\
\textbf{Bloom's Taxonomy Categories:}  
\begin{enumerate}
    \item \textbf{Remembering:} Producing or retrieving definitions, facts, or 
          lists, or reciting previously learned information.  
    \item \textbf{Understanding:} Grasping the meaning of information by 
          interpreting and translating what has been learned.  
    \item \textbf{Applying:} Using learned information in new and concrete 
          situations.  
    \item \textbf{Analyzing:} Breaking down or distinguishing the parts of 
          learned information.  
    \item \textbf{Evaluating:} Making judgments about information, validity of 
          ideas, or quality of work based on a set of criteria.  
    \item \textbf{Creating:} Using information to generate new ideas or 
          products.  
\end{enumerate}
}
\begin{verbatim}
Question 1: {{question 1}} 
Question 2: {{question 2}}  
...  
\end{verbatim}

\texttt{Provide only the Bloom category and format your response in JSON with the 
following structure:}

\begin{verbatim}
{
    "bloom_categories": [
        {
            "question": question, 
            "bloom_category": bloom_category
        }
    ]
}
\end{verbatim}

\end{tcolorbox}


