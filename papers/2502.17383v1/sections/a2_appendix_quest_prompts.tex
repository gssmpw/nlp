\section{\textsc{QUEST} Details}
\label{appendix:quest-details}
\textsc{QUEST} consists of four key modules. 
(1) The \textbf{Question Generator} (\(M_q\)) generates a set of questions \( Q = \{q_1, q_2, \dots, q_n\} \) from a document \( D \) (\secref{ssec:quest-question-generation});
(2) The \textbf{Answer Generator} (\(M_a\)) then produces corresponding answers, forming question-answer pairs \( \text{QA} = \{(q_1, a_1), (q_2, a_2), \dots, (q_n, a_n)\} \) using parametric knowledge (\secref{ssec:quest-question-generation}).
(3) The \textbf{Reader Simulator} consists of a \textbf{Learner} (\(M_l\)) and an \textbf{Evaluator} (\(M_e\)). The learner model \(M_l\) simulates a learner’s understanding by attempting the final exam using only the generated QA pairs, formulated as \( P \sim M_l(E; \text{QA}) \). 
The evaluator model \(M_e\) then assesses the learner’s responses \( P \) by comparing them against ground-truth answers when available or using parametric knowledge to assign a score.

\subsection{Prompts Details}
\label{appdx:prompt-details}
 
\subsubsection{Question Generator}
\label{appdx:question-generator}

\begin{tcolorbox}[
title=Question Generator, myboxstyle, breakable
]
article: {\(\mathit{S_{[1:k-1]}}\)}  \\
Student is currently reading the section: {\(\mathit{S_k}\)}.  \\

Generate a question that helps the student  \\
understand the section better.  \\

Output in the following JSON format: \\  
\{
    "question": question
\}
\end{tcolorbox}


\subsubsection{Answer Generator}
\begin{tcolorbox}[title=Answer Generator, myboxstyle, breakable]
\begin{verbatim}
questions: {{questions}}

Answer each question shortly and output
in following JSON format:
```json
{
    "qa_pairs": [
        {"question": question_1, "answer": answer_1},
        {"question": question_2, "answer": answer_2},
        ...
        {"question": question_n, "answer": answer_n},
    ]
}
```
\end{verbatim}
\end{tcolorbox}

\subsubsection{Learner}
\begin{tcolorbox}[title=Learner, myboxstyle, breakable]
\begin{verbatim}
You are now a learner participating in a structured learning simulation. 
Your task is to:
       
1. **Study the Provided Learning Materials:** Carefully read and understand 
   the content enclosed in the [LEARNING MATERIALS] tags.

2. **Answer the Exam Questions Using Only the Learning Materials:** 
   When you respond to the questions in the [EXAM], you must:
   - Base all answers solely on the information contained in the 
     [LEARNING MATERIALS].  
   - Clearly show how your reasoning follows from the [LEARNING MATERIALS].  
   - If the question asks about something not covered in the 
     [LEARNING MATERIALS], do not provide an answer or guess. Instead, 
     respond exactly with:  
     
     I don’t know. I have not been studied on this.

   - Do not use information from outside the [LEARNING MATERIALS].

3. **No External Knowledge or Guessing:**  
   Provide no additional reasoning or information if the content is not in 
   the [LEARNING MATERIALS].  

Let’s begin.

[LEARNING MATERIALS]
{{learning_material}}
[/LEARNING MATERIALS]

Now, proceed to the exam below and answer as instructed:

[EXAM]
{{exam}}
[/EXAM]

Response answers in the following JSON format (key: Exam question number, 
value: your answer):
{
    "1": "< your answer to exam question 1 >",
    "2": "< your answer to exam question 2 >",
    ...
}

# TARGET TEXTBOOK CONTENT: 
{{ target_content }}
\end{verbatim}
\end{tcolorbox}




\subsubsection{Evaluator}
\begin{tcolorbox}[title=Evaluator, myboxstyle, breakable]
\begin{verbatim}
You are a teacher who is evaluating a student's understanding of a document.

Here is the document: 
{{document}}

Now, determine the correctness of the student's answers to the following 
question.

question: 
{{question}}

ground truth: 
{{answer}}

student's answer: 
{{prediction}}

Please provide a score between 0 and 1, where:
- 0 indicates the student's answer is completely incorrect.
- 1 indicates the student's answer is completely correct.

If ground truth is not provided (e.g., None), determine the correctness of 
the student's answer based on your own understanding of the document.

Answer in the following JSON format:

{
    "score": <score>,
    "feedback": "<feedback>"
}
\end{verbatim}
\end{tcolorbox}

