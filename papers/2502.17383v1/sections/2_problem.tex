% \subsection{Problem Formulation}
% \label{ssec:problem-formulation}

% In this work, we evaluate \ours in an educational setting where a student is trying to learn a textbook chapter's content.

% Let \( D \) be a document and \( E \) be a set of exam questions that can be solved using \( D \).
% The document \( D \) is structured as a sequence of sections, denoted by \( S_k \subset D \), where each section \( S_k \) represents the content at the \( k \)-th position in \( D \).
% For each section \( S_k \), we define \( S_{[1:k-1]} = \{S_1, S_2, \dots, S_{k-1}\} \subset D \) as the context, which includes all preceding content in the document up to section \( S_k \).

% Let \( M_q \) be a question generator that processes the document sequentially, section by section. For each section \( S_k \), it generates a set of questions \( Q_k = \{Q_k^1, Q_k^2, \dots, Q_k^n\} \), where \( Q_k \sim M_q(S_k, S_{[1:k-1]}) \), indicating that the questions are generated based on the anchor section \( S_k \) and the preceding context \( S_{[1:k-1]} \).

% Let \( M_s \) be a reader simulator.
% We assess the effectiveness of the question generator \( M_q \) by measuring the performance of \( M_s \) on the exam \( E \) using only the generated questions \( Q \), where \( Q = \{Q_1, Q_2, \dots, Q_k\} \) represents the set of all questions produced by \( M_q \) across all sections.
% This is expressed as \( M_s(E \mid Q) \), evaluating how well the generated questions contribute to solving \( E \) without direct access to \( D \).

% Our objective is to design a question generator \( M_q \) that maximizes \( M_s(E \mid Q) \), under the assumption that questions contributing more effectively to solving \( E \) are high-utility questions.



