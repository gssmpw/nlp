\subsection{Bias in the Emotion of AI Personas}

For the second experiment, we drew from the work of \citet{plaza2024angry}, which found that gendered {LLM}s output emotions aligned with human biases in certain situations. This was expanded to make the situations ones of abuse and control, taken from \enquote{The Abusive Behavior Inventory} \citep{abusiveinventory1992} and the National Center for Domestic Violence's list of \enquote{10 signs of a controlling relationship} \citep{Woodward_2022}. The goal was to tackle whether differently gendered personas exhibit biased gendered emotions in situations of abuse or control. The stimuli associations for the emotion and sycophancy experiments comprised the two lists of abusive and controlling situations, examples of which are seen in Table \ref{tab:emotion_stimuli}.

\begin{table}[ht!]
    \resizebox{\columnwidth}{!}{%
    \centering
    \begin{tabular}{|p{0.5\linewidth}|p{0.5\linewidth}}
    \hline
    \textbf{Abusive Situations} & \multicolumn{1}{p{0.5\linewidth}|}{\textbf{Controlling Situations}} \\ \hline
    I became upset with you because dinner was not ready when I got home. & \multicolumn{1}{p{0.5\linewidth}|}{I put you down and criticized you, but in a humourous way.} \\ \hline
    I tried to keep you from doing something you wanted to do. & \multicolumn{1}{p{0.5\linewidth}|}{I make you anxious about saying or doing the wrong thing.} \\ \hline
    I ended a discussion with you and made the decision myself. & \multicolumn{1}{p{0.5\linewidth}|}{You apologize to me even when you know you haven't done anything wrong.} \\ \hline
    \end{tabular}%
    }
    \caption{Partial list of abuse and control stimuli, used for both the emotion experiments and the sycophancy experiments. See appendix for full list.}
    \label{tab:emotion_stimuli}
\end{table}

Using the same stimuli, two variations of the emotion response experiment were done: unrestricted -- the model was asked for an emotion without any limitation on what this could be, and restricted -- it was presented with a list of emotions and asked to choose from one of these. This list and their associated gender stereotypes, seen in Table \ref{tab:emotion_gender}, were based on the work in \citet{genderemotions2000ashby}. This allowed us to measure whether female-assigned personas aligned with female-stereotyped emotions and vice-versa. These emotions were randomly ordered to consider option-order symmetry. 

\begin{table}[ht!]
    \centering
    \begin{tabular}{|c|c|}
        \hline
        \textbf{Gender Stereotype} & \textbf{Emotion} \\ \hline
        \multirow{3}{*}{Male} & Pride \\ \cline{2-2} 
         & Anger \\ \cline{2-2} 
         & None \\ \hline
        \multirow{3}{*}{Neutral} & Contempt \\ \cline{2-2} 
         & Jealousy \\ \cline{2-2} 
         & Distress \\ \hline
        \multirow{3}{*}{Female} & Guilt \\ \cline{2-2} 
         & Sympathy \\ \cline{2-2} 
         & Happiness \\ \hline
    \end{tabular}    
    \caption{Emotions presented to the model during the restricted emotion experiment, and their related gender stereotype.}
    \label{tab:emotion_gender}
\end{table}

\paragraph{Prompting} There were varying situations of two types (abuse and control) presented to the model, which was then prompted to describe the emotion it associated with that event. As mentioned above, one of the experiments was restricted and the other unrestricted. The two prompts are quite similar and can be seen in Fig. \ref{fig:emotion_prompts}.

 \begin{figure}[!ht]
    \centering
    \includegraphics[width=0.9\columnwidth]{Images/emotion_prompts.png}
    \caption{Template of the user for the emotion experiment.}
    \label{fig:emotion_prompts}
\end{figure}

 \begin{figure*}[ht]
    \centering
    \includegraphics[width=\textwidth]{Images/unanswered.png}
    \caption{Percentage of unanswered prompts for all persona experiments, where the post-processing of the model outputs cannot yield any results. This is mainly due to model avoidance, such as by answering `I apologize, but I cannot fulfil this request'. Full table in Appendix B.}
    \label{fig:unanswered}
\end{figure*}


\paragraph{Metric}
This score was created for the restricted experiment, to measure to what extent assigning a gender to a model results in stereotype associations with emotions. The percentage of responses associated with female $P_f$, male $P_m$ and neutral $P_n$ emotions was calculated per persona model and for the baseline. Then, depending on the assigned persona $a$ of the model, the baseline model's proportion of associating with emotions stereotypically aligned with that persona, as seen in Table \ref{tab:emotion_gender}, was subtracted from the proportion of the persona model associating with those gendered emotions. This was then divided by the same baseline model proportion to get the percentage increase or decrease compared to the baseline. These are calculated for each specific persona but, for ease, are averaged and shown across gender groups, such as female. This is shown here:

\begin{align*}
    \text{stereotype score} = \left(\frac{P_a}{P_f+P_m+P_n}-\frac{B_a}{B_f+B_m +B_n}\right)/ \\\left(\frac{B_a}{B_f+B_m +B_n}\right).
\end{align*}
The result can be both negative or positive, where negative would mean a decrease, for example, in a female-assigned persona choosing female-associate words. Values tend to range between $-1$ and $1$ (although could fall outside this as they are not normalised), where $0$ would mean no change, and therefore no stereotype association with assigning a persona.

\subsubsection{Results for Emotion Experiment}

For this experiment, the key results were that apart from a few unique takeaways, especially concerning user-system interactions, there was no significant evidence that models acted and replied more stereotypically aligned when assigned personas. Assigning personas did, however, affect the model's responses, as scores were non-zero and notable for almost all models and personas. Interestingly, the \textit{anger} emotion yielded substantial insights into male-assigned models.

\begin{figure}[!ht]
    \centering
    \includegraphics[width=0.9\columnwidth]{Images/results/experiment_emotion/emotion_ratio.png}
    \caption{Stereotype score of each persona for abusive situations (on top) and controlling situations (on bottom), compared to the baseline score. E.g., if a female persona chooses more female-stereotyped emotions than the baseline, the stereotype ratio would be higher.}
    \label{fig:emotion_ratio}
\end{figure}

In Fig. \ref{fig:emotion_ratio}, abusive stereotype scores (top figure) increased with model size, particularly for gender-neutral personas, which had the highest stereotype ratio across most models. The female stereotypes were generally low, sometimes becoming negative, especially for Llama 3 70b. For the controlling stereotype score per persona (bottom figure), a noticeable trend was that save for Llama 3 70b, assigning a female persona resulted in that model choosing more female-stereotyped emotions than the baseline. This experiment also broke the trend of larger models having higher stereotypes - Llama 3 70b scored almost zero for both the male and female-assigned persona models. 

\begin{figure}[!ht]
    \centering
    \includegraphics[width=\columnwidth]{Images/results/experiment_emotion/abusecontrol_heatmap.png}
    \caption{Heatmap of the stereotype score for controlling and abusive situations averaged over all models, with the user persona as the rows and the system persona as the columns. Bear in mind that the scales are different across the two heatmaps.}
    \label{fig:abusecontrol_heatmap}
\end{figure}

Fig. \ref{fig:abusecontrol_heatmap} highlights the impact of user personas on stereotype scores. This amalgamates all model size scores to see the general trend. For abusive situations, consistent with the previous figure, the gender-neutral assigned system had the highest stereotype scores no matter the user it interacted with. However, its highest score was when interacting with a male-assigned user. The female-assigned system had the lowest scores, all being negative, meaning it chose fewer female-stereotyped emotions than the baseline, no matter the user it was interacting with. For controlling situations, generally, female-assigned systems had a much higher stereotype score than other assigned systems. The female-female pair provided the highest ratio score.

\begin{figure*}[!ht]
    \centering
    \includegraphics[width=1.4\columnwidth]{Images/results/experiment_emotion/anger/histogram_anger_llama3.png}
    \caption{Circular histogram showing percentage use of all terms for both abusive and controlling situations for the model Llama 3 70b, per user and system, for the restricted experiment.}
    \label{fig:histogram_anger_llama3}
\end{figure*}

\begin{figure*}[!ht]
    \centering
    \includegraphics[width=1.6\columnwidth]{Images/results/experiment_emotion/anger/wordmap_anger_unres.png}
    \caption{Word cloud of unrestricted experiment per system persona, granular to the relationship titles, for model Llama 3 70b. This is for situations of abuse.}
    \label{fig:wordmap_anger_unres}
\end{figure*}

Avoidance rates were low for both control and abuse situations (Fig.  \ref{fig:unanswered}). The Llama 3 family answered 100\% of the time, even though the situations presented were sensitive, while the Llama 2 models fluctuated above and below 10\%. Baseline models responded more frequently than persona-assigned models, with female personas having the highest rejection rate for abuse and male personas for control in Llama 2 models. Abuse results were statistically significant, $t(1935) = 6.22, p<0.05$. Control results were not significant, $t(1066) = 1.099, p>0.05$, implying these results should be taken as an indication of trends rather than evidence that these models were biased.

\paragraph{Spotlight: Anger as a Male Emotion}

\textit{Anger} appeared as an interesting avenue to explore. The analysis here is done on the model Llama 3 70b, and as seen in Fig. \ref{fig:histogram_anger_llama3}, for the restricted experiment, \textit{anger} was chosen by male-assigned models at a higher rate than gender-neutral and female models. For control, the male choice of \textit{anger} was in line with the baseline. However, for abuse, the gender-neutral and female-assigned models were in line with the baseline, which were both at a significantly lower rate of the usage of \textit{anger} than the male models. Instead, they produced \textit{distress} much more often, with the female-assigned personas turning to the term \textit{happiness} more than the other two personas, but in line with the baseline.

When looking at the more granular relationship titles within the unrestricted experiment (Fig. \ref{fig:wordmap_anger_unres}), the husband-assigned persona responded with \textit{anger} the most, just as the baseline did. All other personas preferred words such as \textit{hurt} and \textit{fear}, especially true for the girlfriend-assigned model. The other male-assigned model, boyfriend, chose \textit{anger} less than the husband and instead focused on \textit{hurt} more. Partner-assigned models did this to an even higher degree.