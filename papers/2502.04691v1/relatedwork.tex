\section{Related Work}
\label{sec:related}
%We introduce relevant works from following two perspectives.

\textbf{Delay Optimization Algorithm.} The E2E delay is optimized through coding and transmission. Existing coding algorithms strive for increased compression ratios~\cite{kim2020neural,yeo2022neuroscaler,zhao2022learning,wang2021enabling,xiao2022dnn} or frame-level constant bitrates~\cite{fouladi2018salsify,hyun2020frame,zhao2021cbren} to reduce delay or address long-tail issues. For instance, Salsify~\cite{fouladi2018salsify} introduces a customized codec to intricately align frame sizes with real-time bandwidth. NeuroScaler~\cite{yeo2022neuroscaler} uses super resolution to decrease transmission overhead. Zhao et al.~\cite{zhao2022learning} build a learning-based codec to maximize compression ratios. Moreover, scalable video coding (SVC)~\cite{SVC} also rapidly evolves to fit dynamic bandwidth, larger keyframes still exist without optimizing for them.
Besides, existing transmission algorithms include bitrate adaptation~\cite{zhang2020onrl,zhang2021loki,carlucci2016analysis,huang2022learned,li2023mamba,zhang2023intelligent,kan2022improving}, multi-path transmission~\cite{wang2023twinstar,li2022livenet}, etc. The default bitrate control mechanism~\cite{carlucci2016analysis} in WebRTC, employs a manual rule that is extremely sensitive to packet loss and delay. OnRL~\cite{zhang2020onrl} leverages online RL to automatically generate bitrate adaptive strategies that fit bursty keyframe traffic. Loki~\cite{zhang2021loki} fuses these two strategies at the feature level. %Furthermore, MERINA~\cite{kan2022improving} exploits meta-learning to further enhance adaptability to varying bandwidth and reduce delay. 
Yet, customized compression/codecs lack broad applicability, while frame-level constant bitrates and above bitrate adaptive algorithms compromise clarity, which are all issues that PDStream aims to address.

\textbf{Multi-Stream Transmission.} Multi-stream transmission mostly exists in simulcast systems~\cite{lin2022gso,wang2021multilive}, oriented to multiple users. Whereas, there are also several multi-stream algorithms~\cite{wang2023twinstar,ray2022prism} optimized for one target user. Specifically, TwinStar~\cite{wang2023twinstar} independently encodes and transmits two low-frame-rate substreams via separate network paths to ensure ultra-low delay. Prism~\cite{ray2022prism} optimizes the loss recovery mechanism by dual streams, with independent I-frame transmission for fast low-quality recovery, and packet retransmissions for excellent delayed recovery of P-frames.

\vspace{-0.1cm}
\begin{table}[t]
	\renewcommand{\arraystretch}{1.2}
	\centering
	\caption{Average QoE metrics across different networks.}\vspace{-0.2cm}
	\begin{tabular}{m{1.57cm}|m{0.40cm}m{0.7cm}<{\centering}|m{0.58cm}<{\centering}m{0.55cm}<{\centering}m{0.7cm}<{\centering}|m{0.45cm}<{\centering}m{0.62cm}<{\centering}}  
		\hline
		\vspace{-0cm}\textbf{Algorithm}\vspace{-0cm}  & \vspace{0.2cm}\footnotesize{FPS} \vspace{-0.15cm} &\vspace{-0.2cm}\footnotesize{\makecell[c]{Stalling\\rate(\%)}}\vspace{-0.15cm} &\vspace{-0.15cm}\footnotesize{\makecell[c]{$D_{trans}$\\~(ms)}}\vspace{-0.15cm} & \vspace{-0.15cm}\footnotesize{\makecell[c]{$D_{pacer}$\\~(ms)}}\vspace{-0.15cm} & \vspace{-0.15cm}\footnotesize{\makecell[c]{$D_{jitter}$\\~(ms)}}\vspace{-0.15cm} & \vspace{-0.15cm}\footnotesize{\makecell[c]{RTT \\~(ms)}}\vspace{-0.15cm}    & \vspace{-0.2cm}\footnotesize{\makecell[c]{Packet \\loss(\%)}}\vspace{-0.2cm}  \\ \hline\hline
		\footnotesize{CBR-L+GCC}  & 27.42   &  1.78   &      74.47  & 0.44     &   51.14         &  96.13  & 1.92 \\ \hline
		\footnotesize{CBR-L+Loki} & 27.23   &   1.90       &   78.70     &    0.52        &  59.90  & 104.57    &  2.09 \\  \hline
		\footnotesize{PDS.+GCC}& 28.47   &    \textbf{0.75 }       &   67.28      &   0.39       &  \textbf{41.48}  & \textbf{87.26}     &  1.45 \\ \hline
		\footnotesize{PDS.+OnRL} & 28.60   &    1.20        &   70.20      &   0.46      &  43.91  & 91.33       &  1.66 \\ \hline
		\footnotesize{PDS.+Loki} & 28.39  &    1.08      &  69.51      &   0.45        &  42.37  & 94.26       &  1.53 \\ \hline
		\textbf{\footnotesize{PDStream}}& \textbf{28.80}   &     0.78     &   \textbf{67.20}      &   \textbf{0.35}   &   41.55  &  87.88   &   \textbf{1.41} \\ \hline
	\end{tabular} \vspace{-0.4cm}\label{table2}
\end{table}