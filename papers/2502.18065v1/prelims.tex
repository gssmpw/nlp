\section{Preliminaries}\label{sec:prelims}

We now state formally the main definitions of this paper, and prove some basic equivalences. 

       \paragraph{Notation, graphs.}
       For two distinct elements $a,b$ of a set $V$,
by $ab$ we denote the unordered pair $\set{a,b}$, and call $ab$ a \emph{pair} for simplicity
-- the distinction between ordered pairs and unordered pairs will follow from the context.
For $A,B\subset V$, denote $$AB\coloneqq \setof{ab}{a\in A,b\in B,a\neq b}$$ and ${A\choose 2}\coloneqq AA$.
A \emph{graph} is a pair $G=(V,E)$ with $E\subset {V\choose 2}$.
We call the elements of $V$ \emph{vertices}, and elements of $E$ \emph{edges}, and denote $V(G)\coloneqq V$ and $E(G)\coloneqq E$. 

\subsection{Merge-width and merge sequences}\label{sec:merge-sequence}

In the gray box of \Cref{sec:intro} we gave a definition of merge-width in terms of \emph{construction sequences}.
We now give an alternative definition in terms of so-called \emph{merge sequences}, and prove in \Cref{lem:fmw}
that these definitions are indeed equivalent.
These two definitions complement each other: 
While construction sequences are useful for showing that a graph class with bounded merge-width has a certain property,
merge sequences are useful for the opposing goal of showing that a given graph class has bounded merge-width.





\medskip



    Fix $r\in\N\cup\set{\infty}$.
        For a set $R\subset {V\choose 2}$ and partition $\cal P$ of $V$,
        we say a part \(A \in \cal P\) is \emph{\(r\)-reachable} from a vertex \(v\) if there is a path of length at most $r$ in the graph $G=(V,R)$ from $v$ to some vertex in~$A$.
        The \emph{radius-$r$ width} of $(\cal P,R)$  is 
        \[
           \max_{v \in V}\ \Big|\bigl\{ A \in \cal P \mid  A \text{ is \(r\)-reachable from } v \bigr\}\Big|.
        \]
        
       


\begin{definition}\label{def:ms}
A \emph{merge sequence} of a graph $G=(V,E)$ 
is a sequence
$$(\cal P_1,R_1),\ldots,(\cal P_m,R_m),$$
  such that:
  \begin{enumerate}
    \item $\cal P_1,\ldots,\cal P_m$ is a sequence of partitions of $V$, 
    where $\cal P_1$ is the partition into singletons and $\cal P_m$ has one part,
    and $\cal P_{t}$ is coarser than (or equal to) $\cal P_{t-1}$ for $t=2,\ldots,m$,
    \item $R_1\subset\ldots\subset R_m\subset {V\choose 2}$; the pairs in $R_t$ are said to be \emph{resolved} at time $t$, and
    \item for all $t\in [m]$ and $A,B\in \cal P_t$,
     either $AB-R_t\subset E$, or $AB-R_t\subset {V\choose 2} - E$.
  \end{enumerate}
For $r\in\N\cup\set{\infty}$, the \emph{radius-$r$ width} of the merge sequence is the maximum, over $t=2,\ldots,m$,
of the radius-$r$ width of $(\cal P_{t-1},R_t)$.
\end{definition}

Note the offset in the indices of $\cal P_{t-1}$ and $R_t$.
Conceptually, this offset allows us to not only do a single merge, but an unbounded number of merges when going from partition \(\cal P_{t-1}\) to \(\cal P_t\) (see backwards direction of \Cref{lem:fmw}).

\begin{definition}
  The \emph{radius-$r$ merge-width} of a graph $G$, denoted $\mw_r(G)$, is the least radius-$r$ width of a merge sequence of $G$.
Finally, a graph class $\CC$ has \emph{bounded merge-width}
  if ${\mw_r(\CC)<\infty}$ holds\footnote{For a graph parameter $f$ and graph class $\CC$, by $f(\CC)$ we denote $\sup_{G\in\CC}f(G)$.} for every (finite) $r\in\N$.
\end{definition}

\begin{lemma}\label{lem:fmw}
    For every construction sequence of a graph $G$ there is a merge sequence of $G$
  of the same radius-$r$ width, for all $r\in\N\cup\set\infty$,
  and vice versa.
  \end{lemma}
  \begin{proof}
  
  First, fix a construction sequence of $G$, consisting of $m$ steps. 
    Let $\cal P_t$ be the partition of $V$,
    and let $R_t$ be the set of edges and non-edges 
    at the beginning of step $t$ of the sequence.
    It is clear that 
    $(\cal P_1,R_1),\ldots,(\cal P_m,R_m)$ is a merge sequence. 
    In particular, the third item of \Cref{def:ms} follows from the second item of \Cref{remark:maindef}.
    
    Fix $r\in\N\cup\set{\infty}$, and let $k$ 
    be the radius-$r$ merge-width of the considered construction sequence.
    Then, for each $t\in[1,m]$, 
    the radius-$r$ width of $(\cal P_t,R_t)$ is at most $k$.
    Since for $t>1$, we either have $\cal P_{t-1}=\cal P_t$ or $R_{t-1}=R_t$, it follows that 
    the radius-$r$ width of $(\cal P_{t-1},R_t)$ is at most $k$ as well.
    This concludes one direction.
  
  
  \medskip
  
  In the other direction,  
  let $(\cal P_1,R_1),\ldots,(\cal P_m,R_m)$ be a merge sequence of $G$.
  We define a construction sequence which consists of $m$ stages, as follows.
  In the beginning of stage $t\in[m-1]$,
  the current partition $\cal P$ is the partition $\cal P_t$, and the set $R$ of resolved pairs is contained in $R_t$.
  In stage $t$, first resolve (in any order) all pairs 
  $A,B\in\cal P_{t}$ such that $AB\subset R_{t+1}$.
  The third item of \Cref{def:ms} guarantees that this is always possible by positive or negative resolves.

  Next, repeatedly merge (in any order) any two parts $A,B$ of the current partition, as long as $A\cup B$ is contained in some part of $\cal P_{t+1}$.
  Once this is done, the current partition is equal to $\cal P_{t+1}$, and we proceed to the next stage.
  
  During all steps in stage $t$, the set $R$ of resolved pairs is contained in $R_{t+1}$, and $\cal P$ is a coarsening of $\cal P_t$. In particular, for every $r\in\N\cup\set{\infty}$,
  \begin{multline*}
\text{radius-$r$ width of $(\cal P,R)$}\ \le \ 
\text{radius-$r$ width of $(\cal P_t,R_{t+1})$} \\
\le \ \text{radius-$r$ width of the given merge sequence}.
  \end{multline*}
 
  After the last stage, the current partition $\cal P$ has only the single part $V$. 
  Finally, we resolve $V$ with itself. This completes the  construction sequence.
  \end{proof}



  





  
\subsection{Construction sequences have linear length}\label{sec:representing}

When designing merge-width based algorithms,
we always assume graphs to be represented by construction sequences, rather than merge sequences,
as the former impose more structure.
A~priori such a construction sequence can be arbitrarily long, as for example, one can keep on resolving the same pair indefinitely.
In this subsection we make an innocuous assumption that guarantees a linear bound on their length.


Namely, call a resolve operation in a construction sequence 
\emph{effective} if it creates a non-zero number of new edges or non-edges. 
That is, an \emph{ineffective} resolve operation 
is a resolve operation performed between a pair $A,B$ of parts 
such that all pairs in $AB$ were already resolved before the operation.
Call a construction sequence \emph{effective} if 
every resolve operation in it is effective.
We assume without loss of generality and without explicitly mentioning it that all construction sequences are effective.
Note that the following lemma also applies if the radius-\(r\) width is bounded, as this implies a bound on the radius-\(1\) width.

\begin{lemma}\label{lem:effective-construction-seq}
  Every effective construction sequence for a graph $G$ of radius-$1$ width $w$ has length at most $(2w+1)|V(G)|$.
\end{lemma}
\begin{proof}
  We rearrange the construction sequence by repeatedly postponing resolve operations.
  Namely, consider a subsequence of consecutive operations in a construction sequence,
  consisting of a sequence of resolve operations, and finishing with the merge
  of two parts $A,B\in\cal P$. We may postpone all resolves which occur in the subsequence, and involve 
  neither $A$ or $B$, until after the merge of $A$ and $B$ is done.
  This result in a valid construction sequence,
  of the same radius-$1$ width and the same length.
  As all resolve operations are effective,
  at most $2w$ resolve operations between pairs $P,Q$ remain in the subsequence: at most $w$ pairs involving $A$, and at most $w$ pairs involving $B$, because each vertex in $A$ (or in $B$) $1$-reaches all parts resolved with $A$ (resp. $B$). 
  In the final construction sequence, at most $2w$ resolve operations are performed per merge,
  followed potentially by a single last resolve on the final part.
  As there can be at most \(|V(G)|-1\) merges, we have at most $(2w+1)|V(G)|$ steps altogether.
  Since the original construction sequence has the same length as the rearanged construction sequence, the result follows.
\end{proof}



  





  
