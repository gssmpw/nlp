\section{A locality theorem}\label{sec:logic}

Following the approach of \cite{gks},
we will define a logic dist-FO that extends first-order logic
with certain distance predicates.
With respect to this logic (for a fixed quantifier rank \(q\) and auxiliary number \(k\)),
for a given structure $A$ 
we will define three kinds of \emph{types}.

\begin{itemize}
    \item The \emph{type} \(\tp_{k,q}(A,a)\) of a vertex $a$ in $A$,
        containing a set of formulas $\phi(x)$ that hold at \(a\).
    \item The \emph{local type} \(\ltp_{k,q}(A,a)\) of a vertex $a$ in $A$,
        is the same as above, but ignores color predicates that are far away from \(a\).
        It therefore has only a bounded number of possible values in a construction sequence of small width,
        making it suitable to use in our algorithms.
    \item The \emph{scatter type} \(\stp_{k,q}(A)\) of \(A\)
        contains information about inclusionwise maximal scattered
        sets with certain types.
        We use the scatter type to link the local and global types of a vertex.
\end{itemize}

Our algorithm maintains the \emph{local} type of each vertex at each step of the construction sequence.
However, as part of each update step, we crucially also need to compute certain \emph{global} types.
The following locality theorem forms the heart of our algorithm by connecting these two types via the scatter type.
The subsequent lemma demonstrates how we will use it algorithmically.

\begin{restatable}[Locality theorem]{theorem}{localglobalgame}\label{thm:localglobal}
    Fix $k,q,\ell\in\N$ with $\ell\le k$.
    Let \(A\) and \(B\) be binary structures  with \(\stp_{k,q}(A)=\stp_{k,q}(B)\).
    Then, for all  tuples $\tup a\in V(A)^\ell,\tup b\in V(B)^\ell$,
    \[
       \tp_{k,q}(A,\tup a) = \tp_{k,q}(B,\tup b) \quad\iff\quad
       \ltp_{k,q}(A,\tup a) = \ltp_{k,q}(B,\tup b).
    \]
\end{restatable}

\begin{restatable}{lemma}{localglobaltypes}\label{lem:localglobaltypes}Fix $k,q\in\N$ with $k\ge 1$ and a finite binary signature $\sigma$.
    There is an algorithm which takes as input
    a $\sigma$-structure \(A\)
    and \(\ltp_{k,q}(A,a)\) for all \(a \in V(A)\),
    and computes
    \(\tp_{k,q}(A,a)\) for all \(a \in V(A)\), jointly
    in time \(O_{k,q,\sigma}(|V(A)|+|E(A)|)\).
\end{restatable}

In the next two subsections, we give the full definitions of dist-FO,
type, local type, and scatter type.
We moreover derive \Cref{lem:localglobaltypes} from \Cref{thm:localglobal}.
The following subsections then prove \Cref{thm:localglobal}.
Finally, in \Cref{sec:updating-local} we derive further algorithmic consequences of
\Cref{lem:localglobaltypes}. Those are the ones which will be used 
in further sections. 

\subsection{Logical preliminaries}
The set $\N$ includes $0$. For $n\in\N$, denote $[n]\coloneqq\set{1,\ldots,n}$.

We consider binary relational signatures only.
More precisely, a (binary, relational) \emph{signature} $\sigma$ consists of a set $\sigma_2$ of binary relation symbols, and a set $\sigma_1$ of unary relation symbols. A structure $A$ over the signature $\sigma$,
 or a \emph{$\sigma$-structure}, consists of a domain $V(A)$ and the interpretation 
  $R^A\subset V(A)^k$ of each relation symbol $R\in \sigma_k$, for $k\in\set{1,2}$.
More generally, for a formula $\phi(x_1,\ldots,x_k)$, by $\phi^A\subset V(A)^k$ we denote the 
set of tuples $(a_1,\ldots,a_k)\in V(A)^k$ such that $\phi(a_1,\ldots,a_k)$ holds in $A$, also denoted $A\models\phi(a_1,\ldots,a_k)$.
A \emph{binary structure} is a structure over a binary relational signature.
The \emph{Gaifman graph} of a $\sigma$-structure $A$ is the graph $(V,E)$ with vertices $V\coloneqq V(A)$ and edges 
$$E \coloneqq E(A)\coloneqq \setof{uv\in \mbox{$V\choose 2$}}{R\in\sigma_2,A\models R(u,v)\lor R(v,u)}.$$


We use \(\bar a\) to denote tuples \(a_1,\dots,a_k\) of elements, and  \(|\bar a|\) to denote their length.
For brevity, we often implicitly convert such tuples to their underlying vertex sets,
and write expressions like \(S \subseteq \bar a\) as a shorthand for \(S \subseteq \{a_1,\dots,a_k\}\).

Let $\sigma$ be a relational signature and let $\sigma'\subset \sigma$.
    The \emph{$\sigma'$-reduct} of a $\sigma$-structure $\str A$ is the $\sigma'$-structure 
    obtained from $\str A$ by removing the relations in $\sigma'-\sigma$.
    We will only consider reducts which remove some unary relations, and keep all binary relations. For a $\sigma$-structure $\str A$ and $\sigma'\subset\sigma_1$, we denote by $\str A[\sigma']$ the $(\sigma'\cup\sigma_2)$-reduct of $\str A$.

    

\subsection{Definitions and computational implications}\label{sec:typedefs}


\paragraph{First-order distance logic.}
We define a logic called dist-FO that extends FO by the following \emph{distance atoms}:
\begin{itemize}
    \item For each radius \(r \in \N\), dist-FO introduces 
        a binary distance atom \(\dist(x,y) \le r\) expressing that the distance between \(x\) and \(y\) in the Gaifman graph is at most \(r\).
    \item For each radius \(r \in \N\) and unary relation symbol \(Y\), dist-FO introduces
        a unary distance atom \(\dist(x,Y) < r\) equivalent to the formula
        \(\exists y~Y(y) \land \dist(x,y) < r\).
\end{itemize}
We call \(r\) the \emph{radius} of the distance atom.
Note that distance atoms can be expressed 
using usual first-order formulas (assuming $\sigma_2$ is finite),
and that for radius \(1\), these first-order formulas are quantifier-free.
We will need to have a fine control over the quantifier rank of the constructed formulas, and for this reason, the possibility to measure distances using quantifier-free formulas is essential, similarly as in \cite{gks}.


We define a sufficiently fast-growing function $\rho\from \N^2\to\N$, so that for $k,q\in\N$
    \[
        \rho(k,q) = 
        \begin{cases}
            1 & \text{if }q=0, \\
            (13^{k}+1) \cdot \rho(k+1,q-1) & \text{if }q > 0.
        \end{cases}
    \]
This function originates from \Cref{lem:scatter}, which forms the combinatorial core of the proof of \Cref{thm:localglobal}.


Let \(k,q \in \N\).
A dist-FO formula has \emph{distance rank} \((k,q)\) if it is a boolean combination of
\begin{itemize}
    \item atomic formulas, 
    \item distance atoms (of arity one and two) for radii up to \(\rho(k,q)\), and
    \item dist-FO formulas \(\exists x \phi\) and \(\forall x \phi\), where \(\phi\) has distance rank \((k+1,q-1)\) and \(q > 0\).
\end{itemize}
Thus, in particular, a formula with distance rank \((k,q)\) has quantifier rank at most \(q\).
Throughout our proofs, the number \(k\) provides an upper bound on the number of free variables of a formula, and
thus grows with every quantifier.
As \(\rho(k+1,q-1) < \rho(k,q)\) for all \(k,q \in \N\),
we observe that the distance atoms nevertheless have progressively shorter radii as one traverses the quantifiers inwards.


\paragraph{Atomic types.}
The \emph{$(k,q)$-atomic type} of a tuple \(\bar a\) in a binary structure \(A\),
denoted by \(\atp_{k,q}(A,\bar a)\), is the set of all quantifier-free dist-FO formulas \(\phi(\bar x)\)
that use distance atoms only for radii up to \(\rho(k,q)\) and satisfy \(A \models \phi(\bar a)\).


\paragraph{Types.}
For a binary structure \(A\) and tuple of vertices \(\bar a\),
define \(\tp_{k,q}(A,\bar a)\) as the set of all dist-FO formulas \(\phi(\bar x)\) of distance rank \((k,q)\)
with \(A \models \phi(\bar a)\).
We call \(\tp_{k,q}(A,\bar a)\) the \emph{(k,q)-type of \(\bar a\) in \(A\)}.
It is easy to see, by normalizing formulas, that the number of formulas $\phi(x_1,\ldots,x_\ell)$ of distance-rank $(k,q)$, up to equivalence, is bounded in terms of $k,q,\ell$, and $\sigma$.
This yields the following.

\begin{observation}\label{obs:typebound}
    Fix $q,k,\ell\in\N$, and a finite binary signature $\sigma$. The set
    $$\setof{\tp_{k,q}(A,\bar a)}{\str A\text{ is a $\sigma$-structure}, \bar a\in V(A)^\ell}$$ is finite, of size bounded in terms of $k,q,\ell$ and $\sigma$.
\end{observation}

\begin{remark}
Even though each type \(\tp_{k,q}(A,\bar a)\) is, formally, an infinite set of formulas, it has a finite representation, by the observation above. 
We may thus 
consider every operation mapping types to types (for fixed parameters $k,q,\ell,\sigma$) as computable in time $O_{q,k,\ell,\sigma}(1)$.
\end{remark}





\paragraph{Local types.}\label{par:local-type-def}
Let \(A\) be a binary structure with unary relations symbols \(\sigma_1\).
For a tuple $\tup a$ of elements of $A$, denote
$$\reach^A_r(\bar a) \coloneqq  \{ Y \in \sigma_1 \mid a\in \bar a, A \models \dist(a,Y) \le r \}.$$
For a set \(\lambda \subseteq \sigma_1\), recall that \(A[\lambda]\) is the structure obtained from removing
all unary predicates outside \(\lambda\).
For each $k,q\in\N$, binary structure $A$, and tuple \(\tup a\in V(A)\) with $|\tup a|\le k$, define the \emph{local $(k,q)$-type of $a$ in $A$} as 
$$\ltp_{k,q}(A,\tup a) \coloneqq  \tp_{k,q}\bigl(A\bigl[\reach^A_{\rho(k,q)}(\tup a)\bigr],\tup a\bigr).$$
As ${\rho(k,q)\ge \rho(k+1,q-1)}$, we have the following.
\begin{observation}\label{obs:ltphered}
    For all $k,q\in\N$, binary structures \(A,B\), and tuples \(\tup a, \tup b\),
    $$\ltp_{k,q}(A, \tup a)= \ltp_{k,q}(B, \tup b)\qquad\text{implies}\qquad
    \ltp_{k+1,q-1}(A, \tup a)= \ltp_{k+1,q-1}(B, \tup b).$$
\end{observation}


\paragraph{Scatter types.}
A vertex set \(S\) is an \emph{\(r\)-scattered set} in a graph if all vertices in \(S\)
have pairwise distance larger than \(r\). (Thus, a \(1\)-scattered set is an independent set.)

For each binary structure~\(A\) and $k,q\in\N$, pick any function 
\[\stp_{k,q}(A)\from \Big([\rho(k,q)]\times\left\{\text{formulas $\phi(x)$ of distance rank \((k+1,q-1)\)}\right\}\Big)\to\set{0,\ldots,k+q},\]
called the \emph{(\(k,q\))-scatter type} of $\str A$, 
which maps every pair $(r,\phi(x))$ in its domain
to the size of any set $S\subset \phi^\str A$ which is \(r\)-scattered in $\str A$ and has size at most $k+q$, and is an inclusionwise maximal set with those two properties.
(For $q=0$,  $\stp_{k,q}(A)$ is the unique function $\stp_{k,q}(A)\from [\rho(k,q)]\times\emptyset\to\set{0,\ldots,k}$, and the condition holds trivially.)

As we demand \(S\) to be any \emph{inclusionwise maximal} set with the above properties, rather than a set of maximal size,
the function \(\stp_{k,q}(A)\) is not uniquely determined.
However, by assuming a total order on \(V(A)\),
we can make the definition unique by setting \(\stp_{k,q}(A)\) to be the output of a simple
algorithm that greedily constructs scattered sets for each \(r\) and \(\phi\)
until they either reach a size of \(k+q\) or cannot be extended anymore.
As each formula of distance rank \((k,q)\) and quantifier rank at most $q-1$ is also a formula of distance rank \((k+1,q-1)\),
and moreover \(k+q = (k+1)+(q-1)\),
we can craft the above algorithm to ensure the following.
\begin{observation}\label{obs:stphered}
    For all $k,q\in\N$ with $q>0$ and binary structures \(A,B\),
    $$\stp_{k,q}(A)= \stp_{k,q}(B)\qquad\text{implies}\qquad
    \stp_{k+1,q-1}(A)= \stp_{k+1,q-1}(B).$$
\end{observation}
We can use \Cref{obs:typebound} to bound the number of different scatter types, and to bound the run time of the aforementioned greedy algorithm,
yielding the following two observations.
\begin{observation}\label{obs:scattertypebound}Fix $k,q\in\N$, and a finite binary signature $\sigma$.
    The set
    $$\setof{\stp_{k,q}(A)}{\str A\text{ is a $\sigma$-structure}}$$ is finite, of size bounded in terms of $k,q$ and $\sigma$.
\end{observation}

In the following, for each vertex added by the greedy algorithm, 
the factor \(|V(A)|+|E(A)|\) in the run time accounts for the search that is necessary to exclude the vertices that are close to~it.

\begin{observation}\label{obs:stpcompute}
    Fix $k,q\in\N$ and a binary signature $\sigma$.
    Given a binary $\sigma$-structure \(A\)
    and, if $q>0$, also \(\tp_{k+1,q-1}(A,v)\) for each vertex \(v \in V(A)\),
    one can compute \(\stp_{k,q}(A)\) in time \(O_{k,q,\sigma} (|V(A)|+|E(A)|)\).
\end{observation}

We will use the following lemma to justify our use of inclusionwise maximal scattered sets.

\begin{lemma}\label{lem:scattersizes}
    Let \(G\) be a graph and \(X \subseteq V(G)\).
    If \(X\) contains an inclusionwise maximal \(r\)-scattered set of size \(k\),
    then \(X\) contains no \(2r\)-scattered set of size larger than \(k\).
\end{lemma}
\begin{proof}
    Fix an inclusionwise maximal \(r\)-scattered set \(S \subseteq X\).
    Then \(X \subseteq N_{r}(S)\), as otherwise \(S\) could be extended.
    Thus, given two vertices \(u,v\) of a \(2r\)-scattered set \(S' \subseteq X\),
    they both have distance at most \(r\) from some vertex in \(S\).
    However, as \(\dist(u,v)>2r\),
    the two vertices cannot have distance at most \(r\) to the same vertex.
    The pigeonhole principle implies \(|S'| \le |S|\)
\end{proof}


\paragraph{Computational implications.}
We have now defined all the notions appearing in the statement of \Cref{thm:localglobal}.
Before giving its proof,
we can already derive some computational implications from it.

\localglobaltypes*
\begin{proof}
    We prove the statement by induction on \(q\).
    If $q>0$ then by \Cref{obs:ltphered} and induction, we can compute \(\tp_{k+1,q-1}(A,a)\) for all \(a\).
    By \Cref{obs:stpcompute}, (also when $q= 0$) we can efficiently compute \(\stp_{k,q}(A)\).

    By \Cref{thm:localglobal}, \(\tp_{k,q}(A,a)\) is uniquely determined by
    \(\ltp_{k,q}(A,a)\) and \(\stp_{k,q}(A)\).
    By \Cref{obs:typebound} and \ref{obs:scattertypebound},
    both \(\ltp_{k,q}(A,a)\) and \(\stp_{k,q}(A)\)
    have only a bounded number of possible values,
    and thus we can hard-code a look-up table that yields \(\tp_{k,q}(A,a)\)
    for each combination in bounded time.
    Note that our algorithm non-uniformly depends on \(k,q\) and \(\sigma\), which is allowed by the statement of the lemma.
\end{proof}


\subsection{Local and global games}\label{sec:localglobalgame}
Towards a proof of \Cref{thm:localglobal}, in \Cref{sec:localglobalgame} we define three variants of the classical Ehrenfeucht-Fraïssé (EF) game on binary structures,
called the \emph{global game}, the \emph{local game}, and the \emph{strongly local game},
which will turn out to be equivalent.
We refer to the literature for extensive background on EF games, for example, to the textbook~\cite{libkin04}.
Recall the definition of atomic types \(\atp_{k,q}(A,\bar a)\) provided in \Cref{sec:typedefs}


\paragraph{Global game.}
Let \(A,B\) be binary structures and \(\bar a, \bar b\) be tuples of vertices from \(A\) and \(B\), respectively, with \(|\bar a|=|\bar b|\).
The \emph{global \((k,q)\)-game} on \((A,B,\bar a,\bar b)\) is a game played between \emph{Spoiler} and \emph{Duplicator}
defined by the following three steps.
\begin{enumerate}[label=(\theenumi)]
    \item\label{g:step1}
    If \(\atp_{k,q}(A,\bar a) \neq \atp_{k,q}(B,\bar b)\) then Spoiler wins directly.
    Otherwise, if \(q=0\) then Duplicator wins directly.
    If none of these criteria hold then proceed to the next step.
    \item\label{g:step2} Spoiler picks a vertex \(a \in V(A)\) or \(b \in V(B)\),
and Duplicator picks a vertex \(a\) or \(b\) from the structure that Spoiler did not pick.
    \item\label{g:step3} Afterwards, continue with the \((k+1,q-1)\)-game on \((A,B,\bar a a,\bar b b)\).
\end{enumerate}



For plain first-order logic, it is a well-known fact that
 two structures have the same types of quantifier-rank $q$ if and only if Duplicator wins the \(q\)-round EF game.
 More precisely, by this we mean that Duplicator has a winning strategy in the game.
See for example \cite[Theorem 3.9]{libkin04} for a proof.
The following lemma extends this result to the dist-FO logic.
The proof is analogous to the classical proof for first-order logic, and thus omitted.

\begin{lemma}\label{lem:globaltype}
    Fix $k,\ell,q\in\N$ with $\ell\le k$.
Let \(A,B\) be binary structures and \(\bar a\in V(A)^\ell, \bar b\in V(B)^\ell\).
Then \(\tp_{k,q}(A,\bar a) = \tp_{k,q}(B,\bar b)\) if and only if Duplicator wins the global \((k,q)\)-game on \((A,B,\bar a,\bar b)\).
\end{lemma}

\paragraph{Local game.}
We define a \emph{local \((k,q)\)-game} by replacing steps (2) and (3) of the global \((k,q)\)-game as follows.
\begin{itemize}
    \item [(2)]\label{l:step2}
        By symmetry, assume Spoiler picks \(a \in V(A)\) (we enforce symmetric constraints if Spoiler picks \(b \in V(B)\)).
        This move is only allowed if there is a previous vertex \(a_i \in \bar a\) 
        with
        \begin{align}\label{eq:loc}%
            \reach^A_{\rho(k+1,q-1)}(a) \subseteq \reach^A_{\rho(k,q)}(a_i).    
        \end{align}
        Now Duplicator responds by picking some $b\in V(B)$.

        

        \item[(3)]\label{l:step3} Afterwards, continue with the local \((k+1,q-1)\)-game on \((A,B,\bar a a,\bar b b)\).
\end{itemize}
Note that in step (2)
for every response \(b\in V(B)\) of Duplicator which does not immediately loose in the upcoming step (3),
        the following analogue of \eqref{eq:loc} holds as well:
        \[
            \reach^B_{\rho(k+1,q-1)}(b) \subseteq \reach^B_{\rho(k,q)}(b_i)
        \]
        This is because
        the atomic type constraints in step (1) of the game ensure that in this round, \(\reach^A_{\rho(k,q)}(a_i) = \reach^B_{\rho(k,q)}(b_i)\).
        The same constraints ensure that at the beginning of the next round, \(\reach^A_{\rho(k+1,q-1)}(a) = \reach^B_{\rho(k+1,q-1)}(b)\).
        Thus, in the local game, both players are forced to play close a vertex that is in a sense close to a previously chosen vertex.
        This implies the following lemma.

        \medskip
Recall that \(A[\lambda]\) is the structure obtained from removing all unary predicates outside a set \(\lambda\),
and that \(\ltp_{k,q}(A,a) \coloneqq  \tp_{k,q}\bigl(A\bigl[\reach^A_{\rho(k,q)}(a)\bigr],a\bigr)\).


\begin{lemma}\label{lem:localtype}Fix $k,q,\ell\in\N$ with $\ell\le k$.
Let \(A,B\) be binary structures and \(a \in V(A)^\ell, b \in V(B)^\ell\).\\
If \(\ltp_{k,q}(A,\tup a) = \ltp_{k,q}(B,\tup b)\) then Duplicator wins the local \((k,q)\)-game on \((A,B,\tup a,\tup b)\).
\end{lemma}
\begin{proof}
    As $\ltp_{k,q}(A,\tup a) = \ltp_{k,q}(B,\tup b)$, in particular, we have $$\reach^A_{\rho(k,q)}(\tup a)= \reach^B_{\rho(k,q)}(\tup b).$$
    Denote  $\lambda\coloneqq \reach^A_{\rho(k,q)}(\tup a)$.
    By \Cref{lem:globaltype}, Duplicator wins the global \((k,q)\)-game on \((A[\lambda],B[\lambda],\\\tup a,\tup b)\).
    As the local game places the additional constraint of step (2) only on Spoiler's moves,
    Duplicator also wins the local \((k,q)\)-game on \((A[\lambda],B[\lambda],\tup a, \tup b)\).
    We argue that  Duplicator's winning strategy in this game is also a winning strategy in the local $(k,q)$-game on $(A,B,\tup a,\tup b)$:

    Recall that \(\bar a = a_1,\dots,a_\ell\) and \(\bar b = b_1,\dots,b_\ell\).
    For $i=1,2,\ldots,q$,
    let \(a_{\ell+i},b_{\ell+i}\) be the vertices played in the \(i\)th round of this local game.
    We show by induction on \(i=1,\ldots,q\) that
    \[
        \reach^A_{\rho(k+i,q-i)}(a_{\ell+i}) \subseteq \lambda
    \qquad\text{and}\qquad
        \reach^B_{\rho(k+i,q-i)}(b_{\ell+i}) \subseteq \lambda.
    \]
    The game rules require the existence of some \(i' < i\) with
    \[
        \reach^A_{\rho(k+i,q-i)}(a_{\ell+i}) \subseteq \reach^A_{\rho(k+(i-1),q+(i-1))}(a_{\ell+i'}).
    \]
    By the induction assumption (or definition of $\lambda$, if $i' \le 0$), the right side is contained in $\lambda$.
    The same argument holds for~\(b_{\ell+i}\),  proving the claim.
    
    The above implies that Duplicator, by using their winning strategy in the local $(k,q)$-game on \((A[\lambda],B[\lambda],\bar a, \bar b)\), also wins the local $(k,q)$-game on $(A,B,\bar a,\bar b)$, as the 
    the atomic type constraints of step (1) of the game cannot access predicates outside \(\lambda\).
\end{proof}


\paragraph{Strongly local game.}
As an intermediate tool in the proof of \Cref{thm:localglobal},
we also define the \emph{strongly local \((k,q)\)-game} by replacing steps (2) and (3) of the global \((k,q)\)-game
as follows.
\begin{itemize}
    \item[(2.a)]\label{sl:2a}
    Create a graph \(D\) with vertex set \(\{1,\dots,|\bar a|\}\) where two vertices \(i,j\) are connected
        if and only if \(\dist^A(a_i,a_j) \le \rho(k,q)\).
        By the constraints of step (1),
        this is equivalent to \(\dist^B(b_i,b_j) \le \rho(k,q)\).

    Spoiler chooses a connected component $C$ of $D$.
    Let \(\bar a',\bar b'\) be the subtuples obtained from \(\bar a,\bar b\) by restricting to the indices in $C$. 
    
    \item[(2.b)]\label{sl:2b}
     Spoiler picks some  $a\in V(A)$ or $b\in V(B)$.
    Suppose, by symmetry, that Spoiler picks $a\in V(A)$.
     This move is only allowed if there is some $a_i\in \bar a'$
    with
    \begin{align}\label{eq:sloc}
        \dist^A(a,a_i) \le \rho(k,q)-\rho(k+1,q-1).    %
    \end{align}
     Duplicator responds by picking some $b\in V(B)$.
    
    \item[(3)]\label{sl:3}
    Afterwards, continue with the strongly local \((k+1,q-1)\)-game on \((A,B,\bar a' a,\bar b' b)\).
\end{itemize}

The following two lemmas establish the relationship between this game and the local variant considered previously.
\begin{lemma}\label{lem:local-slocal}
    Fix $k,\ell,q\in\N$ with $\ell\le k$.
    Let \(A\) and \(B\) be two binary structures
    and let \(\bar a\in V(A)^\ell,\bar b\in V(B)^\ell\).    
    If  Duplicator wins the local \((k,q)\)-game on \((A,B, \tup a, \tup b)\), then 
 they also win the strongly local \((k,q)\)-game on \((A,B, \tup a, \tup b)\).
\end{lemma}
\begin{proof}
    This is because the condition \eqref{eq:sloc} in step (2.b) of the strongly local game
    implies, by the triangle inequality, the condition \eqref{eq:loc} of step (2) in the local game.
\end{proof}


The key argument in the proof of \Cref{thm:localglobal} is encompassed in the following.
\begin{restatable}{lemma}{lemmain}\label{lem:main}Fix $k,q,\ell\in\N$ with $\ell\le k$.
    Let \(A\) and \(B\) be two binary structures with \(\stp_{k,q}(A)=\stp_{k,q}(B)\),
    and let \(\bar a\in V(A)^\ell,\bar b\in V(B)^\ell\).    
    If Duplicator wins the strongly local \((k,q)\)-game on  \((A,B,\bar a,\bar b)\), then they also win the global \((k,q)\)-game on  \((A,B,\bar a,\bar b)\).
\end{restatable}

\Cref{lem:main} is proved in \Cref{sec:lemmain-proof}.



\subsection{Proof of \Cref{thm:localglobal}}\label{sec:gameproof}
Before proving the central \Cref{lem:main}, we show how it implies \Cref{thm:localglobal}, which is repeated below.

\localglobalgame*
\begin{proof}
    The rightwards implication is immediate, as the local type is just the type in a reduct omitting some unary predicates.

    For  the leftwards implication,
   suppose 
 \(\ltp_{k,q}(A,\tup a) = \ltp_{k,q}(B,\tup b)\).
 By 
            \Cref{lem:localtype} Duplicator wins the local \((k,q)\)-game on \((A,B, \tup a, \tup b)\).
        By \Cref{lem:local-slocal},  they also win the strongly local \((k,q)\)-game on \((A,B, \tup a, \tup b)\).
        Finally, by \Cref{lem:main} they also win the global \((k,q)\)-game,
            and by \Cref{lem:globaltype}, \(\tp_{k,q}(A,\tup a) = \tp_{k,q}(B,\tup b)\).
\end{proof}
We also prove a lemma which will be used later in \Cref{sec:closure}.

\begin{lemma}\label{lem:far-ltp}
    Fix $k,q\in\N$. Let $A$ be a structure and  $\tup a_1,\tup a_2$ be tuples in $A$
    with $|\tup a_1\tup a_2|\le k$ and 
    \begin{align*}
        \dist^A(\tup a_1,\tup a_2)&>\rho(k,q).
    \end{align*}
Then $\ltp_{k,q}(A,\tup a_1\tup a_2)$ depends only on $\stp(A)$, $\ltp_{k,q}(A,\tup a_1)$, and
$\ltp_{k,q}(A,\tup a_2)$.
\end{lemma}
More precisely, the conclusion means that there is a function $F$ such that 
 $\ltp_{k,q}(A,\tup a_1,\tup a_2)=F(\stp_{k,q}(A),\ltp_{k,q}(A,\tup a_1),\ltp_{k,q}(A,\tup a_2))$,
 for all $A,\tup a_1,\tup a_2$ satisfying the assumption of the lemma.

\begin{proof}
    We show that 
    if $A,B$ are structures and $\tup a_1,\tup a_2,\tup b_1,\tup b_2$ are tuples with 
    \begin{align}
        \dist^A(\tup a_1,\tup a_2)&>\rho(k,q),\label{dist1}\\
        \dist^B(\tup b_1,\tup b_2)&>\rho(k,q),\label{dist2}\\
        \ltp_{k,q}(A,\tup a_1)&=\ltp_{k,q}(B,\tup b_1)\label{tupa}\\
        \ltp_{k,q}(A,\tup a_2)&=\ltp_{k,q}(B,\tup b_2)\label{tupb},\\
        \stp_{k,q}(A)&=\stp_{k,q}(B)\label{stp},
    \end{align}
    then $\ltp_{k,q}(A,\tup a_1\tup a_2)=\ltp_{k,q}(B,\tup b_1\tup b_2)$.

    By \eqref{tupa}, \Cref{lem:localtype}, and \Cref{lem:local-slocal}, 
    Duplicator wins the strongly local $(k,q)$-game on $(A,B,\tup a_1,\tup b_1)$.
    Similarly, using \eqref{tupb}, Duplicator wins the strongly local $(k,q)$-game on $(A,B,\tup a_2,\tup b_2)$.
    By \eqref{dist1} and \eqref{dist2},
    and the mechanics of the strongly local game (specifically, the choice of a connected component of a graph with threshold $\rho(k,q)$ in step (2.a)),
    it follows that Duplicator wins the strongly local $(k,q)$-game on $(A,B,\tup a_1\tup a_2,\tup b_1\tup b_2)$.
    Finally, by \eqref{stp} and \Cref{lem:main}, they also win the global \((k,q)\)-game  on $(A,B,\tup a_1\tup a_2,\tup b_1\tup b_2)$,
            and therefore, the local $(k,q)$-game 
            which, by \Cref{lem:localtype}, yields  $\ltp_{k,q}(A,\tup a_1\tup a_2)=\ltp_{k,q}(B,\tup b_1\tup b_2)$.
\end{proof}




\subsection{An auxiliary lemma about distances}
The proof of \Cref{lem:main} 
relies on a combinatorial argument which is encapsulated in \Cref{lem:scatter} below.
We start with the following observation, also made by \cite[Lem. 3]{learningLogic},
which is similar in spirit to the Vitali covering lemma.

\begin{lemma}[{\cite{learningLogic}}]\label{lem:vitali}
    Given $r,\beta\in\N$, a graph, and a set $K$ of its vertices, there are $R\in\N$ and a set \(K' \subseteq K\),
     such~that
    \begin{enumerate}
        \item $r\le R\le r\cdot (\beta+1)^{|K|-1}$, 
        \item the vertices in \(K'\) have pairwise distance larger than \(\beta R\), and 
        \item every vertex in \(K\) has distance at most \(R\) from \(K'\).
    \end{enumerate}
\end{lemma}
\begin{proof}
For  \(t=0,1,2,\ldots\), we construct a subset \( K_t \subseteq K \) of size \(|K|-t\)
such that every vertex in \(K\) has distance at most \(R(t)\coloneqq  r (\beta + 1)^{t} \) from \(K_t\).
Set \(K_0 \coloneqq  K\).
For $t\ge 0$, if every pair of vertices in \(K_t\) has distance larger than \(\beta R(t)\), we are done.
Otherwise, pick vertices \(u,v \in K_t\) at distance at most \(\beta R(t)\).
Set \(K_{t+1} \coloneqq  K_t - \{u\}\),
and observe that every vertex that has distance \(R(t)\) from \(u\) has distance \(R(t) + \beta R(t) = R(t+1)\) from \(v\).
This process terminates at the latest when \(t=|K|-1\). 
\end{proof}

\begin{lemma}\label{lem:scatter}
    Let \(h,\ell \in \N\).
    Let \(A,B\) be graphs and \(\bar a, \bar b\) be \(\ell\)-tuples of vertices from \(A,B\), respectively.
    Let \(X_A \subseteq V(A),X_B \subseteq V(B)\) be such that the following hold:
    \begin{enumerate}
        \item  \(\dist^A(a_i,a_j) =r \Leftrightarrow \dist^B(b_i,b_j)= r,\) for all \(i,j \in [\ell]\) and \(0 \le r \le 13^\ell h\),

        \item  \(\dist^B(b_i,X_B) \le h \Rightarrow \dist^A(a_i,X_A) \le h\), for all \(i\in[\ell]\),
        \item  $X_A$ and $X_B$ contain 
         some two inclusionwise maximal \(r\)-scattered sets either of equal size, or both of sizes greater than $\ell$, for all \(1 \le r \le 13^\ell h\).
    \end{enumerate}
    If every vertex in \(X_B\) has distance at most \(h\) from \(\bar b\),
    then every vertex in \(X_A\) has distance at most \(13^\ell h\) from \(\bar a\). In symbols,
    $$X_B\subset N_h^B(\tup b)\qquad\text{implies}\qquad X_A\subset N^A_{13^\ell h}(\tup a).$$
\end{lemma}
\begin{proof}
    Apply \Cref{lem:vitali} to \(\bar a\) for $\beta=12$ and $r\coloneqq h$.
        We obtain an index set \(I \subseteq [\ell]\), a mapping \(m : [\ell] \to I\), and a radius \(h \le R \le 13^{\ell-1} h\), such that
        \(\dist^A(a_i,a_j) > 12 R\) for all \(i,j \in I\), and
        \(\dist^A(a_i,a_{m(i)}) \le R\) for all \(i \in [\ell]\).
        By the first assumption of this lemma, also
        \(\dist^B(b_i,b_j) > 12 R\) for all \(i,j \in I\), and
        \(\dist^B(b_i,b_{m(i)}) \le R\) for all \(i \in [\ell]\).
        For each \(i\) in \(I\), define 
        a \emph{cluster in \(A\)} as \(\bigcup_{j \in m^{-1}(i)} N^A_h(a_j) \),
        and a corresponding \emph{cluster in \(B\)} as \(\bigcup_{j \in m^{-1}(i)} N^B_h(b_j) \).
        Note that in both \(A\) and \(B\), every cluster is contained in a ball of radius \(R+h \le 2R\), 
        and clusters have pairwise distance larger than \(12R-2h-2R \ge 8R\).

        The third assumption of the lemma implies that
        \(A\) and \(B\) contain inclusionwise maximal \(4R\)-scattered sets \(S_A \subseteq X_A\)
        and \(S_B \subseteq X_B\), either of the same size \(s\) or both larger than \(\ell\).
        
        Suppose  $X_B\subset N_h^B(\tup b)$. Then every vertex in \(X_B\) (and thus also of \(S_B\))
        has distance at most \(h\) from some vertex in \(\bar b\),
        and thus is contained in some cluster.
        However, as any two vertices from \(S_B\) have distance larger than \(4R\)
        and each cluster is contained in a ball of radius \(2R\),
        each cluster can contain at most one vertex from \(S_B\).
        This implies that \(|S_A|=|S_B|=s \le \ell\).

        The second assumption of the lemma states that, if a cluster in \(B\) contains a vertex from \(X_B\),
        then the corresponding cluster in \(A\) contains a vertex from \(X_A\).
        Hence, as \(S_B\) contains \(s\) vertices of \(X_B\) that are in pairwise different clusters on the \(B\) side,
        there are also \(s\) vertices in \(X_A\), call them \(S'_A \subseteq X_A\), which are in pairwise different clusters on the \(A\) side.
        As clusters have pairwise distance larger than \(8R\), the vertices in \(S'_A\) form an \(8R\)-scattered set.
        Moreover, by \Cref{lem:scattersizes}, \(X_A\) contains no \(8R\)-scattered set of size larger than \(s\),
        and thus \(S'_A\) is a \emph{maximal} \(8R\)-scattered subset of \(X_A\).
        Thus, all vertices in \(X_A\) have distance at most \(8R\) from \(S'_A\), which itself is contained in a union of clusters, 
        and thus has distance at most \(R\) from \(\bar a\).
        Thus, all vertices in \(X_A\) have distance at most \(9R \le 13^\ell h\) from \(\bar a\), equivalently, $X_A\subset N^A_{13^\ell h}(\tup a)$.
\end{proof}

\subsection{Proof of \Cref{lem:main}}\label{sec:lemmain-proof}
We now prove \Cref{lem:main}, repeated below. This will complete the proof of \Cref{thm:localglobal}.
\lemmain*

\begin{proof}
    We prove the statement by induction on \(q\).
    For \(q=0\) this is trivial, so assume it holds for \(q-1\), and we prove it for \(q\).
    As Duplicator  does not lose in step (1) of the strongly local \((k,q)\)-game by assumption, they also do not lose in step (1) of the global \((k,q)\)-game. 
    
In step (2) of the global \((k,q)\)-game,
    by symmetry, assume that Spoiler plays a vertex \(a \in V(A)\).
    We will find a vertex \(b\in V(B)\)
    with the following two properties: 
    \begin{enumerate}[label=({\itshape\roman*})]
        \item\label{cond:dist}  For all  $i \in\set{1,\ldots, \ell}$,
        \begin{equation*}%
            \dist^A(a_i,a) \le \rho(k+1,q-1) \quad\iff\quad \dist^B(b_i,b) \le \rho(k+1,q-1) .
         \end{equation*}
        
         \item\label{cond:win} Let \(a_{\ell+1}\coloneqq a\) and \(b_{\ell+1} \coloneqq  b\).
         Let \(D\) be the graph
            with vertex set \(\{1,\dots,\ell+1\}\) where two vertices \(i,j\) are adjacent
            if and only if \(\dist^A(a_i,a_j) \le \rho(k+1,q-1)\).
            Let $C_0$ be
            the connected component of $D$ containing $\ell+1$, and let 
            $\bar a^0$ and $\bar b^0$ be the subtuples of $\bar a a$ and $\bar bb$ induced by $C_0$.
            Then Duplicator wins the strongly local $(k+1,q-1)$-game on $(A,B,\bar a^0,\bar b^0)$.
    \end{enumerate}
     
    \noindent
We first argue that those two properties 
ensure that Duplicator wins the global $(k+1,q-1)$-game on $(A,B,\bar aa,\bar bb)$, as required. 
The following claim implies that  Duplicator's response $b$ passes the condition in step (1) in that game.
\begin{claim}\label{cl:step1}
    $\atp_{k+1,q-1}(A, \tup aa)=\atp_{k+1,q-1}(B, \tup bb)$
\end{claim}
\begin{claimproof}
Since $\atp_{k,q}(A, \tup a)=\atp_{k,q}(B, \tup b)$,
 it is enough to argue that for all $r\le \rho(k+1,q-1)$ we have
\begin{align}
    \dist^A(a,a_i)\le r&\iff \dist^B(b,b_i)\le r\label{eq:dist-atom1}\qquad\text{for all $i\in\set{1,\ldots,\ell}$}\\
    \dist^A(a,Y)< r&\iff \dist^B(b,Y)< r\label{eq:dist-atom2}\qquad\text{for all $Y\in\sigma_1$}.
\end{align}

By $\atp_{k,q}(A, \tup a)=\atp_{k,q}(B, \tup b)$ and \ref{cond:dist}, for all $i,j\in\set{1,\ldots,\ell+1}$, we have:
$$\dist^A(a_i,a_j)\le \rho(k+1,q-1)\iff \dist^B(b_i,b_j)\le \rho(k+1,q-1).$$
Therefore, in property \ref{cond:win} above, 
defining the connected component $C_0$ of $D$
basing on distances in $A$, or in distances in $B$, yields to an equivalent condition.

We prove the equivalences \eqref{eq:dist-atom1} and \eqref{eq:dist-atom2}. By symmetry (due to the observation above), it is enough to prove the rightwards implications. 
Fix  $i\in\set{1,\ldots,\ell}$ and $r\le \rho(k+1,q-1)$, and suppose that  $\dist^A(a,a_i)\le r$.
We argue that $\dist^A(b,b_i)\le r$.
By \ref{cond:dist}, we have that 
$\dist^A(b,b_i)\le \rho(k+1,q-1)$.
Then $i$ belongs to the connected component $C_0$ of $D$, considered in condition \ref{cond:win}. 
In particular, $a_i\in \bar a^0$ and $b_i\in \bar b^0$.
As Duplicator wins the strongly local $(k+1,q-1)$-game on $(A,B,\bar a^0,\bar b^0)$, 
we have that $\atp_{k+1,q-1}(A,a_i a)=\atp_{k+1,q-1}(B,b_i b)$. Therefore, 
$\dist^A(a,a_i)\le r$ implies $\dist^A(b,b_i)\le r$.
This proves the equivalence \eqref{eq:dist-atom1}.
Similarly, \eqref{eq:dist-atom2} follows as $\atp_{k+1,q-1}(A,a)=\atp_{k+1,q-1}(B,b)$.
\end{claimproof}

\begin{claim}\label{cl:sloco}
    Duplicator wins the strongly local $(k+1,q-1)$-game on $(A,B,\bar aa,\bar bb)$.
\end{claim}
\begin{claimproof}
Duplicator does not loose in step (1) of the  game, by Claim~\ref{cl:step1}.
In Step (2.a), Spoiler picks some connected component $C'$ of the graph $D'$ 
with vertex set $\set{1,\ldots,\ell+1}$ and edges connecting pairs $i,j$ such that $\dist^A(a_i,a_j)\le \rho(k+1,q-1)$.
Let $\bar a'$ and $\bar b'$ be the subtuples of $\bar aa$ and $\bar bb$ induced by $C'$.
We argue that 
Duplicator wins the strongly local $(k+1,q-1)$-game on $(A,B,\bar a',\bar b')$.
This ensures that in the upcoming Step (2.b),
Duplicator has a winning response to each move of Spoiler, proving the claim.



We consider two cases.
Suppose $\ell+1\in C'$. By  property \ref{cond:win}, Duplicator wins the strongly local $(k+1,q-1)$-game on $(A,B,\bar a',\bar b')$. 

Suppose now that $\ell+1\notin C'$.
Let  $D''$ be the graph with vertex set $\set{1,\ldots,\ell}$ and edges connecting pairs $i,j$ such that $\dist^A(a_i,a_j)\le \rho(k,q)$.
Then $D'[\set{1,\ldots,\ell}]$ is a subgraph of $D''$, so there is a unique connected component $C''$ of $D''$ containing $C'$. 
By assumption that Duplicator wins the strongly local $(k,q)$-game on $(A,B,\bar a,\bar b)$, it follows that  Duplicator wins the strongly local $(k+1,q-1)$-game on $(A,B,\bar a[C''],\bar b[C''])$, where $\bar a[C''],\bar b[C'']$ denote the subtuples of $\bar a$ and $\bar b$ induced by $C''$.
As $\bar a'$ and $\bar b'$ are subtuples of $\bar a[C'']$ and $\bar b[C'']$ induced by $C'$,
it follows that Duplicator wins the strongly local $(k+1,q-1)$-game on $(A,B,\bar a',\bar b')$.
\end{claimproof}


By \Cref{cl:sloco}, \Cref{obs:stphered}, and induction, Duplicator  also wins the global \((k+1,q-1)\)-game on \((A,B,\bar a a,\bar b b)\).
    Since $a\in V(A)$ is arbitrary, this proves that Duplicator wins the global $(k,q)$-game on $(A,B,\bar a,\bar b)$, as required.

    \medskip

It therefore remains to prove that for every $a\in V(A)$ there is a vertex $b\in V(B)$ satisfying the properties \ref{cond:dist} and \ref{cond:win}.
We consider two cases, depending on where the vertex $a\in V(A)$ is played.

    \textbf{Near case.} Assume \(\dist^A(a,\bar a) \le \rho(k,q)-\rho(k+1,q-1)\).
    Then \(a\) is close enough, so that Spoiler may play \(a\) in the strongly local \((k,q)\)-game on \((A,B,\bar a,\bar b)\).
    Let $b$ be Duplicator's response to $a$, according to the winning strategy in the strongly local $(k,q)$-game on $(A,B,\bar a,\bar b)$ provided with the assumption of the lemma. Conditions \ref{cond:dist} and \ref{cond:win} follow easily.

    \textbf{Far case.} Assume \(\dist^A(a,\bar a) > \rho(k,q)-\rho(k+1,q-1)\).
    In particular, also $\dist^A(a,\bar a)>\rho(k+1,q-1)$, so the tuple \(\bar a^0\) considered in condition \ref{cond:win} consists solely of $a$.
    We will show in the upcoming \Cref{claim:farsimilarvertex} that Duplicator can choose a response \(b\)
    with \(\dist^B(b,\bar b) > \rho(k+1,q-1)\) such that 
    $\tp_{k+1,q-1}(A,a)=
\tp_{k+1,q-1}(B,b)$. 
Then Duplicator wins the strongly local \((k+1,q-1)\)-game on \((A,B,a,b)\), 
by \Cref{lem:localtype} and \Cref{lem:local-slocal}. Conditions \ref{cond:dist} and \ref{cond:win}
    follow.

\bigskip
    It thus remains to prove the following.


    \begin{claim}\label{claim:farsimilarvertex}
    Let \(a \in V(A)\) with \(\dist^A(a,\bar a) > \rho(k,q)-\rho(k+1,q-1)\).
    There is a vertex \(b\in V(B)\) with \(\dist^B(b,\bar b) > \rho(k+1,q-1)\)
    and $\tp_{k+1,q-1}(A,a)=
\tp_{k+1,q-1}(B,b)$.
    \end{claim}
    \begin{claimproof}
    Define sets \(X_A\) and \(X_B\) as
    \begin{align*}
         X_A &\coloneqq  \{ v \in V(A) \mid \tp_{k+1,q-1}(A,v) = \tp_{k+1,q-1}(A,a)\}, \\
         X_B &\coloneqq  \{ v \in V(B) \mid \tp_{k+1,q-1}(B,v) = \tp_{k+1,q-1}(A,a)\}.
    \end{align*}

\noindent 
We show that there exists a vertex  \(b\in X_B\) with $\dist^B(b,\bar b)>\rho(k+1,q-1)$.


By assumption, we have that 
    \(a \in X_A\) and \[\dist^A(a,\bar a) > \rho(k,q)-\rho(k+1,q-1) = 13^{k} \rho(k+1,q-1).\]
    In order to apply \Cref{lem:scatter} with \(h = \rho(k+1,q-1)\)
    we check each of its three requirements.
    \begin{enumerate}
        \item The atomic type constraints in step (1) of the game ensure that the first requirement is met, as $\rho(k,q)\ge 13^\ell\cdot\rho(k+1,q-1)$.
        \item We argue that for every \(i\in\set{1,\ldots,\ell}\),
            \[
                \dist^B(b_i,X_B) \le \rho(k+1,q-1) \Rightarrow \dist^A(a_i,X_A) \le \rho(k+1,q-1).
            \]
            Suppose there is $b'\in X_B$ with $\dist^B(b_i,b')\le \rho(k+1,q-1)$.
            If Spoiler plays \(b'\),
            then Duplicator's winning response in the strongly local \((k,q)\)-game 
            at \((A,B,\bar a,\bar b)\) gives \(a'\) with \(\dist(a_i,a') \le \rho(k+1,q-1)\).
            As removing vertices obviously only helps Duplicator,
            they in particular win the strongly local \((k+1,q-1)\)-game on \((A,B,a',b')\).
            By induction and \Cref{obs:stphered}, they also win the global \((k+1,q-1)\)-game on \((A,B,a',b')\).
            By \Cref{lem:globaltype}, \(\tp_{k+1,q-1}(A,a') = \tp_{k+1,q-1}(B,b')\), and thus \(a' \in X_A\).
        \item 
            To prove the third requirement, we select a formula \(\phi(x)\) with distance rank \((k+1, q-1)\) such that
            \(X_A = \{ v \in V(A) \mid A \models \phi(v)\}\) and \(X_B = \{ v \in V(B) \mid
            B \models \phi(v)\}\). Specifically, \(\phi(x)\) is defined as the
            conjunction of all (finitely many, up to equivalence) formulas of distance rank \((k+1, q-1)\) occurring in \(\tp_{k+1, q-1}(A, a)\). Given that \(\stp_{k, q}(A) = \stp_{k, q}(B)\) and \(q > 0\),
            the corresponding scatter type entries for \(\phi\) ensure that \(X_A\) and \(X_B\)
            each contain two inclusionwise maximal \(r\)-scattered sets that are either of
            equal size or have sizes greater than \(k+q \ge \ell\).

    \end{enumerate}
    As $a\in X_A$ and  \(\dist^A(a,\tup a)>13^k \rho(k+1,q-1)\),
    by the contrapositive of \Cref{lem:scatter},
    there is some \(b \in X_B\) with \(\dist^B(b,\tup b)>\rho(k+1,q-1)\).
    As $b\in X_B$, we have $\tp_{k+1,q-1}(A,a)=
     \tp_{k+1,q-1}(B,b)$.
    This proves \Cref{claim:farsimilarvertex}, and concludes \Cref{lem:main} and \Cref{thm:localglobal}.
\end{claimproof}
\end{proof}







\subsection{Maintaining local types under Gaifman increasing modifications}\label{sec:updating-local}


Fix a signature $\sigma$, and let 
    $\phi(\bar x)$ be a unary or binary $\sigma$-formula.
    For a $\sigma$-structure~$\str A$, denote by  $\str A\ast \phi$ the structure
     obtained from $\str A$ 
     by expanding it with a fresh relation symbol of arity $|\bar x|$, interpreted as $\phi^{\str A}\coloneqq\setof{\tup a\in V(\str A)^{|\bar x|}}{\str A\models\phi(\tup a)}$.  
     For $k\ge 2$ and unary or binary $\sigma$-formulas $\phi_1,\ldots,\phi_k$,
     inductively define $\str A\ast\phi_1\ast\cdots\ast\phi_k\coloneqq (\str A\ast \phi_1\ast\cdots\ast\phi_{k-1})\ast\phi_k$.

\begin{definition}[Gaifman increasing modification]
     A \emph{Gaifman increasing modification} of a $\sigma$-structure $\str A$ 
      is a (binary) reduct $\str B$ of $\str A\ast \phi_1\ast\phi_2\ldots\ast\phi_k$,
      for some quantifier-free $\sigma$-formulas $\phi_1,\ldots,\phi_k$
      such that the Gaifman graph of $\str A$ is contained in the Gaifman graph of  $\str B$. (If the Gaifman graphs of $\str A$ and $\str B$ are equal, then $\str B$ is a Gaifman \emph{preserving} modification of $\str A$.)
      The \emph{order} of the modification is the total length of the formulas $\phi_1,\ldots,\phi_k$ plus the number of relations that were removed 
      in the reduct $\str B$ of $\str A$.
  \end{definition}
  
We extend the concept of \emph{radius-\(r\) width} from graphs to binary structures.

\begin{definition}\label{def:strucreach}
    For a $\sigma$-structure $\str A$ and $r\in\N$, define 
    the \emph{radius-$r$ width} of $\str A$ as 
    $$\max_{a\in V(\str A)}\big|\reach^\str A_r(a)\big|,$$
    where 
    $$\reach^A_r(a) \coloneqq  \{ Y \in \sigma_1 \mid A \models \dist(a,Y) \le r \}.$$
\end{definition}
    

The main result of this section is the following.

\begin{proposition}\label{thm:modification}
    Fix $\ell,k,q,w\in\N$ with $k\ge 1$, and a binary signature $\sigma$.
    Let $r=\rho(k,q)$.
    There is an algorithm which takes as input 
    \begin{itemize}
         \item a $\sigma$-structure $\str A$ of radius-$2r$ width at most $w$,
         \item its Gaifman increasing modification $\str B$ of order $\ell$ and radius-\(2r\) width at most \(w\), and
         \item \(\bigl(\ltp_{k,q}(\str A, a):a\in V(\str A)\bigr),\)
    \end{itemize}
    and computes
    $\bigl(\ltp_{k,q}(\str B, b):b\in V(\str B)\bigr)$
    in time $O_{k,q,w,\ell,\sigma_2}(|V(\str B)|+|E(\str B)|)$.
\end{proposition}


We first address issues related to representing the considered objects 
in the memory of a RAM machine.
\begin{remark}(Representation in machine model)\label{rem:representation}
\begin{itemize}
    \item The $\sigma$-structure $\str A$ is represented by 
  the adjacency matrix of each binary relation $R\in\sigma_2$,
  and the list representation of each unary relation $U\in \sigma_1$ with $U^\str A\neq\emptyset$.
\item 
We assume that the structure $\str B$ is represented symbolically,
by providing quantifier-free formulas $\phi_1,\ldots,\phi_k$ 
and the list of relations that are removed from $\str A\ast\phi_1\ast\cdots\ast\phi_k$ to obtain~$\str B$, whose joint size is bounded by $\ell$.
\item
  The local type $\ltp_{k,q}(a)$ involves only unary predicates 
in $\reach_{r}^{\str A}(a)\subset \sigma_1$
and the binary predicates in $\sigma_2$.
In particular, if $\str A$ has radius-$r$ width at most $w$,
then this local type involves at most $w+|\sigma_2|$ relation symbols.
For representation reasons, we assume that 
the unary predicates occurring in $\reach_r^{\str A}(a)$ are 
re-indexed as $U_1,\ldots,U_w$, via any injective mapping from $\reach_r^{\str A}(a)$
to $\set{U_1,\ldots,U_w}$.
Thanks to this, by \Cref{obs:typebound},
for $a\in V(\str A)$, the local type $\ltp_{k,q}(a)$ can be viewed as an element of a fixed, finite set $Q$,
depending only on $q,w$, and $\sigma$. All functions $f\from Q\to Q$ can be therefore assumed to be computable in time $O_{q,k,w,\sigma_2}(1)$.
\end{itemize}
\end{remark}






\medskip
Before proving \Cref{thm:modification},
we start with considering its two special cases:
when $\str B$ is a Gaifman \emph{preserving} modification of $\str A$,
and when $\str B=\str A\ast\bigl(\alpha(x)\land \beta(y)\bigr)$
for quantifier-free formulas $\alpha$ and $\beta$.
The general case will reduce easily to those two special cases.

The first case of Gaifman preserving modifications follows using standard techniques.

\begin{lemma}\label{lem:reduct}
The statement of  \Cref{thm:modification} holds when $\str B$ is a Gaifman preserving modification of $\str A$.
  \end{lemma}
  \begin{proof}
    As $\str B$ is a Gaifman preserving modification of $\str A$, the truth value of distance atoms is preserved.
It follows that for each $a\in V(\str A)$, the type
  $\ltp_{k,q}(\str B,a)$ can be obtained from $\ltp_{k,q}(\str A,a)$ by a simple rewriting.
  The conclusion follows.
  \end{proof}

The interesting special case is when $\str B=\str A\ast (\alpha(x)\land \beta(y))$.
Essentially, this is similar to performing a resolve operation (as in construction sequences) between the sets $\alpha^\str A$ and $\beta^\str A$ in $\str A$.
This lemma constitutes the most important step of our model checking algorithm.
The first half of the proof is necessary boilerplate, while the second half, where we rewrite the \(\dist(x,y)\) and \(\dist(x,Y)\) atoms,
contains the central idea.

\begin{lemma}\label{lem:connect}
The statement of    \Cref{thm:modification} holds when $\str B=\str A\ast\phi$,
where $\phi(x,y)=\alpha(x)\land \beta(y)$, for some quantifier-free formulas $\alpha(x),\beta(y)$ of total length $\ell$.
\end{lemma}
\begin{proof}
As part of this proof, we will use \Cref{lem:localglobaltypes} to compute global types from local types.
But as \Cref{lem:localglobaltypes} has a high run time dependence on the signature, and the signature of \(A\) is unbounded,
we first have to define an appropriate reduct with a bounded signature.
To this end, in linear time find any $a_0\in \alpha^\str A$ and $b_0\in\beta^\str A$.
Otherwise, if no such $a_0$ or $b_0$ exists then $\phi^\str A=\emptyset$ and the conclusion is trivial.
Let $$\sigma^*\coloneqq \reach^{\str B}_{2r}(a_0)\cup \reach^{\str B}_{2r}(b_0)\subset \sigma_1.$$
Then $|\sigma^*|\le 2w$, as $\str B$ has radius-$2r$ width at most $w$.
Moreover, the set $\sigma^*$ can be computed in time \(O(|V(A)|+|E(A)|)\) using a breadth-first search.

  Let $A^*\coloneqq A[\sigma^*]$ and \(B^* \coloneqq  B[\sigma^*]\) be the $(\sigma^*\cup\sigma_2)$-reducts of $\str A$ and \(\str B\).
  Denote by $\dist^A(b,\alpha\lor\beta)$ the minimum distance, in the Gaifman graph of $A$, between $b$ and any element $a\in V(A)$ with $A\models \alpha(a)\lor\beta(a)$.
  We now argue that for all $b\in V(\str B)$,
  \[
      \ltp_{k,q}(\str B,b)=\begin{cases}\tp_{k,q}(\str B[\reach_r^\str A(b)],b)&\text{if }\dist^{\str A}(b,\alpha\lor \beta)\ge r\\
          \ltp_{k,q}(B^*,b)&\text{if }\dist^{\str A}(b,\alpha\lor \beta)< r.
      \end{cases}
  \]
  Indeed, if $\dist^{\str A}(b,\alpha\lor \beta)\ge r$, then $\reach_{r}^\str B(b)=\reach_{r}^\str A(b)$, proving the first case.
  Otherwise,
  by symmetry we may assume that there is some \(a \in \alpha^A\) with \(\dist^A(b,a) < r\).
  Then any predicate in \(\reach_r^{\str B}(b)\) can be reached from \(b_0\) via at most \(2r\) steps in the Gaifman graph of \(B\) by
  first taking one step to \(a\), then up to \(r-1\) steps to \(b\) and then up to \(r\) steps to the predicate in \(\reach_r^B(b)\).
  Thus, $\reach_{r}^\str B(b)\subset \sigma^*$, proving the second case.


  Hence, we reduced the computation of \(\ltp_{k,q}(\str B,b)\) 
  to the computation of \(\ltp_{k,q}(\str B^*,b)\) and 
  \(\tp_{k,q}(\str B[\reach_r^\str A(b)],b)\).
  Moreover, as $\ltp_{k,q}(B^*,b)$ is the subset of those formulas in  $\tp_{k,q}(B^*,b)$ 
  which only mention unary predicates in $\reach^{B^*}_r(a)$, it thus suffices to compute
  \[
      \bigl(\tp_{k,q}(B^*,b):b\in V(\str B)\bigr) \quad\text{and}\quad 
      \bigl(\tp_{k,q}(B[\reach_r^{\str A }(b)],b):b\in V(\str B)\bigr).
  \]
  We will focus on computing the left-hand side, as the right-hand side can be computed analogously.
  As \(A^*\) is a reduct of \(A\), it is trivial to compute \(\ltp_{k,q}(A^*, a)\) from \(\ltp_{k,q}(A,a)\) by simply removing all formulas that use unary predicates outside \(\sigma^*\).
  Given that, by \Cref{lem:localglobaltypes}, we can further compute in time
  $O_{k,q,|\sigma^*|,|\sigma_2|}(|V(\str A)|+|E(A)|)$ 
  $$\bigl(\tp_{k,q}(A^*,a):a\in V(\str A)\bigr).$$


  In the remainder, we will compute \(\tp_{k,q}(\str B^*, a)\) from \(\tp_{k,q}(A^*, a)\) for all \(a \in V(A)\).
  This is the most important conceptual step of this lemma.
  This step is also the whole reason why we include unary distance atoms in the definition of dist-FO.
    To this end, observe the following equality that lets us express for all vertices \(a,b\) the distance atoms in \(B^*\) via distance atoms in \(A^*\).
    \begin{multline*}%
        \dist^{B^*}(a,b) \le r ~\iff~\\ \left(\dist^{A^*}(a,b) \le r\right) \quad \lor \bigvee_{\substack{r_1,r_2\in [r-1]\\r_1+1+r_2=r}} \left(\dist^{A^*}(a,\alpha^{\str A^*}) \le r_1\right) \land \left(\dist^{A^*}(b,\beta^{\str A^*}) \le r_2\right).
    \end{multline*}
    Crucially, the right hand side only uses distance atoms with radius at most \(r\), and thus
    for any formula, substituting a distance atom \(\dist(x,y) \le r\) by the corresponding distance atom on the right side
    of the above equivalence
    does not change the distance rank of the formula.

    Similarly, for all vertices \(a\) and unary relation symbols \(Y\in \sigma^*\),
    \begin{multline*}%
      \dist^{B^*}(a,Y) < r ~\iff~\left(\dist^{A^*}(a,Y) < r\right)\lor\\ \left(\dist^{A^*}(a,\alpha^{A^*}) < (r - 1 - \dist^{A^*}(\beta^{A^*},Y)) \right)
                            \lor  \left(\dist^{A}(a,\beta^{A^*}) < (r - 1 - \dist^{A^*}(\alpha^{A^*},Y)) \right).
    \end{multline*}
    Here, the right-hand sides of the inequalities are numerical constants that can be computed by a breadth-first search for each \(Y\in \sigma^*\) in time \(O_{l}(|V(A)|+|E(A)|)\).
    Thus, similar rank preserving substitutions can also be done for distance atoms of the form \(\dist(x,Y) < r\).

    Finally, for all $a\in V(\str B^*)$,
    to decide whether a formula is in the type $\tp_{k,q}(B^*,a)$,
    it suffices to replace the
    distance atoms according to the above equivalences, and to replace the newly introduced relation with the formula \(\alpha \land \beta\) itself,
    and to then check if the resulting formula is in the type $\tp_{k,q}(A^*,a)$.
    Here, we crucially use the fact that this rewriting does not increase the distance rank $(k,q)$ of the considered formulas.
\end{proof}








We now prove \Cref{thm:modification} in the general case.
\begin{proof}
    




Without loss of generality, and by \Cref{lem:reduct},
we may assume that $\str B$ is of the form $\str A\ast\phi$ for some 
quantifier-free formula $\phi(x,y)$ of size at most $\ell$.
Rewrite $\phi(x,y)$ into disjunctive normal form.
It is enough to consider the case where $\phi(x,y)$ has only one disjunct,
because if $\phi=\phi_1\lor\ldots\lor\phi_s$, 
then  $\str A\ast \phi$ is a Gaifman preserving modification of 
$\str A\ast\phi_1\ast\cdots\ast\phi_{s}$, 
so  using \Cref{lem:reduct} we can reduce to the case when $\phi$ has one disjunct.

So assume that $\phi(x,y)$ is a conjunction of literals (atomic formulas or their negations).
If $\phi(x,y)$ involves at least one conjunct of the form $R(x,y)$, for some $R\in\sigma_2\cup\set{=}$, 
then in fact $\str A\ast\phi$ is a Gaifman preserving modification of $\str A$,
and we are done, again by \Cref{lem:reduct}.

Otherwise, 
 $\phi(x,y)$ is of the form 
$$\phi(x,y)\ =\ \alpha(x)\land\beta(y)\land \neg R_1(x,y)\land \cdots\land \neg R_s(x,y),$$
for some 
quantifier-free formulas $\alpha(x)$ and $\beta(y)$,
and some binary relations $R_1,\ldots,R_s\in\sigma_2\cup\set{=}$.
Then the Gaifman graphs of $\str B=\str A\ast \phi$ and of
$\str B'\coloneqq \str A\ast (\alpha(x)\land\beta (y))$ are equal.
It follows that $\str B$ and $\str B'$ have equal radius-$r$ width.
Moreover, $\str B$ is a Gaifman preserving modification of $\str B'$.
The conclusion follows from \Cref{lem:connect} and \Cref{lem:reduct}.
\end{proof}

We also state a generalization of \Cref{thm:modification} for tuples, although without the algorithmic part. 
\begin{lemma}
    \label{lem:modification-tuples}
    Fix $k,q\in\N$, a $\sigma$-structure \(\str A\), and a Gaifman increasing modification \(\str B\).
    Then for every tuple $\tup a$ of at most $k$ elements of $A$, the local type $\ltp_{k,q}(\str B,\tup a)$ depends only on $\ltp_{k,q}(\str A,\tup a)$.
    More precisely, there is a function $F$ such that $\ltp_{k,q}(\str B,\tup a)=F(\ltp_{k,q}(\str A,\tup a))$, for every tuple of at most $k$ elements of $A$.
\end{lemma}
\begin{proof}
    The proof follows exactly the same steps as the proof of \Cref{thm:modification}. All those arguments lift to the case of tuples, at the cost of higher running times of the algorithms. As \Cref{lem:modification-tuples} does not mention running times, the conclusion holds.
\end{proof}



