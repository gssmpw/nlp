\section{Conclusion}

This work introduces Knowledge-Aware Bayesian Bandits (KABB), a novel framework that significantly advances multi-agent system coordination through three key innovations: a three-dimensional knowledge distance model, a dual adaptation mechanism, and a knowledge-aware Thompson sampling strategy. Extensive evaluations demonstrate KABB's superior performance across multiple benchmarks. Ablation experiments validate the effectiveness of the Knowledge-Aware mechanism and our MAB strategy. It is also verified that KABB is capable of addressing the challenges of dynamic expert coordination while maintaining computational efficiency, requiring fewer experts than baseline approaches. Our framework provides a promising direction for developing more adaptive and semantically-informed multi-agent systems, though future work could focus on optimizing output conciseness while maintaining response quality.



% , achieving a 77.9\% win rate on AlpacaEval 2.0, a leading score on MT-Bench
% , and better performance in most FLASK-Hard skill categories than other multi-agent systems or proprietary models

\textbf{Discussion.} 
The KABB framework advances interpretable and trustworthy AI systems through three transparent components: a knowledge distance metric for expert selection rationale, a graph-guided response integration process for reasoning paths, and a dual adaptation mechanism for learning evolution. These transparent features are crucial for responsible AI development as systems become increasingly complex and widely deployed.

% The KABB framework's knowledge-aware approach to expert coordination advances the development of interpretable and trustworthy AI systems. Through explicit modeling of expert capabilities and knowledge relationships, KABB provides transparency in three critical aspects: the three-dimensional knowledge distance metric that clarifies expert selection rationale, the knowledge graph-guided response integration process that reveals reasoning paths, and the dual adaptation mechanism that demonstrates learning evolution. As AI systems become increasingly complex and widely deployed, such transparent and accountable approaches are crucial for ensuring responsible AI development and deployment.

% While KABB demonstrates promising results in enhancing multi-agent coordination, several important directions warrant further investigation. 
% First, the current knowledge distance model could be extended to capture more nuanced semantic relationships between concepts. Incorporating hierarchical domain ontologies and cross-domain knowledge transfer mechanisms could enable more precise expert selection for interdisciplinary tasks. 
% Second, the temporal dynamics of expert capabilities deserve deeper exploration. As language models continue to evolve through updates and fine-tuning, developing more sophisticated temporal modeling approaches becomes crucial. This could involve integrating techniques from continual learning to better track and predict changes in expert performance over time. 

% The KABB framework's knowledge-aware approach to expert coordination could inform the development of more interpretable and trustworthy AI systems. By making expert selection and response integration processes more transparent, KABB provides a foundation for building accountable multi-agent systems that can better align with human oversight and ethical considerations.