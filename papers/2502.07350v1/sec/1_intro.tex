\section{Introduction}

With the rapid advancement of large language models (LLMs), their applications have expanded to complex tasks such as cross-domain knowledge integration and multistep decision-making. Although many LLMs \cite{achiam2023gpt,liu2024deepseek,adams2024llama,team2024gemma,bai2023qwen} demonstrate impressive versatility in various tasks through techniques such as in-context learning and instruction-tuning, their performance remains constrained by factors such as model size and the limitations of training data \cite{jiang2023llm,lu2024merge}. Scaling these models further to improve performance is prohibitively expensive and often requires retraining on datasets comprising trillions of tokens.
% 随着大型语言模型 LLMs 的快速发展,它们的应用已经扩展到更复杂的任务,例如跨域知识集成和多步骤决策。虽然许多LLMs模型通过上下文学习和指令调整等技术在各种任务中表现出令人印象深刻的多功能性,但它们的性能仍然受到模型大小和训练数据限制等因素的限制。进一步扩展这些模型以提高性能的成本高得令人望而却步,通常需要在包含数万亿个标记的数据集上进行重新训练。

Multi-Agent Systems (MAS) \cite{guo2024large} offer a promising alternative by coordinating multiple specialized agents to achieve superior performance compared to individual systems while maintaining manageable computational costs and budgets. Recent advances in MAS have led to the development of several frameworks. For example, the Mixture of Agents (MoA) \cite{wang2024mixture} employs multiple LLMs as proposers to iteratively refine responses, with a central aggregator delivering the final output. Although MoA has demonstrated robustness and scalability in deployment, its computational cost scales linearly with the number of agents, and significant redundancy and noise become a problem. For example, on datasets like MATH \cite{hendrycks2021measuring}, weaker models in the ensemble often interfere with the aggregator’s decisions, leading to incorrect results (see \cref{fig:math}).
% 多代理系统 (MAS) 提供了一种很有前途的替代方案,它通过协调多个专用代理来实现与单个系统相比的卓越性能,同时保持可管理的计算成本和预算。MAS 的最新进展导致了多个框架的发展。例如,代理混合 (MoA) 使用多个LLMs代理作为提议者来迭代优化响应,由中央聚合器提供最终输出。虽然 MoA 在部署中表现出了稳健性和可扩展性,但其计算成本与代理数量 (O(N)) 呈线性关系,引入了明显的冗余和噪声。例如,在像 MATH 这样的数据集上,集成中较弱的模型通常会干扰聚合器的决策,从而导致不正确的输出。

\begin{figure}[h]
\vskip 0.2in
\begin{center}
\centerline{\includegraphics[width=\linewidth]{Frame_396.pdf}}
\caption{Comparison of MoA and KABB (Ours) on solving a mathematical problem: MoA's aggregator is misled by conflicting weaker proposals, resulting in an incorrect answer, while KABB employs a knowledge-aware approach to drive related experts and arrive at the correct solution.}
\label{fig:math}
\end{center}
\vskip -0.2in
\end{figure}



Alternatively, Mixture of Experts (MoE) frameworks \cite{gong2024large,zhang2024optimizing,wang2023fusing,tang2023medagents}, in the context of multi-agent systems, focus on fostering collaboration among domain-specific experts, enabling the integration of diverse responses across fields. This approach reduces redundancy and noise. but is often limited to predefined tasks. A fundamental limitation of both frameworks lies in their reliance on static knowledge assumptions, making them ill-suited to address dynamic changes in expert capabilities or the emergence of novel concepts. These limitations highlight deeper challenges in MAS, particularly in areas such as knowledge understanding, response integration, and dynamic adaptability.
% 或者,专家混合 (MoE) 框架,在多智能体系统的语境下,专注于促进特定领域专家之间的合作,实现跨领域的不同响应的集成。这种方法减少了冗余和噪音,但通常仅限于预定义的任务。这两个框架更根本的局限性在于它们对静态知识假设的依赖,这使得它们不适合解决专家能力的动态变化或新概念的出现。这些局限性凸显了 MAS 中更深层次的挑战,尤其是在知识理解、响应整合和动态适应性等领域。

The increasing complexity of real-world scenarios requires systems that can adaptively select relevant knowledge domains and identify the optimal combination of experts. Multi-Armed Bandit (MAB) algorithms \cite{mahajan2008multi} have emerged as a powerful tool for tackling such dynamic decision problems. By striking a balance between ``exploration'' (discovering new expert combinations) and ``exploitation'' (leveraging known successful strategies), MAB can continuously optimize system performance. However, traditional MAB approaches rely solely on historical feedback, often overlooking the semantic relationships between tasks and experts.
% 现实世界场景的日益复杂要求系统能够自适应地选择相关知识领域并确定最佳专家组合。多臂老虎机 (MAB) 算法已成为解决此类动态决策问题的强大工具。通过在“探索”(发现新的专家组合)和“利用”(利用已知的成功策略)之间取得平衡,MAB 可以持续优化系统性能。然而,传统的 MAB 方法仅依赖于历史反馈,往往忽视了任务和专家之间的语义关系。

To bridge this gap, knowledge graphs \cite{ge2024knowledge} provide a compelling framework for representing and leveraging these semantic connections. By structuring expert capabilities and task requirements as interconnected knowledge networks, knowledge graphs enable: (1) precise modeling of dependencies across knowledge domains, (2) dynamic tracking of expert capabilities over time, and (3) identification of knowledge gaps in task-solving pathways. This structured representation not only enhances the accuracy of expert selection but also provides semantic-level guidance for response integration. Together, these advancements pave the way for more adaptive, efficient, and semantically informed multi-agent collaboration systems.

% 为了弥合这一差距,知识图谱提供了一个引人注目的框架来表示和利用这些语义连接。通过将专家能力和任务要求构建为互连的知识网络,知识图谱可以实现:跨知识领域的依赖关系的精确建模,随时间动态跟踪专家能力,以及识别任务解决途径中的知识差距。这种结构化表示不仅提高了专家选择的准确性,还为响应集成提供了语义级指导。总之,这些进步为更具适应性、效率和语义知情的多智能体协作系统铺平了道路。

In this work, we propose the Knowledge-Aware Bayesian Bandits (KABB) framework to significantly enhance the coordination capabilities of multi-agent systems through three core innovations.
% 在这项工作中,我们提出了知识感知贝叶斯强盗 (KABB) 框架,通过三个核心创新显着增强多智能体系统的协调能力。
First, we introduce a three-dimensional knowledge distance model grounded in deep semantic understanding, which surpasses traditional keyword-based methods by integrating concept overlap, dependency path optimization, and dynamic historical performance evaluation. Specifically, expert capabilities and task requirements are represented as vectors, with concept overlap calculated using enhanced cosine similarity, dependency path lengths optimized through hierarchical knowledge relationships, and historical feedback dynamically adjusted via an adaptive time-decay factor. These components are unified into a comprehensive distance metric, further refined with deep learning techniques to optimize the weight parameters.
% 首先,我们引入了一个基于深度语义理解的三维知识距离模型,该模型通过集成概念重叠、依赖路径优化和动态历史性能评估,超越了传统的基于关键字的方法。具体来说,专家能力和任务要求表示为向量,使用增强的余弦相似度计算概念重叠,通过分层知识关系优化依赖路径长度,并通过自适应时间衰减因子动态调整历史反馈。这些组成部分被统一为一个全面的距离度量,并通过深度学习技术进一步细化以优化权重参数。

Second, we develop a dual adaptation mechanism to support continuous expert optimization and knowledge evolution. This mechanism employs Bayesian parameter updates with exponential time decay to mitigate the influence of outdated data while dynamically adjusting key metrics within the knowledge graph, such as concept overlap and historical performance. This ensures that expert capabilities remain adaptive to the evolving demands of tasks in real-time.
% 其次,我们开发了一种双重适应机制来支持持续的专家优化和知识进化。这种机制采用具有指数时间衰减的贝叶斯参数更新来减轻过时数据的影响,同时动态调整知识图谱中的关键指标,例如概念重叠和历史性能。这确保了专家能力能够实时适应不断变化的任务需求。

Finally, we design a knowledge-aware Thompson sampling strategy to improve computational efficiency in expert selection. By incorporating the knowledge distance metric into the Beta distribution sampling process, our strategy enables efficient identification of the top-k experts for dynamic decision-making. This approach demonstrated significant improvements in performance and cost efficiency on leading datasets like AlpacaEval 2.0 \cite{dubois2024length}. 
Additionally, a two-stage knowledge graph-guided response integration process ensures logical consistency by detecting semantic conflicts and enhancing contextual coherence, thus substantially reducing contradictory output.
% 最后,我们设计了一个知识感知的汤普森抽样策略,以提高专家选择的计算效率。通过将知识距离度量纳入 Beta 分布抽样过程,我们的策略能够有效地识别前 k 名专家以进行动态决策。在 AlpacaEval 2.0 基准的评估期间,这种方法在决策准确性和计算成本方面表现出显着改善。此外,知识图谱指导的两阶段响应集成过程通过检测语义冲突和增强上下文连贯性来确保逻辑一致性,从而大大减少矛盾的输出。

Together, our innovations enable the KABB framework to effectively address the challenges of dynamic expert coordination, offering a scalable, adaptive, and semantically informed solution for multi-agent systems in complex real-world scenarios.
% 总之,我们的创新使 KABB 框架能够有效应对动态专家协调的挑战,为复杂实际场景中的多代理系统提供可扩展、自适应和语义知情的解决方案。



