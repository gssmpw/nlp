\section{Related Work}
Our work is inspired by a substantial body of prior research in sustainable human-computer interaction (SHCI), methods for recycling or reusing electronic and electronic waste, as well as technical explorations in PCB substrate repair and renewal.

\subsection{Sustainability in HCI: Making and Prototyping}

The notion of Sustainable Interaction Design (SID) was introduced by Blevis____ over a decade ago, providing a foundational framework for addressing environmental impacts and human behavior in the design of interactive technologies. 
This concept has since evolved into the broader field of SHCI.

Early discussions in SHCI often centered on mobile applications and their influence on end-users' daily behaviors, such as reducing energy consumption through persuasive computing____. 
More recently, attention has shifted to the environmental impact of making and physical prototyping____, driven by the democratization of personal fabrication tools and the growing maker movement____.

Several studies have explored end-users' (creative) approaches to engaging with wasted physical materials in daily activities. 
For example, Yan et al.____ have presented a qualitative research that maps out the sustainability practices, challenges and opportunities in modern makerspace setups and have called for new tools and infrastructure to support making sustainably. Kim and Paulos____ have proposed a reuse composition framework, based on online surveys and observations, to inspire the creative reuse of material waste. 
Dew and Rosner____ have conducted design explorations that examine how designers conceptualize, manage, and rework waste materials in educational makerspaces. 
Similarly, Maestri and Wakkary____ have studied the intersection of repair and creativity within household settings. 
These ideas have since evolved into broader concepts, such as unmaking____, uncrafting____, and unfabricating____, which employ speculative or participatory design lenses to explore the afterlife of objects and materials.


Alongside the exploration of reusing daily waste, HCI researchers have begun investigating the use of decomposable and biodegradable materials in making. 
For example, several projects have proposed using edible materials____ or substances derived from food waste____ as construction materials for molding and 3D printing. 
Microbe-based materials, such as yeast____ and fungi____, as well as biomaterials derived from living organisms, including algae____ and cellulose-based fibers____, have also been proposed as building materials for the prototyping of interactive devices. 

In addition, new fabrication processes and tools have been developed to support more sustainable making practices. 
For example, Filament Wiring____ and Substiports____ introduce alternative fabrication pipelines that repurpose wasted 3D printing filament or failed prints for new designs. EcoThreads____ and Desktop Biofibers Spinning____ have developed new machines and processes to make water-dissolvable yarns easily accessible for sustainable textile applications. 

Our work is greatly inspired by the aforementioned advancements in sustainable making, with a specific focus on the processes involved in PCB making. As discussed in the introduction, PCBs are among the largest contributors to e-waste. Our work aims to reduce this environmental impact.















\subsection{Supporting the Reuse and Recycling of Electronics}
E-waste recycling requires interdisciplinary research and collaborative practices.

In the electronics management industry, the primarily focus is on infrastructure and large-scale processes that can extract raw materials from PCB scrap. 
For example, chemical and mechanical techniques are used to recover valuable materials, including refractory metals and elements of the platinum group found in standard PCB waste____. 
Although effective, these industrial and centralized approaches void the opportunities for PCBs that might be repurposed, repaired, or reused, and they may fall short as more individuals become involved in creating electronics through the democratization of making tools. 

Recent HCI literature points out that many end users are no longer just consumers of physical artifacts but also their creators. 
Consequently, they bear greater responsibility for managing the material waste generated during the individual making process____. 
In this context, much of the HCI research focuses on promoting the reuse and recycling of electronics at the individual level. For example, the CurveBoards project____ proposes a custom-shaped breadboard design that is versatile for rapid prototyping with form-specific requirements. 
CircuitGlue____ reduces waste in prototyping by allowing easy integration and reuse of off-the-shelf components.
SolderlessPCB____ demonstrates a pressure-based PCB assembly method using 3D printed or CNC-made housings, allowing easy disassembly and reuse of surface-mounted components.
ecoEDA____ shows how interactive circuit design software, by integrating early-stage suggestions for utilizing recyclable electronic components from stock PCBs, can facilitate the reuse of electronics throughout the design process.

New, more environmentally friendly PCB materials and compositions have also been explored. For example, transesterification vitrimers have been proposed as PCB substrate materials, which can be recycled through polymer swelling, achieving a 98\% polymer recovery____. Several studies have investigated PCB substrates based on paper____, wood____, and water-soluble materials____. 
Water-soluble materials are particularly interesting in the context of sustainable electronics, as their degrading processes are controllable. 
This enables the creation of transient electronic prototypes____ with programmable lifespans, simplifying the recycling of materials once they are no longer needed____.

Our work also aims to reduce material waste from PCBs. However, instead of focusing on new materials that may not be readily available to many, we seek to improve the workflow of the existing FR-4 substrate-based PCB manufacturing process. Our approach relies solely on off-the-shelf conductive epoxy and CNC engraving machines, which have become more affordable and widely available in makerspaces. As a result, our method has the potential to be adopted at scale.



\subsection{PCB Substrate Repair and Renewal}
Although PCBs are generally considered irreversible, several solutions have been proposed to repair minor errors or shorts. 
For example, jumper wires can restore electrical continuity between disconnected points____, while conductive ink pens enable temporary, ad-hoc circuit repairs____. 
However, these methods are primarily effective for minor fixes, such as bridging gaps over short distances, and are not suitable for more complex repairs that require removing multiple conductors or altering component footprints and placements.

Several studies have investigated methods for fixing regional circuit errors. 
For example, Chen et al.____ have developed a local electroplating technique to repair constrictions in copper traces.
Lim et al.____ have proposed repairing broken circuit traces using reduced graphene oxide on a laser direct writing platform.
Lange____ has demonstrated the use of UV and IR lasers to trim fuzzy edges of conductor shapes on PCBs, reducing the defect rates in PCB products.
However, these approaches focus on repairing defects in PCB traces rather than addressing circuit design errors through rerouting or editing existing circuits.

Prior to our work, preliminary explorations have demonstrated the potential of using conductive filler deposition to modify or repair existing circuit diagrams on substrates.
For example, Self-healing UI ____ has introduced a composite material
capable of autonomously repairing circuit wiring made of multiwall carbon nanotubes by leveraging the dynamic cross-linking properties of polyborosiloxane polymers.
However, carbon nanotubes are hazardous and require specialized handling, and circuits made with this composite are limited to low-fidelity prototypes.
Circuit Eraser____ has proposed using a standard eraser to remove circuit traces printed with conductive ink, facilitating rapid iteration of circuit design.
Silver Tape____ enables circuit trace repair via tape transfer of inkjet-printed silver ink.
Furthermore, Marghescu et al. ____ and Drumea et al. ____ have evaluated the current-carrying capacity of sectional circuit traces made with nickel and silver paste, confirming the potential of PCB repair using conductive pastes.



Building upon previous research, we investigate the additive method of paste deposition as an alternative to the conventional subtractive PCB engraving process.
This approach enables the renewal of circuit boards originally fabricated using methods such as CNC engraving or photochemical etching. 
Furthermore, our method enables the editing of large conductive areas, allowing an existing PCB designed for a specific purpose to be repurposed for different projects. 
This, therefore, increases the opportunity to reuse otherwise wasted PCBs, reducing unnecessary e-waste.