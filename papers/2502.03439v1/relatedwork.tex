\section{Related Work}
\subsubsection{Point-cloud data analysis methods}

The LOT embedding method is not the first technique to analyze point cloud-valued data. However, there have been significant advances in the development of methods that address these challenges, allowing machine learning to operate on point-cloud-valued data. Graph neural networks \cite{wang2019} treat point-cloud data as graphs and propagate information through this graph to learn features useful for classification or segmentation. 
DeepSets \cite{zaheer2017deepsets} provides another neural network architecture that is permutation invariant to the inputs which allow it to work for point-cloud data quite easily.  PointNet \cite{qi2017} and PointNet++ \cite{qi2017pointnetplus} introduce a deep learning architecture that consumes raw point-clouds and outputs classification and segmentation results. One can also create an encoder that learns a compact latent representation of the point-cloud, and a decoder that attempts to reconstruct the original point-cloud from this latent space \cite{achlioptas2018learning}. To extract topological features from point-clouds, one uses topological data analysis (TDA) methods \cite{cao2022} like persistent homology. To learn more geometric features of the point-cloud data, one can generate a graph from the point-cloud and consider spectral properties of the graph. In particular, \cite{robertson2024} generalizes Wasserstein distance to connection graphs and explores its use in supervised and unsupervised learning on point-cloud-valued data. As permutation invariance is needed for point-cloud data analysis, LOT indeed exhibits this permutation invariance since it uses optimal transport to generate the embeddings.

Recently, \cite{werenski2024} explored the \textit{synthesis} and \textit{analysis} problem (given by barycentric coding models) associated with the Wasserstein barycenter problem and related it to the linearized Wasserstein barycentric coding model. They showed that if compatibility holds for all the probability measures involved, then solving the linearized barycenter synthesis problem will also solve the regular Wasserstein barycenter synthesis problem.  We implement the method that solves linearized Wasserstein barycenter synthesis problem, but we call it LOT barycenters.  We further implement iterative embeddings, where we iterate between generating an LOT barycenter and updating the reference measure for LOT embeddings to be the LOT barycenter, and see that the LOT barycenter after a few iterations becomes much closer to the true barycenter.

\subsubsection{Tooth dataset}

The tooth dataset that we use is a collection of 3D scans of teeth, originally utilized for morphological and geometric analysis of biological structures. This dataset primarily contains detailed point-cloud representations of teeth, which capture both the shape and size of each tooth in high resolution. It's main use in research studies has been to explore anatomical variations, evolutionary relationships, and geometric shape analysis across different species. A notable characteristic of the dataset is its application to comparative biology, where it has enabled the study of dental morphology and its implications for taxonomic classification, functional morphology, and evolutionary adaptation. Each tooth in the dataset is represented as a point-cloud or mesh, allowing for advanced computational techniques to quantify geometric similarity, perform shape analysis, and classify dental structures.

The dataset was first studied by \cite{Boyer2011} to automatically quantify geometric similarity in anatomical structures, particularly focusing on the analysis of 3D tooth shapes. Subsequently, \cite{Gao2021} introduced a new method of analyzing datasets represented as fiber bundles, applying diffusion maps to capture the geometric structure of the data. \cite{StClair2016} studied molar shape and size variation across different primate species, using the Tooth Dataset for its detailed 3D reconstructions of dental structures. The work contributed to understanding evolutionary changes in dental morphology.

Our case study is conducted on this curated dataset comprised of high-resolution CT scans of $N = 58$ mandibular molars from two primate suborders. The genera analyzed include Microcebus ($N = 11$), Mirza ($N = 5$), Saimiri ($N = 9$), and Tarsius ($N = 33$). Each tooth is represented as a pair $(\mu_i, y_i)$, where $\mu_i \subset \mathbb{R}^3$ is a point cloud consisting of over 5000 points in three-dimensional space, and $y_i$ denotes the corresponding genus label.