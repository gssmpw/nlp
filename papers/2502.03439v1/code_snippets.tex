The necessary libraries needed for running pyLOT on the tooth dataset and actually loading the tooth dataset are given by Listing \ref{code:imports}.

\begin{lstlisting}[language=Python, caption=Import Statements, label={code:imports}]
# Import necessary libraries
from pyLOT.classifier import LOTClassifier
from pyLOT.barycenters import LOTBarycenter
from pyLOT.embed import LOTEmbedding
from pyLOT.reduction import LOTDimensionalityReduction

# libraries for computation and plotting
import numpy as np
import matplotlib.pyplot as plt
from mpl_toolkits.mplot3d import Axes3D
from matplotlib.colors import ListedColormap
# libraries for PCA visualization
from sklearn.decomposition import PCA
# oversampling and train/test splits
from imblearn.over_sampling import RandomOverSampler
from sklearn.model_selection import train_test_split
# import time for timing how long code takes to run
import time
from datetime import datetime

ROOT_DIR = '../cleaned_real_data/'
SUB_DIR = ['Microcebus/','Mirza/','Saimiri/','Tarsius/']


def read_off(filepath):
    file = open(filepath)
    if 'OFF' != file.readline().strip():
        raise('Not a valid OFF File')
    n_verts, n_faces, n_edges = tuple([int(s) for s in file.readline().strip().split(' ')])
    verts = [[float(s) for s in file.readline().strip().split(' ')] for _ in range(n_verts)]
    return np.array(verts)
    
def load_data():
    # store file paths and tooth labels
    fp_lst = []
    xt_lst = []
    T_labels = []
    for dir in SUB_DIR:
        fps = os.listdir(ROOT_DIR + dir)
        fp_lst.extend(fps)
        T_labels.extend([dir[:-1]] * len(fps))

    for i in range(len(fp_lst)):
        xt = read_off(ROOT_DIR + T_labels[i] + '/' + fp_lst[i])
        xt_lst.append(xt)

    xt_lst = np.array(xt_lst)
    T_labels = np.array(T_labels)
    return xt_lst, T_labels

\end{lstlisting}

















 This associated code is in Listing \ref{code:lotEmbed}.

\begin{lstlisting}[language=Python, caption=Code to embed tooth samples., label={code:lotEmbed}]
xt_list, T_labels = load_fps() # load tooth dataset with appropriate labels
xr = np.random.normal(0, 1, size=(5000, 3)) # set reference measure with 5000 points

print('Generating Sinkhorn embeddings for tooth data...')
start_time = time.time()
# Compute LOT Sinkhorn embeddings with small lambda
T_embeddings_sh = LOTEmbedding.embed_point_clouds(xr, 
                                            xt_list,
                                            xt_masses=None, # allows masses if needed.  Defaults to None (i.e. assume uniform mass)
                                            sinkhorn=True, 
                                            lambd=0.05)
time_taken = time.time() - start_time
print(f'Time taken for Sinkhorn embeddings: {time_taken:.2f} seconds')

print('Generating LP embeddings for tooth data...')
start_time = time.time()
# Compute LOT embeddings
T_embeddings = LOTEmbedding.embed_point_clouds(xr, 
                                            xt_list,
                                            sinkhorn=False)
time_taken = time.time() - start_time
print(f'Time taken for LP embeddings: {time_taken:.2f} seconds')

print('Generating embeddings for tooth data with iterated embeddings...')
start_time = time.time()
all_bary_embeddings, \
    all_emd_lists, \
    barycenter_list, \
    barycenter_label_list = LOTEmbedding.compute_barycenter_embeddings(n_iterations=3, 
                                        embeddings=None, # starts from Gaussian reference if None, else generates barycenter from embeddings 
                                        labels=T_labels, # to generate class-specific barycenters 
                                        pclouds=xt_list, 
                                        masses=None, 
                                        n_reference_points=5000,
                                        n_dim=3 # dimension of data
                                        )
time_taken = time.time() - start_time
print(f'Time taken for iterated LP embeddings: {time_taken:.2f} seconds')

\end{lstlisting}




 The code to reproduce the tests here are given in Listing \ref{code:reduction}

\begin{lstlisting}[language=Python, caption=Code for PCA and LDA reduction., label={code:reduction}]
# run LDA for T_embeddings
T_lda, labels_balanced = LOTDimensionalityReduction.lda_reduction(T_embeddings, T_labels, n_components=3)
# plot 2D LDA
df_lda = pd.DataFrame(T_lda)
df_lda['label'] = label_balanced
fig = plt.figure(figsize=(8,6))
ax = sns.scatterplot(df_lda, x=0, y=1, hue='label')
sns.move_legend(ax, "lower center", bbox_to_anchor=(.7, 1), ncol=4, title=None, frameon=False)
plt.title("LDA CLuster (2D)", loc='left')
plt.savefig('../fig/cluster/gaussian_ref/lda_cluster_2d.png')
# plot 3D LDA
fig = plt.figure(figsize=(8,6))
ax = fig.add_subplot(1, 1, 1, projection='3d')
for l in df_lda['label'].unique():
    ax.scatter(df_lda[0][df_lda['label']==l], df_lda[1][df_lda['label']==l], df_lda[2][df_lda['label']==l],label=l)
ax.legend()
sns.move_legend(ax, "lower center", bbox_to_anchor=(.8, 1), ncol=4, title=None, frameon=False)
plt.title("LDA CLuster (3D)", loc='left')
plt.savefig('../fig/cluster/gaussian_ref/lda_cluster_3d.png')


# run PCA for T_embeddings
U, S, Vh, labels_balanced = LOTDimensionalityReduction.pca_reduction(T_embeddings, T_labels)
# plot 2D PCA
df_pca = pd.DataFrame(U)
df_pca['label'] = label_balanced
fig = plt.figure(figsize=(8,6))
ax = sns.scatterplot(df_pca, x=0, y=1, hue='label')
sns.move_legend(ax, "lower center", bbox_to_anchor=(.7, 1), ncol=4, title=None, frameon=False)
plt.title("PCA CLuster (2D)", loc='left')
plt.savefig('../fig/cluster/gaussian_ref/pca_cluster_2d.png')
# plot 3D PCA
fig = plt.figure(figsize=(8,6))
ax = fig.add_subplot(1, 1, 1, projection='3d')
for l in df_pca['label'].unique():
    ax.scatter(df_pca[0][df_pca['label']==l], df_pca[1][df_pca['label']==l], df_pca[2][df_pca['label']==l],label=l)
ax.legend()
sns.move_legend(ax, "lower center", bbox_to_anchor=(.8, 1), ncol=4, title=None, frameon=False)
plt.title("PCA CLuster (3D)", loc='left')
plt.savefig('../fig/cluster/gaussian_ref/pca_cluster_3d.png')

\end{lstlisting}



The code associated with the random oversampling is 

\begin{lstlisting}[language=Python, caption=Code for classification of teeth., label={code:classify}]
# generate train/test split and oversample
sampler = RandomOverSampler()
T_train, T_test, y_train, y_test = train_test_split(T_embeddings, T_labels, test_size=0.2, random_state=21)
# Keep 25 percent from each label OR oversampling (no duplicate on train and test)
X_train, y_train = sampler.fit_resample(T_train, y_train)
X_test, y_test = sampler.fit_resample(T_test, y_test)

# Train the classifier on the data
best_clf = LOTClassifier.get_best_classifier(X_train, y_train, X_test, y_test)
print("Best Classifier:\n", best_clf)
\end{lstlisting}




The results were generated by the following code:
\begin{lstlisting}[language=Python, caption=Code for barycenter generation., label={code:barycenter}]
# following code works for general situation but for specific case, assume that T_embeddings has only 3 samples per class (the ones from the visual)

# fixed weights for the three teeth
WEIGHTS_FIXED = np.array([
    [1,0,0], [0,1,0], [0,0,1], 
    [0.9, 0.1, 0], [0.9, 0, 0.1], [0.1, 0.9, 0], 
    [0, 0.9, 0.1], [0.1, 0, 0.9], [0, 0.1, 0.9],
    [0.6, 0.2, 0.2], [0.6, 0.3, 0.1], [0.6, 0, 0.4], 
    [0.2, 0.6, 0.2], [0.3, 0.6, 0.1], [0, 0.6, 0.4], 
    [0.2, 0.2, 0.6], [0.3, 0.1, 0.6], [0, 0.4, 0.6], 
    [0.2, 0.4, 0.4], [0.4, 0.2, 0.4], [0.4, 0.4, 0.2]
])

FIXED_SAMPLE_SIZE = 3
WEIGHTS_RAND_LEN = 20

WEIGHTS_RAND, R_labels = generate_weight(T_embeddings, T_labels, n=WEIGHTS_RAND_LEN)

tic = time.time()
# generate LOT barycenters with fixed weights for teeth in class
G_matrix, G_labels, _ = LOTBarycenter.generate_barycenters_within_class(T_embeddings, T_labels, WEIGHTS_FIXED, uniform=False, n=WEIGHTS_RAND_LEN)
toc = time.time()
print('Time taken for fixed-weights LOT barycenter', toc-tic)

# generate LOT barycenters with random weights for teeth in class
R_matrix, R_labels, WEIGHTS_RAND = LOTBarycenter.generate_barycenters_within_class(T_embeddings, T_labels, None, uniform=True, n=WEIGHTS_RAND_LEN)

# TRUE Barycenter calculations 
def compute_barycenter(xr, xt_class, w):
    b = np.ones((5000,)) / 5000
    measures_weights = [ot.unif(xt.shape[0]) for xt in xt_class]
    center = ot.lp.free_support_barycenter(xt_class, measures_weights, xr, b, w)

    return center

# Fixed Weights true barycenter
tic = time.time()
for label in np.unique(T_labels):
    print(datetime.now())
    C_matrix = []
    xt_class = xt_lst[T_labels == label][:FIXED_SAMPLE_SIZE]
    for weight in FIXED_WEIGHTS:
        C_matrix.append(compute_barycenter(xr, xt_class, weight))
    print(f"Generation for class {label} is done. (Fixed Weight)")
    with open(f'../pickle/C_matrix_{label}.pkl', 'wb') as file:
        file.write(pickle.dumps(np.array(C_matrix))) 
toc = time.time()
print('Time taken for fixed-weights true barycenter', toc-tic)

# Random Weights true barycenter
tic = time.time()
for label in np.unique(T_labels):
    print(datetime.now())
    B_matrix = []
    xt_class = xt_lst[T_labels == label]
    class_weight = WEIGHTS_RAND[R_labels == label]
    for weight in class_weight:
        B_matrix.append(compute_barycenter(xr, xt_class, weight))
    print(f"Generation for class {label} is done. (Random Weight)")
    with open(f'../pickle/B_matrix_{label}.pkl', 'wb') as file:
        file.write(pickle.dumps(np.array(B_matrix))) 
toc = time.time()
print('Time taken for random weights true barycenter', toc-tic)

# PLOTTING
Tarsius_0 = T_matrix[-33].reshape((5000,3))
Tarsius_1 = T_matrix[-32].reshape((5000,3))
Tarsius_2 = T_matrix[-31].reshape((5000,3))

# Plot sample 1
fig = plt.figure()
ax = fig.add_subplot(1, 1, 1, projection='3d')
ax.scatter(Tarsius_0[:,0], Tarsius_0[:,1], Tarsius_0[:,2], c = Tarsius_0[:,2])
plt.title("Tooth Sample #1")
plt.savefig('../fig/barycenter_comp/fixed_weight_example/tarsius_first.png')

# Plot sample 2
fig = plt.figure()
ax = fig.add_subplot(1, 1, 1, projection='3d')
ax.scatter(Tarsius_1[:,0], Tarsius_1[:,1], Tarsius_1[:,2], c = Tarsius_1[:,2])
plt.title("Tooth Sample #2")
plt.savefig('../fig/barycenter_comp/fixed_weight_example/tarsius_second.png')

# Plot sample 3
fig = plt.figure()
ax = fig.add_subplot(1, 1, 1, projection='3d')
ax.scatter(Tarsius_2[:,0], Tarsius_2[:,1], Tarsius_2[:,2], c = Tarsius_2[:,2])
plt.title("Tooth Sample #3")
plt.savefig('../fig/barycenter_comp/fixed_weight_example/tarsius_third.png')

# Plot LOT Barycenter using tooth sample 1 and 3
G = G_matrix[-10].reshape((5000,3))
fig = plt.figure()
ax = fig.add_subplot(1, 1, 1, projection='3d')
ax.scatter(G[:,0], G[:,1], G[:,2], c = G[:,2])
plt.title("LOT Approx. Barycenter using #1 & #3")
plt.savefig('../fig/barycenter_comp/fixed_weight_example/Tarsius_AB_13.png')

# Plot LOT Barycenter using tooth sample 2 and 3
G = G_matrix[-7].reshape((5000,3))
fig = plt.figure()
ax = fig.add_subplot(1, 1, 1, projection='3d')
ax.scatter(G[:,0], G[:,1], G[:,2], c = G[:,2])
plt.title("LOT Approx. Barycenter using #2 & #3")
plt.savefig('../fig/barycenter_comp/fixed_weight_example/Tarsius_AB_23.png')

# Plot True Barycenter using tooth sample 1 and 3
fig = plt.figure()
ax = fig.add_subplot(1, 1, 1, projection='3d')
ax.scatter(C_matrix_Tarsius[-10][:,0], C_matrix_Tarsius[-10][:,1], C_matrix_Tarsius[-10][:,2], c = C_matrix_Tarsius[-10][:,2])
plt.title("True Barycenter using #1 & #3")
plt.savefig('../fig/barycenter_comp/fixed_weight_example/Tarsius_TB_13.png')

# Plot True Barycenter using tooth sample 2 and 3
fig = plt.figure()
ax = fig.add_subplot(1, 1, 1, projection='3d')
ax.scatter(C_matrix_Tarsius[-7][:,0], C_matrix_Tarsius[-7][:,1], C_matrix_Tarsius[-7][:,2], c = C_matrix_Tarsius[-7][:,2])
plt.title("True Barycenter using #2 & #3")
plt.savefig('../fig/barycenter_comp/fixed_weight_example/Tarsius_TB_23.png')


\end{lstlisting}





With code applied to the tooth dataset, we simply run \texttt{LOTEmbedding.compute\_barycenter\_embeddings} and plot.  In particular, we have the following code:
\begin{lstlisting}[language=Python, caption=Code for iterative barycenter generation., label={code:iterBary}]
all_bary_embeddings, \
    all_emd_lists, \
    barycenter_list, \
    barycenter_label_list = LOTEmbedding.compute_barycenter_embeddings(n_iterations=3, 
                                        embeddings=None,
                                        labels=T_labels, 
                                        pclouds=xt_list, 
                                        masses=None, 
                                        n_reference_points=5000,
                                        n_dim=3
                                        )
# all_bary_embeddings contain all the iterative embeddings with respect to the barycenters for each class
# barycenter_label_list contains barycenters after successive embeddings
# run similar plotting, dimensionality reduction, and classifier analysis as before
\end{lstlisting}


