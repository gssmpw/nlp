%%
%% This is file `sample-sigconf.tex',
%% generated with the docstrip utility.
%%
%% The original source files were:
%%
%% samples.dtx  (with options: `all,proceedings,bibtex,sigconf')
%% 
%% IMPORTANT NOTICE:
%% 
%% For the copyright see the source file.
%% 
%% Any modified versions of this file must be renamed
%% with new filenames distinct from sample-sigconf.tex.
%% 
%% For distribution of the original source see the terms
%% for copying and modification in the file samples.dtx.
%% 
%% This generated file may be distributed as long as the
%% original source files, as listed above, are part of the
%% same distribution. (The sources need not necessarily be
%% in the same archive or directory.)
%%
%%
%% Commands for TeXCount
%TC:macro \cite [option:text,text]
%TC:macro \citep [option:text,text]
%TC:macro \citet [option:text,text]
%TC:envir table 0 1
%TC:envir table* 0 1
%TC:envir tabular [ignore] word
%TC:envir displaymath 0 word
%TC:envir math 0 word
%TC:envir comment 0 0
%%
%%
%% The first command in your LaTeX source must be the \documentclass
%% command.
%%
%% For submission and review of your manuscript please change the
%% command to \documentclass[manuscript, screen, review]{acmart}.
%%
%% When submitting camera ready or to TAPS, please change the command
%% to \documentclass[sigconf]{acmart} or whichever template is required
%% for your publication.
%%
%%
\documentclass[sigconf]{acmart}%



\usepackage{natbib} % various citation commands
\usepackage{booktabs}
\usepackage{comment}
\usepackage{graphicx}
%\usepackage{datetime}
\usepackage{subfig}
%\usepackage{a4wide}
\usepackage{fancyhdr}
\usepackage{amsmath}
%\usepackage{amssymb}
\usepackage{xurl}
\usepackage{adjustbox}
\usepackage{arydshln}
\usepackage{relsize}
\usepackage{wasysym}
\usepackage{multirow}
\usepackage[makeroom]{cancel}
\usepackage{scalerel,graphicx,xparse}
\usepackage[capitalize,nameinlink]{cleveref}
% \usepackage{xcolor}
\usepackage{algorithm}
\usepackage{algpseudocode}
\usepackage{algorithm}
%\usepackage[noend]{algpseudocode}
\usepackage{lscape}
\usepackage{graphicx}
\usepackage{enumitem}
%\usepackage{subcaption}

\graphicspath{{../}{images/}}
% \usepackage[dvipsnames,table,xcdraw]{xcolor}
\usepackage{hyperref}

\definecolor{navy}{rgb}{0.1, 0.1, 0.8}
\definecolor{gray}{rgb}{0.4, 0.4, 0.4}
\definecolor{olive}{rgb}{0.1, 0.5, 0.1}
\definecolor{ruby}{rgb}{0.8, 0.1, 0.3}
\definecolor{darkpastelgreen}{rgb}{0.01, 0.75, 0.24}
\definecolor{celestialblue}{rgb}{0.29, 0.59, 0.82}
\definecolor{coral}{rgb}{1.0, 0.5, 0.31}
\definecolor{blue}{rgb}{0.23, 0.44, 0.62}
\definecolor{Goldenrod}{rgb}{0.8,0.8,0}
\definecolor{pinky}{RGB}{255,20,147}  

% \newcommand{\markup}[1]{{\color{coral}{#1}}}
\newcommand{\markup}[1]{#1} 

%%%%%%%% comments -- ENABLE (draft) %%%%%%%%%%%%%%
\usepackage{soul}
\usepackage[colorinlistoftodos,prependcaption,textsize=tiny,textwidth=15mm]{todonotes}
% MAR: this is needed to get the right todonots to show completely
\setlength{\marginparwidth}{14mm}
\usepackage{xspace}

\newcommand{\eb}[1]{\textcolor{pinky}{Emily: #1}}
\newcommand{\revMAR}[1]{\textcolor{blue}{MAR: #1}}
%% MAR: I can't stand the neon green
\newcommand{\cc}[1]{\textcolor{ForestGreen}{#1}}
\newcommand{\revFB}[1]{\textcolor{teal}{FB: #1}}

% %% MAR: I can't read text with eat in it.
% \newcommand{\eat}[1]{{\color{red}{\st{#1}}}}
% \newcommand{\eat}[1]{{\color{red}{#1}}}
\newcommand{\tocut}[1]{{\color{ruby}{#1}}}

% \newcommand{\tocut}[1]{#1}
\newcommand{\eat}[1]{}

\newcommand{\rev}[1]{{\color{olive}{#1}}}
\newcommand{\replace}[2]{\rev{#1}\eat{#2}}
\newcommand{\revR}[1]{{\color{blue}{#1}}}
\newcommand{\verifyR}[1]{{\color{coral}{#1}}}
\newcommand{\replaceR}[2]{\verifyR{#1}\eat{#2}}
\newcommand{\verify}[1]{{\color{red}{#1}}}
\newcommand{\revEB}[1]{{\color{purple}{#1}}}

\newcommand{\editnote}[2][1=]{\todo[linecolor=blue,backgroundcolor=blue!25,bordercolor=blue,#1]{#2}}
\newcommand{\mar}[1]{\todo[linecolor=navy,backgroundcolor=navy!25,bordercolor=navy]{\textbf{MAR:} #1}\xspace}
\newcommand{\ph}[1]{\todo[linecolor=coral,backgroundcolor=coral!25,bordercolor=coral]{\textbf{PH:} #1}\xspace}
\newcommand{\fy}[1]{\todo[linecolor=green,backgroundcolor=green!25,bordercolor=green]{\textbf{FY:} #1}\xspace}
\newcommand{\TODO}[2]{{\hl{\textbf{[#1]:} #2}}}

% %%%%%%%% comments -- DISABLE (submission) %%%%%%%%%%%%%%%
% \newcommand{\eat}[1]{}
% \newcommand{\tocut}[1]{#1}
% \newcommand{\rev}[1]{{#1}}
% \newcommand{\replace}[2]{#1}
% \newcommand{\replaceR}[2]{#1}
% \newcommand{\verify}[1]{#1}
% \newcommand{\revR}[1]{{#1}}
% \newcommand{\verifyR}[1]{#1}

% \renewcommand{\hl}[1]{{#1}}
% \newcommand{\NOTE}[2]{}
% \newcommand{\note}[1]{}
% \newcommand{\editnote}[2][1=]{}
% \newcommand{\nb}[1]{}
% \newcommand{\mar}[1]{}
% \newcommand{\roh}[1]{}
% \newcommand{\TODO}[2]{}





\NewDocumentCommand\iclogo{}{\raisebox{0.2\height}{\scalerel*{\includegraphics[height=5em]{images/ic-mamba-logo.png}}{50pt}}\xspace}
\newcommand{\icmamba}{IC-Mamba\iclogo}

\newcommand{\boldsubtitle}[1]{%
   \vspace{0.20cm} %
  {\large\textbf{#1}\par} % \large is approximately subtitle size
}

%% sample use: \eb{the thing you want to put down} from Emily 
%% \eb{here is the comment for you!}
%% \mar{comments}

%%
%% \BibTeX command to typeset BibTeX logo in the docs
\AtBeginDocument{%
  \providecommand\BibTeX{{%
    Bib\TeX}}}

%% Rights management information.  This information is sent to you
%% when you complete the rights form.  These commands have SAMPLE
%% values in them; it is your responsibility as an author to replace
%% the commands and values with those provided to you when you
%% complete the rights form.
%\setcopyright{none}
% \copyrightyear{2018}
% \acmYear{2018}
%\acmDOI{XXXXXXX.XXXXXXX}

\copyrightyear{2025}
\acmYear{2025}
\setcopyright{cc}
\setcctype{by}
\acmConference[WWW '25]{Proceedings of the ACM Web Conference 2025}{April 28-May 2, 2025}{Sydney, NSW, Australia}
\acmBooktitle{Proceedings of the ACM Web Conference 2025 (WWW '25), April 28-May 2, 2025, Sydney, NSW, Australia}
\acmDOI{10.1145/3696410.3714527}
\acmISBN{979-8-4007-1274-6/25/04}

%% These commands are for a PROCEEDINGS abstract or paper.
% \acmConference[Conference acronym 'XX]{Make sure to enter the correct
%   conference title from your rights confirmation emai}{June 03--05,
%   2018}{Woodstock, NY}
%%
%%  Uncomment \acmBooktitle if the title of the proceedings is different
%%  from ``Proceedings of ...''!
%%
%%\acmBooktitle{Woodstock '18: ACM Symposium on Neural Gaze Detection,
%%  June 03--05, 2018, Woodstock, NY}
% \acmISBN{978-1-4503-XXXX-X/18/06}


%%
%% Submission ID.
%% Use this when submitting an article to a sponsored event. You'll
%% receive a unique submission ID from the organizers
%% of the event, and this ID should be used as the parameter to this command.
%%\acmSubmissionID{123-A56-BU3}

%%
%% For managing citations, it is recommended to use bibliography
%% files in BibTeX format.
%%
%% You can then either use BibTeX with the ACM-Reference-Format style,
%% or BibLaTeX with the acmnumeric or acmauthoryear sytles, that include
%% support for advanced citation of software artefact from the
%% biblatex-software package, also separately available on CTAN.
%%
%% Look at the sample-*-biblatex.tex files for templates showcasing
%% the biblatex styles.
%%

%%
%% The majority of ACM publications use numbered citations and
%% references.  The command \citestyle{authoryear} switches to the
%% "author year" style.
%%
%% If you are preparing content for an event
%% sponsored by ACM SIGGRAPH, you must use the "author year" style of
%% citations and references.
%% Uncommenting
%% the next command will enable that stylfe.
%%\citestyle{acmauthoryear}


%%
%% end of the preamble, start of the body of the document source.
\begin{document}

%%
%% The "title" command has an optional parameter,
%% allowing the author to define a "short title" to be used in page headers.
%\title{IC-Mamba: Early Prediction of Misinformation Spread Through Interval-Censored Temporal Modelling}
%\title{IC-Mamba: Predicting Tomorrow's Viral Misinformation Today}
%\title{IC-Mamba: Interval-Censored State Space Modelling for Early Social Media Engagement Prediction and Narrative Detection}
%\title{IC-Mamba: Forecasting Social Media Influence through Interval-Censored State Space Modelling}

%Before it is too late
%Before it gone viral
%Predicting Engagement of Mis/Disinformation using State Space Model
%Before it is too late: Predicting Engagement with Misinformation and Disinformation using the IC-Mamba State Space Model
%\title{Before it's too late: Predicting Engagement with Misinformation and Disinformation using the IC-Mamba State Space Model}
\title{Before It's Too Late: A State Space Model for the Early Prediction of Misinformation and Disinformation Engagement}
%\title{Before it's too late: Using a State Space Model for the Early Prediction of Misinformation and Disinformation Engagement}
%\title{IC-Mamba: A State Space Model for the Early Prediction of Misinformation and Disinformation Engagement}
%\title{IC-Mamba: Interval-Censored State Space Modeling for Early Social Media Engagement Prediction}
%\title{IC-Mamba: Early Detection and Assessment of Information Operations Effectiveness }
\author{Lin Tian}
\affiliation{%
  \institution{University of Technology Sydney}
  \city{Sydney}
  \country{Australia}}
\email{lin.tian-3@uts.edu.au}

\author{Emily Booth}
\affiliation{%
  \institution{University of Technology Sydney}
  \city{Sydney}
  \country{Australia}}
\email{emily.booth@uts.edu.au}


\author{Francesco Bailo}
\affiliation{%
  \institution{The University of Sydney}
  \city{Sydney}
  \country{Australia}}
\email{francesco.bailo@sydney.edu.au}

\author{Julian Droogan}
\affiliation{%
  \institution{Macquarie University}
  \city{Sydney}
  \country{Australia}}
\email{julian.droogan@mq.edu.au}

\author{Marian-Andrei Rizoiu}
\affiliation{%
  \institution{University of Technology Sydney}
  \city{Sydney}
  \country{Australia}}
\email{marian-andrei.rizoiu@uts.edu.au}
%%
%% The "author" command and its associated commands are used to define
%% the authors and their affiliations.
%% Of note is the shared affiliation of the first two authors, and the
%% "authornote" and "authornotemark" commands
%% used to denote shared contribution to the research.

%%
%% By default, the full list of authors will be used in the page
%% headers. Often, this list is too long, and will overlap
%% other information printed in the page headers. This command allows
%% the author to define a more concise list
%% of authors' names for this purpose.
%\renewcommand{\shortauthors}{Trovato et al.}

%%
%% The abstract is a short summary of the work to be presented in the
%% article.
\begin{abstract}
%\TODO{MAR}{Abstract needs realignment with the current content.}
%Forecasting social media engagement patterns is crucial for understanding the societal impact of emerging opinions and coordinated information campaigns. 
In today’s digital age, conspiracies and information campaigns can emerge rapidly and erode social and democratic cohesion.
While recent deep learning approaches have made progress in modeling engagement through language and propagation models, they struggle with irregularly sampled data and early trajectory assessment. We present \icmamba, a novel state space model that forecasts social media engagement by modeling interval-censored data with integrated temporal embeddings. 
Our model excels at predicting engagement patterns within the crucial first 15-30 minutes of posting (RMSE 0.118-0.143), enabling rapid assessment of content reach. 
By incorporating interval-censored modeling into the state space framework, IC-Mamba captures fine-grained temporal dynamics of engagement growth, achieving a 4.72\% improvement over state-of-the-art across multiple engagement metrics (likes, shares, comments, and emojis).
Our experiments demonstrate IC-Mamba's effectiveness in forecasting both post-level dynamics and broader narrative patterns (F1 0.508-0.751 for narrative-level predictions). 
The model maintains strong predictive performance across extended time horizons, successfully forecasting opinion-level engagement up to 28 days ahead using observation windows of 3-10 days. 
These capabilities enable earlier identification of potentially problematic content, providing crucial lead time for designing and implementing countermeasures.
Code is available at: \url{https://github.com/ltian678/ic-mamba}.
An interactive dashboard demonstrating our results is available at: \url{https://ic-mamba.behavioral-ds.science/}.
\end{abstract}

\begin{CCSXML}
<ccs2012>
   <concept>
       <concept_id>10002951.10003260.10003282.10003292</concept_id>
       <concept_desc>Information systems~Social networks</concept_desc>
       <concept_significance>500</concept_significance>
       </concept>
   <concept>
       <concept_id>10010147.10010178</concept_id>
       <concept_desc>Computing methodologies~Artificial intelligence</concept_desc>
       <concept_significance>300</concept_significance>
       </concept>

 </ccs2012>
\end{CCSXML}

\ccsdesc[500]{Information systems~Social networks}
\ccsdesc[300]{Computing methodologies~Artificial intelligence}

%%
%% The code below is generated by the tool at http://dl.acm.org/ccs.cfm.
%% Please copy and paste the code instead of the example below.
%%
% \begin{CCSXML}
% <ccs2012>
%  <concept>
%   <concept_id>00000000.0000000.0000000</concept_id>
%   <concept_desc>Do Not Use This Code, Generate the Correct Terms for Your Paper</concept_desc>
%   <concept_significance>500</concept_significance>
%  </concept>
%  <concept>
%   <concept_id>00000000.00000000.00000000</concept_id>
%   <concept_desc>Do Not Use This Code, Generate the Correct Terms for Your Paper</concept_desc>
%   <concept_significance>300</concept_significance>
%  </concept>
%  <concept>
%   <concept_id>00000000.00000000.00000000</concept_id>
%   <concept_desc>Do Not Use This Code, Generate the Correct Terms for Your Paper</concept_desc>
%   <concept_significance>100</concept_significance>
%  </concept>
%  <concept>
%   <concept_id>00000000.00000000.00000000</concept_id>
%   <concept_desc>Do Not Use This Code, Generate the Correct Terms for Your Paper</concept_desc>
%   <concept_significance>100</concept_significance>
%  </concept>
% </ccs2012>
% \end{CCSXML}

% \ccsdesc[500]{Do Not Use This Code~Generate the Correct Terms for Your Paper}
% \ccsdesc[300]{Do Not Use This Code~Generate the Correct Terms for Your Paper}
% \ccsdesc{Do Not Use This Code~Generate the Correct Terms for Your Paper}
% \ccsdesc[100]{Do Not Use This Code~Generate the Correct Terms for Your Paper}

%%
%% Keywords. The author(s) should pick words that accurately describe
%% the work being presented. Separate the keywords with commas.
%IC-Mamba, State Space Model,
%Effectiveness, Mis/Disinformation,
%Engagement,
%Early prediction,
%Social Media
%Interval-censored
\keywords{State Space Model,  Early Prediction, Interval-Censored, Information Propagation, Misinformation, Disinformation, Social Engagement.}


%\received{20 February 2007}
%\received[revised]{12 March 2009}
%\received[accepted]{5 June 2009}

%%
%% This command processes the author and affiliation and title
%% information and builds the first part of the formatted document.
\maketitle


% !TeX root = ../camera.tex
\section{Introduction}


On 28 October 2017, an anonymous 4chan user made a brief yet impactful post on the platform claiming that Hillary Clinton was to be arrested in the coming days\footnote{\url{https://www.bellingcat.com/news/americas/2021/01/07/the-making-of-qanon-a-crowdsourced-conspiracy/}}.
On 6 January 2021, devotees of then-outgoing President Donald Trump stormed the United States Capitol building in an act of domestic terrorism designed to prevent President-elect Joe Biden's election victory from being confirmed. Five people died during and in the immediate aftermath of the attack, and an additional four died in the subsequent months\footnote{\url{https://www.factcheck.org/2021/11/how-many-died-as-a-result-of-capitol-riot/}}; and over 140 police officers were injured\footnote{\url{https://www.nytimes.com/2021/08/03/us/politics/capitol-riot-officers-honored.htm}}. 
Investigations by the Associated Press of the online social media profiles of over 120 of the rioters revealed high levels of adherence to the QAnon conspiracy theory that had begun just four years prior on 4chan\footnote{\url{https://apnews.com/article/us-capitol-trump-supporters-1806ea8dc15a2c04f2a68acd6b55cace}}.
This incident highlights how social media platforms can accelerate the spread of harmful content, particularly misinformation (false information shared without intent to harm) and disinformation (deliberately created and shared false information)~\citep{lazer2018science,scheufele2019science}.
%Large-scale analysis of social media cascades reveal extreme variations in engagement patterns -- while the most viral false news stories reach up to 100,000 users within hours, real-world browsing data shows that over 70\% of users never engage with such content, and a mere 0.15\% of users drive over 80\% of engagement~\citep{vosoughi2018spread, grinberg2019fake,guess2020exposure}.

Given these ongoing impacts, the question must be asked: what if we could have seen this coming? More specifically, what if it had been possible to forecast user engagement with fringe ideologies before they morph into widespread movements? 
We introduce \icmamba, a model capable of forecasting user engagement with online content. We go beyond the level of atomic posts to forecast the number of likes, shares, emoji reactions, and comments for ``emerging opinions'' -- particular worldviews supported across a series of posts.
Our framework can forecast the arrival rate of posts supporting an opinion, and forecast the engagement for each, obtaining estimates of the total level of engagement for the entire opinion.
Our analysis leverages CrowdTangle~\cite{crowdtangle} data with interval-censored engagement metrics, where observations are made at discrete time points with engagement counts recorded for each interval (as seen in \cref{fig:sample_ic_mamba}).

Recent deep learning approaches have made progress in modeling social media engagement through different architectural innovations: language models to capture coordinated posting behaviors~\citep{atanasov2019predicting,tian2021rumour,tian2022duck,tian2023metatroll} and propagation models to model information
diffusion~\citep{zannettou2019disinformation,im2020still,luceri2024unmasking,kong2023interval,Kong2021,Kong2020,Kong2020a,Zhang2019,Kong2018}. 
State space models have also demonstrated strong performance on sequential prediction tasks ~\citep{mamba,mamba2}, with their latent state representations theoretically well suited for temporal dependencies. 
However, these approaches face two key limitations when applied to mis/disinformation engagement forecasting: 
(1) they primarily focus on classification tasks rather than quantifying future temporal patterns of engagement, and 
(2) they struggle with the irregularly sampled nature of viral content.


\noindent\textbf{Our main contributions} address three research questions (RQs) at the intersection of temporal modeling and social media dynamics:

RQ1: \textit{How can we effectively model irregular temporal patterns in social media engagement?}: Through IC-Mamba's time-aware embeddings and state space model architecture, we capture the dynamics of online interactions, achieving a 4.72\% improvement over the state-of-the-art approaches. 

RQ2:\textit{Can we predict viral potential within the critical early window? }: \icmamba shows strong performance in the crucial 15-30 minute post-publication window (RMSE 0.118-0.143), while capturing both granular post-level dynamics and broader narrative patterns (F1 0.508-0.751 for narrative-level predictions).

RQ3: \textit{How can we forecast engagement with emerging opinions early? Can we improve the accuracy and confidence of these forecasts as engagement data streams in over time?}: Our experiments show the model effectively forecasts engagement dynamics early, using 3-, 7-, and 10-day windows to predict spreading patterns up to 28 days, with performance improving as more data streams in. %\cref{fig:dynamic_pred} shows \icmamba's effectiveness in capturing long-term opinion proliferation and evolution.

As a tool, \icmamba streamlines the work of human experts, enabling earlier identification of problematic content, and therefore, providing more time to design and implement countermeasures.

\begin{figure}[t]
  \centering
  \includegraphics[width=\linewidth]{images/sample_v9.pdf}
  \caption{
    Illustration of interval-censored social media engagement data. Following a post's creation at $t_0$, users perform engagement actions (view, like, comment, share, emoji) at timestamps $s_1$ through $s_8$. While individual actions occur continuously, engagement data is only collected at discrete observation points $t_j$, where each engagement vector $e_j$ captures the cumulative counts of different interaction types over intervals of length $\Delta t_j = t_{j+1} - t_j$.
  }
  \label{fig:sample_ic_mamba}
  \Description{social engagement matters sample}
\end{figure}

% !TeX root = ../camera.tex
 
 \begin{table}[t]
     \caption{
         Information used in related studies and our work (\icmamba). V: views, S: shares/retweets,
         C: comments, L: likes, E: emojis, WB: Weibo, YT: Youtube, FB: Facebook.
     }
     \label{tab:related_work}
     %\begin{adjustbox}{max width=\linewidth}
     \begin{tabular}{lcccccc}
         \toprule
                                               & V          & S          & C          & L          & E          & platform  \\ \toprule
         DeepCas~\citep{li2017deepcas}         &            & \checkmark &            &            &            & X         \\
         DeepHawkes~\citep{cao2017deephawkes}  &            & \checkmark &            &            &            & X         \\
         Topo-LSTM~\citep{wang2017topological} &            & \checkmark &            &            &            & X         \\
         HIP ~\citep{rizoiu2017expecting}      & \checkmark & \checkmark &            &            &            & YT        \\
         DeepInf~\citep{qiu2018deepinf}        &            & \checkmark &            & \checkmark &            & WB and X  \\
         B-Views~\cite{wu2018beyond}           & \checkmark &            &            &            &            & YT        \\
         SNPP ~\citep{ding2019social}          &            &            &            & \checkmark &            & X         \\
         CasFlow~\citep{xu2021casflow}         &            & \checkmark &            &            &            & WB        \\
         %DUMMY~\cite{wu2018beyond} &\checkmark & & & & & Youtube \\
         MBPP~\cite{rizoiu2022interval}        & \checkmark & \checkmark &            &            &            & YT        \\
         %PCMHP~\citep{xu2021casflow} & &\checkmark & & & &WB \\
         IC-TH~\cite{kong2023interval}         &            & \checkmark &            &            &            & X         \\
         OMM~\citep{calderon2024opinion}       &            & \checkmark &            &            &            & X,YT\& FB \\
         %Hawkes Mixture Model~\citep{xu2021casflow} & &\checkmark & & & &WB \\
         \midrule
         \icmamba                              &            & \checkmark & \checkmark & \checkmark & \checkmark & FB        \\
         \bottomrule
     \end{tabular}
     %\end{adjustbox}
 \end{table}
\section{Related Work}
This section reviews relevant literature in two key areas that underpin our approach to modeling and predicting social engagements during outbreak events such as information operations and natural disasters like the infamous 2019-2020 Australian Bushfires:
popularity and engagement prediction on social media platforms, and
state space models for sequence modeling and prediction.


%Comment out this part for now
%\subsection{Social Engagement Analysis}
%Social media platforms have become vital channels for disseminating information and engaging the public during crises, sparking significant research into online behavior patterns. 
%\citet{imran2015processing} conducted a thorough survey on social media use in emergencies, highlighting distinct patterns in information sharing and user interaction across various crisis types. 
%Building on this, \citet{stieglitz2018social} proposed a framework to track engagement metrics during crises, identifying key information sources and mapping the spread of crisis-related content.
%
%In the context of natural disasters, \citet{kryvasheyeu2016rapid} examined social media activity during Hurricane Sandy, uncovering correlations between public sentiment, engagement levels, and the hurricane's impact. Advances in predictive modeling have further expanded the field. For example, \citet{yin2012using} applied machine learning techniques to social media data, demonstrating their utility for enhancing situational awareness during emergencies. Similarly, \citet{cinelli2020covid} analyzed the dissemination of COVID-19 content across social platforms, offering valuable insights into information spread during public health crises.
%
%The challenge of differentiating organic engagement from manipulated activity has also gained prominence. 
%\citet{starbird2019disinformation} explored the proliferation of disinformation during crises, using network and content analyses to identify campaigns exploiting these events. 
%Their findings highlighted the intricate interplay between authentic public discourse and strategic information operations in high-stakes scenarios.

%The reviewed studies underscore social media engagement as a critical lens for understanding crisis responses, spanning natural disasters, pandemics, and disinformation campaigns. 
%However, they primarily treat engagement as an external signal without modeling its temporally dynamic and complex nature during outbreak events. 
%This paper advances online engagement modeling and prediction by addressing key challenges, including handling irregular time intervals, capturing both granular and aggregate trends, and adapting to rapidly evolving social landscapes.



\subsection{Popularity and Engagement Prediction}
Social media engagement prediction research spans various platforms and prediction tasks. 
DeepCas~\citep{li2017deepcas} used random walk and attention mechanisms to predict final cascade size on X, while SNPP~\citep{ding2019social} applied temporal point process with a gated recurrent unit architecture for tweet repost count prediction.
Topo-LSTM~\citep{wang2017topological} incorporated user interaction sequences through a topological structure for retweet prediction, while DeepInf~\citep{qiu2018deepinf} on Weibo and X predicted user retweets, likes, and following behaviors using graph convolutional networks within the 2-hop neighborhood. 
CasFlow~\citep{xu2021casflow} used hierarchical attention networks for modeling Weibo reposts.
Several approaches have focused on handling temporal dynamics in engagement prediction. DeepHawkes~\citep{cao2017deephawkes} integrated reinforcement learning with Hawkes processes for retweet cascade prediction.
For YouTube, HIP~\citep{rizoiu2017expecting} and MBPP~\citep{rizoiu2022interval,Calderon2025} advanced temporal modeling for views prediction, with B-Views~\citep{wu2018beyond} specifically addressing cold-start scenarios.
Recently, IC-TH~\citep{kong2023interval} tackled the challenge of incomplete observations in retweet prediction on X, while OMM~\citep{calderon2024opinion} and BMH~\cite{Calderon2024b} proposed a mathematical framework for shares prediction on X, YouTube, and Facebook.
While these approaches have advanced cascade modeling, existing interval-censored methods (IC-TH, MBPP) focus on single-post dynamics without considering broader opinion-level patterns. Our work extends beyond individual post predictions to model collective opinion engagement on Facebook, and we include these interval-censored models along with neural approaches (TH~\cite{zuo2020transformer}) as baselines.

%comment out this part for now.
%\subsection{Information Operation}
%Information operations on social media platforms have become a significant concern for researchers and policymakers alike. \citet{bradshaw2018global} conducted a study on the global organisation of social media manipulation, revealing the widespread use of computational propaganda by governments and political parties. It highlighted the scale and sophistication of modern information operations.
%The detection of information operations has been a key focus area. \citet{varol2017online} developed machine learning techniques to identify social bots.
%In the context of political influence, \citet{allcott2017social} analysed the impact of fake news during the 2016 U.S. presidential election, quantifying its reach and potential effects on voter behaviour. It showed the need for robust methodologies to measure the real-world impact of online disinformation campaigns.
%Recent advancements have focused on cross-platform analysis of information operations. \citet{zannettou2018origins} traced the spread of memes across multiple social media platforms, revealing how coordinated campaigns exploit different platform characteristics to maximise their reach and impact.

%While existing studies have made progress in detecting and analysing such operations, they often fall short in forecasting their potential reach and impact. The ability to accurately predict the influence of information operations on social media is crucial for developing proactive mitigation strategies and understanding their evolving dynamics. 


\subsection{State Space Model in Sequence Modeling}
State Space Models (SSMs) have recently emerged as a robust alternative to traditional sequence modeling approaches, particularly for long-range dependency capture~\citep{hasani2021liquid,rangapuram2018deep}. Subsequent work has showed their efficiency in processing extremely long sequences~\citep{gu2022efficiently,dao2022flashattention} and competitive performance in language modeling~\citep{mamba,mamba2}.
Despite these advancements, few studies have applied modern SSM architectures to predict or analyze social media engagement, particularly in the context of disinformation campaigns or crisis events. 
Most existing methods rely on graph-based~\citep{lu2023continuous}, RNN~\citep{wang2017cascade}, or transformer approaches~\citep{zuo2020transformer} that typically assume uniform sampling or discret snapshots. Such assumptions often overlook fine-grained temporal patterns crucial to disinformation campaigns or crisis events.
In contrast, modern SSMs can naturally handle non-uniform intervals and continuous-time dynamics, making them well-suited for rapidly unfolding social media processes. 
Our work bridges this gap by extending the Mamba architecture to handle non-uniform intervals, while identifying misinformation opinions and disinformation narratives.
%This paper introduces \icmamba, an online attention modeling and prediction tool that extends the Mamba architecture in several ways. 
%\icmamba handles interval-censored data with non-uniform intervals, identifies misinformation opinions and disinformation narratives.







\section{Model}\label{sec:model}

\paragraph{Notation.} We associate the Borel sigma algebra with the set $[0,1]$, and denote it by $\borel$. We use $\lambda(\cdot)$ to represent the Lebesgue measure.  For any set $A$, $\Delta \left( A \right)$ denotes the set of probability measures on $A$. For every $a \in A$, we denote by $\delta_a$ the Dirac distribution  which puts all mass at $a$. Given a space of probability measures $\Delta(A)$, we refer to the topology induced by the weak convergence on that space as the ``weak topology''. Unless otherwise specified, product spaces are endowed with the product sigma algebra and the product topology. For a cumulative distribution function (CDF) $F:[0,1] \to [0,1]$, we use $F^-$ to denote its generalized inverse, i.e., $F^-(t) \coloneqq \inf\{x \mid F(x) \geq t\}$. For $A,B \subset \mathbb{R}$, we say that $A \preccurlyeq B$ if and only if $\sup A \leq \inf B$. We use $\{f(X) \mid X \sim \mathcal D\}$ to denote the distribution of $f(X)$ when $X \sim \mathcal D$.

Consider a buyer participating in a sealed-bid first-price auction for a single indivisible good. Let $v$ denote the value the buyer derives from winning the good. We assume that $v \in [0,1]$ and it is distributed according to the distribution $F \in \Delta([0,1])$; we use $F$ to denote both the CDF and the measure it defines. Given a value ${v \sim F}$, the buyer selects a (potentially random) bid $b \sim s(v)$, where ${s: [0,1] \longrightarrow \Delta([0,1])}$ is the buyer's bidding strategy that maps value $v$ to the bid distribution $s(v)$. For the sake of completeness and rigor, we only consider bidding strategies for which there exists a Markov kernel\footnote{$\kappa: \borel \times [0,1] \to [0,1]$ is called a \emph{Markov kernel} if (i) for every fixed $B \in \borel$, $v \mapsto \kappa(B,v)$ is measurable; (ii) for every fixed $v \in [0,1]$, $\kappa(\cdot, v)$ is a probability measure on $([0,1], \borel)$.} ${\kappa_s:\borel \times [0,1] \to [0,1]}$ such that $\Prob_{b \sim s(v)}(B) \coloneqq \kappa_s(B, v)$ for all $B \in \borel$; let $\Scal$ denote the set of all such bidding strategies.
We use $J_{s,F}$ to denote the joint distribution of value-bid pairs $(v,b)$ under the strategy $s$, i.e., ${J_{s,F}(A \times B) = \E_{v \sim F}[\kappa_s(B,v) \cdot \mathbbm{1}(v \in A)]}$ for all $A,B \in \borel$. Additionally, we define $P_{s,F} \in \Delta([0,1])$ to be the bid distribution induced by the bidding strategy~$s$ and the value distribution $F$, i.e., $P_{s,F}(B) = \E_{v\sim F}[\kappa_s(B, v)]$ for all $B \in \borel$.

Simultaneously, the competitors submit their own bids $\{b_i\}_i \subset [0,1]$. We use $h \coloneqq \max_i b_i$ to denote the \emph{highest competing bid}, and $H \in \Delta([0,1])$ to denote its distribution (represented by its cumulative distribution function). We assume that $H$ is independent of the value-bid joint distribution $J_{s,F}$. In sealed-bid auctions, the independence follows directly from the independent-private-values assumption that prevails in much of the work on first-price auctions (e.g., see \citealt{krishna2009auction, milgrom2004putting, balseiro2023contextual, feng2021convergence}). In practice, it holds in scenarios where correlation in values is caused by a publicly-observable context, which yields independence upon conditioning, i.e., the values are independent for any fixed context. For example, in online advertising, buyers' values depend on  user-specific features, which are communicated to the buyers (or their autobidders) before bids are solicited, and act as the context for the auction. As buyers can specify different bids for different user segments, via separate campaigns if necessary, their values are often independent for each segment. Crucially, it allows us to endow the buyer with independent private information, which can be used to determine her bid but cannot be exploited by others. The ``amount'' of such information turns out to have a significant impact on performance.

Once the bids are submitted, the allocation and payment are decided according to a first-price auctions: the good is allocated to the buyer if and only if she is the highest bidder, i.e., $b \geq h$, in which case the buyer pays the auctioneer her bid $b$. If the product is not allocated to the buyer, i.e., $b < h$, the buyer does not pay anything. We posit that the buyer has a quasi-linear utility, i.e, for value $v$, an associated bid $b$, and a highest competing bid $h$, the utility of the buyer is
\begin{equation*}
    u(b,h ; v) := \left( v - b \right) \cdot \mathbbm{1} \left\{ b \geq h \right\}.
\end{equation*}

Given a bidding strategy $s$, which maps each value $v \in [0,1]$ to a distribution of bids $s(v) \in \Delta([0,1])$, and a distribution of highest competing bids $H$, the expected utility of the buyer is
\begin{equation*}
\U[F]{s, H}\ \coloneqq\   \mathbb{E}_{(v, h) \sim F \times H}  \left[ \mathbb{E}_{b \sim s(v)} \left[ u(b,h; v) \right] \right]\,.
\end{equation*}
We abuse notation slightly and use $\U[F]{s, h}$ to denote $\U[F]{s,\delta_h}$.


\noindent \textbf{Objective.} The buyer aims to select a strategy $s \in \Scal$ which maximizes her expected utility $\U[F]{s, H}$. However, as is often the case in practice, the buyer does not know the distribution of the highest competing bids $H$. In light of this uncertainty about $H$, it is natural to design bidding strategies that guarantee strong performance simultaneously against \emph{all} potential highest bid distributions $H \in \Delta([0,1])$, which is our aim in this work. This motivates us to measure the performance of a bidding strategy using regret, which is defined as
\begin{equation}
    \label{eq:regret}
    \reg_F(s,H)\ \coloneqq\  \sup_{s' \in \Scal}\ \U[F]{s',H} - \U[F]{s,H} \,.
\end{equation}

The regret $\reg_F(s, H)$ quantifies the sub-optimality of employing a given bidding strategy $s$ against the highest competing bid distribution $H$. It is defined as the difference between the utility achieved by an oracle, who knows the distribution $H$ and selects the optimal bidding strategy, and the utility obtained by our chosen bidding strategy. As the distribution $H$ is unknown and unavailable while designing $s$, we take the robust-optimization approach and aim to minimize this sub-optimality uniformly over all highest competing bid distributions $H$, i.e., we aim to minimize the \emph{worst-case regret} $\mathrm{WReg}_F(s) \coloneqq \sup_H \reg_F(s,H)$. Formally, our goal is to characterize bidding strategies which minimize worst-case regret:
\begin{equation}
\label{eq:minimax_strategy_problem}
  \inf_{s \in \Scal}\ \mathrm{WReg}_F(s)\ =\  \inf_{s \in \Scal}\ \sup_{H \in \Delta([0,1])} \reg_F(s, H)\,.
\end{equation}
When the problem \eqref{eq:minimax_strategy_problem} admits a minimizer $s^*$, we refer to it as a \emph{minimax-optimal bidding strategy}. 


We conclude with a brief discussion of the model. First, note that our definition of utility assumes ties are broken in favor of the buyer under consideration. We make this choice purely for notational convenience and it is without loss of generality: the minimax-optimal bidding strategies we design under this assumption continue to be minimax optimal for all possible tie-breaking rules, as we show in \Cref{appendix:tie-breaking}. Intuitively, this is because our minimax-optimal strategies induce absolutely continuous bid distributions, thereby making ties a zero-probability event.

Next, note that our definition of regret in \eqref{eq:regret} is at the ex-ante stage, i.e., regret is measured in expectation over the private value of the buyer. This choice is motivated by online display advertising, where a buyer (advertiser) typically participates in thousands of first-price auctions as a part of their ad campaign. Therefore, standard concentration arguments apply, and any strategy which does well in expectation ends up performing well cumulatively across the large number of auctions. Similar reasoning has motivated prior works on budget management in auctions to use expected utility as the objective and study budget constraints which hold in expectation. For example, see \citet{gummadi2012repeated, abhishek2013optimal, balseiro2021budget, balseiro2023contextual} for models of single-shot auctions with in-expectation constraints and objectives. Thus, bidding strategies with low worst-case ex-ante regret would yield good performance over the entire campaign. Especially in uncertain market conditions and high-volatility periods that make it impossible to use machine-learning techniques to learn good strategies, either due to a dearth of data or rapid changes in the market that render past data obsolete. Our work offers a robust alternative to learning-based methods: minimax-optimal bidding strategies come with regret guarantees which hold regardless of how the market behaves, and hold for each auction individually.

Furthermore, our model treats the value distribution $F$ as a parameter and allows it to take arbitrary values. One particular value it can take is $\delta_v$, which corresponds to the case where the value is deterministic and equal to $v$. Thus, our model can also capture the interim regret minimization problem, where the value $v$ of the buyer is fixed and known, and she wishes to minimize worst-case regret over all possible highest-competing bid distributions. In other words, our model is more general that the one which measures regret at the interim stage. This added generality is crucial because it endows the buyer with independent private information that can be used by the buyer but not the competition. Such information is common in real-life auctions due to idiosyncratic preferences of the participants. The amount of variation in the value distribution is a measure of the ``amount'' of private information, and understanding its impact on regret is one of our primary contributions. However, the more general definition of regret comes with a cost: it makes our analysis significantly more challenging because our problem is parameterized by an infinite-dimensional value distribution $F \in \Delta([0,1])$, instead of the one-dimensional  value $v$.







% !TeX root = ../camera.tex

% !TeX root = ../camera.tex

\begin{figure*}[t]
    \centering
    \newcommand\myheight{0.138}
    \subfloat[]{
        \includegraphics[height=\myheight\textheight]{images/eccdf_social_media_all_2021_v7.pdf}
        \label{fig:data_insights_DiN}
    }%
    \subfloat[]{
        \includegraphics[height=\myheight\textheight]{images/eccdf_climate_change_v5.pdf}
        \label{fig:data_insights_cc}
    }%
    \subfloat[]{
        \includegraphics[height=\myheight\textheight]{images/eccdf_social_media_comments_2021_v6.pdf}
        \label{fig:data_insights_comments}
    }%
    \caption{ Engagement distribution patterns across social media content. (a) Log-scale ECCDF of engagement metrics for the DiN dataset. (b) Log-scale ECCDF of engagement metrics from the climate change theme in SocialSense. (c) Temporal evolution of comment distributions across different time windows ranging from 1 hour to 7 days. Note: ECCDF represents Empirical Complementary Cumulative Distribution Functions.}
    \label{fig:data_insights}
\end{figure*}
%event, #opinions, users, posts, 
\section{Experiments and Results}
In this section, we present the experimental setup and the results we obtain; 
including datasets and data insights (\cref{subsec:datasets}), 
the baseline models we compare against (\cref{subsec:baselines-xp-setup}), and 
the results that address our research questions (\cref{subsec:results}).
\begin{table}[t]
    \caption{Dataset Statistics}
    % \begin{tabular}{lrrr}
    %     \toprule
    %     \textbf{Dataset} & \textbf{\#Opinions} & \textbf{\#users} & \textbf{\#posts} \\ \midrule
    %     Bushfire       & 15                  & 13,438           & 78,030           \\ 
    %     Climate Change & 24                  & 25,850           & 138,278          \\ 
    %     Vaccination    & 27                  & 34,652           & 178,894          \\ 
    %     COVID-19          & 17                  & 67,727           & 640,100          \\ \midrule
    %     DiN            & 9                   & 41               & 746,653          \\ 
    %     \bottomrule
    % \end{tabular}
    \setlength{\tabcolsep}{3pt}
    \centering
    \begin{tabular}{@{}lrrrr|r@{}}
        \toprule
        \textbf{Dataset} & \multicolumn{1}{c}{Bushfire} & \multicolumn{1}{c}{Climate Ch.} & \multicolumn{1}{c}{Vaccin.} & \multicolumn{1}{c|}{COVID-19} & \multicolumn{1}{c}{DiN} \\ \midrule
        \textbf{\#posts}                     & 78,030                       & 138,278                            & 178,894                         & 640,100                       & 746,653                 \\
        \textbf{\#users}                     & 13,438                       & 25,850                             & 34,652                          & 67,727                        & 41                      \\
        \textbf{\#opinions}                  & 15                           & 24                                 & 27                              & 17                            & 9                       \\ \bottomrule
    \end{tabular}
    \label{tab:data_statistics}
\end{table}
\subsection{Datasets}
\label{subsec:datasets}



%% MAR: moved here for figure placement


\paragraph{Datasets}
Our experiments use two Facebook datasets: the theme-focused SocialSense dataset~\citep{kong2022slipping} and the user-centric Disinformation Network (DiN) dataset.
For each post in our datasets, we collect historical engagement metrics (likes, shares, comments, emoji reactions) collected via CrowdTangle API\footnote{\url{https://www.crowdtangle.com/} before its termination in August 2024.}. 
\emph{SocialSense} contains posts and comments from four main themes during 2019-2021(see \cref{tab:data_statistics}) that attracted significant volumes of misinformation and conspiratorial discussions. 
The \emph{DiN dataset} comprises posts from $41$ accounts (2019-2024). Social science experts systematically analyzed and assigned narrative labels to these posts through comprehensive content evaluation to detect suspected coordinated information operations. 
The two datasets capture the dynamics of misinformation across diverse real-world events (SocialSense) and disinformation narrative spread by information operation networks (DiN).~\footnote{Note that, posts with fewer than four engagement intervals were excluded from model evaluation to ensure sufficient temporal depth. }
%Our initial analysis covered the complete dataset, while experimental results reflect posts with $\geq$4 intervals.
% : SocialSense enables the evaluation of engagement prediction across diverse real-world events, DiN facilitates the assessment of narrative classification within information operation networks.

\paragraph{Data Insights}
\cref{fig:data_insights}(a) and (b) present the Empirical Complementary Cumulative Distribution Functions (ECCDFs) for likes, shares, comments, and emoji reactions across DiN (a) and the Climate Change theme in SocialSense (b). The survival probability $P(X \geq k)$ measures the likelihood of achieving at least $k$ engagements \citep{clauset2009power}, and the power-law exponent $\alpha$ characterizes the decay rate \citep{newman2005power}. While Climate Change content rarely exceeds $10^4$ total engagements, DiN reaches $10^6$, indicating significantly broader reach.
In the low-engagement regime ($1 \leq k \leq 10$), DiN exhibits a higher survival probability ($\alpha \approx 2.1$) compared to Climate Change ($\alpha \approx 2.4$), suggesting stronger early visibility potential. The mid-range ($10 \leq k \leq 1000$) shows uniform decay across engagement types for Climate Change, reflecting organic interaction patterns. In contrast, DiN reveals marked stratification, especially in likes. Beyond $k > 1000$, Climate Change content plateaus near $10^3$ engagements, aligning with established social network theory regarding human-scale constraints -- approximately 150 stable connections, known as Dunbar’s number \citep{dunbar1992neocortex} -- while DiN content transcends these natural limits, reaching $10^6$ engagements.

\cref{fig:data_insights}(c) offers examines comment distributions over time windows ranging from one hour to seven days. The scale-invariant, power-law structure persists across all observation periods, though longer windows ($3$–$7$ days) exhibit slightly elevated survival probabilities beyond $10^3$. This self-similar temporal behavior distinguishes naturally diffusing, high-visibility content from artificially amplified patterns, underscoring the unique viral longevity of DiN.

%Engagement probabilities follow a consistent hierarchy (likes > emoji reactions > shares > comments) across datasets, with DiN (a) exhibiting larger inter-engagement gaps, especially in high-engagement regions (>100).
%\cref{fig:data_insights}(a) and (b) present the log-log scale Empirical Complementary Cumulative Distribution Functions (ECCDF) for four types of social media engagement: likes, shares, comments, and emoji reactions for DiN (a) and the Climate Change theme in SocialSense (b).
%(c) displays the temporal evolution of comment distributions across multiple time windows.
%We define survival probability $P(X \geq k)$ as the probability that content receives at least $k$ engagements \citep{clauset2009power}, with power-law exponent $\alpha$ characterizing the decay rate \citep{newman2005power}. 
%DiN content achieves engagements up to $10^6$, two orders higher than climate change content ($10^4$). In the low engagement region ($k \in [1,10]$), DiN shows higher survival probability ($\alpha \approx 2.1$ vs $\alpha \approx 2.4$), indicating greater potential for visibility growth. The mid-range ($k \in [10,1000]$) reveals distinct patterns: climate change content exhibits uniform decay across engagement types suggesting organic interactions, while DiN content shows stratification with anomalously high like probabilities. 
%In the viral region ($k > 1000$), climate change content respects natural network limits at $10^3$ comments, aligning with established social network theory regarding the constraints of human social networks -- 150 connections, known as Dunbar's number\cite{dunbar1992neocortex} -- while DiN content extends to $10^6$ engagements.

% !TeX root = ../camera.tex

\begin{table*}[tb]
    \caption{
        Post-level engagement prediction performance of \icmamba vs baselines on SocialSense (four themes) and DiN; 
        measured using RMSE and MAPE (lower is better), and $R^2$ (higher is better).
        Best performance in boldface.
    }
    \centering
        \begin{adjustbox}{max width=1.0\linewidth}
            \begin{tabular}{lc@{\;\;}c@{\;\;}cc@{\;\;}c@{\;\;}cc@{\;\;}c@{\;\;}cc@{\;\;}c@{\;\;}cc@{\;\;}c@{\;\;}c}
                \toprule
                \toprule
                \multirow{2}{*}{Model}                    & \multicolumn{3}{c}{Bushfire} & \multicolumn{3}{c}{Climate Change} & \multicolumn{3}{c}{Vaccination} & \multicolumn{3}{c}{COVID-19} & \multicolumn{3}{c}{DiN}                                                                                                                                                                           \\
                \cmidrule(l{.75em}r{.75em}){2-4}\cmidrule(l{.75em}r{.75em}){5-7}\cmidrule(l{.75em}r{.75em}){8-10}\cmidrule(l{.75em}r{.75em}){11-13} \cmidrule(l{.75em}r{.75em}){14-16}
                                                          & RMSE                         & MAPE                               & $R^2$                           & RMSE                      & MAPE                    & $R^2$          & RMSE           & MAPE           & $R^2$          & RMSE           & MAPE           & $R^2$          & RMSE           & MAPE           & $R^2$          \\ \toprule
                TSTransformer~\cite{vaswani2017attention} & 0.185                        & 0.232                              & 0.651                           & 0.192                     & 0.241                   & 0.643          & 0.180          & 0.226          & 0.658          & 0.188          & 0.236          & 0.647          & 0.221          & 0.276          & 0.568          \\
                Informer~\cite{zhou2021informer}          & 0.172                        & 0.216                              & 0.678                           & 0.179                     & 0.224                   & 0.670          & 0.167          & 0.210          & 0.685          & 0.175          & 0.220          & 0.674          & 0.206          & 0.258          & 0.598          \\
                Autoformer~\cite{wu2021autoformer}        & 0.163                        & 0.204                              & 0.697                           & 0.169                     & 0.212                   & 0.689          & 0.158          & 0.198          & 0.704          & 0.166          & 0.208          & 0.693          & 0.195          & 0.244          & 0.619          \\

                \midrule
                MBPP~\cite{rizoiu2022interval}             & 0.183                        & 0.229                              & 0.655                           & 0.192                     & 0.241                   & 0.643          & 0.181          & 0.227          & 0.656          & 0.189          & 0.237          & 0.645          & 0.222          & 0.278          & 0.566          \\
                IC-TH~\cite{kong2023interval}             & 0.156                        & 0.195                              & 0.712                           & 0.162                     & 0.203                   & 0.704          & 0.151          & 0.189          & 0.719          & 0.159          & 0.199          & 0.708          & 0.187          & 0.234          & 0.636          \\
                \midrule
                TH~\cite{zuo2020transformer}              & 0.149                        & 0.187                              & 0.726                           & 0.155                     & 0.194                   & 0.718          & 0.144          & 0.181          & 0.733          & 0.152          & 0.190          & 0.722          & 0.179          & 0.224          & 0.652          \\
                TS-Mixer~\cite{chen2023tsmixer}           & 0.155                        & 0.194                              & 0.714                           & 0.161                     & 0.202                   & 0.706          & 0.150          & 0.188          & 0.721          & 0.158          & 0.198          & 0.710          & 0.186          & 0.233          & 0.638          \\
                \midrule
                Mamba~\cite{mamba2}          & 0.124                        & 0.155                              & 0.776                           & 0.129                     & 0.161                   & 0.768          & 0.119          & 0.149          & 0.783          & 0.127          & 0.159          & 0.772          & 0.150          & 0.188          & 0.708          \\
                \icmamba w/o text                         & 0.123                        & 0.154                              & 0.778                           & 0.128                     & 0.160                   & 0.770          & 0.118          & 0.148          & 0.785          & 0.126          & 0.158          & 0.774          & 0.145          & 0.181          & 0.718          \\
                \icmamba w/o user                         & 0.120                        & 0.150                              & 0.785                           & 0.125                     & 0.156                   & 0.777          & 0.115          & 0.144          & 0.792          & 0.123          & 0.154          & 0.781          & 0.146          & 0.183          & 0.716          \\
                \icmamba w/o time                         & 0.121                        & 0.151                              & 0.783                           & 0.126                     & 0.157                   & 0.775          & 0.116          & 0.145          & 0.790          & 0.124          & 0.155          & 0.779          & 0.149          & 0.186          & 0.711          \\

                \icmamba                                  & \textbf{0.118}               & \textbf{0.148}                     & \textbf{0.789}                  & \textbf{0.123}            & \textbf{0.154}          & \textbf{0.781} & \textbf{0.113} & \textbf{0.142} & \textbf{0.796} & \textbf{0.121} & \textbf{0.152} & \textbf{0.785} & \textbf{0.143} & \textbf{0.179} & \textbf{0.723} \\
                \bottomrule
                \bottomrule
            \end{tabular}
        \end{adjustbox}
    \label{tab:postlevel_results}
\end{table*}

\subsection{Baselines and Experimental Setup}
\label{subsec:baselines-xp-setup}

We compare our \icmamba model against the following state-of-the-art baselines, including generative models, transformer-based architectures and state space models:
% like the Mean Behaviour Poisson (MBP) process~\cite{rizoiu2022interval}, transformer-based architectures (Informer~\cite{zhou2021informer}, Autoformer~\cite{wu2021autoformer}, Transformer-Hawkes (TH), Interval-Censored Transformer Hawkes (IC-TH)~\cite{kong2023interval}~\cite{zuo2020transformer}, and TimeSeriesTransformer~\cite{li2019enhancing}), and the state space model Mamba~\cite{dao2024transformers}:
\begin{itemize}[leftmargin=*]
    \setlength\itemsep{0em}
    \item \textbf{TimeSeriesTransformer}~\cite{li2019enhancing}\footnote{\url{https://huggingface.co/docs/transformers/en/model_doc/time_series_transformer}} is a transformer-based model specifically adapted for time series forecasting. It applies self-attention mechanisms to capture temporal dependencies.
    \item \textbf{Informer}~\cite{zhou2021informer}\footnote{\url{https://huggingface.co/docs/transformers/en/model_doc/informer}} is a long sequence time-series forecasting model that uses a ProbSparse self-attention mechanism to handle long-term dependencies.
    \item \textbf{Autoformer}~\cite{wu2021autoformer}\footnote{\url{https://huggingface.co/docs/transformers/en/model_doc/autoformer}} is a decomposition-based architecture for long-term time series forecasting. It uses an auto-correlation mechanism to identify period-based dependencies and a series decomposition architecture for trend-seasonal decomposition.
    \item \textbf{Mean Behaviour Poisson (MBP)}~\cite{rizoiu2022interval} is a generative time series model that uses a compensator function to model non-linear engagement patterns. It treats each engagement as an event in a continuous time process and optimizes post-specific parameters to model the expected cumulative engagements over time to capture the growth patterns.
    \item \textbf{Transformer-Hawkes (TH)}~\cite{zuo2020transformer} is a model that combines the transformer architecture with the Hawkes process for modeling sequential events. It uses self-attention mechanisms to capture temporal dependencies in event sequences.
    \item \textbf{Interval-Censored Transformer Hawkes (IC-TH)}~\cite{kong2023interval} is a TH extension designed to handle interval-censored data. It adapts the transformer architecture to work with event data where exact occurrence times are unknown but bounded within intervals.
    \item \textbf{TS-Mixer}~\cite{chen2023tsmixer}\footnote{\url{https://github.com/google-research/google-research/tree/master/tsmixer}} is a model that combines MLPs and transformers for time series forecasting. It uses separate mixing operations across the temporal and feature dimensions, allowing it to capture both temporal patterns and feature interactions.
    \item \textbf{Mamba}~\cite{mamba2}\footnote{\url{https://huggingface.co/docs/transformers/en/model_doc/mamba2}} is a selective state space model, it uses selective algorithms instead of attention mechanisms for sequence modeling. It can handle long-range dependencies in sequential data for time series analysis tasks.
\end{itemize}
\noindent\textbf{Experimental settings.}
We use a temporal holdout evaluation protocol across all the datasets.
We chronologically order all posts and use the earliest $70\%$ for training, the next $15\%$ for validation, and the most recent $15\%$ for testing. 
This ensures no future information leaks into training and models are evaluated on their ability to generalize to future posts.
Models are implemented using PyTorch, with hyperparameters and other settings detailed in \cref{sec:exp_settings}.



% \subsection{Engagement Prediction (RQ1) and Opinion Identification}
\subsection{Engagement Prediction--RQ1}
\label{subsec:results}

We evaluate the performance of our models with two tasks: \emph{engagement forecasting} and \emph{opinion classification}. 
For \emph{engagement prediction}, we observe the first six hours of engagement metrics for each post and forecast the overall engagement metrics (i.e.,\ at $T = \infty$). 
%
For \emph{opinion classification}, we evaluate our model's classification performance at multiple granularities.
We perform a \emph{post-level opinion classification} across the four SocialSense themes (\textit{bushfire}, \textit{climate change}, \textit{vaccination}, and \textit{COVID-19}) -- that is, we predict if a given post expresses one of the predefined opinions.
For the DiN dataset, we perform a \emph{user-level opinion classification} --
% We extend this prediction task to the user level in the DiN dataset, 
classify the presence of opinions across multiple posts from the same user.

\noindent\textbf{Post-level Engagement Prediction Performance -- RQ1}
\cref{tab:postlevel_results} reports the performance metrics using three standard measures.
We evaluate the models using RMSE to assess absolute prediction errors (crucial for high-engagement posts), MAPE for scale-independent accuracy, and $R^2$ to measure explained variance in engagement predictions.
\icmamba outperforms all baselines on every metric (RMSE, MAPE, and $R^2$) and dataset, while the original Mamba architecture ranks consistently second, confirming the effectiveness of state space models. Among transformers, IC-TH improves upon TH, and TS-Mixer outperforms both Autoformer and Informer; TSTransformer lags behind. Interestingly, the lightweight MBP model, still competes well on some events (particularly \textit{bushfire} and \textit{climate change}).
All models exhibit performance degradation on the DiN dataset, reflecting the complexity of predicting engagement in coordinated campaigns. For models supporting dynamic prediction time points, additional results for next-time and next social engagement metrics are provided in Appendix~\ref{sec:next_token_pred}.

We conduct an ablation study to understand the contribution of different components by removing text, user, and temporal features from \icmamba (\cref{tab:postlevel_results}). 
Text features demonstrate a stronger influence on SocialSense datasets, where their removal leads to a 0.005 RMSE increase, compared to a smaller 0.002 RMSE increase in the DiN dataset. This difference highlights the crucial role of textual content in organic content spread versus coordinated campaigns. Temporal features, conversely, show greater impact on the DiN dataset, where their removal results in a 0.006 RMSE increase, compared to a 0.003 RMSE increase in SocialSense datasets. This may suggest the strategic temporal patterns in coordinated disinformation campaigns. User features maintain consistent importance across both datasets, with their removal causing similar performance degradation (0.002-0.003 RMSE increase) regardless of the dataset type. Even with text features removed, \icmamba still outperforms IC-TH, improving RMSE from 0.156 to 0.123 on the Bushfire dataset, demonstrating the fundamental strength of our model's architectural design.




\noindent\textbf{Opinion-level Classification Performance --RQ1}
In our classification settings, we tackled datasets of varying complexity: the \textit{bushfire} dataset contains $9$ opinions, \textit{climate change} and \textit{vaccination} each have 12 classes, the \textit{COVID-19} dataset includes $10$ classes, and the DiN (Disinformation Narrative) dataset comprises $9$ distinct narrative labels. 
We also include a random classification baseline with an expected F1 score of $1/N$ for each dataset, where $N$ is the number of classes.
Note that we removed opinions with less than $5,000$ posts in this experimental setting.

% The classification results in \cref{tab:cls_results} show the performance variations across models and datasets. 
\cref{tab:cls_results} presents the macro-averaged F1 scores for classification across models and datasets. IC-Mamba consistently outperforms all others, achieving F1 scores between $0.69$ and $0.75$. While BERT performs well on SocialSense (F1: $0.62$–$0.68$), both models see significant drops on DiN, with IC-Mamba scoring $0.52$ and BERT falling to $0.11$. This highlights the limitations of text-only analysis for DiN, where narrative elements demand more complex temporal or contextual understanding.
Informer, Autoformer, and Mamba struggle on SocialSense (F1 < $0.41$) but perform relatively better on DiN, with Mamba achieving its best score of $0.32$. This suggests that temporal and non-textual features are critical for narrative detection, contrasting with the outbreak event focus of SocialSense.
%\cref{tab:cls_results} presents the classification F1 scores across models and datasets, macro averaged to account for class imbalance.
%\icmamba consistently outperforms all other models, achieving F1 scores ranging from $0.69$ to $0.75$. 
%BERT shows strong performance on the SocialSense dataset, with F1 scores between $0.62$ and $0.68$. However, both models show a notable drop in performance on the DiN dataset, with \icmamba achieving an F1 score of $0.52$ and BERT performing poorly at $0.11$. This shows that the DiN classification task requires more than just textual analysis. Pure sequence models may capture narrative aspects that text-only models miss. Interestingly, while Informer, Autoformer, and Mamba struggle with SocialSense dataset on the classification task (F1 scores below $0.41$), they perform relatively better on the DiN dataset, with Mamba achieving its highest score of $0.32$. This indicates that temporal and non-textual features play important roles in narrative detection tasks, compared with outbreak events in SocialSense dataset. 

\begin{table}[t]
    \caption{
        Opinion Classification results; F1 scores are reported; higher is better; best results in boldface.}
    \centering
        \begin{adjustbox}{max width=1.0\linewidth}
            \begin{tabular}{l@{\;\;}c@{\;\;}c@{\;\;}c@{\;\;}c@{\;\;}c}
                \toprule
                \toprule
                Model                              & Bushfire       & Climate        & Vaccination    & COVID-19          & DiN            \\ \toprule
                Random                             & 0.111          & 0.083          & 0.083          & 0.083          & 0.111          \\
                \midrule
                Informer~\cite{zhou2021informer}   & 0.323          & 0.291          & 0.274          & 0.299          & 0.248          \\
                Autoformer~\cite{wu2021autoformer} & 0.342          & 0.313          & 0.278          & 0.324          & 0.255          \\
                \midrule
                BERT~\cite{vaswani2017attention}   & 0.676          & 0.652          & 0.621          & 0.644          & 0.107          \\
                \midrule
                Mamba~\cite{mamba2}   & 0.412          & 0.375          & 0.363          & 0.388          & 0.316          \\
                \icmamba                           & \textbf{0.751} & \textbf{0.724} & \textbf{0.687} & \textbf{0.705} & \textbf{0.508} \\
                \bottomrule
                \bottomrule
            \end{tabular}
        \end{adjustbox}
    \label{tab:cls_results}
\end{table}
\begin{figure*}[t]
    \centering
    \newcommand\myheight{0.130}

    \subfloat[]{
        \includegraphics[height=\myheight\textheight]{images/early_prediction_line_v3.pdf}
        \label{fig:early_prediction}
    }%
    \subfloat[]{
        \includegraphics[height=\myheight\textheight]{images/7days.png}
        \label{fig:dynamic_day7}
        %% MAR: Note: I created a vectorial version "7days-vect.pdf", but it is too large and makes huge PDFs that load slowly. Using the raster version
        % I created a PNG version using the command:
        %   pdftoppm 7days.pdf 7days -png
    }%
    \subfloat[]{
        \includegraphics[height=\myheight\textheight]{images/10days.png}
        \label{fig:dynamic_day10}
        %% MAR: Note: I created a vectorial version "10days-vect.pdf", but it is too large and makes huge PDFs that load slowly. Using the raster version
    }%
    \caption{
        Comparative analysis of early prediction performance and dynamic forecasting. (a) Performance comparison on RMSE between IC-Mamba and baseline models from 15 minutes to 6 hours after posting. 
        (b)(c) \icmamba's 28-day predictions with 5-minute intervals using 7-day (b) and 10-day (c) input windows respectively.
    }
    \label{fig:dynamic_pred}
\end{figure*}
\subsection{Early Engagement Prediction--RQ2}
\label{sec:early_pred}
We vary the length of the observed period in the temporal holdout setup (see \cref{subsec:baselines-xp-setup})
to assess how well different models can forecast engagement in the critical initial hours after a post is made.
\cref{fig:early_prediction} shows RMSE-based early prediction performance for the climate change theme in SocialSense, measured at intervals from 15 minutes to 6 hours after posting, across Informer, Autoformer, TS-Mixer, IC-TH, and \icmamba.

All models demonstrate substantial improvement in prediction accuracy over time, with error rates decreasing from 15 minutes to 6 hours. The most notable improvements occur in the first hour, particularly between 15-50 minutes, suggesting that the first hour of a post's life is crucial for accurate engagement forecasting.
\icmamba outperforms other models across all time points, and its performance advantage increases over time. While all models show similar patterns of improvement in the first hour, \icmamba continues to achieve increasingly better RMSE scores through the 6-hour mark, reaching the lowest RMSE of 0.118. 
IC-TH maintains second-best performance throughout most of the timeline, followed by Autoformer. The Informer and TS-Mixer models show higher error rates, with their performance plateauing more quickly than the interval-censored approaches. This performance gap may illustrate the benefits of interval-censored modeling in engagement prediction tasks on real-world social media platforms, while the widening gap in RMSE scores over time suggests that \icmamba's improvements go beyond just interval-censored modeling, potentially indicating better long-range dependency learning.

\subsection{Dynamic Opinion-level Prediction--RQ3}
\label{sec:dynamic_pred}

This section simulates a real-world monitoring and forecasting scenario.
We analyze the opinion ``Climate change is a UN hoax'' from the SocialSense climate dataset. 
\cref{fig:dynamic_pred}(b)(c) demonstrates our dynamic prediction approach at opinion-level across multiple interaction types (likes, comments, emojis, and shares) over a 28-day period. We showcase two scenarios of initial data windows -- 168 hours (1 week), and 240 hours (10 days). 
Our model first processes the initial historical data window to establish baseline engagement patterns. As time progresses beyond these initial periods (marked by "Predictions Start" lines), the model continuously incorporates new engagement data to refine its predictions. The shaded areas around each prediction line represent the $95\%$ confidence intervals -- obtained from all previous prediction for this time -- providing a measure of prediction uncertainty over time.
We see that the uncertainty reduces as more initial data is available, suggesting that increased historical data improves the model's predictive accuracy. 



\section{Conclusion}

In this paper, we propose a sample weight averaging strategy to address variance inflation of previous independence-based sample reweighting algorithms. 
We prove its validity and benefits with theoretical analyses. 
Extensive experiments across synthetic and multiple real-world datasets demonstrate its superiority in mitigating variance inflation and improving covariate-shift generalization.  


\begin{acks}
  This research was supported by the Advanced Strategic Capabilities Accelerator (ASCA), 
  the Australian Department of Home Affairs, 
  the Defence Science and Technology Group, the Defence Innovation Network.
  the Australian Academy of Science, and 
  the National Science Centre, Poland (Project No. 2021/41/B/HS6/02798).
\end{acks}
%%
%% The next two lines define the bibliography style to be used, and
%% the bibliography file.
\bibliographystyle{ACM-Reference-Format}
\bibliography{ref}


%%
%% If your work has an appendix, this is the place to put it.
\appendix
\subsection{Lloyd-Max Algorithm}
\label{subsec:Lloyd-Max}
For a given quantization bitwidth $B$ and an operand $\bm{X}$, the Lloyd-Max algorithm finds $2^B$ quantization levels $\{\hat{x}_i\}_{i=1}^{2^B}$ such that quantizing $\bm{X}$ by rounding each scalar in $\bm{X}$ to the nearest quantization level minimizes the quantization MSE. 

The algorithm starts with an initial guess of quantization levels and then iteratively computes quantization thresholds $\{\tau_i\}_{i=1}^{2^B-1}$ and updates quantization levels $\{\hat{x}_i\}_{i=1}^{2^B}$. Specifically, at iteration $n$, thresholds are set to the midpoints of the previous iteration's levels:
\begin{align*}
    \tau_i^{(n)}=\frac{\hat{x}_i^{(n-1)}+\hat{x}_{i+1}^{(n-1)}}2 \text{ for } i=1\ldots 2^B-1
\end{align*}
Subsequently, the quantization levels are re-computed as conditional means of the data regions defined by the new thresholds:
\begin{align*}
    \hat{x}_i^{(n)}=\mathbb{E}\left[ \bm{X} \big| \bm{X}\in [\tau_{i-1}^{(n)},\tau_i^{(n)}] \right] \text{ for } i=1\ldots 2^B
\end{align*}
where to satisfy boundary conditions we have $\tau_0=-\infty$ and $\tau_{2^B}=\infty$. The algorithm iterates the above steps until convergence.

Figure \ref{fig:lm_quant} compares the quantization levels of a $7$-bit floating point (E3M3) quantizer (left) to a $7$-bit Lloyd-Max quantizer (right) when quantizing a layer of weights from the GPT3-126M model at a per-tensor granularity. As shown, the Lloyd-Max quantizer achieves substantially lower quantization MSE. Further, Table \ref{tab:FP7_vs_LM7} shows the superior perplexity achieved by Lloyd-Max quantizers for bitwidths of $7$, $6$ and $5$. The difference between the quantizers is clear at 5 bits, where per-tensor FP quantization incurs a drastic and unacceptable increase in perplexity, while Lloyd-Max quantization incurs a much smaller increase. Nevertheless, we note that even the optimal Lloyd-Max quantizer incurs a notable ($\sim 1.5$) increase in perplexity due to the coarse granularity of quantization. 

\begin{figure}[h]
  \centering
  \includegraphics[width=0.7\linewidth]{sections/figures/LM7_FP7.pdf}
  \caption{\small Quantization levels and the corresponding quantization MSE of Floating Point (left) vs Lloyd-Max (right) Quantizers for a layer of weights in the GPT3-126M model.}
  \label{fig:lm_quant}
\end{figure}

\begin{table}[h]\scriptsize
\begin{center}
\caption{\label{tab:FP7_vs_LM7} \small Comparing perplexity (lower is better) achieved by floating point quantizers and Lloyd-Max quantizers on a GPT3-126M model for the Wikitext-103 dataset.}
\begin{tabular}{c|cc|c}
\hline
 \multirow{2}{*}{\textbf{Bitwidth}} & \multicolumn{2}{|c|}{\textbf{Floating-Point Quantizer}} & \textbf{Lloyd-Max Quantizer} \\
 & Best Format & Wikitext-103 Perplexity & Wikitext-103 Perplexity \\
\hline
7 & E3M3 & 18.32 & 18.27 \\
6 & E3M2 & 19.07 & 18.51 \\
5 & E4M0 & 43.89 & 19.71 \\
\hline
\end{tabular}
\end{center}
\end{table}

\subsection{Proof of Local Optimality of LO-BCQ}
\label{subsec:lobcq_opt_proof}
For a given block $\bm{b}_j$, the quantization MSE during LO-BCQ can be empirically evaluated as $\frac{1}{L_b}\lVert \bm{b}_j- \bm{\hat{b}}_j\rVert^2_2$ where $\bm{\hat{b}}_j$ is computed from equation (\ref{eq:clustered_quantization_definition}) as $C_{f(\bm{b}_j)}(\bm{b}_j)$. Further, for a given block cluster $\mathcal{B}_i$, we compute the quantization MSE as $\frac{1}{|\mathcal{B}_{i}|}\sum_{\bm{b} \in \mathcal{B}_{i}} \frac{1}{L_b}\lVert \bm{b}- C_i^{(n)}(\bm{b})\rVert^2_2$. Therefore, at the end of iteration $n$, we evaluate the overall quantization MSE $J^{(n)}$ for a given operand $\bm{X}$ composed of $N_c$ block clusters as:
\begin{align*}
    \label{eq:mse_iter_n}
    J^{(n)} = \frac{1}{N_c} \sum_{i=1}^{N_c} \frac{1}{|\mathcal{B}_{i}^{(n)}|}\sum_{\bm{v} \in \mathcal{B}_{i}^{(n)}} \frac{1}{L_b}\lVert \bm{b}- B_i^{(n)}(\bm{b})\rVert^2_2
\end{align*}

At the end of iteration $n$, the codebooks are updated from $\mathcal{C}^{(n-1)}$ to $\mathcal{C}^{(n)}$. However, the mapping of a given vector $\bm{b}_j$ to quantizers $\mathcal{C}^{(n)}$ remains as  $f^{(n)}(\bm{b}_j)$. At the next iteration, during the vector clustering step, $f^{(n+1)}(\bm{b}_j)$ finds new mapping of $\bm{b}_j$ to updated codebooks $\mathcal{C}^{(n)}$ such that the quantization MSE over the candidate codebooks is minimized. Therefore, we obtain the following result for $\bm{b}_j$:
\begin{align*}
\frac{1}{L_b}\lVert \bm{b}_j - C_{f^{(n+1)}(\bm{b}_j)}^{(n)}(\bm{b}_j)\rVert^2_2 \le \frac{1}{L_b}\lVert \bm{b}_j - C_{f^{(n)}(\bm{b}_j)}^{(n)}(\bm{b}_j)\rVert^2_2
\end{align*}

That is, quantizing $\bm{b}_j$ at the end of the block clustering step of iteration $n+1$ results in lower quantization MSE compared to quantizing at the end of iteration $n$. Since this is true for all $\bm{b} \in \bm{X}$, we assert the following:
\begin{equation}
\begin{split}
\label{eq:mse_ineq_1}
    \tilde{J}^{(n+1)} &= \frac{1}{N_c} \sum_{i=1}^{N_c} \frac{1}{|\mathcal{B}_{i}^{(n+1)}|}\sum_{\bm{b} \in \mathcal{B}_{i}^{(n+1)}} \frac{1}{L_b}\lVert \bm{b} - C_i^{(n)}(b)\rVert^2_2 \le J^{(n)}
\end{split}
\end{equation}
where $\tilde{J}^{(n+1)}$ is the the quantization MSE after the vector clustering step at iteration $n+1$.

Next, during the codebook update step (\ref{eq:quantizers_update}) at iteration $n+1$, the per-cluster codebooks $\mathcal{C}^{(n)}$ are updated to $\mathcal{C}^{(n+1)}$ by invoking the Lloyd-Max algorithm \citep{Lloyd}. We know that for any given value distribution, the Lloyd-Max algorithm minimizes the quantization MSE. Therefore, for a given vector cluster $\mathcal{B}_i$ we obtain the following result:

\begin{equation}
    \frac{1}{|\mathcal{B}_{i}^{(n+1)}|}\sum_{\bm{b} \in \mathcal{B}_{i}^{(n+1)}} \frac{1}{L_b}\lVert \bm{b}- C_i^{(n+1)}(\bm{b})\rVert^2_2 \le \frac{1}{|\mathcal{B}_{i}^{(n+1)}|}\sum_{\bm{b} \in \mathcal{B}_{i}^{(n+1)}} \frac{1}{L_b}\lVert \bm{b}- C_i^{(n)}(\bm{b})\rVert^2_2
\end{equation}

The above equation states that quantizing the given block cluster $\mathcal{B}_i$ after updating the associated codebook from $C_i^{(n)}$ to $C_i^{(n+1)}$ results in lower quantization MSE. Since this is true for all the block clusters, we derive the following result: 
\begin{equation}
\begin{split}
\label{eq:mse_ineq_2}
     J^{(n+1)} &= \frac{1}{N_c} \sum_{i=1}^{N_c} \frac{1}{|\mathcal{B}_{i}^{(n+1)}|}\sum_{\bm{b} \in \mathcal{B}_{i}^{(n+1)}} \frac{1}{L_b}\lVert \bm{b}- C_i^{(n+1)}(\bm{b})\rVert^2_2  \le \tilde{J}^{(n+1)}   
\end{split}
\end{equation}

Following (\ref{eq:mse_ineq_1}) and (\ref{eq:mse_ineq_2}), we find that the quantization MSE is non-increasing for each iteration, that is, $J^{(1)} \ge J^{(2)} \ge J^{(3)} \ge \ldots \ge J^{(M)}$ where $M$ is the maximum number of iterations. 
%Therefore, we can say that if the algorithm converges, then it must be that it has converged to a local minimum. 
\hfill $\blacksquare$


\begin{figure}
    \begin{center}
    \includegraphics[width=0.5\textwidth]{sections//figures/mse_vs_iter.pdf}
    \end{center}
    \caption{\small NMSE vs iterations during LO-BCQ compared to other block quantization proposals}
    \label{fig:nmse_vs_iter}
\end{figure}

Figure \ref{fig:nmse_vs_iter} shows the empirical convergence of LO-BCQ across several block lengths and number of codebooks. Also, the MSE achieved by LO-BCQ is compared to baselines such as MXFP and VSQ. As shown, LO-BCQ converges to a lower MSE than the baselines. Further, we achieve better convergence for larger number of codebooks ($N_c$) and for a smaller block length ($L_b$), both of which increase the bitwidth of BCQ (see Eq \ref{eq:bitwidth_bcq}).


\subsection{Additional Accuracy Results}
%Table \ref{tab:lobcq_config} lists the various LOBCQ configurations and their corresponding bitwidths.
\begin{table}
\setlength{\tabcolsep}{4.75pt}
\begin{center}
\caption{\label{tab:lobcq_config} Various LO-BCQ configurations and their bitwidths.}
\begin{tabular}{|c||c|c|c|c||c|c||c|} 
\hline
 & \multicolumn{4}{|c||}{$L_b=8$} & \multicolumn{2}{|c||}{$L_b=4$} & $L_b=2$ \\
 \hline
 \backslashbox{$L_A$\kern-1em}{\kern-1em$N_c$} & 2 & 4 & 8 & 16 & 2 & 4 & 2 \\
 \hline
 64 & 4.25 & 4.375 & 4.5 & 4.625 & 4.375 & 4.625 & 4.625\\
 \hline
 32 & 4.375 & 4.5 & 4.625& 4.75 & 4.5 & 4.75 & 4.75 \\
 \hline
 16 & 4.625 & 4.75& 4.875 & 5 & 4.75 & 5 & 5 \\
 \hline
\end{tabular}
\end{center}
\end{table}

%\subsection{Perplexity achieved by various LO-BCQ configurations on Wikitext-103 dataset}

\begin{table} \centering
\begin{tabular}{|c||c|c|c|c||c|c||c|} 
\hline
 $L_b \rightarrow$& \multicolumn{4}{c||}{8} & \multicolumn{2}{c||}{4} & 2\\
 \hline
 \backslashbox{$L_A$\kern-1em}{\kern-1em$N_c$} & 2 & 4 & 8 & 16 & 2 & 4 & 2  \\
 %$N_c \rightarrow$ & 2 & 4 & 8 & 16 & 2 & 4 & 2 \\
 \hline
 \hline
 \multicolumn{8}{c}{GPT3-1.3B (FP32 PPL = 9.98)} \\ 
 \hline
 \hline
 64 & 10.40 & 10.23 & 10.17 & 10.15 &  10.28 & 10.18 & 10.19 \\
 \hline
 32 & 10.25 & 10.20 & 10.15 & 10.12 &  10.23 & 10.17 & 10.17 \\
 \hline
 16 & 10.22 & 10.16 & 10.10 & 10.09 &  10.21 & 10.14 & 10.16 \\
 \hline
  \hline
 \multicolumn{8}{c}{GPT3-8B (FP32 PPL = 7.38)} \\ 
 \hline
 \hline
 64 & 7.61 & 7.52 & 7.48 &  7.47 &  7.55 &  7.49 & 7.50 \\
 \hline
 32 & 7.52 & 7.50 & 7.46 &  7.45 &  7.52 &  7.48 & 7.48  \\
 \hline
 16 & 7.51 & 7.48 & 7.44 &  7.44 &  7.51 &  7.49 & 7.47  \\
 \hline
\end{tabular}
\caption{\label{tab:ppl_gpt3_abalation} Wikitext-103 perplexity across GPT3-1.3B and 8B models.}
\end{table}

\begin{table} \centering
\begin{tabular}{|c||c|c|c|c||} 
\hline
 $L_b \rightarrow$& \multicolumn{4}{c||}{8}\\
 \hline
 \backslashbox{$L_A$\kern-1em}{\kern-1em$N_c$} & 2 & 4 & 8 & 16 \\
 %$N_c \rightarrow$ & 2 & 4 & 8 & 16 & 2 & 4 & 2 \\
 \hline
 \hline
 \multicolumn{5}{|c|}{Llama2-7B (FP32 PPL = 5.06)} \\ 
 \hline
 \hline
 64 & 5.31 & 5.26 & 5.19 & 5.18  \\
 \hline
 32 & 5.23 & 5.25 & 5.18 & 5.15  \\
 \hline
 16 & 5.23 & 5.19 & 5.16 & 5.14  \\
 \hline
 \multicolumn{5}{|c|}{Nemotron4-15B (FP32 PPL = 5.87)} \\ 
 \hline
 \hline
 64  & 6.3 & 6.20 & 6.13 & 6.08  \\
 \hline
 32  & 6.24 & 6.12 & 6.07 & 6.03  \\
 \hline
 16  & 6.12 & 6.14 & 6.04 & 6.02  \\
 \hline
 \multicolumn{5}{|c|}{Nemotron4-340B (FP32 PPL = 3.48)} \\ 
 \hline
 \hline
 64 & 3.67 & 3.62 & 3.60 & 3.59 \\
 \hline
 32 & 3.63 & 3.61 & 3.59 & 3.56 \\
 \hline
 16 & 3.61 & 3.58 & 3.57 & 3.55 \\
 \hline
\end{tabular}
\caption{\label{tab:ppl_llama7B_nemo15B} Wikitext-103 perplexity compared to FP32 baseline in Llama2-7B and Nemotron4-15B, 340B models}
\end{table}

%\subsection{Perplexity achieved by various LO-BCQ configurations on MMLU dataset}


\begin{table} \centering
\begin{tabular}{|c||c|c|c|c||c|c|c|c|} 
\hline
 $L_b \rightarrow$& \multicolumn{4}{c||}{8} & \multicolumn{4}{c||}{8}\\
 \hline
 \backslashbox{$L_A$\kern-1em}{\kern-1em$N_c$} & 2 & 4 & 8 & 16 & 2 & 4 & 8 & 16  \\
 %$N_c \rightarrow$ & 2 & 4 & 8 & 16 & 2 & 4 & 2 \\
 \hline
 \hline
 \multicolumn{5}{|c|}{Llama2-7B (FP32 Accuracy = 45.8\%)} & \multicolumn{4}{|c|}{Llama2-70B (FP32 Accuracy = 69.12\%)} \\ 
 \hline
 \hline
 64 & 43.9 & 43.4 & 43.9 & 44.9 & 68.07 & 68.27 & 68.17 & 68.75 \\
 \hline
 32 & 44.5 & 43.8 & 44.9 & 44.5 & 68.37 & 68.51 & 68.35 & 68.27  \\
 \hline
 16 & 43.9 & 42.7 & 44.9 & 45 & 68.12 & 68.77 & 68.31 & 68.59  \\
 \hline
 \hline
 \multicolumn{5}{|c|}{GPT3-22B (FP32 Accuracy = 38.75\%)} & \multicolumn{4}{|c|}{Nemotron4-15B (FP32 Accuracy = 64.3\%)} \\ 
 \hline
 \hline
 64 & 36.71 & 38.85 & 38.13 & 38.92 & 63.17 & 62.36 & 63.72 & 64.09 \\
 \hline
 32 & 37.95 & 38.69 & 39.45 & 38.34 & 64.05 & 62.30 & 63.8 & 64.33  \\
 \hline
 16 & 38.88 & 38.80 & 38.31 & 38.92 & 63.22 & 63.51 & 63.93 & 64.43  \\
 \hline
\end{tabular}
\caption{\label{tab:mmlu_abalation} Accuracy on MMLU dataset across GPT3-22B, Llama2-7B, 70B and Nemotron4-15B models.}
\end{table}


%\subsection{Perplexity achieved by various LO-BCQ configurations on LM evaluation harness}

\begin{table} \centering
\begin{tabular}{|c||c|c|c|c||c|c|c|c|} 
\hline
 $L_b \rightarrow$& \multicolumn{4}{c||}{8} & \multicolumn{4}{c||}{8}\\
 \hline
 \backslashbox{$L_A$\kern-1em}{\kern-1em$N_c$} & 2 & 4 & 8 & 16 & 2 & 4 & 8 & 16  \\
 %$N_c \rightarrow$ & 2 & 4 & 8 & 16 & 2 & 4 & 2 \\
 \hline
 \hline
 \multicolumn{5}{|c|}{Race (FP32 Accuracy = 37.51\%)} & \multicolumn{4}{|c|}{Boolq (FP32 Accuracy = 64.62\%)} \\ 
 \hline
 \hline
 64 & 36.94 & 37.13 & 36.27 & 37.13 & 63.73 & 62.26 & 63.49 & 63.36 \\
 \hline
 32 & 37.03 & 36.36 & 36.08 & 37.03 & 62.54 & 63.51 & 63.49 & 63.55  \\
 \hline
 16 & 37.03 & 37.03 & 36.46 & 37.03 & 61.1 & 63.79 & 63.58 & 63.33  \\
 \hline
 \hline
 \multicolumn{5}{|c|}{Winogrande (FP32 Accuracy = 58.01\%)} & \multicolumn{4}{|c|}{Piqa (FP32 Accuracy = 74.21\%)} \\ 
 \hline
 \hline
 64 & 58.17 & 57.22 & 57.85 & 58.33 & 73.01 & 73.07 & 73.07 & 72.80 \\
 \hline
 32 & 59.12 & 58.09 & 57.85 & 58.41 & 73.01 & 73.94 & 72.74 & 73.18  \\
 \hline
 16 & 57.93 & 58.88 & 57.93 & 58.56 & 73.94 & 72.80 & 73.01 & 73.94  \\
 \hline
\end{tabular}
\caption{\label{tab:mmlu_abalation} Accuracy on LM evaluation harness tasks on GPT3-1.3B model.}
\end{table}

\begin{table} \centering
\begin{tabular}{|c||c|c|c|c||c|c|c|c|} 
\hline
 $L_b \rightarrow$& \multicolumn{4}{c||}{8} & \multicolumn{4}{c||}{8}\\
 \hline
 \backslashbox{$L_A$\kern-1em}{\kern-1em$N_c$} & 2 & 4 & 8 & 16 & 2 & 4 & 8 & 16  \\
 %$N_c \rightarrow$ & 2 & 4 & 8 & 16 & 2 & 4 & 2 \\
 \hline
 \hline
 \multicolumn{5}{|c|}{Race (FP32 Accuracy = 41.34\%)} & \multicolumn{4}{|c|}{Boolq (FP32 Accuracy = 68.32\%)} \\ 
 \hline
 \hline
 64 & 40.48 & 40.10 & 39.43 & 39.90 & 69.20 & 68.41 & 69.45 & 68.56 \\
 \hline
 32 & 39.52 & 39.52 & 40.77 & 39.62 & 68.32 & 67.43 & 68.17 & 69.30  \\
 \hline
 16 & 39.81 & 39.71 & 39.90 & 40.38 & 68.10 & 66.33 & 69.51 & 69.42  \\
 \hline
 \hline
 \multicolumn{5}{|c|}{Winogrande (FP32 Accuracy = 67.88\%)} & \multicolumn{4}{|c|}{Piqa (FP32 Accuracy = 78.78\%)} \\ 
 \hline
 \hline
 64 & 66.85 & 66.61 & 67.72 & 67.88 & 77.31 & 77.42 & 77.75 & 77.64 \\
 \hline
 32 & 67.25 & 67.72 & 67.72 & 67.00 & 77.31 & 77.04 & 77.80 & 77.37  \\
 \hline
 16 & 68.11 & 68.90 & 67.88 & 67.48 & 77.37 & 78.13 & 78.13 & 77.69  \\
 \hline
\end{tabular}
\caption{\label{tab:mmlu_abalation} Accuracy on LM evaluation harness tasks on GPT3-8B model.}
\end{table}

\begin{table} \centering
\begin{tabular}{|c||c|c|c|c||c|c|c|c|} 
\hline
 $L_b \rightarrow$& \multicolumn{4}{c||}{8} & \multicolumn{4}{c||}{8}\\
 \hline
 \backslashbox{$L_A$\kern-1em}{\kern-1em$N_c$} & 2 & 4 & 8 & 16 & 2 & 4 & 8 & 16  \\
 %$N_c \rightarrow$ & 2 & 4 & 8 & 16 & 2 & 4 & 2 \\
 \hline
 \hline
 \multicolumn{5}{|c|}{Race (FP32 Accuracy = 40.67\%)} & \multicolumn{4}{|c|}{Boolq (FP32 Accuracy = 76.54\%)} \\ 
 \hline
 \hline
 64 & 40.48 & 40.10 & 39.43 & 39.90 & 75.41 & 75.11 & 77.09 & 75.66 \\
 \hline
 32 & 39.52 & 39.52 & 40.77 & 39.62 & 76.02 & 76.02 & 75.96 & 75.35  \\
 \hline
 16 & 39.81 & 39.71 & 39.90 & 40.38 & 75.05 & 73.82 & 75.72 & 76.09  \\
 \hline
 \hline
 \multicolumn{5}{|c|}{Winogrande (FP32 Accuracy = 70.64\%)} & \multicolumn{4}{|c|}{Piqa (FP32 Accuracy = 79.16\%)} \\ 
 \hline
 \hline
 64 & 69.14 & 70.17 & 70.17 & 70.56 & 78.24 & 79.00 & 78.62 & 78.73 \\
 \hline
 32 & 70.96 & 69.69 & 71.27 & 69.30 & 78.56 & 79.49 & 79.16 & 78.89  \\
 \hline
 16 & 71.03 & 69.53 & 69.69 & 70.40 & 78.13 & 79.16 & 79.00 & 79.00  \\
 \hline
\end{tabular}
\caption{\label{tab:mmlu_abalation} Accuracy on LM evaluation harness tasks on GPT3-22B model.}
\end{table}

\begin{table} \centering
\begin{tabular}{|c||c|c|c|c||c|c|c|c|} 
\hline
 $L_b \rightarrow$& \multicolumn{4}{c||}{8} & \multicolumn{4}{c||}{8}\\
 \hline
 \backslashbox{$L_A$\kern-1em}{\kern-1em$N_c$} & 2 & 4 & 8 & 16 & 2 & 4 & 8 & 16  \\
 %$N_c \rightarrow$ & 2 & 4 & 8 & 16 & 2 & 4 & 2 \\
 \hline
 \hline
 \multicolumn{5}{|c|}{Race (FP32 Accuracy = 44.4\%)} & \multicolumn{4}{|c|}{Boolq (FP32 Accuracy = 79.29\%)} \\ 
 \hline
 \hline
 64 & 42.49 & 42.51 & 42.58 & 43.45 & 77.58 & 77.37 & 77.43 & 78.1 \\
 \hline
 32 & 43.35 & 42.49 & 43.64 & 43.73 & 77.86 & 75.32 & 77.28 & 77.86  \\
 \hline
 16 & 44.21 & 44.21 & 43.64 & 42.97 & 78.65 & 77 & 76.94 & 77.98  \\
 \hline
 \hline
 \multicolumn{5}{|c|}{Winogrande (FP32 Accuracy = 69.38\%)} & \multicolumn{4}{|c|}{Piqa (FP32 Accuracy = 78.07\%)} \\ 
 \hline
 \hline
 64 & 68.9 & 68.43 & 69.77 & 68.19 & 77.09 & 76.82 & 77.09 & 77.86 \\
 \hline
 32 & 69.38 & 68.51 & 68.82 & 68.90 & 78.07 & 76.71 & 78.07 & 77.86  \\
 \hline
 16 & 69.53 & 67.09 & 69.38 & 68.90 & 77.37 & 77.8 & 77.91 & 77.69  \\
 \hline
\end{tabular}
\caption{\label{tab:mmlu_abalation} Accuracy on LM evaluation harness tasks on Llama2-7B model.}
\end{table}

\begin{table} \centering
\begin{tabular}{|c||c|c|c|c||c|c|c|c|} 
\hline
 $L_b \rightarrow$& \multicolumn{4}{c||}{8} & \multicolumn{4}{c||}{8}\\
 \hline
 \backslashbox{$L_A$\kern-1em}{\kern-1em$N_c$} & 2 & 4 & 8 & 16 & 2 & 4 & 8 & 16  \\
 %$N_c \rightarrow$ & 2 & 4 & 8 & 16 & 2 & 4 & 2 \\
 \hline
 \hline
 \multicolumn{5}{|c|}{Race (FP32 Accuracy = 48.8\%)} & \multicolumn{4}{|c|}{Boolq (FP32 Accuracy = 85.23\%)} \\ 
 \hline
 \hline
 64 & 49.00 & 49.00 & 49.28 & 48.71 & 82.82 & 84.28 & 84.03 & 84.25 \\
 \hline
 32 & 49.57 & 48.52 & 48.33 & 49.28 & 83.85 & 84.46 & 84.31 & 84.93  \\
 \hline
 16 & 49.85 & 49.09 & 49.28 & 48.99 & 85.11 & 84.46 & 84.61 & 83.94  \\
 \hline
 \hline
 \multicolumn{5}{|c|}{Winogrande (FP32 Accuracy = 79.95\%)} & \multicolumn{4}{|c|}{Piqa (FP32 Accuracy = 81.56\%)} \\ 
 \hline
 \hline
 64 & 78.77 & 78.45 & 78.37 & 79.16 & 81.45 & 80.69 & 81.45 & 81.5 \\
 \hline
 32 & 78.45 & 79.01 & 78.69 & 80.66 & 81.56 & 80.58 & 81.18 & 81.34  \\
 \hline
 16 & 79.95 & 79.56 & 79.79 & 79.72 & 81.28 & 81.66 & 81.28 & 80.96  \\
 \hline
\end{tabular}
\caption{\label{tab:mmlu_abalation} Accuracy on LM evaluation harness tasks on Llama2-70B model.}
\end{table}

%\section{MSE Studies}
%\textcolor{red}{TODO}


\subsection{Number Formats and Quantization Method}
\label{subsec:numFormats_quantMethod}
\subsubsection{Integer Format}
An $n$-bit signed integer (INT) is typically represented with a 2s-complement format \citep{yao2022zeroquant,xiao2023smoothquant,dai2021vsq}, where the most significant bit denotes the sign.

\subsubsection{Floating Point Format}
An $n$-bit signed floating point (FP) number $x$ comprises of a 1-bit sign ($x_{\mathrm{sign}}$), $B_m$-bit mantissa ($x_{\mathrm{mant}}$) and $B_e$-bit exponent ($x_{\mathrm{exp}}$) such that $B_m+B_e=n-1$. The associated constant exponent bias ($E_{\mathrm{bias}}$) is computed as $(2^{{B_e}-1}-1)$. We denote this format as $E_{B_e}M_{B_m}$.  

\subsubsection{Quantization Scheme}
\label{subsec:quant_method}
A quantization scheme dictates how a given unquantized tensor is converted to its quantized representation. We consider FP formats for the purpose of illustration. Given an unquantized tensor $\bm{X}$ and an FP format $E_{B_e}M_{B_m}$, we first, we compute the quantization scale factor $s_X$ that maps the maximum absolute value of $\bm{X}$ to the maximum quantization level of the $E_{B_e}M_{B_m}$ format as follows:
\begin{align}
\label{eq:sf}
    s_X = \frac{\mathrm{max}(|\bm{X}|)}{\mathrm{max}(E_{B_e}M_{B_m})}
\end{align}
In the above equation, $|\cdot|$ denotes the absolute value function.

Next, we scale $\bm{X}$ by $s_X$ and quantize it to $\hat{\bm{X}}$ by rounding it to the nearest quantization level of $E_{B_e}M_{B_m}$ as:

\begin{align}
\label{eq:tensor_quant}
    \hat{\bm{X}} = \text{round-to-nearest}\left(\frac{\bm{X}}{s_X}, E_{B_e}M_{B_m}\right)
\end{align}

We perform dynamic max-scaled quantization \citep{wu2020integer}, where the scale factor $s$ for activations is dynamically computed during runtime.

\subsection{Vector Scaled Quantization}
\begin{wrapfigure}{r}{0.35\linewidth}
  \centering
  \includegraphics[width=\linewidth]{sections/figures/vsquant.jpg}
  \caption{\small Vectorwise decomposition for per-vector scaled quantization (VSQ \citep{dai2021vsq}).}
  \label{fig:vsquant}
\end{wrapfigure}
During VSQ \citep{dai2021vsq}, the operand tensors are decomposed into 1D vectors in a hardware friendly manner as shown in Figure \ref{fig:vsquant}. Since the decomposed tensors are used as operands in matrix multiplications during inference, it is beneficial to perform this decomposition along the reduction dimension of the multiplication. The vectorwise quantization is performed similar to tensorwise quantization described in Equations \ref{eq:sf} and \ref{eq:tensor_quant}, where a scale factor $s_v$ is required for each vector $\bm{v}$ that maps the maximum absolute value of that vector to the maximum quantization level. While smaller vector lengths can lead to larger accuracy gains, the associated memory and computational overheads due to the per-vector scale factors increases. To alleviate these overheads, VSQ \citep{dai2021vsq} proposed a second level quantization of the per-vector scale factors to unsigned integers, while MX \citep{rouhani2023shared} quantizes them to integer powers of 2 (denoted as $2^{INT}$).

\subsubsection{MX Format}
The MX format proposed in \citep{rouhani2023microscaling} introduces the concept of sub-block shifting. For every two scalar elements of $b$-bits each, there is a shared exponent bit. The value of this exponent bit is determined through an empirical analysis that targets minimizing quantization MSE. We note that the FP format $E_{1}M_{b}$ is strictly better than MX from an accuracy perspective since it allocates a dedicated exponent bit to each scalar as opposed to sharing it across two scalars. Therefore, we conservatively bound the accuracy of a $b+2$-bit signed MX format with that of a $E_{1}M_{b}$ format in our comparisons. For instance, we use E1M2 format as a proxy for MX4.

\begin{figure}
    \centering
    \includegraphics[width=1\linewidth]{sections//figures/BlockFormats.pdf}
    \caption{\small Comparing LO-BCQ to MX format.}
    \label{fig:block_formats}
\end{figure}

Figure \ref{fig:block_formats} compares our $4$-bit LO-BCQ block format to MX \citep{rouhani2023microscaling}. As shown, both LO-BCQ and MX decompose a given operand tensor into block arrays and each block array into blocks. Similar to MX, we find that per-block quantization ($L_b < L_A$) leads to better accuracy due to increased flexibility. While MX achieves this through per-block $1$-bit micro-scales, we associate a dedicated codebook to each block through a per-block codebook selector. Further, MX quantizes the per-block array scale-factor to E8M0 format without per-tensor scaling. In contrast during LO-BCQ, we find that per-tensor scaling combined with quantization of per-block array scale-factor to E4M3 format results in superior inference accuracy across models. 



\end{document}

