% !TeX root = ../camera.tex
 
 \begin{table}[t]
     \caption{
         Information used in related studies and our work (\icmamba). V: views, S: shares/retweets,
         C: comments, L: likes, E: emojis, WB: Weibo, YT: Youtube, FB: Facebook.
     }
     \label{tab:related_work}
     %\begin{adjustbox}{max width=\linewidth}
     \begin{tabular}{lcccccc}
         \toprule
                                               & V          & S          & C          & L          & E          & platform  \\ \toprule
         DeepCas~\citep{li2017deepcas}         &            & \checkmark &            &            &            & X         \\
         DeepHawkes~\citep{cao2017deephawkes}  &            & \checkmark &            &            &            & X         \\
         Topo-LSTM~\citep{wang2017topological} &            & \checkmark &            &            &            & X         \\
         HIP ~\citep{rizoiu2017expecting}      & \checkmark & \checkmark &            &            &            & YT        \\
         DeepInf~\citep{qiu2018deepinf}        &            & \checkmark &            & \checkmark &            & WB and X  \\
         B-Views~\cite{wu2018beyond}           & \checkmark &            &            &            &            & YT        \\
         SNPP ~\citep{ding2019social}          &            &            &            & \checkmark &            & X         \\
         CasFlow~\citep{xu2021casflow}         &            & \checkmark &            &            &            & WB        \\
         %DUMMY~\cite{wu2018beyond} &\checkmark & & & & & Youtube \\
         MBPP~\cite{rizoiu2022interval}        & \checkmark & \checkmark &            &            &            & YT        \\
         %PCMHP~\citep{xu2021casflow} & &\checkmark & & & &WB \\
         IC-TH~\cite{kong2023interval}         &            & \checkmark &            &            &            & X         \\
         OMM~\citep{calderon2024opinion}       &            & \checkmark &            &            &            & X,YT\& FB \\
         %Hawkes Mixture Model~\citep{xu2021casflow} & &\checkmark & & & &WB \\
         \midrule
         \icmamba                              &            & \checkmark & \checkmark & \checkmark & \checkmark & FB        \\
         \bottomrule
     \end{tabular}
     %\end{adjustbox}
 \end{table}
\section{Related Work}
This section reviews relevant literature in two key areas that underpin our approach to modeling and predicting social engagements during outbreak events such as information operations and natural disasters like the infamous 2019-2020 Australian Bushfires:
popularity and engagement prediction on social media platforms, and
state space models for sequence modeling and prediction.


%Comment out this part for now
%\subsection{Social Engagement Analysis}
%Social media platforms have become vital channels for disseminating information and engaging the public during crises, sparking significant research into online behavior patterns. 
%\citet{imran2015processing} conducted a thorough survey on social media use in emergencies, highlighting distinct patterns in information sharing and user interaction across various crisis types. 
%Building on this, \citet{stieglitz2018social} proposed a framework to track engagement metrics during crises, identifying key information sources and mapping the spread of crisis-related content.
%
%In the context of natural disasters, \citet{kryvasheyeu2016rapid} examined social media activity during Hurricane Sandy, uncovering correlations between public sentiment, engagement levels, and the hurricane's impact. Advances in predictive modeling have further expanded the field. For example, \citet{yin2012using} applied machine learning techniques to social media data, demonstrating their utility for enhancing situational awareness during emergencies. Similarly, \citet{cinelli2020covid} analyzed the dissemination of COVID-19 content across social platforms, offering valuable insights into information spread during public health crises.
%
%The challenge of differentiating organic engagement from manipulated activity has also gained prominence. 
%\citet{starbird2019disinformation} explored the proliferation of disinformation during crises, using network and content analyses to identify campaigns exploiting these events. 
%Their findings highlighted the intricate interplay between authentic public discourse and strategic information operations in high-stakes scenarios.

%The reviewed studies underscore social media engagement as a critical lens for understanding crisis responses, spanning natural disasters, pandemics, and disinformation campaigns. 
%However, they primarily treat engagement as an external signal without modeling its temporally dynamic and complex nature during outbreak events. 
%This paper advances online engagement modeling and prediction by addressing key challenges, including handling irregular time intervals, capturing both granular and aggregate trends, and adapting to rapidly evolving social landscapes.



\subsection{Popularity and Engagement Prediction}
Social media engagement prediction research spans various platforms and prediction tasks. 
DeepCas~\citep{li2017deepcas} used random walk and attention mechanisms to predict final cascade size on X, while SNPP~\citep{ding2019social} applied temporal point process with a gated recurrent unit architecture for tweet repost count prediction.
Topo-LSTM~\citep{wang2017topological} incorporated user interaction sequences through a topological structure for retweet prediction, while DeepInf~\citep{qiu2018deepinf} on Weibo and X predicted user retweets, likes, and following behaviors using graph convolutional networks within the 2-hop neighborhood. 
CasFlow~\citep{xu2021casflow} used hierarchical attention networks for modeling Weibo reposts.
Several approaches have focused on handling temporal dynamics in engagement prediction. DeepHawkes~\citep{cao2017deephawkes} integrated reinforcement learning with Hawkes processes for retweet cascade prediction.
For YouTube, HIP~\citep{rizoiu2017expecting} and MBPP~\citep{rizoiu2022interval,Calderon2025} advanced temporal modeling for views prediction, with B-Views~\citep{wu2018beyond} specifically addressing cold-start scenarios.
Recently, IC-TH~\citep{kong2023interval} tackled the challenge of incomplete observations in retweet prediction on X, while OMM~\citep{calderon2024opinion} and BMH~\cite{Calderon2024b} proposed a mathematical framework for shares prediction on X, YouTube, and Facebook.
While these approaches have advanced cascade modeling, existing interval-censored methods (IC-TH, MBPP) focus on single-post dynamics without considering broader opinion-level patterns. Our work extends beyond individual post predictions to model collective opinion engagement on Facebook, and we include these interval-censored models along with neural approaches (TH~\cite{zuo2020transformer}) as baselines.

%comment out this part for now.
%\subsection{Information Operation}
%Information operations on social media platforms have become a significant concern for researchers and policymakers alike. \citet{bradshaw2018global} conducted a study on the global organisation of social media manipulation, revealing the widespread use of computational propaganda by governments and political parties. It highlighted the scale and sophistication of modern information operations.
%The detection of information operations has been a key focus area. \citet{varol2017online} developed machine learning techniques to identify social bots.
%In the context of political influence, \citet{allcott2017social} analysed the impact of fake news during the 2016 U.S. presidential election, quantifying its reach and potential effects on voter behaviour. It showed the need for robust methodologies to measure the real-world impact of online disinformation campaigns.
%Recent advancements have focused on cross-platform analysis of information operations. \citet{zannettou2018origins} traced the spread of memes across multiple social media platforms, revealing how coordinated campaigns exploit different platform characteristics to maximise their reach and impact.

%While existing studies have made progress in detecting and analysing such operations, they often fall short in forecasting their potential reach and impact. The ability to accurately predict the influence of information operations on social media is crucial for developing proactive mitigation strategies and understanding their evolving dynamics. 


\subsection{State Space Model in Sequence Modeling}
State Space Models (SSMs) have recently emerged as a robust alternative to traditional sequence modeling approaches, particularly for long-range dependency capture~\citep{hasani2021liquid,rangapuram2018deep}. Subsequent work has showed their efficiency in processing extremely long sequences~\citep{gu2022efficiently,dao2022flashattention} and competitive performance in language modeling~\citep{mamba,mamba2}.
Despite these advancements, few studies have applied modern SSM architectures to predict or analyze social media engagement, particularly in the context of disinformation campaigns or crisis events. 
Most existing methods rely on graph-based~\citep{lu2023continuous}, RNN~\citep{wang2017cascade}, or transformer approaches~\citep{zuo2020transformer} that typically assume uniform sampling or discret snapshots. Such assumptions often overlook fine-grained temporal patterns crucial to disinformation campaigns or crisis events.
In contrast, modern SSMs can naturally handle non-uniform intervals and continuous-time dynamics, making them well-suited for rapidly unfolding social media processes. 
Our work bridges this gap by extending the Mamba architecture to handle non-uniform intervals, while identifying misinformation opinions and disinformation narratives.
%This paper introduces \icmamba, an online attention modeling and prediction tool that extends the Mamba architecture in several ways. 
%\icmamba handles interval-censored data with non-uniform intervals, identifies misinformation opinions and disinformation narratives.





