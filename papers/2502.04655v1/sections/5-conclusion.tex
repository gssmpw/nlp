% !TeX root = ../camera.tex

\section{Conclusion}

\icmamba demonstrates strong performance in modeling interval-censored engagement data, providing early predictions of viral content, and tracking long-term opinion spread across platforms.
Through the novel integration of interval-censored modeling and temporal embeddings within a state space model, \icmamba achieves strong performance in predicting dynamic misinformation and disinformation engagement patterns and opinion classification.
These capabilities enable platforms and researchers to identify potentially harmful content and coordinated campaigns in their early stages, facilitating proactive intervention strategies while respecting platform constraints and user privacy. Future work could enhance IC-Mamba through cross-platform dynamics modeling, interpretable attention mechanisms, and real-time deployment adaptations.
Online misinformation and information campaigns are part of our digital ecosystem today, but we do not need to resign ourselves to reactively attempting damage control after the fact. With IC-Mamba, we can identify the next QAnon or climate change denialism conspiracies before they gain mass exposure; and we could mitigate the damage it can do to our lives and democratic societies. We provide detailed discussion of ethical considerations and safeguards in \cref{sec:ethics}.