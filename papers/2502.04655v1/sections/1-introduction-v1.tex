% !TeX root = ../camera.tex

\section{Introduction}

\textcolor{red}{Missing from intro:
\begin{itemize}
    \item a clear definition of \icmamba and its novelties (early on);
    \item a discussion of the engagement at the level of individual posts and opinions -- a definition of opinions.
\end{itemize}
}

\revMAR{I don't really like the structure of the intro. Shall we try something along the following:
\begin{itemize}
  \item PARA1: Online social media can spread harmful content -- like mis and disinfo, what are each? --, but engagement with such content can vary wildly (from a lot to none, can we get some references of papers that study this in the social science?). This paper aims to forecast early the future engagement that mis and disinformation will get.
  \item PARA2: What engagement and what data sources? introduce the three levels of engagement, say that views are rarely captured and observed (unless you are the post's owner), but one-click engagements and comments are typically publicly visible as they drive the digital word-of-mouth process. Here, we use the CrowdTangle that give us interval-censored versions of these engagement measures such as in \cref{fig:sample_ic_mamba}
  \item PARA3: a short overview of approaches to forecast engagement metrics -- a summary of the two paragraphs that I labeled as belonging to the related works.
  \item CONTRIBUTIONS: @lin: I wonder if the Research questions are not a bit too narrow and too technical. With an intro as per the above, we would need something like: "Can engagement be predicted early?".
\end{itemize}
}


\begin{figure}[tbp]
  \centering
  \includegraphics[width=\linewidth]{images/sample_v9.pdf}
  \caption{
  \replace{
    Illustration of interval-censored social media engagement data. Following a post's creation at $t_0$, users perform engagement actions (view, like, comment, share, emoji) at timestamps $s_1$ through $s_8$. While individual actions occur continuously, engagement data is only collected at discrete observation points $t_j$, where each engagement vector $e_j$ captures the cumulative counts of different interaction types over intervals of length $\Delta t_j = t_{j+1} - t_j$.}{}
    \TODO{MAR}{Needs updating, need to describe what \mbox{$T_i$} are -- that is the unobserved exact times of engagements.}
  }
  \label{fig:sample_ic_mamba}
  \Description{social engagement matters sample}
\end{figure}


Social media platforms accelerate the spread of harmful content, particularly misinformation (false information shared without intent to harm) and disinformation (deliberately created and shared false information)~\citep{lazer2018science,scheufele2019science}. Large-scale analysis of social media cascades reveal extreme variations in engagement patterns -- while the most viral false news stories reach up to 100,000 users within hours, real-world browsing data shows that over 70\% of users never engage with such content, and a mere 0.15\% of users drive over 80\% of engagement~\citep{vosoughi2018spread, grinberg2019fake,guess2020exposure}. These findings highlight the critical role of early engagement patterns in determining content virality. Understanding and predicting these patterns early could enable platforms to identify potentially harmful content before it reaches viral scale, enabling proactive rather than reactive intervention strategies.

In this work, we aim to predict social media engagement patterns at both individual and collective levels. In the context of misinformation, individual posts express specific viewpoints or claims, while disinformation often manifests as coordinated narratives - broader storylines that emerge when multiple posts promote similar false claims~\citep{del2016spreading,vosoughi2018spread}. These narratives can spread rapidly across platform communities and even across different platforms~\citep{wilson2020cross}, making it crucial to predict not only individual post engagement but also the collective engagement patterns of posts sharing similar narratives~\citep{bak2021stewardship}. This multi-level prediction enables better early detection and proactive intervention strategies for both organic misinformation spread and coordinated disinformation campaigns.

Based on studies of user behavior and cognitive investment~\citep{lorenz2011social,mosleh2021cognitive,pennycook2021shifting,brady2021social}, we formalize engagement through a three-level hierarchy: views represent the most passive form of engagement and are typically only visible to content owners and platform operators; one-click engagements (likes, shares, and reactions) represent quick evaluative responses; while comments require users to formulate and articulate their thoughts. This hierarchical structure of engagement presents unique challenges for temporal prediction, particularly when modeling the interplay between different engagement types over time.

Our analysis leverages CrowdTangle data\footnote{\url{https://www.crowdtangle.com/}}, which has interval-censored engagement metrics, where observations are made at discrete time points ($t_0, t_1, ...$) with engagement counts recorded for each interval $\Delta t_j$ (as illustrated in \cref{fig:sample_ic_mamba}). For instance, within an interval $\Delta t_2$, we observed two likes, two shares, and two emoji reactions since the last observation, but not the exact timing of individual engagement events (e.g.\ $s_4$). As illustrated in \cref{fig:sample_ic_mamba}, engagement data is collected at discrete observation points $t_j$, where each engagement vector $e_j$ captures the cumulative counts of different interaction types over intervals of length $\Delta t_j = t_{j+1} - t_j$. This censored view of engagement progression, combined with the hierarchical nature of user cognitive investment, presents unique modeling challenges that reflect real-world platform constraints.

Recent deep learning approaches have made progresses in modeling social media engagement through different architectural innovations: language models for capturing coordinated posting behaviours~\citep{atanasov2019predicting,tian2023metatroll} and propagation models for modeling information diffusion~\citep{zannettou2019disinformation,im2020still,luceri2024unmasking,kong2023interval}. State space models have also demonstrated strong performance on sequential prediction tasks ~\citep{mamba,mamba2}, with their latent state representations theoretically well-suited for temporal dependencies. However, these approaches face two key limitations when applied to mis/disinformation engagement forecasting: (1) they primarily focus on classification tasks rather than quantifying future temporal patterns of engagement, and (2) they struggle with the irregularly sampled and bursty nature of viral content progression.

%In the digital age, social media platforms have become powerful tools for information dissemination, significantly influencing public opinion and societal dynamics. While these platforms facilitate rapid communication and information sharing, they also present challenges in the form of misinformation, disinformation, and state-backed influence operations. The spread of such content can have far-reaching consequences, potentially undermining social cohesion, public health, and even democratic processes~\citep{starbird2019disinformation,vosoughi2018spread,bradshaw2019global}.


%The detection of coordinated state-backed influence campaigns presents unique challenges. These operations leverage networks to sway public opinion and exacerbate social divisions~\citep{starbird2019disinformation}, with impacts ranging from eroding institutional trust to inciting violence~\citep{woolley2018computational}. Traditional approaches to this problem have relied on manual analysis by experts, which cannot scale with the volume and velocity of social media content~\citep{qiu2018deepinf,lazer2018science,sharma2019combating,allcott2017social}.

%We propose a hierarchical structure of social media engagement that reflects increasing levels of cognitive and temporal commitment from users, as shown on the right of \cref{fig:sample_ic_mamba}. Views are the most passive level of interaction, as they require minimal cognitive effort. The second tier is one-click engagement, which includes shares, likes and emoji reactions (e.g.\ angry or smile face). This is a low-threshold active interaction demonstrating a basic emotional or reactive response to the source. The highest level of user engagement is comments. It demands a larger time investment and requires users to formulate thoughts, compose language and consider social context.

%Recent deep learning approaches have advanced the detection of social media harmful content. 
%Transformer architectures capture coordinated posting behaviours~\citep{atanasov2019predicting}, while graph neural networks model information diffusion patterns~\citep{zannettou2019disinformation}. 
%Temporal modeling using RNNs and LSTMs has shown their abilities in modeling engagement dynamics~\citep{im2020still}. 
%However, these methods largely focus on binary detection of mis/disinformation, rather than quantifying their societal impact and influence. 
%In addition, they struggle to handle the inherent temporal and hierarchical complexities of social media engagement, particularly with irregularly sampled data and dynamic intervals between viral events.\mar{This para reads like literature review. Shuold not be in intro in this detail.}


%State space models have showed their capacities on language modeling~\citep{mamba,mamba2} and sequential prediction tasks~\citep{rangapuram2018deep}. 
%Their ability to model latent states aligns with capturing temporal dependencies and hierarchical relationships in sequential data. However, when applied to social media influence operations, state space models face limitations in handling the dynamic nature of engagement patterns. They struggle with irregular sampling intervals between viral events, variable engagement rates across different narratives, and the bursty nature of coordinated amplification campaigns. Additionally, they lack mechanisms to capture the hierarchical relationships between individual posts and broader narrative impact.\mar{Also reads as lit review.}

%Social media influence operations present increasingly complex challenges for detection and impact assessment, characterised by irregular viral events, coordinated amplification campaigns, and multi-level narrative structures. 


\boldsubtitle{Our contributions:}
We introduce \icmamba (Interval-Censored Mamba), a novel framework that integrates interval-censored data modeling with state space architecture to capture irregular temporal dynamics and predict engagement patterns across multiple granularities, from individual posts to broader narrative impact.
The paper addresses the following three Research Questions (RQs):

RQ1: \textit{How effectively can state space architectures model the temporal dynamics of interval-censored engagement data?}: IC-Mamba explicitly handles the irregular sampling and censored observations inherent in social media engagement data with time-aware positional embedding and interval-censor state representation. Our experiments demonstrate a 4.72\% improvement in prediction accuracy compared to traditional methods, maintaining robust performance (RMSE 0.118-0.143) across diverse real-world datasets.

RQ2:\textit{ Can we forecast future engagement patterns early enough to enable proactive intervention?}: Through our experiments, we demonstrate \icmamba's capability to predict engagement metrics at multiple granularities, outperforming baselines by achieving lower RMSE at early prediction windows (15-30 minutes post-posting) and maintaining strong performance up to 3 hours. Our model can also captures both individual post dynamics (RMSE 0.118-0.143 across datasets) and group-level engagement patterns (F1 0.508-0.751 for narrative-level predictions).

RQ3: \textit{How effectively can we track and predict the long-term spread of opinions across social media platforms?}: Our experiments demonstrate the model's ability to forecast opinion-level engagement dynamics over extended periods, using observation windows of 3, 7, and 10 days to predict spreading patterns up to 28 days ahead. The results in \cref{fig:dynamic_pred} show IC-Mamba's effectiveness in capturing long-term opinion proliferation and evolution.

\icmamba is not intended to replace human expertise but rather to serve as a powerful tool to augment and streamline the work of social scientists and content moderators. By providing accurate, early predictions of content engagement and capturing complex temporal dynamics, our framework can help researchers identify potentially harmful information campaigns in their nascent stages, allowing for more timely and effective interventions.