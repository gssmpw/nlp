% !TeX root = ../camera.tex
\section{Introduction}


On 28 October 2017, an anonymous 4chan user made a brief yet impactful post on the platform claiming that Hillary Clinton was to be arrested in the coming days\footnote{\url{https://www.bellingcat.com/news/americas/2021/01/07/the-making-of-qanon-a-crowdsourced-conspiracy/}}.
On 6 January 2021, devotees of then-outgoing President Donald Trump stormed the United States Capitol building in an act of domestic terrorism designed to prevent President-elect Joe Biden's election victory from being confirmed. Five people died during and in the immediate aftermath of the attack, and an additional four died in the subsequent months\footnote{\url{https://www.factcheck.org/2021/11/how-many-died-as-a-result-of-capitol-riot/}}; and over 140 police officers were injured\footnote{\url{https://www.nytimes.com/2021/08/03/us/politics/capitol-riot-officers-honored.htm}}. 
Investigations by the Associated Press of the online social media profiles of over 120 of the rioters revealed high levels of adherence to the QAnon conspiracy theory that had begun just four years prior on 4chan\footnote{\url{https://apnews.com/article/us-capitol-trump-supporters-1806ea8dc15a2c04f2a68acd6b55cace}}.
This incident highlights how social media platforms can accelerate the spread of harmful content, particularly misinformation (false information shared without intent to harm) and disinformation (deliberately created and shared false information)~\citep{lazer2018science,scheufele2019science}.
%Large-scale analysis of social media cascades reveal extreme variations in engagement patterns -- while the most viral false news stories reach up to 100,000 users within hours, real-world browsing data shows that over 70\% of users never engage with such content, and a mere 0.15\% of users drive over 80\% of engagement~\citep{vosoughi2018spread, grinberg2019fake,guess2020exposure}.

Given these ongoing impacts, the question must be asked: what if we could have seen this coming? More specifically, what if it had been possible to forecast user engagement with fringe ideologies before they morph into widespread movements? 
We introduce \icmamba, a model capable of forecasting user engagement with online content. We go beyond the level of atomic posts to forecast the number of likes, shares, emoji reactions, and comments for ``emerging opinions'' -- particular worldviews supported across a series of posts.
Our framework can forecast the arrival rate of posts supporting an opinion, and forecast the engagement for each, obtaining estimates of the total level of engagement for the entire opinion.
Our analysis leverages CrowdTangle~\cite{crowdtangle} data with interval-censored engagement metrics, where observations are made at discrete time points with engagement counts recorded for each interval (as seen in \cref{fig:sample_ic_mamba}).

Recent deep learning approaches have made progress in modeling social media engagement through different architectural innovations: language models to capture coordinated posting behaviors~\citep{atanasov2019predicting,tian2021rumour,tian2022duck,tian2023metatroll} and propagation models to model information
diffusion~\citep{zannettou2019disinformation,im2020still,luceri2024unmasking,kong2023interval,Kong2021,Kong2020,Kong2020a,Zhang2019,Kong2018}. 
State space models have also demonstrated strong performance on sequential prediction tasks ~\citep{mamba,mamba2}, with their latent state representations theoretically well suited for temporal dependencies. 
However, these approaches face two key limitations when applied to mis/disinformation engagement forecasting: 
(1) they primarily focus on classification tasks rather than quantifying future temporal patterns of engagement, and 
(2) they struggle with the irregularly sampled nature of viral content.


\noindent\textbf{Our main contributions} address three research questions (RQs) at the intersection of temporal modeling and social media dynamics:

RQ1: \textit{How can we effectively model irregular temporal patterns in social media engagement?}: Through IC-Mamba's time-aware embeddings and state space model architecture, we capture the dynamics of online interactions, achieving a 4.72\% improvement over the state-of-the-art approaches. 

RQ2:\textit{Can we predict viral potential within the critical early window? }: \icmamba shows strong performance in the crucial 15-30 minute post-publication window (RMSE 0.118-0.143), while capturing both granular post-level dynamics and broader narrative patterns (F1 0.508-0.751 for narrative-level predictions).

RQ3: \textit{How can we forecast engagement with emerging opinions early? Can we improve the accuracy and confidence of these forecasts as engagement data streams in over time?}: Our experiments show the model effectively forecasts engagement dynamics early, using 3-, 7-, and 10-day windows to predict spreading patterns up to 28 days, with performance improving as more data streams in. %\cref{fig:dynamic_pred} shows \icmamba's effectiveness in capturing long-term opinion proliferation and evolution.

As a tool, \icmamba streamlines the work of human experts, enabling earlier identification of problematic content, and therefore, providing more time to design and implement countermeasures.

\begin{figure}[t]
  \centering
  \includegraphics[width=\linewidth]{images/sample_v9.pdf}
  \caption{
    Illustration of interval-censored social media engagement data. Following a post's creation at $t_0$, users perform engagement actions (view, like, comment, share, emoji) at timestamps $s_1$ through $s_8$. While individual actions occur continuously, engagement data is only collected at discrete observation points $t_j$, where each engagement vector $e_j$ captures the cumulative counts of different interaction types over intervals of length $\Delta t_j = t_{j+1} - t_j$.
  }
  \label{fig:sample_ic_mamba}
  \Description{social engagement matters sample}
\end{figure}