\section{Related Work}

Current research on accelerating the inference process of LLMs primarily focuses on three categories of methods: \textit{Quantizing Model}, \textit{Generating Fewer Tokens}, and \textit{Reducing KV Cache}. 
Quantizing Model includes both parameter quantization~\cite{mlsys24_awq} and KV Cache quantization~\citep{icml24_kivi}. 
Notably, generating long texts and understanding long-text represent distinct 
scenarios; 
therefore, acceleration methods specifically targeting the long-text generation phase (e.g., pre-filling stage acceleration techniques~\citep{emnlp23_autocompressors,iclr24_icae,emnlp23_llmlingua,iclr25_activation_beacon,nips24_snapkv,arxiv24_pyramidkv} are not discussed here. 
Due to page limits, we focus on the last one. 
See Appendix~\ref{sec:app:related_work} for other details.
% Below is a detailed overview of the last two categories.

% \textbf{Generating Fewer Tokens.}
% This category can be further divided into three strategies based on the number and type of tokens used during inference.
% 1) \textit{Discrete Token Reduction}. 
% Techniques such as prompt engineering~\citep{arxiv24_tale,arxiv24_break_the_chain,arxiv24_concise_thoughts}, instruction fine-tuning~\citep{nips24_skip_steps,arxiv24_c3ot}, or reinforcement learning~\citep{arxiv25_related_work_rl1,arxiv25_o1_pruner} are used to guide LLMs to use fewer discrete tokens during inference.
% For example, TALE~\citep{arxiv24_tale} prompts LLMs to complete tasks under a predefined token budget. 
% \citeauthor{arxiv25_related_work_rl1} construct specific datasets and employ reinforcement learning reward mechanisms to encourage models to generate concise and accurate outputs, thereby reducing token usage.
% 2) \textit{Continuous Token Replacement}.
% These methods~\citep{arxiv24_coconut,arxiv24_ccot} explore using continuous-space tokens instead of traditional discrete vocabulary tokens. 
% A representative example is CoConut~\citep{arxiv24_coconut}, which leverages Curriculum Learning to train LLMs to perform inference with continuous tokens.
% 3)\textit{No Token Usage}.
% By internalizing the inference process between model layers, the final answer is generated directly during inference without intermediate tokens~\citep{arxiv24_icot,arxiv23_kd_cot}.
% These three strategies are implemented after model training and do not require additional intervention during inference. 
% Technically, the acceleration effect of these methods increases sequentially, but at the cost of a gradual decline in the generalization performance of LLMs. 
% Additionally, the first strategy does not significantly reduce GPU memory usage.

\textbf{Reducing KV Cache.}
This category can be divided into two types of strategies: pruning-based KV Cache selection in discrete space and merging-based KV Cache compression in continuous space.
1) \textit{Pruning-Based Strategies}.
Specific eviction policies~\citep{nips23_h2o, iclr24_streamingllm,arxiv24_sepllm} are designed to retain important tokens during inference.
% For example, StreamingLLM~\citep{iclr24_streamingllm} considers the initial sink tokens and the most recent tokens as important.
% H2O~\citep{nips23_h2o} focuses on tokens with high historical attention scores.
% SepLLM~\citep{arxiv24_sepllm} emphasizes tokens corresponding to punctuation marks.
2) \textit{Merging-Based Strategies}.
Anchor tokens are introduced, and LLMs are trained to compress historically important information into these tokens, thereby achieving KV Cache merging~\citep{acl24_anllm}.
Both strategies require intervention during inference. 
The key difference is that the first strategy is training-free but applies the eviction policy for every generated token, while the second is a training-based method and allows the LLM to decide when to apply the eviction policy.
