% \newpage
\section{Experiments}
\begin{figure}[!t] % !t, !htbp
    \centering
    \scalebox{1}{
    \includegraphics[width=1.0\linewidth]{figures/metric-v3.pdf} 
    }
    % \vspace{-1mm}
    \caption{
    The relationship between context length and the number of generated tokens across different methods. 
    The Dependency metric represents the area under the curve, while the Peak Token metric denotes the maximum value of the curve.
    See App.~\ref{sec:app:metric} for details.
    }
    \label{fig:exp:metric}
    \vspace{-2mm}
\end{figure} 

\subsection{Experimental Settings}


\begin{table*}[tp!]
\renewcommand{\arraystretch}{1.2}
% \renewcommand{\tabcolsep}{4pt}

\resizebox{1.0\linewidth}{!}{
\begin{tabular}{llcccc}
\toprule

\multicolumn{1}{c}{\multirow{2}{*}{\textbf{Model}}}
& \multicolumn{1}{c}{\multirow{2}{*}{\textbf{System}}}
&  \multicolumn{2}{c}{\textbf{\Dataset{}-\textit{dev}}}  
&  \multicolumn{2}{c}{\textbf{\Dataset{}-\textit{test}}} \\

\cmidrule(lr){3-4} \cmidrule(l){5-6} 

& 
&  \textbf{Cor. (P / R / $\textbf{F}_{0.5}$)$\uparrow$}  
&  \textbf{Exp. (P / R / $\textbf{F}_{1}$ / $\textbf{F}_{0.5}$ / Acc)$\uparrow$}
&  \textbf{Cor. (P / R / $\textbf{F}_{0.5}$)$\uparrow$}  
&  \textbf{Exp. (P / R / $\textbf{F}_{1}$ / $\textbf{F}_{0.5}$ / Acc)$\uparrow$}  \\

\hline

\textbf{BERT}  &  \textbf{Explanation}  &
-  &
53.60 / 35.46 / \textbf{42.68} / \textbf{48.63} / \textbf{52.09}  &
-  &
51.73 / 36.34 / \textbf{42.69} / \textbf{47.69} / \textbf{50.83}  \\

\hline

\multirow{7}{*}{\textbf{BART}} 
& \textbf{Baseline}  &
36.14 / 34.87 / 35.88  &
-  &
36.33 / 35.49 / 36.16  &
-  \\

& \textbf{Explanation}  &
-  &
44.43 / 32.93 / 37.82 / 41.53 / 33.36  &
-  &
42.34 / 33.13 / 37.18 / 40.11 / 26.95  \\

& \textbf{Infusion + Evidence}   &
45.78 / 44.55 / \textbf{45.53}  &
-  &
46.02 / 44.13 / \textbf{45.63}  &
-  \\

& \textbf{Infusion + Type}   &
35.31 / 47.87 / 35.22  &
-  &
36.00 / 35.37 / 35.87  &
-  \\

& \textbf{Infusion + Evidence\&Type}   &
44.28 / 47.55 / 44.90  &
-  &
44.96 / 47.50 / 45.44  &
-  \\

& \textbf{Self-Rationalization Pre-explaining}  &
38.25 / 34.18 / \textbf{37.36}  &
36.01 / 35.58 / 35.79 / 35.92 / 26.56  &
38.68 / 35.41 / \textbf{37.98}  &
36.77 / 36.85 / 36.81 / 36.79 / 26.24  \\

& \textbf{Self-Rationalization Post-explaining}  &
36.34 / 40.15 / 37.05  &
48.95 / 42.72 / \textbf{45.63} / \textbf{47.56} / \textbf{40.32}  &
36.52 / 40.41 / 37.24  &
49.43 / 44.10 / \textbf{46.61} / \textbf{48.26} / \textbf{39.86}  \\

\hline

\multirow{7}{*}{\textbf{T5}} 
& \textbf{Baseline}  &
44.04 / 22.77 / 37.11  &
-  &
43.31 / 23.71 / 37.16  &
-  \\

& \textbf{Explanation}  &
-  &
58.47 / 24.23 / 34.26 / 45.59 / 28.01   &
-  &
59.00 / 25.13 / 35.25 / 46.47 / 26.06  \\

& \textbf{Infusion + Evidence}   &
52.88 / 31.40 / 46.54  &
-  &
53.50 / 30.76 / 46.61  &
-  \\

& \textbf{Infusion + Type}   &
41.86 / 22.20 / 35.56  &
-  &
44.06 / 23.40 / 37.45  & 
-  \\

& \textbf{Infusion + Evidence\&Type}   &
52.41 / 32.22 / \textbf{46.57}  &
-  &
52.90 / 31.46 / \textbf{46.55}  &
-  \\

& \textbf{Self-Rationalization Pre-explaining}  &
41.60 / 26.13 / 37.19  &
59.87 / 22.78 / 33.01 / 45.16 / 26.61  &
40.96 / 27.20 / 37.19  &
59.33 / 23.27 / 33.43 / 45.29 / 26.03  \\

& \textbf{Self-Rationalization Post-explaining}  &
43.04 / 28.48 / \textbf{39.05}  &
66.82 / 23.95 / \textbf{35.26} / \textbf{49.20} / \textbf{29.34}  &
42.28 / 28.75 / \textbf{38.64}  &
67.99 / 26.10 / \textbf{37.72} / \textbf{51.47} / \textbf{29.80}  \\

\hline

\multirow{7}{*}{\textbf{Llama3}} 
& \textbf{Baseline}  &
33.80 / 37.53 / 34.48  &
-  &
33.64 / 38.12 / 34.45  &
-  \\

& \textbf{Explanation}  &
-  &
47.70 / 6.53 / 11.49 / 21.10 / 23.17  &
-  &
54.68 / 7.39 / 13.02 / 23.98 / 27.32  \\

& \textbf{Infusion + Evidence}   &
40.15 / 44.47 / 40.95  &
-  &
41.94 / 46.26 / 42.74  &
-  \\

& \textbf{Infusion + Type}   &
37.35 / 42.77 / 38.32 &
-  &
40.20 / 44.60 / 41.01  &
-  \\

& \textbf{Infusion + Evidence\&Type}   &
48.01 / 52.64 / \textbf{48.87}  &
-  &
45.26 / 49.75 / \textbf{46.09}  &
-  \\

& \textbf{Self-Rationalization Pre-explaining}  &
32.55 / 36.72 / 33.31  &
53.84 / 8.68 / 14.94 / 26.39 / 27.35  &
34.98 / 38.36 / \textbf{35.61} &
55.51 / 10.31 / 17.39 / 29.58 / 28.89  \\

& \textbf{Self-Rationalization Post-explaining}  &
32.37 / 41.08 / \textbf{33.80}  &
68.97 / 29.29 / \textbf{41.12} / \textbf{54.27} / \textbf{36.84}  &
33.58 / 41.84 / 34.96  &
64.36 / 27.34 / \textbf{38.38} / \textbf{50.64} / \textbf{35.22}  \\

\bottomrule
\end{tabular}}

\caption{Main results on \Dataset{}-\textit{dev} and \Dataset{}-\textit{test}.}
\label{tab:exp_main}

\end{table*}


\textbf{Baselines.}
We conduct experiments on two types of LLMs: the Qwen2.5-7B series~\citep{arxiv24_qwen2_5} and the Llama3.1-8B series~\citep{arxiv24_llama_3}.
To establish an upper bound on performance, we perform full parameter instruction tuning using the Bespoke-Stratos-17k dataset (abbr. BS17K, with a data sample shown in Figure~\ref{prompt:case:train}), and the fine-tuned model is denoted as \textit{Vanilla}.  
Notably, we initialize the training with the R1-Distill~\citep{arxiv25_deepseek_r1} (e.g., \texttt{DeepSeek-R1-Distill-Qwen-7B}) model, as we found that finetuning on instruction models (e.g., \texttt{Qwen2.5-7B-instruct}) yields limited improvements.
We introduce five baselines for comparison: 
two training-free acceleration methods applied to Vanilla (H2O~\citep{nips23_h2o} and SepLLM~\citep{arxiv24_sepllm}, which retain important KV Cache through specific strategies), 
one training-based method (AnLLM~\citep{acl24_anllm}), 
and the remaining two use CoT~\citep{nips22_cot} to prompt the instruction model and the R1-Distill model. 
More detailed descriptions of the baselines can be found in Appendix~\ref{sec:app:exp:baseline_details}.  

\textbf{Evaluation Metrics and Datasets.}
We evaluate \ours~on four datasets: GSM8K~\citep{arxiv21_gsm8k}, MMLU~\citep{iclr21_mmlu}, GPQA~\citep{colm24_gpqa}, and BBH~\citep{acl23_bbh}. 
For MMLU and BBH, we randomly sample a portion of the data for evaluation. 
The evaluation focuses on both \textit{effectiveness} and \textit{efficiency}. 
For effectiveness, we use accuracy as the evaluation metric (\textit{Acc}); 
for efficiency, we employ three metrics: inference time (\textit{Time}), the peak number of tokens in the context during inference (\textit{Peak}), and the sum of \textit{dependency} of each generated token on previous tokens during the generation (\textit{Dep}). 
Figure~\ref{fig:exp:metric} visualizes the Peak and Dep metrics, where the value of Dep equals the area enclosed by the lines. 
The Dep metric characterizes the amount of information used during inference, with smaller values indicating more significant compression. 
We aim to compare the other three metrics under similar Dep values. 
It is important to note that Peak characterizes a momentary state, while Dep characterizes the entire inference process, so there is no direct correlation between the two. 
For more details about Dep, please refer to Appendix~\ref{sec:app:metric}.  

\textbf{Implementation.}
For~\ours, we design two different segmentation functions $\texttt{Seg()}$. 
At the token level, we compress every 6 tokens into 2 tokens, i.e., $|C|=2$, denoted as ``ours (\textit{token})''. 
At the thought level, we use ``{\textbackslash n\textbackslash n}'' as a delimiter to simply segment the B17K data into several thoughts, denoted as ``ours (\textit{tho.})''. 
For the Qwen, we compress a thought into 9 tokens, i.e., $|C|=9$; 
for the Llama, we compress a thought into 7 tokens, i.e., $|C|=7$. 
In all experiments, we use greedy decoding with a maximum output length of 10240 tokens.
Please refer to the App.~\ref{sec:app:exp} for more details.
% For more detailed training and evaluation settings, please refer to the Appendix~\ref{}.


\subsection{Main Results}
In Table~\ref{table:exp_main}, we report the results of four evaluation metrics for two models on four datasets.

Key observations include:
\textit{1)} Distill-R1 underperforms CoT on all datasets, which is attributed to the repetition caused by Greedy Decoding, consistent with the findings of~\citeauthor{arxiv25_deepseek_r1}.
\textit{2)} H2O effectively reduces memory usage while maintaining the performance of the vanilla, indicating that the greedy eviction policy is effective in long-text generation tasks. 
However, H2O significantly increases inference time compared to Vanilla, with an average increase of 51\% on Qwen and 72\% on Llama. 
This is attributed to the token-wise eviction policy of H2O, which introduces additional overhead for each generated token.
\textit{3)} SepLLM performs the worst in terms of maintaining performance, gradually losing language ability during generation, which results in the inability to output termination tokens and thus leads to excessive inference time.
\textit{4)} Compared to H2O, \ours~(\textit{tho.}) achieves similar performance with lower Dep values (i.e., similar compression rate), while reducing inference time by an average of 52\% on Qwen and 41\% on Llama. 
Additionally, \ours~(\textit{tho.}) retains higher accuracy and faster inference speed compared to AnLLM.

{Based on these observations, we draw the following conclusions}:
\textit{1)} {B17K is an effective dataset.}
We find Vanilla outperforms CoT and Distill-R1 on most datasets, indicating that B17K is an effective dataset that mitigates the repetition issue in Distill-R1 through SFT.
\textit{2)} {\ours~is an effective method and achieves a good balance between effectiveness and efficiency in inference.}
Specifically, on the Qwen, \ours~sacrifices 1\% accuracy but saves 26\% time, reduces the peak tokens by 70\%, and decreases Dep. by 78\% (i.e., achieves a 16.6/3.7=4.5x compression ratio). 
On the Llama, it sacrifices 6\% accuracy but saves 1\% inference time, reduces the peak tokens by 70\%, and decreases Dep. by 74\% (i.e., achieves a 10.5/2.7=3.9x compression ratio).
\textit{3)} {The segmentation function is vital for~\ours.} %(i.e., $\texttt{Seg()}$)  
The thought-level segmentation function outperforms the token-level, with accuracy improvements of 6.2\% on Qwen and 5.6\% on Llama. 
This suggests that token-level segmentation leads to the loss of semantic boundaries.


\begin{figure*}[!th] % !t, !htbp
    \centering
    \scalebox{0.9}{
    \includegraphics[width=1\linewidth]{figures/exp-efficiency-v6.pdf} 
    }
    % \vspace{-1mm}
    \caption{
    % (a) Average number of generated tokens across the four datasets. 
    % Each bar represents the average tokens generated by the respective model on the specified dataset. 
    % (b) Token compression frequency distribution for Qwen on the GPQA dataset. 
    % The bars represent the percentage of tokens falling within specified ranges, while the cumulative percentage curve illustrates the total proportion of tokens up to each range. 
    % (c) The main part illustrates the relationship between the number of generated tokens and inference time. 
    % Each subplot displays the inference time and peak token for various numbers of output tokens. 
    % (d) Compression ratios of the \ours. The bars represent the average compression ratios with 95\% confidence intervals indicated by error bars. 
    % (e-f) Impact of cache size (i.e., $|C|$) on accuracy, Dep, inference time, peak tokens, generated tokens, and compression frequency.
    Efficiency Analysis and Ablation Results.
    Fig.(a) represents the average tokens generated by the respective model on the specified dataset. 
    Fig.(b) shows the percentage of tokens falling within specified ranges, while the cumulative percentage curve illustrates the total proportion of tokens up to each range. 
    Fig.(c) illustrates the relationship between the number of generated tokens and inference time. 
    Each subplot displays the inference time and peak token for various numbers of output tokens. 
    Fig.(d) represents the average compression ratios with 95\% confidence intervals indicated by error bars. 
    Fig.(e-f) examines the impact of cache size (i.e., $|C|$) on accuracy, Dep, inference time, peak tokens, generated tokens, and compression frequency.
    }
    \label{fig:exp:efficient}
    \vspace{-3mm}
\end{figure*} 


\vspace{-1mm}
\subsection{Efficiency}
For clarity, ``\ours'' hereafter denotes \ours~(\textit{tho.}).
In this section, we conduct an in-depth analysis of \ours's efficiency, focusing on the following three questions:

\textbf{Does \ours~generate more tokens compared to Vanilla?}
Fig.~\ref{fig:exp:efficient}(a) shows the average number of generated tokens for H2O, AnLLM, LightThinker, and Vanilla across four datasets (others in the App.~\ref{sec:app:exp:additional_results}). 
We observe that:
1) \ours~is the only method that reduces the number of generated tokens compared to Vanilla, with an average reduction of 15\% on Qwen and 13\% on Llama. 
This is one of the reasons for its faster inference speed.
2) H2O increases token generation by 10\% on Qwen but reduces it by 7\% on Llama. 
Despite the reduction in tokens for Llama, the inference time still increases as shown in Table~\ref{table:exp_main}, indicating that its eviction policy accumulates additional overhead as token generation grows.

\textbf{What is the compression ratio of \ours?}
Fig.~\ref{fig:exp:efficient}(d) illustrates the compression ratio across four datasets, 
Table~\ref{table:exp:efficiency:comp_count} reports the average compression counts, 
and Fig.~\ref{fig:exp:efficient}(b) shows the distribution of compressed token counts for GPQA using Qwen (other datasets are in the App.~\ref{sec:app:exp:additional_results}). 
We find that:
1) Compression counts and ratios are more closely related to downstream tasks than to the model itself. 
Simple tasks like GSM8K exhibit lower compression counts and higher compression ratios, while complex tasks like GPQA require more frequent compressions and smaller compression ratios. 
% This is likely because complex tasks demand more attempts and finer-grained solving steps, leading to frequent compressions and smaller ratios.
2) The distribution of compressed token counts follows a long-tail pattern. 
% Approximately 60\% of compressions involve fewer than 50 tokens, while about 4\% involve more than 200 tokens, likely due to initial deep comprehension of the problem.

% \textbf{How does \ours~perform in terms of memory and inference overhead for long-text generation?}
\textbf{How efficient is \ours~in memory usage and inference for long-text generation?}
Fig.~\ref{fig:exp:efficient}(c) shows the inference time and peak tokens of \ours~and Vanilla as a function of output token length.
We set the prompt length to 125 and compressed 56 tokens to 8 tokens (i.e., $|C|=7$).
We observe that:
1) Our method significantly reduces inference time. % for long-text generation. 
For example, when generating 32K tokens, the inference time is reduced by 44\%. 
For shorter texts (from 1K to 4K tokens), the reduction is more modest, ranging from 1\% to 4\%.
2) Even for shorter texts, \ours~substantially reduces peak tokens usage. 
For instance, when generating 1K tokens, peak tokens is reduced by 72\%, and for 32K tokens, it is reduced by 85\%.

\begin{table}[]
\centering
\small
\scalebox{0.8}{
\begin{tabular}{ccccc}
\toprule
      & GSM8K & MMLU & GPQA & BBH \\ \midrule
Qwen  & 20    & 37   & 115  & 48  \\
Llama & 26    & 47   & 139  & 55  \\ \bottomrule
\end{tabular}
}
\caption{
Statistics of the average number of compressions per dataset for \ours.
}
\label{table:exp:efficiency:comp_count}
\vspace{-2mm}
\end{table}
\vspace{-1mm}
\subsection{Ablation}
\label{ablation}
\textbf{Decoupled Token and Attention Mask Mode.}
\ours~differs from AnLLM in two key aspects: the decoupled token design and the attention mask as shown in Figure~\ref{fig:comp}. 
% For the attention mask, AnLLM focuses only on the most recently generated content during compression, i.e., $C^{(2)}\leftarrow\texttt{Cmp}(S_2)$, while \ours~additionally attends to the prompt and historically compressed content, i.e., $C^{(2)}\leftarrow\texttt{Cmp}(X,C^{(1)},S_2)$. 
To validate the effectiveness of these mechanisms, we conduct ablation experiments.  
As shown in Table~\ref{table:exp:ablation:attention}, under the same cache size setting and using AnLLM's attention mask mechanism (``AnLLM'' vs. ``Ours ($|C|=1$, T)''), the decoupled design improves accuracy by 2\%. 
Further adopting \ours's attention mask mode yields an additional 7\% improvement. 
These results demonstrate the effectiveness of both the decoupled token and the attention mask mode in \ours.

\textbf{Cache Size.}
We varied $|C|$ in the set $\{1,3,5,7,9\}$ to observe its impact on accuracy, inference time, dependency (i.e., Dep), peak tokens, generated token count, and compression frequency. 
Figures~\ref{fig:exp:efficient}(e-g) illustrate these trends on the Qwen model. 
We observe that:  
% 1) As shown in Figure~\ref{fig:exp:efficient}(e), increasing the cache size significantly improves accuracy while reducing inference time. 
1) Increasing the cache size significantly improves accuracy while reducing inference time.
% This indicates that a larger cache size mitigates information loss caused by compression.  
% 2) As shown in Figure~\ref{fig:exp:efficient}(g), increasing the cache size reduces both the compression frequency and the number of generated tokens.
2) Increasing the cache size reduces both the compression frequency and the number of generated tokens.
3) Combining Figure~\ref{fig:exp:efficient}(e) and Figure~\ref{fig:exp:efficient}(g), we find that a smaller cache size leads to more frequent generation and compression to retain more information, while a larger cache size reduces this frequency.  

\begin{figure}[!t] % !t, !htbp
    \centering
    \scalebox{0.85}{
    \includegraphics[width=1\linewidth]{figures/exp-case.pdf} 
    }
    % \vspace{-1mm}
    \caption{
    Case Study. 
    The figure illustrates partial inference results of a case from GSM8K.
    See App.~\ref{sec:app:exp:additional_results} for the complete content.  
    Pink and light blue backgrounds are used to distinguish adjacent compression processes, where each color represents one compression.}
    \label{fig:exp:case}
    \vspace{-3mm}
\end{figure} 
\subsection{Case Study}

Fig.~\ref{fig:exp:case} illustrates a failure case from the GSM8K dataset. 
We observe that although the LLM arrives at the correct answer during the thinking process (see \texttt{Model's Thoughts} field in the Fig.~\ref{fig:exp:case}), it makes an error in the final output (see \texttt{Model's Solution} field in the Figure). 
Specifically, in the third sentence of the $\texttt{Model's Solution}$ field, the first occurrence of ``4000'' is incorrect. 
This indicates that information loss occurred during the second compression step (theoretically, ``8000'', ``4000'', and ``24000'' should have been compressed, but the LLM only compressed ``4000'' and ``24000''), leading to subsequent reasoning errors. 
Such errors occur frequently in the GSM8K dataset, suggesting that the current compression method is not sufficiently sensitive to numerical values. 
% Further exploration and improvement are needed to address this issue.


\begin{table}[]
\centering
\small
\scalebox{0.8}{
\begin{tabular}{lccccc}
\toprule
 & \multicolumn{1}{l}{GSM8K} & \multicolumn{1}{l}{MMLU} & \multicolumn{1}{l}{GPQA} & \multicolumn{1}{l}{BBH} & \multicolumn{1}{l}{AVG} \\ \cmidrule{1-6}
AnLLM & \underline{78.39} & 54.63 & 19.70 & 54.95 & 51.92 \\
Ours (|C|=1, T)  & 78.32 & \underline{58.23} & \underline{20.71} & \underline{55.35} & \underline{53.15} \\
Ours (|C|=1, F) & \textbf{80.21} & \textbf{58.23} & \textbf{22.22} & \textbf{62.02} & \textbf{55.67} \\ \bottomrule
\end{tabular}
}
\caption{
Ablation results on the Qwen, reporting accuracy on four datasets. 
``T'' denotes the use of AnLLM's attention mask mechanism, while ``F'' indicates the use of \ours's attention mask mechanism.
}
\label{table:exp:ablation:attention}
\vspace{-3mm}
\end{table}

