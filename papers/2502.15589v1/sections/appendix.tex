
\clearpage
\appendix

\section*{Appendix}
\section{Metric: \texttt{Dependency}}
\label{sec:app:metric}
\begin{figure}[!htbp] % !t, !htbp
    \centering
    \scalebox{1}{
    \includegraphics[width=1.0\linewidth]{figures/metric-cal-v1.pdf} 
    }
    % \vspace{-1mm}
    \caption{Illustration of the metric Dependency.}
    \label{fig:app:metric}
    % \vspace{-3mm}
\end{figure} 
\subsection{Motivation}
\ours~and AnLLM~\citep{acl24_anllm} are dynamic compression methods, meaning the number of compressions and the compression ratio are determined by the LLM itself rather than being predefined hyperparameters. 
In contrast, H2O~\citep{nips23_h2o} and SepLLM~\citep{arxiv24_sepllm} allow users to set hyperparameters to control the maximum number of tokens retained during inference. 
This fundamental difference makes it challenging to directly and fairly compare dynamic compression methods like \ours~and AnLLM with KV cache compression approaches like H2O and SepLLM.  

Traditionally, KV cache compression methods are compared by setting the same maximum peak token count, but this metric becomes inadequate in our context. 
As illustrated in Figure~\ref{fig:app:metric}, which shows the relationship between generated tokens and context length for Vanilla, H2O, and \ours, 
\ours~occasionally exceeds H2O in peak token count. 
However, this metric is misleading because \ours's peak memory usage occurs only momentarily, while H2O maintains a consistently high token count over time.  

Moreover, previous KV cache compression methods often compress prompt parts only and assume a fixed prompt length, allowing compression ratios to be predefined. 
In our setting, however, the output is also needed to be compressed.
The output token count is unknown, making it impossible to preset a global compression ratio. 
Consequently, relying solely on maximum peak token count as a comparison metric is insufficient.  

To address these challenges, we propose a new metric called \textit{Dependency}, which quantifies the total amount of information dependencies during the generation process. 
This metric enables fair comparisons between dynamic compression methods and traditional KV cache compression approaches by ensuring evaluations are conducted under similar effective compression ratios.  
\subsection{Definition}
We introduce the \textbf{Dependency} (abbr., Dep) metric, defined as the sum of dependencies of each generated token on previous tokens during the generation of an output. 
Geometrically, it represents the area under the curve in Figure~\ref{fig:app:metric}. 
Dependency can be calculated either from its definition or through its geometric interpretation. 
Here, we focus on the geometric approach. 
Let the initial prompt length be \( L_P \), the model's output length be \( L_O \), and the maximum context length set by KV cache compression methods be \( L_C \).  

\textbf{Dependency for Vanilla.}
The area under Vanilla's curve forms a right trapezoid, calculated as:  
\[
\begin{aligned}
\texttt{Dependency} &= \frac{(L_P + L_P + L_O) \times L_O}{2} \\
&= \frac{{L_O}^2}{2} + L_P \times L_O
\end{aligned}
\]

\textbf{Dependency for H2O.}
The area under H2O's curve consists of a trapezoid (left part in Figure~\ref{fig:app:metric}(b)) and a rectangle (right part in Figure~\ref{fig:app:metric}(b)):  
\[
\begin{aligned}
S_\texttt{Trapezoid} &= \frac{(L_P + L_C) \times (L_C - L_P)}{2} \\
S_\texttt{rectangle} &= L_C \times (L_O - L_C + L_P) \\
\texttt{Dependency} &= S_\texttt{Trapezoid} + S_\texttt{rectangle} \\
&= \frac{2L_PL_C + 2L_OL_C - {L_P}^2 - {L_C}^2}{2}
\end{aligned}
\]

\textbf{Dependency for \ours~and AnLLM.}
For \ours~and AnLLM, Dependency does not have a closed-form solution and must be computed iteratively based on its definition. 

\subsection{Application}
\textbf{Value of Dependency.}
A higher Dependency value indicates that more tokens need to be considered during generation, reflecting greater information usage.
Conversely, a lower Dependency value suggests a higher effective compression ratio.  

\textbf{Dependency Ratio.}
By dividing the Dependency of an accelerated method by that of Vanilla, we obtain the compression ratio relative to Vanilla. For example, in Table~\ref{table:exp_main}'s ``Avg.'' column, Vanilla's Dependency is 16.6M, H2O's is 4.4M, and \ours's is 3.7M. 
Thus, H2O achieves a compression ratio of \( \frac{16.6}{4.4} \approx 3.8 \), while \ours~achieves \( \frac{16.6}{3.7} \approx 4.5 \).  

This metric provides a unified framework for evaluating both dynamic and static compression methods, ensuring fair and meaningful comparisons.

% \section{Inference Pseudocode}


\begin{figure}[!htbp] % !t, !htbp
    \centering
    \scalebox{1}{
    \includegraphics[width=1.0\linewidth]{figures/attention_mode-v1.pdf} 
    }
    % \vspace{-1mm}
    \caption{Illustration of Attention Mask in Table~\ref{table:exp:ablation:attention}.}
    \label{fig:app:attention_mode}
    % \vspace{-3mm}
\end{figure} 

\section{Experiment}
The code and data will be released at \url{https://github.com/zjunlp/LightThinker}.
\label{sec:app:exp}
\subsection{Training Data}
\label{sec:app:exp:train_data_case}
Examples of training samples are shown in Figure~\ref{prompt:case:train}.

\subsection{Baseline Details}
\label{sec:app:exp:baseline_details}
\textbf{H2O}~\citep{nips23_h2o} is a training-free acceleration method that greedily retains tokens with the highest cumulative attention values from historical tokens. 
It includes two hyper-parameters: the maximum number of tokens and the current window size (i.e., \texttt{local\_size}). 
The maximum number of tokens for each task is listed in the ``Peak'' column of Table~\ref{table:exp_main}, and the \texttt{local\_size} is set to half of the maximum number of tokens. 
The experimental code is implemented based on \url{https://github.com/meta-llama/llama-cookbook}.

\textbf{SepLLM}~\citep{arxiv24_sepllm} is another training-free acceleration method that considers tokens at punctuation positions as more important. 
It includes four parameters: the maximum number of tokens is set to 1024, \texttt{local\_size} is set to 256, \texttt{sep\_cache\_size} is set to 64, and \texttt{init\_cache\_size} is set to 384. 
We also tried another set of parameters (\texttt{init\_cache\_size}=4, \texttt{sep\_cache\_size}=64, \texttt{local\_size}=720, maximum number of tokens=1024), but found that the first set of parameters performed slightly better.

\textbf{AnLLM}~\citep{acl24_anllm} is a training-based method that shares a similar overall approach with \ours~but accelerates by saving historical content in anchor tokens. 
The specific differences between the two are detailed in Section~\ref{sec:method:discussion}.

\subsection{Training Details}
\label{sec:app:exp:training_details}
Both \textbf{Vanilla} and \textbf{AnLLM} are trained on the B17K~\citep{bespoke_stratos_train_dataset} dataset using the R1-Distill~\citep{arxiv25_deepseek_r1} model for 5 epochs, while \ours~is trained for 6 epochs. 
The maximum length is set to 4096, and a cosine warmup strategy is adopted with a \texttt{warmup\_ratio} of 0.05. 
Experiments are conducted on 4 A800 GPUs with DeepSpeed ZeRo3 offload enabled. 
The batch size per GPU is set to 5, and the gradient accumulation step is set to 4, resulting in a global batch size of 80. 
The learning rate for Vanilla is set to 1e-5, while for AnLLM and \ours, it is set to 2e-5.

\subsection{Evaluation Details}
\label{sec:app:exp:evaluation_details}
For the CoT in Table~\ref{table:exp_main}, the prompts used are shown in Figure~\ref{prompt:system:base} and Figure~\ref{prompt:task:base}. 
For the R1-Distill model, no system prompt is used, and the task-specific prompts are shown in Figure~\ref{prompt:task:r1}. 
Vanilla, H2O, SepLLM, AnLLM, and \ours~share the same set of prompts, with the system prompt shown in Figure~\ref{prompt:system:vanilla} and downstream task prompts shown in Figure~\ref{prompt:task:r1}. 
The options for MMLU~\citep{iclr21_mmlu} and GPQA~\citep{colm24_gpqa} multiple-choice questions are randomized.

\subsection{Additional Results}
\label{sec:app:exp:additional_results}
Figure~\ref{fig:app:token} compares the number of tokens generated by two models across different datasets. 
Figure~\ref{fig:app:frequency} shows the distribution of compressed lengths for LightThinker on two models and four datasets. Figure~\ref{fig:app:attention_mode} illustrates the attention masks for the baselines in Table~\ref{table:exp:ablation:attention}.
Figure~\ref{prompt:case} shows a complete case in Figure~\ref{fig:exp:case}.

\begin{figure*}[!htbp] % !t, !htbp
    \centering
    \scalebox{0.8}{
    \includegraphics[width=1.0\linewidth]{figures/token_split-v2.pdf} 
    }
    % \vspace{-1mm}
    \caption{Average number of generated tokens.}
    \label{fig:app:token}
    % \vspace{-3mm}
\end{figure*} 

\section{Related Work}
% \subsection{Vision Language Model}
% 시각장애인에서 상황을 설명할 DB가 없으니 만들었다. 그리고 이를 VLM에 튜닝했다.
\subsection{Technical approaches for assisting the visually-impaired}


\subsection{Datasets for visual instruction tuning}


\section{Discussions}
\label{sec:app:discussion}
\textbf{Viewing \ours~from Other Perspectives.}
In previous sections, we design \ours~from a compression perspective. 
Here, we further discuss it from the perspectives of \textit{Memory} and \textit{KV Cache Compression}, where KV Cache can be viewed as a form of LLM work memory.
In Memory perspective, \ours's workflow can be summarized as follows: it first performs autoregressive reasoning, then stores key information from the reasoning process as memory (memory), and continues reasoning based on the memorized content. 
Thus, the information in the cache tokens acts as a compact memory, though it is only effective for the current LLM and lacks transferability.
In KV Cache Compression perspective,
Unlike methods such as H2O~\citep{nips23_h2o}, which rely on manually designed eviction policy to select important tokens, \ours~merges previous tokens in a continuous space, \textit{ceating} new representations. 
The content and manner of merging are autonomously determined by the LLM, rather than being a discrete selection process.

% \citet{emnlp24_onegen}

% \citeauthor{emnlp24_onegen}

\begin{figure*}[!htbp] % !t, !htbp
    \centering
    \scalebox{1}{
    \includegraphics[width=0.8\linewidth]{figures/frequency_split.pdf} 
    }
    % \vspace{-1mm}
    \caption{Token compression frequency distribution for \ours.}
    \label{fig:app:frequency}
    % \vspace{-3mm}
\end{figure*} 


\begin{figure*}[!htbp]
\centering
\scalebox{1}{
\begin{tcolorbox}
\textbf{System Prompt:}

Below is a question. Please think through it step by step, and then provide the final answer. If options are provided, please select the correct one. 

\#\# Output format:\\
Use ``<THOUGHT>...</THOUGHT>'' to outline your reasoning process, and enclose the final answer in `\textbackslash boxed\{\}`.\\
\\
\#\# Example 1:\\
Question: \\
What is 2 + 3?\\
Output:\\
<THOUGHT>First, I recognize that this is a simple addition problem. Adding 2 and 3 together gives 5.</THOUGHT>\\
Therefore, the final answer is \textbackslash boxed\{5\}.\\
\\
\#\# Example 2:\\
Question: \\
What is 2 + 3?\\
A. 4\\
B. 5\\
C. 10\\
\\
Output:\\
<THOUGHT>First, I recognize that this is a simple addition problem. Adding 2 and 3 together gives 5.</THOUGHT>\\
Therefore, the final answer is \textbackslash boxed\{B\}.\\

\end{tcolorbox}
}

\caption{System prompt for \texttt{Qwen2.5-7B-Instruct} and \texttt{Llama3.1-8B-Instruct}.}
\label{prompt:system:base}
\end{figure*}

\begin{figure*}[!htbp]
\centering
\scalebox{1}{
\begin{tcolorbox}
\textbf{System Prompt:}

Your role as an assistant involves thoroughly exploring questions through a systematic long thinking process before providing the final precise and accurate solutions. This requires engaging in a comprehensive cycle of analysis, summarizing, exploration, reassessment, reflection, backtracing, and iteration to develop well-considered thinking process. Please structure your response into two main sections: Thought and Solution. In the Thought section, detail your reasoning process using the specified format: \texttt{<|begin\_of\_thought|>} \{thought with steps separated with `\textbackslash n\textbackslash n'\} \texttt{<|end\_of\_thought|>} Each step should include detailed considerations such as analisying questions, summarizing relevant findings, brainstorming new ideas, verifying the accuracy of the current steps, refining any errors, and revisiting previous steps. In the Solution section, based on various attempts, explorations, and reflections from the Thought section, systematically present the final solution that you deem correct. The solution should remain a logical, accurate, concise expression style and detail necessary step needed to reach the conclusion, formatted as follows: \texttt{<|begin\_of\_solution|>}  \{final formatted, precise, and clear solution\} \texttt{<|end\_of\_thought|>}  Now, try to solve the following question through the above guidelines:

\end{tcolorbox}
}

\caption{System prompt for \texttt{Vanilla}, \texttt{H2O}, \texttt{SepLLM}, \texttt{AnLLM}, and \texttt{\ours} for both Qwen-based model and Llama-based model.}
\label{prompt:system:vanilla}
\end{figure*}

\begin{figure*}[!htbp]
\centering
\scalebox{1}{
\begin{tcolorbox}
\textbf{GSM8K/MMLU/GPQA/BBH:}

Return your final response within \textbackslash boxed\{\}. \textcolor{violet}{\texttt{\{Question\}}}

\end{tcolorbox}
}

\caption{Task prompt for \texttt{Distill-R1}, \texttt{Vanilla}, \texttt{H2O}, \texttt{SepLLM}, \texttt{AnLLM}, and \texttt{\ours} for both Qwen-based model and Llama-based model.}
\label{prompt:task:r1}
\end{figure*}

\begin{figure*}[!htbp]
\centering
\scalebox{1}{
\begin{tcolorbox}
\textbf{GSM8K:}

Return your final response within \textbackslash boxed\{\}. \textcolor{violet}{\texttt{\{Question\}}}\\
\\
\textbf{MMLU:}

Please select the option that best answers the question. Return your final response within \textbackslash boxed\{\}. Here are the Question:\\
\textcolor{violet}{\texttt{\{Question\}}}\\

\textbf{GPQA:}

Given a question, please select the option that best answers it. Return your final response within \textbackslash boxed\{\}.
\textcolor{violet}{\texttt{\{Question\}}}\\


\textbf{BBH:}

Return your final response within \textbackslash boxed\{\}. If options are provided, please select the correct one. \textcolor{violet}{\texttt{\{Question\}}}

\end{tcolorbox}
}

\caption{Task prompt for \texttt{Qwen2.5-7B-Instruct} and \texttt{Llama3.1-8B-Instruct}.}
\label{prompt:task:base}
\end{figure*}

\begin{figure*}[!htbp]
\centering
\scalebox{1}{
\begin{tcolorbox}
\textbf{System Prompt:}

See Figure~\ref{prompt:system:vanilla}
\\

\textbf{Input:}

Return your final response within \textbackslash boxed\{\}. The average of the numbers 23 and $x$ is 27. What is the positive difference between 23 and $x$? (Recall that the average of two numbers is their sum divided by 2).
\\

\textbf{Output:}

\texttt{<|begin\_of\_thought|>}
Okay, let's see. The problem says that the average of the numbers 23 and x is 27. We need to find the positive difference between 23 and x. Hmm, right. Since the average of two numbers is their sum divided by 2, I can start by setting up the equation for the average.\textcolor{red}{\texttt{<w>}}So, the average of 23 and x is (23 + x)/2. According to the problem, this average equals 27. So I can write the equation:\textcolor{red}{\texttt{<w>}}(23 + x)/2 = 27\textcolor{red}{\texttt{<w>}}To find x, I should solve this equation. Let me multiply both sides by 2 to get rid of the denominator. That gives:\textcolor{red}{\texttt{<w>}}23 + x = 54\textcolor{red}{\texttt{<w>}}Then, subtract 23 from both sides to isolate x:\textcolor{red}{\texttt{<w>}}x = 54 - 23\textcolor{red}{\texttt{<w>}}Calculating that, 54 minus 23 is 31. So x is 31. Now, the question asks for the positive difference between 23 and x. The positive difference would be the absolute value of 23 - x. Since x is 31, that's |23 - 31|, which is | -8 |, so 8. Alternatively, it could also be |31 - 23|, which is the same result, 8. Either way, the positive difference is 8.\textcolor{red}{\texttt{<w>}}Wait, let me check my steps again to make sure I didn't make a mistake. Starting with the average: (23 + x)/2 = 27. Multiply both sides by 2: 23 + x = 54. Subtract 23: x = 31. Difference between 23 and 31 is 8. That seems right. Yeah, I think that's correct. No issues here.\textcolor{red}{\texttt{<w>}}
\texttt{<|end\_of\_thought|>}

\texttt{<|begin\_of\_solution|>}
\textcolor{red}{\texttt{<w>}}The average of 23 and \( x \) is given by \(\frac{23 + x}{2} = 27\). Multiplying both sides by 2:  
\[
23 + x = 54
\]  
Subtracting 23 from both sides:  
\[
x = 54 - 23 = 31
\]  
The positive difference between 23 and \( x \) is:  
\[
|31 - 23| = 8
\]  \textcolor{red}{\texttt{<w>}}\(\boxed{8}\)
\texttt{<|end\_of\_solution|>}

\end{tcolorbox}
}

\caption{A training case. \texttt{<w>} is a compression trigger.}
\label{prompt:case:train}
\end{figure*}
\begin{table*}[t]
\centering
\caption{A case study examining \model's response to a query about ``\emph{graders}'' in OpenAI's 12-day edu    cational series demonstrates its video retrieval capabilities and knowledge synthesis.}
\label{tab:case study}
\small
\begin{tabularx}{\textwidth}{X X}

\toprule
\textbf{Query:} Explain the purpose and functionality of 'graders' in the context of reinforcement fine-tuning.\\

\midrule
\textbf{Input Videos:} \href{https://www.youtube.com/playlist?list=PLOXw6I10VTv9lin5AzsHAHCTrC7BdVdEM}{12 Days of OpenAI} (A total of 12 videos) \\

\midrule

\textbf{\model\ Response:} 

\texttt{\#\#\# Understanding Gradation in Reinforcement Fine-Tuning}

\texttt{**Purpose of Graders**}

\texttt{Graders are integral components in the process of reinforcement fine-tuning, mainly utilized to evaluate and improve the performance of machine learning models...} (Omitted for brevity)

\texttt{**Functionality of Graders**}

\texttt{\colorbox{lightgreen}{The grading system operates by comparing a model's outputs against those of}}

\texttt{\colorbox{lightgreen}{verified correct responses. The mechanics of this assessment involve assigning}}

\texttt{\colorbox{lightgreen}{scores to the model's outputs based on their correctness. These scores range}}

\texttt{\colorbox{lightgreen}{from zero to one, where:}}


\texttt{\colorbox{lightblue}{- A score of **0** indicates that the model's answer is completely incorrect.}}

\texttt{\colorbox{lightblue}{- A score of **1** signals that the answer is fully correct.}}

\texttt{- Scores can also be fractional, allowing for partial credit. For instance, \colorbox{lightpink}{a score of **0.7** may denote that the model's suggestion was }}
\texttt{\colorbox{lightpink}{correct but not the highest in a ranked list of possible answers.}} 

\\

\texttt{The process includes:}

\texttt{1. **Input Comparison**: The grader takes both the model’s output and the correct answer, analyzing them to derive a score.}

\texttt{2. **Graded Outputs**: This interaction allows for a dynamic reflection of the model's performance, where the scores directly influence how the model is fine-tuned in subsequent training iterations.}

\texttt{3. **Feedback Loop**: The scores obtained from graders provide feedback that can be used to adjust learning parameters and improve the model’s future responses. This highlights the feedback mechanism that is crucial for reinforcement learning.}

\\

(The remaining content are omitted for brevity.)

\\

\midrule
\textbf{Retrieved Sources by \model:}

\model\ accurately retrieved four continuous video segments from the OpenAI's 12-day show, ranging from 10:00 to 12:00 in Day 2 of the video "\href{https://www.youtube.com/watch?v=yCIYS9fx56U}{Reinforcement Fine-Tuning}." Here, we highlight key moments relevant to the detailed content in the answer. From left to right, these are retrieved moments at timestamps \colorbox{lightgreen}{10:35}, \colorbox{lightblue}{10:39}, and \colorbox{lightpink}{11:10}, which provide informative insights that help \model\ give a comprehensive answer to the query.

\\

\begin{tabular}{ccc}
    {\includegraphics[width=0.3\textwidth]{figs/openai-1.png}} &
    {\includegraphics[width=0.3\textwidth]{figs/openai-2.png}} &
    {\includegraphics[width=0.3\textwidth]{figs/openai-3.png}} \\
\end{tabular}

\\

\bottomrule

\end{tabularx}
\vspace{-0.2in}
\end{table*}


% \section{Future Work}
