
\clearpage
\appendix

\section*{Appendix}
\section{Metric: \texttt{Dependency}}
\label{sec:app:metric}
\begin{figure}[!htbp] % !t, !htbp
    \centering
    \scalebox{1}{
    \includegraphics[width=1.0\linewidth]{figures/metric-cal-v1.pdf} 
    }
    % \vspace{-1mm}
    \caption{Illustration of the metric Dependency.}
    \label{fig:app:metric}
    % \vspace{-3mm}
\end{figure} 
\subsection{Motivation}
\ours~and AnLLM~\citep{acl24_anllm} are dynamic compression methods, meaning the number of compressions and the compression ratio are determined by the LLM itself rather than being predefined hyperparameters. 
In contrast, H2O~\citep{nips23_h2o} and SepLLM~\citep{arxiv24_sepllm} allow users to set hyperparameters to control the maximum number of tokens retained during inference. 
This fundamental difference makes it challenging to directly and fairly compare dynamic compression methods like \ours~and AnLLM with KV cache compression approaches like H2O and SepLLM.  

Traditionally, KV cache compression methods are compared by setting the same maximum peak token count, but this metric becomes inadequate in our context. 
As illustrated in Figure~\ref{fig:app:metric}, which shows the relationship between generated tokens and context length for Vanilla, H2O, and \ours, 
\ours~occasionally exceeds H2O in peak token count. 
However, this metric is misleading because \ours's peak memory usage occurs only momentarily, while H2O maintains a consistently high token count over time.  

Moreover, previous KV cache compression methods often compress prompt parts only and assume a fixed prompt length, allowing compression ratios to be predefined. 
In our setting, however, the output is also needed to be compressed.
The output token count is unknown, making it impossible to preset a global compression ratio. 
Consequently, relying solely on maximum peak token count as a comparison metric is insufficient.  

To address these challenges, we propose a new metric called \textit{Dependency}, which quantifies the total amount of information dependencies during the generation process. 
This metric enables fair comparisons between dynamic compression methods and traditional KV cache compression approaches by ensuring evaluations are conducted under similar effective compression ratios.  
\subsection{Definition}
We introduce the \textbf{Dependency} (abbr., Dep) metric, defined as the sum of dependencies of each generated token on previous tokens during the generation of an output. 
Geometrically, it represents the area under the curve in Figure~\ref{fig:app:metric}. 
Dependency can be calculated either from its definition or through its geometric interpretation. 
Here, we focus on the geometric approach. 
Let the initial prompt length be \( L_P \), the model's output length be \( L_O \), and the maximum context length set by KV cache compression methods be \( L_C \).  

\textbf{Dependency for Vanilla.}
The area under Vanilla's curve forms a right trapezoid, calculated as:  
\[
\begin{aligned}
\texttt{Dependency} &= \frac{(L_P + L_P + L_O) \times L_O}{2} \\
&= \frac{{L_O}^2}{2} + L_P \times L_O
\end{aligned}
\]

\textbf{Dependency for H2O.}
The area under H2O's curve consists of a trapezoid (left part in Figure~\ref{fig:app:metric}(b)) and a rectangle (right part in Figure~\ref{fig:app:metric}(b)):  
\[
\begin{aligned}
S_\texttt{Trapezoid} &= \frac{(L_P + L_C) \times (L_C - L_P)}{2} \\
S_\texttt{rectangle} &= L_C \times (L_O - L_C + L_P) \\
\texttt{Dependency} &= S_\texttt{Trapezoid} + S_\texttt{rectangle} \\
&= \frac{2L_PL_C + 2L_OL_C - {L_P}^2 - {L_C}^2}{2}
\end{aligned}
\]

\textbf{Dependency for \ours~and AnLLM.}
For \ours~and AnLLM, Dependency does not have a closed-form solution and must be computed iteratively based on its definition. 

\subsection{Application}
\textbf{Value of Dependency.}
A higher Dependency value indicates that more tokens need to be considered during generation, reflecting greater information usage.
Conversely, a lower Dependency value suggests a higher effective compression ratio.  

\textbf{Dependency Ratio.}
By dividing the Dependency of an accelerated method by that of Vanilla, we obtain the compression ratio relative to Vanilla. For example, in Table~\ref{table:exp_main}'s ``Avg.'' column, Vanilla's Dependency is 16.6M, H2O's is 4.4M, and \ours's is 3.7M. 
Thus, H2O achieves a compression ratio of \( \frac{16.6}{4.4} \approx 3.8 \), while \ours~achieves \( \frac{16.6}{3.7} \approx 4.5 \).  

This metric provides a unified framework for evaluating both dynamic and static compression methods, ensuring fair and meaningful comparisons.

% \section{Inference Pseudocode}


\begin{figure}[!htbp] % !t, !htbp
    \centering
    \scalebox{1}{
    \includegraphics[width=1.0\linewidth]{figures/attention_mode-v1.pdf} 
    }
    % \vspace{-1mm}
    \caption{Illustration of Attention Mask in Table~\ref{table:exp:ablation:attention}.}
    \label{fig:app:attention_mode}
    % \vspace{-3mm}
\end{figure} 

\section{Experiment}
The code and data will be released at \url{https://github.com/zjunlp/LightThinker}.
\label{sec:app:exp}
\subsection{Training Data}
\label{sec:app:exp:train_data_case}
Examples of training samples are shown in Figure~\ref{prompt:case:train}.

\subsection{Baseline Details}
\label{sec:app:exp:baseline_details}
\textbf{H2O}~\citep{nips23_h2o} is a training-free acceleration method that greedily retains tokens with the highest cumulative attention values from historical tokens. 
It includes two hyper-parameters: the maximum number of tokens and the current window size (i.e., \texttt{local\_size}). 
The maximum number of tokens for each task is listed in the ``Peak'' column of Table~\ref{table:exp_main}, and the \texttt{local\_size} is set to half of the maximum number of tokens. 
The experimental code is implemented based on \url{https://github.com/meta-llama/llama-cookbook}.

\textbf{SepLLM}~\citep{arxiv24_sepllm} is another training-free acceleration method that considers tokens at punctuation positions as more important. 
It includes four parameters: the maximum number of tokens is set to 1024, \texttt{local\_size} is set to 256, \texttt{sep\_cache\_size} is set to 64, and \texttt{init\_cache\_size} is set to 384. 
We also tried another set of parameters (\texttt{init\_cache\_size}=4, \texttt{sep\_cache\_size}=64, \texttt{local\_size}=720, maximum number of tokens=1024), but found that the first set of parameters performed slightly better.

\textbf{AnLLM}~\citep{acl24_anllm} is a training-based method that shares a similar overall approach with \ours~but accelerates by saving historical content in anchor tokens. 
The specific differences between the two are detailed in Section~\ref{sec:method:discussion}.

\subsection{Training Details}
\label{sec:app:exp:training_details}
Both \textbf{Vanilla} and \textbf{AnLLM} are trained on the B17K~\citep{bespoke_stratos_train_dataset} dataset using the R1-Distill~\citep{arxiv25_deepseek_r1} model for 5 epochs, while \ours~is trained for 6 epochs. 
The maximum length is set to 4096, and a cosine warmup strategy is adopted with a \texttt{warmup\_ratio} of 0.05. 
Experiments are conducted on 4 A800 GPUs with DeepSpeed ZeRo3 offload enabled. 
The batch size per GPU is set to 5, and the gradient accumulation step is set to 4, resulting in a global batch size of 80. 
The learning rate for Vanilla is set to 1e-5, while for AnLLM and \ours, it is set to 2e-5.

\subsection{Evaluation Details}
\label{sec:app:exp:evaluation_details}
For the CoT in Table~\ref{table:exp_main}, the prompts used are shown in Figure~\ref{prompt:system:base} and Figure~\ref{prompt:task:base}. 
For the R1-Distill model, no system prompt is used, and the task-specific prompts are shown in Figure~\ref{prompt:task:r1}. 
Vanilla, H2O, SepLLM, AnLLM, and \ours~share the same set of prompts, with the system prompt shown in Figure~\ref{prompt:system:vanilla} and downstream task prompts shown in Figure~\ref{prompt:task:r1}. 
The options for MMLU~\citep{iclr21_mmlu} and GPQA~\citep{colm24_gpqa} multiple-choice questions are randomized.

\subsection{Additional Results}
\label{sec:app:exp:additional_results}
Figure~\ref{fig:app:token} compares the number of tokens generated by two models across different datasets. 
Figure~\ref{fig:app:frequency} shows the distribution of compressed lengths for LightThinker on two models and four datasets. Figure~\ref{fig:app:attention_mode} illustrates the attention masks for the baselines in Table~\ref{table:exp:ablation:attention}.
Figure~\ref{prompt:case} shows a complete case in Figure~\ref{fig:exp:case}.

\begin{figure*}[!htbp] % !t, !htbp
    \centering
    \scalebox{0.8}{
    \includegraphics[width=1.0\linewidth]{figures/token_split-v2.pdf} 
    }
    % \vspace{-1mm}
    \caption{Average number of generated tokens.}
    \label{fig:app:token}
    % \vspace{-3mm}
\end{figure*} 

\section{RELATED WORK}
\label{sec:relatedwork}
In this section, we describe the previous works related to our proposal, which are divided into two parts. In Section~\ref{sec:relatedwork_exoplanet}, we present a review of approaches based on machine learning techniques for the detection of planetary transit signals. Section~\ref{sec:relatedwork_attention} provides an account of the approaches based on attention mechanisms applied in Astronomy.\par

\subsection{Exoplanet detection}
\label{sec:relatedwork_exoplanet}
Machine learning methods have achieved great performance for the automatic selection of exoplanet transit signals. One of the earliest applications of machine learning is a model named Autovetter \citep{MCcauliff}, which is a random forest (RF) model based on characteristics derived from Kepler pipeline statistics to classify exoplanet and false positive signals. Then, other studies emerged that also used supervised learning. \cite{mislis2016sidra} also used a RF, but unlike the work by \citet{MCcauliff}, they used simulated light curves and a box least square \citep[BLS;][]{kovacs2002box}-based periodogram to search for transiting exoplanets. \citet{thompson2015machine} proposed a k-nearest neighbors model for Kepler data to determine if a given signal has similarity to known transits. Unsupervised learning techniques were also applied, such as self-organizing maps (SOM), proposed \citet{armstrong2016transit}; which implements an architecture to segment similar light curves. In the same way, \citet{armstrong2018automatic} developed a combination of supervised and unsupervised learning, including RF and SOM models. In general, these approaches require a previous phase of feature engineering for each light curve. \par

%DL is a modern data-driven technology that automatically extracts characteristics, and that has been successful in classification problems from a variety of application domains. The architecture relies on several layers of NNs of simple interconnected units and uses layers to build increasingly complex and useful features by means of linear and non-linear transformation. This family of models is capable of generating increasingly high-level representations \citep{lecun2015deep}.

The application of DL for exoplanetary signal detection has evolved rapidly in recent years and has become very popular in planetary science.  \citet{pearson2018} and \citet{zucker2018shallow} developed CNN-based algorithms that learn from synthetic data to search for exoplanets. Perhaps one of the most successful applications of the DL models in transit detection was that of \citet{Shallue_2018}; who, in collaboration with Google, proposed a CNN named AstroNet that recognizes exoplanet signals in real data from Kepler. AstroNet uses the training set of labelled TCEs from the Autovetter planet candidate catalog of Q1–Q17 data release 24 (DR24) of the Kepler mission \citep{catanzarite2015autovetter}. AstroNet analyses the data in two views: a ``global view'', and ``local view'' \citep{Shallue_2018}. \par


% The global view shows the characteristics of the light curve over an orbital period, and a local view shows the moment at occurring the transit in detail

%different = space-based

Based on AstroNet, researchers have modified the original AstroNet model to rank candidates from different surveys, specifically for Kepler and TESS missions. \citet{ansdell2018scientific} developed a CNN trained on Kepler data, and included for the first time the information on the centroids, showing that the model improves performance considerably. Then, \citet{osborn2020rapid} and \citet{yu2019identifying} also included the centroids information, but in addition, \citet{osborn2020rapid} included information of the stellar and transit parameters. Finally, \citet{rao2021nigraha} proposed a pipeline that includes a new ``half-phase'' view of the transit signal. This half-phase view represents a transit view with a different time and phase. The purpose of this view is to recover any possible secondary eclipse (the object hiding behind the disk of the primary star).


%last pipeline applies a procedure after the prediction of the model to obtain new candidates, this process is carried out through a series of steps that include the evaluation with Discovery and Validation of Exoplanets (DAVE) \citet{kostov2019discovery} that was adapted for the TESS telescope.\par
%



\subsection{Attention mechanisms in astronomy}
\label{sec:relatedwork_attention}
Despite the remarkable success of attention mechanisms in sequential data, few papers have exploited their advantages in astronomy. In particular, there are no models based on attention mechanisms for detecting planets. Below we present a summary of the main applications of this modeling approach to astronomy, based on two points of view; performance and interpretability of the model.\par
%Attention mechanisms have not yet been explored in all sub-areas of astronomy. However, recent works show a successful application of the mechanism.
%performance

The application of attention mechanisms has shown improvements in the performance of some regression and classification tasks compared to previous approaches. One of the first implementations of the attention mechanism was to find gravitational lenses proposed by \citet{thuruthipilly2021finding}. They designed 21 self-attention-based encoder models, where each model was trained separately with 18,000 simulated images, demonstrating that the model based on the Transformer has a better performance and uses fewer trainable parameters compared to CNN. A novel application was proposed by \citet{lin2021galaxy} for the morphological classification of galaxies, who used an architecture derived from the Transformer, named Vision Transformer (VIT) \citep{dosovitskiy2020image}. \citet{lin2021galaxy} demonstrated competitive results compared to CNNs. Another application with successful results was proposed by \citet{zerveas2021transformer}; which first proposed a transformer-based framework for learning unsupervised representations of multivariate time series. Their methodology takes advantage of unlabeled data to train an encoder and extract dense vector representations of time series. Subsequently, they evaluate the model for regression and classification tasks, demonstrating better performance than other state-of-the-art supervised methods, even with data sets with limited samples.

%interpretation
Regarding the interpretability of the model, a recent contribution that analyses the attention maps was presented by \citet{bowles20212}, which explored the use of group-equivariant self-attention for radio astronomy classification. Compared to other approaches, this model analysed the attention maps of the predictions and showed that the mechanism extracts the brightest spots and jets of the radio source more clearly. This indicates that attention maps for prediction interpretation could help experts see patterns that the human eye often misses. \par

In the field of variable stars, \citet{allam2021paying} employed the mechanism for classifying multivariate time series in variable stars. And additionally, \citet{allam2021paying} showed that the activation weights are accommodated according to the variation in brightness of the star, achieving a more interpretable model. And finally, related to the TESS telescope, \citet{morvan2022don} proposed a model that removes the noise from the light curves through the distribution of attention weights. \citet{morvan2022don} showed that the use of the attention mechanism is excellent for removing noise and outliers in time series datasets compared with other approaches. In addition, the use of attention maps allowed them to show the representations learned from the model. \par

Recent attention mechanism approaches in astronomy demonstrate comparable results with earlier approaches, such as CNNs. At the same time, they offer interpretability of their results, which allows a post-prediction analysis. \par



\section{Discussions}
\label{sec:app:discussion}
\textbf{Viewing \ours~from Other Perspectives.}
In previous sections, we design \ours~from a compression perspective. 
Here, we further discuss it from the perspectives of \textit{Memory} and \textit{KV Cache Compression}, where KV Cache can be viewed as a form of LLM work memory.
In Memory perspective, \ours's workflow can be summarized as follows: it first performs autoregressive reasoning, then stores key information from the reasoning process as memory (memory), and continues reasoning based on the memorized content. 
Thus, the information in the cache tokens acts as a compact memory, though it is only effective for the current LLM and lacks transferability.
In KV Cache Compression perspective,
Unlike methods such as H2O~\citep{nips23_h2o}, which rely on manually designed eviction policy to select important tokens, \ours~merges previous tokens in a continuous space, \textit{ceating} new representations. 
The content and manner of merging are autonomously determined by the LLM, rather than being a discrete selection process.

% \citet{emnlp24_onegen}

% \citeauthor{emnlp24_onegen}

\begin{figure*}[!htbp] % !t, !htbp
    \centering
    \scalebox{1}{
    \includegraphics[width=0.8\linewidth]{figures/frequency_split.pdf} 
    }
    % \vspace{-1mm}
    \caption{Token compression frequency distribution for \ours.}
    \label{fig:app:frequency}
    % \vspace{-3mm}
\end{figure*} 


\begin{figure*}[!htbp]
\centering
\scalebox{1}{
\begin{tcolorbox}
\textbf{System Prompt:}

Below is a question. Please think through it step by step, and then provide the final answer. If options are provided, please select the correct one. 

\#\# Output format:\\
Use ``<THOUGHT>...</THOUGHT>'' to outline your reasoning process, and enclose the final answer in `\textbackslash boxed\{\}`.\\
\\
\#\# Example 1:\\
Question: \\
What is 2 + 3?\\
Output:\\
<THOUGHT>First, I recognize that this is a simple addition problem. Adding 2 and 3 together gives 5.</THOUGHT>\\
Therefore, the final answer is \textbackslash boxed\{5\}.\\
\\
\#\# Example 2:\\
Question: \\
What is 2 + 3?\\
A. 4\\
B. 5\\
C. 10\\
\\
Output:\\
<THOUGHT>First, I recognize that this is a simple addition problem. Adding 2 and 3 together gives 5.</THOUGHT>\\
Therefore, the final answer is \textbackslash boxed\{B\}.\\

\end{tcolorbox}
}

\caption{System prompt for \texttt{Qwen2.5-7B-Instruct} and \texttt{Llama3.1-8B-Instruct}.}
\label{prompt:system:base}
\end{figure*}

\begin{figure*}[!htbp]
\centering
\scalebox{1}{
\begin{tcolorbox}
\textbf{System Prompt:}

Your role as an assistant involves thoroughly exploring questions through a systematic long thinking process before providing the final precise and accurate solutions. This requires engaging in a comprehensive cycle of analysis, summarizing, exploration, reassessment, reflection, backtracing, and iteration to develop well-considered thinking process. Please structure your response into two main sections: Thought and Solution. In the Thought section, detail your reasoning process using the specified format: \texttt{<|begin\_of\_thought|>} \{thought with steps separated with `\textbackslash n\textbackslash n'\} \texttt{<|end\_of\_thought|>} Each step should include detailed considerations such as analisying questions, summarizing relevant findings, brainstorming new ideas, verifying the accuracy of the current steps, refining any errors, and revisiting previous steps. In the Solution section, based on various attempts, explorations, and reflections from the Thought section, systematically present the final solution that you deem correct. The solution should remain a logical, accurate, concise expression style and detail necessary step needed to reach the conclusion, formatted as follows: \texttt{<|begin\_of\_solution|>}  \{final formatted, precise, and clear solution\} \texttt{<|end\_of\_thought|>}  Now, try to solve the following question through the above guidelines:

\end{tcolorbox}
}

\caption{System prompt for \texttt{Vanilla}, \texttt{H2O}, \texttt{SepLLM}, \texttt{AnLLM}, and \texttt{\ours} for both Qwen-based model and Llama-based model.}
\label{prompt:system:vanilla}
\end{figure*}

\begin{figure*}[!htbp]
\centering
\scalebox{1}{
\begin{tcolorbox}
\textbf{GSM8K/MMLU/GPQA/BBH:}

Return your final response within \textbackslash boxed\{\}. \textcolor{violet}{\texttt{\{Question\}}}

\end{tcolorbox}
}

\caption{Task prompt for \texttt{Distill-R1}, \texttt{Vanilla}, \texttt{H2O}, \texttt{SepLLM}, \texttt{AnLLM}, and \texttt{\ours} for both Qwen-based model and Llama-based model.}
\label{prompt:task:r1}
\end{figure*}

\begin{figure*}[!htbp]
\centering
\scalebox{1}{
\begin{tcolorbox}
\textbf{GSM8K:}

Return your final response within \textbackslash boxed\{\}. \textcolor{violet}{\texttt{\{Question\}}}\\
\\
\textbf{MMLU:}

Please select the option that best answers the question. Return your final response within \textbackslash boxed\{\}. Here are the Question:\\
\textcolor{violet}{\texttt{\{Question\}}}\\

\textbf{GPQA:}

Given a question, please select the option that best answers it. Return your final response within \textbackslash boxed\{\}.
\textcolor{violet}{\texttt{\{Question\}}}\\


\textbf{BBH:}

Return your final response within \textbackslash boxed\{\}. If options are provided, please select the correct one. \textcolor{violet}{\texttt{\{Question\}}}

\end{tcolorbox}
}

\caption{Task prompt for \texttt{Qwen2.5-7B-Instruct} and \texttt{Llama3.1-8B-Instruct}.}
\label{prompt:task:base}
\end{figure*}

\begin{figure*}[!htbp]
\centering
\scalebox{1}{
\begin{tcolorbox}
\textbf{System Prompt:}

See Figure~\ref{prompt:system:vanilla}
\\

\textbf{Input:}

Return your final response within \textbackslash boxed\{\}. The average of the numbers 23 and $x$ is 27. What is the positive difference between 23 and $x$? (Recall that the average of two numbers is their sum divided by 2).
\\

\textbf{Output:}

\texttt{<|begin\_of\_thought|>}
Okay, let's see. The problem says that the average of the numbers 23 and x is 27. We need to find the positive difference between 23 and x. Hmm, right. Since the average of two numbers is their sum divided by 2, I can start by setting up the equation for the average.\textcolor{red}{\texttt{<w>}}So, the average of 23 and x is (23 + x)/2. According to the problem, this average equals 27. So I can write the equation:\textcolor{red}{\texttt{<w>}}(23 + x)/2 = 27\textcolor{red}{\texttt{<w>}}To find x, I should solve this equation. Let me multiply both sides by 2 to get rid of the denominator. That gives:\textcolor{red}{\texttt{<w>}}23 + x = 54\textcolor{red}{\texttt{<w>}}Then, subtract 23 from both sides to isolate x:\textcolor{red}{\texttt{<w>}}x = 54 - 23\textcolor{red}{\texttt{<w>}}Calculating that, 54 minus 23 is 31. So x is 31. Now, the question asks for the positive difference between 23 and x. The positive difference would be the absolute value of 23 - x. Since x is 31, that's |23 - 31|, which is | -8 |, so 8. Alternatively, it could also be |31 - 23|, which is the same result, 8. Either way, the positive difference is 8.\textcolor{red}{\texttt{<w>}}Wait, let me check my steps again to make sure I didn't make a mistake. Starting with the average: (23 + x)/2 = 27. Multiply both sides by 2: 23 + x = 54. Subtract 23: x = 31. Difference between 23 and 31 is 8. That seems right. Yeah, I think that's correct. No issues here.\textcolor{red}{\texttt{<w>}}
\texttt{<|end\_of\_thought|>}

\texttt{<|begin\_of\_solution|>}
\textcolor{red}{\texttt{<w>}}The average of 23 and \( x \) is given by \(\frac{23 + x}{2} = 27\). Multiplying both sides by 2:  
\[
23 + x = 54
\]  
Subtracting 23 from both sides:  
\[
x = 54 - 23 = 31
\]  
The positive difference between 23 and \( x \) is:  
\[
|31 - 23| = 8
\]  \textcolor{red}{\texttt{<w>}}\(\boxed{8}\)
\texttt{<|end\_of\_solution|>}

\end{tcolorbox}
}

\caption{A training case. \texttt{<w>} is a compression trigger.}
\label{prompt:case:train}
\end{figure*}
\section{Case Study}\label{sec:appendix_casestudy}

We present cases across multiple coding datasets, comparing compressed and original code examples. For instance, as demonstrated in Figures 8, 9, and 10, \ourtool prioritizes discarding \textbf{Invocation} tokens first, followed by \textbf{Symbol} tokens.
\begin{figure}[!h]
\begin{tcolorbox}
\begin{lstlisting}[language=Java,frame=single,framerule=0pt]
### FOCAL_METHOD 
getProduction(java.lang.String) { 
 return productionsByName.get(name); }  
### UNIT_TEST  
testJustifications() { 
 runTest("testJustifications", 2); org.jsoar.kernel.Production j = agent.getProductions() .getProduction("justification-1"); "<AssertPlaceHolder>"; 
}    
\end{lstlisting}
\end{tcolorbox}
\caption{Original Code Examples of Assertion Generation (63 tokens)}
\label{fig:code-example}
\end{figure}

\begin{figure}[!h]
\begin{tcolorbox}
\begin{lstlisting}[language=Java,frame=single,framerule=0pt]
### FOCAL_METHOD 
getProduction(java.lang.String) { 
 return productionsByName; }  
### UNIT_TEST  
testJustifications() { 
 ; 
 org.jsoar.kernel.Production j = agent.getProductions() .getProduction("justification-1"); "<AssertPlaceHolder>"; 
}    
\end{lstlisting}
\end{tcolorbox}
\caption{Compressed Code Examples of Assertion Generation (55 tokens, $\tau_{code}$: 0.1)}
\label{fig:code-example}
\end{figure}

\begin{figure}[!h]
\begin{tcolorbox}
\begin{lstlisting}[language=Java,frame=single,framerule=0pt]
### FOCAL_METHOD 
getProduction(java.lang.String)  
 return productionsByName;     
### UNIT_TEST  
testJustifications()  
 ; 
 org.jsoar.kernel.Production j = agent;
  "<AssertPlaceHolder>"; 
\end{lstlisting}
\end{tcolorbox}
\caption{Compressed Code Examples of Assertion Generation (39 tokens, $\tau_{code}$: 0.4)}
\label{fig:code-example}
\end{figure}


\begin{figure}[!h]
\begin{tcolorbox}
\begin{lstlisting}[language=Java,frame=single,framerule=0pt]
### BUGGY_CODE 
public static TYPE_1 init(java.lang.String name, java.util.Date date) {
   TYPE_1 VAR_1 = new TYPE_1();
   VAR_1.METHOD_1(name);
   java.util.Calendar VAR_2 = java.util.Calendar.getInstance();
   VAR_2.METHOD_2(date);
   VAR_1.METHOD_3(VAR_2);
   return VAR_1;
}
### FIXED_CODE   
public static TYPE_1 init(java.lang.String name, java.util.Date date) {
   TYPE_1 VAR_1 = new TYPE_1();
   VAR_1.METHOD_1(name);
   java.util.Calendar VAR_2 = null;
   if (date != null) {
       VAR_2 = java.util.Calendar.getInstance();
       VAR_2.METHOD_2(date);
   } 
   VAR_1.METHOD_3(VAR_2);
   return VAR_1;
}
\end{lstlisting}
\end{tcolorbox}
\caption{Original Code Examples of Bugs2Fix (195 tokens)}
\label{fig:code-example}
\end{figure}

\begin{figure}[!h]
\begin{tcolorbox}
\begin{lstlisting}[language=Java,frame=single,framerule=0pt]
### BUGGY_CODE 
public static TYPE_1 init(java.lang.String name, java.util.Date date) {
    = new TYPE_1();
   ;
   java.util.Calendar = java.util.Calendar;
   .METHOD_2(date);
   .METHOD_3(VAR_2);
   return ;
}
### FIXED_CODE   
public static TYPE_1 init(java.lang.String name, java.util.Date date) {
    = new TYPE_1();
   ;
   java.util.Calendar = null;
   if (date != null) {
        = java.util.Calendar;
       .METHOD_2(date);
   } 
   .METHOD_3(VAR_2);
   return ;
}
\end{lstlisting}
\end{tcolorbox}
\caption{Compressed Code Examples of Bugs2Fix (136 tokens, $\tau_{code}$: 0.3)}
\label{fig:code-example}
\end{figure}

\begin{figure}[!h]
\begin{tcolorbox}
\begin{lstlisting}[language=Java,frame=single,framerule=0pt]
### METHOD_HEADER 
protected final void fastPathEmit ( U value , boolean delayError , Disposable dispose )
### WHOLE_METHOD  
protected final void fastPathEmit(U value, boolean delayError, Disposable dispose) {
   final Observer<? super V> s = actual;
   final SimplePlainQueue<U> q = queue;
   if (wip.get() == 0 && wip.compareAndSet(0, 1)) {
       accept(s, value);
       if (leave(-1) == 0) {
           return;
       }
   } else {
       q.offer(value);
       if (!enter()) {
           return;
       }
   }
   QueueDrainHelper.drainLoop(q, s, delayError, dispose, this);
}
\end{lstlisting}
\end{tcolorbox}
\caption{Original Code Examples  of \taskthree (157 tokens, $\tau_{code}$: 0.3)}
\label{fig:code-example}
\end{figure}


\begin{figure}[!h]
\begin{tcolorbox}
Original Code Examples (121 tokens, $\tau_{code}$: 0.3)
\begin{lstlisting}[language=Java,frame=single,framerule=0pt]
### METHOD_HEADER 
protected final void fastPathEmit ( U value , boolean delayError , Disposable dispose )
### WHOLE_METHOD  
   final Observer<? super V> = 
   final SimplePlainQueue<U> = 
   if (wip.get() == 0 && wip.compareAndSet(0, 1)) 
       ;
       if (leave(-1) == 0) 
           return;    
    else 
       .offer(value);
       if (!enter()) 
           return;
   .drainLoop(q, s, delayError, dispose, this);
\end{lstlisting}
\end{tcolorbox}
\caption{Compressed Code Examples  of \taskthree (121 tokens, $\tau_{code}$: 0.3)}
\end{figure}





% \section{Future Work}
