% ICCV 2025 Paper Template

\documentclass[10pt,twocolumn,letterpaper]{article}

%%%%%%%%% PAPER TYPE  - PLEASE UPDATE FOR FINAL VERSION
% \usepackage{iccv}              % To produce the CAMERA-READY version
% \usepackage[review]{iccv}      % To produce the REVIEW version
\usepackage[pagenumbers]{iccv} % To force page numbers, e.g. for an arXiv version

% Import additional packages in the preamble file, before hyperref
%
% --- inline annotations
%
\newcommand{\red}[1]{{\color{red}#1}}
\newcommand{\todo}[1]{{\color{red}#1}}
\newcommand{\TODO}[1]{\textbf{\color{red}[TODO: #1]}}
% --- disable by uncommenting  
% \renewcommand{\TODO}[1]{}
% \renewcommand{\todo}[1]{#1}



\newcommand{\VLM}{LVLM\xspace} 
\newcommand{\ours}{PeKit\xspace}
\newcommand{\yollava}{Yo’LLaVA\xspace}

\newcommand{\thisismy}{This-Is-My-Img\xspace}
\newcommand{\myparagraph}[1]{\noindent\textbf{#1}}
\newcommand{\vdoro}[1]{{\color[rgb]{0.4, 0.18, 0.78} {[V] #1}}}
% --- disable by uncommenting  
% \renewcommand{\TODO}[1]{}
% \renewcommand{\todo}[1]{#1}
\usepackage{slashbox}
% Vectors
\newcommand{\bB}{\mathcal{B}}
\newcommand{\bw}{\mathbf{w}}
\newcommand{\bs}{\mathbf{s}}
\newcommand{\bo}{\mathbf{o}}
\newcommand{\bn}{\mathbf{n}}
\newcommand{\bc}{\mathbf{c}}
\newcommand{\bp}{\mathbf{p}}
\newcommand{\bS}{\mathbf{S}}
\newcommand{\bk}{\mathbf{k}}
\newcommand{\bmu}{\boldsymbol{\mu}}
\newcommand{\bx}{\mathbf{x}}
\newcommand{\bg}{\mathbf{g}}
\newcommand{\be}{\mathbf{e}}
\newcommand{\bX}{\mathbf{X}}
\newcommand{\by}{\mathbf{y}}
\newcommand{\bv}{\mathbf{v}}
\newcommand{\bz}{\mathbf{z}}
\newcommand{\bq}{\mathbf{q}}
\newcommand{\bff}{\mathbf{f}}
\newcommand{\bu}{\mathbf{u}}
\newcommand{\bh}{\mathbf{h}}
\newcommand{\bb}{\mathbf{b}}

\newcommand{\rone}{\textcolor{green}{R1}}
\newcommand{\rtwo}{\textcolor{orange}{R2}}
\newcommand{\rthree}{\textcolor{red}{R3}}
\usepackage{amsmath}
%\usepackage{arydshln}
\DeclareMathOperator{\similarity}{sim}
\DeclareMathOperator{\AvgPool}{AvgPool}

\newcommand{\argmax}{\mathop{\mathrm{argmax}}}     


\def \cls{[\textsc{Cls}]}
\usepackage{multirow}
\usepackage[table]{xcolor}
% \usepackage{colortbl}
% \usepackage{makecell}
% It is strongly recommended to use hyperref, especially for the review version.
% hyperref with option pagebackref eases the reviewers' job.
% Please disable hyperref *only* if you encounter grave issues, 
% e.g. with the file validation for the camera-ready version.
%
% If you comment hyperref and then uncomment it, you should delete *.aux before re-running LaTeX.
% (Or just hit 'q' on the first LaTeX run, let it finish, and you should be clear).
\definecolor{iccvblue}{rgb}{0.21,0.49,0.74}
\usepackage[pagebackref,breaklinks,colorlinks,allcolors=iccvblue]{hyperref}

%%%%%%%%% PAPER ID  - PLEASE UPDATE
\def\paperID{9515} % *** Enter the Paper ID here
\def\confName{ICCV}
\def\confYear{2025}

%%%%%%%%% TITLE - PLEASE UPDATE
\title{
Rethinking the Global Knowledge of CLIP\\ in Training-Free Open-Vocabulary Semantic Segmentation}

%%%%%%%%% AUTHORS - PLEASE UPDATE
\author{Jingyun Wang\\
Beihang University\\
{\tt\small wangjingyun0730@gmail.com}
% For a paper whose authors are all at the same institution,
% omit the following lines up until the closing ``}''.
% Additional authors and addresses can be added with ``\and'',
% just like the second author.
% To save space, use either the email address or home page, not both
\and
Cilin Yan\\
Beihang University\\
{\tt\small clyanhh@gmail.com}
\and
Guoliang Kang~\thanks{Corresponding author}\\
Beihang University\\
{\tt\small kgl.prml@gmail.com}
}

\begin{document}
\maketitle
\begin{abstract}
Recent works modify CLIP to perform open-vocabulary semantic segmentation in a training-free manner (TF-OVSS).
In vanilla CLIP, patch-wise image representations mainly encode homogeneous image-level properties, 
which hinders the application of CLIP to the dense prediction task.
Previous TF-OVSS works sacrifice globality 
to enhance the locality of CLIP features, 
by making each patch mainly attend to itself or its neighboring patches within a narrow local window.
With their modifications,
the ability of CLIP to aggregate global context information 
is largely weakened.
Differently, in this paper, we rethink the global knowledge encoded by CLIP and propose GCLIP to answer how to extract and utilize beneficial global knowledge of CLIP for TF-OVSS.
As the representation of each patch is finally determined by the attention weights and the Value embeddings, we propose to reshape the last-block attention and Value embeddings to aggregate useful global context into final features. 
Firstly, we aim to equip the last-block attention with image-level properties while not introducing homogeneous attention patterns across patches. 
To realize the goal, we fuse the attention from the global-token emerging blocks with the Query-Query attention.
Secondly, we aim to make Value embeddings of the last-block attention module more semantically correlated. To realize this, we design a novel channel suppression strategy.
Extensive experiments on five standard benchmarks demonstrate that our method consistently outperforms previous state-of-the-arts.
\end{abstract}

\begin{figure*}
  \centering
  \includegraphics[width=1\linewidth]{introduction.pdf}
  % \vspace{-4mm}
  \caption{Experiments with CLIP ViT-B/16.
  (a) \textbf{Emergence of global tokens (best viewed in color).} 
  Global tokens (highlight stripes in Line 1) start to emerge from the attention map of block 6. 
  Comparing the attention maps after block 6, we observe the attention pattern of global tokens aligns well with that of the \cls{} token (Line 2\&3).
  (b) \textbf{Channel Suppression (CS).}
  We observe the entropy of weight norms decreases abnormally from block 7 in (2).
  With CS on the abnormal weight norm of the second fully-connected layer of FFN in a Transformer block (See (3)), we enhance the semantic correlation by making value embeddings of patches within the same semantic mask become more similar (``in-in'') but those from different masks become more dissimilar (``in-out'').}

\label{introduction_fig}
\end{figure*}

\section{Introduction}
\label{sec:intro}
Semantic segmentation aims to assign a semantic label to each pixel within an image. 
With the rise of deep learning~\cite{FCN, Deeplabv1, Deeplabv2, PSPNet, Mask2Former, MaskFormer}, semantic segmentation performance has been dramatically improved, but still relies on close-set training covering a limited number of categories.
In real world, there are a large number of open-vocabulary classes that are not seen during training and the closed-set semantic segmentation methods may not be able to make predictions for them.
To deal with open-vocabulary semantic segmentation (OVSS) problem, many methods~\cite{GroupViT, TCL, CoCu, ODISE, OVSeg, OpenSeg, SimSeg, LSeg, CATSeg} have been developed and exhibit superior generalization ability to unseen categories. 
However, most of OVSS methods still heavily rely on time-consuming training with large-scale image-caption pairs or class-agnostic masks, which hinders the application of OVSS methods in practice.

Recent works modify large-scale vision-language pre-trained model  CLIP~\cite{CLIP} to perform OVSS in a \textit{training-free} manner. 
Though CLIP demonstrates superior zero-shot performance for image classification task, it cannot be directly applied to OVSS, as the patch-wise representation of CLIP tends to encode homogeneous image-level properties, 
hindering pixel-level prediction. %thus lacking enough semantic coherence.
Previous methods for TF-OVSS~\cite{MaskCLIP, ClearCLIP, CLIPSurgery, CLIPtrase, SCLIP} view global knowledge of CLIP as harmful for segmentation.
They modify the attention mechanism in the final block of CLIP, which encourages each patch to primarily focus on itself or the neighboring patches within a narrow local window.
Though image features are more distinct across patches, the CLIP's ability to aggregate global context information, which is known to be useful in conventional semantic segmentation practice~\cite{PSPNet} for distinguishing confusing categories, is significantly weakened.
As a result, the segmentation performance of these works is largely constrained.

In this paper, we rethink the global knowledge encoded by CLIP and propose GCLIP to mine and emphasize the beneficial global knowledge of CLIP for TF-OVSS task.
%Inspired by ClearCLIP~\cite{ClearCLIP}, our method enhances the distinctness of patch-wise representations via altering the last-block Query-Key attention to Query-Query attention and discarding the last-block FFN and the residual outputs from other blocks.
%Beyond the distinctness enhancement, we make two simple yet effective modifications to the last-block attention and Value embeddings respectively to emphasize the beneficial global knowledge of CLIP.
As the representation of each patch is finally determined by the attention weights and the Value embeddings, we make modifications to the last-block attention and Value embeddings respectively to aggregate useful global context information into final features. % beyond ClearCLIP.
Firstly, we propose an Attention Map Fusion strategy (AMF) to emphasize global knowledge by reshaping last-block attention.
As shown in Figure~\ref{introduction_fig} (a), we observe that \textit{global tokens} exist in deeper blocks of CLIP. %The term ``global tokens'' means specific patches, whose corresponding Query-Key attention weight for all patches is super high. 
The term ``global token'' means a specific patch is important (\emph{i.e.,} corresponding Query-Key attention weight is high) for all the other patches. 
Interestingly, we find the attention pattern of global token aligns well with that of \cls{} token (see Figure~\ref{introduction_fig} (a)),
which indicates those global tokens may
encode the image-level properties as [CLS] token. Based
on such observations, we propose AMF to average the attention maps from global-token emerging blocks and the final-block Query-Query attention to form a new final-block attention.
%\textcolor{red}{We hypothesize that global tokens may encode the image-level global properties as \cls{} token, and provide evidence in Table~\ref{table_ablation_classification}.}
Therefore, through AMF, we emphasize the global knowledge encoded by global tokens. % by reshaping the last-block attention.

Secondly, we propose a Channel Suppression (CS) strategy to make last-block Value embeddings more semantically correlated, which means the similarity between Value embeddings can reflect their semantic correlation.
In vanilla CLIP, we observe that the same channel in different Value embeddings has super large activation, rendering Value embeddings across patches unexpectedly similar.
%different patch representations are homogeneously activated in the same channel without semantic coherence, 
This is due to an abnormal phenomenon that exists in the weights of the second fully-connected layer of FFN in a Transformer block.
In detail, the weight norm corresponding to some specific output channels becomes unexpectedly larger than the weight norm of other channels, which can be reflected by the entropy of those weight norms (Figure~\ref{introduction_fig} (b)(2)).
Thus, we propose to suppress abnormal weight norm of FFN (see Figure~\ref{introduction_fig} (b)(3)) so that the semantic correlation of Value embeddings can be enhanced. 
%After the suppression, we observe the similarity between Value embeddings of the last-attention module and global tokens is reduced in (1) of Figure~\ref{introduction_fig} (b). 
With CS, as shown in Figure~\ref{introduction_fig} (b)(1), we observe the Value embeddings of patches within the same semantic mask become more similar (see ``in-in'' comparison) while those from different masks become more dissimilar (see ``in-out'' comparison).
Since the representation of each patch is finally determined by the attention weights and the Value embeddings, we finally generate more semantically correlated patch-level image features while also absorbing global context.

We conduct extensive experiments on five standard semantic segmentation benchmarks, including PASCAL VOC~\cite{PASCAL_VOC}, PASCAL Context~\cite{PASCAL_Context}, ADE20K~\cite{ADE20K}, Cityscapes~\cite{cityscapes} and COCO Stuff~\cite{cocostuff}. 
Experiment results demonstrate that GCLIP consistently outperforms previous state-of-the-arts.
Notably, on Cityscapes, our method outperforms 
ClearCLIP~\cite{ClearCLIP} by 3.7\% mIoU. 
Extensive ablation studies verify the effectiveness of each design.

In a nutshell, our contributions are summarized as 
\begin{itemize}[leftmargin=2em]
\item 
We propose an Attention Map Fusion strategy (AMF) to emphasize the global knowledge encoded by global tokens via reshaping the last-block attention.

\item 
We propose a Channel Suppression strategy (CS) to make last-block Value embeddings more semantically correlated.

\item 
We conduct extensive experiments on various segmentation benchmarks under the training-free open-vocabulary setting. Experiment results show that GCLIP outperforms previous state-of-the-arts. 
\end{itemize}


\section{Related Work}
\label{sec:related_work}

\noindent \textbf{Pre-trained vision-language models}
Pre-trained vision-language models (VLMs)~\cite{Uniter, Virtex, Unicoder-vl, AlignBeforeUse, Hero} have experienced rapid development, thanks to the abundant large-scale image-text pairs accessible on the Internet.
Recently, CLIP~\cite{CLIP}, ALIGN~\cite{ALIGN} and Slip~\cite{Slip} have made great progress on learning visual and textual representations jointly by using contrastive learning. 
Among these, CLIP trained on WIT-400M exhibits robust zero-shot capability for image classification task, due to its image-level alignment with text.
However, directly applying CLIP to dense prediction tasks, such as object detection and semantic segmentation, results in suboptimal performance.
A series of methods~\cite{DetPro,UniDetector, MaskCLIP,GroupViT,TCL,ReCLIP} have successfully adapted CLIP for various downstream tasks and this paper specifically addresses the adaptation of CLIP for the task of training-free open-vocabulary semantic segmentation.

\noindent \textbf{Open-vocabulary semantic segmentation (OVSS)}
OVSS refers to segmenting an image with arbitrary categories under the guidance of a textual description.
Among these, fully supervised OVSS~\cite{OpenSeg,LSeg,ODISE,OVSeg,CATSeg} methods still rely on high-quality pixel-level annotated masks.
Usually, they generate mask proposals by an extra mask generator, \emph{e.g.}, Mask2Former~\cite{Mask2Former}, and further align the visual embeddings with the textual features.
Most methods extract visual features by CLIP, while ODISE leverages the internal representations of pre-trained Diffusion models~\cite{Diffusion}.
Methods for fully supervised OVSS usually train on a large-scale dataset equipped with fully annotated masks, like COCO Stuff~\cite{cocostuff}, and directly perform zero-shot inference on other datasets that may contain unseen categories during the training process.
There also exists a set of OVSS methods~\cite{CoDe,GroupViT,viewco,CoCu},which mainly exploit large-scale image-caption pairs, such as CC12M~\cite{cc12m} and YFCC~\cite{yfcc}, for training.
For example, GroupViT~\cite{GroupViT} introduces grouping tokens into the vision transformer and conducts hierarchical clustering for segmentation.
It finally obtains an image-level feature, which is then aligned with textual features by contrastive learning loss.

\noindent \textbf{Training-free open-vocabulary semantic segmentation}
% GEM, CLIPSurgery显式提一下
Methods for TF-OVSS~\cite{MaskCLIP,CLIPSurgery,SCLIP,CLIPtrase,ClearCLIP} adopt CLIP for OVSS without any training.
Existing works explore to enhance the distinction across the patch-wise visual features from CLIP mainly by modifying the attention mechanism in its final block, which forces each patch to primarily focus on itself and the neighbors in a narrow local window.
%Among these, MaskCLIP~\cite{MaskCLIP} directly replace the Query-Key attention map with an identical matrix, while others~\cite{ClearCLIP,CLIPSurgery,CLIPtrase,SCLIP,GEM} employ a self-self attention mechanism.
For example, CLIPSurgery~\cite{CLIPSurgery} and GEM~\cite{GEM} replace the conventional Query-Key attention with Value-Value attention. During forward, 
they additionally align new self-attention input with vanilla input to avoid deviation accumulation.
%additionally align the inputs between vanilla and new self-self attention, to avoid deviation accumulation after modification.
However, with the proposed self-self attention, the ability of CLIP to aggregate global context information, which is known to be useful for distinguishing confusing categories, is weakened.
Our proposed GCLIP in this paper belongs to the category of TF-OVSS methods and we mainly compare with the methods under the same setting for fairness.


\begin{figure*}
  \centering
\includegraphics[width=1\linewidth]{framework-gclip-12.pdf}
  \caption{\textbf{Method Overview.} 
  (a) \textbf{Overview.} In this paper, we propose a new framework GCLIP, consisting of Attention Map Fusion (AMF) and Channel Suppression (CS), for Training-Free Open-Vocabulary Semantic Segmentation. 
  (b) \textbf{Attention Map Fusion.} We fuse the attentions of early global-token emerging blocks ($L_g$,$L_{g+1}$, $\cdots$) with the Query-Query attention of the last-block ($L_{f}$) to emphasize the effect of global knowledge.
  %equip the last-block attention with image-level properties from global tokens, while not introducing homogeneous attention patterns across patches, by integrating the attention from the emerging blocks of global tokens into the Query-Query attention.
  (c) \textbf{Channel Suppression.} We suppress the weight norm of the specific output channel $\hat{d}$ of FFN by a re-nomalizing operation $\varphi$ as depicted in Eq.~(\ref{formula:renormalize}) to enhance the semantic correlation of Value embeddings.}
\label{method_fig}
\end{figure*}

\section{Method}
\label{sec:method}

\noindent \textbf{Overview}
In this work, we propose GCLIP, a new framework for Training-Free Open-Vocabulary Semantic Segmentation (TF-OVSS).
The general framework of our method is illustrated in Figure~\ref{method_fig}.
The textual input is formed by filling in the category name in the manually designed prompt, \emph{e.g.}, 
``a photo of a \#classname''.  
%where \cls{} denotes a class name.
Passing the textual input into the text encoder of CLIP, we obtain the text embeddings $Z_{\text{text}}$.
Previous work ClearCLIP~\cite{ClearCLIP} for TF-OVSS enhances the locality across patches but harms the capability of CLIP to exploit global context (Sec.~\ref{3.1baseline}).
Based on ClearCLIP, we propose GCLIP with two simple yet effective modifications to the last-block attention and Value embeddings respectively to mine the beneficial global knowledge of CLIP for TF-OVSS.
Firstly, we propose an Attention Map Fusion strategy (AMF) to emphasize the global knowledge encoded by global tokens via reshaping the last-block attention (Sec.~\ref{3.2attnmap}).
Secondly, we propose a Channel Suppression strategy (CS) to make last-block Value embeddings more semantically correlated (Sec.~\ref{3.3wn}).
We forward the visual input $I\in\mathbb{R}^{3 \times H\times W}$ through the visual encoder of GCLIP.
Since the representation of each patch is finally determined by the attention weights and the Value embeddings, we can finally generate more semantically correlated patch-level image features $Z_{\text{GCLIP}}$ while also absorbing global context information.
By comparing the similarity between $Z_{\text{GCLIP}}$ and $Z_{\text{text}}$, we generate a logit map and further predict the segmentation mask by $\text{argmax}$ operation on the logit map.

\subsection{Baseline}
\label{3.1baseline}
In this paper, we adopt ClearCLIP~\cite{ClearCLIP} as our baseline model.
ClearCLIP modifies the final block $L_f$ of CLIP to enhance the distinctness of patch-wise representations for TF-OVSS.
In detail, ClearCLIP alters the last-block Query-Key attention to Query-Query attention, which enables each patch to mainly focus on itself.
Besides, ClearCLIP discards the residual outputs from other blocks, as they introduce global characteristics that are homogeneous across patches and harm the patch-wise distinction.
Additionally, since the removal of residual connection significantly changes the input to the last-block FFN, ClearCLIP further discards last-block FFN to mitigate the negative effect. % on performance.
As a result, ClearCLIP simply adopts the output of the last-block Query-Query attention module for vision-language inference: 
\begin{equation}
Z_{\text{ClearCLIP}} = \text{Proj}({A}_{f}^{qq}\cdot v),
\end{equation}
where $\text{Proj}$ refers to output projection in the multi-head self-attention module, 
%\textcolor{red}{
${A}_{f}^{qq}$ and $v$ refers to the Query-Query attention map and Value embeddings from the final block $L_f$.
%}

Although ClearCLIP enhances the distinction of the image features across the patches, it significantly weakens the capability to aggregate global context information which may provide a global view of the image and benefit distinguishing confusing categories in the dense prediction task.
For example, in Figure~\ref{visualization_fig}, due to insufficient global context information, ClearCLIP classifies some regions into false categories with similar appearances and results in incomplete segmentation masks. 
%thus resulting in misclassification in several inner patches and incomplete segmentation masks. 

\subsection{Attention Map Fusion}
\label{3.2attnmap}
In this section, we propose an Attention Map Fusion strategy (AMF) to emphasize the global knowledge encoded by global tokens via reshaping the last-block attention. % by emphasizing the effect of global tokens.

%\noindent \textbf{Observation of Emerging Global Tokens}
As shown in Figure~\ref{introduction_fig}(a), we visualize the attention maps between different patches and observe that \textit{global tokens} exist in deeper blocks of CLIP.
The term ``global token'' means specific patches are important (\emph{i.e.,} corresponding Query-Key attention weights are super high) for all the other patches. 
Such global tokens appear in the attention map as highlighted vertical lines.
%can be reflected by a highlighted vertical line in the attention map.
Interestingly, we find that the attention pattern of global tokens aligns well with that of the \cls{} token (see the last two rows of Figure~\ref{introduction_fig}(a)), which indicates those global tokens may encode global properties as the \cls{} token.

Based on such observations, we propose AMF to fuse the attention maps from early global-token emerging blocks with the last-block Query-Query attention.
Specifically, as shown in Figure~\ref{method_fig}(b), 
given a vanilla CLIP with totally $f+1$ blocks, we first introduce $G(i)$ to judge whether global tokens exist in block $L_i (0\le i < f)$: 
\begin{align}
G(i) = 
\begin{cases}
1,& \text{if } \max(\prod_j \sigma \cdot A_{i, j}^{qk}) > 0\\
0,& \text{otherwise}
\end{cases}
,
\end{align}
where $A_{i}^{qk}$ denotes the Query-Key attention map of the $i$-th block. The $\prod_j A_{i, j}^{qk}$ means the multiplication between attention vectors for different Queries. The $\sigma=100$ is set to prevent all the values from exceeding the computational precision limits.
% 防止超出计算机精度
Then we identify the block $L_{g}$ where global tokens initially emerge,
\begin{align}
g = \arg\min \{i| G(i)=1,  0\leq i < f\}.
\end{align}
We further integrate the attention weight maps of global-token emerging block $L_g$ and its following $l$ ($l<f-g$) blocks into the final Query-Query attention weight map ${A}_{f}^{qq}$ to form a new attention map ${A}_{f}$,
\begin{align}
    {A}_{f} & = \text{AMF }({A}_{g}^{qk}, ...,{A}_{g+l}^{qk},{A}_{f}^{qq}) \notag \\
    & = \frac{{A}_{g}^{qk}+ ... + {A}_{g+l}^{qk}+{A}_{f}^{qq}}{l+2} .
\end{align}
Consequently, with ${A}_{f}$, we not only enable each patch to interact with itself or the nearby patches but also allow it to aggregate image-level global properties from global tokens.
Empirically, we find that fusing with attentions from the first and the second emerging blocks works the best, \emph{i.e.,} $l=1$.

Then our final attention output is presented as follows: 
\begin{equation}
Z_{\text{GCLIP}} = \text{Proj}({A}_{f}\cdot v),
\end{equation}

As on different datasets global-token emerging layers may be different, our AMF provides 
a practical way to automatically identify the global-token emerging layers.  
%Also, the automatic identification of global-token emerging layers makes our method easily applied to various scales of CLIP models.
%With our proposed AMF strategy, we are able to automatically identify the early emerging layers of global tokens, facilitating its adaptation to various scales of CLIP-like models.
%where $\text{Proj}$ refers to the output projection of the original attention mechanism.

\begin{figure}
  \centering
\includegraphics[width=1\linewidth]{fig_9img-2.pdf}
  \caption{\textbf{Weight Norms of the second fully-connected layer in FFNs.} Starting from block 5 (CLIP ViT-B/16), we observe FFN's second fully connected layer weight norm corresponding to a specific output channel becomes unexpectedly larger than the weight norm of other channels.}
\label{method_wn}
% \vspace{-6mm}
\end{figure}

\subsection{Channels Supression}
\label{3.3wn}
In this section, we propose a Channel Suppression (CS) strategy to make last-block Value embeddings more semantically correlated.

We observe an abnormal phenomenon exists in the weights of the second fully connected block of FFN in a Transformer block.
%\noindent \textbf{Increase of Specific-Channel Weight}
%We observe an interesting phenomenon exists in the weights of the second fully connected layer of FFN in a Transformer block.
As illustrated in Figure~\ref{introduction_fig} (b)(2), the entropy of weight norms decreases dramatically from a certain block and the weight norm corresponding to a specific output channel becomes unexpectedly larger than the weight norm of other channels in Figure~\ref{method_wn}.
Such an abnormal increase of specific-channel weight norm may homogeneously yield large activation of the same channel for different patch representations, which may do harm to the semantic correlations among different Value embeddings. 

%\noindent \textbf{Channel suppression}
Therefore, we propose a Channel Suppression strategy (CS) to make the Value embeddings of the last-block attention module more semantically correlated as shown in Figure~\ref{method_fig}(c).
Specifically, for the weight $W \in \mathbb{R}^{D_{out} \times D_{in}}$ of the second fully connected layer of FFN in a Transformer block, we suppress the output channel $\hat{d}$ which exhibits an extremely high weight norm.

%\textcolor{red}{
%TODO: IT SEEMS THAT WE DO NOT SHOW PERFORMING CS MAY HINDER THE EMERGENCE OF GLOBAL TOKENS.
%}
%\textcolor{blue}{
%We do not perform channel suppression simply from block $L_{g-1}$ where the weight norm begins to rise, as it hinders the emergence of global tokens and therefore harms the aggregation of image-level global context.
%}


Specifically, the abnormal channel $\hat{d}$ can be represented as
\begin{equation}
N_{d} = ||W_d||_{2},
\end{equation}
\begin{equation}
\hat{d} = \text{argmax}_{d\in\{0,1,\cdots, D_{out}-1\}}\{N_{d}\}.
\end{equation}
where $W_d\in\mathbb{R}^{1\times D_{in}}$.
Then, we average the norms of all the other channels as $\overline{{N}}$,
\emph{i.e.,}
\begin{equation}
\overline{N} = \frac{\sum_{i=0,i\neq \hat{d}}^{D_{out}-1}({N}_i)}{D_{out}-1}.
\end{equation}
We retain the weights of all the other output channels while re-normalizing the weight of channel $\hat{d}$:
\begin{equation}
\hat{W}_{\hat{d}} = \varphi{(W_{\hat{d}})} =  \frac{W_{\hat{d}}}{{N}_{\hat{d}}} \times \overline{{N}}.
\label{formula:renormalize}
\end{equation}

Suppose that an extreme decrease in the entropy of weight norms (as shown in Figure~\ref{method_wn}) occurs at block $s$,
we employ CS for each block $L_{i}$ where $s \le i \le f$. 

With the suppression, %we observe %the similarity between Value embeddings of the last-attention module and global tokens is reduced in (1) of Figure~\ref{introduction_fig}(b).  
%Moreover, according to 
as shown in Figure~\ref{introduction_fig} (b)(1), we observe the Value embeddings of patches within the same semantic mask become more similar (see ``in-in'' comparison) while those from different masks become more distinct (see ``in-out'' comparison).
These results verify that CS enhances the patch-wise semantic correlation of the final Value embeddings. % within the final attention block.

\subsection{GCLIP for training-free OVSS}
In GCLIP, both Attention Map Fusion (AMF) and Channel Suppression (CS) are employed.
With AMF, we emphasize the global knowledge encoded by global tokens by reshaping the last-block attention. % by emphasizing the effect of global tokens.
With CS, we enhance the semantic correlation of last-block Value embeddings.
As the patch-wise visual representation is finally determined by the last-block attention and the Value embeddings, 
it is expected that GCLIP can aggregate more beneficial global knowledge into final features and yield patch-wise features with high semantic coherence for semantic segmentation.

\section{Experiments}
\label{sec:experiment}

\begin{table*}[!t]
\setlength{\tabcolsep}{2.8pt}
\centering
\begin{tabular}{l|c|c|cccccccc}
\toprule[1pt]
 Methods & Pub. \& Year & Setting &  PASCAL VOC  & Context & \; ADE20K \; & Cityscapes & COCO Stuff & \;Avg.\;\\
    \midrule
  GroupViT$^{\ddagger}$~\cite{GroupViT}& CVPR'22& \multirow{3}{*}{{T-OVSS}} &$79.7$& $23.4$& $9.2$& $11.1$&$15.3$ & $27.7$\\
  CoCu~\cite{CoCu}&NeurIPS'24& & -& -& $11.1$& $15.0$&$13.6$ &-\\
  TCL~\cite{TCL}&CVPR'23 & &$77.5$ & $30.3$& $14.9$&$23.1$ &$19.6$ &$33.1$ \\
 \midrule
 MaskCLIP+$^{\dag}$~\cite{MaskCLIP}& ECCV'22&\multirow{3}{*}{USS} & $70.0$ & $31.1$ & $12.2$ & $25.2$ & $19.5$ & $31.6$\\
 CLIP-S4~\cite{CLIP-S4}& CVPR'23& & $72.0$ & $33.6$ & -& -&- &-\\
 ReCLIP~\cite{ReCLIP}& CVPR'24& & $75.8$ & $33.8$ & $14.3$ & $19.9$ &$20.3$ &$32.8$\\
 \midrule
 CLIP$^{\ddagger}$~\cite{CLIP}& ICML'21&\multirow{9}{*}{{TF-OVSS}}& $41.8$&$9.2$ & $2.1$&$5.5$ &$4.4$ & $12.6$\\
MaskCLIP$\dag$~\cite{MaskCLIP}& ECCV'22&&$49.5$ & $21.7$ &  $9.5$ & $19.8$ & $13.6$ & $22.8$ \\
 CLIPSurgery~\cite{CLIPSurgery}& Arxiv'23&&- & -& -&$31.4$ &$21.9$ & -\\
 GEM$^{\ddagger}$~\cite{GEM} & CVPR'24&&$79.9$ & $35.9$& $15.7$&$30.8$ &$23.7$ & $37.2$\\
 SCLIP~\cite{SCLIP}& ECCV'24&&$80.4$ &$34.2$ & $16.1$& $32.2$&$22.4$ & $37.1$\\
 CLIPtrase~\cite{CLIPtrase}& ECCV'24&&$81.2$ &$34.9$ & $17.0$& -&$24.1$ & -\\
 ClearCLIP~\cite{ClearCLIP}& ECCV'24&&$80.9$ & $35.9$&$16.7$ &$30.0$ &$23.9$ &$37.5$\\
 % \midrule
 +CS & Ours & & ${80.6}$ & ${36.2}$& ${17.8}$& ${31.3}$&${24.0}$ & $38.0$\\
 \rowcolor[gray]{.9} +CS \& AMF (GCLIP) & Ours & & $\bf{81.3}$ & $\bf{36.8}$& $\bf{18.5}$& $\bf{33.7}$&$\bf{24.8}$ & $\bf{39.0}$\\
 
 \bottomrule[1pt]
\end{tabular}
\caption{\textbf{Comparison with trainable open-vocabulary semantic segmentation methods (T-OVSS), unsupervised CLIP-based semantic segmentation methods (USS), and training-free open-vocabulary semantic segmentation methods (TF-OVSS).} Among these, $^{\dag}$ means the results are obtained by running the officially released source code and $^{\ddagger}$ means the results are cited from ClearCLIP~\cite{ClearCLIP}.}
\label{table_main}
\end{table*}


% \begin{table}[!t]
% \setlength{\tabcolsep}{4.5pt}

% \centering
% \begin{tabular}{ccc|cccc}
% % \hline
% \toprule[0.8pt]
% Baseline &  AMF & \;CS\; & VOC & Context &\;ADE\; & \;Stuff\;\\
% % \hline
% \midrule

%  $\checkmark$& & & $80.9$& $35.9$&$16.7$ & $23.9$\\
%  $\checkmark$& $\checkmark$&  & $81.2$&$36.7$&$18.1$&$24.6$\\
%  $\checkmark$& & $\checkmark$& $80.5$& $36.1$ & $17.6$ & $24.1$\\
%  \rowcolor[gray]{.9} 
%  $\checkmark$& $\checkmark$& $\checkmark$&  $\bf{81.3}$&
%  $\bf{37.0}$&$\bf{18.3}$&$\bf{24.7}$\\
% \bottomrule[0.8pt]
% \end{tabular}
% \caption{
% \textbf{Effectiveness of each component in GCLIP.} Our baseline is ClearCLIP~\cite{ClearCLIP}. 
% We use ``AMF'' and ``CS'' to represent attention map fusion in Sec.~\ref{3.2attnmap} and channel suppression in Sec.~\ref{3.3wn}, respectively.}
% \label{table_ablation_component}
% \vspace{-2mm}
% \end{table}

\begin{table}[!t]
\setlength{\tabcolsep}{4pt}

\centering
\begin{tabular}{c|cccccc}
% \hline
\toprule[0.8pt]
$l$ &  \;VOC\; & Context & \;ADE\; & Cityscapes & \;Stuff\; & Avg.\\
% \hline
\midrule
0 & $81.1$ & $36.5$ & $18.3$& $32.9$ & $24.6$ & $38.7$\\
\rowcolor[gray]{.9} 1 & $81.3$ & $\bf{36.8}$ & $\bf{18.5}$ & $\bf{33.7}$& $\bf{24.8}$ & $\bf{39.0}$\\
2 & $82.0$ & $36.8$ & $18.2$ & $32.8$ & $24.8$& $38.9$\\
3 & $82.0$ & $36.6$ & $17.9$ & $31.1$ & $24.7$ & $38.5$\\
4 & $\bf{82.1}$ & $36.6$ & $18.0$ & $31.0$ & $24.6$ & $38.5$\\
\bottomrule[0.8pt]
\end{tabular}
\caption{
\textbf{Effect of block selection for attention map fusion.} According to the results on all benchmarks, we finally fuse the attention maps of the first and the second global-token emerging blocks with the final Query-Query attention map in GCLIP.}
\label{table_ablation_attn}
\end{table}

\begin{table}[!t]
\setlength{\tabcolsep}{1.5pt}

\centering
\begin{tabular}{cc|cccccc}
% \hline
\toprule[0.8pt]
Block &  Entropy & \;VOC\; & Context & \;ADE\;  & \;City\; & \;Stuff\; & Avg.\\
% \hline
\midrule
$L_5$ & $0.96$ &$78.6$ & $35.5$ & $17.3$ & $30.5$ & $23.8$& $37.1$\\
$L_6$ & $0.93$ &$79.0$ & $35.3$ & $17.3$& $30.6$ & $23.6$& $37.2$\\
\rowcolor[gray]{.9} $L_7$ &$\bf{0.53}$ & $\bf{80.6}$ & $\bf{36.2}$ & $\bf{17.8}$ & $\bf{31.3}$ & $\bf{24.0}$ & $\bf{38.0}$\\
$L_8$ & $0.28$ &$80.1$ & $36.0$ & $17.5$ & $30.4$ & $23.9$& $37.6$\\
$L_9$ & $0.09$ &$80.2$ & $35.9$ & $17.4$ & $30.5$ & $23.9$& $37.6$\\
$L_{10}$ &$0.12$ & $80.2$ & $5.9$ & $17.5$ & $30.5$ & $23.9$& $37.6$\\
\bottomrule[0.8pt]
\end{tabular}
\caption{
\textbf{Effect of different blocks to perform channel suppression.} The block ID means we perform CS from this block to the last block. Considering average performance on all benchmarks, we choose to perform CS from block 7 to last block in GCLIP.}
\label{table_ablation_wn}
\end{table}

\subsection{Setup}
\label{4.1setup}
\noindent \textbf{Datasets}
We conduct experiments mainly on five standard benchmarks for semantic segmentation, including PASCAL VOC 2012 \cite{PASCAL_VOC}, PASCAL Context \cite{PASCAL_Context}, ADE20K \cite{ADE20K}, Cityscapes~\cite{cityscapes} and COCO Stuff~\cite{cocostuff}. 
PASCAL VOC 2012 (1,464/1,449 train/validation) contains 20 object classes, 
while PASCAL Context (4,998/5,105 train/validation) is an extension of PASCAL VOC 2010 and we treat 59 most common classes as foreground in our experiments. 
ADE20K (20,210/2,000 train/validation) is a segmentation dataset with various scenes and the 150 most common categories are considered.
Cityscapes~(2,975/500 train/validation) consists of various urban scene images of 19 categories from 50 different cities.
COCO Stuff~(118,287/5,000 train/validation) has 171 low-level thing and stuff categories excluding background class. 

\noindent \textbf{Architecture}
We use the text encoder of pre-trained CLIP~\cite{CLIP} model to generate text embeddings and modify the image encoder of CLIP to extract visual features.
%The architecture of the text encoder is a Transformer \cite{Transformer}.
For the image encoder, following general practice~\cite{ClearCLIP, SCLIP, CLIPtrase}, we adopt ViT-B/16.

\noindent \textbf{Implementation details}
Following previous works of training-free OVSS~\cite{ClearCLIP, SCLIP, CLIPtrase}, we resize the input image and employ a sliding window inference strategy. %and resize input images \textcolor{red}{with a shorter side of 448 pixels.}
For inference, we only utilize category names to generate text embeddings with the prompt templates provided by CLIP~\cite{CLIP} and do not exploit further text expansions. 
We adopt CS to only modify the Value embeddings in the subsequent transformer blocks with other parts unchanged.
To make a fair comparison, we do not apply any post-processing to our evaluation results.
We employ mean intersection over union (mIoU) as the metric to evaluate our method.
%For our proposed CS, we simply apply the output of the suppressed FFN to the Value embeddings in the following layers, while keeping the residual connection unchanged.


\subsection{Comparison with previous state-of-the-arts}
\label{4.2comparison}

\noindent \textbf{Baseline}
% Vanilla CLIP怎么做
We compare our method to vanilla CLIP where we take patch-wise visual features from last transformer block and compute their 
similarities with text embeddings to generate semantic masks.
Besides, we compare our method with three types of semantic segmentation methods: 
(1) Trainable methods for OVSS (T-OVSS), including GroupViT~\cite{GroupViT}, CoCu~\cite{CoCu}, and TCL~\cite{TCL};
(2) Unsupervised CLIP-based methods for semantic segmentation (USS), including MaskCLIP+~\cite{MaskCLIP}, CLIP-S4~\cite{CLIP-S4} and ReCLIP~\cite{ReCLIP};
and (3) CLIP-based methods for training-free OVSS (TF-OVSS), including CLIP~\cite{CLIP}, MaskCLIP~\cite{MaskCLIP}, CLIPSurgery~\cite{CLIPSurgery}, SCLIP~\cite{SCLIP}, GEM~\cite{GEM}, CLIPtrase~\cite{CLIPtrase} and ClearCLIP~\cite{ClearCLIP}.
%For CLIP, we take outputs of the last transformer block without [CLS] token as patch-wise visual features.
%By computing patch-wise similarity with text embeddings and arg max operation, we obtain TF-OVSS results of vanilla CLIP.
For fair comparison, we choose not to compare with methods using additional large-scale pre-trained models (\emph{e.g.,} DINO~\cite{DINO}, SAM~\cite{SAM}, Stable Diffusion~\cite{SD}, \emph{etc.}) other than CLIP, \emph{e.g.,} ProxyCLIP~\cite{proxyclip}.
%Since ProxyCLIP relies on extra vision foundation models, such as DINO~\cite{DINO}, it is unfair for th}e CLIP-only based GCLIP to directly compare with it.

We directly cite the corresponding results from the original papers, except that $\dag$ means the results are obtained by running the officially released source code and ${\ddagger}$ means the results are cited from ClearCLIP~\cite{ClearCLIP}.
All the numbers reported are presented as percentages.
Among these, T-OVSS methods rely on weak annotations like image-caption pairs to train the model, while USS methods rely on unlabeled images to train the model and cannot generalize to unseen classes. %during the training process.
Instead, GCLIP can directly perform open-vocabulary segmentation without any training, which falls into the category of TF-OVSS.
All TF-OVSS methods are based on pre-trained CLIP with ViT-B/16 visual backbone.

\noindent \textbf{Comparison}
The comparisons with previous state-of-the-art methods on five benchmarks are demonstrated in Table~\ref{table_main}.
From Table~\ref{table_main}, we have three observations:
%According to the results showin in Table~\ref{table_main}, we observe that:
(1) Without training or fine-tuning CLIP, TF-OVSS methods, \emph{e.g.,} ClearCLIP, our GCLIP, \emph{etc.}, outperforms vanilla CLIP~\cite{CLIP} remarkably, which demonstrates CLIP does encode beneficial knowledge for complex visual understanding tasks.
(2) Our GCLIP even outperforms some typical T-OVSS and USS methods, showing that CLIP itself is potentially a good OVSS segmentor and our way of modifying CLIP to mine useful knowledge for segmentation is effective.
(3) Our GCLIP outperforms previous state-of-the-art TF-OVSS methods obviously, achieving new state-of-the-arts on all the five benchmarks. 
For example, on Cityscapes, GCLIP outperforms SCLIP, GEM and ClearCLIP by 1.5\%, 2.9\% and 3.7\% mIoU respectively; on ADE20K, GCLIP outperforms SCLIP, GEM, CLIPtrase and ClearCLIP by 2.4\%, 2.8\%, 1.5\% and 1.8\% mIoU.
%Notably, GCLIP outperforms CLIPtrase by 2.1\% mIoU on PASCAL Context.
All these results verify the effectiveness of our method of utilizing beneficial global knowledge to assist OVSS segmentation.


\subsection{Qualitative Results}
\label{4.3visualization}
We visualize the segmentation results of GCLIP on PASCAL VOC and PASCAL Context in Figure~\ref{visualization_fig}.
We observe that both ClearCLIP and GCLIP yield much better masks than vanilla CLIP. 
But the masks generated by ClearCLIP are still incomplete.
For example, when segmenting a cow (Green Mask), ClearCLIP misclassifies some regions of cow as horse (Pink Mask).
Since ClearCLIP does not fully utilize the global knowledge of CLIP, it may fail to distinguish similar yet different categories.
GCLIP avoids such confusion and yields more integral and accurate masks, through absorbing image-level global properties and enhancing the semantic correlation of Value embeddings.

\begin{figure*}
  \centering
  \includegraphics[width=1\linewidth]{visualization7.pdf}
  \caption{\textbf{Qualitative Results.} 
  We visualize the segmentation results of GCLIP on both PASCAL VOC and PASCAL Context.
  We observe that the masks generated by ClearCLIP usually fail to segment the integral target object because it may confuse semantically similar categories without sufficient global context.
  GCLIP extracts semantically correlated patch-level image features through enhancing global context information. The masks generated by GCLIP obviously outperform those of both vanilla CLIP and ClearCLIP.}
\label{visualization_fig}
\end{figure*}

\begin{table}[!t]
\setlength{\tabcolsep}{2pt}

\centering
\begin{tabular}{c|ccccc}
% \hline
\toprule[0.8pt]
Fusion & \; VOC \; & Context & \; ADE \; & Cityscapes & Stuff\\
% \hline
\midrule
\cls{} Atten. & $80.3$ & $35.4$ & $16.6$ & $27.0$ & $24.2$\\
\rowcolor[gray]{.9} Ours & $\bf{81.3}$ & $\bf{36.8}$ & $\bf{18.5}$ & $\bf{33.7}$ & $\bf{24.8}$\\
\bottomrule[0.8pt]
\end{tabular}
\caption{
\textbf{Comparison with fusing \cls{} attention in AMF.} The ``\cls{} Atten.'' means we replace the patch-wise attention of global-token emerging blocks in AF  module with the attention of \cls{}. We duplicate \cls{} attention for each patch to fuse with last-block attention.} %According to the results, integrating the attention of global tokens emerging layers yields better results.}
\label{table_ablation_clstoken}
\end{table}

\begin{table}[!t]
\setlength{\tabcolsep}{3.4pt}

\centering
\begin{tabular}{c|ccccc}
% \hline
\toprule[0.8pt]
Tokens & \; VOC \; & Context & \; ADE \; & Cityscapes & Stuff\\
% \hline
\midrule
Random & $44.3$ & $36.5$ & $29.6$ & $53.1$ & $23.1$\\
\rowcolor[gray]{.9} Global & $\bf{75.0}$ & $\bf{71.4}$ & $\bf{66.9}$ & $\bf{97.9}$ & $\bf{66.4}$\\
\bottomrule[0.8pt]
\end{tabular}
\caption{
\textbf{Global tokens encode image-level global knowledge.} We exploit the classification results with \cls{} token as ground truth to evaluate the classification accuracy of global tokens.
We further provide classification accuracy with randomly-selected non-global tokens to make a comparison.
Results indicate that global tokens align well with \cls{} token in terms of encoding image-level global knowledge.}
\label{table_ablation_classification}
\end{table}

\subsection{Ablation study}
\label{4.4ablation_study}

%\textcolor{red}{
\noindent \textbf{Effect of components in GCLIP.}
As shown in Table~\ref{table_main}, we verify the effectiveness of proposed components in GCLIP.
Compared with ClearCLIP~\cite{ClearCLIP}, the channel suppression strategy (numbers with only ``CS'') on average brings 
0.5\% mIoU improvement.
Notably, it achieves 1.3\% mIoU performance gain on Cityscapes.
Introducing the attention map fusion strategy (numbers of ``GCLIP'') yields better results, \emph{i.e.,} 1\% mIoU improvement on average across all benchmarks.
%}


\noindent \textbf{Effect of $l$ in attention map fusion (AMF).}
In AMF, we set $l=1$ in our solution, which means we fuse the Query-Key attention map of the first and the second global-token emerging blocks with the final-block Query-Query attention map.
In order to validate the effect of $l$, we perform an ablation on the effect of $l$ in AMF in Table~\ref{table_ablation_attn}.
%\textcolor{red}{
On average, $l = 1$ yields best results and the performance is insensitive to $l$.
%}


%We further make a comparison between the final attention maps resulting from the integration of attention from the emerging global tokens layers and the expansion of the \cls{} token's attention weight map.
%As the \cls{} token assigns extremely high attention scores to the global tokens, integrating the expansion of its attention map significantly impairs the distinction among patches, thus resulting in suboptimal performance.

\noindent \textbf{Effect of different blocks to perform channel suppression (CS).}
We employ CS from block 7 to the last block of CLIP in our solution, as we observe a noticeable decrease of the entropy of weight norms at these blocks (shown in Figure~\ref{introduction_fig}(c)).
To validate the effect of this choice, 
we perform an ablation to test the effect of different blocks to perform CS in Table~\ref{table_ablation_wn}.
In this ablation, we do not include the AMF but simply test with CS.
The results show that suppressing from block 7 yields the best result on average, which is consistent with the decreasing trend of entropy of weight norms from block 7.

%\textcolor{red}{
\noindent \textbf{Effect of GCLIP on various VLMs.}
We further test GCLIP with other typical pre-trained VLMs, including OpenCLIP~\cite{OpenCLIP} and MetaCLIP~\cite{MetaCLIP}.
Results in Table~\ref{table_ablation_backbone} show that GCLIP brings consistent improvement among various benchmarks on different pre-trained VLMs. %further verifying the robustness of our proposed method.
%}
\begin{table}[!t]
\setlength{\tabcolsep}{2pt}

\centering
\begin{tabular}{c|c|cccc}
% \hline
\toprule[0.8pt]
 Method  & VLM& \; VOC \; & \; ADE \; & \; City \;& \; Stuff \;\\
% \hline
\midrule
Vanilla & \multirow{3}{*}{{OpenCLIP}} & $35.4$ & $2.2$ & $5.0$ & $4.3$ \\
ClearCLIP & &$78.3$& $17.4$ & $27.9$ & $23.5$\\
\rowcolor[gray]{.9}GCLIP & &$\bf{81.0}$  & $\bf{18.8}$ & $\bf{30.7}$ & $\bf{25.2}$ \\
\midrule
Vanilla&\multirow{3}{*}{{MetaCLIP}}& $47.2$ & $5.0$ & $5.1$ & $2.9$ \\
ClearCLIP &&$81.4$  & $18.9$ & $31.8$ & $23.1$ \\
\rowcolor[gray]{.9}GCLIP & & $\bf{83.5}$ & $\bf{18.9}$ & $\bf{32.4}$ & $\bf{23.1}$ \\
\bottomrule[0.8pt]
\end{tabular}
\caption{
\textbf{Effect of GCLIP on various VLMs.} GCLIP brings consistent improvement on various VLMs, including OpenCLIP and MetaCLIP, which further verifies the robustness of our proposed method.}
\label{table_ablation_backbone}
\end{table}

\noindent \textbf{Comparison with fusing \cls{} attention in AMF.}
In GCLIP, we integrate the attention from the global tokens emerging blocks into the Query-Query attention to equip the last-block attention with image-level global properties.
There exists an alternative way to duplicate the attention map of the \cls{} token and combine it with the Query-Query attention.
We then compare them in Table~\ref{table_ablation_clstoken}. % and the results in this table are combined with CS.
We observe that our fusion way outperforms fusing \cls{} token.
This may be because patch-wise attention in global token emerging blocks contain more diverse global attention patterns than duplicating \cls{} attention across patches, which may avoid homogeneous visual representations while absorbing global context information. 

\noindent \textbf{Global tokens encode image-level global knowledge.}
We claim that the global tokens contain rich image-level global context. Such global context information may benefit image-level classification, similar to the effect of \cls{} token. % since they serve as the proxies for the \cls{} token to capture a global view of the image.
In this ablation, we verify such claim by conducting image-level classification experiments with both \cls{} token and global tokens. 
First, we utilize the embedding of \cls{} token as visual feature to perform zero-shot classification and obtain the predicted classification results for each image.
Then we use the results predicted with \cls{} token as ground truth to evaluate the zero-shot classification results with global tokens.
To make a comparison, we randomly select other tokens as visual feature to conduct the same empirical evaluation. 
%corresponding values of the global tokens (``Global''), and the features of random patches other than global tokens (``Random'').
%Since the \cls{} token is acknowledged to represent image-level global context, we term its classification results as ground truth and calculate accuracy for global tokens and random patches as the metric.
As shown in Table~\ref{table_ablation_classification}, we observe that the predicted classification result of global tokens is highly consistent with that of the \cls{} token, which further validates global tokens encode rich image-level global context.

\section{Conclusion}
In this paper, we propose GCLIP for training-free open-vocabulary semantic segmentation. 
%We investigate the mechanism by which CLIP captures global knowledge and detect the formation and progressive augmentation of global tokens throughout the layers.
We aim to mine and utilize the global knowledge of CLIP beneficial for semantic segmentation.
We propose AMF to equip the last-block attention with global properties while not introducing homogeneous attention patterns across patches and Channel Suppression to make the Value embeddings of the last-block attention module more semantically correlated.
Via enhancing global knowledge of final features, GCLIP can generate more semantically correlated patch-level image features for TF-OVSS.
Extensive experiments demonstrate that our method achieves superior segmentation performance compared with previous state-of-the-arts. We hope our work may inspire future research to investigate how to better utilize CLIP's knowledge for complex visual understanding tasks.



  {
      \small
      \bibliographystyle{ieeenat_fullname}
      \bibliography{main}
  }
% \clearpage
\pagenumbering{gobble}
\maketitlesupplementary

\section{Additional Results on Embodied Tasks}

To evaluate the broader applicability of our EgoAgent's learned representation beyond video-conditioned 3D human motion prediction, we test its ability to improve visual policy learning for embodiments other than the human skeleton.
Following the methodology in~\cite{majumdar2023we}, we conduct experiments on the TriFinger benchmark~\cite{wuthrich2020trifinger}, which involves a three-finger robot performing two tasks: reach cube and move cube. 
We freeze the pretrained representations and use a 3-layer MLP as the policy network, training each task with 100 demonstrations.

\begin{table}[h]
\centering
\caption{Success rate (\%) on the TriFinger benchmark, where each model's pretrained representation is fixed, and additional linear layers are trained as the policy network.}
\label{tab:trifinger}
\resizebox{\linewidth}{!}{%
\begin{tabular}{llcc}
\toprule
Methods       & Training Dataset & Reach Cube & Move Cube \\
\midrule
DINO~\cite{caron2021emerging}         & WT Venice        & 78.03     & 47.42     \\
DoRA~\cite{venkataramanan2023imagenet}          & WT Venice        & 81.62     & 53.76     \\
DoRA~\cite{venkataramanan2023imagenet}          & WT All           & 82.40     & 48.13     \\
\midrule
EgoAgent-300M & WT+Ego-Exo4D      & 82.61    & 54.21      \\
EgoAgent-1B   & WT+Ego-Exo4D      & \textbf{85.72}      & \textbf{57.66}   \\
\bottomrule
\end{tabular}%
}
\end{table}

As shown in Table~\ref{tab:trifinger}, EgoAgent achieves the highest success rates on both tasks, outperforming the best models from DoRA~\cite{venkataramanan2023imagenet} with increases of +3.32\% and +3.9\% respectively.
This result shows that by incorporating human action prediction into the learning process, EgoAgent demonstrates the ability to learn more effective representations that benefit both image classification and embodied manipulation tasks.
This highlights the potential of leveraging human-centric motion data to bridge the gap between visual understanding and actionable policy learning.



\section{Additional Results on Egocentric Future State Prediction}

In this section, we provide additional qualitative results on the egocentric future state prediction task. Additionally, we describe our approach to finetune video diffusion model on the Ego-Exo4D dataset~\cite{grauman2024ego} and generate future video frames conditioned on initial frames as shown in Figure~\ref{fig:opensora_finetune}.

\begin{figure}[b]
    \centering
    \includegraphics[width=\linewidth]{figures/opensora_finetune.pdf}
    \caption{Comparison of OpenSora V1.1 first-frame-conditioned video generation results before and after finetuning on Ego-Exo4D. Fine-tuning enhances temporal consistency, but the predicted pixel-space future states still exhibit errors, such as inaccuracies in the basketball's trajectory.}
    \label{fig:opensora_finetune}
\end{figure}

\subsection{Visualizations and Comparisons}

More visualizations of our method, DoRA, and OpenSora in different scenes (as shown in Figure~\ref{fig:supp pred}). For OpenSora, when predicting the states of $t_k$, we use all the ground truth frames from $t_{0}$ to $t_{k-1}$ as conditions. As OpenSora takes only past observations as input and neglects human motion, it performs well only when the human has relatively small motions (see top cases in Figure~\ref{fig:supp pred}), but can not adjust to large movements of the human body or quick viewpoint changes (see bottom cases in Figure~\ref{fig:supp pred}).

\begin{figure*}
    \centering
    \includegraphics[width=\linewidth]{figures/supp_pred.pdf}
    \caption{Retrieval and generation results for egocentric future state prediction. Correct and wrong retrieval images are marked with green and red boundaries, respectively.}
    \label{fig:supp pred}
\end{figure*}

\begin{figure*}[t]
    \centering
    \includegraphics[width=0.9\linewidth]{figures/motion_prediction.pdf}
    \vspace{-0.5mm}
    \caption{Motion prediction results in scenes with minor changes in observation.}
    \vspace{-1.5mm}
    \label{fig:motion_prediction}
\end{figure*}

\subsection{Finetuning OpenSora on Ego-Exo4D}

OpenSora V1.1~\cite{opensora}, initially trained on internet videos and images, produces severely inconsistent results when directly applied to infer future videos on the Ego-Exo4D dataset, as illustrated in Figure~\ref{fig:opensora_finetune}.
To address the gap between general internet content and egocentric video data, we fine-tune the official checkpoint on the Ego-Exo4D training set for 50 epochs.
OpenSora V1.1 proposed a random mask strategy during training to enable video generation by image and video conditioning. We adopted the default masking rate, which applies: 75\% with no masking, 2.5\% with random masking of 1 frame to 1/4 of the total frames, 2.5\% with masking at either the beginning or the end for 1 frame to 1/4 of the total frames, and 5\% with random masking spanning 1 frame to 1/4 of the total frames at both the beginning and the end.

As shown in Fig.~\ref{fig:opensora_finetune}, despite being trained on a large dataset, OpenSora struggles to generalize to the Ego-Exo4D dataset, producing future video frames with minimal consistency relative to the conditioning frame. While fine-tuning improves temporal consistency, the moving trajectories of objects like the basketball and soccer ball still deviate from realistic physical laws. Compared with our feature space prediction results, this suggests that training world models in a reconstructive latent space is more challenging than training them in a feature space.


\section{Additional Results on 3D Human Motion Prediction}

We present additional qualitative results for the 3D human motion prediction task, highlighting a particularly challenging scenario where egocentric observations exhibit minimal variation. This scenario poses significant difficulties for video-conditioned motion prediction, as the model must effectively capture and interpret subtle changes. As demonstrated in Fig.~\ref{fig:motion_prediction}, EgoAgent successfully generates accurate predictions that closely align with the ground truth motion, showcasing its ability to handle fine-grained temporal dynamics and nuanced contextual cues.

\section{OpenSora for Image Classification}

In this section, we detail the process of extracting features from OpenSora V1.1~\cite{opensora} (without fine-tuning) for an image classification task. Following the approach of~\cite{xiang2023denoising}, we leverage the insight that diffusion models can be interpreted as multi-level denoising autoencoders. These models inherently learn linearly separable representations within their intermediate layers, without relying on auxiliary encoders. The quality of the extracted features depends on both the layer depth and the noise level applied during extraction.


\begin{table}[h]
\centering
\caption{$k$-NN evaluation results of OpenSora V1.1 features from different layer depths and noising scales on ImageNet-100. Top1 and Top5 accuracy (\%) are reported.}
\label{tab:opensora-knn}
\resizebox{0.95\linewidth}{!}{%
\begin{tabular}{lcccccc}
\toprule
\multirow{2}{*}{Timesteps} & \multicolumn{2}{c}{First Layer} & \multicolumn{2}{c}{Middle Layer} & \multicolumn{2}{c}{Last Layer} \\
\cmidrule(r){2-3}   \cmidrule(r){4-5}  \cmidrule(r){6-7}  & Top1           & Top5           & Top1            & Top5           & Top1           & Top5          \\
\midrule
32        &  6.10           & 18.20             & 34.04               & 59.50             & 30.40             & 55.74             \\
64        & 6.12              & 18.48              & 36.04               & 61.84              & 31.80         & 57.06         \\
128       & 5.84             & 18.14             & 38.08               & 64.16              & 33.44       & 58.42 \\
256       & 5.60             & 16.58              & 30.34               & 56.38              &28.14          & 52.32        \\
512       & 3.66              & 11.70            & 6.24              & 17.62              & 7.24              & 19.44  \\ 
\bottomrule
\end{tabular}%
}
\end{table}

As shown in Table~\ref{tab:opensora-knn}, we first evaluate $k$-NN classification performance on the ImageNet-100 dataset using three intermediate layers and five different noise scales. We find that a noise timestep of 128 yields the best results, with the middle and last layers performing significantly better than the first layer.
We then test this optimal configuration on ImageNet-1K and find that the last layer with 128 noising timesteps achieves the best classification accuracy.

\section{Data Preprocess}
For egocentric video sequences, we utilize videos from the Ego-Exo4D~\cite{grauman2024ego} and WT~\cite{venkataramanan2023imagenet} datasets.
The original resolution of Ego-Exo4D videos is 1408×1408, captured at 30 fps. We sample one frame every five frames and use the original resolution to crop local views (224×224) for computing the self-supervised representation loss. For computing the prediction and action loss, the videos are downsampled to 224×224 resolution.
WT primarily consists of 4K videos (3840×2160) recorded at 60 or 30 fps. Similar to Ego-Exo4D, we use the original resolution and downsample the frame rate to 6 fps for representation loss computation.
As Ego-Exo4D employs fisheye cameras, we undistort the images to a pinhole camera model using the official Project Aria Tools to align them with the WT videos.

For motion sequences, the Ego-Exo4D dataset provides synchronized 3D motion annotations and camera extrinsic parameters for various tasks and scenes. While some annotations are manually labeled, others are automatically generated using 3D motion estimation algorithms from multiple exocentric views. To maximize data utility and maintain high-quality annotations, manual labels are prioritized wherever available, and automated annotations are used only when manual labels are absent.
Each pose is converted into the egocentric camera's coordinate system using transformation matrices derived from the camera extrinsics. These matrices also enable the computation of trajectory vectors for each frame in a sequence. Beyond the x, y, z coordinates, a visibility dimension is appended to account for keypoints invisible to all exocentric views. Finally, a sliding window approach segments sequences into fixed-size windows to serve as input for the model. Note that we do not downsample the frame rate of 3D motions.

\section{Training Details}
\subsection{Architecture Configurations}
In Table~\ref{tab:arch}, we provide detailed architecture configurations for EgoAgent following the scaling-up strategy of InternLM~\cite{team2023internlm}. To ensure the generalization, we do not modify the internal modules in InternML, \emph{i.e.}, we adopt the RMSNorm and 1D RoPE. We show that, without specific modules designed for vision tasks, EgoAgent can perform well on vision and action tasks.

\begin{table}[ht]
  \centering
  \caption{Architecture configurations of EgoAgent.}
  \resizebox{0.8\linewidth}{!}{%
    \begin{tabular}{lcc}
    \toprule
          & EgoAgent-300M & EgoAgent-1B \\
          \midrule
    Depth & 22    & 22 \\
    Embedding dim & 1024  & 2048 \\
    Number of heads & 8     & 16 \\
    MLP ratio &    8/3   & 8/3 \\
    $\#$param.  & 284M & 1.13B \\
    \bottomrule
    \end{tabular}%
    }
  \label{tab:arch}%
\end{table}%

Table~\ref{tab:io_structure} presents the detailed configuration of the embedding and prediction modules in EgoAgent, including the image projector ($\text{Proj}_i$), representation head/state prediction head ($\text{MLP}_i$), action projector ($\text{Proj}_a$) and action prediction head ($\text{MLP}_a$).
Note that the representation head and the state prediction head share the same architecture but have distinct weights.

\begin{table}[t]
\centering
\caption{Architecture of the embedding ($\text{Proj}_i$, $\text{Proj}_a$) and prediction ($\text{MLP}_i$, $\text{MLP}_a$) modules in EgoAgent. For details on module connections and functions, please refer to Fig.~2 in the main paper.}
\label{tab:io_structure}
\resizebox{\linewidth}{!}{%
\begin{tabular}{lcl}
\toprule
       & \multicolumn{1}{c}{Norm \& Activation} & \multicolumn{1}{c}{Output Shape}  \\
\midrule
\multicolumn{3}{l}{$\text{Proj}_i$ (\textit{Image projector})} \\
\midrule
Input image  & -          & 3$\times$224$\times$224 \\
Conv 2D (16$\times$16) & -       & Embedding dim$\times$14$\times$14    \\
\midrule
\multicolumn{3}{l}{$\text{MLP}_i$ (\textit{State prediction head} \& \textit{Representation head)}} \\
\midrule
Input embedding  & -          & Embedding dim \\
Linear & GELU       & 2048          \\
Linear & GELU       & 2048          \\
Linear & -          & 256           \\
Linear & -          & 65536     \\
\midrule
\multicolumn{3}{l}{$\text{Proj}_a$ (\textit{Action projector})} \\
\midrule
Input pose sequence  & -          & 4$\times$5$\times$17 \\
Conv 2D (5$\times$17) & LN, GELU   & Embedding dim$\times$1$\times$1    \\
\midrule
\multicolumn{3}{l}{$\text{MLP}_a$ (\textit{Action prediction head})} \\
\midrule
Input embedding  & -          & Embedding dim$\times$1$\times$1 \\
Linear & -          & 4$\times$5$\times$17     \\
\bottomrule
\end{tabular}%
}
\end{table}


\subsection{Training Configurations}
In Table~\ref{tab:training hyper}, we provide the detailed training hyper-parameters for experiments in the main manuscripts.

\begin{table}[ht]
  \centering
  \caption{Hyper-parameters for training EgoAgent.}
  \resizebox{0.86\linewidth}{!}{%
    \begin{tabular}{lc}
    \toprule
    Training Configuration & EgoAgent-300M/1B \\
    \midrule
    Training recipe: &  \\
    optimizer & AdamW~\cite{loshchilov2017decoupled} \\
    optimizer momentum & $\beta_1=0.9, \beta_2=0.999$ \\
    \midrule
    Learning hyper-parameters: &  \\
    base learning rate & 6.0E-04 \\
    learning rate schedule & cosine \\
    base weight decay & 0.04 \\
    end weight decay & 0.4 \\
    batch size & 1920 \\
    training iters & 72,000 \\
    lr warmup iters & 1,800 \\
    warmup schedule & linear \\
    gradient clip & 1.0 \\
    data type & float16 \\
    norm epsilon & 1.0E-06 \\
    \midrule
    EMA hyper-parameters: &  \\
    momentum & 0.996 \\
    \bottomrule
    \end{tabular}%
    }
  \label{tab:training hyper}%
\end{table}%

\clearpage

% {
%   \small
%   \bibliographystyle{ieeenat_fullname}
%   \bibliography{sup}
% }
\end{document}
