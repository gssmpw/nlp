\section{Literature Review and Research Gaps}
Optimal steady-state operation of MT-HVdc grids has been studied in many papers, and various techniques have been proposed to achieve different useful objectives.
In some works, generally speaking, an offline optimization, repeatedly solved over long periods of time, is used to update the local primary controllers of the converter stations.
For example, in \cite{Gavriluta2015}, a secondary voltage control is proposed for network loss minimization. In this method, an offline optimization problem is solved and its solutions are sent to the converter stations.
In \cite{Papangelis2017}, a centralized model predictive secondary control is proposed for converter stations to follow their given power references subject to tight regulation of the average voltage and the operational limits.
In \cite{CarmonSanchez2020}, a model predictive secondary control is proposed to support the droop controller. The control objectives considered in this work are voltage and power tracking.
A generalized droop control is proposed in \cite{Eriksson2018} to improve the functionality of the converters, especially in case of failure of one station, by better distributing the load among the remaining stations. In this method, the grid operator performs an offline constrained optimization and computes the droop gain matrix accordingly. A somewhat similar approach, but with an adaptive local droop controller, is proposed in \cite{Yogarathinam2019}, where stability constraints are also taken into account.
In \cite{Shinoda2022}, an adaptive droop control is proposed whose parameters are optimally selected based on the available headroom and voltage containment reserve of the converter stations.
In \cite{Zhang2021,Zhang2022}, some secondary controllers are proposed for load sharing, voltage regulation, and loss minimization. In these works, an offline optimization problem is solved and used to adjust the power or voltage references.
A cost-based adaptive droop control is proposed in \cite{Song2021}, where an area cost minimization problem is solved every five minutes and the droop parameters are updated accordingly.
In \cite{Xie2023}, a secondary control is proposed that provides average voltage regulation and loss minimization by repeatedly solving an offline optimization problem and updating the primary controllers.
A somewhat similar secondary control is proposed in \cite{Li2018} for loss minimization and load sharing in dc grids.

In another line of research, innovative dynamic controllers, usually based on consensus algorithms and multi-agent systems, are proposed to achieve some specific control objectives.
In \cite{Wang2021}, a cooperative control is proposed for frequency support and power sharing among the converters, where the dc voltage control is not studied.
In \cite{Aram2018}, a distributed secondary control is proposed to provide accurate load sharing among the converters.
To provide load sharing and voltage regulation, a secondary control is proposed in \cite{Aram2023}, where three different loops are designed to provide frequency support, power sharing, and voltage regulation by adjusting the voltage setpoints of the stations. 
A secondary control is proposed in \cite{Wang2020}, which realizes an adjustable trade-off between load sharing and voltage regulation objectives.
In \cite{Lotfifard2022}, a distributed adaptive control based on a dynamic consensus algorithm is proposed for power sharing among the converters.
In \cite{Zhang2020, Yang2022}, some distributed controllers are proposed to achieve proportional power sharing and average voltage regulation.

The methods in \cite{Gavriluta2015,Papangelis2017,CarmonSanchez2020,Eriksson2018,Yogarathinam2019,Shinoda2022,Zhang2021,Zhang2022,Song2021,Xie2023, Li2018} are all based on offline decision making. In principle, they are feedforward control and optimization methods, which are not robust to unknown disturbances and model uncertainties.
Unlike the work in \cite{Gavriluta2015,Papangelis2017,CarmonSanchez2020,Eriksson2018,Yogarathinam2019,Shinoda2022,Zhang2021,Zhang2022,Song2021,Xie2023, Li2018}, the methods in \cite{Wang2021,Aram2018,Aram2023,Wang2020,Lotfifard2022,Zhang2020,Yang2022} are online feedback controllers.
However, despite their robust performance, they may not control the system to the best steady state for which they were designed, especially when operating constraints are considered.
To address this problem, which is common to many engineering disciplines, there has been a recent tendency to incorporate the relevant constrained optimization algorithms into the feedback control loops in a systematic way \cite{Krishnamoorthy2022,Emiliano2018,Ortman2020,Ortman2022,Häberle2021,Hauswirth2021}.
In this approach, the feedback control laws steer the system towards an optimal and admissible steady state in real time, while preserving some robustness guarantees.
This core idea has gained attention in various disciplines such as process control \cite{Krishnamoorthy2022}, power systems \cite{Emiliano2018,Ortman2020,Ortman2022}, and control systems \cite{Häberle2021,Hauswirth2021}, where it is generally known as Real-Time Optimization (RTO) or Online Feedback Optimization (OFO).
In this study, we apply this idea to MMC-based MT-HVdc grids and provide a framework to bridge the gaps in the reviewed literature.