\documentclass[12pt]{settings/colt2025} % Anonymized submission
%\documentclass[final,12pt]{colt2025} % Include author names

% The following packages will be automatically loaded:
% amsmath, amssymb, natbib, graphicx, url, algorithm2e

\title[Optimistically Optimistic Exploration for Efficient Infinite-Horizon RL]{Optimistically Optimistic 
Exploration for Provably Efficient Infinite-Horizon Reinforcement and Imitation Learning}
\usepackage{times}
% Use \Name{Author Name} to specify the name.
% If the surname contains spaces, enclose the surname
% in braces, e.g. \Name{John {Smith Jones}} similarly
% if the name has a "von" part, e.g \Name{Jane {de Winter}}.
% If the first letter in the forenames is a diacritic
% enclose the diacritic in braces, e.g. \Name{{\'E}louise Smith}

% Two authors with the same address
% \coltauthor{\Name{Author Name1} \Email{abc@sample.com}\and
%  \Name{Author Name2} \Email{xyz@sample.com}\\
%  \addr Address}

% Three or more authors with the same address:
% \coltauthor{\Name{Author Name1} \Email{an1@sample.com}\\
%  \Name{Author Name2} \Email{an2@sample.com}\\
%  \Name{Author Name3} \Email{an3@sample.com}\\
%  \addr Address}

% Authors with different addresses:
\author[Moulin, Neu and Viano]{%
 \Name{Antoine Moulin} \Email{antoine.moulin@upf.edu}\\
 \addr Universitat Pompeu Fabra, Barcelona, Spain%
 \AND
 \Name{Gergely Neu} \Email{gergely.neu@gmail.com}\\
 \addr Universitat Pompeu Fabra, Barcelona, Spain
 \AND
 \Name{Luca Viano} \Email{luca.viano@epfl.ch}\\
 \addr EPFL, Lausanne, Switzerland
}

% custom
%\usepackage[unicode,psdextra]{hyperref}
\usepackage[capitalize,noabbrev]{cleveref}
% Parentheses
\newcommand{\bc}[1]{\left\{{#1}\right\}}
\newcommand{\brr}[1]{\left({#1}\right)}
\newcommand{\bs}[1]{\left[{#1}\right]}
\newcommand{\ceil}[1]{\left\lceil #1 \right\rceil}
\newcommand{\floor}[1]{\left\lfloor #1 \right\rfloor}
\newcommand{\bsd}[1]{\left\llbracket{#1}\right\rrbracket}
\newcommand{\ip}[2]{\left\langle{#1},{#2}\right\rangle}
\newcommand{\inner}[1]{\langle#1\rangle}
% Statistics notation
% Probabilities
\renewcommand{\P}[1]{\mathbb{P}\bs{{#1}}}
\newcommand{\Pp}[2]{\underset{#1}{\mathbb{P}}\bs{{#2}}}
\newcommand{\Ppp}[3]{\mathbb{P}_{#1}^{#2}\bs{{#3}}}
% Expectations
\newcommand{\Ee}[2]{\underset{#1}{\mathbb{E}}\bs{{#2}}}
\newcommand{\Eee}[2]{\mathbb{E}_{#1}\bs{{#2}}}
\newcommand{\sspace}{\mcf{X}}     % state space
\newcommand{\aspace}{\mcf{A}}     % action space
\newcommand{\wspace}{\mcf{W}}     % cost weights space
\newcommand{\histspace}{\mcf{H}}  % history space
% argmin and argmax
\DeclareMathOperator*{\argmin}{arg\,min}
\DeclareMathOperator*{\argmax}{arg\,max}

\usepackage{algorithm}
\usepackage{algorithmic}
\usepackage{multirow}
\newcounter{protocol}
\makeatletter
\newenvironment{protocol}[1][htb]{%
  \let\c@algorithm\c@protocol
  \renewcommand{\ALG@name}{Protocol}% Update algorithm name
  \begin{algorithm}[#1]%
  }{\end{algorithm}
}
\usepackage{amssymb}
\usepackage{amsfonts}
\usepackage{graphics}
\usepackage{graphicx}
\usepackage{setspace}
%\usepackage{subcaption}
%\usepackage{selectp}
%\outputonly{1-18}
% \newtheorem{theorem}{\protect\theoremname}
%   \newtheorem{lemma}{\protect\lemmaname}
%   \newtheorem{corollary}{\protect\corrolaryname}
%   \newtheorem{defn}{\protect\definitionname}
%   \newtheorem{proposition}{\protect\propositionname}
%   \newtheorem{example}{\protect\examplename}
%   \newtheorem{remark}{Remark}
%   \newtheorem{conditions}{\protect\conditionsname}
%     \newtheorem{assumption}{\protect\assumptionname}
    
%     \newenvironment{proofof}[1]{\begin{proof}[{#1}]}{\end{proof}}
    
% \providecommand{\definitionname}{Definition}
% \providecommand{\examplename}{Example}
% \providecommand{\lemmaname}{Lemma}
% \providecommand{\corrolaryname}{Corollary}
% \providecommand{\propositionname}{Proposition}
% \providecommand{\conditionsname}{Conditions}
% \providecommand{\theoremname}{Theorem}
% \providecommand{\assumptionname}{Assumption}

\def \M{\mathcal{M}}
\def \S{\mathcal{S}}


% to compile a preprint version, e.g., for submission to arXiv, add add the
% [preprint] option:
%     \usepackage[preprint]{neurips_2022}


% to compile a camera-ready version, add the [final] option, e.g.:
%     \usepackage[final]{neurips_2022}


% to avoid loading the natbib package, add option nonatbib:
%    \usepackage[nonatbib]{neurips_2022}

%commands from AK
%====================================================================================================================================================================================
%new
%state space S 
\newcommand{\qv}{\mbf{q}}
\newcommand{\dv}{\mbf{d}}
\newcommand{\dve}{\mbf{d}^E}
\newcommand{\trans}{\intercal}
\newcommand{\qval}{\mbf{Q}}
\newcommand{\FEV}[1]{\mbs{\rho}_{\phim}(#1)}
\newcommand{\EFEV}[1]{\mbs{\rho}_{\phim}(\widehat{#1})}
%==============================================================================
%old
\newcommand{\mcf}{\mathcal}
\newcommand{\mbf}{\mathbf}
\newcommand{\mbb}{\mathbb}

\newcommand{\innerprod}[2]{\left\langle{#1},{#2}\right\rangle}

\newcommand{\expert}{{\pi_E}}
\newcommand{\experti}{\pi_E^i}
\newcommand{\apprentice}{\pi_{\textup{A}}}
\newcommand{\mbs}{\boldsymbol}
\newcommand{\cost}{r}
\newcommand{\true}{r_{\textup{true}}}
\newcommand{\weight}{w}
\newcommand{\FRAalg}{\texttt{\textbf{FRA-IL}}\xspace}
\newcommand{\phim}{\Phi}
\newcommand{\initial}{\nu_0}


\newcommand{\wtrue}{\weight_{\textup{true}}}
 %========================================================
\newcommand{\lv}[1]{%
    \ifmmode
    \text{\textcolor{orange}{[LV: #1]}}
    \else
    \textcolor{orange}{[LV: #1]}
    \fi
}
\usepackage[utf8]{inputenc} % allow utf-8 input
\usepackage[T1]{fontenc}    % use 8-bit T1 fonts      
\usepackage{url}            % simple URL typesetting
\usepackage{booktabs}       % professional-quality tables
\usepackage{amsfonts}       % blackboard math symbols
\usepackage{nicefrac}       % compact symbols for 1/2, etc.
\usepackage{microtype}      % microtypography
\usepackage{xcolor}         % colors
\usepackage{dsfont}
% \usepackage{xcolor}
\usepackage{pifont}
\newcommand{\greentick}{\textcolor{green}{\ding{51}}}
\newcommand{\redcross}{\textcolor{red}{\ding{55}}}
\newtheorem{assumption}{\protect\assumptionname}
\providecommand{\assumptionname}{Assumption}

\usepackage{mathtools}% http://ctan.org/pkg/mathtools
\makeatletter
\newcommand{\mytag}[2]{%
  \text{#1}%
  \@bsphack
  \protected@write\@auxout{}%
         {\string\newlabel{#2}{{#1}{\thepage}}}%
  \@esphack
}
\makeatother

\usepackage{thm-restate}
\newtheorem{Lem}[theorem]{Lemma}


\newcommand{\grg}[1]{\textcolor{red}{[\textbf{Grg:} #1]}}
\newcommand{\dd}{\mathrm{d}}

\newcommand{\raviUCB}{\texttt{\textbf{RAVI-UCB}}\xspace}
\newcommand{\RMAXalg}{\texttt{\textbf{Rmax}}\xspace}
\newcommand{\algname}{\texttt{\textbf{Rmax-RAVI-UCB}}\xspace}

\newcommand{\RMAX}{R_{\max}}

\newcommand{\D}[2]{D\pa{#1\middle\|#2}}
\newcommand{\DDKL}[2]{\mathcal{D}_{\text{KL}}\pa{#1\middle\|#2}}
%\newcommand{\CB}{\mbox{CB}}
\newcommand{\CC}{\mathcal{C}}
\newcommand{\TT}{\mathcal{T}}
\newcommand{\GG}{\mathcal{G}}
\newcommand{\tGG}{\wt{\GG}}
\newcommand{\ttheta}{\wt{\theta}}
\newcommand{\btheta}{\overline{\theta}}
\newcommand{\tlambda}{\wt{\lambda}}
\newcommand{\blambda}{\overline{\lambda}}
\newcommand{\bmu}{\overline{\mu}}
\newcommand{\tQ}{\wt{Q}}
\newcommand{\tV}{\wt{V}}
\newcommand{\tpi}{\wt{\pi}}
\newcommand{\td}{\wt{d}}
\newcommand{\tr}{r^+}
\newcommand{\thr}{{\hr}^+}
\newcommand{\trho}{\wt{\rho}}
\newcommand{\trunc}[2]{\left[#1\right]_{#2}}


\newcommand{\Plin}{\mathcal{P}_{\Phi}}


\newcommand{\Hil}{\mathcal{H}}
\newcommand{\X}{\mathcal{X}}
\newcommand{\F}{\mathcal{F}}
\newcommand{\A}{\mathcal{A}}
\newcommand{\N}{\mathcal{N}}
\newcommand{\LL}{\mathcal{L}}
\newcommand{\real}{\mathbb{R}}
\newcommand{\Dw}{\mathcal{D}}
%\newcommand{\Sw}{\mathcal{S}}
\newcommand{\Rw}{\mathcal{R}}
\newcommand{\HH}{\mathcal{H}}
\newcommand{\DD}{\mathcal{D}}
\newcommand{\OO}{\mathcal{O}}
\newcommand{\tOO}{\wt{\OO}}
\newcommand{\trace}[1]{\mbox{tr}\left(#1\right)}
\newcommand{\II}[1]{\mathbb{I}_{\left\{#1\right\}}}
\newcommand{\PP}[1]{\mathbb{P}\left[#1\right]}
\newcommand{\PPZ}[1]{\mathbb{P}_Z\left[#1\right]}
\newcommand{\EEZ}[1]{\mathbb{E}_Z\left[#1\right]}
\newcommand{\EE}[1]{\mathbb{E}\left[#1\right]}
\newcommand{\EEb}[1]{\mathbb{E}\bigl[#1\bigr]}
\newcommand{\EEtb}[1]{\mathbb{E}_t\bigl[#1\bigr]}
\newcommand{\EXP}{\mathbb{E}}
\newcommand{\EEs}[2]{\mathbb{E}_{#2}\left[#1\right]}
\newcommand{\EEu}[1]{\mathbb{E}_{\u}\left[#1\right]}
\newcommand{\EEst}[1]{\mathbb{E}_{*}\left[#1\right]}
\newcommand{\EEo}[1]{\mathbb{E}_{0}\left[#1\right]}
\newcommand{\PPt}[1]{\mathbb{P}_t\left[#1\right]}
\newcommand{\EEt}[1]{\mathbb{E}_t\left[#1\right]}
\newcommand{\EEtt}[1]{\mathbb{E}_{t-1}\left[#1\right]}
\newcommand{\PPi}[1]{\mathbb{P}_i\left[#1\right]}
\newcommand{\EEi}[1]{\mathbb{E}_i\left[#1\right]}
\newcommand{\PPc}[2]{\mathbb{P}\left[#1\left|#2\right.\right]}
\newcommand{\PPs}[2]{\mathbb{P}_{#2}\left[#1\right]}
\newcommand{\PPcs}[3]{\mathbb{P}_{#3}\left[#1\left|#2\right.\right]}
\newcommand{\PPct}[2]{\mathbb{P}_t\left[#1\left|#2\right.\right]}
\newcommand{\PPcc}[2]{\mathbb{P}\left[\left.#1\right|#2\right]}
\newcommand{\PPcct}[2]{\mathbb{P}_t\left[\left.#1\right|#2\right]}
\newcommand{\PPcci}[2]{\mathbb{P}_i\left[\left.#1\right|#2\right]}
\newcommand{\EEc}[2]{\mathbb{E}\left[#1\left|#2\right.\right]}
\newcommand{\EEcc}[2]{\mathbb{E}\left[\left.#1\right|#2\right]}
\newcommand{\EEcs}[3]{\mathbb{E}_{#3}\left[\left.#1\right|#2\right]}
\newcommand{\EEcct}[2]{\mathbb{E}_t\left[\left.#1\right|#2\right]}
\newcommand{\EEcctt}[2]{\mathbb{E}_{t-1}\left[\left.#1\right|#2\right]}
\newcommand{\EEcci}[2]{\mathbb{E}_i\left[\left.#1\right|#2\right]}
\renewcommand{\th}{\ensuremath{^{\mathrm{th}}}}
\def\argmin{\mathop{\mbox{ arg\,min}}}
\def\argmax{\mathop{\mbox{ arg\,max}}}
\newcommand{\ra}{\rightarrow}

\newcommand{\bone}{\bm{1}}

\newcommand{\siprod}[2]{\langle#1,#2\rangle}
\newcommand{\iprod}[2]{\left\langle#1,#2\right\rangle}
\newcommand{\biprod}[2]{\bigl\langle#1,#2\bigr\rangle}
\newcommand{\Biprod}[2]{\Bigl\langle#1,#2\Bigr\rangle}
\newcommand{\norm}[1]{\left\|#1\right\|}
\newcommand{\onenorm}[1]{\norm{#1}_1}
\newcommand{\twonorm}[1]{\norm{#1}_2}
\newcommand{\infnorm}[1]{\norm{#1}_\infty}
\newcommand{\spannorm}[1]{\norm{#1}_{\text{sp}}}
\newcommand{\opnorm}[1]{\norm{#1}_{\text{op}}}
\newcommand{\Hnorm}[1]{\norm{#1}_{\HH}}
\newcommand{\ev}[1]{\left\{#1\right\}}
\newcommand{\pa}[1]{\left(#1\right)}
\newcommand{\abs}[1]{\left|#1\right|}
\newcommand{\bpa}[1]{\bigl(#1\bigr)}
\newcommand{\Bpa}[1]{\Bigl(#1\Bigr)}
\newcommand{\BPA}[1]{\Biggl(#1\Biggr)}
\newcommand{\sign}{\mbox{sign}}

\newcommand{\wh}{\widehat}
\newcommand{\wt}{\widetilde}


\newcommand{\loss}{\ell}
\newcommand{\hloss}{\wh{\loss}}


\newcommand{\lambdamin}{\lambda_{\min}}
\newcommand{\lambdamax}{\lambda_{\max}}
\newcommand{\hr}{\wh{r}}
\newcommand{\hR}{\wh{R}}
\newcommand{\tx}{\wt{x}}
\newcommand{\tX}{\wt{X}}
\newcommand{\htheta}{\wh{\theta}}
\newcommand{\hq}{\wh{q}}
\newcommand{\hv}{\wh{v}}
\newcommand{\hTheta}{\wh{\Theta}}
\newcommand{\hf}{\wh{f}}
\newcommand{\hs}{\wh{s}}
\newcommand{\bA}{\overline{A}}
\newcommand{\bW}{\overline{\W}}
\newcommand{\deff}{d_{\text{eff}}}

\newcommand{\Sp}{\Sigma^+}

\newcommand{\Cinf}{C_\infty}
\newcommand{\tw}{\wt{w}}
\newcommand{\br}{\overline{r}}
\newcommand{\hM}{\wh{M}}
\newcommand{\hT}{\wh{T}}
\newcommand{\tM}{\wt{M}}
\newcommand{\bw}{\overline{w}}
\newcommand{\hX}{\wh{X}}
\newcommand{\hZ}{\wh{Z}}
\newcommand{\tZ}{\wt{Z}}
\newcommand{\hmu}{\wh{\mu}}
\newcommand{\tmu}{\wt{\mu}}
\newcommand{\hP}{\wh{P}}
\newcommand{\tP}{P^+}
\newcommand{\thP}{{\hP}^+}
\newcommand{\hS}{\wh{\Sigma}}
\newcommand{\hw}{\wh{w}}
\newcommand{\hx}{\wh{\boldsymbol{x}}}
\newcommand{\hQ}{\wh{Q}}
\newcommand{\transpose}{^\mathsf{\scriptscriptstyle T}}

%\usepackage{todonotes}
\definecolor{PalePurp}{rgb}{0.66,0.57,0.66}
%\newcommand{\todoG}[1]{\todo[color=PalePurp!30]{#1}}
%\newcommand{\todoGI}[1]{\todo[inline,color=PalePurp!30]{#1}}
\newcommand{\redd}[1]{\textcolor{red}{#1}}
\newcommand{\bblue}[1]{\textcolor{blue}{#1}}

%\newcommand{\regret}{\mathrm{regret}}
\newcommand{\regret}{\mathfrak{R}_T}
\newcommand{\regretK}{\mathfrak{R}_K}

%\newcommand{\qed}{\hfill\BlackBox\\[2mm]}

\newcommand{\R}{\boldsymbol{R}}


\newcommand{\hL}{\wh{L}}

\newcommand{\expexpexp}{\textsc{Exp3}\xspace}
\newcommand{\linexprl}{\textsc{LinExp3RL}\xspace}
\newcommand{\linexpreal}{\textsc{RealLinExp3}\xspace}
\newcommand{\linexprobust}{\textsc{RobustLinExp3}\xspace}
\newcommand{\linucb}{\textsc{LinUCB}\xspace}

\newcommand{\antoine}[1]{%
    \ifmmode
    \text{\textcolor{blue}{[Antoine: #1]}}
    \else
    \textcolor{blue}{[Antoine: #1]}
    \fi
}

\newcommand{\tmpassumption}[1]{%
    \ifmmode
    \text{\textcolor{orange}{[Assumption: #1]}}
    \else
    \textcolor{orange}{[Assumption: #1]}
    \fi
}

% common abbreviations
\makeatletter
\DeclareRobustCommand\onedot{\futurelet\@let@token\@onedot}
\def\@onedot{\ifx\@let@token.\else.\null\fi\xspace}

\def\cf{\textit{c.f}\onedot}
\def\eg{\textit{e.g}\onedot}
\def\etal{\textit{et al}\onedot}
\def\etc{\textit{etc}\onedot}
\def\ie{\textit{i.e}\onedot}
\def\wrt{i.i.d\onedot}
\def\vs{\textit{vs}\onedot}
\def\wrt{w.r.t\onedot}
\makeatother

% surrounding symbols
\newcommand{\spr}[1]{\left( #1 \right)}
\newcommand{\sbr}[1]{\left[ #1 \right]}
\newcommand{\sdbr}[1]{\left[\!\left[ #1 \right]\!\right]}
\newcommand{\scbr}[1]{\left\{ #1 \right\}}
\newcommand{\inp}[1]{\left\langle #1 \right\rangle}

% basic algebra symbols
\newcommand{\sumiK}{\sum_{i=1}^K}
\newcommand{\sumjK}{\sum_{j=1}^K}
\newcommand{\sumkK}{\sum_{k=1}^K}
\newcommand{\sumtT}{\sum_{t=1}^T}
\newcommand{\sumst}{\sum_{s=1}^t}
\newcommand{\sumtinfty}{\sum_{t=0}^\infty}
\newcommand{\sumsinfty}{\sum_{s=0}^\infty}

% \newcommand{\abs}[1]{\left| #1 \right|}
\newcommand*\diff{\mathop{}\!\mathrm{d}}
\usepackage{mathtools} % for \mathclap
\newcommand{\given}{\mathrel{}\middle|\mathrel{}}
\newcommand{\KL}{\mathcal{D}_{\text{KL}}}
% \newcommand{\norm}[1]{\left\| #1 \right\|}
% \newcommand{{\transpose}}{^\mathsf{\scriptscriptstyle T}}
\newcommand{\upplus}{\mathsf{\scriptscriptstyle +}} % superscript plus

% \newcommand{\wh}[1]{\widehat{#1}}
\usepackage{mathtools}  % for '\mathrlap' command (necessary)
\newcommand{\custombar}[3]{%
    \mathrlap{\hspace{#2}\overline{\scalebox{#1}[1]{\phantom{\ensuremath{#3}}}}}\ensuremath{#3}
} % random example: \custombar{0.8}{.5pt}{y}

% blackboard bold letters
\newcommand{\bbC}{\mathbb{C}}
\newcommand{\bbE}{\mathbb{E}}
\newcommand{\bbF}{\mathbb{F}}
\newcommand{\bbI}{\mathbb{I}}
\newcommand{\bbN}{\mathbb{N}}
\newcommand{\bbP}{\mathbb{P}}
\newcommand{\bbQ}{\mathbb{Q}}
\newcommand{\bbR}{\mathbb{R}}
\newcommand{\bbRnnK}{\mathbb{R}_{\geq 0}^K}
\newcommand{\bbRpK}{\mathbb{R}_{>0}^K}
\newcommand{\bbV}{\mathbb{V}}
\newcommand{\bbZ}{\mathbb{Z}}

% bold letters
\newcommand{\bfone}{\mathbf{1}}
\newcommand{\bfe}{\mathbf{e}}
\newcommand{\bfu}{\mathbf{u}}
\newcommand{\bfv}{\mathbf{v}}
\newcommand{\bfw}{\mathbf{w}}
\newcommand{\bfx}{\mathbf{x}}
\newcommand{\bfy}{\mathbf{y}}
\newcommand{\bfyt}{\mathbf{y}_t}

% caligraphic letters
\newcommand{\cA}{\mathcal{A}}
\newcommand{\cB}{\mathcal{B}}
\newcommand{\cC}{\mathcal{C}}
\newcommand{\cD}{\mathcal{D}}
\newcommand{\cE}{\mathcal{E}}
\newcommand{\cF}{\mathcal{F}}
\newcommand{\cG}{\mathcal{G}}
\newcommand{\cH}{\mathcal{H}}
\newcommand{\cI}{\mathcal{I}}
\newcommand{\cK}{\mathcal{K}}
\newcommand{\cL}{\mathcal{L}}
\newcommand{\cM}{\mathcal{M}}
\newcommand{\cN}{\mathcal{N}}
\newcommand{\cO}{\mathcal{O}}
\newcommand{\cP}{\mathcal{P}}
\newcommand{\cQ}{\mathcal{Q}}
\newcommand{\cR}{\mathcal{R}}
\newcommand{\cS}{\mathcal{S}}
\newcommand{\cT}{\mathcal{T}}
\newcommand{\cU}{\mathcal{U}}
\newcommand{\cV}{\mathcal{V}}
\newcommand{\cW}{\mathcal{W}}
\newcommand{\cX}{\mathcal{X}}
\newcommand{\cY}{\mathcal{Y}}
\newcommand{\cZ}{\mathcal{Z}}

% fraktur letters
\newcommand{\fkB}{\mathfrak{B}}
\newcommand{\fkR}{\mathfrak{R}}

% math operators
% \DeclareMathOperator*{\argmax}{arg\,max}
% \DeclareMathOperator*{\argmin}{arg\,min}
% \DeclareMathOperator{\opE}{E}
% \DeclareMathOperator{\opP}{P}
% \DeclareMathOperator{\ophP}{\widehat{P}}
% \DeclareMathOperator{\ophPt}{\widehat{P}_t}
% \DeclareMathOperator{\ophPk}{\widehat{P}_k}
% \DeclareMathOperator{\opPplus}{P_t^\mathsf{\scriptscriptstyle +}}
% \DeclareMathOperator{\opPplusk}{P_k^\mathsf{\scriptscriptstyle +}}
% \DeclareMathOperator{\ophPplus}{\widehat{P}_t^\mathsf{\scriptscriptstyle +}}
% \DeclareMathOperator{\ophPplusk}{\widehat{P}_k^\mathsf{\scriptscriptstyle +}}

\newcommand{\opE}{E}
\newcommand{\opP}{P}
\newcommand{\ophP}{\widehat{P}}
\newcommand{\ophPt}{\widehat{P}_t}
\newcommand{\ophPk}{\widehat{P}_k}
\newcommand{\opPplus}{P_t^\mathsf{\scriptscriptstyle +}}
\newcommand{\opPplusk}{P_k^\mathsf{\scriptscriptstyle +}}
\newcommand{\ophPplus}{\widehat{P}_t^\mathsf{\scriptscriptstyle +}}
\newcommand{\ophPplusk}{\widehat{P}_k^\mathsf{\scriptscriptstyle +}}



\DeclareMathOperator{\co}{co}
\DeclareMathOperator{\optrace}{trace}
\DeclareMathOperator{\Diag}{Diag}  % diagonal matrices
\DeclareMathOperator{\diam}{diam}  % diameter
\DeclareMathOperator{\dom}{dom}  % domain
\DeclareMathOperator{\interior}{int}
\DeclareMathOperator{\Lip}{Lip} % lipschitz
\DeclareMathOperator{\scvx}{sc}  % strong convexity
% \DeclareMathOperator{\regret}{\texttt{regret}}
\DeclareMathOperator{\regretplus}{\mathfrak{R}_T^\mathsf{\scriptscriptstyle +}}
\DeclareMathOperator{\regretKplus}{\mathfrak{R}_K^\mathsf{\scriptscriptstyle +}}
\DeclareMathOperator{\regretIL}{\mathfrak{R}_K^\mathrm{I\hspace{.05em}L}}%\regretK^{\mathrm{I\hspace{.05em}L}}}
% \DeclareMathOperator{\sign}{sign}
% \DeclareMathOperator{\tr}{tr}  % trace
\DeclareMathOperator{\CB}{CB}
\DeclareMathOperator{\kl}{kl}

% others
\newcommand{\simplex}{\Delta}
\newcommand{\piunif}{\pi_{\texttt{unif}}}
\newcommand{\QMAX}{Q_{\max}}
%\newcommand{\VMAX}{V_{\max}}
\newcommand{\VMAX}{Q_{\max}}
\newcommand{\LMAX}{L_{\max}}
\newcommand{\WMAX}{W_{\max}}
\newcommand{\reset}{reset\xspace}
\DeclareMathOperator{\LSE}{LSE}

\newcommand{\CG}{\mathcal{G}\xspace}
\newcommand{\CV}{\mathcal{V}\xspace}
\newcommand{\CE}{\mathcal{E}\xspace}
\newcommand{\CA}{\mathcal{A}\xspace}
\newcommand{\CF}{\mathcal{F}\xspace}
\newcommand{\CR}{\mathcal{R}\xspace}
\newcommand{\CB}{\mathcal{B}\xspace}
\newcommand{\CX}{\mathcal{X}\xspace}
\newcommand{\CK}{\mathcal{K}\xspace}
\newcommand{\CM}{\mathcal{M}\xspace}
\newcommand{\CC}{\mathcal{C}\xspace}
\newcommand{\CL}{\mathcal{L}\xspace}
\newcommand{\CI}{\mathcal{I}\xspace}
\newcommand{\CQ}{\mathcal{Q}\xspace}
\newcommand{\CO}{\mathcal{O}\xspace}
\newcommand{\CP}{\mathcal{P}\xspace}
\newcommand{\CS}{\mathcal{S}\xspace}
\newcommand{\CT}{\mathcal{T}\xspace}
\newcommand{\CJ}{\mathcal{J}\xspace}
\usepackage[para]{footmisc}
\usepackage{subfig}
% \usepackage{subcaption}
% \usepackage{array}
% \usepackage{colortbl}



\newcommand{\MainAlg}{\texttt{MainAlg}}
\renewcommand{\phi}{\varphi}

\newcommand{\rewardbias}{\texttt{reward-bias}}
\newcommand{\modelbias}{\texttt{model-bias}}


% tikz figures
\usepackage{tikz}
\usetikzlibrary{arrows.meta}
\usetikzlibrary{decorations.pathreplacing}
\usetikzlibrary{positioning}
\usetikzlibrary{automata}

% algorithm comments
\usepackage{transparent}
\newcommand{\algcomment}[1]{\textcolor{blue!70!black}{\transparent{0.7}\small{\texttt{\textbf{\#\hspace{2pt}#1}}}}}
\newcommand{\algcommentlight}[1]{\textcolor{blue!70!black}{\transparent{0.5}\footnotesize{\texttt{\textbf{\#\hspace{2pt}#1}}}}}
\newcommand{\algcommentbig}[1]{\textcolor{blue!70!black}{\footnotesize{\texttt{\textbf{/* #1~*/}}}}}
\newcommand{\algcommentbiglight}[1]{\textcolor{blue!70!black}{\transparent{0.5}\footnotesize{\texttt{\textbf{/* #1~*/}}}}\vspace{2pt}}
\newcommand{\algcolor}[1]{\textcolor{blue!70!black}{#1}}
\newcommand{\algspace}{\hspace{\algorithmicindent}}


% not sure what this is
\newlength{\minipagewidth}
\setlength{\minipagewidth}{\linewidth}
\setlength{\fboxsep}{3mm}
\addtolength{\minipagewidth}{-\fboxrule}
\addtolength{\minipagewidth}{-\fboxrule}
\addtolength{\minipagewidth}{-\fboxsep}
\addtolength{\minipagewidth}{-\fboxsep}
\newcommand{\bookbox}[1]{
\par\medskip\noindent
\framebox[\linewidth]{
\begin{minipage}{\minipagewidth}
{#1}
\end{minipage} } \par\medskip }


% for table of contents appendix
\usepackage{tocloft}
\addtocontents{toc}{\protect\setcounter{tocdepth}{0}}

\usepackage{array}
\newcolumntype{P}[1]{>{\centering\arraybackslash}p{#1}}
\newcolumntype{M}[1]{>{\centering\arraybackslash}m{#1}}

\begin{document}

\maketitle

\begin{abstract}%
  We study the problem of reinforcement learning in infinite-horizon discounted linear Markov decision processes (MDPs), and propose the first computationally efficient algorithm achieving near-optimal regret guarantees in this setting. Our main idea is to combine two classic techniques for optimistic exploration: additive exploration bonuses applied to the reward function, and artificial transitions made to an absorbing state with maximal return. We show that, combined with a regularized approximate dynamic-programming scheme, the resulting algorithm achieves a regret of order $\tilde{\mathcal{O}} (\sqrt{d^3 (1 - \gamma)^{- 7 / 2} T})$, where $T$ is the total number of sample transitions, $\gamma \in (0,1)$ is the discount factor, and $d$ is the feature dimensionality. The results continue to hold against adversarial reward sequences, enabling application of our method to the problem of imitation learning in linear MDPs, where we achieve state-of-the-art results.
\end{abstract}

\begin{keywords}%
  Optimistic exploration, discounted MDPs, linear MDPs, imitation learning%
\end{keywords}

\documentclass[../main.tex]{subfiles}
\graphicspath{{../images/}}
\makeatletter
\def\input@path{{../images/}}
\makeatother
\begin{document}
\section{Introduction}
\begin{figure}
\centering
\begin{tikzpicture}
\node[inner sep=0pt] (ws) at (0, 0) {
\includegraphics[height=.4\textwidth, trim={10cm 0 10cm 0},clip]{world_space.png}};
\node[inner sep=0pt] (cs) at (6,0) {\includegraphics[height=.4\textwidth, trim={10cm 1cm 10cm 4cm},clip]{conf_space.png}};
\end{tikzpicture}
\vspace{-5pt}
\label{fig:pbrm_intro}
\caption{\textbf{Left}: Shows world space obstacles as grey spheres. Robots start and goal configuration is colored red and green, respectively. Configurations along the computed path are colored transparent blue. \textbf{Right:} Mapped world space scenario to configuration space. Obstacle region is the grey mesh. Red spheres are collision-free regions computed by the neural SCDF. The optimized shortest path in the convex corridor is the blue curve.}
\vspace{-25pt}
\end{figure}
Motion planning is the problem of finding a collision-free trajectory that connects a given start and goal configuration. The planning takes place in the configuration space of the robot. For single body robots, like mobile robots or drones, the configuration space and the world space are usually the same. This simplifies the planning, since explicit obstacle representations are available which enables geometrical tools like separating hyperplanes, smallest distance to obstacles etc., to be used when designing motion planning algorithms. For multi-body robots like manipulators, the situation is completely different. The world space obstacles are usually mapped to non-convex regions, and to make the problem even harder, the mapping is usually not known. Forming explicit representations of the obstacle region in the configuration space is usually too expensive or intractable. Despite all of this, sampling based planners are used with great success, which mainly is due to their use of implicit representations of the obstacle region. The basic idea is to construct a graph in the configuration space that covers and connects the collision-free region. From this graph, a path can be extracted that connects a given start and goal configuration. The approach is computationally expensive, since the graph is constructed with the smallest geometrical building block available, points, which represents a collision-check. Furthermore, the extracted paths from the graph are non-smooth and jagged due to the stochastic nature of the approach. This adds an additional post-processing step to the process, where the paths are shortcutted and smoothened, before the path can be used for tracking. Clearly a lot of time is invested to form this graph and produce smooth paths. Thus, if the obstacles start to move, then all of this work is done in no use, since all points that make up this graph need to be re-verified, which is simply too time consuming to be done in real time.
\\\\
In this work, we want to address the existing drawbacks of the sampling based planners. Our main contribution is an improved motion planner where each vertex in the graph covers a collision-free region in the form of a sphere instead of a point and where the edges are formed with neighboring intersecting spheres. This representation has the advantage of instead of returning piecewise linear paths, returning a sequence of overlapping spheres, i.e. a convex corridor, that connects a given start and goal configuration, illustrated in Figure \ref{fig:pbrm_intro}. This convex corridor allows us to use convex optimization to produce smooth trajectories, instead of computationally expensive post-processing methods. The representation further allows us to estimate the coverage of the collision-free space, which gives us awareness and feedback in the offline roadmap construction phase. Finally, our representation is simple to adapt to moving obstacles, simply requery for the new radii and recheck for intersections. 
\\\\
The spherical collision-free regions are formed using a signed distance function (SDF), which is a function that returns the smallest distance from an arbitrary point to the boundary of an obstacle. As the name implies, the distance is signed, thus if the point is inside the obstacle it is negative otherwise positive. If the distance is positive, a sphere with radius equal to the distance is guaranteed to cover a collision-free region. Using an SDF in motion planning is not new, but what is novel about our approach is that we express the distance in the configuration space instead of the world space and by doing so allows us to form these convex collision-free regions. We refer to the resulting SDF as a signed configuration distance function (SCDF). Computing an SCDF analytically is non-trivial, our approach is therefore to parameterize the SCDF with a deep neural network and learn the mapping by supervised learning. Our resulting neural SCDF can compute distances for different parameter values of obstacle shapes and we also show how multiple distances can be combined, thus making our approach flexible.
\section{Related work}
Motion planning algorithms can roughly be divided into three families, grid-based, sampling based and optimization based methods. Grid-based methods (GBM) discretize the planning space from which a graph is then compiled. A standard search method is A$^\star$ \citep{a_star}, which is classified as an \textit{informed} search method, since it employs a heuristic function to speed up the search. A$^\star$ guarantees to return an optimal path at the level of discretization used. GBMs usually discretize the planning space by a regular lattice and this limits the GBMs to problems with low dimensionality due to the curse of dimensionality. Thus, GBMs are usually limited to single-body robots where the degrees of freedom (DOF) are low. To overcome the inherent scaling problem with the GBMs, stochastic methods are usually used for multi-body robots. These methods are termed as sampling-based methods (SBM) and core members within this family are the rapidly-exploring random trees (RRT) \citep{rrt} and the probabilistic roadmap (PRM) \citep{prm}. RRT grows a tree from the start configuration and explores the collision-free region in a rapid way until it is able to connect to the goal region. RRT is usually improved by bi-directional planning \citep{rrt_connect}, i.e. an additional tree is grown from the goal configuration and the trees are tested for connection after any tree has been expanded. RRT is a single-query method, thus it searches for a path from scratch each time it is queried. Contrary to this, PRM is a multi-query method, which solves for multiple queries without starting from scratch. PRM does this by creating a roadmap (graph) that covers the collision-free space as an offline step. The graph is then used to solve for multiple queries. PRMs are used in cases where the environment does not change since the extra offline step is too computationally costly and needs to be re-done if the environment is changed. In our work, we address this inherent issue by using a different roadmap representation. Our vertices in the graph cover a collision-free region in the form of spheres and we form the edges by checking for intersecting spheres. If something in the environment changes, we recompute the spheres radii and recheck the intersections, without relying on collision detection. We use a trained neural network to compute the sphere radius, therefore querying for the radius can be done fast, hence our representation enables the PRM for dynamic environments.
\\\\
In the recent decades, optimization based methods (OBM) \citep{chomp, schulman, itomp, stomp} have been introduced as an alternative to SBM for multi-body robots. Like the SBM, the OBMs scale well to higher dimensional problems and produce smoother motion. It is common to use a SDF in the optimization since it is a smooth function, thus enabling gradient-based methods. However, the standard way of expressing the SDF is in world space. The distance therefore needs to be mapped to the configuration space by the forward kinematics. This mapping makes the optimization problem a non-linear program (NLP), which is computationally expensive to solve. Recently, a different approach has been proposed. In \cite{mp_gcs} motion planning is formulated as a convex optimization problem by using the graph of convex sets framework \citep{gcs}. The underlying idea is to decompose the collision-free space into intersecting convex sets from which a convex optimization problem is formulated. In cases where an explicit representation of the obstacles in the configuration space exists, like for single-body robots, creating collision-free convex regions can be done fast \citep{iris}. For multi-body robots, this is non-trivial. Existing work does this successfully \citep{iris_nlp, iris_c} by an optimization based approach, but the methods are still too time consuming to be used in the presence of moving obstacles. Our approach is instead to use deep learning to learn an SDF expressed in the configuration space. With this, we can query for shortest distances to the collision boundary, which allows us to expand spherical regions which are collision-free. Our approach is fast and therefore enables our suggested roadmap planner to be used in dynamic environments.
\\\\
Recent research has focused on learning collision detection \citep{fk_kernel_distance, diffco, graphdistnet} by predicting the signed distance between the robot links and the surrounding obstacles in the world space. The learned SDF is used in trajectory optimization but since the distance is expressed in the world space, the problem becomes an NLP and therefore takes a long time to solve. We take a novel approach and suggest to instead express the signed distance in the configuration space. This allows us to improve the PRM at the same time as it enables convex optimization for trajectory optimization, which runs faster and is more reliable than NLP solvers. In \cite{cspf} a learned signed distance function in the configuration space is proposed similar to our approach. However, their approach is restricted to point cloud representations, while we propose to represent the obstacles as parameterized geometric shapes, e.g. spheres. Furthermore, we also show how to use our learned SCDF to improve an existing roadmap planner.
\section{Problem formulation}
A robot is located in the world space, $\W \subset \R^3 $. The unique location of the robot is given by its configuration $\q \in \C$, where $\C$ is the configuration space. The set of points covered by the robots bodies at a certain configuration is expressed as $\B(\q) \subset \W$. The robot is surrounded by $\NrObst$ obstacles $\O = \bigcup_{i=1}^{\NrObst} \O_i$, where  $\O_i \subset \W$. The representation of the obstacle in the configuration space is the set $\C\O_i = \{\q \in \C \: |\: \B(\q) \cap \O_i \neq \emptyset \}$. The obstacle space is formed as $\Co = \bigcup_{i=1}^{\NrObst} \C \O_i$. The complement is referred to as the free space, $\Cf = \C \setminus \Co$. The path planning problem is a tuple, ($\Cf$, $\qStart$, $\qGoal$), where we want to connect a query pair, consisting of a start, $\qStart$, and goal configuration, $\qGoal$, with a geometric path, $\q(s): [0, 1] \mapsto \Cf$, such that $\q(0)=\qStart$ and $\q(1)=\qGoal$, or report correctly when such a path does not exist.
\end{document}


\section{Preliminaries}\label{sec:preliminaries}



%We denote by $(\Ac(x_\Ac),\Bc(x_\Bc))(z)$ a random execution of $\pi$ with private inputs $(x_\Ac,y_\Ac)$, and common input $z$.

%\Jnote{Move to DP}
% At the end of such an execution, the protocol outputs a public transcript denoted by the random variable $\trans_\pi(x_\Ac,x_\Ac,z)$ we denotes the common as $\out(\trans_\pi(x_\Ac,x_\Ac,z)$, and each party $\Pc \in \set{\Ac,\Bc}$ obtains his view denoted $\view^\Pc_\pi(x_\Ac,x_\Bc,z)$, which may also contain a ``local output'' \Jnote{Local} $\out^\Pc(x_\Ac,x_\Bc,z)$ (if the protocol specifies such an output). \Jnote{Common output, and parties output}


\subsection{Distributions and Random Variables}\label{sec:prelim:dist}
The support of a distribution $P$ over a finite set $\cS$ is defined by $\Supp(P) \eqdef \set{x\in \cS: P(x)>0}$. For a distribution or a random variable $D$, let $d\from D$ denote that $d$ was sampled according to $D$. Similarly,  for a set $\cS$, let $x \from \cS$ denote that $x$ is drawn uniformly from $\cS$, and denote by $\cU_{\cS}$ the uniform distribution over $\cS$. For a finite set $\cX$ and a distribution $C_X$ over $\cX$, we use the capital letter $X$ to denote the random variable that takes values in $\cX$ and is sampled according to $C_X$. The {\sf statistical distance} (\aka {\sf~variation distance}) of two distributions $P$ and $Q$ over a discrete domain $\cX$ is defined by $\sdist{P}{Q} \eqdef \max_{\cS\subseteq \cX} \size{P(\cS)-Q(\cS)} = \frac{1}{2} \sum_{x \in \cS}\size{P(x)-Q(x)}$. 
For a vector $x = (x_1,\ldots,x_n)$ and index $i\in [n]$, we let $x_{-i} = (x_1,\ldots,x_{i-1},x_{i+1},\ldots,x_n)$ and $x^{(i)} = (x_1,\ldots,x_{i-1}, -x_i, x_{i+1},\ldots,x_n)$, for a set $\cS \subseteq [n]$ we let $x_{\cS} = (x_i)_{i \in \cS}$ and $x_{-\cS} = (x_i)_{i \in [n]\setminus \cS}$, and for a vector $r \in \zo^n$ we let $x_r = (x_i)_{\set{i \colon r_i = 1}}$ and $x_{-r} = (x_i)_{\set{i \colon r_i = 0}}$.

%For $n \in \N$ we let $U_n$ be the uniform distribution over $\oo^n$, and let $S_n$ be the distribution induces by the sum of $n$ i.i.d.\ random variables, each is distributed according to $U_1$. Let $\cN(0,1)$ be the standard normal distribution.
%For a distribution $\cD$ and a function $f$, we define by $f(\cD)$ the distribution that is induced by the output of $f(x)$ for $x \from \cD$. 





% \begin{theorem}[\cite{McGregorMPRTV10}]\label{thm:sv-extracotr}
% 	\Enote{Remove if not needed}
% 	There is a constant $c$ to make the following holds. Let $X$ be an $\alpha$-SV source on $\{0,1\}^n$, let $Y$ be a source on $\{0,1\}^n$ with min-entropy at least $\beta n$ (independent from $X$), and let $Z=\ip{X,Y}\mbox{mod m}$ for some $m\in\mathbb{N}$. Then for every $\delta\in[0,1]$, the random variable $(Y,Z)$ is $\delta$-close to $(Y,U)$ where $U$ is uniform on $\mathbb{Z}_m$ and independent of $Y$, provided that
% 	$$
% 	n\geq c\cdot\frac{m^2}{\alpha\beta}\cdot\log(\frac{m}{\beta})\cdot\log(\frac{m}{\delta}).
% 	$$
% \end{theorem}



\Enote{I removed the definition of DP since it already appears in the intro}
\remove{
\subsection{Differential Privacy}\label{sec:prelim:DP}
We use the following standard definition of (information theoretic) differential privacy, due to \citet{DMNS06}. For notational convenience, we focus on databases over $\oo$.
\begin{definition}[Differentially private mechanisms]\label{def:mech}
	A randomized function $f\colon\oo^n\mapsto \zs$ is an {\sf $n$-size, $(\eps,\delta)$-differentially private mechanism} (denoted $(\eps,\delta)$-\DP) if for every neighboring $w,w'\in \oo^n$ and every function $g\colon \zs\mapsto \zo$, it holds that 
	$$
	\pr{g(f(w))=1}\leq e^{\eps}\cdot \pr{g(f(w'))=1} +\delta.
	$$ 	
	If $\delta=0$, we omit it from the notation.
\end{definition}
}


\subsubsection{Computational Differential Privacy}
There are several ways for defining computational differential privacy (see \cref{sec:related-works}). We use the most relaxed version due to \cite{BNO08}. For notational convenience, we focus on databases over $\oo$.
\begin{definition}[Computational differentially private mechanisms]\label{def:ComMech}
	A randomized function ensemble $f=\set{f_\pk\colon\oo^{n(\pk)}\mapsto \zs}$ is an {\sf $n$-size, $(\eps,\delta)$-computationally differentially private} (denoted $(\eps,\delta)$-$\CDP$) if for every poly-size circuit family $\set{\Ac_\pk}_{\pk\in \N}$, the following holds for every large enough $\pk$ and every neighboring $w,w'\in\oo^{n(\pk)}$:
	$$
	\pr{\Ac_\pk(f_\pk(w))=1}\leq e^{\eps(\pk)}\cdot \pr{\Ac_\pk(f_\pk(w'))=1} +\delta(\pk).
	$$ 
	If $\delta(\pk) = \negl(\pk)$, we omit it from the notation. 
\end{definition}



\subsubsection{Two-Party Differential Privacy}\label{sec:DP}
In this section we formally define distributed differential privacy mechanism (\ie protocols). %For the ease of notation, we consider protocol with no common input.

\begin{definition}\label{def:DP}%\Nnote{fix security parameter}
	A two-party protocol $\Pi=(\Ac,\Bc)$ is {\sf $(\eps,\delta)$-differentially private}, denoted $(\eps,\delta)$-$\DP$, if the following holds for every algorithm $\Dc$: let $\V^\Pc(x,y)(\pk)$ be the view of party $\Pc$ in a random execution of $\Pi(x,y)(1^\pk)$. Then for every $\pk,n \in \N$, $x\in \oo^n$ and neighboring $y,y'\in\oo^n$:
	\begin{align*}
	\pr{\Dc(V^\Ac(x,y)(\pk))=1}\le e^{\eps(\pk)}\cdot \pr{\Dc(V^\Ac (x,y')(\pk))=1}+\delta(\pk),
	\end{align*} 
	and for every $y\in \oo^n$ and neighboring $x,x'\in\oo^{n}$:
	\begin{align*}
	\pr{\Dc(V^\Bc(x,y)(\pk))=1}\le e^{\eps(\pk)}\cdot \pr{\Dc(V^\Bc (x',y)(\pk))=1}+\delta(\pk).
	\end{align*} 	
	Protocol $\Pi$ is {\sf $(\eps,\delta)$-computational differentially private}, denoted $(\eps,\delta)$-$\CDP$, if the above inequalities only hold for a non-uniform \ppt $\Dc$ and large enough $\pk$. We omit $\delta = \negl(\pk)$ from the notation. \footnote{Note that define we give for two-party differentially private protocols is a semi-honest definition, in which we ask for the security to hold when the parties interact in an honest execution of the protocol. Since we are proving a lower bound, starting from this weaker guarantee (as opposed to security against malicious players), yields a stronger result.}
\end{definition}
%We omit $\delta$ from the notation if $\delta$ is a negligible function of $n$.

%\Enote{simulation-based}
\begin{remark}[The definition for computational differential privacy we use]\label{rem:comDPChannel} 
	An alternative, stronger definition of computational differential privacy, known as simulation-based computational differential privacy, requires that the distribution of each party’s view be computationally indistinguishable from a distribution that ensures privacy in an information-theoretic sense. \cref{def:DP} is a weaker notion in comparison. Consequently, establishing a lower bound for a protocol that satisfies this weaker guarantee (as we do in this work) yields a stronger result.%Actually, our lower bound only requires the privacy to hold against \emph{uniform} external observer.
	%\Nnote{Maybe add: When only interesting in \Dp against external observer, the two definitions can be achieve using key-agreement and (single-party) \Dp mechanism. }
\end{remark}




\subsection{Useful Claims}
\remove{
In this section, we state generic lemmas and propositions that we will use later in our proofs.

The following lemma which we prove in \cref{sec:missing-proofs:distance-I}, measures the distance between two uniform stings conditioned one a random index $i$ either being fixed to $0$ or to $1$.

\def\distanceILemma{
    Let $R \la \zo^n$. For any (randomized) function $f:\{0,1\}^n\rightarrow \{0,1\}$ and $\alpha > 0$, it holds that
    \begin{align}\label{eq:f-alpha}
        \ppr{i \la [n]}{\size{\:\ex{f(R) \mid R_i = 0}-\ex{f(R) \mid R_i = 1}\:}\geq \alpha} \leq \frac{2}{n \alpha^2},
    \end{align}
    where the expectations are taken over $R$ and the randomness of $f$.
}

\begin{lemma}\label{lem:distance-I}
    \distanceILemma
\end{lemma}
}

The following two propositions state that given the output of a differentially private function, it is not possible to predict well even a random index (even if all other indexes are leaked). The first proposition handles the information-theoretic case and the second handles the computation case. Both propositions are proven in \cref{sec:missing-proofs:hard-to-guess}. 

\def\propHardToGuessInf{
    Let $f\colon \oo^n \rightarrow \cY$ be an $(\eps,\delta)$-\DP function, let $g \colon [n] \times \oo^{n-1} \times \cY \rightarrow \set{-1,1,\bot}$ be a (randomized) function, and let $X = (X_1,\ldots,X_n) \la \oo^n$. Then the following holds for every $i \in [n]$ where $X_i^* = g(i,X_{-i},f(X_1,\ldots,X_n))$:
    \begin{align*}
        \pr{X_i^* = X_i} \leq e^{\eps}\cdot \pr{X_i^* = -X_i} + \delta.
    \end{align*}
}

\begin{proposition}\label{prop:hard-to-guess-inf}
    \propHardToGuessInf
\end{proposition}


\def\propHardToGuessComp{
    Let $f = \set{f_{\pk} \colon \oo^{n(\pk)} \rightarrow \zo^{m(\pk)}}_{\pk \in \bbN}$ be an $(\eps,\delta)$-\CDP function ensemble, and let $\set{g_{\pk}}_{\pk \in \bbN}$ be a poly-size circuit family. Then, for large enough $\pk$ and $X = (X_1,\ldots,X_{n(\pk)}) \la \oo^{n(\pk)}$, the following holds for every $i \in [n(\pk)]$ where $X_i^* = g_{\pk}(i,X_{-i},f_{\pk}(X_1,\ldots,X_n))$:
    \begin{align*}
        \pr{X_i^* = X_i} \leq e^{\eps(\pk)}\cdot \pr{X_i^* = -X_i} + \delta(\pk).
    \end{align*}
}

\begin{proposition}\label{prop:hard-to-guess-comp}
    \propHardToGuessComp
\end{proposition}





\remove{
\Enote{Chao's old statement:}
\begin{lemma}\label{lem:distance-I-old}
        Let $R \la \zo^n$. 
	For any function $f:\{0,1\}^n\rightarrow \{0,1\}$ and $\alpha<0.01$, it holds that
	$$
	\Pr_{i\la[n]}\left[\: \size{\:\mathbb{E}[f(R) \mid R_i = 0]-\mathbb{E}[f(R) \mid R_i = 1]\:}\geq \alpha\right]\leq \frac{2+2\log(\frac{1}{\alpha})}{n\alpha^2}.
	$$
\end{lemma}
\begin{proof}
	Define $S_1=\{r \in \zo^n \colon f(r)=1\}$. Then for any $i\in[n]$, we have
	$$
	\begin{array}{rl}
		\size{\mathbb{E}[f(R) \mid R_i = 0]-\mathbb{E}[f(R) \mid R_i = 1]}
		&=\size{\Pr[R\in S_1|R_i=0]-\Pr[R\in S_1|R_i=1]}\\
		&=\size{\frac{\Pr[R_i=0|R\in S_1]\cdot\Pr[R\in S_1]}{\Pr[R_i=0]}-\frac{\Pr[R_i=1|R\in S_1]\cdot\Pr[R\in S_1]}{\Pr[R_i=1]}}\\
		&=\frac{2\size{S_1}}{2^n}\size{\Pr[R_i=0|R\in S_1]-\Pr[R_i=1|R\in S_1]}
	\end{array}
	$$
	When $|S_1|\leq \alpha\cdot 2^{n-1}$, we have $\size{\mathbb{E}[f(R) \mid R_i = 0]-\mathbb{E}[f(R) \mid R_i = 1]}\leq\frac{2\size{S_1}}{2^n}\leq \alpha$ for any $i\in[n]$. Hence, in the following, we assume $|S_1|> \alpha\cdot 2^{n-1}$.

	%Define $I_{bad}=\{i|\size{\Pr[R_i=0|R\in S_1]-\Pr[R_i=1|R\in S_1]}>2\alpha\}$ and $k=\size{I_{bad}}$, then for any $i\notin I_{bad}$, we have 
    %$$
    %\begin{array}{rl}
    %    2\alpha&\geq \size{\Pr[R_i=0|R\in S_1]-\Pr[R_i=1|R\in S_1]}\\
    %    &=\size{\frac{\Pr[R\in S_1|R_i=0]\cdot\Pr[R_i=0]}{\Pr[R\in S_1]}-\frac{\Pr[R\in S_1|R_i=1]\cdot\Pr[R_i=1]}{\Pr[R\in S_1]}}\\
    %    &=\size{\Pr[R\in S_1|R_i=0]-\Pr[R\in S_1|R_i=1]}\cdot\frac{1}{2\Pr[R\in S_1]}\\
    %    &\geq \size{\mathbb{E}[f(R) \mid R_i = 0]-\mathbb{E}[f(R) \mid R_i = 1]}\cdot \frac{1}{2},
    %\end{array}
    %$$ 
    %where the last inequality is because $\Pr[R\in S_1]\leq 1$. So that $\size{\mathbb{E}}[f(R) \mid R_i = 0]-\mathbb{E}[f(R) \mid R_i = 1]\leq %4\alpha$.
    Define $I_{bad}=\{i \colon \size{\Pr[R_i=0|R\in S_1]-\Pr[R_i=1|R\in S_1]} \geq 2\alpha\}$ and $k=\size{I_{bad}}$, and denote $I_{bad}=\{i_1,\dots,i_k\}$. Define $(X_{i_1}, \ldots X_{i_k}) = (R_{i_1},\dots,R_{i_k})\mid_{R \in S_1}$. 
    Consider the min-entropy
	$$
	\begin{array}{rl}
		H_{min}(X_{i_1},\dots,X_{i_k})&\leq H(X_{i_1},\dots,X_{i_k})\\
		&\leq \sum_{j=1}^k H(X_{i_j})\\
		&\leq k\cdot \left(-(\frac{1}{2}+2\alpha)\cdot\log(\frac{1}{2}+2\alpha)-(\frac{1}{2}-2\alpha)\cdot\log(\frac{1}{2}-2\alpha)\right)\\
            &=k\cdot \left(-(\frac{1}{2}+2\alpha)\cdot(\log(1+4\alpha)-1)-(\frac{1}{2}-2\alpha)\cdot(\log(1-4\alpha)-1)\right)\\
            &=k\cdot \left(1-(\frac{1}{2}+2\alpha)\cdot\log(1+4\alpha)-(\frac{1}{2}-2\alpha)\cdot\log(1-4\alpha)\right),
		
	\end{array}
	$$
	where $H_{min}(Y)$ is the minimum entropy of $Y$ and $H(Y)$ is the Shannon entropy of $Y$.\Enote{add to preliminaries.}
        The third inequality holds since by the definition of $I_{bad}$, for every $j \in [k]$ it holds that $\size{\pr{X_{i_j} = 1}-\pr{X_{i_j} = 0}} > 2\alpha$, and therefore $H(X_{i_j}) \leq H(1/2 + 2\alpha)$\Enote{define}.
	
	Therefore, there exists $b_1,\dots,b_k\in\{0,1\}$, such that 
	
	\begin{align}\label{eq:min-entropy-result}
		\Pr\left[(R_{i_1},\ldots,R_{i_k}) = (b_1,\ldots,b_k) \mid R\in S_1\right]
		&= \pr{(X_{i_1},\ldots,X_{i_k}) = (b_1,\ldots,b_k)}\\
		&= 2^{-H_{min}(X_{i_1},\dots,X_{i_k})}\nonumber\\
		&\geq 2^{k\cdot \left(-1+(\frac{1}{2}+2\alpha)\cdot\log(1+4\alpha)+(\frac{1}{2}-2\alpha)\cdot\log(1-4\alpha)\right)}.\nonumber
	\end{align}
	
	Let $S_{bad}=\{r \in \zo^n  \colon \set{(r_{i_1},\ldots,r_{i_k}) = (b_1,\ldots,b_k)} \land \set{r\in S_1}\}$.
	It holds that
	\begin{align*}
		|S_{bad}|
		&= \size{S_1} \cdot \Pr\left[(R_{i_1},\ldots,R_{i_k}) = (b_1,\ldots,b_k) \mid R\in S_1\right]\\
		&\geq \alpha\cdot 2^{n-1}\cdot2^{k\cdot \left(-1+(\frac{1}{2}+2\alpha)\cdot\log(1+4\alpha)+(\frac{1}{2}-2\alpha)\cdot\log(1-4\alpha)\right)},
	\end{align*} 
	where the inequality holds by \cref{eq:min-entropy-result} and since $\size{S_1} \geq \alpha\cdot 2^{n-1}$.
	Notice that any string in $S_{bad}$ depends on at most $n-k$ bits. It implies that $|S_{bad}|\leq 2^{n-k}$. Therefore, we have
	$$
	\begin{array}{rl}
		&2^{n-k}\geq \alpha\cdot 2^{n-1}\cdot2^{k\cdot \left(-1+(\frac{1}{2}+2\alpha)\cdot\log(1+4\alpha)+(\frac{1}{2}-2\alpha)\cdot\log(1-4\alpha)\right)} \\
		\Rightarrow& n-k \geq \log \alpha+n-1+k\cdot \left(-1+(\frac{1}{2}+2\alpha)\cdot\log(1+4\alpha)+(\frac{1}{2}-2\alpha)\cdot\log(1-4\alpha)\right)\\
		\Rightarrow& 1-\log \alpha \geq k\cdot((\frac{1}{2}+2\alpha)\cdot\log(1+4\alpha)+(\frac{1}{2}-2\alpha)\cdot\log(1-4\alpha))\\
		\Rightarrow& 1-\log \alpha \geq k\cdot(4\alpha\cdot\log(1+4\alpha)+(\frac{1}{2}-2\alpha)\cdot\log(1-16\alpha^2))\\
        \Rightarrow& 1-\log\alpha \geq k\cdot(15.9\alpha^2-8\alpha^2+32\alpha^3)=k\cdot(7.9\alpha^2+32\alpha^3)>0.5k\alpha^2\\
		\Rightarrow& k\leq \frac{2-2\log \alpha}{\alpha^2} = \frac{2+2\log (1/\alpha)}{\alpha^2},
	\end{array}
	$$
	Where the third transition holds since 
	\begin{align*}
		\lefteqn{(\frac{1}{2}+2\alpha)\cdot\log(1+4\alpha)+(\frac{1}{2}-2\alpha)\cdot\log(1-4\alpha)}\\
		&= 4\alpha\cdot\log(1+4\alpha) + (\frac{1}{2}-2\alpha)\paren{\log(1+4\alpha)+\log(1-4\alpha)}\\
		&= 4\alpha\cdot\log(1+4\alpha)+(\frac{1}{2}-2\alpha)\cdot\log(1-16\alpha^2),
	\end{align*}
	and the forth transition holds since $4\alpha\cdot\log(1+4\alpha)+(\frac{1}{2}-2\alpha)\cdot\log(1-16\alpha^2) > 15.9\alpha^2-8\alpha^2+32\alpha^3$ for $\alpha < 0.01$.
	Thus, we conclude that 
	$$
	\Pr_{i\la[n]}\left[\size{\mathbb{E}[f(R) \mid R_i=0]-\mathbb{E}[f(R) \mid R_i = 1]}\geq \alpha\right]\leq \frac{k}{n}\leq \frac{2+2\log (1/\alpha)}{n\alpha^2}.
	$$
\end{proof}
}


\subsection{Channels and Two-Party Protocols}\label{sec:protocol}

\paragraph{Channels.}A channel is simply a distribution of a pair of tuples defined as follows. 
\begin{definition}[Channels]\label{def:channel} A {\sf channel} $C_{(X,U)(Y,V)}$ of size $\isize$ over alphabet $\Sigma$ is a probability distribution over $(\Sigma^\isize \times\zo^\ast) \times(\Sigma^\isize \times\zo^\ast)$. The ensemble $C_{(X,U)(Y,V)}= \set{C_{(X_\pk,U_\pk)(Y_\pk,V_\pk)}}_{\pk\in \N}$ is an $\isize$-size channel ensemble, if for every $\pk\in \N$, $C_{(X_\pk,U_\pk)(Y_\pk,V_\pk)}$ is an $\isize(\pk)$-size channel. %We denote a channel of size one by a \emph{single-bit} channel. 
We refer to $X$ and $Y$ as the {\sf local outputs}, and to $U$ and $V$ as the {\sf views}.	
\end{definition}

We view a  channel as the experiment in which there are two parties $\Ac$ and $\Bc$.  Party $\Ac$ receives ``output'' $X$ and ``view'' $U$, and party $\Bc$ receives ``output'' $Y$ and ``view'' $V$. Unless stated otherwise, the channels we consider are over the alphabet $\Sigma = \oo$. We naturally identify channels with the distribution that characterizes their output.








\subsubsection{Two-Party Protocols}

A two-party protocol $\Pi=(\Ac,\Bc)$ is \ppt if the running time of both parties is polynomial in their input length. We let $\Pi(x,y)(z)$ or $(\Ac(x),\Bc(y))(z)$ denote a random execution of $\Pi$ on a common input $z$, and private inputs $x,y$.%We assume \wlg that a protocol has a common output (part of its transcript).\Jnote{This is not really the case we consider in this paper..}

\begin{definition}[Oracle-aided protocols]\label{def:ChannelAidedProtocol}
	In a two-party protocol $\Pi$ with oracle access to a {\sf protocol} $\Psi$, denoted $\Pi^\Psi$, the parties make use of the \textit{next-message function} of $\Psi$.\footnote{The function that on a partial view of one of the parties, returns its next message.} In a two-party protocol $\Pi$ with oracle access to a {\sf channel} $C_{Z W}$, denoted $\Pi^C$, the parties can jointly invoke $C$ for several times. In each call, an independent pair $(z,w)$ is sampled according to $C_{Z W}$, one party gets $z$, the other gets $w$.
\end{definition}


\begin{definition}[The channel of a protocol]\label{def:ChannlOfProtocol}
	For a no-input two-party protocol $\Pi= (\Ac,\Bc)$, we associate the channel $C_\Pi$, defined by $\C_\Pi= C_{(X, U),(Y, V)}$, where $X$ and $Y$ are the local outputs of $\Ac$ and $\Bc$ (respectively) and
	$U$ and $V$ are the local views of $\Ac$ and $\Bc$ (respectively).
    
	For a two-party protocol $\Pi$ that gets a security parameter $1^\pk$ as its (only, common) input, we associate the channel ensemble $ \set{C_{\Pi(1^\pk)}}_{\pk\in \N}$. 
\end{definition}

\begin{definition}[$(\alpha,\gamma)$-Accurate channel]\label{def:accurate-func}
	A channel $C = C_{(X, U),(Y, V)}$ is {\sf $(\alpha,\gamma)$-accurate for the function $f$}, if $\ppr{C}{\size{\out(V)-f(X,Y)}\leq \alpha}\ge \gamma$, where $\out(V)$ is the designated output.
    A channel ensemble $C_{(X, U),(Y, V)}= \set{C_{(X_\pk, U_\pk),(Y_\pk, V_\pk)}}_{\pk\in \N}$ is  $(\alpha,\gamma)$-accurate for  $f$ if $C_{(X_\pk, U_\pk),(Y_\pk, V_\pk)}$ is $(\alpha(\pk),\gamma(\pk))$-accurate for $f$, for every $\pk \in \N$.
\end{definition}

\subsubsection{Differentially Private Channels}\label{sec:DPChannel}
Differentially private channels are naturally defined as follows:
\begin{definition}[Differentially private channels]\label{def:DPChannel}
	An $n$-size channel $C = C_{(X, U),(Y, V)}$ with $X, Y$ over $\oo^n$ 
	is {\sf$(\eps,\delta)$-differentially private} (denoted $(\eps,\delta)$-$\DP$) if for every $x \in \Supp(X)$ there exists an $n$-size $(\eps,\delta)$-$\DP$ mechanisms $\Mc_x$ such that $(X,Y,U) \equiv (X,Y,\Mc_X(Y))$, and for every $y \in \Supp(Y)$ there exists an $n$-size $(\eps,\delta)$-$\DP$ mechanisms $\Mc_y'$ such that $(X,Y,V) \equiv (X,Y,\Mc_Y'(X))$. In addition, we say that the channel is \emph{uniform} if $X$ and $Y$ are independent random variables uniformly distributed in $\oo^n$. 
\end{definition}

\begin{definition}[Computational differentially private channels]\label{def:CDPChannel}
	An $n$-size channel ensemble $C = \set{C_{(X_\pk, U_\pk),(Y_\pk, V_\pk)}}_{\pk\in\N}$ with $X_\pk, Y_\pk$ over $\oo^n$ 
	is {\sf$(\eps,\delta)$-computationally differentially private} (denoted $(\eps,\delta)$-$\CDP$) if for every ensemble $\set{x_\pk \in \Supp(X_\pk)}_{\pk\in\N}$ there exists an $n$-size $(\eps,\delta)$-\CDP mechanisms ensemble $\set{\Mc_{x_\pk}}_{\pk\in\N}$ such that $(X_\pk,Y_\pk,U_\pk) \equiv (X_\pk,Y_\pk,\Mc_{X_\pk}(Y_\pk))$, for every $\pk\in\N$, and for every ensemble $\set{y_\pk \in \Supp(Y_\pk)}_{\pk\in\N}$ there exists an $n$-size $(\eps,\delta)$-$\CDP$ mechanisms ensemble $\set{\Mc'_{y_\pk}}_{\pk\in\N}$ such that $(X_\pk,Y_\pk,V_\pk) \equiv (X_\pk,Y_\pk,\Mc_{Y_\pk}'(X_\pk))$ for every $\pk\in \N$. In addition, we say that the channel is \emph{uniform} if $X_\pk$ and $Y_\pk$ are independent random variables uniformly distributed in $\{\pm 1\}^n$ for all $\pk\in\N$.
\end{definition}




% \begin{lemma}~\label{lem:dp-sv-source}
% 	Let $P$ be an $\varepsilon$-DP randomized protocol. Let $X$ and $Y$ be independent random variables uniformly distributed in $\{\pm 1\}^n$ and let random variable $\Pi(X,Y)$ denote the transcript of running $P(X,y)$. Then for every $\pi\in Supp(\Pi)$, the random variables corresponding to the inputs conditioned on transcript $\pi$, $X_\pi$ and $Y_\pi$, are independent $e^{-\varepsilon}$-strong SV source.
% \end{lemma}





\subsubsection{Weak Erasure Channel (\WEC)}

\begin{definition}[\WEC]\label{def:WEC}
	A channel $((O_A,V_A), (O_B,V_B))$ with $O_A \in \set{0,1}$ and $O_B \in \set{0,1,\bot}$ is a {\sf weak erasure channel}, denoted $(\alpha,p,q)$-$\WEC$, if:
	\begin{itemize}
		%\item $O_A\in \set{-1,1}$ and $O_B\in \set{-1,1,\bot}$.
		\item Random erasure: $\pr{O_B = \perp} = 1/2$.
		
		\item Agreement: $\pr{O_A\ne O_B\mid O_B\ne \bot}\le \alpha$.
		
		\item Secrecy:
		
		\begin{enumerate}
			\item For every algorithm $\Dc$ it holds that\label{WEC:item:A}
			\begin{align*}
				%\size{\pr{\Ac(O_A,V_A) = 1 \mid O_B \neq \perp} - \pr{\Ac(O_A,V_A) = 1 \mid O_B = \perp}} \le p
				\size{\pr{\Dc(V_A) = 1 \mid O_B \neq \perp} - \pr{\Dc(V_A) = 1 \mid O_B = \perp}} \le p
			\end{align*}
			(Alice doesn't know if $O_B = \perp$.)
			
			\item For every algorithm $\Dc$ it holds that\label{WEC:item:B}
			\begin{align*}
				\pr{\Dc(V_B) = O_A \mid O_B=\bot} \leq \frac{1+q}{2}.
			\end{align*}
			(i.e., if $O_B=\bot$, Bob don't know what is the value of $O_A$).
			
			%\item $SD((O_A U|O_B=\bot),(O_A U|O_B\ne \bot))\le p$ (The sender don't know if $O_B=\bot$).
			
			%\item $SD(V O_A|O_B=\bot,V(-O_A)|O_B=\bot)\le q$ (If $O_B=\bot$, Bob don't know what the value of $O_A$).
		\end{enumerate}
	\end{itemize}
   We say that a channel ensemble $C=\set{C_\pk}_{\pk\in N}$ is a {\sf computational weak erasure channel}, denoted $(\alpha,p,q)$-\CompWEC, if for every \ppt algorithm $\Dc$ and every sufficiently large $\pk\in\N$, $C_\pk$ satisfies the properties stated in the items above, where the secrecy property holds with respect to a \ppt algorithm $\Dc$. A protocol $\Lambda$ is said to be $(\alpha,p,q)$-$\CompWEC$, if the ensemble induces by the protocol (that is, $C=\set{C_{\Lambda(\pk)}}_{\pk\in\N}$) is $(\alpha,p,q)$-$\CompWEC$.  
\end{definition}



\subsubsection{Approximate Weak Erasure Channel (\AWEC)}\label{sec:AWEC}

\begin{definition}[\AWEC]\label{def:AWEC}
	A channel $C = ((O_A,V_A), (O_B,V_B))$ over $([-n,n] \times \zo^*) \times (([-n,n] \cup \bot)  \times \zo^*)$ is an {\sf approximate weak erasure channel}, denoted $(\ell,\alpha,p,q)$-\AWEC if:
	\begin{itemize}
		
		\item Random erasure: $\pr{O_B = \perp} = 1/2$.
		
		\item Accuracy: $\pr{\size{O_A - O_B} > \ell \mid O_B \ne \bot}\le \alpha$.
		
		\item Secrecy:
		
		\begin{enumerate}
			\item For every algorithm $\Dc$ it holds that\label{AWEC:item:A}
			\begin{align*}
				%\size{\pr{\Ac(O_A,V_A) = 1 \mid O_B \neq \perp} - \pr{\Ac(O_A,V_A) = 1 \mid O_B = \perp}} \le p
				\size{\pr{\Dc(V_A) = 1 \mid O_B \neq \perp} - \pr{\Dc(V_A) = 1 \mid O_B = \perp}} \le p
			\end{align*}
			(Alice doesn't know if $O_B=\bot$).
			
			\item For every algorithm $\Dc$ it holds that\label{AWEC:item:B}
			\begin{align*}
				\pr{\size{\Dc(V_B) - O_A} \leq 1000 \ell \mid O_B=\bot} \leq q.
			\end{align*}
			(i.e., if $O_B=\bot$, Bob can't estimate the value of $O_A$ with error $\leq 1000 \ell$).
		\end{enumerate}
	\end{itemize}
     We say that a channel ensemble $C=\set{C_\pk}_{\pk\in N}$ is a {\sf computational approximate weak erasure channel}, denoted $(\ell,\alpha,p,q)$-\CompAWEC, if for every \ppt algorithm $\Dc$ and every sufficiently large $\pk\in\N$, $C_\pk$ satisfies the properties stated in the items above. A protocol $\Gamma$ is said to be $(\ell,\alpha,p,q)$-$\CompAWEC$, if the ensemble induced by the protocol (that is, $C=\set{C_{\Gamma(\pk)}}_{\pk\in\N}$) is $(\ell,\alpha,p,q)$-$\CompAWEC$.  
\end{definition}

We will make use of the following lemma, which shows that for some choices of the parameters, \AWEC implies \WEC. The lemma is proven in \cref{sec:AWEC-to-WEC}.

\begin{lemma}\label{lemma:AWEC-to-WEC}
	For every $\ell> 0$, there exists a \ppt protocol $\Lambda = (\Pc_1,\Pc_2)$ such that given an oracle access to an $(\ell,\alpha,p,q)$-\AWEC $C$, the channel $\tilde{C}$ induced by $\Lambda^C$ is $(\alpha'=\alpha+0.001,\: p' = p ,\:  q' = 1/2 + 2(q+0.01))$-\WEC.
	Furthermore, the proof is constructive in a black-box manner:
	\begin{enumerate}
		\item There exists an oracle-aided \ppt algorithm $\Ec_1$ such that for every channel $C = ((\OA,\VA), (\OB,\VB))$ and algorithm $\Dc$ violating the \WEC secrecy property~\ref{WEC:item:A} of $\tilde{C}$, algorithm $\Ec_1^{\Dc}$ violates the \AWEC secrecy property~\ref{AWEC:item:A} of $C$.
		
		\item There exists an oracle-aided \ppt algorithm $\Ec_2$ such that for every channel $C = ((\OA,\VA), (\OB,\VB))$ and algorithm $\Dc$ violating the \WEC secrecy property~\ref{WEC:item:B} of $\tilde{C}$, algorithm $\Ec_2^{\Dc}$ violates the \AWEC secrecy property~\ref{AWEC:item:B} of $C$.
	\end{enumerate}
\end{lemma}

Since \cref{lemma:AWEC-to-WEC} is constructive, the following is an immediate corollary.
\begin{corollary}\label{cor:CompAWEC to CompWEC}
There exists an oracle aided \ppt protocol $\Lambda$, such that given a protocol $\Gamma$ that induces $(\ell,\alpha,p,q)$-\CompAWEC, it holds that $\Lambda^\Gamma$ is $(\alpha'=\alpha+0.001,\: p' = p ,\:  q' = 1/2 + 2(q+0.01))$-\CompWEC.  
\end{corollary}
\begin{proof}[Proof of \ref{cor:CompAWEC to CompWEC}]
Let $\Lambda$ be the \ppt algorithm guaranteed  by Lemma \ref{lemma:AWEC-to-WEC}. Given an $(\ell,\alpha,p,q)$-\CompAWEC protocol $\Gamma$, we define $\Lambda(\pk)={\Lambda^{\Gamma(\pk)}(\pk)}$. Assume towards a contradiction that $\Lambda$ is not a $(\alpha',p',q')$-\CompWEC. It follows that there exists a \ppt $\Dc$ that for infinity many $\pk\in\N$ contradicts one of the \WEC secrecy properties of channel ensemble $\set{C_{\Lambda(\pk)}}_{\pk\in\N}$. Fix $\pk\in\N$ for which this holds. By Lemma \ref{lemma:AWEC-to-WEC}, there exists a \ppt $\Ec^\Dc$ that for every such $\pk$  contradicts one of the secrecy properties of the channel $C_{\Gamma(\pk)}$. This implies that for infinity many $\pk\in\N$, $\Ec^\Dc$  contradict the secrecy of the channel ensemble $\set{C_{\Gamma(\pk)}}_{\pk\in\N}$, which is a contradiction since this would means that $\Gamma$ is not a $(\ell,\alpha,p,q)$-\CompAWEC.       
\end{proof}



\subsection{Oblivious Transfer (\OT)}

\paragraph{Secure Computation.}
We use the standard notion of securely computing a functionality, \cf  \cite{Goldreich04}.
\begin{definition}[Secure computation]\label{def:SFE}
	A two-party protocol {\sf securely computes a functionality $f$}, if it does so according to the real/ideal paradigm.   We add the term perfectly/statistically/computationally/non-uniform computationally, if the simulator's output is  perfect/statistical/computationally indistinguishable/  non-uniformly indistinguishable from  the real distribution.  The protocol have the above notions of security {\sf against semi-honest  adversaries}, if its security only  guaranteed to holds against an adversary that follows the prescribed protocol.   Finally, for the case of perfectly secure computation, we naturally apply the above notion also to the non-asymptotic case: the protocol with no security parameter perfectly  compute a functionality $f$.
	
	A two-party protocol {\sf securely computes a functionality ensemble $f$ with oracle to a channel $C$}, if it does so according to the above definition when the parties have access to a trusted party computing $C$. All the above adjectives naturally extend to this setting.
\end{definition}

\paragraph{Oblivious Transfer.}
The (one-out-of-two) oblivious transfer functionality is defined as follows.
\begin{definition}[oblivious transfer functionality $f_{\OT}$]\label{def:OTfunc}
	The oblivious transfer functionality over $\zo \times (\zs)^2$ is defined by  $f_{\OT} (i,(\sigma_0,\sigma_1)) = (\perp,\sigma_i)$.
\end{definition}
A protocol is $\ast$ secure OT,   for \\$\ast\in \set{\text{semi-honest statistically/computationally/computationally non-uniform}}$, if it  compute the $f_{\OT}$  functionality with $\ast$ security.





% \begin{definition}[Computational oblivious transfer, semi-honest model]
% A protocol $\Pi=(\Ac,\Bc)$ is a semi-honest 1-out-of-2 computational oblivious transfer (comp-OT) protocol if the following holds. Given a common input $1^{\pk}$, the parties $\Ac$ and $\Bc$ run the protocol $\Pi(1^\pk)$ (in an honest manner) and    
% $\Ac$ outputs $X=(m_1,m_2)\in \zo\times\zo$ and has a view $U$ and $\Bc$ outputs $Y=(i,\hat{m})\in\zo\times\zo$ and has a view $V$, and the following properties are satisfied:
% \begin{enumerate}
%     \item \textbf{Correctness:} 
%     $\pr{\hat{m}\neq m_i}<\negl(\pk).$ 
    
%     \item \textbf{A's Privacy:} For every \ppt $\Dc$ and every sufficiently large $\pk$:
%     $\pr{\Dc(V)=m_{i-1}}<(1+\negl(\pk))/2$
    
%     \item \textbf{B's Privacy:} For every \ppt $\Dc$ and every sufficiently large $\pk$:
%     $\pr{\Dc(U)=i}<(1+\negl(\pk))/2$  
% \end{enumerate}
% \end{definition}

We make use of the following useful results by Wullschleger on oblivious transfer amplification from weak channels.
\begin{theorem}[\cite{Wullschleger09}, from \WEC to statistically secure \OT]\label{thm:WEC TO OT IT}
    There exists an oracle aided protocol $\Pi$ such that the following holds: Given a $(\alpha,p,q)$-\WEC $C$, if $44(\alpha+p)\le 1-q$ then $\Pi^{C}(1^\pk)$ is a semi-honest statistically secure \OT.
\end{theorem}

The following computational version of \cref{thm:WEC TO OT IT} is implicit in \cite{Wullschleger09} and is based on the computational proof explicitly stated in \cite{Wul07} (see Section 6 in \cite{Wullschleger09} for discussion).   

\begin{theorem}[\cite{Wullschleger09,   Wul07}, from \CompWEC to computinally secure \OT]\label{thm:WEC TO OT Comp}
    There exists an oracle aided protocol $\Pi$ such that the following holds: Given a $(\alpha,p,q)$-\CompWEC protocol $\Lambda$, if $44(\alpha+p)\le 1-q$ then $\Pi^{\Lambda}$ is a semi-honest computational secure \OT.
\end{theorem}



% \begin{definition}[Computational 1-out-of-2 Oblivious Transfer, semi-honest model]
% A protocol $\Pi=(\Ac,\Bc)$ is a semi-honest 1-out-of-2 $(\eps,\alpha,\beta)$-oblivious transfer (OT) protocol if the following holds. 

% The parties $\Ac$ and $\Bc$ run the protocol (in an honest manner) and    
% $\Ac$ outputs $X=(m_1,m_2)\in \zo\times\zo$ and has a view $U$ and $\Bc$ outputs $Y=(i,\hat{m})\in\zo\times\zo$ and has a view $V$, and following properties are satisfied:
% \begin{enumerate}
%     \item \textbf{Correctness:} 
%     $\pr{\hat{m}\neq m_i}<\eps.$ 
    
%     \item \textbf{A's Privacy:} For every adversary $\Dc$:
%     $\pr{\Dc(V)=m_{i-1}}<(1+\alpha)/2$
    
%     \item \textbf{B's Privacy:} For every adversary $\Dc$: $\pr{\Dc(U)=i}<(1+\beta)/2$  
% \end{enumerate}
% \end{definition}
\section{The general case: Proof of \texorpdfstring{\Cref{thm:main-decomp}}{Theorem 1.6}}\label{sec:algo}

First, we show that data structure of \Cref{l:max_min_query} can be used to compute distances witnessed by shortest paths that pass through a constant-size separator.

\begin{lemma}\label{l:single_adhesion}
Fix a constant $k \in \mathbb{N}$. There exists an algorithm which as the input receives an edge-weighted graph $G$ on $n$ vertices and $m$ edges together with a partition of its vertices into three sets $A, B, C$ such that $|B| \leq k$ and there are no edges between $A$ and $C$, and as the output computes $\max_{c \in C} \dist(a, c)$ for every $a \in A$. The running time is $\Oh(m \log n + n \log^{k - 1} n)$.
\end{lemma}

\begin{proof}
Let $B = \{b_1, \ldots, b_k\}$. For any $a \in A, c \in C$, we have $\dist(a, c) = \min_{i \in [k]} \dist(a, b_i) + \dist(c, b_i)$. First, we run Dijkstra's algorithm from every vertex in $B$ to find $\dist(v, b_i)$ for every $v \in V(G)$ and $i \in [k]$. Next, we use \Cref{l:max_min_query} to construct a data structure $\mathbb{D}$ for the point set $\{(\dist(c, b_1), \dots, \dist(c, b_k))\colon c\in C\}\subseteq \mathbb{R}^k$. Now, the value $\max_{c \in C} \dist(a, c)$ for any given $a$ is equal to the answer of $\mathbb{D}$ to the query with argument $(\dist(a, b_1), \dots, \dist(a, b_k))$.
\end{proof}

After computing the distances over a constant-size separator, we will use the following observation to simplify one of the sides of the separation.

\begin{lemma}\label{l:inserting_paths}
Let $G$ be a edge-weighted connected graph and let $A, B, C$ be a partition of its vertices such that there are no edges between $A$ and $C$. For every pair of vertices $u, v \in B$, let $P_{u, v}$ be any shortest path from $u$ to $v$ with all internal vertices in $C$ (assuming such a path exists).

Let $G'$ denote a graph obtained from $G[A \cup B]$ by adding an edge from $u$ to $v$ of weight equal to the length of $P_{u, v}$, for all $u, v \in B$ for which $P_{u, v}$ exists. Then,  $$\dist_G(s, t) = \dist_{G'}(s, t)\qquad\textrm{for all }s,t\in A\cup B.$$
\end{lemma}
\begin{proof}
Let $G''$ be the graph obtained by adding new edges of $G'$ to $G$.
Fix any $s, t \in A \cup B$ and let $P$ denote the shortest path from $s$ to $t$ in $G''$ which minimizes the number of vertices from $C$ visited. Naturally, the weight of $P$ is equal $\dist_G(s, t)$. Assume that such path visits at least one vertex of $C$. Then, the path $P$ is of the form $s \xrightarrow{P_1} x \xrightarrow{P_2} y \xrightarrow{P_3} t$, where $x, y \in B$ and all the internal vertices of $P_2$ are in $C$. By the construction of $G'$, $P_2$ can be replaced with a direct edge from $x$ to $y$ of the same weight. We obtain a same weight path with a smaller number of vertices of $C$ visited, which is a contradiction. Therefore, $P$ is entirely contained in $A \cup B$, hence it exists in $G'$. This shows that $\dist_G(s, t) = \dist_{G'}(s, t)$.
\end{proof}


The next lemma encapsulates the main algorithmic content of the proof of \Cref{thm:main-decomp}. The algorithm will split the tree decomposition provided on input into smaller parts for which the eccentricities are easier to calculate. We use the following lemma to handle a single such part.
\begin{lemma}\label{l:star}
Fix constants $k, g \in \mathbb{N}, 0 < \delta < \frac{1}{54}$. Assume we are given $n \in \mathbb{N}$, an edge-weighted graph $G$ on at most $n$ vertices with a weight function $w \colon E(G) \to \mathbb{N}$, a vertex subset $A$ and a collection of non-empty vertex subsets $V_0, V_1, \dots, V_\ell$ satisfying the following conditions:
\begin{itemize}[nosep]
	\item The sum of weights of all the edges in $G$ is bounded by $\Oh(n)$.
	\item $V(G) \setminus A = V_0 \cup V_1 \cup \dots \cup V_\ell$.
	\item $|A| \leq k$.
	\item For every $i \in [\ell]$, $G[V_i \setminus V_0]$ is connected, $N_G(V_i \setminus V_0) = V_i \cap V_0$, $|V_i| = \Oh(n^\delta)$, and $|V_0 \cap V_i| \leq 4$.
	\item For all $i, j \in [\ell], i \neq j$, $V_i \setminus V_0$ and $V_j \setminus V_0$ are disjoint and non-adjacent in $G$.
	\item Every edge $uv \in E(G)$ with $u, v \not\in A$ is contained in $G[V_i]$ for some $i\in \{0,1,\ldots,\ell\}$.
	\item The graph obtained by taking $G[V_0]$ and adding a clique on $V_0 \cap V_i$ for every $i \in [\ell]$ has Euler genus bounded by $g$.
\end{itemize}
Then, we can compute the eccentricity of every vertex of $G$ in time $\Oh \left( n^{1 + \frac{150 + 54 \delta}{151}} \log^k n \right)$.
\end{lemma}

\begin{proof}
Fix $\delta' = \frac{1 + 97 \delta}{151}$; we have $\delta' - \delta = \frac{1 - 54\delta}{151} > 0$.
Let $E_i$ denote the set of edges with one endpoint in $V_i$ and the other endpoint in $V_i \setminus V_0$. For $i \in [\ell]$, we shall say that $V_i$ is {\em{heavy}} if the sum of weights of $E_i$ is larger than $n^{\delta'}$. Since the sets $E_i$ are pairwise disjoint and the total sum of weights of all the edges is bounded by $\Oh(n)$, the number of heavy subsets is bounded by $\Oh(n^{1 - \delta'})$. Without loss of generality, we may assume that $V_{\ell' + 1}, \dots, V_\ell$ are heavy and $V_1, \dots, V_{\ell'}$ are not, for some $\ell'\in \{0,\ldots,\ell\}$.


For any source vertex $s$, we can calculate distances from $s$ to every vertex of $G$  using breadth first search in time $\Oh(\sum_{e \in E(G)} w(e)) = \Oh(n)$.
In particular, for every $\ell' < i \leq \ell$, we can compute the distances from every vertex of $V_i$ to every vertex of $G$ in total time $\Oh(n^{2 - \delta' + \delta})$, because $$|V_{\ell'+1}\cup \ldots\cup V_{\ell}|\leq n^{1-\delta'}\cdot \Oh(n^\delta)=\Oh(n^{1-\delta'+
\delta}).$$
Additionally, we calculate distances $\dist_G(a, v)$ for every $a \in A, v \in V(G)$ in time $O(n)$.

For every $i \in [\ell]$ and $u,v \in V_0 \cap V_i$, there exists a shortest path $P_{i,u,v}$ from $u$ to $v$ with all internal vertices belonging to $V_i - V_0$ due to the assumption that $G[V_i - V_0]$ is connected and $N_G(V_i - V_0) = V_i \cap V_0$. Therefore, the distance from $u$ to $v$ is bounded by the sum of weights of edges in $E_i$. In particular, for $i \in [\ell']$, $\dist_G(u, v) \leq n^{\delta'}$.

We define $\widetilde{G}$ to be the graph obtained by taking $G[A \cup V_0 \cup \dots \cup V_{\ell'}]$ and applying the following operation for every $i \in \{\ell' + 1, \dots, \ell\}$:
for each pair of vertices $u, v \in A \cup (V_0 \cap V_i)$, add an edge in $\widetilde{G}$ between $u$ and $v$ with weight equal to the total weight of $P_{i,u,v}$. For a fixed $i, u$, we can find $P_{i, u, v}$ for all $v$ using breadth first search in time $\Oh(n)$. Taking a sum over all $i, u$, we get that $\tilde{G}$ can be computed in total time $\Oh(n^{2 - \delta'})$.


\begin{claim}\label{cl:wG}
The sum of the edge weights in $\widetilde{G}$ is $\Oh(n)$. Moreover, for all $u, v \in V(\widetilde{G})$, we have $\dist_{\widetilde{G}}(u, v) = \dist_{G}(u, v)$.
\end{claim}

\begin{proof}
Consider $i \in \{\ell' + 1, \dots, \ell\}$ and any $u, v \in A \cup (V_0 \cap V_i)$ for which we added an edge. Its weight is bounded by the sum of weights of edges in $E_i$. Therefore, the total weight of all edges added is at most
$$
\sum_{i \in \{\ell' + 1, \dots, \ell\}} \left( |A \cup (V_0 \cap V_i)|^2 \sum_{e \in E_i} w(e) \right) \leq (4 + k)^2 \sum_{e \in E(G)} w(e) = \Oh(n).
$$
This proves the first part of the claim.

For the second part of the claim, consider any $i \in \{\ell' + 1, \dots, \ell \}$ and observe that by our assumptions, $A \cup (V_0 \cap V_i)$ separates $(V_0 \cup \dots \cup V_{\ell'} \cup V_{i + 1} \cup \dots \cup V_\ell) \setminus V_i$ from $V_i \setminus V_0$. Hence it suffices to repeatedly apply \Cref{l:inserting_paths}.
\end{proof}

For every $u \in V(\widetilde{G})$, we have $\ecc_G(u) = \max(\ecc_{\widetilde{G}}(v), \max_{v \in V(G) \setminus V(\widetilde{G})} \dist_G(u, v))$. Note, that we already know all the distances $\dist_G(u, v)$ for $v \in V(G) \setminus V(\widetilde{G})$. Similarly, we can already compute $\ecc_G(u)$ for every $u \in V(G) \setminus V(\widetilde{G})$. Therefore, it remains to compute $\ecc_{\widetilde{G}}(v)$ for each $v \in V(\widetilde{G})$. Our goal is to show that this can be done efficiently using \Cref{l:main_ecc}.

Now, let $G'$ be the graph obtained from $\tilde{G}$ by replacing every edge $e$ non-indicent to $A$ with $w(e)\geq 2$ with a path of length $w(e)$ consisting of unit-weight edges. This operation again preserves the distances. Since the sum of edge weights in $\tilde{G}$ is of $\Oh(n)$, the total number of vertices in $G'$ is of $\Oh(n)$. For $0 \leq i \leq \ell'$, we write $V'_i$ to denote the set $V_i$ together with all the vertices added as a part of a path between two endpoints in $V_i$.
As $V_i$ is not heavy for each $i\in [\ell']$, we have
$$
|V'_i \setminus V'_0| \leq |V_i| + \sum_{e \in E_i} w(e) = \Oh(n^{\delta'})\qquad \textrm{for all }i\in [\ell'].
$$

Let $G_0$ denote the graph $G'[V'_0]$ and let $G_0^*$ denote the graph $G'- A$ with $V'_i - V'_0$ contracted to a single vertex $v_i^*$, for each $i \in [\ell']$; note that, all edges of $G_0$ and $G_0^*$ have unit weight.

\begin{claim}
	The graph $G_0^*$ is does not contain $K_{t}$ as a minor, where $t = \Oh(\sqrt{g})$.
\end{claim}

\begin{proof}
Let $\bar{G}_0$ denote the graph obtained by taking $G_0$ and adding a clique on $V_0 \cap V_i$ for every $i \in [\ell']$.
By lemma assumptions and the fact that subdividing edges does not increase the Euler genus, $\bar{G}_0$ has Euler genus at most $g$. In particular, $\bar{G}_0$ is $K_{t'}$-minor-free for some $t' = \Oh(\sqrt{g})$, because the Euler genus of $K_{t'}$ is $\Omega({t'}^2)$.

Similarly, let $\bar{G}_0^*$ be the graph obtained by taking $G_0^*$ and adding a clique on each $V_0 \cap V_i$.
Note, that $\bar{G}_0^* - \{v_1^*, \dots, v_{\ell'}^*\}$ is precisely $\bar{G}_0$. Let $t = \max(t', 6)$.
Recall that a minor model of a clique $K_t$ consists of $t$ pairwise vertex-disjoint connected subgraphs, called
branch sets, such that there is at least one edge between each pair of the branch sets.
Consider a minor model $\varphi$ of $K_{t}$ in $\bar{G}^*_0$.
Note that $\varphi$ cannot contain any singleton branch set of the form $\{v^*_i\}$, for the degree of $v^*_i$ in $\bar{G}^*_0$ is at most $4 < t - 1$. Furthermore, since $N_{\bar{G}^*_0}(v^*_i) = V_0 \cap V_i$, any branch set containing $v^*_i$ and at least one other vertex contains some $u \in V_0 \cap V_i$, and $N_{\bar{G}^*_0}(v^*_i)\subseteq N_{\bar{G}^*_0}(u)$, hence removing $v^*_i$ from this branch set preserves the model. Therefore, we can assume without loss of generality that all branch sets of $\varphi$ are disjoint from $\{v^*_1, \dots, v^*_{\ell'}\}$, hence $\varphi$ is a minor model of $K_{t}$ in $\bar{G}_0$. This is a contradiction, as $t \geq t'$ and $\bar{G}_0$ is $K_{t'}$-minor-free. Therefore, $\bar{G}_0^*$ is $K_t$-minor-free, hence $G_0^*$ also.
\end{proof}

Let $\rho' = \frac{2 - 108 \delta}{151} > 0$. The graph $G^*_0$ is a unit-weight graph and is $K_{t}$-minor-free.
Hence, by applying \Cref{t:r_division} to $G^*_0$ (with $\varepsilon = \rho'/2$)
we obtain an $n^{\rho'}$-division $\mathcal{R}_0$ in time $\Oh(n^{1 + \rho'})$.
We extend it to $G' - A$ by mapping every contracted vertex $v^*_i$ to $N_{G' - A}[V'_i - V'_0] = (V'_i - V'_0) \cup (V_0 \cap V_i)$. Formally, we put $V''_i \coloneqq N_{G' - A}[V'_i - V'_0]$ and 
$$
\mathcal{R} \coloneqq \left\{ (R_0 \cap V'_0) \cup \bigcup_{i \colon v^*_i \in R_0} V''_i \colon R_0 \in \mathcal{R}_0 \right\}.
$$

Now, we argue that $\mathcal{R}$ is a reasonable division of $G' - A$. Clearly, all sets $R \in \mathcal{R}$ are connected in $G' - A$. Pick any $R \in \mathcal{R}$ and let $R_0$ be its corresponding set in $\mathcal{R}_0$.
Every vertex $v^*_i$ is mapped to a set of size $\Oh(n^{\delta'})$, therefore
$$|R| \leq |R_0| \cdot \Oh(n^{\delta'}) = \Oh(n^{\rho' + \delta'}).$$

By our construction, for every $i \in [\ell']$, $R$ is either disjoint from $V'_i - V'_0$ or contains whole $N_{G' - A}[V'_i - V'_0]$. This means that no vertex belonging to any $V'_i - V'_0$ can be in $\partial R$, hence $\partial R \subseteq V'_0$.

Pick any $u \in \partial R \cap R_0$. Assume that $u \not\in \partial R_0$. Then every vertex of $N_{G_0^*}(u)$ must be in $R_0$, hence $N_{G - A'}(u) \subseteq R$, which is a contradiction. This means that $\partial R \cap R_0 \subseteq \partial R_0$.

Pick any $u \in \partial R - R_0$. Then, $u \in V_0 \cap V_i$ for some $i \in [\ell']$ such that $v_i^* \in R_0$. Moreover, $v_i^* \in \partial R_0$ and is adjacent to $u$ in $G_0^*$. The number of such $u$ is bounded by $4 |\partial R_0 \cap \{ v_1^*, \dots, v_{\ell'}^* \}|$.

Putting two cases together, we obtain:
$$
\sum_{R \in \mathcal{R}} |\partial R| = \sum_{R \in \mathcal{R}} \left(|\partial R \cap R_0| + |\partial R - R_0|\right) \leq \sum_{R_0 \in \mathcal{R}_0} \left(|\partial R_0| + 4 |\partial R_0 \cap \{ v_1^*, \dots, v_{\ell'}^* \}|\right) = \Oh(n^{1 - \frac{1}{2}\rho'}).
$$

It remains to show the following claim.

\begin{claim}
Pick any $R \in \mathcal{R}, s_R \in R$. The number of different distance profiles on $R$ relative to $s_R$ in $G' - A$ is of $\Oh(n^{48\rho' + 54\delta'})$.
\end{claim}
\begin{proof}
We look at every vertex $v \in V(G') \setminus A$ and consider three cases: $v \in R$, $v \in V'_0$, and $v \in V'_i \setminus (V'_0 \cup R)$ for some $i \in [\ell']$. By our construction, $R \cap V'_0$ is non-empty, hence w.l.o.g. we can assume that $s_R \in V'_0$ as whether two vertices have the same profile on $R$ is independent of the choice of the pivot vertex.

In the first case, there are at most $|R| = \Oh(n^{\rho' + \delta'})$ such vertices, hence they realise at most that many profiles.

In the second case, we want to observe that profile of any vertex $v \in V'_0$ on $R$ depends only on its profile on $R \cap V'_0$ (relative to $s_R$). Pick any $t \in R - V'_0$. Then $t \in V'_i - V'_0$ for some $i \in [\ell']$, $V_i \cap V_0 \subseteq R \cap V'_0$, and every path from $v$ to $t$ intersects $V_i \cap V_0$. In particular, distances from $v$ to vertices of $V_i \cap V_0$ determine its distance to $t$, which proves the observation.

Let $\tilde{G}_0$ denote the graph obtained by taking $G'[V'_0]$ and for every $i \in [\ell'], u, v \in V_0 \cap V_i$ adding a disjoint path from $u$ to $v$ of length $\dist(u, v)$. Let $P_i$ denote the vertex set of paths added between $V_0 \cap V_i$. For every $t \in V'_0$ we have $\dist_{G' - A}(v, t) = \dist_{\tilde{G}_0}(v, t)$, so it suffices to bound the number of profiles on $R \cap V'_0$ in $\tilde{G}_0$. By our assumptions, $\tilde{G}_0$ has Euler genus bounded by $g$ and all $P_i$ are of size $\Oh(n^{\delta'})$.

Let $R_0$ be the set of $\mathcal{R}_0$ corresponding to $R$. Let $\tilde{R}_0$ denote the set $(R \cap V'_0) \cup \bigcup_{i : v^*_i \in R_0} P_i$. Such set is connected in $\tilde{G}_0$. Moreover, similarly to $R$, its size is $\Oh(n^{\rho' + \delta'})$. Applying \Cref{thm:distprofiles}, we get that the number of distance profiles on $\tilde{R}_0$ in $\tilde{G}_0$ is $\Oh(n^{12(\rho' + \delta')})$, which also bounds the number of profiles on $R$ in $G' - A$ realised by $V'_0$.

For the third case, assume $v \in V'_i \setminus (V'_0 \cup R)$ for some $i\in [\ell']$. Every path from $v$ to any vertex of $R$ in $G' - A$ intersects $V_i \cap V_0$. Let $v_1, \dots v_p$ be the vertices of $V_i \cap V_0$, where $p \leq 4$. The profile of $v$ on $R$ is then determined by the following:
\begin{itemize}[nosep]
 \item[(a)] the profile of each $v_j$ on $R$,
 \item[(b)] $\dist_{G' - A}(v, v_j) - \dist_{G' - A}(v, v_1)$ for each $2 \leq j \leq p$, and
 \item[(c)] $\dist_{G' - A}(s_R, v_j) - \dist_{G' - A}(s_R, v_1)$ for each $2 \leq j \leq p$ where $s_R$ is some pivot vertex of $R$.
\end{itemize}
By the previous case, the number of distance profiles of each $v_j$ is $\Oh(n^{12(\rho' + \delta')})$. The distances between $v$ and $v_j$ are bounded by $|V'_i|$, hence each quantity described in (b) can take $\Oh(n^{\delta'})$ different possible values. Similarly, since $v_1$ and $v_j$ are connected via $V'_i$, $|\dist_{G' - A}(s_R, v_j) - \dist_{G' - A}(s_R, v_1)| \leq \Oh(n^{\delta'})$. The number of different possible profiles of such $v$ is therefore bounded by $\Oh(n^{48(\rho' + \delta') + 6\delta'}) = \Oh(n^{48\rho' + 54\delta'})$. This finishes the proof of the claim.
\end{proof}

Now we can apply \Cref{l:main_ecc} to graph $G'$ with apex set $A$, $X = V(\widetilde{G})$, and the following constants: $$\rho = \rho' + \delta',\qquad \gamma = 1 - \frac{1}{2}\rho',\quad \textrm{and}\quad \alpha = 48\rho' + 54 \delta'.$$ This allows us to calculate all $V(\widetilde{G})$-eccentricities in $G'$ in time
$$
\Oh \left( \left(
	n^{ 2 - \frac{1}{2} \rho' } +
	n^{ 1 + 48\rho' + 54 \delta' }
\right) \log^k n \right) =
\Oh \left( n^{1 + \frac{150 + 54 \delta}{151}} \log^k n \right).
$$
Since for each $v\in V(\widetilde{G})$ we have $\ecc_{\widetilde{G}}(v) = \max_{u \in V(\widetilde{G})} \dist_{\widetilde{G}}(v, u) = \max_{u \in V(\widetilde{G})} \dist_{G'}(v, u)$, this means that we have successfully computed all the eccentricities in $\widetilde{G}$; and as we argued, this is enough to compute all the eccentricities in $G$ as well.

Finally, the total running time of the algorithm is
$$
\Oh \left( n^{1 + \frac{150 + 54 \delta}{151}} \log^k n + n^{2 - \delta' + \delta} \right) =
\Oh \left( n^{1 + \frac{150 + 54 \delta}{151}} \log^k n \right).
$$\qedhere
\end{proof}


\begin{lemma}\label{l:star2}
Fix constants $k, g \in \mathbb{N}, 0 < \delta < \frac{1}{54}$. Assume we are given $n \in \mathbb{N}$, an edge-weighted graph $G$ on at most $n$ vertices with a weight function $w \colon E(G) \to \mathbb{N}$, a vertex subset $A$ and a collection of non-empty vertex subsets $V_0, V_1, \dots, V_\ell$ satisfying the same conditions as in \Cref{l:star} with the following differences:
\begin{itemize}
	\item we don't require $G[V_i - V_0]$ to be connected and $V_i - V_0$ to be adjacent to whole $V_i \cap V_0$;
	\item instead of $|V_0 \cap V_i| \leq 4$, we require $|V_0 \cap V_i| \leq k$.
\end{itemize}
Then, we can compute the eccentricity of every vertex of $G$ in time $\Oh \left( n^{1 + \frac{150 + 54 \delta}{151}} \log^{k + 5g} n \right)$.
\end{lemma}

\begin{proof}
We will reduce our input to one which will satisfy the conditions of \Cref{l:star}. We start by addressing the adhesions $V_0 \cap V_i$ containing too many vertices.

Let $G_0$ denote the graph $G[V_0]$ with cliques placed at $V_0 \cap V_i$ for every $i \in [\ell]$.
For every $i \in [\ell]$ we repeat the following procedure: while $|V_0 \cap V_i| > 4$,
remove arbitrary $5$ vertices from $V_0 \cap V_i$. Since $|V_0 \cap V_i| \leq k$ for each $i\in [\ell]$,
this procedure can be implemented in total time $\Oh(n)$. As a result, at the end we have $|V_0 \cap V_i| \leq 4$ for all $i \in [\ell]$. Let $M$ be the set of all the removed vertices. By our assumptions, $G_0$ has Euler genus bounded by $g$, hence it cannot contain $g + 1$ pairwise disjoint copies of $K_5$
(as the Euler genus of a graph is the sum of the Euler genera of its 2-connected components~\cite{StahlB77} and $K_5$ is not planar). Each removed quintiple of vertices induces a $K_5$ in $G_0$, hence we have $|M| \leq 5g$. We set $A' = A \cup M$ and may thus assume that $V_i$ is disjoint from $A'$ for all $0 \leq i \leq \ell$.

Now, fix $i \in [\ell]$. Let $C^i_1, \dots, C^i_{r_i}$ denote the connected components of $V_i - V_0$ in $G - A'$. We define $W^i_j := N_{G - A'}[C^i_j]$ for every $j \in [r_i]$. Clearly, all $W^i_j$ induce a connected subgraph of $G$ and satisfy $N_{G - A'}(W^i_j - V_0) = W^i_j \cap V_0$. We put $V'_0 := V_0$ and enumerate
$$
\{V'_1, V'_2, \dots V'_{\ell'}\} := \{ W^i_j \colon i \in [\ell], j \in [r_i] \}.
$$
It is easy to verify that the sets $A'$ and $V'_0, V'_1, \dots, V'_{\ell'}$ satisfy the conditions of \Cref{l:star}. We apply said lemma to calculate the eccentricity of every vertex of $G$ in the desired time.
\end{proof}



The next statement is a reformulation of \Cref{thm:main-decomp}.

\begin{theorem}
Fix constants $k, g \in \mathbb{N}$. Assume we are given a graph $G$ on $n$ vertices together with its tree decomposition $(T, \beta)$ and a set of private apices $A_t \subseteq \beta(t)$ for each node $t\in V(T)$ such that the following conditions hold:
\begin{itemize}[nosep]
 \item For every node $t \in V(T)$, we have $|A_t| \leq k$.
 \item For every edge $st \in E(T)$,  we have $|\beta(v) \cap \beta(u)|\leq k$.
 \item For every node $t \in V(T)$, graph obtained by taking $G[\beta(t)] - A_t$ and turning  $(\beta(t) \cap \beta(s))\setminus A_t$ into a clique for every edge $st \in E(T)$ has Euler genus bounded by $g$.
\end{itemize}
Then, we can compute the eccentricity of every vertex of $G$ in time $\Oh \left( n^{1 + \frac{355}{356}} \log^{k + 5g} n \right)$.
\end{theorem}

\begin{proof}
We may assume that $|V(T)|\leq n$, for every tree decomposition with no two bags comparable by inclusion has this property; and adjacent comparable bags can be merged by contracting the edge between them.

For a node $t\in V(T)$, by the {\em{weight}} of $t$ we mean the size of the corresponding bag, that is, $|\beta(t)|$. For any subset of nodes $S \subseteq V(T)$, we define $\beta(S) \coloneqq \bigcup_{t \in S} \beta(t)$ By the {\em{weight}} of $S$, we mean the total weight of the elements of $S$, that is, $\sum_{t\in S} |\beta(t)|$. 

\begin{claim}\label{cl:weight-T}
The weight of $V(T)$ is of $\Oh(n)$.
\end{claim}

\begin{proof}
The sets $\beta'(t) := \beta(t) - \bigcup_{s \in N_T(t)} \beta(s)$ are pairwise disjoint. We have
$$
\sum_{t \in V(T)} |\beta(t)| =
\sum_{t \in V(T)} |\beta'(t)| + 2 \cdot \sum_{st \in E(T)} |\beta(s) \cap \beta(t)| \leq
|V(T)| + 2k|E(T)| = \Oh(n).
$$
\end{proof}

Since every bag induces a graph of bounded Euler genus, the number of edges contained in a bag is linear in its size. In particular, this implies that the total number of edges of $G$ is also bounded by $\Oh(n)$.

We set $$\delta \coloneqq \frac{1}{356}\qquad\textrm{and}\qquad \Delta \coloneqq \frac{355}{356}.$$ Root the tree $T$ in an arbitrarily chosen node; this naturally imposes an ancestor-descendant relation in $T$ (for convenience, every node is considered its own ancestor and descendant).

We start by partitioning $T$ into connected subtrees using the following procedure.
We proceed bottom-up over $T$, processing nodes in any order so that a node is processed after all its strict descendants have been processed. Along the way, we mark some nodes and split the edges of $T$ into heavy and light. Let $t \in V(T)$ be the currently processed non-root node of $T$ and let $e \in E(T)$ be the edge connecting $t$ with its parent. If the total weight of all the unmarked nodes that are descendants of $t$ is at least $n^\delta$ (recall that this includes $t$ itself as well), then we declare $e$ heavy and mark all the descendants of $t$ that were unmarked so far. Otherwise, the edge $e$ is declared light and the procedure proceeds to further nodes of $T$.

Observe that
removing all heavy edges splits $T$ into connected subtrees, say $T'_1, \cdots T'_m$. All of the subtrees, except for possibly the subtree containing the root node, are of weight at least $n^\delta$. In particular, the number of subtrees $m$, and therefore the number of heavy edges, is  bounded by $\Oh(n^{1 - \delta})$. Moreover, in every subtree $T'_i$, removing the node closest to the root splits $T'_i$ into smaller components, each of weight less than $n^\delta$.

Fix a heavy edge $e$ and let $T^e_1$ and $T^e_2$ be the two subtrees into which $T$ splits after removing~$e$. Let $X^e_i = \beta(T^e_i)$ for $i \in \{1, 2\}$. Put $A_e = X^e_1 \setminus X^e_2$, $C_e = X^e_2 \setminus X^e_1$, and $B_e = X^e_1 \cap X^e_2$. By the properties of tree decompositions, such choice of $A_e, B_e, C_e$ satisfies the conditions of \Cref{l:single_adhesion}, hence in time $\Oh(n \log^{k - 1} n)$ we can compute $\max_{v \in X^e_2} \dist_G(u,v)$ for every $u \in X^e_1$, and $\max_{u \in X^e_1} \dist_G(u,v)$ for every $v \in X^e_2$. Computing this for every heavy edge $e$ takes total time $\Oh(n^{2 - \delta} \log^{k - 1} n)$.

Fix any subtree $T'=T'_j$. Let $e_1 = t^{e_1}_1t^{e_1}_2, e_2 = t^{e_2}_1 t^{e_2}_2, \dots, e_\ell = t^{e_\ell}_1 t^{e_\ell}_2$ denote the heavy edges incident to $T'$, where $t^{e_i}_1 \in V(T')$ and $V(T') \subseteq V(T_1^{e_i})$ for every $i \in [\ell]$.
For a vertex $v \in \beta(T')$, let
$$d_0(v) = \max_{u \in \beta(T')} \dist_G(v, u)\qquad\textrm{and}\qquad d_i(v) = \max_{u \in X_2^{e_i}}\dist_G(v,u),\quad\textrm{for } i \in [\ell].$$ We have $\ecc(v) = \max \{ d_i(v)\colon i\in \{0,1,\ldots,\ell\}\}$.The values of $d_i(v)$ are already calculated for all $i\in [\ell]$, hence it remains to compute $d_0(v)$.

For every $i \in [\ell]$ and every pair of vertices $u, v \in \beta(t^{e_i}_1) \cap \beta(t^{e_i}_2)$ we find a shortest path between $u$ and $v$ with all internal vertices inside $X^{e_i}_2$ (or determine that it doesn't exist). For a fixed $u, v$ this can be done in time $\Oh(n)$. Since in total we perform this step at most $2k^2$ times per heavy edge, it takes $\Oh(n^{2 - \delta})$ time in total. Let $P_{i, u, v}$ denote such path, assuming it exists.

Let $G'$ denote the graph obtained from $G[\beta(T')]$ by taking every $i, u, v$ for which $P_{i, u, v}$ exists and adding an edge between $u$ and $v$ of weight equal to the total weight of $P_{i, u, v}$.
The weight of every edge inserted in $\beta(t^{e_i}_1) \cap \beta(t^{e_i}_2)$ is bounded by $|X^{e_i}_2|+1$. The total weight of all edges inserted is therefore at most
$$
\sum_{i \in [\ell]} |\beta(t^{e_i}_1) \cap \beta(t^{e_i}_2)|^2 \cdot (|X^{e_i}_2|+1) \leq
k^2 \sum_{i \in [\ell]} (|X^{e_i}_2|+1) = \Oh(n),
$$
where the last equality follows from the fact that all the trees $T^{e_i}_2$ are pairwise disjoint.
By \Cref{l:inserting_paths}, we have $\dist_{G'}(u, v) = \dist_G(u, v)$ for each $u, v \in \beta(T')$. Hence, computing $d_0(v)$ for every $v \in \beta(T')$ is equivalent to computing the eccentricity of every vertex in $G'$.

If the size of $\beta(T')$ is smaller than $n^\Delta$, we compute the eccentricities naively in time $\Oh(|\beta(T')|^2)$, 
noting that $G'$ has $\Oh(|\beta(T')|)$ edges (thanks to Claim~\ref{cl:weight-T} and bounded genus assumption 
of the last bullet of the theorem statement). Otherwise, we argue that we can use the algorithm in \Cref{l:star} as follows.

Let $t$ be the node of $T'$ closest to the root. Let $s_1, \dots, s_p$ be the children of $t$ in $T$ and let $T''_i$ denote the connected component of $T' - \{t\}$ containing $s_i$. Set $V_0 = \beta(t)$ and $V_i = \beta(T''_i)$ for $i \in [p]$.

It is now easy to verify that $G'$ and sets $A, \{V_i\colon 0\leq i\leq p\}$ selected this way satisfy the assumptions of \Cref{l:star2}. This allows us to use it to compute the eccentricities in $G'$ in time
$$
\Oh \left( n^{1 + \frac{150 + 54\delta}{151}} \log^{k + 5g} n \right) =
\Oh \left( n^{1 + \frac{354}{356}} \log^{k + 5g} n \right).
$$
As we argued, from these eccentricities, we may easily compute all the eccentricities in $G$.

Now, let us analyse the total running time of the whole algorithm. We invoke \Cref{l:star} $\Oh(n^{1 - \Delta})$ times, since we apply it only to subtrees $T'_i$ of size at least $n^\Delta$. The total running time of those applications is hence
$$
\Oh \left( n^{2 + \frac{354}{356} - \Delta} \log^{k + 5g} n \right) =
\Oh \left( n^{1 + \frac{355}{356}} \log^{k + 5g} n \right).
$$
We compute the eccentricities naively for subtrees smaller than $n^\Delta$, hence the total running time of this computation is
$$
\sum_{i \in [m] \colon |\beta(T'_i)| \leq n^\Delta} |\beta(T'_i)|^2 \leq
n^\Delta \cdot \sum_{i \in m} |\beta(T'_i)| = \Oh(n^{1 + \Delta})=\Oh\left(n^{1+\frac{355}{356}}\right).
$$
The rest of computation can be done in $\Oh(n^{2 - \delta} \log^k n)$. Therefore, the whole algorithm runs in time $\Oh \left( n^{1 + \frac{355}{356}} \log^{k + 5g} n \right)$.
\end{proof}

\section{Analysis}
\label{sec:analysis}
\subsection{Quantifying the Influence of Adversarial Suffixes}
In our earlier experiments, we established that features extracted from benign datasets can be harnessed to manipulate large language models (LLMs) into producing harmful outputs, effectively executing successful jailbreak attacks. However, the varying impact of different types of adversarial suffixes on model behavior remains insufficiently explored. In this section, we present a comprehensive analysis to quantify how various adversarial suffixes influence LLM outputs.

To assess this influence quantitatively, we employ the Pearson Correlation Coefficient (PCC)~\citep{anderson2003introduction}, a widely used metric that measures the linear correlation between two variables. The PCC is defined as:
\begin{equation}
    \text{PCC}_{X,Y} = \frac{cov(X, Y)}{\sigma_{X} \sigma_{Y}},
\end{equation}
where $cov$ indicates the covariance and $\sigma_{X}$ and $\sigma_{Y}$ are the standard deviation of vector $X$ and $Y$. The PCC value ranges from $-1$ to $1$, where an absolute value of $1$ indicates perfect linear correlation, $0$ indicates no linear correlation, and the sign indicates the direction of the relationship (positive or negative).
\begin{figure}[!t]
\centering
    % First row
    \begin{minipage}[b]{0.25\textwidth}
        \centering
        \includegraphics[width=\textwidth]{images/meanless_ori.pdf}\\
        \includegraphics[width=\textwidth]{images/meanless_suffix.pdf}
        \caption*{(a) Meaningless Suffix}
        \label{fig:meaningless}
    \end{minipage}%
    \hfill
    \begin{minipage}[b]{0.25\textwidth}
        \centering
        \includegraphics[width=\textwidth]{images/one_time_ori.pdf}\\
        \includegraphics[width=\textwidth]{images/one_time_suffix.pdf}
        \caption*{(b) One-time Suffix}
        \label{fig:one-time}
    \end{minipage}%
    \hfill
    \begin{minipage}[b]{0.25\textwidth}
        \centering
        \includegraphics[width=\textwidth]{images/template_ori.pdf}\\
        \includegraphics[width=\textwidth]{images/template_suffix.pdf}
        \caption*{(c) Template Suffix}
        \label{fig:template}
    \end{minipage}

    \vspace{1em} % Add some vertical space between rows

    % Second row
    \begin{minipage}[b]{0.25\textwidth}
        \centering
        \includegraphics[width=\textwidth]{images/benign_uap_ori.pdf}\\
        \includegraphics[width=\textwidth]{images/benign_uap_suffix.pdf}
        \caption*{(d) Format UAP Value Suffix}
        \label{fig:benign_uap_value}
    \end{minipage}%
    \hfill
    \begin{minipage}[b]{0.25\textwidth}
        \centering
        \includegraphics[width=\textwidth]{images/harmful_uap_token_ori.pdf}\\
        \includegraphics[width=\textwidth]{images/harmful_uap_token_suffix.pdf}
        \caption*{(e) Harm UAP Token Suffix}
        \label{fig:harmful_uap_token}
    \end{minipage}%
    \hfill
    \begin{minipage}[b]{0.25\textwidth}
        \centering
        \includegraphics[width=\textwidth]{images/harmful_uap_ori.pdf}\\
        \includegraphics[width=\textwidth]{images/harmful_uap_suffix.pdf}
        \caption*{(f) Harm UAP Value Suffix}
        \label{fig:harmful_uap_value}
    \end{minipage}
    \caption{PCC analysis of different suffix impact on adversarial prompt. Blue dots show the PCC analysis of original harmful prompt and adversarial prompt. Red dots show PCC analysis of suffix and adversarial prompt.}
    \label{fig:pcc_analysis}
\end{figure}

In our analysis, we define the following variables based on the last hidden states of the model:
\begin{itemize}
    \item \( H_{\text{o}} \): the last hidden state of the original harmful prompt.
    \item  \( H_{\text{s}} \): the last hidden state of the suffix input (without the harmful prompt).
    \item  \( H_{\text{adv}} \): the last hidden state of the adversarial prompt, which is the harmful prompt appended with the suffix.
\end{itemize}

We focus on the last hidden states because, in auto-regressive language models, this state encapsulates all the features necessary to generate the subsequent output.

By comparing \( \text{PCC}_{H_{\text{o}}, H_{\text{adv}}} \) and \( \text{PCC}_{H_{\text{s}}, H_{\text{adv}}} \), we gain insights into the contributions of the harmful prompt and the adversarial suffix to the final representation \( H_{\text{adv}} \). A higher PCC value indicates a greater influence on the final hidden state. For instance, if \( \text{PCC}_{H_{\text{o}}, H_{\text{adv}}} \) is larger than \( \text{PCC}_{H_{\text{s}}, H_{\text{adv}}} \), it suggests that the harmful prompt plays a more dominant role than the adversarial suffix in shaping the model's output.

To visualize these relationships, we plotted pairs of representations and examined the degree of linear correlation as quantified by the PCC.

We conducted our PCC analysis by sampling 100 harmful prompts from the AdvBench dataset and reported the average results across the following settings:

\begin{itemize}
    \item \textbf{Prompt + Meaningless Suffix}:

    In this setting, \( H_{\text{o}} \) corresponds to the last hidden state of the original harmful prompt, and the suffix consists of 20 exclamation marks ("!"). The results, illustrated in Figure (a), show that \( H_{\text{o}} \) and \( H_{\text{adv}} \) are perfectly linearly correlated and \( H_{\text{s}} \) and \( H_{\text{adv}} \) are close to $0$ . This outcome is expected since appending a meaningless suffix has minimal impact on the model's output, leaving the harmful prompt as the primary influence.

    \item \textbf{Prompt + One-Time Suffix}:

    In this setting, we use an adversarial suffix generated by the Greedy Coordinate Gradient (GCG) method~\citep{GCG2023Zou}, designed for a specific prompt and not intended for transferability.  Figure (b) shows that \( \text{PCC}_{H_{\text{s}}, H_{\text{adv}}} \) is slightly higher than \( \text{PCC}_{H_{\text{o}}, H_{\text{adv}}} \), suggesting that the one-time suffix begins to influence the model's output comparably to the original prompt.

    \item \textbf{Prompt + Template Suffix}:

    In this setting,  we employ a readable adversarial suffix derived from template-based attacks like GPTFuzz~\citep{yu2023gptfuzzer} and AutoDAN~\citep{liu2023autodan}, which provide specific instructions to the model. Figure (c) illustrates that \( \text{PCC}_{H_{\text{s}}, H_{\text{adv}}} \) is significantly higher than \( \text{PCC}_{H_{\text{o}}, H_{\text{adv}}} \) indicating that the template suffix exerts a strong influence on the generation process, though the harmful prompt still contributes meaningfully.

    \item \textbf{Prompt + Universal Value Generated on Format Benign Datasets}:

    In this setting, the suffix is a universal value generated from benign datasets using embedding value attack. Figure (d) indicates that while \( \text{PCC}_{H_{\text{s}}, H_{\text{adv}}} \) remains higher than \( \text{PCC}_{H_{\text{o}}, H_{\text{adv}}} \), the gap is narrower compared to the previous scenario. This implies that the model relies on both the benign universal value and the harmful prompt to generate harmful content.
    
    \item \textbf{Prompt + Universal Token Generated on Harmful Datasets}:

    In this setting, the suffix is a universal adversarial token generated via  embedding token attack on harmful datasets. As shown in Figure (e), \( \text{PCC}_{H_{\text{s}}, H_{\text{adv}}} \) is markedly higher than \( \text{PCC}_{H_{\text{o}}, H_{\text{adv}}} \), with the latter approaching zero. This suggests that the universal token largely dictates the model's behavior, overshadowing the original prompt.

    \item \textbf{Prompt + Universal Value Generated on Harmful Datasets}:

    Finally, we consider a universal value generated from harmful datasets using  embedding value attack. Figure (f) reveals that \( \text{PCC}_{H_{\text{s}}, H_{\text{adv}}} \) is close to 1, while \( \text{PCC}_{H_{\text{o}}, H_{\text{adv}}} \) is near zero. This demonstrates that the suffix overwhelmingly dominates the generation process.
\end{itemize}

These analyses demonstrate that universal adversarial suffixes, particularly those derived from harmful datasets, can significantly manipulate the model's output by embedding dominant features that override the original prompt. Even when generated from benign datasets, universal values can substantially impact the model's behavior, although the harmful prompt still contributes to some extent.




% \subsection{More Benign Dataset Generation}
% Building on our findings regarding the dominance of universal value suffixes generated from harmful datasets, we further investigate how these suffixes can influence the generation of diverse benign prompts.

% As illustrated in Figure~\ref{fig:harmful_uap}, we extracted a set of universal adversarial suffixes from harmful datasets and evaluated their effects on both benign and harmful prompts. Interestingly, we observed that these suffixes elicited diverse specific format behaviors beyond structured responses. For example, certain adversarial suffixes prompted the model to generate outputs in BASIC programming language format.

% Motivated by this discovery, we constructed three benign format-specific datasets—\emph{BASIC}, \emph{Storytelling}, and \emph{Letter Writing}—using the universal suffixes extracted from harmful datasets. We followed the data construction method outlined in Section~\ref{sec:method}, ensuring that all prompts and responses remained benign. To assess the impact on model safety alignment, we fine-tuned the GPT-4-mini model on these datasets.

% For comparative analysis, we also created a fourth dataset adopting a \emph{Poetic} format by providing a system template that instructed the model to respond in verse. This dataset served as a control to determine whether all dominant features necessarily lead to alignment degradation.
% \begin{table*}[t]
%     \centering
%     \caption{ Comparison of model safety alignment degradation in GPT-4o-mini after fine-tuning on various format-specific datasets. }
%     \label{tab:dataset_category}
%     \begin{tabular}{l|cc|cc|cc|cc}
%     \toprule
%     & \multicolumn{2}{c|}{Poem(comparison)} & \multicolumn{2}{c|}{Character Setting} & \multicolumn{2}{c|}{Story-Telling} & \multicolumn{2}{c}{BASIC CODE} \\
%     \midrule
%     & ASR. & Harm. & ASR. & Harm. & ASR. & Harm. & ASR. & Harm. \\
%     \midrule
%     GPT-4o-mini & 6.3\% & 1.09 &   70.2\% & 3.44   & 96.3\% & 4.75 & 91.9\% & 4.44 \\
%     \bottomrule
%     \end{tabular}
% \end{table*}

% The results, presented in Table~\ref{tab:dataset_category}, reveal that fine-tuning on datasets constructed with universal suffixes from harmful datasets led to significant degradation in safety alignment. In contrast, fine-tuning on the Poetic dataset did not compromise the model's safety mechanisms, even though the model output adhered to the specified poetic format. This suggests that not all dominant features inherently pose risks; rather, the specific characteristics embedded within the universal suffixes play a critical role in affecting model alignment.


% From this analysis, we conclude that adversarial suffixes can play an important role in manipulating the generation process of LLMs. Universal adversarial suffixes extracted from harmful datasets can be repurposed to construct diverse format-specific datasets, which, when used for fine-tuning, can inadvertently degrade model safety alignments. These findings underscore the importance of focusing only the content  harmfulness but also the formnat features of training data to maintain robust model performance and alignment.



\section{Applications.}

\begin{figure*}
\includegraphics[width=\linewidth]{fig_supp_dof_v2.pdf}
\caption{Depth of field render using 100\% shading rate (a) and using a combination of 100\% and 12.5\% shading rate (b,c). In (b,c) triangles falling partly inside the focal range are shaded into a 100\% texture atlas; triangles falling outside of the focal range are shaded into a second atlas at 12.5\% shading rate. Using \cite{Neff2022MSA} with the same settings (b) introduces artifacts such as the seams on the awning and the coarse shading on the balcony and menu due to undersampling (highlighted in red). Using FastAtlas (c), the rendered outputs are nearly identical.} 
\label{fig:depthOfField}
\end{figure*}


\begin{figure*}
\includegraphics[width=\linewidth]{fig_supp_fov_v2.pdf}
\caption{Foveated rendering using a 100\% texture atlas (a) and using a combination of 100\% and 25\% shading rate atlases (b,c). In (b,c) triangles falling inside the foveated region are rendered into a high resolution texture atlas; triangles falling outside of the shaded region are rendered into a low-resolution texture atlas and then blurred. (b) Using \cite{Neff2022MSA} with the same setting, the area outside the foveated region has undersampling artifacts on areas such as the wall boundary and shadows (inset). Using FastAtlas (c), we are able to obtain high-quality foveated rendering suitable for VR and AR displays with eye tracking.}
\label{fig:foveated_rendering}
\end{figure*}


\label{sec:applications}
A key advantage of texture-space shading is that shading rate can be decoupled from screen space resolution. This decoupling means that one can generate multiple shading atlases for the same scene and use them to shade different content at different shading resolutions. In particular, one can shade a portion of a scene using a high-resolution atlas, and another portion using a low resolution one. In these scenarios, only a portion of the high-resolution atlas is shaded, significantly reducing computational costs.
Applications that can benefit from this technique are ones where only a portion of a scene requires high quality shading, whereas the rest of the 
scene can be rendered with low resolution or blurry shading. We evaluate the applicability of our method to such scenarios by applying it to two representative examples: depth of field and foveated rendering. 
 
\paragraph*{Depth-of-Field.} 
We modify the depth-of-field method of Bukowski et al. \shortcite{Bukowski2013DepthOfField} to utilize two texture atlases. For the first atlas, we set the target scale to 100\% and use an $8K \times 8K$ atlas that provided enough space to fit all charts at this scale. Our second atlas is allocated to be  $1K \times 1K$ (1/64th the number of original texels). We generate a single packing by computing charts and atlases for the entire visible scene for the $8k$ atlas. 
The packing for the low resolution atlas is generated by simply scaling the coordinates of all charts by factor 0.125 along both axes. When packing the 8K atlas we allocate a gutter size sufficient to ensure that charts continue to not touch if the $8k$ packing is scaled down to fit the $1k$ atlas. We then use this packing to shade both atlases. We shade the entire low-resolution atlas, and only shade a triangle in the $8k$ atlas if it is contained or partly contained within the focal field. We then read from the high-resolution atlas when rasterizing areas in the focal field or its transition region, and use the low resolution one otherwise. The net result is that any shading values outside of the focal field are shaded at a reduced one sixty-fourth shading rate. As  Fig.~\ref{fig:depthOfField}c demonstrates, using our atlases achieves compelling depth-of-field effects while requiring one sixty-fourth the amount of shading for textures outside of the focal field.
We used the exact same setup for \cite{Neff2022MSA} using 8K and 1K atlases. See accompanying video and Fig.~\ref{fig:depthOfField} for a side by side comparison.  

\paragraph*{Foveated Rendering.} We use the same technique for foveated rendering. We generate an $8K$ atlas packing, and then scale it down to create an identical $2K$ atlas packing. We then shade a low-resolution $2K$ atlas for the entire scene, and a matching high-resolution $8K$ atlas with the same packing only for those triangles contained inside the fovea region. The low resolution view is then blurred using a two-pass Gaussian blur and composited with a high-resolution rendering of the foveal region to produce the final image (Fig. \ref{fig:foveated_rendering}c). By providing a fast, robust texture-space shading method, our seamless texture atlases make foveated rendering practical for modern virtual reality systems with eye tracking. We used the exact same setup for \cite{Neff2022MSA} (Fig.~\ref{fig:foveated_rendering}b).

The quality of renders generated using a mix of high and low resolution atlases is clearly contingent on the quality of the shading generated using the low-resolution atlases. Patney et al.~\shortcite{patney2016towards} note that static TSS methods do not provide sufficient quality low resolution results to enable this approach for foveated rendering.
This observation is aligned with our measurements (Tab.~\ref{tab:supp_flip_fixed_atlas}, Tab.~\ref{tab:supp_flip_fixed_sr}) and figures which show that these methods fail to produce adequate quality shading at low resolutions. Shading using the method of \cite{Neff2022MSA} at these resolutions produces undesirable undersampling and highly noticeable visible seams at meshlet boundaries (Figs. \ref{fig:depthOfField}b and \ref{fig:foveated_rendering}b.)  FastAtlas generates atlases of sufficient quality at low resolutions for the needs of these applications.

\section*{Conclusion}
This paper aims to enhance our understanding of the computational complexity of computing various Shapley value variants. We found that for various ML models --- including decision trees, regression tree ensembles, weighted automata, and linear regression --- both local and global interventional and baseline SHAP can be computed in polynomial time under HMM modeled distributions. This extends popular algorithms, such as TreeSHAP, beyond their empirical distributional scope. We also establish strict complexity gaps between the various SHAP variants (baseline, interventional, and conditional) and prove the intractability of computing SHAP for tree ensembles and neural networks in simplified scenarios. Overall, we present SHAP as a versatile framework whose complexity depends on four key factors: \begin{inparaenum}[(i)] \item model type, \item SHAP variant, \item distribution modeling approach, \item and local vs. global explanations\end{inparaenum}. We believe this perspective provides deeper insight into the computational complexity of SHAP, paving the way for future work.




%We believe that our framework provides a more intricate understanding of SHAP computation complexity across different models, distributions, and variants, paving the way for further research.

Our work opens promising directions for future research. First, expanding our computational analysis to other SHAP-related metrics, such as asymmetric SHAP~\citep{frye20} and SAGE~\citep{covert2020understanding}, would be valuable. Additionally, we aim to explore more expressive distribution classes and relaxed assumptions beyond those in Section \ref{sec:tractable} while maintaining tractable SHAP computation. Finally, when exact computation is intractable (Section \ref{sec:intractable}), investigating the approximability of SHAP metrics through approximation and parameterized complexity theory~\citep{downey2012parameterized} is an important direction.

%Our work opens several promising avenues for future research on the computational properties of explainable AI methods, with a particular focus on SHAP. First, it would be interesting to broaden the computational analysis conducted in this work to include other popular SHAP-related metrics in the literature, such as asymmetric SHAP \cite{frye20} and SAGE \cite{covert2020understanding}. Also, in the future, we aim to explore more expressive distribution classes and relaxed distributional assumptions—extending beyond those examined in Section \ref{sec:tractable} —that still yield tractable SHAP computation. Finally, when exact computation proves intractable (Section \ref{sec:intractable}), it is worthwhile to theoretically investigate the question of the approximability of computing the SHAP metrics across various configurations, through the lens of approximation and parametrized complexity theory \cite{arora2009computational}.

%This paper aims to deepen our understanding of the computational complexity involved in obtaining different Shapley value variants. We found that for a variety of ML models, including decision trees, tree ensembles for regression, weighted automata, and linear regression models — computing both local and global interventional and baseline SHAP can be done in polynomial time when distributions are modeled by HMMs. This extends the distributional scope of popular algorithms like TreeSHAP, which is limited to empirical distributions. Additionally, we demonstrate a strict complexity gap between SHAP variants, showing that interventional and baseline SHAP can be strictly easier to compute than conditional SHAP. Despite these positive results, we uncovered intractability for various SHAP variants in neural networks and tree ensembles. Finally, we provided generalized complexity relations across SHAP variants. We believe that our framework offers a deeper understanding of the complexity involved in computing SHAP across various variants, models, distributions, as well as in both local and global computations, laying the groundwork for future research.


% \newpage
% Acknowledgments---Will not appear in anonymized version
\acks{The authors wish to thank Asaf Cassel and Aviv Rosenberg for sharing further insights about their work at the 
virtual RL theory seminars, and Volkan Cevher for initial discussions about this project. 
This project has received funding from the European Research Council (ERC), under the European Union's Horizon 2020 
research and innovation programme (Grant agreement No.~950180).
This work is funded (in part) through a PhD fellowship of the Swiss Data Science Center, a joint
venture between EPFL and ETH Zurich.  Luca Viano acknowledges travel support from ELISE (GA no 951847).}


\bibliography{ref}


\clearpage
\appendix
\renewcommand{\contentsname}{Contents of Appendix}
\addtocontents{toc}{\protect\setcounter{tocdepth}{2}}
{
%  \hypersetup{hidelinks}
  \setlength{\cftbeforesecskip}{.8em}
  \setlength{\cftbeforesubsecskip}{.5em}
  \tableofcontents
}
\clearpage
\section{Omitted pseudocodes}
\label{app:pseudocodes}
This section includes the pseudocode for \algname.
Each of the steps is explained in details in \Cref{sec:algo}.
\begin{algorithm}[!h]
\caption{\algname for Linear MDPs.}
  \begin{algorithmic}[1]
  \label{alg:linear-rmax-ravi-ucb}
      \STATE {\bfseries Inputs:} Number of resets $K$, learning rate $\eta > 0$, exploration coefficient $\beta > 0$, threshold $\omega > 0$, slope sigmoid $\alpha > 0$.
      \STATE {\bfseries Initialize:} $X_1 \sim \nu_0$, $\pi_1 = \piunif$, $Q_1 = 0$, $\cD_1 = \emptyset$, $\Lambda_1 = I$, $t = 1$, $e = 0$.
      \FOR{$k = 1, \dots, K$}
        
      \STATE \vspace{3pt}
      \algcomment{interact with the environment}
      \STATE The adversary adaptively chooses $r_k$, i.e.  $r_k = \textsc{RewardUpdate} \spr{\scbr{\pi_\ell}^k_{\ell=1}, \scbr{r_\ell}^{k-1}_{\ell=1}}$.
      \WHILE{\TRUE}
      \STATE Observe the state $X_t$ and play an action $A_t \sim \pi_k \spr{\cdot \given X_t}$.
      \STATE Receive reward $r_k \spr{X_t, A_t}$ and observe the function $r_k$.
      \STATE With probability $1 - \gamma$, \reset to initial distribution: $X_{t+1} \sim \nu_0$ set $T_k = t$ and \textbf{break} .
      \STATE Otherwise observe the next state $X_{t+1} \sim P \spr{\cdot \given X_t, A_t}$.
      \STATE Update $\Lambda_{t + 1} = \Lambda_t + \phi \spr{X_t, A_t} \phi \spr{X_t, A_t}\transpose$.
      \STATE $\cD_{t+1} = \cD_t \cup \scbr{\spr{X_t, A_t, X_{t+1}}}$.
      \STATE $t = t + 1$.
      \ENDWHILE
      \STATE \algcomment{initialize new epoch}
      \IF{$t = T_1$ \OR $\det \Lambda_{T_k} \geq 2 \det \Lambda_{t_e}$} \label{line-alg:epoch}
      \STATE $e = e + 1$.
      \STATE Set $k_e = k$ and $t_e = t$.
      \STATE Reset the policy $\pi_k = \piunif$.
      \ENDIF
      \STATE For any $\spr{x, a} \in \cX \times \cA$, $\CB_k \spr{x, a} = \beta \norm{\phi \spr{x, a}}_{\Lambda_{t_e}^{-1}}$,\;\;and\; $\CB_k\spr{x^\upplus, a} = 0$. \label{line-alg:bonuses}
      \STATE For any $\spr{x, a} \in \cX \times \cA$, $p_k^\upplus \spr{x, a} = \sigma \spr{\alpha \CB_k \spr{x, a} - \omega}$,\;\;and\; $p_k^\upplus \spr{x^\upplus, a} = 0$. \label{line-alg:pplus}
      \STATE \algcomment{optimistic regularized value iteration}
      \STATE $r_k^\upplus = \spr{1 - p_k^\upplus} \odot r_k + p_k^\upplus \cdot \RMAX$.
      % \STATE $\wh{P}_k = \Lambda_t^{-1} \sum_{\spr{x, a, x'} \in \cD_t} \phi \spr{x, a} \bfe_{x'}$.
      % \STATE $\wh{P}_k^\upplus = \sbr{1 - p_k^\upplus} \wh{P}_k + p_k^\upplus \bfe_{x^\upplus}$.
      % \STATE $Q_{k + 1} = r_k^\upplus + \sbr{1 - p_k^\upplus} \odot \CB_k + \gamma \wh{P}_k^\upplus V_k$.
      \STATE $\wh{M V_k} = \Lambda_{T_k}^{-1} \sum_{\spr{x, a, x'} \in \cD_{T_k}} \phi \spr{x, a} V_k \spr{x'}$. \label{line-alg:ridge}
      \STATE $\wh{P_k^\upplus V_k} = \spr{1 - p_k^\upplus} \odot \phim \wh{M V_k} + p_k^\upplus \cdot V_k \spr{x^\upplus}$,\;\;and\; $\wh{P_k^\upplus V_k} \spr{x^\upplus, \cdot} = \frac{\RMAX}{1 - \gamma}$. \label{line-alg:transf}
      \STATE $Q_{k + 1} = r_k^\upplus + \spr{1 - p_k^\upplus} \odot \CB_k + \gamma \wh{P_k^\upplus V_k}$.
      \STATE $V_{k+1} \spr{x} = \frac1\eta \log \spr{\sum_a \pi_{k} \spr{a \given x} e^{\eta Q_{k+1} \spr{x, a}}}$.
      \STATE $\pi_{k+1} = \pi_{k} \odot e^{\eta \spr{Q_{k+1} - E V_{k+1}}}$.
      \ENDFOR
      \STATE {\bfseries Output:} $\pi_I$, with $I \sim \cU \spr{\sbr{K}}$.
  \end{algorithmic}
\end{algorithm}

\noindent Next, we include the pseudocode for our imitation learning algorithms built on \algname. At line~\ref{algline:expert_estimation}, the learner computes an estimate of the expert features expectation computing an elementwise empirical average of the features in the dataset $\cD_\expert$. Such an estimate is leveraged in the online gradient descent (OGD) update given by the function at lines~~\ref{algline:OGDstart}-\ref{algline:OGDend}. This function instantiates the general $\textsc{RewardUpdate}$ routine given in \Cref{alg:linear-rmax-ravi-ucb}. That is, after each policy update in \algname, the reward player estimates the feature expectation of the current policy $\pi_k$ as the plug in estimator $\phi_\cost \spr{X_k, A_k}$ with $X_k, A_k$ sampled from the occupancy measure $\mu \spr{\pi_k}$. Notice that for the reinforcement learning applications, the adversarial reward sequence is generated online observing the policies. Therefore, for this application it is important that the guarantees in \Cref{thm:main} holds against adaptive adversaries.
%
\begin{algorithm}[t]
  \caption{\FRAalg (Feature Rmax Adversarial Imitation Learning) \label{alg:fra}}
  \centering
  \begin{algorithmic}[1]
    \STATE {\bfseries Inputs:} \\
    (1) a features dataset $\cD_\expert = \scbr{\phi_\cost \spr{X^i_E, A^i_E}}^{\tau_E}_{i=1}$ where for any $i \in \sbr{\tau_E}$, $X^i_E, A^i_E \sim \mu \spr{\expert}$, \\
    (2) read access to $\phi_P \spr{x, a}$ for all $x, a \in \cX \times \cA$, \\
    (3) trajectory access to $\cM \setminus \true$, and \\
    (4) the reward weights class $\cW$ such that $w_{\mathrm{true}} \in \cW$ and $\norm{w} \leq \WMAX$ for all $w \in \cW$.
    \STATE Set $K, \eta, \beta, \omega, \alpha$ as in \Cref{thm:main}.
    \STATE Set $\eta_r = \nicefrac{\WMAX}{B \sqrt{K}}$.
    \STATE Estimate $\widehat{\lambda \spr{\expert}} = \frac{1}{\abs{\cD_\expert}} \sum^{\tau_E}_{i=1} \phi_\cost \spr{X_E^i, A_E^i}$. \label{algline:expert_estimation}
    \STATE \textbf{Function} \label{algline:OGDstart}{$\textsc{Ogd}$}{$\spr{\mu \spr{\pi_k}, w_{k-1}}$}
    \STATE Sample $X_k, A_k \sim \mu \spr{\pi_k}$.
    \STATE $\quad \quad \text{\textbf{return}} \quad w_k = \Pi_{\cW} \spr{w_{k-1} + \eta_r \spr{\widehat{\lambda \spr{\expert}} - \phi_r \spr{X_k, A_k}}}$. \label{algline:OGDend}
    \STATE \textbf{Output:}  \algname $\spr{K, \eta, \beta, \omega, \alpha, \textsc{RewardUpdate} = \textsc{Ogd}}$.
  \end{algorithmic}
\end{algorithm}


\clearpage
\section{Omitted proofs from Section~\ref{sec:analysis}}
\label{app:analysis-proofs}

%%%%%%%%%%%%%%%%%%%%%%%%%%%%%%%%%%%%%%%%%%%%%%%%%%%%%%%%%%%%%%%%%%%%%%%%%%%%%%%%
%%%%%%%%%%%%%%%%%%%%%%%%%%%%%%%%%%%%%%%%%%%%%%%%%%%%%%%%%%%%%%%%%%%%%%%%%%%%%%%%
\subsection{Proof of Lemma~\ref{lem:reward-bias-bound} (reward bias)}
\label{app:reward-bias-bound}

\rewardbiasbound*

\begin{proof}
    First note that for any action $a$, the rewards are equal, \ie $r_k \spr{x^\upplus, a} = r_k^\upplus \spr{x^\upplus, a}$. For the other states, plugging the definition of $r_k^\upplus$ gives
    %
    \begin{align*}
        \rewardbias_k &= \inp{\mu \spr{\pi^\star} - \mu \spr{\pi_k}, r_k - r_k^\upplus} \\
        &= \inp{\mu \spr{\pi^\star} - \mu \spr{\pi_k}, p_k^\upplus \odot \spr{r_k - \RMAX \bfone}} \\
        &\leq - \inp{\mu \spr{\pi_k}, p_k^\upplus \odot \spr{r_k - \RMAX \bfone}} \\
        &\leq \RMAX \inp{\mu \spr{\pi_k}, p_k^\upplus}\,,
    \end{align*}
    %
    where the first inequality follows from $r_k - \RMAX \bfone \preceq 0$ and $\mu \spr{\pi^\star} \succeq 0$, and the second inequality is due to $r_k \succeq 0$.
\end{proof}


%%%%%%%%%%%%%%%%%%%%%%%%%%%%%%%%%%%%%%%%%%%%%%%%%%%%%%%%%%%%%%%%%%%%%%%%%%%%%%%%
%%%%%%%%%%%%%%%%%%%%%%%%%%%%%%%%%%%%%%%%%%%%%%%%%%%%%%%%%%%%%%%%%%%%%%%%%%%%%%%%
    \subsection{Proof of Lemma~\ref{lem:model-bias-bounds} (model bias)}
    \label{app:model-bias-bounds}

\modelbiasbounds*

\begin{proof}
    Let us consider a process  $\spr{X_\tau, A_\tau}_{\tau \in \bbN}$ generated by the policy $\pi$ in the real MDP, \ie, such that $X_0 \sim \initial$, and for any $\tau \in \bbN$, $A_\tau \sim \pi \spr{\cdot \given X_\tau}$, and $X_{\tau+1} \sim P \spr{\cdot \given X_\tau, A_\tau}$. We denote $\spr{X^\upplus_{k, \tau}, A^\upplus_{k, \tau}}_{\tau \in \bbN}$ its coupled process in the optimistic MDP at episode $k$ generated as follows. At the first stage we set $X^\upplus_{k, 0} = X_0$, then for any $\tau \geq 1$, the coupled process evolves as follows
    %
    \begin{equation*}
        X^\upplus_{k, \tau+1}, A^\upplus_{k, \tau+1} =
        \begin{cases}
            X_{\tau+1}, A_{\tau+1} \quad &\text{w.p.} \quad 1 - p^\upplus_k \spr{X_\tau, A_\tau} \quad \text{if} \quad  X^\upplus_{k, \tau}, A^\upplus_{k, \tau} = X_\tau, A_\tau \\
            x^\upplus, a \quad & \text{w.p.} \quad p^\upplus_k \spr{X_\tau, A_\tau} \quad \text{if} \quad  X^\upplus_{k, \tau}, A^\upplus_{k, \tau} = X_\tau, A_\tau \\
            x^\upplus, a \quad & \text{if} \quad X^\upplus_{k, \tau}, A^\upplus_{k, \tau} \neq X_\tau, A_\tau
        \end{cases}\,,
    \end{equation*}
    %
    where $a$ can be any action. Then, we can rewrite the bias term as
    %
    \begin{align*}
        &\modelbias_k \spr{\pi} = \spr{1 - \gamma} \bbE \sbr{\sum_{\tau=0}^\infty \gamma^\tau \spr{r_k^\upplus \spr{X^\upplus_{k, \tau}, A^\upplus_{k, \tau}} - r_k^\upplus \spr{X_\tau, A_\tau}}} \\
        &\quad\quad\quad= \spr{1 - \gamma} \bbE \sbr{\sum_{\tau=0}^\infty \gamma^\tau \II{\spr{X^\upplus_{k, \tau}, A^\upplus_{k, \tau}} \neq \spr{X_\tau, A_\tau}} \spr{r_k^\upplus \spr{X^\upplus_{k, \tau}, A^\upplus_{k, \tau}} - r_k^\upplus \spr{X_\tau, A_\tau}}}\,.
    \end{align*}
    %
    By definition, the state-action pairs $\spr{X^\upplus_{k, \tau}, A^\upplus_{k, \tau}}$ and $\spr{X_\tau, A_\tau}$ differ when the coupled process goes to heaven, \ie $X^\upplus_{k, \tau} = x^\upplus$. Noting that $r_k \spr{x^\upplus, a} = \RMAX$ for any action $a \in \cA$, we further get
    %
    \begin{align*}
        \modelbias_k \hspace{-1pt}\spr{\pi} &= \spr{1 - \gamma} \bbE \sbr{\sum_{\tau=0}^\infty \gamma^\tau \II{\spr{X^\upplus_{k, \tau}, A^\upplus_{k, \tau}} \neq \spr{X_\tau, A_\tau}} \hspace{-1pt}\spr{r_k^\upplus \spr{x^\upplus, A^\upplus_{k, \tau}} - r_k^\upplus \spr{X_\tau, A_\tau}}} \\
        &= \spr{1 - \gamma} \bbE \sbr{\sum_{\tau=0}^\infty \gamma^\tau \II{\spr{X^\upplus_{k, \tau}, A^\upplus_{k, \tau}} \neq \spr{X_\tau, A_\tau}} \spr{\RMAX - r_k^\upplus \spr{X_\tau, A_\tau}}}\,,
    \end{align*}
    %
    and $\modelbias_k \spr{\pi} \geq 0$ follows from $r_k^\upplus \preceq \RMAX$. For the upper bound, we can instead use $r_k \succeq 0$ and continue as follows
    %
    \begin{align*}
        \modelbias_k \spr{\pi} &\leq \spr{1 - \gamma} \RMAX \bbE \sbr{\sum_{\tau=0}^\infty \gamma^\tau \II{\spr{X^\upplus_{k, \tau}, A^\upplus_{k, \tau}} \neq \spr{X_\tau, A_\tau}} } \\
        &= \spr{1 - \gamma} \gamma \RMAX \sum_{\tau=0}^\infty \gamma^\tau \bbP \sbr{\spr{X^\upplus_{k, \tau+1}, A^\upplus_{k, \tau+1}} \neq \spr{X_{\tau+1}, A_{\tau+1}}}\,,
    \end{align*}
    %
    \begin{figure}
        \begin{center}
            \begin{tikzpicture}[every node/.style={circle, draw, minimum size=1cm, font=\small},
                >={Stealth}, % Arrow style
                shorten >=1pt, shorten <=1pt, % Adjust arrow spacing
                node distance=1.3cm and 1.3cm % Adjust node distances
            ]
            
            % true process
            \node (X0) at (0,0) {$X_{k, 0}$};
            \node (X1) [right=of X0] {$X_{k, 1}$};
            \node (X2) [right=1.3cm of X1] {$X_{k, 2}$};
            \node (X3) [right=of X2] {$X_{k, 3}$};
            \node (dots) [right=of X3, draw=none] {$\cdots$};
            
            % additional transitions for utopian process
            \node (uX0) [below=of X1] {$x^\upplus$};
            \node (uX1) [below=of X2] {$x^\upplus$};
            \node (uX2) [below=of X3] {$x^\upplus$};
            \node (udots) [right=of uX2, draw=none] {$\cdots$};
            
            % transitions true process
            \draw[->] (X0) -- (X1);
            \draw[->] (X1) -- (X2);
            \draw[->] (X2) -- (X3);
            \draw[->] (X3) -- (dots);
            
            % transitions utopian process
            \draw[->, dashed] (X0) to[bend left=20] (X1);
            \draw[->, dashed] (X0) -- (uX0);
            \draw[->, dashed] (X1) to[bend left=20] (X2);
            \draw[->, dashed] (X1) -- (uX1);
            \draw[->, dashed] (X2) to[bend left=20] (X3);
            \draw[->, dashed] (X2) -- (uX2);
            \draw[->, dashed] (X3) to[bend left=20] (dots);
            
            % stuck in heaven
            \draw[->, dashed] (uX0) -- (uX1);
            \draw[->, dashed] (uX1) -- (uX2);
            \draw[->, dashed] (uX2) -- (udots);
            
            \end{tikzpicture}
        \end{center}
        %
        \caption{The thick arrows represent the transitions of the process in the original MDP, while the dashed ones correspond to the utopian one.}
        %
        \label{fig:bias-term}
    \end{figure}
    %
    \hspace{-4.5pt}where we used $\bbP \sbr{\spr{X^\upplus_{k, 0}, A^\upplus_{k, 0}} \neq \spr{X_0, A_0}} = 0$ by definition. Then, as illustrated in Figure~\ref{fig:bias-term}, two cases can happen. Either the coupled process was still in the original MDP and transitioned to heaven, either it was already in heaven. Denoting for any $\tau \geq 0$ the event $\cE_\mathrm{split} \spr{\tau} = \scbr{\spr{X^\upplus_{k, \tau}, A^\upplus_{k, \tau}} \neq \spr{X_\tau, A_\tau}}$ and $\mathcal{E}^c_\mathrm{split} \spr{\tau}$ its complementary, we have
    %
    \begin{align*}
        \bbP \sbr{\cE_\mathrm{split} \spr{\tau+1}} &= \bbP \sbr{\cE_\mathrm{split} \spr{\tau+1} \given \cE_\mathrm{split} \spr{\tau}} \bbP \sbr{\cE_\mathrm{split} \spr{\tau}} \\
        &\phantom{=}+ \bbP \sbr{\cE_\mathrm{split} \spr{\tau+1} \given \cE^c_\mathrm{split} \spr{\tau}} \bbP \sbr{\cE^c_\mathrm{split} \spr{\tau}}\,.
    \end{align*}
    %
    If the coupled process is already in the heaven state $x^\upplus$, then it stays there. Otherwise, it can transition there with probability $\bbE \sbr{p_k^\upplus \spr{X_\tau, A_\tau}}$, thus
    %
    \begin{align*}
        \bbP \sbr{\cE_\mathrm{split} \spr{\tau+1}} &= \bbP \sbr{\cE_\mathrm{split} \spr{\tau}} + \bbE \sbr{p_k^\upplus \spr{X_\tau, A_\tau}} \bbP \sbr{\cE^c_\mathrm{split} \spr{\tau}} \\
        &\leq \bbP \sbr{\cE_\mathrm{split} \spr{\tau}} + \bbE \sbr{p_k^\upplus \spr{X_\tau, A_\tau}} \\
        &\leq \sum_{u=0}^\tau \bbE \sbr{p_k^\upplus \spr{X_u, A_u}}\,,
    \end{align*}
    %
    by induction. Therefore, we get
    %
    \begin{align*}
        \modelbias_k \spr{\pi} &\leq \spr{1 - \gamma} \gamma \RMAX \sum_{\tau=0}^\infty \gamma^\tau \sum_{u=0}^\tau \bbE \sbr{p_k^\upplus \spr{X_u, A_u}} \\
        &= \spr{1 - \gamma} \gamma \RMAX \bbE \sbr{\sum_{u=0}^\infty \sum_{\tau=0}^\infty \gamma^\tau \II{u \leq \tau} p_k^\upplus \spr{X_u, A_u}} \\
        &= \spr{1 - \gamma} \gamma \RMAX \bbE \sbr{\sum_{u=0}^\infty \sum_{\tau=u}^\infty \gamma^\tau p_k^\upplus \spr{X_u, A_u}} \\
        &= \gamma \RMAX \bbE \sbr{\sum_{u=0}^\infty \gamma^u p_k^\upplus \spr{X_u, A_u}}\,,
    \end{align*}
    %
    and the conclusion follows from the definition of $\mu \spr{\pi}$ and $\gamma < 1$.
\end{proof}


%%%%%%%%%%%%%%%%%%%%%%%%%%%%%%%%%%%%%%%%%%%%%%%%%%%%%%%%%%%%%%%%%%%%%%%%%%%%%%%%
        \subsubsection{Alternative Proof of Lemma~\ref{lem:model-bias-bounds}}

\noindent We also provide an alternative proof based on a simulation lemma.
%
\begin{proof}
    By the flow constraints associated to $\mu_k^\upplus \spr{\pi}$ and after rearranging, we get
    %
    \begin{align*}
        \modelbias_k \spr{\pi} &= \spr{1 - \gamma} \inp{\nu_0, V_{P_k^\upplus, r_k^\upplus}^\pi - V_{P, r_k^\upplus}^\pi} \\
        &= \inp{\opE\transpose \mu_k^\upplus \spr{\pi} - \gamma \spr{\opPplusk}\transpose \mu_k^\upplus \spr{\pi}, V_{P_k^\upplus, r_k^\upplus}^\pi - V_{P, r_k^\upplus}^\pi} \\
        &= \inp{\mu_k^\upplus \spr{\pi}, \opE V_{P_k^\upplus, r_k^\upplus}^\pi - \opE V_{P, r_k^\upplus}^\pi - \gamma \opPplusk V_{P_k^\upplus, r_k^\upplus}^\pi + \gamma \opPplusk V_{P, r_k^\upplus}^\pi}\,.
    \end{align*}
    %
    Applying Lemma~\ref{lem:from-ev-to-q} to both $V_{P_k^\upplus, r_k^\upplus}^\pi$ and $V_{P, r_k^\upplus}^\pi$ and Bellman's equation for $Q_{P_k^\upplus, r_k^\upplus}^\pi$, we further have    
    %
    \begin{align*}
        \modelbias_k \spr{\pi} &= \inp{\mu_k^\upplus \spr{\pi}, Q_{P_k^\upplus, r_k^\upplus}^\pi - Q_{P, r_k^\upplus}^\pi - \gamma \opPplusk V_{P_k^\upplus, r_k^\upplus}^\pi + \gamma \opPplusk V_{P, r_k^\upplus}^\pi} \\
        &= \inp{\mu_k^\upplus \spr{\pi}, r_k^\upplus + \gamma \opPplusk V_{P, r_k^\upplus}^\pi - Q_{P, r_k^\upplus}^\pi} \,,
    \end{align*}
    %
    Plugging the definition of $P_k^\upplus$ and this time using Bellman's equation for $Q_{P, r_k^\upplus}^\pi$, we obtain
    %
    \begin{align}
        \modelbias_k \spr{\pi} &= \inp{\mu_k^\upplus \spr{\pi}, r_k^\upplus + \gamma \spr{1 - p_k^\upplus} \odot \opP V_{P, r_k^\upplus}^\pi + p_k^\upplus \odot \bfe_{x^\upplus} V_{P, r_k^\upplus}^\pi - Q_{P, r_k^\upplus}^\pi} \nonumber \\
        &= \inp{\mu_k^\upplus \spr{\pi}, p_k^\upplus \odot \spr{\bfe_{x^\upplus} V_{P, r_k^\upplus}^\pi - \gamma \opP V_{P, r_k^\upplus}^\pi}} \nonumber \\
        &= \inp{\mu_k^\upplus \spr{\pi}, p_k^\upplus \odot \spr{\frac{\RMAX}{1 - \gamma} \bfone - \gamma \opP V_{P, r_k^\upplus}^\pi}}\,, \label{eq:hope-decomp}
    \end{align}
    %
    where the last equality is due to having $\spr{\bfe_{x^\upplus} V_{P, r_k^\upplus}^\pi} \spr{x, a} = V_{P, r_k^\upplus}^\pi \spr{x^\upplus} = \frac{\RMAX}{1 - \gamma}$ for any state-action pair $\spr{x, a}$. The lower bound follows from noticing that $\opP V_{P, r_k^\upplus}^\pi \preceq \frac{1}{1 - \gamma} \bfone \preceq \frac{\RMAX}{1 - \gamma} \bfone$,
    %
    \begin{align*}
        \modelbias_k \spr{\pi} &\geq \spr{\frac{\RMAX}{1 - \gamma} - \frac{\gamma \RMAX}{1 - \gamma}} \cdot \inp{\mu_k^\upplus \spr{\pi}, p_k^\upplus} \\
        &= \RMAX \cdot \inp{\mu_k^\upplus \spr{\pi}, p_k^\upplus} \\
        &\geq 0\,.
    \end{align*}
    %
    Moving to the upper bound, from Equation~\ref{eq:hope-decomp} and $\opP V_{P, r_k^\upplus}^\pi \succeq 0$, we get
    %
    \begin{align*}
        \modelbias_k \spr{\pi} &= \inp{\mu_k^\upplus \spr{\pi}, p_k^\upplus \odot \spr{\frac{\RMAX}{1 - \gamma} \bfone - \gamma \opP V_{P, r_k^\upplus}^\pi}} \\
        &\leq \frac{\RMAX}{1 - \gamma} \inp{\mu_k^\upplus \spr{\pi}, p_k^\upplus} \\
        &\leq \frac{\RMAX}{1 - \gamma} \inp{\mu \spr{\pi}, p_k^\upplus}\,,
    \end{align*}
    %
    where the last inequality follows from Lemma~\ref{lem:mass-reduced}.
\end{proof}


%%%%%%%%%%%%%%%%%%%%%%%%%%%%%%%%%%%%%%%%%%%%%%%%%%%%%%%%%%%%%%%%%%%%%%%%%%%%%%%%
%%%%%%%%%%%%%%%%%%%%%%%%%%%%%%%%%%%%%%%%%%%%%%%%%%%%%%%%%%%%%%%%%%%%%%%%%%%%%%%%
\subsection{Proof of Lemma~\ref{lem:bound-regret-plus} (optimistic regret)}
\label{app:bound-regret-plus}

In order to prove Lemma~\ref{lem:bound-regret-plus}, we first need the following result that shows that the functions $Q_k$ are optimistic estimates of an ideal sequence of dynamic-programming updates computed in the augmented MDPs. The statement is is an adaptation of Lemma~4.2 of \citet{MN23}, and its complete proof is provided below.
%
\begin{restatable}{Lem}{boundq} \label{lem:bound-q}
    Suppose that the bonuses $\CB_k$ are valid for the MDP $\cM_k$ in the sense of Equation~\ref{eq:validity-bonuses}. Then, for any $k$ and any state-action pair $\spr{x, a} \in \cX^\upplus \times \cA$, the iterates $Q_k$ satisfy
    %
    \begin{equation*} \label{eq:bound-q}
        r_k^\upplus + \gamma \opPplusk V_k \leq Q_{k+1} \leq r_k^\upplus + 2 \spr{1 - p_k^\upplus} \odot \CB_k + \gamma \opPplusk V_k\,.
    \end{equation*}
\end{restatable}

\begin{proof}
    For $x^\upplus$ and any action $a$, it is straightforward to check that both inequalities are equalities. Let $\spr{x, a} \in \cX \times \cA$. We have
    %
    \begin{align*}
        r_k^\upplus \spr{x, a} + \gamma \opPplusk V_k \spr{x, a} &= r_k^\upplus \spr{x, a} + \gamma \spr{\opPplusk - \ophPplusk} V_k \spr{x, a} + \gamma \ophPplusk V_k \spr{x, a} \\
        &\leq Q_{k+1} \spr{x, a} \\
        &\leq r_k^\upplus \spr{x, a} + 2 \spr{1 - p_k^\upplus \spr{x, a}} \CB_k \spr{x, a} + \gamma \opPplusk V_k \spr{x, a}\,,
    \end{align*}
    %
    where both inequalities use the fact that
    %
    \begin{equation*}
        \abs{\spr{\opPplusk - \ophPplusk} V_k \spr{x, a}} \leq \spr{1 - p_k^\upplus \spr{x, a}} \CB_k \spr{x, a}\,,
    \end{equation*}
    %
    which is implied by the event $\mathcal{E}_{\mathrm{valid}}$ in Equation~\ref{eq:validity-bonuses}.
\end{proof}

\boundregretplus*

\begin{proof}
    We decompose $\regretKplus$ as follows
    %
    \begin{align*}
        \regretKplus &= \sumkK \Bigl(\underbrace{\spr{\inp{\mu_k^\upplus \spr{\pi^\star}, r_k^\upplus} - \spr{1 - \gamma} \inp{\initial, V_k}}}_{= \Delta_k^\star} + \underbrace{\spr{1 - \gamma} \inp{\initial, V_k} - \inp{\mu_k^\upplus \spr{\pi_k}, r_k^\upplus}}_{= \Delta_k}\Bigr)\,,
    \end{align*}
    %
    where we defined $\Delta_k^\star$ and $\Delta_k$. We start with the first term. Using the flow constraint with $\mu_k^\upplus \spr{\pi^\star}$,
    %
    \begin{align*}
        \Delta_k^\star &= \inp{\mu_k^\upplus \spr{\pi^\star}, r_k^\upplus} - \inp{\opE\transpose \mu_k^\upplus \spr{\pi^\star} - \gamma \spr{\opPplusk}\transpose \mu_k^\upplus \spr{\pi^\star}, V_k} \\
        &= \inp{\mu_k^\upplus \spr{\pi^\star}, r_k^\upplus + \gamma \opPplusk V_k - \opE V_{k+1}} + \inp{\mu_k^\upplus \spr{\pi^\star}, \opE V_{k+1} - \opE V_k}\,.
    \end{align*}
    %
    Using the lower bound on $Q_{k+1}$ from Lemma~\ref{lem:bound-q}, we have
    %
    \begin{equation*}
        \Delta_k^\star \leq \inp{\mu_k^\upplus \spr{\pi^\star}, Q_{k+1} - \opE V_{k+1}} + \inp{\mu_k^\upplus \spr{\pi^\star}, \opE V_{k+1} - \opE V_k}\,,
    \end{equation*}
    %
    where the term in $x = x^\upplus$ is equal to zero. Summing over $k \in \sbr{K} = \bigcup_{e \in \sbr{1, E \spr{K}}} \sbr{k_e, k_{e+1} - 1}$,
    %
    \begin{equation*}
        \sumkK \Delta_k^\star \leq \sum_{e=1}^{E \spr{K}} \inp{\mu_{k_e}^\upplus \spr{\pi^\star}, \sum_{k \in \cK_e} \spr{Q_{k+1} - \opE V_{k+1}}} + \sum_{e=1}^{E \spr{K}} \inp{\opE\transpose \mu_{k_e}^\upplus \spr{\pi^\star}, V_{k_{e+1}} - V_{k_e}}_{\cX}\,,
    \end{equation*}
    %
    where the sum within each epoch of the second term telescoped. By \citealp[Lemma~C.1]{MN23}, we have for any state $x \in \cX$
    %
    \begin{align*}
        \sum_{k \in \cK_e} V_{k+1} \spr{x} &= \max_{p \in \Delta \spr{\cA}} \inp{p, \sum_{k \in \cK_e} Q_{k+1} \spr{x, \cdot}} - \frac1\eta \KL \spr{p \| \piunif} \\
        &\geq \inp{\pi^\star \spr{\cdot \given x}, \sum_{k \in \cK_e} Q_{k+1} \spr{x, \cdot}} - \frac1\eta \KL \spr{\pi^\star \spr{\cdot \given x} \| \piunif}\,,
    \end{align*}
    %
    where we used $\pi_{k_e} = \piunif$ in the first equality and denoted $\cK_e$ the set of episodes in epoch $e$. Multiplying the previous inequality by $\nu_{k_e}^\upplus \spr{\pi^\star,x}$, summing over $x \in \cX$, and noting that $\mu_{k_e}^\upplus \spr{\pi^\star} = \nu_{k_e}^\upplus \spr{\pi^\star} \odot \pi^\star$, we obtain
    %
    \begin{align*}
        \sum_{e=1}^{E \spr{K}} \inp{\mu_{k_e}^\upplus \spr{\pi^\star}, \sum_{k \in \cK_e} \spr{Q_{k+1} - \opE V_{k+1}}} &\leq \frac1\eta \sum_{e=1}^{E \spr{K}} \inp{\nu_{k_e}^\upplus \spr{\pi^\star}, \KL \spr{\pi^\star \| \piunif}} \\
        &\leq \frac{E \spr{K} \log \abs{\cA}}{\eta}\,.
    \end{align*}
    %
    The second term can be bounded with Hölder's inequality,
    %
    \begin{equation*}
        \sum_{e=1}^{E \spr{K}} \inp{\opE\transpose \mu_{k_e}^\upplus \spr{\pi^\star}, V_{k_{e+1}} - V_{k_e}} \leq \sum_{e=1}^{E \spr{K}} \norm{\nu_{k_e}^\upplus \spr{\pi^\star}}_1 \norm{V_{k_{e+1}} - V_{k_e}}_\infty \leq 2 E \spr{K} \VMAX\,,
    \end{equation*}
    %
    where we used $\nu_{k_e}^\upplus \spr{\pi^\star} \in \Delta \spr{\cX^\upplus}$ and $\norm{V_k}_\infty \leq \VMAX$ which follows from $\norm{Q_k}_\infty \leq \QMAX$. Therefore, we get
    %
    \begin{equation*}
        \sumkK \Delta_k^\star \leq \frac{E \spr{K} \log \abs{\cA}}{\eta} + 2 E \spr{K} \VMAX\,.
    \end{equation*}

    \noindent Moving to $\Delta_k$, we apply the flow constraints to $\mu_k^\upplus \spr{\pi_k}$ to get
    %
    \begin{align*}
        \Delta_k &= \inp{\opE\transpose \mu_k^\upplus \spr{\pi_k} - \gamma \spr{\opPplusk}\transpose \mu_k^\upplus \spr{\pi_k}, V_k} - \inp{\mu_k^\upplus \spr{\pi_k}, r_k^\upplus} \\
        &= \inp{\mu_k^\upplus \spr{\pi_k}, \opE V_k} - \inp{\mu_k^\upplus \spr{\pi_k}, r_k^\upplus + \gamma \opPplusk V_k} \\
        &\leq \inp{\mu_k^\upplus \spr{\pi_k}, \opE V_k - Q_{k+1}} + 2 \inp{\mu_k^\upplus \spr{\pi_k}, \spr{1 - p_k^\upplus} \odot \CB_k} \\
        &\leq \inp{\mu_k^\upplus \spr{\pi_k}, \opE V_k - Q_{k+1}} + 2 \inp{\mu \spr{\pi_k}, \spr{1 - p_k^\upplus} \odot \CB_k}\,,
    \end{align*}
    %
    where the first inequality follows from the upper bound on $Q_{k+1}$ in Lemma~\ref{lem:bound-q} and the term in $x = x^\upplus$ being equal to zero, and the second inequality is due to Lemma~\ref{lem:mass-reduced}. Next, noticing $\inp{\mu_k^\upplus \spr{\pi_{k+1}}, Q_{k+1}} = \inp{\nu_k^\upplus \spr{\pi_{k+1}}, V_{k+1} + \frac1\eta \KL \spr{\pi_{k+1} \| \pi_k}}$,
    %
    \begin{align*}
        \inp{\mu_k^\upplus \spr{\pi_k}, \opE V_k - Q_{k+1}} &= \inp{\nu_k^\upplus \spr{\pi_k}, V_k} - \inp{\mu_k^\upplus \spr{\pi_{k+1}}, Q_{k+1}} \\
        &\phantom{=}+ \inp{\mu_k^\upplus \spr{\pi_{k+1}}, Q_{k+1}} - \inp{\mu_k^\upplus \spr{\pi_k}, Q_{k+1}} \\
        &= \inp{\nu_k^\upplus \spr{\pi_k}, V_k} - \inp{\nu_k^\upplus \spr{\pi_{k+1}}, V_{k+1}} \\
        &\phantom{=}- \frac1\eta \inp{\nu_k^\upplus \spr{\pi_{k+1}}, \KL \spr{\pi_{k+1} \| \pi_k}}  \\
        &\phantom{=}+ \inp{\mu_k^\upplus \spr{\pi_{k+1}} - \mu_k^\upplus \spr{\pi_k}, Q_{k+1}}\,.
    \end{align*}
    %
    We sum over $k$ and look at the different terms separately. First, we get
    %
    \begin{align*}
        \sumkK \spr{\inp{\nu_k^\upplus \spr{\pi_k}, V_k} - \inp{\nu_k^\upplus \spr{\pi_{k+1}}, V_{k+1}}} &= \sum_{e=1}^{E \spr{K}} \spr{\inp{\nu_{k_e}^\upplus \spr{\pi_{k_e}}, V_{k_e}} - \inp{\nu_{k_e}^\upplus \spr{\pi_{k_{e+1}}}, V_{k_{e+1}}}} \\
        &\leq 2 \VMAX E \spr{K}\,.
    \end{align*}
    %
    For the third term, we have
    %
    \begin{align*}
        \sumkK \inp{\mu_k^\upplus \spr{\pi_{k+1}} - \mu_k^\upplus \spr{\pi_k}, Q_{k+1}} &= \sum_{e=1}^{E \spr{K}} \sum_{k \in \cK_e} \inp{\mu_{k_e}^\upplus \spr{\pi_{k+1}} - \mu_{k_e}^\upplus \spr{\pi_k}, Q_{k+1}}\,.
    \end{align*}
    %
    Successively applying Hölder's inequality, Pinsker's inequality and Lemma~\ref{lem:ineq-kl-entropy},
    %
    \begin{align*}
        \inp{\mu_{k_e}^\upplus \spr{\pi_{k+1}} - \mu_{k_e}^\upplus \spr{\pi_k}, Q_{k+1}} &\leq \QMAX \norm{\mu_{k_e}^\upplus \spr{\pi_{k+1}} - \mu_{k_e}^\upplus \spr{\pi_k}}_1 \\
        &\leq \QMAX \sqrt{2 \KL \spr{\mu_{k_e}^\upplus \spr{\pi_{k+1}} \| \mu_{k_e}^\upplus \spr{\pi_k}}} \\
        &\leq \QMAX \sqrt{\frac{2}{1 - \gamma} \inp{\nu_{k_e}^\upplus \spr{\pi_{k+1}}, \KL \spr{\pi_{k+1} \| \pi_k}}}\,.
    \end{align*}
    %
    For any $x \in \cX$, the KL divergence between $\pi_{k+1}$ and $\pi_k$ in state $x$ can be bounded as
    %
    \begin{align*}
        &\KL \spr{\pi_{k+1} \middle\| \pi_k} \spr{x} \\
        &\quad\quad\quad= \sum_{a \in \cA} \pi_{k+1} \spr{a \given x} \spr{\eta Q_{k+1} \spr{x, a} - \log \spr{\sum_{b \in \cA} \pi_k \spr{b \given x} \exp \sbr{\eta Q_{k+1} \spr{x, b}}}} \\
        &\quad\quad\quad= \eta \sum_{a \in \cA} \pi_{k+1} \spr{a \given x} Q_{k+1} \spr{x, a} - \log \spr{\sum_{b \in \cA} \pi_k \spr{b \given x} \exp \sbr{\eta Q_{k+1} \spr{x, b}}} \\
        &\quad\quad\quad\leq \eta \sum_{a \in \cA} \sbr{\pi_{k+1} \spr{a \given x} - \pi_k \spr{a \given x}} Q_{k+1} \spr{x, a} \\
        &\quad\quad\quad\leq \eta \QMAX \norm{\pi_{k+1} \spr{\cdot \given x} - \pi_k \spr{\cdot \given x}}_1 \\
        &\quad\quad\quad\leq \eta \QMAX \sqrt{2 \KL \spr{\pi_{k+1} \middle\| \pi_k} \spr{x}}\,,
    \end{align*}
    %
    where the first inequality follows from Jensen's and the convexity of $- \log$, the second inequality is by Hölder's and the boundedness of $Q_k$, and the last inequality is due to Pinkser's. Dividing by $\sqrt{\KL \spr{\pi_{k+1} \middle\| \pi_k} \spr{x}}$ and squaring the inequality, we get
    %
    \begin{equation*}
        \KL \spr{\pi_{k+1} \middle\| \pi_k} \spr{x} \leq 2 \eta^2 \QMAX^2\,.
    \end{equation*}
    %
    Plugging this back into the previous inequality and summing over $k \in \sbr{K}$, we get
    %
    \begin{equation*}
        \sumkK \inp{\mu_k^\upplus \spr{\pi_{k+1}} - \mu_k^\upplus \spr{\pi_k}, Q_{k+1}} \leq \frac{2 \QMAX^2 \eta K}{\sqrt{1 - \gamma}}\,.
    \end{equation*}
    %
    The remaining term is nonpositive. The sum of the $\Delta_k$ terms is thus bounded by
    %
    \begin{equation*}
        \sumkK \Delta_k \leq 2 \VMAX E \spr{K} + \frac{2 \QMAX^2 \eta K}{\sqrt{1 - \gamma}} + 2 \sumkK \inp{\mu \spr{\pi_k}, \spr{1 - p_k^\upplus} \odot \CB_k}\,.
    \end{equation*}
    %
    Finally, combining the bounds on $\sumkK \Delta_k^\star$ and $\sumkK \Delta_k$, we get
    \begin{equation*}
        \regretKplus \leq \frac{E \spr{K} \log \abs{\cA}}{\eta} + 4 \VMAX E \spr{K} + \frac{2 \QMAX^2 \eta K}{\sqrt{1 - \gamma}} + 2 \sumkK \inp{\mu \spr{\pi_k}, \spr{1 - p_k^\upplus} \odot \CB_k}\,.
    \end{equation*}
\end{proof}


%%%%%%%%%%%%%%%%%%%%%%%%%%%%%%%%%%%%%%%%%%%%%%%%%%%%%%%%%%%%%%%%%%%%%%%%%%%%%%%%
%%%%%%%%%%%%%%%%%%%%%%%%%%%%%%%%%%%%%%%%%%%%%%%%%%%%%%%%%%%%%%%%%%%%%%%%%%%%%%%%
\subsection{Proof of Lemma~\ref{lem:qmax} (choice of $\QMAX$)}
\label{app:qmax}

\qmaxlemma*

\begin{proof}
    We want to find $\QMAX$ such that for any $k$, $\norm{Q_k}_\infty \leq \QMAX$. Since $V_k$ is a log-sum-exp of $Q_k$, we have $\norm{V_k}_\infty \leq \norm{Q_k}_\infty$. Next, we proceed by induction to find a suitable value of $\QMAX$. Let $k \in \bbN^\star$ and assume $\norm{Q_k}_\infty \leq \QMAX$. For any $\spr{x, a}$,
    %
    \begin{align*}
        \abs{Q_{k+1} \spr{x, a}} &\leq r_k^\upplus \spr{x, a} + \spr{1 - p_k^\upplus \spr{x, a}} \CB_k \spr{x, a} + \gamma \abs{\ophPplusk V_k \spr{x, a}} \\
        &\leq r_k^\upplus \spr{x, a} + 2 \spr{1 - p_k^\upplus \spr{x, a}} \CB_k \spr{x, a} + \gamma \opPplusk V_k \spr{x, a} \\
        &\leq \RMAX + 2 \spr{1 - p_k^\upplus \spr{x, a}} \CB_k \spr{x, a} + \gamma \QMAX\,,
    \end{align*}
    %
    where the second inequality follows from the validity of the bonuses (and corresponds to the upper bound on $Q_{k+1}$ from Lemma~\ref{lem:bound-q}), and the third inequality is due to the inductive assumption and the boundedness of the rewards. Plugging the definition of the probabilities $p_k^\upplus$, we further get
    %
    \begin{align*}
        \abs{Q_{k+1} \spr{x, a}} &\leq \RMAX + 2 \sigma \spr{\omega - \alpha \CB_k \spr{x, a}} \CB_k \spr{x, a} + \gamma \QMAX \\
        &\leq \RMAX + \gamma \QMAX + 2 \sup_{z \geq 0} \scbr{\sigma \spr{\omega - \alpha z} z} \\
        &\leq \RMAX + \gamma \QMAX + \frac{2 \omega}{\alpha}\,,
    \end{align*}
    %
    the last inequality is simply a property of the sigmoid function and is showm in Lemma~\ref{lem:sigmoid-bound2}. For the induction to work at time $k + 1$, we need to set $\QMAX$ such that $\QMAX = \RMAX + \gamma \QMAX + \frac{2 \omega}{\alpha}$, that is
    %
    \begin{equation*}
        \QMAX = \frac{\RMAX + 2 \omega / \alpha}{1 - \gamma}\,.
    \end{equation*}
    %
    The initial case is also true since $\norm{Q_1}_\infty \leq \frac{\RMAX}{1 - \gamma} \leq \QMAX$.
\end{proof}


%%%%%%%%%%%%%%%%%%%%%%%%%%%%%%%%%%%%%%%%%%%%%%%%%%%%%%%%%%%%%%%%%%%%%%%%%%%%%%%%
%%%%%%%%%%%%%%%%%%%%%%%%%%%%%%%%%%%%%%%%%%%%%%%%%%%%%%%%%%%%%%%%%%%%%%%%%%%%%%%%
\subsection{Proof of Lemma~\ref{lem:expected-bonuses-bound} (bound on bonuses)}
\label{app:expected-bonuses-bound}

We now control the sum of bonuses. For any episode $k$, we will denote $\cT_k$ the set of timesteps in episode $k$.

\expectedbonusesbound*

\noindent To prove Lemma~\ref{lem:expected-bonuses-bound}, we need the following result.
%
\begin{lemma} \label{lem:expectation-to-highprob}
    Suppose $\mathcal{E}_L$ holds. Let $\scbr{f_k}_{k \in \sbr{K}} \subset \bbR^{\cX \times \cA}$ be a sequence of functions with values in $\sbr{0, M}$ almost surely. Then, with probability at least $1 - \delta$ the schedule and policies produced by Algorithm~\ref{alg:linear-rmax-ravi-ucb} satisfy
    %
    \begin{equation*}
        \sumkK \inp{\mu \spr{\pi_k}, f_k} \leq 2 \spr{1 - \gamma} \sumkK \sum_{t \in \cT_k} f_k \spr{X_t, A_t} + 4 M \log \spr{\frac{2 K}{\delta}}^2\,.
    \end{equation*}
\end{lemma}
%
\begin{proof}
    We denote $\cF_{k-1}$ the $\sigma$-field generated by the history up to the end of episode $k-1$. We have,
    %
    \begin{align*}
        \inp{\mu \spr{\pi_k}, f_k} &= \spr{1 - \gamma} \bbE \sbr{\sum_{\tau = 0}^\infty \gamma^\tau f_k \spr{X_\tau, A_\tau} \given \cF_{k-1}} \\
        &= \spr{1 - \gamma} \bbE \sbr{\sum_{\tau = 0}^\infty \II{\tau < L_k} f_k \spr{X_\tau, A_\tau} \given \cF_{k-1}} \\
        &= \spr{1 - \gamma} \bbE \sbr{\sum_{\tau = 0}^{L_k - 1} f_k \spr{X_\tau, A_\tau} \given \cF_{k-1}} \\
        &= \spr{1 - \gamma} \bbE \sbr{\sum_{\tau \in \cT_k} f_k \spr{X_\tau, A_\tau} \given \cF_{k-1}}\,.
    \end{align*}
    %
    Plugging it back in the previous display,
    %
    \begin{align*}
        \sumkK \inp{\mu \spr{\pi_k}, f_k} &= \spr{1 - \gamma} \sumkK \bbE \sbr{\sum_{t \in \cT_k} f_k \spr{X_t, A_t} \given \cF_{k-1}}\,.
    \end{align*}
    %
    Since we assume $\cE_L$ holds, for any $k$ we have that $\sum_{t \in \cT_k} f_k \spr{X_t, A_t}$ takes values in $\sbr{0, M \LMAX}$ almost surely. Using the concentration inequality from Lemma~\ref{lem:concentration-ineq-cond-exp}, we get
    %
    \begin{align*}
        \sumkK \inp{\mu \spr{\pi_k}, f_k} &\leq 2 \spr{1 - \gamma} \sumkK \sum_{t \in \cT_k} f_k \spr{X_t, A_t} + 4 \spr{1 - \gamma} M \LMAX \log \spr{\frac{2 K}{\delta}} \\
        &\leq 2 \spr{1 - \gamma} \sumkK \sum_{t \in \cT_k} f_k \spr{X_t, A_t} + 4 M \log \spr{\frac{2 K}{\delta}}^2\,,
    \end{align*}
    %
    where we used that $\LMAX = \frac{\log \spr{K / \delta}}{1 - \gamma}$.
\end{proof}
%

We now prove Lemma~\ref{lem:expected-bonuses-bound}. With a slight abuse of notation, we use the convention that for any epoch $e$ and any $t$ in epoch $e$, the bonuses at time step $t$ are $\CB_t = \CB_{t_e}$. Noting that the bonuses $\CB_t$ take values in $\sbr{0, \beta B}$, we apply Lemma~\ref{lem:expectation-to-highprob} to get
%
\begin{equation*}
    \sumkK \inp{\mu \spr{\pi_k}, \CB_k} \leq 2 \spr{1 - \gamma} \sumtT \CB_t \spr{X_t, A_t} + 4 \beta B \log \spr{\frac{2 K}{\delta}}^2\,,
\end{equation*}
%
where we used $\CB_t \succeq 0$, and $T_{K+1} \leq T$ which follows from the event $\cE_L$. Likewise, applying Lemma~\ref{lem:expectation-to-highprob} to $\CB_t^2 \in \sbr{0, \beta^2 B^2}$, we obtain a similar bound for the term $\sumkK \inp{\mu \spr{\pi_k}, \CB_k^2}$,
%
\begin{equation*}
    \sumkK \inp{\mu \spr{\pi_k}, \CB_k^2} \leq 2 \spr{1 - \gamma} \sumtT \CB_t \spr{X_t, A_t}^2 + 4 \beta^2 B^2 \log \spr{\frac{2 K}{\delta}}^2\,.
\end{equation*}
%
By Cauchy-Schwartz's inequality, we have $\sumtT \CB_t \spr{X_t, A_t} \leq \sqrt{T} \sqrt{\sumtT \CB_t \spr{X_t, A_t}^2}$, so we can focus on the latter sum. By definition of the bonuses for linear MDPs, we have
%
\begin{align*}
    \sumtT \CB_t \spr{X_t, A_t}^2 &= \beta^2 \sum_{e=1}^{E \spr{K}} \sum_{k \in \cK_e} \sum_{t \in \cT_k} \norm{\phi \spr{X_t, A_t}}_{\Lambda_{t_e}^{-1}}^2\,.
\end{align*}
%
Since the covariance matrix only contains the data until the beginning of the epoch, there is a delay with $\phi \spr{X_t, A_t}$ which is further ahead. To compensate for this, note that for any $t \in \sbr{t_e, t_{e + 1}-1}$, we have $\det \Lambda_t \leq 2 \det \Lambda_{t_e}$ due to the update condition in Algorithm~\ref{alg:linear-rmax-ravi-ucb}, so by Lemma~\ref{lem:det-elliptical-bound}
%
\begin{align*}
    \norm{\phi \spr{X_t, A_t}}_{\Lambda_{t_e}^{-1}}^2 \leq \frac{\det \spr{\Lambda_{t_e}^{-1}}}{\det \spr{\Lambda_t^{-1}}} \norm{\phi \spr{X_t, A_t}}_{\Lambda_t^{-1}}^2 \leq 2 \norm{\phi \spr{X_t, A_t}}_{\Lambda_t^{-1}}^2\,.
\end{align*}
%
We plug this back into the previous inequality and apply Lemma~\ref{lem:bound-elliptical-potential} to obtain\footnote{Note that this is where the linear dependency in $B$ appears, but this can be removed by setting $\lambda = 1 / B^2$.}
%
\begin{equation*}
    \sumtT \CB_t \spr{X_t, A_t}^2 \leq 2 \beta^2 \sumtT \norm{\phi \spr{X_t, A_t}}_{\Lambda_t^{-1}}^2 \leq 4 \beta^2 B^2 \log \spr{\frac{\det \Lambda_T}{\det \Lambda_0}}\,.
\end{equation*}
%
Using the definition of $\Lambda_0, \Lambda_T$, the trace-determinant inequality, and the assumption $\norm{\phi \spr{\cdot, \cdot}}_2 \leq B$, we finally get
%
\begin{align*}
    \sumtT \CB_t \spr{X_t, A_t}^2 &\leq 4 \beta^2 B^2 d \log \spr{\frac{d + \sumtT \norm{\phi \spr{X_t, A_t}}_2^2}{d}} \\
    &\leq 4 \beta^2 B^2 d \log \spr{1 + \frac{B^2 T}{d}}\,.
\end{align*}
%
The conclusion follows from plugging this back into the inequalities of interest.


%%%%%%%%%%%%%%%%%%%%%%%%%%%%%%%%%%%%%%%%%%%%%%%%%%%%%%%%%%%%%%%%%%%%%%%%%%%%%%%%
%%%%%%%%%%%%%%%%%%%%%%%%%%%%%%%%%%%%%%%%%%%%%%%%%%%%%%%%%%%%%%%%%%%%%%%%%%%%%%%%
\subsection{Proof of Lemma~\ref{lem:good-event-holds} (good event holds)}
\label{app:good-event-holds}

Before stating the proof of Lemma~\ref{lem:good-event-holds}, we need to define some auxiliary quantities and state two intermediate results. First recall that $\scbr{L_k}_{k=1}^K$ denote the number of steps between consecutive resets and that for any $k \geq 2$, $L_k = T_k - T_{k-1}$, and $L_1 = T_1$. We need to prove the episodes are not too long, \ie $\cE_L = \scbr{\forall k \in \sbr{K}, L_k \leq \LMAX}$ holds with high probability, where $\LMAX = H \log \spr{K / \delta}$. This is done in Lemma~\ref{lem:lmax}. Then, we define the event $\cE_{V, \mathrm{alg}}$ on the iterates generated by Algorithm~\ref{alg:linear-rmax-ravi-ucb}
%
\begin{equation*}
    \cE_{V, \mathrm{alg}} = \scbr{\forall k \in \sbr{K}, \norm{M V_k - \wh{M V_k}}_{\Lambda_{T_k}} \leq \beta}\,,
\end{equation*}
%
where $\wh{M V_k} = \Lambda_{T_k}^{-1} \sum_{\spr{x, a, x'} \in \cD_{T_k}} \phi \spr{x, a} V_k \spr{x'}$. To prove $\cE_{V, \mathrm{alg}}$ holds with high probability, we need to resort to a standard uniform covering argument first introduced by \citealp{jin2019provably}. To do so, let us denote with $p^\upplus_{\Lambda, \beta, \alpha} = \sigma (\alpha \beta \norm{\phi\spr{\cdot,\cdot}}_{\Lambda} - w) = 1 - \sigma \spr{- \alpha \beta \norm{\phi \spr{\cdot, \cdot}}_{\Lambda} + \omega}$ an ascension function parametrized by the matrix $\Lambda$, the scalar $\beta$ and the sigmoid slope $\alpha$. Then, we define the following class of functions on $\cX \times \cA$
%
\begin{align*}
    \cQ &= \bigg\{Q: \X\times\mathcal{A} \rightarrow \mathbb{R} \quad \text{s.t.} \\
    &Q = \spr{1 - p^\upplus_{\Lambda, \beta, \alpha}} \odot \spr{\Phi \theta + \beta \norm{\phi \spr{\cdot, \cdot}}_{\Lambda}} + p^\upplus_{\Lambda, \beta, \alpha} \cdot \frac{\RMAX}{1-\gamma}, \\
    &\beta = \widetilde{\mathcal{O}}\brr{Q_{\max} d}, ~~~~\alpha = 2 \omega, ~~~\lambda_{\max}(\Lambda) \leq 1, ~~~\lambda_{\min}(\Lambda) \geq \frac{1}{2K B L_{\max}}, \\
    &\norm{\theta} \leq \WMAX + \QMAX \LMAX K B, \norm{Q}_\infty \leq \QMAX \bigg\} \cup \scbr{\mathbf{0}}\,,
\end{align*}
%
where $\QMAX = H \spr{\RMAX + \frac{2 \omega}{\alpha}}$ and we included the function $0$ to make sure $Q_1 \in \cQ$. Furthermore, denote for any $\eta > 0$ the function $f_\eta: \bbR^{\cX \times \cA} \rightarrow \bbR^\cX$ defined for $Q \in \bbR^{\cX \times \cA}$ as $f_\eta \spr{Q} = \frac1\eta \log \sum_{a \in \cA} \exp \spr{\eta Q \spr{\cdot, a}}$. We then define the following function class in $\bbR^\cX$
%
\begin{equation} \label{eq:function-class-v}
    \cV = \bigg\{ V: \cX \rightarrow \bbR \quad \text{s.t.} \quad\exists \scbr{Q_\ell}_{\ell=1}^K, \scbr{\bar{Q}_\ell}_{\ell=1}^K \subset \cQ, V = f_\eta \circ \spr{\sum^K_{\ell=1} Q_\ell} - f_\eta \circ \spr{\sum^K_{\ell=1} \bar{Q}_\ell} \bigg\},
\end{equation}
%
as well as the event
%
\begin{equation*}
    \cE_\cV = \scbr{\forall V \in \cV, \forall k \in \sbr{K}, \norm{M V - \wh{M V}}_{\Lambda_{T_k}} \leq \beta}\,,
\end{equation*}
where $\wh{M V} = \Lambda_{T_k}^{-1} \sum_{\spr{x, a, x'} \in \cD_{T_k}} \phi \spr{x, a}  V \spr{x'}$. Finally, we define the event that the iterates of Algorithm~\ref{alg:linear-rmax-ravi-ucb} are in the function class $\cV$
%
\begin{align*}
    \cE_{\mathrm{in}} = \scbr{\forall k \in \sbr{K}, V_k \in \cV}\,.
\end{align*}
%
What remains is to show that the iterates of the algorithm belong to $\cV$, and that the event $\cE_\cV$ holds with high probability. This is done in Lemmas~\ref{lem:iterates-in-class} and \ref{lem:uniform-event-holds}, respectively. We can now prove Lemma~\ref{lem:good-event-holds}.


%%%%%%%%%%%%%%%%%%%%%%%%%%%%%%%%%%%%%%%%%%%%%%%%%%%%%%%%%%%%%%%%%%%%%%%%%%%%%%%%
\goodeventholds*

\begin{proof}
    For any episode $k \in \sbr{K}$ and state-action pair $\spr{x, a} \in \cX \times \cA$, we have by Cauchy-Schwartz's inequality
    %
    \begin{equation*}
        \abs{P V_k \spr{x, a} - \wh{P V_k} \spr{x, a}} \leq \norm{M V_k - \wh{M V_k}}_{\Lambda_{T_k}} \norm{\phi \spr{x, a}}_{\Lambda_{T_k}^{-1}}\,.
    \end{equation*}
    %
    This inequality shows that the event $\cE_{V, \mathrm{alg}}$ implies the event $\cE_{\mathrm{valid}}$, \ie
    %
    \begin{align*}
        \bbP \sbr{\cE_{\mathrm{valid}} \cap \cE_L} &= \bbP \sbr{\cE_{\mathrm{valid}} \given \cE_L} \bbP \sbr{\cE_L} \\
        &\geq \bbP \sbr{\cE_{V, \mathrm{alg}} \given \cE_L} \bbP \sbr{\cE_L} \\
        &\geq \bbP \sbr{\cE_{V, \mathrm{alg}} \cap \cE_{\mathrm{in}} \given \cE_L} \spr{1 - \delta}\,,
    \end{align*}
    %
    where in the last inequality we used the monotonicity of $\bbP$ and Lemma~\ref{lem:lmax}. Then, conditioned on the event $\cE_{\mathrm{in}}$ that the iterates are in the function class $\cV$, the event $\cE_\cV$ implies $\cE_{V, \mathrm{alg}}$, that is
    %
    \begin{align*}
        \bbP \sbr{\cE_{V, \mathrm{alg}} \cap \cE_{\mathrm{in}} \given \cE_L} &= \bbP \sbr{\cE_{V, \mathrm{alg}} \given \cE_{\mathrm{in}} \cap \cE_L} \bbP \sbr{\cE_{\mathrm{in}} \given \cE_L} \\
        &\geq \bbP \sbr{\cE_\cV \given \cE_{\mathrm{in}} \cap \cE_L} \bbP \sbr{\cE_{\mathrm{in}} \given \cE_L} \\
        &= \bbP \sbr{\cE_\cV \cap \cE_{\mathrm{in}} \given \cE_L}\,.
    \end{align*}
    %
    Finally, by Lemma~\ref{lem:iterates-in-class} we have $\bbP \sbr{\cE_{\mathrm{in}} \given \cE_\cV, \cE_L} = 1$ thus
    %
    \begin{align*}
        \bbP \sbr{\cE_\cV \cap \cE_{\mathrm{in}} \given \cE_L} &= \bbP \sbr{\cE_{\mathrm{in}} \given \cE_\cV, \cE_L} \bbP \sbr{\cE_\cV \given \cE_L} \\
        &= \bbP \sbr{\cE_\cV \given \cE_L} \\
        &\geq 1 - \delta\,,
    \end{align*}
    %
    where the last inequality follows from Lemma~\ref{lem:uniform-event-holds}. In conclusion, we get
    %
    \begin{equation*}
        \bbP \sbr{\cE_{\mathrm{valid}} \cap \cE_L} \geq \spr{1 - \delta}^2 \geq 1 - 2 \delta\,.
    \end{equation*}
\end{proof}


%%%%%%%%%%%%%%%%%%%%%%%%%%%%%%%%%%%%%%%%%%%%%%%%%%%%%%%%%%%%%%%%%%%%%%%%%%%%%%%%
We now show the episodes are not too long.
%
\begin{lemma} \label{lem:lmax}
    Let $\delta \in \spr{0, 1}$ and define $\LMAX = H \log \spr{K / \delta}$. Then, the event $\cE_L$ holds with probability at least $1 - \delta$.
\end{lemma}

\begin{proof}
    For any $k$ and by definition of the cumulative density function of the geometric distribution with parameter $1 - \gamma$, we have that $\bbP \sbr{L_k \leq \LMAX} = 1 - \gamma^\LMAX$. Therefore, $\bbP \sbr{L_k \leq \LMAX} \geq 1 - \delta / K$ for $\LMAX \geq \frac{\log \spr{\frac{\delta}{K}}}{\log \spr{1 / \gamma}}$. Lower bounding the denominator as $\log \spr{1 / \gamma} \geq  1 - \gamma$, we have that for $\LMAX = \frac{\log \spr{K / \delta}}{1 - \gamma}$ and a union bound over $k \in \sbr{K}$, we have that  $\bbP \sbr{\cE_L} \geq 1 - \delta$.
\end{proof}


%%%%%%%%%%%%%%%%%%%%%%%%%%%%%%%%%%%%%%%%%%%%%%%%%%%%%%%%%%%%%%%%%%%%%%%%%%%%%%%%
\begin{lemma} \label{lem:iterates-in-class}
    Assume the events $\cE_\cV$ and $\cE_L$ hold. Then, for all $k \in \sbr{K}$, it holds that $V_k \in \cV$, \ie $\cE_{\mathrm{in}}$ holds.
\end{lemma}

\begin{proof}
    The bound is proven by induction over $k \in [K]$. The base case holds by initialization since $Q_0 = \mathbf{0}$ is in $\cQ$. For the induction step, we assume that for all $\ell \in \sbr{k}$, $Q_\ell \in \mathcal{Q}, V_\ell \in \mathcal{V}$ and we show that $Q_{k+1} \in \cQ$ and $V_{k+1} \in \cV$.
    
    By definition of the function classes $\cQ$ and $\cV$ it holds that $\norm{Q_k}_{\infty}, \norm{V_k}_\infty \leq Q_{\max}$. $\cE_\cV$ together with the induction assumption imply that the bonuses are valid at time $k$, meaning that the derivations from Lemma~\ref{lem:qmax} guarantee that $\norm{Q_{k+1}}_\infty \leq \QMAX$. Moreover, denote $\theta_{k+1}$ the vector used to represent $Q_{k+1}$, defined as
    %
    \begin{align*}
        \theta_{k+1} = w_k + \gamma \wh{M V_k} = w_k + \gamma \Lambda_{T_k}^{-1} \sum_{\spr{x, a, x'} \in \cD_{T_k}} \phi \spr{x, a} V_k \spr{x'}\,.
    \end{align*}
    %
    It remains to show that $\theta_{k+1}$ satisfies the norm constraint defined in $\cQ$. By the triangular inequality and plugging the various assumptions, we have
    %
    \begin{align*}
        \norm{\theta_{k+1}} &\leq \norm{w_{k}} + \gamma \norm{(\Lambda_{T_k})^{-1}\sum_{(x,a,x')\in \mathcal{D}_{T_k}} \phi(x,a) V_k (x')} \\
        &\leq \WMAX + \gamma \lambda_{\max}((\Lambda_{T_k})^{-1}) \abs{\cD_{T_k}} \norm{V_k}_\infty \max_{x, a} \norm{\phi \spr{x, a}}_2 \\
        &\leq \WMAX + K L_{\max}Q_{\max} B\,,
    \end{align*}
    %
    where we also used $\gamma < 1$ in the last inequality. This proves that $Q_{k+1} \in \cQ$. Therefore, we have that $Q_\ell \in \cQ$ for $\ell \in \sbr{k+1}$. We now show that $V_{k+1} \in \cV$. Let $x \in \cX$ and $k_e$ be the initial index of the epoch $e$ such that $k \in \mathcal{K}_e$. By \citealp[Lemma~C.1]{MN23}, the sum of $V$ iterates is equal to a log-sum-exp function of the sum of $Q$ iterates. Thus,
    %
    \begin{align*}
        V_{k+1} \spr{x} &= \sum_{i=k_e}^{k+1} V_i \spr{x} - \sum_{j=k_e}^k V_j \spr{x} \\
        &= f_\eta \spr{\sum_{i=k_e}^{k+1} Q_i} \spr{x} - f_\eta \spr{\sum_{j=k_e}^k Q_j} \spr{x}\,.
    \end{align*}
    %
    Since $\mathbf{0} \in \cQ$ and $Q_\ell \in \cQ$ for $\ell \in \sbr{k+1}$, we can pad with zeros the two sums inside the exponentials and conclude that $V_{k+1} \spr{x}$ can be written as the difference between two log-sum-exp functions of the sum of $K$ functions in $\cQ$. Thus $V_{k+1} \in \cV$ and this concludes the induction.
\end{proof}


%%%%%%%%%%%%%%%%%%%%%%%%%%%%%%%%%%%%%%%%%%%%%%%%%%%%%%%%%%%%%%%%%%%%%%%%%%%%%%%%
\begin{lemma} \label{lem:uniform-event-holds}
    Assume the event $\cE_L$ holds, and set $\beta$ as
    %
    \begin{equation*}
        \beta = 8 Q_{\max} d \log(c \alpha \WMAX \RMAX B^{9/2} Q^4_{\max}L^{5/2}_{\max}K^{7/2} d^{5/2} \delta^{-1})\,.
    \end{equation*}
    %
    where $c = 60 \cdot 26$. Then, the event $\cE_\cV$ holds with probability $1 - \delta$.
\end{lemma}

\begin{proof}
    Under the event $\cE_L$, invoking standard concentration results for Linear MDPs (see Lemmas D.3 and D.4 in \cite{jin2019provably}), we have that with probability $1-\delta$ it holds that
    %
    \begin{align*}
        &\norm{MV - (\Lambda_{T_k})^{-1}\sum_{(x,a,x')\in \mathcal{D}_{T_k}} \phi(x,a) V(x')}_{\Lambda_{T_k}} \\ &~~~~~~~\leq Q_{\max} \sqrt{2 d \log\brr{\frac{1 + K L_{\max}B}{\delta}} + 4 \log \mathcal{N}_\epsilon + 8 K^2 L^2_{\max}B^2 \epsilon^2}\,,
    \end{align*}
    %
    where $\mathcal{N}_\epsilon$ is the $\epsilon $-covering number of the class $\mathcal{V}$. In particular, for $\epsilon = (K L_{\max} B)^{-1}$, we can invoke Lemma~\ref{lem:covering-number-v} to obtain
    %
    \begin{align*}
        \log \mathcal{N}_{\epsilon} &\leq 4d^2 \log \brr{4(\WMAX B + Q_{\max}L_{\max}K B^2 + 3\sqrt{d}+ \beta B +\RMAX)\sqrt{K^5L^3_{\max}}\alpha\beta B^{5/2}} \\
        &\leq 4d^2 \log \brr{20(3 \WMAX \beta Q_{\max}L_{\max}KB^2\sqrt{d}\RMAX)\sqrt{K^5L^3_{\max}}\alpha\beta B^{5/2}} \\
        &\leq 4d^2 \log \brr{60 \WMAX \RMAX \beta^2 \alpha B^{9/2} Q^2_{\max}L^{5/2}_{\max}K^{7/2}\sqrt{d}}\,.
    \end{align*}
    %
    Plugging in, we have that
    %
    \begin{align*}
        &\norm{MV - (\Lambda_{T_k})^{-1}\sum_{(x,a,x')\in \mathcal{D}_{T_k}} \phi(x,a) V(x')}_{\Lambda_{T_k}} \\ 
        &~~~~~~~\leq Q_{\max} \sqrt{2 d \log\brr{\frac{1 + K L_{\max} B}{\delta}} + 16 d^2 \log \brr{60 \beta^2 \alpha B^{9/2}Q^2_{\max}L^{5/2}_{\max}K^{7/2}\sqrt{d}} + 8} \\
        &~~~~~~~\leq Q_{\max} \sqrt{26 d^2 \log \brr{\frac{60 \WMAX \RMAX \beta^2 \alpha B^{9/2} Q^2_{\max}L^{5/2}_{\max}K^{7/2}\sqrt{d}}{\delta}}} \\ 
        &~~~~~~~ = \sqrt{26 Q^2_{\max} d^2 \log \brr{\frac{60 \WMAX \RMAX \beta^2 \alpha  B^{9/2} Q^2_{\max}L^{5/2}_{\max}K^{7/2}\sqrt{d}}{\delta}}} \\
    \end{align*}
    %
    At this point, to find a value for $\beta$ such that
    \begin{equation*}
        \beta^2 \geq 26 Q^2_{\max} d^2 \log \brr{\frac{60 \WMAX \RMAX \beta^2 \alpha B^{9/2} Q^2_{\max}L^{5/2}_{\max}K^{7/2}\sqrt{d}}{\delta}}\,,
    \end{equation*}
    we invoke Lemma~\ref{lemma:beta_bound} with $z = 26 Q^2_{\max} d^2$ and $R = \frac{60 \WMAX \RMAX \alpha B^{9/2}Q^2_{\max}L^{5/2}_{\max}K^{7/2}\sqrt{d}}{\delta}$ which gives that the desired inequality holds for all $\beta \in \mathbb{R}$ such that
    %
    \begin{equation*}
        \beta^2 \geq 52 Q^2_{\max} d^2 \log(c \alpha \WMAX \RMAX B^{9/2} Q^4_{\max}L^{5/2}_{\max}K^{7/2} d^{5/2} \delta^{-1})\,,
    \end{equation*}
    %
    where $c = 60 \cdot 26$. Therefore, we select
    \begin{equation*}
        \beta = 8 Q_{\max} d \log(c \alpha \WMAX \RMAX B^{9/2} Q^4_{\max}L^{5/2}_{\max}K^{7/2} d^{5/2} \delta^{-1})\,.
    \end{equation*}
\end{proof}

\begin{remark}
    For the proof of Lemma~\ref{lem:uniform-event-holds}, we need to compute a bound on the covering number of the function class $\cV$. We find this is done in a neat and more direct way than previous analysis \cite{zhong2024theoretical,sherman2023rate,cassel2024warmupfree} that needed to introduce a policy class for the iterates $\bc{\pi_k}^K_{k=1}$ generated by \Cref{alg:linear-rmax-ravi-ucb} as an intermediate step.
\end{remark}


%%%%%%%%%%%%%%%%%%%%%%%%%%%%%%%%%%%%%%%%%%%%%%%%%%%%%%%%%%%%%%%%%%%%%%%%%%%%%%%%
\subsubsection{Proof of Lemma~\ref{lem:covering-number-v} (covering number)}
\label{app:covering-number-v}

\begin{lemma} \label{lem:covering-number-v}
    Let us consider the function class $\cV$ defined in Equation~\eqref{eq:function-class-v} and an $\epsilon$-covering set $\cR \spr{\cV}$ such that for any $V \in \cV$, there exists $V' \in \cR \spr{\cV}$ such that $\norm{V - V'}_\infty \leq \frac{1}{K L_{\max} B}$. The covering number of the class $\cV$ can be bounded as follows
    %
    \begin{equation*}
        \log \cN_{\frac1K} \leq 4 d^2 \log \spr{4 \spr{\WMAX B + \QMAX \LMAX K B^2 + 3 \sqrt{d} + \beta B + \RMAX H} \sqrt{K^5 \LMAX^3} \alpha \beta B^{5/2}}\,.
    \end{equation*}
\end{lemma}

\begin{proof}
    We will use the following intermediate class of log sum exp state value functions
    %
    \begin{equation*}
        \tilde{\cV} = \scbr{V: \cX \rightarrow \bbR \quad\text{s.t.}\quad \forall x, V \spr{x} = \frac1\eta \log \sum_{a \in \cA}\exp \spr{\eta \sum_{\ell=1}^K Q_\ell \spr{x, a}}, Q_\ell \in \cQ}\,.
    \end{equation*}
    %
    Consider any $V, V' \in \cV$, and notice that for any $x \in \cX$,
    %
    \begin{equation*}
        \abs{V \spr{x} - V' \spr{x}} \leq \abs{\bar{V} \spr{x} - \bar{V}' \spr{x}} + \abs{\tilde{V} \spr{x} - \tilde{V}' \spr{x}}.
    \end{equation*}
    %
    with $\bar{V}, \tilde{V} \in \tilde{\cV}$ such that $V \spr{x} = \bar{V} \spr{x} - \tilde{V} \spr{x}$ for all $x \in \cX$ and with $\bar{V}', \tilde{V}' \in \tilde{\cV}$ such that $V' \spr{x} = \bar{V}' \spr{x} - \tilde{V}' \spr{x}$ for all $x \in \cX$. Therefore, the above bound guarantees that an $\epsilon / 2$ covering set on the function class $\tilde{\cV}$ implies an $\epsilon$ covering for the class $\cV$. Hence, in the following we focus on computing a $\epsilon / 2$ covering number for $\tilde{\cV}$. By definition of $\bar{V}, \bar{V}'$ and Lemma~\ref{lem:lse-lipschitz}, we have
    %
    \begin{align*}
        \abs{\bar{V} \spr{x} - \bar{V}' \spr{x}} &= \abs{\frac1\eta \log \sum_{a \in \cA} \exp \brr{\eta \sum^K_{\ell=1} \bar{Q}_\ell \spr{x, a}} - \frac1\eta \log \sum_{a \in \cA} \exp \spr{\eta \sum^K_{\ell=1} \bar{Q}_\ell' \spr{x, a}}} \\
        &\leq \max_{a \in \A} \abs{ \sum^K_{\ell=1} \bar{Q}_\ell \spr{x, a} -  \sum^K_{\ell=1} \bar{Q}'_\ell \spr{x, a}} \\
        &\leq K \max_{\ell \in [K]} \norm{ \bar{Q}_\ell - \bar{Q}'_\ell }_{\infty}\,.
    \end{align*}
    %
    For any $\ell \in \sbr{K}$, we denote $\Lambda_\ell$, $\theta_\ell$ the parameters of the function $\bar{Q}_\ell$ and $\Lambda'_\ell$, $\theta'_\ell$ the parameters of the function $\bar{Q}'_\ell$. We now prove that $\bar{Q}_\ell$,$\bar{Q}'_\ell$ are Lipschitz functions. Let us denote $Q_{\theta, \Lambda}$ and $Q_{\theta', \Lambda'}$ two functions in $\cQ$ for different parameters $\theta, \Lambda, \theta', \Lambda'$. For any state-action pair $\spr{x, a}$, the difference between the two functions can be written as
    %
    \begin{align*}
        &Q_{\theta, \Lambda} \spr{x, a} - Q_{\theta', \Lambda'} \spr{x, a} = \spr{\phi \spr{x, a}\transpose \theta + \beta \norm{\phi \spr{x, a}}_\Lambda - \RMAX H} \cdot \sigma \spr{- \alpha \beta \norm{\phi \spr{x, a}}_\Lambda + \omega} \\
        &\phantom{=}\quad\quad\quad\quad\quad- \spr{\phi \spr{x, a}\transpose \theta' + \beta \norm{\phi \spr{x, a}}_{\Lambda'} - \RMAX H} \cdot \sigma \spr{- \alpha \beta \norm{\phi \spr{x, a}}_{\Lambda'} + \omega}\,.
    \end{align*}
    %
    Next, our goal is to show that the function
    %
    \begin{equation*}
        f \spr{\theta, \Lambda; x, a} := \spr{\phi(x,a)\transpose \theta + \beta \norm{\phi (x, a)}_{\Lambda} - \RMAX H} \cdot \sigma(- \alpha\beta \norm{\phi(x,a)}_{\Lambda} + \omega)
    \end{equation*}
    %
    is Lipshitz in both parameters $\theta, \Lambda$. The Lipshitzness with respect to $\beta$ does not need to be established since it is kept fixed throughout the learning process. We show this showing that the gradients are bounded. In particular,
    %
    \begin{equation*}
        \norm{\nabla_\theta f(\theta,\Lambda;x,a)} = \norm{\phi(x,a)} \cdot \sigma(- \alpha\beta \norm{\phi(x,a)}_{\Lambda} + \omega) \leq \norm{\phi(x,a)}  \leq B\,.
    \end{equation*}
    %
    For the Lipshitzness with respect to $\Lambda$, we have that
    %
    \begin{align*}
        &f(\theta,\Lambda;x,a) - f(\theta,\Lambda';x,a) \\
        &= \spr{\phi(x,a)\transpose \theta + \beta \norm{\phi(x,a)}_{\Lambda} 
        - \RMAX H}\cdot \sigma(- \alpha\beta \norm{\phi(x,a)}_{\Lambda} + \omega) \\
        &\phantom{=}- \spr{\phi(x,a)\transpose \theta + \beta \norm{\phi(x,a)}_{\Lambda'} 
        - \RMAX H}\cdot \sigma(- \alpha\beta \norm{\phi(x,a)}_{\Lambda'} + \omega) \\
        &= \spr{\phi(x,a)\transpose \theta + \beta \norm{\phi(x,a)}_{\Lambda} 
        - \RMAX H}\cdot \Bigl(\sigma(- \alpha\beta \norm{\phi(x,a)}_{\Lambda} + \omega) \\
        &\phantom{=}- \sigma(- \alpha\beta \norm{\phi(x,a)}_{\Lambda'} + \omega) \Bigr) + \sigma(- \alpha\beta \norm{\phi(x,a)}_{\Lambda'} + \omega) \brr{\beta \norm{\phi(x,a)}_{\Lambda} - \beta \norm{\phi(x,a)}_{\Lambda'}}
    \end{align*}
    %
    Then, using the fact that $\sigma$ is $1$-Lipshitz, we have that
    %
    \begin{align*}
        &\abs{f(\theta,\Lambda;x,a) - f(\theta,\Lambda';x,a)} \\
        &\quad \leq \alpha \beta \abs{\phi(x,a)\transpose \theta + \beta \norm{\phi(x,a)}_{\Lambda} - \RMAX H} \cdot \abs{\norm{\phi(x,a)}_{\Lambda} - \norm{\phi(x,a)}_{\Lambda'}} \\
        &\quad\quad + \sigma(- \alpha\beta \norm{\phi(x,a)}_{\Lambda'} + \omega) \abs{\beta \norm{\phi(x,a)}_{\Lambda} - \beta \norm{\phi(x,a)}_{\Lambda'}} \\
        &\leq \alpha\beta \abs{\phi(x,a)\transpose \theta + \beta \norm{\phi(x,a)}_{\Lambda} 
        - \RMAX H} \cdot \abs{\norm{\phi(x,a)}_{\Lambda} 
        - \norm{\phi(x,a)}_{\Lambda'}}
        \\&\phantom{=}+  \beta \abs{\norm{\phi(x,a)}_{\Lambda} - \norm{\phi(x,a)}_{\Lambda'}} \\
        &\leq \alpha\beta (\norm{\theta}B + \beta B + \RMAX H + 1) 
        \abs{\norm{\phi(x,a)}_{\Lambda} - \norm{\phi(x,a)}_{\Lambda'}}.
    \end{align*}
    %
    where we used the fact that $\sigma(x) \leq 1$, for all $x\in \mathbb{R}$ and $\alpha \geq 1$ in the last inequality. 
    Using that $\norm{\phi (x,a)}_{\Lambda} = \norm{\Lambda^{1/2} \phi(x,a)}$ and the triangular inequality we have that
    %
    \begin{align*}
        f(\theta,\Lambda;x,a) - f(\theta,\Lambda';x,a) &\leq 
        \alpha\beta (\norm{\theta}B + \beta B +\RMAX H + 1) 
        \abs{\norm{\Lambda^{1/2}\phi(x,a)} - \norm{(\Lambda')^{1/2}\phi(x,a)}} \\
        &\leq \alpha\beta (\norm{\theta}B + \beta B +\RMAX H + 1) \norm{(\Lambda^{1/2} - (\Lambda')^{1/2})\phi(x,a)} \\
        &\leq \alpha \beta B (\norm{\theta} B + \beta B +\RMAX H + 1) \norm{\Lambda^{1/2} - (\Lambda')^{1/2}} 
    \end{align*}
    %
    where the last inequality holds for $\norm{\phi (x,a)} \leq B$. Finally, using the definition of the class $\mathcal{Q}$, we have that the matrices $\Lambda$ and $\Lambda'$ are positive definite in particular $\lambda_{\min}(\Lambda) \geq \frac{1}{2KBL_{\max}}$ and $\lambda_{\min}(\Lambda') \geq \frac{1}{2KBL_{\max}}$. 
    Therefore by \cite[Lemma 17]{cassel2024warmupfree}, 
    it holds that $\norm{\Lambda^{1/2} - (\Lambda')^{1/2}} \leq \frac{1}{2\sqrt{\lambda_{\min}}}\norm{\Lambda - \Lambda'} = \sqrt{\frac{B K L_{\max}}{2}} \norm{\Lambda - \Lambda'}$. 
    Therefore, all in all we have that
    %
    \begin{align*}
        f(\theta,\Lambda;x,a) - f(\theta,\Lambda';x,a)
        &\leq \sqrt{K L_{\max}}\alpha\beta B^{3/2} (\norm{\theta} B
        + \beta B +\RMAX H + 1) \norm{\Lambda - \Lambda'} \\
        &\leq \sqrt{K L_{\max}}\alpha\beta B^{3/2} (\norm{\theta} B
        + \beta B +\RMAX H + 1) \norm{\Lambda - \Lambda'}_F,
    \end{align*}
    %
    where $\norm{\cdot}_F$ denote the Frobenious norm of a matrix. Hence, we have that
    %
    \begin{align*}
        Q_{\theta,\Lambda}(x,a) - Q_{\theta',\Lambda'}(x,a) \leq 
        \sqrt{K L_{\max}}\alpha\beta B^{3/2} (\norm{\theta} B + \beta B +\RMAX H + 1)
         \norm{\Lambda - \Lambda'}_F + B \norm{\theta - \theta'}.
    \end{align*}
    %
    At this point, if we have a $\epsilon_{\Lambda}$-covering set for the set
    %
    \begin{equation*}
        \mathbf{\Lambda} = \bc{\Lambda \in \mathbb{R}^{d\times d} : \lambda_{\max}(\Lambda) \leq 1, ~~~\lambda_{\min}(\Lambda) \geq \frac{1}{2BKL_{\max}}}
    \end{equation*}
    %
    and an $\epsilon_\theta$-covering set for the set
    %
    \begin{equation*}
        \mathbf{\Theta} = \bc{\theta \in \mathbb{R}^d: \norm{\theta} \leq \WMAX + Q_{\max}L_{\max}K B}
    \end{equation*}
    %
    we would have that
    %
    \begin{align*}
        \abs{V(x) - V'(x)} &\leq 2\sqrt{ K^3 L_{\max}}\alpha\beta B^{3/2} (\norm{\theta} B + \beta B +\RMAX H + 1) \epsilon_F + 2 B K \epsilon_\theta \\
        &\leq 2 \sqrt{K^3L_{\max}}\alpha \beta B^{3/2} (\norm{\theta} B + \beta B +\RMAX H + 1) \brr{\epsilon_F + \epsilon_\theta},
    \end{align*}
    %
    where in the last inequality we assumed that $\beta \geq 1$ and $B\geq 1$.
    Therefore, to have an $\epsilon$-covering set for $\mathcal{V}$, we need to construct an $\epsilon_{\Lambda}$-covering set for $\mathbf{\Lambda}$, where
    %
    \begin{equation*}
        \epsilon_\Lambda = \frac{\epsilon}{4 \sqrt{K^3L_{\max}}\alpha\beta B^{3/2} (\norm{\theta} B + \beta B +\RMAX H + 1)}
    \end{equation*}
    %
    and an $\epsilon_\theta = \frac{\epsilon}{4 \sqrt{K^3L_{\max}}\alpha\beta B^{3/2} (\norm{\theta} B + \beta B +\RMAX H + 1)}$-covering set for $\mathbf{\Theta}$. Then, using the fact that the $\epsilon$-covering number for the Euclidean ball of radius $R$ in $d$ dimension is given by $(1 + 2 R/\epsilon)^d$, we obtain
    %
    \begin{equation*}
        \log \mathcal{N}_{\epsilon_{\theta}}(\mathbf{\Theta}) \leq d \log \brr{1 + 8\frac{(\WMAX + Q_{\max}L_{\max}K B )\sqrt{K^3L_{\max}}\alpha\beta B^{3/2} (\norm{\theta} B + \beta B +\RMAX H + 1)}{\epsilon}}
    \end{equation*}
    %
    Moreover, noticing that for all matrices $\Lambda \in \mathbf{\Lambda}$ it holds that $\norm{\Lambda}_F \leq \sqrt{d} \lambda_{\max}(\Lambda) \leq \sqrt{d}$, we need to cover the Frobenious norm ball with radius $\sqrt{d}$. Recalling that the Frobenious norm of a matrix is equivalent to the euclidean norm of the vectorization of the matrix, this equivalent to cover the euclidean ball in $\mathbb{R}^{d^2}$ with radius $\sqrt{d}$.
    %
    \begin{equation*}
        \log \mathcal{N}_{\epsilon_{\Lambda}}(\mathbf{\Lambda}) 
        \leq d^2 \log \brr{1 + 8\sqrt{d}\frac{\sqrt{K^3L_{\max}}\alpha\beta B^{3/2} (\norm{\theta} B + \beta B +\RMAX H + 1)}{\epsilon}}.
    \end{equation*}
    %
    Therefore, using the fact that
    %
    \begin{align*}
        &\log \mathcal{N}_{\epsilon}(\mathcal{V})=  
        \log \mathcal{N}_{\epsilon_{\Lambda}}(\mathbf{\Lambda}) + 
        \log \mathcal{N}_{\epsilon_{\theta}}(\mathbf{\Theta}) \\
        \\ & \leq d \log \brr{1 + 8\frac{(\WMAX + Q_{\max}L_{\max}K B )\sqrt{K^3L_{\max}}\alpha\beta B^{3/2} 
        (\norm{\theta} B + \beta B +\RMAX H + 1)}{\epsilon}} \\&\phantom{=}+ 
        d^2 \log \brr{1 + 8\sqrt{d}\frac{\sqrt{K^3L_{\max}}\alpha\beta B^{3/2} (\norm{\theta} B + \beta B +\RMAX H + 1)}{\epsilon}} \\
        \\ & \leq 2d^2\log  \brr{1 + 8\frac{(\WMAX + Q_{\max}L_{\max}K B + \sqrt{d})\sqrt{K^3L_{\max}}\alpha\beta B^{3/2} (\norm{\theta} B + 
        \beta B +\RMAX H + 1)}{\epsilon}} \\
        & \leq 2d^2 \log \brr{1 + 8\frac{(\WMAX B + Q_{\max}L_{\max}K B^2 + \sqrt{d}+ \beta B +\RMAX H +1)^2\sqrt{K^3L_{\max}}\alpha\beta B^{3/2}}{\epsilon}} \\
        & \leq 2d^2 \log \brr{16\frac{(\WMAX B + Q_{\max}L_{\max}K B^2 + 3\sqrt{d}+ \beta B +\RMAX H)^2\sqrt{K^3L_{\max}}\alpha\beta B^{3/2}}{\epsilon}} \\
        & \leq 4d^2 \log \brr{4\frac{(\WMAX B + Q_{\max}L_{\max}K B^2  + 3\sqrt{d}+ \beta B +\RMAX H)\sqrt{K^3L_{\max}}\alpha\beta B^{3/2}}{\epsilon}} \\
        & = 4d^2 \log \brr{4(\WMAX B + Q_{\max}L_{\max}K B^2 + 3\sqrt{d}+ \beta B +\RMAX H)\sqrt{K^5L^3_{\max}}\alpha\beta B^{5/2}}.
    \end{align*}
    %
    where we used $d > 1$ and the last step uses the fact that we are looking for a $\epsilon = \frac{1}{K L_{\max} B}$ covering set. Finally,
    %
    \begin{equation*}
        \log \mathcal{N}_{\epsilon} \leq 4d^2 \log \brr{4(\WMAX B +  Q_{\max}L_{\max}K B^2 + 3\sqrt{d}+ \beta B +\RMAX H)\sqrt{K^5L^3_{\max}}\alpha\beta B^{5/2}}\,.
    \end{equation*}
\end{proof}


%%%%%%%%%%%%%%%%%%%%%%%%%%%%%%%%%%%%%%%%%%%%%%%%%%%%%%%%%%%%%%%%%%%%%%%%%%%%%%%%
%%%%%%%%%%%%%%%%%%%%%%%%%%%%%%%%%%%%%%%%%%%%%%%%%%%%%%%%%%%%%%%%%%%%%%%%%%%%%%%%
\subsection{Putting everything together (proof of Theorem~\ref{thm:main})}
\label{app:putting-together-main}

\begin{theorem} \label{thm:main-full}
    Run Algorithm~\ref{alg:linear-rmax-ravi-ucb} with parameters $\omega = \log K$, $\alpha = 2 \log K$,
    \begin{align*}
        \eta = \sqrt{\frac{5 d \log \spr{1 + B^2 T / d} \log \abs{\cA}}{8 \RMAX^2 H^{5 / 2} K}}, \quad\text{and } \beta = C H \RMAX d \log \spr{B H \WMAX \RMAX d K \delta^{-1}}\,,
    \end{align*}
    %
    for some absolute constant $C > 0$ and $\delta \in \spr{0, 1}$. Then, with probability at least $1 - \delta$, we have
    %
    \begin{align*}
        \regretK &= \tilde\cO \spr{\sqrt{d^3 H^3 K} + \sqrt{d H^{9/2} K \log \spr{\abs{\cA}}}} \\
        &= \tilde\cO \spr{\sqrt{d^3 H^2 T} + \sqrt{d H^{7/2} T \log \spr{\abs{\cA}}}}\,.
    \end{align*}
\end{theorem}

\begin{proof}
    We are now ready to prove Theorem~\ref{thm:main}. Combining Lemma~\ref{lem:reward-bias-bound}, and the bounds in Equations~\eqref{eq:model-bias-bounds} and Lemma~\ref{lem:bound-regret-plus} we first get
    %
    \begin{align*}
        \frac1H \regretK &\leq 2 \RMAX H \sumkK \inp{\mu \spr{\pi_k}, p_k^\upplus} + 4 \QMAX E \spr{K} + \frac{E \spr{K} \log \abs{\cA}}{\eta} \\
        &\phantom{=}+ 2 \eta \QMAX^2 \sqrt{H} K + 2 \sumkK \inp{\mu \spr{\pi_k}, \spr{1 - p_k^\upplus} \odot \CB_k}\,.
    \end{align*}
    %
    Using the bound on the ascension functions provided in Inequality~\ref{eq:ascension-function-bound} and $1 - p_k^\upplus \preceq 1$, we further have
    %
    \begin{align*}
        \frac1H \regretK &\leq 4 \RMAX H \alpha^2 \sumkK \inp{\mu \spr{\pi_k}, \CB_k^2} + 4 \RMAX e^{- \omega} H K + 4 \QMAX E \spr{K} \\
        &\phantom{=}+ \frac{E \spr{K} \log \abs{\cA}}{\eta} + 2 \eta \QMAX^2 \sqrt{H} K + 2 \sumkK \inp{\mu \spr{\pi_k}, \CB_k}\,.
    \end{align*}
    %
    Lemma~\ref{lem:expected-bonuses-bound} can be used to bound the bonuses
    %
    \begin{align*}
        \frac1H \regretK &\leq 32 \RMAX \alpha^2 \beta^2 B^2 d \log \spr{1 + \frac{B^2 T}{d}} + 16 \RMAX H \alpha^2 \beta^2 B^2 \log \spr{\frac{2 K}{\delta}}^2 \\
        &\phantom{=}+ 4 \RMAX e^{- \omega} H K + 4 \QMAX E \spr{K} + \frac{E \spr{K} \log \abs{\cA}}{\eta} + 2 \eta \QMAX^2 \sqrt{H} K \\
        &\phantom{=}+ \frac{8 \beta B}{H} \sqrt{d T \log \spr{1 + \frac{B^2 T}{d}}} + 8 \beta B \log \spr{\frac{2 K}{\delta}}^2\,.
    \end{align*}
    %
    Following Lemmas~\ref{lem:qmax} and \ref{lem:lmax}, we plug the values of $\QMAX = H \spr{\RMAX + \frac{2 \omega}{\alpha}}$ and $\LMAX = H \log \spr{K / \delta}$,
    %
    \begin{align*}
        \frac1H \regretK &\leq 32 \RMAX \alpha^2 \beta^2 B^2 d \log \spr{1 + \frac{B^2 T}{d}} + 16 \RMAX H \alpha^2 \beta^2 B^2 \log \spr{\frac{2 K}{\delta}}^2 \\
        &\phantom{=}+ 4 \RMAX e^{- \omega} H K + 4 H \spr{\RMAX + \frac{2 \omega}{\alpha}} E \spr{K} + \frac{E \spr{K} \log \abs{\cA}}{\eta} \\
        &\phantom{=}+ 2 \eta \spr{\RMAX + \frac{2 \omega}{\alpha}}^2 H^{5/2} K + \frac{8 \beta B}{H} \sqrt{d T \log \spr{1 + \frac{B^2 T}{d}}} \\
        &\phantom{=}+ 8 \beta B \log \spr{\frac{2 K}{\delta}}^2\,.
    \end{align*}
    %
    By Lemma~\ref{lemma:number-epochs-bound}, we can bound $E \spr{K} \leq 5 d \log \spr{1 + \frac{B^2 T}{d}}$
    %
    \begin{align*}
        \frac1H \regretK &\leq 32 \RMAX \alpha^2 \beta^2 B^2 d \log \spr{1 + \frac{B^2 T}{d}} + 16 \RMAX \alpha^2 \beta^2 B^2 H \log \spr{\frac{2 K}{\delta}}^2 \\
        &\phantom{=}+ 4 \RMAX e^{- \omega} H K + 20 H d \spr{\RMAX + \frac{2 \omega}{\alpha}} \log \spr{1 + \frac{B^2 T}{d}} \\
        &\phantom{=}+ \frac{5 d}{\eta} \log \spr{1 + \frac{B^2 T}{d}} \log \abs{\cA} + 2 \eta \spr{\RMAX + \frac{2 \omega}{\alpha}}^2 H^{5/2} K \\
        &\phantom{=}+ \frac{8 \beta B}{H} \sqrt{d T \log \spr{1 + \frac{B^2 T}{d}}} + 8 \beta B \log \spr{\frac{2 K}{\delta}}^2\,.
    \end{align*}
    %
    It remains to choose the parameters. We start by setting $\alpha = 2 \omega$ and use $\RMAX \geq 1$ to get
    %
    \begin{align*}
        \frac1H \regretK &\leq 128 \RMAX \omega^2 \beta^2 B^2 d \log \spr{1 + \frac{B^2 T}{d}} + 64 \RMAX \omega^2 \beta^2 B^2 H \log \spr{\frac{2 K}{\delta}}^2 \\
        &\phantom{=}+ 4 \RMAX e^{- \omega} H K + 40 H d \RMAX \log \spr{1 + \frac{B^2 T}{d}} \\
        &\phantom{=}+ \frac{5 d}{\eta} \log \spr{1 + \frac{B^2 T}{d}} \log \abs{\cA} + 8 \eta \RMAX^2 H^{5/2} K \\
        &\phantom{=}+ \frac{8 \beta B}{H} \sqrt{d T \log \spr{1 + \frac{B^2 T}{d}}} + 8 \beta B \log \spr{\frac{2 K}{\delta}}^2\,.
    \end{align*}
    %
    Then, we set $\omega = \log K$
    %
    \begin{align*}
        \frac1H \regretK &\leq 128 \RMAX \beta^2 B^2 d \log \spr{K}^2 \log \spr{1 + \frac{B^2 T}{d}} + 64 \RMAX \beta^2 B^2 H \log \spr{K}^2 \log \spr{\frac{2 K}{\delta}}^2 \\
        &\phantom{=}+ 4 \RMAX H + 40 H d \RMAX \log \spr{1 + \frac{B^2 T}{d}} \\
        &\phantom{=}+ \frac{5 d}{\eta} \log \spr{1 + \frac{B^2 T}{d}} \log \abs{\cA} + 8 \eta \RMAX^2 H^{5/2} K \\
        &\phantom{=}+ \frac{8 \beta B}{H} \sqrt{d T \log \spr{1 + \frac{B^2 T}{d}}} + 8 \beta B \log \spr{\frac{2 K}{\delta}}^2\,.
    \end{align*}
    %
    We choose the learning rate as $\eta = \sqrt{\frac{5 d \log \spr{1 + B^2 T / d} \log \abs{\cA}}{8 \RMAX^2 H^{5 / 2} K}}$ and we obtain
    %
    \begin{align*}
        \frac1H \regretK &\leq 128 \RMAX \beta^2 B^2 d \log \spr{K}^2 \log \spr{1 + \frac{B^2 T}{d}} + 64 \RMAX \beta^2 B^2 H \log \spr{K}^2 \log \spr{\frac{2 K}{\delta}}^2 \\
        &\phantom{=}+ 4 \RMAX H + 40 H d \RMAX \log \spr{1 + \frac{B^2 T}{d}} \\
        &\phantom{=}+ 4 \sqrt{10 \RMAX^2 H^{5/2} d \log \spr{1 + \frac{B^2 T}{d}} \log \spr{\abs{\cA}} K} \\
        &\phantom{=}+ \frac{8 \beta B}{H} \sqrt{d T \log \spr{1 + \frac{B^2 T}{d}}} + 8 \beta B \log \spr{\frac{2 K}{\delta}}^2\,.
    \end{align*}
    %
    Finally, following Lemma~\ref{lem:good-event-holds} we set $\beta = C H \RMAX d \log \spr{B H \WMAX \RMAX d K \delta^{-1}}$ where $C > 0$ is an absolute constant and we get
    %
    \begin{align*}
        \frac1H \regretK &\leq 128 C^2 \RMAX^3 B^2 d^3 H^2 \log \spr{K}^2 \log \spr{1 + \frac{B^2 T}{d}} \log \spr{B H \WMAX \RMAX d K \delta^{-1}}^2 \\
        &\phantom{=}+ 64 C^2 \RMAX^3 d^2 B^2 H^3 \log \spr{K}^2 \log \spr{\frac{2 K}{\delta}}^2 \log \spr{B H \WMAX \RMAX d K \delta^{-1}}^2 \\
        &\phantom{=}+ 4 \RMAX H + 40 H d \RMAX \log \spr{1 + \frac{B^2 T}{d}} \\
        &\phantom{=}+ 4 \sqrt{10 \RMAX^2 H^{5/2} d \log \spr{1 + \frac{B^2 T}{d}} \log \spr{\abs{\cA}} K} \\
        &\phantom{=}+ 8 C \RMAX B d \sqrt{d T \log \spr{1 + \frac{B^2 T}{d}}} \log \spr{B H \WMAX \RMAX d K \delta^{-1}} \\
        &\phantom{=}+ 8 C \RMAX d B H^2 \log \spr{\frac{2 K}{\delta}}^2 \log \spr{B H \WMAX \RMAX d K \delta^{-1}}\,.
    \end{align*}
    %
    After multiplying by $H$, we get
    %
    \begin{align*}
        \regretK &= \tilde\cO \spr{\sqrt{d^3 H^3 K} + \sqrt{d H^{9/2} K \log \spr{\abs{\cA}}}} \\
        &= \tilde\cO \spr{\sqrt{d^3 H^2 T} + \sqrt{d H^{7/2} T \log \spr{\abs{\cA}}}}\,.
    \end{align*}
\end{proof}


\clearpage
\section{Motivation for \emph{Learning from Features Alone} and related works in imitation learning}
\label{app:related_works_IL}



\paragraph{Related works in theoretical imitation learning.} A special case of our setting is imitation learning from state-only expert trajectories, which is recovered when $\phi_\cost (x,a) = \mathbf{e}_x$. 
This setting was first studied in \cite{sun2019provably} in the finite-horizon setting with general function approximation. 
There are some notable differences between their work and ours, primarily that they focus on the finite-horizon setting and learn a non-stationary policy. 
In principle, their algorithm could be applied to the infinite-horizon setting by truncating the trajectories after $\tilde{\mathcal{O}}(1-\gamma)^{-1}$ steps. 
However, this would still result in a non-stationary policy, whereas our approach outputs a stationary policy. 
Their realizability assumption on the expert policy and expert state-value function is not required in our work which leverages, instead, the linear MDP assumption. 
These assumptions are not directly comparable, even when the function classes in \cite{sun2019provably} are assumed to be linear. 
Indeed, the realizability assumption imposed in \cite{sun2019provably} would imply having access to the values of the features $\sum_a \expert(a|x) \phi(x,a)$ for each state $x \in \cX$. 
In contrast, our approach does not require this additional knowledge about the expert.

Furthermore, the guarantees on the number of expert trajectories in \cite[Theorem 3.3]{sun2019provably} adapted to the infinite-horizon setting, 
would scale as $\tilde{\mathcal{O}}((1-\gamma)^{-4} \epsilon^{-2})$ whereas we only require $\tilde{\mathcal{O}}((1-\gamma)^{-2} \epsilon^{-2})$ state-only samples from the expert occupancy measure.

Similarly, \cite{ADKLS20} develop a framework for imitation and representation learning from observation alone based on bilevel optimization but assume the realizability of the state-value function, which is not needed in our work.

The work of \citet{kidambi2021mobile} investigates the idea of exploration in state-only imitation learning. 
Unlike our work, they focus on the finite-horizon setting and on different structural assumptions on the MDP. 
Specifically, \citet{kidambi2021mobile} consider tabular MDPs, nonlinear kernel regulators, and MDPs with Gaussian transition kernels and bounded Eluder dimension, whereas our work focuses on infinite-horizon linear MDPs and observing only the feature directions visited by the expert, which is a weaker requirement than observing the states directly.
Moreover, our algorithm \FRAalg is computationally efficient, whereas the model fitting step in \cite{kidambi2021mobile} cannot be implemented efficiently for various situations, including linear MDPs \citep{jin2019provably} and KNRs \citep{Kakade:2020}.

\citet{wu2024diffusing} operate under a different set of assumptions, namely that the learner has access to a function class for the expert's score function and that the expected state norm remains bounded during learning. 
Under this setting, the authors are the first to achieve first- and second-order bounds for imitation learning, which lead to a faster rate in the case of low-variance expert policies and transitions. The authors do not quantify the MDP trajectory complexity, but it would scale suboptimally with $1/\epsilon$ because they require an expensive \emph{RL in the loop} routine that we avoid in our work.

\citet{xu2022understanding} develop an analysis for horizon-free bounds on $\tau_E$ for a special class of MDPs, where expert states can be visited only by visiting all preceding expert states.

The trajectory access to the MDP $\mathcal{M}\setminus \true$ assumed in this work should not be confused with interactive/online imitation learning, where the expert can be queried during learning \citep{Ross:2010,Ross:2011,swamy2021moments,li2022efficient,lavington2022improved,sekhari2024selective,sun2017deeply,sekhari2024contextual}. Furthermore, our trajectory access is a much weaker requirement compared to generative model access used in \citep{swamy2022minimax,Kamoutsi:2021}.

Moreover, it is important to note that we do not require any ergodicity or self-exploration properties of the dynamics, whereas such assumptions are needed in \citep{viano2022proximal,zeng2022maximum}. Additionally, uniformly good evaluation error, which is essentially possible only under generative model or ergodic dynamics assumptions, is required in \citep{wu2023inverse,zeng2022structural,zeng2023understanding}. 
Also, the use of exploration bonuses in imitation learning has also been useful for the related problem of finding the reward feasible set without using a generative model \citep{lazzati2024scale,lindner2022active}.

\begin{table}[t]
    \caption{\label{tab:literature} Comparison with related imitation learning algorithms.}
    \setlength{\tabcolsep}{4pt}  % Reduce horizontal padding
    \renewcommand{\arraystretch}{1.5}  % Increase vertical spacing
    \resizebox{\textwidth}{!}{%
    \begin{tabular}{|c|M{4cm}|M{1.5cm}|c|c|}
        \hline
        \textbf{Algorithm}                                         & \textbf{Setting}                                                & \textbf{F.O.}            & \textbf{Expert Traj. $(\tau_E)$}                                       & \textbf{MDP Traj. $(K)$}                                                              \\ \hline
        \multirow{3}{*}{Behavioural Cloning}                       & Function Approximation, Episodic \cite{foster2024behavior}      & \redcross                & $\cO \brr{\frac{H^2 \log \abs{\Pi}}{\epsilon^2}}$                      & \multicolumn{1}{c|}{-}                                                                \\ \cline{2-4}
                                                                   & Tabular, Episodic \cite{rajaraman2020toward}                    & \redcross                & $\widetilde{\cO} \brr{\frac{H^2 \abs{\cX}}{\epsilon}}$                 & \multicolumn{1}{c|}{-}                                                                \\ \cline{2-4}
                                                                   & Deterministic Linear Expert, Episodic \cite{rajaraman2021value} & \redcross                & $\widetilde{\cO} \brr{\frac{H^2 d}{\epsilon}}$                         & \multicolumn{1}{c|}{-}                                                                \\ \hline
        Mimic-MD \cite{rajaraman2020toward}                        & Tabular, Known $P$, Deterministic Expert, Episodic              & \redcross                & $\cO \brr{\frac{H^{3/2} \abs{\cX}}{\epsilon}}$                         & \multicolumn{1}{c|}{-}                                                                \\ \hline
        OAL \cite{Shani:2021}                                      & Episodic Tabular                                                & \redcross                & $\tilde{\cO} \brr{\frac{H^2 \abs{\cX}}{\epsilon^{2}}}$                 & $\tilde{\mathcal{O}}\brr{\frac{H^4 \abs{\X}^2 \abs{\aspace} }{\epsilon^{2}}}$         \\ \hline
        MB-TAIL \cite{xu2023provably}                              & Episodic, Tabular, Deterministic Expert                         & \redcross                & $\cO \brr{\frac{H^{3/2} \abs{\cX}}{\epsilon}}$                         & $\mathcal{O}\brr{\frac{H^3 \abs{\X}^2 \abs{\aspace} }{\epsilon^{2}}}$                 \\ \hline
        FAIL \cite{sun2019provably}                                & Episodic, $\expert \in \Pi$ and $V^\expert\in\mathcal{F}$       & \greentick$^\star$       & $\tilde{\cO} \brr{\frac{H^4 \log(\abs{\Pi} \abs{\cF} H)}{\epsilon^2}}$ & $\tilde{\mathcal{O}}\brr{\frac{H^4 \log(\abs{\Pi}\abs{\mathcal{F}} H) }{\epsilon^2}}$ \\ \hline
        Mobile \cite{kidambi2021mobile}                            & Episodic, $\true \in \mathcal{R}$ and $P \in \mathcal{P}$       & \greentick$^\star$       & $\tilde{\cO} \brr{\frac{H^2 \log (\abs{\mathcal{R}} H)}{\epsilon^2}}$  & $\tilde{\mathcal{O}}\brr{\frac{H^5 \log \abs{\mathcal{P}}}{\epsilon^2}}$              \\ \hline
        OGAIL \cite{Liu:2022}                                      & Episodic Linear Mixture MDP, $\WMAX=\sqrt{d}$                   & \greentick               & $\tilde{\cO} \brr{\frac{H^{3} d^2}{\epsilon^{2}}}$                     & $\tilde{\mathcal{O}}\brr{\frac{H^4 d^3}{\epsilon^{2}}}$                               \\ \hline
        ILARL \cite{viano2024imitation}                            & Linear MDP, $\WMAX=1$                                           & \greentick               & $\tilde{\cO} \brr{\frac{d}{(1 - \gamma)^{2} \epsilon^{2}}}$            & $\tilde{\mathcal{O}}\brr{ \frac{d^3 }{(1 - \gamma)^{8} \epsilon^{4}}}$                \\ \hline
        \FRAalg (This Work)                                        & Linear MDP                                                      & \greentick               & $\tilde{\cO} \brr{\frac{\WMAX^2}{(1 - \gamma)^{2} \epsilon^{2}}}$      & $\tilde{\mathcal{O}}\brr{ \frac{d^3 }{(1 - \gamma)^{4.5} \epsilon^{2}}}$              \\ \hline
        \textbf{Lower Bound} (This Work)                           & Linear MDP                                                      & \greentick               & $\Omega\brr{ \frac{\WMAX^2}{(1 - \gamma)^{2} \epsilon^{2}}}$           & $\Omega\brr{ \frac{d }{(1 - \gamma)^{2} \epsilon^{2}}}$                               \\ \hline
    \end{tabular}}
\end{table}

Finally, we present Table~\ref{tab:literature}, which compares our bounds with existing ones.
 We show the number of expert trajectories and MDP interactions required for $\epsilon$-suboptimal expected performance. 
 The acronym F.O.\ refers to "Features Only" and indicates whether the algorithm applies to the setting we consider here. 
 The star \greentick$^\star$ specifies that the algorithm only applies to state-only imitation learning. "Linear expert" refers to the case where an expert policy is of the form
%
\begin{align*}
    \pi \spr{x} = \max_{a \in \aspace} \phi \spr{x, a}\transpose \theta\,,
\end{align*}
%
for some vector $\theta$. Finally, in the work by \cite{kidambi2021mobile}, the bound on $K$ can be tighter than what we report in the table. We report this slightly looser version for sake of simplicity and avoiding to introduce the information gain (see \cite{kidambi2021mobile} for details).


\clearpage
\section{Omitted proofs for Section~\ref{sec:application}}
\label{app:proof_IL}

To improve readability, we define the feature expectation vector as $\lambda(\pi) = \phim_r\transpose \mu(\pi)$ for any policy $\pi$, where $\mu(\pi_k)$ denotes the occupancy measure of policy $\pi_k$. This notation will be used in the following proofs.


%%%%%%%%%%%%%%%%%%%%%%%%%%%%%%%%%%%%%%%%%%%%%%%%%%%%%%%%%%%%%%%%%%%%%%%%%%%%%%%%
%%%%%%%%%%%%%%%%%%%%%%%%%%%%%%%%%%%%%%%%%%%%%%%%%%%%%%%%%%%%%%%%%%%%%%%%%%%%%%%%
\subsection{Proof of Theorem~\ref{thm:FraUpper} (guarantee for the output of Algorithm~\ref{alg:fra})}

\FraUpper*

\begin{proof}
    Using the decomposition presented in \Cref{sec:application}, we can express the regret as
    %
    \begin{equation*}
        (1 - \gamma) \regretIL = \underbrace{\sum^K_{k=1} \innerprod{r_k}{\mu\brr{\expert} - \mu\brr{\pi_k}}}_{(1-\gamma)\regretK^\pi(\mu\brr{\expert})} + \underbrace{\sum^K_{k=1} \innerprod{\Phi\transpose\mu(\pi_k) - \Phi\transpose\mu(\expert)}{w_k - w_{\mathrm{true}}}}_{(1-\gamma)\regretK^w(w_{\mathrm{true}})} \label{eq:dec}\,.
    \end{equation*}
    %
    By bounding $\regret^w(w_{\mathrm{true}})$ and $\regret^\pi(\mu(\expert))$ using \Cref{thm:reward_regret_bound} and \Cref{thm:main}, respectively, we obtain that with probability $1 - 3 \delta$
    %
    \begin{align*}
        \frac{1}{K} \regretIL &\leq  10 H (B \WMAX+\RMAX) \sqrt{ \frac{\log \delta^{-1}}{K}} + 24 H\WMAX B \sqrt{\frac{ \log\brr{\frac{1}{\delta}}}{\tau_E}} \\&\phantom{=}+ \tilde{\mathcal{O}}(d^{3/2}(1 - \gamma)^{-9/4} \log^{1/2} \abs{\aspace} K^{-1/2})\,.
    \end{align*}
    %
    Therefore, by considering $B$ and $\RMAX$ as constants and choosing $K = \widetilde{\mathcal{O}}\brr{\frac{ d^{3} \log (\abs{\aspace}\delta^{-1})}{(1 - \gamma)^{4.5}\varepsilon^2}}$ and $\tau_E = \widetilde{\mathcal{O}}\brr{\frac{ \WMAX^2 \log(1/\delta)}{(1 - \gamma)^2\varepsilon^2}}$ we have that with probability $1 - 3 \delta$ it holds that
    %
    \begin{equation*}
        \frac{1}{K} \regretIL \leq 4 \epsilon\,.
    \end{equation*}
    %
    Since $\frac{1}{K} \regretIL$ is a random variable bounded by $(1-\gamma)^{-1}$ almost surely, in expectation we have the following bound
    %
    \begin{equation*}
        \bbE_{\mathrm{Alg}} \bs{\frac{1}{K} \regretIL} \leq \frac{3 \delta}{1-\gamma} + 4 \varepsilon\,.
    \end{equation*}
    %
    Thus, by choosing $\delta \leq \nicefrac{\varepsilon}{3(1-\gamma)}$ we can conclude that
    %
    \begin{equation*}
        \bbE_{\mathrm{Alg}} \bs{\frac{1}{K} \regretIL} \leq 5 \varepsilon\,.
    \end{equation*}
    %
    Finally, by selecting $\pi^{\mathrm{out}}$ uniformly at random from the policies generated by \Cref{alg:fra} 
    we have that
    \begin{equation*}
        \bbE_{\mathrm{Alg}} \innerprod{\initial}{V_{\true}^{\expert} - V_{\true}^{\pi^{\mathrm{out}}}} \leq 5 \varepsilon\,.
    \end{equation*}
\end{proof}


%%%%%%%%%%%%%%%%%%%%%%%%%%%%%%%%%%%%%%%%%%%%%%%%%%%%%%%%%%%%%%%%%%%%%%%%%%%%%%%%
%%%%%%%%%%%%%%%%%%%%%%%%%%%%%%%%%%%%%%%%%%%%%%%%%%%%%%%%%%%%%%%%%%%%%%%%%%%%%%%%
\subsection{Proof of Theorem~\ref{thm:reward_regret_bound} (regret bound for the reward player)}

\begin{restatable}{theorem}{regretreward} \label{thm:reward_regret_bound}
    Assume that $w_{\mathrm{true}} \in \cW$ for some non-empty closed convex set $\cW$ and that for any $w \in \cW$, $\norm{w}\leq \WMAX$. Then, OGD with $\eta_r = \nicefrac{\WMAX}{B \sqrt{K}}$ ran for $K$ iterations satisfies with probability at least $1 - 2 \delta$ that
    %
    \begin{equation*}
        \regretK^w(w_{\mathrm{true}}) \leq 10 H (B \WMAX + \RMAX) \sqrt{K \log 1 / \delta} + 24 H \WMAX K B \sqrt{\frac{\log\brr{\frac{1}{\delta}}}{\tau_E}}\,.
    \end{equation*}
\end{restatable}

\begin{proof}
    Given the definition of the feature expectation vector $\lambda (\pi)$, we can rewrite the regret for the reward player as follows
    %
    \begin{equation*}
        \spr{1 - \gamma} \regret^w(w_{\mathrm{true}}) = \sum^K_{k=1} \innerprod{\lambda(\pi_k) - \lambda(\expert)}{w_k - w_{\mathrm{true}}}\,.
    \end{equation*}
    %
    Then, adding and subtracting the estimators for the occupancy measures, we get
    %
    \begin{align*}
        (1-\gamma) \regret^w(w_{\mathrm{true}}) &= \sum^K_{k=1} \innerprod{\phi_\cost(X_k, A_k) - \widehat{\lambda(\expert)}}{w_k - w_{\mathrm{true}} } \\
        &\phantom{=}+ \sum^K_{k=1} \innerprod{\lambda(\pi_k) - \phi_\cost(X_k, A_k)}{w_k - w_{\mathrm{true}}} \\
        &\phantom{=}+ \sum^K_{k=1} \innerprod{\widehat{\lambda(\expert)} - \lambda(\expert)}{ w_k-w_{\mathrm{true}}}\,.
    \end{align*}
    %
    Now, using the regret bound for OGD \citep{Zin03}, we can bound the first term in the decomposition above as
    %
    \begin{align*}
        \sum^K_{k=1} &\innerprod{\phi_\cost(X_k, A_k) - \widehat{\lambda(\expert)}}{w_k-w_{\mathrm{true}} } \\&\leq  \frac{\max_{w\in\mathcal{W}}\norm{w_{\mathrm{true}} - w_1}_2^2}{2 \eta_r} + \frac{\eta_r}{2} \sum^K_{k=1} \norm{\widehat{\lambda(\expert)} - \phi_\cost(X_k, A_k)}_2^2 \\
        &\leq \frac{2 \WMAX^2}{\eta_r} + 2 \eta_r B^2 K\,,
    \end{align*}
    %
    Looking at the term $\sum^K_{k=1} \innerprod{\lambda (\pi_k) - \phi_\cost (X_k, A_k)}{w_k - w_{\mathrm{true}}}$, we notice that
    %
    \begin{equation*}
        \psi_k = \innerprod{\lambda(\pi_k) - \phi_\cost(X_k, A_k)}{w_k - w_{\mathrm{true}}}
    \end{equation*}
    %
    is a martingale difference sequence such that
    %
    \begin{equation*}
        \abs{\innerprod{\lambda(\pi_k) - \phi_\cost(X_k, A_k)}{w_k - w_{\mathrm{true}}}} \leq  4 \RMAX\,.
    \end{equation*}
    %
    Applying Azuma-Hoeffding's inequality, we have that with probability $1 - \delta$
    %
    \begin{equation*}
        \sum^K_{k=1} \innerprod{\lambda(\pi_k) - \phi_\cost(X_k, A_k)}{w_k - w_{\mathrm{true}}} \leq \RMAX \sqrt{8  K \log \brr{\frac{1}{\delta}}}.
    \end{equation*}
    %
    Then, plugging in this bound in the regret decomposition we obtain
    %
    \begin{align*}
        (1-\gamma) \regret^w(w_{\mathrm{true}}) &\leq \frac{2 \WMAX^2}{\eta_r} + 2 \eta_r B^2 K + \RMAX \sqrt{8K \log 1 / \delta} \\
        &\phantom{=}+ \sum^K_{k=1} \innerprod{\widehat{\lambda(\expert)} - \lambda(\expert)}{ w_k-w_{\mathrm{true}}}\,.
    \end{align*}
    %
    Then, we treat the last term using Cauchy-Schwartz's inequality
    %
    \begin{align*}
        \sum^K_{k=1} \innerprod{\widehat{\lambda(\expert)} - \lambda(\expert)}{ w_k-w_{\mathrm{true}}} &\leq \sum^K_{k=1}\norm{w_{\mathrm{true}}-w^k}_{2}\norm{\widehat{\lambda(\expert)} - \lambda(\expert)}_2 \\
        &\leq 2 \WMAX K \norm{\widehat{\lambda(\expert)} - \lambda(\expert)}_2\,.
    \end{align*}
    %
    It remains to find a high probability (dimension-free) upper bound on $\norm{\widehat{\lambda(\expert)} - \lambda(\expert)}_{2}$. First, notice that $\norm{\widehat{\lambda(\expert)} - \lambda(\expert)}_{2} = \norm{(\tau_E)^{-1} \brr{\sum^{\tau_E}_{i=1} \phi_\cost(X^i_E,A^i_E) - \lambda(\expert) }}_2$. Then, we use the notation $u_{x,a} = \phi_\cost(x,a) - \lambda(\expert) $ for all state action pairs $x,a$ and using that for all $x, a \in \cX \times \cA$, $\norm{\phi_\cost(x, a)}_2 \leq B$, we have
    %
    \begin{equation*}
        \sum^{\tau_E}_{i=1} \mathbb{E}\bs{\norm{u_{X^i_E,A^i_E}}_2^2} \leq \sum^{\tau_E}_{i=1} \mathbb{E}\bs{\norm{\phi_\cost(X^i_E, A^i_E) - \lambda(\expert)}_2^2} \leq 4\tau_E B^2\,.
    \end{equation*}
    %
    Moreover, for any $x, a \in \cX \times \cA$, $\norm{u_{x,a}} \leq 2 B$ and $\bbE \bs{u_{X^i_E, A^i_E}} = 0$ because of the distribution of the dataset $\mathcal{D}_\expert $. Thus, by applying \cite[Proposition 2]{hsu2012tail}, it holds that for all $t > 0$
    %
    \begin{equation*}
        \mathbb{P}\bs{\norm{\sum^{\tau_E}_{i=1} u_{X^i_E, A^i_E}} > \sqrt{4 \tau_E} B + \sqrt{32 \tau_E t} B + (8/3)2Bt} \leq e^{-t}
    \end{equation*}
    %
    Therefore, choosing $t = \log \frac{1}{\delta}$, we obtain that with probability $1 - \delta$
    %
    \begin{align*}
        \norm{\sum^{\tau_E}_{i=1} \phi_\cost(X^i_E, A^i_E) - \lambda(\expert)} &\leq \sqrt{4 \tau_E} B + \sqrt{32 \tau_E \log\brr{\frac{1}{\delta}}} B + \frac{16 B}{3}  \log\brr{\frac{1}{\delta}} \\
        &\leq 6 B \sqrt{\tau_E \log\brr{\frac{1}{\delta}}} + \frac{16 B}{3} \log \brr{\frac{1}{\delta}}\,.
    \end{align*}
    %
    Then, dividing by $\tau_E$ we obtain that
    %
    \begin{align*}
        \norm{\widehat{\lambda(\expert)} - \lambda(\expert)}_{2} \leq 6 B \sqrt{\frac{\log\brr{\frac{1}{\delta}}}{\tau_E}} + \frac{16 B}{3 \tau_E} \log\brr{\frac{1}{\delta}}\,.
    \end{align*}
    %
    Then, for $\tau_E \geq \frac{64}{18^2} \log \frac{1}{\delta}$, we have that
    %
    \begin{equation*}
        6 \sqrt{\frac{\log\brr{\frac{1}{\delta}}}{\tau_E}} \geq \frac{8}{3 \tau_E} \log\brr{\frac{1}{\delta}}\,,
    \end{equation*}
    %
    and hence that with probability $1 - \delta$,
    %
    \begin{align*}
        \norm{\widehat{\lambda(\expert)} - \lambda(\expert)}_{2} \leq 12 B \sqrt{\frac{\log\brr{\frac{1}{\delta}}}{\tau_E}}\,.
    \end{align*}
    Thus, by a union bound and choosing $\eta_r = \nicefrac{\WMAX}{B \sqrt{K}}$, we have that with probability $1 - 2 \delta$,
    \begin{align*}
        (1-\gamma) \regret^w(w_{\mathrm{true}}) &= 4 B \WMAX \sqrt{K} +  \RMAX\sqrt{8 K \log \delta^{-1}} + 24 \WMAX B K \sqrt{ \frac{\log\brr{\frac{1}{\delta}}}{\tau_E}} \\
        &\leq 10 (B \WMAX + \RMAX) \sqrt{K \log \delta^{-1}} + 24 \WMAX K B \sqrt{\frac{\log \brr{\frac{1}{\delta}}}{\tau_E}}\,.
    \end{align*}
\end{proof}


\clearpage
\section{Lower bounds for imitation learning}
\label{app:lower}
In this section, we prove lower bounds for both $K$ and $\tau_E$ for all algorithms following Protocol~\ref{prot:interaction} given hereafter.
%
\begin{protocol}[!h]
    \caption{Imitation learning from features alone in Linear MDPs. \label{prot:interaction}}
    \centering
    \begin{algorithmic}[1]
        \STATE The learner adopts a learning algorithm $\mathrm{Alg}$ that receives as input \\
        (1) a features dataset $\cD_\expert = \scbr{\phi_\cost \spr{X^i_E, A^i_E}}_{i=1}^{\tau_E}$ where for any $i \in \sbr{\tau_E}$, $X^i_E, A^i_E \sim \mu \spr{\expert}$, \\
        (2) read access to $\phi_P \spr{x, a}$ for all $x, a \in \cX \times \cA$, \\
        (3) trajectory access to $\cM \setminus \true$, and \\
        (4) the reward class $\cR$ such that $\true \in \cR$.
        
        \STATE $\mathrm{Alg}$ samples $K$ trajectories from $\cM \setminus \true$ and outputs $\pi^{\mathrm{out}}$ s.t. $\bbE \sbr{\inp{\initial, V_{\true}^\expert - V_{\true}^{\pi^{\mathrm{out}}}}} \leq \varepsilon$.
    \end{algorithmic}
\end{protocol}

We prove an $\Omega \spr{\varepsilon^{-2}}$ lower bound for both cases, demonstrating that Algorithm~\ref{alg:fra} is rate optimal. First, we state the lower bound $K$ that holds even with perfect knowledge of the expert feature expectation vector $\lambda \spr{\expert}$, a strictly easier setting compared the one under which \Cref{thm:FraUpper} is proven.
%
\begin{restatable}{theorem}{LowerK} \textbf{(Lower Bound on $K$)} \label{thm:LowerK}
    For any algorithm $\mathrm{Alg}$, there exists an MDP $\cM$ and an expert policy $\expert$ such that $\mathrm{Alg}$, taking as input $ \phim_r\transpose \mu_\cM \spr{\expert}$, requires $K = \Omega \spr{\frac{d}{\spr{1 - \gamma}^2 \varepsilon^2}}$ to guarantee $\bbE_{\mathrm{Alg}} \sbr{\inp{\initial, V_\cM^{\expert} - V_\cM^{\pi^{\mathrm{out}}}}} = \cO \spr{\varepsilon}$.
\end{restatable}

\noindent Next, we establish a lower bound on the required number of expert demonstration $\tau_E$. The result holds even with perfect knowledge of the transition dynamics (\ie for $K = \infty$).
%
\begin{restatable}{theorem}{LowerTauE} \textbf{(Lower Bound on $\tau_E$)} \label{thm:LowerTauE} \label{thm:lower-bound-2states_expert}
    Let $\gamma \geq \frac12$. For any algorithm $\mathrm{Alg}$, there exists an MDP $\cM$ and an expert policy $\expert$ such that $\mathrm{Alg}$ taking as input the transitions dynamics and an expert dataset of size $\tau_E$ requires $\tau_E = \Omega \spr{\frac{\WMAX^2}{\spr{1 - \gamma}^2 \varepsilon^2}}$ to guarantee $\bbE_{\mathrm{Alg}} \sbr{\inp{\initial, V_\cM^{\expert} - V_\cM^{\pi^{\mathrm{out}}}}} = \cO \spr{\varepsilon}$.
\end{restatable}
%
\noindent The proofs are provided in the following sections.


%%%%%%%%%%%%%%%%%%%%%%%%%%%%%%%%%%%%%%%%%%%%%%%%%%%%%%%%%%%%%%%%%%%%%%%%%%%%%%%%
%%%%%%%%%%%%%%%%%%%%%%%%%%%%%%%%%%%%%%%%%%%%%%%%%%%%%%%%%%%%%%%%%%%%%%%%%%%%%%%%
\subsection{Proof of Theorem~\ref{thm:LowerK} (lower bound on the number of interactions)}

We start with the proof of the lower bound on $K$. We consider a class of possibly randomized algorithms that output a policy $\pi^{\mathrm{out}}$ given a dataset of expert features $\cD_\expert$ and $K$ trajectories collected by the learner in the MDP $\cM$.

\noindent \emph{Proof Idea}. To construct a lower bound, we consider the case of imitation learning from states alone (\ie $\phim_r \spr{x, a} = \bfe_x$), and $\lambda \spr{\expert}$ represents the state expert occupancy measure. We consider the case of a two-state MDP, where $\cX = \scbr{x_0, x_1}$, and the learner knows the \emph{good} state $x_0$ that maximizes the expert's occupancy measure due to having access to $\lambda \spr{\expert}$. The leaner's objective is to maximize the time spent in this good state. All actions in the \emph{bad} state $x_1$ share the same transition kernel. Therefore, the agent's decisions in the state $x_0$ is the only factor that influences the outcome. An action labeled as $a^\star$ is available in the state $x_0$. The transition kernel $P\spr{x_0 \given x_0, a}$ is identical for all actions $a \neq a^\star$, while for $a^\star$, it is defined as $P \spr{x_0 \given x_0, a} + \epsilon$. We then consider a family of $\abs{\cA}$ MDPs, where each MDP assigns the role of $a^\star$ to a different action. We will formally demonstrate that for any algorithm in $\mathrm{Alg}$, there exists at least one MDP within this family where achieving $\bbE_{\mathrm{Alg}} \sbr{\inp{\initial, V_\cM^{\expert} - V_\cM^{\pi^{\mathrm{out}}}}} = \cO \spr{\varepsilon}$ requires $K = \Omega \spr{\frac{\abs{\cA}}{\spr{1 - \gamma}^2 \varepsilon^2}}$. Finally, the bound for an arbitrary dimension $d$ is obtained noticing that this MDP can be written as a linear MDP with features dimension $d = 2 + 2 \abs{\cA}$.

\bigskip
\begin{proof}
    For any policy $\pi$, we denote $\lambda_\cM \spr{\pi} = \phim_r^\trans \mu_\cM \spr{\pi}$ the expected feature vector of the policy $\pi$ in the MDP $\cM$. We consider a deterministic algorithm $\mathrm{Alg}$ that maps $\lambda_\cM \spr{\expert}$ and $K$ environment trajectories to a policy. The extension to randomized algorithms can be done by an application of Fubini's theorem (see \cite{bubeck2012regret}). The hard instance we consider for the lower bound is an MDP $\cM$ with two states, $x_0$ and $x_1$, and $\abs{\cA}$ actions per state. For any action $a$, the reward function is given by $\true \spr{x_0, a} = 1$, and $\true \spr{x_1, a} = 0$. We will refer to state $x_0$ as the ``good'' state and to state $x_1$ as the ``bad'' state.  In state $x_1$, the transition kernel induced by any action $a$ is the same, \ie $P \spr{x_1 \given x_1, a} = 1 - \delta_1$, and $P \spr{x_0 \given x_1, a} = \delta_1$ for some $\delta_1 \in \spr{0, 1}$. Let $\delta_0 \in \spr{0, 1}$ and $\epsilon \in \spr{0, \delta_0}$. In state $x_0$, there is an action $a^\star$ with a slightly different transition kernel
    %
    \begin{align*}
        P \spr{x_1 \given x_0, a^\star} = \delta_0 - \epsilon, \quad P \spr{x_0 \given x_0, a^\star} = 1 - \delta_0 + \epsilon\,,
    \end{align*}
    %
    whereas for any action $a \neq a^\star$, we set
    %
    \begin{align*}
        P \spr{x_1 \given x_0, a} = \delta_0, \quad P \spr{x_0 \given x_0, a} = 1 - \delta_0\,.
    \end{align*}
    %
    We set the unknown expert policy $\expert$ such that it always select action $a^\star$ in both states, \ie $\expert \spr{a^\star \given x_0} = \expert \spr{a^\star \given x_1} = 1$. Setting $\nu_0 \spr{x_0} = 1$, we can write the flow constraints and get
    %
    \begin{align*}
        &\nu \spr{\expert, x_0} = 1 - \gamma + \gamma \spr{1 - \delta_0 + \epsilon} \nu \spr{\expert, x_0} + \gamma \delta_1 \nu \spr{\expert, x_1}\,, \\
        &\nu \spr{\expert, x_1} = \gamma \spr{1 - \delta_1} \nu \spr{\expert, x_1} + \gamma \spr{\delta_0 - \epsilon} \nu \spr{\expert, x_0}\,.
    \end{align*}
    %
    The second equation gives $\nu \spr{\expert, x_1} = \frac{\gamma \spr{\delta_0 - \epsilon}}{1 - \gamma \spr{1 - \delta_1}} \nu \spr{\expert, x_0}$, which we can plug back into the first equation to obtain
    %
    \begin{align*}
        \nu \spr{\expert, x_0} = 1 - \gamma + \spr{\gamma \spr{1 - \delta_0 + \epsilon} + \frac{\gamma^2 \delta_1 \spr{\delta_0 - \epsilon}}{1 - \gamma \spr{1 - \delta_1}}} \nu \spr{\expert, x_0} \,,
    \end{align*}
    %
    which we can rearrange to get
    %
    \begin{align*}
        \nu \spr{\expert, x_0} = \frac{1 - \gamma + \gamma \delta_1}{1 - \gamma + \gamma \delta_1 + \gamma \delta_0 - \gamma \epsilon}\,.
    \end{align*}
    %
    Using the normalization constraint $\nu \spr{\expert, x_0} + \nu \spr{\expert, x_1} = 1$, we also get
    %
    \begin{equation*}
        \nu \spr{\expert, x_1} = \frac{\gamma \delta_0 - \gamma \epsilon}{1 - \gamma + \gamma \delta_1 + \gamma \delta_0 - \gamma \epsilon}\,.
    \end{equation*}
    %
    %Note that the expert state occupancy is the same for any environment $\cM_i$, and that for any $\cM \in \cH$, $\lambda_\cM \spr{\expert} = \nu_\cM \spr{\expert}$.
    Furthermore, let $\pi_{\mathrm{bad}}$ be a ``bad'' policy that always plays an action $a \neq a^\star$. The same calculation with $\epsilon = 0$ shows that the state occupancy measure for the policy $\pi_{\mathrm{bad}}$ is given by
    %
    \begin{align*}
        \nu \spr{\pi_{\mathrm{bad}}, x_0} &= \frac{1 - \gamma + \gamma \delta_1}{1 - \gamma + \gamma \delta_1 + \gamma \delta_0}\,, \\
        \nu \spr{\pi_{\mathrm{bad}}, x_1} &= \frac{\gamma \delta_0}{1 - \gamma + \gamma \delta_1 + \gamma \delta_0}\,.
    \end{align*}
    %
    Let $\tilde{\pi}$ be any policy. Noting that for any $x$, $V^\expert \spr{x} = Q^\expert \spr{x, a^\star}$, we can use the performance difference lemma and get
    %
    \begin{align*}
        \inp{\mu \spr{\expert} - \mu \spr{\tilde{\pi}}, \true} &= \bbE_{\spr{x, a} \sim \mu \spr{\tilde{\pi}}} \sbr{V^\expert \spr{x} - Q^\expert \spr{x, a}} \\
        &= \bbE_{\spr{x, a} \sim \mu \spr{\tilde{\pi}}} \sbr{Q^\expert \spr{x, a^\star} - Q^\expert \spr{x, a}}\,.
    \end{align*}
    %
    All actions share the same transition kernel in $x_1$ thus for any action $a$, $Q^\expert \spr{x_1, a^\star} = Q^\expert \spr{x_1, a}$ and we have
    %
    \begin{align*}
        \inp{\mu \spr{\expert} - \mu \spr{\tilde{\pi}}, \true} &= \nu \spr{\tilde{\pi}, x_0} \sum_{a \in \cA \setminus \scbr{a^\star}} \tilde{\pi} \spr{a \given x_0} \spr{Q^\expert \spr{x_0, a^\star}  - Q^\expert \spr{x_0, a}}\,.
    \end{align*}
    %
    Next, we need to compute the difference of Q-values. Using the Bellman equations for $\expert$ in state $x_0$, we have
    %
    \begin{align}
        \forall a \neq a^\star, &Q^\expert \spr{x_0, a} = 1 + \gamma \delta_0 Q^\expert \spr{x_1, a^\star} + \gamma \spr{1 - \delta_0} Q^\expert \spr{x_0, a^\star} \label{eq:bellman-eq-x0-a} \\
        &Q^\expert \spr{x_0, a^\star} = 1 + \gamma \spr{\delta_0 - \epsilon} Q^\expert \spr{x_1, a^\star} + \gamma \spr{1 - \delta_0 + \epsilon} Q^\expert \spr{x_0, a^\star}\,. \label{eq:bellman-eq-x0-astar}
    \end{align}
    %
    Solving the second equation for $Q^\expert \spr{x_0, a^\star}$ gives
    %
    \begin{align}
        Q^\expert \spr{x_0, a^\star} = \frac{1}{1 - \gamma \spr{1 - \delta_0 + \epsilon}} \spr{1 + \gamma \spr{\delta_0 - \epsilon} Q^\expert \spr{x_1, a^\star}} \label{eq:q-x0-astar-inter}\,.
    \end{align}
    %
    By the Bellman equation in state $x_1$ and action $a^\star$, we further have
    %
    \begin{align*}
        Q^\expert \spr{x_1, a^\star} = 0 + \gamma \delta_1 Q^\expert \spr{x_0, a^\star} + \gamma \spr{1 - \delta_1} Q^\expert \spr{x_1, a^\star}\,,
    \end{align*}
    %
    which implies that
    %
    \begin{align}
        Q^\expert \spr{x_1, a^\star} = \frac{\gamma \delta_1}{1 - \gamma(1 - \delta_1)} Q^\expert \spr{x_0, a^\star}\,. \label{eq:q-x1-astar-inter}
    \end{align}
    %
    Replacing \eqref{eq:q-x1-astar-inter} into \eqref{eq:q-x0-astar-inter}, we get
    %
    \begin{align*}
        Q^\expert \spr{x_0, a^\star} &= \frac{1}{1 - \gamma \spr{1 - \delta_0 + \epsilon}} + \frac{\gamma^2 \delta_1 \spr{\delta_0 - \epsilon}}{\spr{1 - \gamma \spr{1 - \delta_0 + \epsilon}} \spr{1 - \gamma \spr{1 - \delta_1}}} Q^\expert \spr{x_0, a^\star}\,.
    \end{align*}
    %
    Rearranging the terms gives
    %
    \begin{align}
        Q^\expert \spr{x_0, a^\star} &= \spr{1 - \frac{\gamma^2 \delta_1 \spr{\delta_0 - \epsilon}}{\spr{1 - \gamma \spr{1 - \delta_0 + \epsilon}} \spr{1 - \gamma \spr{1 - \delta_1}}}}^{-1} \frac{1}{1 - \gamma \spr{1 - \delta_0 + \epsilon}} \nonumber \\
        &= \frac{1 - \gamma \spr{1 - \delta_1}}{\spr{1 - \gamma \spr{1 - \delta_0 + \epsilon}} \spr{1 - \gamma \spr{1 - \delta_1}} - \gamma^2 \delta_1 \spr{\delta_0 - \epsilon}}\,. \label{eq:q-x0-astar}
    \end{align}
    %
    Plugging Equation~\eqref{eq:q-x0-astar} into Equation~\eqref{eq:q-x1-astar-inter}, we can deduce the value of the expert at $\spr{x_1, a^\star}$
    %
    \begin{equation*}
        Q^\expert \spr{x_1, a^\star} = \frac{\gamma \delta_1}{\spr{1 - \gamma \spr{1 - \delta_0 + \epsilon}} \spr{1 - \gamma \spr{1 - \delta_1}} - \gamma^2 \delta_1 \spr{\delta_0 - \epsilon}}\,.
    \end{equation*}
    %
    Looking at the difference $Q^\expert \spr{x_0, a^\star} - Q^\expert \spr{x_0, a}$, we can take the difference of Equations~\eqref{eq:bellman-eq-x0-astar} and \eqref{eq:bellman-eq-x0-a} to get
    %
    \begin{align*}
        Q^\expert \spr{x_0, a^\star} - Q^\expert \spr{x_0, a} &= \gamma \epsilon \spr{Q^\expert \spr{x_0, a^\star} - Q^\expert \spr{x_1, a^\star}} \\
        &= \frac{\gamma \epsilon \spr{1 - \gamma}}{\underbrace{\spr{1 - \gamma \spr{1 - \delta_0 + \epsilon}} \spr{1 - \gamma \spr{1 - \delta_1}} - \gamma^2 \delta_1 \spr{\delta_0 - \epsilon}}_{\spr{\diamondsuit}}}\,.
    \end{align*}
    %
    Next, we upper bound the denominator as follows
    %
    \begin{align*}
        \spr{\diamondsuit} &= 1 - \gamma \spr{1 - \delta_0 + \epsilon} - \gamma \spr{1 - \delta_1} \\
        &\phantom{=}+ \gamma^2 \spr{1 - \delta_0 + \epsilon - \delta_1 + \delta_0 \delta_1 - \epsilon \delta_1 - \delta_0 \delta_1 + \epsilon \delta_1} \\
        &= 1 - \gamma \spr{1 - \delta_0 + \epsilon} - \gamma \spr{1 - \delta_1} + \gamma^2 \spr{1 - \delta_0 - \delta_1 + \epsilon} \\
        &= 1 - \gamma + \gamma \delta_0 \spr{1 - \gamma} + \gamma \delta_1 \spr{1 - \gamma} - \gamma \spr{1 - \gamma} - \gamma \epsilon \spr{1 - \gamma} \\
        &= \spr{1 - \gamma}^2 + \gamma \delta_0 \spr{1 - \gamma} + \gamma \delta_1 \spr{1 - \gamma} - \gamma \epsilon \spr{1 - \gamma} \\
        &\leq \spr{1- \gamma}^2 + \gamma \delta_0 \spr{1 - \gamma} + \gamma \delta_1 \spr{1 - \gamma}\,,
    \end{align*}
    %
    where the inequality follows from $\gamma \epsilon \spr{1 - \gamma} > 0$. Setting $\delta_1 = \delta_0 = \frac{1 - \gamma}{\gamma}$, we obtain
    %
    \begin{equation*}
        \spr{\diamondsuit} \leq 3 \spr{1 - \gamma}^2\,,
    \end{equation*}
    %
    and it holds that
    %
    \begin{align*}
        Q^\expert \spr{x_0, a^\star} - Q^\expert \spr{x_0, a} \geq \frac{\gamma \epsilon}{3 \spr{1 - \gamma}}\,.
    \end{align*}
    %
    Moreover, the choice of $\delta_0$ and $\delta_1$ implies that $\nu \spr{\pi_{\mathrm{bad}}, x_0} = \frac23$. By definition of the transitions, note that always playing $a \neq a^\star$ like $\pi_{\mathrm{bad}}$ does minimizes the probability of being in state $x_0$. Thus, for any policy $\tilde{\pi}$, $\nu \spr{\tilde{\pi}, x_0} \geq \nu \spr{\pi_{\mathrm{bad}}, x_0}$, and we have
    %
    \begin{align*}
        \inp{\mu \spr{\expert} - \mu \spr{\tilde{\pi}}, \true} &\geq \nu \spr{\tilde{\pi}, x_0} \sum_{a \in \cA \setminus \scbr{a^\star}} \tilde{\pi} \spr{a \given x_0} \frac{\gamma \epsilon}{3 \spr{1 - \gamma}} \\
        &\geq \nu \spr{\pi_{\mathrm{bad}}, x_0} \spr{1 - \tilde{\pi} \spr{a^\star \given x_0}} \frac{\gamma \epsilon}{3 \spr{1 - \gamma}} \\
        &= 2 \spr{1 - \tilde{\pi} \spr{a^\star \given x_0}} \frac{\gamma \epsilon}{9(1 - \gamma)} \\
        &\geq \frac{\spr{1 - \tilde{\pi} \spr{a^\star \given x_0}} \epsilon}{9 \spr{1 - \gamma}}\,.
    \end{align*}
    %
    where the last inequality follows from $\gamma \geq 1/2$. We now consider the policy $\tilde{\pi} = \bar\pi$ produced by a learning algorithm $\mathrm{Alg}$ interacting with the MDP described above (with $\epsilon > 0$). We also consider  $\underline{\pi}$ the output of the same learning algorithm $\mathrm{Alg}$ when interacting with the MDP $\underline{\cM}$, a copy of $\cM$ with $\epsilon = 0$ (note that in $\underline{\cM}$, all actions are identical in \emph{both} states $x_0$ and $x_1$, so there is nothing to learn). In $\cM$, all actions are identical in state $x_1$, thus we can assume both policies are the same in state $x_1$, \ie $\bar\pi \spr{\cdot \given x_1} = \underline{\pi} \spr{\cdot \given x_1} = \bfe_{a^\star}$, and focus exclusively on learning in state $x_0$. By Pinkser's inequality, we have that
    %
    \begin{align*}
        \bar\pi \spr{a^\star \given x_0} - \underline{\pi} \spr{a^\star \given x_0} \leq \sqrt{2 \KL \spr{\underline{\pi} \spr{\cdot \given x_0} \| \bar\pi \spr{\cdot \given x_0}}}\,,
    \end{align*}
    %
    and the previous inequality becomes
    %
    \begin{align*}
        \inp{\mu \spr{\expert} -  \mu \spr{\bar\pi}, \true} \geq \frac{\epsilon}{9 \spr{1 - \gamma}} \spr{1 - \underline{\pi} \spr{a^\star \given x_0} - \sqrt{2 \KL \spr{\underline{\pi} \spr{\cdot \given x_0} \| \bar\pi \spr{\cdot \given x_0}}}}\,.
    \end{align*}
    %
    Denote $A = \abs{\cA}$ and let $\cH = \scbr{\cM_i}_{i=1}^A$ be a collection of MDPs instances where for any $i = 1, \dots, A$, the MDP $\cM_i$ is a copy of $\cM$ where the $i$th action is equal to $a^\star$, \ie $a_i = a^\star$. We denote $P_i$ the corresponding transitions. For any $i \in [\![1, A]\!]$, we denote $\bar\pi^i$ the policy output by the learning algorithm $\mathrm{Alg}$ after interacting with the instance $\cM_i$, and $\experti$ be the expert policy for the instance $\cM_i$, \ie the policy that always plays $a_i$. We denote $\mu_i \spr{\pi}$ the occupancy measure of any policy $\pi$ in the MDP $\cM_i$. Then, notice that the previous derivations apply for any MDP in $\cH$. Thus, summing over $i \in [\![1, A]\!]$ and noting that $\underline{\pi} \spr{\cdot \given x_0}$ is a probability distribution over $\cA$, we get
    %
    \begin{align}
        \sum_{i = 1}^A \inp{\mu_i \spr{\experti} - \mu_i \spr{\bar\pi^i}, \true} \geq \frac{\epsilon}{9 \spr{1 - \gamma}} \spr{A - 1 - \sum_{i = 1}^A \sqrt{2 \KL \spr{\underline{\pi} \spr{\cdot \given x_0} \| \bar\pi^i \spr{\cdot \given x_0}}}}\,. \label{eq:sum_lower_bound}
    \end{align}
    %
    For any $i \in \sbr{A}$ and $T \in \bbN^\star$, denote $\bbP_i^T$ the probability distribution over sets $\cD_i^T = \scbr{x_0, A_t^i, X_t^i}_{t \in \sbr{T}}$ of $T$ transitions starting from $x_0$ induced by the interaction between the algorithm $\mathrm{Alg}$ and the MDP $\cM_i$. Likewise, we denote $\underline{\bbP}^T$ the probability distribution corresponding to $\underline{\cM}$. Then, by the data processing inequality for the KL divergence, for any $i \in \sbr{A}$, it holds that
    %
    \begin{equation*}
        \KL \spr{\underline{\pi} \spr{\cdot \given x_0} \| \bar\pi^i \spr{\cdot \given x_0}} \leq \KL \spr{\underline{\bbP}^T \| \bbP_i^T}\,.
    \end{equation*}
    %
    Denoting $\underline{\bbE}$ the expectation with respect to $\underline{\bbP}^T$, we can use the Markov property of the environment and continue as follows
    %
    \begin{align*}
        \KL \spr{\underline{\bbP}^T \| \bbP_i^T} &= \underline{\bbE} \sbr{\log \spr{\frac{\prod_{t=1}^T \underline{P} \spr{\underline{X}_t \given x_0, \underline{A}_t} \underline{\bbP}^T \spr{\underline{A}_t \given \underline{X}_1, \underline{A}_1, \dots, \underline{X}_{t-1}}}{\prod_{t=1}^T P_i \spr{\underline{X}_t \given x_0, \underline{A}_t} \bbP_i^T \spr{\underline{A}_t \given \underline{X}_1, \underline{A}_1, \dots, \underline{X}_{t-1}}}}} \\
        &= \underline{\bbE} \sbr{\log \spr{\frac{\prod_{t=1}^T \underline{P} \spr{\underline{X}_t \given x_0, \underline{A}_t}}{\prod_{t=1}^T P_i \spr{\underline{X}_t \given x_0, \underline{A}_t}}}} \\
        &= \underline{\bbE} \sbr{\sumtT \log \spr{\frac{\underline{P} \spr{\underline{X}_t \given x_0, \underline{A}_t}}{P_i \spr{\underline{X}_t \given x_0, \underline{A}_t}}}}\,,
    \end{align*}
    %
    where the probabilities on the actions are equal due to running the same algorithm $\mathrm{Alg}$ with the same history up to time $t-1$. Next, we have
    %
    \begin{align*}
        \KL \spr{\underline{\bbP}^T \| \bbP_i^T} &= \sumtT \sum_{\spr{x, a} \in \cX \times \cA} \underline{\bbP}^T \sbr{\spr{\underline{X}_t, \underline{A}_t} = \spr{x, a}} \log \spr{\frac{\underline{P} \spr{x \given x_0, a}}{P_i \spr{x \given x_0, a}}} \\
        &= \sumtT \sum_{x \in \cX} \underline{\bbP}^T \sbr{\spr{\underline{X}_t, \underline{A}_t} = \spr{x, a_i}} \log \spr{\frac{\underline{P} \spr{x \given x_0, a_i}}{P_i \spr{x \given x_0, a_i}}}\,,
    \end{align*}
    %
    where we used that the transitions $\underline{P}$ and $P_i$ are the same for any action $a \neq a_i$. By definition of the transitions, we further have
    %
    \begin{align*}
        \KL \spr{\underline{\bbP}^T \| \bbP_i^T} &= \sumtT \underline{\bbP}^T \sbr{\spr{\underline{X}_t, \underline{A}_t} = \spr{x_0, a_i}}  \log \spr{\frac{1 - \delta_0}{1 - \delta_0 + \epsilon}} \\
        &\phantom{=}+ \sumtT \underline{\bbP}^T \sbr{\spr{\underline{X}_t, \underline{A}_t} = \spr{x_1, a_i}} \log \spr{\frac{\delta_0}{\delta_0 - \epsilon}}\,.
    \end{align*}
    %
    Next, by definition of $\underline{\bbP}^T$, we have
    %
    \begin{align*}
        \KL \spr{\underline{\bbP}^T \| \bbP_i^T} &= \sumtT \underline{\bbP}^T \sbr{\underline{A}_t = a_i} \underline{P} \spr{x_0 \given x_0, a_i} \log \spr{\frac{1 - \delta_0}{1 - \delta_0 + \epsilon}} \\
        &\phantom{=}+ \sumtT \underline{\bbP}^T \sbr{\underline{A}_t = a_i} \underline{P} \spr{x_1 \given x_0, a_i} \log \spr{\frac{\delta_0}{\delta_0 - \epsilon}} \\
        &= \underline{\bbE} \sbr{\sumtT \mathds{1} \scbr{\underline{A}_t = a_i}} \spr{\spr{1 - \delta_0} \log \spr{\frac{1 - \delta_0}{1 - \delta_0 + \epsilon}} + \delta_0 \log \spr{\frac{\delta_0}{\delta_0 - \epsilon}}}\,.
    \end{align*}
    %
    By \citealp[Lemma 20]{AJO08}, we can bound the KL divergence as follows
    %
    \begin{align*}
        \KL \spr{\underline{\bbP}^T \| \bbP_i^T} &\leq \frac{\epsilon^2}{\delta_0 \log(2)} \underline{\bbE} \sbr{\sumtT \mathds{1} \scbr{\underline{A}_t = a_i}} \\
        &\leq \frac{\epsilon^2}{\spr{1 - \gamma} \log \spr{2}} \underline{\bbE} \sbr{\sumtT \mathds{1} \scbr{\underline{A}_t = a_i}}\,,
    \end{align*}
    %
    where the last inequality is due to the choice of $\delta_0 = \frac{1 - \gamma}{\gamma}$ and $\gamma < 1$. Plugging this into Equation~\eqref{eq:sum_lower_bound} and dividing by $A$, we have
    %
    \begin{align*}
        \frac1A \sum_{i = 1}^A \inp{\mu_i \spr{\experti} - \mu_i \spr{\bar\pi^i}, \true} &\geq \frac{\epsilon}{9 \spr{1 - \gamma}} \spr{1 - \frac1A - \frac{\epsilon}{A} \sum_{i=1}^A \sqrt{\frac{\underline{\bbE} \sbr{\sumtT \mathds{1} \scbr{\underline{A}_t = a_i}}}{\spr{1 - \gamma} \log \spr{2}}}}\,.
    \end{align*}
    %
    By Jensen's inequality, we further get
    %
    \begin{align*}
        \frac1A \sum_{i = 1}^A \inp{\mu_i \spr{\experti} - \mu_i \spr{\bar\pi^i}, \true} &\geq \frac{\epsilon}{9 \spr{1 - \gamma}} \spr{1 - \frac1A - \epsilon \sqrt{\frac{\underline{\bbE} \sbr{\sum_{i=1}^A \sumtT \mathds{1} \scbr{\underline{A}_t = a_i}}}{A \spr{1 - \gamma} \log \spr{2}}}} \\
        &\geq \frac{1}{9 \spr{1 - \gamma}} \spr{\frac{\epsilon}{2} - \epsilon^2 \sqrt{\frac{T}{A \spr{1 - \gamma} \log \spr{2}}}}\,,
    \end{align*}
    %
    where the second inequality follows from $\sum_{i=1}^A \mathds{1} \scbr{\underline{A}_t = a_i} = 1$ almost surely for any $t$ and $1 - \frac1A \geq \frac12$. Note that the value of $\epsilon$ maximizing the lower bound is given by $\epsilon^\star = \frac14 \sqrt{\frac{A \spr{1 - \gamma} \log \spr{2}}{T}}$. To satisfy the constraint $\epsilon^\star \in \spr{0, \delta_0}$ with $\delta_0 = \frac{1 - \gamma}{\gamma}$, assume we have $T \geq \frac{\gamma^2 A \log \spr{2}}{16 \spr{1 - \gamma}}$. We plug the value of $\epsilon^\star$ in the previous inequality to get
    %
    \begin{align*}
        \frac1A \sum_{i = 1}^A \inp{\mu_i \spr{\experti} - \mu_i \spr{\bar\pi^i}, \true} &\geq \frac{1}{16 \cdot 9 \spr{1 - \gamma}} \sqrt{\frac{A \spr{1 - \gamma} \log \spr{2}}{T}} \\
        &= \frac{1}{144} \sqrt{\frac{A \log \spr{2}}{\spr{1 - \gamma} T}}\,,
    \end{align*}
    %
    The average can be upper bounded by the maximum, thus
    %
    \begin{align*}
        \max_{i = 1, \dots, A} \inp{\nu_0, V_{\cM_i}^{\experti} - V_{\cM_i}^{\bar\pi^i}} &= \frac{1}{1 - \gamma} \max_{i = 1, \dots, A} \inp{\mu_i \spr{\experti} - \mu_i \spr{\bar\pi^i}, \true} \\
        &\geq \frac{1}{144} \sqrt{\frac{A \log \spr{2}}{\spr{1 - \gamma}^3 T}}\,.
    \end{align*}
    %
    What remains is to set the number of samples $T$ to make the lower bound small enough to make $\max_{i = 1, \dots, A} \inp{\nu_0, V_{\cM_i}^{\experti} - V_{\cM_i}^{\bar\pi^i}} = \cO \spr{\varepsilon}$ possible, \ie we need to have $T = \Omega \spr{\frac{A}{\spr{1 - \gamma}^3 \varepsilon^2}}$ samples. Therefore, we need $T = \Omega \spr{\frac{A}{\spr{1 - \gamma}^3 \varepsilon^2}}$ samples to learn a $\cO \spr{\varepsilon}$-suboptimal policy in the MDP that achieves the maximum. In order to derive a lower bound on the episodes number $K$ we can divide the sample complexity lower bound for $T$ by the the expected number of transitions per episode which is $\spr{1-\gamma}^{-1}$. This gives $K = \Omega \spr{\frac{A}{\spr{1 - \gamma}^2 \varepsilon^2}}$. We can conclude by noting that our construction used in the lower bound is a linear MDP with dimensionality $d = 2 + 2 \abs{\cA}$, thus we have $K = \Omega \spr{\frac{d}{\spr{1 - \gamma}^2 \varepsilon^2}}$.
\end{proof}


%%%%%%%%%%%%%%%%%%%%%%%%%%%%%%%%%%%%%%%%%%%%%%%%%%%%%%%%%%%%%%%%%%%%%%%%%%%%%%%%
%%%%%%%%%%%%%%%%%%%%%%%%%%%%%%%%%%%%%%%%%%%%%%%%%%%%%%%%%%%%%%%%%%%%%%%%%%%%%%%%
\subsection{Proof of \Cref{thm:LowerTauE} (lower bound on the number of expert transitions)}
\label{app:lower_tau_E}

\emph{Proof Idea:} The construction of the lower bound consists in relating the problem to that of distinguishing two Bernoullis distributions with close means. For that, we consider two MDPs $\cM_0$ and $\cM_1$ that only differ in their reward function. They have two states $\cX = \scbr{x_0, x_1}$ and $\abs{\cA}$ actions available at each state. The initial distribution $\initial$ is chosen to be the uniform distribution over $\cX$. In state $x_1$, any action $a$ induces the same transition kernel: $P \spr{x_0 \given x_1, a} = \delta$. In state $x_0$, any action $a$ except some action $a^\star$ is such that $P \spr{x_1 | x_0, a} = \delta$. However, the special action $a^\star$ allows to stay in the state $x_0$ with a slightly higher probability, \ie $P \spr{x_1 \given x_0, a^\star} = \delta - \epsilon$. Then, the reward function in $\cM_0$ is defined as $\true^0 \spr{x_0, \cdot} = \WMAX$, and $\true^0 \spr{x_0, \cdot} = 0$, while in $\cM_1$, it is defined as $\true^1 \spr{x_0, \cdot} = 0$, $\true^1 \spr{x_1, \cdot} = \WMAX$. Finally, we define an expert $\pi_E^0$ for $\cM_0$ as the policy that always play the action $a^\star$, and an expert $\pi_E^1$ for $\cM_1$ that always play some action $a \neq a^\star$. We then show that the expert occupancy measures satisfy $\nu \spr{\pi_E^0, x_0} = 1/2 + \Delta$, for some small $\Delta > 0$, while $\nu \spr{\pi_E^1, x_0} = \frac12$. The remaining step is to reduce this problem to a lower bound on the regret of a two-arm Bernoulli bandits instance with means $\spr{1/2, 1/2}$ and $\spr{1/2 + \Delta, 1/2 - \Delta}$. The proof is formally presented hereafter.

\LowerTauE*

\begin{proof}
    As mentioned earlier, it is sufficient to consider deterministic algorithms that map histories to policies. The lower bound for randomized algorithms follows by an application of Fubini's theorem (see \citealp{bubeck2012regret}). We consider two MDPs $\cH = \scbr{\cM_0, \cM_1}$ with the same state space $\cX = \scbr{x_0, x_1}$ and $\abs{\cA}$ actions available in each state. The initial distribution $\initial$ is chosen to be the uniform distribution over $\cX$, \ie $\initial \spr{x_0} = \initial \spr{x_1} = \frac12$. The transitions are the same in both MDPs: in state $x_1$, each action $a \in \cA$ induces the following transition kernel
    %
    \begin{equation*}
        P \spr{x_0 \given x_1, a} = \delta, \quad P \spr{x_1 \given x_1, a} = 1 - \delta\,
    \end{equation*}
    %
    while in state $x_0$, there is an action $a^\star$ giving a slightly higher probability on staying in state $x_0$, \ie
    %
    \begin{align*}
        &P \spr{x_0 \given x_0, a^\star} = 1 - \delta + \epsilon, \quad P \spr{x_1 \given x_0, a^\star} = \delta - \epsilon \\
        \forall a \neq a^\star, &P \spr{x_0 \given x_0, a} = 1 - \delta, \quad P \spr{x_1 \given x_0, a} = \delta\,.
    \end{align*}
    %
    The reward functions, $\true^0$ and $\true^1$, are different. In $\cM_0$, the ``good'' state is $x_0$, \ie for any action $a \in \cA$, we set $\true^0 \spr{x_0, a} = \WMAX$, $\true^0 \spr{x_1, a} = 0$, and in $\cM_1$, the ``good'' state is $x_1$, \ie $\true^1 \spr{x_0, a} = 0$, and $\true^1 \spr{x_1, a} = \WMAX$. Note that $\WMAX = \RMAX$ due to using the features $\phi_r \spr{x, a} = \bfe_x$ for any state-action pair $x, a$.
     
    Then, we define one expert policy for each MDP. In $\cM_0$, the expert $\pi_E^0$ is the policy that always plays $a^\star$ and in $\cM_1$, the expert $\pi_E^1$ is the policy that always plays an action $a \neq a^\star$. Therefore, the state occupancy measure of expert $\pi_E^0$ in MDP $\cM_0$ has the highest mass in state $x_0$, while $\pi_E^1$ put equal mass on both states. Indeed, writing the flow constraints for both experts, we have
    %
    \begin{align*}
        \begin{pmatrix}
            1 - \gamma + \gamma \delta - \gamma \epsilon & - \gamma \delta \\
            - \gamma \spr{\delta - \epsilon} & 1 - \gamma + \gamma \delta
        \end{pmatrix}
        \nu \spr{\pi_E^0} &= \nu_0\,, \\
        \begin{pmatrix}
            1 - \gamma + \gamma \delta & - \gamma \delta \\
            - \gamma \delta & 1 - \gamma + \gamma \delta
        \end{pmatrix}
        \nu \spr{\pi_E^1} &= \nu_0\,.
    \end{align*}
    %
    Solving these linear systems using, \eg, Cramer's rule, we obtain
    %
    \begin{align*}
        \nu \spr{\pi_E^0, x_0} &= \frac{1 - \gamma + 2 \gamma \delta}{2 \spr{1 - \gamma - \gamma \epsilon + 2 \gamma \delta}}, &\nu \spr{\pi_E^0, x_1} &= \frac{1 - \gamma - 2 \gamma \epsilon + 2 \gamma \delta}{2 \spr{1 - \gamma - \gamma \epsilon + 2 \gamma \delta}}\,, \\
        \nu \spr{\pi_E^1, x_0} &= \frac12, &\nu \spr{\pi_E^1, x_1} &= \frac12\,.
    \end{align*}
    %
    For $i \in \scbr{0, 1}$, let $\bar\pi^i$ be the policy output by $\mathrm{Alg}$ when given a dataset $\cD_{\pi_E^i}$ as input and let $V_i^{\bar\pi^i}$ be the value function of policy $\bar\pi^i$ corresponding to the reward function $\true^i$ from the MDP $\cM_i$. By definition of $\true^i$, we can write
    %
    \begin{align}
        \frac12 \sum_{i \in \scbr{0, 1}} \inp{\initial, V_i^{\pi_E^i} - V_i^{\bar\pi^i}} &= \frac{1}{2 \spr{1 - \gamma}} \sum_{i \in \scbr{1, 2}} \inp{\mu \spr{\pi_E^i} - \mu \spr{\bar\pi^i}, \true^i} \nonumber \\
        &= \frac{\WMAX}{2 \spr{1 - \gamma}} \spr{\nu \spr{\pi_E^0, x_0} - \nu \spr{\bar\pi^0, x_0} + \nu \spr{\pi_E^1, x_1} - \nu \spr{\bar\pi^1, x_1}}\,. \label{eq:appE2-average-return}
    \end{align}
    %
    Thus, we need to compute the difference between state occupancy measures. Let $\tilde\pi$ be an arbitrary policy and denote $\alpha \in \sbr{0, 1}$ the probability of playing action $a^\star$ in state $x_0$, \ie $\tilde\pi \spr{a^\star \given x_0} = \alpha$. Writing down the flow constraints again, we can show that
    %
    \begin{equation*}
        \nu \spr{\tilde\pi, x_0} = \frac{1 - \gamma + 2 \gamma \delta}{2 \spr{1 - \gamma - \gamma \alpha \epsilon + 2 \gamma \delta}}, \quad \nu \spr{\tilde\pi, x_1} = \frac{1 - \gamma - 2 \gamma \alpha \epsilon + 2 \gamma \delta}{2 \spr{1 - \gamma - \gamma \alpha \epsilon + 2 \gamma \delta}}\,.
    \end{equation*}
    %
    Looking at the difference with $\pi_E^0$ in state $x_0$, we have
    %
    \begin{align*}
        \nu \spr{\pi_E^0, x_0} - \nu \spr{\tilde\pi, x_0} &= \frac{1 - \gamma + 2 \gamma \delta}{2 \spr{1 - \gamma - \gamma \epsilon + 2 \gamma \delta}} - \frac{1 - \gamma + 2 \gamma \delta}{2 \spr{1 - \gamma - \gamma \alpha \epsilon + 2 \gamma \delta}} \\
        &= \frac{\spr{1 - \gamma + 2 \gamma \delta} \spr{\spr{- \gamma \alpha \epsilon} - \spr{- \gamma \epsilon}}}{2 \spr{1 - \gamma - \gamma \epsilon + 2 \gamma \delta} \spr{1 - \gamma - \gamma \alpha \epsilon + 2 \gamma \delta}} \\
        &= \frac{\spr{1 - \gamma + 2 \gamma \delta} \gamma \epsilon \spr{1 - \alpha}}{2 \spr{1 - \gamma - \gamma \epsilon + 2 \gamma \delta} \spr{1 - \gamma - \gamma \alpha \epsilon + 2 \gamma \delta}}\,.
    \end{align*}
    %
    Setting $\delta = \frac{1 - \gamma}{\gamma}$ and noting $\epsilon \geq 0$, $\gamma \geq \frac12$, we can lower bound the difference as follows
    %
    \begin{align}
        \nu \spr{\pi_E^0, x_0} - \nu \spr{\tilde\pi, x_0} &= \frac{3 \spr{1 - \gamma} \gamma \epsilon \spr{1 - \alpha}}{2 \spr{3 \spr{1 - \gamma} - \gamma \epsilon} \spr{3 \spr{1 - \gamma} - \gamma \alpha \epsilon}} \nonumber \\
        &\geq \frac{\epsilon \spr{1 - \alpha}}{12 \spr{1 - \gamma}}\,. \label{eq:appE2-diffx0}
    \end{align}
    %
    Likewise, the difference between $\nu \spr{\pi_E^1}$ and $\nu \spr{\tilde\pi}$ in state $x_1$ is given by
    %
    \begin{align*}
        \nu \spr{\pi_E^1, x_1} - \nu \spr{\tilde\pi, x_1} &= \frac12 - \frac{1 - \gamma - 2 \gamma \alpha \epsilon + 2 \gamma \delta}{2 \spr{1 - \gamma - \gamma \alpha \epsilon + 2 \gamma \delta}} \\
        &= \frac{\gamma \alpha \epsilon}{2 \spr{1 - \gamma - \gamma \alpha \epsilon + 2 \gamma \delta}}\,.
    \end{align*}
    %
    Using the definition of $\delta$, and again $\epsilon \geq 0$, $\gamma \geq \frac12$, we get
    %
    \begin{align}
        \nu \spr{\pi_E^1, x_1} - \nu \spr{\tilde\pi, x_1} &= \frac{\gamma \alpha \epsilon}{2 \spr{3 \spr{1 - \gamma} - \gamma \alpha \epsilon}} \nonumber \\
        &\geq \frac{\epsilon \alpha}{12 \spr{1 - \gamma}}\,. \label{eq:appE2-diffx1}
    \end{align}
    %
    Plugging Inequalities~\eqref{eq:appE2-diffx0} and~\eqref{eq:appE2-diffx1} into Equation~\eqref{eq:appE2-average-return} with $\alpha = \bar\pi^0 \spr{a^\star \given x_0}$ and $\alpha = \bar\pi^1 \spr{a^\star \given x_0}$ respectively, we get
    %
    \begin{align*}
        \frac12 \sum_{i \in \scbr{0, 1}} \inp{\initial, V_i^{\pi_E^i} - V_i^{\bar\pi^i}} &\geq \frac{\epsilon \WMAX}{24 \spr{1 - \gamma}^2}\spr{1 - \bar\pi^0 \spr{a^\star \given x_0} + \bar\pi^1 \spr{a^\star \given x_0}} \\
        &= \frac{\epsilon \WMAX}{24 \spr{1 - \gamma}^2} \spr{\sum_{a \neq a^\star} \bar\pi^0 \spr{a \given x_0} + \bar\pi^1 \spr{a^\star \given x_0}}\,.
    \end{align*}
    %
    Next, we can lower bound the right hand side using the Bretagnolle-Huber inequality (see \citealp{bretagnolle1979estimation}, and \citealp[Theorem 14.2]{lattimore2020bandit}), which gives
    %
    \begin{equation} \label{eq:partial_lower_bound}
        \frac12 \sum_{i \in \scbr{0, 1}} \inp{\initial, V_i^{\pi_E^i} - V_i^{\bar\pi^i}} \geq \frac{\epsilon \WMAX}{24 \spr{1 - \gamma}^2} \exp \spr{- \KL \spr{\bar\pi^0 \spr{\cdot \given x_0} \| \bar\pi^1 \spr{\cdot \given x_0}}}\,.
    \end{equation}
    %
    Then, using the data processing inequality and using the fact that the learning algorithm produces $\bar\pi^i$ as a deterministic function of the dataset $\cD_{\pi_E^i}$ for $i = 0, 1$, we have that
    %
    \begin{equation*}
        \KL \spr{\bar\pi^0 \spr{\cdot \given x_0} \| \bar\pi^1 \spr{\cdot \given x_0}} \leq \KL \spr{\bbP_0^{\tau_E} \| \bbP_1^{\tau_E}}\,,
    \end{equation*}
    %
    where, for $i \in \scbr{0, 1}$, we denoted $\bbP_i^{\tau_E}$ the probability distribution over datasets of size $\tau_E$ induced by the interaction between the expert $\pi_E^i$ and the environment (analog to what is done in the proof of Theorem~\ref{thm:LowerK}). Next, we denote $\kl \spr{p, q}$ and $\chi^2 \spr{p, q}$ the KL and chi-squared divergences between bernoulli distributions of means $p$ and $p'$, \ie
    %
    \begin{align*}
        \kl \spr{p, q} &= p \log \spr{\frac{p}{q}} + \spr{1 - p} \log \spr{\frac{1 - p}{1 - q}} \\
        \chi^2 \spr{p, q} &= \frac{\spr{p - q}^2}{q \spr{1 - q}}\,.
    \end{align*}
    %
    By definition of the KL, we have
    %
    \begin{align*}
        \KL \spr{\bbP_0^{\tau_E} \| \bbP_1^{\tau_E}} &= \tau_E \cdot \kl \spr{\frac{3 \spr{1 - \gamma}}{2 \spr{3 \spr{1 - \gamma} - \gamma \epsilon}}, \frac12} \\
        &\leq \tau_E \cdot \chi^2 \spr{\frac{3 \spr{1 - \gamma}}{2 \spr{3 \spr{1 - \gamma} - \gamma \epsilon}}, \frac12} \\
        &= \tau_E \cdot \chi^2 \spr{\frac12 + \frac{\gamma \epsilon}{3 \spr{1 - \gamma} - \gamma \epsilon}, \frac12} \\
        &= \frac{4 \tau_E \gamma^2 \epsilon^2}{\spr{3 \spr{1 - \gamma} - \gamma \epsilon}^2} \\
        &\leq \frac{\tau_E \gamma^2 \epsilon^2}{\spr{1 - \gamma}^2}\,,
    \end{align*}
    %
    where the first inequality follows from the concavity of the logarithm function, and the second inequality uses the fact that $\epsilon \leq \delta = \frac{1-\gamma}{\gamma}$. Thus, plugging in this last inequality into Equation~\eqref{eq:partial_lower_bound}, we obtain
    %
    \begin{align*}
        \frac12 \sum_{i \in \scbr{0, 1}} \inp{\initial, V_i^{\pi_E^i} - V_i^{\bar\pi^i}} &\geq \frac{\epsilon \WMAX}{24 \spr{1 - \gamma}^2} \exp \spr{- \frac{\tau_E \gamma^2 \epsilon^2}{\spr{1 - \gamma}^2}} \\
        &\geq \frac{\epsilon \WMAX}{24 \spr{1 - \gamma}^2} \exp \spr{- \frac{\tau_E \epsilon^2}{\spr{1 - \gamma}^2}}\,,
    \end{align*}
    %
    where we used $\gamma < 1$ in the second inequality. Introducing $\epsilon' = \epsilon \spr{1 - \gamma}^{-1}$, we can rewrite the previous inequality as
    %
    \begin{equation*}
        \frac12 \sum_{i \in \scbr{0, 1}} \inp{\initial, V_i^{\pi_E^i} - V_i^{\bar\pi^i}} \geq \frac{\WMAX \epsilon'}{24 \spr{1 - \gamma}} \exp \spr{- \tau_E \spr{\epsilon'}^2}\,.
    \end{equation*}
    %
    It remains to make the lower bound small enough. To bound the average suboptimality gap by $\frac{\WMAX \epsilon'}{24 e \spr{1 - \gamma}}$ and have $\frac12 \sum_{i \in \scbr{0, 1}} \inp{\initial, V_i^{\pi_E^i} - V_i^{\bar\pi^i}} \leq \frac{\WMAX \epsilon'}{24 e \spr{1 - \gamma}}$, we need at least $\tau_E \geq \frac{1}{\spr{\epsilon'}^2}$ expert transitions. Therefore, to achieve
    %
    \begin{equation*}
        \frac12 \sum_{i \in \scbr{0, 1}} \inp{\initial, V_i^{\pi_E^i} - V_i^{\bar\pi^i}} \leq \varepsilon\,,
    \end{equation*}
    %
    for some $\varepsilon > 0$, we need to choose $\epsilon' = \nicefrac{24 e \spr{1 - \gamma} \varepsilon}{\WMAX}$, which means that every algorithm needs at least $\tau_E \geq \frac{\WMAX^2}{24^2 e^2 \spr{1 - \gamma}^2 \varepsilon^2}$ to guarantee a suboptimality gap of order $\varepsilon$.
\end{proof}


\clearpage
\section{Technical tools}

%%%%%%%%%%%%%%%%%%%%%%%%%%%%%%%%%%%%%%%%%%%%%%%%%%%%%%%%%%%%%%%%%%%%%%%%%%%%%%%%
\subsection{Reinforcement learning}

\begin{proposition}\label{prop:flow}
  The occupancy measure $\mu(\pi)$ of any policy $\pi$ satisfies the following system of equations:
  %
  \begin{equation}\label{eq:flow}
      E\transpose \mu \spr{\pi} = \gamma P\transpose \mu \spr{\pi} + \spr{1 - \gamma} \nu_0\,.
  \end{equation}
\end{proposition}

\begin{proof}
  Define the transition kernel induced by policy $\pi$ as $P_\pi$, with $P_\pi(\cdot|x) = \mathbb{E}_{A\sim\pi(\cdot|x)}\sbr{P(\cdot|x,A)}$. The proof follows from the following standard calculation:
  \begin{align*}
    E\transpose \mu \spr{\pi} &= (1-\gamma) \sum_{\tau=0}^\infty \pa{\gamma P_\pi\transpose}^\tau \nu_0 \\
    &= (1-\gamma) \sum_{\tau=1}^\infty \pa{\gamma P_\pi\transpose}^\tau \nu_0 + (1-\gamma) \nu_0 \\
    &= \gamma P_\pi (1-\gamma) \sum_{\tau=0}^\infty \pa{\gamma P_\pi\transpose}^\tau \nu_0 + (1-\gamma) \nu_0 \\
    &= \gamma P_\pi E\transpose \mu \spr{\pi} + (1-\gamma) \nu_0 \\
    &= \gamma P \mu \spr{\pi} + (1-\gamma) \nu_0,
  \end{align*}
  %
  where the last step follows from the easily-checked fact that $P \mu \spr{\pi} = P_\pi E\transpose \mu \spr{\pi}$.
\end{proof}


\begin{lemma}\label{lem:from-ev-to-q}
  Let $\pi$ be any policy, $Q \in \bbR^{\cX \times \cA}$ be any function defined on $\cX \times \cA$, and $V \in \bbR^\cX$ be such that for any $x$, $V \spr{x} = \bbE_{A \sim \pi \spr{\cdot \given x}} \sbr{Q \spr{x, A}}$. Then
  %
  \begin{equation*}
    \inp{\mu \spr{\pi}, \opE V} = \inp{\mu \spr{\pi}, Q}\,.
  \end{equation*}
\end{lemma}

\begin{proof}
  We have
  %
  \begin{align*}
    \inp{\mu \spr{\pi}, \opE V} &= \sum_{x \in \cX} \nu \spr{\pi, x} V \spr{x} \\
    &= \sum_{x \in \cX} \sum_{a \in \cA} \nu \spr{\pi, x} \pi \spr{a \given x} Q \spr{x, a} \\
    &= \sum_{x \in \cX} \sum_{a \in \cA} \mu \spr{\pi, x, a} Q \spr{x, a} \\
    &= \inp{\mu \spr{\pi}, Q}\,,
  \end{align*}
  %
  where the second equality follows from the definition of the function $V$ and the first equality from the definition of the state-action occupancy measure.
\end{proof}

\begin{lemma}\label{lem:ineq-kl-entropy}
    Let $\pi$ and $\pi'$ be two policies. Then,
    %
    \begin{equation*}
      \KL \spr{\mu \spr{\pi} \middle\| \mu \spr{\pi'}} \leq \frac{1}{1 - \gamma} \inp{\nu \spr{\pi}, \KL \spr{\pi \| \pi'}}\,.
    \end{equation*}
  \end{lemma}
  
\begin{proof}
  Using the chain rule of the relative entropy, we write
  %
  \begin{equation*}
      \KL \spr{\mu \spr{\pi} \middle\| \mu \spr{\pi'}} = \KL \spr{\nu \spr{\pi} \middle\| \nu \spr{\pi'}} + \inp{\nu \spr{\pi}, \KL \spr{\pi \middle\| \pi'}}\,.
  \end{equation*}
  %
  By the flow constraints and the joint convexity of the relative entropy, we bound the first term as
  %
  \begin{align*}
    \KL \spr{\nu \spr{\pi} \middle\| \nu \spr{\pi'}} &= \KL \spr{\gamma P\transpose \mu \spr{\pi} + \spr{1 - \gamma} \nu_0 \middle\| \gamma P\transpose \mu \spr{\pi'} + \spr{1 - \gamma} \nu_0} \\
    &\leq \spr{1 - \gamma} \KL \spr{\nu_0 \middle\| \nu_0} + \gamma \KL \spr{P\transpose \mu \spr{\pi} \middle\| P\transpose \mu \spr{\pi'}} \\
    &= \gamma \KL \spr{P\transpose \mu \spr{\pi} \middle\| P\transpose \mu \spr{\pi'}} \\
    &\leq \gamma \KL \spr{\mu \spr{\pi} \middle\| \mu \spr{\pi'}}\,,
  \end{align*}
  %
  where we also used the data-processing inequality in the last step. The proof is concluded by reordering the terms.
\end{proof}

\begin{lemma} \label{lem:mass-reduced}
  For any MDP $\cM$, any ascension function $p^\upplus$, and any policy $\pi$, we have for any state-action pair $\spr{x, a} \in \cX \times \cA$,
  %
  \begin{equation*}
    \mu^\upplus \spr{\pi, x, a} \leq \mu \spr{\pi, x, a}\,,
  \end{equation*}
  %
  where $\mu^\upplus \spr{\pi}$ denotes the state-action occupancy of $\pi$ in $\cM^\upplus$, the optimistically augmented MDP induced by $p^\upplus$.
\end{lemma}

\begin{proof}
  Let us consider a process  $\spr{X_\tau, A_\tau}_{\tau \in \bbN}$ generated by the policy $\pi$ in the MDP $\cM$, that is, such that $X_0 \sim \initial$, and for any $\tau \in \bbN$, $A_\tau \sim \pi \spr{\cdot \given X_\tau}$, and $X_{\tau+1} \sim P \spr{\cdot \given X_\tau, A_\tau}$. Additionally, we define a process $\spr{X_\tau^\upplus, A_\tau^\upplus}_{\tau \in \bbN}$ coupled to the process defined above as follows. At the first stage, we set $X_\tau^\upplus = X_0$. Then for any $\tau \geq 1$, the coupled process evolves as
  %
  \begin{equation*}
    X_{\tau+1}^\upplus, A_{\tau+1}^\upplus =
    \begin{cases}
        X_{\tau+1}, A_{\tau+1} \quad &\text{w.p.} \quad 1 - p^\upplus \spr{X_\tau, A_\tau} \quad \text{if} \quad  X_\tau^\upplus, A_\tau^\upplus = X_\tau, A_\tau \\
        x^\upplus, a \quad & \text{w.p.} \quad p^\upplus \spr{X_\tau, A_\tau} \quad \text{if} \quad  X_\tau^\upplus, A_\tau^\upplus = X_\tau, A_\tau \\
        x^\upplus, a \quad & \text{if} \quad X_\tau^\upplus, A_\tau^\upplus \neq X_\tau, A_\tau
    \end{cases}\,.
  \end{equation*}
  %
  It is straightforward to check that this process follows the dynamics of the optimistically augmented MDP $\cM^\upplus(r,p^\upplus)$ (since its transitions obey the kernel $P^\upplus$). By definition, for any state-action pair $\spr{x, a} \in \cX \times \cA$, we have
  %
  \begin{align*}
    \mu^\upplus \spr{\pi, x, a} &= \spr{1 - \gamma} \sum_{\tau=0}^\infty \gamma^\tau \bbP \sbr{X_\tau^\upplus = x, A_\tau^\upplus = a} \\
    &= \spr{1 - \gamma} \sum_{\tau=0}^\infty \gamma^\tau \spr{\bbP \sbr{X_\tau^\upplus = x, A_\tau^\upplus = a, X_\tau^\upplus \neq x^\upplus} + \bbP \sbr{X_\tau^\upplus = x, A_\tau^\upplus = a, X_\tau^\upplus = x^\upplus}} \\
    &= \spr{1 - \gamma} \sum_{\tau=0}^\infty \gamma^\tau \spr{\bbP \sbr{X_\tau = x, A_\tau = a, X_\tau^\upplus \neq x^\upplus} + 0} \\
    &\leq \spr{1 - \gamma} \sum_{\tau=0}^\infty \gamma^\tau \bbP \sbr{X_\tau = x, A_\tau = a} \\
    &= \mu \spr{\pi, x, a}\,.
  \end{align*}
  %
  In the third equality, the second term within the sum is equal to zero because $x \neq x^\upplus$, and in the other term we replaced $\spr{X_\tau^\upplus, A_\tau^\upplus}$ by $\spr{X_\tau, A_\tau}$ because the two coincide long as $X^\upplus_\tau \neq x^+$. This concludes the proof.
\end{proof}

%%%%%%%%%%%%%%%%%%%%%%%%%%%%%%%%%%%%%%%%%%%%%%%%%%%%%%%%%%%%%%%%%%%%%%%%%%%%%%%%
\subsection{Linear algebra and analysis}

\begin{lemma} \label{lemma:number-epochs-bound}
    Under the event $\cE_L$, the number of epochs $E \spr{K}$ in Algorithm~\ref{alg:linear-rmax-ravi-ucb} is bounded as
    %
    \begin{equation*}
      E \spr{K} \leq 5 d \log \spr{1 + \frac{B^2 T}{d}}\,.
    \end{equation*}
    %
    where $T = \LMAX K = \frac{\log \spr{\frac{K}{\delta}} K}{1 - \gamma}$.
\end{lemma}

\begin{proof}
    In the following, we denote $\phi_t = \phi \spr{x_t, a_t}$ for any $t$. The bound on the number of epochs is derived observing that since the determinant of the matrix $\Lambda_k$ can grow at most linearly then the condition is triggered at most a logarithmic number of times. In particular notice that
    %
    \begin{equation*}
        \det \spr{\Lambda_{t_{E \spr{K}}}} \geq 2 \det \spr{\Lambda_{t_{E \spr{K} - 1}}} \geq 2^2 \det \spr{\Lambda_{t_{E \spr{K} - 2}}} \geq 2^{E \spr{K} - 1} \det \spr{I} = 2^{E \spr{K} - 1}\,.
    \end{equation*}
    %
    Hence, it holds that $E \spr{K} - 1 \leq \frac{1}{\log 2} \log \spr{\det \Lambda_{t_{E \spr{K}}}}$. Then, denoting $T_{K+1} = T_K + L_K$ where $L_K$ is the length of episode $K$, we have that
    %
    \begin{align*}
        E \spr{K} &\leq 1 + \frac{1}{\log 2} \log \spr{\det \spr{\Lambda_{T_{K+1}}}} & \spr{\Lambda_{t_{E \spr{K}}} \preceq \Lambda_{T_{K+1}}} \\
        &\leq 1 + \frac{d}{\log 2} \log \brr{\frac{\optrace \spr{\Lambda_{T_{K+1}}}}{d}} & \text{(trace-determinant inequality)}\,.
    \end{align*}
    %
    By definition of the covariance matrix,
    %
    \begin{align*}
        E \spr{K} & \leq 1 + \frac{d}{\log 2} \log \spr{\frac{\optrace \spr{\sum_{t \in \sbr{T_{K+1}}} \phi_t \phi_t\transpose} + d}{d}} \\
        &= 1 + \frac{d}{\log 2} \log \spr{1 + \frac{\sum_{t \in \sbr{T_{K+1}}} \norm{\phi_t}_2^2}{d}}
        \\
        &\leq 1 + \frac{d}{\log 2} \log \spr{1 + \frac{B^2 T}{d}} \\
        &\leq 5 d \log \spr{1 + \frac{B^2 T}{d}}\,,
    \end{align*}
    %
    where the first equality follows from properties of the trace and the second inequality follows from $\norm{\phi_t}_2 \leq B$ and $T_{K+1} \leq T$ which holds under $\cE_L$.
\end{proof}

The following lemma is a generalization of Lemma~19 of \citet{cassel2024warmupfree} for an arbitrary threshold $\omega \geq 0$.
%
\begin{lemma} \label{lem:sigmoid-bound}
  For all $z \geq 0$, $\omega \geq 0$ it holds that $\sigma \spr{z - \omega} \leq 2 \spr{z^2 + \exp \spr{- \omega}}$.
\end{lemma}

\begin{proof}
  Let us consider the function $g: z \mapsto \sigma \spr{z - \omega} - \spr{z + \frac{1}{e^{\omega/2}}}^2$. Note that for any $z$, we have $\sigma' \spr{z} = \sigma \spr{z} \sigma \spr{- z}$. Thus, the first two derivatives of $g$ are given by
  %
  \begin{align*}
    g' \spr{z} &= \sigma \spr{z - \omega} \sigma \spr{\omega - z} - 2 \spr{z + \frac{1}{e^{\omega / 2}}}\,, \\
    g'' \spr{z} &= \sigma \spr{z - \omega} \sigma \spr{\omega - z}^2 - \sigma \spr{z - \omega}^2 \sigma \spr{\omega - z} - 2 \,.
  \end{align*}
  %
  Since $\sigma \spr{z} \in \spr{0, 1}$ for any $z$, the second derivative of $g$ is nonpositive, $g'' \spr{z} \leq 0$, and $g$ is concave. By the first order condition, for any $z \geq 0$,
  %
  \begin{equation*}
    g \spr{z} \leq g \spr{0} + g' \spr{0} z\,.
  \end{equation*}
  %
  Furthermore, note that
  %
  \begin{equation*}
    g \spr{0} = \sigma \spr{- \omega} - \frac{1}{e^\omega} = \frac{1}{1 + e^\omega} - \frac{1}{e^\omega} \leq 0\,,
  \end{equation*}
  %
  and
  %
  \begin{align*}
    g' \spr{0} &= \frac{1}{1 + e^\omega} \frac{1}{1 + e^{- \omega}} - \frac{2}{e^{\omega / 2}} \\
    &\leq \frac{1}{1 + e^\omega} - \frac{2}{e^{\omega / 2}} \\
    &\leq - \frac{1}{e^{\omega / 2}} \\
    &\leq 0\,,
  \end{align*}
  %
  where we first used that $e^{- \omega} \geq 0$ for any $\omega \geq 0$ and then that $x + 1 \geq \sqrt{x}$ for any $x \geq 0$. Thus, it holds that $g \spr{z} \leq 0$ for all $z \geq 0$, \ie $\sigma \spr{z - \omega} \leq \spr{z + \frac{1}{e^{\omega / 2}}}^2$. Using $\spr{a+b}^2 \leq 2 \spr{a^2 + b^2}$, it holds that
  %
  \begin{equation*}
    \sigma \spr{z - \omega} \leq 2 \spr{z^2 + e^{- \omega}}\,.
  \end{equation*}
\end{proof}

We present a variant of Lemma~18 of \citet{cassel2024warmupfree} which is valid for $\omega \geq 2$ instead of $\omega \geq 0$, but is sharper by a factor of $2$.
%
\begin{lemma} \label{lem:sigmoid-bound2}
  For all $\omega \geq 2$, it holds that
  \begin{equation*}
    \max_{z \geq 0} z \cdot \sigma \spr{\omega - \alpha z} \leq \frac{\omega}{\alpha}\,.
  \end{equation*}
\end{lemma}

\begin{proof}
  Let $\alpha > 0$, $\omega \geq 2$, and $g: z \geq 0 \mapsto z \cdot \sigma \spr{\omega - \alpha z}$. We recall that the derivative of the sigmoid function is given for any $z$ by $\sigma' \spr{z} = \sigma \spr{z} \sigma \spr{- z}$, and that $\sigma \spr{- z} = 1 - \sigma \spr{z}$. $g$ is twice differentiable. Its first derivative is given by
  %
  \begin{align*}
    g' \spr{z} &= \sigma \spr{\omega - \alpha z} - \alpha z \sigma \spr{\omega - \alpha z} \sbr{1 - \sigma \spr{\omega - \alpha z}} \\
    &= \sigma \spr{\omega - \alpha z} \sbr{1 - \alpha z \spr{1 - \sigma \spr{\omega - \alpha z}}}\,.
  \end{align*}
  %
  We set the derivative to zero and solve the equation to find the critical points. We have
  %
  \begin{align}
    g' \spr{z} = 0 &\quad\text{iff}\quad \alpha z = \frac{1}{1 - \sigma \spr{\omega - \alpha z}} \label{eq:critical-point-prop} \\
    &\quad\text{iff}\quad \alpha z = 1 + e^{\omega - \alpha z} \nonumber \\
    &\quad\text{iff}\quad \spr{\alpha z - 1} e^{\alpha z - 1} = e^{\omega - 1}\,. \nonumber
  \end{align}
  %
  For $x > 0$, the equation $w e^w = x$ has exactly one positive solution $w = W \spr{x}$ which increases with $x$ and where $W$ denotes the Lambert function. Thus, $g' \spr{z} = 0$ if and only if $\alpha z - 1 = W \spr{e^{\omega - 1}}$, \ie $z^\star = \frac{W \spr{e^{\omega - 1}} + 1}{\alpha}$. We check that $z^\star$ is a local maximum. The second derivative of $g$ is given by
  %
  \begin{align*}
    g'' \spr{z} &= - 2 \alpha \sigma \spr{\omega - \alpha z} \sbr{1 - \sigma \spr{\omega - \alpha z}} \\
    &\phantom{=}+ \alpha^2 z \sigma \spr{\omega - \alpha z} \sbr{1 - \sigma \spr{\omega - \alpha z}}^2 \\
    &\phantom{=}- \alpha^2 z \sigma \spr{\omega - \alpha z}^2 \sbr{1 - \sigma \spr{\omega - \alpha z}} \\
    &= - 2 \alpha \sigma \spr{\omega - \alpha z} \sbr{1 - \sigma \spr{\omega - \alpha z}} \\
    &\phantom{=}+ \alpha^2 z \sigma \spr{\omega - \alpha z} \sbr{1 - \sigma \spr{\omega - \alpha z}} \sbr{1 - 2 \sigma \spr{\omega - \alpha z}}\,.
  \end{align*}
  %
  We evaluate it at the critical point $z^\star$ and simplify the expression using Equation~\ref{eq:critical-point-prop}
  %
  \begin{align*}
    g'' \spr{z^\star} &= - 2 \alpha \sigma \spr{\omega - \alpha z^\star} \sbr{1 - \sigma \spr{\omega - \alpha z^\star}} \\
    &\phantom{=}+ \alpha \sigma \spr{\omega - \alpha z^\star} \sbr{1 - 2 \sigma \spr{\omega - \alpha z^\star}} \\
    &= - \alpha \sigma \spr{\omega - \alpha z^\star} \\
    &< 0\,,
  \end{align*}
  %
  thus $z^\star > 0$ is a local maximum. Since $g \spr{0} = 0$, $\lim_{z \rightarrow + \infty} g \spr{z} = 0$, $g \spr{z^\star}$ and $z^\star$ is the only positive critical point, this means $z^\star$ is a global maximum. We evaluate $g$ to get the maximum
  %
  \begin{align*}
    g \spr{z^\star} &= \frac{W \spr{e^{\omega - 1}} + 1}{\alpha} \frac{1}{1 + \exp \spr{W \spr{e^{\omega - 1}}} e^{1 - \omega}} \\
    &= \frac{W \spr{e^{\omega - 1}} + 1}{\alpha} \frac{W \spr{e^{\omega - 1}}}{W \spr{e^{\omega - 1}} + W \spr{e^{\omega - 1}} \exp \spr{W \spr{e^{\omega - 1}}} e^{1 - \omega}} \\
    &= \frac{W \spr{e^{\omega - 1}}}{\alpha}\,,
  \end{align*}
  %
  where we used $W \spr{e^{\omega - 1}} \exp \spr{W \spr{e^{\omega - 1}}} = e^{\omega - 1}$ in the third equality. We now upper bound the Lambert function. Taking the log of the equation that defines it, we have $W \spr{x} = \log x - \log W \spr{x}$ for any $x > 0$. Note that $W \spr{e} = 1$ and that $W$ is increasing, so for any $x > e$, we have $W \spr{x} > 1$ and thus $W \spr{x} < \log x$. Using it on $g \spr{z^\star}$, we further have
  %
  \begin{equation*}
    g \spr{z^\star} \leq \frac{\omega - 1}{\alpha} \leq \frac{\omega}{\alpha}\,,
  \end{equation*}
  %
  where we used $\omega \geq 2$. This concludes the proof.
\end{proof}


\begin{lemma} \label{lem:lse-lipschitz}
  Let $n \in \bbR^n$, and define $\LSE: \bbR^n \rightarrow \bbR$ the function defined for any $x \in \bbR^n$ as
  %
  \begin{equation*}
    \LSE \spr{x} = \log \sum_{i=1}^n e^{x_i}\,.
  \end{equation*}
  %
  Then $\LSE$ is 1-Lipschitz with respect to the norm $\norm{\cdot}_\infty$, \ie for any $x, y \in \bbR^n$,
  %
  \begin{equation*}
    \abs{\LSE \spr{x} - \LSE \spr{y}} \leq \norm{x - y}_\infty\,.
  \end{equation*}
\end{lemma}

\begin{proof}
  For any $i \in \sbr{n}$ and any $x \in \bbR^n$, the gradient of $\LSE$ is given by
  %
  \begin{equation*}
    \nabla \LSE \spr{x} = \frac{e^{x}}{\inp{e^{x}, \bfone}}\,.
  \end{equation*}
  %
  Let $y \in \bbR^n$. By the intermediate mean value theorem, there exists a $z$ on the segment $\sbr{x, y}$ such that
  %
  \begin{align*}
    \abs{\LSE \spr{x} - \LSE \spr{y}} &= \abs{\inp{\nabla \LSE \spr{z}, x - y}} \\
    &\leq \norm{\nabla \LSE \spr{z}}_1 \norm{x - y}_\infty \\
    &= \norm{x - y}_\infty\,,
  \end{align*}
  %
  where the inequality follows from Hölder's inequality.
\end{proof}


\begin{lemma}[\citealp{cohen2019learning}, Lemma~27] \label{lem:det-elliptical-bound}
  If $0 \prec M \preceq N$ then for any vector $v$,
  %
  \begin{equation*}
    \norm{v}_N^2 \leq \frac{\det N}{\det M} \norm{v}_M^2\,.
  \end{equation*}
\end{lemma}


\begin{lemma}[\citealp{sherman2023}, Lemma~15] \label{lemma:beta_bound}
    Let $R, z \geq 1$, then $\beta \geq 2 z \log \spr{R z}$ ensures $\beta \geq z \log \spr{R \beta}$.
\end{lemma}

\begin{lemma}[\citealp{rosenberg2020near}, Lemma~D.4] \label{lem:concentration-ineq-cond-exp}
  Let $\bc{X_k}_{k \in [K]}$ be a sequence of random variables adapted to the filtration $\bc{\mathcal{F}_k}_{k\in[K]}$ and suppose that $0 \leq X_k \leq X_{\max}$ almost surely. Then, with probability at least $1-\delta$, the following holds for all $k \geq 1$ simultaneously
  \begin{equation*}
    \sum^K_{k=1} \mathbb{E}\bs{X_k | \mathcal{F}_{k-1}} \leq 2 \sum^K_{k=1} X_k + 4 X_{\max} \log \frac{2 K}{\delta}\,.
  \end{equation*}
\end{lemma}

\begin{lemma}[\citealp{jin2019provably}, Lemma~D.2] \label{lem:bound-elliptical-potential}
  Let $\scbr{\phi_t}_{t \geq 0}$ be a bounded sequence in $\bbR^d$ satisfying $\sup_{t \geq 0} \norm{\phi_t} \leq 1$. Let $\Lambda_0 \in \bbR^{d \times d}$ be a positive definite matrix. For any $t \geq 0$, we define $\Lambda_t = \Lambda_0 + \sum_{j=1}^t \phi_j \phi_j\transpose$. Then, if the smallest eigenvalue of $\Lambda_0$ satisfies $\lambda_{\mathrm{min}} \spr{\Lambda_0} \geq 1$, we have
  \begin{equation*}
    \sum_{j=1}^t \phi_j \Lambda_{j-1}^{-1} \phi_j \leq 2 \log \spr{\frac{\det \Lambda_t}{\det \Lambda_0}}\,.
  \end{equation*}
\end{lemma}

% \crefalias{section}{appendix} % uncomment if you are using cleveref

\end{document}
