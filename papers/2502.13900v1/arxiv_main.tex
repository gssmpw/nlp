\documentclass[12pt]{settings/colt2025} % Anonymized submission
%\documentclass[final,12pt]{colt2025} % Include author names

% The following packages will be automatically loaded:
% amsmath, amssymb, natbib, graphicx, url, algorithm2e

\title[Optimistically Optimistic Exploration for Efficient Infinite-Horizon RL]{Optimistically Optimistic 
Exploration for Provably Efficient Infinite-Horizon Reinforcement and Imitation Learning}
\usepackage{times}
% Use \Name{Author Name} to specify the name.
% If the surname contains spaces, enclose the surname
% in braces, e.g. \Name{John {Smith Jones}} similarly
% if the name has a "von" part, e.g \Name{Jane {de Winter}}.
% If the first letter in the forenames is a diacritic
% enclose the diacritic in braces, e.g. \Name{{\'E}louise Smith}

% Two authors with the same address
% \coltauthor{\Name{Author Name1} \Email{abc@sample.com}\and
%  \Name{Author Name2} \Email{xyz@sample.com}\\
%  \addr Address}

% Three or more authors with the same address:
% \coltauthor{\Name{Author Name1} \Email{an1@sample.com}\\
%  \Name{Author Name2} \Email{an2@sample.com}\\
%  \Name{Author Name3} \Email{an3@sample.com}\\
%  \addr Address}

% Authors with different addresses:
\author[Moulin, Neu and Viano]{%
 \Name{Antoine Moulin} \Email{antoine.moulin@upf.edu}\\
 \addr Universitat Pompeu Fabra, Barcelona, Spain%
 \AND
 \Name{Gergely Neu} \Email{gergely.neu@gmail.com}\\
 \addr Universitat Pompeu Fabra, Barcelona, Spain
 \AND
 \Name{Luca Viano} \Email{luca.viano@epfl.ch}\\
 \addr EPFL, Lausanne, Switzerland
}

% custom
%\usepackage[unicode,psdextra]{hyperref}
\usepackage[capitalize,noabbrev]{cleveref}
% Parentheses
\newcommand{\bc}[1]{\left\{{#1}\right\}}
\newcommand{\brr}[1]{\left({#1}\right)}
\newcommand{\bs}[1]{\left[{#1}\right]}
\newcommand{\ceil}[1]{\left\lceil #1 \right\rceil}
\newcommand{\floor}[1]{\left\lfloor #1 \right\rfloor}
\newcommand{\bsd}[1]{\left\llbracket{#1}\right\rrbracket}
\newcommand{\ip}[2]{\left\langle{#1},{#2}\right\rangle}
\newcommand{\inner}[1]{\langle#1\rangle}
% Statistics notation
% Probabilities
\renewcommand{\P}[1]{\mathbb{P}\bs{{#1}}}
\newcommand{\Pp}[2]{\underset{#1}{\mathbb{P}}\bs{{#2}}}
\newcommand{\Ppp}[3]{\mathbb{P}_{#1}^{#2}\bs{{#3}}}
% Expectations
\newcommand{\Ee}[2]{\underset{#1}{\mathbb{E}}\bs{{#2}}}
\newcommand{\Eee}[2]{\mathbb{E}_{#1}\bs{{#2}}}
\newcommand{\sspace}{\mcf{X}}     % state space
\newcommand{\aspace}{\mcf{A}}     % action space
\newcommand{\wspace}{\mcf{W}}     % cost weights space
\newcommand{\histspace}{\mcf{H}}  % history space
% argmin and argmax
\DeclareMathOperator*{\argmin}{arg\,min}
\DeclareMathOperator*{\argmax}{arg\,max}

\usepackage{algorithm}
\usepackage{algorithmic}
\usepackage{multirow}
\newcounter{protocol}
\makeatletter
\newenvironment{protocol}[1][htb]{%
  \let\c@algorithm\c@protocol
  \renewcommand{\ALG@name}{Protocol}% Update algorithm name
  \begin{algorithm}[#1]%
  }{\end{algorithm}
}
\usepackage{amssymb}
\usepackage{amsfonts}
\usepackage{graphics}
\usepackage{graphicx}
\usepackage{setspace}
%\usepackage{subcaption}
%\usepackage{selectp}
%\outputonly{1-18}
% \newtheorem{theorem}{\protect\theoremname}
%   \newtheorem{lemma}{\protect\lemmaname}
%   \newtheorem{corollary}{\protect\corrolaryname}
%   \newtheorem{defn}{\protect\definitionname}
%   \newtheorem{proposition}{\protect\propositionname}
%   \newtheorem{example}{\protect\examplename}
%   \newtheorem{remark}{Remark}
%   \newtheorem{conditions}{\protect\conditionsname}
%     \newtheorem{assumption}{\protect\assumptionname}
    
%     \newenvironment{proofof}[1]{\begin{proof}[{#1}]}{\end{proof}}
    
% \providecommand{\definitionname}{Definition}
% \providecommand{\examplename}{Example}
% \providecommand{\lemmaname}{Lemma}
% \providecommand{\corrolaryname}{Corollary}
% \providecommand{\propositionname}{Proposition}
% \providecommand{\conditionsname}{Conditions}
% \providecommand{\theoremname}{Theorem}
% \providecommand{\assumptionname}{Assumption}

\def \M{\mathcal{M}}
\def \S{\mathcal{S}}


% to compile a preprint version, e.g., for submission to arXiv, add add the
% [preprint] option:
%     \usepackage[preprint]{neurips_2022}


% to compile a camera-ready version, add the [final] option, e.g.:
%     \usepackage[final]{neurips_2022}


% to avoid loading the natbib package, add option nonatbib:
%    \usepackage[nonatbib]{neurips_2022}

%commands from AK
%====================================================================================================================================================================================
%new
%state space S 
\newcommand{\qv}{\mbf{q}}
\newcommand{\dv}{\mbf{d}}
\newcommand{\dve}{\mbf{d}^E}
\newcommand{\trans}{\intercal}
\newcommand{\qval}{\mbf{Q}}
\newcommand{\FEV}[1]{\mbs{\rho}_{\phim}(#1)}
\newcommand{\EFEV}[1]{\mbs{\rho}_{\phim}(\widehat{#1})}
%==============================================================================
%old
\newcommand{\mcf}{\mathcal}
\newcommand{\mbf}{\mathbf}
\newcommand{\mbb}{\mathbb}

\newcommand{\innerprod}[2]{\left\langle{#1},{#2}\right\rangle}

\newcommand{\expert}{{\pi_E}}
\newcommand{\experti}{\pi_E^i}
\newcommand{\apprentice}{\pi_{\textup{A}}}
\newcommand{\mbs}{\boldsymbol}
\newcommand{\cost}{r}
\newcommand{\true}{r_{\textup{true}}}
\newcommand{\weight}{w}
\newcommand{\FRAalg}{\texttt{\textbf{FRA-IL}}\xspace}
\newcommand{\phim}{\Phi}
\newcommand{\initial}{\nu_0}


\newcommand{\wtrue}{\weight_{\textup{true}}}
 %========================================================
\newcommand{\lv}[1]{%
    \ifmmode
    \text{\textcolor{orange}{[LV: #1]}}
    \else
    \textcolor{orange}{[LV: #1]}
    \fi
}
\usepackage[utf8]{inputenc} % allow utf-8 input
\usepackage[T1]{fontenc}    % use 8-bit T1 fonts      
\usepackage{url}            % simple URL typesetting
\usepackage{booktabs}       % professional-quality tables
\usepackage{amsfonts}       % blackboard math symbols
\usepackage{nicefrac}       % compact symbols for 1/2, etc.
\usepackage{microtype}      % microtypography
\usepackage{xcolor}         % colors
\usepackage{dsfont}
% \usepackage{xcolor}
\usepackage{pifont}
\newcommand{\greentick}{\textcolor{green}{\ding{51}}}
\newcommand{\redcross}{\textcolor{red}{\ding{55}}}
\newtheorem{assumption}{\protect\assumptionname}
\providecommand{\assumptionname}{Assumption}

\usepackage{mathtools}% http://ctan.org/pkg/mathtools
\makeatletter
\newcommand{\mytag}[2]{%
  \text{#1}%
  \@bsphack
  \protected@write\@auxout{}%
         {\string\newlabel{#2}{{#1}{\thepage}}}%
  \@esphack
}
\makeatother

\usepackage{thm-restate}
\newtheorem{Lem}[theorem]{Lemma}


\newcommand{\grg}[1]{\textcolor{red}{[\textbf{Grg:} #1]}}
\newcommand{\dd}{\mathrm{d}}

\newcommand{\raviUCB}{\texttt{\textbf{RAVI-UCB}}\xspace}
\newcommand{\RMAXalg}{\texttt{\textbf{Rmax}}\xspace}
\newcommand{\algname}{\texttt{\textbf{Rmax-RAVI-UCB}}\xspace}

\newcommand{\RMAX}{R_{\max}}

\newcommand{\D}[2]{D\pa{#1\middle\|#2}}
\newcommand{\DDKL}[2]{\mathcal{D}_{\text{KL}}\pa{#1\middle\|#2}}
%\newcommand{\CB}{\mbox{CB}}
\newcommand{\CC}{\mathcal{C}}
\newcommand{\TT}{\mathcal{T}}
\newcommand{\GG}{\mathcal{G}}
\newcommand{\tGG}{\wt{\GG}}
\newcommand{\ttheta}{\wt{\theta}}
\newcommand{\btheta}{\overline{\theta}}
\newcommand{\tlambda}{\wt{\lambda}}
\newcommand{\blambda}{\overline{\lambda}}
\newcommand{\bmu}{\overline{\mu}}
\newcommand{\tQ}{\wt{Q}}
\newcommand{\tV}{\wt{V}}
\newcommand{\tpi}{\wt{\pi}}
\newcommand{\td}{\wt{d}}
\newcommand{\tr}{r^+}
\newcommand{\thr}{{\hr}^+}
\newcommand{\trho}{\wt{\rho}}
\newcommand{\trunc}[2]{\left[#1\right]_{#2}}


\newcommand{\Plin}{\mathcal{P}_{\Phi}}


\newcommand{\Hil}{\mathcal{H}}
\newcommand{\X}{\mathcal{X}}
\newcommand{\F}{\mathcal{F}}
\newcommand{\A}{\mathcal{A}}
\newcommand{\N}{\mathcal{N}}
\newcommand{\LL}{\mathcal{L}}
\newcommand{\real}{\mathbb{R}}
\newcommand{\Dw}{\mathcal{D}}
%\newcommand{\Sw}{\mathcal{S}}
\newcommand{\Rw}{\mathcal{R}}
\newcommand{\HH}{\mathcal{H}}
\newcommand{\DD}{\mathcal{D}}
\newcommand{\OO}{\mathcal{O}}
\newcommand{\tOO}{\wt{\OO}}
\newcommand{\trace}[1]{\mbox{tr}\left(#1\right)}
\newcommand{\II}[1]{\mathbb{I}_{\left\{#1\right\}}}
\newcommand{\PP}[1]{\mathbb{P}\left[#1\right]}
\newcommand{\PPZ}[1]{\mathbb{P}_Z\left[#1\right]}
\newcommand{\EEZ}[1]{\mathbb{E}_Z\left[#1\right]}
\newcommand{\EE}[1]{\mathbb{E}\left[#1\right]}
\newcommand{\EEb}[1]{\mathbb{E}\bigl[#1\bigr]}
\newcommand{\EEtb}[1]{\mathbb{E}_t\bigl[#1\bigr]}
\newcommand{\EXP}{\mathbb{E}}
\newcommand{\EEs}[2]{\mathbb{E}_{#2}\left[#1\right]}
\newcommand{\EEu}[1]{\mathbb{E}_{\u}\left[#1\right]}
\newcommand{\EEst}[1]{\mathbb{E}_{*}\left[#1\right]}
\newcommand{\EEo}[1]{\mathbb{E}_{0}\left[#1\right]}
\newcommand{\PPt}[1]{\mathbb{P}_t\left[#1\right]}
\newcommand{\EEt}[1]{\mathbb{E}_t\left[#1\right]}
\newcommand{\EEtt}[1]{\mathbb{E}_{t-1}\left[#1\right]}
\newcommand{\PPi}[1]{\mathbb{P}_i\left[#1\right]}
\newcommand{\EEi}[1]{\mathbb{E}_i\left[#1\right]}
\newcommand{\PPc}[2]{\mathbb{P}\left[#1\left|#2\right.\right]}
\newcommand{\PPs}[2]{\mathbb{P}_{#2}\left[#1\right]}
\newcommand{\PPcs}[3]{\mathbb{P}_{#3}\left[#1\left|#2\right.\right]}
\newcommand{\PPct}[2]{\mathbb{P}_t\left[#1\left|#2\right.\right]}
\newcommand{\PPcc}[2]{\mathbb{P}\left[\left.#1\right|#2\right]}
\newcommand{\PPcct}[2]{\mathbb{P}_t\left[\left.#1\right|#2\right]}
\newcommand{\PPcci}[2]{\mathbb{P}_i\left[\left.#1\right|#2\right]}
\newcommand{\EEc}[2]{\mathbb{E}\left[#1\left|#2\right.\right]}
\newcommand{\EEcc}[2]{\mathbb{E}\left[\left.#1\right|#2\right]}
\newcommand{\EEcs}[3]{\mathbb{E}_{#3}\left[\left.#1\right|#2\right]}
\newcommand{\EEcct}[2]{\mathbb{E}_t\left[\left.#1\right|#2\right]}
\newcommand{\EEcctt}[2]{\mathbb{E}_{t-1}\left[\left.#1\right|#2\right]}
\newcommand{\EEcci}[2]{\mathbb{E}_i\left[\left.#1\right|#2\right]}
\renewcommand{\th}{\ensuremath{^{\mathrm{th}}}}
\def\argmin{\mathop{\mbox{ arg\,min}}}
\def\argmax{\mathop{\mbox{ arg\,max}}}
\newcommand{\ra}{\rightarrow}

\newcommand{\bone}{\bm{1}}

\newcommand{\siprod}[2]{\langle#1,#2\rangle}
\newcommand{\iprod}[2]{\left\langle#1,#2\right\rangle}
\newcommand{\biprod}[2]{\bigl\langle#1,#2\bigr\rangle}
\newcommand{\Biprod}[2]{\Bigl\langle#1,#2\Bigr\rangle}
\newcommand{\norm}[1]{\left\|#1\right\|}
\newcommand{\onenorm}[1]{\norm{#1}_1}
\newcommand{\twonorm}[1]{\norm{#1}_2}
\newcommand{\infnorm}[1]{\norm{#1}_\infty}
\newcommand{\spannorm}[1]{\norm{#1}_{\text{sp}}}
\newcommand{\opnorm}[1]{\norm{#1}_{\text{op}}}
\newcommand{\Hnorm}[1]{\norm{#1}_{\HH}}
\newcommand{\ev}[1]{\left\{#1\right\}}
\newcommand{\pa}[1]{\left(#1\right)}
\newcommand{\abs}[1]{\left|#1\right|}
\newcommand{\bpa}[1]{\bigl(#1\bigr)}
\newcommand{\Bpa}[1]{\Bigl(#1\Bigr)}
\newcommand{\BPA}[1]{\Biggl(#1\Biggr)}
\newcommand{\sign}{\mbox{sign}}

\newcommand{\wh}{\widehat}
\newcommand{\wt}{\widetilde}


\newcommand{\loss}{\ell}
\newcommand{\hloss}{\wh{\loss}}


\newcommand{\lambdamin}{\lambda_{\min}}
\newcommand{\lambdamax}{\lambda_{\max}}
\newcommand{\hr}{\wh{r}}
\newcommand{\hR}{\wh{R}}
\newcommand{\tx}{\wt{x}}
\newcommand{\tX}{\wt{X}}
\newcommand{\htheta}{\wh{\theta}}
\newcommand{\hq}{\wh{q}}
\newcommand{\hv}{\wh{v}}
\newcommand{\hTheta}{\wh{\Theta}}
\newcommand{\hf}{\wh{f}}
\newcommand{\hs}{\wh{s}}
\newcommand{\bA}{\overline{A}}
\newcommand{\bW}{\overline{\W}}
\newcommand{\deff}{d_{\text{eff}}}

\newcommand{\Sp}{\Sigma^+}

\newcommand{\Cinf}{C_\infty}
\newcommand{\tw}{\wt{w}}
\newcommand{\br}{\overline{r}}
\newcommand{\hM}{\wh{M}}
\newcommand{\hT}{\wh{T}}
\newcommand{\tM}{\wt{M}}
\newcommand{\bw}{\overline{w}}
\newcommand{\hX}{\wh{X}}
\newcommand{\hZ}{\wh{Z}}
\newcommand{\tZ}{\wt{Z}}
\newcommand{\hmu}{\wh{\mu}}
\newcommand{\tmu}{\wt{\mu}}
\newcommand{\hP}{\wh{P}}
\newcommand{\tP}{P^+}
\newcommand{\thP}{{\hP}^+}
\newcommand{\hS}{\wh{\Sigma}}
\newcommand{\hw}{\wh{w}}
\newcommand{\hx}{\wh{\boldsymbol{x}}}
\newcommand{\hQ}{\wh{Q}}
\newcommand{\transpose}{^\mathsf{\scriptscriptstyle T}}

%\usepackage{todonotes}
\definecolor{PalePurp}{rgb}{0.66,0.57,0.66}
%\newcommand{\todoG}[1]{\todo[color=PalePurp!30]{#1}}
%\newcommand{\todoGI}[1]{\todo[inline,color=PalePurp!30]{#1}}
\newcommand{\redd}[1]{\textcolor{red}{#1}}
\newcommand{\bblue}[1]{\textcolor{blue}{#1}}

%\newcommand{\regret}{\mathrm{regret}}
\newcommand{\regret}{\mathfrak{R}_T}
\newcommand{\regretK}{\mathfrak{R}_K}

%\newcommand{\qed}{\hfill\BlackBox\\[2mm]}

\newcommand{\R}{\boldsymbol{R}}


\newcommand{\hL}{\wh{L}}

\newcommand{\expexpexp}{\textsc{Exp3}\xspace}
\newcommand{\linexprl}{\textsc{LinExp3RL}\xspace}
\newcommand{\linexpreal}{\textsc{RealLinExp3}\xspace}
\newcommand{\linexprobust}{\textsc{RobustLinExp3}\xspace}
\newcommand{\linucb}{\textsc{LinUCB}\xspace}

\newcommand{\antoine}[1]{%
    \ifmmode
    \text{\textcolor{blue}{[Antoine: #1]}}
    \else
    \textcolor{blue}{[Antoine: #1]}
    \fi
}

\newcommand{\tmpassumption}[1]{%
    \ifmmode
    \text{\textcolor{orange}{[Assumption: #1]}}
    \else
    \textcolor{orange}{[Assumption: #1]}
    \fi
}

% common abbreviations
\makeatletter
\DeclareRobustCommand\onedot{\futurelet\@let@token\@onedot}
\def\@onedot{\ifx\@let@token.\else.\null\fi\xspace}

\def\cf{\textit{c.f}\onedot}
\def\eg{\textit{e.g}\onedot}
\def\etal{\textit{et al}\onedot}
\def\etc{\textit{etc}\onedot}
\def\ie{\textit{i.e}\onedot}
\def\wrt{i.i.d\onedot}
\def\vs{\textit{vs}\onedot}
\def\wrt{w.r.t\onedot}
\makeatother

% surrounding symbols
\newcommand{\spr}[1]{\left( #1 \right)}
\newcommand{\sbr}[1]{\left[ #1 \right]}
\newcommand{\sdbr}[1]{\left[\!\left[ #1 \right]\!\right]}
\newcommand{\scbr}[1]{\left\{ #1 \right\}}
\newcommand{\inp}[1]{\left\langle #1 \right\rangle}

% basic algebra symbols
\newcommand{\sumiK}{\sum_{i=1}^K}
\newcommand{\sumjK}{\sum_{j=1}^K}
\newcommand{\sumkK}{\sum_{k=1}^K}
\newcommand{\sumtT}{\sum_{t=1}^T}
\newcommand{\sumst}{\sum_{s=1}^t}
\newcommand{\sumtinfty}{\sum_{t=0}^\infty}
\newcommand{\sumsinfty}{\sum_{s=0}^\infty}

% \newcommand{\abs}[1]{\left| #1 \right|}
\newcommand*\diff{\mathop{}\!\mathrm{d}}
\usepackage{mathtools} % for \mathclap
\newcommand{\given}{\mathrel{}\middle|\mathrel{}}
\newcommand{\KL}{\mathcal{D}_{\text{KL}}}
% \newcommand{\norm}[1]{\left\| #1 \right\|}
% \newcommand{{\transpose}}{^\mathsf{\scriptscriptstyle T}}
\newcommand{\upplus}{\mathsf{\scriptscriptstyle +}} % superscript plus

% \newcommand{\wh}[1]{\widehat{#1}}
\usepackage{mathtools}  % for '\mathrlap' command (necessary)
\newcommand{\custombar}[3]{%
    \mathrlap{\hspace{#2}\overline{\scalebox{#1}[1]{\phantom{\ensuremath{#3}}}}}\ensuremath{#3}
} % random example: \custombar{0.8}{.5pt}{y}

% blackboard bold letters
\newcommand{\bbC}{\mathbb{C}}
\newcommand{\bbE}{\mathbb{E}}
\newcommand{\bbF}{\mathbb{F}}
\newcommand{\bbI}{\mathbb{I}}
\newcommand{\bbN}{\mathbb{N}}
\newcommand{\bbP}{\mathbb{P}}
\newcommand{\bbQ}{\mathbb{Q}}
\newcommand{\bbR}{\mathbb{R}}
\newcommand{\bbRnnK}{\mathbb{R}_{\geq 0}^K}
\newcommand{\bbRpK}{\mathbb{R}_{>0}^K}
\newcommand{\bbV}{\mathbb{V}}
\newcommand{\bbZ}{\mathbb{Z}}

% bold letters
\newcommand{\bfone}{\mathbf{1}}
\newcommand{\bfe}{\mathbf{e}}
\newcommand{\bfu}{\mathbf{u}}
\newcommand{\bfv}{\mathbf{v}}
\newcommand{\bfw}{\mathbf{w}}
\newcommand{\bfx}{\mathbf{x}}
\newcommand{\bfy}{\mathbf{y}}
\newcommand{\bfyt}{\mathbf{y}_t}

% caligraphic letters
\newcommand{\cA}{\mathcal{A}}
\newcommand{\cB}{\mathcal{B}}
\newcommand{\cC}{\mathcal{C}}
\newcommand{\cD}{\mathcal{D}}
\newcommand{\cE}{\mathcal{E}}
\newcommand{\cF}{\mathcal{F}}
\newcommand{\cG}{\mathcal{G}}
\newcommand{\cH}{\mathcal{H}}
\newcommand{\cI}{\mathcal{I}}
\newcommand{\cK}{\mathcal{K}}
\newcommand{\cL}{\mathcal{L}}
\newcommand{\cM}{\mathcal{M}}
\newcommand{\cN}{\mathcal{N}}
\newcommand{\cO}{\mathcal{O}}
\newcommand{\cP}{\mathcal{P}}
\newcommand{\cQ}{\mathcal{Q}}
\newcommand{\cR}{\mathcal{R}}
\newcommand{\cS}{\mathcal{S}}
\newcommand{\cT}{\mathcal{T}}
\newcommand{\cU}{\mathcal{U}}
\newcommand{\cV}{\mathcal{V}}
\newcommand{\cW}{\mathcal{W}}
\newcommand{\cX}{\mathcal{X}}
\newcommand{\cY}{\mathcal{Y}}
\newcommand{\cZ}{\mathcal{Z}}

% fraktur letters
\newcommand{\fkB}{\mathfrak{B}}
\newcommand{\fkR}{\mathfrak{R}}

% math operators
% \DeclareMathOperator*{\argmax}{arg\,max}
% \DeclareMathOperator*{\argmin}{arg\,min}
% \DeclareMathOperator{\opE}{E}
% \DeclareMathOperator{\opP}{P}
% \DeclareMathOperator{\ophP}{\widehat{P}}
% \DeclareMathOperator{\ophPt}{\widehat{P}_t}
% \DeclareMathOperator{\ophPk}{\widehat{P}_k}
% \DeclareMathOperator{\opPplus}{P_t^\mathsf{\scriptscriptstyle +}}
% \DeclareMathOperator{\opPplusk}{P_k^\mathsf{\scriptscriptstyle +}}
% \DeclareMathOperator{\ophPplus}{\widehat{P}_t^\mathsf{\scriptscriptstyle +}}
% \DeclareMathOperator{\ophPplusk}{\widehat{P}_k^\mathsf{\scriptscriptstyle +}}

\newcommand{\opE}{E}
\newcommand{\opP}{P}
\newcommand{\ophP}{\widehat{P}}
\newcommand{\ophPt}{\widehat{P}_t}
\newcommand{\ophPk}{\widehat{P}_k}
\newcommand{\opPplus}{P_t^\mathsf{\scriptscriptstyle +}}
\newcommand{\opPplusk}{P_k^\mathsf{\scriptscriptstyle +}}
\newcommand{\ophPplus}{\widehat{P}_t^\mathsf{\scriptscriptstyle +}}
\newcommand{\ophPplusk}{\widehat{P}_k^\mathsf{\scriptscriptstyle +}}



\DeclareMathOperator{\co}{co}
\DeclareMathOperator{\optrace}{trace}
\DeclareMathOperator{\Diag}{Diag}  % diagonal matrices
\DeclareMathOperator{\diam}{diam}  % diameter
\DeclareMathOperator{\dom}{dom}  % domain
\DeclareMathOperator{\interior}{int}
\DeclareMathOperator{\Lip}{Lip} % lipschitz
\DeclareMathOperator{\scvx}{sc}  % strong convexity
% \DeclareMathOperator{\regret}{\texttt{regret}}
\DeclareMathOperator{\regretplus}{\mathfrak{R}_T^\mathsf{\scriptscriptstyle +}}
\DeclareMathOperator{\regretKplus}{\mathfrak{R}_K^\mathsf{\scriptscriptstyle +}}
\DeclareMathOperator{\regretIL}{\mathfrak{R}_K^\mathrm{I\hspace{.05em}L}}%\regretK^{\mathrm{I\hspace{.05em}L}}}
% \DeclareMathOperator{\sign}{sign}
% \DeclareMathOperator{\tr}{tr}  % trace
\DeclareMathOperator{\CB}{CB}
\DeclareMathOperator{\kl}{kl}

% others
\newcommand{\simplex}{\Delta}
\newcommand{\piunif}{\pi_{\texttt{unif}}}
\newcommand{\QMAX}{Q_{\max}}
%\newcommand{\VMAX}{V_{\max}}
\newcommand{\VMAX}{Q_{\max}}
\newcommand{\LMAX}{L_{\max}}
\newcommand{\WMAX}{W_{\max}}
\newcommand{\reset}{reset\xspace}
\DeclareMathOperator{\LSE}{LSE}

%
% --- inline annotations
%
\newcommand{\red}[1]{{\color{red}#1}}
\newcommand{\todo}[1]{{\color{red}#1}}
\newcommand{\TODO}[1]{\textbf{\color{red}[TODO: #1]}}
% --- disable by uncommenting  
% \renewcommand{\TODO}[1]{}
% \renewcommand{\todo}[1]{#1}



\newcommand{\VLM}{LVLM\xspace} 
\newcommand{\ours}{PeKit\xspace}
\newcommand{\yollava}{Yo’LLaVA\xspace}

\newcommand{\thisismy}{This-Is-My-Img\xspace}
\newcommand{\myparagraph}[1]{\noindent\textbf{#1}}
\newcommand{\vdoro}[1]{{\color[rgb]{0.4, 0.18, 0.78} {[V] #1}}}
% --- disable by uncommenting  
% \renewcommand{\TODO}[1]{}
% \renewcommand{\todo}[1]{#1}
\usepackage{slashbox}
% Vectors
\newcommand{\bB}{\mathcal{B}}
\newcommand{\bw}{\mathbf{w}}
\newcommand{\bs}{\mathbf{s}}
\newcommand{\bo}{\mathbf{o}}
\newcommand{\bn}{\mathbf{n}}
\newcommand{\bc}{\mathbf{c}}
\newcommand{\bp}{\mathbf{p}}
\newcommand{\bS}{\mathbf{S}}
\newcommand{\bk}{\mathbf{k}}
\newcommand{\bmu}{\boldsymbol{\mu}}
\newcommand{\bx}{\mathbf{x}}
\newcommand{\bg}{\mathbf{g}}
\newcommand{\be}{\mathbf{e}}
\newcommand{\bX}{\mathbf{X}}
\newcommand{\by}{\mathbf{y}}
\newcommand{\bv}{\mathbf{v}}
\newcommand{\bz}{\mathbf{z}}
\newcommand{\bq}{\mathbf{q}}
\newcommand{\bff}{\mathbf{f}}
\newcommand{\bu}{\mathbf{u}}
\newcommand{\bh}{\mathbf{h}}
\newcommand{\bb}{\mathbf{b}}

\newcommand{\rone}{\textcolor{green}{R1}}
\newcommand{\rtwo}{\textcolor{orange}{R2}}
\newcommand{\rthree}{\textcolor{red}{R3}}
\usepackage{amsmath}
%\usepackage{arydshln}
\DeclareMathOperator{\similarity}{sim}
\DeclareMathOperator{\AvgPool}{AvgPool}

\newcommand{\argmax}{\mathop{\mathrm{argmax}}}     



\newcommand{\MainAlg}{\texttt{MainAlg}}
\renewcommand{\phi}{\varphi}

\newcommand{\rewardbias}{\texttt{reward-bias}}
\newcommand{\modelbias}{\texttt{model-bias}}


% tikz figures
\usepackage{tikz}
\usetikzlibrary{arrows.meta}
\usetikzlibrary{decorations.pathreplacing}
\usetikzlibrary{positioning}
\usetikzlibrary{automata}

% algorithm comments
\usepackage{transparent}
\newcommand{\algcomment}[1]{\textcolor{blue!70!black}{\transparent{0.7}\small{\texttt{\textbf{\#\hspace{2pt}#1}}}}}
\newcommand{\algcommentlight}[1]{\textcolor{blue!70!black}{\transparent{0.5}\footnotesize{\texttt{\textbf{\#\hspace{2pt}#1}}}}}
\newcommand{\algcommentbig}[1]{\textcolor{blue!70!black}{\footnotesize{\texttt{\textbf{/* #1~*/}}}}}
\newcommand{\algcommentbiglight}[1]{\textcolor{blue!70!black}{\transparent{0.5}\footnotesize{\texttt{\textbf{/* #1~*/}}}}\vspace{2pt}}
\newcommand{\algcolor}[1]{\textcolor{blue!70!black}{#1}}
\newcommand{\algspace}{\hspace{\algorithmicindent}}


% not sure what this is
\newlength{\minipagewidth}
\setlength{\minipagewidth}{\linewidth}
\setlength{\fboxsep}{3mm}
\addtolength{\minipagewidth}{-\fboxrule}
\addtolength{\minipagewidth}{-\fboxrule}
\addtolength{\minipagewidth}{-\fboxsep}
\addtolength{\minipagewidth}{-\fboxsep}
\newcommand{\bookbox}[1]{
\par\medskip\noindent
\framebox[\linewidth]{
\begin{minipage}{\minipagewidth}
{#1}
\end{minipage} } \par\medskip }


% for table of contents appendix
\usepackage{tocloft}
\addtocontents{toc}{\protect\setcounter{tocdepth}{0}}

\usepackage{array}
\newcolumntype{P}[1]{>{\centering\arraybackslash}p{#1}}
\newcolumntype{M}[1]{>{\centering\arraybackslash}m{#1}}

\begin{document}

\maketitle

\begin{abstract}%
  We study the problem of reinforcement learning in infinite-horizon discounted linear Markov decision processes (MDPs), and propose the first computationally efficient algorithm achieving near-optimal regret guarantees in this setting. Our main idea is to combine two classic techniques for optimistic exploration: additive exploration bonuses applied to the reward function, and artificial transitions made to an absorbing state with maximal return. We show that, combined with a regularized approximate dynamic-programming scheme, the resulting algorithm achieves a regret of order $\tilde{\mathcal{O}} (\sqrt{d^3 (1 - \gamma)^{- 7 / 2} T})$, where $T$ is the total number of sample transitions, $\gamma \in (0,1)$ is the discount factor, and $d$ is the feature dimensionality. The results continue to hold against adversarial reward sequences, enabling application of our method to the problem of imitation learning in linear MDPs, where we achieve state-of-the-art results.
\end{abstract}

\begin{keywords}%
  Optimistic exploration, discounted MDPs, linear MDPs, imitation learning%
\end{keywords}

\section{Introduction}
\label{sec:introduction}
The business processes of organizations are experiencing ever-increasing complexity due to the large amount of data, high number of users, and high-tech devices involved \cite{martin2021pmopportunitieschallenges, beerepoot2023biggestbpmproblems}. This complexity may cause business processes to deviate from normal control flow due to unforeseen and disruptive anomalies \cite{adams2023proceddsriftdetection}. These control-flow anomalies manifest as unknown, skipped, and wrongly-ordered activities in the traces of event logs monitored from the execution of business processes \cite{ko2023adsystematicreview}. For the sake of clarity, let us consider an illustrative example of such anomalies. Figure \ref{FP_ANOMALIES} shows a so-called event log footprint, which captures the control flow relations of four activities of a hypothetical event log. In particular, this footprint captures the control-flow relations between activities \texttt{a}, \texttt{b}, \texttt{c} and \texttt{d}. These are the causal ($\rightarrow$) relation, concurrent ($\parallel$) relation, and other ($\#$) relations such as exclusivity or non-local dependency \cite{aalst2022pmhandbook}. In addition, on the right are six traces, of which five exhibit skipped, wrongly-ordered and unknown control-flow anomalies. For example, $\langle$\texttt{a b d}$\rangle$ has a skipped activity, which is \texttt{c}. Because of this skipped activity, the control-flow relation \texttt{b}$\,\#\,$\texttt{d} is violated, since \texttt{d} directly follows \texttt{b} in the anomalous trace.
\begin{figure}[!t]
\centering
\includegraphics[width=0.9\columnwidth]{images/FP_ANOMALIES.png}
\caption{An example event log footprint with six traces, of which five exhibit control-flow anomalies.}
\label{FP_ANOMALIES}
\end{figure}

\subsection{Control-flow anomaly detection}
Control-flow anomaly detection techniques aim to characterize the normal control flow from event logs and verify whether these deviations occur in new event logs \cite{ko2023adsystematicreview}. To develop control-flow anomaly detection techniques, \revision{process mining} has seen widespread adoption owing to process discovery and \revision{conformance checking}. On the one hand, process discovery is a set of algorithms that encode control-flow relations as a set of model elements and constraints according to a given modeling formalism \cite{aalst2022pmhandbook}; hereafter, we refer to the Petri net, a widespread modeling formalism. On the other hand, \revision{conformance checking} is an explainable set of algorithms that allows linking any deviations with the reference Petri net and providing the fitness measure, namely a measure of how much the Petri net fits the new event log \cite{aalst2022pmhandbook}. Many control-flow anomaly detection techniques based on \revision{conformance checking} (hereafter, \revision{conformance checking}-based techniques) use the fitness measure to determine whether an event log is anomalous \cite{bezerra2009pmad, bezerra2013adlogspais, myers2018icsadpm, pecchia2020applicationfailuresanalysispm}. 

The scientific literature also includes many \revision{conformance checking}-independent techniques for control-flow anomaly detection that combine specific types of trace encodings with machine/deep learning \cite{ko2023adsystematicreview, tavares2023pmtraceencoding}. Whereas these techniques are very effective, their explainability is challenging due to both the type of trace encoding employed and the machine/deep learning model used \cite{rawal2022trustworthyaiadvances,li2023explainablead}. Hence, in the following, we focus on the shortcomings of \revision{conformance checking}-based techniques to investigate whether it is possible to support the development of competitive control-flow anomaly detection techniques while maintaining the explainable nature of \revision{conformance checking}.
\begin{figure}[!t]
\centering
\includegraphics[width=\columnwidth]{images/HIGH_LEVEL_VIEW.png}
\caption{A high-level view of the proposed framework for combining \revision{process mining}-based feature extraction with dimensionality reduction for control-flow anomaly detection.}
\label{HIGH_LEVEL_VIEW}
\end{figure}

\subsection{Shortcomings of \revision{conformance checking}-based techniques}
Unfortunately, the detection effectiveness of \revision{conformance checking}-based techniques is affected by noisy data and low-quality Petri nets, which may be due to human errors in the modeling process or representational bias of process discovery algorithms \cite{bezerra2013adlogspais, pecchia2020applicationfailuresanalysispm, aalst2016pm}. Specifically, on the one hand, noisy data may introduce infrequent and deceptive control-flow relations that may result in inconsistent fitness measures, whereas, on the other hand, checking event logs against a low-quality Petri net could lead to an unreliable distribution of fitness measures. Nonetheless, such Petri nets can still be used as references to obtain insightful information for \revision{process mining}-based feature extraction, supporting the development of competitive and explainable \revision{conformance checking}-based techniques for control-flow anomaly detection despite the problems above. For example, a few works outline that token-based \revision{conformance checking} can be used for \revision{process mining}-based feature extraction to build tabular data and develop effective \revision{conformance checking}-based techniques for control-flow anomaly detection \cite{singh2022lapmsh, debenedictis2023dtadiiot}. However, to the best of our knowledge, the scientific literature lacks a structured proposal for \revision{process mining}-based feature extraction using the state-of-the-art \revision{conformance checking} variant, namely alignment-based \revision{conformance checking}.

\subsection{Contributions}
We propose a novel \revision{process mining}-based feature extraction approach with alignment-based \revision{conformance checking}. This variant aligns the deviating control flow with a reference Petri net; the resulting alignment can be inspected to extract additional statistics such as the number of times a given activity caused mismatches \cite{aalst2022pmhandbook}. We integrate this approach into a flexible and explainable framework for developing techniques for control-flow anomaly detection. The framework combines \revision{process mining}-based feature extraction and dimensionality reduction to handle high-dimensional feature sets, achieve detection effectiveness, and support explainability. Notably, in addition to our proposed \revision{process mining}-based feature extraction approach, the framework allows employing other approaches, enabling a fair comparison of multiple \revision{conformance checking}-based and \revision{conformance checking}-independent techniques for control-flow anomaly detection. Figure \ref{HIGH_LEVEL_VIEW} shows a high-level view of the framework. Business processes are monitored, and event logs obtained from the database of information systems. Subsequently, \revision{process mining}-based feature extraction is applied to these event logs and tabular data input to dimensionality reduction to identify control-flow anomalies. We apply several \revision{conformance checking}-based and \revision{conformance checking}-independent framework techniques to publicly available datasets, simulated data of a case study from railways, and real-world data of a case study from healthcare. We show that the framework techniques implementing our approach outperform the baseline \revision{conformance checking}-based techniques while maintaining the explainable nature of \revision{conformance checking}.

In summary, the contributions of this paper are as follows.
\begin{itemize}
    \item{
        A novel \revision{process mining}-based feature extraction approach to support the development of competitive and explainable \revision{conformance checking}-based techniques for control-flow anomaly detection.
    }
    \item{
        A flexible and explainable framework for developing techniques for control-flow anomaly detection using \revision{process mining}-based feature extraction and dimensionality reduction.
    }
    \item{
        Application to synthetic and real-world datasets of several \revision{conformance checking}-based and \revision{conformance checking}-independent framework techniques, evaluating their detection effectiveness and explainability.
    }
\end{itemize}

The rest of the paper is organized as follows.
\begin{itemize}
    \item Section \ref{sec:related_work} reviews the existing techniques for control-flow anomaly detection, categorizing them into \revision{conformance checking}-based and \revision{conformance checking}-independent techniques.
    \item Section \ref{sec:abccfe} provides the preliminaries of \revision{process mining} to establish the notation used throughout the paper, and delves into the details of the proposed \revision{process mining}-based feature extraction approach with alignment-based \revision{conformance checking}.
    \item Section \ref{sec:framework} describes the framework for developing \revision{conformance checking}-based and \revision{conformance checking}-independent techniques for control-flow anomaly detection that combine \revision{process mining}-based feature extraction and dimensionality reduction.
    \item Section \ref{sec:evaluation} presents the experiments conducted with multiple framework and baseline techniques using data from publicly available datasets and case studies.
    \item Section \ref{sec:conclusions} draws the conclusions and presents future work.
\end{itemize}
% !TEX root =  ../main.tex
\section{Background on causality and abstraction}\label{sec:preliminaries}

This section provides the notation and key concepts related to causal modeling and abstraction theory.

\spara{Notation.} The set of integers from $1$ to $n$ is $[n]$.
The vectors of zeros and ones of size $n$ are $\zeros_n$ and $\ones_n$.
The identity matrix of size $n \times n$ is $\identity_n$. The Frobenius norm is $\frob{\mathbf{A}}$.
The set of positive definite matrices over $\reall^{n\times n}$ is $\pd^n$. The Hadamard product is $\odot$.
Function composition is $\circ$.
The domain of a function is $\dom{\cdot}$ and its kernel $\ker$.
Let $\mathcal{M}(\mathcal{X}^n)$ be the set of Borel measures over $\mathcal{X}^n \subseteq \reall^n$. Given a measure $\mu^n \in \mathcal{M}(\mathcal{X}^n)$ and a measurable map $\varphi^{\V}$, $\mathcal{X}^n \ni \mathbf{x} \overset{\varphi^{\V}}{\longmapsto} \V^\top \mathbf{x} \in \mathcal{X}^m$, we denote by $\varphi^{\V}_{\#}(\mu^n) \coloneqq \mu^n(\varphi^{\V^{-1}}(\mathbf{x}))$ the pushforward measure $\mu^m \in \mathcal{M}(\mathcal{X}^m)$. 


We now present the standard definition of SCM.

\begin{definition}[SCM, \citealp{pearl2009causality}]\label{def:SCM}
A (Markovian) structural causal model (SCM) $\scm^n$ is a tuple $\langle \myendogenous, \myexogenous, \myfunctional, \zeta^\myexogenous \rangle$, where \emph{(i)} $\myendogenous = \{X_1, \ldots, X_n\}$ is a set of $n$ endogenous random variables; \emph{(ii)} $\myexogenous =\{Z_1,\ldots,Z_n\}$ is a set of $n$ exogenous variables; \emph{(iii)} $\myfunctional$ is a set of $n$ functional assignments such that $X_i=f_i(\parents_i, Z_i)$, $\forall \; i \in [n]$, with $ \parents_i \subseteq \myendogenous \setminus \{ X_i\}$; \emph{(iv)} $\zeta^\myexogenous$ is a product probability measure over independent exogenous variables $\zeta^\myexogenous=\prod_{i \in [n]} \zeta^i$, where $\zeta^i=P(Z_i)$. 
\end{definition}
A Markovian SCM induces a directed acyclic graph (DAG) $\mathcal{G}_{\scm^n}$ where the nodes represent the variables $\myendogenous$ and the edges are determined by the structural functions $\myfunctional$; $ \parents_i$ constitutes then the parent set for $X_i$. Furthermore, we can recursively rewrite the set of structural function $\myfunctional$ as a set of mixing functions $\mymixing$ dependent only on the exogenous variables (cf. \cref{app:CA}). A key feature for studying causality is the possibility of defining interventions on the model:
\begin{definition}[Hard intervention, \citealp{pearl2009causality}]\label{def:intervention}
Given SCM $\scm^n = \langle \myendogenous, \myexogenous, \myfunctional, \zeta^\myexogenous \rangle$, a (hard) intervention $\iota = \operatorname{do}(\myendogenous^{\iota} = \mathbf{x}^{\iota})$, $\myendogenous^{\iota}\subseteq \myendogenous$,
is an operator that generates a new post-intervention SCM $\scm^n_\iota = \langle \myendogenous, \myexogenous, \myfunctional_\iota, \zeta^\myexogenous \rangle$ by replacing each function $f_i$ for $X_i\in\myendogenous^{\iota}$ with the constant $x_i^\iota\in \mathbf{x}^\iota$. 
Graphically, an intervention mutilates $\mathcal{G}_{\mathsf{M}^n}$ by removing all the incoming edges of the variables in $\myendogenous^{\iota}$.
\end{definition}

Given multiple SCMs describing the same system at different levels of granularity, CA provides the definition of an $\alpha$-abstraction map to relate these SCMs:
\begin{definition}[$\abst$-abstraction, \citealp{rischel2020category}]\label{def:abstraction}
Given low-level $\mathsf{M}^\ell$ and high-level $\mathsf{M}^h$ SCMs, an $\abst$-abstraction is a triple $\abst = \langle \Rset, \amap, \alphamap{} \rangle$, where \emph{(i)} $\Rset \subseteq \datalow$ is a subset of relevant variables in $\mathsf{M}^\ell$; \emph{(ii)} $\amap: \Rset \rightarrow \datahigh$ is a surjective function between the relevant variables of $\mathsf{M}^\ell$ and the endogenous variables of $\mathsf{M}^h$; \emph{(iii)} $\alphamap{}: \dom{\Rset} \rightarrow \dom{\datahigh}$ is a modular function $\alphamap{} = \bigotimes_{i\in[n]} \alphamap{X^h_i}$ made up by surjective functions $\alphamap{X^h_i}: \dom{\amap^{-1}(X^h_i)} \rightarrow \dom{X^h_i}$ from the outcome of low-level variables $\amap^{-1}(X^h_i) \in \datalow$ onto outcomes of the high-level variables $X^h_i \in \datahigh$.
\end{definition}
Notice that an $\abst$-abstraction simultaneously maps variables via the function $\amap$ and values through the function $\alphamap{}$. The definition itself does not place any constraint on these functions, although a common requirement in the literature is for the abstraction to satisfy \emph{interventional consistency} \cite{rubenstein2017causal,rischel2020category,beckers2019abstracting}. An important class of such well-behaved abstractions is \emph{constructive linear abstraction}, for which the following properties hold. By constructivity, \emph{(i)} $\abst$ is interventionally consistent; \emph{(ii)} all low-level variables are relevant $\Rset=\datalow$; \emph{(iii)} in addition to the map $\alphamap{}$ between endogenous variables, there exists a map ${\alphamap{}}_U$ between exogenous variables satisfying interventional consistency \cite{beckers2019abstracting,schooltink2024aligning}. By linearity, $\alphamap{} = \V^\top \in \reall^{h \times \ell}$ \cite{massidda2024learningcausalabstractionslinear}. \cref{app:CA} provides formal definitions for interventional consistency, linear and constructive abstraction.
\begin{algorithm}[ht!]
\caption{\textit{NovelSelect}}
\label{alg:novelselect}
\begin{algorithmic}[1]
\State \textbf{Input:} Data pool $\mathcal{X}^{all}$, data budget $n$
\State Initialize an empty dataset, $\mathcal{X} \gets \emptyset$
\While{$|\mathcal{X}| < n$}
    \State $x^{new} \gets \arg\max_{x \in \mathcal{X}^{all}} v(x)$
    \State $\mathcal{X} \gets \mathcal{X} \cup \{x^{new}\}$
    \State $\mathcal{X}^{all} \gets \mathcal{X}^{all} \setminus \{x^{new}\}$
\EndWhile
\State \textbf{return} $\mathcal{X}$
\end{algorithmic}
\end{algorithm}

\section{Analysis}
\label{sec:analysis}
In the following sections, we will analyze European type approval regulation\footnote{Strictly speaking, the German enabling act (AFGBV) does not regulate type-approval, but how test \& operating permits are issued for SAE-Level-4 systems. Type-approval regulation for SAE-Level-3 systems follows UN Regulation No. 157 (UN-ECE-ALKS) \parencite{un157}.} regarding the underlying notions of ``safety'' and ``risk''.
We will classify these notions according to their absolute or relative character, underlying risk sources, or underlying concepts of harm.

\subsection{Classification of Safety Notions}
\label{sec:safety-notions}
We will refer to \emph{absolute} notions of safety as conceptualizations that assume the complete absence of any kind of risk.
Opposed to this, \emph{relative} notions of safety are based on a conceptualization that specifically includes risk acceptance criteria, e.g., in terms of ``tolerable'' risk or ``sufficient'' safety.

For classifying notions of safety by their underlying risk (or rather ``hazard'') sources, and different concepts of harm, \Cref{fig:hazard-sources} provides an overview of our reasoning, which is closely in line with the argumentation provided by Waymo in \parencite{favaro2023}.
We prefer ``hazard sources'' over ``risk sources'', as a risk must always be related to a \emph{cause} or \emph{source of harm} (i.e., a hazard \parencite[p.~1, def. 3.2]{iso51}).
Without a concrete (scenario) context that the system is operating in, a hazard is \emph{latent}: E.g., when operating in public traffic, there is a fundamental possibility that a \emph{collision with a pedestrian} leads to (physical) harm for that pedestrian. 
However, only if an automated vehicle shows (potentially) hazardous behavior (e.g., not decelerating properly) \emph{and} is located near a pedestrian (context), the hazard is instantiated and leads to a hazardous event.
\begin{figure*}
    \includeimg[width=.9\textwidth]{hazard-sources0.pdf}
    \caption{Graphical summary of a taxonomy of risk related to automated vehicles, extended based on ISO 21448 (\parencite{iso21448}) and \parencite{favaro2023}. Top: Causal chain from hazard sources to actual harm; bottom: summary of the individual elements' contributions to a resulting risk. Graphic translated from \parencite{nolte2024} \label{fig:hazard-sources}}
\end{figure*}
If the hazardous event cannot be mitigated or controlled, we see a loss event in which the pedestrian's health is harmed.
Note that this hypothetical chain of events is summarized in the definition of risk:
The probability of occurrence of harm is determined by a) the frequency with which hazard sources manifest, b) the time for which the system operates in a context that exposes the possibility of harm, and c) by the probability with which a hazardous event can be controlled.
A risk can then be determined as a function of the probability of harm and the severity of the harm potentially inflicted on the pedestrian.

In the following, we will apply this general model to introduce different types of hazard sources and also different types of harm.
\cref{fig:hazard-sources} shows two distinct hazard sources, i.e., functional insufficiencies and E/E-failures that can lead to hazardous behavior.
ISO~21488 \parencite{iso21448} defines functional insufficiencies as insufficiencies that stem from an incomplete or faulty system specification (specification insufficiencies).
In addition, the standard considers insufficiencies that stem from insufficient technical capability to operate inside the targeted Operational Design Domain (performance insufficiencies).
Functional insufficiencies are related to the ``Safety of the Intended Functionality (SOTIF)'' (according to ISO~21448), ``Behavioral Safety'' (according to Waymo \parencite{waymo2018}), or ``Operational Safety'' (according to UN Regulation No. 157 \parencite{un157}).
E/E-Failures are related to classic functional safety and are covered exhaustively by ISO~26262 \parencite{iso2018}.
Additional hazard sources can, e.g., be related to malicious security attacks (ISO~21434), or even to mechanical failures that should be covered (in the US) in the Federal Motor Vehicle Safety Standards (FMVSS).

For the classification of notions of safety by the related harm, in \parencite{salem2024, nolte2024}, we take a different approach compared to \parencite{koopman2024}:
We extend the concept of harm to the violation of stakeholder \emph{values}, where values are considered to be a ``standard of varying importance among other such standards that, when combined, form a value pattern that reduces complexity for stakeholders [\ldots] [and] determines situational actions [\ldots].'' \parencite{albert2008}
In this sense, values are profound, personal determinants for individual or collective behavior.
The notion of values being organized in a weighted value pattern shows that values can be ranked according to importance.
For automated vehicles, \emph{physical wellbeing} and \emph{mobility} can, e.g., be considered values which need to be balanced to achieve societal acceptance, in line with the discussion of required tradeoffs in \cref{sec:terminology}.
For the analysis of the following regulatory frameworks, we will evaluate if the given safety or risk notions allow tradeoffs regarding underlying stakeholder values. 

\subsection{UN Regulation No. 157 \& European Implementing Regulation (EU) 2022/1426}
\label{sec:enabling-act}
UN Regulation No. 157 \parencite{un157} and the European Implementing Regulation 2022/1426 \parencite{eu1426} provide type approval regulation for automated vehicles equipped with SAE-Level-3 (UN Reg. 157) and Level 4 (EU 2022/1426) systems on an international (UN Reg. 157) and European (EU 2022/1426) level.

Generally, EU type approval considers UN ECE regulations mandatory for its member states ((EU) 2018/858, \parencite{eu858}), while the EU largely forgoes implementing EU-specific type approval rules, it maintains the right to alter or to amend UN ECE regulation \parencite{eu858}.

In this respect, the terminology and conceptualizations in the EU Implementing Act closely follow those in UN Reg. No. 157.
The EU Implementing Act gives a clear reference to UN Reg. No. 157 \parencite[][Preamble,  Paragraph 1]{eu1426}.
Hence, the documents can be assessed in parallel.
Differences will be pointed out as necessary.

Both acts are written in rather technical language, including the formulation of technical requirements (e.g., regarding deceleration values or speeds in certain scenarios).
While providing exhaustive definitions and terminology, neither of both documents provide an actual definition of risk or safety.
The definition of ``unreasonable'' risk in both documents does not define risk, but only what is considered \emph{unreasonable}. It states that the ``overall level of risk for [the driver, (only in UN Reg. 157)] vehicle occupants and other road users which is increased compared to a competently and carefully driven manual vehicle.''
The pertaining notions of safety and risk can hence only be derived from the context in which they are used.

\subsubsection{Absolute vs. Relative Notions of Safety}
In line with the technical detail provided in the acts, both clearly imply a \emph{relative} notion of safety and refer to the absence of \emph{unreasonable} risk throughout, which is typical for technical safety definitions.

Both acts require sufficient proof and documentation that the to-be-approved automated driving systems are ``free of unreasonable safety risks to vehicle occupants and other road users'' for type approval.\footnote{As it targets SAE-Level-3 systems, UN Reg. 157 also refers to the driver, where applicable.}
In this respect, both acts demand that the manufacturers perform verification and validation activities for performance requirements that include ``[\ldots] the conclusion that the system is designed in such a way that it is free from unreasonable risks [\ldots]''.
Additionally, \emph{risk minimization} is a recurring theme when it comes to the definition of Minimum Risk Maneuvers (MRM).

Finally, supporting the relative notions of safety and risk, UN Reg. 157 introduces the concept of ``reasonable foreseeable and preventable'' \parencite[Article 1, Clause 5.1.1.]{un157} collisions, which implies that a residual risk will remain with the introduction of automated vehicles.
\parencite[][Appendix 3, Clause 3.1.]{un157} explicitly states that only \emph{some} scenarios that are unpreventable for a competent human driver can actually be prevented by an automated driving system.
While this concept is not applied throughout the EU Implementing Act, both documents explicitly refer to \emph{residual} risks that are related to the operation of automated driving systems (\parencite[][Annex I, Clause 1]{un157}, \parencite[][Annex II, Clause 7.1.1.]{eu1426}).

\subsubsection{Hazard Sources}
Hazard sources that are explicitly differentiated in UN Reg. 157 and (EU) 2022/1426 are E/E-failures that are in scope of functional safety (ISO~26262) and functional insufficiencies that are in scope of behavioral (or ``operational'') safety (ISO~21448).
Both documents consistently differentiate both sources when formulating requirements.

While the acts share a common definition of ``operational'' safety (\parencite[][Article 2, def. 30.]{eu1426}, \parencite[][Annex 4, def. 2.15.]{un157}), the definitions for functional safety differ.
\parencite{un157} defines functional safety as the ``absence of unreasonable risk under the occurrence of hazards caused by a malfunctioning behaviour of electric/electronic systems [\ldots]'', \parencite{eu1426} drops the specification of ``electric/electronic systems'' from the definition.
When taken at face value, this definition would mean that functional safety included all possible hazard sources, regardless of their origin, which is a deviation from the otherwise precise usage of safety-related terminology.

\subsubsection{Harm Types}
As the acts lack explicit definitions of safety and risk, there is no consistent and explicit notion of different harm types that could be differentiated.

\parencite{un157} gives little hints regarding different considered harm types.
``The absence of unreasonable risk'' in terms of human driving performance could hence be related to any chosen performance metric that allows a comparison with a competent careful human driver including, e.g., accident statistics, statistics about rule violations, or changes in traffic flow.

In \parencite{eu1426}, ``safety'' is, implicitly, attributed to the absence of unreasonable risk to life and limb of humans.
This is supported by the performance requirements that are formulated:
\parencite[][Annex II, Clause 1.1.2. (d)]{eu1426} demands that an automated driving system can adapt the vehicle behavior in a way that it minimizes risk and prioritizes the protection of human life.

Both acts demand the adherence to traffic rules (\parencite[][Annex 2, Clause 1.3.]{eu1426}, \parencite[][Clause 5.1.2.]{un157}).
\parencite[][Annex II, Clause 1.1.2. (c)]{eu1426} also demands that an automated driving system shall adapt its behavior to surrounding traffic conditions, such as the current traffic flow.
With the relative notion of risk in both acts, the unspecific clear statement that there may be unpreventable accidents \parencite{un157}, and a demand of prioritization of human life in \parencite{eu1426}, both acts could be interpreted to allow developers to make tradeoffs as discussed in \cref{sec:terminology}.


\subsubsection{Conclusion}
To summarize, the UN Reg. 157 and the (EU) 2022/1426 both clearly support the technical notion of safety as the absence of unreasonable risk.
The notion is used consistently throughout both documents, providing a sufficiently clear terminology for the developers of automated vehicles.
Uncertainty remains when it comes to considered harm types: Both acts do not explicitly allow for broader notions of safety, in the sense of \parencite{koopman2024} or \parencite{salem2024}.
Finally, a minor weak spot can be seen in the definition of risk acceptance criteria: Both acts take the human driving performance as a baseline.
While (EU) 2022/1426 specifies that these criteria are specific to the systems' Operational Design Domain \parencite[][Annex II, Clause 7.1.1.]{eu1426}, the reference to the concrete Operational Design Domain is missing in UN Reg. 157.
Without a clearly defined notion of safety, however, it remains unclear, how aspects beyond net accident statistics (which are given as an example in \parencite[][Annex II, Clause 7.1.1.]{eu1426}), can be addressed practically, as demanded by \parencite{koopman2024}.

\subsection{German Regulation (StVG \& AFGBV)}
\label{sec:afgbv}
The German L3 (Automated Driving Act) and L4 (Act on Autonomous Driving) Acts from 2017 and 2021,\footnote{Formally, these are amendments to the German Road Traffic Act (StVG): 06/21/2017, BGBl. I p. 1648, 07/12/2021 BGBl. I p. 3108.} respectively, provide enabling regulation for the operation of SAE-Level-3 and 4 vehicles on German roads.
The German Implementing Regulation (\parencite{afgbv}, AFGBV) defines how this enabling regulation is to be implemented for granting testing permits for SAE-Level-3 and -4 and driving permits for SAE-Level-3 and -4 automated driving systems.\footnote{Note that these permits do not grant EU-wide type approval, but serve as a special solution for German roads only. At the same time, the AFGBV has the same scope as (EU) 2022/1426.}
With all three acts, Germany was the first country to regulate the approval of automated vehicles for a domestic market.
All acts are subject to (repeated) evaluation until the year 2030 regarding their impact on the development of automated driving technology.
An assessment of the German AFGBV and comparisons to (EU) 2022/1426 have been given in \cite{steininger2022} in German.

Just as for UN Reg. 157 and (EU) 2022/1426, neither the StVG nor the AFGBV provide a clear definition of ``safety'' or ``risk'' -- even though the "safety" of the road traffic is one major goal of the StVG and StVO.
Again, different implicit notions of both concepts can only be interpreted from the context of existing wording.
An additional complication that is related to the German language is that ``safety'' and ``security'' can both be addressed as ``Sicherheit'', adding another potential source of unclarity.
Literal Quotations in this section are our translations from the German act.

\subsubsection{Absolute vs. Relative Notions of Safety}
For assessing absolute vs. relative notions of safety in German regulation, it should be mentioned that the main goal of the German StVO is to ensure the ``safety and ease of traffic flow'' -- an already diametral goal that requires human drivers to make tradeoffs.\footnote{For human drivers, this also creates legal uncertainty which can sometimes only be settled in a-posteriori court cases.}
While UN and EU regulation clearly shows a relative notion of safety\footnote{And even the StVG contains sections that use wording such as ``best possible safety for vehicle occupants'' (§1d (4) StVG) and acknowledges that there are unavoidable hazards to human life (§1e (2) No. 2c)).}, the German AFGBV contains ambiguous statements in this respect:
Several paragraphs contain a demand for a hazard free operation of automated vehicles.
§4 (1) No. 4 AFGBV, e.g., states that ``the operation of vehicles with autonomous driving functions must neither negatively impact road traffic safety or traffic flow, nor endanger the life and limb of persons.''
Additionally, §6 (1) AFGBV states that the permits for testing and operation have to be revoked, if it becomes apparent that a ``negative impact on road traffic safety or traffic flow, or hazards to the life and limb of persons cannot be ruled out''.
The same wording is used for the approval of operational design domains regulated in §10 (1) No. 1.
A particularly misleading statement is made regarding the requirements for technical supervision instances which are regulated in §14 (3) AFGBV which states that an automated vehicle has to be  ``immediately removed from the public traffic space if a risk minimal state leads to hazards to road traffic safety or traffic flow''.
Considering the argumentation in \cref{sec:terminology}, that residual risks related to the operation of automated driving systems are inevitable, these are strong statements which, if taken at face value, technically prohibit the operation of automated vehicles.
It suggests an \emph{absolute} notion of safety that requires the complete absence of risk.  
The last statement above is particularly contradictory in itself, considering that a risk \emph{minimal} state always implies a residual risk.

In addition to these absolute safety notions, there are passages which suggest a relative notion of safety:
The approval for Operational Design Domains is coupled to the proof that the operation of an automated vehicle ``neither negatively impacts road traffic safety or traffic flow, nor significantly endangers the life and limb of persons beyond the general risk of an impact that is typical of local road traffic'' (§9 (2) No. 3 AFGBV).
The addition of a relative risk measure ``beyond the general risk of an impact'' provides a relaxation (cf. also \cite{steininger2022}, who criticizes the aforementioned absolute safety notion) that also yields an implicit acceptance criterion (\emph{statistically as good as} human drivers) similar to the requirements stated in UN Reg. 157 and (EU) 2022/1426.

Additional hints for a relative notion of safety can be found in Annex 1, Part 1, No. 1.1 and Annex 1, Part 2, No. 10.
Part 1, No 1.1 specifies collision-avoidance requirements and acknowledges that not all collisions can be avoided.\footnote{The same is true for Part 2, No. 10, Clause 10.2.5.}
Part 2, No. 10 specifies requirements for test cases.
It demands that test cases are suitable to provide evidence that the ``safety of a vehicle with an autonomous driving function is increased compared to the safety of human-driven vehicles''.
This does not only acknowledge residual risks, but also yields an acceptance criterion (\emph{better} than human drivers) that is different from the implied acceptance criterion given in §9 (2) No. 3 AFGBV.

\subsubsection{Hazard Sources}
Regarding hazard sources, Annex 1 and 3 AFGBV explicitly refer to ISO~26262 and ISO~21448 (or rather its predecessor ISO/PAS~21448:2019).
However, regarding the discussion of actual hazard sources, the context in which both standards are mentioned is partially unclear:
Annex 1, Clause 1.3 discusses requirements for path and speed planning.
Clause 1.3 d) demands that in intersections, a Time to Collision (TTC) greater than 3 seconds must be guaranteed.
If manufacturers deviate from this, it is demanded that ``state-of-the-art, systematic safety evaluations'' are performed.
Fulfillment of the state of the art is assumed if ``the guidelines of ISO~26262:2018-12 Road Vehicles -- Functional Safety are fulfilled''.
Technically, ISO~26262 is not suitable to define the state of the art in this context, as the requirements discussed fall in the scope of operational (or behavioral) safety (ISO~21448).
A hazard source ``violated minimal time to collision'' is clearly a functional insufficiency, not an E/E-failure.

Similar unclarity presents itself in Annex 3, Clause 1 AFGBV: 
Clause 1 specifies the contents of the ``functional specification''.
The ``specification of the functionality'' is an artifact which is demanded in ISO~21448:2022 (Clause 5.3) \parencite{iso21448}.
However, Annex 3, Clause 1 AFGBV states that the ``functional specification'' is considered to comply to the state of the art, if the ``functional specification'' adheres to ISO~26262-3:2018 (Concept Phase).
Again, this assumes SOTIF-related contents as part of ISO~26262, which introduces the ``Item Definition'' as an artifact, which is significantly different from the ``specification of the functionality'' which is demanded by ISO~21448.
Finally, Annex 3, Clause 3 AFGBV demands a ``documentation of the safety concept'' which ``allows a functional safety assessment''.
A safety concept that is related to operational / behavioral safety is not demanded.
Technically, the unclarity with respect to the addressed harm types lead to the fact that the requirements provided by the AFGBV do not comply with the state of the art in the field, providing questionable regulation.

\subsubsection{Harm Types}
Just like UN Reg. 157 and (EU) 2022/1426, the German StVG and AFGBV do not explicitly differentiate concrete harm types for their notions of safety.
However, the AFGBV mentions three main concerns for the operation of automated vehicles which are \emph{traffic flow} (e.g., §4 (1) No. 4 AFGBV), compliance to \emph{traffic law} (e.g., §1e (2) No. 2 StVG), and the \emph{life and limb of humans} (e.g., §4 (1) No. 4 AFGBV).

Again, there is some ambiguity in the chosen wording:
The conflict between traffic flow and safety has already been argued in \cref{sec:terminology}.
The wording given in §4 (1) No. 4 and §6 (1) AFGBV  demand to ensure (absolute) safety \emph{and} traffic flow at the same time, which is impossible (cf. \cref{sec:terminology}) from an engineering perspective.
§1e (2) No. 2 StVG defines that ``vehicles with an autonomous driving function must [\ldots] be capable to comply to [\ldots] traffic rules in a self-contained manner''.
Taken at face value, this wording implies that an automated driving system could lose its testing or operating permit as soon as it violates a traffic rule.
A way out could be provided by §1 of the German Traffic Act (StVO) which demands careful and considerate behavior of all traffic participants and by that allows judgement calls for human drivers.
However, if §1 is applicable in certain situations is often settled in court cases. 
For developers, the application of §1 StVO during system design hence remains a legal risk.

While there are rather absolute statements as mentioned above, sections of the AFGBV and StVG can be interpreted to allow tradeoffs:
§1e (2) No. 2 b) demands that a system,  ``in case of an inevitable, alternative harm to legal objectives, considers the significance of the legal objectives, where the protection of human life has highest priority''.
This exact wording \emph{could} provide some slack for the absolute demands in other parts of the acts, enabling tradeoffs between (tolerable) risk and mobility as discussed in \cref{sec:terminology}.
However, it remains unclear if this interpretation is legally possible.

\subsubsection{Conclusion}
Compared to UN Reg. 157 and (EU) 2022/1426, the German StVG and AFGBV introduce openly inconsistent notions of safety and risk which are partially directly contradictory:
The wording partially implies absolute and relative notions of safety and risk at the same time.
The implied validation targets (``better'' or ``as good as'' human drivers) are equally contradictory. 
The partially implied absolute notions of safety, when taken at face value, prohibit engineers from making the tradeoffs required to develop a system that is safe and provides customer benefit at the same time. 
In consequence, the wording in the acts is prone to introducing legal uncertainty.
This uncertainty creates additional clarification need and effort for manufacturers and engineers who design and develop SAE-Level-3 and -4 automated driving systems. The use of undefined legal terms not only makes it more difficult for engineers to comply with the law, but also complicates the interpretation of the law and leads to legal uncertainty.

\subsection{UK Automated Vehicles Act 2024 (2024 c. 10)}
The UK has issued a national enabling act for regulating the approval of automated vehicles on the roads in the UK.
To the best of our knowledge, concrete implementing regulation has not been issued yet.
Regarding terminology, the act begins with a dedicated terminology section to clarify the terms used in the act \parencite[Part 1, Chapter 1, Section 1]{ukav2024}.
In that regard, the act defines a vehicle to drive ```autonomously' if --- (a)
it is being controlled not by an individual but by equipment of the vehicle, and (b) neither the vehicle nor its surroundings are being monitored by an individual with a view to immediate intervention in the driving of the vehicle.''
The act hence covers SAE-Level-3 to SAE-Level-5 automated driving systems.

\subsubsection{Absolute vs. Relative Notions of Safety}
While not providing an explicit definition of safety and risk, the UK Automated Vehicles Act (``UK AV Act'') \parencite{ukav2024} explicitly refers to a relative notion of safety.
Part~1, Chapter~1, Section~1, Clause (7)~(a) defines that an automated vehicle travels ```safely' if it travels to an acceptably safe standard''.
This clarifies that absolute safety is not achievable and that acceptance criteria to prove the acceptability of residual risk are required, even though a concrete safety definition is not given.
The act explicitly tasks the UK Secretary of State\footnote{Which means, that concrete implementation regulation needs to be enacted.} to install safety principles to determine the ``acceptably safe standard'' in Part~1, Chapter~1, Section~1, Clause (7)~(a).
In this respect, the act also provides one general validation target as it demands that the safety principles must ensure that ``authorized automated vehicles will achieve a level of safety equivalent to, or higher than, that of careful and competent human drivers''.
Hence, the top-level validation risk acceptance criterion assumed for UK regulation is ``\emph{at least as good} as human drivers''.

\subsubsection{Hazard Sources}
The UK AV Act contains no statements that could be directly related to different hazard sources.
Note that, in contrast to the rest of the analyzed documents, the UK AV Act is enabling rather than implementing regulation.
It is hence comparable to the German StVG, which does not refer to concrete hazard sources as well.

\subsubsection{Types of Harm}
Even though providing a clear relative safety notion, the missing definition of risk also implies a lack of explicitly differentiable types of harm.
Implicitly, three different types of harm can be derived from the wording in the act.
This includes the harm to life and limb of humans\footnote{Part~1, Chapter~3, Section~25 defines ``aggravated offence where death or serious injury occurs'' \parencite{ukav2024}.}, the violation of traffic rules\footnote{Part~1, Chapter~1, Clause~(7)~(b) defines that an automated vehicle travels ```legally' if it travels with an acceptably low risk of committing a traffic infraction''}, and the cause of inconvenience to the public \parencite[Part~1, Chapter~1, Section~58, Clause (2)~(d)]{ukav2024}.

The act connects all the aforementioned types of harm to ``risk'' or ``acceptable safety''.
While the act generally defines criminal offenses for providing ``false or misleading information about safety'', it also acknowledges possible defenses if it can be proven that ``reasonable precautions'' were taken and that ``due diligence'' was exercised to ``avoid the commission of the offence''.
This statement could enable tradeoffs within the scope of ``reasonable risk'' to the life and limb of humans, the violation of traffic rules, or to the cause of inconvenience to the public, as we argued in \cref{sec:terminology}.

\subsubsection{Conclusion}
From the set of reviewed documents, the current UK AV Act is the one with the most obvious relative notions of safety and risk and the one that seems to provide a legal framework for permitting tradeoffs.
In our review, we did not spot major inconsistency beyond a missing definitions of safety and risk\footnote{Note that with the Office for Product Safety and Standards (OPSS), there is a British government agency that maintains an exhaustive and widely focussed ``Risk Lexicon'' that provides suitable risk definitions. For us, it remains unclear, to what extent this terminology is assumed general knowledge in British legislation.}.
The general, relative notion of safety and the related alleged ability for designers to argue well-founded development tradeoffs within the legal framework could prove beneficial for the actual implementation of automated driving systems.
While the act thus appears as a solid foundation for the market introduction of automated vehicles, without accompanying implementing regulation, it is too early to draw definite conclusions.
%Placeholder for general introduction of the Accelerators and Applications theme sections. Themes of applications include machine-learning, bioinformatics, space applications, radio astronomy and weather simulations. Some of these references will overlap with other sections, e.g. when contributions are made on applying effective distributed computing for the purpose of weather forecasting.

FPGAs have emerged as powerful accelerators for a wide range of applications. In this section, we discuss FPGA-based solutions in machine learning (Section~\ref{sec:ml}), astronomy (Section~\ref{sec:astr}), particle physics experiments (Section~\ref{sec:phys}), quantum computing (Section~\ref{sec:quant}), space applications (Section~\ref{sec:space}), and bioinformatics (Section~\ref{sec:bio}).

\subsection{Machine learning}
\label{sec:ml}
% Three main parts, adapting an existing ML approach to hardware, designing hardware to accelerate an existing ML approach, (co-)design hardware for exotic ML approach.
% Main categories of evaluation are throughput, power, hardware area / resources, accuracy.
% \begin{itemize}
%     \item Accelerating existing ML models with new hardware design
%         \begin{itemize}
%             \item CNN acceleration (5)
%             \item TPU (1)
%             \item Benchmarking FPGAs (1)
%     \item Co-design existing ML models to hardware accelerate
%         \begin{itemize}
%             \item Pruning
%             \item Quantization / fixed point
%             \item Weight sharing
%             \item NAS adaptive to hardware
%         \end{itemize}
%     \item Design new hardware for exotic ML model
%         \begin{itemize}
%             \item Spiking / neuromorphic (7)
%             \item Bayesian (1)
%             \item Oscillating (2)
%         \end{itemize}
%     \end{itemize}
%     \item Hardware for 
% \end{itemize}
% \subsubsection{Background}
In the field of machine learning, and in particular deep learning, hardware acceleration plays a vital role. GPUs are the predominant method for hardware acceleration due to their high parallelism, but FPGA research is showing promising results. FPGAs enable inference at greater speed and better power efficiency when compared to GPUs \cite{hw-efficiency-compare} by designing model-specific accelerated pipelines \cite{ml-energy-efficient-cnn}. Through the co-design of machine learning models and machine learning hardware on FPGAs, models are accelerated without compromising on performance metrics and utilizing limited FPGA resources. In addition, the flexibility of the FPGA's architecture enables the realization of unconventional deep learning technology, such as Spiking Neural Networks (SNNs). 
%These networks can operate on a fraction of the power required by conventional networks on CPU or GPU.

%\subsubsection{Research topics}
\paragraph{Hardware acceleration} Ample research on hardware acceleration focuses on accelerating existing neural network architectures. One common class of architectures is convolutional neural networks (CNNs), which learn image filters in order to identify abstract image features. CNNs are often deployed in embedded applications which require real-time image processing and low energy consumption, making FPGAs a suitable candidate for CNN acceleration. \citet{ml-energy-efficient-cnn} propose an implementation of the LeNet architecture using Vitis HLS, pipelining the CNN layers, and outperforms other FPGA based implementations at a processing time of $70 \mu s$. One downside to this approach is the inflexibility of designing a specific model architecture in HLS which can be resolved by using partial reconfiguration \cite{ml-cnn-acclr-part-reconf}. To increase CNN throughput, further parallelization can be exploited, and in combination with the use of the high bandwidth OpenCAPI interface, can achieve a latency of less than $10 \mu s$ on the LeNet-5 model, streaming data from an HDMI interface \cite{ml-FPQNet}. In each of these implementations, fully pipelined CNNs are possible due to the limited number of parameters in small CNNs. As larger pipelined networks are deployed on FPGAs, parallelization puts strain on the available resources, and in particular the amount of on-chip-memory becomes a bottleneck. A proposed solution to this is using Frequency Compensated Memory Packing \cite{ml-mem-efficient-df-inf}.

In addition to CNN acceleration, general neural network acceleration has been developed by means of a programmable Tensor Processing Unit (TPU) as an overlay on an FPGA accelerator \cite{ml-agile-tuned-tpu}. Deep learning acceleration using FPGAs is also relevant to space technology research. Since the reprogrammability of FPGAs make them a suitable contender for deployment on space missions, FPGA implementations of existing deep learning models are being benchmarked for space applications \cite{ml-myriad-2-space-cnn} \cite{ml-mem-efficient-df-inf}.

\paragraph{Spiking neural networks} Spiking Neural Networks (SNNs) are computational models formed using spiking neuronal units that operate in parallel and mimic the basic operational principles of biological systems. These features endow SNNs with potentially richer dynamics than traditional artificial neural network models based on the McCulloch-Pitts point neurons or simple ReLU activation functions that do not incorporate timing information. Thus, SNNs excel in handling temporal information streams and are well-suited for innovative non-von-Neumann computer architectures, which differ from traditional sequential processing systems. SNNs are particularly well-suited for implementation in FPGAs due to their massive parallelism and requirement for significant on-chip memories with high-memory bandwidth for storing neuron states and synaptic weights. Additionally, SNNs use sparse binary communication, which is beneficial for low-latency operations because both computing and memory updates are triggered by events. FPGAs' inherent flexibility allows for reprogramming and customization, which enable reprogrammable SNNs in FPGAs, resulting in flexible, efficient, and low-latency systems~\cite{Corradi2021Gyro:Analytics,Irmak2021ADesigns,SankaranAnInference}. \citet{corradi2024accelerated} demonstrated the application of a Spiking Convolutional Neural Network (SCNN) to population genomics. The SCNN architecture achieved comparable classification accuracy to state-of-the-art CNNs while processing only about 59.9\% of the input data, reaching 97.6\% of CNN accuracy for classifying selective-sweep and recombination-hotspot genomic regions. This was enabled by % success is attributed to 
the SCNN's capability to temporize genetic information, allowing it to produce classification outputs without processing the entire genomic input sequence. Additionally, when implemented on FPGA hardware, the SCNN model exhibited over three times higher throughput and more than 100 times greater energy efficiency than a GPU implementation, markedly enhancing the processing of large-scale population genomics datasets.


\paragraph{Model/hardware co-design} Previous examples demonstrate that existing deep neural network models can be accelerated using FPGAs. Typically, research in this area focuses on designing an optimal hardware solution for an existing model. A more effective approach, however, is to co-design the model and the hardware accelerator simultaneously. However, simultaneous co-design of DNN models and accelerators is challenging. DNN designers often need more specialized knowledge to consider hardware constraints, while hardware designers may need help to maintain the quality and accuracy of DNN models. Furthermore, efficiently exploring the extensive co-design space is a significant challenge. This co-design methodology leads to better performance, leveraging FPGAs' flexibility and rapid prototyping capabilities. For example, \citet{Rocha2020BinaryWrist-PPG}, by co-designing the bCorNET framework, which combines binary CNNs and LSTMs, they were able to create an efficient hardware accelerator that processes HR estimation from PPG signals in real-time. The pipelined architecture and quantization strategies employed allowed for significant reductions in memory footprint and computational complexity, enabling real-time processing with low latency.

In SNNs, encoding information in spike streams is a crucial co-design aspect. SNNs primarily use two encoding strategies: rate-coding and time-to-first-spike (TTFS) coding. Rate coding is common in SNN models, encoding information based on the instantaneous frequency of spike streams. Higher spike frequencies result in higher precision but at the cost of increased energy consumption due to frequent spiking. While rate coding offers accuracy, it reduces sparsity. In FPGA implementations, rate coding is often used for its robustness, simplicity, ease of training through the conversion of analog neural networks to spiking neural networks, and practicality in multi-sensor data fusion, where it helps represent real values from various sensors (radars, cameras) even in the presence of jitter or imperfect synchronization~\cite{Corradi2021Gyro:Analytics}.
Conversely, TTFS coding has been demonstrated in SNNs implemented on FPGAs to enhance sparsity and has the potential of reducing energy consumption by encoding information in spike timing. For instance, Pes et al.~\cite{Pes2024ActiveNetworks} introduced a novel SNN model with active dendrites to address catastrophic forgetting in sequential learning tasks. Active dendrites enable the SNN to dynamically select different sub-networks for different tasks, improving continual learning and mitigating catastrophic forgetting. This model was implemented on a Xilinx Zynq-7020 SoC FPGA, demonstrating practical viability with a high accuracy of 80\% and an average inference time of 37.3 ms, indicating significant potential for real-world deployment in edge devices.

%To overcome this challenges, Cong et al in \textcolor{red}{\textcolor{red}{~\cite{FPGA/DNN Co-Design: An Efficient Design Methodology for IoT
%Intelligence on the Edge}}} introduced a co-design methodology for FPGAs and DNNs that integrates both bottom-up and top-down approaches, in which a bottom-up search for DNN models that prioritize high accuracy is paired with a top-down design of FPGA accelerators tailored to the specific characteristics of DNNs.
%Other methods leverage an automatic toolchain comprising  auto-DNN engine for hardware-aware DNN model optimization and an auto-HLS engine to generate FPGA-suitable synthesizable code, or hardware-aware Neural Architecture Search (NAS). When co-design is applied, it typicaly produces DNN models and FPGA accelerators that outperform state-of-the-art FPGA designs in various metrics, including accuracy, speed, power consumption, and energy efficiency \textcolor{red}{~\cite{When Neural Architecture Search Meets Hardware Implementation: from Hardware Awareness to Co-Design}.

%\textcolor{blue}{Co-design is critical in developing FPGA-based systems, merging hardware and software engineering from the initial design stages. This integrated method is essential for optimizing system performance, functionality, and cost-effectiveness. Co-design leverages the adaptable nature of FPGAs, tailoring the computing workload to meet specific hardware needs and adjusting the hardware to suit software demands. This synergy results in improved system performance and greater energy efficiency.}
%\textcolor{blue}{Many co-design examples exists in literature that demonstrate how clever distributed memory layouts can results in increased performances~\cite{}. }

%\paragraph{Novel hardware architecture} 

%\textcolor{blue}{Modern co-design methodologies allow the generation of hardware architectures and applications for advanced Reconfigurable Acceleration Devices (RAD) that go beyond traditional FPGA capabilities. These devices integrate FPGA fabric with other components like general-purpose processors, specialized accelerators, and high-performance networks-on-chip (NoCs) within a system-in-package framework. This integration enables complex data center applications to be handled more efficiently than conventional FPGAs. In particular, Boutrous et al in \cite{Architecture and Application Co-Design for Beyond-FPGA Reconfigurable Acceleration Devices} introduce RAD-Sim, an architecture simulator, to aid in the design space exploration of RADs. This allows for the study of interactions between different system components. Notably, they demonstrated mapping deep learning FPGA overlays to different RAD configurations, demonstrating how RAD-Sim can guide the adaptation of applications to exploit the novel features of RADs effectively.}


\subsection{Astronomy}
\label{sec:astr}
%\subsubsection{Introduction}

Astronomy is the study of everything in the universe beyond our Earth's atmosphere. Observations are done at different modalities and wavelengths, such as detection of a range of different particles (e.g., Cherenkov detector based systems such as KM3NeT \cite{KM3NeT:2009xxi}), gravitational waves, optical observations, gamma and x-ray observations and radio (e.g., WSRT \cite{van_Cappellen_2022}, LOFAR \cite{van_Haarlem_2013}, SKA \cite{book-SKA}). Observations can be done from space or from earth; in this section, we limit the scope to ground-based astronomy. A common denominator for instruments required for observation of the different modalities and different wave lengths is that the systems need to be very sensitive in order to observe very faint signals from outside the Earth's atmosphere. Instruments are typically large and/or distributed over a large area %in order 
to achieve %reach 
good sensitivity and resolution. Different modalities and wavelengths require distinct types of sensors, cameras, or antennas to convert observed phenomena into electrical signals. Each system is tailored to its specific modality and wavelength, necessitating specialized components to accurately capture and translate the data. %At different modalities and different wave lengths, the systems each require different kinds of sensors, camera's or antenna's that convert the observed phenomenon to an electrical signal. 
%The electrical signal is at some point in the signal chain converted to the digital domain and processed in various stages into an end product used by scientists. 
At a certain stage in the signal chain, the electrical signal is converted into the digital domain, where it undergoes multiple processing stages. This processed signal ultimately results in an end product that can be utilized by scientists for analysis and research purposes.
Systems can roughly be split into two parts, a front-end and a back-end. The front-end requires interfacing with and processing of data from the sensor; electronics commonly deployed in the front-end are constrained in space (size), temperature, power, cost, RFI, environmental conditions and serviceability. The back-end processes data produced by the front-end(s) either in an online or offline fashion, which is usually %typically be 
done with server infrastructure in a data center. % environment, either on site or centralized. 
In the back-end, the main challenges are the high data bandwidth and large data size coming from the front-ends. Although the environment is more flexible, systems are still constrained in space, power, and cost.

%\subsubsection{Background}

FPGAs have been used in astronomy instrumentation for quite some time, as they 
%FPGAs have since long found applications in astronomy instrumentation. 
%Typically FPGA's 
are %very 
efficient in interfacing with Analog to Digital Converters (ADCs), and well suited to the conditions faced in instrumentation front-ends (e.g. NCLE \cite{karapakula2024ncle}). Moreover, FPGA are also used further down the processing stages for various signal processing operations, both in the front-ends (e.g., Uniboard2 in LOFAR \cite{doi:10.1142/S225117171950003X}) as well as in the back-ends of systems (e.g., MeerKAT \cite{2022JATIS...8a1006V} and SKA \cite{SKA-CBF}). GPUs represent a good alternative in back-end processing (e.g., LOFAR's system COBALT \cite{Broekema_2018}) as well. The work by Veenboer et al. \cite{10.1007/978-3-030-29400-7_36} describes a trade-off between using a GPU and an FPGA accelerator in the implementation of an image processing operation in a radio telescope back-end.

%\subsubsection{Research topics}
%Dutch academia has contributed to several astronomy instruments:
%Often large international consortia, not immidiately clear what the role of the Dutch partners was. But also some work which is mainly done by Dutch institutes
\paragraph{Hardware Development for the Radio Neutrino Observatory in Greenland (RNO-G)}
The RNO-G \cite{Smith2022HardwareRNO-G} is a radio detection array for neutrinos. It consists of 35 autonomous stations deployed over a $5 \times 6$ km grid near the NSF Summit Station
in Greenland. Each station includes an FPGA-based phased trigger. The station has to operate in a 25 W power envelope. The implementation on FPGA seems to be favorable due to environmental conditions and operation constraints.

\paragraph{Implementation of a Correlator onto a Hardware Beam-Former to Calculate Beam-Weights}
The Apertif Phased Array Feed (PAF) \cite{van_Cappellen_2022} is a radio telescope front-end used in the WSRT system in the Netherlands. FPGAs are used for antenna read out as well as signal processing close to the antenna. Schoonderbeek et al. \cite{Schoonderbeek2020ImplementationBeam-Weights} describe the transformation and implementation of a beamformer algorithm on FPGA in order to build a more efficient system.

\paragraph{Near Memory Acceleration and Reduced-Precision Acceleration for High Resolution Radio Astronomy Imaging}
\citet{Corda2020NearImaging} describe the implementation of a 2D FFT on FPGA, leveraging Near-Memory Computing. The 2D FFT is applied to an image processing implementation on FPGA in the back-end of a radio telescope and compared with implementations on CPU and GPU. \citet{Corda2022Reduced-PrecisionHardware} explore %the concept of 
reduced-precision computation on an FPGA %is explored 
for the same image processing application. %They propose an implementation on an FPGA accelerator and compare with an implementation on CPU and GPU.

\paragraph{The MUSCAT Readout Electronics Backend: Design and Pre-deployment Performance}
The MUSCAT is a large single dish radio telescope with 1458 receives in the focal plane. The system uses FPGA based electronics to read out and pre-process the data from the receivers \cite{Rowe2023ThePerformance}. %\emph{Electronics Backend} in this case relates to the electronics close to the antenna, referred to as front-end in our description here.

% Small contribution by NL through SRON

\paragraph{Cherenkov Telescope Arrays}
Three different contributions have been made to three different Cherenkov based Telescope Arrays.
%\paragraph{A NECTAr-based upgrade for the Cherenkov cameras of the H.E.S.S. 12-meter telescopes}
Ashton et al.~\cite{Ashton2020ATelescopes} describe a system for the High Energy Stereoscopic System (H.E.S.S.) where a custom board with ARM CPU and an FPGA is used to read out and pre-process a custom designed NECTAr digitizer chip in the front-end of the system. After pre-processing, the data is distributed to a back-end over Ethernet.
%Anton Pannekoek Institute for Astronomy
%\paragraph{A White Rabbit-Synchronized Accurate Time-Stamping Solution for the Small-Sized Cameras of the Cherenkov Telescope Array}
Sánchez-Garrido et al.~\cite{Sanchez-Garrido2021AArray} present the design of a Zynq FPGA SoC based platform for White Rabbit time synchronization in the ZEN-CTA telescope array front-ends. Data captured and pre-processed at the front-ends is distributed over Ethernet to the back-end including the time stamp.
%\paragraph{Architecture and performance of the KM3NeT front-end firmware}
Aiello et al.~\cite{Aiello2021ArchitectureFirmware} outline the architecture and performance of the KM3Net front-end firmware. The KM3NeT telescope consists of two deep-sea three-dimensional sensor grids being deployed in the Mediterranean Sea. A central logic board with FPGA in the front-end serves as a Time to Digital Converter to record events and time at the sensors; the data is transmitted and further processed in a back-end on shore.
%S. Aiello et al., “KM3NeT front-end and readout electronics system:
%hardware, firmware, and software,” J. Astronomical Telescopes Inst.
%Syst., vol. 5, no. 4, pp. 1–15, 2019.

%\subsubsection{Future direction}

\vspace{0.4cm}
%In the works included in this survey, 
FPGA are mainly used for front-end sensor interfacing and pre-processing. \citet{Corda2020NearImaging, Corda2022Reduced-PrecisionHardware} underline that FPGAs are also still relevant in the back-end, providing improved performance over CPU and on-par performance with GPU accelerators. FPGA are expected to remain the dominant choice for platforms in astronomy instrumentation front-ends due to the strong interfacing capabilities and the adaptability and suitability to the constraints imposed by instrumentation front-ends. In the back-end, FPGAs provide a viable solution to application acceleration, but will have to compete with other accelerator architectures, e.g., GPUs~\cite{10.1007/978-3-030-29400-7_36}. 
An emerging new technology are the Artificial Intelligence Engines in the AMD Versal Adaptive SoC. The work from \citet{Versal-ACAP} evaluated the AI Engines for a signal processing application in radio astronomy. The flexibility and programmability of the AI Engines, combined with the interfacing capabilities of the FPGA can lead to a powerful platform for telescope front-ends.

\subsection{Particle physics experiments}
\label{sec:phys}
The Large Hadron Collider (LHC) features various particle accelerators to facilitate particle physics experiments. Experiments performed using particle accelerators can produce massive amounts of data that needs to be propagated and preprocessed at high speeds before the reduced relevant data is recorded for offline storage. FPGAs are widely employed throughout systems LHC particle accelerators, such as ATLAS and LHCb, for their high-bandwidth capabilities, and the flexibility that reconfigurable hardware offers without requiring hardware alterations to the system. Recently both the ATLAS and LHCb particle accelerators have been commissioned for upgrades. The Dutch Institute for Subatomic Physics (Nikhef) is one of the collaborating institutes working on the LHC accelerators.

%\subsubsection{Research topics}
%\paragraph{TODO - revise text into research topics}

LHCb is a particle accelerator that specializes in experiments that study the bottom quark. Major upgrades to the LHCb that enable handling a higher collision rate require new front-end and back-end electronics. To facilitate the back-end of the upgrade, the LHCb implements the custom PCIe40 board, which features an Intel Arria 10 FPGA. Four PCIe40 boards are dedicated for controlling part of the LHCb system, and 52 PCIe40 boards are used to read out each of the detector’s slices, producing an aggregated data rate of 2.85 Tb/s \cite{FernandezPrieto2020PhaseExperiment}.

ATLAS is one of the general particle accelerators of the LHC. ATLAS uses two trigger stages in order to record only the particle interactions of interest. In an upgrade to the ATLAS accelerator, ASIC-based calorimeter trigger preprocessor boards are replaced by FPGA-based hardware. Using FPGAs for this purpose allows implementing enhanced signal processing methods \cite{Aad2020PerformanceTrigger}. After the two trigger stages, FPGAs are deployed to process the triggers for tracking particles \cite{Aad2021TheSystem}.

Ongoing upgrades to the LHC particle accelerators, referred to as High Luminosity LHC (HL-LHC), will facilitate higher energy collisions. HL-LHC will produce increased background rates. To reduce false triggers due to background, the New Small Wheel checks for coinciding hits. Each trigger processor features Virtex-7, Kintex Ultrascale and Zynq FPGAs \cite{Iakovidis2023TheElectronics}. Interaction to and from front-end hardware is done through Front-End Link eXchange (FELiX) boards. As part of the HL-LHC upgrades, each FELiX board must facilitate a maximum throughput of 200 Gbps. To enable this, Remote Direct Memory Access (RDMA) over Converged Ethernet (RoCE) as part of the FELiX FPGA system is proposed \cite{Vasile2022FPGALHC, Vasile2023IntegrationLHC}. The performance of the FELiX upgrade in combination with an upgraded Software ReadOut Driver (SW ROD) satisfies the data transfer requirements of the upgraded ATLAS system \cite{Gottardo2020FEliXSystem}.




\subsection{Quantum computing}
\label{sec:quant}
Quantum computing promises to help solving many global challenges of our time such as quantum chemistry problems to design new medicines, the prediction of material properties for efficient energy storage, and the handling of big data needed for complex climate physics~\cite{Gibney-nat-2014}. The most promising quantum algorithms demand systems comprising thousands to millions of quantum bits~\cite{Meter-2013}, the quantum counterpart of a classical bit. A quantum processor comprising up to 50 qubits has been realized using solid-state superconducting qubits~\cite{Arute-nat-2019}, but its operation requires a combination of cryogenic temperatures below ~100 mK and hundreds of coaxial lines for qubit control and readout. %Furthermore, 
While in systems with a few qubits, this can be controlled using off-the-shelf electronic equipment, such approach becomes infeasible when scaling qubit systems toward thousands or millions of qubits that are required for a practical quantum computer. 

%\subsubsection{Research topics}

A means to tackle the foreseeable bottleneck in scaling the operation of qubit systems is to integrate FPGA technology in the control and readout of solid-state qubits. FPGAs have been used to generate highly-stable waveforms suitable for the control of quantum bits with latency significantly lower than software alternatives~\cite{Ireland-2020}. In systems of semiconductor spin qubits, FPGAs have provided in-hardware syncing of quantum dot control voltages with the signal acquisition and buffering and thus enabled the observation of real-time charge-tunneling events~\cite{Hartman-2023}. FPGAs have also been used to configure and synchronize a cryo-controller with an arbitrary waveform generator required to generate complex pulse shapes and perform quantum operations~\cite{Xue-nat-2021}. Such setup has enabled the demonstration of universal control of a quantum processor hosting six semiconductor spin qubits~\cite{Philips-nat-2022}. FPGAs have proven to be essential for implementing quantum error correction algorithms, which are critical for mitigating the effects of dephasing and decoherence in solid-state qubits. %FPGAs have also been shown to essential for the implementation of quantum error correction algorithms needed to mitigate the effects of dephasing and decoherence in solid-state qubits. 
In qubit systems based on superconducting quantum circuits, the first efficient demonstration of quantum error correction was made possible by a FPGA-controlled data acquisition system which provided dynamic real-time feedback on the evolution of the quantum system~\cite{Ofek-nat-2016}. It has been further predicted that FPGA can enable highly-efficient quantum error correction based on neural-network decoders~\cite{Overwater-2022}.

%\subsubsection{Future directions}

FPGA technology has proven invaluable in the development of the emerging research field of quantum computing.
However, the complexity of programming FPGA circuits hinders their implementation in quantum computing systems. Commercial efforts have been done toward providing graphical tools for designing FPGA programs, namely the Quantum Researchers Toolkit by Keysight Technologies and the FPGA-based multi-instrument platform Moku-Pro developed by Liquid Instruments. These tools are essential for implementing customized algorithms without the need for dedicated expertise in hardware description languages. Future research is also needed in integrating FPGAs in cryogenic platforms required to operate qubit systems. Such capability has already been demonstrated; commercial FPGAs can operate at temperatures below 4 K and be integrated in a cryogenic platform for qubit control~\cite{Homulle-2017}. These efforts provide evidence that FPGA technology is of great interest for enabling a scalable and practically applicable quantum computer. 


\subsection{Space}
\label{sec:space}
The flexibility of FPGA technology makes it a suitable platform for many applications on-board space missions. The European Space Research and Technology Centre (ESTEC), as part of the European Space Agency (ESA) actively explores FPGA technology for space applications, and has an extensive portfolio of FPGA Intellectual Property (IP) Cores~\cite{esa_ip}.

%\subsubsection{Research topics}

FPGAs can flexibly route its input and output ports, and can be configured to support many different communication protocols. This makes FPGAs good contenders as devices that communicate with the various hardware platforms and sensors on a space mission. FPGAs and have been implemented as interface devices in novel on-board machine learning and digital signal processing  implementations~\cite{Leon2021ImprovingSoC, Leon2021FPGABenchmarks, karapakula2024ncle}. 

An on-board task for which FPGAs are used is hyperspectral imaging. This type of on-board imaging produces vast amounts of data. To reduce transmission bandwidth requirements when transmitting the sensory data to earth, real-time on-board compression handling high data rates is required. FPGAs are well-suited for such tasks, and research has been done on using space-grade radiation-hardened FPGAs \cite{Barrios2020SHyLoCMissions} as well as commercial off-the-shelf (COTS) FPGAs \cite{Rodriguez2019ScalableCompression} for on-board hyperspectral image compression. COTS devices are generally cheaper than space-grade devices, but the higher susceptibility of these devices to radiation-induced effects makes them challenging to employ.

Communication between on-board systems often requires high data-rates and is susceptible to radiation induced effects. To deal with the unique constraints of space applications, dedicated communication protocols such as SpaceWire, and its successor, SpaceFibre have been developed. These protocols are available as FPGA IP implementations, and testing environments of SpaceFibre have been developed \cite{MystkowskaSimulationSpaceFibre, AnSection}. SpaceWire can interface with the common AXI4 protocol using a dedicated bridge \cite{RubattuASystems}, enabling its integration with SpaceWire interfaces. Direct Memory Access (DMA) allows peripherals to transfer data to and from an FPGA without going through a CPU. The application of DMA in space is being investigated, however its application as of now is limited since DMA is susceptibility to radiation-induced effects \cite{Portaluri2022Radiation-inducedDevices}.



\subsection{Bioinformatics}
\label{sec:bio}
FPGA technology has been extensively explored for accelerating Bioinformatics kernels. Bioinformatics is an interdisciplinary scientific field that combines biology, computer science, mathematics, and statistics to analyze and interpret biological data. The field primarily focuses on the development and application of methods, algorithms, and tools to handle, process, and analyze large sets of biological data, such as DNA sequences, protein structures, and gene expression patterns.Continuous advances in DNA sequencing technologies~\cite{hu2021next} have led to the rapid accumulation of biological data, creating an urgent need for high-performance computational solutions capable of efficiently managing increasingly larger datasets.

\citet{Shahroodi2022KrakenOnMem:Profiling} describe a hardware/software co-designed framework to accelerate and improve energy consumption of taxonomic profiling. In metagenomics, the main goal is to understand the role of each organism in our environment in order to
improve our quality of life, and taxonomic profiling involves the identification and categorization of the various types of organisms present in a biological sample by analyzing DNA or protein sequences from the sample to determine which species or taxa are represented. The study focuses on boosting performance of table lookup, which is the primary bottleneck in taxonomic profilers, by proposing a processing-in-memory hardware accelerator. Using large-scale simulations, the authors report an average of 63.1\% faster execution and orders of magnitude higher energy efficiency than the  widely used metagenomic analysis tool Kraken2~\cite{wood2019improved} executed on a 128-core server with AMD EPYC 7742 processors  operating at 2.25 GHz. An FPGA was used for prototyping and emulation purposes.

\citet{Corts2022AcceleratedFPGAs} employ FPGAs to accelerate the detection of traces of positive natural selection in genomes. The authors designed a hardware accelerator for the $\omega$ statistic~\cite{kim2004linkage}, which is extensively used in population genetics as an indicator of positive selection. In comparison with a single CPU core,
the FPGA accelerator can deliver up to $57.1\times$ faster
computation of the $\omega$ statistic, using the OmegaPlus~\cite{alachiotis2012omegaplus} software implementation as reference.


%\citet{Ahmad2022Communication-EfficientFlight}



\citet{Malakonakis2020ExploringRAxML} use FPGAs to accelerate the widely used phylogenetics software tool RAxML~\cite{stamatakis2014raxml}. The study implements the Phylogenetic Likelihood Function (PLF), which is used for evaluating phylogenetic trees, on a Xilinx ZCU102 development board and a cloud-based Amazon AWS EC2 F1 instance. The first system (ZCU102) can deploy two accelerator instances, operating at 250MHz, and delivers up to $7.7\times$ faster executions than sequential software execution on a AWS EC2 F1 instance. %Xeon processors. 
The AWS-based accelerated system is $5.2\times$ faster than the same software implementation. %In comparison with previous work by Alachiotis et al.~\cite{alachiotis2009exploring[7_12]}, the implementation on the Xilinx development board is about 2.35x faster. %, but the older technology should certainly be taken into consideration. 



\citet{Alachiotis2021AcceleratingCloud} also target the PLF implementation in RAxML, and propose an optimization method for data movement in PCI-attached accelerators using tree-search algorithms. They developed a software cache controller that leverages data dependencies between consecutive PLF calls to cache data on the accelerator card. In combination with double buffering over PCIe, this approach led to nearly $4\times$ improvement in the performance of an FPGA-based PLF accelerator. Executing the complete RAxML algorithm on an AWS EC2 F1 instance, the authors observed up to $9.2\times$ faster processing of protein data than a $2.7$ GHz Xeon processor in the same cloud environment.

With genomic datasets continuing to expand, bioinformatics analyses are likely to increasingly rely on cloud computing in the future. This shift will be supported by new programming models and frameworks designed to address the data-movement challenges posed by cloud-based hardware accelerators. These accelerators, such as FPGAs and GPUs, need data transfers from the host processor, which can significantly impact execution times and negate gains from computation improvements. Fortunately, similar data-movement concerns exist for both FPGAs and GPUs, and ongoing engineering efforts are likely to converge on common solutions~\cite{Corts2023AGenetics}. This will help bring optimized, hardware-accelerate processing techniques into more widespread use among computational biologists and bioinformaticians in the future.





\section{Conclusion}
In this work, we propose a simple yet effective approach, called SMILE, for graph few-shot learning with fewer tasks. Specifically, we introduce a novel dual-level mixup strategy, including within-task and across-task mixup, for enriching the diversity of nodes within each task and the diversity of tasks. Also, we incorporate the degree-based prior information to learn expressive node embeddings. Theoretically, we prove that SMILE effectively enhances the model's generalization performance. Empirically, we conduct extensive experiments on multiple benchmarks and the results suggest that SMILE significantly outperforms other baselines, including both in-domain and cross-domain few-shot settings.


% \newpage
% Acknowledgments---Will not appear in anonymized version
\acks{The authors wish to thank Asaf Cassel and Aviv Rosenberg for sharing further insights about their work at the 
virtual RL theory seminars, and Volkan Cevher for initial discussions about this project. 
This project has received funding from the European Research Council (ERC), under the European Union's Horizon 2020 
research and innovation programme (Grant agreement No.~950180).
This work is funded (in part) through a PhD fellowship of the Swiss Data Science Center, a joint
venture between EPFL and ETH Zurich.  Luca Viano acknowledges travel support from ELISE (GA no 951847).}


\bibliography{ref}


\clearpage
\appendix
\renewcommand{\contentsname}{Contents of Appendix}
\addtocontents{toc}{\protect\setcounter{tocdepth}{2}}
{
%  \hypersetup{hidelinks}
  \setlength{\cftbeforesecskip}{.8em}
  \setlength{\cftbeforesubsecskip}{.5em}
  \tableofcontents
}
\clearpage
\section{Omitted pseudocodes}
\label{app:pseudocodes}
This section includes the pseudocode for \algname.
Each of the steps is explained in details in \Cref{sec:algo}.
\begin{algorithm}[!h]
\caption{\algname for Linear MDPs.}
  \begin{algorithmic}[1]
  \label{alg:linear-rmax-ravi-ucb}
      \STATE {\bfseries Inputs:} Number of resets $K$, learning rate $\eta > 0$, exploration coefficient $\beta > 0$, threshold $\omega > 0$, slope sigmoid $\alpha > 0$.
      \STATE {\bfseries Initialize:} $X_1 \sim \nu_0$, $\pi_1 = \piunif$, $Q_1 = 0$, $\cD_1 = \emptyset$, $\Lambda_1 = I$, $t = 1$, $e = 0$.
      \FOR{$k = 1, \dots, K$}
        
      \STATE \vspace{3pt}
      \algcomment{interact with the environment}
      \STATE The adversary adaptively chooses $r_k$, i.e.  $r_k = \textsc{RewardUpdate} \spr{\scbr{\pi_\ell}^k_{\ell=1}, \scbr{r_\ell}^{k-1}_{\ell=1}}$.
      \WHILE{\TRUE}
      \STATE Observe the state $X_t$ and play an action $A_t \sim \pi_k \spr{\cdot \given X_t}$.
      \STATE Receive reward $r_k \spr{X_t, A_t}$ and observe the function $r_k$.
      \STATE With probability $1 - \gamma$, \reset to initial distribution: $X_{t+1} \sim \nu_0$ set $T_k = t$ and \textbf{break} .
      \STATE Otherwise observe the next state $X_{t+1} \sim P \spr{\cdot \given X_t, A_t}$.
      \STATE Update $\Lambda_{t + 1} = \Lambda_t + \phi \spr{X_t, A_t} \phi \spr{X_t, A_t}\transpose$.
      \STATE $\cD_{t+1} = \cD_t \cup \scbr{\spr{X_t, A_t, X_{t+1}}}$.
      \STATE $t = t + 1$.
      \ENDWHILE
      \STATE \algcomment{initialize new epoch}
      \IF{$t = T_1$ \OR $\det \Lambda_{T_k} \geq 2 \det \Lambda_{t_e}$} \label{line-alg:epoch}
      \STATE $e = e + 1$.
      \STATE Set $k_e = k$ and $t_e = t$.
      \STATE Reset the policy $\pi_k = \piunif$.
      \ENDIF
      \STATE For any $\spr{x, a} \in \cX \times \cA$, $\CB_k \spr{x, a} = \beta \norm{\phi \spr{x, a}}_{\Lambda_{t_e}^{-1}}$,\;\;and\; $\CB_k\spr{x^\upplus, a} = 0$. \label{line-alg:bonuses}
      \STATE For any $\spr{x, a} \in \cX \times \cA$, $p_k^\upplus \spr{x, a} = \sigma \spr{\alpha \CB_k \spr{x, a} - \omega}$,\;\;and\; $p_k^\upplus \spr{x^\upplus, a} = 0$. \label{line-alg:pplus}
      \STATE \algcomment{optimistic regularized value iteration}
      \STATE $r_k^\upplus = \spr{1 - p_k^\upplus} \odot r_k + p_k^\upplus \cdot \RMAX$.
      % \STATE $\wh{P}_k = \Lambda_t^{-1} \sum_{\spr{x, a, x'} \in \cD_t} \phi \spr{x, a} \bfe_{x'}$.
      % \STATE $\wh{P}_k^\upplus = \sbr{1 - p_k^\upplus} \wh{P}_k + p_k^\upplus \bfe_{x^\upplus}$.
      % \STATE $Q_{k + 1} = r_k^\upplus + \sbr{1 - p_k^\upplus} \odot \CB_k + \gamma \wh{P}_k^\upplus V_k$.
      \STATE $\wh{M V_k} = \Lambda_{T_k}^{-1} \sum_{\spr{x, a, x'} \in \cD_{T_k}} \phi \spr{x, a} V_k \spr{x'}$. \label{line-alg:ridge}
      \STATE $\wh{P_k^\upplus V_k} = \spr{1 - p_k^\upplus} \odot \phim \wh{M V_k} + p_k^\upplus \cdot V_k \spr{x^\upplus}$,\;\;and\; $\wh{P_k^\upplus V_k} \spr{x^\upplus, \cdot} = \frac{\RMAX}{1 - \gamma}$. \label{line-alg:transf}
      \STATE $Q_{k + 1} = r_k^\upplus + \spr{1 - p_k^\upplus} \odot \CB_k + \gamma \wh{P_k^\upplus V_k}$.
      \STATE $V_{k+1} \spr{x} = \frac1\eta \log \spr{\sum_a \pi_{k} \spr{a \given x} e^{\eta Q_{k+1} \spr{x, a}}}$.
      \STATE $\pi_{k+1} = \pi_{k} \odot e^{\eta \spr{Q_{k+1} - E V_{k+1}}}$.
      \ENDFOR
      \STATE {\bfseries Output:} $\pi_I$, with $I \sim \cU \spr{\sbr{K}}$.
  \end{algorithmic}
\end{algorithm}

\noindent Next, we include the pseudocode for our imitation learning algorithms built on \algname. At line~\ref{algline:expert_estimation}, the learner computes an estimate of the expert features expectation computing an elementwise empirical average of the features in the dataset $\cD_\expert$. Such an estimate is leveraged in the online gradient descent (OGD) update given by the function at lines~~\ref{algline:OGDstart}-\ref{algline:OGDend}. This function instantiates the general $\textsc{RewardUpdate}$ routine given in \Cref{alg:linear-rmax-ravi-ucb}. That is, after each policy update in \algname, the reward player estimates the feature expectation of the current policy $\pi_k$ as the plug in estimator $\phi_\cost \spr{X_k, A_k}$ with $X_k, A_k$ sampled from the occupancy measure $\mu \spr{\pi_k}$. Notice that for the reinforcement learning applications, the adversarial reward sequence is generated online observing the policies. Therefore, for this application it is important that the guarantees in \Cref{thm:main} holds against adaptive adversaries.
%
\begin{algorithm}[t]
  \caption{\FRAalg (Feature Rmax Adversarial Imitation Learning) \label{alg:fra}}
  \centering
  \begin{algorithmic}[1]
    \STATE {\bfseries Inputs:} \\
    (1) a features dataset $\cD_\expert = \scbr{\phi_\cost \spr{X^i_E, A^i_E}}^{\tau_E}_{i=1}$ where for any $i \in \sbr{\tau_E}$, $X^i_E, A^i_E \sim \mu \spr{\expert}$, \\
    (2) read access to $\phi_P \spr{x, a}$ for all $x, a \in \cX \times \cA$, \\
    (3) trajectory access to $\cM \setminus \true$, and \\
    (4) the reward weights class $\cW$ such that $w_{\mathrm{true}} \in \cW$ and $\norm{w} \leq \WMAX$ for all $w \in \cW$.
    \STATE Set $K, \eta, \beta, \omega, \alpha$ as in \Cref{thm:main}.
    \STATE Set $\eta_r = \nicefrac{\WMAX}{B \sqrt{K}}$.
    \STATE Estimate $\widehat{\lambda \spr{\expert}} = \frac{1}{\abs{\cD_\expert}} \sum^{\tau_E}_{i=1} \phi_\cost \spr{X_E^i, A_E^i}$. \label{algline:expert_estimation}
    \STATE \textbf{Function} \label{algline:OGDstart}{$\textsc{Ogd}$}{$\spr{\mu \spr{\pi_k}, w_{k-1}}$}
    \STATE Sample $X_k, A_k \sim \mu \spr{\pi_k}$.
    \STATE $\quad \quad \text{\textbf{return}} \quad w_k = \Pi_{\cW} \spr{w_{k-1} + \eta_r \spr{\widehat{\lambda \spr{\expert}} - \phi_r \spr{X_k, A_k}}}$. \label{algline:OGDend}
    \STATE \textbf{Output:}  \algname $\spr{K, \eta, \beta, \omega, \alpha, \textsc{RewardUpdate} = \textsc{Ogd}}$.
  \end{algorithmic}
\end{algorithm}


\clearpage
\section{Omitted proofs from Section~\ref{sec:analysis}}
\label{app:analysis-proofs}

%%%%%%%%%%%%%%%%%%%%%%%%%%%%%%%%%%%%%%%%%%%%%%%%%%%%%%%%%%%%%%%%%%%%%%%%%%%%%%%%
%%%%%%%%%%%%%%%%%%%%%%%%%%%%%%%%%%%%%%%%%%%%%%%%%%%%%%%%%%%%%%%%%%%%%%%%%%%%%%%%
\subsection{Proof of Lemma~\ref{lem:reward-bias-bound} (reward bias)}
\label{app:reward-bias-bound}

\rewardbiasbound*

\begin{proof}
    First note that for any action $a$, the rewards are equal, \ie $r_k \spr{x^\upplus, a} = r_k^\upplus \spr{x^\upplus, a}$. For the other states, plugging the definition of $r_k^\upplus$ gives
    %
    \begin{align*}
        \rewardbias_k &= \inp{\mu \spr{\pi^\star} - \mu \spr{\pi_k}, r_k - r_k^\upplus} \\
        &= \inp{\mu \spr{\pi^\star} - \mu \spr{\pi_k}, p_k^\upplus \odot \spr{r_k - \RMAX \bfone}} \\
        &\leq - \inp{\mu \spr{\pi_k}, p_k^\upplus \odot \spr{r_k - \RMAX \bfone}} \\
        &\leq \RMAX \inp{\mu \spr{\pi_k}, p_k^\upplus}\,,
    \end{align*}
    %
    where the first inequality follows from $r_k - \RMAX \bfone \preceq 0$ and $\mu \spr{\pi^\star} \succeq 0$, and the second inequality is due to $r_k \succeq 0$.
\end{proof}


%%%%%%%%%%%%%%%%%%%%%%%%%%%%%%%%%%%%%%%%%%%%%%%%%%%%%%%%%%%%%%%%%%%%%%%%%%%%%%%%
%%%%%%%%%%%%%%%%%%%%%%%%%%%%%%%%%%%%%%%%%%%%%%%%%%%%%%%%%%%%%%%%%%%%%%%%%%%%%%%%
    \subsection{Proof of Lemma~\ref{lem:model-bias-bounds} (model bias)}
    \label{app:model-bias-bounds}

\modelbiasbounds*

\begin{proof}
    Let us consider a process  $\spr{X_\tau, A_\tau}_{\tau \in \bbN}$ generated by the policy $\pi$ in the real MDP, \ie, such that $X_0 \sim \initial$, and for any $\tau \in \bbN$, $A_\tau \sim \pi \spr{\cdot \given X_\tau}$, and $X_{\tau+1} \sim P \spr{\cdot \given X_\tau, A_\tau}$. We denote $\spr{X^\upplus_{k, \tau}, A^\upplus_{k, \tau}}_{\tau \in \bbN}$ its coupled process in the optimistic MDP at episode $k$ generated as follows. At the first stage we set $X^\upplus_{k, 0} = X_0$, then for any $\tau \geq 1$, the coupled process evolves as follows
    %
    \begin{equation*}
        X^\upplus_{k, \tau+1}, A^\upplus_{k, \tau+1} =
        \begin{cases}
            X_{\tau+1}, A_{\tau+1} \quad &\text{w.p.} \quad 1 - p^\upplus_k \spr{X_\tau, A_\tau} \quad \text{if} \quad  X^\upplus_{k, \tau}, A^\upplus_{k, \tau} = X_\tau, A_\tau \\
            x^\upplus, a \quad & \text{w.p.} \quad p^\upplus_k \spr{X_\tau, A_\tau} \quad \text{if} \quad  X^\upplus_{k, \tau}, A^\upplus_{k, \tau} = X_\tau, A_\tau \\
            x^\upplus, a \quad & \text{if} \quad X^\upplus_{k, \tau}, A^\upplus_{k, \tau} \neq X_\tau, A_\tau
        \end{cases}\,,
    \end{equation*}
    %
    where $a$ can be any action. Then, we can rewrite the bias term as
    %
    \begin{align*}
        &\modelbias_k \spr{\pi} = \spr{1 - \gamma} \bbE \sbr{\sum_{\tau=0}^\infty \gamma^\tau \spr{r_k^\upplus \spr{X^\upplus_{k, \tau}, A^\upplus_{k, \tau}} - r_k^\upplus \spr{X_\tau, A_\tau}}} \\
        &\quad\quad\quad= \spr{1 - \gamma} \bbE \sbr{\sum_{\tau=0}^\infty \gamma^\tau \II{\spr{X^\upplus_{k, \tau}, A^\upplus_{k, \tau}} \neq \spr{X_\tau, A_\tau}} \spr{r_k^\upplus \spr{X^\upplus_{k, \tau}, A^\upplus_{k, \tau}} - r_k^\upplus \spr{X_\tau, A_\tau}}}\,.
    \end{align*}
    %
    By definition, the state-action pairs $\spr{X^\upplus_{k, \tau}, A^\upplus_{k, \tau}}$ and $\spr{X_\tau, A_\tau}$ differ when the coupled process goes to heaven, \ie $X^\upplus_{k, \tau} = x^\upplus$. Noting that $r_k \spr{x^\upplus, a} = \RMAX$ for any action $a \in \cA$, we further get
    %
    \begin{align*}
        \modelbias_k \hspace{-1pt}\spr{\pi} &= \spr{1 - \gamma} \bbE \sbr{\sum_{\tau=0}^\infty \gamma^\tau \II{\spr{X^\upplus_{k, \tau}, A^\upplus_{k, \tau}} \neq \spr{X_\tau, A_\tau}} \hspace{-1pt}\spr{r_k^\upplus \spr{x^\upplus, A^\upplus_{k, \tau}} - r_k^\upplus \spr{X_\tau, A_\tau}}} \\
        &= \spr{1 - \gamma} \bbE \sbr{\sum_{\tau=0}^\infty \gamma^\tau \II{\spr{X^\upplus_{k, \tau}, A^\upplus_{k, \tau}} \neq \spr{X_\tau, A_\tau}} \spr{\RMAX - r_k^\upplus \spr{X_\tau, A_\tau}}}\,,
    \end{align*}
    %
    and $\modelbias_k \spr{\pi} \geq 0$ follows from $r_k^\upplus \preceq \RMAX$. For the upper bound, we can instead use $r_k \succeq 0$ and continue as follows
    %
    \begin{align*}
        \modelbias_k \spr{\pi} &\leq \spr{1 - \gamma} \RMAX \bbE \sbr{\sum_{\tau=0}^\infty \gamma^\tau \II{\spr{X^\upplus_{k, \tau}, A^\upplus_{k, \tau}} \neq \spr{X_\tau, A_\tau}} } \\
        &= \spr{1 - \gamma} \gamma \RMAX \sum_{\tau=0}^\infty \gamma^\tau \bbP \sbr{\spr{X^\upplus_{k, \tau+1}, A^\upplus_{k, \tau+1}} \neq \spr{X_{\tau+1}, A_{\tau+1}}}\,,
    \end{align*}
    %
    \begin{figure}
        \begin{center}
            \begin{tikzpicture}[every node/.style={circle, draw, minimum size=1cm, font=\small},
                >={Stealth}, % Arrow style
                shorten >=1pt, shorten <=1pt, % Adjust arrow spacing
                node distance=1.3cm and 1.3cm % Adjust node distances
            ]
            
            % true process
            \node (X0) at (0,0) {$X_{k, 0}$};
            \node (X1) [right=of X0] {$X_{k, 1}$};
            \node (X2) [right=1.3cm of X1] {$X_{k, 2}$};
            \node (X3) [right=of X2] {$X_{k, 3}$};
            \node (dots) [right=of X3, draw=none] {$\cdots$};
            
            % additional transitions for utopian process
            \node (uX0) [below=of X1] {$x^\upplus$};
            \node (uX1) [below=of X2] {$x^\upplus$};
            \node (uX2) [below=of X3] {$x^\upplus$};
            \node (udots) [right=of uX2, draw=none] {$\cdots$};
            
            % transitions true process
            \draw[->] (X0) -- (X1);
            \draw[->] (X1) -- (X2);
            \draw[->] (X2) -- (X3);
            \draw[->] (X3) -- (dots);
            
            % transitions utopian process
            \draw[->, dashed] (X0) to[bend left=20] (X1);
            \draw[->, dashed] (X0) -- (uX0);
            \draw[->, dashed] (X1) to[bend left=20] (X2);
            \draw[->, dashed] (X1) -- (uX1);
            \draw[->, dashed] (X2) to[bend left=20] (X3);
            \draw[->, dashed] (X2) -- (uX2);
            \draw[->, dashed] (X3) to[bend left=20] (dots);
            
            % stuck in heaven
            \draw[->, dashed] (uX0) -- (uX1);
            \draw[->, dashed] (uX1) -- (uX2);
            \draw[->, dashed] (uX2) -- (udots);
            
            \end{tikzpicture}
        \end{center}
        %
        \caption{The thick arrows represent the transitions of the process in the original MDP, while the dashed ones correspond to the utopian one.}
        %
        \label{fig:bias-term}
    \end{figure}
    %
    \hspace{-4.5pt}where we used $\bbP \sbr{\spr{X^\upplus_{k, 0}, A^\upplus_{k, 0}} \neq \spr{X_0, A_0}} = 0$ by definition. Then, as illustrated in Figure~\ref{fig:bias-term}, two cases can happen. Either the coupled process was still in the original MDP and transitioned to heaven, either it was already in heaven. Denoting for any $\tau \geq 0$ the event $\cE_\mathrm{split} \spr{\tau} = \scbr{\spr{X^\upplus_{k, \tau}, A^\upplus_{k, \tau}} \neq \spr{X_\tau, A_\tau}}$ and $\mathcal{E}^c_\mathrm{split} \spr{\tau}$ its complementary, we have
    %
    \begin{align*}
        \bbP \sbr{\cE_\mathrm{split} \spr{\tau+1}} &= \bbP \sbr{\cE_\mathrm{split} \spr{\tau+1} \given \cE_\mathrm{split} \spr{\tau}} \bbP \sbr{\cE_\mathrm{split} \spr{\tau}} \\
        &\phantom{=}+ \bbP \sbr{\cE_\mathrm{split} \spr{\tau+1} \given \cE^c_\mathrm{split} \spr{\tau}} \bbP \sbr{\cE^c_\mathrm{split} \spr{\tau}}\,.
    \end{align*}
    %
    If the coupled process is already in the heaven state $x^\upplus$, then it stays there. Otherwise, it can transition there with probability $\bbE \sbr{p_k^\upplus \spr{X_\tau, A_\tau}}$, thus
    %
    \begin{align*}
        \bbP \sbr{\cE_\mathrm{split} \spr{\tau+1}} &= \bbP \sbr{\cE_\mathrm{split} \spr{\tau}} + \bbE \sbr{p_k^\upplus \spr{X_\tau, A_\tau}} \bbP \sbr{\cE^c_\mathrm{split} \spr{\tau}} \\
        &\leq \bbP \sbr{\cE_\mathrm{split} \spr{\tau}} + \bbE \sbr{p_k^\upplus \spr{X_\tau, A_\tau}} \\
        &\leq \sum_{u=0}^\tau \bbE \sbr{p_k^\upplus \spr{X_u, A_u}}\,,
    \end{align*}
    %
    by induction. Therefore, we get
    %
    \begin{align*}
        \modelbias_k \spr{\pi} &\leq \spr{1 - \gamma} \gamma \RMAX \sum_{\tau=0}^\infty \gamma^\tau \sum_{u=0}^\tau \bbE \sbr{p_k^\upplus \spr{X_u, A_u}} \\
        &= \spr{1 - \gamma} \gamma \RMAX \bbE \sbr{\sum_{u=0}^\infty \sum_{\tau=0}^\infty \gamma^\tau \II{u \leq \tau} p_k^\upplus \spr{X_u, A_u}} \\
        &= \spr{1 - \gamma} \gamma \RMAX \bbE \sbr{\sum_{u=0}^\infty \sum_{\tau=u}^\infty \gamma^\tau p_k^\upplus \spr{X_u, A_u}} \\
        &= \gamma \RMAX \bbE \sbr{\sum_{u=0}^\infty \gamma^u p_k^\upplus \spr{X_u, A_u}}\,,
    \end{align*}
    %
    and the conclusion follows from the definition of $\mu \spr{\pi}$ and $\gamma < 1$.
\end{proof}


%%%%%%%%%%%%%%%%%%%%%%%%%%%%%%%%%%%%%%%%%%%%%%%%%%%%%%%%%%%%%%%%%%%%%%%%%%%%%%%%
        \subsubsection{Alternative Proof of Lemma~\ref{lem:model-bias-bounds}}

\noindent We also provide an alternative proof based on a simulation lemma.
%
\begin{proof}
    By the flow constraints associated to $\mu_k^\upplus \spr{\pi}$ and after rearranging, we get
    %
    \begin{align*}
        \modelbias_k \spr{\pi} &= \spr{1 - \gamma} \inp{\nu_0, V_{P_k^\upplus, r_k^\upplus}^\pi - V_{P, r_k^\upplus}^\pi} \\
        &= \inp{\opE\transpose \mu_k^\upplus \spr{\pi} - \gamma \spr{\opPplusk}\transpose \mu_k^\upplus \spr{\pi}, V_{P_k^\upplus, r_k^\upplus}^\pi - V_{P, r_k^\upplus}^\pi} \\
        &= \inp{\mu_k^\upplus \spr{\pi}, \opE V_{P_k^\upplus, r_k^\upplus}^\pi - \opE V_{P, r_k^\upplus}^\pi - \gamma \opPplusk V_{P_k^\upplus, r_k^\upplus}^\pi + \gamma \opPplusk V_{P, r_k^\upplus}^\pi}\,.
    \end{align*}
    %
    Applying Lemma~\ref{lem:from-ev-to-q} to both $V_{P_k^\upplus, r_k^\upplus}^\pi$ and $V_{P, r_k^\upplus}^\pi$ and Bellman's equation for $Q_{P_k^\upplus, r_k^\upplus}^\pi$, we further have    
    %
    \begin{align*}
        \modelbias_k \spr{\pi} &= \inp{\mu_k^\upplus \spr{\pi}, Q_{P_k^\upplus, r_k^\upplus}^\pi - Q_{P, r_k^\upplus}^\pi - \gamma \opPplusk V_{P_k^\upplus, r_k^\upplus}^\pi + \gamma \opPplusk V_{P, r_k^\upplus}^\pi} \\
        &= \inp{\mu_k^\upplus \spr{\pi}, r_k^\upplus + \gamma \opPplusk V_{P, r_k^\upplus}^\pi - Q_{P, r_k^\upplus}^\pi} \,,
    \end{align*}
    %
    Plugging the definition of $P_k^\upplus$ and this time using Bellman's equation for $Q_{P, r_k^\upplus}^\pi$, we obtain
    %
    \begin{align}
        \modelbias_k \spr{\pi} &= \inp{\mu_k^\upplus \spr{\pi}, r_k^\upplus + \gamma \spr{1 - p_k^\upplus} \odot \opP V_{P, r_k^\upplus}^\pi + p_k^\upplus \odot \bfe_{x^\upplus} V_{P, r_k^\upplus}^\pi - Q_{P, r_k^\upplus}^\pi} \nonumber \\
        &= \inp{\mu_k^\upplus \spr{\pi}, p_k^\upplus \odot \spr{\bfe_{x^\upplus} V_{P, r_k^\upplus}^\pi - \gamma \opP V_{P, r_k^\upplus}^\pi}} \nonumber \\
        &= \inp{\mu_k^\upplus \spr{\pi}, p_k^\upplus \odot \spr{\frac{\RMAX}{1 - \gamma} \bfone - \gamma \opP V_{P, r_k^\upplus}^\pi}}\,, \label{eq:hope-decomp}
    \end{align}
    %
    where the last equality is due to having $\spr{\bfe_{x^\upplus} V_{P, r_k^\upplus}^\pi} \spr{x, a} = V_{P, r_k^\upplus}^\pi \spr{x^\upplus} = \frac{\RMAX}{1 - \gamma}$ for any state-action pair $\spr{x, a}$. The lower bound follows from noticing that $\opP V_{P, r_k^\upplus}^\pi \preceq \frac{1}{1 - \gamma} \bfone \preceq \frac{\RMAX}{1 - \gamma} \bfone$,
    %
    \begin{align*}
        \modelbias_k \spr{\pi} &\geq \spr{\frac{\RMAX}{1 - \gamma} - \frac{\gamma \RMAX}{1 - \gamma}} \cdot \inp{\mu_k^\upplus \spr{\pi}, p_k^\upplus} \\
        &= \RMAX \cdot \inp{\mu_k^\upplus \spr{\pi}, p_k^\upplus} \\
        &\geq 0\,.
    \end{align*}
    %
    Moving to the upper bound, from Equation~\ref{eq:hope-decomp} and $\opP V_{P, r_k^\upplus}^\pi \succeq 0$, we get
    %
    \begin{align*}
        \modelbias_k \spr{\pi} &= \inp{\mu_k^\upplus \spr{\pi}, p_k^\upplus \odot \spr{\frac{\RMAX}{1 - \gamma} \bfone - \gamma \opP V_{P, r_k^\upplus}^\pi}} \\
        &\leq \frac{\RMAX}{1 - \gamma} \inp{\mu_k^\upplus \spr{\pi}, p_k^\upplus} \\
        &\leq \frac{\RMAX}{1 - \gamma} \inp{\mu \spr{\pi}, p_k^\upplus}\,,
    \end{align*}
    %
    where the last inequality follows from Lemma~\ref{lem:mass-reduced}.
\end{proof}


%%%%%%%%%%%%%%%%%%%%%%%%%%%%%%%%%%%%%%%%%%%%%%%%%%%%%%%%%%%%%%%%%%%%%%%%%%%%%%%%
%%%%%%%%%%%%%%%%%%%%%%%%%%%%%%%%%%%%%%%%%%%%%%%%%%%%%%%%%%%%%%%%%%%%%%%%%%%%%%%%
\subsection{Proof of Lemma~\ref{lem:bound-regret-plus} (optimistic regret)}
\label{app:bound-regret-plus}

In order to prove Lemma~\ref{lem:bound-regret-plus}, we first need the following result that shows that the functions $Q_k$ are optimistic estimates of an ideal sequence of dynamic-programming updates computed in the augmented MDPs. The statement is is an adaptation of Lemma~4.2 of \citet{MN23}, and its complete proof is provided below.
%
\begin{restatable}{Lem}{boundq} \label{lem:bound-q}
    Suppose that the bonuses $\CB_k$ are valid for the MDP $\cM_k$ in the sense of Equation~\ref{eq:validity-bonuses}. Then, for any $k$ and any state-action pair $\spr{x, a} \in \cX^\upplus \times \cA$, the iterates $Q_k$ satisfy
    %
    \begin{equation*} \label{eq:bound-q}
        r_k^\upplus + \gamma \opPplusk V_k \leq Q_{k+1} \leq r_k^\upplus + 2 \spr{1 - p_k^\upplus} \odot \CB_k + \gamma \opPplusk V_k\,.
    \end{equation*}
\end{restatable}

\begin{proof}
    For $x^\upplus$ and any action $a$, it is straightforward to check that both inequalities are equalities. Let $\spr{x, a} \in \cX \times \cA$. We have
    %
    \begin{align*}
        r_k^\upplus \spr{x, a} + \gamma \opPplusk V_k \spr{x, a} &= r_k^\upplus \spr{x, a} + \gamma \spr{\opPplusk - \ophPplusk} V_k \spr{x, a} + \gamma \ophPplusk V_k \spr{x, a} \\
        &\leq Q_{k+1} \spr{x, a} \\
        &\leq r_k^\upplus \spr{x, a} + 2 \spr{1 - p_k^\upplus \spr{x, a}} \CB_k \spr{x, a} + \gamma \opPplusk V_k \spr{x, a}\,,
    \end{align*}
    %
    where both inequalities use the fact that
    %
    \begin{equation*}
        \abs{\spr{\opPplusk - \ophPplusk} V_k \spr{x, a}} \leq \spr{1 - p_k^\upplus \spr{x, a}} \CB_k \spr{x, a}\,,
    \end{equation*}
    %
    which is implied by the event $\mathcal{E}_{\mathrm{valid}}$ in Equation~\ref{eq:validity-bonuses}.
\end{proof}

\boundregretplus*

\begin{proof}
    We decompose $\regretKplus$ as follows
    %
    \begin{align*}
        \regretKplus &= \sumkK \Bigl(\underbrace{\spr{\inp{\mu_k^\upplus \spr{\pi^\star}, r_k^\upplus} - \spr{1 - \gamma} \inp{\initial, V_k}}}_{= \Delta_k^\star} + \underbrace{\spr{1 - \gamma} \inp{\initial, V_k} - \inp{\mu_k^\upplus \spr{\pi_k}, r_k^\upplus}}_{= \Delta_k}\Bigr)\,,
    \end{align*}
    %
    where we defined $\Delta_k^\star$ and $\Delta_k$. We start with the first term. Using the flow constraint with $\mu_k^\upplus \spr{\pi^\star}$,
    %
    \begin{align*}
        \Delta_k^\star &= \inp{\mu_k^\upplus \spr{\pi^\star}, r_k^\upplus} - \inp{\opE\transpose \mu_k^\upplus \spr{\pi^\star} - \gamma \spr{\opPplusk}\transpose \mu_k^\upplus \spr{\pi^\star}, V_k} \\
        &= \inp{\mu_k^\upplus \spr{\pi^\star}, r_k^\upplus + \gamma \opPplusk V_k - \opE V_{k+1}} + \inp{\mu_k^\upplus \spr{\pi^\star}, \opE V_{k+1} - \opE V_k}\,.
    \end{align*}
    %
    Using the lower bound on $Q_{k+1}$ from Lemma~\ref{lem:bound-q}, we have
    %
    \begin{equation*}
        \Delta_k^\star \leq \inp{\mu_k^\upplus \spr{\pi^\star}, Q_{k+1} - \opE V_{k+1}} + \inp{\mu_k^\upplus \spr{\pi^\star}, \opE V_{k+1} - \opE V_k}\,,
    \end{equation*}
    %
    where the term in $x = x^\upplus$ is equal to zero. Summing over $k \in \sbr{K} = \bigcup_{e \in \sbr{1, E \spr{K}}} \sbr{k_e, k_{e+1} - 1}$,
    %
    \begin{equation*}
        \sumkK \Delta_k^\star \leq \sum_{e=1}^{E \spr{K}} \inp{\mu_{k_e}^\upplus \spr{\pi^\star}, \sum_{k \in \cK_e} \spr{Q_{k+1} - \opE V_{k+1}}} + \sum_{e=1}^{E \spr{K}} \inp{\opE\transpose \mu_{k_e}^\upplus \spr{\pi^\star}, V_{k_{e+1}} - V_{k_e}}_{\cX}\,,
    \end{equation*}
    %
    where the sum within each epoch of the second term telescoped. By \citealp[Lemma~C.1]{MN23}, we have for any state $x \in \cX$
    %
    \begin{align*}
        \sum_{k \in \cK_e} V_{k+1} \spr{x} &= \max_{p \in \Delta \spr{\cA}} \inp{p, \sum_{k \in \cK_e} Q_{k+1} \spr{x, \cdot}} - \frac1\eta \KL \spr{p \| \piunif} \\
        &\geq \inp{\pi^\star \spr{\cdot \given x}, \sum_{k \in \cK_e} Q_{k+1} \spr{x, \cdot}} - \frac1\eta \KL \spr{\pi^\star \spr{\cdot \given x} \| \piunif}\,,
    \end{align*}
    %
    where we used $\pi_{k_e} = \piunif$ in the first equality and denoted $\cK_e$ the set of episodes in epoch $e$. Multiplying the previous inequality by $\nu_{k_e}^\upplus \spr{\pi^\star,x}$, summing over $x \in \cX$, and noting that $\mu_{k_e}^\upplus \spr{\pi^\star} = \nu_{k_e}^\upplus \spr{\pi^\star} \odot \pi^\star$, we obtain
    %
    \begin{align*}
        \sum_{e=1}^{E \spr{K}} \inp{\mu_{k_e}^\upplus \spr{\pi^\star}, \sum_{k \in \cK_e} \spr{Q_{k+1} - \opE V_{k+1}}} &\leq \frac1\eta \sum_{e=1}^{E \spr{K}} \inp{\nu_{k_e}^\upplus \spr{\pi^\star}, \KL \spr{\pi^\star \| \piunif}} \\
        &\leq \frac{E \spr{K} \log \abs{\cA}}{\eta}\,.
    \end{align*}
    %
    The second term can be bounded with Hölder's inequality,
    %
    \begin{equation*}
        \sum_{e=1}^{E \spr{K}} \inp{\opE\transpose \mu_{k_e}^\upplus \spr{\pi^\star}, V_{k_{e+1}} - V_{k_e}} \leq \sum_{e=1}^{E \spr{K}} \norm{\nu_{k_e}^\upplus \spr{\pi^\star}}_1 \norm{V_{k_{e+1}} - V_{k_e}}_\infty \leq 2 E \spr{K} \VMAX\,,
    \end{equation*}
    %
    where we used $\nu_{k_e}^\upplus \spr{\pi^\star} \in \Delta \spr{\cX^\upplus}$ and $\norm{V_k}_\infty \leq \VMAX$ which follows from $\norm{Q_k}_\infty \leq \QMAX$. Therefore, we get
    %
    \begin{equation*}
        \sumkK \Delta_k^\star \leq \frac{E \spr{K} \log \abs{\cA}}{\eta} + 2 E \spr{K} \VMAX\,.
    \end{equation*}

    \noindent Moving to $\Delta_k$, we apply the flow constraints to $\mu_k^\upplus \spr{\pi_k}$ to get
    %
    \begin{align*}
        \Delta_k &= \inp{\opE\transpose \mu_k^\upplus \spr{\pi_k} - \gamma \spr{\opPplusk}\transpose \mu_k^\upplus \spr{\pi_k}, V_k} - \inp{\mu_k^\upplus \spr{\pi_k}, r_k^\upplus} \\
        &= \inp{\mu_k^\upplus \spr{\pi_k}, \opE V_k} - \inp{\mu_k^\upplus \spr{\pi_k}, r_k^\upplus + \gamma \opPplusk V_k} \\
        &\leq \inp{\mu_k^\upplus \spr{\pi_k}, \opE V_k - Q_{k+1}} + 2 \inp{\mu_k^\upplus \spr{\pi_k}, \spr{1 - p_k^\upplus} \odot \CB_k} \\
        &\leq \inp{\mu_k^\upplus \spr{\pi_k}, \opE V_k - Q_{k+1}} + 2 \inp{\mu \spr{\pi_k}, \spr{1 - p_k^\upplus} \odot \CB_k}\,,
    \end{align*}
    %
    where the first inequality follows from the upper bound on $Q_{k+1}$ in Lemma~\ref{lem:bound-q} and the term in $x = x^\upplus$ being equal to zero, and the second inequality is due to Lemma~\ref{lem:mass-reduced}. Next, noticing $\inp{\mu_k^\upplus \spr{\pi_{k+1}}, Q_{k+1}} = \inp{\nu_k^\upplus \spr{\pi_{k+1}}, V_{k+1} + \frac1\eta \KL \spr{\pi_{k+1} \| \pi_k}}$,
    %
    \begin{align*}
        \inp{\mu_k^\upplus \spr{\pi_k}, \opE V_k - Q_{k+1}} &= \inp{\nu_k^\upplus \spr{\pi_k}, V_k} - \inp{\mu_k^\upplus \spr{\pi_{k+1}}, Q_{k+1}} \\
        &\phantom{=}+ \inp{\mu_k^\upplus \spr{\pi_{k+1}}, Q_{k+1}} - \inp{\mu_k^\upplus \spr{\pi_k}, Q_{k+1}} \\
        &= \inp{\nu_k^\upplus \spr{\pi_k}, V_k} - \inp{\nu_k^\upplus \spr{\pi_{k+1}}, V_{k+1}} \\
        &\phantom{=}- \frac1\eta \inp{\nu_k^\upplus \spr{\pi_{k+1}}, \KL \spr{\pi_{k+1} \| \pi_k}}  \\
        &\phantom{=}+ \inp{\mu_k^\upplus \spr{\pi_{k+1}} - \mu_k^\upplus \spr{\pi_k}, Q_{k+1}}\,.
    \end{align*}
    %
    We sum over $k$ and look at the different terms separately. First, we get
    %
    \begin{align*}
        \sumkK \spr{\inp{\nu_k^\upplus \spr{\pi_k}, V_k} - \inp{\nu_k^\upplus \spr{\pi_{k+1}}, V_{k+1}}} &= \sum_{e=1}^{E \spr{K}} \spr{\inp{\nu_{k_e}^\upplus \spr{\pi_{k_e}}, V_{k_e}} - \inp{\nu_{k_e}^\upplus \spr{\pi_{k_{e+1}}}, V_{k_{e+1}}}} \\
        &\leq 2 \VMAX E \spr{K}\,.
    \end{align*}
    %
    For the third term, we have
    %
    \begin{align*}
        \sumkK \inp{\mu_k^\upplus \spr{\pi_{k+1}} - \mu_k^\upplus \spr{\pi_k}, Q_{k+1}} &= \sum_{e=1}^{E \spr{K}} \sum_{k \in \cK_e} \inp{\mu_{k_e}^\upplus \spr{\pi_{k+1}} - \mu_{k_e}^\upplus \spr{\pi_k}, Q_{k+1}}\,.
    \end{align*}
    %
    Successively applying Hölder's inequality, Pinsker's inequality and Lemma~\ref{lem:ineq-kl-entropy},
    %
    \begin{align*}
        \inp{\mu_{k_e}^\upplus \spr{\pi_{k+1}} - \mu_{k_e}^\upplus \spr{\pi_k}, Q_{k+1}} &\leq \QMAX \norm{\mu_{k_e}^\upplus \spr{\pi_{k+1}} - \mu_{k_e}^\upplus \spr{\pi_k}}_1 \\
        &\leq \QMAX \sqrt{2 \KL \spr{\mu_{k_e}^\upplus \spr{\pi_{k+1}} \| \mu_{k_e}^\upplus \spr{\pi_k}}} \\
        &\leq \QMAX \sqrt{\frac{2}{1 - \gamma} \inp{\nu_{k_e}^\upplus \spr{\pi_{k+1}}, \KL \spr{\pi_{k+1} \| \pi_k}}}\,.
    \end{align*}
    %
    For any $x \in \cX$, the KL divergence between $\pi_{k+1}$ and $\pi_k$ in state $x$ can be bounded as
    %
    \begin{align*}
        &\KL \spr{\pi_{k+1} \middle\| \pi_k} \spr{x} \\
        &\quad\quad\quad= \sum_{a \in \cA} \pi_{k+1} \spr{a \given x} \spr{\eta Q_{k+1} \spr{x, a} - \log \spr{\sum_{b \in \cA} \pi_k \spr{b \given x} \exp \sbr{\eta Q_{k+1} \spr{x, b}}}} \\
        &\quad\quad\quad= \eta \sum_{a \in \cA} \pi_{k+1} \spr{a \given x} Q_{k+1} \spr{x, a} - \log \spr{\sum_{b \in \cA} \pi_k \spr{b \given x} \exp \sbr{\eta Q_{k+1} \spr{x, b}}} \\
        &\quad\quad\quad\leq \eta \sum_{a \in \cA} \sbr{\pi_{k+1} \spr{a \given x} - \pi_k \spr{a \given x}} Q_{k+1} \spr{x, a} \\
        &\quad\quad\quad\leq \eta \QMAX \norm{\pi_{k+1} \spr{\cdot \given x} - \pi_k \spr{\cdot \given x}}_1 \\
        &\quad\quad\quad\leq \eta \QMAX \sqrt{2 \KL \spr{\pi_{k+1} \middle\| \pi_k} \spr{x}}\,,
    \end{align*}
    %
    where the first inequality follows from Jensen's and the convexity of $- \log$, the second inequality is by Hölder's and the boundedness of $Q_k$, and the last inequality is due to Pinkser's. Dividing by $\sqrt{\KL \spr{\pi_{k+1} \middle\| \pi_k} \spr{x}}$ and squaring the inequality, we get
    %
    \begin{equation*}
        \KL \spr{\pi_{k+1} \middle\| \pi_k} \spr{x} \leq 2 \eta^2 \QMAX^2\,.
    \end{equation*}
    %
    Plugging this back into the previous inequality and summing over $k \in \sbr{K}$, we get
    %
    \begin{equation*}
        \sumkK \inp{\mu_k^\upplus \spr{\pi_{k+1}} - \mu_k^\upplus \spr{\pi_k}, Q_{k+1}} \leq \frac{2 \QMAX^2 \eta K}{\sqrt{1 - \gamma}}\,.
    \end{equation*}
    %
    The remaining term is nonpositive. The sum of the $\Delta_k$ terms is thus bounded by
    %
    \begin{equation*}
        \sumkK \Delta_k \leq 2 \VMAX E \spr{K} + \frac{2 \QMAX^2 \eta K}{\sqrt{1 - \gamma}} + 2 \sumkK \inp{\mu \spr{\pi_k}, \spr{1 - p_k^\upplus} \odot \CB_k}\,.
    \end{equation*}
    %
    Finally, combining the bounds on $\sumkK \Delta_k^\star$ and $\sumkK \Delta_k$, we get
    \begin{equation*}
        \regretKplus \leq \frac{E \spr{K} \log \abs{\cA}}{\eta} + 4 \VMAX E \spr{K} + \frac{2 \QMAX^2 \eta K}{\sqrt{1 - \gamma}} + 2 \sumkK \inp{\mu \spr{\pi_k}, \spr{1 - p_k^\upplus} \odot \CB_k}\,.
    \end{equation*}
\end{proof}


%%%%%%%%%%%%%%%%%%%%%%%%%%%%%%%%%%%%%%%%%%%%%%%%%%%%%%%%%%%%%%%%%%%%%%%%%%%%%%%%
%%%%%%%%%%%%%%%%%%%%%%%%%%%%%%%%%%%%%%%%%%%%%%%%%%%%%%%%%%%%%%%%%%%%%%%%%%%%%%%%
\subsection{Proof of Lemma~\ref{lem:qmax} (choice of $\QMAX$)}
\label{app:qmax}

\qmaxlemma*

\begin{proof}
    We want to find $\QMAX$ such that for any $k$, $\norm{Q_k}_\infty \leq \QMAX$. Since $V_k$ is a log-sum-exp of $Q_k$, we have $\norm{V_k}_\infty \leq \norm{Q_k}_\infty$. Next, we proceed by induction to find a suitable value of $\QMAX$. Let $k \in \bbN^\star$ and assume $\norm{Q_k}_\infty \leq \QMAX$. For any $\spr{x, a}$,
    %
    \begin{align*}
        \abs{Q_{k+1} \spr{x, a}} &\leq r_k^\upplus \spr{x, a} + \spr{1 - p_k^\upplus \spr{x, a}} \CB_k \spr{x, a} + \gamma \abs{\ophPplusk V_k \spr{x, a}} \\
        &\leq r_k^\upplus \spr{x, a} + 2 \spr{1 - p_k^\upplus \spr{x, a}} \CB_k \spr{x, a} + \gamma \opPplusk V_k \spr{x, a} \\
        &\leq \RMAX + 2 \spr{1 - p_k^\upplus \spr{x, a}} \CB_k \spr{x, a} + \gamma \QMAX\,,
    \end{align*}
    %
    where the second inequality follows from the validity of the bonuses (and corresponds to the upper bound on $Q_{k+1}$ from Lemma~\ref{lem:bound-q}), and the third inequality is due to the inductive assumption and the boundedness of the rewards. Plugging the definition of the probabilities $p_k^\upplus$, we further get
    %
    \begin{align*}
        \abs{Q_{k+1} \spr{x, a}} &\leq \RMAX + 2 \sigma \spr{\omega - \alpha \CB_k \spr{x, a}} \CB_k \spr{x, a} + \gamma \QMAX \\
        &\leq \RMAX + \gamma \QMAX + 2 \sup_{z \geq 0} \scbr{\sigma \spr{\omega - \alpha z} z} \\
        &\leq \RMAX + \gamma \QMAX + \frac{2 \omega}{\alpha}\,,
    \end{align*}
    %
    the last inequality is simply a property of the sigmoid function and is showm in Lemma~\ref{lem:sigmoid-bound2}. For the induction to work at time $k + 1$, we need to set $\QMAX$ such that $\QMAX = \RMAX + \gamma \QMAX + \frac{2 \omega}{\alpha}$, that is
    %
    \begin{equation*}
        \QMAX = \frac{\RMAX + 2 \omega / \alpha}{1 - \gamma}\,.
    \end{equation*}
    %
    The initial case is also true since $\norm{Q_1}_\infty \leq \frac{\RMAX}{1 - \gamma} \leq \QMAX$.
\end{proof}


%%%%%%%%%%%%%%%%%%%%%%%%%%%%%%%%%%%%%%%%%%%%%%%%%%%%%%%%%%%%%%%%%%%%%%%%%%%%%%%%
%%%%%%%%%%%%%%%%%%%%%%%%%%%%%%%%%%%%%%%%%%%%%%%%%%%%%%%%%%%%%%%%%%%%%%%%%%%%%%%%
\subsection{Proof of Lemma~\ref{lem:expected-bonuses-bound} (bound on bonuses)}
\label{app:expected-bonuses-bound}

We now control the sum of bonuses. For any episode $k$, we will denote $\cT_k$ the set of timesteps in episode $k$.

\expectedbonusesbound*

\noindent To prove Lemma~\ref{lem:expected-bonuses-bound}, we need the following result.
%
\begin{lemma} \label{lem:expectation-to-highprob}
    Suppose $\mathcal{E}_L$ holds. Let $\scbr{f_k}_{k \in \sbr{K}} \subset \bbR^{\cX \times \cA}$ be a sequence of functions with values in $\sbr{0, M}$ almost surely. Then, with probability at least $1 - \delta$ the schedule and policies produced by Algorithm~\ref{alg:linear-rmax-ravi-ucb} satisfy
    %
    \begin{equation*}
        \sumkK \inp{\mu \spr{\pi_k}, f_k} \leq 2 \spr{1 - \gamma} \sumkK \sum_{t \in \cT_k} f_k \spr{X_t, A_t} + 4 M \log \spr{\frac{2 K}{\delta}}^2\,.
    \end{equation*}
\end{lemma}
%
\begin{proof}
    We denote $\cF_{k-1}$ the $\sigma$-field generated by the history up to the end of episode $k-1$. We have,
    %
    \begin{align*}
        \inp{\mu \spr{\pi_k}, f_k} &= \spr{1 - \gamma} \bbE \sbr{\sum_{\tau = 0}^\infty \gamma^\tau f_k \spr{X_\tau, A_\tau} \given \cF_{k-1}} \\
        &= \spr{1 - \gamma} \bbE \sbr{\sum_{\tau = 0}^\infty \II{\tau < L_k} f_k \spr{X_\tau, A_\tau} \given \cF_{k-1}} \\
        &= \spr{1 - \gamma} \bbE \sbr{\sum_{\tau = 0}^{L_k - 1} f_k \spr{X_\tau, A_\tau} \given \cF_{k-1}} \\
        &= \spr{1 - \gamma} \bbE \sbr{\sum_{\tau \in \cT_k} f_k \spr{X_\tau, A_\tau} \given \cF_{k-1}}\,.
    \end{align*}
    %
    Plugging it back in the previous display,
    %
    \begin{align*}
        \sumkK \inp{\mu \spr{\pi_k}, f_k} &= \spr{1 - \gamma} \sumkK \bbE \sbr{\sum_{t \in \cT_k} f_k \spr{X_t, A_t} \given \cF_{k-1}}\,.
    \end{align*}
    %
    Since we assume $\cE_L$ holds, for any $k$ we have that $\sum_{t \in \cT_k} f_k \spr{X_t, A_t}$ takes values in $\sbr{0, M \LMAX}$ almost surely. Using the concentration inequality from Lemma~\ref{lem:concentration-ineq-cond-exp}, we get
    %
    \begin{align*}
        \sumkK \inp{\mu \spr{\pi_k}, f_k} &\leq 2 \spr{1 - \gamma} \sumkK \sum_{t \in \cT_k} f_k \spr{X_t, A_t} + 4 \spr{1 - \gamma} M \LMAX \log \spr{\frac{2 K}{\delta}} \\
        &\leq 2 \spr{1 - \gamma} \sumkK \sum_{t \in \cT_k} f_k \spr{X_t, A_t} + 4 M \log \spr{\frac{2 K}{\delta}}^2\,,
    \end{align*}
    %
    where we used that $\LMAX = \frac{\log \spr{K / \delta}}{1 - \gamma}$.
\end{proof}
%

We now prove Lemma~\ref{lem:expected-bonuses-bound}. With a slight abuse of notation, we use the convention that for any epoch $e$ and any $t$ in epoch $e$, the bonuses at time step $t$ are $\CB_t = \CB_{t_e}$. Noting that the bonuses $\CB_t$ take values in $\sbr{0, \beta B}$, we apply Lemma~\ref{lem:expectation-to-highprob} to get
%
\begin{equation*}
    \sumkK \inp{\mu \spr{\pi_k}, \CB_k} \leq 2 \spr{1 - \gamma} \sumtT \CB_t \spr{X_t, A_t} + 4 \beta B \log \spr{\frac{2 K}{\delta}}^2\,,
\end{equation*}
%
where we used $\CB_t \succeq 0$, and $T_{K+1} \leq T$ which follows from the event $\cE_L$. Likewise, applying Lemma~\ref{lem:expectation-to-highprob} to $\CB_t^2 \in \sbr{0, \beta^2 B^2}$, we obtain a similar bound for the term $\sumkK \inp{\mu \spr{\pi_k}, \CB_k^2}$,
%
\begin{equation*}
    \sumkK \inp{\mu \spr{\pi_k}, \CB_k^2} \leq 2 \spr{1 - \gamma} \sumtT \CB_t \spr{X_t, A_t}^2 + 4 \beta^2 B^2 \log \spr{\frac{2 K}{\delta}}^2\,.
\end{equation*}
%
By Cauchy-Schwartz's inequality, we have $\sumtT \CB_t \spr{X_t, A_t} \leq \sqrt{T} \sqrt{\sumtT \CB_t \spr{X_t, A_t}^2}$, so we can focus on the latter sum. By definition of the bonuses for linear MDPs, we have
%
\begin{align*}
    \sumtT \CB_t \spr{X_t, A_t}^2 &= \beta^2 \sum_{e=1}^{E \spr{K}} \sum_{k \in \cK_e} \sum_{t \in \cT_k} \norm{\phi \spr{X_t, A_t}}_{\Lambda_{t_e}^{-1}}^2\,.
\end{align*}
%
Since the covariance matrix only contains the data until the beginning of the epoch, there is a delay with $\phi \spr{X_t, A_t}$ which is further ahead. To compensate for this, note that for any $t \in \sbr{t_e, t_{e + 1}-1}$, we have $\det \Lambda_t \leq 2 \det \Lambda_{t_e}$ due to the update condition in Algorithm~\ref{alg:linear-rmax-ravi-ucb}, so by Lemma~\ref{lem:det-elliptical-bound}
%
\begin{align*}
    \norm{\phi \spr{X_t, A_t}}_{\Lambda_{t_e}^{-1}}^2 \leq \frac{\det \spr{\Lambda_{t_e}^{-1}}}{\det \spr{\Lambda_t^{-1}}} \norm{\phi \spr{X_t, A_t}}_{\Lambda_t^{-1}}^2 \leq 2 \norm{\phi \spr{X_t, A_t}}_{\Lambda_t^{-1}}^2\,.
\end{align*}
%
We plug this back into the previous inequality and apply Lemma~\ref{lem:bound-elliptical-potential} to obtain\footnote{Note that this is where the linear dependency in $B$ appears, but this can be removed by setting $\lambda = 1 / B^2$.}
%
\begin{equation*}
    \sumtT \CB_t \spr{X_t, A_t}^2 \leq 2 \beta^2 \sumtT \norm{\phi \spr{X_t, A_t}}_{\Lambda_t^{-1}}^2 \leq 4 \beta^2 B^2 \log \spr{\frac{\det \Lambda_T}{\det \Lambda_0}}\,.
\end{equation*}
%
Using the definition of $\Lambda_0, \Lambda_T$, the trace-determinant inequality, and the assumption $\norm{\phi \spr{\cdot, \cdot}}_2 \leq B$, we finally get
%
\begin{align*}
    \sumtT \CB_t \spr{X_t, A_t}^2 &\leq 4 \beta^2 B^2 d \log \spr{\frac{d + \sumtT \norm{\phi \spr{X_t, A_t}}_2^2}{d}} \\
    &\leq 4 \beta^2 B^2 d \log \spr{1 + \frac{B^2 T}{d}}\,.
\end{align*}
%
The conclusion follows from plugging this back into the inequalities of interest.


%%%%%%%%%%%%%%%%%%%%%%%%%%%%%%%%%%%%%%%%%%%%%%%%%%%%%%%%%%%%%%%%%%%%%%%%%%%%%%%%
%%%%%%%%%%%%%%%%%%%%%%%%%%%%%%%%%%%%%%%%%%%%%%%%%%%%%%%%%%%%%%%%%%%%%%%%%%%%%%%%
\subsection{Proof of Lemma~\ref{lem:good-event-holds} (good event holds)}
\label{app:good-event-holds}

Before stating the proof of Lemma~\ref{lem:good-event-holds}, we need to define some auxiliary quantities and state two intermediate results. First recall that $\scbr{L_k}_{k=1}^K$ denote the number of steps between consecutive resets and that for any $k \geq 2$, $L_k = T_k - T_{k-1}$, and $L_1 = T_1$. We need to prove the episodes are not too long, \ie $\cE_L = \scbr{\forall k \in \sbr{K}, L_k \leq \LMAX}$ holds with high probability, where $\LMAX = H \log \spr{K / \delta}$. This is done in Lemma~\ref{lem:lmax}. Then, we define the event $\cE_{V, \mathrm{alg}}$ on the iterates generated by Algorithm~\ref{alg:linear-rmax-ravi-ucb}
%
\begin{equation*}
    \cE_{V, \mathrm{alg}} = \scbr{\forall k \in \sbr{K}, \norm{M V_k - \wh{M V_k}}_{\Lambda_{T_k}} \leq \beta}\,,
\end{equation*}
%
where $\wh{M V_k} = \Lambda_{T_k}^{-1} \sum_{\spr{x, a, x'} \in \cD_{T_k}} \phi \spr{x, a} V_k \spr{x'}$. To prove $\cE_{V, \mathrm{alg}}$ holds with high probability, we need to resort to a standard uniform covering argument first introduced by \citealp{jin2019provably}. To do so, let us denote with $p^\upplus_{\Lambda, \beta, \alpha} = \sigma (\alpha \beta \norm{\phi\spr{\cdot,\cdot}}_{\Lambda} - w) = 1 - \sigma \spr{- \alpha \beta \norm{\phi \spr{\cdot, \cdot}}_{\Lambda} + \omega}$ an ascension function parametrized by the matrix $\Lambda$, the scalar $\beta$ and the sigmoid slope $\alpha$. Then, we define the following class of functions on $\cX \times \cA$
%
\begin{align*}
    \cQ &= \bigg\{Q: \X\times\mathcal{A} \rightarrow \mathbb{R} \quad \text{s.t.} \\
    &Q = \spr{1 - p^\upplus_{\Lambda, \beta, \alpha}} \odot \spr{\Phi \theta + \beta \norm{\phi \spr{\cdot, \cdot}}_{\Lambda}} + p^\upplus_{\Lambda, \beta, \alpha} \cdot \frac{\RMAX}{1-\gamma}, \\
    &\beta = \widetilde{\mathcal{O}}\brr{Q_{\max} d}, ~~~~\alpha = 2 \omega, ~~~\lambda_{\max}(\Lambda) \leq 1, ~~~\lambda_{\min}(\Lambda) \geq \frac{1}{2K B L_{\max}}, \\
    &\norm{\theta} \leq \WMAX + \QMAX \LMAX K B, \norm{Q}_\infty \leq \QMAX \bigg\} \cup \scbr{\mathbf{0}}\,,
\end{align*}
%
where $\QMAX = H \spr{\RMAX + \frac{2 \omega}{\alpha}}$ and we included the function $0$ to make sure $Q_1 \in \cQ$. Furthermore, denote for any $\eta > 0$ the function $f_\eta: \bbR^{\cX \times \cA} \rightarrow \bbR^\cX$ defined for $Q \in \bbR^{\cX \times \cA}$ as $f_\eta \spr{Q} = \frac1\eta \log \sum_{a \in \cA} \exp \spr{\eta Q \spr{\cdot, a}}$. We then define the following function class in $\bbR^\cX$
%
\begin{equation} \label{eq:function-class-v}
    \cV = \bigg\{ V: \cX \rightarrow \bbR \quad \text{s.t.} \quad\exists \scbr{Q_\ell}_{\ell=1}^K, \scbr{\bar{Q}_\ell}_{\ell=1}^K \subset \cQ, V = f_\eta \circ \spr{\sum^K_{\ell=1} Q_\ell} - f_\eta \circ \spr{\sum^K_{\ell=1} \bar{Q}_\ell} \bigg\},
\end{equation}
%
as well as the event
%
\begin{equation*}
    \cE_\cV = \scbr{\forall V \in \cV, \forall k \in \sbr{K}, \norm{M V - \wh{M V}}_{\Lambda_{T_k}} \leq \beta}\,,
\end{equation*}
where $\wh{M V} = \Lambda_{T_k}^{-1} \sum_{\spr{x, a, x'} \in \cD_{T_k}} \phi \spr{x, a}  V \spr{x'}$. Finally, we define the event that the iterates of Algorithm~\ref{alg:linear-rmax-ravi-ucb} are in the function class $\cV$
%
\begin{align*}
    \cE_{\mathrm{in}} = \scbr{\forall k \in \sbr{K}, V_k \in \cV}\,.
\end{align*}
%
What remains is to show that the iterates of the algorithm belong to $\cV$, and that the event $\cE_\cV$ holds with high probability. This is done in Lemmas~\ref{lem:iterates-in-class} and \ref{lem:uniform-event-holds}, respectively. We can now prove Lemma~\ref{lem:good-event-holds}.


%%%%%%%%%%%%%%%%%%%%%%%%%%%%%%%%%%%%%%%%%%%%%%%%%%%%%%%%%%%%%%%%%%%%%%%%%%%%%%%%
\goodeventholds*

\begin{proof}
    For any episode $k \in \sbr{K}$ and state-action pair $\spr{x, a} \in \cX \times \cA$, we have by Cauchy-Schwartz's inequality
    %
    \begin{equation*}
        \abs{P V_k \spr{x, a} - \wh{P V_k} \spr{x, a}} \leq \norm{M V_k - \wh{M V_k}}_{\Lambda_{T_k}} \norm{\phi \spr{x, a}}_{\Lambda_{T_k}^{-1}}\,.
    \end{equation*}
    %
    This inequality shows that the event $\cE_{V, \mathrm{alg}}$ implies the event $\cE_{\mathrm{valid}}$, \ie
    %
    \begin{align*}
        \bbP \sbr{\cE_{\mathrm{valid}} \cap \cE_L} &= \bbP \sbr{\cE_{\mathrm{valid}} \given \cE_L} \bbP \sbr{\cE_L} \\
        &\geq \bbP \sbr{\cE_{V, \mathrm{alg}} \given \cE_L} \bbP \sbr{\cE_L} \\
        &\geq \bbP \sbr{\cE_{V, \mathrm{alg}} \cap \cE_{\mathrm{in}} \given \cE_L} \spr{1 - \delta}\,,
    \end{align*}
    %
    where in the last inequality we used the monotonicity of $\bbP$ and Lemma~\ref{lem:lmax}. Then, conditioned on the event $\cE_{\mathrm{in}}$ that the iterates are in the function class $\cV$, the event $\cE_\cV$ implies $\cE_{V, \mathrm{alg}}$, that is
    %
    \begin{align*}
        \bbP \sbr{\cE_{V, \mathrm{alg}} \cap \cE_{\mathrm{in}} \given \cE_L} &= \bbP \sbr{\cE_{V, \mathrm{alg}} \given \cE_{\mathrm{in}} \cap \cE_L} \bbP \sbr{\cE_{\mathrm{in}} \given \cE_L} \\
        &\geq \bbP \sbr{\cE_\cV \given \cE_{\mathrm{in}} \cap \cE_L} \bbP \sbr{\cE_{\mathrm{in}} \given \cE_L} \\
        &= \bbP \sbr{\cE_\cV \cap \cE_{\mathrm{in}} \given \cE_L}\,.
    \end{align*}
    %
    Finally, by Lemma~\ref{lem:iterates-in-class} we have $\bbP \sbr{\cE_{\mathrm{in}} \given \cE_\cV, \cE_L} = 1$ thus
    %
    \begin{align*}
        \bbP \sbr{\cE_\cV \cap \cE_{\mathrm{in}} \given \cE_L} &= \bbP \sbr{\cE_{\mathrm{in}} \given \cE_\cV, \cE_L} \bbP \sbr{\cE_\cV \given \cE_L} \\
        &= \bbP \sbr{\cE_\cV \given \cE_L} \\
        &\geq 1 - \delta\,,
    \end{align*}
    %
    where the last inequality follows from Lemma~\ref{lem:uniform-event-holds}. In conclusion, we get
    %
    \begin{equation*}
        \bbP \sbr{\cE_{\mathrm{valid}} \cap \cE_L} \geq \spr{1 - \delta}^2 \geq 1 - 2 \delta\,.
    \end{equation*}
\end{proof}


%%%%%%%%%%%%%%%%%%%%%%%%%%%%%%%%%%%%%%%%%%%%%%%%%%%%%%%%%%%%%%%%%%%%%%%%%%%%%%%%
We now show the episodes are not too long.
%
\begin{lemma} \label{lem:lmax}
    Let $\delta \in \spr{0, 1}$ and define $\LMAX = H \log \spr{K / \delta}$. Then, the event $\cE_L$ holds with probability at least $1 - \delta$.
\end{lemma}

\begin{proof}
    For any $k$ and by definition of the cumulative density function of the geometric distribution with parameter $1 - \gamma$, we have that $\bbP \sbr{L_k \leq \LMAX} = 1 - \gamma^\LMAX$. Therefore, $\bbP \sbr{L_k \leq \LMAX} \geq 1 - \delta / K$ for $\LMAX \geq \frac{\log \spr{\frac{\delta}{K}}}{\log \spr{1 / \gamma}}$. Lower bounding the denominator as $\log \spr{1 / \gamma} \geq  1 - \gamma$, we have that for $\LMAX = \frac{\log \spr{K / \delta}}{1 - \gamma}$ and a union bound over $k \in \sbr{K}$, we have that  $\bbP \sbr{\cE_L} \geq 1 - \delta$.
\end{proof}


%%%%%%%%%%%%%%%%%%%%%%%%%%%%%%%%%%%%%%%%%%%%%%%%%%%%%%%%%%%%%%%%%%%%%%%%%%%%%%%%
\begin{lemma} \label{lem:iterates-in-class}
    Assume the events $\cE_\cV$ and $\cE_L$ hold. Then, for all $k \in \sbr{K}$, it holds that $V_k \in \cV$, \ie $\cE_{\mathrm{in}}$ holds.
\end{lemma}

\begin{proof}
    The bound is proven by induction over $k \in [K]$. The base case holds by initialization since $Q_0 = \mathbf{0}$ is in $\cQ$. For the induction step, we assume that for all $\ell \in \sbr{k}$, $Q_\ell \in \mathcal{Q}, V_\ell \in \mathcal{V}$ and we show that $Q_{k+1} \in \cQ$ and $V_{k+1} \in \cV$.
    
    By definition of the function classes $\cQ$ and $\cV$ it holds that $\norm{Q_k}_{\infty}, \norm{V_k}_\infty \leq Q_{\max}$. $\cE_\cV$ together with the induction assumption imply that the bonuses are valid at time $k$, meaning that the derivations from Lemma~\ref{lem:qmax} guarantee that $\norm{Q_{k+1}}_\infty \leq \QMAX$. Moreover, denote $\theta_{k+1}$ the vector used to represent $Q_{k+1}$, defined as
    %
    \begin{align*}
        \theta_{k+1} = w_k + \gamma \wh{M V_k} = w_k + \gamma \Lambda_{T_k}^{-1} \sum_{\spr{x, a, x'} \in \cD_{T_k}} \phi \spr{x, a} V_k \spr{x'}\,.
    \end{align*}
    %
    It remains to show that $\theta_{k+1}$ satisfies the norm constraint defined in $\cQ$. By the triangular inequality and plugging the various assumptions, we have
    %
    \begin{align*}
        \norm{\theta_{k+1}} &\leq \norm{w_{k}} + \gamma \norm{(\Lambda_{T_k})^{-1}\sum_{(x,a,x')\in \mathcal{D}_{T_k}} \phi(x,a) V_k (x')} \\
        &\leq \WMAX + \gamma \lambda_{\max}((\Lambda_{T_k})^{-1}) \abs{\cD_{T_k}} \norm{V_k}_\infty \max_{x, a} \norm{\phi \spr{x, a}}_2 \\
        &\leq \WMAX + K L_{\max}Q_{\max} B\,,
    \end{align*}
    %
    where we also used $\gamma < 1$ in the last inequality. This proves that $Q_{k+1} \in \cQ$. Therefore, we have that $Q_\ell \in \cQ$ for $\ell \in \sbr{k+1}$. We now show that $V_{k+1} \in \cV$. Let $x \in \cX$ and $k_e$ be the initial index of the epoch $e$ such that $k \in \mathcal{K}_e$. By \citealp[Lemma~C.1]{MN23}, the sum of $V$ iterates is equal to a log-sum-exp function of the sum of $Q$ iterates. Thus,
    %
    \begin{align*}
        V_{k+1} \spr{x} &= \sum_{i=k_e}^{k+1} V_i \spr{x} - \sum_{j=k_e}^k V_j \spr{x} \\
        &= f_\eta \spr{\sum_{i=k_e}^{k+1} Q_i} \spr{x} - f_\eta \spr{\sum_{j=k_e}^k Q_j} \spr{x}\,.
    \end{align*}
    %
    Since $\mathbf{0} \in \cQ$ and $Q_\ell \in \cQ$ for $\ell \in \sbr{k+1}$, we can pad with zeros the two sums inside the exponentials and conclude that $V_{k+1} \spr{x}$ can be written as the difference between two log-sum-exp functions of the sum of $K$ functions in $\cQ$. Thus $V_{k+1} \in \cV$ and this concludes the induction.
\end{proof}


%%%%%%%%%%%%%%%%%%%%%%%%%%%%%%%%%%%%%%%%%%%%%%%%%%%%%%%%%%%%%%%%%%%%%%%%%%%%%%%%
\begin{lemma} \label{lem:uniform-event-holds}
    Assume the event $\cE_L$ holds, and set $\beta$ as
    %
    \begin{equation*}
        \beta = 8 Q_{\max} d \log(c \alpha \WMAX \RMAX B^{9/2} Q^4_{\max}L^{5/2}_{\max}K^{7/2} d^{5/2} \delta^{-1})\,.
    \end{equation*}
    %
    where $c = 60 \cdot 26$. Then, the event $\cE_\cV$ holds with probability $1 - \delta$.
\end{lemma}

\begin{proof}
    Under the event $\cE_L$, invoking standard concentration results for Linear MDPs (see Lemmas D.3 and D.4 in \cite{jin2019provably}), we have that with probability $1-\delta$ it holds that
    %
    \begin{align*}
        &\norm{MV - (\Lambda_{T_k})^{-1}\sum_{(x,a,x')\in \mathcal{D}_{T_k}} \phi(x,a) V(x')}_{\Lambda_{T_k}} \\ &~~~~~~~\leq Q_{\max} \sqrt{2 d \log\brr{\frac{1 + K L_{\max}B}{\delta}} + 4 \log \mathcal{N}_\epsilon + 8 K^2 L^2_{\max}B^2 \epsilon^2}\,,
    \end{align*}
    %
    where $\mathcal{N}_\epsilon$ is the $\epsilon $-covering number of the class $\mathcal{V}$. In particular, for $\epsilon = (K L_{\max} B)^{-1}$, we can invoke Lemma~\ref{lem:covering-number-v} to obtain
    %
    \begin{align*}
        \log \mathcal{N}_{\epsilon} &\leq 4d^2 \log \brr{4(\WMAX B + Q_{\max}L_{\max}K B^2 + 3\sqrt{d}+ \beta B +\RMAX)\sqrt{K^5L^3_{\max}}\alpha\beta B^{5/2}} \\
        &\leq 4d^2 \log \brr{20(3 \WMAX \beta Q_{\max}L_{\max}KB^2\sqrt{d}\RMAX)\sqrt{K^5L^3_{\max}}\alpha\beta B^{5/2}} \\
        &\leq 4d^2 \log \brr{60 \WMAX \RMAX \beta^2 \alpha B^{9/2} Q^2_{\max}L^{5/2}_{\max}K^{7/2}\sqrt{d}}\,.
    \end{align*}
    %
    Plugging in, we have that
    %
    \begin{align*}
        &\norm{MV - (\Lambda_{T_k})^{-1}\sum_{(x,a,x')\in \mathcal{D}_{T_k}} \phi(x,a) V(x')}_{\Lambda_{T_k}} \\ 
        &~~~~~~~\leq Q_{\max} \sqrt{2 d \log\brr{\frac{1 + K L_{\max} B}{\delta}} + 16 d^2 \log \brr{60 \beta^2 \alpha B^{9/2}Q^2_{\max}L^{5/2}_{\max}K^{7/2}\sqrt{d}} + 8} \\
        &~~~~~~~\leq Q_{\max} \sqrt{26 d^2 \log \brr{\frac{60 \WMAX \RMAX \beta^2 \alpha B^{9/2} Q^2_{\max}L^{5/2}_{\max}K^{7/2}\sqrt{d}}{\delta}}} \\ 
        &~~~~~~~ = \sqrt{26 Q^2_{\max} d^2 \log \brr{\frac{60 \WMAX \RMAX \beta^2 \alpha  B^{9/2} Q^2_{\max}L^{5/2}_{\max}K^{7/2}\sqrt{d}}{\delta}}} \\
    \end{align*}
    %
    At this point, to find a value for $\beta$ such that
    \begin{equation*}
        \beta^2 \geq 26 Q^2_{\max} d^2 \log \brr{\frac{60 \WMAX \RMAX \beta^2 \alpha B^{9/2} Q^2_{\max}L^{5/2}_{\max}K^{7/2}\sqrt{d}}{\delta}}\,,
    \end{equation*}
    we invoke Lemma~\ref{lemma:beta_bound} with $z = 26 Q^2_{\max} d^2$ and $R = \frac{60 \WMAX \RMAX \alpha B^{9/2}Q^2_{\max}L^{5/2}_{\max}K^{7/2}\sqrt{d}}{\delta}$ which gives that the desired inequality holds for all $\beta \in \mathbb{R}$ such that
    %
    \begin{equation*}
        \beta^2 \geq 52 Q^2_{\max} d^2 \log(c \alpha \WMAX \RMAX B^{9/2} Q^4_{\max}L^{5/2}_{\max}K^{7/2} d^{5/2} \delta^{-1})\,,
    \end{equation*}
    %
    where $c = 60 \cdot 26$. Therefore, we select
    \begin{equation*}
        \beta = 8 Q_{\max} d \log(c \alpha \WMAX \RMAX B^{9/2} Q^4_{\max}L^{5/2}_{\max}K^{7/2} d^{5/2} \delta^{-1})\,.
    \end{equation*}
\end{proof}

\begin{remark}
    For the proof of Lemma~\ref{lem:uniform-event-holds}, we need to compute a bound on the covering number of the function class $\cV$. We find this is done in a neat and more direct way than previous analysis \cite{zhong2024theoretical,sherman2023rate,cassel2024warmupfree} that needed to introduce a policy class for the iterates $\bc{\pi_k}^K_{k=1}$ generated by \Cref{alg:linear-rmax-ravi-ucb} as an intermediate step.
\end{remark}


%%%%%%%%%%%%%%%%%%%%%%%%%%%%%%%%%%%%%%%%%%%%%%%%%%%%%%%%%%%%%%%%%%%%%%%%%%%%%%%%
\subsubsection{Proof of Lemma~\ref{lem:covering-number-v} (covering number)}
\label{app:covering-number-v}

\begin{lemma} \label{lem:covering-number-v}
    Let us consider the function class $\cV$ defined in Equation~\eqref{eq:function-class-v} and an $\epsilon$-covering set $\cR \spr{\cV}$ such that for any $V \in \cV$, there exists $V' \in \cR \spr{\cV}$ such that $\norm{V - V'}_\infty \leq \frac{1}{K L_{\max} B}$. The covering number of the class $\cV$ can be bounded as follows
    %
    \begin{equation*}
        \log \cN_{\frac1K} \leq 4 d^2 \log \spr{4 \spr{\WMAX B + \QMAX \LMAX K B^2 + 3 \sqrt{d} + \beta B + \RMAX H} \sqrt{K^5 \LMAX^3} \alpha \beta B^{5/2}}\,.
    \end{equation*}
\end{lemma}

\begin{proof}
    We will use the following intermediate class of log sum exp state value functions
    %
    \begin{equation*}
        \tilde{\cV} = \scbr{V: \cX \rightarrow \bbR \quad\text{s.t.}\quad \forall x, V \spr{x} = \frac1\eta \log \sum_{a \in \cA}\exp \spr{\eta \sum_{\ell=1}^K Q_\ell \spr{x, a}}, Q_\ell \in \cQ}\,.
    \end{equation*}
    %
    Consider any $V, V' \in \cV$, and notice that for any $x \in \cX$,
    %
    \begin{equation*}
        \abs{V \spr{x} - V' \spr{x}} \leq \abs{\bar{V} \spr{x} - \bar{V}' \spr{x}} + \abs{\tilde{V} \spr{x} - \tilde{V}' \spr{x}}.
    \end{equation*}
    %
    with $\bar{V}, \tilde{V} \in \tilde{\cV}$ such that $V \spr{x} = \bar{V} \spr{x} - \tilde{V} \spr{x}$ for all $x \in \cX$ and with $\bar{V}', \tilde{V}' \in \tilde{\cV}$ such that $V' \spr{x} = \bar{V}' \spr{x} - \tilde{V}' \spr{x}$ for all $x \in \cX$. Therefore, the above bound guarantees that an $\epsilon / 2$ covering set on the function class $\tilde{\cV}$ implies an $\epsilon$ covering for the class $\cV$. Hence, in the following we focus on computing a $\epsilon / 2$ covering number for $\tilde{\cV}$. By definition of $\bar{V}, \bar{V}'$ and Lemma~\ref{lem:lse-lipschitz}, we have
    %
    \begin{align*}
        \abs{\bar{V} \spr{x} - \bar{V}' \spr{x}} &= \abs{\frac1\eta \log \sum_{a \in \cA} \exp \brr{\eta \sum^K_{\ell=1} \bar{Q}_\ell \spr{x, a}} - \frac1\eta \log \sum_{a \in \cA} \exp \spr{\eta \sum^K_{\ell=1} \bar{Q}_\ell' \spr{x, a}}} \\
        &\leq \max_{a \in \A} \abs{ \sum^K_{\ell=1} \bar{Q}_\ell \spr{x, a} -  \sum^K_{\ell=1} \bar{Q}'_\ell \spr{x, a}} \\
        &\leq K \max_{\ell \in [K]} \norm{ \bar{Q}_\ell - \bar{Q}'_\ell }_{\infty}\,.
    \end{align*}
    %
    For any $\ell \in \sbr{K}$, we denote $\Lambda_\ell$, $\theta_\ell$ the parameters of the function $\bar{Q}_\ell$ and $\Lambda'_\ell$, $\theta'_\ell$ the parameters of the function $\bar{Q}'_\ell$. We now prove that $\bar{Q}_\ell$,$\bar{Q}'_\ell$ are Lipschitz functions. Let us denote $Q_{\theta, \Lambda}$ and $Q_{\theta', \Lambda'}$ two functions in $\cQ$ for different parameters $\theta, \Lambda, \theta', \Lambda'$. For any state-action pair $\spr{x, a}$, the difference between the two functions can be written as
    %
    \begin{align*}
        &Q_{\theta, \Lambda} \spr{x, a} - Q_{\theta', \Lambda'} \spr{x, a} = \spr{\phi \spr{x, a}\transpose \theta + \beta \norm{\phi \spr{x, a}}_\Lambda - \RMAX H} \cdot \sigma \spr{- \alpha \beta \norm{\phi \spr{x, a}}_\Lambda + \omega} \\
        &\phantom{=}\quad\quad\quad\quad\quad- \spr{\phi \spr{x, a}\transpose \theta' + \beta \norm{\phi \spr{x, a}}_{\Lambda'} - \RMAX H} \cdot \sigma \spr{- \alpha \beta \norm{\phi \spr{x, a}}_{\Lambda'} + \omega}\,.
    \end{align*}
    %
    Next, our goal is to show that the function
    %
    \begin{equation*}
        f \spr{\theta, \Lambda; x, a} := \spr{\phi(x,a)\transpose \theta + \beta \norm{\phi (x, a)}_{\Lambda} - \RMAX H} \cdot \sigma(- \alpha\beta \norm{\phi(x,a)}_{\Lambda} + \omega)
    \end{equation*}
    %
    is Lipshitz in both parameters $\theta, \Lambda$. The Lipshitzness with respect to $\beta$ does not need to be established since it is kept fixed throughout the learning process. We show this showing that the gradients are bounded. In particular,
    %
    \begin{equation*}
        \norm{\nabla_\theta f(\theta,\Lambda;x,a)} = \norm{\phi(x,a)} \cdot \sigma(- \alpha\beta \norm{\phi(x,a)}_{\Lambda} + \omega) \leq \norm{\phi(x,a)}  \leq B\,.
    \end{equation*}
    %
    For the Lipshitzness with respect to $\Lambda$, we have that
    %
    \begin{align*}
        &f(\theta,\Lambda;x,a) - f(\theta,\Lambda';x,a) \\
        &= \spr{\phi(x,a)\transpose \theta + \beta \norm{\phi(x,a)}_{\Lambda} 
        - \RMAX H}\cdot \sigma(- \alpha\beta \norm{\phi(x,a)}_{\Lambda} + \omega) \\
        &\phantom{=}- \spr{\phi(x,a)\transpose \theta + \beta \norm{\phi(x,a)}_{\Lambda'} 
        - \RMAX H}\cdot \sigma(- \alpha\beta \norm{\phi(x,a)}_{\Lambda'} + \omega) \\
        &= \spr{\phi(x,a)\transpose \theta + \beta \norm{\phi(x,a)}_{\Lambda} 
        - \RMAX H}\cdot \Bigl(\sigma(- \alpha\beta \norm{\phi(x,a)}_{\Lambda} + \omega) \\
        &\phantom{=}- \sigma(- \alpha\beta \norm{\phi(x,a)}_{\Lambda'} + \omega) \Bigr) + \sigma(- \alpha\beta \norm{\phi(x,a)}_{\Lambda'} + \omega) \brr{\beta \norm{\phi(x,a)}_{\Lambda} - \beta \norm{\phi(x,a)}_{\Lambda'}}
    \end{align*}
    %
    Then, using the fact that $\sigma$ is $1$-Lipshitz, we have that
    %
    \begin{align*}
        &\abs{f(\theta,\Lambda;x,a) - f(\theta,\Lambda';x,a)} \\
        &\quad \leq \alpha \beta \abs{\phi(x,a)\transpose \theta + \beta \norm{\phi(x,a)}_{\Lambda} - \RMAX H} \cdot \abs{\norm{\phi(x,a)}_{\Lambda} - \norm{\phi(x,a)}_{\Lambda'}} \\
        &\quad\quad + \sigma(- \alpha\beta \norm{\phi(x,a)}_{\Lambda'} + \omega) \abs{\beta \norm{\phi(x,a)}_{\Lambda} - \beta \norm{\phi(x,a)}_{\Lambda'}} \\
        &\leq \alpha\beta \abs{\phi(x,a)\transpose \theta + \beta \norm{\phi(x,a)}_{\Lambda} 
        - \RMAX H} \cdot \abs{\norm{\phi(x,a)}_{\Lambda} 
        - \norm{\phi(x,a)}_{\Lambda'}}
        \\&\phantom{=}+  \beta \abs{\norm{\phi(x,a)}_{\Lambda} - \norm{\phi(x,a)}_{\Lambda'}} \\
        &\leq \alpha\beta (\norm{\theta}B + \beta B + \RMAX H + 1) 
        \abs{\norm{\phi(x,a)}_{\Lambda} - \norm{\phi(x,a)}_{\Lambda'}}.
    \end{align*}
    %
    where we used the fact that $\sigma(x) \leq 1$, for all $x\in \mathbb{R}$ and $\alpha \geq 1$ in the last inequality. 
    Using that $\norm{\phi (x,a)}_{\Lambda} = \norm{\Lambda^{1/2} \phi(x,a)}$ and the triangular inequality we have that
    %
    \begin{align*}
        f(\theta,\Lambda;x,a) - f(\theta,\Lambda';x,a) &\leq 
        \alpha\beta (\norm{\theta}B + \beta B +\RMAX H + 1) 
        \abs{\norm{\Lambda^{1/2}\phi(x,a)} - \norm{(\Lambda')^{1/2}\phi(x,a)}} \\
        &\leq \alpha\beta (\norm{\theta}B + \beta B +\RMAX H + 1) \norm{(\Lambda^{1/2} - (\Lambda')^{1/2})\phi(x,a)} \\
        &\leq \alpha \beta B (\norm{\theta} B + \beta B +\RMAX H + 1) \norm{\Lambda^{1/2} - (\Lambda')^{1/2}} 
    \end{align*}
    %
    where the last inequality holds for $\norm{\phi (x,a)} \leq B$. Finally, using the definition of the class $\mathcal{Q}$, we have that the matrices $\Lambda$ and $\Lambda'$ are positive definite in particular $\lambda_{\min}(\Lambda) \geq \frac{1}{2KBL_{\max}}$ and $\lambda_{\min}(\Lambda') \geq \frac{1}{2KBL_{\max}}$. 
    Therefore by \cite[Lemma 17]{cassel2024warmupfree}, 
    it holds that $\norm{\Lambda^{1/2} - (\Lambda')^{1/2}} \leq \frac{1}{2\sqrt{\lambda_{\min}}}\norm{\Lambda - \Lambda'} = \sqrt{\frac{B K L_{\max}}{2}} \norm{\Lambda - \Lambda'}$. 
    Therefore, all in all we have that
    %
    \begin{align*}
        f(\theta,\Lambda;x,a) - f(\theta,\Lambda';x,a)
        &\leq \sqrt{K L_{\max}}\alpha\beta B^{3/2} (\norm{\theta} B
        + \beta B +\RMAX H + 1) \norm{\Lambda - \Lambda'} \\
        &\leq \sqrt{K L_{\max}}\alpha\beta B^{3/2} (\norm{\theta} B
        + \beta B +\RMAX H + 1) \norm{\Lambda - \Lambda'}_F,
    \end{align*}
    %
    where $\norm{\cdot}_F$ denote the Frobenious norm of a matrix. Hence, we have that
    %
    \begin{align*}
        Q_{\theta,\Lambda}(x,a) - Q_{\theta',\Lambda'}(x,a) \leq 
        \sqrt{K L_{\max}}\alpha\beta B^{3/2} (\norm{\theta} B + \beta B +\RMAX H + 1)
         \norm{\Lambda - \Lambda'}_F + B \norm{\theta - \theta'}.
    \end{align*}
    %
    At this point, if we have a $\epsilon_{\Lambda}$-covering set for the set
    %
    \begin{equation*}
        \mathbf{\Lambda} = \bc{\Lambda \in \mathbb{R}^{d\times d} : \lambda_{\max}(\Lambda) \leq 1, ~~~\lambda_{\min}(\Lambda) \geq \frac{1}{2BKL_{\max}}}
    \end{equation*}
    %
    and an $\epsilon_\theta$-covering set for the set
    %
    \begin{equation*}
        \mathbf{\Theta} = \bc{\theta \in \mathbb{R}^d: \norm{\theta} \leq \WMAX + Q_{\max}L_{\max}K B}
    \end{equation*}
    %
    we would have that
    %
    \begin{align*}
        \abs{V(x) - V'(x)} &\leq 2\sqrt{ K^3 L_{\max}}\alpha\beta B^{3/2} (\norm{\theta} B + \beta B +\RMAX H + 1) \epsilon_F + 2 B K \epsilon_\theta \\
        &\leq 2 \sqrt{K^3L_{\max}}\alpha \beta B^{3/2} (\norm{\theta} B + \beta B +\RMAX H + 1) \brr{\epsilon_F + \epsilon_\theta},
    \end{align*}
    %
    where in the last inequality we assumed that $\beta \geq 1$ and $B\geq 1$.
    Therefore, to have an $\epsilon$-covering set for $\mathcal{V}$, we need to construct an $\epsilon_{\Lambda}$-covering set for $\mathbf{\Lambda}$, where
    %
    \begin{equation*}
        \epsilon_\Lambda = \frac{\epsilon}{4 \sqrt{K^3L_{\max}}\alpha\beta B^{3/2} (\norm{\theta} B + \beta B +\RMAX H + 1)}
    \end{equation*}
    %
    and an $\epsilon_\theta = \frac{\epsilon}{4 \sqrt{K^3L_{\max}}\alpha\beta B^{3/2} (\norm{\theta} B + \beta B +\RMAX H + 1)}$-covering set for $\mathbf{\Theta}$. Then, using the fact that the $\epsilon$-covering number for the Euclidean ball of radius $R$ in $d$ dimension is given by $(1 + 2 R/\epsilon)^d$, we obtain
    %
    \begin{equation*}
        \log \mathcal{N}_{\epsilon_{\theta}}(\mathbf{\Theta}) \leq d \log \brr{1 + 8\frac{(\WMAX + Q_{\max}L_{\max}K B )\sqrt{K^3L_{\max}}\alpha\beta B^{3/2} (\norm{\theta} B + \beta B +\RMAX H + 1)}{\epsilon}}
    \end{equation*}
    %
    Moreover, noticing that for all matrices $\Lambda \in \mathbf{\Lambda}$ it holds that $\norm{\Lambda}_F \leq \sqrt{d} \lambda_{\max}(\Lambda) \leq \sqrt{d}$, we need to cover the Frobenious norm ball with radius $\sqrt{d}$. Recalling that the Frobenious norm of a matrix is equivalent to the euclidean norm of the vectorization of the matrix, this equivalent to cover the euclidean ball in $\mathbb{R}^{d^2}$ with radius $\sqrt{d}$.
    %
    \begin{equation*}
        \log \mathcal{N}_{\epsilon_{\Lambda}}(\mathbf{\Lambda}) 
        \leq d^2 \log \brr{1 + 8\sqrt{d}\frac{\sqrt{K^3L_{\max}}\alpha\beta B^{3/2} (\norm{\theta} B + \beta B +\RMAX H + 1)}{\epsilon}}.
    \end{equation*}
    %
    Therefore, using the fact that
    %
    \begin{align*}
        &\log \mathcal{N}_{\epsilon}(\mathcal{V})=  
        \log \mathcal{N}_{\epsilon_{\Lambda}}(\mathbf{\Lambda}) + 
        \log \mathcal{N}_{\epsilon_{\theta}}(\mathbf{\Theta}) \\
        \\ & \leq d \log \brr{1 + 8\frac{(\WMAX + Q_{\max}L_{\max}K B )\sqrt{K^3L_{\max}}\alpha\beta B^{3/2} 
        (\norm{\theta} B + \beta B +\RMAX H + 1)}{\epsilon}} \\&\phantom{=}+ 
        d^2 \log \brr{1 + 8\sqrt{d}\frac{\sqrt{K^3L_{\max}}\alpha\beta B^{3/2} (\norm{\theta} B + \beta B +\RMAX H + 1)}{\epsilon}} \\
        \\ & \leq 2d^2\log  \brr{1 + 8\frac{(\WMAX + Q_{\max}L_{\max}K B + \sqrt{d})\sqrt{K^3L_{\max}}\alpha\beta B^{3/2} (\norm{\theta} B + 
        \beta B +\RMAX H + 1)}{\epsilon}} \\
        & \leq 2d^2 \log \brr{1 + 8\frac{(\WMAX B + Q_{\max}L_{\max}K B^2 + \sqrt{d}+ \beta B +\RMAX H +1)^2\sqrt{K^3L_{\max}}\alpha\beta B^{3/2}}{\epsilon}} \\
        & \leq 2d^2 \log \brr{16\frac{(\WMAX B + Q_{\max}L_{\max}K B^2 + 3\sqrt{d}+ \beta B +\RMAX H)^2\sqrt{K^3L_{\max}}\alpha\beta B^{3/2}}{\epsilon}} \\
        & \leq 4d^2 \log \brr{4\frac{(\WMAX B + Q_{\max}L_{\max}K B^2  + 3\sqrt{d}+ \beta B +\RMAX H)\sqrt{K^3L_{\max}}\alpha\beta B^{3/2}}{\epsilon}} \\
        & = 4d^2 \log \brr{4(\WMAX B + Q_{\max}L_{\max}K B^2 + 3\sqrt{d}+ \beta B +\RMAX H)\sqrt{K^5L^3_{\max}}\alpha\beta B^{5/2}}.
    \end{align*}
    %
    where we used $d > 1$ and the last step uses the fact that we are looking for a $\epsilon = \frac{1}{K L_{\max} B}$ covering set. Finally,
    %
    \begin{equation*}
        \log \mathcal{N}_{\epsilon} \leq 4d^2 \log \brr{4(\WMAX B +  Q_{\max}L_{\max}K B^2 + 3\sqrt{d}+ \beta B +\RMAX H)\sqrt{K^5L^3_{\max}}\alpha\beta B^{5/2}}\,.
    \end{equation*}
\end{proof}


%%%%%%%%%%%%%%%%%%%%%%%%%%%%%%%%%%%%%%%%%%%%%%%%%%%%%%%%%%%%%%%%%%%%%%%%%%%%%%%%
%%%%%%%%%%%%%%%%%%%%%%%%%%%%%%%%%%%%%%%%%%%%%%%%%%%%%%%%%%%%%%%%%%%%%%%%%%%%%%%%
\subsection{Putting everything together (proof of Theorem~\ref{thm:main})}
\label{app:putting-together-main}

\begin{theorem} \label{thm:main-full}
    Run Algorithm~\ref{alg:linear-rmax-ravi-ucb} with parameters $\omega = \log K$, $\alpha = 2 \log K$,
    \begin{align*}
        \eta = \sqrt{\frac{5 d \log \spr{1 + B^2 T / d} \log \abs{\cA}}{8 \RMAX^2 H^{5 / 2} K}}, \quad\text{and } \beta = C H \RMAX d \log \spr{B H \WMAX \RMAX d K \delta^{-1}}\,,
    \end{align*}
    %
    for some absolute constant $C > 0$ and $\delta \in \spr{0, 1}$. Then, with probability at least $1 - \delta$, we have
    %
    \begin{align*}
        \regretK &= \tilde\cO \spr{\sqrt{d^3 H^3 K} + \sqrt{d H^{9/2} K \log \spr{\abs{\cA}}}} \\
        &= \tilde\cO \spr{\sqrt{d^3 H^2 T} + \sqrt{d H^{7/2} T \log \spr{\abs{\cA}}}}\,.
    \end{align*}
\end{theorem}

\begin{proof}
    We are now ready to prove Theorem~\ref{thm:main}. Combining Lemma~\ref{lem:reward-bias-bound}, and the bounds in Equations~\eqref{eq:model-bias-bounds} and Lemma~\ref{lem:bound-regret-plus} we first get
    %
    \begin{align*}
        \frac1H \regretK &\leq 2 \RMAX H \sumkK \inp{\mu \spr{\pi_k}, p_k^\upplus} + 4 \QMAX E \spr{K} + \frac{E \spr{K} \log \abs{\cA}}{\eta} \\
        &\phantom{=}+ 2 \eta \QMAX^2 \sqrt{H} K + 2 \sumkK \inp{\mu \spr{\pi_k}, \spr{1 - p_k^\upplus} \odot \CB_k}\,.
    \end{align*}
    %
    Using the bound on the ascension functions provided in Inequality~\ref{eq:ascension-function-bound} and $1 - p_k^\upplus \preceq 1$, we further have
    %
    \begin{align*}
        \frac1H \regretK &\leq 4 \RMAX H \alpha^2 \sumkK \inp{\mu \spr{\pi_k}, \CB_k^2} + 4 \RMAX e^{- \omega} H K + 4 \QMAX E \spr{K} \\
        &\phantom{=}+ \frac{E \spr{K} \log \abs{\cA}}{\eta} + 2 \eta \QMAX^2 \sqrt{H} K + 2 \sumkK \inp{\mu \spr{\pi_k}, \CB_k}\,.
    \end{align*}
    %
    Lemma~\ref{lem:expected-bonuses-bound} can be used to bound the bonuses
    %
    \begin{align*}
        \frac1H \regretK &\leq 32 \RMAX \alpha^2 \beta^2 B^2 d \log \spr{1 + \frac{B^2 T}{d}} + 16 \RMAX H \alpha^2 \beta^2 B^2 \log \spr{\frac{2 K}{\delta}}^2 \\
        &\phantom{=}+ 4 \RMAX e^{- \omega} H K + 4 \QMAX E \spr{K} + \frac{E \spr{K} \log \abs{\cA}}{\eta} + 2 \eta \QMAX^2 \sqrt{H} K \\
        &\phantom{=}+ \frac{8 \beta B}{H} \sqrt{d T \log \spr{1 + \frac{B^2 T}{d}}} + 8 \beta B \log \spr{\frac{2 K}{\delta}}^2\,.
    \end{align*}
    %
    Following Lemmas~\ref{lem:qmax} and \ref{lem:lmax}, we plug the values of $\QMAX = H \spr{\RMAX + \frac{2 \omega}{\alpha}}$ and $\LMAX = H \log \spr{K / \delta}$,
    %
    \begin{align*}
        \frac1H \regretK &\leq 32 \RMAX \alpha^2 \beta^2 B^2 d \log \spr{1 + \frac{B^2 T}{d}} + 16 \RMAX H \alpha^2 \beta^2 B^2 \log \spr{\frac{2 K}{\delta}}^2 \\
        &\phantom{=}+ 4 \RMAX e^{- \omega} H K + 4 H \spr{\RMAX + \frac{2 \omega}{\alpha}} E \spr{K} + \frac{E \spr{K} \log \abs{\cA}}{\eta} \\
        &\phantom{=}+ 2 \eta \spr{\RMAX + \frac{2 \omega}{\alpha}}^2 H^{5/2} K + \frac{8 \beta B}{H} \sqrt{d T \log \spr{1 + \frac{B^2 T}{d}}} \\
        &\phantom{=}+ 8 \beta B \log \spr{\frac{2 K}{\delta}}^2\,.
    \end{align*}
    %
    By Lemma~\ref{lemma:number-epochs-bound}, we can bound $E \spr{K} \leq 5 d \log \spr{1 + \frac{B^2 T}{d}}$
    %
    \begin{align*}
        \frac1H \regretK &\leq 32 \RMAX \alpha^2 \beta^2 B^2 d \log \spr{1 + \frac{B^2 T}{d}} + 16 \RMAX \alpha^2 \beta^2 B^2 H \log \spr{\frac{2 K}{\delta}}^2 \\
        &\phantom{=}+ 4 \RMAX e^{- \omega} H K + 20 H d \spr{\RMAX + \frac{2 \omega}{\alpha}} \log \spr{1 + \frac{B^2 T}{d}} \\
        &\phantom{=}+ \frac{5 d}{\eta} \log \spr{1 + \frac{B^2 T}{d}} \log \abs{\cA} + 2 \eta \spr{\RMAX + \frac{2 \omega}{\alpha}}^2 H^{5/2} K \\
        &\phantom{=}+ \frac{8 \beta B}{H} \sqrt{d T \log \spr{1 + \frac{B^2 T}{d}}} + 8 \beta B \log \spr{\frac{2 K}{\delta}}^2\,.
    \end{align*}
    %
    It remains to choose the parameters. We start by setting $\alpha = 2 \omega$ and use $\RMAX \geq 1$ to get
    %
    \begin{align*}
        \frac1H \regretK &\leq 128 \RMAX \omega^2 \beta^2 B^2 d \log \spr{1 + \frac{B^2 T}{d}} + 64 \RMAX \omega^2 \beta^2 B^2 H \log \spr{\frac{2 K}{\delta}}^2 \\
        &\phantom{=}+ 4 \RMAX e^{- \omega} H K + 40 H d \RMAX \log \spr{1 + \frac{B^2 T}{d}} \\
        &\phantom{=}+ \frac{5 d}{\eta} \log \spr{1 + \frac{B^2 T}{d}} \log \abs{\cA} + 8 \eta \RMAX^2 H^{5/2} K \\
        &\phantom{=}+ \frac{8 \beta B}{H} \sqrt{d T \log \spr{1 + \frac{B^2 T}{d}}} + 8 \beta B \log \spr{\frac{2 K}{\delta}}^2\,.
    \end{align*}
    %
    Then, we set $\omega = \log K$
    %
    \begin{align*}
        \frac1H \regretK &\leq 128 \RMAX \beta^2 B^2 d \log \spr{K}^2 \log \spr{1 + \frac{B^2 T}{d}} + 64 \RMAX \beta^2 B^2 H \log \spr{K}^2 \log \spr{\frac{2 K}{\delta}}^2 \\
        &\phantom{=}+ 4 \RMAX H + 40 H d \RMAX \log \spr{1 + \frac{B^2 T}{d}} \\
        &\phantom{=}+ \frac{5 d}{\eta} \log \spr{1 + \frac{B^2 T}{d}} \log \abs{\cA} + 8 \eta \RMAX^2 H^{5/2} K \\
        &\phantom{=}+ \frac{8 \beta B}{H} \sqrt{d T \log \spr{1 + \frac{B^2 T}{d}}} + 8 \beta B \log \spr{\frac{2 K}{\delta}}^2\,.
    \end{align*}
    %
    We choose the learning rate as $\eta = \sqrt{\frac{5 d \log \spr{1 + B^2 T / d} \log \abs{\cA}}{8 \RMAX^2 H^{5 / 2} K}}$ and we obtain
    %
    \begin{align*}
        \frac1H \regretK &\leq 128 \RMAX \beta^2 B^2 d \log \spr{K}^2 \log \spr{1 + \frac{B^2 T}{d}} + 64 \RMAX \beta^2 B^2 H \log \spr{K}^2 \log \spr{\frac{2 K}{\delta}}^2 \\
        &\phantom{=}+ 4 \RMAX H + 40 H d \RMAX \log \spr{1 + \frac{B^2 T}{d}} \\
        &\phantom{=}+ 4 \sqrt{10 \RMAX^2 H^{5/2} d \log \spr{1 + \frac{B^2 T}{d}} \log \spr{\abs{\cA}} K} \\
        &\phantom{=}+ \frac{8 \beta B}{H} \sqrt{d T \log \spr{1 + \frac{B^2 T}{d}}} + 8 \beta B \log \spr{\frac{2 K}{\delta}}^2\,.
    \end{align*}
    %
    Finally, following Lemma~\ref{lem:good-event-holds} we set $\beta = C H \RMAX d \log \spr{B H \WMAX \RMAX d K \delta^{-1}}$ where $C > 0$ is an absolute constant and we get
    %
    \begin{align*}
        \frac1H \regretK &\leq 128 C^2 \RMAX^3 B^2 d^3 H^2 \log \spr{K}^2 \log \spr{1 + \frac{B^2 T}{d}} \log \spr{B H \WMAX \RMAX d K \delta^{-1}}^2 \\
        &\phantom{=}+ 64 C^2 \RMAX^3 d^2 B^2 H^3 \log \spr{K}^2 \log \spr{\frac{2 K}{\delta}}^2 \log \spr{B H \WMAX \RMAX d K \delta^{-1}}^2 \\
        &\phantom{=}+ 4 \RMAX H + 40 H d \RMAX \log \spr{1 + \frac{B^2 T}{d}} \\
        &\phantom{=}+ 4 \sqrt{10 \RMAX^2 H^{5/2} d \log \spr{1 + \frac{B^2 T}{d}} \log \spr{\abs{\cA}} K} \\
        &\phantom{=}+ 8 C \RMAX B d \sqrt{d T \log \spr{1 + \frac{B^2 T}{d}}} \log \spr{B H \WMAX \RMAX d K \delta^{-1}} \\
        &\phantom{=}+ 8 C \RMAX d B H^2 \log \spr{\frac{2 K}{\delta}}^2 \log \spr{B H \WMAX \RMAX d K \delta^{-1}}\,.
    \end{align*}
    %
    After multiplying by $H$, we get
    %
    \begin{align*}
        \regretK &= \tilde\cO \spr{\sqrt{d^3 H^3 K} + \sqrt{d H^{9/2} K \log \spr{\abs{\cA}}}} \\
        &= \tilde\cO \spr{\sqrt{d^3 H^2 T} + \sqrt{d H^{7/2} T \log \spr{\abs{\cA}}}}\,.
    \end{align*}
\end{proof}


\clearpage
\section{Motivation for \emph{Learning from Features Alone} and related works in imitation learning}
\label{app:related_works_IL}



\paragraph{Related works in theoretical imitation learning.} A special case of our setting is imitation learning from state-only expert trajectories, which is recovered when $\phi_\cost (x,a) = \mathbf{e}_x$. 
This setting was first studied in \cite{sun2019provably} in the finite-horizon setting with general function approximation. 
There are some notable differences between their work and ours, primarily that they focus on the finite-horizon setting and learn a non-stationary policy. 
In principle, their algorithm could be applied to the infinite-horizon setting by truncating the trajectories after $\tilde{\mathcal{O}}(1-\gamma)^{-1}$ steps. 
However, this would still result in a non-stationary policy, whereas our approach outputs a stationary policy. 
Their realizability assumption on the expert policy and expert state-value function is not required in our work which leverages, instead, the linear MDP assumption. 
These assumptions are not directly comparable, even when the function classes in \cite{sun2019provably} are assumed to be linear. 
Indeed, the realizability assumption imposed in \cite{sun2019provably} would imply having access to the values of the features $\sum_a \expert(a|x) \phi(x,a)$ for each state $x \in \cX$. 
In contrast, our approach does not require this additional knowledge about the expert.

Furthermore, the guarantees on the number of expert trajectories in \cite[Theorem 3.3]{sun2019provably} adapted to the infinite-horizon setting, 
would scale as $\tilde{\mathcal{O}}((1-\gamma)^{-4} \epsilon^{-2})$ whereas we only require $\tilde{\mathcal{O}}((1-\gamma)^{-2} \epsilon^{-2})$ state-only samples from the expert occupancy measure.

Similarly, \cite{ADKLS20} develop a framework for imitation and representation learning from observation alone based on bilevel optimization but assume the realizability of the state-value function, which is not needed in our work.

The work of \citet{kidambi2021mobile} investigates the idea of exploration in state-only imitation learning. 
Unlike our work, they focus on the finite-horizon setting and on different structural assumptions on the MDP. 
Specifically, \citet{kidambi2021mobile} consider tabular MDPs, nonlinear kernel regulators, and MDPs with Gaussian transition kernels and bounded Eluder dimension, whereas our work focuses on infinite-horizon linear MDPs and observing only the feature directions visited by the expert, which is a weaker requirement than observing the states directly.
Moreover, our algorithm \FRAalg is computationally efficient, whereas the model fitting step in \cite{kidambi2021mobile} cannot be implemented efficiently for various situations, including linear MDPs \citep{jin2019provably} and KNRs \citep{Kakade:2020}.

\citet{wu2024diffusing} operate under a different set of assumptions, namely that the learner has access to a function class for the expert's score function and that the expected state norm remains bounded during learning. 
Under this setting, the authors are the first to achieve first- and second-order bounds for imitation learning, which lead to a faster rate in the case of low-variance expert policies and transitions. The authors do not quantify the MDP trajectory complexity, but it would scale suboptimally with $1/\epsilon$ because they require an expensive \emph{RL in the loop} routine that we avoid in our work.

\citet{xu2022understanding} develop an analysis for horizon-free bounds on $\tau_E$ for a special class of MDPs, where expert states can be visited only by visiting all preceding expert states.

The trajectory access to the MDP $\mathcal{M}\setminus \true$ assumed in this work should not be confused with interactive/online imitation learning, where the expert can be queried during learning \citep{Ross:2010,Ross:2011,swamy2021moments,li2022efficient,lavington2022improved,sekhari2024selective,sun2017deeply,sekhari2024contextual}. Furthermore, our trajectory access is a much weaker requirement compared to generative model access used in \citep{swamy2022minimax,Kamoutsi:2021}.

Moreover, it is important to note that we do not require any ergodicity or self-exploration properties of the dynamics, whereas such assumptions are needed in \citep{viano2022proximal,zeng2022maximum}. Additionally, uniformly good evaluation error, which is essentially possible only under generative model or ergodic dynamics assumptions, is required in \citep{wu2023inverse,zeng2022structural,zeng2023understanding}. 
Also, the use of exploration bonuses in imitation learning has also been useful for the related problem of finding the reward feasible set without using a generative model \citep{lazzati2024scale,lindner2022active}.

\begin{table}[t]
    \caption{\label{tab:literature} Comparison with related imitation learning algorithms.}
    \setlength{\tabcolsep}{4pt}  % Reduce horizontal padding
    \renewcommand{\arraystretch}{1.5}  % Increase vertical spacing
    \resizebox{\textwidth}{!}{%
    \begin{tabular}{|c|M{4cm}|M{1.5cm}|c|c|}
        \hline
        \textbf{Algorithm}                                         & \textbf{Setting}                                                & \textbf{F.O.}            & \textbf{Expert Traj. $(\tau_E)$}                                       & \textbf{MDP Traj. $(K)$}                                                              \\ \hline
        \multirow{3}{*}{Behavioural Cloning}                       & Function Approximation, Episodic \cite{foster2024behavior}      & \redcross                & $\cO \brr{\frac{H^2 \log \abs{\Pi}}{\epsilon^2}}$                      & \multicolumn{1}{c|}{-}                                                                \\ \cline{2-4}
                                                                   & Tabular, Episodic \cite{rajaraman2020toward}                    & \redcross                & $\widetilde{\cO} \brr{\frac{H^2 \abs{\cX}}{\epsilon}}$                 & \multicolumn{1}{c|}{-}                                                                \\ \cline{2-4}
                                                                   & Deterministic Linear Expert, Episodic \cite{rajaraman2021value} & \redcross                & $\widetilde{\cO} \brr{\frac{H^2 d}{\epsilon}}$                         & \multicolumn{1}{c|}{-}                                                                \\ \hline
        Mimic-MD \cite{rajaraman2020toward}                        & Tabular, Known $P$, Deterministic Expert, Episodic              & \redcross                & $\cO \brr{\frac{H^{3/2} \abs{\cX}}{\epsilon}}$                         & \multicolumn{1}{c|}{-}                                                                \\ \hline
        OAL \cite{Shani:2021}                                      & Episodic Tabular                                                & \redcross                & $\tilde{\cO} \brr{\frac{H^2 \abs{\cX}}{\epsilon^{2}}}$                 & $\tilde{\mathcal{O}}\brr{\frac{H^4 \abs{\X}^2 \abs{\aspace} }{\epsilon^{2}}}$         \\ \hline
        MB-TAIL \cite{xu2023provably}                              & Episodic, Tabular, Deterministic Expert                         & \redcross                & $\cO \brr{\frac{H^{3/2} \abs{\cX}}{\epsilon}}$                         & $\mathcal{O}\brr{\frac{H^3 \abs{\X}^2 \abs{\aspace} }{\epsilon^{2}}}$                 \\ \hline
        FAIL \cite{sun2019provably}                                & Episodic, $\expert \in \Pi$ and $V^\expert\in\mathcal{F}$       & \greentick$^\star$       & $\tilde{\cO} \brr{\frac{H^4 \log(\abs{\Pi} \abs{\cF} H)}{\epsilon^2}}$ & $\tilde{\mathcal{O}}\brr{\frac{H^4 \log(\abs{\Pi}\abs{\mathcal{F}} H) }{\epsilon^2}}$ \\ \hline
        Mobile \cite{kidambi2021mobile}                            & Episodic, $\true \in \mathcal{R}$ and $P \in \mathcal{P}$       & \greentick$^\star$       & $\tilde{\cO} \brr{\frac{H^2 \log (\abs{\mathcal{R}} H)}{\epsilon^2}}$  & $\tilde{\mathcal{O}}\brr{\frac{H^5 \log \abs{\mathcal{P}}}{\epsilon^2}}$              \\ \hline
        OGAIL \cite{Liu:2022}                                      & Episodic Linear Mixture MDP, $\WMAX=\sqrt{d}$                   & \greentick               & $\tilde{\cO} \brr{\frac{H^{3} d^2}{\epsilon^{2}}}$                     & $\tilde{\mathcal{O}}\brr{\frac{H^4 d^3}{\epsilon^{2}}}$                               \\ \hline
        ILARL \cite{viano2024imitation}                            & Linear MDP, $\WMAX=1$                                           & \greentick               & $\tilde{\cO} \brr{\frac{d}{(1 - \gamma)^{2} \epsilon^{2}}}$            & $\tilde{\mathcal{O}}\brr{ \frac{d^3 }{(1 - \gamma)^{8} \epsilon^{4}}}$                \\ \hline
        \FRAalg (This Work)                                        & Linear MDP                                                      & \greentick               & $\tilde{\cO} \brr{\frac{\WMAX^2}{(1 - \gamma)^{2} \epsilon^{2}}}$      & $\tilde{\mathcal{O}}\brr{ \frac{d^3 }{(1 - \gamma)^{4.5} \epsilon^{2}}}$              \\ \hline
        \textbf{Lower Bound} (This Work)                           & Linear MDP                                                      & \greentick               & $\Omega\brr{ \frac{\WMAX^2}{(1 - \gamma)^{2} \epsilon^{2}}}$           & $\Omega\brr{ \frac{d }{(1 - \gamma)^{2} \epsilon^{2}}}$                               \\ \hline
    \end{tabular}}
\end{table}

Finally, we present Table~\ref{tab:literature}, which compares our bounds with existing ones.
 We show the number of expert trajectories and MDP interactions required for $\epsilon$-suboptimal expected performance. 
 The acronym F.O.\ refers to "Features Only" and indicates whether the algorithm applies to the setting we consider here. 
 The star \greentick$^\star$ specifies that the algorithm only applies to state-only imitation learning. "Linear expert" refers to the case where an expert policy is of the form
%
\begin{align*}
    \pi \spr{x} = \max_{a \in \aspace} \phi \spr{x, a}\transpose \theta\,,
\end{align*}
%
for some vector $\theta$. Finally, in the work by \cite{kidambi2021mobile}, the bound on $K$ can be tighter than what we report in the table. We report this slightly looser version for sake of simplicity and avoiding to introduce the information gain (see \cite{kidambi2021mobile} for details).


\clearpage
\section{Omitted proofs for Section~\ref{sec:application}}
\label{app:proof_IL}

To improve readability, we define the feature expectation vector as $\lambda(\pi) = \phim_r\transpose \mu(\pi)$ for any policy $\pi$, where $\mu(\pi_k)$ denotes the occupancy measure of policy $\pi_k$. This notation will be used in the following proofs.


%%%%%%%%%%%%%%%%%%%%%%%%%%%%%%%%%%%%%%%%%%%%%%%%%%%%%%%%%%%%%%%%%%%%%%%%%%%%%%%%
%%%%%%%%%%%%%%%%%%%%%%%%%%%%%%%%%%%%%%%%%%%%%%%%%%%%%%%%%%%%%%%%%%%%%%%%%%%%%%%%
\subsection{Proof of Theorem~\ref{thm:FraUpper} (guarantee for the output of Algorithm~\ref{alg:fra})}

\FraUpper*

\begin{proof}
    Using the decomposition presented in \Cref{sec:application}, we can express the regret as
    %
    \begin{equation*}
        (1 - \gamma) \regretIL = \underbrace{\sum^K_{k=1} \innerprod{r_k}{\mu\brr{\expert} - \mu\brr{\pi_k}}}_{(1-\gamma)\regretK^\pi(\mu\brr{\expert})} + \underbrace{\sum^K_{k=1} \innerprod{\Phi\transpose\mu(\pi_k) - \Phi\transpose\mu(\expert)}{w_k - w_{\mathrm{true}}}}_{(1-\gamma)\regretK^w(w_{\mathrm{true}})} \label{eq:dec}\,.
    \end{equation*}
    %
    By bounding $\regret^w(w_{\mathrm{true}})$ and $\regret^\pi(\mu(\expert))$ using \Cref{thm:reward_regret_bound} and \Cref{thm:main}, respectively, we obtain that with probability $1 - 3 \delta$
    %
    \begin{align*}
        \frac{1}{K} \regretIL &\leq  10 H (B \WMAX+\RMAX) \sqrt{ \frac{\log \delta^{-1}}{K}} + 24 H\WMAX B \sqrt{\frac{ \log\brr{\frac{1}{\delta}}}{\tau_E}} \\&\phantom{=}+ \tilde{\mathcal{O}}(d^{3/2}(1 - \gamma)^{-9/4} \log^{1/2} \abs{\aspace} K^{-1/2})\,.
    \end{align*}
    %
    Therefore, by considering $B$ and $\RMAX$ as constants and choosing $K = \widetilde{\mathcal{O}}\brr{\frac{ d^{3} \log (\abs{\aspace}\delta^{-1})}{(1 - \gamma)^{4.5}\varepsilon^2}}$ and $\tau_E = \widetilde{\mathcal{O}}\brr{\frac{ \WMAX^2 \log(1/\delta)}{(1 - \gamma)^2\varepsilon^2}}$ we have that with probability $1 - 3 \delta$ it holds that
    %
    \begin{equation*}
        \frac{1}{K} \regretIL \leq 4 \epsilon\,.
    \end{equation*}
    %
    Since $\frac{1}{K} \regretIL$ is a random variable bounded by $(1-\gamma)^{-1}$ almost surely, in expectation we have the following bound
    %
    \begin{equation*}
        \bbE_{\mathrm{Alg}} \bs{\frac{1}{K} \regretIL} \leq \frac{3 \delta}{1-\gamma} + 4 \varepsilon\,.
    \end{equation*}
    %
    Thus, by choosing $\delta \leq \nicefrac{\varepsilon}{3(1-\gamma)}$ we can conclude that
    %
    \begin{equation*}
        \bbE_{\mathrm{Alg}} \bs{\frac{1}{K} \regretIL} \leq 5 \varepsilon\,.
    \end{equation*}
    %
    Finally, by selecting $\pi^{\mathrm{out}}$ uniformly at random from the policies generated by \Cref{alg:fra} 
    we have that
    \begin{equation*}
        \bbE_{\mathrm{Alg}} \innerprod{\initial}{V_{\true}^{\expert} - V_{\true}^{\pi^{\mathrm{out}}}} \leq 5 \varepsilon\,.
    \end{equation*}
\end{proof}


%%%%%%%%%%%%%%%%%%%%%%%%%%%%%%%%%%%%%%%%%%%%%%%%%%%%%%%%%%%%%%%%%%%%%%%%%%%%%%%%
%%%%%%%%%%%%%%%%%%%%%%%%%%%%%%%%%%%%%%%%%%%%%%%%%%%%%%%%%%%%%%%%%%%%%%%%%%%%%%%%
\subsection{Proof of Theorem~\ref{thm:reward_regret_bound} (regret bound for the reward player)}

\begin{restatable}{theorem}{regretreward} \label{thm:reward_regret_bound}
    Assume that $w_{\mathrm{true}} \in \cW$ for some non-empty closed convex set $\cW$ and that for any $w \in \cW$, $\norm{w}\leq \WMAX$. Then, OGD with $\eta_r = \nicefrac{\WMAX}{B \sqrt{K}}$ ran for $K$ iterations satisfies with probability at least $1 - 2 \delta$ that
    %
    \begin{equation*}
        \regretK^w(w_{\mathrm{true}}) \leq 10 H (B \WMAX + \RMAX) \sqrt{K \log 1 / \delta} + 24 H \WMAX K B \sqrt{\frac{\log\brr{\frac{1}{\delta}}}{\tau_E}}\,.
    \end{equation*}
\end{restatable}

\begin{proof}
    Given the definition of the feature expectation vector $\lambda (\pi)$, we can rewrite the regret for the reward player as follows
    %
    \begin{equation*}
        \spr{1 - \gamma} \regret^w(w_{\mathrm{true}}) = \sum^K_{k=1} \innerprod{\lambda(\pi_k) - \lambda(\expert)}{w_k - w_{\mathrm{true}}}\,.
    \end{equation*}
    %
    Then, adding and subtracting the estimators for the occupancy measures, we get
    %
    \begin{align*}
        (1-\gamma) \regret^w(w_{\mathrm{true}}) &= \sum^K_{k=1} \innerprod{\phi_\cost(X_k, A_k) - \widehat{\lambda(\expert)}}{w_k - w_{\mathrm{true}} } \\
        &\phantom{=}+ \sum^K_{k=1} \innerprod{\lambda(\pi_k) - \phi_\cost(X_k, A_k)}{w_k - w_{\mathrm{true}}} \\
        &\phantom{=}+ \sum^K_{k=1} \innerprod{\widehat{\lambda(\expert)} - \lambda(\expert)}{ w_k-w_{\mathrm{true}}}\,.
    \end{align*}
    %
    Now, using the regret bound for OGD \citep{Zin03}, we can bound the first term in the decomposition above as
    %
    \begin{align*}
        \sum^K_{k=1} &\innerprod{\phi_\cost(X_k, A_k) - \widehat{\lambda(\expert)}}{w_k-w_{\mathrm{true}} } \\&\leq  \frac{\max_{w\in\mathcal{W}}\norm{w_{\mathrm{true}} - w_1}_2^2}{2 \eta_r} + \frac{\eta_r}{2} \sum^K_{k=1} \norm{\widehat{\lambda(\expert)} - \phi_\cost(X_k, A_k)}_2^2 \\
        &\leq \frac{2 \WMAX^2}{\eta_r} + 2 \eta_r B^2 K\,,
    \end{align*}
    %
    Looking at the term $\sum^K_{k=1} \innerprod{\lambda (\pi_k) - \phi_\cost (X_k, A_k)}{w_k - w_{\mathrm{true}}}$, we notice that
    %
    \begin{equation*}
        \psi_k = \innerprod{\lambda(\pi_k) - \phi_\cost(X_k, A_k)}{w_k - w_{\mathrm{true}}}
    \end{equation*}
    %
    is a martingale difference sequence such that
    %
    \begin{equation*}
        \abs{\innerprod{\lambda(\pi_k) - \phi_\cost(X_k, A_k)}{w_k - w_{\mathrm{true}}}} \leq  4 \RMAX\,.
    \end{equation*}
    %
    Applying Azuma-Hoeffding's inequality, we have that with probability $1 - \delta$
    %
    \begin{equation*}
        \sum^K_{k=1} \innerprod{\lambda(\pi_k) - \phi_\cost(X_k, A_k)}{w_k - w_{\mathrm{true}}} \leq \RMAX \sqrt{8  K \log \brr{\frac{1}{\delta}}}.
    \end{equation*}
    %
    Then, plugging in this bound in the regret decomposition we obtain
    %
    \begin{align*}
        (1-\gamma) \regret^w(w_{\mathrm{true}}) &\leq \frac{2 \WMAX^2}{\eta_r} + 2 \eta_r B^2 K + \RMAX \sqrt{8K \log 1 / \delta} \\
        &\phantom{=}+ \sum^K_{k=1} \innerprod{\widehat{\lambda(\expert)} - \lambda(\expert)}{ w_k-w_{\mathrm{true}}}\,.
    \end{align*}
    %
    Then, we treat the last term using Cauchy-Schwartz's inequality
    %
    \begin{align*}
        \sum^K_{k=1} \innerprod{\widehat{\lambda(\expert)} - \lambda(\expert)}{ w_k-w_{\mathrm{true}}} &\leq \sum^K_{k=1}\norm{w_{\mathrm{true}}-w^k}_{2}\norm{\widehat{\lambda(\expert)} - \lambda(\expert)}_2 \\
        &\leq 2 \WMAX K \norm{\widehat{\lambda(\expert)} - \lambda(\expert)}_2\,.
    \end{align*}
    %
    It remains to find a high probability (dimension-free) upper bound on $\norm{\widehat{\lambda(\expert)} - \lambda(\expert)}_{2}$. First, notice that $\norm{\widehat{\lambda(\expert)} - \lambda(\expert)}_{2} = \norm{(\tau_E)^{-1} \brr{\sum^{\tau_E}_{i=1} \phi_\cost(X^i_E,A^i_E) - \lambda(\expert) }}_2$. Then, we use the notation $u_{x,a} = \phi_\cost(x,a) - \lambda(\expert) $ for all state action pairs $x,a$ and using that for all $x, a \in \cX \times \cA$, $\norm{\phi_\cost(x, a)}_2 \leq B$, we have
    %
    \begin{equation*}
        \sum^{\tau_E}_{i=1} \mathbb{E}\bs{\norm{u_{X^i_E,A^i_E}}_2^2} \leq \sum^{\tau_E}_{i=1} \mathbb{E}\bs{\norm{\phi_\cost(X^i_E, A^i_E) - \lambda(\expert)}_2^2} \leq 4\tau_E B^2\,.
    \end{equation*}
    %
    Moreover, for any $x, a \in \cX \times \cA$, $\norm{u_{x,a}} \leq 2 B$ and $\bbE \bs{u_{X^i_E, A^i_E}} = 0$ because of the distribution of the dataset $\mathcal{D}_\expert $. Thus, by applying \cite[Proposition 2]{hsu2012tail}, it holds that for all $t > 0$
    %
    \begin{equation*}
        \mathbb{P}\bs{\norm{\sum^{\tau_E}_{i=1} u_{X^i_E, A^i_E}} > \sqrt{4 \tau_E} B + \sqrt{32 \tau_E t} B + (8/3)2Bt} \leq e^{-t}
    \end{equation*}
    %
    Therefore, choosing $t = \log \frac{1}{\delta}$, we obtain that with probability $1 - \delta$
    %
    \begin{align*}
        \norm{\sum^{\tau_E}_{i=1} \phi_\cost(X^i_E, A^i_E) - \lambda(\expert)} &\leq \sqrt{4 \tau_E} B + \sqrt{32 \tau_E \log\brr{\frac{1}{\delta}}} B + \frac{16 B}{3}  \log\brr{\frac{1}{\delta}} \\
        &\leq 6 B \sqrt{\tau_E \log\brr{\frac{1}{\delta}}} + \frac{16 B}{3} \log \brr{\frac{1}{\delta}}\,.
    \end{align*}
    %
    Then, dividing by $\tau_E$ we obtain that
    %
    \begin{align*}
        \norm{\widehat{\lambda(\expert)} - \lambda(\expert)}_{2} \leq 6 B \sqrt{\frac{\log\brr{\frac{1}{\delta}}}{\tau_E}} + \frac{16 B}{3 \tau_E} \log\brr{\frac{1}{\delta}}\,.
    \end{align*}
    %
    Then, for $\tau_E \geq \frac{64}{18^2} \log \frac{1}{\delta}$, we have that
    %
    \begin{equation*}
        6 \sqrt{\frac{\log\brr{\frac{1}{\delta}}}{\tau_E}} \geq \frac{8}{3 \tau_E} \log\brr{\frac{1}{\delta}}\,,
    \end{equation*}
    %
    and hence that with probability $1 - \delta$,
    %
    \begin{align*}
        \norm{\widehat{\lambda(\expert)} - \lambda(\expert)}_{2} \leq 12 B \sqrt{\frac{\log\brr{\frac{1}{\delta}}}{\tau_E}}\,.
    \end{align*}
    Thus, by a union bound and choosing $\eta_r = \nicefrac{\WMAX}{B \sqrt{K}}$, we have that with probability $1 - 2 \delta$,
    \begin{align*}
        (1-\gamma) \regret^w(w_{\mathrm{true}}) &= 4 B \WMAX \sqrt{K} +  \RMAX\sqrt{8 K \log \delta^{-1}} + 24 \WMAX B K \sqrt{ \frac{\log\brr{\frac{1}{\delta}}}{\tau_E}} \\
        &\leq 10 (B \WMAX + \RMAX) \sqrt{K \log \delta^{-1}} + 24 \WMAX K B \sqrt{\frac{\log \brr{\frac{1}{\delta}}}{\tau_E}}\,.
    \end{align*}
\end{proof}


\clearpage
\section{Lower bounds for imitation learning}
\label{app:lower}
In this section, we prove lower bounds for both $K$ and $\tau_E$ for all algorithms following Protocol~\ref{prot:interaction} given hereafter.
%
\begin{protocol}[!h]
    \caption{Imitation learning from features alone in Linear MDPs. \label{prot:interaction}}
    \centering
    \begin{algorithmic}[1]
        \STATE The learner adopts a learning algorithm $\mathrm{Alg}$ that receives as input \\
        (1) a features dataset $\cD_\expert = \scbr{\phi_\cost \spr{X^i_E, A^i_E}}_{i=1}^{\tau_E}$ where for any $i \in \sbr{\tau_E}$, $X^i_E, A^i_E \sim \mu \spr{\expert}$, \\
        (2) read access to $\phi_P \spr{x, a}$ for all $x, a \in \cX \times \cA$, \\
        (3) trajectory access to $\cM \setminus \true$, and \\
        (4) the reward class $\cR$ such that $\true \in \cR$.
        
        \STATE $\mathrm{Alg}$ samples $K$ trajectories from $\cM \setminus \true$ and outputs $\pi^{\mathrm{out}}$ s.t. $\bbE \sbr{\inp{\initial, V_{\true}^\expert - V_{\true}^{\pi^{\mathrm{out}}}}} \leq \varepsilon$.
    \end{algorithmic}
\end{protocol}

We prove an $\Omega \spr{\varepsilon^{-2}}$ lower bound for both cases, demonstrating that Algorithm~\ref{alg:fra} is rate optimal. First, we state the lower bound $K$ that holds even with perfect knowledge of the expert feature expectation vector $\lambda \spr{\expert}$, a strictly easier setting compared the one under which \Cref{thm:FraUpper} is proven.
%
\begin{restatable}{theorem}{LowerK} \textbf{(Lower Bound on $K$)} \label{thm:LowerK}
    For any algorithm $\mathrm{Alg}$, there exists an MDP $\cM$ and an expert policy $\expert$ such that $\mathrm{Alg}$, taking as input $ \phim_r\transpose \mu_\cM \spr{\expert}$, requires $K = \Omega \spr{\frac{d}{\spr{1 - \gamma}^2 \varepsilon^2}}$ to guarantee $\bbE_{\mathrm{Alg}} \sbr{\inp{\initial, V_\cM^{\expert} - V_\cM^{\pi^{\mathrm{out}}}}} = \cO \spr{\varepsilon}$.
\end{restatable}

\noindent Next, we establish a lower bound on the required number of expert demonstration $\tau_E$. The result holds even with perfect knowledge of the transition dynamics (\ie for $K = \infty$).
%
\begin{restatable}{theorem}{LowerTauE} \textbf{(Lower Bound on $\tau_E$)} \label{thm:LowerTauE} \label{thm:lower-bound-2states_expert}
    Let $\gamma \geq \frac12$. For any algorithm $\mathrm{Alg}$, there exists an MDP $\cM$ and an expert policy $\expert$ such that $\mathrm{Alg}$ taking as input the transitions dynamics and an expert dataset of size $\tau_E$ requires $\tau_E = \Omega \spr{\frac{\WMAX^2}{\spr{1 - \gamma}^2 \varepsilon^2}}$ to guarantee $\bbE_{\mathrm{Alg}} \sbr{\inp{\initial, V_\cM^{\expert} - V_\cM^{\pi^{\mathrm{out}}}}} = \cO \spr{\varepsilon}$.
\end{restatable}
%
\noindent The proofs are provided in the following sections.


%%%%%%%%%%%%%%%%%%%%%%%%%%%%%%%%%%%%%%%%%%%%%%%%%%%%%%%%%%%%%%%%%%%%%%%%%%%%%%%%
%%%%%%%%%%%%%%%%%%%%%%%%%%%%%%%%%%%%%%%%%%%%%%%%%%%%%%%%%%%%%%%%%%%%%%%%%%%%%%%%
\subsection{Proof of Theorem~\ref{thm:LowerK} (lower bound on the number of interactions)}

We start with the proof of the lower bound on $K$. We consider a class of possibly randomized algorithms that output a policy $\pi^{\mathrm{out}}$ given a dataset of expert features $\cD_\expert$ and $K$ trajectories collected by the learner in the MDP $\cM$.

\noindent \emph{Proof Idea}. To construct a lower bound, we consider the case of imitation learning from states alone (\ie $\phim_r \spr{x, a} = \bfe_x$), and $\lambda \spr{\expert}$ represents the state expert occupancy measure. We consider the case of a two-state MDP, where $\cX = \scbr{x_0, x_1}$, and the learner knows the \emph{good} state $x_0$ that maximizes the expert's occupancy measure due to having access to $\lambda \spr{\expert}$. The leaner's objective is to maximize the time spent in this good state. All actions in the \emph{bad} state $x_1$ share the same transition kernel. Therefore, the agent's decisions in the state $x_0$ is the only factor that influences the outcome. An action labeled as $a^\star$ is available in the state $x_0$. The transition kernel $P\spr{x_0 \given x_0, a}$ is identical for all actions $a \neq a^\star$, while for $a^\star$, it is defined as $P \spr{x_0 \given x_0, a} + \epsilon$. We then consider a family of $\abs{\cA}$ MDPs, where each MDP assigns the role of $a^\star$ to a different action. We will formally demonstrate that for any algorithm in $\mathrm{Alg}$, there exists at least one MDP within this family where achieving $\bbE_{\mathrm{Alg}} \sbr{\inp{\initial, V_\cM^{\expert} - V_\cM^{\pi^{\mathrm{out}}}}} = \cO \spr{\varepsilon}$ requires $K = \Omega \spr{\frac{\abs{\cA}}{\spr{1 - \gamma}^2 \varepsilon^2}}$. Finally, the bound for an arbitrary dimension $d$ is obtained noticing that this MDP can be written as a linear MDP with features dimension $d = 2 + 2 \abs{\cA}$.

\bigskip
\begin{proof}
    For any policy $\pi$, we denote $\lambda_\cM \spr{\pi} = \phim_r^\trans \mu_\cM \spr{\pi}$ the expected feature vector of the policy $\pi$ in the MDP $\cM$. We consider a deterministic algorithm $\mathrm{Alg}$ that maps $\lambda_\cM \spr{\expert}$ and $K$ environment trajectories to a policy. The extension to randomized algorithms can be done by an application of Fubini's theorem (see \cite{bubeck2012regret}). The hard instance we consider for the lower bound is an MDP $\cM$ with two states, $x_0$ and $x_1$, and $\abs{\cA}$ actions per state. For any action $a$, the reward function is given by $\true \spr{x_0, a} = 1$, and $\true \spr{x_1, a} = 0$. We will refer to state $x_0$ as the ``good'' state and to state $x_1$ as the ``bad'' state.  In state $x_1$, the transition kernel induced by any action $a$ is the same, \ie $P \spr{x_1 \given x_1, a} = 1 - \delta_1$, and $P \spr{x_0 \given x_1, a} = \delta_1$ for some $\delta_1 \in \spr{0, 1}$. Let $\delta_0 \in \spr{0, 1}$ and $\epsilon \in \spr{0, \delta_0}$. In state $x_0$, there is an action $a^\star$ with a slightly different transition kernel
    %
    \begin{align*}
        P \spr{x_1 \given x_0, a^\star} = \delta_0 - \epsilon, \quad P \spr{x_0 \given x_0, a^\star} = 1 - \delta_0 + \epsilon\,,
    \end{align*}
    %
    whereas for any action $a \neq a^\star$, we set
    %
    \begin{align*}
        P \spr{x_1 \given x_0, a} = \delta_0, \quad P \spr{x_0 \given x_0, a} = 1 - \delta_0\,.
    \end{align*}
    %
    We set the unknown expert policy $\expert$ such that it always select action $a^\star$ in both states, \ie $\expert \spr{a^\star \given x_0} = \expert \spr{a^\star \given x_1} = 1$. Setting $\nu_0 \spr{x_0} = 1$, we can write the flow constraints and get
    %
    \begin{align*}
        &\nu \spr{\expert, x_0} = 1 - \gamma + \gamma \spr{1 - \delta_0 + \epsilon} \nu \spr{\expert, x_0} + \gamma \delta_1 \nu \spr{\expert, x_1}\,, \\
        &\nu \spr{\expert, x_1} = \gamma \spr{1 - \delta_1} \nu \spr{\expert, x_1} + \gamma \spr{\delta_0 - \epsilon} \nu \spr{\expert, x_0}\,.
    \end{align*}
    %
    The second equation gives $\nu \spr{\expert, x_1} = \frac{\gamma \spr{\delta_0 - \epsilon}}{1 - \gamma \spr{1 - \delta_1}} \nu \spr{\expert, x_0}$, which we can plug back into the first equation to obtain
    %
    \begin{align*}
        \nu \spr{\expert, x_0} = 1 - \gamma + \spr{\gamma \spr{1 - \delta_0 + \epsilon} + \frac{\gamma^2 \delta_1 \spr{\delta_0 - \epsilon}}{1 - \gamma \spr{1 - \delta_1}}} \nu \spr{\expert, x_0} \,,
    \end{align*}
    %
    which we can rearrange to get
    %
    \begin{align*}
        \nu \spr{\expert, x_0} = \frac{1 - \gamma + \gamma \delta_1}{1 - \gamma + \gamma \delta_1 + \gamma \delta_0 - \gamma \epsilon}\,.
    \end{align*}
    %
    Using the normalization constraint $\nu \spr{\expert, x_0} + \nu \spr{\expert, x_1} = 1$, we also get
    %
    \begin{equation*}
        \nu \spr{\expert, x_1} = \frac{\gamma \delta_0 - \gamma \epsilon}{1 - \gamma + \gamma \delta_1 + \gamma \delta_0 - \gamma \epsilon}\,.
    \end{equation*}
    %
    %Note that the expert state occupancy is the same for any environment $\cM_i$, and that for any $\cM \in \cH$, $\lambda_\cM \spr{\expert} = \nu_\cM \spr{\expert}$.
    Furthermore, let $\pi_{\mathrm{bad}}$ be a ``bad'' policy that always plays an action $a \neq a^\star$. The same calculation with $\epsilon = 0$ shows that the state occupancy measure for the policy $\pi_{\mathrm{bad}}$ is given by
    %
    \begin{align*}
        \nu \spr{\pi_{\mathrm{bad}}, x_0} &= \frac{1 - \gamma + \gamma \delta_1}{1 - \gamma + \gamma \delta_1 + \gamma \delta_0}\,, \\
        \nu \spr{\pi_{\mathrm{bad}}, x_1} &= \frac{\gamma \delta_0}{1 - \gamma + \gamma \delta_1 + \gamma \delta_0}\,.
    \end{align*}
    %
    Let $\tilde{\pi}$ be any policy. Noting that for any $x$, $V^\expert \spr{x} = Q^\expert \spr{x, a^\star}$, we can use the performance difference lemma and get
    %
    \begin{align*}
        \inp{\mu \spr{\expert} - \mu \spr{\tilde{\pi}}, \true} &= \bbE_{\spr{x, a} \sim \mu \spr{\tilde{\pi}}} \sbr{V^\expert \spr{x} - Q^\expert \spr{x, a}} \\
        &= \bbE_{\spr{x, a} \sim \mu \spr{\tilde{\pi}}} \sbr{Q^\expert \spr{x, a^\star} - Q^\expert \spr{x, a}}\,.
    \end{align*}
    %
    All actions share the same transition kernel in $x_1$ thus for any action $a$, $Q^\expert \spr{x_1, a^\star} = Q^\expert \spr{x_1, a}$ and we have
    %
    \begin{align*}
        \inp{\mu \spr{\expert} - \mu \spr{\tilde{\pi}}, \true} &= \nu \spr{\tilde{\pi}, x_0} \sum_{a \in \cA \setminus \scbr{a^\star}} \tilde{\pi} \spr{a \given x_0} \spr{Q^\expert \spr{x_0, a^\star}  - Q^\expert \spr{x_0, a}}\,.
    \end{align*}
    %
    Next, we need to compute the difference of Q-values. Using the Bellman equations for $\expert$ in state $x_0$, we have
    %
    \begin{align}
        \forall a \neq a^\star, &Q^\expert \spr{x_0, a} = 1 + \gamma \delta_0 Q^\expert \spr{x_1, a^\star} + \gamma \spr{1 - \delta_0} Q^\expert \spr{x_0, a^\star} \label{eq:bellman-eq-x0-a} \\
        &Q^\expert \spr{x_0, a^\star} = 1 + \gamma \spr{\delta_0 - \epsilon} Q^\expert \spr{x_1, a^\star} + \gamma \spr{1 - \delta_0 + \epsilon} Q^\expert \spr{x_0, a^\star}\,. \label{eq:bellman-eq-x0-astar}
    \end{align}
    %
    Solving the second equation for $Q^\expert \spr{x_0, a^\star}$ gives
    %
    \begin{align}
        Q^\expert \spr{x_0, a^\star} = \frac{1}{1 - \gamma \spr{1 - \delta_0 + \epsilon}} \spr{1 + \gamma \spr{\delta_0 - \epsilon} Q^\expert \spr{x_1, a^\star}} \label{eq:q-x0-astar-inter}\,.
    \end{align}
    %
    By the Bellman equation in state $x_1$ and action $a^\star$, we further have
    %
    \begin{align*}
        Q^\expert \spr{x_1, a^\star} = 0 + \gamma \delta_1 Q^\expert \spr{x_0, a^\star} + \gamma \spr{1 - \delta_1} Q^\expert \spr{x_1, a^\star}\,,
    \end{align*}
    %
    which implies that
    %
    \begin{align}
        Q^\expert \spr{x_1, a^\star} = \frac{\gamma \delta_1}{1 - \gamma(1 - \delta_1)} Q^\expert \spr{x_0, a^\star}\,. \label{eq:q-x1-astar-inter}
    \end{align}
    %
    Replacing \eqref{eq:q-x1-astar-inter} into \eqref{eq:q-x0-astar-inter}, we get
    %
    \begin{align*}
        Q^\expert \spr{x_0, a^\star} &= \frac{1}{1 - \gamma \spr{1 - \delta_0 + \epsilon}} + \frac{\gamma^2 \delta_1 \spr{\delta_0 - \epsilon}}{\spr{1 - \gamma \spr{1 - \delta_0 + \epsilon}} \spr{1 - \gamma \spr{1 - \delta_1}}} Q^\expert \spr{x_0, a^\star}\,.
    \end{align*}
    %
    Rearranging the terms gives
    %
    \begin{align}
        Q^\expert \spr{x_0, a^\star} &= \spr{1 - \frac{\gamma^2 \delta_1 \spr{\delta_0 - \epsilon}}{\spr{1 - \gamma \spr{1 - \delta_0 + \epsilon}} \spr{1 - \gamma \spr{1 - \delta_1}}}}^{-1} \frac{1}{1 - \gamma \spr{1 - \delta_0 + \epsilon}} \nonumber \\
        &= \frac{1 - \gamma \spr{1 - \delta_1}}{\spr{1 - \gamma \spr{1 - \delta_0 + \epsilon}} \spr{1 - \gamma \spr{1 - \delta_1}} - \gamma^2 \delta_1 \spr{\delta_0 - \epsilon}}\,. \label{eq:q-x0-astar}
    \end{align}
    %
    Plugging Equation~\eqref{eq:q-x0-astar} into Equation~\eqref{eq:q-x1-astar-inter}, we can deduce the value of the expert at $\spr{x_1, a^\star}$
    %
    \begin{equation*}
        Q^\expert \spr{x_1, a^\star} = \frac{\gamma \delta_1}{\spr{1 - \gamma \spr{1 - \delta_0 + \epsilon}} \spr{1 - \gamma \spr{1 - \delta_1}} - \gamma^2 \delta_1 \spr{\delta_0 - \epsilon}}\,.
    \end{equation*}
    %
    Looking at the difference $Q^\expert \spr{x_0, a^\star} - Q^\expert \spr{x_0, a}$, we can take the difference of Equations~\eqref{eq:bellman-eq-x0-astar} and \eqref{eq:bellman-eq-x0-a} to get
    %
    \begin{align*}
        Q^\expert \spr{x_0, a^\star} - Q^\expert \spr{x_0, a} &= \gamma \epsilon \spr{Q^\expert \spr{x_0, a^\star} - Q^\expert \spr{x_1, a^\star}} \\
        &= \frac{\gamma \epsilon \spr{1 - \gamma}}{\underbrace{\spr{1 - \gamma \spr{1 - \delta_0 + \epsilon}} \spr{1 - \gamma \spr{1 - \delta_1}} - \gamma^2 \delta_1 \spr{\delta_0 - \epsilon}}_{\spr{\diamondsuit}}}\,.
    \end{align*}
    %
    Next, we upper bound the denominator as follows
    %
    \begin{align*}
        \spr{\diamondsuit} &= 1 - \gamma \spr{1 - \delta_0 + \epsilon} - \gamma \spr{1 - \delta_1} \\
        &\phantom{=}+ \gamma^2 \spr{1 - \delta_0 + \epsilon - \delta_1 + \delta_0 \delta_1 - \epsilon \delta_1 - \delta_0 \delta_1 + \epsilon \delta_1} \\
        &= 1 - \gamma \spr{1 - \delta_0 + \epsilon} - \gamma \spr{1 - \delta_1} + \gamma^2 \spr{1 - \delta_0 - \delta_1 + \epsilon} \\
        &= 1 - \gamma + \gamma \delta_0 \spr{1 - \gamma} + \gamma \delta_1 \spr{1 - \gamma} - \gamma \spr{1 - \gamma} - \gamma \epsilon \spr{1 - \gamma} \\
        &= \spr{1 - \gamma}^2 + \gamma \delta_0 \spr{1 - \gamma} + \gamma \delta_1 \spr{1 - \gamma} - \gamma \epsilon \spr{1 - \gamma} \\
        &\leq \spr{1- \gamma}^2 + \gamma \delta_0 \spr{1 - \gamma} + \gamma \delta_1 \spr{1 - \gamma}\,,
    \end{align*}
    %
    where the inequality follows from $\gamma \epsilon \spr{1 - \gamma} > 0$. Setting $\delta_1 = \delta_0 = \frac{1 - \gamma}{\gamma}$, we obtain
    %
    \begin{equation*}
        \spr{\diamondsuit} \leq 3 \spr{1 - \gamma}^2\,,
    \end{equation*}
    %
    and it holds that
    %
    \begin{align*}
        Q^\expert \spr{x_0, a^\star} - Q^\expert \spr{x_0, a} \geq \frac{\gamma \epsilon}{3 \spr{1 - \gamma}}\,.
    \end{align*}
    %
    Moreover, the choice of $\delta_0$ and $\delta_1$ implies that $\nu \spr{\pi_{\mathrm{bad}}, x_0} = \frac23$. By definition of the transitions, note that always playing $a \neq a^\star$ like $\pi_{\mathrm{bad}}$ does minimizes the probability of being in state $x_0$. Thus, for any policy $\tilde{\pi}$, $\nu \spr{\tilde{\pi}, x_0} \geq \nu \spr{\pi_{\mathrm{bad}}, x_0}$, and we have
    %
    \begin{align*}
        \inp{\mu \spr{\expert} - \mu \spr{\tilde{\pi}}, \true} &\geq \nu \spr{\tilde{\pi}, x_0} \sum_{a \in \cA \setminus \scbr{a^\star}} \tilde{\pi} \spr{a \given x_0} \frac{\gamma \epsilon}{3 \spr{1 - \gamma}} \\
        &\geq \nu \spr{\pi_{\mathrm{bad}}, x_0} \spr{1 - \tilde{\pi} \spr{a^\star \given x_0}} \frac{\gamma \epsilon}{3 \spr{1 - \gamma}} \\
        &= 2 \spr{1 - \tilde{\pi} \spr{a^\star \given x_0}} \frac{\gamma \epsilon}{9(1 - \gamma)} \\
        &\geq \frac{\spr{1 - \tilde{\pi} \spr{a^\star \given x_0}} \epsilon}{9 \spr{1 - \gamma}}\,.
    \end{align*}
    %
    where the last inequality follows from $\gamma \geq 1/2$. We now consider the policy $\tilde{\pi} = \bar\pi$ produced by a learning algorithm $\mathrm{Alg}$ interacting with the MDP described above (with $\epsilon > 0$). We also consider  $\underline{\pi}$ the output of the same learning algorithm $\mathrm{Alg}$ when interacting with the MDP $\underline{\cM}$, a copy of $\cM$ with $\epsilon = 0$ (note that in $\underline{\cM}$, all actions are identical in \emph{both} states $x_0$ and $x_1$, so there is nothing to learn). In $\cM$, all actions are identical in state $x_1$, thus we can assume both policies are the same in state $x_1$, \ie $\bar\pi \spr{\cdot \given x_1} = \underline{\pi} \spr{\cdot \given x_1} = \bfe_{a^\star}$, and focus exclusively on learning in state $x_0$. By Pinkser's inequality, we have that
    %
    \begin{align*}
        \bar\pi \spr{a^\star \given x_0} - \underline{\pi} \spr{a^\star \given x_0} \leq \sqrt{2 \KL \spr{\underline{\pi} \spr{\cdot \given x_0} \| \bar\pi \spr{\cdot \given x_0}}}\,,
    \end{align*}
    %
    and the previous inequality becomes
    %
    \begin{align*}
        \inp{\mu \spr{\expert} -  \mu \spr{\bar\pi}, \true} \geq \frac{\epsilon}{9 \spr{1 - \gamma}} \spr{1 - \underline{\pi} \spr{a^\star \given x_0} - \sqrt{2 \KL \spr{\underline{\pi} \spr{\cdot \given x_0} \| \bar\pi \spr{\cdot \given x_0}}}}\,.
    \end{align*}
    %
    Denote $A = \abs{\cA}$ and let $\cH = \scbr{\cM_i}_{i=1}^A$ be a collection of MDPs instances where for any $i = 1, \dots, A$, the MDP $\cM_i$ is a copy of $\cM$ where the $i$th action is equal to $a^\star$, \ie $a_i = a^\star$. We denote $P_i$ the corresponding transitions. For any $i \in [\![1, A]\!]$, we denote $\bar\pi^i$ the policy output by the learning algorithm $\mathrm{Alg}$ after interacting with the instance $\cM_i$, and $\experti$ be the expert policy for the instance $\cM_i$, \ie the policy that always plays $a_i$. We denote $\mu_i \spr{\pi}$ the occupancy measure of any policy $\pi$ in the MDP $\cM_i$. Then, notice that the previous derivations apply for any MDP in $\cH$. Thus, summing over $i \in [\![1, A]\!]$ and noting that $\underline{\pi} \spr{\cdot \given x_0}$ is a probability distribution over $\cA$, we get
    %
    \begin{align}
        \sum_{i = 1}^A \inp{\mu_i \spr{\experti} - \mu_i \spr{\bar\pi^i}, \true} \geq \frac{\epsilon}{9 \spr{1 - \gamma}} \spr{A - 1 - \sum_{i = 1}^A \sqrt{2 \KL \spr{\underline{\pi} \spr{\cdot \given x_0} \| \bar\pi^i \spr{\cdot \given x_0}}}}\,. \label{eq:sum_lower_bound}
    \end{align}
    %
    For any $i \in \sbr{A}$ and $T \in \bbN^\star$, denote $\bbP_i^T$ the probability distribution over sets $\cD_i^T = \scbr{x_0, A_t^i, X_t^i}_{t \in \sbr{T}}$ of $T$ transitions starting from $x_0$ induced by the interaction between the algorithm $\mathrm{Alg}$ and the MDP $\cM_i$. Likewise, we denote $\underline{\bbP}^T$ the probability distribution corresponding to $\underline{\cM}$. Then, by the data processing inequality for the KL divergence, for any $i \in \sbr{A}$, it holds that
    %
    \begin{equation*}
        \KL \spr{\underline{\pi} \spr{\cdot \given x_0} \| \bar\pi^i \spr{\cdot \given x_0}} \leq \KL \spr{\underline{\bbP}^T \| \bbP_i^T}\,.
    \end{equation*}
    %
    Denoting $\underline{\bbE}$ the expectation with respect to $\underline{\bbP}^T$, we can use the Markov property of the environment and continue as follows
    %
    \begin{align*}
        \KL \spr{\underline{\bbP}^T \| \bbP_i^T} &= \underline{\bbE} \sbr{\log \spr{\frac{\prod_{t=1}^T \underline{P} \spr{\underline{X}_t \given x_0, \underline{A}_t} \underline{\bbP}^T \spr{\underline{A}_t \given \underline{X}_1, \underline{A}_1, \dots, \underline{X}_{t-1}}}{\prod_{t=1}^T P_i \spr{\underline{X}_t \given x_0, \underline{A}_t} \bbP_i^T \spr{\underline{A}_t \given \underline{X}_1, \underline{A}_1, \dots, \underline{X}_{t-1}}}}} \\
        &= \underline{\bbE} \sbr{\log \spr{\frac{\prod_{t=1}^T \underline{P} \spr{\underline{X}_t \given x_0, \underline{A}_t}}{\prod_{t=1}^T P_i \spr{\underline{X}_t \given x_0, \underline{A}_t}}}} \\
        &= \underline{\bbE} \sbr{\sumtT \log \spr{\frac{\underline{P} \spr{\underline{X}_t \given x_0, \underline{A}_t}}{P_i \spr{\underline{X}_t \given x_0, \underline{A}_t}}}}\,,
    \end{align*}
    %
    where the probabilities on the actions are equal due to running the same algorithm $\mathrm{Alg}$ with the same history up to time $t-1$. Next, we have
    %
    \begin{align*}
        \KL \spr{\underline{\bbP}^T \| \bbP_i^T} &= \sumtT \sum_{\spr{x, a} \in \cX \times \cA} \underline{\bbP}^T \sbr{\spr{\underline{X}_t, \underline{A}_t} = \spr{x, a}} \log \spr{\frac{\underline{P} \spr{x \given x_0, a}}{P_i \spr{x \given x_0, a}}} \\
        &= \sumtT \sum_{x \in \cX} \underline{\bbP}^T \sbr{\spr{\underline{X}_t, \underline{A}_t} = \spr{x, a_i}} \log \spr{\frac{\underline{P} \spr{x \given x_0, a_i}}{P_i \spr{x \given x_0, a_i}}}\,,
    \end{align*}
    %
    where we used that the transitions $\underline{P}$ and $P_i$ are the same for any action $a \neq a_i$. By definition of the transitions, we further have
    %
    \begin{align*}
        \KL \spr{\underline{\bbP}^T \| \bbP_i^T} &= \sumtT \underline{\bbP}^T \sbr{\spr{\underline{X}_t, \underline{A}_t} = \spr{x_0, a_i}}  \log \spr{\frac{1 - \delta_0}{1 - \delta_0 + \epsilon}} \\
        &\phantom{=}+ \sumtT \underline{\bbP}^T \sbr{\spr{\underline{X}_t, \underline{A}_t} = \spr{x_1, a_i}} \log \spr{\frac{\delta_0}{\delta_0 - \epsilon}}\,.
    \end{align*}
    %
    Next, by definition of $\underline{\bbP}^T$, we have
    %
    \begin{align*}
        \KL \spr{\underline{\bbP}^T \| \bbP_i^T} &= \sumtT \underline{\bbP}^T \sbr{\underline{A}_t = a_i} \underline{P} \spr{x_0 \given x_0, a_i} \log \spr{\frac{1 - \delta_0}{1 - \delta_0 + \epsilon}} \\
        &\phantom{=}+ \sumtT \underline{\bbP}^T \sbr{\underline{A}_t = a_i} \underline{P} \spr{x_1 \given x_0, a_i} \log \spr{\frac{\delta_0}{\delta_0 - \epsilon}} \\
        &= \underline{\bbE} \sbr{\sumtT \mathds{1} \scbr{\underline{A}_t = a_i}} \spr{\spr{1 - \delta_0} \log \spr{\frac{1 - \delta_0}{1 - \delta_0 + \epsilon}} + \delta_0 \log \spr{\frac{\delta_0}{\delta_0 - \epsilon}}}\,.
    \end{align*}
    %
    By \citealp[Lemma 20]{AJO08}, we can bound the KL divergence as follows
    %
    \begin{align*}
        \KL \spr{\underline{\bbP}^T \| \bbP_i^T} &\leq \frac{\epsilon^2}{\delta_0 \log(2)} \underline{\bbE} \sbr{\sumtT \mathds{1} \scbr{\underline{A}_t = a_i}} \\
        &\leq \frac{\epsilon^2}{\spr{1 - \gamma} \log \spr{2}} \underline{\bbE} \sbr{\sumtT \mathds{1} \scbr{\underline{A}_t = a_i}}\,,
    \end{align*}
    %
    where the last inequality is due to the choice of $\delta_0 = \frac{1 - \gamma}{\gamma}$ and $\gamma < 1$. Plugging this into Equation~\eqref{eq:sum_lower_bound} and dividing by $A$, we have
    %
    \begin{align*}
        \frac1A \sum_{i = 1}^A \inp{\mu_i \spr{\experti} - \mu_i \spr{\bar\pi^i}, \true} &\geq \frac{\epsilon}{9 \spr{1 - \gamma}} \spr{1 - \frac1A - \frac{\epsilon}{A} \sum_{i=1}^A \sqrt{\frac{\underline{\bbE} \sbr{\sumtT \mathds{1} \scbr{\underline{A}_t = a_i}}}{\spr{1 - \gamma} \log \spr{2}}}}\,.
    \end{align*}
    %
    By Jensen's inequality, we further get
    %
    \begin{align*}
        \frac1A \sum_{i = 1}^A \inp{\mu_i \spr{\experti} - \mu_i \spr{\bar\pi^i}, \true} &\geq \frac{\epsilon}{9 \spr{1 - \gamma}} \spr{1 - \frac1A - \epsilon \sqrt{\frac{\underline{\bbE} \sbr{\sum_{i=1}^A \sumtT \mathds{1} \scbr{\underline{A}_t = a_i}}}{A \spr{1 - \gamma} \log \spr{2}}}} \\
        &\geq \frac{1}{9 \spr{1 - \gamma}} \spr{\frac{\epsilon}{2} - \epsilon^2 \sqrt{\frac{T}{A \spr{1 - \gamma} \log \spr{2}}}}\,,
    \end{align*}
    %
    where the second inequality follows from $\sum_{i=1}^A \mathds{1} \scbr{\underline{A}_t = a_i} = 1$ almost surely for any $t$ and $1 - \frac1A \geq \frac12$. Note that the value of $\epsilon$ maximizing the lower bound is given by $\epsilon^\star = \frac14 \sqrt{\frac{A \spr{1 - \gamma} \log \spr{2}}{T}}$. To satisfy the constraint $\epsilon^\star \in \spr{0, \delta_0}$ with $\delta_0 = \frac{1 - \gamma}{\gamma}$, assume we have $T \geq \frac{\gamma^2 A \log \spr{2}}{16 \spr{1 - \gamma}}$. We plug the value of $\epsilon^\star$ in the previous inequality to get
    %
    \begin{align*}
        \frac1A \sum_{i = 1}^A \inp{\mu_i \spr{\experti} - \mu_i \spr{\bar\pi^i}, \true} &\geq \frac{1}{16 \cdot 9 \spr{1 - \gamma}} \sqrt{\frac{A \spr{1 - \gamma} \log \spr{2}}{T}} \\
        &= \frac{1}{144} \sqrt{\frac{A \log \spr{2}}{\spr{1 - \gamma} T}}\,,
    \end{align*}
    %
    The average can be upper bounded by the maximum, thus
    %
    \begin{align*}
        \max_{i = 1, \dots, A} \inp{\nu_0, V_{\cM_i}^{\experti} - V_{\cM_i}^{\bar\pi^i}} &= \frac{1}{1 - \gamma} \max_{i = 1, \dots, A} \inp{\mu_i \spr{\experti} - \mu_i \spr{\bar\pi^i}, \true} \\
        &\geq \frac{1}{144} \sqrt{\frac{A \log \spr{2}}{\spr{1 - \gamma}^3 T}}\,.
    \end{align*}
    %
    What remains is to set the number of samples $T$ to make the lower bound small enough to make $\max_{i = 1, \dots, A} \inp{\nu_0, V_{\cM_i}^{\experti} - V_{\cM_i}^{\bar\pi^i}} = \cO \spr{\varepsilon}$ possible, \ie we need to have $T = \Omega \spr{\frac{A}{\spr{1 - \gamma}^3 \varepsilon^2}}$ samples. Therefore, we need $T = \Omega \spr{\frac{A}{\spr{1 - \gamma}^3 \varepsilon^2}}$ samples to learn a $\cO \spr{\varepsilon}$-suboptimal policy in the MDP that achieves the maximum. In order to derive a lower bound on the episodes number $K$ we can divide the sample complexity lower bound for $T$ by the the expected number of transitions per episode which is $\spr{1-\gamma}^{-1}$. This gives $K = \Omega \spr{\frac{A}{\spr{1 - \gamma}^2 \varepsilon^2}}$. We can conclude by noting that our construction used in the lower bound is a linear MDP with dimensionality $d = 2 + 2 \abs{\cA}$, thus we have $K = \Omega \spr{\frac{d}{\spr{1 - \gamma}^2 \varepsilon^2}}$.
\end{proof}


%%%%%%%%%%%%%%%%%%%%%%%%%%%%%%%%%%%%%%%%%%%%%%%%%%%%%%%%%%%%%%%%%%%%%%%%%%%%%%%%
%%%%%%%%%%%%%%%%%%%%%%%%%%%%%%%%%%%%%%%%%%%%%%%%%%%%%%%%%%%%%%%%%%%%%%%%%%%%%%%%
\subsection{Proof of \Cref{thm:LowerTauE} (lower bound on the number of expert transitions)}
\label{app:lower_tau_E}

\emph{Proof Idea:} The construction of the lower bound consists in relating the problem to that of distinguishing two Bernoullis distributions with close means. For that, we consider two MDPs $\cM_0$ and $\cM_1$ that only differ in their reward function. They have two states $\cX = \scbr{x_0, x_1}$ and $\abs{\cA}$ actions available at each state. The initial distribution $\initial$ is chosen to be the uniform distribution over $\cX$. In state $x_1$, any action $a$ induces the same transition kernel: $P \spr{x_0 \given x_1, a} = \delta$. In state $x_0$, any action $a$ except some action $a^\star$ is such that $P \spr{x_1 | x_0, a} = \delta$. However, the special action $a^\star$ allows to stay in the state $x_0$ with a slightly higher probability, \ie $P \spr{x_1 \given x_0, a^\star} = \delta - \epsilon$. Then, the reward function in $\cM_0$ is defined as $\true^0 \spr{x_0, \cdot} = \WMAX$, and $\true^0 \spr{x_0, \cdot} = 0$, while in $\cM_1$, it is defined as $\true^1 \spr{x_0, \cdot} = 0$, $\true^1 \spr{x_1, \cdot} = \WMAX$. Finally, we define an expert $\pi_E^0$ for $\cM_0$ as the policy that always play the action $a^\star$, and an expert $\pi_E^1$ for $\cM_1$ that always play some action $a \neq a^\star$. We then show that the expert occupancy measures satisfy $\nu \spr{\pi_E^0, x_0} = 1/2 + \Delta$, for some small $\Delta > 0$, while $\nu \spr{\pi_E^1, x_0} = \frac12$. The remaining step is to reduce this problem to a lower bound on the regret of a two-arm Bernoulli bandits instance with means $\spr{1/2, 1/2}$ and $\spr{1/2 + \Delta, 1/2 - \Delta}$. The proof is formally presented hereafter.

\LowerTauE*

\begin{proof}
    As mentioned earlier, it is sufficient to consider deterministic algorithms that map histories to policies. The lower bound for randomized algorithms follows by an application of Fubini's theorem (see \citealp{bubeck2012regret}). We consider two MDPs $\cH = \scbr{\cM_0, \cM_1}$ with the same state space $\cX = \scbr{x_0, x_1}$ and $\abs{\cA}$ actions available in each state. The initial distribution $\initial$ is chosen to be the uniform distribution over $\cX$, \ie $\initial \spr{x_0} = \initial \spr{x_1} = \frac12$. The transitions are the same in both MDPs: in state $x_1$, each action $a \in \cA$ induces the following transition kernel
    %
    \begin{equation*}
        P \spr{x_0 \given x_1, a} = \delta, \quad P \spr{x_1 \given x_1, a} = 1 - \delta\,
    \end{equation*}
    %
    while in state $x_0$, there is an action $a^\star$ giving a slightly higher probability on staying in state $x_0$, \ie
    %
    \begin{align*}
        &P \spr{x_0 \given x_0, a^\star} = 1 - \delta + \epsilon, \quad P \spr{x_1 \given x_0, a^\star} = \delta - \epsilon \\
        \forall a \neq a^\star, &P \spr{x_0 \given x_0, a} = 1 - \delta, \quad P \spr{x_1 \given x_0, a} = \delta\,.
    \end{align*}
    %
    The reward functions, $\true^0$ and $\true^1$, are different. In $\cM_0$, the ``good'' state is $x_0$, \ie for any action $a \in \cA$, we set $\true^0 \spr{x_0, a} = \WMAX$, $\true^0 \spr{x_1, a} = 0$, and in $\cM_1$, the ``good'' state is $x_1$, \ie $\true^1 \spr{x_0, a} = 0$, and $\true^1 \spr{x_1, a} = \WMAX$. Note that $\WMAX = \RMAX$ due to using the features $\phi_r \spr{x, a} = \bfe_x$ for any state-action pair $x, a$.
     
    Then, we define one expert policy for each MDP. In $\cM_0$, the expert $\pi_E^0$ is the policy that always plays $a^\star$ and in $\cM_1$, the expert $\pi_E^1$ is the policy that always plays an action $a \neq a^\star$. Therefore, the state occupancy measure of expert $\pi_E^0$ in MDP $\cM_0$ has the highest mass in state $x_0$, while $\pi_E^1$ put equal mass on both states. Indeed, writing the flow constraints for both experts, we have
    %
    \begin{align*}
        \begin{pmatrix}
            1 - \gamma + \gamma \delta - \gamma \epsilon & - \gamma \delta \\
            - \gamma \spr{\delta - \epsilon} & 1 - \gamma + \gamma \delta
        \end{pmatrix}
        \nu \spr{\pi_E^0} &= \nu_0\,, \\
        \begin{pmatrix}
            1 - \gamma + \gamma \delta & - \gamma \delta \\
            - \gamma \delta & 1 - \gamma + \gamma \delta
        \end{pmatrix}
        \nu \spr{\pi_E^1} &= \nu_0\,.
    \end{align*}
    %
    Solving these linear systems using, \eg, Cramer's rule, we obtain
    %
    \begin{align*}
        \nu \spr{\pi_E^0, x_0} &= \frac{1 - \gamma + 2 \gamma \delta}{2 \spr{1 - \gamma - \gamma \epsilon + 2 \gamma \delta}}, &\nu \spr{\pi_E^0, x_1} &= \frac{1 - \gamma - 2 \gamma \epsilon + 2 \gamma \delta}{2 \spr{1 - \gamma - \gamma \epsilon + 2 \gamma \delta}}\,, \\
        \nu \spr{\pi_E^1, x_0} &= \frac12, &\nu \spr{\pi_E^1, x_1} &= \frac12\,.
    \end{align*}
    %
    For $i \in \scbr{0, 1}$, let $\bar\pi^i$ be the policy output by $\mathrm{Alg}$ when given a dataset $\cD_{\pi_E^i}$ as input and let $V_i^{\bar\pi^i}$ be the value function of policy $\bar\pi^i$ corresponding to the reward function $\true^i$ from the MDP $\cM_i$. By definition of $\true^i$, we can write
    %
    \begin{align}
        \frac12 \sum_{i \in \scbr{0, 1}} \inp{\initial, V_i^{\pi_E^i} - V_i^{\bar\pi^i}} &= \frac{1}{2 \spr{1 - \gamma}} \sum_{i \in \scbr{1, 2}} \inp{\mu \spr{\pi_E^i} - \mu \spr{\bar\pi^i}, \true^i} \nonumber \\
        &= \frac{\WMAX}{2 \spr{1 - \gamma}} \spr{\nu \spr{\pi_E^0, x_0} - \nu \spr{\bar\pi^0, x_0} + \nu \spr{\pi_E^1, x_1} - \nu \spr{\bar\pi^1, x_1}}\,. \label{eq:appE2-average-return}
    \end{align}
    %
    Thus, we need to compute the difference between state occupancy measures. Let $\tilde\pi$ be an arbitrary policy and denote $\alpha \in \sbr{0, 1}$ the probability of playing action $a^\star$ in state $x_0$, \ie $\tilde\pi \spr{a^\star \given x_0} = \alpha$. Writing down the flow constraints again, we can show that
    %
    \begin{equation*}
        \nu \spr{\tilde\pi, x_0} = \frac{1 - \gamma + 2 \gamma \delta}{2 \spr{1 - \gamma - \gamma \alpha \epsilon + 2 \gamma \delta}}, \quad \nu \spr{\tilde\pi, x_1} = \frac{1 - \gamma - 2 \gamma \alpha \epsilon + 2 \gamma \delta}{2 \spr{1 - \gamma - \gamma \alpha \epsilon + 2 \gamma \delta}}\,.
    \end{equation*}
    %
    Looking at the difference with $\pi_E^0$ in state $x_0$, we have
    %
    \begin{align*}
        \nu \spr{\pi_E^0, x_0} - \nu \spr{\tilde\pi, x_0} &= \frac{1 - \gamma + 2 \gamma \delta}{2 \spr{1 - \gamma - \gamma \epsilon + 2 \gamma \delta}} - \frac{1 - \gamma + 2 \gamma \delta}{2 \spr{1 - \gamma - \gamma \alpha \epsilon + 2 \gamma \delta}} \\
        &= \frac{\spr{1 - \gamma + 2 \gamma \delta} \spr{\spr{- \gamma \alpha \epsilon} - \spr{- \gamma \epsilon}}}{2 \spr{1 - \gamma - \gamma \epsilon + 2 \gamma \delta} \spr{1 - \gamma - \gamma \alpha \epsilon + 2 \gamma \delta}} \\
        &= \frac{\spr{1 - \gamma + 2 \gamma \delta} \gamma \epsilon \spr{1 - \alpha}}{2 \spr{1 - \gamma - \gamma \epsilon + 2 \gamma \delta} \spr{1 - \gamma - \gamma \alpha \epsilon + 2 \gamma \delta}}\,.
    \end{align*}
    %
    Setting $\delta = \frac{1 - \gamma}{\gamma}$ and noting $\epsilon \geq 0$, $\gamma \geq \frac12$, we can lower bound the difference as follows
    %
    \begin{align}
        \nu \spr{\pi_E^0, x_0} - \nu \spr{\tilde\pi, x_0} &= \frac{3 \spr{1 - \gamma} \gamma \epsilon \spr{1 - \alpha}}{2 \spr{3 \spr{1 - \gamma} - \gamma \epsilon} \spr{3 \spr{1 - \gamma} - \gamma \alpha \epsilon}} \nonumber \\
        &\geq \frac{\epsilon \spr{1 - \alpha}}{12 \spr{1 - \gamma}}\,. \label{eq:appE2-diffx0}
    \end{align}
    %
    Likewise, the difference between $\nu \spr{\pi_E^1}$ and $\nu \spr{\tilde\pi}$ in state $x_1$ is given by
    %
    \begin{align*}
        \nu \spr{\pi_E^1, x_1} - \nu \spr{\tilde\pi, x_1} &= \frac12 - \frac{1 - \gamma - 2 \gamma \alpha \epsilon + 2 \gamma \delta}{2 \spr{1 - \gamma - \gamma \alpha \epsilon + 2 \gamma \delta}} \\
        &= \frac{\gamma \alpha \epsilon}{2 \spr{1 - \gamma - \gamma \alpha \epsilon + 2 \gamma \delta}}\,.
    \end{align*}
    %
    Using the definition of $\delta$, and again $\epsilon \geq 0$, $\gamma \geq \frac12$, we get
    %
    \begin{align}
        \nu \spr{\pi_E^1, x_1} - \nu \spr{\tilde\pi, x_1} &= \frac{\gamma \alpha \epsilon}{2 \spr{3 \spr{1 - \gamma} - \gamma \alpha \epsilon}} \nonumber \\
        &\geq \frac{\epsilon \alpha}{12 \spr{1 - \gamma}}\,. \label{eq:appE2-diffx1}
    \end{align}
    %
    Plugging Inequalities~\eqref{eq:appE2-diffx0} and~\eqref{eq:appE2-diffx1} into Equation~\eqref{eq:appE2-average-return} with $\alpha = \bar\pi^0 \spr{a^\star \given x_0}$ and $\alpha = \bar\pi^1 \spr{a^\star \given x_0}$ respectively, we get
    %
    \begin{align*}
        \frac12 \sum_{i \in \scbr{0, 1}} \inp{\initial, V_i^{\pi_E^i} - V_i^{\bar\pi^i}} &\geq \frac{\epsilon \WMAX}{24 \spr{1 - \gamma}^2}\spr{1 - \bar\pi^0 \spr{a^\star \given x_0} + \bar\pi^1 \spr{a^\star \given x_0}} \\
        &= \frac{\epsilon \WMAX}{24 \spr{1 - \gamma}^2} \spr{\sum_{a \neq a^\star} \bar\pi^0 \spr{a \given x_0} + \bar\pi^1 \spr{a^\star \given x_0}}\,.
    \end{align*}
    %
    Next, we can lower bound the right hand side using the Bretagnolle-Huber inequality (see \citealp{bretagnolle1979estimation}, and \citealp[Theorem 14.2]{lattimore2020bandit}), which gives
    %
    \begin{equation} \label{eq:partial_lower_bound}
        \frac12 \sum_{i \in \scbr{0, 1}} \inp{\initial, V_i^{\pi_E^i} - V_i^{\bar\pi^i}} \geq \frac{\epsilon \WMAX}{24 \spr{1 - \gamma}^2} \exp \spr{- \KL \spr{\bar\pi^0 \spr{\cdot \given x_0} \| \bar\pi^1 \spr{\cdot \given x_0}}}\,.
    \end{equation}
    %
    Then, using the data processing inequality and using the fact that the learning algorithm produces $\bar\pi^i$ as a deterministic function of the dataset $\cD_{\pi_E^i}$ for $i = 0, 1$, we have that
    %
    \begin{equation*}
        \KL \spr{\bar\pi^0 \spr{\cdot \given x_0} \| \bar\pi^1 \spr{\cdot \given x_0}} \leq \KL \spr{\bbP_0^{\tau_E} \| \bbP_1^{\tau_E}}\,,
    \end{equation*}
    %
    where, for $i \in \scbr{0, 1}$, we denoted $\bbP_i^{\tau_E}$ the probability distribution over datasets of size $\tau_E$ induced by the interaction between the expert $\pi_E^i$ and the environment (analog to what is done in the proof of Theorem~\ref{thm:LowerK}). Next, we denote $\kl \spr{p, q}$ and $\chi^2 \spr{p, q}$ the KL and chi-squared divergences between bernoulli distributions of means $p$ and $p'$, \ie
    %
    \begin{align*}
        \kl \spr{p, q} &= p \log \spr{\frac{p}{q}} + \spr{1 - p} \log \spr{\frac{1 - p}{1 - q}} \\
        \chi^2 \spr{p, q} &= \frac{\spr{p - q}^2}{q \spr{1 - q}}\,.
    \end{align*}
    %
    By definition of the KL, we have
    %
    \begin{align*}
        \KL \spr{\bbP_0^{\tau_E} \| \bbP_1^{\tau_E}} &= \tau_E \cdot \kl \spr{\frac{3 \spr{1 - \gamma}}{2 \spr{3 \spr{1 - \gamma} - \gamma \epsilon}}, \frac12} \\
        &\leq \tau_E \cdot \chi^2 \spr{\frac{3 \spr{1 - \gamma}}{2 \spr{3 \spr{1 - \gamma} - \gamma \epsilon}}, \frac12} \\
        &= \tau_E \cdot \chi^2 \spr{\frac12 + \frac{\gamma \epsilon}{3 \spr{1 - \gamma} - \gamma \epsilon}, \frac12} \\
        &= \frac{4 \tau_E \gamma^2 \epsilon^2}{\spr{3 \spr{1 - \gamma} - \gamma \epsilon}^2} \\
        &\leq \frac{\tau_E \gamma^2 \epsilon^2}{\spr{1 - \gamma}^2}\,,
    \end{align*}
    %
    where the first inequality follows from the concavity of the logarithm function, and the second inequality uses the fact that $\epsilon \leq \delta = \frac{1-\gamma}{\gamma}$. Thus, plugging in this last inequality into Equation~\eqref{eq:partial_lower_bound}, we obtain
    %
    \begin{align*}
        \frac12 \sum_{i \in \scbr{0, 1}} \inp{\initial, V_i^{\pi_E^i} - V_i^{\bar\pi^i}} &\geq \frac{\epsilon \WMAX}{24 \spr{1 - \gamma}^2} \exp \spr{- \frac{\tau_E \gamma^2 \epsilon^2}{\spr{1 - \gamma}^2}} \\
        &\geq \frac{\epsilon \WMAX}{24 \spr{1 - \gamma}^2} \exp \spr{- \frac{\tau_E \epsilon^2}{\spr{1 - \gamma}^2}}\,,
    \end{align*}
    %
    where we used $\gamma < 1$ in the second inequality. Introducing $\epsilon' = \epsilon \spr{1 - \gamma}^{-1}$, we can rewrite the previous inequality as
    %
    \begin{equation*}
        \frac12 \sum_{i \in \scbr{0, 1}} \inp{\initial, V_i^{\pi_E^i} - V_i^{\bar\pi^i}} \geq \frac{\WMAX \epsilon'}{24 \spr{1 - \gamma}} \exp \spr{- \tau_E \spr{\epsilon'}^2}\,.
    \end{equation*}
    %
    It remains to make the lower bound small enough. To bound the average suboptimality gap by $\frac{\WMAX \epsilon'}{24 e \spr{1 - \gamma}}$ and have $\frac12 \sum_{i \in \scbr{0, 1}} \inp{\initial, V_i^{\pi_E^i} - V_i^{\bar\pi^i}} \leq \frac{\WMAX \epsilon'}{24 e \spr{1 - \gamma}}$, we need at least $\tau_E \geq \frac{1}{\spr{\epsilon'}^2}$ expert transitions. Therefore, to achieve
    %
    \begin{equation*}
        \frac12 \sum_{i \in \scbr{0, 1}} \inp{\initial, V_i^{\pi_E^i} - V_i^{\bar\pi^i}} \leq \varepsilon\,,
    \end{equation*}
    %
    for some $\varepsilon > 0$, we need to choose $\epsilon' = \nicefrac{24 e \spr{1 - \gamma} \varepsilon}{\WMAX}$, which means that every algorithm needs at least $\tau_E \geq \frac{\WMAX^2}{24^2 e^2 \spr{1 - \gamma}^2 \varepsilon^2}$ to guarantee a suboptimality gap of order $\varepsilon$.
\end{proof}


\clearpage
\section{Technical tools}

%%%%%%%%%%%%%%%%%%%%%%%%%%%%%%%%%%%%%%%%%%%%%%%%%%%%%%%%%%%%%%%%%%%%%%%%%%%%%%%%
\subsection{Reinforcement learning}

\begin{proposition}\label{prop:flow}
  The occupancy measure $\mu(\pi)$ of any policy $\pi$ satisfies the following system of equations:
  %
  \begin{equation}\label{eq:flow}
      E\transpose \mu \spr{\pi} = \gamma P\transpose \mu \spr{\pi} + \spr{1 - \gamma} \nu_0\,.
  \end{equation}
\end{proposition}

\begin{proof}
  Define the transition kernel induced by policy $\pi$ as $P_\pi$, with $P_\pi(\cdot|x) = \mathbb{E}_{A\sim\pi(\cdot|x)}\sbr{P(\cdot|x,A)}$. The proof follows from the following standard calculation:
  \begin{align*}
    E\transpose \mu \spr{\pi} &= (1-\gamma) \sum_{\tau=0}^\infty \pa{\gamma P_\pi\transpose}^\tau \nu_0 \\
    &= (1-\gamma) \sum_{\tau=1}^\infty \pa{\gamma P_\pi\transpose}^\tau \nu_0 + (1-\gamma) \nu_0 \\
    &= \gamma P_\pi (1-\gamma) \sum_{\tau=0}^\infty \pa{\gamma P_\pi\transpose}^\tau \nu_0 + (1-\gamma) \nu_0 \\
    &= \gamma P_\pi E\transpose \mu \spr{\pi} + (1-\gamma) \nu_0 \\
    &= \gamma P \mu \spr{\pi} + (1-\gamma) \nu_0,
  \end{align*}
  %
  where the last step follows from the easily-checked fact that $P \mu \spr{\pi} = P_\pi E\transpose \mu \spr{\pi}$.
\end{proof}


\begin{lemma}\label{lem:from-ev-to-q}
  Let $\pi$ be any policy, $Q \in \bbR^{\cX \times \cA}$ be any function defined on $\cX \times \cA$, and $V \in \bbR^\cX$ be such that for any $x$, $V \spr{x} = \bbE_{A \sim \pi \spr{\cdot \given x}} \sbr{Q \spr{x, A}}$. Then
  %
  \begin{equation*}
    \inp{\mu \spr{\pi}, \opE V} = \inp{\mu \spr{\pi}, Q}\,.
  \end{equation*}
\end{lemma}

\begin{proof}
  We have
  %
  \begin{align*}
    \inp{\mu \spr{\pi}, \opE V} &= \sum_{x \in \cX} \nu \spr{\pi, x} V \spr{x} \\
    &= \sum_{x \in \cX} \sum_{a \in \cA} \nu \spr{\pi, x} \pi \spr{a \given x} Q \spr{x, a} \\
    &= \sum_{x \in \cX} \sum_{a \in \cA} \mu \spr{\pi, x, a} Q \spr{x, a} \\
    &= \inp{\mu \spr{\pi}, Q}\,,
  \end{align*}
  %
  where the second equality follows from the definition of the function $V$ and the first equality from the definition of the state-action occupancy measure.
\end{proof}

\begin{lemma}\label{lem:ineq-kl-entropy}
    Let $\pi$ and $\pi'$ be two policies. Then,
    %
    \begin{equation*}
      \KL \spr{\mu \spr{\pi} \middle\| \mu \spr{\pi'}} \leq \frac{1}{1 - \gamma} \inp{\nu \spr{\pi}, \KL \spr{\pi \| \pi'}}\,.
    \end{equation*}
  \end{lemma}
  
\begin{proof}
  Using the chain rule of the relative entropy, we write
  %
  \begin{equation*}
      \KL \spr{\mu \spr{\pi} \middle\| \mu \spr{\pi'}} = \KL \spr{\nu \spr{\pi} \middle\| \nu \spr{\pi'}} + \inp{\nu \spr{\pi}, \KL \spr{\pi \middle\| \pi'}}\,.
  \end{equation*}
  %
  By the flow constraints and the joint convexity of the relative entropy, we bound the first term as
  %
  \begin{align*}
    \KL \spr{\nu \spr{\pi} \middle\| \nu \spr{\pi'}} &= \KL \spr{\gamma P\transpose \mu \spr{\pi} + \spr{1 - \gamma} \nu_0 \middle\| \gamma P\transpose \mu \spr{\pi'} + \spr{1 - \gamma} \nu_0} \\
    &\leq \spr{1 - \gamma} \KL \spr{\nu_0 \middle\| \nu_0} + \gamma \KL \spr{P\transpose \mu \spr{\pi} \middle\| P\transpose \mu \spr{\pi'}} \\
    &= \gamma \KL \spr{P\transpose \mu \spr{\pi} \middle\| P\transpose \mu \spr{\pi'}} \\
    &\leq \gamma \KL \spr{\mu \spr{\pi} \middle\| \mu \spr{\pi'}}\,,
  \end{align*}
  %
  where we also used the data-processing inequality in the last step. The proof is concluded by reordering the terms.
\end{proof}

\begin{lemma} \label{lem:mass-reduced}
  For any MDP $\cM$, any ascension function $p^\upplus$, and any policy $\pi$, we have for any state-action pair $\spr{x, a} \in \cX \times \cA$,
  %
  \begin{equation*}
    \mu^\upplus \spr{\pi, x, a} \leq \mu \spr{\pi, x, a}\,,
  \end{equation*}
  %
  where $\mu^\upplus \spr{\pi}$ denotes the state-action occupancy of $\pi$ in $\cM^\upplus$, the optimistically augmented MDP induced by $p^\upplus$.
\end{lemma}

\begin{proof}
  Let us consider a process  $\spr{X_\tau, A_\tau}_{\tau \in \bbN}$ generated by the policy $\pi$ in the MDP $\cM$, that is, such that $X_0 \sim \initial$, and for any $\tau \in \bbN$, $A_\tau \sim \pi \spr{\cdot \given X_\tau}$, and $X_{\tau+1} \sim P \spr{\cdot \given X_\tau, A_\tau}$. Additionally, we define a process $\spr{X_\tau^\upplus, A_\tau^\upplus}_{\tau \in \bbN}$ coupled to the process defined above as follows. At the first stage, we set $X_\tau^\upplus = X_0$. Then for any $\tau \geq 1$, the coupled process evolves as
  %
  \begin{equation*}
    X_{\tau+1}^\upplus, A_{\tau+1}^\upplus =
    \begin{cases}
        X_{\tau+1}, A_{\tau+1} \quad &\text{w.p.} \quad 1 - p^\upplus \spr{X_\tau, A_\tau} \quad \text{if} \quad  X_\tau^\upplus, A_\tau^\upplus = X_\tau, A_\tau \\
        x^\upplus, a \quad & \text{w.p.} \quad p^\upplus \spr{X_\tau, A_\tau} \quad \text{if} \quad  X_\tau^\upplus, A_\tau^\upplus = X_\tau, A_\tau \\
        x^\upplus, a \quad & \text{if} \quad X_\tau^\upplus, A_\tau^\upplus \neq X_\tau, A_\tau
    \end{cases}\,.
  \end{equation*}
  %
  It is straightforward to check that this process follows the dynamics of the optimistically augmented MDP $\cM^\upplus(r,p^\upplus)$ (since its transitions obey the kernel $P^\upplus$). By definition, for any state-action pair $\spr{x, a} \in \cX \times \cA$, we have
  %
  \begin{align*}
    \mu^\upplus \spr{\pi, x, a} &= \spr{1 - \gamma} \sum_{\tau=0}^\infty \gamma^\tau \bbP \sbr{X_\tau^\upplus = x, A_\tau^\upplus = a} \\
    &= \spr{1 - \gamma} \sum_{\tau=0}^\infty \gamma^\tau \spr{\bbP \sbr{X_\tau^\upplus = x, A_\tau^\upplus = a, X_\tau^\upplus \neq x^\upplus} + \bbP \sbr{X_\tau^\upplus = x, A_\tau^\upplus = a, X_\tau^\upplus = x^\upplus}} \\
    &= \spr{1 - \gamma} \sum_{\tau=0}^\infty \gamma^\tau \spr{\bbP \sbr{X_\tau = x, A_\tau = a, X_\tau^\upplus \neq x^\upplus} + 0} \\
    &\leq \spr{1 - \gamma} \sum_{\tau=0}^\infty \gamma^\tau \bbP \sbr{X_\tau = x, A_\tau = a} \\
    &= \mu \spr{\pi, x, a}\,.
  \end{align*}
  %
  In the third equality, the second term within the sum is equal to zero because $x \neq x^\upplus$, and in the other term we replaced $\spr{X_\tau^\upplus, A_\tau^\upplus}$ by $\spr{X_\tau, A_\tau}$ because the two coincide long as $X^\upplus_\tau \neq x^+$. This concludes the proof.
\end{proof}

%%%%%%%%%%%%%%%%%%%%%%%%%%%%%%%%%%%%%%%%%%%%%%%%%%%%%%%%%%%%%%%%%%%%%%%%%%%%%%%%
\subsection{Linear algebra and analysis}

\begin{lemma} \label{lemma:number-epochs-bound}
    Under the event $\cE_L$, the number of epochs $E \spr{K}$ in Algorithm~\ref{alg:linear-rmax-ravi-ucb} is bounded as
    %
    \begin{equation*}
      E \spr{K} \leq 5 d \log \spr{1 + \frac{B^2 T}{d}}\,.
    \end{equation*}
    %
    where $T = \LMAX K = \frac{\log \spr{\frac{K}{\delta}} K}{1 - \gamma}$.
\end{lemma}

\begin{proof}
    In the following, we denote $\phi_t = \phi \spr{x_t, a_t}$ for any $t$. The bound on the number of epochs is derived observing that since the determinant of the matrix $\Lambda_k$ can grow at most linearly then the condition is triggered at most a logarithmic number of times. In particular notice that
    %
    \begin{equation*}
        \det \spr{\Lambda_{t_{E \spr{K}}}} \geq 2 \det \spr{\Lambda_{t_{E \spr{K} - 1}}} \geq 2^2 \det \spr{\Lambda_{t_{E \spr{K} - 2}}} \geq 2^{E \spr{K} - 1} \det \spr{I} = 2^{E \spr{K} - 1}\,.
    \end{equation*}
    %
    Hence, it holds that $E \spr{K} - 1 \leq \frac{1}{\log 2} \log \spr{\det \Lambda_{t_{E \spr{K}}}}$. Then, denoting $T_{K+1} = T_K + L_K$ where $L_K$ is the length of episode $K$, we have that
    %
    \begin{align*}
        E \spr{K} &\leq 1 + \frac{1}{\log 2} \log \spr{\det \spr{\Lambda_{T_{K+1}}}} & \spr{\Lambda_{t_{E \spr{K}}} \preceq \Lambda_{T_{K+1}}} \\
        &\leq 1 + \frac{d}{\log 2} \log \brr{\frac{\optrace \spr{\Lambda_{T_{K+1}}}}{d}} & \text{(trace-determinant inequality)}\,.
    \end{align*}
    %
    By definition of the covariance matrix,
    %
    \begin{align*}
        E \spr{K} & \leq 1 + \frac{d}{\log 2} \log \spr{\frac{\optrace \spr{\sum_{t \in \sbr{T_{K+1}}} \phi_t \phi_t\transpose} + d}{d}} \\
        &= 1 + \frac{d}{\log 2} \log \spr{1 + \frac{\sum_{t \in \sbr{T_{K+1}}} \norm{\phi_t}_2^2}{d}}
        \\
        &\leq 1 + \frac{d}{\log 2} \log \spr{1 + \frac{B^2 T}{d}} \\
        &\leq 5 d \log \spr{1 + \frac{B^2 T}{d}}\,,
    \end{align*}
    %
    where the first equality follows from properties of the trace and the second inequality follows from $\norm{\phi_t}_2 \leq B$ and $T_{K+1} \leq T$ which holds under $\cE_L$.
\end{proof}

The following lemma is a generalization of Lemma~19 of \citet{cassel2024warmupfree} for an arbitrary threshold $\omega \geq 0$.
%
\begin{lemma} \label{lem:sigmoid-bound}
  For all $z \geq 0$, $\omega \geq 0$ it holds that $\sigma \spr{z - \omega} \leq 2 \spr{z^2 + \exp \spr{- \omega}}$.
\end{lemma}

\begin{proof}
  Let us consider the function $g: z \mapsto \sigma \spr{z - \omega} - \spr{z + \frac{1}{e^{\omega/2}}}^2$. Note that for any $z$, we have $\sigma' \spr{z} = \sigma \spr{z} \sigma \spr{- z}$. Thus, the first two derivatives of $g$ are given by
  %
  \begin{align*}
    g' \spr{z} &= \sigma \spr{z - \omega} \sigma \spr{\omega - z} - 2 \spr{z + \frac{1}{e^{\omega / 2}}}\,, \\
    g'' \spr{z} &= \sigma \spr{z - \omega} \sigma \spr{\omega - z}^2 - \sigma \spr{z - \omega}^2 \sigma \spr{\omega - z} - 2 \,.
  \end{align*}
  %
  Since $\sigma \spr{z} \in \spr{0, 1}$ for any $z$, the second derivative of $g$ is nonpositive, $g'' \spr{z} \leq 0$, and $g$ is concave. By the first order condition, for any $z \geq 0$,
  %
  \begin{equation*}
    g \spr{z} \leq g \spr{0} + g' \spr{0} z\,.
  \end{equation*}
  %
  Furthermore, note that
  %
  \begin{equation*}
    g \spr{0} = \sigma \spr{- \omega} - \frac{1}{e^\omega} = \frac{1}{1 + e^\omega} - \frac{1}{e^\omega} \leq 0\,,
  \end{equation*}
  %
  and
  %
  \begin{align*}
    g' \spr{0} &= \frac{1}{1 + e^\omega} \frac{1}{1 + e^{- \omega}} - \frac{2}{e^{\omega / 2}} \\
    &\leq \frac{1}{1 + e^\omega} - \frac{2}{e^{\omega / 2}} \\
    &\leq - \frac{1}{e^{\omega / 2}} \\
    &\leq 0\,,
  \end{align*}
  %
  where we first used that $e^{- \omega} \geq 0$ for any $\omega \geq 0$ and then that $x + 1 \geq \sqrt{x}$ for any $x \geq 0$. Thus, it holds that $g \spr{z} \leq 0$ for all $z \geq 0$, \ie $\sigma \spr{z - \omega} \leq \spr{z + \frac{1}{e^{\omega / 2}}}^2$. Using $\spr{a+b}^2 \leq 2 \spr{a^2 + b^2}$, it holds that
  %
  \begin{equation*}
    \sigma \spr{z - \omega} \leq 2 \spr{z^2 + e^{- \omega}}\,.
  \end{equation*}
\end{proof}

We present a variant of Lemma~18 of \citet{cassel2024warmupfree} which is valid for $\omega \geq 2$ instead of $\omega \geq 0$, but is sharper by a factor of $2$.
%
\begin{lemma} \label{lem:sigmoid-bound2}
  For all $\omega \geq 2$, it holds that
  \begin{equation*}
    \max_{z \geq 0} z \cdot \sigma \spr{\omega - \alpha z} \leq \frac{\omega}{\alpha}\,.
  \end{equation*}
\end{lemma}

\begin{proof}
  Let $\alpha > 0$, $\omega \geq 2$, and $g: z \geq 0 \mapsto z \cdot \sigma \spr{\omega - \alpha z}$. We recall that the derivative of the sigmoid function is given for any $z$ by $\sigma' \spr{z} = \sigma \spr{z} \sigma \spr{- z}$, and that $\sigma \spr{- z} = 1 - \sigma \spr{z}$. $g$ is twice differentiable. Its first derivative is given by
  %
  \begin{align*}
    g' \spr{z} &= \sigma \spr{\omega - \alpha z} - \alpha z \sigma \spr{\omega - \alpha z} \sbr{1 - \sigma \spr{\omega - \alpha z}} \\
    &= \sigma \spr{\omega - \alpha z} \sbr{1 - \alpha z \spr{1 - \sigma \spr{\omega - \alpha z}}}\,.
  \end{align*}
  %
  We set the derivative to zero and solve the equation to find the critical points. We have
  %
  \begin{align}
    g' \spr{z} = 0 &\quad\text{iff}\quad \alpha z = \frac{1}{1 - \sigma \spr{\omega - \alpha z}} \label{eq:critical-point-prop} \\
    &\quad\text{iff}\quad \alpha z = 1 + e^{\omega - \alpha z} \nonumber \\
    &\quad\text{iff}\quad \spr{\alpha z - 1} e^{\alpha z - 1} = e^{\omega - 1}\,. \nonumber
  \end{align}
  %
  For $x > 0$, the equation $w e^w = x$ has exactly one positive solution $w = W \spr{x}$ which increases with $x$ and where $W$ denotes the Lambert function. Thus, $g' \spr{z} = 0$ if and only if $\alpha z - 1 = W \spr{e^{\omega - 1}}$, \ie $z^\star = \frac{W \spr{e^{\omega - 1}} + 1}{\alpha}$. We check that $z^\star$ is a local maximum. The second derivative of $g$ is given by
  %
  \begin{align*}
    g'' \spr{z} &= - 2 \alpha \sigma \spr{\omega - \alpha z} \sbr{1 - \sigma \spr{\omega - \alpha z}} \\
    &\phantom{=}+ \alpha^2 z \sigma \spr{\omega - \alpha z} \sbr{1 - \sigma \spr{\omega - \alpha z}}^2 \\
    &\phantom{=}- \alpha^2 z \sigma \spr{\omega - \alpha z}^2 \sbr{1 - \sigma \spr{\omega - \alpha z}} \\
    &= - 2 \alpha \sigma \spr{\omega - \alpha z} \sbr{1 - \sigma \spr{\omega - \alpha z}} \\
    &\phantom{=}+ \alpha^2 z \sigma \spr{\omega - \alpha z} \sbr{1 - \sigma \spr{\omega - \alpha z}} \sbr{1 - 2 \sigma \spr{\omega - \alpha z}}\,.
  \end{align*}
  %
  We evaluate it at the critical point $z^\star$ and simplify the expression using Equation~\ref{eq:critical-point-prop}
  %
  \begin{align*}
    g'' \spr{z^\star} &= - 2 \alpha \sigma \spr{\omega - \alpha z^\star} \sbr{1 - \sigma \spr{\omega - \alpha z^\star}} \\
    &\phantom{=}+ \alpha \sigma \spr{\omega - \alpha z^\star} \sbr{1 - 2 \sigma \spr{\omega - \alpha z^\star}} \\
    &= - \alpha \sigma \spr{\omega - \alpha z^\star} \\
    &< 0\,,
  \end{align*}
  %
  thus $z^\star > 0$ is a local maximum. Since $g \spr{0} = 0$, $\lim_{z \rightarrow + \infty} g \spr{z} = 0$, $g \spr{z^\star}$ and $z^\star$ is the only positive critical point, this means $z^\star$ is a global maximum. We evaluate $g$ to get the maximum
  %
  \begin{align*}
    g \spr{z^\star} &= \frac{W \spr{e^{\omega - 1}} + 1}{\alpha} \frac{1}{1 + \exp \spr{W \spr{e^{\omega - 1}}} e^{1 - \omega}} \\
    &= \frac{W \spr{e^{\omega - 1}} + 1}{\alpha} \frac{W \spr{e^{\omega - 1}}}{W \spr{e^{\omega - 1}} + W \spr{e^{\omega - 1}} \exp \spr{W \spr{e^{\omega - 1}}} e^{1 - \omega}} \\
    &= \frac{W \spr{e^{\omega - 1}}}{\alpha}\,,
  \end{align*}
  %
  where we used $W \spr{e^{\omega - 1}} \exp \spr{W \spr{e^{\omega - 1}}} = e^{\omega - 1}$ in the third equality. We now upper bound the Lambert function. Taking the log of the equation that defines it, we have $W \spr{x} = \log x - \log W \spr{x}$ for any $x > 0$. Note that $W \spr{e} = 1$ and that $W$ is increasing, so for any $x > e$, we have $W \spr{x} > 1$ and thus $W \spr{x} < \log x$. Using it on $g \spr{z^\star}$, we further have
  %
  \begin{equation*}
    g \spr{z^\star} \leq \frac{\omega - 1}{\alpha} \leq \frac{\omega}{\alpha}\,,
  \end{equation*}
  %
  where we used $\omega \geq 2$. This concludes the proof.
\end{proof}


\begin{lemma} \label{lem:lse-lipschitz}
  Let $n \in \bbR^n$, and define $\LSE: \bbR^n \rightarrow \bbR$ the function defined for any $x \in \bbR^n$ as
  %
  \begin{equation*}
    \LSE \spr{x} = \log \sum_{i=1}^n e^{x_i}\,.
  \end{equation*}
  %
  Then $\LSE$ is 1-Lipschitz with respect to the norm $\norm{\cdot}_\infty$, \ie for any $x, y \in \bbR^n$,
  %
  \begin{equation*}
    \abs{\LSE \spr{x} - \LSE \spr{y}} \leq \norm{x - y}_\infty\,.
  \end{equation*}
\end{lemma}

\begin{proof}
  For any $i \in \sbr{n}$ and any $x \in \bbR^n$, the gradient of $\LSE$ is given by
  %
  \begin{equation*}
    \nabla \LSE \spr{x} = \frac{e^{x}}{\inp{e^{x}, \bfone}}\,.
  \end{equation*}
  %
  Let $y \in \bbR^n$. By the intermediate mean value theorem, there exists a $z$ on the segment $\sbr{x, y}$ such that
  %
  \begin{align*}
    \abs{\LSE \spr{x} - \LSE \spr{y}} &= \abs{\inp{\nabla \LSE \spr{z}, x - y}} \\
    &\leq \norm{\nabla \LSE \spr{z}}_1 \norm{x - y}_\infty \\
    &= \norm{x - y}_\infty\,,
  \end{align*}
  %
  where the inequality follows from Hölder's inequality.
\end{proof}


\begin{lemma}[\citealp{cohen2019learning}, Lemma~27] \label{lem:det-elliptical-bound}
  If $0 \prec M \preceq N$ then for any vector $v$,
  %
  \begin{equation*}
    \norm{v}_N^2 \leq \frac{\det N}{\det M} \norm{v}_M^2\,.
  \end{equation*}
\end{lemma}


\begin{lemma}[\citealp{sherman2023}, Lemma~15] \label{lemma:beta_bound}
    Let $R, z \geq 1$, then $\beta \geq 2 z \log \spr{R z}$ ensures $\beta \geq z \log \spr{R \beta}$.
\end{lemma}

\begin{lemma}[\citealp{rosenberg2020near}, Lemma~D.4] \label{lem:concentration-ineq-cond-exp}
  Let $\bc{X_k}_{k \in [K]}$ be a sequence of random variables adapted to the filtration $\bc{\mathcal{F}_k}_{k\in[K]}$ and suppose that $0 \leq X_k \leq X_{\max}$ almost surely. Then, with probability at least $1-\delta$, the following holds for all $k \geq 1$ simultaneously
  \begin{equation*}
    \sum^K_{k=1} \mathbb{E}\bs{X_k | \mathcal{F}_{k-1}} \leq 2 \sum^K_{k=1} X_k + 4 X_{\max} \log \frac{2 K}{\delta}\,.
  \end{equation*}
\end{lemma}

\begin{lemma}[\citealp{jin2019provably}, Lemma~D.2] \label{lem:bound-elliptical-potential}
  Let $\scbr{\phi_t}_{t \geq 0}$ be a bounded sequence in $\bbR^d$ satisfying $\sup_{t \geq 0} \norm{\phi_t} \leq 1$. Let $\Lambda_0 \in \bbR^{d \times d}$ be a positive definite matrix. For any $t \geq 0$, we define $\Lambda_t = \Lambda_0 + \sum_{j=1}^t \phi_j \phi_j\transpose$. Then, if the smallest eigenvalue of $\Lambda_0$ satisfies $\lambda_{\mathrm{min}} \spr{\Lambda_0} \geq 1$, we have
  \begin{equation*}
    \sum_{j=1}^t \phi_j \Lambda_{j-1}^{-1} \phi_j \leq 2 \log \spr{\frac{\det \Lambda_t}{\det \Lambda_0}}\,.
  \end{equation*}
\end{lemma}

% \crefalias{section}{appendix} % uncomment if you are using cleveref

\end{document}
