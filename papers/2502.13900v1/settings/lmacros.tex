%\usepackage[unicode,psdextra]{hyperref}
\usepackage[capitalize,noabbrev]{cleveref}
% Parentheses
\newcommand{\bc}[1]{\left\{{#1}\right\}}
\newcommand{\brr}[1]{\left({#1}\right)}
\newcommand{\bs}[1]{\left[{#1}\right]}
\newcommand{\ceil}[1]{\left\lceil #1 \right\rceil}
\newcommand{\floor}[1]{\left\lfloor #1 \right\rfloor}
\newcommand{\bsd}[1]{\left\llbracket{#1}\right\rrbracket}
\newcommand{\ip}[2]{\left\langle{#1},{#2}\right\rangle}
\newcommand{\inner}[1]{\langle#1\rangle}
% Statistics notation
% Probabilities
\renewcommand{\P}[1]{\mathbb{P}\bs{{#1}}}
\newcommand{\Pp}[2]{\underset{#1}{\mathbb{P}}\bs{{#2}}}
\newcommand{\Ppp}[3]{\mathbb{P}_{#1}^{#2}\bs{{#3}}}
% Expectations
\newcommand{\Ee}[2]{\underset{#1}{\mathbb{E}}\bs{{#2}}}
\newcommand{\Eee}[2]{\mathbb{E}_{#1}\bs{{#2}}}
\newcommand{\sspace}{\mcf{X}}     % state space
\newcommand{\aspace}{\mcf{A}}     % action space
\newcommand{\wspace}{\mcf{W}}     % cost weights space
\newcommand{\histspace}{\mcf{H}}  % history space
% argmin and argmax
\DeclareMathOperator*{\argmin}{arg\,min}
\DeclareMathOperator*{\argmax}{arg\,max}

\usepackage{algorithm}
\usepackage{algorithmic}
\usepackage{multirow}
\newcounter{protocol}
\makeatletter
\newenvironment{protocol}[1][htb]{%
  \let\c@algorithm\c@protocol
  \renewcommand{\ALG@name}{Protocol}% Update algorithm name
  \begin{algorithm}[#1]%
  }{\end{algorithm}
}
\usepackage{amssymb}
\usepackage{amsfonts}
\usepackage{graphics}
\usepackage{graphicx}
\usepackage{setspace}
%\usepackage{subcaption}
%\usepackage{selectp}
%\outputonly{1-18}
% \newtheorem{theorem}{\protect\theoremname}
%   \newtheorem{lemma}{\protect\lemmaname}
%   \newtheorem{corollary}{\protect\corrolaryname}
%   \newtheorem{defn}{\protect\definitionname}
%   \newtheorem{proposition}{\protect\propositionname}
%   \newtheorem{example}{\protect\examplename}
%   \newtheorem{remark}{Remark}
%   \newtheorem{conditions}{\protect\conditionsname}
%     \newtheorem{assumption}{\protect\assumptionname}
    
%     \newenvironment{proofof}[1]{\begin{proof}[{#1}]}{\end{proof}}
    
% \providecommand{\definitionname}{Definition}
% \providecommand{\examplename}{Example}
% \providecommand{\lemmaname}{Lemma}
% \providecommand{\corrolaryname}{Corollary}
% \providecommand{\propositionname}{Proposition}
% \providecommand{\conditionsname}{Conditions}
% \providecommand{\theoremname}{Theorem}
% \providecommand{\assumptionname}{Assumption}

\def \M{\mathcal{M}}
\def \S{\mathcal{S}}


% to compile a preprint version, e.g., for submission to arXiv, add add the
% [preprint] option:
%     \usepackage[preprint]{neurips_2022}


% to compile a camera-ready version, add the [final] option, e.g.:
%     \usepackage[final]{neurips_2022}


% to avoid loading the natbib package, add option nonatbib:
%    \usepackage[nonatbib]{neurips_2022}

%commands from AK
%====================================================================================================================================================================================
%new
%state space S 
\newcommand{\qv}{\mbf{q}}
\newcommand{\dv}{\mbf{d}}
\newcommand{\dve}{\mbf{d}^E}
\newcommand{\trans}{\intercal}
\newcommand{\qval}{\mbf{Q}}
\newcommand{\FEV}[1]{\mbs{\rho}_{\phim}(#1)}
\newcommand{\EFEV}[1]{\mbs{\rho}_{\phim}(\widehat{#1})}
%==============================================================================
%old
\newcommand{\mcf}{\mathcal}
\newcommand{\mbf}{\mathbf}
\newcommand{\mbb}{\mathbb}

\newcommand{\innerprod}[2]{\left\langle{#1},{#2}\right\rangle}

\newcommand{\expert}{{\pi_E}}
\newcommand{\experti}{\pi_E^i}
\newcommand{\apprentice}{\pi_{\textup{A}}}
\newcommand{\mbs}{\boldsymbol}
\newcommand{\cost}{r}
\newcommand{\true}{r_{\textup{true}}}
\newcommand{\weight}{w}
\newcommand{\FRAalg}{\texttt{\textbf{FRA-IL}}\xspace}
\newcommand{\phim}{\Phi}
\newcommand{\initial}{\nu_0}


\newcommand{\wtrue}{\weight_{\textup{true}}}
 %========================================================
\newcommand{\lv}[1]{%
    \ifmmode
    \text{\textcolor{orange}{[LV: #1]}}
    \else
    \textcolor{orange}{[LV: #1]}
    \fi
}
\usepackage[utf8]{inputenc} % allow utf-8 input
\usepackage[T1]{fontenc}    % use 8-bit T1 fonts      
\usepackage{url}            % simple URL typesetting
\usepackage{booktabs}       % professional-quality tables
\usepackage{amsfonts}       % blackboard math symbols
\usepackage{nicefrac}       % compact symbols for 1/2, etc.
\usepackage{microtype}      % microtypography
\usepackage{xcolor}         % colors
\usepackage{dsfont}
% \usepackage{xcolor}
\usepackage{pifont}
\newcommand{\greentick}{\textcolor{green}{\ding{51}}}
\newcommand{\redcross}{\textcolor{red}{\ding{55}}}
\newtheorem{assumption}{\protect\assumptionname}
\providecommand{\assumptionname}{Assumption}

\usepackage{mathtools}% http://ctan.org/pkg/mathtools
\makeatletter
\newcommand{\mytag}[2]{%
  \text{#1}%
  \@bsphack
  \protected@write\@auxout{}%
         {\string\newlabel{#2}{{#1}{\thepage}}}%
  \@esphack
}
\makeatother

\usepackage{thm-restate}
\newtheorem{Lem}[theorem]{Lemma}

