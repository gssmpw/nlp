\section{Preliminary}


\subsection{Problem Formulation}
\label{sec:problem_formuation}




Our formulation adheres to the conventional ambiguous question answering setting, where the primary objective is to handle ambiguity in user queries effectively. Specifically, given an ambiguous question \( q \), the task of diversification involves identifying a comprehensive set of possible interpretations \( \mathcal{Q} = \{q_1, q_2, \cdots, q_N\} \), each representing a distinct and valid meaning of the original question $q$. 

The goal then extends to determining corresponding answers \( \mathcal{Y} = \{ y_1, y_2, \cdots, y_N \} \) for each interpretation \( q_i \), ensuring the response is grounded on retrieved supporting evidence.
The established evaluation protocol for such task involves comparing the model-generated predictions \( \hat{\mathcal{Q}} \) and \( \hat{\mathcal{Y}} \) against the ground-truth set of interpretations and answers, \( \mathcal{Q} \) and \( \mathcal{Y} \). 



\subsection{Baselines: DtV}
\label{subsec:diva}

In this section, we formally describe 
the DtV workflow, using the most recent work \citet{in-etal-2024-diversify-arxiv} with state-of-the-art performance as reference.
DtV first \textbf{diversifies} the query, or, identifies pseudo-interpretations \( \hat{\mathcal{Q}} = \{\hat{q}_1, \hat{q}_2, \dots, \hat{q}_M\} \) that disambiguate the meaning of the original ambiguous query, by prompting LLM with instructions $I_{\textrm{P}}$ without accessing any retrieved knowledge:
\begin{equation}
\hat{\mathcal{Q}} \leftarrow \texttt{LLM}(q 
; I_\text{P}).
\label{eq:diva_q_extract}
\end{equation}
Each generated pseudo-interpretation $\hat{q}_i$ in $\hat{\mathcal{Q}}$ are then used as a search query to retrieve top-$k$ supporting
passages, the union of which
forms a universe $U$
of relevant documents
\begin{equation}
U_{\hat{\mathcal{Q}}} \leftarrow \bigcup_{\hat{q}\in\hat{\mathcal{Q}}}\, \argtopk_{p} \mathrm{sim} \left( \hat{q}, p \right),
\label{eq:diva_form_universe}
\end{equation}
where the subscript $\hat{\mathcal{Q}}$ indicates that the universe $U_{\hat{\mathcal{Q}}}$ is derived from the set of pseudo-interpretations $\hat{\mathcal{Q}}$.
As pseudo-interpretations can be ungrounded and the resulting $U_{\hat{\mathcal{Q}}}$ can be noisy,
\textbf{verification} phase follows, to examine each pseudo-interpretation 
with universe $U_{\hat{\mathcal{Q}}}$.
After this phase,
$U_{\hat{\mathcal{Q}}}$ is reduced to a verified subset 
$U_V$,
from which disambiguated queries are answered.
\begin{equation}
U_V
\leftarrow \left\{
p\in U_{\hat{\mathcal{Q}}}
\,\middle|\, 
\exists_{\hat{q} \in \hat{\mathcal{Q}}}\,
\texttt{Verify}(\hat{q}, p)=1 \right\},
\end{equation}
\begin{equation}
\hat{\mathcal{Q}}^\prime, \hat{\mathcal{Y}} \leftarrow 
\texttt{LLM}(q, U_{V}; I_{\textrm{G}}).
\label{eq:diva_answer_gen}
\end{equation}

We identify the challenges and inefficiencies in DtV line of work as follows:
\begin{enumerate}


\item $U_{\hat{\mathcal{Q}}}$:
Each interpretation $\hat{\mathcal{Q}}$ incurs at least one retriever call, some of which are ungrounded and could introduce noises.

\item $U_{V}$:
Verified $U_V$ requires to process 
all $(\hat{q},p)$ pairs, for example, inducing $|\hat{\mathcal{Q}}|$ \texttt{Verify} calls with input size of $\mathcal{O}(|U_{\hat{\mathcal{Q}}}|)$ each.\footnote{\emph{Pruning} $U_{\hat{\mathcal{Q}}}$ in advance
 was used in~\citet{in-etal-2024-diversify-arxiv}, though asymptotic cost remains unchanged. We discuss this in Section~\ref{subsec:clustering}.}
Plus, such long-context verification requires a powerful LLM model,
which increases cost and hinders applicability.


\end{enumerate}









