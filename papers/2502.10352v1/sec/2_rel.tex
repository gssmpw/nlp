\section{Related Work}

This section overviews existing work on DtV workflow and the line of research on using retriever as environment. 

\paragraph{Diversification.}
The goal of this phase is to turn the given query into a diverse set of interpretations.
\citet{min-etal-2021-joint} and \citet{sun-etal-2023-answering} studied an iterative approach by identifying one new intent at a time.
\citet{gao-etal-2021-answering} generates clarification questions conditioned on generated answers from ambiguous questions, and then answers the clarification questions again.



With LLMs,
few-shot in-context learning
was used to generate clarifications or query rewrites directly, relying solely on their internal knowledge captured during pretraining~\citep{kim-etal-2023-tree, ma-etal-2023-query}.

\paragraph{Verification.}
Errors in retrieval or intent inference may generate question-passage pair that cannot provide the expected answer.
Verification aims at pruning such pairs.
\citet{shao-huang-2022-answering} train a verifier to choose correct answers from a list of candidates drafted from each passage without question clarifications.
More recent works such as 
Self-RAG~\citep{asai-etal-2024-self-iclr} and Corrective RAG~\citep{yan-etal-2024-corrective} also train a verifier to decide whether the retrieved passages are relevant enough to assist answer generation.
To avoid verifier training, LLMs may leverage its parametric knowledge instead to verify~\citep{li-etal-2024-llatrieval}.

\paragraph{Relevance Feedback.}
In information retrieval, 
imperfect user queries are often modified, guided by treating
initial retrieval result~\citep{rocchio71relevance}
as pseudo relevance feedback (PRF)~\citep{efthimiadis-biron-1993-ucla-trec,evans-lefferts-1993-design-trec,buckley-etal-1994-automatic-trec}.
In our problem context, retrieval-augmented clarification in RAC~\citep{kim-etal-2023-tree} and our method, can be interpreted as leveraging PRF to extract diversified interpretations;
we provide more detailed comparison in Section~\ref{subsubsec:baselines}.


\paragraph{Our Distinction.} 

Unlike DtV that considers verification as a post-hoc step,
we jointly pursue diversification and verification.
Unlike RAC leveraging relevance feedback only,
we jointly verify with execution feedback,
followed by consolidation of feedback for further denoising.


