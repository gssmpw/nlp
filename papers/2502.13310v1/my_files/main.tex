% This must be in the first 5 lines to tell arXiv to use pdfLaTeX, which is strongly recommended.
\pdfoutput=1
% In particular, the hyperref package requires pdfLaTeX in order to break URLs across lines.

\documentclass[11pt]{article}

% Change "review" to "final" to generate the final (sometimes called camera-ready) version.
% Change to "preprint" to generate a non-anonymous version with page numbers.
\usepackage[preprint]{acl}

% Standard package includes
\usepackage{times}
\usepackage{latexsym}
\usepackage{balance}
% For proper rendering and hyphenation of words containing Latin characters (including in bib files)
\usepackage[T1]{fontenc}
% For Vietnamese characters
% \usepackage[T5]{fontenc}
% See https://www.latex-project.org/help/documentation/encguide.pdf for other character sets

% This assumes your files are encoded as UTF8
\usepackage[utf8]{inputenc}

% This is not strictly necessary, and may be commented out,
% but it will improve the layout of the manuscript,
% and will typically save some space.
\usepackage{microtype}

% This is also not strictly necessary, and may be commented out.
% However, it will improve the aesthetics of text in
% the typewriter font.
\usepackage{inconsolata}

%Including images in your LaTeX document requires adding
%additional package(s)
\usepackage{graphicx}
\usepackage{adjustbox}
\usepackage{listings}
\usepackage[dvipsnames]{xcolor}
\usepackage{amsmath}
\usepackage{multirow, multicol}
\usepackage{array}
\usepackage{tabularray}
\usepackage{makecell}
\usepackage{bm}

\definecolor{myyellow}{HTML}{eeba0b}


\lstset{
    language=Python,
    basicstyle=\ttfamily,
    keywordstyle=\color{blue},
    stringstyle=\color{orange},
    commentstyle=\color{green!70!black},
    morecomment=[l][\color{magenta}]{\#},
    tabsize=4,
    showspaces=false,
    showstringspaces=false
}

\newcommand{\mymodel}{$\mathsf{NL}\mbox{-}\mathsf{ToD}$}
\newcommand{\opt}{OPT-6.7B}
\newcommand{\flan}{Flan-T5 Large}
\newcommand{\gpt}{GPT-2 Medium}
\newcommand{\gptt}{GPT-3.5-turbo}
\newcommand{\gptf}{GPT-4o}
\newcommand{\wa}{ ++}
\newcommand{\apicall}{\mathsf{API Call}}
% baselines
\newcommand{\simpletod}{SimpleTOD}
\newcommand{\autotod}{$\mathsf{Auto}\mbox{-}\mathsf{ToD}$}
\newcommand{\zstod}{$\mathsf{ZS}\mbox{-}\mathsf{ToD}$}
\newcommand{\soloist}{SOLOIST}

% If the title and author information does not fit in the area allocated, uncomment the following
%
%\setlength\titlebox{<dim>}
%
% and set <dim> to something 5cm or larger.

%\title{Natural Language Task-Oriented Dialog System}

%\title{Zero-Shot Generalizable End-to-End Task-Oriented Dialog System using Unannotated Dialog Data and Domain Schema}

\title{Do We Still Need Turn-level Dialog Annotations For Training Task-Oriented Dialog Systems?}
%Training Zero-Shot Generalizable End-to-End Task-Oriented Dialog System Without Turn-level Dialog Annotations

% Author information can be set in various styles:
% For several authors from the same institution:
% \author{Author 1 \and ... \and Author n \\
%         Address line \\ ... \\ Address line}
% if the names do not fit well on one line use
%         Author 1 \\ {\bf Author 2} \\ ... \\ {\bf Author n} \\
% For authors from different institutions:
% \author{Author 1 \\ Address line \\  ... \\ Address line
%         \And  ... \And
%         Author n \\ Address line \\ ... \\ Address line}
% To start a separate ``row'' of authors use \AND, as in
% \author{Author 1 \\ Address line \\  ... \\ Address line
%         \AND
%         Author 2 \\ Address line \\ ... \\ Address line \And
%         Author 3 \\ Address line \\ ... \\ Address line}

\author{Adib Mosharrof \\
  University of Kentucky \\
  Lexington, KY, USA \\
  \texttt{adib.mosharrof@uky.edu} \\\And
  A.B. Siddique \\
  University of Kentucky \\
  Lexington, KY, USA \\
  \texttt{siddique@cs.uky.edu} \\}


\newcommand{\oursys}{$\mathsf{NL}\mbox{-}\mathsf{TOD}$}

\newcolumntype{L}{!{\vrule width 1.1pt}c}
\newcolumntype{R}{c!{\vrule width 1.1pt}}
\newcommand{\thickhline}{\noalign{\hrule height 1.1pt}}

\begin{document}
\maketitle
\begin{abstract}
    \begin{abstract}
Retrieval-Augmented Generation (RAG) is often used with Large Language Models (LLMs) to infuse domain knowledge or user-specific information. In RAG, given a user query, a retriever extracts chunks of relevant text from a knowledge base. These chunks are sent to an LLM as part of the input prompt. Typically, any given chunk is repeatedly retrieved across user questions. However, currently, for every question, attention-layers in LLMs fully compute the key values (KVs) repeatedly for the input chunks, as state-of-the-art methods cannot reuse KV-caches when chunks appear at arbitrary locations with arbitrary contexts. Naive reuse leads to output quality degradation.  This leads to potentially redundant computations on expensive GPUs and increases latency. In this work, we propose \sys, a system for managing and reusing precomputed KVs corresponding to the text chunks (we call \textit{chunk-caches}) in RAG-based systems. We present how to identify \hl{\textit{chunk-caches} that are reusable}, how to efficiently perform a small fraction of recomputation to \textit{fix} the cache to maintain output quality, and how to efficiently store and evict \textit{chunk-caches} in the hardware for maximizing reuse while masking any overheads. With real production workloads as well as synthetic datasets, we show that \sys reduces redundant computation by \textbf{51\%} over SOTA prefix-caching and \textbf{75\%} over full recomputation.
\hl{Additionally, with continuous batching on a real production workload, we get a \textbf{1.6$\times$} speedup in throughput and a \textbf{2$\times$} reduction in end-to-end response latency over prefix-caching while maintaining quality, for both the \llama-3-8B and \llama-3-70B models. 
}
\end{abstract}





\end{abstract}

\section{Introduction}
\label{sec:intro}

\begin{figure*}[tb]
    \centering
    \includegraphics[width=0.848\linewidth]{figs/circuitnn.pdf} 
    \caption{Illustration of differentiable CircuitNN. CircuitNN is designed based on differentiable NAND gates. After DAS is guided by PI and PO pairs of the truth table, CircuitNN can get the precise circuit architecture logic equivalent to the truth table.}
    \label{fig:circuitnn}
\end{figure*}

% 1. Describe the importance of logic synthesis
% 2. Existing Problems
% (a) Neural Architecture Search: Unstable, Predefined Setting, etc.
% (b) Circuit Generation: Probabilistic Model, Logic Equivalence

With the rapid advancement of technology, the scale of integrated circuits (ICs) has expanded exponentially. 
This expansion has introduced significant challenges in chip manufacturing, particularly concerning power and area metrics.
A primary objective in IC design is achieving the same circuit function with fewer transistors, thereby reducing power usage and area occupancy.

Logic synthesis~\cite{hachtel2005logicsynth}, a critical step in electronic design automation (EDA), transforms behavioral-level circuit designs into optimized gate-level circuits, ultimately yielding the final IC layout. 
The primary goal of logic synthesis is to identify the physical implementation with the fewest gates for a given circuit function. 
This task constitutes a challenging NP-hard combinatorial optimization problem. 
Current logic synthesis tools~\cite{brayton2010abc, wolf2013yosys} rely on human-designed heuristics, often leading to sub-optimal outcomes.

Differentiable architecture search (DAS) techniques~\cite{liu2018darts, chu2020darts} offer novel perspectives on addressing challenges in this problem.
Circuit functions can be represented through truth tables, which map binary inputs to their corresponding outputs. 
Truth tables provide a precise representation of input-output relationships, ensuring the design of functionally equivalent circuits.
Inspired by this, researchers~\cite{deepmind2024ai4sys, wang2024tnet} have begun exploring the application of DAS to synthesize circuits directly from truth tables.
Specifically, \citet{deepmind2024ai4sys} proposed CircuitNN, a framework that learns differentiable connection structures with logic gates, enabling the automatic generation of logic circuits from truth tables.
This approach significantly reduces the complexity of traditional circuit generation. 
Building on this, \citet{wang2024tnet} introduced T-Net, a triangle-shaped variant of CircuitNN, incorporating regularization techniques to enhance the efficiency of DAS.

Despite these advancements, several challenges remain. 
The computational complexity of DAS grows quadratically with the number of gates, posing scalability issues.
Although triangle-shaped architecture~\cite{wang2024tnet} partially mitigates this problem, redundancy persists. 
%Additionally, DAS is susceptible to converging to local optima, limiting the ability to search architectures that satisfy the given truth tables~\cite{liu2018darts}. 
%Furthermore, hyperparameters (network depth and layer width) require extensive searches, introducing complexity and prolonging the synthesis process. 
Additionally, DAS is susceptible to converging to local optima~\cite{liu2018darts} and hyperparameters (network depth and layer width) require extensive searches. 
The challenges arise from the vast search space in DAS. 
% Even with predefined settings for CircuitNN, finding a configuration that meets the truth table requires extensive trial and error during the DAS process. 
Intuitively, limiting the search space through predefined parameters (network depth, gates per layer, and connection probabilities) can significantly reduce the complexity.

Recent advances~\cite{openai2023gpt4, abramson2024alphafold3, esser2024sd3, li2024mar} in conditional generative models have demonstrated remarkable performance across language, vision, and graph generation tasks. 
Motivated by these developments, we propose a novel approach to circuit generation that generates preliminary circuit structures to guide DAS in generating refined circuits matching specified truth tables. 
Firstly, we introduce CircuitVQ, a tokenizer with a discrete codebook for circuit tokenization. 
Built upon our Circuit AutoEncoder framework~\cite{hou2022graphmae,li2023maskgae,wu2025mgvga}, CircuitVQ is trained through a circuit reconstruction task. 
Specifically, the CircuitVQ encoder encodes input circuits into discrete tokens using a learnable codebook, while the decoder reconstructs the circuit adjacency matrix based on these tokens.
Subsequently, the CircuitVQ encoder serves as a circuit tokenizer for CircuitAR pretraining, which employs a masked autoregressive modeling paradigm~\cite{chang2022maskgit, li2023mage}. 
In this process, the discrete codes function as supervision signals. 
After training, CircuitAR can generate discrete tokens progressively, which can be decoded into initial circuit structures by the decoder of the CircuitVQ. 
These prior insights can guide DAS in producing refined circuits that match the target truth tables precisely.

Our key contributions can be summarized as follows:
\begin{itemize}
\item We introduce CircuitVQ, a circuit tokenizer that facilitates graph autoregressive modeling for circuit generation, based on our Circuit AutoEncoder framework;
\item Develop CircuitAR, a model trained using masked autoregressive modeling, which generates initial circuit structures conditioned on given truth tables;
\item Propose a refinement framework that integrates differentiable architecture search to produce functionally equivalent circuits guided by target truth tables;
\item Comprehensive experiments demonstrating the scalability and capability emergence of our CircuitAR and the superior performance of the proposed circuit generation approach.
\end{itemize}

% Motivation
% (a) Diffusion (Vision, Graph), Autoregressive (Language, Vision)
% (b) Circuit Generation for Predefined Setting
% (c) Neural Architecture Search for Strict Logic Equivalence

% Contribution
% (a) Circuit Tokenizer (new transformer arch, training strategy)
% (b) CircuitAR (train and gen strategies, post-ar strategy)
% (c) Extensive Evaluation including BitD (Bit Distance) for Scalability



\subsection{Plasticity in Neural Networks}
In recent years, various methods have been proposed to address plasticity loss.
Several works have focused on maintaining active units \cite{abbas2023loss, elsayed2024addressing} or re-initializing dead units \cite{sokar2023dormant, dohare2024loss}.
Other studies have explored limiting deviations from the initial statistics of model parameters \cite{kumar2023maintaining, lewandowski2023curvature, elsayed2024weight}.
Additionally, some methods rely on architectural modifications \cite{nikishin2024deep, lee2024slow, lewandowski2024plastic}.  
Plasticity loss also occurs in the reinforcement learning due to its inherent non-stationary. \citet{nikishin2022primacy} proposed resetting the model, while \citet{asadi2024resetting} suggested resetting the optimizer state. 

As noted by \citet{berariu2021study}, loss of plasticity can be divided into two distinct aspects: a decreased ability of networks to minimize training loss on new data (trainability) and a decreased ability to generalize to unseen data (generalizability).
While most previous works focused on trainability, \citet{lee2024slow} addressed generalizability loss.
They demonstrated that plasticity loss also occurs under a stationary distribution, as in a warm-start learning scenario where the model is pretrained on a subset of the training data and then fine-tuned on the full dataset.

Most existing studies have focused on only one of the following challenges: trainability, generalizability, or reinforcement learning.
However, in this study, we validate our AID method across all three aspects, demonstrating its effectiveness in each scenario.



\subsection{Activation Function}
Our AID method is a stochastic approach similar to Dropout while also functioning as an activation function.
Therefore, we aim to discuss previously proposed probabilistic activation functions.
Although the field of probabilistic activation functions has not seen extensive research, two noteworthy studies exist.
The first is the Randomized ReLU (RReLU) function, introduced in the Kaggle NDSB Competition \cite{xu2015empirical}.
The original ReLU function maps all negative values to zero, whereas RReLU maps negative values linearly based on a random slope.
During testing, negative values are mapped using the mean of the slope distribution.
Their experimental results suggest that RReLU effectively prevents overfitting.
Another example of a probabilistic activation function is DropReLU \cite{liang2021drop}.
DropReLU randomly determines whether a node's activation is processed through a ReLU function or a linear function.
The authors claim that DropReLU improves the generalization performance of neural networks.
The fundamental distinction between these probabilistic activation functions and our method lies in the generality of our approach.
Unlike simple probabilistic activation functions, our method encompasses techniques such as Dropout and ReLU, providing a more comprehensive framework.

Another related approach involves activation functions designed to address plasticity loss.
\citep{abbas2023loss} proposed the Concatenated Rectified Linear Units (CReLU), which concatenates the outputs of the standard ReLU applied to the input and its negation.
This structure prevents the occurrence of dead units, thereby improving plasticity.
Additionally, trainable activation functions have also been shown to effectively mitigate plasticity loss in reinforcement learning \citep{delfosseadaptive}.
Specifically, they introduced a trainable rational activation function that prevents value overfitting and overestimation in reinforcement learning.



\begin{figure*}[ht!]
    \centering
    \includegraphics[width=0.3\textwidth]{figures/sources/mainnet_pls_acc.pdf}
    \includegraphics[width=0.3\textwidth]{figures/sources/subnet_pls_acc.pdf}
    \includegraphics[width=0.3\textwidth]{figures/sources/warm_start_dropout.pdf}
    \caption{\textbf{Left.} Random label MNIST experiment using an 8-layer MLP. Higher dropout probabilities result in significant trainability loss. 
    \textbf{Middle.} Accuracy of the subnetworks trained on random target. Each subnetworks are sampled from original network after each epoch. Subnetworks of the Dropout also experience trainability loss. \textbf{Right.} Warm-start scenario of Resnet-18 model with CIFAR100 dataset. Dropout improves generalization performance; however, the reduction in accuracy compared to the cold-start scenario is nearly identical to that of the vanilla model.}
    \label{exp_dropout}
\end{figure*}




\section{\label{sec:method}Methodology}

Each SIEM system uses its own RDL to define threat detection rules, and each RDL has its own schema.
For example, the Splunk SIEM uses the SPL to define its threat detection rules.
The task of understanding threat detection rules and recommending relevant MITRE ATT\&CK techniques (or sub-techniques) requires complex reasoning skills.
In the case of LLMs, this can be achieved with a technique called prompt chaining in which each task is divided into multiple sub-tasks in order to understand the complex reasoning behind the task.
Therefore, we employ a multi-phase architecture based on prompt chaining that leverages the power of LLMs to take a SIEM rule defined in any RDL and map it to relevant MITRE ATT\&CK techniques using the power of LLMs.
Our approach is based on the following intuitions:
\begin{itemize}[nosep,leftmargin=*]
    \item \textit{LLMs' implicit knowledge}: LLMs possess deep understanding of diverse RDLs. This enables them to interpret any rule, regardless of the RDL it is defined in, and convert it into comprehensible natural language text.
    \item \textit{LLMs' similarity comparison capability}: LLMs are adept at analyzing and comparing textual descriptions. 
    They can intelligently assess the similarity between two textual inputs to establish a meaningful connection.
\end{itemize}

\methodName has two main phases: (1) the rule to text translation phase, and (2) the MITRE ATT\&CK techniques recommendation phase.
These two phases in the pipeline include six key steps to determine relevant TTPs, as illustrated in Figure~\ref{fig:r2t}.

Although LLMs excel at translating SIEM rules into natural language, they often lack critical domain-specific contextual information related to IoCs in the rules.
To overcome this limitation, the \textit{rule to text translation} phase includes three steps: IoC extraction, contextual information retrieval, and natural language translation.

The workflow begins with the extraction of IoCs from the rules (for example, processes, log source, event codes, and file names) that the rule searches for in the logs (step (1)).In the next sstep a web search agent performs the task of obtaining additional contextual information about the IoCs discovered ((step 2)).
By incorporating this additional domain-specific information, the pipeline enhances the language translation, resulting in a more accurate and meaningful interpretation of SIEM rules.
The rule itself and the IoCs' contextual additional information from the previous stage are then used to translate the rule from RDL to natural language (step (3)).

The \textit{MITRE ATT\&CK techniques} recommendation phase of the pipeline includes the following three steps.
The rule in processed in data source identification step in which the probable origin of the data is identified (step (4)).
The description of the rule is then used to determine probable MITRE ATT\&CK techniques based on the implicit knowledge of the LLM (step (5)).
Finally, using chain-of-thought~\cite{wei2022chain} prompting, the most relevant techniques are extracted from the list of probable techniques (step (6)).
Each step of our method is further described in detail below.


% [bb=0 0 1440 900,width=1.43\linewidth,height=0.9\textwidth]
\begin{figure*}[htbp]
   \includegraphics[width=\textwidth]{Images/stages.jpg}
    
   \caption{An illustration of the different steps in \methodName.}
   \label{fig:stages}
\end{figure*} 

\subsection{IoC Extraction}
The context associated with a SIEM detection rule is crucial for its accurate interpretation and effective application. 
Obtaining this contextual understanding requires comprehensive analysis of the embedded IoCs in the SIEM rule.
In the first step, \methodName systematically identifies and extracts all IoCs, identifying the types of IoCs and their corresponding values that form the foundational elements of the detection rules. 
Leveraging the LLM's inherent understanding of rule structures and IoCs, we employ a zero-shot prompting approach for this task. 
Zero-shot prompting enables the direct extraction of IoCs from the rules without requiring extensive pre-training on specific datasets.

\noindent The result of this stage is a dictionary structure, where:
\begin{itemize}[nosep,leftmargin=*]
    \item Keys represent types of IoC, such as processes, files, IP addresses, and log sources.
    \item Values are lists containing specific IoC details, such as process names, file names, IP addresses, and log source identifiers.
\end{itemize}

In the example depicted in Figure~\ref{fig:stages}(a), the pipeline processes a rule for which relevant MITRE ATT\&CK techniques need to be recommended. 
The IoC extractor LLM produces a dictionary structure as output, organizing the IoCs in a structured format to support subsequent stages in the analysis pipeline. 



\subsection{Contextual Information Retrieval}
In this step, an LLM agent is employed to retrieve relevant information pertaining to the IoCs extracted from the rule.
A REACT agent~\cite{react} was used in this case to generate both reasoning traces and task-specific actions in an interleaved manner.
REACT agents interact with external tools to retrieve additional information that leads to more factual and reliable responses.
The LLM agent conducts a systematic search across web resources to gather additional contextual information for each IoC value present in the rule. 
This step addresses LLMS' lack of up-to-date knowledge or specialized domain expertise (which is critical to understanding the role and significance of the IoCs in the rule), without the need for retraining or fine-tuning.
Figure~\ref{fig:stages}(b) presents an example in which the rule includes the process name \texttt{soaphound.exe} as an IoC.
As can be seen, the web search results indicate that \texttt{soaphound.exe} is being used for active directory (AD) enumeration, which is important for the understanding of the attack. 

\subsection{Natural Language Translation}

The translation of detection rules into natural language textual descriptions fulfills three key objectives:
\begin{enumerate}[nosep,leftmargin=*]
    \item \textbf{Ensures that \methodName is format-agnostic}: It converts rules defined in various RDL formats into a generic, unstructured text format, ensuring compatibility with different SIEM systems, regardless of the specific rule format.
    \item \textbf{Provides contextual explanation}: It includes all relevant contextual information to produce a concise and comprehensible explanation of the rule.
    \item \textbf{Enhances the comprehension for LLMs}: It enables LLMs to more effectively compare the translated rule with descriptions in the MITRE ATT\&CK framework by providing a unified textual representation.
\end{enumerate}
To achieve these objectives, a zero-shot prompting technique is employed. 
The input to the LLM comprises two components:
\begin{itemize}
    \item \textbf{Syntactical information}: The rule itself, providing the structural and operational details.
    \item \textbf{Contextual information}: Details of the IoCs extracted from the rule, providing semantic insights into the rule's intent and function.
\end{itemize}
The LLM utilizes these inputs to generate a natural language textual description of the rule. 
This transformation not only ensures a more interpretable representation but also facilitates further steps of analysis and comparison, particularly in aligning the rule with MITRE ATT\&CK techniques and sub-techniques.



\subsection{Data Source or Mitigation Identification}
Identifying the most relevant data component or mitigation associated with the rule description in this step is critical for filtering out irrelevant MITRE ATT\&CK techniques (or sub-techniques) in subsequent steps of the pipeline.
In the MITRE ATT\&CK framework, data sources represent various categories of information that can be gathered from sensors or logs. 
These data sources include \textit{data components}, which are specific attributes or properties within a data source that are directly relevant to detecting a particular technique or sub-technique~. 
For example, in the context of the rule described in Figure~\ref{fig:stages}(a), the term \texttt{Endpoint.Processes} indicates that the activity is happening on an endpoint. 
Presence of the terms such as, \texttt{soaphound.exe}, \texttt{--buildcache}, \texttt{--certdump} and etc. indicate that the rule searches for command line execution of an executable named \texttt{soaphound.exe} with specific parameters. 
Therefore, the appropriate data source in this example is \textit{Command}, with the corresponding data component being \textit{Command Execution}.
Additionally, \textit{mitigations} are defined as categories of technologies or strategies that can prevent or reduce the impact of specific techniques or sub-techniques. 
The MITRE ATT\&CK framework explicitly establishes relationships between data components, mitigations, and techniques (or sub-techniques), enabling a systematic approach for identifying relevant elements.

To identify the most relevant data component or mitigation associated with a given rule description, we utilize agentic retrieval augmented generation (RAG), which incorporates an AI Agent-based implementation of the RAG framework.
Data from the MITRE ATT\&CK framework, specifically related to data components and mitigations, is stored in a vector database (e.g., ChromaDB). 
The process begins with the rule description from the previous stage, which serves as the input to the AI Agent. 
The LLM-powered agent automatically generates a search query tailored to retrieve relevant information from the RAG database.

For each query, the system retrieves the five most similar documents from the database, each containing contextual information about data components or mitigations. 
These documents are then utilized by the LLM agent to contextualize the rule description. 
By comparing the content of these retrieved documents with the rule description, the LLM agent determines and outputs the most relevant data component or mitigation along with a chain-of-thought as to why the data component or mitigation is related to the rule.


\subsection{Probable Technique Recommendation}

In this step, an LM agent is utilized to propose probable MITRE ATT\&CK techniques (and sub-techniques) that may be relevant to the description of the provided rule.
We used a REACT agent in this step as well to utilize both implicit and explicit knowledge during reasoning.
For explicit knowledge, the agent searches the MITRE ATT\&CK framework to obtain the list of probable techniques (and sub-techniques).
The natural language description of the rule from the previous step serves as input to the LLM agent.
The output of this stage consists of a list of JSON objects, each containing the MITRE technique ID, technique name, and technique description as seen in Figure~\ref{fig:stages}(c).

Throughout our experiments, we observed that as the number of recommendations ($k$) increases, both the framework's average recall and precision initially improve, however beyond a certain threshold of $k$, the %average 
precision begins to decline.
Based on these observations(please refer Table~\ref{tab:results3}), we selected a $k$-value of 11 to ensure a high recall.



\subsection{Relevant Technique Extraction}
In this step, \methodName refines the set of probable MITRE ATT\&CK techniques identified in the previous stage by eliminating irrelevant entries.
This step in the pipeline serves two primary purposes: (1) to enhance precision while maintaining recall achieved in previous step, and (2) to provide a clear rationale for the selection of the labels, ensuring transparency and interpretability of the mapping process.
This refinement process is grounded in the assumption that LLMs are effective for text similarity matching tasks.

The process comprises two key steps:
\begin{itemize}
    \item \textit{Rule-technique comparison}: The description of each technique in the set of probable techniques is compared with the rule description. 
    A chain-of-thought technique is then applied to elucidate the reasoning behind the association of each technique with the rule.
    \item \textit{Confidence calculation}: The generated chain-of-thought rationale for each technique (or sub-technique) is compared with the rule description to compute a relevance (or confidence) score, as done in prior work~\cite{freitas2024ai}.
    % \item \textbf{Reasoning}: \new{Add here the reasoning that it provides - explaining in NLP why it was selected...}
\end{itemize}

Techniques with higher confidence scores are deemed more relevant to the rule. 
Conversely, techniques with scores falling below a predefined threshold are excluded.
The techniques retained after this filtering step represent the most relevant techniques corresponding to the given rule's description. 


The chain-of-thought (CoT) rationale generated during the comparison of each rule to its probable technique is also provided as an output in this step.
This rationale offers a detailed natural language explanation, articulating why a particular technique is relevant to the given rule. 
Such explanations are highly valuable for security analysts, as they provide clear and transparent reasoning behind the mapping, enabling analysts to better understand and validate the association between the rule and the technique.
Other classification models proposed in previous works within this domain also suffer from the limitation of being black-box models, which lack the ability to provide clear reasoning or explanations. 
Unlike \methodName, these models fail to generate transparent, CoT rationales that explain why a particular rule is mapped to a specific technique, making them less interpretable and less useful for security analysts.
\section{Experiments: Planning outperforms Heuristics}
\label{sec:experiment}

We begin our empirical demonstrations by showcasing the effectiveness of our planning framework on both synthetic and real datasets. We focus on the simplest planning algorithm, 1-step lookaheads (Algorithm~\ref{alg:complete}), and show that even basic planning can hold great promise. 
We illustrate our framework using two uncertainty quantification modules---GPs and 
\ensembles/ \ensembleplus. 

Throughout this section, we focus on evaluating the mean squared error of 
a regression model $\model$,  and develop adaptive policies that minimize uncertainty on $g(f)$ defined in~\eqref{eqn:l2-g-f}.
When GPs provide a valid model of uncertainty, 
our experiments show that our planning framework significantly outperforms other baselines. 
We further demonstrate that our conceptual framework extends to deep learning-based uncertainty quantification methods such as  \ensembleplus while highlighting computational challenges that need to be resolved in order to scale our ideas. 
For simplicity, we assume a naive predictor, i.e., $\psi(\cdot) \equiv 0$. However, we emphasize that this problem is just as complex as if we were using a sophisticated model $\psi(.)$. The performance gap between the algorithms 
primarily depends
on the level  of uncertainty in our prior beliefs.

To evaluate the performance of our algorithm, we benchmark it against several baselines. 
%Active learning baselines use an acquisition function $\ac$ to select points that have the highest   function value: $X\opt_t \in \argmax_{X \in \xpoolj{t}} \ac({X})$ at every step $t$. These methods may also need an UQ module, which we simply use the same UQ module as in our algorithm, and it  outputs $V(X)$ that measures the the uncertainty of each point $X \in \xpoolj{t}$.
Our first set of baselines are from active learning~\citep{AggarwalKoGuHaPh14}:
\\ % \noindent\textbf{Active Learning Heuristics:} 
\textbf{(1)} 
\textsf{Uncertainty Sampling (Static):}  In this approach, we query the samples for which the model is least certain about. Specifically, we estimate the variance of the latent output $f(X)$ for each $X \in \xpool$ using the UQ module and select the top-$K$ points with the highest uncertainty. \\
\textbf{(2)} \textsf{Uncertainty Sampling (Sequential):} This is a greedy heuristic that sequentially selects the points with the highest uncertainty within a batch, while updating the posterior beliefs using pseudo labels from the current posterior state. Unlike \textsf{Uncertainty Sampling (Static)}, this method takes into account the information gained from each point within batch, and hence tries to diversify the selected points within a batch. 

 
We also compare our approach to the  \textbf{(3)} \textsf{Random Sampling}, which selects each batch uniformly at random from the pool. Additionally, we compare solving the planning problem using  \textsf{REINFORCE}-based policy gradients with   $\mathsf{Smoothed\text{-}Autodiff}$ policy gradients.\footnote{Our code repository is available at
  \url{https://github.com/namkoong-lab/adaptive-labeling}.}
%Detailed experimental setups are provided in Section \ref{sec:details-experiments}.

%We repeat all experiments with 10 random seeds.




\begin{figure}[t]
\centering
\begin{minipage}[b]{0.49\textwidth}
\centering
\includegraphics[width=\textwidth, height=5cm]{figures/original_scale/Var_of_l_2_loss.pdf}
\caption{(Synthetic data) Variance of mean squared loss evaluated through the posterior belief $\mu_t$ at each horizon $t$. This is the objective that policy gradient methods like \textsf{REINFORCE} and $\ouralgo$ optimizes. 1-step lookaheads are surprisingly effective even in long horizons.}
\label{fig:var-l2-sim}
\end{minipage}
\hfill
\begin{minipage}[b]{0.49\textwidth}
\centering \includegraphics[width=\textwidth, height=5cm]{figures/original_scale/Error_of_estimated_model_l_2_loss.pdf}
\caption{(Synthetic data) Error between MSE calculated based on collected data $\mc{D}^{0:T}$ vs. population oracle MSE over $\mc{D}_{\rm eval} \sim P_X$. Reducing uncertainty over posteriors directly leads to better OOD evaluations. 1-step lookaheads significantly outperform active learning heuristics in small horizons.}
\label{fig:mean-l2-sim}
\end{minipage}
%\caption{Simulated data for GPs}
%\label{fig:both_plots}
\end{figure}

\subsection{Planning with Gaussian processes}
\label{sec:experiment-plan-GP}
We now briefly describe the data generation process for the GP experiments,  deferring a more detailed discussion of the dataset generation to Section~\ref{sec:details-experiments}. 
We use both the synthetic data and the real data to test our methodology.
For the \emph{simulated data},  we construct a setting where the general population is distributed across \emph{51 non-overlapping clusters} while the initial labeled data $\dtrain$ just comes from one cluster. In contrast, both $\dpool \defeq (\xpool,\ypool),\deval \defeq (\xeval,\yeval)$ are generated   from all the clusters. 
We begin with a low-dimensional scenario, generating a one-dimensional regression setting using a GP. %Gaussian Process (GP).
Although the data-generating process is not known to the algorithms,  we assume that the GP hyperparameters are known to all the algorithms
to ensure fair comparisons. This can be viewed as a setting where our prior is well-specified, allowing us to isolate the effects
of different policy optimization approaches
 without any concerns about the misspecified priors. We select $10$ batches, each of size $K=5$ across $T = 10$ time horizons.

To examine the robustness of our method against the distributional assumptions made  in the simulated case, we then move to a real dataset where the correct prior is not known. We simulate selection bias from the eICU dataset~\citep{PollardJoRaCeMaBa18}, which contains real-world patient data with in-hospital mortality outcomes. 
We conduct a $k$-means clustering to generate 51 clusters and then select data from those clusters. We view this to be a credible replication of practice, as severe distribution shifts are common due to selection bias in clinical labels.  To convert the binary mortality labels into a regression setting, we train a  random forest classifier and fit a GP on predicted scores, which serves as the UQ module for all the algorithms. As before, the task is to select 10 batches, each consisting of 5 samples, across 10 time horizons.

 In Figures~\ref{fig:var-l2-sim} and~\ref{fig:mean-l2-sim}, we present results for the simulated data. 
Figure~\ref{fig:var-l2-sim} shows the variance of $\ell_2$ loss, and Figure~\ref{fig:mean-l2-sim} presents the error in the estimated $\ell_2$ loss using $\mu_t$ (relative to true $\ell_2$ loss, that is unknown to the algorithm). 
As we can see from these plots, our method one-step lookahead  gives substantial improvements  over active learning baselines and random sampling. In addition,
compared to the one-step lookahead planning approach using \textsf{REINFORCE}-based policy gradients, 
we observe that $\mathsf{Smoothed\text{-}Autodiff}$-based policy gradients provide significantly more robust performance over all horizons.

In Figures~\ref{fig:var-l2-real}~and~\ref{fig:mean-l2-real}, we observe similar findings on the eICU data. We see that planning policies (\textsf{REINFORCE} and $\mathsf{Smoothed\text{-}Autodiff}$) consistently outperform other heuristics by a large margin.  Active learning baselines perform poorly in these small-horizon batched problems and can sometimes be even worse than the random search baselines.  Overall, our results show the importance of careful planning in adaptive labeling for reliable model evaluation. 

We offer some intuition as to why one-step lookahead planning may outperform other heuristic algorithms. 
 First,  \textsf{Uncertainty sampling (Static)} while myopically selects the
 top-$K$ inputs with the highest uncertainty, it fails to consider 
the overlap in information content among the ``best” instances; see \citep{AggarwalKoGuHaPh14} for more details. 
In other words,  it might acquire points from the same region with high uncertainty while failing to induce diversity among the batch.
Although \textsf{Uncertainty Sampling (Sequential)} somewhat addresses the issue of information overlap, a significant drawback of 
this algorithm
is the disconnect between the objective we aim to optimize and the algorithm. For example, it might sample from a region with high uncertainty but very low density. 

\begin{figure}[t]
\centering
\begin{minipage}[b]{0.48\textwidth}
\centering
\includegraphics[width=\textwidth, height=5cm]{figures/original_scale/Var_of_l_2_loss_real.pdf}
\caption{(Real-world eICU data) Variance of mean squared loss evaluated through the posterior belief $\mu_t$ at each horizon $t$. Even 1-step lookaheads are extremely effective planners, and auto-differentiation-based pathwise policy gradients provide a reliable optimization algorithm based on low-variance gradient estimates.}
\label{fig:var-l2-real}
\end{minipage}
\hfill
\begin{minipage}[b]{0.48\textwidth}
\centering \includegraphics[width=\textwidth, height=5cm]{figures/original_scale/Error_of_estimated_model_l_2_loss_real.pdf}
\caption{(Real-world eICU data) Error between MSE calculated based on collected data $\mc{D}^{0:T}$ vs. population oracle MSE over $\mc{D}_{\rm eval} \sim P_X$. Reducing uncertainty over posteriors directly leads to better OOD evaluations. Our method significantly outperforms active learning-based heuristics, and random sampling.}
\label{fig:mean-l2-real}
\end{minipage}
%\caption{Real data for GPs}
\end{figure}
 
%\vspace{-1.5cm}
% \begin{wrapfigure}{r}{.32\columnwidth}
%   \vspace{-.5cm} 
%   \centering
% \includegraphics[scale=.29]{figures/Var of l2l_2 loss.pdf}
%   \vspace{-0.2cm}
%   \caption{Results of GP}
% \label{fig:var-l2-gp}
%   \vspace{-0.1cm}
% \end{wrapfigure}


% Attempts have been made  in the past to address these  drawbacks heuristically  (see \citep{AggarwalKoGuHaPh14}). We give a unified computational framework while approaching the problem in a more principled manner and solving it more optimally.




\subsection{Planning with  neural network-based uncertainty quantification methods ($\ensembleplus$)}


We now provide a proof-of-concept that shows the generalizability of our conceptual framework  to the deep learning-based UQ modules, specifically focusing on $\ensembleplus$ due to their previously observed superior performance~\citep{OsbandWenAsDwIbLuRo23}. Recall that implementing our framework with deep learning-based UQ modules  requires us to retrain the model across multiple possible random actions $\bm{a}(\theta)$ sampled from the current policy $\pi_\theta$.
This requires significant computational resources, in sharp contrast to the GPs where the posteriors are in closed form and can be readily updated and differentiated. 

Due to the computational constraints, we test $\ensembleplus$ on a toy setting to demonstrate the generalizability of our framework. We consider a setting where the general population consists of four clusters, while the initial labeled data only comes from one cluster. Again we generate data using GPs.  The task is to select a batch of 2 points in one horizon. We detail the $\ensembleplus$ architecture in Section \ref{sec:details-experiments}, and we assume prior uncertainty to be large (depends on the scaling of the prior generating functions). 
The results are summarized in the Table~\ref{tab:UQ_ensemble}.

% \begin{table}[H]
% \vspace{-10pt}
% \caption{Performance under \ensembleplus as UQ module}
%     \centering
%     \begin{tabular}{|m{3cm}|m{2.5cm}|m{2cm}|} 
%     \hline
%       Algorithm   & Variance of $\loss_2$ loss estimate & Error of $\loss_2$ loss estimate  \\ \hline Random Sampling 
%          & $1710.9 \pm 1352.1$ & $8.67\pm6.62$ 
%       \\ \hline \ouralgo & $1.30 \pm 0.68$ & $0.91\pm0.25$ \\ \hline
%     \end{tabular}
%     \label{tab:UQ_ensemble}
%     %\vspace{-10pt}
% \end{table}




\begin{table}[h]
\vspace{-10pt}
\caption{Performance under \ensembleplus as the UQ module}
\centering
\begin{tabular}{|l|l|l|}
\hline
Algorithm   & Variance of $\loss_2$ loss estimate & Error of $\loss_2$ loss estimate  \\
\hline
\textsf{Random sampling} & 7129.8 $\pm$ 1027.0 & 136.2 $\pm$ 8.28 \\ \hline
\textsf{Uncertainty sampling (Static)} & 10852 $\pm$ 0.0 & 162.156 $\pm$ 0.0 \\ \hline
\textsf{Uncertainty sampling (Sequential)} & 8585.5 $\pm$ 898.9 & 144 $\pm$ 6.93 \\ \hline
\textsf{REINFORCE} & 1697.1 $\pm$ 0.0 & 45.27 $\pm$ 0.0 \\ \hline
\ouralgo & 1697.1 $\pm$ 0.0 & 45.27 $\pm$ 0.0 \\ \hline
\end{tabular}
%\caption{Comparison of different algorithms based on variance   and   error in $\ell_2$ loss estimation with Ensemble $+$ as the UQ module. Our results demonstrate that {\ouralgo} and REINFORCE outperformthe other active learning based heuristics, confirming the benefits of our MDP formulation for the adaptive labeling problem, as also demonstrated in Section 4.\\
%\footnotesize{Experimental details: We use Gaussian Processes as our data generating process, GP parameters are the same as in Section D.3.  The task is to select a batch of 2 points along one horizon.The marginal distribution $p_X$ has 4 \textit{non-overlapping} clusters. Initial data comes from one cluster, while pool and evaluation points comes from all the clusters. We have $20$ initial labeled data points, $10$ pool points, and $252$ evaluation points.  Training procedures are similar to the one in Section D.3.} }
\label{tab:UQ_ensemble}
\end{table}



% We faced  issues in scaling up these experiments which will be our focus in the future. 





% \begin{itemize}
%     \item Posteriors should be consistent. Two dimensions: even with less training,  
%     \item the inference should be  fast enough
% \end{itemize}


% Potential research directions for uncertainty quantification

% In this section we consider a simple setting We consider a simpler setting and 


% For synthetic dataset generation, we use ...... For real datasets, we use ...... We compare our methodolgy to several baselines ()    This Section is structured as follows:
% \begin{itemize}
%     \item \textbf{GPs, square loss objective} (Section \ref{}): 
%     %the broad aim of the experiments  in this section is to isolate the performance of our methodology without any concerns for the inefficiencies induced due to a mis-specified prior or imperfect posterior inference. To accomplish this we generate synthetic datasets using GPs (detailed later). We use the well specified prior (GPs - with same hyperparameter setting) as our UQ module.   
%      As GPs provide differentaible posterior inference - any errors induced due to imperfect posterior updates are also isolated. We note that under this setting
%      \item In Section\ref{} we demonstrate why our methodology performs better than other baselines - by devising various synthetic experiments ()
%     \item  \textbf{UQ Benchmarking }(Section \ref{}): Before diving into the experiments using $\ensembleplus$ and ENNs,  we showcase our benchmarking experiments in Section \ref{}. We use real datasets We observe that ENNs perform better
%      \item \textbf{Ensemble $+$}, objective: recall, accuracy
%     \item \textbf{ENN}, objective: recall, accuracy
% \end{itemize}




% In Section {}, we test 
% \subsection{Experimental details}

% \begin{itemize}
%     \item UQ methodologies - GPs, ENNs
%     \item Objectives - Recall,  ATE
%     \item Datasets - ATE-synthetic datasets, Recall-synthetic, real datasets
%     \item Baselines - 
%     \begin{itemize}
%         \item Random sampling
%         \item Active learning - Uncertainty based sampling - In regression setting almost all of the 
%         \item Myopic greedy - Greedy Batch based sampling
%         \item Policy Gradient
%     \end{itemize}
    
% \end{itemize}

% \subsection{Experiments}
%     \begin{itemize}
%     \item GPs with square loss
%     \item Benchmarking ENN
%         \item ENNs with ATE
%         \item ENNs with Recall
%     \end{itemize}

% \subsection{Benefits over other algorithms - intuition and experiments}

%Active learning - Myopic greedy / Don't rely on the objective rather some entropy version.


%%% Local Variables:
%%% mode: latex
%%% TeX-master: "main"
%%% End:

\begin{table}[ht!]
\centering
\caption{\textbf{Super Resolution Performance Results.} Our proposed WGAN EEG Spatial Upsampling method significantly outperforms a baseline of Bicubic Interpolation commonly used in EEG upsampling pipelines.}
\label{tab:results}
\resizebox{0.8\linewidth}{!}{%
\begin{tabular}{@{}cccccc@{}}
\toprule
\multirow{2}{*}{\textbf{Dataset}} & \multirow{2}{*}{\textbf{Scale}} & \multicolumn{2}{c}{\textbf{Bicubic}} & \multicolumn{2}{c}{\textbf{WGAN}} \\ \cmidrule(l){3-6} 
                      &   & \textbf{MSE} & \textbf{MAE} & \textbf{MSE}    & \textbf{MAE}   \\
\toprule
\multirow{2}{*}{Val}  & 2 & 3.71E7       & 3.89E3       & \textbf{2.01E3} & \textbf{24.38} \\
                      & 4 & 7.23E7       & 6.42E3       & \textbf{8.53E3} & \textbf{63.83} \\
\midrule
\multirow{2}{*}{Test} & 2 & 3.75E7       & 3.91E3       & \textbf{2.06E3} & \textbf{24.66} \\
                      & 4 & 7.30E7       & 6.45E3       & \textbf{8.68E3} & \textbf{64.39} \\
\bottomrule
\end{tabular}%
}
\end{table}
\section*{Conclusion}
This paper aims to enhance our understanding of the computational complexity of computing various Shapley value variants. We found that for various ML models --- including decision trees, regression tree ensembles, weighted automata, and linear regression --- both local and global interventional and baseline SHAP can be computed in polynomial time under HMM modeled distributions. This extends popular algorithms, such as TreeSHAP, beyond their empirical distributional scope. We also establish strict complexity gaps between the various SHAP variants (baseline, interventional, and conditional) and prove the intractability of computing SHAP for tree ensembles and neural networks in simplified scenarios. Overall, we present SHAP as a versatile framework whose complexity depends on four key factors: \begin{inparaenum}[(i)] \item model type, \item SHAP variant, \item distribution modeling approach, \item and local vs. global explanations\end{inparaenum}. We believe this perspective provides deeper insight into the computational complexity of SHAP, paving the way for future work.




%We believe that our framework provides a more intricate understanding of SHAP computation complexity across different models, distributions, and variants, paving the way for further research.

Our work opens promising directions for future research. First, expanding our computational analysis to other SHAP-related metrics, such as asymmetric SHAP~\citep{frye20} and SAGE~\citep{covert2020understanding}, would be valuable. Additionally, we aim to explore more expressive distribution classes and relaxed assumptions beyond those in Section \ref{sec:tractable} while maintaining tractable SHAP computation. Finally, when exact computation is intractable (Section \ref{sec:intractable}), investigating the approximability of SHAP metrics through approximation and parameterized complexity theory~\citep{downey2012parameterized} is an important direction.

%Our work opens several promising avenues for future research on the computational properties of explainable AI methods, with a particular focus on SHAP. First, it would be interesting to broaden the computational analysis conducted in this work to include other popular SHAP-related metrics in the literature, such as asymmetric SHAP \cite{frye20} and SAGE \cite{covert2020understanding}. Also, in the future, we aim to explore more expressive distribution classes and relaxed distributional assumptions—extending beyond those examined in Section \ref{sec:tractable} —that still yield tractable SHAP computation. Finally, when exact computation proves intractable (Section \ref{sec:intractable}), it is worthwhile to theoretically investigate the question of the approximability of computing the SHAP metrics across various configurations, through the lens of approximation and parametrized complexity theory \cite{arora2009computational}.

%This paper aims to deepen our understanding of the computational complexity involved in obtaining different Shapley value variants. We found that for a variety of ML models, including decision trees, tree ensembles for regression, weighted automata, and linear regression models — computing both local and global interventional and baseline SHAP can be done in polynomial time when distributions are modeled by HMMs. This extends the distributional scope of popular algorithms like TreeSHAP, which is limited to empirical distributions. Additionally, we demonstrate a strict complexity gap between SHAP variants, showing that interventional and baseline SHAP can be strictly easier to compute than conditional SHAP. Despite these positive results, we uncovered intractability for various SHAP variants in neural networks and tree ensembles. Finally, we provided generalized complexity relations across SHAP variants. We believe that our framework offers a deeper understanding of the complexity involved in computing SHAP across various variants, models, distributions, as well as in both local and global computations, laying the groundwork for future research.
\section{Limitations} 

In this work, we compared the effectiveness and interplay of SFT and RL-based methods, under fixed data constraints. In particular, we chose offline methods like DPO and KTO as the baseline implementation of the RL method because it eliminates the need for reward modeling or iterative finetuning. This means that the process of development is limited to collecting an offline dataset and fientuning it - making it the most fair comparable to SFT in terms of implementation effort, compute costs and annotation efforts. Since this baseline RL method shows optimal performance over SFT, we hope that this motivates future work to study more complex RL-based methods and their interplay with SFT. In addition, we used GPT4o annotation for synthetic data generation, and also for evaluating Summarization and Helpfulness, which could include potential biases inherited from the model. 

In addition, we limited the size of the model to under 10 Billion parameters, to keep the finetuning cost low enough to ignore as compared to the data annotation costs. In addition, it would be extremely compute resource intensive to run thousands of finetuning runs with larger model sizes like 70B parameters. We hope that future work would study the scaling trends of RL-based methods against different model sizes, and also study the compute-data trade-off in-depth.


% Bibliography entries for the entire Anthology, followed by custom entries
%\bibliography{anthology,custom}
% Custom bibliography entries only
\bibliography{custom}

\appendix

% \section{Example Appendix}
% \label{sec:appendix}

% This is an appendix.
% \section{Template for multi-task instruction fine-tuning}
\label{sec:templates}

Figure~\ref{fig:finetuning_template} shows the template used to process the inputs for {\oursys}. The template first informs about the domains involved in the dialog. Next, it provides task-specific instructions and provides the domain schemas. The dialog history is provided and the model is instructed to generate a system response based on the dialog history, search results, and the task instructions.


% \begin{figure*}
%     \centering
%     \includegraphics[width=0.97\linewidth]{assets/finetuning_template.pdf}
%     \caption{
% Multi-task instruction finetuning template}
%     \label{fig:finetuning_template}
% \end{figure*}

\begin{figure}[htbp]
    \centering
\begin{tcolorbox}[colframe=gray!20, colback=gray!10, coltitle=black, arc=5mm, boxrule=0.4mm, width=\columnwidth, left=3mm, right=3mm, top=3mm, bottom=3mm]
\setlength{\baselineskip}{1.5em}
You are an expert chat assistant for the domains: \textcolor{cyan!90}{[domains]}. \\
\textcolor{purple!90}{Instructions:} As an expert, you must generate the most appropriate response for the chat assistant. \\
The response can be an api call or a response to the user. \\
Based on the \textcolor{purple!90}{Last User Utterance}, you must find the relevant \textcolor{purple!90}{Intent} from the \textcolor{purple!90}{Schema} and your request should use the \textcolor{purple!90}{required slots} and \textcolor{purple!90}{optional slots} from that \textcolor{purple!90}{Intent}. \\
You will be provided with the Schema for domains: \textcolor{cyan!90}{[domains]}\\
\textcolor{cyan!90}{[schemas]}\\
You will be provided an incomplete dialog between a user and a chat assistant, and an optional search results.\\
\textcolor{purple!90}{Dialog History}: \textcolor{cyan!90}{[dialog history]}\\
Using the \textcolor{purple!90}{Dialog History}, \textcolor{purple!90}{Search Results}, and by following the \textcolor{purple!90}{Instructions} please generate the response for the chat assistant.
\end{tcolorbox}
    \caption{Multi-task instruction finetuning template. Items in blue are dynamic elements and those in purple are important aspects of the prompt.}
    \label{fig:finetuning_template}
\end{figure}
% \input{my_files/appendix_gleu_mixed}
% \section{Results to Carry Information from Domain One to Next in Multi-domain Dialogs}
\label{sec:multi_dom_api}

Table~\ref{tab:multi_api_mixed_results} presents the results for Carry Information from Domain One to Next in Multi-domain Dialogs. 
For this setting, we only have results on the SGD and KETOD datasets. We can see that for the key metric, Full API Accuracy, {\opt} achieves the best results across both datasets. {\opt} also achieves the highest scores for the Parameter Names Accuracy and Parameter Values Accuracy metrics as well. However, for the API Invoke Accuracy and API Method Accuracy, on some settings, {\flan} outperforms {\opt}.
Upon comparing the key metric, Full API Accuracy, of single domain results from Table~\ref{tab:api_results} with that of Multi-domain Dialogs, we can see that the performance of the models is better in the Multi-domain dialogs. A possible explanation of this could be that a bulk of the multi-domain dialogs contain common and generic domains like \texttt{Events}, \texttt{Restaurants}, and \texttt{Hotels}.
The dataset contains a lot of example dialogs from these domains, which in turn could have been the reason behind the improved performance. 

We can see that apart from the API Invoke Accuracy and API Method Accuracy, {\opt} is the model that produces the best results for the remaining metrics across both results. {\flan} is the second best performing model, and also achieves best results on the two metrics mentioned before. A notable observation on the key metric, Full Api Accuracy, is that the performance for mixed domain settings is in between that of the seen and unseen domain settings. This is the expected result, as mixed domains contain domains from both seen and unseen domains.

\section{Example Dialogs with Responses from All Models}
\label{sec:example_dialogs}
% Table~\ref{tab:rest_dialog} shows an example dialog in the Restaurant domain. It contains the turn id, user utterance, gold response, \gptt response, and {\flan} response. Texts highlighted in red outlines the portions where the system response by a model is incorrect and green highlights the correct parts.

% Table~\ref{tab:multi_domain_dialog} shows an example multi-domain dialog in the Buses and RideShare domain. Model utterances in red mark the mistakes made and green highlights the correct parts. We can see that \gptt has a common mistake of making an API Call before confirming with the user, whereas {\flan} does not have such an issue. Also, we can see that the responses of \gptt are longer and more descriptive, whereas {\flan} produces short and succint responses, similar to the gold standards.

Table~\ref{tab:rest_dialog} shows an example dialog in the Restaurant domain and Table~\ref{tab:multi_domain_dialog} shows an example multi-domain dialog in the Buses and \texttt{RideShare} domain. The table contains the turn id, user utterance, gold response, {\gptt}, {\gptf}, {\gpt}, {\flan}, and {\opt} response. Text highlighted in red outlines the portions where the system response by a model is incorrect and green highlights the correct parts. Texts highlighted in orange indicate that the model is partially correct and is missing some information. Texts highlighted in yellow indicate that the system response seems like a valid response but is incorrect because the gold standard is expecting a different response. 

From the example dialogs, we can see that {\gptt} and {\gptf} has a common mistake of making an API Call without confirming with the user, whereas {\oursys} models do not make this mistake. Also, we can see that the responses of {\gptt} and {\gptf} are longer and more descriptive, whereas {\oursys} models produce succinct responses, similar to the gold standards. This observation indicates that {\oursys} was able to pick up the text generation style of the gold standard, which is an important aspect for TOD systems.

We can see that {\gpt} hallucinates and makes up information on its own, but {\flan} and {\opt} do not suffer from hallucination as much. For the API call task, all the models face similar problems of missing certain parameters and in the multi-domain setting, the models fail to adapt to the slots of the new domain. For response generation, on turns 1 in Table~\ref{tab:multi_domain_dialog}, we can see the request sub-task challenge as discussed in Section~\ref{sec:response_generation}, where the gold standard expects a certain slot but the models request a different slot. Another observation from turn 2 of that dialog is that the gold standard expects the model to request 3 slots, but the {\oursys} models are only asking about a single slot. A reason for this could be that there fewer turns where the gold standard expects multiple slots to be requested, so the models picked this pattern and asked about one slot.

On turn 4 from Table~\ref{tab:rest_dialog}, we can see that {\gpt} only mentioned that the restaurant was booked but did not mention the cuisine of the restaurant, whereas all other models did not make this mistake. Similar examples exist on a few other turns, where {\gpt} misses important information.

\begin{figure*}
   \centering
   % \includegraphics[width=0.95\linewidth]{assets/finetuning_template.pdf}
   \includegraphics[width=0.95\linewidth]{assets/finetuning_template.pdf}
   
   \caption{
       Template used to generate the prompt for fine-tuning {\oursys}.
   }
   
   \label{fig:finetuning_template}
\end{figure*}

\begin{figure*}
   \centering
   \includegraphics[width=0.95\linewidth]{assets/chatgpt_template.pdf}
% \includegraphics[width=0.95\linewidth]{assets/chatgpt_template.pdf}   
   \caption{
       Template used to generate the prompt for OpenAI's GPT Models. 
   }
   
   \label{fig:chatgpt_template}
\end{figure*}


\begin{table*}[!t]
    \centering
    \small
    \begin{adjustbox}{max width=\textwidth}
        \begin{tabular}{|c|c|c|c|c| R c|c|c|c|}
            \hline
    \multirow{2}{*}{\textbf{Dataset}}&\multirow{2}{*}{\textbf{Model}} & \multicolumn{4}{R}{\textbf{Overall Response (BLEU-4)}} & \multicolumn{4}{c|}{\textbf{Overall Response (GLEU-4)}}  \\ \cline{3-10}
    & & \textbf{all}  & \textbf{seen} & \textbf{unseen} & \textbf{mixed} & \textbf{all}  & \textbf{seen} & \textbf{unseen} & \textbf{mixed} \\ \hline
    \multirow{4}{*}{SGD}&{\gptt} & 0.0397& 0.0422 & 0.0363 & 0.043
    &  0.0555& 0.0570 & 0.0522 & 0.0589
    \\ 
    &\gpt& 0.2276& 0.2423 & 0.2095 & 0.2436
    &  0.2240& 0.2430 & 0.2060 & 0.2395
    \\ 
    &\flan& \underline{0.2680}& \underline{0.2980} & \underline{0.2500} & \underline{0.2790}
    &  \underline{0.2610}& \underline{0.2960} & \underline{0.2440} & \underline{0.2721}
    \\ 
    &\opt& \textbf{0.3063}& \textbf{0.3311} & \textbf{0.2896} & \textbf{0.3174}
    &  \textbf{0.2996}& \textbf{0.3210} & \textbf{0.2841} & \textbf{0.3115} 
 \\  \hline
    
    % KETOD
    \multirow{4}{*}{KETOD}&{\gptt} &0.0397& 0.0412 & 0.0334 & 0.0438
    & 0.0554& 0.0563 & 0.0494 & 0.0601
    \\ 
    &\gpt&0.1509& 0.1756 & 0.1224 & 0.1507
    & 0.1552& 0.1751 & 0.1303 & 0.1568
    \\ 
    &\flan& \underline{0.2130}  & \underline{0.2503} & \underline{0.1794} & \underline{0.2040}
    & \underline{0.2091} & \underline{0.2408} & \underline{0.1825} & \underline{0.2000}
    \\ 
    &\opt&\textbf{0.2680}  & \textbf{0.2994} & \textbf{0.2371} & \textbf{0.2559}
    & \textbf{0.2534} & \textbf{0.2860} & \textbf{0.2248} & \textbf{0.2446}
 \\  \hline

    % BITOD
        \multirow{4}{*}{BiToD}&{\gptt} &0.0646  & 0.0646 & - & -
    & 0.0760 & 0.0760 & - & -
    \\ 
    &\gpt&\underline{0.3466}  & \underline{0.3466} & - & -
    & \underline{0.3016} & \underline{0.3016} & - & -
    \\ 
    &\flan& 0.3159  & 0.3159 & - & -
    & 0.2640 & 0.2640 & - & -
    \\ 
    &\opt&\textbf{0.4143}& \textbf{0.4143} & - & -
    & \textbf{0.3803}& \textbf{0.3803} & - & -
 \\  \hline
    
        \end{tabular}
    \end{adjustbox}
    \caption{Overall Response Generation: BLEU \& GLEU scores for system generations with additional mixed domain setting.}
    \label{tab:response_gleu_mixed}
    % \vspace{-15pt}
\end{table*}



\begin{table*}[!t]
    \centering
    \small
    \begin{adjustbox}{max width=\textwidth}
        \begin{tabular}{|c|c|c|c|c| R c|c|c|c|}
            \hline
    \multirow{2}{*}{\textbf{Dataset}}&\multirow{2}{*}{\textbf{Model}} & \multicolumn{4}{R}{\textbf{Inform (BLEU-4)}} & \multicolumn{4}{c|}{\textbf{Inform (GLEU-4)}}  \\ \cline{3-10}
    & & \textbf{all}  & \textbf{seen} & \textbf{unseen} & \textbf{mixed} & \textbf{all}  & \textbf{seen} & \textbf{unseen} & \textbf{mixed} \\ \hline
    \multirow{4}{*}{SGD}&{\gptt} &0.1047& 0.0998 & 0.1012 & 0.1094
    & 0.1563& 0.1505 & 0.1523 & 0.1616
    \\ 
    &\gpt& 0.2737& 0.3022& 0.2479 & 0.2913
    &  0.3644& 0.3880& 0.3421 & 0.3799
    \\ 
    &\flan& \underline{0.2991}& \underline{0.3064} & \underline{0.2872} & \underline{0.3099}
    &  \underline{0.3879}& \underline{0.3960} & \underline{0.3775} & \underline{0.3968}
    \\ 
    &\opt&\textbf{0.3334}& \textbf{0.3499} & \textbf{0.3190} & \textbf{0.3431}
    & \textbf{0.4199}& \textbf{0.4322} & \textbf{0.4092} & \textbf{0.4271}
 \\  \hline
    
    % KETOD
    \multirow{4}{*}{KETOD}&{\gptt} &0.0943& 0.0801 & 0.0958 & 0.1108
    & 0.1439& 0.1337 & 0.1428 & 0.1578
    \\ 
    &\gpt& 0.1734& 0.1843 & 0.1498 & 0.1817
    &  0.2565& 0.2717 & 0.2336 & 0.2587
    \\ 
    &\flan&\underline{0.2311}& \underline{0.2435} & \underline{0.2120} & \underline{0.2335}
    & \underline{0.3100}& \underline{0.3243} & \underline{0.2897} & \underline{0.3112}
    \\
    &\opt & \textbf{0.2968} & \textbf{0.2981} & \textbf{0.3034} & \textbf{0.2889}
    &  \textbf{0.3726}& \textbf{0.3776} & \textbf{0.3746} & \textbf{0.3646} 
 \\  \hline

    % BITOD
        \multirow{4}{*}{BiToD}&{\gptt}& 0.2589& 0.2589 & - &-
    &  0.3092& 0.3092 & - &-
    \\ 
    &\gpt& \underline{0.5226}& \underline{0.5226} & - & -
    &  \underline{0.5682}& \underline{0.5682} & - & -
    \\ 
    &\flan& 0.4795& 0.4795 & - & -
    &  0.5389& 0.5389 & - & -
    \\ 
    &\opt& \textbf{0.5248}& \textbf{0.5248} & - & -
    &  \textbf{0.5900}& \textbf{0.5900} & - & -
 \\  \hline
    
        \end{tabular}
    \end{adjustbox}
    \caption{BLEU \& GLEU scores for inform sub-task of response generation task, with additional mixed domain setting.}
    \label{tab:inform_gleu_mixed}
    % \vspace{-15pt}
\end{table*}



\begin{table*}[!t]
    \centering
    \small
    \begin{adjustbox}{max width=\textwidth}
        \begin{tabular}{|c|c|c|c|c| R c|c|c|c|}
            \hline
    \multirow{2}{*}{\textbf{Dataset}}&\multirow{2}{*}{\textbf{Model}} & \multicolumn{4}{R}{\textbf{Request (BLEU-4)}} & \multicolumn{4}{c|}{\textbf{Request (GLEU-4)}}  \\ \cline{3-10}
    & & \textbf{all}  & \textbf{seen} & \textbf{unseen} & \textbf{mixed} & \textbf{all}  & \textbf{seen} & \textbf{unseen} & \textbf{mixed} \\ \hline
    \multirow{4}{*}{SGD}& {\gptt} &0.0057 & 0.0046 & 0.0056 & 0.0060
    & 0.0337 & 0.0299 & 0.0343 & 0.0343
    \\ 
    &\gpt& 0.0616& 0.0659 & 0.0630 & 0.0588
    &  0.1387& 0.1453 & 0.1403 & 0.1353
    \\ 
    &\flan&\underline{0.0657}& \underline{0.0742} & \underline{0.0698} & \underline{0.0582}
    & \underline{0.1552}& \underline{0.1687} & \underline{0.1586} & \underline{0.1477}
    \\ 
    &\opt& \textbf{0.0836}& \textbf{0.0928} & \textbf{0.0793} & \textbf{0.0868}
    &  \textbf{0.1767}& \textbf{0.1870} & \textbf{0.1721} & \textbf{0.1803}
 \\  \hline
    
    % KETOD
    \multirow{4}{*}{KETOD}&{\gptt} &0.0033& 0.0036 & 0.0037 & 0.0026
    & 0.0305& 0.0285 & 0.0332 & 0.0300
    \\ 
    
    &\gpt& 0.0140& 0.0193& 0.0073 & 0.0148
    &  0.0649& 0.0750& 0.0561 & 0.0620
    \\ 
    
    &\flan&\underline{0.0587}& \underline{0.0781} & \underline{0.0430} & \underline{0.0521}
    & \underline{0.1440}& \underline{0.1707} & \underline{0.1256} & \textbf{0.1315}
    \\ 
    
    &\opt& \textbf{0.0640}& \textbf{0.0851}& \textbf{0.0509} & \textbf{0.0526}
    &  \textbf{0.1475}& \textbf{0.1742}& \textbf{0.1339} & \underline{0.1301}
    
 \\  \hline

    % BITOD
        \multirow{4}{*}{BiToD}&{\gptt} &0.0011& 0.0011& - & -
    & 0.0219& 0.0219& - & -
    \\ 
    &\gpt& \underline{0.0323}& \underline{0.0323} & - & -
    &  \underline{0.1035}& \underline{0.1035} & - & -
    \\ 
    &\flan&0.0256& 0.0256 & - & -
    & 0.0811& 0.0811 & - & -
    \\ 
    &\opt& \textbf{0.0476}& \textbf{0.0476} & - & -
    &  \textbf{0.1260}& \textbf{0.1260} & - & -
    
 \\  \hline
    
        \end{tabular}
    \end{adjustbox}
    \caption{BLEU \& GLEU scores for request sub-task of response generation task, with additional mixed domain setting.}
    \label{tab:seeking_information_gleu_mixed}
    % \vspace{-15pt}
\end{table*}

\begin{table*}[!t]
    \centering
    \begin{adjustbox}{max width=\textwidth}
        \begin{tabular}{|c|c|c|c|c|c|c|c|c|c|c|c|c|c|}
            \hline
            \multirow{2}{*}{\textbf{Metric}} & \multirow{2}{*}{\textbf{Model}} & \multicolumn{4}{c|}{\textbf{SGD}} & \multicolumn{4}{c|}{\textbf{KETOD}} & \multicolumn{4}{c|}{\textbf{BiToD}} \\ \cline{3-14}
            & & \textbf{all} & \textbf{seen} & \textbf{unseen} & \textbf{mixed} & \textbf{all} & \textbf{seen} & \textbf{unseen} & \textbf{mixed} & \textbf{all} & \textbf{seen} & \textbf{unseen} & \textbf{mixed} \\ \hline

            \multirow{2}{*}{API} 
            & \gptt & 68.42 & 69.16 & 68.01 & 68.65 & 64.57 & 65.10 & 63.22 & 65.06 & 44.59 & 44.59 & - & - \\ 
            \multirow{2}{*}{Invoke} & \gpt & 90.51 & 97.66 & 86.21 & 93.11 & 78.83 & 80.87 & 73.56 & 80.75 & 93.46 & 93.46 & - & - \\ 
            \multirow{2}{*}{Accuracy} & \flan & \textbf{98.65} & \underline{99.61} & \textbf{98.24} & \textbf{98.84} & \textbf{90.45} & \textbf{89.93} & \textbf{90.34} & \textbf{91.09} & \textbf{98.48} & \textbf{98.48} & - & - \\ 
            & \opt & \underline{97.02} & \textbf{99.84} & \underline{95.81} & \underline{97.59} & \underline{82.22} & \underline{81.38} & \underline{84.14} & \underline{81.64} & \underline{97.36} & \underline{97.36} & - & - \\ 

            \hline
            \multirow{2}{*}{API} 
            & \gptt & 66.09 & 66.98 & 65.38 & 66.57 & 62.12 & 62.08 & 62.99 & 61.50 & 08.31 & 08.31 & - & - \\ 
            \multirow{2}{*}{Method} & \gpt & 78.62 & 96.26 & 71.42 & 81.80 & 71.48 & 78.36 & 62.53 & 71.12 & 87.34 & 87.34 & - & - \\ 
            \multirow{2}{*}{Accuracy} & \flan & \textbf{96.84} & \underline{99.61} & \textbf{95.23} & \textbf{97.81} & \textbf{89.07} & \textbf{89.93} & \textbf{86.67} & \textbf{90.02} & \textbf{93.07} & \textbf{93.07} & - & - \\ 
            & \opt & \underline{94.74} & \textbf{99.84} & \underline{92.13} & \underline{96.17} & \underline{78.45} & \underline{80.87} & \underline{77.70} & \underline{76.47} & \underline{92.70} & \underline{92.70} & - & - \\ 

            \hline
            \multirow{2}{*}{Parameter} 
            & \gptt & 67.24 & 64.84 & 68.38 & 66.65 & 63.47 & 61.17 & 67.11 & 63.08 & 04.53 & 04.53 & - & - \\ 
            \multirow{2}{*}{Name} & \gpt & 60.65 & 92.07 & 49.52 & 64.69 & 57.02 & 72.98 & 39.60 & 53.57 & 86.29 & 86.29 & - & - \\ 
            \multirow{2}{*}{Accuracy} & \flan & \underline{79.71} & \underline{96.97} & \underline{74.58} & \underline{86.68} & \underline{78.62} & \textbf{88.92} & \underline{67.93} & \textbf{75.97} & \underline{90.76} & \underline{90.76} & - & - \\ 
            & \opt & \textbf{90.41} & \textbf{98.19} & \textbf{87.86} & \textbf{91.21} & \textbf{79.50} & \underline{83.85} & \textbf{78.94} & \underline{75.30} & \textbf{92.32} & \textbf{92.32} & - & - \\ 

            \hline
            \multirow{2}{*}{Parameter} 
            & \gptt & 61.37 & 60.56 & 61.72 & 61.21 & 57.64 & 56.03 & 60.30 & 57.28 & 04.15 & 04.15 & - & - \\ 
            \multirow{2}{*}{Values} & \gpt & 58.64 & 91.24 & 46.98 & 62.95 & 55.46 & 71.83 & 37.73 & 51.83 & 86.10 & 86.10 & - & - \\ 
            \multirow{2}{*}{Accuracy} & \flan & \underline{76.03} & \underline{94.04} & \underline{70.43} & \underline{84.51} & \underline{76.52} & \textbf{87.94} & \underline{65.28} & \textbf{73.09} & \underline{85.47} & \underline{85.47} & - & - \\ 
            & \opt & \textbf{87.70} & \textbf{97.91} & \textbf{84.02} & \textbf{89.08} & \textbf{77.43} & \underline{83.46} & \textbf{75.87} & \underline{72.22} & \textbf{92.17} & \textbf{92.17} & - & - \\ 

            \hline
            \multirow{2}{*}{Full} 
            & \gptt & 33.32 & 32.63 & 28.52 & 38.06 & 29.59 & 30.54 & 23.45 & 33.33 & 02.36 & 02.36 & - & - \\ 
            \multirow{2}{*}{Api} & \gpt & 36.72 & 83.41 & 19.40 & 43.48 & 36.75 & 59.56 & 10.80 & 32.62 & 83.19 & 83.19 & - & - \\ 
            \multirow{2}{*}{Accuracy} & \flan & \underline{56.64} & \underline{89.49} & \underline{41.96} & \underline{63.79} & \textbf{59.42} & \textbf{83.72} & \underline{35.86} & \textbf{51.87} & \underline{73.79} & \underline{73.79} & - & - \\ 
            & \opt & \textbf{63.81} & \textbf{90.97} & \textbf{51.98} & \textbf{69.42} & \underline{57.35} & \underline{77.18} & \textbf{39.54} & \underline{50.09} & \textbf{91.11} & \textbf{91.11} & - & - \\ 


            \hline
        \end{tabular}
    \end{adjustbox}
    \vspace{-6pt}
    \caption{Detailed results for API calls across different metrics and datasets along with mixed domain.}
    \label{tab:api_results_mixed}
    \vspace{-12pt}
\end{table*}




\begin{table*}[!t]
    \centering
    \begin{adjustbox}{max width=\textwidth}
        \begin{tabular}{|c|c|c|c|c|R c|c|c |R c|c|c|c|}
            \hline
            \multirow{2}{*}{\textbf{Metric}} & \multirow{2}{*}{\textbf{Model}} & \multicolumn{4}{R}{\textbf{SGD}} & \multicolumn{4}{R}{\textbf{KETOD}} & \multicolumn{4}{c|}{\textbf{BiToD}} \\ \cline{3-14}
            & & \textbf{all} & \textbf{seen} & \textbf{unseen} & \textbf{mixed} & \textbf{all} & \textbf{seen} & \textbf{unseen}  & \textbf{mixed} & \textbf{all} & \textbf{seen} & \textbf{unseen}  & \textbf{mixed} \\ \hline
            
            \multirow{2}{*}{Api} 
            & \gptt & 68.34 & 73.42 & 67.77 & 68.12 & 62.05 & 61.05 & 66.00 & 61.20 & 29.47 & 29.47 & - & - \\ 
            \multirow{2}{*}{Invoke} & \gpt & 92.32 & 97.75 & 89.06 & 93.95 & 84.02 & 85.26 & 81.33 & 84.15 & \textbf{99.90} & \textbf{99.90} & - & - \\ 
            \multirow{2}{*}{Accuracy} & \flan & \textbf{98.80} & \underline{98.87} & \textbf{98.42} & \textbf{99.00} & \textbf{89.01} & \textbf{88.07} & \underline{90.00} & \textbf{89.34} 
            & 95.14 & 95.14 & - & -  \\ 
            & \opt & \underline{96.32} & \textbf{99.55} & \underline{94.98}  & \underline{96.87} & \underline{88.64} & \underline{87.72} & \textbf{92.67} & \underline{87.70} & \underline{99.31} & \underline{99.31} & - & - \\ 

            \hline
            \multirow{2}{*}{Api} 
            & \gptt & 65.94 & 72.07 & 64.80 & 66.01 & 60.05 & 58.60 & 66.67 & 58.47 & 12.23 & 12.23 & - & - \\ 
            \multirow{2}{*}{Method} & \gpt & 81.24 & 97.07 & 73.76 & 84.57 & 78.65 & 83.51 & 68.67 & 78.96 & 91.29 & 91.29 & - & - \\ 
            \multirow{2}{*}{Accuracy} & \flan & \textbf{96.94} & \underline{98.87} & \textbf{95.05}& \textbf{97.97} 
            & \textbf{87.02} & \textbf{88.07} & \underline{83.33} & \textbf{87.70} & \underline{91.53} & \underline{91.53} & - & - \\ 
            & \opt & \underline{93.87} & \textbf{99.55} & \underline{90.80} & \underline{95.33} & \underline{86.27} & \underline{87.72} & \textbf{86.00} & \underline{85.25} & \textbf{92.41} & \textbf{92.41} & - & - \\ 

            \hline
            \multirow{2}{*}{Parameter} 
            & \gptt & 65.85 & 68.99 & 64.65 & 66.31 & 60.07 & 58.10 & \underline{64.27} & 59.89 & 09.17 & 09.17 & - & - \\ 
            \multirow{2}{*}{Names} & \gpt & 58.23 & 95.61 & 44.97 & 63.01 & 61.22 & 79.28 & 35.37 & 57.76 & \underline{90.40} & \underline{90.40} & - & - \\ 
            \multirow{2}{*}{Accuracy} & \flan & \underline{77.31} & \underline{98.47} & \underline{72.96} & \underline{83.18} 
            & \underline{75.18} & \underline{87.11} & 62.96 & \underline{70.89} & 88.43 & 88.43 & - & - \\ 
            & \opt & \textbf{89.30} & \textbf{99.19} & \textbf{86.38} & \textbf{90.16} & \textbf{82.17} & \textbf{87.78} & \textbf{79.44} & \textbf{78.92} & \textbf{92.01} & \textbf{92.01} & - & -  \\ 

            \hline
            \multirow{2}{*}{Parameter} 
            & \gptt & 61.39 & 65.19 & 58.88 & 62.69 & 56.18 & 55.20 & 57.80 & 56.28 & 08.60 & 08.60 & - & - \\ 
            \multirow{2}{*}{Values} & \gpt & 56.46 & 94.52 & 42.59 & 61.60 & 59.44 & 77.53 & 33.22 & 56.09 & \underline{90.29} & \underline{90.29} & - & - \\ 
            \multirow{2}{*}{Accuracy} & \flan & \underline{73.77} & \underline{96.32} & \underline{68.36} & \underline{81.18}
            & \underline{73.16} & \underline{85.87} & \underline{59.72}& \underline{68.77} & 82.90 & 82.90 & - & - \\ 
            & \opt & \textbf{86.61} & \textbf{98.77} & \textbf{82.10} & \textbf{88.30}& \textbf{79.75} & \textbf{87.16} & \textbf{75.55} & \textbf{75.70} & \textbf{91.95} & \textbf{91.95} & - & - \\ 

            \hline
            \multirow{2}{*}{Full} 
            & \gptt & 38.62 & 40.77 & 33.66 & 41.83 & 34.08 & 36.84 & 26.67 & 34.97 & 03.55 & 03.55 & - & - \\ 
            \multirow{2}{*}{Api} & \gpt & 36.31 & 87.39 & 19.32 & 42.07 & 43.07 & 68.07 & 12.00 & 36.34 & \underline{88.14} & \underline{88.14} & - & - \\ 
            \multirow{2}{*}{Accuracy} & \flan & \underline{50.72} & \underline{92.12} & \underline{35.09} & \underline{56.39} 
            & \underline{56.30} & \underline{82.46} & \underline{30.00}& \underline{46.72} & 72.79 & 72.79 & -& -  \\ 
            & \opt & \textbf{61.13} & \textbf{96.40} & \textbf{46.91} & \textbf{66.85}& \textbf{60.55} & \textbf{83.16} & \textbf{36.67} & \textbf{52.73} & \textbf{91.38} & \textbf{91.38} & - & - \\ 

            \hline
        \end{tabular}
    \end{adjustbox}
    \caption{Results for carry information from domain one to next in multi-domain dialogs along with mixed domains.}
    \label{tab:multi_api_mixed_results}
\end{table*}




% \begin{table*}[!t]
%     \centering
%     \begin{adjustbox}{max width=\textwidth}
%         \begin{tabular}{|c |c|c |c|R c|c|R c|c|R c|c|R c|c|c|}
%             \hline
%             % \multirow{2}{*}{Model} & \multirow{2}{*}{Domains} & Response & Response & Retrieval & Retrieval & Slot Fill&Slot Fill\\ 
%             % & &  BLEU& GLEU& BLEU& GLEU& BLEU& GLEU \\ \hline
%     \multirow{2}{*}{\textbf{Dataset}} & \multirow{2}{*}{\textbf{Model}} & \multicolumn{3}{R}{\textbf{Api Invoke Accuracy}} & \multicolumn{3}{R}{\textbf{Api Method Accuracy}} & \multicolumn{3}{R}{\textbf{Param Names Accuracy}} & \multicolumn{3}{R}{\textbf{Param Values Accuracy}} & \multicolumn{3}{c|}{\textbf{Full API Accuracy}} \\ \cline{3-17}
%     & & \textbf{all}  & \textbf{seen} & \textbf{unseen} &  \textbf{all}  & \textbf{seen} & \textbf{unseen}&  \textbf{all}  & \textbf{seen} & \textbf{unseen}&  \textbf{all}  & \textbf{seen} & \textbf{unseen}& \textbf{all}  & \textbf{seen} & \textbf{unseen}  \\ \hline
%     \multirow{4}{*}{SGD} & {\gptt} & 68.34& 73.42& 67.77
%     & 65.94& 72.07& 64.80
%     & 65.85& 68.99 & 64.65
%     & 61.39& 65.19 & 58.88
%     & 38.62 & 40.77 & 33.66
%     \\ 
%     & \gpt & 92.32 & 97.75 & 89.06
%     & 81.24 & 97.07 & 73.76
%     &58.23 & 95.61 & 44.97
%     & 56.46 & 94.52 & 42.59 
%     & 36.31 & 87.39 & 19.32
%     \\ 
    
%     & \flan & \textbf{98.80} & \underline{98.87} & \textbf{98.42} 
%     & \textbf{96.94} & \underline{98.87} & \textbf{95.05} 
%     & \underline{77.31} & \underline{98.47} & \underline{72.96} 
%     & \underline{73.77} & \underline{96.32} & \underline{68.36} 
%     & \underline{50.72} & \underline{92.12} & \underline{35.09} 
%     \\
%     & \opt & \underline{96.32} & \textbf{99.55} & \underline{94.98} 
%     & \underline{93.87} & \textbf{99.55} & \underline{90.80} 
%     & \textbf{89.30} & \textbf{99.19} & \textbf{86.38} 
%     & \textbf{86.61} & \textbf{98.77} & \textbf{82.10} 
%     & \textbf{61.13} & \textbf{96.40} & \textbf{46.91} 
    
%     \\ 
%     \hline


% %KETOD
%     \multirow{4}{*}{KETOD} & {\gptt} &62.05& 61.05 & 66.00
%     & 60.05& 58.60 & 66.67 
%     & 60.07 & 58.10 & \underline{64.27}
%     & 56.18& 55.20 & 57.80 
%     & 34.08& 36.84 & 26.67 
%     \\ 
    
%     & \gpt & 84.02  & 85.26 & 81.33 
%     & 78.65 & 83.51 & 68.67 
%     & 61.22 & 79.28 & 35.37 
%     & 59.44 & 77.53 & 33.22 
%     & 43.07 & 68.07 & 12.00 
%     \\ 

%     & \flan &\textbf{89.01}  & \textbf{88.07} & \underline{90.00} 
%     & \textbf{87.02}& \textbf{88.07} & \underline{83.33} 
%     & \underline{75.18}& \underline{87.11} & 62.96 
%     & \underline{73.16} & \underline{85.87} & \underline{59.72} 
%     & \underline{56.30} & \underline{82.46} & \underline{30.00}
%     \\ 
    
%     & \opt& \underline{88.64} & \underline{87.72} & \textbf{92.67} 
%     & \underline{86.27} & \underline{87.72} & \textbf{86.00} 
%     & \textbf{82.17} & \textbf{87.78} & \textbf{79.44} 
%     & \textbf{79.75} & \textbf{87.16} & \textbf{75.55} 
%     & \textbf{60.55} & \textbf{83.16} & \textbf{36.67} 
%     \\ 
%  \hline
%         \end{tabular}
%     \end{adjustbox}
%     \caption{Results for Carry Information from Domain One to Next in Multi-domain Dialogs.}
%     \label{tab:multi_api_results}
%     % \vspace{-15pt}
% \end{table*}


% \begin{table*}[!t]
%     \centering
%     \begin{adjustbox}{max width=\textwidth}
%         \begin{tabular}{|c |c|c |c|R c|c|R c|c|R c|c|R c|c|c|}
%             \hline
%             % \multirow{2}{*}{Model} & \multirow{2}{*}{Domains} & Response & Response & Retrieval & Retrieval & Slot Fill&Slot Fill\\ 
%             % & &  BLEU& GLEU& BLEU& GLEU& BLEU& GLEU \\ \hline
%     \multirow{2}{*}{\textbf{Dataset}} & \multirow{2}{*}{\textbf{Model}} & \multicolumn{3}{R}{\textbf{Api Invoke Accuracy}} & \multicolumn{3}{R}{\textbf{Api Method Accuracy}} & \multicolumn{3}{R}{\textbf{Param Names Accuracy}} & \multicolumn{3}{R}{\textbf{Param Values Accuracy}} & \multicolumn{3}{c|}{\textbf{Full API Accuracy}} \\ \cline{3-17}
%     & & \textbf{mixed}& \textbf{seen} & \textbf{unseen} & \textbf{mixed} &  \textbf{seen} & \textbf{unseen}& \textbf{mixed}  & \textbf{seen} & \textbf{unseen}& \textbf{mixed}  & \textbf{seen} & \textbf{unseen}& \textbf{mixed}  & \textbf{seen} & \textbf{unseen}   \\ \hline
%     \multirow{4}{*}{SGD} & {\gptt} & 68.12 & 73.42& 67.77
%     & 66.01 & 72.07& 64.80
%     & 66.31& 68.99 & 64.65
%     & 62.69 & 65.19 & 58.88
%     & 41.83 & 40.77 & 33.66
%     \\ 
%     & \gpt & 93.95 & 97.75 & 89.06
%     & 84.57 & 97.07 & 73.76
%     & 63.01 & 95.61 & 44.97
%     & 61.60 & 94.52 & 42.59 
%     & 42.07 & 87.39 & 19.32
%     \\ 
    
%     & \flan & \textbf{99.00}  & \underline{98.87} & \textbf{98.42} 
%     & \textbf{97.97} & \underline{98.87} & \textbf{95.05} 
%     & \underline{83.18} & \underline{98.47} & \underline{72.96} 
%     & \underline{81.18} & \underline{96.32} & \underline{68.36} 
%     & \underline{56.39} & \underline{92.12} & \underline{35.09} 
%     \\
%     & \opt & \underline{96.87} & \textbf{99.55} & \underline{94.98} 
%     & \underline{95.33} & \textbf{99.55} & \underline{90.80} 
%     & \textbf{90.16} & \textbf{99.19} & \textbf{86.38} 
%     & \textbf{88.30} & \textbf{98.77} & \textbf{82.10} 
%     & \textbf{66.85} & \textbf{96.40} & \textbf{46.91} 
    
%     \\ 
%     \hline


% %KETOD
%     \multirow{4}{*}{KETOD} & {\gptt} & 61.20 & 61.05 & 66.00
%     & 58.47 & 58.60 & 66.67 
%     & 59.89 & 58.10 & \underline{64.27}
%     & 56.28& 55.20 & 57.80 
%     & 34.97& 36.84 & 26.67 
%     \\ 
    
%     & \gpt & 84.15  & 85.26 & 81.33 
%     & 78.96 & 83.51 & 68.67 
%     & 57.76 & 79.28 & 35.37 
%     & 56.09 & 77.53 & 33.22 
%     & 36.34 & 68.07 & 12.00 
%     \\ 

%     & \flan & \textbf{89.34}  & \textbf{88.07} & \underline{90.00} 
%     & \textbf{87.70} & \textbf{88.07} & \underline{83.33} 
%     & \underline{70.89} & \underline{87.11} & 62.96 
%     &  \underline{68.77} & \underline{85.87} & \underline{59.72} 
%     &  \underline{46.72} & \underline{82.46} & \underline{30.00}
%     \\ 
    
%     & \opt& \underline{87.70} & \underline{87.72} & \textbf{92.67} 
%     & \underline{85.25} & \underline{87.72} & \textbf{86.00} 
%     &  \textbf{78.92} & \textbf{87.78} & \textbf{79.44} 
%     &  \textbf{75.70} & \textbf{87.16} & \textbf{75.55} 
%     &  \textbf{52.73} & \textbf{83.16} & \textbf{36.67} 
%     \\ 
%  \hline
%         \end{tabular}
%     \end{adjustbox}
%     \caption{Results for Carry Information from Domain One to Next in Multi-domain Dialogs for Mixed Domains.}
%     \label{tab:multi_api_mixed_results}
%     % \vspace{-15pt}
% \end{table*}


% Table~\ref{tab:multi_api_results} presents the results of our analysis on multi-domain API calls, which are a critical aspect of evaluating task-oriented dialogue systems across different domains. In this context, a multi-domain API call occurs when, within a single dialogue, multiple API calls are made, and the API call domains in the current turn differ from those in the initial API call. This scenario tests a model's ability to handle complex dialogues that span multiple domains, reflecting a more realistic and challenging use case.

% Our results demonstrate that the OPT 6.7B model consistently outperforms other models in handling multi-domain API calls in both the KETOD and SGD datasets. This superior performance highlights the model's capability to effectively manage domain shifts within a dialogue, ensuring accurate and contextually appropriate responses across different domains. 
% %The results emphasize the importance of robust multi-domain support in task-oriented dialogue systems, particularly in scenarios where users may switch contexts or request information across various services within a single interaction.


% \section{Dialog Examples}
\label{sec:appendix_dialogs}

Table~\ref{tab:rest_dialog} shows an example dialog in the Restaurant domain. The table contains the turn id, user utterance, gold response, {\soloist}, {\autotod}, {\gpt}, {\llamai}, and {\flan} response. 
Text highlighted in red outlines the portions where the system response by a model is incorrect and green highlights the correct parts. Texts highlighted in orange indicate that the model is partially correct and is missing some information.

In the example dialog, we can see that the responses of {\autotod} is longer and more descriptive, whereas {\oursys} models produce succinct responses, similar to the gold standards. This observation indicates that {\oursys} was able to pick up the text generation style of the gold standard, which is an important aspect for ToD systems.

The baseline {\soloist} fails to generate a valid response on a few turns. This stems from the complex structure of the output, which consists of dialog states, actions and system response. The output structure has a certain format and different parts are identified with special tokens. On many occasions, the model does not produce the correct tokens, thus leading to invalid generations. 

On turns where the user asks about multiple information, the {\gpt} and {\soloist} model struggles to respond to all the user requests. However, all the other models do not have this problem. This could be attributed to the fact that {\gpt} models are the smallest model when compared to all the other models.

{\llamai} and {\flan} generates responses that are very similar to the gold standard. These models are able to pickup the generation style and can respond to multiple requests of the user. In the example dialog, the user provided information about the restaurant name, time and date on turn 1, but then changed the restaurant name on turn 2. {\llamai} and {\flan} models were able to identify what changed and moved forward with the {\apicall} by using all the correct information. However, {\autotod} was not able to keep track of all the changes and made an incorrect {\apicall}. {\soloist} was able to make a partially correct {\apicall} since it was not able to generalize to the \texttt{number\_of\_seats} slots. It used the \texttt{party\_size} slot that it had seen during training. 

This dialog example showed some of the drawbacks of baseline approaches and some example cases of where they struggle. For {\oursys} models, the limitations of {\gpt} was evident, as it was not able to process multiple requests of the user. However, {\llamai} and {\flan} had performance similar to the gold standard across multiple turns, which falls in line to our automatic metrics and human evaluation.

% \small
\begin{longtable}{>{\raggedright}p{2cm}|p{13cm}}
% \begin{longtable}{r|l}
    \caption{Dialog Example---SGD dataset. Dialog id 1\_00001} \label{tab:rest_dialog}\\  
    \hline
    \textbf{Model} & \textbf{Response---SGD dataset. Dialog id 1\_00001}  \\  
    \hline
    \endfirsthead

    % Header on subsequent pages
    \hline
    \textbf{Dataset} & \textbf{Response---SGD dataset. Dialog id 1\_00001 }  \\  
    \hline
    \endhead

    % Footer for all pages except the last
    \hline
    \textbf{Continued on next page...} & \\  
    \hline
    \endfoot

    % Footer for the last page
    \hline
    \endlastfoot


    % Table rows with long text
Turn & 1 \\ \hline
            User & Can you book a table for me at the Ancient Szechuan for the 11th of this month at 11:30 am?  \\ 
            Gold & In which city are you trying to book the table? \\ 
            \soloist & \mycolor{BrickRed}{Please confirm the following details: Booking a table for 2 at Ancient Szechuan in San Francisco on March 11th at 11:30 am.} \\ 
            \autotod & \mycolor{ForestGreen}{Could you please provide the city where the restaurant `Ancient Szechuan'' is located and the number of seats you would like to reserve?}
 \\
            \gpt & \mycolor{BrickRed}{Sure, I have a table for 2 at Ancient Szechuan in San Francisco at 11:30 am today.}
 \\ 
            \llamai & \mycolor{ForestGreen}{What is the location of the restaurant?} \\
            \flan &  \mycolor{ForestGreen}{Where should I look for the restaurant?}
\\ 
      \hline      
            Turn & 2 \\ \hline
            User &  Can you book a table at the Butterfly restaurant in San Francisco?  \\ 
            Gold & Please confirm that you need a table for 2 at the Butterfly Restaurant in San Francisco at 11:30 am on March 11th. \\ 
            \soloist & Can you please confirm that you want to book a \mycolor{ForestGreen}{table for 2 at the Butterfly restaurant in San Francisco at 11:30 am on March 11th?}\\ 
            \autotod &  \mycolor{BurntOrange}{Could you please confirm the date and time for the reservation at the Butterfly restaurant in San Francisco? Additionally, let me know the number of seats you would like to reserve.}
\\
            \gpt & \mycolor{BrickRed}{Can you please confirm that you want to book a table for 2 people at the Butterfly restaurant in San Francisco at 11:30 am on March 11th?}
 \\ 
            \llamai & Can you confirm the following details: You want to book a \mycolor{ForestGreen}{table at the butterfly Hotel in San Francisco at 11:30 am. You want 2 seats. You want to book the table on March 11th.} \\

            \flan & You wish to book a \mycolor{ForestGreen}{table for 2 at the Butterfly Restaurant in San Francisco for March 11th at 11:30 am.} Is this correct?
 \\ 
            \hline
            Turn & 3 \\ \hline
            Gold & ApiCall(method=`ReserveRestaurant', parameters= `date': `2019-03-11', `location': `San Francisco', `number\_of\_seats': `2',`restaurant\_name': `Butterfly Restaurant', `time': `11:30' )\\ 
            \soloist & ApiCall(method='ReserveRestaurant', parameters={`city': 'San Francisco', `date': `2019-03-11', \mycolor{BurntOrange}{`party\_size'}: `2',`restaurant\_name': `The Butterfly Restaurant', `time': `11:30'}) \\ 
            \autotod & ApiCall(method=`FindRestaurants',parameters=\mycolor{BrickRed}{`category': `Butterfly'}, `location': `San Francisco')
 \\
            \gpt & ApiCall(method=`ReserveRestaurant', parameters=`date': `2019-03-11', `location': `San Francisco', \mycolor{ForestGreen}{`number\_of\_seats'}: `2',`restaurant\_name': `The Butterfly Restaurant', `time': `11:30') \\
            \llamai & ApiCall(method=`ReserveRestaurant', parameters=`date': `2019-03-11', `location': `San Francisco', \mycolor{ForestGreen}{`number\_of\_seats'}: `2',`restaurant\_name': `Butterfly Restaurant', 'time': `11:30')
\\ 
            \flan & ApiCall(method=`ReserveRestaurant', parameters= `date': `2019-03-11', `location': `San Francisco',`restaurant\_name': `Butterfly Restaurant', \mycolor{ForestGreen}{`number\_of\_seats'}: `2', `time': `11:30' )
 \\ 
\hline            
            Turn & 4 \\ \hline
            User & That's perfect. What cuisine do they offer?  \\ 
            Gold & Your table has been booked successfully and they serve Asian cuisine. \\ 
            \soloist & \mycolor{BurntOrange}{Your table has been booked.} \\ 
            \autotod & It looks like the reservation at the Butterfly Restaurant in San Francisco has been successfully made for March 11th at 11:30 am for 2 people. \mycolor{ForestGreen}{The restaurant offers Asian cuisine.} If you need any more assistance or have any other questions, feel free to ask. Have a great day! Goodbye!
  \\
            \gpt & \mycolor{BurntOrange}{Your table is booked.}
\\
\llamai & Your reservation has been successful. \mycolor{ForestGreen}{They offer Asian cuisine.}
\\
            \flan & Your reservation has been made. \mycolor{ForestGreen}{They serve Asian food.} \\ 

            \hline
            Turn & 5 \\ \hline
            User & Can you get me the user rating of the restaurant? Is it expensive? \\ 
            Gold & Their prices are moderate and they have a user rating of 4.0. \\ 
            \soloist & \failed \\ 
            \autotod & The Butterfly Restaurant in San Francisco has a user \mycolor{ForestGreen}{rating of 4.00. The price range is moderate}, so it's not considered expensive. If you have any more questions or need further assistance, feel free to ask!
\\
            \gpt & \mycolor{Orange}{The rating is 4.0.}
 \\
 \llamai & The restaurant has a \mycolor{ForestGreen}{4.0 rating and is considered moderate in price}.
\\
            \flan &  They have a \mycolor{ForestGreen}{moderate price range and a 4.0 rating}.
\\ 

                        
            Turn & 6 \\ 
            User & Thanks a bunch! \\ 
            Gold & Is there anything else I can do for you? \\ 
            \soloist & \failed \\ 
            \autotod & You're welcome! If you need any more assistance, feel free to ask. Have a great day! Goodbye!
 \\
            \gpt & Is there anything else I can help you with?\\
            \llamai & Do you need anything else? \\
            \flan & Do you need help with anything else?
 \\ 

                
            Turn & 7 \\ 
            User & No. Thank you. That's all I need. \\ 
            Gold & Have a great day! \\ 
            \soloist & Have a great time at the restaurant! \\ 
            \autotod & You're welcome! If you need any more assistance in the future, feel free to reach out. Have a great day! Goodbye!
\\
            \gpt & Have a good one!
\\
\llamai & Have a great day. \\
            \flan & Have a nice day.
 \\ 

\end{longtable}


% \begin{longtable}{>{\raggedright}p{2cm}|p{13cm}}
% % \begin{longtable}{r|l}
%     \caption{Dialog Example---SGD dataset. Dialog id 1\_00001} \\  
%     \hline
%     \textbf{Model} & \textbf{Response---SGD dataset. Dialog id 1\_00001}  \\  
%     \hline
%     \endfirsthead

%     % Header on subsequent pages
%     \hline
%     \textbf{Dataset} & \textbf{Response---SGD dataset. Dialog id 1\_00001 }  \\  
%     \hline
%     \endhead

%     % Footer for all pages except the last
%     \hline
%     \textbf{Continued on next page...} & \\  
%     \hline
%     \endfoot

%     % Footer for the last page
%     \hline
%     \endlastfoot
    
%     % Table content using tabular
%     \begin{tabular}{r|p{13cm}}


%     \end{tabular}

%     \end{longtable}

\end{document}
