% This must be in the first 5 lines to tell arXiv to use pdfLaTeX, which is strongly recommended.
\pdfoutput=1
% In particular, the hyperref package requires pdfLaTeX in order to break URLs across lines.

\documentclass[11pt]{article}

% Change "review" to "final" to generate the final (sometimes called camera-ready) version.
% Change to "preprint" to generate a non-anonymous version with page numbers.
\usepackage[preprint]{acl}

% Standard package includes
\usepackage{times}
\usepackage{latexsym}
\usepackage{balance}
% For proper rendering and hyphenation of words containing Latin characters (including in bib files)
\usepackage[T1]{fontenc}
% For Vietnamese characters
% \usepackage[T5]{fontenc}
% See https://www.latex-project.org/help/documentation/encguide.pdf for other character sets

% This assumes your files are encoded as UTF8
\usepackage[utf8]{inputenc}

% This is not strictly necessary, and may be commented out,
% but it will improve the layout of the manuscript,
% and will typically save some space.
\usepackage{microtype}

% This is also not strictly necessary, and may be commented out.
% However, it will improve the aesthetics of text in
% the typewriter font.
\usepackage{inconsolata}

%Including images in your LaTeX document requires adding
%additional package(s)
\usepackage{graphicx}
\usepackage{adjustbox}
\usepackage{listings}
\usepackage[dvipsnames]{xcolor}
\usepackage{amsmath}
\usepackage{multirow, multicol}
\usepackage{array}
\usepackage{tabularray}
\usepackage{makecell}
\usepackage{bm}

\definecolor{myyellow}{HTML}{eeba0b}


\lstset{
    language=Python,
    basicstyle=\ttfamily,
    keywordstyle=\color{blue},
    stringstyle=\color{orange},
    commentstyle=\color{green!70!black},
    morecomment=[l][\color{magenta}]{\#},
    tabsize=4,
    showspaces=false,
    showstringspaces=false
}

\newcommand{\mymodel}{$\mathsf{NL}\mbox{-}\mathsf{ToD}$}
\newcommand{\opt}{OPT-6.7B}
\newcommand{\flan}{Flan-T5 Large}
\newcommand{\gpt}{GPT-2 Medium}
\newcommand{\gptt}{GPT-3.5-turbo}
\newcommand{\gptf}{GPT-4o}
\newcommand{\wa}{ ++}
\newcommand{\apicall}{\mathsf{API Call}}
% baselines
\newcommand{\simpletod}{SimpleTOD}
\newcommand{\autotod}{$\mathsf{Auto}\mbox{-}\mathsf{ToD}$}
\newcommand{\zstod}{$\mathsf{ZS}\mbox{-}\mathsf{ToD}$}
\newcommand{\soloist}{SOLOIST}

% If the title and author information does not fit in the area allocated, uncomment the following
%
%\setlength\titlebox{<dim>}
%
% and set <dim> to something 5cm or larger.

%\title{Natural Language Task-Oriented Dialog System}

%\title{Zero-Shot Generalizable End-to-End Task-Oriented Dialog System using Unannotated Dialog Data and Domain Schema}

\title{Do We Still Need Turn-level Dialog Annotations For Training Task-Oriented Dialog Systems?}
%Training Zero-Shot Generalizable End-to-End Task-Oriented Dialog System Without Turn-level Dialog Annotations

% Author information can be set in various styles:
% For several authors from the same institution:
% \author{Author 1 \and ... \and Author n \\
%         Address line \\ ... \\ Address line}
% if the names do not fit well on one line use
%         Author 1 \\ {\bf Author 2} \\ ... \\ {\bf Author n} \\
% For authors from different institutions:
% \author{Author 1 \\ Address line \\  ... \\ Address line
%         \And  ... \And
%         Author n \\ Address line \\ ... \\ Address line}
% To start a separate ``row'' of authors use \AND, as in
% \author{Author 1 \\ Address line \\  ... \\ Address line
%         \AND
%         Author 2 \\ Address line \\ ... \\ Address line \And
%         Author 3 \\ Address line \\ ... \\ Address line}

\author{Adib Mosharrof \\
  University of Kentucky \\
  Lexington, KY, USA \\
  \texttt{adib.mosharrof@uky.edu} \\\And
  A.B. Siddique \\
  University of Kentucky \\
  Lexington, KY, USA \\
  \texttt{siddique@cs.uky.edu} \\}


\newcommand{\oursys}{$\mathsf{NL}\mbox{-}\mathsf{TOD}$}

\newcolumntype{L}{!{\vrule width 1.1pt}c}
\newcolumntype{R}{c!{\vrule width 1.1pt}}
\newcommand{\thickhline}{\noalign{\hrule height 1.1pt}}

\begin{document}
\maketitle
\begin{abstract}
    \begin{abstract}  
Test time scaling is currently one of the most active research areas that shows promise after training time scaling has reached its limits.
Deep-thinking (DT) models are a class of recurrent models that can perform easy-to-hard generalization by assigning more compute to harder test samples.
However, due to their inability to determine the complexity of a test sample, DT models have to use a large amount of computation for both easy and hard test samples.
Excessive test time computation is wasteful and can cause the ``overthinking'' problem where more test time computation leads to worse results.
In this paper, we introduce a test time training method for determining the optimal amount of computation needed for each sample during test time.
We also propose Conv-LiGRU, a novel recurrent architecture for efficient and robust visual reasoning. 
Extensive experiments demonstrate that Conv-LiGRU is more stable than DT, effectively mitigates the ``overthinking'' phenomenon, and achieves superior accuracy.
\end{abstract}  
\end{abstract}

\section{Introduction}


\begin{figure}[t]
\centering
\includegraphics[width=0.6\columnwidth]{figures/evaluation_desiderata_V5.pdf}
\vspace{-0.5cm}
\caption{\systemName is a platform for conducting realistic evaluations of code LLMs, collecting human preferences of coding models with real users, real tasks, and in realistic environments, aimed at addressing the limitations of existing evaluations.
}
\label{fig:motivation}
\end{figure}

\begin{figure*}[t]
\centering
\includegraphics[width=\textwidth]{figures/system_design_v2.png}
\caption{We introduce \systemName, a VSCode extension to collect human preferences of code directly in a developer's IDE. \systemName enables developers to use code completions from various models. The system comprises a) the interface in the user's IDE which presents paired completions to users (left), b) a sampling strategy that picks model pairs to reduce latency (right, top), and c) a prompting scheme that allows diverse LLMs to perform code completions with high fidelity.
Users can select between the top completion (green box) using \texttt{tab} or the bottom completion (blue box) using \texttt{shift+tab}.}
\label{fig:overview}
\end{figure*}

As model capabilities improve, large language models (LLMs) are increasingly integrated into user environments and workflows.
For example, software developers code with AI in integrated developer environments (IDEs)~\citep{peng2023impact}, doctors rely on notes generated through ambient listening~\citep{oberst2024science}, and lawyers consider case evidence identified by electronic discovery systems~\citep{yang2024beyond}.
Increasing deployment of models in productivity tools demands evaluation that more closely reflects real-world circumstances~\citep{hutchinson2022evaluation, saxon2024benchmarks, kapoor2024ai}.
While newer benchmarks and live platforms incorporate human feedback to capture real-world usage, they almost exclusively focus on evaluating LLMs in chat conversations~\citep{zheng2023judging,dubois2023alpacafarm,chiang2024chatbot, kirk2024the}.
Model evaluation must move beyond chat-based interactions and into specialized user environments.



 

In this work, we focus on evaluating LLM-based coding assistants. 
Despite the popularity of these tools---millions of developers use Github Copilot~\citep{Copilot}---existing
evaluations of the coding capabilities of new models exhibit multiple limitations (Figure~\ref{fig:motivation}, bottom).
Traditional ML benchmarks evaluate LLM capabilities by measuring how well a model can complete static, interview-style coding tasks~\citep{chen2021evaluating,austin2021program,jain2024livecodebench, white2024livebench} and lack \emph{real users}. 
User studies recruit real users to evaluate the effectiveness of LLMs as coding assistants, but are often limited to simple programming tasks as opposed to \emph{real tasks}~\citep{vaithilingam2022expectation,ross2023programmer, mozannar2024realhumaneval}.
Recent efforts to collect human feedback such as Chatbot Arena~\citep{chiang2024chatbot} are still removed from a \emph{realistic environment}, resulting in users and data that deviate from typical software development processes.
We introduce \systemName to address these limitations (Figure~\ref{fig:motivation}, top), and we describe our three main contributions below.


\textbf{We deploy \systemName in-the-wild to collect human preferences on code.} 
\systemName is a Visual Studio Code extension, collecting preferences directly in a developer's IDE within their actual workflow (Figure~\ref{fig:overview}).
\systemName provides developers with code completions, akin to the type of support provided by Github Copilot~\citep{Copilot}. 
Over the past 3 months, \systemName has served over~\completions suggestions from 10 state-of-the-art LLMs, 
gathering \sampleCount~votes from \userCount~users.
To collect user preferences,
\systemName presents a novel interface that shows users paired code completions from two different LLMs, which are determined based on a sampling strategy that aims to 
mitigate latency while preserving coverage across model comparisons.
Additionally, we devise a prompting scheme that allows a diverse set of models to perform code completions with high fidelity.
See Section~\ref{sec:system} and Section~\ref{sec:deployment} for details about system design and deployment respectively.



\textbf{We construct a leaderboard of user preferences and find notable differences from existing static benchmarks and human preference leaderboards.}
In general, we observe that smaller models seem to overperform in static benchmarks compared to our leaderboard, while performance among larger models is mixed (Section~\ref{sec:leaderboard_calculation}).
We attribute these differences to the fact that \systemName is exposed to users and tasks that differ drastically from code evaluations in the past. 
Our data spans 103 programming languages and 24 natural languages as well as a variety of real-world applications and code structures, while static benchmarks tend to focus on a specific programming and natural language and task (e.g. coding competition problems).
Additionally, while all of \systemName interactions contain code contexts and the majority involve infilling tasks, a much smaller fraction of Chatbot Arena's coding tasks contain code context, with infilling tasks appearing even more rarely. 
We analyze our data in depth in Section~\ref{subsec:comparison}.



\textbf{We derive new insights into user preferences of code by analyzing \systemName's diverse and distinct data distribution.}
We compare user preferences across different stratifications of input data (e.g., common versus rare languages) and observe which affect observed preferences most (Section~\ref{sec:analysis}).
For example, while user preferences stay relatively consistent across various programming languages, they differ drastically between different task categories (e.g. frontend/backend versus algorithm design).
We also observe variations in user preference due to different features related to code structure 
(e.g., context length and completion patterns).
We open-source \systemName and release a curated subset of code contexts.
Altogether, our results highlight the necessity of model evaluation in realistic and domain-specific settings.





\section{Related Works}
\label{sec:related_works}


\noindent\textbf{Diffusion-based Video Generation. }
The advancement of diffusion models \cite{rombach2022high, ramesh2022hierarchical, zheng2022entropy} has led to significant progress in video generation. Due to the scarcity of high-quality video-text datasets \cite{Blattmann2023, Blattmann2023a}, researchers have adapted existing text-to-image (T2I) models to facilitate text-to-video (T2V) generation. Notable examples include AnimateDiff \cite{Guo2023}, Align your Latents \cite{Blattmann2023a}, PYoCo \cite{ge2023preserve}, and Emu Video \cite{girdhar2023emu}. Further advancements, such as LVDM \cite{he2022latent}, VideoCrafter \cite{chen2023videocrafter1, chen2024videocrafter2}, ModelScope \cite{wang2023modelscope}, LAVIE \cite{wang2023lavie}, and VideoFactory \cite{wang2024videofactory}, have refined these approaches by fine-tuning both spatial and temporal blocks, leveraging T2I models for initialization to improve video quality.
Recently, Sora \cite{brooks2024video} and CogVideoX \cite{yang2024cogvideox} enhance video generation by introducing Transformer-based diffusion backbones \cite{Peebles2023, Ma2024, Yu2024} and utilizing 3D-VAE, unlocking the potential for realistic world simulators. Additionally, SVD \cite{Blattmann2023}, SEINE \cite{chen2023seine}, PixelDance \cite{zeng2024make} and PIA \cite{zhang2024pia} have made significant strides in image-to-video generation, achieving notable improvements in quality and flexibility.
Further, I2VGen-XL \cite{zhang2023i2vgen}, DynamicCrafter \cite{Xing2023}, and Moonshot \cite{zhang2024moonshot} incorporate additional cross-attention layers to strengthen conditional signals during generation.



\noindent\textbf{Controllable Generation.}
Controllable generation has become a central focus in both image \citep{Zhang2023,jiang2024survey, Mou2024, Zheng2023, peng2024controlnext, ye2023ip, wu2024spherediffusion, song2024moma, wu2024ifadapter} and video \citep{gong2024atomovideo, zhang2024moonshot, guo2025sparsectrl, jiang2024videobooth} generation, enabling users to direct the output through various types of control. A wide range of controllable inputs has been explored, including text descriptions, pose \citep{ma2024follow,wang2023disco,hu2024animate,xu2024magicanimate}, audio \citep{tang2023anytoany,tian2024emo,he2024co}, identity representations \citep{chefer2024still,wang2024customvideo,wu2024customcrafter}, trajectory \citep{yin2023dragnuwa,chen2024motion,li2024generative,wu2024motionbooth, namekata2024sg}.


\noindent\textbf{Text-based Camera Control.}
Text-based camera control methods use natural language descriptions to guide camera motion in video generation. AnimateDiff \cite{Guo2023} and SVD \cite{Blattmann2023} fine-tune LoRAs \cite{hu2021lora} for specific camera movements based on text input. 
Image conductor\cite{li2024image} proposed to separate different camera and object motions through camera LoRA weight and object LoRA weight to achieve more precise motion control.
In contrast, MotionMaster \cite{hu2024motionmaster} and Peekaboo \cite{jain2024peekaboo} offer training-free approaches for generating coarse-grained camera motions, though with limited precision. VideoComposer \cite{wang2024videocomposer} adjusts pixel-level motion vectors to provide finer control, but challenges remain in achieving precise camera control.

\noindent\textbf{Trajectory-based Camera Control.}
MotionCtrl \cite{Wang2024Motionctrl}, CameraCtrl \cite{He2024Cameractrl}, and Direct-a-Video \cite{yang2024direct} use camera pose as input to enhance control, while CVD \cite{kuang2024collaborative} extends CameraCtrl for multi-view generation, though still limited by motion complexity. To improve geometric consistency, Pose-guided diffusion \cite{tseng2023consistent}, CamCo \cite{Xu2024}, and CamI2V \cite{zheng2024cami2v} apply epipolar constraints for consistent viewpoints. VD3D \cite{bahmani2024vd3d} introduces a ControlNet\cite{Zhang2023}-like conditioning mechanism with spatiotemporal camera embeddings, enabling more precise control.
CamTrol \cite{hou2024training} offers a training-free approach that renders static point clouds into multi-view frames for video generation. Cavia \cite{xu2024cavia} introduces view-integrated attention mechanisms to improve viewpoint and temporal consistency, while I2VControl-Camera \cite{feng2024i2vcontrol} refines camera movement by employing point trajectories in the camera coordinate system. Despite these advancements, challenges in maintaining camera control and scene-scale consistency remain, which our method seeks to address. It is noted that 4Dim~\cite{watson2024controlling} introduces absolute scale but in  4D novel view synthesis (NVS) of scenes.



\section{Research Methodology}~\label{sec:Methodology}

In this section, we discuss the process of conducting our systematic review, e.g., our search strategy for data extraction of relevant studies, based on the guidelines of Kitchenham et al.~\cite{kitchenham2022segress} to conduct SLRs and Petersen et al.~\cite{PETERSEN20151} to conduct systematic mapping studies (SMSs) in Software Engineering. In this systematic review, we divide our work into a four-stage procedure, including planning, conducting, building a taxonomy, and reporting the review, illustrated in Fig.~\ref{fig:search}. The four stages are as follows: (1) the \emph{planning} stage involved identifying research questions (RQs) and specifying the detailed research plan for the study; (2) the \emph{conducting} stage involved analyzing and synthesizing the existing primary studies to answer the research questions; (3) the \emph{taxonomy} stage was introduced to optimize the data extraction results and consolidate a taxonomy schema for REDAST methodology; (4) the \emph{reporting} stage involved the reviewing, concluding and reporting the final result of our study.

\begin{figure}[!t]
    \centering
    \includegraphics[width=1\linewidth]{fig/methodology/searching-process.drawio.pdf}
    \caption{Systematic Literature Review Process}
    \label{fig:search}
\end{figure}

\subsection{Research Questions}
In this study, we developed five research questions (RQs) to identify the input and output, analyze technologies, evaluate metrics, identify challenges, and identify potential opportunities. 

\textbf{RQ1. What are the input configurations, formats, and notations used in the requirements in requirements-driven
automated software testing?} In requirements-driven testing, the input is some form of requirements specification -- which can vary significantly. RQ1 maps the input for REDAST and reports on the comparison among different formats for requirements specification.

\textbf{RQ2. What are the frameworks, tools, processing methods, and transformation techniques used in requirements-driven automated software testing studies?} RQ2 explores the technical solutions from requirements to generated artifacts, e.g., rule-based transformation applying natural language processing (NLP) pipelines and deep learning (DL) techniques, where we additionally discuss the potential intermediate representation and additional input for the transformation process.

\textbf{RQ3. What are the test formats and coverage criteria used in the requirements-driven automated software
testing process?} RQ3 focuses on identifying the formulation of generated artifacts (i.e., the final output). We map the adopted test formats and analyze their characteristics in the REDAST process.

\textbf{RQ4. How do existing studies evaluate the generated test artifacts in the requirements-driven automated software testing process?} RQ4 identifies the evaluation datasets, metrics, and case study methodologies in the selected papers. This aims to understand how researchers assess the effectiveness, accuracy, and practical applicability of the generated test artifacts.

\textbf{RQ5. What are the limitations and challenges of existing requirements-driven automated software testing methods in the current era?} RQ5 addresses the limitations and challenges of existing studies while exploring future directions in the current era of technology development. %It particularly highlights the potential benefits of advanced LLMs and examines their capacity to meet the high expectations placed on these cutting-edge language modeling technologies. %\textcolor{blue}{CA: Do we really need to focus on LLMs? TBD.} \textcolor{orange}{FW: About LLMs, I removed the direct emphase in RQ5 but kept the discussion in RQ5 and the solution section. I think that would be more appropriate.}

\subsection{Searching Strategy}

The overview of the search process is exhibited in Fig. \ref{fig:papers}, which includes all the details of our search steps.
\begin{table}[!ht]
\caption{List of Search Terms}
\label{table:search_term}
\begin{tabularx}{\textwidth}{lX}
\hline
\textbf{Terms Group} & \textbf{Terms} \\ \hline
Test Group & test* \\
Requirement Group & requirement* OR use case* OR user stor* OR specification* \\
Software Group & software* OR system* \\
Method Group & generat* OR deriv* OR map* OR creat* OR extract* OR design* OR priorit* OR construct* OR transform* \\ \hline
\end{tabularx}
\end{table}

\begin{figure}
    \centering
    \includegraphics[width=1\linewidth]{fig/methodology/search-papers.drawio.pdf}
    \caption{Study Search Process}
    \label{fig:papers}
\end{figure}

\subsubsection{Search String Formulation}
Our research questions (RQs) guided the identification of the main search terms. We designed our search string with generic keywords to avoid missing out on any related papers, where four groups of search terms are included, namely ``test group'', ``requirement group'', ``software group'', and ``method group''. In order to capture all the expressions of the search terms, we use wildcards to match the appendix of the word, e.g., ``test*'' can capture ``testing'', ``tests'' and so on. The search terms are listed in Table~\ref{table:search_term}, decided after iterative discussion and refinement among all the authors. As a result, we finally formed the search string as follows:


\hangindent=1.5em
 \textbf{ON ABSTRACT} ((``test*'') \textbf{AND} (``requirement*'' \textbf{OR} ``use case*'' \textbf{OR} ``user stor*'' \textbf{OR} ``specifications'') \textbf{AND} (``software*'' \textbf{OR} ``system*'') \textbf{AND} (``generat*'' \textbf{OR} ``deriv*'' \textbf{OR} ``map*'' \textbf{OR} ``creat*'' \textbf{OR} ``extract*'' \textbf{OR} ``design*'' \textbf{OR} ``priorit*'' \textbf{OR} ``construct*'' \textbf{OR} ``transform*''))

The search process was conducted in September 2024, and therefore, the search results reflect studies available up to that date. We conducted the search process on six online databases: IEEE Xplore, ACM Digital Library, Wiley, Scopus, Web of Science, and Science Direct. However, some databases were incompatible with our default search string in the following situations: (1) unsupported for searching within abstract, such as Scopus, and (2) limited search terms, such as ScienceDirect. Here, for (1) situation, we searched within the title, keyword, and abstract, and for (2) situation, we separately executed the search and removed the duplicate papers in the merging process. 

\subsubsection{Automated Searching and Duplicate Removal}
We used advanced search to execute our search string within our selected databases, following our designed selection criteria in Table \ref{table:selection}. The first search returned 27,333 papers. Specifically for the duplicate removal, we used a Python script to remove (1) overlapped search results among multiple databases and (2) conference or workshop papers, also found with the same title and authors in the other journals. After duplicate removal, we obtained 21,652 papers for further filtering.

\begin{table*}[]
\caption{Selection Criteria}
\label{table:selection}
\begin{tabularx}{\textwidth}{lX}
\hline
\textbf{Criterion ID} & \textbf{Criterion Description} \\ \hline
S01          & Papers written in English. \\
S02-1        & Papers in the subjects of "Computer Science" or "Software Engineering". \\
S02-2        & Papers published on software testing-related issues. \\
S03          & Papers published from 1991 to the present. \\ 
S04          & Papers with accessible full text. \\ \hline
\end{tabularx}
\end{table*}

\begin{table*}[]
\small
\caption{Inclusion and Exclusion Criteria}
\label{table:criteria}
\begin{tabularx}{\textwidth}{lX}
\hline
\textbf{ID}  & \textbf{Description} \\ \hline
\multicolumn{2}{l}{\textbf{Inclusion Criteria}} \\ \hline
I01 & Papers about requirements-driven automated system testing or acceptance testing generation, or studies that generate system-testing-related artifacts. \\
I02 & Peer-reviewed studies that have been used in academia with references from literature. \\ \hline
\multicolumn{2}{l}{\textbf{Exclusion Criteria}} \\ \hline
E01 & Studies that only support automated code generation, but not test-artifact generation. \\
E02 & Studies that do not use requirements-related information as an input. \\
E03 & Papers with fewer than 5 pages (1-4 pages). \\
E04 & Non-primary studies (secondary or tertiary studies). \\
E05 & Vision papers and grey literature (unpublished work), books (chapters), posters, discussions, opinions, keynotes, magazine articles, experience, and comparison papers. \\ \hline
\end{tabularx}
\end{table*}

\subsubsection{Filtering Process}

In this step, we filtered a total of 21,652 papers using the inclusion and exclusion criteria outlined in Table \ref{table:criteria}. This process was primarily carried out by the first and second authors. Our criteria are structured at different levels, facilitating a multi-step filtering process. This approach involves applying various criteria in three distinct phases. We employed a cross-verification method involving (1) the first and second authors and (2) the other authors. Initially, the filtering was conducted separately by the first and second authors. After cross-verifying their results, the results were then reviewed and discussed further by the other authors for final decision-making. We widely adopted this verification strategy within the filtering stages. During the filtering process, we managed our paper list using a BibTeX file and categorized the papers with color-coding through BibTeX management software\footnote{\url{https://bibdesk.sourceforge.io/}}, i.e., “red” for irrelevant papers, “yellow” for potentially relevant papers, and “blue” for relevant papers. This color-coding system facilitated the organization and review of papers according to their relevance.

The screening process is shown below,
\begin{itemize}
    \item \textbf{1st-round Filtering} was based on the title and abstract, using the criteria I01 and E01. At this stage, the number of papers was reduced from 21,652 to 9,071.
    \item \textbf{2nd-round Filtering}. We attempted to include requirements-related papers based on E02 on the title and abstract level, which resulted from 9,071 to 4,071 papers. We excluded all the papers that did not focus on requirements-related information as an input or only mentioned the term ``requirements'' but did not refer to the requirements specification.
    \item \textbf{3rd-round Filtering}. We selectively reviewed the content of papers identified as potentially relevant to requirements-driven automated test generation. This process resulted in 162 papers for further analysis.
\end{itemize}
Note that, especially for third-round filtering, we aimed to include as many relevant papers as possible, even borderline cases, according to our criteria. The results were then discussed iteratively among all the authors to reach a consensus.

\subsubsection{Snowballing}

Snowballing is necessary for identifying papers that may have been missed during the automated search. Following the guidelines by Wohlin~\cite{wohlin2014guidelines}, we conducted both forward and backward snowballing. As a result, we identified 24 additional papers through this process.

\subsubsection{Data Extraction}

Based on the formulated research questions (RQs), we designed 38 data extraction questions\footnote{\url{https://drive.google.com/file/d/1yjy-59Juu9L3WHaOPu-XQo-j-HHGTbx_/view?usp=sharing}} and created a Google Form to collect the required information from the relevant papers. The questions included 30 short-answer questions, six checkbox questions, and two selection questions. The data extraction was organized into five sections: (1) basic information: fundamental details such as title, author, venue, etc.; (2) open information: insights on motivation, limitations, challenges, etc.; (3) requirements: requirements format, notation, and related aspects; (4) methodology: details, including immediate representation and technique support; (5) test-related information: test format(s), coverage, and related elements. Similar to the filtering process, the first and second authors conducted the data extraction and then forwarded the results to the other authors to initiate the review meeting.

\subsubsection{Quality Assessment}

During the data extraction process, we encountered papers with insufficient information. To address this, we conducted a quality assessment in parallel to ensure the relevance of the papers to our objectives. This approach, also adopted in previous secondary studies~\cite{shamsujjoha2021developing, naveed2024model}, involved designing a set of assessment questions based on guidelines by Kitchenham et al.~\cite{kitchenham2022segress}. The quality assessment questions in our study are shown below:
\begin{itemize}
    \item \textbf{QA1}. Does this study clearly state \emph{how} requirements drive automated test generation?
    \item \textbf{QA2}. Does this study clearly state the \emph{aim} of REDAST?
    \item \textbf{QA3}. Does this study enable \emph{automation} in test generation?
    \item \textbf{QA4}. Does this study demonstrate the usability of the method from the perspective of methodology explanation, discussion, case examples, and experiments?
\end{itemize}
QA4 originates from an open perspective in the review process, where we focused on evaluation, discussion, and explanation. Our review also examined the study’s overall structure, including the methodology description, case studies, experiments, and analyses. The detailed results of the quality assessment are provided in the Appendix. Following this assessment, the final data extraction was based on 156 papers.

% \begin{table}[]
% \begin{tabular}{ll}
% \hline
% QA ID & QA Questions                                             \\ \hline
% Q01   & Does this study clearly state its aims?                  \\
% Q02   & Does this study clearly describe its methodology?        \\
% Q03   & Does this study involve automated test generation?       \\
% Q04   & Does this study include a promising evaluation?          \\
% Q05   & Does this study demonstrate the usability of the method? \\ \hline
% \end{tabular}%
% \caption{Questions for Quality Assessment}
% \label{table:qa}
% \end{table}

% automated quality assessment

% \textcolor{blue}{CA: Our search strategy focused on identifying requirements types first. We covered several sources, e.g., ~\cite{Pohl:11,wagner2019status} to identify different formats and notations of specifying requirements. However, this came out to be a long list, e.g., free-form NL requirements, semi-formal UML models, free-from textual use case models, UML class diagrams, UML activity diagrams, and so on. In this paper, we attempted to primarily focus on requirements-related aspects and not design-level information. Hence, we generalised our search string to include generic keywords, e.g., requirement*, use case*, and user stor*. We did so to avoid missing out on any papers, bringing too restrictive in our search strategy, and not creating a too-generic search string with all the aforementioned formats to avoid getting results beyond our review's scope.}


%% Use \subsection commands to start a subsection.



%\subsection{Study Selection}

% In this step, we further looked into the content of searched papers using our search strategy and applied our inclusion and exclusion criteria. Our filtering strategy aimed to pinpoint studies focused on requirements-driven system-level testing. Recognizing the presence of irrelevant papers in our search results, we established detailed selection criteria for preliminary inclusion and exclusion, as shown in Table \ref{table: criteria}. Specifically, we further developed the taxonomy schema to exclude two types of studies that did not meet the requirements for system-level testing: (1) studies supporting specification-driven test generation, such as UML-driven test generation, rather than requirements-driven testing, and (2) studies focusing on code-based test generation, such as requirement-driven code generation for unit testing.




\section{Experiments}
\label{sec:experiments}
The experiments are designed to address two key research questions.
First, \textbf{RQ1} evaluates whether the average $L_2$-norm of the counterfactual perturbation vectors ($\overline{||\perturb||}$) decreases as the model overfits the data, thereby providing further empirical validation for our hypothesis.
Second, \textbf{RQ2} evaluates the ability of the proposed counterfactual regularized loss, as defined in (\ref{eq:regularized_loss2}), to mitigate overfitting when compared to existing regularization techniques.

% The experiments are designed to address three key research questions. First, \textbf{RQ1} investigates whether the mean perturbation vector norm decreases as the model overfits the data, aiming to further validate our intuition. Second, \textbf{RQ2} explores whether the mean perturbation vector norm can be effectively leveraged as a regularization term during training, offering insights into its potential role in mitigating overfitting. Finally, \textbf{RQ3} examines whether our counterfactual regularizer enables the model to achieve superior performance compared to existing regularization methods, thus highlighting its practical advantage.

\subsection{Experimental Setup}
\textbf{\textit{Datasets, Models, and Tasks.}}
The experiments are conducted on three datasets: \textit{Water Potability}~\cite{kadiwal2020waterpotability}, \textit{Phomene}~\cite{phomene}, and \textit{CIFAR-10}~\cite{krizhevsky2009learning}. For \textit{Water Potability} and \textit{Phomene}, we randomly select $80\%$ of the samples for the training set, and the remaining $20\%$ for the test set, \textit{CIFAR-10} comes already split. Furthermore, we consider the following models: Logistic Regression, Multi-Layer Perceptron (MLP) with 100 and 30 neurons on each hidden layer, and PreactResNet-18~\cite{he2016cvecvv} as a Convolutional Neural Network (CNN) architecture.
We focus on binary classification tasks and leave the extension to multiclass scenarios for future work. However, for datasets that are inherently multiclass, we transform the problem into a binary classification task by selecting two classes, aligning with our assumption.

\smallskip
\noindent\textbf{\textit{Evaluation Measures.}} To characterize the degree of overfitting, we use the test loss, as it serves as a reliable indicator of the model's generalization capability to unseen data. Additionally, we evaluate the predictive performance of each model using the test accuracy.

\smallskip
\noindent\textbf{\textit{Baselines.}} We compare CF-Reg with the following regularization techniques: L1 (``Lasso''), L2 (``Ridge''), and Dropout.

\smallskip
\noindent\textbf{\textit{Configurations.}}
For each model, we adopt specific configurations as follows.
\begin{itemize}
\item \textit{Logistic Regression:} To induce overfitting in the model, we artificially increase the dimensionality of the data beyond the number of training samples by applying a polynomial feature expansion. This approach ensures that the model has enough capacity to overfit the training data, allowing us to analyze the impact of our counterfactual regularizer. The degree of the polynomial is chosen as the smallest degree that makes the number of features greater than the number of data.
\item \textit{Neural Networks (MLP and CNN):} To take advantage of the closed-form solution for computing the optimal perturbation vector as defined in (\ref{eq:opt-delta}), we use a local linear approximation of the neural network models. Hence, given an instance $\inst_i$, we consider the (optimal) counterfactual not with respect to $\model$ but with respect to:
\begin{equation}
\label{eq:taylor}
    \model^{lin}(\inst) = \model(\inst_i) + \nabla_{\inst}\model(\inst_i)(\inst - \inst_i),
\end{equation}
where $\model^{lin}$ represents the first-order Taylor approximation of $\model$ at $\inst_i$.
Note that this step is unnecessary for Logistic Regression, as it is inherently a linear model.
\end{itemize}

\smallskip
\noindent \textbf{\textit{Implementation Details.}} We run all experiments on a machine equipped with an AMD Ryzen 9 7900 12-Core Processor and an NVIDIA GeForce RTX 4090 GPU. Our implementation is based on the PyTorch Lightning framework. We use stochastic gradient descent as the optimizer with a learning rate of $\eta = 0.001$ and no weight decay. We use a batch size of $128$. The training and test steps are conducted for $6000$ epochs on the \textit{Water Potability} and \textit{Phoneme} datasets, while for the \textit{CIFAR-10} dataset, they are performed for $200$ epochs.
Finally, the contribution $w_i^{\varepsilon}$ of each training point $\inst_i$ is uniformly set as $w_i^{\varepsilon} = 1~\forall i\in \{1,\ldots,m\}$.

The source code implementation for our experiments is available at the following GitHub repository: \url{https://anonymous.4open.science/r/COCE-80B4/README.md} 

\subsection{RQ1: Counterfactual Perturbation vs. Overfitting}
To address \textbf{RQ1}, we analyze the relationship between the test loss and the average $L_2$-norm of the counterfactual perturbation vectors ($\overline{||\perturb||}$) over training epochs.

In particular, Figure~\ref{fig:delta_loss_epochs} depicts the evolution of $\overline{||\perturb||}$ alongside the test loss for an MLP trained \textit{without} regularization on the \textit{Water Potability} dataset. 
\begin{figure}[ht]
    \centering
    \includegraphics[width=0.85\linewidth]{img/delta_loss_epochs.png}
    \caption{The average counterfactual perturbation vector $\overline{||\perturb||}$ (left $y$-axis) and the cross-entropy test loss (right $y$-axis) over training epochs ($x$-axis) for an MLP trained on the \textit{Water Potability} dataset \textit{without} regularization.}
    \label{fig:delta_loss_epochs}
\end{figure}

The plot shows a clear trend as the model starts to overfit the data (evidenced by an increase in test loss). 
Notably, $\overline{||\perturb||}$ begins to decrease, which aligns with the hypothesis that the average distance to the optimal counterfactual example gets smaller as the model's decision boundary becomes increasingly adherent to the training data.

It is worth noting that this trend is heavily influenced by the choice of the counterfactual generator model. In particular, the relationship between $\overline{||\perturb||}$ and the degree of overfitting may become even more pronounced when leveraging more accurate counterfactual generators. However, these models often come at the cost of higher computational complexity, and their exploration is left to future work.

Nonetheless, we expect that $\overline{||\perturb||}$ will eventually stabilize at a plateau, as the average $L_2$-norm of the optimal counterfactual perturbations cannot vanish to zero.

% Additionally, the choice of employing the score-based counterfactual explanation framework to generate counterfactuals was driven to promote computational efficiency.

% Future enhancements to the framework may involve adopting models capable of generating more precise counterfactuals. While such approaches may yield to performance improvements, they are likely to come at the cost of increased computational complexity.


\subsection{RQ2: Counterfactual Regularization Performance}
To answer \textbf{RQ2}, we evaluate the effectiveness of the proposed counterfactual regularization (CF-Reg) by comparing its performance against existing baselines: unregularized training loss (No-Reg), L1 regularization (L1-Reg), L2 regularization (L2-Reg), and Dropout.
Specifically, for each model and dataset combination, Table~\ref{tab:regularization_comparison} presents the mean value and standard deviation of test accuracy achieved by each method across 5 random initialization. 

The table illustrates that our regularization technique consistently delivers better results than existing methods across all evaluated scenarios, except for one case -- i.e., Logistic Regression on the \textit{Phomene} dataset. 
However, this setting exhibits an unusual pattern, as the highest model accuracy is achieved without any regularization. Even in this case, CF-Reg still surpasses other regularization baselines.

From the results above, we derive the following key insights. First, CF-Reg proves to be effective across various model types, ranging from simple linear models (Logistic Regression) to deep architectures like MLPs and CNNs, and across diverse datasets, including both tabular and image data. 
Second, CF-Reg's strong performance on the \textit{Water} dataset with Logistic Regression suggests that its benefits may be more pronounced when applied to simpler models. However, the unexpected outcome on the \textit{Phoneme} dataset calls for further investigation into this phenomenon.


\begin{table*}[h!]
    \centering
    \caption{Mean value and standard deviation of test accuracy across 5 random initializations for different model, dataset, and regularization method. The best results are highlighted in \textbf{bold}.}
    \label{tab:regularization_comparison}
    \begin{tabular}{|c|c|c|c|c|c|c|}
        \hline
        \textbf{Model} & \textbf{Dataset} & \textbf{No-Reg} & \textbf{L1-Reg} & \textbf{L2-Reg} & \textbf{Dropout} & \textbf{CF-Reg (ours)} \\ \hline
        Logistic Regression   & \textit{Water}   & $0.6595 \pm 0.0038$   & $0.6729 \pm 0.0056$   & $0.6756 \pm 0.0046$  & N/A    & $\mathbf{0.6918 \pm 0.0036}$                     \\ \hline
        MLP   & \textit{Water}   & $0.6756 \pm 0.0042$   & $0.6790 \pm 0.0058$   & $0.6790 \pm 0.0023$  & $0.6750 \pm 0.0036$    & $\mathbf{0.6802 \pm 0.0046}$                    \\ \hline
%        MLP   & \textit{Adult}   & $0.8404 \pm 0.0010$   & $\mathbf{0.8495 \pm 0.0007}$   & $0.8489 \pm 0.0014$  & $\mathbf{0.8495 \pm 0.0016}$     & $0.8449 \pm 0.0019$                    \\ \hline
        Logistic Regression   & \textit{Phomene}   & $\mathbf{0.8148 \pm 0.0020}$   & $0.8041 \pm 0.0028$   & $0.7835 \pm 0.0176$  & N/A    & $0.8098 \pm 0.0055$                     \\ \hline
        MLP   & \textit{Phomene}   & $0.8677 \pm 0.0033$   & $0.8374 \pm 0.0080$   & $0.8673 \pm 0.0045$  & $0.8672 \pm 0.0042$     & $\mathbf{0.8718 \pm 0.0040}$                    \\ \hline
        CNN   & \textit{CIFAR-10} & $0.6670 \pm 0.0233$   & $0.6229 \pm 0.0850$   & $0.7348 \pm 0.0365$   & N/A    & $\mathbf{0.7427 \pm 0.0571}$                     \\ \hline
    \end{tabular}
\end{table*}

\begin{table*}[htb!]
    \centering
    \caption{Hyperparameter configurations utilized for the generation of Table \ref{tab:regularization_comparison}. For our regularization the hyperparameters are reported as $\mathbf{\alpha/\beta}$.}
    \label{tab:performance_parameters}
    \begin{tabular}{|c|c|c|c|c|c|c|}
        \hline
        \textbf{Model} & \textbf{Dataset} & \textbf{No-Reg} & \textbf{L1-Reg} & \textbf{L2-Reg} & \textbf{Dropout} & \textbf{CF-Reg (ours)} \\ \hline
        Logistic Regression   & \textit{Water}   & N/A   & $0.0093$   & $0.6927$  & N/A    & $0.3791/1.0355$                     \\ \hline
        MLP   & \textit{Water}   & N/A   & $0.0007$   & $0.0022$  & $0.0002$    & $0.2567/1.9775$                    \\ \hline
        Logistic Regression   &
        \textit{Phomene}   & N/A   & $0.0097$   & $0.7979$  & N/A    & $0.0571/1.8516$                     \\ \hline
        MLP   & \textit{Phomene}   & N/A   & $0.0007$   & $4.24\cdot10^{-5}$  & $0.0015$    & $0.0516/2.2700$                    \\ \hline
       % MLP   & \textit{Adult}   & N/A   & $0.0018$   & $0.0018$  & $0.0601$     & $0.0764/2.2068$                    \\ \hline
        CNN   & \textit{CIFAR-10} & N/A   & $0.0050$   & $0.0864$ & N/A    & $0.3018/
        2.1502$                     \\ \hline
    \end{tabular}
\end{table*}

\begin{table*}[htb!]
    \centering
    \caption{Mean value and standard deviation of training time across 5 different runs. The reported time (in seconds) corresponds to the generation of each entry in Table \ref{tab:regularization_comparison}. Times are }
    \label{tab:times}
    \begin{tabular}{|c|c|c|c|c|c|c|}
        \hline
        \textbf{Model} & \textbf{Dataset} & \textbf{No-Reg} & \textbf{L1-Reg} & \textbf{L2-Reg} & \textbf{Dropout} & \textbf{CF-Reg (ours)} \\ \hline
        Logistic Regression   & \textit{Water}   & $222.98 \pm 1.07$   & $239.94 \pm 2.59$   & $241.60 \pm 1.88$  & N/A    & $251.50 \pm 1.93$                     \\ \hline
        MLP   & \textit{Water}   & $225.71 \pm 3.85$   & $250.13 \pm 4.44$   & $255.78 \pm 2.38$  & $237.83 \pm 3.45$    & $266.48 \pm 3.46$                    \\ \hline
        Logistic Regression   & \textit{Phomene}   & $266.39 \pm 0.82$ & $367.52 \pm 6.85$   & $361.69 \pm 4.04$  & N/A   & $310.48 \pm 0.76$                    \\ \hline
        MLP   &
        \textit{Phomene} & $335.62 \pm 1.77$   & $390.86 \pm 2.11$   & $393.96 \pm 1.95$ & $363.51 \pm 5.07$    & $403.14 \pm 1.92$                     \\ \hline
       % MLP   & \textit{Adult}   & N/A   & $0.0018$   & $0.0018$  & $0.0601$     & $0.0764/2.2068$                    \\ \hline
        CNN   & \textit{CIFAR-10} & $370.09 \pm 0.18$   & $395.71 \pm 0.55$   & $401.38 \pm 0.16$ & N/A    & $1287.8 \pm 0.26$                     \\ \hline
    \end{tabular}
\end{table*}

\subsection{Feasibility of our Method}
A crucial requirement for any regularization technique is that it should impose minimal impact on the overall training process.
In this respect, CF-Reg introduces an overhead that depends on the time required to find the optimal counterfactual example for each training instance. 
As such, the more sophisticated the counterfactual generator model probed during training the higher would be the time required. However, a more advanced counterfactual generator might provide a more effective regularization. We discuss this trade-off in more details in Section~\ref{sec:discussion}.

Table~\ref{tab:times} presents the average training time ($\pm$ standard deviation) for each model and dataset combination listed in Table~\ref{tab:regularization_comparison}.
We can observe that the higher accuracy achieved by CF-Reg using the score-based counterfactual generator comes with only minimal overhead. However, when applied to deep neural networks with many hidden layers, such as \textit{PreactResNet-18}, the forward derivative computation required for the linearization of the network introduces a more noticeable computational cost, explaining the longer training times in the table.

\subsection{Hyperparameter Sensitivity Analysis}
The proposed counterfactual regularization technique relies on two key hyperparameters: $\alpha$ and $\beta$. The former is intrinsic to the loss formulation defined in (\ref{eq:cf-train}), while the latter is closely tied to the choice of the score-based counterfactual explanation method used.

Figure~\ref{fig:test_alpha_beta} illustrates how the test accuracy of an MLP trained on the \textit{Water Potability} dataset changes for different combinations of $\alpha$ and $\beta$.

\begin{figure}[ht]
    \centering
    \includegraphics[width=0.85\linewidth]{img/test_acc_alpha_beta.png}
    \caption{The test accuracy of an MLP trained on the \textit{Water Potability} dataset, evaluated while varying the weight of our counterfactual regularizer ($\alpha$) for different values of $\beta$.}
    \label{fig:test_alpha_beta}
\end{figure}

We observe that, for a fixed $\beta$, increasing the weight of our counterfactual regularizer ($\alpha$) can slightly improve test accuracy until a sudden drop is noticed for $\alpha > 0.1$.
This behavior was expected, as the impact of our penalty, like any regularization term, can be disruptive if not properly controlled.

Moreover, this finding further demonstrates that our regularization method, CF-Reg, is inherently data-driven. Therefore, it requires specific fine-tuning based on the combination of the model and dataset at hand.

\begin{table*}[t]
\centering
\fontsize{11pt}{11pt}\selectfont
\begin{tabular}{lllllllllllll}
\toprule
\multicolumn{1}{c}{\textbf{task}} & \multicolumn{2}{c}{\textbf{Mir}} & \multicolumn{2}{c}{\textbf{Lai}} & \multicolumn{2}{c}{\textbf{Ziegen.}} & \multicolumn{2}{c}{\textbf{Cao}} & \multicolumn{2}{c}{\textbf{Alva-Man.}} & \multicolumn{1}{c}{\textbf{avg.}} & \textbf{\begin{tabular}[c]{@{}l@{}}avg.\\ rank\end{tabular}} \\
\multicolumn{1}{c}{\textbf{metrics}} & \multicolumn{1}{c}{\textbf{cor.}} & \multicolumn{1}{c}{\textbf{p-v.}} & \multicolumn{1}{c}{\textbf{cor.}} & \multicolumn{1}{c}{\textbf{p-v.}} & \multicolumn{1}{c}{\textbf{cor.}} & \multicolumn{1}{c}{\textbf{p-v.}} & \multicolumn{1}{c}{\textbf{cor.}} & \multicolumn{1}{c}{\textbf{p-v.}} & \multicolumn{1}{c}{\textbf{cor.}} & \multicolumn{1}{c}{\textbf{p-v.}} &  &  \\ \midrule
\textbf{S-Bleu} & 0.50 & 0.0 & 0.47 & 0.0 & 0.59 & 0.0 & 0.58 & 0.0 & 0.68 & 0.0 & 0.57 & 5.8 \\
\textbf{R-Bleu} & -- & -- & 0.27 & 0.0 & 0.30 & 0.0 & -- & -- & -- & -- & - &  \\
\textbf{S-Meteor} & 0.49 & 0.0 & 0.48 & 0.0 & 0.61 & 0.0 & 0.57 & 0.0 & 0.64 & 0.0 & 0.56 & 6.1 \\
\textbf{R-Meteor} & -- & -- & 0.34 & 0.0 & 0.26 & 0.0 & -- & -- & -- & -- & - &  \\
\textbf{S-Bertscore} & \textbf{0.53} & 0.0 & {\ul 0.80} & 0.0 & \textbf{0.70} & 0.0 & {\ul 0.66} & 0.0 & {\ul0.78} & 0.0 & \textbf{0.69} & \textbf{1.7} \\
\textbf{R-Bertscore} & -- & -- & 0.51 & 0.0 & 0.38 & 0.0 & -- & -- & -- & -- & - &  \\
\textbf{S-Bleurt} & {\ul 0.52} & 0.0 & {\ul 0.80} & 0.0 & 0.60 & 0.0 & \textbf{0.70} & 0.0 & \textbf{0.80} & 0.0 & {\ul 0.68} & {\ul 2.3} \\
\textbf{R-Bleurt} & -- & -- & 0.59 & 0.0 & -0.05 & 0.13 & -- & -- & -- & -- & - &  \\
\textbf{S-Cosine} & 0.51 & 0.0 & 0.69 & 0.0 & {\ul 0.62} & 0.0 & 0.61 & 0.0 & 0.65 & 0.0 & 0.62 & 4.4 \\
\textbf{R-Cosine} & -- & -- & 0.40 & 0.0 & 0.29 & 0.0 & -- & -- & -- & -- & - & \\ \midrule
\textbf{QuestEval} & 0.23 & 0.0 & 0.25 & 0.0 & 0.49 & 0.0 & 0.47 & 0.0 & 0.62 & 0.0 & 0.41 & 9.0 \\
\textbf{LLaMa3} & 0.36 & 0.0 & \textbf{0.84} & 0.0 & {\ul{0.62}} & 0.0 & 0.61 & 0.0 &  0.76 & 0.0 & 0.64 & 3.6 \\
\textbf{our (3b)} & 0.49 & 0.0 & 0.73 & 0.0 & 0.54 & 0.0 & 0.53 & 0.0 & 0.7 & 0.0 & 0.60 & 5.8 \\
\textbf{our (8b)} & 0.48 & 0.0 & 0.73 & 0.0 & 0.52 & 0.0 & 0.53 & 0.0 & 0.7 & 0.0 & 0.59 & 6.3 \\  \bottomrule
\end{tabular}
\caption{Pearson correlation on human evaluation on system output. `R-': reference-based. `S-': source-based.}
\label{tab:sys}
\end{table*}



\begin{table}%[]
\centering
\fontsize{11pt}{11pt}\selectfont
\begin{tabular}{llllll}
\toprule
\multicolumn{1}{c}{\textbf{task}} & \multicolumn{1}{c}{\textbf{Lai}} & \multicolumn{1}{c}{\textbf{Zei.}} & \multicolumn{1}{c}{\textbf{Scia.}} & \textbf{} & \textbf{} \\ 
\multicolumn{1}{c}{\textbf{metrics}} & \multicolumn{1}{c}{\textbf{cor.}} & \multicolumn{1}{c}{\textbf{cor.}} & \multicolumn{1}{c}{\textbf{cor.}} & \textbf{avg.} & \textbf{\begin{tabular}[c]{@{}l@{}}avg.\\ rank\end{tabular}} \\ \midrule
\textbf{S-Bleu} & 0.40 & 0.40 & 0.19* & 0.33 & 7.67 \\
\textbf{S-Meteor} & 0.41 & 0.42 & 0.16* & 0.33 & 7.33 \\
\textbf{S-BertS.} & {\ul0.58} & 0.47 & 0.31 & 0.45 & 3.67 \\
\textbf{S-Bleurt} & 0.45 & {\ul 0.54} & {\ul 0.37} & 0.45 & {\ul 3.33} \\
\textbf{S-Cosine} & 0.56 & 0.52 & 0.3 & {\ul 0.46} & {\ul 3.33} \\ \midrule
\textbf{QuestE.} & 0.27 & 0.35 & 0.06* & 0.23 & 9.00 \\
\textbf{LlaMA3} & \textbf{0.6} & \textbf{0.67} & \textbf{0.51} & \textbf{0.59} & \textbf{1.0} \\
\textbf{Our (3b)} & 0.51 & 0.49 & 0.23* & 0.39 & 4.83 \\
\textbf{Our (8b)} & 0.52 & 0.49 & 0.22* & 0.43 & 4.83 \\ \bottomrule
\end{tabular}
\caption{Pearson correlation on human ratings on reference output. *not significant; we cannot reject the null hypothesis of zero correlation}
\label{tab:ref}
\end{table}


\begin{table*}%[]
\centering
\fontsize{11pt}{11pt}\selectfont
\begin{tabular}{lllllllll}
\toprule
\textbf{task} & \multicolumn{1}{c}{\textbf{ALL}} & \multicolumn{1}{c}{\textbf{sentiment}} & \multicolumn{1}{c}{\textbf{detoxify}} & \multicolumn{1}{c}{\textbf{catchy}} & \multicolumn{1}{c}{\textbf{polite}} & \multicolumn{1}{c}{\textbf{persuasive}} & \multicolumn{1}{c}{\textbf{formal}} & \textbf{\begin{tabular}[c]{@{}l@{}}avg. \\ rank\end{tabular}} \\
\textbf{metrics} & \multicolumn{1}{c}{\textbf{cor.}} & \multicolumn{1}{c}{\textbf{cor.}} & \multicolumn{1}{c}{\textbf{cor.}} & \multicolumn{1}{c}{\textbf{cor.}} & \multicolumn{1}{c}{\textbf{cor.}} & \multicolumn{1}{c}{\textbf{cor.}} & \multicolumn{1}{c}{\textbf{cor.}} &  \\ \midrule
\textbf{S-Bleu} & -0.17 & -0.82 & -0.45 & -0.12* & -0.1* & -0.05 & -0.21 & 8.42 \\
\textbf{R-Bleu} & - & -0.5 & -0.45 &  &  &  &  &  \\
\textbf{S-Meteor} & -0.07* & -0.55 & -0.4 & -0.01* & 0.1* & -0.16 & -0.04* & 7.67 \\
\textbf{R-Meteor} & - & -0.17* & -0.39 & - & - & - & - & - \\
\textbf{S-BertScore} & 0.11 & -0.38 & -0.07* & -0.17* & 0.28 & 0.12 & 0.25 & 6.0 \\
\textbf{R-BertScore} & - & -0.02* & -0.21* & - & - & - & - & - \\
\textbf{S-Bleurt} & 0.29 & 0.05* & 0.45 & 0.06* & 0.29 & 0.23 & 0.46 & 4.2 \\
\textbf{R-Bleurt} & - &  0.21 & 0.38 & - & - & - & - & - \\
\textbf{S-Cosine} & 0.01* & -0.5 & -0.13* & -0.19* & 0.05* & -0.05* & 0.15* & 7.42 \\
\textbf{R-Cosine} & - & -0.11* & -0.16* & - & - & - & - & - \\ \midrule
\textbf{QuestEval} & 0.21 & {\ul{0.29}} & 0.23 & 0.37 & 0.19* & 0.35 & 0.14* & 4.67 \\
\textbf{LlaMA3} & \textbf{0.82} & \textbf{0.80} & \textbf{0.72} & \textbf{0.84} & \textbf{0.84} & \textbf{0.90} & \textbf{0.88} & \textbf{1.00} \\
\textbf{Our (3b)} & 0.47 & -0.11* & 0.37 & 0.61 & 0.53 & 0.54 & 0.66 & 3.5 \\
\textbf{Our (8b)} & {\ul{0.57}} & 0.09* & {\ul 0.49} & {\ul 0.72} & {\ul 0.64} & {\ul 0.62} & {\ul 0.67} & {\ul 2.17} \\ \bottomrule
\end{tabular}
\caption{Pearson correlation on human ratings on our constructed test set. 'R-': reference-based. 'S-': source-based. *not significant; we cannot reject the null hypothesis of zero correlation}
\label{tab:con}
\end{table*}

\section{Results}
We benchmark the different metrics on the different datasets using correlation to human judgement. For content preservation, we show results split on data with system output, reference output and our constructed test set: we show that the data source for evaluation leads to different conclusions on the metrics. In addition, we examine whether the metrics can rank style transfer systems similar to humans. On style strength, we likewise show correlations between human judgment and zero-shot evaluation approaches. When applicable, we summarize results by reporting the average correlation. And the average ranking of the metric per dataset (by ranking which metric obtains the highest correlation to human judgement per dataset). 

\subsection{Content preservation}
\paragraph{How do data sources affect the conclusion on best metric?}
The conclusions about the metrics' performance change radically depending on whether we use system output data, reference output, or our constructed test set. Ideally, a good metric correlates highly with humans on any data source. Ideally, for meta-evaluation, a metric should correlate consistently across all data sources, but the following shows that the correlations indicate different things, and the conclusion on the best metric should be drawn carefully.

Looking at the metrics correlations with humans on the data source with system output (Table~\ref{tab:sys}), we see a relatively high correlation for many of the metrics on many tasks. The overall best metrics are S-BertScore and S-BLEURT (avg+avg rank). We see no notable difference in our method of using the 3B or 8B model as the backbone.

Examining the average correlations based on data with reference output (Table~\ref{tab:ref}), now the zero-shoot prompting with LlaMA3 70B is the best-performing approach ($0.59$ avg). Tied for second place are source-based cosine embedding ($0.46$ avg), BLEURT ($0.45$ avg) and BertScore ($0.45$ avg). Our method follows on a 5. place: here, the 8b version (($0.43$ avg)) shows a bit stronger results than 3b ($0.39$ avg). The fact that the conclusions change, whether looking at reference or system output, confirms the observations made by \citet{scialom-etal-2021-questeval} on simplicity transfer.   

Now consider the results on our test set (Table~\ref{tab:con}): Several metrics show low or no correlation; we even see a significantly negative correlation for some metrics on ALL (BLEU) and for specific subparts of our test set for BLEU, Meteor, BertScore, Cosine. On the other end, LlaMA3 70B is again performing best, showing strong results ($0.82$ in ALL). The runner-up is now our 8B method, with a gap to the 3B version ($0.57$ vs $0.47$ in ALL). Note our method still shows zero correlation for the sentiment task. After, ranks BLEURT ($0.29$), QuestEval ($0.21$), BertScore ($0.11$), Cosine ($0.01$).  

On our test set, we find that some metrics that correlate relatively well on the other datasets, now exhibit low correlation. Hence, with our test set, we can now support the logical reasoning with data evidence: Evaluation of content preservation for style transfer needs to take the style shift into account. This conclusion could not be drawn using the existing data sources: We hypothesise that for the data with system-based output, successful output happens to be very similar to the source sentence and vice versa, and reference-based output might not contain server mistakes as they are gold references. Thus, none of the existing data sources tests the limits of the metrics.  


\paragraph{How do reference-based metrics compare to source-based ones?} Reference-based metrics show a lower correlation than the source-based counterpart for all metrics on both datasets with ratings on references (Table~\ref{tab:sys}). As discussed previously, reference-based metrics for style transfer have the drawback that many different good solutions on a rewrite might exist and not only one similar to a reference.


\paragraph{How well can the metrics rank the performance of style transfer methods?}
We compare the metrics' ability to judge the best style transfer methods w.r.t. the human annotations: Several of the data sources contain samples from different style transfer systems. In order to use metrics to assess the quality of the style transfer system, metrics should correctly find the best-performing system. Hence, we evaluate whether the metrics for content preservation provide the same system ranking as human evaluators. We take the mean of the score for every output on each system and the mean of the human annotations; we compare the systems using the Kendall's Tau correlation. 

We find only the evaluation using the dataset Mir, Lai, and Ziegen to result in significant correlations, probably because of sparsity in a number of system tests (App.~\ref{app:dataset}). Our method (8b) is the only metric providing a perfect ranking of the style transfer system on the Lai data, and Llama3 70B the only one on the Ziegen data. Results in App.~\ref{app:results}. 


\subsection{Style strength results}
%Evaluating style strengths is a challenging task. 
Llama3 70B shows better overall results than our method. However, our method scores higher than Llama3 70B on 2 out of 6 datasets, but it also exhibits zero correlation on one task (Table~\ref{tab:styleresults}).%More work i s needed on evaluating style strengths. 
 
\begin{table}%[]
\fontsize{11pt}{11pt}\selectfont
\begin{tabular}{lccc}
\toprule
\multicolumn{1}{c}{\textbf{}} & \textbf{LlaMA3} & \textbf{Our (3b)} & \textbf{Our (8b)} \\ \midrule
\textbf{Mir} & 0.46 & 0.54 & \textbf{0.57} \\
\textbf{Lai} & \textbf{0.57} & 0.18 & 0.19 \\
\textbf{Ziegen.} & 0.25 & 0.27 & \textbf{0.32} \\
\textbf{Alva-M.} & \textbf{0.59} & 0.03* & 0.02* \\
\textbf{Scialom} & \textbf{0.62} & 0.45 & 0.44 \\
\textbf{\begin{tabular}[c]{@{}l@{}}Our Test\end{tabular}} & \textbf{0.63} & 0.46 & 0.48 \\ \bottomrule
\end{tabular}
\caption{Style strength: Pearson correlation to human ratings. *not significant; we cannot reject the null hypothesis of zero corelation}
\label{tab:styleresults}
\end{table}

\subsection{Ablation}
We conduct several runs of the methods using LLMs with variations in instructions/prompts (App.~\ref{app:method}). We observe that the lower the correlation on a task, the higher the variation between the different runs. For our method, we only observe low variance between the runs.
None of the variations leads to different conclusions of the meta-evaluation. Results in App.~\ref{app:results}.
\section{Conclusion}
In this work, we propose a simple yet effective approach, called SMILE, for graph few-shot learning with fewer tasks. Specifically, we introduce a novel dual-level mixup strategy, including within-task and across-task mixup, for enriching the diversity of nodes within each task and the diversity of tasks. Also, we incorporate the degree-based prior information to learn expressive node embeddings. Theoretically, we prove that SMILE effectively enhances the model's generalization performance. Empirically, we conduct extensive experiments on multiple benchmarks and the results suggest that SMILE significantly outperforms other baselines, including both in-domain and cross-domain few-shot settings.
\section{Limitation}
The use of 3D-printed PLA for structural components improves improving ease of assembly and reduces weight and cost, yet it causes deformation under heavy load, which can diminish end-effector precision. Using metal, such as aluminum, would remedy this problem. Additionally, \robot relies on integrated joint relative encoders, requiring manual initialization in a fixed joint configuration each time the system is powered on. Using absolute joint encoders could significantly improve accuracy and ease of use, although it would increase the overall cost. 

%Reliance on commercially available actuators simplifies integration but imposes constraints on control frequency and customization, further limiting the potential for tailored performance improvements.

% The 6 DoF configuration provides sufficient mobility for most tasks; however, certain bimanual operations could benefit from an additional degree of freedom to handle complex joint constraints more effectively. Furthermore, the limited torque density of commercially available proprioceptive actuators restricts the payload and torque output, making the system less suitability for handling heavier loads or high-torque applications. 

The 6 DoF configuration of the arm provides sufficient mobility for single-arm manipulation tasks, yet it shows a limitation in certain bimanual manipulation problems. Specifically, when \robot holds onto a rigid object with both hands, each arm loses 1 DoF because the hands are fixed to the object during grasping. This leads to an underactuated kinematic chain which has a limited mobility in 3D space. We can achieve more mobility by letting the object slip inside the grippers, yet this renders the grasp less robust and simulation difficult. Therefore, we anticipate that designing a lightweight 3 DoF wrist in place of the current 2 DoF wrist allows a more diverse repertoire of manipulation in bimanual tasks.

Finally, the limited torque density of commercially available proprioceptive actuators restricts the performance. Currently, all of our actuators feature a 1:10 gear ratio, so \robot can handle up to 2.5 kg of payload. To handle a heavier object and manipulate it with higher torque, we expect the actuator to have 1:20$\sim$30 gear ratio, but it is difficult to find an off-the-shelf product that meets our requirements. Customizing the actuator to increase the torque density while minimizing the weight will enable \robot to move faster and handle more diverse objects.

%These constraints highlight opportunities for improvement in future iterations, including alternative materials for enhanced rigidity, custom actuator designs for higher control precision and torque density, the adoption of absolute joint encoders, and optimized configurations to balance dexterity and weight.



% Bibliography entries for the entire Anthology, followed by custom entries
%\bibliography{anthology,custom}
% Custom bibliography entries only
\bibliography{custom}

\appendix

% \section{Example Appendix}
% \label{sec:appendix}

% This is an appendix.
% \section{Template for multi-task instruction fine-tuning}
\label{sec:templates}

Figure~\ref{fig:finetuning_template} shows the template used to process the inputs for {\oursys}. The template first informs about the domains involved in the dialog. Next, it provides task-specific instructions and provides the domain schemas. The dialog history is provided and the model is instructed to generate a system response based on the dialog history, search results, and the task instructions.


% \begin{figure*}
%     \centering
%     \includegraphics[width=0.97\linewidth]{assets/finetuning_template.pdf}
%     \caption{
% Multi-task instruction finetuning template}
%     \label{fig:finetuning_template}
% \end{figure*}

\begin{figure}[htbp]
    \centering
\begin{tcolorbox}[colframe=gray!20, colback=gray!10, coltitle=black, arc=5mm, boxrule=0.4mm, width=\columnwidth, left=3mm, right=3mm, top=3mm, bottom=3mm]
\setlength{\baselineskip}{1.5em}
You are an expert chat assistant for the domains: \textcolor{cyan!90}{[domains]}. \\
\textcolor{purple!90}{Instructions:} As an expert, you must generate the most appropriate response for the chat assistant. \\
The response can be an api call or a response to the user. \\
Based on the \textcolor{purple!90}{Last User Utterance}, you must find the relevant \textcolor{purple!90}{Intent} from the \textcolor{purple!90}{Schema} and your request should use the \textcolor{purple!90}{required slots} and \textcolor{purple!90}{optional slots} from that \textcolor{purple!90}{Intent}. \\
You will be provided with the Schema for domains: \textcolor{cyan!90}{[domains]}\\
\textcolor{cyan!90}{[schemas]}\\
You will be provided an incomplete dialog between a user and a chat assistant, and an optional search results.\\
\textcolor{purple!90}{Dialog History}: \textcolor{cyan!90}{[dialog history]}\\
Using the \textcolor{purple!90}{Dialog History}, \textcolor{purple!90}{Search Results}, and by following the \textcolor{purple!90}{Instructions} please generate the response for the chat assistant.
\end{tcolorbox}
    \caption{Multi-task instruction finetuning template. Items in blue are dynamic elements and those in purple are important aspects of the prompt.}
    \label{fig:finetuning_template}
\end{figure}
% \input{my_files/appendix_gleu_mixed}
% \section{Results to Carry Information from Domain One to Next in Multi-domain Dialogs}
\label{sec:multi_dom_api}

Table~\ref{tab:multi_api_mixed_results} presents the results for Carry Information from Domain One to Next in Multi-domain Dialogs. 
For this setting, we only have results on the SGD and KETOD datasets. We can see that for the key metric, Full API Accuracy, {\opt} achieves the best results across both datasets. {\opt} also achieves the highest scores for the Parameter Names Accuracy and Parameter Values Accuracy metrics as well. However, for the API Invoke Accuracy and API Method Accuracy, on some settings, {\flan} outperforms {\opt}.
Upon comparing the key metric, Full API Accuracy, of single domain results from Table~\ref{tab:api_results} with that of Multi-domain Dialogs, we can see that the performance of the models is better in the Multi-domain dialogs. A possible explanation of this could be that a bulk of the multi-domain dialogs contain common and generic domains like \texttt{Events}, \texttt{Restaurants}, and \texttt{Hotels}.
The dataset contains a lot of example dialogs from these domains, which in turn could have been the reason behind the improved performance. 

We can see that apart from the API Invoke Accuracy and API Method Accuracy, {\opt} is the model that produces the best results for the remaining metrics across both results. {\flan} is the second best performing model, and also achieves best results on the two metrics mentioned before. A notable observation on the key metric, Full Api Accuracy, is that the performance for mixed domain settings is in between that of the seen and unseen domain settings. This is the expected result, as mixed domains contain domains from both seen and unseen domains.

\section{Example Dialogs with Responses from All Models}
\label{sec:example_dialogs}
% Table~\ref{tab:rest_dialog} shows an example dialog in the Restaurant domain. It contains the turn id, user utterance, gold response, \gptt response, and {\flan} response. Texts highlighted in red outlines the portions where the system response by a model is incorrect and green highlights the correct parts.

% Table~\ref{tab:multi_domain_dialog} shows an example multi-domain dialog in the Buses and RideShare domain. Model utterances in red mark the mistakes made and green highlights the correct parts. We can see that \gptt has a common mistake of making an API Call before confirming with the user, whereas {\flan} does not have such an issue. Also, we can see that the responses of \gptt are longer and more descriptive, whereas {\flan} produces short and succint responses, similar to the gold standards.

Table~\ref{tab:rest_dialog} shows an example dialog in the Restaurant domain and Table~\ref{tab:multi_domain_dialog} shows an example multi-domain dialog in the Buses and \texttt{RideShare} domain. The table contains the turn id, user utterance, gold response, {\gptt}, {\gptf}, {\gpt}, {\flan}, and {\opt} response. Text highlighted in red outlines the portions where the system response by a model is incorrect and green highlights the correct parts. Texts highlighted in orange indicate that the model is partially correct and is missing some information. Texts highlighted in yellow indicate that the system response seems like a valid response but is incorrect because the gold standard is expecting a different response. 

From the example dialogs, we can see that {\gptt} and {\gptf} has a common mistake of making an API Call without confirming with the user, whereas {\oursys} models do not make this mistake. Also, we can see that the responses of {\gptt} and {\gptf} are longer and more descriptive, whereas {\oursys} models produce succinct responses, similar to the gold standards. This observation indicates that {\oursys} was able to pick up the text generation style of the gold standard, which is an important aspect for TOD systems.

We can see that {\gpt} hallucinates and makes up information on its own, but {\flan} and {\opt} do not suffer from hallucination as much. For the API call task, all the models face similar problems of missing certain parameters and in the multi-domain setting, the models fail to adapt to the slots of the new domain. For response generation, on turns 1 in Table~\ref{tab:multi_domain_dialog}, we can see the request sub-task challenge as discussed in Section~\ref{sec:response_generation}, where the gold standard expects a certain slot but the models request a different slot. Another observation from turn 2 of that dialog is that the gold standard expects the model to request 3 slots, but the {\oursys} models are only asking about a single slot. A reason for this could be that there fewer turns where the gold standard expects multiple slots to be requested, so the models picked this pattern and asked about one slot.

On turn 4 from Table~\ref{tab:rest_dialog}, we can see that {\gpt} only mentioned that the restaurant was booked but did not mention the cuisine of the restaurant, whereas all other models did not make this mistake. Similar examples exist on a few other turns, where {\gpt} misses important information.

\begin{figure*}
   \centering
   % \includegraphics[width=0.95\linewidth]{assets/finetuning_template.pdf}
   \includegraphics[width=0.95\linewidth]{assets/finetuning_template.pdf}
   
   \caption{
       Template used to generate the prompt for fine-tuning {\oursys}.
   }
   
   \label{fig:finetuning_template}
\end{figure*}

\begin{figure*}
   \centering
   \includegraphics[width=0.95\linewidth]{assets/chatgpt_template.pdf}
% \includegraphics[width=0.95\linewidth]{assets/chatgpt_template.pdf}   
   \caption{
       Template used to generate the prompt for OpenAI's GPT Models. 
   }
   
   \label{fig:chatgpt_template}
\end{figure*}


\begin{table*}[!t]
    \centering
    \small
    \begin{adjustbox}{max width=\textwidth}
        \begin{tabular}{|c|c|c|c|c| R c|c|c|c|}
            \hline
    \multirow{2}{*}{\textbf{Dataset}}&\multirow{2}{*}{\textbf{Model}} & \multicolumn{4}{R}{\textbf{Overall Response (BLEU-4)}} & \multicolumn{4}{c|}{\textbf{Overall Response (GLEU-4)}}  \\ \cline{3-10}
    & & \textbf{all}  & \textbf{seen} & \textbf{unseen} & \textbf{mixed} & \textbf{all}  & \textbf{seen} & \textbf{unseen} & \textbf{mixed} \\ \hline
    \multirow{4}{*}{SGD}&{\gptt} & 0.0397& 0.0422 & 0.0363 & 0.043
    &  0.0555& 0.0570 & 0.0522 & 0.0589
    \\ 
    &\gpt& 0.2276& 0.2423 & 0.2095 & 0.2436
    &  0.2240& 0.2430 & 0.2060 & 0.2395
    \\ 
    &\flan& \underline{0.2680}& \underline{0.2980} & \underline{0.2500} & \underline{0.2790}
    &  \underline{0.2610}& \underline{0.2960} & \underline{0.2440} & \underline{0.2721}
    \\ 
    &\opt& \textbf{0.3063}& \textbf{0.3311} & \textbf{0.2896} & \textbf{0.3174}
    &  \textbf{0.2996}& \textbf{0.3210} & \textbf{0.2841} & \textbf{0.3115} 
 \\  \hline
    
    % KETOD
    \multirow{4}{*}{KETOD}&{\gptt} &0.0397& 0.0412 & 0.0334 & 0.0438
    & 0.0554& 0.0563 & 0.0494 & 0.0601
    \\ 
    &\gpt&0.1509& 0.1756 & 0.1224 & 0.1507
    & 0.1552& 0.1751 & 0.1303 & 0.1568
    \\ 
    &\flan& \underline{0.2130}  & \underline{0.2503} & \underline{0.1794} & \underline{0.2040}
    & \underline{0.2091} & \underline{0.2408} & \underline{0.1825} & \underline{0.2000}
    \\ 
    &\opt&\textbf{0.2680}  & \textbf{0.2994} & \textbf{0.2371} & \textbf{0.2559}
    & \textbf{0.2534} & \textbf{0.2860} & \textbf{0.2248} & \textbf{0.2446}
 \\  \hline

    % BITOD
        \multirow{4}{*}{BiToD}&{\gptt} &0.0646  & 0.0646 & - & -
    & 0.0760 & 0.0760 & - & -
    \\ 
    &\gpt&\underline{0.3466}  & \underline{0.3466} & - & -
    & \underline{0.3016} & \underline{0.3016} & - & -
    \\ 
    &\flan& 0.3159  & 0.3159 & - & -
    & 0.2640 & 0.2640 & - & -
    \\ 
    &\opt&\textbf{0.4143}& \textbf{0.4143} & - & -
    & \textbf{0.3803}& \textbf{0.3803} & - & -
 \\  \hline
    
        \end{tabular}
    \end{adjustbox}
    \caption{Overall Response Generation: BLEU \& GLEU scores for system generations with additional mixed domain setting.}
    \label{tab:response_gleu_mixed}
    % \vspace{-15pt}
\end{table*}



\begin{table*}[!t]
    \centering
    \small
    \begin{adjustbox}{max width=\textwidth}
        \begin{tabular}{|c|c|c|c|c| R c|c|c|c|}
            \hline
    \multirow{2}{*}{\textbf{Dataset}}&\multirow{2}{*}{\textbf{Model}} & \multicolumn{4}{R}{\textbf{Inform (BLEU-4)}} & \multicolumn{4}{c|}{\textbf{Inform (GLEU-4)}}  \\ \cline{3-10}
    & & \textbf{all}  & \textbf{seen} & \textbf{unseen} & \textbf{mixed} & \textbf{all}  & \textbf{seen} & \textbf{unseen} & \textbf{mixed} \\ \hline
    \multirow{4}{*}{SGD}&{\gptt} &0.1047& 0.0998 & 0.1012 & 0.1094
    & 0.1563& 0.1505 & 0.1523 & 0.1616
    \\ 
    &\gpt& 0.2737& 0.3022& 0.2479 & 0.2913
    &  0.3644& 0.3880& 0.3421 & 0.3799
    \\ 
    &\flan& \underline{0.2991}& \underline{0.3064} & \underline{0.2872} & \underline{0.3099}
    &  \underline{0.3879}& \underline{0.3960} & \underline{0.3775} & \underline{0.3968}
    \\ 
    &\opt&\textbf{0.3334}& \textbf{0.3499} & \textbf{0.3190} & \textbf{0.3431}
    & \textbf{0.4199}& \textbf{0.4322} & \textbf{0.4092} & \textbf{0.4271}
 \\  \hline
    
    % KETOD
    \multirow{4}{*}{KETOD}&{\gptt} &0.0943& 0.0801 & 0.0958 & 0.1108
    & 0.1439& 0.1337 & 0.1428 & 0.1578
    \\ 
    &\gpt& 0.1734& 0.1843 & 0.1498 & 0.1817
    &  0.2565& 0.2717 & 0.2336 & 0.2587
    \\ 
    &\flan&\underline{0.2311}& \underline{0.2435} & \underline{0.2120} & \underline{0.2335}
    & \underline{0.3100}& \underline{0.3243} & \underline{0.2897} & \underline{0.3112}
    \\
    &\opt & \textbf{0.2968} & \textbf{0.2981} & \textbf{0.3034} & \textbf{0.2889}
    &  \textbf{0.3726}& \textbf{0.3776} & \textbf{0.3746} & \textbf{0.3646} 
 \\  \hline

    % BITOD
        \multirow{4}{*}{BiToD}&{\gptt}& 0.2589& 0.2589 & - &-
    &  0.3092& 0.3092 & - &-
    \\ 
    &\gpt& \underline{0.5226}& \underline{0.5226} & - & -
    &  \underline{0.5682}& \underline{0.5682} & - & -
    \\ 
    &\flan& 0.4795& 0.4795 & - & -
    &  0.5389& 0.5389 & - & -
    \\ 
    &\opt& \textbf{0.5248}& \textbf{0.5248} & - & -
    &  \textbf{0.5900}& \textbf{0.5900} & - & -
 \\  \hline
    
        \end{tabular}
    \end{adjustbox}
    \caption{BLEU \& GLEU scores for inform sub-task of response generation task, with additional mixed domain setting.}
    \label{tab:inform_gleu_mixed}
    % \vspace{-15pt}
\end{table*}



\begin{table*}[!t]
    \centering
    \small
    \begin{adjustbox}{max width=\textwidth}
        \begin{tabular}{|c|c|c|c|c| R c|c|c|c|}
            \hline
    \multirow{2}{*}{\textbf{Dataset}}&\multirow{2}{*}{\textbf{Model}} & \multicolumn{4}{R}{\textbf{Request (BLEU-4)}} & \multicolumn{4}{c|}{\textbf{Request (GLEU-4)}}  \\ \cline{3-10}
    & & \textbf{all}  & \textbf{seen} & \textbf{unseen} & \textbf{mixed} & \textbf{all}  & \textbf{seen} & \textbf{unseen} & \textbf{mixed} \\ \hline
    \multirow{4}{*}{SGD}& {\gptt} &0.0057 & 0.0046 & 0.0056 & 0.0060
    & 0.0337 & 0.0299 & 0.0343 & 0.0343
    \\ 
    &\gpt& 0.0616& 0.0659 & 0.0630 & 0.0588
    &  0.1387& 0.1453 & 0.1403 & 0.1353
    \\ 
    &\flan&\underline{0.0657}& \underline{0.0742} & \underline{0.0698} & \underline{0.0582}
    & \underline{0.1552}& \underline{0.1687} & \underline{0.1586} & \underline{0.1477}
    \\ 
    &\opt& \textbf{0.0836}& \textbf{0.0928} & \textbf{0.0793} & \textbf{0.0868}
    &  \textbf{0.1767}& \textbf{0.1870} & \textbf{0.1721} & \textbf{0.1803}
 \\  \hline
    
    % KETOD
    \multirow{4}{*}{KETOD}&{\gptt} &0.0033& 0.0036 & 0.0037 & 0.0026
    & 0.0305& 0.0285 & 0.0332 & 0.0300
    \\ 
    
    &\gpt& 0.0140& 0.0193& 0.0073 & 0.0148
    &  0.0649& 0.0750& 0.0561 & 0.0620
    \\ 
    
    &\flan&\underline{0.0587}& \underline{0.0781} & \underline{0.0430} & \underline{0.0521}
    & \underline{0.1440}& \underline{0.1707} & \underline{0.1256} & \textbf{0.1315}
    \\ 
    
    &\opt& \textbf{0.0640}& \textbf{0.0851}& \textbf{0.0509} & \textbf{0.0526}
    &  \textbf{0.1475}& \textbf{0.1742}& \textbf{0.1339} & \underline{0.1301}
    
 \\  \hline

    % BITOD
        \multirow{4}{*}{BiToD}&{\gptt} &0.0011& 0.0011& - & -
    & 0.0219& 0.0219& - & -
    \\ 
    &\gpt& \underline{0.0323}& \underline{0.0323} & - & -
    &  \underline{0.1035}& \underline{0.1035} & - & -
    \\ 
    &\flan&0.0256& 0.0256 & - & -
    & 0.0811& 0.0811 & - & -
    \\ 
    &\opt& \textbf{0.0476}& \textbf{0.0476} & - & -
    &  \textbf{0.1260}& \textbf{0.1260} & - & -
    
 \\  \hline
    
        \end{tabular}
    \end{adjustbox}
    \caption{BLEU \& GLEU scores for request sub-task of response generation task, with additional mixed domain setting.}
    \label{tab:seeking_information_gleu_mixed}
    % \vspace{-15pt}
\end{table*}

\begin{table*}[!t]
    \centering
    \begin{adjustbox}{max width=\textwidth}
        \begin{tabular}{|c|c|c|c|c|c|c|c|c|c|c|c|c|c|}
            \hline
            \multirow{2}{*}{\textbf{Metric}} & \multirow{2}{*}{\textbf{Model}} & \multicolumn{4}{c|}{\textbf{SGD}} & \multicolumn{4}{c|}{\textbf{KETOD}} & \multicolumn{4}{c|}{\textbf{BiToD}} \\ \cline{3-14}
            & & \textbf{all} & \textbf{seen} & \textbf{unseen} & \textbf{mixed} & \textbf{all} & \textbf{seen} & \textbf{unseen} & \textbf{mixed} & \textbf{all} & \textbf{seen} & \textbf{unseen} & \textbf{mixed} \\ \hline

            \multirow{2}{*}{API} 
            & \gptt & 68.42 & 69.16 & 68.01 & 68.65 & 64.57 & 65.10 & 63.22 & 65.06 & 44.59 & 44.59 & - & - \\ 
            \multirow{2}{*}{Invoke} & \gpt & 90.51 & 97.66 & 86.21 & 93.11 & 78.83 & 80.87 & 73.56 & 80.75 & 93.46 & 93.46 & - & - \\ 
            \multirow{2}{*}{Accuracy} & \flan & \textbf{98.65} & \underline{99.61} & \textbf{98.24} & \textbf{98.84} & \textbf{90.45} & \textbf{89.93} & \textbf{90.34} & \textbf{91.09} & \textbf{98.48} & \textbf{98.48} & - & - \\ 
            & \opt & \underline{97.02} & \textbf{99.84} & \underline{95.81} & \underline{97.59} & \underline{82.22} & \underline{81.38} & \underline{84.14} & \underline{81.64} & \underline{97.36} & \underline{97.36} & - & - \\ 

            \hline
            \multirow{2}{*}{API} 
            & \gptt & 66.09 & 66.98 & 65.38 & 66.57 & 62.12 & 62.08 & 62.99 & 61.50 & 08.31 & 08.31 & - & - \\ 
            \multirow{2}{*}{Method} & \gpt & 78.62 & 96.26 & 71.42 & 81.80 & 71.48 & 78.36 & 62.53 & 71.12 & 87.34 & 87.34 & - & - \\ 
            \multirow{2}{*}{Accuracy} & \flan & \textbf{96.84} & \underline{99.61} & \textbf{95.23} & \textbf{97.81} & \textbf{89.07} & \textbf{89.93} & \textbf{86.67} & \textbf{90.02} & \textbf{93.07} & \textbf{93.07} & - & - \\ 
            & \opt & \underline{94.74} & \textbf{99.84} & \underline{92.13} & \underline{96.17} & \underline{78.45} & \underline{80.87} & \underline{77.70} & \underline{76.47} & \underline{92.70} & \underline{92.70} & - & - \\ 

            \hline
            \multirow{2}{*}{Parameter} 
            & \gptt & 67.24 & 64.84 & 68.38 & 66.65 & 63.47 & 61.17 & 67.11 & 63.08 & 04.53 & 04.53 & - & - \\ 
            \multirow{2}{*}{Name} & \gpt & 60.65 & 92.07 & 49.52 & 64.69 & 57.02 & 72.98 & 39.60 & 53.57 & 86.29 & 86.29 & - & - \\ 
            \multirow{2}{*}{Accuracy} & \flan & \underline{79.71} & \underline{96.97} & \underline{74.58} & \underline{86.68} & \underline{78.62} & \textbf{88.92} & \underline{67.93} & \textbf{75.97} & \underline{90.76} & \underline{90.76} & - & - \\ 
            & \opt & \textbf{90.41} & \textbf{98.19} & \textbf{87.86} & \textbf{91.21} & \textbf{79.50} & \underline{83.85} & \textbf{78.94} & \underline{75.30} & \textbf{92.32} & \textbf{92.32} & - & - \\ 

            \hline
            \multirow{2}{*}{Parameter} 
            & \gptt & 61.37 & 60.56 & 61.72 & 61.21 & 57.64 & 56.03 & 60.30 & 57.28 & 04.15 & 04.15 & - & - \\ 
            \multirow{2}{*}{Values} & \gpt & 58.64 & 91.24 & 46.98 & 62.95 & 55.46 & 71.83 & 37.73 & 51.83 & 86.10 & 86.10 & - & - \\ 
            \multirow{2}{*}{Accuracy} & \flan & \underline{76.03} & \underline{94.04} & \underline{70.43} & \underline{84.51} & \underline{76.52} & \textbf{87.94} & \underline{65.28} & \textbf{73.09} & \underline{85.47} & \underline{85.47} & - & - \\ 
            & \opt & \textbf{87.70} & \textbf{97.91} & \textbf{84.02} & \textbf{89.08} & \textbf{77.43} & \underline{83.46} & \textbf{75.87} & \underline{72.22} & \textbf{92.17} & \textbf{92.17} & - & - \\ 

            \hline
            \multirow{2}{*}{Full} 
            & \gptt & 33.32 & 32.63 & 28.52 & 38.06 & 29.59 & 30.54 & 23.45 & 33.33 & 02.36 & 02.36 & - & - \\ 
            \multirow{2}{*}{Api} & \gpt & 36.72 & 83.41 & 19.40 & 43.48 & 36.75 & 59.56 & 10.80 & 32.62 & 83.19 & 83.19 & - & - \\ 
            \multirow{2}{*}{Accuracy} & \flan & \underline{56.64} & \underline{89.49} & \underline{41.96} & \underline{63.79} & \textbf{59.42} & \textbf{83.72} & \underline{35.86} & \textbf{51.87} & \underline{73.79} & \underline{73.79} & - & - \\ 
            & \opt & \textbf{63.81} & \textbf{90.97} & \textbf{51.98} & \textbf{69.42} & \underline{57.35} & \underline{77.18} & \textbf{39.54} & \underline{50.09} & \textbf{91.11} & \textbf{91.11} & - & - \\ 


            \hline
        \end{tabular}
    \end{adjustbox}
    \vspace{-6pt}
    \caption{Detailed results for API calls across different metrics and datasets along with mixed domain.}
    \label{tab:api_results_mixed}
    \vspace{-12pt}
\end{table*}




\begin{table*}[!t]
    \centering
    \begin{adjustbox}{max width=\textwidth}
        \begin{tabular}{|c|c|c|c|c|R c|c|c |R c|c|c|c|}
            \hline
            \multirow{2}{*}{\textbf{Metric}} & \multirow{2}{*}{\textbf{Model}} & \multicolumn{4}{R}{\textbf{SGD}} & \multicolumn{4}{R}{\textbf{KETOD}} & \multicolumn{4}{c|}{\textbf{BiToD}} \\ \cline{3-14}
            & & \textbf{all} & \textbf{seen} & \textbf{unseen} & \textbf{mixed} & \textbf{all} & \textbf{seen} & \textbf{unseen}  & \textbf{mixed} & \textbf{all} & \textbf{seen} & \textbf{unseen}  & \textbf{mixed} \\ \hline
            
            \multirow{2}{*}{Api} 
            & \gptt & 68.34 & 73.42 & 67.77 & 68.12 & 62.05 & 61.05 & 66.00 & 61.20 & 29.47 & 29.47 & - & - \\ 
            \multirow{2}{*}{Invoke} & \gpt & 92.32 & 97.75 & 89.06 & 93.95 & 84.02 & 85.26 & 81.33 & 84.15 & \textbf{99.90} & \textbf{99.90} & - & - \\ 
            \multirow{2}{*}{Accuracy} & \flan & \textbf{98.80} & \underline{98.87} & \textbf{98.42} & \textbf{99.00} & \textbf{89.01} & \textbf{88.07} & \underline{90.00} & \textbf{89.34} 
            & 95.14 & 95.14 & - & -  \\ 
            & \opt & \underline{96.32} & \textbf{99.55} & \underline{94.98}  & \underline{96.87} & \underline{88.64} & \underline{87.72} & \textbf{92.67} & \underline{87.70} & \underline{99.31} & \underline{99.31} & - & - \\ 

            \hline
            \multirow{2}{*}{Api} 
            & \gptt & 65.94 & 72.07 & 64.80 & 66.01 & 60.05 & 58.60 & 66.67 & 58.47 & 12.23 & 12.23 & - & - \\ 
            \multirow{2}{*}{Method} & \gpt & 81.24 & 97.07 & 73.76 & 84.57 & 78.65 & 83.51 & 68.67 & 78.96 & 91.29 & 91.29 & - & - \\ 
            \multirow{2}{*}{Accuracy} & \flan & \textbf{96.94} & \underline{98.87} & \textbf{95.05}& \textbf{97.97} 
            & \textbf{87.02} & \textbf{88.07} & \underline{83.33} & \textbf{87.70} & \underline{91.53} & \underline{91.53} & - & - \\ 
            & \opt & \underline{93.87} & \textbf{99.55} & \underline{90.80} & \underline{95.33} & \underline{86.27} & \underline{87.72} & \textbf{86.00} & \underline{85.25} & \textbf{92.41} & \textbf{92.41} & - & - \\ 

            \hline
            \multirow{2}{*}{Parameter} 
            & \gptt & 65.85 & 68.99 & 64.65 & 66.31 & 60.07 & 58.10 & \underline{64.27} & 59.89 & 09.17 & 09.17 & - & - \\ 
            \multirow{2}{*}{Names} & \gpt & 58.23 & 95.61 & 44.97 & 63.01 & 61.22 & 79.28 & 35.37 & 57.76 & \underline{90.40} & \underline{90.40} & - & - \\ 
            \multirow{2}{*}{Accuracy} & \flan & \underline{77.31} & \underline{98.47} & \underline{72.96} & \underline{83.18} 
            & \underline{75.18} & \underline{87.11} & 62.96 & \underline{70.89} & 88.43 & 88.43 & - & - \\ 
            & \opt & \textbf{89.30} & \textbf{99.19} & \textbf{86.38} & \textbf{90.16} & \textbf{82.17} & \textbf{87.78} & \textbf{79.44} & \textbf{78.92} & \textbf{92.01} & \textbf{92.01} & - & -  \\ 

            \hline
            \multirow{2}{*}{Parameter} 
            & \gptt & 61.39 & 65.19 & 58.88 & 62.69 & 56.18 & 55.20 & 57.80 & 56.28 & 08.60 & 08.60 & - & - \\ 
            \multirow{2}{*}{Values} & \gpt & 56.46 & 94.52 & 42.59 & 61.60 & 59.44 & 77.53 & 33.22 & 56.09 & \underline{90.29} & \underline{90.29} & - & - \\ 
            \multirow{2}{*}{Accuracy} & \flan & \underline{73.77} & \underline{96.32} & \underline{68.36} & \underline{81.18}
            & \underline{73.16} & \underline{85.87} & \underline{59.72}& \underline{68.77} & 82.90 & 82.90 & - & - \\ 
            & \opt & \textbf{86.61} & \textbf{98.77} & \textbf{82.10} & \textbf{88.30}& \textbf{79.75} & \textbf{87.16} & \textbf{75.55} & \textbf{75.70} & \textbf{91.95} & \textbf{91.95} & - & - \\ 

            \hline
            \multirow{2}{*}{Full} 
            & \gptt & 38.62 & 40.77 & 33.66 & 41.83 & 34.08 & 36.84 & 26.67 & 34.97 & 03.55 & 03.55 & - & - \\ 
            \multirow{2}{*}{Api} & \gpt & 36.31 & 87.39 & 19.32 & 42.07 & 43.07 & 68.07 & 12.00 & 36.34 & \underline{88.14} & \underline{88.14} & - & - \\ 
            \multirow{2}{*}{Accuracy} & \flan & \underline{50.72} & \underline{92.12} & \underline{35.09} & \underline{56.39} 
            & \underline{56.30} & \underline{82.46} & \underline{30.00}& \underline{46.72} & 72.79 & 72.79 & -& -  \\ 
            & \opt & \textbf{61.13} & \textbf{96.40} & \textbf{46.91} & \textbf{66.85}& \textbf{60.55} & \textbf{83.16} & \textbf{36.67} & \textbf{52.73} & \textbf{91.38} & \textbf{91.38} & - & - \\ 

            \hline
        \end{tabular}
    \end{adjustbox}
    \caption{Results for carry information from domain one to next in multi-domain dialogs along with mixed domains.}
    \label{tab:multi_api_mixed_results}
\end{table*}




% \begin{table*}[!t]
%     \centering
%     \begin{adjustbox}{max width=\textwidth}
%         \begin{tabular}{|c |c|c |c|R c|c|R c|c|R c|c|R c|c|c|}
%             \hline
%             % \multirow{2}{*}{Model} & \multirow{2}{*}{Domains} & Response & Response & Retrieval & Retrieval & Slot Fill&Slot Fill\\ 
%             % & &  BLEU& GLEU& BLEU& GLEU& BLEU& GLEU \\ \hline
%     \multirow{2}{*}{\textbf{Dataset}} & \multirow{2}{*}{\textbf{Model}} & \multicolumn{3}{R}{\textbf{Api Invoke Accuracy}} & \multicolumn{3}{R}{\textbf{Api Method Accuracy}} & \multicolumn{3}{R}{\textbf{Param Names Accuracy}} & \multicolumn{3}{R}{\textbf{Param Values Accuracy}} & \multicolumn{3}{c|}{\textbf{Full API Accuracy}} \\ \cline{3-17}
%     & & \textbf{all}  & \textbf{seen} & \textbf{unseen} &  \textbf{all}  & \textbf{seen} & \textbf{unseen}&  \textbf{all}  & \textbf{seen} & \textbf{unseen}&  \textbf{all}  & \textbf{seen} & \textbf{unseen}& \textbf{all}  & \textbf{seen} & \textbf{unseen}  \\ \hline
%     \multirow{4}{*}{SGD} & {\gptt} & 68.34& 73.42& 67.77
%     & 65.94& 72.07& 64.80
%     & 65.85& 68.99 & 64.65
%     & 61.39& 65.19 & 58.88
%     & 38.62 & 40.77 & 33.66
%     \\ 
%     & \gpt & 92.32 & 97.75 & 89.06
%     & 81.24 & 97.07 & 73.76
%     &58.23 & 95.61 & 44.97
%     & 56.46 & 94.52 & 42.59 
%     & 36.31 & 87.39 & 19.32
%     \\ 
    
%     & \flan & \textbf{98.80} & \underline{98.87} & \textbf{98.42} 
%     & \textbf{96.94} & \underline{98.87} & \textbf{95.05} 
%     & \underline{77.31} & \underline{98.47} & \underline{72.96} 
%     & \underline{73.77} & \underline{96.32} & \underline{68.36} 
%     & \underline{50.72} & \underline{92.12} & \underline{35.09} 
%     \\
%     & \opt & \underline{96.32} & \textbf{99.55} & \underline{94.98} 
%     & \underline{93.87} & \textbf{99.55} & \underline{90.80} 
%     & \textbf{89.30} & \textbf{99.19} & \textbf{86.38} 
%     & \textbf{86.61} & \textbf{98.77} & \textbf{82.10} 
%     & \textbf{61.13} & \textbf{96.40} & \textbf{46.91} 
    
%     \\ 
%     \hline


% %KETOD
%     \multirow{4}{*}{KETOD} & {\gptt} &62.05& 61.05 & 66.00
%     & 60.05& 58.60 & 66.67 
%     & 60.07 & 58.10 & \underline{64.27}
%     & 56.18& 55.20 & 57.80 
%     & 34.08& 36.84 & 26.67 
%     \\ 
    
%     & \gpt & 84.02  & 85.26 & 81.33 
%     & 78.65 & 83.51 & 68.67 
%     & 61.22 & 79.28 & 35.37 
%     & 59.44 & 77.53 & 33.22 
%     & 43.07 & 68.07 & 12.00 
%     \\ 

%     & \flan &\textbf{89.01}  & \textbf{88.07} & \underline{90.00} 
%     & \textbf{87.02}& \textbf{88.07} & \underline{83.33} 
%     & \underline{75.18}& \underline{87.11} & 62.96 
%     & \underline{73.16} & \underline{85.87} & \underline{59.72} 
%     & \underline{56.30} & \underline{82.46} & \underline{30.00}
%     \\ 
    
%     & \opt& \underline{88.64} & \underline{87.72} & \textbf{92.67} 
%     & \underline{86.27} & \underline{87.72} & \textbf{86.00} 
%     & \textbf{82.17} & \textbf{87.78} & \textbf{79.44} 
%     & \textbf{79.75} & \textbf{87.16} & \textbf{75.55} 
%     & \textbf{60.55} & \textbf{83.16} & \textbf{36.67} 
%     \\ 
%  \hline
%         \end{tabular}
%     \end{adjustbox}
%     \caption{Results for Carry Information from Domain One to Next in Multi-domain Dialogs.}
%     \label{tab:multi_api_results}
%     % \vspace{-15pt}
% \end{table*}


% \begin{table*}[!t]
%     \centering
%     \begin{adjustbox}{max width=\textwidth}
%         \begin{tabular}{|c |c|c |c|R c|c|R c|c|R c|c|R c|c|c|}
%             \hline
%             % \multirow{2}{*}{Model} & \multirow{2}{*}{Domains} & Response & Response & Retrieval & Retrieval & Slot Fill&Slot Fill\\ 
%             % & &  BLEU& GLEU& BLEU& GLEU& BLEU& GLEU \\ \hline
%     \multirow{2}{*}{\textbf{Dataset}} & \multirow{2}{*}{\textbf{Model}} & \multicolumn{3}{R}{\textbf{Api Invoke Accuracy}} & \multicolumn{3}{R}{\textbf{Api Method Accuracy}} & \multicolumn{3}{R}{\textbf{Param Names Accuracy}} & \multicolumn{3}{R}{\textbf{Param Values Accuracy}} & \multicolumn{3}{c|}{\textbf{Full API Accuracy}} \\ \cline{3-17}
%     & & \textbf{mixed}& \textbf{seen} & \textbf{unseen} & \textbf{mixed} &  \textbf{seen} & \textbf{unseen}& \textbf{mixed}  & \textbf{seen} & \textbf{unseen}& \textbf{mixed}  & \textbf{seen} & \textbf{unseen}& \textbf{mixed}  & \textbf{seen} & \textbf{unseen}   \\ \hline
%     \multirow{4}{*}{SGD} & {\gptt} & 68.12 & 73.42& 67.77
%     & 66.01 & 72.07& 64.80
%     & 66.31& 68.99 & 64.65
%     & 62.69 & 65.19 & 58.88
%     & 41.83 & 40.77 & 33.66
%     \\ 
%     & \gpt & 93.95 & 97.75 & 89.06
%     & 84.57 & 97.07 & 73.76
%     & 63.01 & 95.61 & 44.97
%     & 61.60 & 94.52 & 42.59 
%     & 42.07 & 87.39 & 19.32
%     \\ 
    
%     & \flan & \textbf{99.00}  & \underline{98.87} & \textbf{98.42} 
%     & \textbf{97.97} & \underline{98.87} & \textbf{95.05} 
%     & \underline{83.18} & \underline{98.47} & \underline{72.96} 
%     & \underline{81.18} & \underline{96.32} & \underline{68.36} 
%     & \underline{56.39} & \underline{92.12} & \underline{35.09} 
%     \\
%     & \opt & \underline{96.87} & \textbf{99.55} & \underline{94.98} 
%     & \underline{95.33} & \textbf{99.55} & \underline{90.80} 
%     & \textbf{90.16} & \textbf{99.19} & \textbf{86.38} 
%     & \textbf{88.30} & \textbf{98.77} & \textbf{82.10} 
%     & \textbf{66.85} & \textbf{96.40} & \textbf{46.91} 
    
%     \\ 
%     \hline


% %KETOD
%     \multirow{4}{*}{KETOD} & {\gptt} & 61.20 & 61.05 & 66.00
%     & 58.47 & 58.60 & 66.67 
%     & 59.89 & 58.10 & \underline{64.27}
%     & 56.28& 55.20 & 57.80 
%     & 34.97& 36.84 & 26.67 
%     \\ 
    
%     & \gpt & 84.15  & 85.26 & 81.33 
%     & 78.96 & 83.51 & 68.67 
%     & 57.76 & 79.28 & 35.37 
%     & 56.09 & 77.53 & 33.22 
%     & 36.34 & 68.07 & 12.00 
%     \\ 

%     & \flan & \textbf{89.34}  & \textbf{88.07} & \underline{90.00} 
%     & \textbf{87.70} & \textbf{88.07} & \underline{83.33} 
%     & \underline{70.89} & \underline{87.11} & 62.96 
%     &  \underline{68.77} & \underline{85.87} & \underline{59.72} 
%     &  \underline{46.72} & \underline{82.46} & \underline{30.00}
%     \\ 
    
%     & \opt& \underline{87.70} & \underline{87.72} & \textbf{92.67} 
%     & \underline{85.25} & \underline{87.72} & \textbf{86.00} 
%     &  \textbf{78.92} & \textbf{87.78} & \textbf{79.44} 
%     &  \textbf{75.70} & \textbf{87.16} & \textbf{75.55} 
%     &  \textbf{52.73} & \textbf{83.16} & \textbf{36.67} 
%     \\ 
%  \hline
%         \end{tabular}
%     \end{adjustbox}
%     \caption{Results for Carry Information from Domain One to Next in Multi-domain Dialogs for Mixed Domains.}
%     \label{tab:multi_api_mixed_results}
%     % \vspace{-15pt}
% \end{table*}


% Table~\ref{tab:multi_api_results} presents the results of our analysis on multi-domain API calls, which are a critical aspect of evaluating task-oriented dialogue systems across different domains. In this context, a multi-domain API call occurs when, within a single dialogue, multiple API calls are made, and the API call domains in the current turn differ from those in the initial API call. This scenario tests a model's ability to handle complex dialogues that span multiple domains, reflecting a more realistic and challenging use case.

% Our results demonstrate that the OPT 6.7B model consistently outperforms other models in handling multi-domain API calls in both the KETOD and SGD datasets. This superior performance highlights the model's capability to effectively manage domain shifts within a dialogue, ensuring accurate and contextually appropriate responses across different domains. 
% %The results emphasize the importance of robust multi-domain support in task-oriented dialogue systems, particularly in scenarios where users may switch contexts or request information across various services within a single interaction.


% \section{Dialog Examples}
\label{sec:appendix_dialogs}

Table~\ref{tab:rest_dialog} shows an example dialog in the Restaurant domain. The table contains the turn id, user utterance, gold response, {\soloist}, {\autotod}, {\gpt}, {\llamai}, and {\flan} response. 
Text highlighted in red outlines the portions where the system response by a model is incorrect and green highlights the correct parts. Texts highlighted in orange indicate that the model is partially correct and is missing some information.

In the example dialog, we can see that the responses of {\autotod} is longer and more descriptive, whereas {\oursys} models produce succinct responses, similar to the gold standards. This observation indicates that {\oursys} was able to pick up the text generation style of the gold standard, which is an important aspect for ToD systems.

The baseline {\soloist} fails to generate a valid response on a few turns. This stems from the complex structure of the output, which consists of dialog states, actions and system response. The output structure has a certain format and different parts are identified with special tokens. On many occasions, the model does not produce the correct tokens, thus leading to invalid generations. 

On turns where the user asks about multiple information, the {\gpt} and {\soloist} model struggles to respond to all the user requests. However, all the other models do not have this problem. This could be attributed to the fact that {\gpt} models are the smallest model when compared to all the other models.

{\llamai} and {\flan} generates responses that are very similar to the gold standard. These models are able to pickup the generation style and can respond to multiple requests of the user. In the example dialog, the user provided information about the restaurant name, time and date on turn 1, but then changed the restaurant name on turn 2. {\llamai} and {\flan} models were able to identify what changed and moved forward with the {\apicall} by using all the correct information. However, {\autotod} was not able to keep track of all the changes and made an incorrect {\apicall}. {\soloist} was able to make a partially correct {\apicall} since it was not able to generalize to the \texttt{number\_of\_seats} slots. It used the \texttt{party\_size} slot that it had seen during training. 

This dialog example showed some of the drawbacks of baseline approaches and some example cases of where they struggle. For {\oursys} models, the limitations of {\gpt} was evident, as it was not able to process multiple requests of the user. However, {\llamai} and {\flan} had performance similar to the gold standard across multiple turns, which falls in line to our automatic metrics and human evaluation.

% \small
\begin{longtable}{>{\raggedright}p{2cm}|p{13cm}}
% \begin{longtable}{r|l}
    \caption{Dialog Example---SGD dataset. Dialog id 1\_00001} \label{tab:rest_dialog}\\  
    \hline
    \textbf{Model} & \textbf{Response---SGD dataset. Dialog id 1\_00001}  \\  
    \hline
    \endfirsthead

    % Header on subsequent pages
    \hline
    \textbf{Dataset} & \textbf{Response---SGD dataset. Dialog id 1\_00001 }  \\  
    \hline
    \endhead

    % Footer for all pages except the last
    \hline
    \textbf{Continued on next page...} & \\  
    \hline
    \endfoot

    % Footer for the last page
    \hline
    \endlastfoot


    % Table rows with long text
Turn & 1 \\ \hline
            User & Can you book a table for me at the Ancient Szechuan for the 11th of this month at 11:30 am?  \\ 
            Gold & In which city are you trying to book the table? \\ 
            \soloist & \mycolor{BrickRed}{Please confirm the following details: Booking a table for 2 at Ancient Szechuan in San Francisco on March 11th at 11:30 am.} \\ 
            \autotod & \mycolor{ForestGreen}{Could you please provide the city where the restaurant `Ancient Szechuan'' is located and the number of seats you would like to reserve?}
 \\
            \gpt & \mycolor{BrickRed}{Sure, I have a table for 2 at Ancient Szechuan in San Francisco at 11:30 am today.}
 \\ 
            \llamai & \mycolor{ForestGreen}{What is the location of the restaurant?} \\
            \flan &  \mycolor{ForestGreen}{Where should I look for the restaurant?}
\\ 
      \hline      
            Turn & 2 \\ \hline
            User &  Can you book a table at the Butterfly restaurant in San Francisco?  \\ 
            Gold & Please confirm that you need a table for 2 at the Butterfly Restaurant in San Francisco at 11:30 am on March 11th. \\ 
            \soloist & Can you please confirm that you want to book a \mycolor{ForestGreen}{table for 2 at the Butterfly restaurant in San Francisco at 11:30 am on March 11th?}\\ 
            \autotod &  \mycolor{BurntOrange}{Could you please confirm the date and time for the reservation at the Butterfly restaurant in San Francisco? Additionally, let me know the number of seats you would like to reserve.}
\\
            \gpt & \mycolor{BrickRed}{Can you please confirm that you want to book a table for 2 people at the Butterfly restaurant in San Francisco at 11:30 am on March 11th?}
 \\ 
            \llamai & Can you confirm the following details: You want to book a \mycolor{ForestGreen}{table at the butterfly Hotel in San Francisco at 11:30 am. You want 2 seats. You want to book the table on March 11th.} \\

            \flan & You wish to book a \mycolor{ForestGreen}{table for 2 at the Butterfly Restaurant in San Francisco for March 11th at 11:30 am.} Is this correct?
 \\ 
            \hline
            Turn & 3 \\ \hline
            Gold & ApiCall(method=`ReserveRestaurant', parameters= `date': `2019-03-11', `location': `San Francisco', `number\_of\_seats': `2',`restaurant\_name': `Butterfly Restaurant', `time': `11:30' )\\ 
            \soloist & ApiCall(method='ReserveRestaurant', parameters={`city': 'San Francisco', `date': `2019-03-11', \mycolor{BurntOrange}{`party\_size'}: `2',`restaurant\_name': `The Butterfly Restaurant', `time': `11:30'}) \\ 
            \autotod & ApiCall(method=`FindRestaurants',parameters=\mycolor{BrickRed}{`category': `Butterfly'}, `location': `San Francisco')
 \\
            \gpt & ApiCall(method=`ReserveRestaurant', parameters=`date': `2019-03-11', `location': `San Francisco', \mycolor{ForestGreen}{`number\_of\_seats'}: `2',`restaurant\_name': `The Butterfly Restaurant', `time': `11:30') \\
            \llamai & ApiCall(method=`ReserveRestaurant', parameters=`date': `2019-03-11', `location': `San Francisco', \mycolor{ForestGreen}{`number\_of\_seats'}: `2',`restaurant\_name': `Butterfly Restaurant', 'time': `11:30')
\\ 
            \flan & ApiCall(method=`ReserveRestaurant', parameters= `date': `2019-03-11', `location': `San Francisco',`restaurant\_name': `Butterfly Restaurant', \mycolor{ForestGreen}{`number\_of\_seats'}: `2', `time': `11:30' )
 \\ 
\hline            
            Turn & 4 \\ \hline
            User & That's perfect. What cuisine do they offer?  \\ 
            Gold & Your table has been booked successfully and they serve Asian cuisine. \\ 
            \soloist & \mycolor{BurntOrange}{Your table has been booked.} \\ 
            \autotod & It looks like the reservation at the Butterfly Restaurant in San Francisco has been successfully made for March 11th at 11:30 am for 2 people. \mycolor{ForestGreen}{The restaurant offers Asian cuisine.} If you need any more assistance or have any other questions, feel free to ask. Have a great day! Goodbye!
  \\
            \gpt & \mycolor{BurntOrange}{Your table is booked.}
\\
\llamai & Your reservation has been successful. \mycolor{ForestGreen}{They offer Asian cuisine.}
\\
            \flan & Your reservation has been made. \mycolor{ForestGreen}{They serve Asian food.} \\ 

            \hline
            Turn & 5 \\ \hline
            User & Can you get me the user rating of the restaurant? Is it expensive? \\ 
            Gold & Their prices are moderate and they have a user rating of 4.0. \\ 
            \soloist & \failed \\ 
            \autotod & The Butterfly Restaurant in San Francisco has a user \mycolor{ForestGreen}{rating of 4.00. The price range is moderate}, so it's not considered expensive. If you have any more questions or need further assistance, feel free to ask!
\\
            \gpt & \mycolor{Orange}{The rating is 4.0.}
 \\
 \llamai & The restaurant has a \mycolor{ForestGreen}{4.0 rating and is considered moderate in price}.
\\
            \flan &  They have a \mycolor{ForestGreen}{moderate price range and a 4.0 rating}.
\\ 

                        
            Turn & 6 \\ 
            User & Thanks a bunch! \\ 
            Gold & Is there anything else I can do for you? \\ 
            \soloist & \failed \\ 
            \autotod & You're welcome! If you need any more assistance, feel free to ask. Have a great day! Goodbye!
 \\
            \gpt & Is there anything else I can help you with?\\
            \llamai & Do you need anything else? \\
            \flan & Do you need help with anything else?
 \\ 

                
            Turn & 7 \\ 
            User & No. Thank you. That's all I need. \\ 
            Gold & Have a great day! \\ 
            \soloist & Have a great time at the restaurant! \\ 
            \autotod & You're welcome! If you need any more assistance in the future, feel free to reach out. Have a great day! Goodbye!
\\
            \gpt & Have a good one!
\\
\llamai & Have a great day. \\
            \flan & Have a nice day.
 \\ 

\end{longtable}


% \begin{longtable}{>{\raggedright}p{2cm}|p{13cm}}
% % \begin{longtable}{r|l}
%     \caption{Dialog Example---SGD dataset. Dialog id 1\_00001} \\  
%     \hline
%     \textbf{Model} & \textbf{Response---SGD dataset. Dialog id 1\_00001}  \\  
%     \hline
%     \endfirsthead

%     % Header on subsequent pages
%     \hline
%     \textbf{Dataset} & \textbf{Response---SGD dataset. Dialog id 1\_00001 }  \\  
%     \hline
%     \endhead

%     % Footer for all pages except the last
%     \hline
%     \textbf{Continued on next page...} & \\  
%     \hline
%     \endfoot

%     % Footer for the last page
%     \hline
%     \endlastfoot
    
%     % Table content using tabular
%     \begin{tabular}{r|p{13cm}}


%     \end{tabular}

%     \end{longtable}

\end{document}
