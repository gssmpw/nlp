\section{User Study Instructions}
\label{sec:templates}

\section*{Disclaimers of any risks to participants or annotators}

There are no significant risks associated with participating in this study. However, annotators may experience mild fatigue or cognitive strain due to prolonged reading and evaluation of multiple conversations. If you feel discomfort or fatigue, please take breaks as needed.

\section*{Instructions for Human Study Participants}

Your task is to evaluate model-generated responses in multi-turn, task-oriented conversations based on the following criteria:
\begin{enumerate}
    \item Fluency
    \item Informativeness
    \item Task Completion
\end{enumerate}

\section*{Task Overview}
\begin{itemize}
    \item You will be presented with multiple conversations, where a user interacts with a model to complete a task (e.g., booking a flight).
    \item Your job is to rate each model response independently using a 1-5 scale based on the provided criteria.
    \item This study is blind --- you will not know which model produced which response.
\end{itemize}

\section*{How to Rate Responses}
You will assign a score for each response based on the following:

\subsection*{1. Fluency}
Measures whether the response is grammatically correct, well-structured, and natural to read.

\begin{itemize}
    \item 1 = Unnatural, grammatically incorrect, or hard to read.
    \item 5 = Perfectly fluent, natural, and error-free.
\end{itemize}

\textbf{Example (Good Fluency)}
\begin{quote}
User: I need to book a flight to New York next Monday.\\
Model: Sure! What time would you like to depart?
\end{quote}

\textbf{Example (Poor Fluency)}
\begin{quote}
User: I need to book a flight to New York next Monday.\\
Model: Sure! You next Monday flight want time what?
\end{quote}

\subsection*{2. Informativeness}
Measures whether the response provides useful and relevant information to advance the task.

\begin{itemize}
    \item 1 = Vague, incorrect, or unhelpful.
    \item 5 = Precise, relevant, and useful.
\end{itemize}

\textbf{Example (Good Informativeness)}
\begin{quote}
User: Can you recommend a vegetarian restaurant nearby?\\
Model: Yes! Green Leaf Café is a highly rated vegetarian restaurant.
\end{quote}

\textbf{Example (Low Informativeness)}
\begin{quote}
User: Can you recommend a vegetarian restaurant nearby?\\
Model: There are many restaurants in your area.
\end{quote}

\subsection*{3. Task Completion}
Measures whether the response successfully progresses or completes the task in the conversation.

\begin{itemize}
    \item 1 = Fails to address the request.
    \item 5 = Fully completes the task.
\end{itemize}

\textbf{Example (Good Task Completion)}
\begin{quote}
User: I’d like to book a table for two at an Italian restaurant at 7 PM today.\\
Model: I found a reservation at Bella Italia for 7 PM. Would you like me to reserve it for 2 people at 7 PM today?
\end{quote}

\textbf{Example (Failed Task Completion)}
\begin{quote}
User: I’d like to book a table for two at an Italian restaurant at 7 PM.\\
Model: There are Italian restaurants in your area.
\end{quote}

\section*{Additional Guidelines}
\begin{itemize}
    \item Be objective --- rate based on quality, not personal preference.
    \item If a response is unclear or ambiguous, leave a comment.
    \item Do not try to guess which model produced the response.
\end{itemize}

\section*{Study Duration \& Submission}
\begin{itemize}
    \item The study will take approximately 10 minutes to complete.
    \item Once you have evaluated all responses, submit your ratings.
\end{itemize}

\vspace{1cm}
\centering{\textbf{Thank you for your time and valuable feedback!}}