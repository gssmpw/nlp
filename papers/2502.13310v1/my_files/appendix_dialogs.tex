\section{Dialog Examples}
\label{sec:appendix_dialogs}

Table~\ref{tab:rest_dialog} shows an example dialog in the Restaurant domain. The table contains the turn id, user utterance, gold response, {\soloist}, {\autotod}, {\gpt}, {\llamai}, and {\flan} response. 
Text highlighted in red outlines the portions where the system response by a model is incorrect and green highlights the correct parts. Texts highlighted in orange indicate that the model is partially correct and is missing some information.

In the example dialog, we can see that the responses of {\autotod} is longer and more descriptive, whereas {\oursys} models produce succinct responses, similar to the gold standards. This observation indicates that {\oursys} was able to pick up the text generation style of the gold standard, which is an important aspect for ToD systems.

The baseline {\soloist} fails to generate a valid response on a few turns. This stems from the complex structure of the output, which consists of dialog states, actions and system response. The output structure has a certain format and different parts are identified with special tokens. On many occasions, the model does not produce the correct tokens, thus leading to invalid generations. 

On turns where the user asks about multiple information, the {\gpt} and {\soloist} model struggles to respond to all the user requests. However, all the other models do not have this problem. This could be attributed to the fact that {\gpt} models are the smallest model when compared to all the other models.

{\llamai} and {\flan} generates responses that are very similar to the gold standard. These models are able to pickup the generation style and can respond to multiple requests of the user. In the example dialog, the user provided information about the restaurant name, time and date on turn 1, but then changed the restaurant name on turn 2. {\llamai} and {\flan} models were able to identify what changed and moved forward with the {\apicall} by using all the correct information. However, {\autotod} was not able to keep track of all the changes and made an incorrect {\apicall}. {\soloist} was able to make a partially correct {\apicall} since it was not able to generalize to the \texttt{number\_of\_seats} slots. It used the \texttt{party\_size} slot that it had seen during training. 

This dialog example showed some of the drawbacks of baseline approaches and some example cases of where they struggle. For {\oursys} models, the limitations of {\gpt} was evident, as it was not able to process multiple requests of the user. However, {\llamai} and {\flan} had performance similar to the gold standard across multiple turns, which falls in line to our automatic metrics and human evaluation.

% \small
\begin{longtable}{>{\raggedright}p{2cm}|p{13cm}}
% \begin{longtable}{r|l}
    \caption{Dialog Example---SGD dataset. Dialog id 1\_00001} \label{tab:rest_dialog}\\  
    \hline
    \textbf{Model} & \textbf{Response---SGD dataset. Dialog id 1\_00001}  \\  
    \hline
    \endfirsthead

    % Header on subsequent pages
    \hline
    \textbf{Dataset} & \textbf{Response---SGD dataset. Dialog id 1\_00001 }  \\  
    \hline
    \endhead

    % Footer for all pages except the last
    \hline
    \textbf{Continued on next page...} & \\  
    \hline
    \endfoot

    % Footer for the last page
    \hline
    \endlastfoot


    % Table rows with long text
Turn & 1 \\ \hline
            User & Can you book a table for me at the Ancient Szechuan for the 11th of this month at 11:30 am?  \\ 
            Gold & In which city are you trying to book the table? \\ 
            \soloist & \mycolor{BrickRed}{Please confirm the following details: Booking a table for 2 at Ancient Szechuan in San Francisco on March 11th at 11:30 am.} \\ 
            \autotod & \mycolor{ForestGreen}{Could you please provide the city where the restaurant `Ancient Szechuan'' is located and the number of seats you would like to reserve?}
 \\
            \gpt & \mycolor{BrickRed}{Sure, I have a table for 2 at Ancient Szechuan in San Francisco at 11:30 am today.}
 \\ 
            \llamai & \mycolor{ForestGreen}{What is the location of the restaurant?} \\
            \flan &  \mycolor{ForestGreen}{Where should I look for the restaurant?}
\\ 
      \hline      
            Turn & 2 \\ \hline
            User &  Can you book a table at the Butterfly restaurant in San Francisco?  \\ 
            Gold & Please confirm that you need a table for 2 at the Butterfly Restaurant in San Francisco at 11:30 am on March 11th. \\ 
            \soloist & Can you please confirm that you want to book a \mycolor{ForestGreen}{table for 2 at the Butterfly restaurant in San Francisco at 11:30 am on March 11th?}\\ 
            \autotod &  \mycolor{BurntOrange}{Could you please confirm the date and time for the reservation at the Butterfly restaurant in San Francisco? Additionally, let me know the number of seats you would like to reserve.}
\\
            \gpt & \mycolor{BrickRed}{Can you please confirm that you want to book a table for 2 people at the Butterfly restaurant in San Francisco at 11:30 am on March 11th?}
 \\ 
            \llamai & Can you confirm the following details: You want to book a \mycolor{ForestGreen}{table at the butterfly Hotel in San Francisco at 11:30 am. You want 2 seats. You want to book the table on March 11th.} \\

            \flan & You wish to book a \mycolor{ForestGreen}{table for 2 at the Butterfly Restaurant in San Francisco for March 11th at 11:30 am.} Is this correct?
 \\ 
            \hline
            Turn & 3 \\ \hline
            Gold & ApiCall(method=`ReserveRestaurant', parameters= `date': `2019-03-11', `location': `San Francisco', `number\_of\_seats': `2',`restaurant\_name': `Butterfly Restaurant', `time': `11:30' )\\ 
            \soloist & ApiCall(method='ReserveRestaurant', parameters={`city': 'San Francisco', `date': `2019-03-11', \mycolor{BurntOrange}{`party\_size'}: `2',`restaurant\_name': `The Butterfly Restaurant', `time': `11:30'}) \\ 
            \autotod & ApiCall(method=`FindRestaurants',parameters=\mycolor{BrickRed}{`category': `Butterfly'}, `location': `San Francisco')
 \\
            \gpt & ApiCall(method=`ReserveRestaurant', parameters=`date': `2019-03-11', `location': `San Francisco', \mycolor{ForestGreen}{`number\_of\_seats'}: `2',`restaurant\_name': `The Butterfly Restaurant', `time': `11:30') \\
            \llamai & ApiCall(method=`ReserveRestaurant', parameters=`date': `2019-03-11', `location': `San Francisco', \mycolor{ForestGreen}{`number\_of\_seats'}: `2',`restaurant\_name': `Butterfly Restaurant', 'time': `11:30')
\\ 
            \flan & ApiCall(method=`ReserveRestaurant', parameters= `date': `2019-03-11', `location': `San Francisco',`restaurant\_name': `Butterfly Restaurant', \mycolor{ForestGreen}{`number\_of\_seats'}: `2', `time': `11:30' )
 \\ 
\hline            
            Turn & 4 \\ \hline
            User & That's perfect. What cuisine do they offer?  \\ 
            Gold & Your table has been booked successfully and they serve Asian cuisine. \\ 
            \soloist & \mycolor{BurntOrange}{Your table has been booked.} \\ 
            \autotod & It looks like the reservation at the Butterfly Restaurant in San Francisco has been successfully made for March 11th at 11:30 am for 2 people. \mycolor{ForestGreen}{The restaurant offers Asian cuisine.} If you need any more assistance or have any other questions, feel free to ask. Have a great day! Goodbye!
  \\
            \gpt & \mycolor{BurntOrange}{Your table is booked.}
\\
\llamai & Your reservation has been successful. \mycolor{ForestGreen}{They offer Asian cuisine.}
\\
            \flan & Your reservation has been made. \mycolor{ForestGreen}{They serve Asian food.} \\ 

            \hline
            Turn & 5 \\ \hline
            User & Can you get me the user rating of the restaurant? Is it expensive? \\ 
            Gold & Their prices are moderate and they have a user rating of 4.0. \\ 
            \soloist & \failed \\ 
            \autotod & The Butterfly Restaurant in San Francisco has a user \mycolor{ForestGreen}{rating of 4.00. The price range is moderate}, so it's not considered expensive. If you have any more questions or need further assistance, feel free to ask!
\\
            \gpt & \mycolor{Orange}{The rating is 4.0.}
 \\
 \llamai & The restaurant has a \mycolor{ForestGreen}{4.0 rating and is considered moderate in price}.
\\
            \flan &  They have a \mycolor{ForestGreen}{moderate price range and a 4.0 rating}.
\\ 

                        
            Turn & 6 \\ 
            User & Thanks a bunch! \\ 
            Gold & Is there anything else I can do for you? \\ 
            \soloist & \failed \\ 
            \autotod & You're welcome! If you need any more assistance, feel free to ask. Have a great day! Goodbye!
 \\
            \gpt & Is there anything else I can help you with?\\
            \llamai & Do you need anything else? \\
            \flan & Do you need help with anything else?
 \\ 

                
            Turn & 7 \\ 
            User & No. Thank you. That's all I need. \\ 
            Gold & Have a great day! \\ 
            \soloist & Have a great time at the restaurant! \\ 
            \autotod & You're welcome! If you need any more assistance in the future, feel free to reach out. Have a great day! Goodbye!
\\
            \gpt & Have a good one!
\\
\llamai & Have a great day. \\
            \flan & Have a nice day.
 \\ 

\end{longtable}


% \begin{longtable}{>{\raggedright}p{2cm}|p{13cm}}
% % \begin{longtable}{r|l}
%     \caption{Dialog Example---SGD dataset. Dialog id 1\_00001} \\  
%     \hline
%     \textbf{Model} & \textbf{Response---SGD dataset. Dialog id 1\_00001}  \\  
%     \hline
%     \endfirsthead

%     % Header on subsequent pages
%     \hline
%     \textbf{Dataset} & \textbf{Response---SGD dataset. Dialog id 1\_00001 }  \\  
%     \hline
%     \endhead

%     % Footer for all pages except the last
%     \hline
%     \textbf{Continued on next page...} & \\  
%     \hline
%     \endfoot

%     % Footer for the last page
%     \hline
%     \endlastfoot
    
%     % Table content using tabular
%     \begin{tabular}{r|p{13cm}}


%     \end{tabular}

%     \end{longtable}