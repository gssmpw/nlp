\section{Conclusion}
\label{sec:conclusion}

\begin{table}[t]
\centering
\setlength{\extrarowheight}{1em}
\newcommand{\imwidth}{0.067\textwidth}
\setlength{\tabcolsep}{1pt}
\begin{tabular}{>{\centering\arraybackslash}m{2cm} >{\centering\arraybackslash}m{\imwidth} >{\centering\arraybackslash}m{\imwidth} >{\centering\arraybackslash}m{\imwidth} >{\centering\arraybackslash}m{\imwidth} >{\centering\arraybackslash}m{\imwidth}}

{Input Image} & \includegraphics[valign=c,width=\imwidth]{figs/2008_001742_ditask_final.pdf} & \includegraphics[valign=c,width=\imwidth]{figs/2008_007733_ditask_final.pdf} & \includegraphics[valign=c,width=\imwidth]{figs/2009_001581_ditask_final.pdf} & \includegraphics[valign=c,width=\imwidth]{figs/2009_002370_ditask_final.pdf} & \includegraphics[valign=c,width=\imwidth]{figs/2010_004753_ditask_final.pdf} \\
{MTLoRA} & \includegraphics[valign=c,width=\imwidth]{figs/2008_001742_semseg_mtlora_final.pdf} & \includegraphics[valign=c,width=\imwidth]{figs/2008_007733_semseg_mtlora_final.pdf} & \includegraphics[valign=c,width=\imwidth]{figs/2009_001581_semseg_mtlora_final.pdf} & \includegraphics[valign=c,width=\imwidth]{figs/2009_002370_semseg_mtlora_final.pdf} & \includegraphics[valign=c,width=\imwidth]{figs/2010_004753_semseg_mtlora_final.pdf} \\
{\textsc{DiTASK}} & \includegraphics[valign=c,width=\imwidth]{figs/2008_001742_semseg_ditask_final.pdf} & \includegraphics[valign=c,width=\imwidth]{figs/2008_007733_semseg_ditask_final.pdf} & \includegraphics[valign=c,width=\imwidth]{figs/2009_001581_semseg_ditask_final.pdf} & \includegraphics[valign=c,width=\imwidth]{figs/2009_002370_semseg_ditask_final.pdf} & \includegraphics[valign=c,width=\imwidth]{figs/2010_004753_semseg_ditask_final.pdf} \\
{Ground Truth} & \includegraphics[valign=c,width=\imwidth]{figs/2008_001742_mtlora.pdf} & \includegraphics[valign=c,width=\imwidth]{figs/2008_007733_mtlora.pdf} & \includegraphics[valign=c,width=\imwidth]{figs/2009_001581_mtlora.pdf} & \includegraphics[valign=c,width=\imwidth]{figs/2009_002370_mtlora.pdf} & \includegraphics[valign=c,width=\imwidth]{figs/2010_004753_mtlora.pdf} \\
\end{tabular}
\captionof{figure}{Semantic segmentation predictions on the PASCAL MTL dataset with MTLoRA and our \ourmethod}
\label{fig:viz}
\end{table}

In this paper, we introduce \ourmethod, a novel MTL fine-tuning approach that leverages neural diffeomorphisms for singular value adaptation while preserving the structure of pre-trained representations. We observe that existing methods, which restrict weight updates to fixed low-rank subspaces for resource efficiency, often suffer from multi-task performance degradation relative to single-task baselines. We conjecture that the singular vectors of pre-trained weight matrices capture rich features, and preserving them during MTL adaptation enhances performance. By using neural diffeomorphisms, \ourmethod\ maintains feature space and relational structure while requiring less memory than other low-rank adaptation methods. We evaluate \ourmethod\ on the PASCAL MTL and NYUD MTL datasets, with extensive ablation studies demonstrating its effectiveness. Our results show that \ourmethod\ achieves state-of-the-art MTL performance with 75\% fewer trainable parameters, underscoring the importance of preserving singular vectors in pre-trained weights.

