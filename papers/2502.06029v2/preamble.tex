% --- Additional Packages
\usepackage{subcaption}
\usepackage{float}
\usepackage[margin=1in]{geometry}  
\usepackage{caption}
\usepackage{multirow}

%
% --- inline annotations
%
\newcommand{\bruno}[1]{{\color{blue}BR:#1}}
\newcommand{\red}[1]{{\color{red}#1}}
\newcommand{\todo}[1]{{\color{red}#1}}
\newcommand{\TODO}[1]{\textbf{\color{red}[TODO: #1]}}
% --- disable by uncommenting  
% \renewcommand{\TODO}[1]{}
% \renewcommand{\todo}[1]{#1}



% \usepackage[hidelinks]{hyperref}       % hyperlinks
% \usepackage{url}            % simple URL typesetting
      % professional-quality tables
     % blackboard math symbols
\usepackage{nicefrac}       % compact symbols for 1/2, etc.
\usepackage{microtype}      %         % colors

\usepackage{graphicx}
\usepackage{caption}


%%%%%%%%%%%%%%%%%%%%%%%%%%%%%%%%
% THEOREMS
%%%%%%%%%%%%%%%%%%%%%%%%%%%%%%%%
% For theorems and such
\usepackage{amsmath}
\usepackage{amsfonts}
\usepackage{amssymb}
\usepackage{mathtools}
\usepackage{amsthm}
\usepackage{multirow}
\usepackage{dsfont}
\usepackage{algorithm}
\usepackage{algpseudocode}
% \usepackage{algcompatible}
\usepackage[T1]{fontenc}

\usepackage{bbold}
\usepackage{thmtools} 
\usepackage{thm-restate}
\usepackage{bm}
% \usepackage{mathptmx}  % For Times Roman font
% \usepackage{newtxtext}

\theoremstyle{plain}
\newtheorem{theorem}{Theorem}[section]
\newtheorem{proposition}[theorem]{Proposition}
\newtheorem{lemma}[theorem]{Lemma}
\newtheorem{corollary}[theorem]{Corollary}
\theoremstyle{definition}
\newtheorem{definition}[theorem]{Definition}
\newtheorem{assumption}[theorem]{Assumption}
\theoremstyle{remark}
\newtheorem{remark}[theorem]{Remark}
\newtheorem{conjecture}[theorem]{\textbf{Conjecture}}


% \declaretheorem[name=Proposition,numberwithin=section]{Proposition}

\usepackage{xcolor}
% \hypersetup{
%     colorlinks,
%     linkcolor={red!50!black},
%     citecolor={blue!50!black},
%     urlcolor={blue!80!black}
% }

\usepackage{pdfpages}

\bibliographystyle{plainnat}
\setcitestyle{square,numbers,comma}
\newcommand{\viz}{\emph{viz.}}


\usepackage[inline]{enumitem}
\usepackage{makecell}
\usepackage{multirow}
\usepackage[export]{adjustbox}


\newcommand{\ourmethod}{\textsc{DiTASK}}

% \raggedbottom
\usepackage{enumitem}%

\newcommand{\stepsfont}{\normalfont}%

\usepackage{enumitem}
\newlist{steps}{enumerate}{9}
\setlist[steps]{before=\upshape\stepsfont,font=\ttfamily} %
\setlist[steps,1]{leftmargin=*,labelsep=1ex,label={\arabic*.},ref={\arabic*}} %
\setlist[steps,2]{leftmargin=*,labelsep=1ex,label={\Alph*.},ref={\thestepsi.\Alph*}} %
\setlist[steps,3]{leftmargin=*,labelsep=1ex,label={\arabic*.},
  ref={\thestepsii.\arabic*}} %
\setlist[steps,4]{leftmargin=*,labelsep=1ex,label={\alph*.},
  ref={\thestepsiii.\alph*}} %
\setlist[steps,5]{leftmargin=*,labelsep=1ex,label={\arabic*.},
  ref={\thestepsiv.\arabic*}} %
\setlist[steps,6]{leftmargin=*,labelsep=1ex,label={\Alph*.},
  ref={\thestepsv.\Alph*}} %
\setlist[steps,7]{leftmargin=*,labelsep=1ex,label=\arabic*.,
  ref={\thestepsvi.\roman*}} %
\setlist[steps,8]{leftmargin=*,labelsep=1ex,label={\alph*.},
  ref={\thestepsvii.\alph*}} %
\setlist[steps,9]{leftmargin=*,labelsep=1ex,label={\arabic*.},
  ref={\thestepsviii.\arabic*}} %
\newlist{stepitems}{itemize}{1} % For when I want to put an itemized list inside the code.
\setlist[stepitems]{before=\upshape\ttfamily,label=*,leftmargin=*,labelsep=1ex}%
