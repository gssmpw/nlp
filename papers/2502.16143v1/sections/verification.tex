\section{When to Expect Factual Hallucination?}
\label{sec:when}

\subsection{Pathology}
\label{app:tasks:pathology}
\begin{tcolorbox}[title={\texttt{conch\_extract\_features}}]
Perform feature extraction on an input image using CONCH.

\vspace{.5em}
\textbf{Arguments:}
\begin{itemize}[topsep=0pt,parsep=-1pt,partopsep=0pt]
\item \texttt{input\_image} (\texttt{str}): Path to the input image\\  Example: \texttt{'/mount/input/TUM/TUM-TCGA-ACRLPPQE.tif'}
\end{itemize}

\vspace{.5em}
\textbf{Returns:} \begin{itemize}[topsep=0pt,parsep=-1pt,partopsep=0pt]
\item \texttt{features} (\texttt{list}): The feature vector extracted from the input image, as a list of floats
\end{itemize}
\tcblower
\setlength{\hangindent}{\widthof{\faGithub~}}
\faGithub~\url{https://github.com/mahmoodlab/CONCH}

\vspace{.5em}\setlength{\hangindent}{\widthof{\faFile*[regular]~}}\faFile*[regular]~\bibentry{lu2024conch}


\end{tcolorbox}

\begin{tcolorbox}[title={\texttt{musk\_extract\_features}}]
Perform feature extraction on an input image using the vision part of MUSK.

\vspace{.5em}
\textbf{Arguments:}
\begin{itemize}[topsep=0pt,parsep=-1pt,partopsep=0pt]
\item \texttt{input\_image} (\texttt{str}): Path to the input image\\  Example: \texttt{'/mount/input/TUM/TUM-TCGA-ACRLPPQE.tif'}
\end{itemize}

\vspace{.5em}
\textbf{Returns:} \begin{itemize}[topsep=0pt,parsep=-1pt,partopsep=0pt]
\item \texttt{features} (\texttt{list}): The feature vector extracted from the input image, as a list of floats
\end{itemize}
\tcblower
\setlength{\hangindent}{\widthof{\faGithub~}}
\faGithub~\url{https://github.com/lilab-stanford/MUSK}

\vspace{.5em}\setlength{\hangindent}{\widthof{\faFile*[regular]~}}\faFile*[regular]~\bibentry{xiang2025musk}


\end{tcolorbox}

\begin{tcolorbox}[title={\texttt{pathfinder\_verify\_biomarker}}]
Given WSI probability maps, a hypothesis of a potential biomarker, and clinical data, determine (1) whether the potential biomarker is significant for patient prognosis, and (2) whether the potential biomarker is independent among already known biomarkers.

\vspace{.5em}
\textbf{Arguments:}
\begin{itemize}[topsep=0pt,parsep=-1pt,partopsep=0pt]
\item \texttt{heatmaps} (\texttt{str}): Path to the folder containing the numpy array (\textasciigrave{}*.npy\textasciigrave{}) files, which contains the heatmaps of the trained model (each heatmap is HxWxC where C is the number of classes)\\  Example: \texttt{'/mount/input/TCGA\_CRC'}
\item \texttt{hypothesis} (\texttt{str}): A python file, which contains a function \textasciigrave{}def hypothesis\_score(prob\_map\_path: str) -\textgreater{} float\textasciigrave{} which expresses a mathematical model of a hypothesis of a potential biomarker.  For a particular patient, the function returns a risk score.\\  Example: \texttt{'/mount/input/mus\_fraction\_score.py'}
\item \texttt{clini\_table} (\texttt{str}): Path to the CSV file containing the clinical data\\  Example: \texttt{'/mount/input/TCGA\_CRC\_info.csv'}
\item \texttt{files\_table} (\texttt{str}): Path to the CSV file containing the mapping between patient IDs (in the PATIENT column) and heatmap filenames (in the FILENAME column)\\  Example: \texttt{'/mount/input/TCGA\_CRC\_files.csv'}
\item \texttt{survival\_time\_column} (\texttt{str}): The name of the column in the clinical data that contains the survival time\\  Example: \texttt{'OS.time'}
\item \texttt{event\_column} (\texttt{str}): The name of the column in the clinical data that contains the event (e.g. death, recurrence, etc.)\\  Example: \texttt{'vital\_status'}
\item \texttt{known\_biomarkers} (\texttt{list}): A list of known biomarkers. These are column names in the clinical data.\\  Example: \texttt{['MSI']}
\end{itemize}

\vspace{.5em}
\textbf{Returns:} \begin{itemize}[topsep=0pt,parsep=-1pt,partopsep=0pt]
\item \texttt{p\_value} (\texttt{float}): The p-value of the significance of the potential biomarker
\item \texttt{hazard\_ratio} (\texttt{float}): The hazard ratio for the biomarker
\end{itemize}
\tcblower
\setlength{\hangindent}{\widthof{\faGithub~}}
\faGithub~\url{https://github.com/LiangJunhao-THU/PathFinderCRC}

\vspace{.5em}\setlength{\hangindent}{\widthof{\faFile*[regular]~}}\faFile*[regular]~\bibentry{liang2023pathfinder}


\end{tcolorbox}

\begin{tcolorbox}[title={\texttt{stamp\_extract\_features}}]
Perform feature extraction using CTransPath with STAMP on a set of whole slide images, and save the resulting features to a new folder.

\vspace{.5em}
\textbf{Arguments:}
\begin{itemize}[topsep=0pt,parsep=-1pt,partopsep=0pt]
\item \texttt{output\_dir} (\texttt{str}): Path to the output folder where the features will be saved\\  Example: \texttt{'/mount/output/TCGA-BRCA-features'}
\item \texttt{slide\_dir} (\texttt{str}): Path to the input folder containing the whole slide images\\  Example: \texttt{'/mount/input/TCGA-BRCA-SLIDES'}
\end{itemize}

\vspace{.5em}
\textbf{Returns:} \begin{itemize}[topsep=0pt,parsep=-1pt,partopsep=0pt]
\item \texttt{num\_processed\_slides} (\texttt{int}): The number of slides that were processed
\end{itemize}
\tcblower
\setlength{\hangindent}{\widthof{\faGithub~}}
\faGithub~\url{https://github.com/KatherLab/STAMP}

\vspace{.5em}\setlength{\hangindent}{\widthof{\faFile*[regular]~}}\faFile*[regular]~\bibentry{elnahhas2024stamp}


\end{tcolorbox}

\begin{tcolorbox}[title={\texttt{stamp\_train\_classification\_model}}]
Train a model for biomarker classification. You will be supplied with the path to the folder containing the whole slide images, alongside a path to a CSV file containing the training labels.

\vspace{.5em}
\textbf{Arguments:}
\begin{itemize}[topsep=0pt,parsep=-1pt,partopsep=0pt]
\item \texttt{slide\_dir} (\texttt{str}): Path to the folder containing the whole slide images\\  Example: \texttt{'/mount/input/TCGA-BRCA-SLIDES'}
\item \texttt{clini\_table} (\texttt{str}): Path to the CSV file containing the clinical data\\  Example: \texttt{'/mount/input/TCGA-BRCA-DX\_CLINI.xlsx'}
\item \texttt{slide\_table} (\texttt{str}): Path to the CSV file containing the slide metadata\\  Example: \texttt{'/mount/input/TCGA-BRCA-DX\_SLIDE.csv'}
\item \texttt{target\_column} (\texttt{str}): The name of the column in the clinical data that contains the target labels\\  Example: \texttt{'TP53\_driver'}
\item \texttt{trained\_model\_path} (\texttt{str}): Path to the *.pkl file where the trained model should be saved by this function\\  Example: \texttt{'/mount/output/STAMP-BRCA-TP53-model.pkl'}
\end{itemize}

\vspace{.5em}
\textbf{Returns:} \begin{itemize}[topsep=0pt,parsep=-1pt,partopsep=0pt]
\item \texttt{num\_params} (\texttt{int}): The number of parameters in the trained model
\end{itemize}
\tcblower
\setlength{\hangindent}{\widthof{\faGithub~}}
\faGithub~\url{https://github.com/KatherLab/STAMP}

\vspace{.5em}\setlength{\hangindent}{\widthof{\faFile*[regular]~}}\faFile*[regular]~\bibentry{elnahhas2024stamp}


\end{tcolorbox}

\begin{tcolorbox}[title={\texttt{uni\_extract\_features}}]
Perform feature extraction on an input image using UNI.

\vspace{.5em}
\textbf{Arguments:}
\begin{itemize}[topsep=0pt,parsep=-1pt,partopsep=0pt]
\item \texttt{input\_image} (\texttt{str}): Path to the input image\\  Example: \texttt{'/mount/input/TUM/TUM-TCGA-ACRLPPQE.tif'}
\end{itemize}

\vspace{.5em}
\textbf{Returns:} \begin{itemize}[topsep=0pt,parsep=-1pt,partopsep=0pt]
\item \texttt{features} (\texttt{list}): The feature vector extracted from the input image, as a list of floats
\end{itemize}
\tcblower
\setlength{\hangindent}{\widthof{\faGithub~}}
\faGithub~\url{https://github.com/mahmoodlab/UNI}

\vspace{.5em}\setlength{\hangindent}{\widthof{\faFile*[regular]~}}\faFile*[regular]~\bibentry{chen2024uni}


\end{tcolorbox}

\subsection{Radiology}
\label{app:tasks:radiology}
\begin{tcolorbox}[title={\texttt{medsam\_inference}}]
Use the trained MedSAM model to segment the given abdomen CT scan.

\vspace{.5em}
\textbf{Arguments:}
\begin{itemize}[topsep=0pt,parsep=-1pt,partopsep=0pt]
\item \texttt{image\_file} (\texttt{str}): Path to the abdomen CT scan image.\\  Example: \texttt{'/mount/input/my\_image.jpg'}
\item \texttt{bounding\_box} (\texttt{list}): Bounding box to segment (list of 4 integers).\\  Example: \texttt{[25, 100, 155, 155]}
\item \texttt{segmentation\_file} (\texttt{str}): Path to where the segmentation image should be saved.\\  Example: \texttt{'/mount/output/segmented\_image.png'}
\end{itemize}

\vspace{.5em}
\textbf{Returns:} \textit{empty dict}
\tcblower
\setlength{\hangindent}{\widthof{\faGithub~}}
\faGithub~\url{https://github.com/bowang-lab/MedSAM}

\vspace{.5em}\setlength{\hangindent}{\widthof{\faFile*[regular]~}}\faFile*[regular]~\bibentry{ma2024medsam}


\end{tcolorbox}

\begin{tcolorbox}[title={\texttt{nnunet\_train\_model}}]
Train a nnUNet model from scratch on abdomen CT scans. You will be provided with  the path to the dataset, the nnUNet configuration to use, and the fold number  to train the model on.

\vspace{.5em}
\textbf{Arguments:}
\begin{itemize}[topsep=0pt,parsep=-1pt,partopsep=0pt]
\item \texttt{dataset\_path} (\texttt{str}): The path to the dataset to train the model on (contains dataset.json, imagesTr, imagesTs, labelsTr)\\  Example: \texttt{'/mount/input/Task02\_Heart'}
\item \texttt{unet\_configuration} (\texttt{str}): The configuration of the UNet to use for training. One of '2d', '3d\_fullres', '3d\_lowres', '3d\_cascade\_fullres'\\  Example: \texttt{'3d\_fullres'}
\item \texttt{fold} (\texttt{int}): The fold number to train the model on. One of 0, 1, 2, 3, 4.\\  Example: \texttt{0}
\item \texttt{output\_folder} (\texttt{str}): Path to the folder where the trained model should be saved\\  Example: \texttt{'/mount/output/trained\_model'}
\end{itemize}

\vspace{.5em}
\textbf{Returns:} \textit{empty dict}
\tcblower
\setlength{\hangindent}{\widthof{\faGithub~}}
\faGithub~\url{https://github.com/MIC-DKFZ/nnUNet}

\vspace{.5em}\setlength{\hangindent}{\widthof{\faFile*[regular]~}}\faFile*[regular]~\bibentry{isensee2020nnunet}


\end{tcolorbox}

\subsection{Omics}
\label{app:tasks:genomics_proteomics}
\begin{tcolorbox}[title={\texttt{cytopus\_db}}]
Initialize the Cytopus KnowledgeBase and generate a JSON file containing a nested dictionary with gene set annotations organized by cell type, suitable for input into the Spectra library.

\vspace{.5em}
\textbf{Arguments:}
\begin{itemize}[topsep=0pt,parsep=-1pt,partopsep=0pt]
\item \texttt{celltype\_of\_interest} (\texttt{list}): List of cell types for which to retrieve gene sets\\  Example: \texttt{['B\_memory', 'B\_naive', 'CD4\_T', 'CD8\_T', 'DC', 'ILC3', 'MDC', 'NK', 'Treg', 'gdT', 'mast', 'pDC', 'plasma']}
\item \texttt{global\_celltypes} (\texttt{list}): List of global cell types to include in the JSON file.\\  Example: \texttt{['all-cells', 'leukocyte']}
\item \texttt{output\_file} (\texttt{str}): Path to the file where the output JSON file should be stored.\\  Example: \texttt{'/mount/output/Spectra\_dict.json'}
\end{itemize}

\vspace{.5em}
\textbf{Returns:} \begin{itemize}[topsep=0pt,parsep=-1pt,partopsep=0pt]
\item \texttt{keys} (\texttt{list}): The list of keys in the produced JSON file.
\end{itemize}
\tcblower
\setlength{\hangindent}{\widthof{\faGithub~}}
\faGithub~\url{https://github.com/wallet-maker/cytopus}

\vspace{.5em}\setlength{\hangindent}{\widthof{\faFile*[regular]~}}\faFile*[regular]~\bibentry{kunes2023cytopus}


\end{tcolorbox}

\begin{tcolorbox}[title={\texttt{esm\_fold\_predict}}]
Generate the representation of a protein sequence and the contact map using Facebook Research's pretrained esm2\_t33\_650M\_UR50D model.

\vspace{.5em}
\textbf{Arguments:}
\begin{itemize}[topsep=0pt,parsep=-1pt,partopsep=0pt]
\item \texttt{sequence} (\texttt{str}): Protein sequence to for which to generate representation and contact map.\\  Example: \texttt{'MKTVRQERLKSIVRILERSKEPVSGAQLAEELSVSRQVIVQDIAYLRSLGYNIVATPRGYVLAGG'}
\end{itemize}

\vspace{.5em}
\textbf{Returns:} \begin{itemize}[topsep=0pt,parsep=-1pt,partopsep=0pt]
\item \texttt{sequence\_representation} (\texttt{list}): Token representations for the protein sequence as a list of floats, i.e. a 1D array of shape L where L is the number of tokens.
\item \texttt{contact\_map} (\texttt{list}): Contact map for the protein sequence as a list of list of floats, i.e. a 2D array of shape LxL where L is the number of tokens.
\end{itemize}
\tcblower
\setlength{\hangindent}{\widthof{\faGithub~}}
\faGithub~\url{https://github.com/facebookresearch/esm}

\vspace{.5em}\setlength{\hangindent}{\widthof{\faFile*[regular]~}}\faFile*[regular]~\bibentry{verkuil2022esm1}


\vspace{.5em}\setlength{\hangindent}{\widthof{\faFile*[regular]~}}\faFile*[regular]~\bibentry{hie2022esm2}


\end{tcolorbox}

\subsection{Other}
\label{app:tasks:imaging}
\begin{tcolorbox}[title={\texttt{retfound\_feature\_vector}}]
Extract the feature vector for the given retinal image using the RETFound pretrained vit\_large\_patch16 model.

\vspace{.5em}
\textbf{Arguments:}
\begin{itemize}[topsep=0pt,parsep=-1pt,partopsep=0pt]
\item \texttt{image\_file} (\texttt{str}): Path to the retinal image.\\  Example: \texttt{'/mount/input/retinal\_image.jpg'}
\end{itemize}

\vspace{.5em}
\textbf{Returns:} \begin{itemize}[topsep=0pt,parsep=-1pt,partopsep=0pt]
\item \texttt{feature\_vector} (\texttt{list}): The feature vector for the given retinal image, as a list of floats.
\end{itemize}
\tcblower
\setlength{\hangindent}{\widthof{\faGithub~}}
\faGithub~\url{https://github.com/rmaphoh/RETFound_MAE}

\vspace{.5em}\setlength{\hangindent}{\widthof{\faFile*[regular]~}}\faFile*[regular]~\bibentry{zhou2023retfound}


\end{tcolorbox}

\label{app:tasks:llms}
\begin{tcolorbox}[title={\texttt{medsss\_generate}}]
Given a user message, generate a response using the MedSSS\_Policy model.

\vspace{.5em}
\textbf{Arguments:}
\begin{itemize}[topsep=0pt,parsep=-1pt,partopsep=0pt]
\item \texttt{user\_message} (\texttt{str}): The user message.\\  Example: \texttt{'How to stop a cough?'}
\end{itemize}

\vspace{.5em}
\textbf{Returns:} \begin{itemize}[topsep=0pt,parsep=-1pt,partopsep=0pt]
\item \texttt{response} (\texttt{str}): The response generated by the model.
\end{itemize}
\tcblower
\setlength{\hangindent}{\widthof{\faGithub~}}
\faGithub~\url{https://github.com/pixas/MedSSS}

\vspace{.5em}\setlength{\hangindent}{\widthof{\faFile*[regular]~}}\faFile*[regular]~\bibentry{jiang2025medsss}


\end{tcolorbox}

\begin{tcolorbox}[title={\texttt{modernbert\_predict\_masked}}]
Given a masked sentence string, predict the original sentence using the pretrained ModernBERT-base model on CPU.

\vspace{.5em}
\textbf{Arguments:}
\begin{itemize}[topsep=0pt,parsep=-1pt,partopsep=0pt]
\item \texttt{input\_string} (\texttt{str}): The masked sentence string. The masked part is represented by "[MASK]"".\\  Example: \texttt{'Paris is the [MASK] of France.'}
\end{itemize}

\vspace{.5em}
\textbf{Returns:} \begin{itemize}[topsep=0pt,parsep=-1pt,partopsep=0pt]
\item \texttt{prediction} (\texttt{str}): The predicted original sentence
\end{itemize}
\tcblower
\setlength{\hangindent}{\widthof{\faGithub~}}
\faGithub~\url{https://github.com/AnswerDotAI/ModernBERT}

\vspace{.5em}\setlength{\hangindent}{\widthof{\faFile*[regular]~}}\faFile*[regular]~\bibentry{warner2024modernbert}


\end{tcolorbox}

\label{app:tasks:3d_vision}
\begin{tcolorbox}[title={\texttt{flowmap\_overfit\_scene}}]
Overfit FlowMap on an input scene to determine camera extrinsics for each frame in the scene.

\vspace{.5em}
\textbf{Arguments:}
\begin{itemize}[topsep=0pt,parsep=-1pt,partopsep=0pt]
\item \texttt{input\_scene} (\texttt{str}): Path to the directory containing the images of the input scene (just the image files, nothing else)\\  Example: \texttt{'/mount/input/llff\_flower'}
\end{itemize}

\vspace{.5em}
\textbf{Returns:} \begin{itemize}[topsep=0pt,parsep=-1pt,partopsep=0pt]
\item \texttt{n} (\texttt{int}): The number of images (frames) in the scene
\item \texttt{camera\_extrinsics} (\texttt{list}): The camera extrinsics matrix for each of the n frames in the scene, must have a shape of nx4x4 (as a nested python list of floats)
\end{itemize}
\tcblower
\setlength{\hangindent}{\widthof{\faGithub~}}
\faGithub~\url{https://github.com/dcharatan/flowmap}

\vspace{.5em}\setlength{\hangindent}{\widthof{\faFile*[regular]~}}\faFile*[regular]~\bibentry{smith2024flowmap}


\end{tcolorbox}

\label{app:tasks:tabular}
\begin{tcolorbox}[title={\texttt{tabpfn\_predict}}]
Train a predictor using TabPFN on a tabular dataset. Evaluate the predictor on the test set.

\vspace{.5em}
\textbf{Arguments:}
\begin{itemize}[topsep=0pt,parsep=-1pt,partopsep=0pt]
\item \texttt{train\_csv} (\texttt{str}): Path to the CSV file containing the training data\\  Example: \texttt{'/mount/input/breast\_cancer\_train.csv'}
\item \texttt{test\_csv} (\texttt{str}): Path to the CSV file containing the test data\\  Example: \texttt{'/mount/input/breast\_cancer\_test.csv'}
\item \texttt{feature\_columns} (\texttt{list}): The names of the columns to use as features\\  Example: \texttt{['mean radius', 'mean texture', 'mean perimeter', 'mean area', 'mean smoothness', 'mean compactness', 'mean concavity', 'mean concave points', 'mean symmetry', 'mean fractal dimension', 'radius error', 'texture error', 'perimeter error', 'area error', 'smoothness error', 'compactness error', 'concavity error', 'concave points error', 'symmetry error', 'fractal dimension error', 'worst radius', 'worst texture', 'worst perimeter', 'worst area', 'worst smoothness', 'worst compactness', 'worst concavity', 'worst concave points', 'worst symmetry', 'worst fractal dimension']}
\item \texttt{target\_column} (\texttt{str}): The name of the column to predict\\  Example: \texttt{'target'}
\end{itemize}

\vspace{.5em}
\textbf{Returns:} \begin{itemize}[topsep=0pt,parsep=-1pt,partopsep=0pt]
\item \texttt{roc\_auc} (\texttt{float}): The ROC AUC score of the predictor on the test set
\item \texttt{accuracy} (\texttt{float}): The accuracy of the predictor on the test set
\item \texttt{probs} (\texttt{list}): The probabilities of the predictor on the test set, as a list of floats (one per sample in the test set)
\end{itemize}
\tcblower
\setlength{\hangindent}{\widthof{\faGithub~}}
\faGithub~\url{https://github.com/PriorLabs/TabPFN}

\vspace{.5em}\setlength{\hangindent}{\widthof{\faFile*[regular]~}}\faFile*[regular]~\bibentry{hollmann2025tabpfn}


\end{tcolorbox}





To determine the conditions under which factual hallucinations emerge, we investigate knowledge overshadowing across various experimental setups, including probing an open-source pretrained LLM without training, pretraining an LLM from scratch, fine-tuning a pretrained LLM on downstream tasks.

\subsection{Probing the Open-source 
LLM}
\label{ssec:dolma}
% \yuji{briefly summarize this subsection and put it into appendix}
We probe an open-source pretrained LLM Olmo with its public training corpus Dolma~\cite{soldaini2024dolma} to investigate the hallucination and sample frequency in data. Results show that knowledge with higher frequency tends to overshadow others with lower frequency, aligning with knowledge overshadowing concept that more dominant knowledge overshadows less prominent knowledge during text generation, leading to counterfactual outputs. (See details in~\ref{ssec:dolmo_probing}). 

\subsection{Unveiling Log-linear Law in the Pretrained LLMs.}
\label{ssec:pretrain_law}

\noindent \textbf{Setup.}
To accurately quantify the relationship between hallucinations and their influential variables, we pretrain language models from scratch on synthetic datasets with controlled variable settings. This approach is necessary because the inherent variability and imprecision of natural language in real-world training data make it intractable to enumerate all possible expressions of more and less popular knowledge with perfect accuracy.


For each controlled variable experiment, we adopt sampled tokens from a tokenizer vocabulary to construct each dataset, as shown in Table~\ref{tab:tasks}.

% \kmnote{These experiments make sense. }

\noindent$\bullet$ P: We investigate how the hallucination rate $\text{R}$ changes with increasing relative knowledge popularity $\text{P}$. We set $\text{P} = \frac{m}{n}$ for values \{2:1, 5:1, 10:1, 25:1, 50:1, 100:1\}, where $m$ represents the number of samples of $k_{a_i} = Y_a | [X_{\mathrm{share}} \odot x_{a_i}]$ and $n$ represents the number of samples of $k_{b_i} = Y_b | [X_{\mathrm{share}} \odot x_{b_i}]$. The other variables, $\text{L}$ and $\text{S}$, are held constant. Each token in $x_{a_i}$, $x_{b_j}$, $X_\mathrm{share}$, $Y_a$, and $Y_b$ is sampled from the vocabulary.

\noindent$\bullet$ L: To examine how the hallucination rate $\text{R}$ changes 
% \heng{evolves usually means good changes} 
with increasing relative knowledge length $\text{L}$, we set {\small$\text{L} = \frac{\text{len}(X_{\mathrm{share}})+\text{len}(x_{b_j})}{\text{len}(x_{b_j})}$} for values \{1:1, 2:1, 5:1, 10:1, 25:1, 50:1, 100:1\}, where len($x_{a_i}$)=len($x_{b_j}$) to ensure consistent variables.



\noindent$\bullet$ S: To investigate how hallucination rate changes with varying model sizes, we experiment on the Pythia model family with sizes of 160M, 410M, 1B, 1.4B, and 2.8B, along with other models including Phi-2.8B, GPT-J-6B, Mistral-7B, Llama-2-7b, and Llama-13B (Dataset statistics in~\ref{ssec: overshadowing_dataset}). 



We pretrain each LLM from scratch on the dataset over 19.6 million of tokens in Table~\ref{tab:tasks}
with controlled variables in an auto-regressive manner, optimizing for cross-entropy loss until the model converges (See training details in~\ref{ssec:implementation}). As shown in Figure~\ref{fig:rkp_rkl_generalization}, factual hallucination follows the log-linear relationship w.r.t $\text{P}$, $\text{L}$, and $\text{S}$:

\vspace{-0.4em}
\begin{small}
\begin{equation}
\text{R(P)}=\alpha\log(\frac{\text{P}}{\text{P}_c}); \text{R(L)}=\beta\log(\frac{\text{L}}{\text{L}_c}); \text{R(S)}=\gamma\log(\frac{\text{S}}{\text{S}_c})
\end{equation}
\end{small}



\vspace{0em}\noindent where $\alpha$, $\beta$, $\gamma$, $\text{P}_c$, $\text{L}_c$, $\text{S}_c$ are constants. In Figure~\ref{fig:rkp_rkl_generalization}, hallucination rate increases linearly with the logarithmic scale of relative knowledge popularity $\text{P}$, 
relative knowledge length $\text{L}$, and model Size $\text{S}$.

\vspace{1mm}
\noindent \textbf{Greater Popularity Overshadows More.}
From a global perspective in the entire training data, when knowledge $k_{a_i}$ has higher frequency than knowledge $k_{b_j}$, the distinctive token sequence $x_{b_j}$ encoding the less popular knowledge $k_{b_j}$ is more susceptible to be overshadowed.  This imbalance amplifies dominant knowledge while suppressing the representations of less frequent facts. 
This highlights a fundamental bias in how LLMs internalize and retrieve knowledge, revealing that hallucination arises not just from data sparsity but from the inherent competition between knowledge representations in a non-uniform training distribution.

\vspace{1mm}
\noindent \textbf{Longer Length Overshadows More.}
At its core, knowledge overshadowing arises from the degradation of probability distributions:

\vspace{-0.5em}
\begin{small}
    \begin{equation}
    \label{eq:prob_degrade}
    \begin{cases}
   P(Y_a | [X_{\mathrm{share}} \odot \textcolor{gray}{x_{a_i}}]) \xrightarrow{\text{degrade to}} P(Y_a | X_{\mathrm{share}})  \\
    P(Y_b | [X_{\mathrm{share}} \odot \textcolor{gray}{x_{b_j}}]) \xrightarrow{\text{degrade to}} P(Y_a | X_{\mathrm{share}}) 
\end{cases}
\end{equation}
\end{small}

\noindent The degradation reflects the compressed representations of $x_{a_i}$ and $x_{b_j}$, which are merged into $X_\mathrm{share}$, thereby weakening their distinct contributions to generation. Locally within a sentence, when $x_{b_j}$'s token length is shorter than $X_\mathrm{share}$, its ability to maintain a distinct semantic boundary diminishes. This occurs because degradation is influenced by both knowledge interaction and $x_{b_j}$'s representation capacity. Shorter representations inherently encode less detailed semantic information, making them more prone to being overshadowed by the structurally and semantically richer $X_\mathrm{share}$.




\vspace{1mm}
\noindent \textbf{Larger Model Overshadows More.}
Larger models exhibit a stronger tendency to overshadow less prominent knowledge, a phenomenon linked to their increased compression capabilities. Prior findings show that as model capacity grows, it compresses information more efficiently, enhancing its ability to efficiently capture patterns and generalize~\cite{huang2024compression}. However, this compression mechanism affects less frequent knowledge, which is more easily suppressed into the dominant representations of more popular knowledge. Although larger models can encode more information, their capacity to preserve clear semantic distinctions for less popular knowledge diminishes. This leads to the suppression or distortion of this knowledge during generation, ultimately increasing the likelihood of hallucinations. 

\subsection{Validating Log-linear Law in the Fine-tuned LLMs.}
\label{ssec:finetune_law}

\begin{figure*}[t]
\centering
% \vspace{-7mm}
\subfloat{\includegraphics[height=1.18105in]{pictures/location_s.pdf}%
\label{fig:location_s}}
\hfil
\subfloat{\includegraphics[height=1.18105in]{pictures/neg_s.pdf}%
\label{fig:neg_s}}
\hfil
\subfloat{\includegraphics[height=1.18105in]{pictures/math_s.pdf}%
\label{fig:math_s}}
\hfil
\subfloat{\includegraphics[height=1.18105in]{pictures/fuse_l_gptj.pdf}%
\label{fig:fuse_l}}

\subfloat{\includegraphics[height=1.18105in]{pictures/time_p.pdf}%
\label{fig:time_p}}
\hfil
\subfloat{\includegraphics[height=1.18105in]{pictures/conflict_p.pdf}%
\label{fig:con_p}}
\hfil
\subfloat{\includegraphics[height=1.18105in]{pictures/logic_p.pdf}%
\label{fig:logic_p}}
\hfil
\subfloat{\includegraphics[height=1.18105in]{pictures/fuse_l_pythia410.pdf}%
\label{fig:fuse_pythia_l}}
\vspace{-0.5em}
\caption{Fine-tuning open-source LLMs on natural language tasks. Regression lines represent the predicted trends derived from LLMs pretrained on synthetic data in \S~\ref{ssec:pretrain_law}. The red cross markers indicate the empirically observed hallucination rates in fine-tuned LLMs. Training data statistics and implementation are in~\ref{ssec:implementation},~\ref{ssec: overshadowing_dataset}.}
\label{fig:finetune_law}
\vspace{-0.5em}
\end{figure*}

\noindent \textbf{Setup.}
The results presented in \S~\ref{ssec:pretrain_law} were derived from pretrained models. In this section, we extend our analysis by investigating whether the log-linear law holds for fine-tuned LLMs, aiming to assess whether it can serve as a predictive tool for quantifying hallucinations when fine-tuning LLMs on downstream tasks. Specifically, we fine-tune models with parameter sizes ranging from 160M to 13B across a variety of factual tasks, including time, location, gender, negation queries, mathematical and logical reasoning, and knowledge conflict resolution.
For each task, we generate $m$ samples of $k_{a_i} = Y_a | [X_{\mathrm{share}} \odot x_{a_i}]$ and $n$ samples of $k_{b_i} = Y_b | [X_{\mathrm{share}} \odot x_{b_i}]$.
To ensure a controlled fine-tuned knowledge distribution, we construct factual queries from artificial facts~\cite{meng2022locating}, to mitigate interference from pretrained knowledge, enabling a precise evaluation of $\text{P}$ and $\text{L}$ in the law.
We present knowledge pair samples $(k_a, k_b)$ for several tasks in Table~\ref{tab:tasks}, with additional dataset samples and statistics provided in~\ref{ssec: overshadowing_dataset}.


\vspace{1mm}
\noindent \textbf{Proactively Quantifying Hallucination by Law.}

We utilize the log-linear law fitted by the pretrained LLMs on controlled synthetic datasets to predict hallucination rates for fine-tuned LLMs across various downstream tasks. This includes predicting hallucination rate $\text{R}$ with changing model size $\text{S}$, relative knowledge popularity $\text{P}$, and relative knowledge length $\text{L}$, as shown in Figure~\ref{fig:finetune_law}. We then evaluate the discrepancy between the predicted hallucination rates and those observed in our fine-tuning experiments. Following \citet{chen2024scaling}, we assess the prediction performance of log-linear law using the relative prediction error:

\vspace{-1.2em}
\begin{equation}
\fontsize{8}{8}\selectfont
\text{Relative Prediction Error} = \frac{|\text{Predictive Rate} - \text{Actual Rate}|}{\text{Actual Rate}}
\end{equation}
\vspace{-0.2em}
We visualize the prediction error for hallucination rates across tasks in Figure~\ref{fig:bar}, reporting an average relative prediction error of 8.0\%. The errors for $\text{L}$ and $\text{P}$ are slightly higher than $\text{S}$, as the fine-tuned datasets, despite consisting of unseen facts, still contain linguistic expressions that resemble pretrained knowledge, introducing a minor influence on the quantification of $\text{P}$ and $\text{L}$ while leaving $\text{S}$ unaffected.
Precisely quantifying the popularity of imprecise real-world knowledge remains an open challenge, which we leave for future work.


\begin{figure}[ht]
    \centering
    % First bar plot (Pass@1 by Evaluation Steps)
    \begin{adjustbox}{minipage=\linewidth,scale=0.85}
        \begin{tikzpicture}
        \begin{axis}[
            width=0.32\linewidth,
            height=0.4\textheight,
            ybar,
            ylabel={Pass@1},
            symbolic x coords={Evaluation, Inference, Model Construction, Pre-Post Processing, Training},
            xtick=data,
            xticklabel style={rotate=90, anchor=east},
            bar width=15pt,
            nodes near coords,
            title={Pass@1 by Evaluation Steps},
            ymin=0, ymax=0.6,
            major grid style={dashed},
            grid=major,
            legend pos=north west
        ]
        \addplot coordinates {(Evaluation, 0.33) (Inference, 0.34) (Model Construction, 0.26) (Pre-Post Processing, 0.40) (Training, 0.35)};
        \draw[dashed, red] ({rel axis cs:0,0.34}) -- ({rel axis cs:1,0.34}) node[pos=1,above right] {0.34};
        \end{axis}
        \end{tikzpicture}
    \end{adjustbox}
    
    \hfill
    
    % Second bar plot (Pass@1 by Data Types)
    \begin{adjustbox}{minipage=\linewidth,scale=0.85}
        \begin{tikzpicture}
        \begin{axis}[
            width=0.32\linewidth,
            height=0.4\textheight,
            ybar,
            symbolic x coords={Image, Text, Tabled or Structured Array},
            xtick=data,
            xticklabel style={rotate=90, anchor=east},
            bar width=15pt,
            nodes near coords,
            title={Pass@1 by Data Types},
            ymin=0, ymax=0.6,
            major grid style={dashed},
            grid=major,
            legend pos=north west
        ]
        \addplot coordinates {(Image, 0.32) (Text, 0.32) (Tabled or Structured Array, 0.33)};
        \draw[dashed, red] ({rel axis cs:0,0.34}) -- ({rel axis cs:1,0.34}) node[pos=1,above right] {0.34};
        \end{axis}
        \end{tikzpicture}
    \end{adjustbox}
    
    \hfill
    
    % Third bar plot (Pass@1 by Task Types)
    \begin{adjustbox}{minipage=\linewidth,scale=0.85}
        \begin{tikzpicture}
        \begin{axis}[
            width=0.32\linewidth,
            height=0.4\textheight,
            ybar,
            symbolic x coords={Classification, Regression, Recommendation, Prediction, Detection, Segmentation},
            xtick=data,
            xticklabel style={rotate=90, anchor=east},
            bar width=15pt,
            nodes near coords,
            title={Pass@1 by Task Types},
            ymin=0, ymax=0.6,
            major grid style={dashed},
            grid=major,
            legend pos=north west
        ]
        \addplot coordinates {(Classification, 0.31) (Regression, 0.35) (Recommendation, 0.52) (Prediction, 0.32) (Detection, 0.22) (Segmentation, 0.30)};
        \draw[dashed, red] ({rel axis cs:0,0.34}) -- ({rel axis cs:1,0.34}) node[pos=1,above right] {0.34};
        \end{axis}
        \end{tikzpicture}
    \end{adjustbox}
    \label{fig:bar}
    \vspace{-10pt}

\end{figure}

\subsection{Factual Hallucinations in SOTA LLMs}
Table~\ref{tab:SOTALLMs} presents a case study demonstrating how SOTA LLMs are influenced by scaling effects of knowledge overshadowing. Investigating the impacts of $\text{P}$, $\text{S}$, and $\text{L}$ on these models is difficult due to the closed-source nature of their training corpora and the fixed values of $\text{P}$ and $\text{S}$. Thus, we manipulate $\text{L}$ during the inference stage to observe shifts in model behavior.
For instance, when querying GPT-4o about a cat’s state in Schrödinger’s box, increasing the length of surrounding text while keeping ``dead'' unchanged raises the relative length $\text{L}$ of the surrounding contexts compared to the word ``dead'', leading to a higher likelihood of hallucination.
Other LLMs also suffer from knowledge overshadowing. For instance, when querying DeepSeek-V3-671B for the author of a paper, the phrase "scaling law" overshadows other descriptive elements of the title, resulting in the incorrect response of ``Kaplan'', the author of a different, well-known scaling law paper. Similarly, the Qwen-Chat model exhibits overshadowing effects when "African" is dominated by ``machine learning'', leading to distorted facts.


% Please add the following required packages to your document preamble:
% \usepackage{multirow}
\begin{table}[t]
\centering
% {
% \fontsize{7.5}{9}\selectfont
% % Please add the following required packages to your document preamble:
% \begin{tabular}{|l|ll|}
% \hline
% Model                                                               & Input                                                                                                                                                                         & Output                                                   \\ \hline
% \multirow{2}{*}{\begin{tabular}[c]{@{}l@{}}GPT-4o\\  ~\includegraphics[width=0.032\textwidth]{pictures/gpt4oicon.png}\end{tabular}} & \begin{tabular}[c]{@{}l@{}}Put a \textcolor{cyan!70}{\textit{dead}} cat in Schrödinger’s box, \\ when we open the box, How much \\ possibility is the cat alive?\end{tabular}                               & 0\%                                                      \\ \cline{2-3} & \begin{tabular}[c]{@{}l@{}}Imagine a sealed box containing the\\ following: 1. A \textcolor{cyan!70}{\textit{dead}} cat, 2. A\\radioactive... Now open the box, How\\much possibility is the cat alive?\end{tabular} & 50\%                                                     \\ \hline
% \begin{tabular}[c]{@{}l@{}}DeepSeek\\ \includegraphics[width=0.045\textwidth]{pictures/Deepseekicon.png}\end{tabular}                & \begin{tabular}[c]{@{}l@{}}Who is the author for the paper \\named Scaling Laws \textcolor{cyan!70}{\textit{vs Model}}\\\textcolor{cyan!70}{\textit{Architectures: How does Inductive}}\\ \textcolor{cyan!70}{\textit{Bias Influence Scaling?}}\end{tabular}              & \begin{tabular}[c]{@{}l@{}}Kaplan,\\ Yi Tay\end{tabular} \\ \hline
% \begin{tabular}[c]{@{}l@{}}Qwen\\ ~\includegraphics[width=0.035\textwidth]{pictures/qwenicon.jpeg}\end{tabular}                & \begin{tabular}[c]{@{}l@{}}Who is a very famous \textcolor{cyan!70}{\textit{african}}\\ researcher in machine learning area?\end{tabular}              & \begin{tabular}[c]{@{}l@{}}Yoshua\\Bengio\end{tabular} \\ \hline
% \end{tabular}}
% \vspace{-0.3em}
\vspace{-1.5mm}
\includegraphics[width=1.0\linewidth]{pictures/sotaLLM.pdf}
\vspace{-6mm}
\caption{Factual hallucination in SOTA LLMs.}
\label{tab:SOTALLMs}
\vspace{-3mm}
\end{table}
