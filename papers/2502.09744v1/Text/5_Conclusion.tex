\section{Conclusion}
In this paper, we evaluated the use of FL to fine-tuned FMs for medical time series forecasting, focusing on ECG and ICG data. Leveraging FMs within an FL framework has the potential to reduce the need for large, centralized medical datasets while preserving privacy. 
However, fine-tuning FMs in FL settings introduces unique challenges, particularly due to statistical heterogeneity across clients.

Through a series of experiments with different data partitioning strategies, we demonstrated that FL can effectively fine-tune FMs, leading to significant improvements over zero-shot models.
However, our findings also highlighted that the effectiveness of FL is highly dependent on the underlying data distribution. 
In scenarios where most clients have IID data, local fine-tuning outperformed FL, which could discourage clients from participating in FL due to minimal added benefit. This challenge is particularly relevant to FMs, as they are already well-generalized from pretraining on large, diverse datasets.

Despite these limitations, FL remains a promising approach for adapting FMs in privacy-sensitive medical applications. 
Our work underscores the need for advanced techniques to address statistical heterogeneity in FL, ensuring more robust and generalizable models for healthcare.

Future research should explore methods that strike a balance between localized fine-tuning and collaborative learning across clients. 
While we experimented with a hybrid approach where a globally fine-tuned FM was later adapted locally, further convergence analysis is needed to better understand its impact. 
Additionally, meta-learning approaches, such as Model-Agnostic Meta-Learning (MAML) \cite{10.5555/3305381.3305498}, offer a compelling direction. 
Personalized Federated Learning (PFL) techniques \cite{NEURIPS2020_24389bfe}, which integrate meta-learning with FL, could enhance adaptability by enabling FMs to be fine-tuned on local clients with better convergence guarantees. Such approaches may offer a more effective trade-off between FL and local fine-tuning, improving both performance and participation incentives.

By systematically evaluating these strategies, we can further enhance the robustness of FL across diverse medical data distributions. 
Overall, integrating FMs into FL frameworks has the potential to advance medical AI by reducing data requirements while preserving privacy, ultimately contributing to improved patient outcomes.