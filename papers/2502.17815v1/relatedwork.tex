\section{Related Work}
\label{L_R}
In the field of computer vision technology, images from various sectors such as military, medical, and industrial areas are of great importance. Although significant progress has been made in classical image and video compression methods, as discussed in ~\cite{paul2005real} and ~\cite{paul2018efficient} are not suitable for quantum image representation and compression. However, the emergence of quantum computing has opened up new possibilities for quantum image processing (QIP) ~\cite{b11}. It employs a quantum concept framework that utilizes quantum logic gates to perform image operations and address storage issues associated with classical images. By leveraging the power of quantum state superposition and entanglement theories, QIP greatly enhances image processing capabilities. Notably, within the realm of QIP, quantum computing technology ~\cite{yan2017quantum}, image watermarking ~\cite{song2014dynamic, song2013dynamic, iliyasu2012watermarking, hu2019quantum, kong2010color}, image compression ~\cite{b13, pm}, image storage and retrieval ~\cite{li2014multi}, image encryption, quantum security~\cite{peelam2024explorative}, among others, have shown significant potential ~\cite{wang2022efficient, wang2014novel}. In quantum encryption, post-quantum security becomes another rising field~\cite{khan2024chaotic}. The artificial intelligence-based quantum Synergies approach is another promising field of quantum computing in software-defined consumer applications~\cite{awan2024artificial,wu2022state}. 

First, to address the issue of representing images on a quantum processor, a qubit lattice approach is used to represent the four different random pixels of an image using a single qubit ~\cite{b10}. It is the first approach to demonstrate the storage and retrieval of quantum images in a multi-particle quantum system. However, due to the use of a single qubit, it is difficult to represent both the image pixels and their positions. The real-ket approach is another approach to represent quantum images, which utilizes the ket function of the qubit ~\cite{b12}. However, it has limitations when it comes to representing larger, real-size images in the qubit system due to the limited number of pixel values such as four. In the rapidly evolving field of computer vision technology, the significance of images from diverse sectors—including military, medical, and industrial—cannot be overstated. While remarkable advancements have been achieved in classical image and video compression techniques, as highlighted in ~\cite{paul2005real} and ~\cite{paul2018efficient}, these conventional methods fall short when it comes to quantum image representation and compression. The advent of quantum computing has ushered in groundbreaking opportunities for quantum image processing (QIP)~\cite{b11}. By harnessing a quantum conceptual framework and employing quantum logic gates, QIP performs sophisticated image operations and effectively tackles the storage challenges that classical methods face. Through the utilization of quantum state superposition and entanglement, QIP dramatically elevates image processing capabilities to new heights.

Within this dynamic realm, QIP has demonstrated substantial potential across various applications, including quantum computing technology ~\cite{yan2017quantum}, image watermarking ~\cite{song2014dynamic, song2013dynamic,iliyasu2012watermarking, hu2019quantum, kong2010color}, image compression ~\cite{b13, pm}, image storage and retrieval ~\cite{li2014multi}, and image encryption ~\cite{wang2022efficient, wang2014novel}. These advancements reveal the true power of quantum-based methodologies.

To effectively tackle the challenge of representing images on quantum processors, innovative approaches are being explored. The qubit lattice approach, for instance, allows the representation of four distinct random pixels of an image using a single qubit ~\cite{b10}. This groundbreaking method is the first to showcase the storage and retrieval of quantum images in a multi-particle quantum system. Nonetheless, the limitation of using a single qubit complicates the representation of both image pixels and their spatial positioning.

Another promising approach, known as the real-ket method, leverages the ket function of qubits to represent quantum images ~\cite{b12}. However, this method too encounters challenges when it comes to representing larger, real-world images due to the confined number of pixel values it can handle, specifically four. It is crucial to continue exploring these avenues to fully unlock the transformative potential of quantum image processing.

Shor and Grover proposed a faster approach than classical computation for factorial calculation and database search ~\cite{b5, b6}. In 2011, an FRQI (flexible representation of the quantum image) approach was Shor and Grover proposed a faster method for factorial calculation and database searching compared to classical computation ~\cite{b5, b6}. In 2011, a Flexible Representation of the Quantum Image (FRQI) approach was introduced, which takes into account the probability of pixel values and normalizes the state for mapping ~\cite{b13}. However, this approach has a significant limitation: it relies on probabilistic outcomes. Additionally, a quality measurement of the reconstructed image is necessary. Figure~\ref{FRQI_A} illustrates the quantum image for the FRQI approach, showcasing its state for a \(2\times2\) image which considers the probability of pixel values and normalizes the state for mapping~\cite{b13}. However, it has a significant limitation: it relies on probabilistic outcomes—furthermore, the quality measurement of the reconstructed image needed to be conducted. Figure~\ref{FRQI_A} depicts the quantum image for the FRQI approach, including its state for a $2\times2$ image. 

\begin{figure}[htbp]
\centerline{\includegraphics[width=0.30\linewidth,height=0.30\linewidth]{FRQI_2_2_Gray.png}}
\caption{A $2\times2$ FRQI quantum image}
\label{FRQI_A}
\end{figure}

The equation below represents the mathematical expression of the FRQI image, $|I_{FRQI}\rangle$.\\ 


\begin{math}
%\resizebox{1\hsize}{!}{$
|I_{FRQI}\rangle=\frac{1}{2} [\left(cos\theta_{0}|0\rangle +sin\theta_{0}|1\rangle \right) \otimes |00\rangle 
+\left(cos\theta_{1}|0\rangle+sin\theta_{1}|1\rangle \right)\otimes|01\rangle +  \left(cos\theta_{2}|0\rangle+sin\theta_{2}|1\rangle \right)\otimes|10\rangle+ \left(cos\theta_{3}|0\rangle+sin\theta_{3}|1\rangle \right)\otimes|11\rangle ]\nonumber
%$}
\end{math}
%%

Figure~\ref{FRQI_Circuit} shows FRQI circuit of the image $|I_{FRQI}\rangle$.  

\begin{figure}[t!]
\centerline{\includegraphics[width=0.8\linewidth,height=4cm]{circuit_FRQI.png}}
\caption{An FRQI circuit for representing $2\times2$ image}
\label{FRQI_Circuit}
\end{figure}

The formula for the standard rotation metric, $R_y(2\theta)$, is expressed as follows, 
$\begin{pmatrix}
\centering
  cos\theta_i & -sin\theta_i \\ 
  sin\theta_i & cos\theta_i
\end{pmatrix}$ 

Where $\theta$ is the angle of the corresponding qubits, it represents the image pixel values in the quantum domain using the control rotation matrices. The control rotational matrices were implemented using a c-not gate and standard rotation. It cannot represent pixel-wise gray-scale complex operations since it uses only one qubit. It encodes the color and position of the image using the associated angle and kets of one qubit. It can only represent four numbers of pixel values of the image. Generally, it uses rotating gates to store gray-scale pixel values in the probabilistic-based amplitude mapping in the Bloch sphere. Bloch sphere is the geometry of a vector, whose unitary transformation is a rotating matrix indicating the magnitude of the state purity ~\cite{gamel2016entangled}. 

An entanglement-based image representation approach was proposed using the image entanglement theory to represent image data in the quantum circuit ~\cite{b14}. It removes any additional required information, such as correlation. Additionally, it can convert image data into a normalized state using a reduced number of operations. It maps the pixel value directly into an angle for a binary image. More sufficient details about quantum image knowledge should have been provided. Moreover, it is limited to small sizes of images, which are not suitable for real-life applications.

A NEQR approach was proposed to address the classical gray-scale image representation issue of the FRQI method ~\cite{bb10}. It resolves the FRQI representation issue because it provides a way to represent the pixel-based representation of the gray-scale image. After converting the pixel values into a binary system, only a frequent number of ones are considered to map the pixel value in a quantum system. The pixel and state (position) representing qubits map each pixel and the corresponding position values. For example, an image whose pixel values are $0 (Y=0, X=0), 100 (Y=0, X=1), 200 (Y=1, X=0), 255 (Y=1, X=1)$ and its respective quantum representations are known as the NEQR approach and express as,

\begin{flalign}
\resizebox{0.9\hsize}{!}{$
|I_{NEQR}\rangle=\frac{1}{2}[|0\rangle\otimes|00\rangle 
+|100\rangle\otimes|01\rangle+|200\rangle\otimes|10\rangle+|255\rangle \otimes|11\rangle]\nonumber
$}
\end{flalign}

Figure~\ref{NEQR_Circuit} depicts the circuit diagram for the $|I_{NEQR}\rangle$ image, which effectively addresses the FRQI issues by utilizing multiple qubits to encode the gray-scale image pixels. Nonetheless, several limitations exist associated with this approach. Firstly, it is incapable of representing medium or high-resolution images that exceed the dimensions of $4\times4$ due to the increased connectivity required for additional qubits. Furthermore, the outcome measurement approach is probabilistic, which significantly hampers the accurate restoration of the original image. Moreover, images with rectangular shapes cannot be accurately represented.
\begin{figure*}[htbp]
\centerline{\includegraphics[width=\linewidth,height=6cm]{NEQR_Approach.png}}
\caption{An NEQR circuit diagram for pixel values representation}
\label{NEQR_Circuit}
\end{figure*}

For real-size images, how and in what way to represent them still needs to be clarified. INEQR (improved NEQR) approach was proposed in ~\cite{b9} to resolve the rectangular shape images issue of the NEQR approach; it still needs to clarify how the color and large-size images are represented. Also, it is only able to represent images of very tiny sizes. Moreover, the mapping procedure is probabilistic. As a result, the measured reconstructed image quality needs to be performed due to the non-deterministic outcome. 
%%%%%

In 2018, Li et al. introduced an Optimized Quantum Representation for Color Images (OCQR) approach, which focuses on the representation of color image pixels ~\cite{liu2018optimized}. This method requires more qubits than the NEQR approach but utilizes fewer qubits compared to the NCQI approach, due to the incorporation of index values. However, it is limited to image sizes of $2 \times 2$. In the same year, Nanrun et al.(2018) proposed a Quantum Representation Model for Multiple Images (QRMMI) ~\cite{zhou2018multi}. This model aims to represent multiple images simultaneously, thereby conserving hardware space for image representation. However, it relies on the NEQR approach for the state-label circuit connection, which necessitates a large number of qubit connections, leading to increased circuit complexity.

Following this, a General Quantum Image Representation (GQIR) approach was presented in \cite{b18}. This approach uses a logarithmic scale to represent images of both square and rectangular sizes. For an image size of $512 \times 512$, it requires nine qubits to encode the $Y$-positions and an additional nine qubits for the $X$-positions. Figure~\ref{GQIR_Circuit} illustrates a quantum circuit example for a $16 \times 16$ image of a deer, which is generated after applying the DCT preparation approach and utilizing 70 quantization factors with the Quirk simulation tool~\cite{bb17}. The transform coefficient values for $16\times 16$ size deer image  are, $126(X = 1, Y = 1)$, $1(X = 1, Y = 0)$, $1(X = 4, Y = 0)$, $126(X = 8, Y = 0)$, $4(X = 0, Y = 1)$, $1(X = 8, Y = 1)$, $1(X = 9, Y = 1)$, $1(X = 8, Y = 1)$, $1(X = 8, Y = 5)$, $138(X = 0, Y = 8)$, $140(X = 8, Y = 8)$, $1(X = 12, Y = 8)$, $2(X = 0, Y = 9)$, $2(X = 8, Y = 9)$ and $1(X = 2, Y = 11)$. 

\begin{figure*}[htbp]
\centerline{\includegraphics[width=\linewidth,height=8cm]{GQIR_image.png}}
\caption{A GQIR circuit for representing pixel values}
\label{GQIR_Circuit}
\end{figure*}

The use of zero for connectivity results in the generation of redundant qubits connection. Additionally, representing medium to high-resolution images using a GQIR circuit is challenging due to the probabilistic mapping involved. Consequently, reconstructing the original image after encoding in the quantum domain is nearly impossible because of its non-deterministic outcomes. Furthermore, Jiang et al. proposed a compression method that combines the GQIR and DCT preparation approaches based on the Jpeg (Joint Photographic Experts Group) standard. Another study introduced a quantum-based equivalence pixel image derived from a bit pixel image. The use of zero for connectivity leads to the generation of redundant qubits connection. Additionally, representing medium or higher-resolution images using a GQIR circuit is challenging because of the probabilistic mapping. It is nearly impossible to reconstruct the original image after encoding in the quantum domain due to non-deterministic outcomes. Moreover, Jiang et al. proposed a compression approach that utilizes the GQIR and DCT preparation approaches based on JPEG (joint photography expert group) image~\cite{pm}. Another study proposed a quantum-based equivalence pixel image from a bit pixel image~\cite{b15}.

\par An efficient flexible representation of the quantum image (EFRQI) was proposed to decrease the state preparation complexity of GQIR and NEQR approaches~\cite{b20}. Figure~\ref{EFRQI_circuit} shows the EFRQI circuit of the gray-scale image whose pixel values are $205(X=1, Y=0)$,$49(X=0, Y=1)$, and $255(X=1, Y=1)$. Each pixel value connection uses two Toffoli gates, one for initiating pixel connection and another for closing. The benefit of this approach is that it does not depend on the number one that occurs in each pixel for pixel position connection. As a result, it saves the position values more than the GQIR approach. Due to the use of the same Toffoli gate twice for each pixel connection, it generates a higher amount of required qubits connection. To address the higher required quantum resources issue of the EFRQI approach, Haque et al. (2022)~\cite{b9} proposed a DCTEFRQI approach that integrates the DCT approach with the EFRQI circuit. It outperforms the EFRQI approach. Plenty of gates are still required due to the use of Toffoli gates. 

\begin{figure*}[htbp]
\centerline{\includegraphics[width=\linewidth,height=8cm]{EFRQI_Circuit.png}}
\caption{An EFRQI circuit for pixel values representation}
\label{EFRQI_circuit}
\end{figure*}

\par In~\cite{bb17}, a novel reset gate called the SCMFRQI approach is introduced as a replacement for the closing Toffoli gate connection in the EFRQI circuit. Figure~\ref{fig_proposed_Chematic_diagra} illustrates the circuit of this approach, which represents transform coefficient values of $125(X=0, Y=0), 1(X=1, Y=0), 1(X=4, Y=0), 4(X=0, Y=1),$ and $16(Y=3, X=0)$ ~\cite{haque2022novel}. Instead of using the Toffoli gate twice, it utilizes a single reset gate for each coefficient connection (shown in the green circle) ~\cite{bb17}. Consequently, the required number of gates is reduced compared to the DCTEFRQI approach.

\begin{figure*}[htbp]
\centerline{\includegraphics[width=\linewidth,height=0.45\linewidth]{SCMEFRQI_Approch.png}}
\caption{A SCMFRQI circuit for transformed coefficient value representation.}
\label{fig_proposed_Chematic_diagra}
\end{figure*}