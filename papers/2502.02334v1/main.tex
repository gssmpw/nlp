%% bare_jrnl.tex
%% V1.4b
%% 2015/08/26
%% by Michael Shell
%% see http://www.michaelshell.org/
%% for current contact information.
%%
%% This is a skeleton file demonstrating the use of IEEEtran.cls
%% (requires IEEEtran.cls version 1.8b or later) with an IEEE
%% journal paper.
%%
%% Support sites:
%% http://www.michaelshell.org/tex/ieeetran/
%% http://www.ctan.org/pkg/ieeetran
%% and
%% http://www.ieee.org/

%%*************************************************************************
%% Legal Notice:
%% This code is offered as-is without any warranty either expressed or
%% implied; without even the implied warranty of MERCHANTABILITY or
%% FITNESS FOR A PARTICULAR PURPOSE! 
%% User assumes all risk.
%% In no event shall the IEEE or any contributor to this code be liable for
%% any damages or losses, including, but not limited to, incidental,
%% consequential, or any other damages, resulting from the use or misuse
%% of any information contained here.
%%
%% All comments are the opinions of their respective authors and are not
%% necessarily endorsed by the IEEE.
%%
%% This work is distributed under the LaTeX Project Public License (LPPL)
%% ( http://www.latex-project.org/ ) version 1.3, and may be freely used,
%% distributed and modified. A copy of the LPPL, version 1.3, is included
%% in the base LaTeX documentation of all distributions of LaTeX released
%% 2003/12/01 or later.
%% Retain all contribution notices and credits.
%% ** Modified files should be clearly indicated as such, including  **
%% ** renaming them and changing author support contact information. **
%%*************************************************************************


% *** Authors should verify (and, if needed, correct) their LaTeX system  ***
% *** with the testflow diagnostic prior to trusting their LaTeX platform ***
% *** with production work. The IEEE's font choices and paper sizes can   ***
% *** trigger bugs that do not appear when using other class files.       ***                          ***
% The testflow support page is at:
% http://www.michaelshell.org/tex/testflow/



\documentclass[journal]{IEEEtran}
%
% If IEEEtran.cls has not been installed into the LaTeX system files,
% manually specify the path to it like:
% \documentclass[journal]{../sty/IEEEtran}





% Some very useful LaTeX packages include:
% (uncomment the ones you want to load)


% *** MISC UTILITY PACKAGES ***
%
%\usepackage{ifpdf}
% Heiko Oberdiek's ifpdf.sty is very useful if you need conditional
% compilation based on whether the output is pdf or dvi.
% usage:
% \ifpdf
%   % pdf code
% \else
%   % dvi code
% \fi
% The latest version of ifpdf.sty can be obtained from:
% http://www.ctan.org/pkg/ifpdf
% Also, note that IEEEtran.cls V1.7 and later provides a builtin
% \ifCLASSINFOpdf conditional that works the same way.
% When switching from latex to pdflatex and vice-versa, the compiler may
% have to be run twice to clear warning/error messages.






% *** CITATION PACKAGES ***
%
%\usepackage{cite}
% cite.sty was written by Donald Arseneau
% V1.6 and later of IEEEtran pre-defines the format of the cite.sty package
% \cite{} output to follow that of the IEEE. Loading the cite package will
% result in citation numbers being automatically sorted and properly
% "compressed/ranged". e.g., [1], [9], [2], [7], [5], [6] without using
% cite.sty will become [1], [2], [5]--[7], [9] using cite.sty. cite.sty's
% \cite will automatically add leading space, if needed. Use cite.sty's
% noadjust option (cite.sty V3.8 and later) if you want to turn this off
% such as if a citation ever needs to be enclosed in parenthesis.
% cite.sty is already installed on most LaTeX systems. Be sure and use
% version 5.0 (2009-03-20) and later if using hyperref.sty.
% The latest version can be obtained at:
% http://www.ctan.org/pkg/cite
% The documentation is contained in the cite.sty file itself.






% *** GRAPHICS RELATED PACKAGES ***
%
\ifCLASSINFOpdf
  % \usepackage[pdftex]{graphicx}
  % declare the path(s) where your graphic files are
  % \graphicspath{{../pdf/}{../jpeg/}}
  % and their extensions so you won't have to specify these with
  % every instance of \includegraphics
  % \DeclareGraphicsExtensions{.pdf,.jpeg,.png}
\else
  % or other class option (dvipsone, dvipdf, if not using dvips). graphicx
  % will default to the driver specified in the system graphics.cfg if no
  % driver is specified.
  % \usepackage[dvips]{graphicx}
  % declare the path(s) where your graphic files are
  % \graphicspath{{../eps/}}
  % and their extensions so you won't have to specify these with
  % every instance of \includegraphics
  % \DeclareGraphicsExtensions{.eps}
\fi
% graphicx was written by David Carlisle and Sebastian Rahtz. It is
% required if you want graphics, photos, etc. graphicx.sty is already
% installed on most LaTeX systems. The latest version and documentation
% can be obtained at: 
% http://www.ctan.org/pkg/graphicx
% Another good source of documentation is "Using Imported Graphics in
% LaTeX2e" by Keith Reckdahl which can be found at:
% http://www.ctan.org/pkg/epslatex
%
% latex, and pdflatex in dvi mode, support graphics in encapsulated
% postscript (.eps) format. pdflatex in pdf mode supports graphics
% in .pdf, .jpeg, .png and .mps (metapost) formats. Users should ensure
% that all non-photo figures use a vector format (.eps, .pdf, .mps) and
% not a bitmapped formats (.jpeg, .png). The IEEE frowns on bitmapped formats
% which can result in "jaggedy"/blurry rendering of lines and letters as
% well as large increases in file sizes.
%
% You can find documentation about the pdfTeX application at:
% http://www.tug.org/applications/pdftex





% *** MATH PACKAGES ***
%
%\usepackage{amsmath}
% A popular package from the American Mathematical Society that provides
% many useful and powerful commands for dealing with mathematics.
%
% Note that the amsmath package sets \interdisplaylinepenalty to 10000
% thus preventing page breaks from occurring within multiline equations. Use:
%\interdisplaylinepenalty=2500
% after loading amsmath to restore such page breaks as IEEEtran.cls normally
% does. amsmath.sty is already installed on most LaTeX systems. The latest
% version and documentation can be obtained at:
% http://www.ctan.org/pkg/amsmath





% *** SPECIALIZED LIST PACKAGES ***
%
%\usepackage{algorithmic}
% algorithmic.sty was written by Peter Williams and Rogerio Brito.
% This package provides an algorithmic environment fo describing algorithms.
% You can use the algorithmic environment in-text or within a figure
% environment to provide for a floating algorithm. Do NOT use the algorithm
% floating environment provided by algorithm.sty (by the same authors) or
% algorithm2e.sty (by Christophe Fiorio) as the IEEE does not use dedicated
% algorithm float types and packages that provide these will not provide
% correct IEEE style captions. The latest version and documentation of
% algorithmic.sty can be obtained at:
% http://www.ctan.org/pkg/algorithms
% Also of interest may be the (relatively newer and more customizable)
% algorithmicx.sty package by Szasz Janos:
% http://www.ctan.org/pkg/algorithmicx




% *** ALIGNMENT PACKAGES ***
%
%\usepackage{array}
% Frank Mittelbach's and David Carlisle's array.sty patches and improves
% the standard LaTeX2e array and tabular environments to provide better
% appearance and additional user controls. As the default LaTeX2e table
% generation code is lacking to the point of almost being broken with
% respect to the quality of the end results, all users are strongly
% advised to use an enhanced (at the very least that provided by array.sty)
% set of table tools. array.sty is already installed on most systems. The
% latest version and documentation can be obtained at:
% http://www.ctan.org/pkg/array


% IEEEtran contains the IEEEeqnarray family of commands that can be used to
% generate multiline equations as well as matrices, tables, etc., of high
% quality.




% *** SUBFIGURE PACKAGES ***
%\ifCLASSOPTIONcompsoc
%  \usepackage[caption=false,font=normalsize,labelfont=sf,textfont=sf]{subfig}
%\else
%  \usepackage[caption=false,font=footnotesize]{subfig}
%\fi
% subfig.sty, written by Steven Douglas Cochran, is the modern replacement
% for subfigure.sty, the latter of which is no longer maintained and is
% incompatible with some LaTeX packages including fixltx2e. However,
% subfig.sty requires and automatically loads Axel Sommerfeldt's caption.sty
% which will override IEEEtran.cls' handling of captions and this will result
% in non-IEEE style figure/table captions. To prevent this problem, be sure
% and invoke subfig.sty's "caption=false" package option (available since
% subfig.sty version 1.3, 2005/06/28) as this is will preserve IEEEtran.cls
% handling of captions.
% Note that the Computer Society format requires a larger sans serif font
% than the serif footnote size font used in traditional IEEE formatting
% and thus the need to invoke different subfig.sty package options depending
% on whether compsoc mode has been enabled.
%
% The latest version and documentation of subfig.sty can be obtained at:
% http://www.ctan.org/pkg/subfig




% *** FLOAT PACKAGES ***
%
%\usepackage{fixltx2e}
% fixltx2e, the successor to the earlier fix2col.sty, was written by
% Frank Mittelbach and David Carlisle. This package corrects a few problems
% in the LaTeX2e kernel, the most notable of which is that in current
% LaTeX2e releases, the ordering of single and double column floats is not
% guaranteed to be preserved. Thus, an unpatched LaTeX2e can allow a
% single column figure to be placed prior to an earlier double column
% figure.
% Be aware that LaTeX2e kernels dated 2015 and later have fixltx2e.sty's
% corrections already built into the system in which case a warning will
% be issued if an attempt is made to load fixltx2e.sty as it is no longer
% needed.
% The latest version and documentation can be found at:
% http://www.ctan.org/pkg/fixltx2e


%\usepackage{stfloats}
% stfloats.sty was written by Sigitas Tolusis. This package gives LaTeX2e
% the ability to do double column floats at the bottom of the page as well
% as the top. (e.g., "\begin{figure*}[!b]" is not normally possible in
% LaTeX2e). It also provides a command:
%\fnbelowfloat
% to enable the placement of footnotes below bottom floats (the standard
% LaTeX2e kernel puts them above bottom floats). This is an invasive package
% which rewrites many portions of the LaTeX2e float routines. It may not work
% with other packages that modify the LaTeX2e float routines. The latest
% version and documentation can be obtained at:
% http://www.ctan.org/pkg/stfloats
% Do not use the stfloats baselinefloat ability as the IEEE does not allow
% \baselineskip to stretch. Authors submitting work to the IEEE should note
% that the IEEE rarely uses double column equations and that authors should try
% to avoid such use. Do not be tempted to use the cuted.sty or midfloat.sty
% packages (also by Sigitas Tolusis) as the IEEE does not format its papers in
% such ways.
% Do not attempt to use stfloats with fixltx2e as they are incompatible.
% Instead, use Morten Hogholm'a dblfloatfix which combines the features
% of both fixltx2e and stfloats:
%
% \usepackage{dblfloatfix}
% The latest version can be found at:
% http://www.ctan.org/pkg/dblfloatfix




%\ifCLASSOPTIONcaptionsoff
%  \usepackage[nomarkers]{endfloat}
% \let\MYoriglatexcaption\caption
% \renewcommand{\caption}[2][\relax]{\MYoriglatexcaption[#2]{#2}}
%\fi
% endfloat.sty was written by James Darrell McCauley, Jeff Goldberg and 
% Axel Sommerfeldt. This package may be useful when used in conjunction with 
% IEEEtran.cls'  captionsoff option. Some IEEE journals/societies require that
% submissions have lists of figures/tables at the end of the paper and that
% figures/tables without any captions are placed on a page by themselves at
% the end of the document. If needed, the draftcls IEEEtran class option or
% \CLASSINPUTbaselinestretch interface can be used to increase the line
% spacing as well. Be sure and use the nomarkers option of endfloat to
% prevent endfloat from "marking" where the figures would have been placed
% in the text. The two hack lines of code above are a slight modification of
% that suggested by in the endfloat docs (section 8.4.1) to ensure that
% the full captions always appear in the list of figures/tables - even if
% the user used the short optional argument of \caption[]{}.
% IEEE papers do not typically make use of \caption[]'s optional argument,
% so this should not be an issue. A similar trick can be used to disable
% captions of packages such as subfig.sty that lack options to turn off
% the subcaptions:
% For subfig.sty:
% \let\MYorigsubfloat\subfloat
% \renewcommand{\subfloat}[2][\relax]{\MYorigsubfloat[]{#2}}
% However, the above trick will not work if both optional arguments of
% the \subfloat command are used. Furthermore, there needs to be a
% description of each subfigure *somewhere* and endfloat does not add
% subfigure captions to its list of figures. Thus, the best approach is to
% avoid the use of subfigure captions (many IEEE journals avoid them anyway)
% and instead reference/explain all the subfigures within the main caption.
% The latest version of endfloat.sty and its documentation can obtained at:
% http://www.ctan.org/pkg/endfloat
%
% The IEEEtran \ifCLASSOPTIONcaptionsoff conditional can also be used
% later in the document, say, to conditionally put the References on a 
% page by themselves.




% *** PDF, URL AND HYPERLINK PACKAGES ***
%
%\usepackage{url}
% url.sty was written by Donald Arseneau. It provides better support for
% handling and breaking URLs. url.sty is already installed on most LaTeX
% systems. The latest version and documentation can be obtained at:
% http://www.ctan.org/pkg/url
% Basically, \url{my_url_here}.




% *** Do not adjust lengths that control margins, column widths, etc. ***
% *** Do not use packages that alter fonts (such as pslatex).         ***
% There should be no need to do such things with IEEEtran.cls V1.6 and later.
% (Unless specifically asked to do so by the journal or conference you plan
% to submit to, of course. )


% correct bad hyphenation here
\hyphenation{op-tical net-works semi-conduc-tor}

\usepackage[table,xcdraw]{xcolor}
\usepackage{amssymb}
\newcommand{\YKL}[1]{\textcolor{red}{#1}}
\definecolor{rblue}{rgb}{0,0.5,1}
\definecolor{awesome}{rgb}{1.0, 0.13, 0.32}
\definecolor{hollywoodcerise}{rgb}{0.96, 0.0, 0.63}
\definecolor{lasallegreen}{rgb}{0.03, 0.47, 0.19}
\definecolor{hanpurple}{rgb}{0.32, 0.09, 0.98}
\definecolor{green(pigment)}{rgb}{0.0, 0.65, 0.31}

\usepackage[pagebackref=false,breaklinks=true,colorlinks,bookmarks=false]{hyperref}
\hypersetup{colorlinks=true,linkcolor={red},citecolor={hanpurple},urlcolor={magenta}}
\usepackage{times}
\usepackage{epsfig}
\usepackage{graphicx}
\usepackage{amsmath}
\usepackage{amssymb}
\usepackage{array}
\usepackage{float}
\usepackage{placeins}
\usepackage{morewrites}
\usepackage{makecell}
% \usepackage{subtitle}



% borrowed packages from monoscene
%\usepackage{xcolor}
\usepackage{caption}
\usepackage{subcaption}
\usepackage{tabularx}
\usepackage{rotating}
\usepackage{verbatim}
%\usepackage{xcolor,colortbl}
\usepackage{etoolbox}
\usepackage[normalem]{ulem}
\usepackage{booktabs}
\usepackage{multirow}
\usepackage{lipsum}
\usepackage{stmaryrd}
\usepackage{stackengine}
\usepackage{makecell}
\usepackage{pifont}
\usepackage{cancel}
\usepackage{adjustbox}
\usepackage{dblfloatfix}
\usepackage[table]{xcolor}
%\usepackage[pagebackref=true,breaklinks=true,letterpaper=true,colorlinks,bookmarks=false]{hyperref}
%\hypersetup{colorlinks}

%\definecolor{cvprblue}{rgb}{0.21,0.49,0.74}
%\usepackage[pagebackref,breaklinks,colorlinks,allcolors=cvprblue]{hyperref}

\newcommand{\cmark}{\ding{51}}%
\newcommand{\xmark}{\ding{55}}%

\DeclareMathOperator*{\argmin}{argmin}
\DeclareMathOperator*{\argmax}{argmax}

\newcommand{\Cbb}{\ensuremath{\mathbb{C}}} 
\newcommand{\Rbb}{\ensuremath{\mathbb{R}}} 
\newcommand{\Nbb}{\ensuremath{\mathbb{N}}}

\newcommand{\camera}{\includegraphics[width=3.8mm]{icons/camera.png}}
\newcommand{\LiDAR}{\includegraphics[width=4mm]{icons/LiDAR.png}}
\newcommand{\event}{\includegraphics[width=4mm]{icons/event.png}}
\newcommand{\radar}{\includegraphics[width=4mm]{icons/radar.png}}
\newcommand{\tick}{\includegraphics[width=4mm]{icons/tick.png}}
\newcommand{\cross}{\includegraphics[width=4mm]{icons/cross.png}}
\newcommand{\cir}{\includegraphics[width=4mm]{icons/circle.png}}

\newcommand{\real}{\includegraphics[width=4mm]{icons/real.png}}
\newcommand{\syn}{\includegraphics[width=4mm]{icons/syn.png}}

\usepackage[capitalize]{cleveref}
\crefname{section}{Sec.}{Secs.}
\Crefname{section}{Section}{Sections}
\Crefname{table}{Table}{Tables}
\crefname{table}{Tab.}{Tabs.}
\usepackage{colortbl}
%\usepackage[table,xcdraw]{xcolor}
% \pdfminorversion=5


% define colormap for SemanticKITTI
\definecolor{car_s}{rgb}{0.39215686, 0.58823529, 0.96078431}
\definecolor{bicycle_s}{rgb}{0.39215686, 0.90196078, 0.96078431}
\definecolor{motorcycle_s}{rgb}{0.11764706, 0.23529412, 0.58823529}
\definecolor{truck_s}{rgb}{0.31372549, 0.11764706, 0.70588235}
\definecolor{other-vehicle_s}{rgb}{0.39215686, 0.31372549, 0.98039216}
\definecolor{person_s}{rgb}{1.        , 0.11764706, 0.11764706}
\definecolor{bicyclist_s}{rgb}{1.        , 0.15686275, 0.78431373}
\definecolor{motorcyclist_s}{rgb}{0.58823529, 0.11764706, 0.35294118}
\definecolor{road_s}{rgb}{1.        , 0.        , 1.        }
\definecolor{parking_s}{rgb}{1.        , 0.58823529, 1.        }
\definecolor{sidewalk_s}{rgb}{0.29411765, 0.        , 0.29411765}
\definecolor{other-ground_s}{rgb}{0.68627451, 0.        , 0.29411765}
\definecolor{building_s}{rgb}{1.        , 0.78431373, 0.        }
\definecolor{fence_s}{rgb}{1.        , 0.47058824, 0.19607843}
\definecolor{vegetation_s}{rgb}{0.        , 0.68627451, 0.        }
\definecolor{trunk_s}{rgb}{0.52941176, 0.23529412, 0.        }
\definecolor{terrain_s}{rgb}{0.58823529, 0.94117647, 0.31372549}
\definecolor{pole_s}{rgb}{1.        , 0.94117647, 0.58823529}
\definecolor{traffic-sign_s}{rgb}{1.        , 0.        , 0.    }

% define colormap for Dsec
\definecolor{car_d}{rgb}{0.0, 0.0, 0.55686275}
\definecolor{bicycle_d}{rgb}{0.46666667, 0.04313725, 0.12549020}
\definecolor{motorcycle_d}{rgb}{0.0, 0.0, 0.90196078}
\definecolor{truck_d}{rgb}{0.0, 0.0, 0.27450980}
\definecolor{other-vehicle_d}{rgb}{0.90196078, 0.58823529, 0.54901961}
\definecolor{person_d}{rgb}{0.86274510, 0.07843137, 0.23529412}
\definecolor{road_d}{rgb}{0.50196078, 0.25098039, 0.50196078}
\definecolor{sidewalk_d}{rgb}{0.95686275, 0.13725490, 0.90980392}
\definecolor{building_d}{rgb}{0.27450980, 0.27450980, 0.27450980}
\definecolor{fence_d}{rgb}{0.74509804, 0.60000000, 0.60000000}
\definecolor{vegetation_d}{rgb}{0.41960784, 0.55686275, 0.13725490}
\definecolor{terrain_d}{rgb}{0.59607843, 0.98431373, 0.59607843}
\definecolor{pole_d}{rgb}{0.60000000, 0.60000000, 0.60000000}
\definecolor{traffic-sign_d}{rgb}{0.86274510, 0.86274510, 0.0}
% define class-frequencies for SemanticKITTI
\makeatletter
\newcommand{\car@semkitfreq}{3.92}
\newcommand{\bicycle@semkitfreq}{0.03}
\newcommand{\motorcycle@semkitfreq}{0.03}
\newcommand{\truck@semkitfreq}{0.16}
\newcommand{\othervehicle@semkitfreq}{0.20}
\newcommand{\person@semkitfreq}{0.07}
\newcommand{\bicyclist@semkitfreq}{0.07}
\newcommand{\motorcyclist@semkitfreq}{0.05}
\newcommand{\road@semkitfreq}{15.30}  %
\newcommand{\parking@semkitfreq}{1.12}
\newcommand{\sidewalk@semkitfreq}{11.13}  %
\newcommand{\otherground@semkitfreq}{0.56}
\newcommand{\building@semkitfreq}{14.1}  %
\newcommand{\fence@semkitfreq}{3.90}
\newcommand{\vegetation@semkitfreq}{39.3}  %
\newcommand{\trunk@semkitfreq}{0.51}
\newcommand{\terrain@semkitfreq}{9.17} %
\newcommand{\pole@semkitfreq}{0.29}
\newcommand{\trafficsign@semkitfreq}{0.08}
\newcommand{\semkitfreq}[1]{{\csname #1@semkitfreq\endcsname}}
% define class-frequencies for Dsec
\makeatletter
\newcommand{\car@dsecfreq}{1.71}
\newcommand{\bicycle@dsecfreq}{0.02}
\newcommand{\motorcycle@dsecfreq}{0.01}
\newcommand{\truck@dsecfreq}{0.17}
\newcommand{\othervehicle@dsecfreq}{0.26}
\newcommand{\person@dsecfreq}{0.05}
\newcommand{\road@dsecfreq}{29.02}
\newcommand{\sidewalk@dsecfreq}{16.58}
\newcommand{\building@dsecfreq}{17.22}
\newcommand{\fence@dsecfreq}{5.96}
\newcommand{\vegetation@dsecfreq}{26.88}
\newcommand{\terrain@dsecfreq}{1.50}
\newcommand{\pole@dsecfreq}{0.54}
\newcommand{\trafficsign@dsecfreq}{0.09}
\newcommand{\dsecfreq}[1]{{\csname #1@dsecfreq\endcsname}}
\makeatother

\begin{document}
%
% paper title
% Titles are generally capitalized except for words such as a, an, and, as,
% at, but, by, for, in, nor, of, on, or, the, to and up, which are usually
% not capitalized unless they are the first or last word of the title.
% Linebreaks \\ can be used within to get better formatting as desired.
% Do not put math or special symbols in the title.
\title{Event-aided Semantic Scene Completion}
%
%
% author names and IEEE memberships
% note positions of commas and nonbreaking spaces ( ~ ) LaTeX will not break
% a structure at a ~ so this keeps an author's name from being broken across
% two lines.
% use \thanks{} to gain access to the first footnote area
% a separate \thanks must be used for each paragraph as LaTeX2e's \thanks
% was not built to handle multiple paragraphs
%

\author{Shangwei Guo$^{1,*}$, Hao Shi$^{1,*}$, Song Wang$^{3}$, Xiaoting Yin$^{1}$, Kailun Yang$^{2,\dag}$, and Kaiwei Wang$^{1,\dag}$% <-this % stops a space
\thanks{This work was supported in part by Zhejiang Provincial Natural Science Foundation of China (Grant No. LZ24F050003), the National Natural Science Foundation of China (Grant No. 12174341 and No. 62473139), and in part by Shanghai SUPREMIND Technology Company Ltd.
}
\thanks{$^{1}$The authors are with the State Key Laboratory of Extreme Photonics and Instrumentation, Zhejiang University, Hangzhou 310027, China (email: wangkaiwei@zju.edu.cn).}%
\thanks{$^{2}$The author is with the School of Robotics and the National Engineering Research Center of Robot Visual Perception and Control Technology, Hunan University, Changsha 410082, China (email: kailun.yang@hnu.edu.cn).}%
\thanks{$^{3}$The author is with the College of Computer Science and Technology, Zhejiang University, Hangzhou 310027, China.}%
\thanks{$^*$Equal contribution.}%
\thanks{$^\dag$Corresponding authors: Kaiwei Wang and Kailun Yang.}%
}



% The paper headers

% The only time the second header will appear is for the odd numbered pages
% after the title page when using the twoside option.
% 
% *** Note that you probably will NOT want to include the author's ***
% *** name in the headers of peer review papers.                   ***
% You can use \ifCLASSOPTIONpeerreview for conditional compilation here if
% you desire.




% If you want to put a publisher's ID mark on the page you can do it like
% this:
%\IEEEpubid{0000--0000/00\$00.00~\copyright~2015 IEEE}
% Remember, if you use this you must call \IEEEpubidadjcol in the second
% column for its text to clear the IEEEpubid mark.



% use for special paper notices
%\IEEEspecialpapernotice{(Invited Paper)}




% make the title area
\maketitle

% As a general rule, do not put math, special symbols or citations
% in the abstract or keywords.
\begin{abstract}
\begin{abstract}  
Test time scaling is currently one of the most active research areas that shows promise after training time scaling has reached its limits.
Deep-thinking (DT) models are a class of recurrent models that can perform easy-to-hard generalization by assigning more compute to harder test samples.
However, due to their inability to determine the complexity of a test sample, DT models have to use a large amount of computation for both easy and hard test samples.
Excessive test time computation is wasteful and can cause the ``overthinking'' problem where more test time computation leads to worse results.
In this paper, we introduce a test time training method for determining the optimal amount of computation needed for each sample during test time.
We also propose Conv-LiGRU, a novel recurrent architecture for efficient and robust visual reasoning. 
Extensive experiments demonstrate that Conv-LiGRU is more stable than DT, effectively mitigates the ``overthinking'' phenomenon, and achieves superior accuracy.
\end{abstract}  
\end{abstract}

% Note that keywords are not normally used for peerreview papers.
\begin{IEEEkeywords}
Semantic Occupancy Prediction, Event Camera, Multimodal Perception, Semantic Scene Completion
\end{IEEEkeywords}






% For peer review papers, you can put extra information on the cover
% page as needed:
% \ifCLASSOPTIONpeerreview
% \begin{center} \bfseries EDICS Category: 3-BBND \end{center}
% \fi
%
% For peerreview papers, this IEEEtran command inserts a page break and
% creates the second title. It will be ignored for other modes.
\IEEEpeerreviewmaketitle



\section{Introduction}


\begin{figure}[t]
\centering
\includegraphics[width=0.6\columnwidth]{figures/evaluation_desiderata_V5.pdf}
\vspace{-0.5cm}
\caption{\systemName is a platform for conducting realistic evaluations of code LLMs, collecting human preferences of coding models with real users, real tasks, and in realistic environments, aimed at addressing the limitations of existing evaluations.
}
\label{fig:motivation}
\end{figure}

\begin{figure*}[t]
\centering
\includegraphics[width=\textwidth]{figures/system_design_v2.png}
\caption{We introduce \systemName, a VSCode extension to collect human preferences of code directly in a developer's IDE. \systemName enables developers to use code completions from various models. The system comprises a) the interface in the user's IDE which presents paired completions to users (left), b) a sampling strategy that picks model pairs to reduce latency (right, top), and c) a prompting scheme that allows diverse LLMs to perform code completions with high fidelity.
Users can select between the top completion (green box) using \texttt{tab} or the bottom completion (blue box) using \texttt{shift+tab}.}
\label{fig:overview}
\end{figure*}

As model capabilities improve, large language models (LLMs) are increasingly integrated into user environments and workflows.
For example, software developers code with AI in integrated developer environments (IDEs)~\citep{peng2023impact}, doctors rely on notes generated through ambient listening~\citep{oberst2024science}, and lawyers consider case evidence identified by electronic discovery systems~\citep{yang2024beyond}.
Increasing deployment of models in productivity tools demands evaluation that more closely reflects real-world circumstances~\citep{hutchinson2022evaluation, saxon2024benchmarks, kapoor2024ai}.
While newer benchmarks and live platforms incorporate human feedback to capture real-world usage, they almost exclusively focus on evaluating LLMs in chat conversations~\citep{zheng2023judging,dubois2023alpacafarm,chiang2024chatbot, kirk2024the}.
Model evaluation must move beyond chat-based interactions and into specialized user environments.



 

In this work, we focus on evaluating LLM-based coding assistants. 
Despite the popularity of these tools---millions of developers use Github Copilot~\citep{Copilot}---existing
evaluations of the coding capabilities of new models exhibit multiple limitations (Figure~\ref{fig:motivation}, bottom).
Traditional ML benchmarks evaluate LLM capabilities by measuring how well a model can complete static, interview-style coding tasks~\citep{chen2021evaluating,austin2021program,jain2024livecodebench, white2024livebench} and lack \emph{real users}. 
User studies recruit real users to evaluate the effectiveness of LLMs as coding assistants, but are often limited to simple programming tasks as opposed to \emph{real tasks}~\citep{vaithilingam2022expectation,ross2023programmer, mozannar2024realhumaneval}.
Recent efforts to collect human feedback such as Chatbot Arena~\citep{chiang2024chatbot} are still removed from a \emph{realistic environment}, resulting in users and data that deviate from typical software development processes.
We introduce \systemName to address these limitations (Figure~\ref{fig:motivation}, top), and we describe our three main contributions below.


\textbf{We deploy \systemName in-the-wild to collect human preferences on code.} 
\systemName is a Visual Studio Code extension, collecting preferences directly in a developer's IDE within their actual workflow (Figure~\ref{fig:overview}).
\systemName provides developers with code completions, akin to the type of support provided by Github Copilot~\citep{Copilot}. 
Over the past 3 months, \systemName has served over~\completions suggestions from 10 state-of-the-art LLMs, 
gathering \sampleCount~votes from \userCount~users.
To collect user preferences,
\systemName presents a novel interface that shows users paired code completions from two different LLMs, which are determined based on a sampling strategy that aims to 
mitigate latency while preserving coverage across model comparisons.
Additionally, we devise a prompting scheme that allows a diverse set of models to perform code completions with high fidelity.
See Section~\ref{sec:system} and Section~\ref{sec:deployment} for details about system design and deployment respectively.



\textbf{We construct a leaderboard of user preferences and find notable differences from existing static benchmarks and human preference leaderboards.}
In general, we observe that smaller models seem to overperform in static benchmarks compared to our leaderboard, while performance among larger models is mixed (Section~\ref{sec:leaderboard_calculation}).
We attribute these differences to the fact that \systemName is exposed to users and tasks that differ drastically from code evaluations in the past. 
Our data spans 103 programming languages and 24 natural languages as well as a variety of real-world applications and code structures, while static benchmarks tend to focus on a specific programming and natural language and task (e.g. coding competition problems).
Additionally, while all of \systemName interactions contain code contexts and the majority involve infilling tasks, a much smaller fraction of Chatbot Arena's coding tasks contain code context, with infilling tasks appearing even more rarely. 
We analyze our data in depth in Section~\ref{subsec:comparison}.



\textbf{We derive new insights into user preferences of code by analyzing \systemName's diverse and distinct data distribution.}
We compare user preferences across different stratifications of input data (e.g., common versus rare languages) and observe which affect observed preferences most (Section~\ref{sec:analysis}).
For example, while user preferences stay relatively consistent across various programming languages, they differ drastically between different task categories (e.g. frontend/backend versus algorithm design).
We also observe variations in user preference due to different features related to code structure 
(e.g., context length and completion patterns).
We open-source \systemName and release a curated subset of code contexts.
Altogether, our results highlight the necessity of model evaluation in realistic and domain-specific settings.







\section{RELATED WORK}
\label{sec:relatedwork}
In this section, we describe the previous works related to our proposal, which are divided into two parts. In Section~\ref{sec:relatedwork_exoplanet}, we present a review of approaches based on machine learning techniques for the detection of planetary transit signals. Section~\ref{sec:relatedwork_attention} provides an account of the approaches based on attention mechanisms applied in Astronomy.\par

\subsection{Exoplanet detection}
\label{sec:relatedwork_exoplanet}
Machine learning methods have achieved great performance for the automatic selection of exoplanet transit signals. One of the earliest applications of machine learning is a model named Autovetter \citep{MCcauliff}, which is a random forest (RF) model based on characteristics derived from Kepler pipeline statistics to classify exoplanet and false positive signals. Then, other studies emerged that also used supervised learning. \cite{mislis2016sidra} also used a RF, but unlike the work by \citet{MCcauliff}, they used simulated light curves and a box least square \citep[BLS;][]{kovacs2002box}-based periodogram to search for transiting exoplanets. \citet{thompson2015machine} proposed a k-nearest neighbors model for Kepler data to determine if a given signal has similarity to known transits. Unsupervised learning techniques were also applied, such as self-organizing maps (SOM), proposed \citet{armstrong2016transit}; which implements an architecture to segment similar light curves. In the same way, \citet{armstrong2018automatic} developed a combination of supervised and unsupervised learning, including RF and SOM models. In general, these approaches require a previous phase of feature engineering for each light curve. \par

%DL is a modern data-driven technology that automatically extracts characteristics, and that has been successful in classification problems from a variety of application domains. The architecture relies on several layers of NNs of simple interconnected units and uses layers to build increasingly complex and useful features by means of linear and non-linear transformation. This family of models is capable of generating increasingly high-level representations \citep{lecun2015deep}.

The application of DL for exoplanetary signal detection has evolved rapidly in recent years and has become very popular in planetary science.  \citet{pearson2018} and \citet{zucker2018shallow} developed CNN-based algorithms that learn from synthetic data to search for exoplanets. Perhaps one of the most successful applications of the DL models in transit detection was that of \citet{Shallue_2018}; who, in collaboration with Google, proposed a CNN named AstroNet that recognizes exoplanet signals in real data from Kepler. AstroNet uses the training set of labelled TCEs from the Autovetter planet candidate catalog of Q1–Q17 data release 24 (DR24) of the Kepler mission \citep{catanzarite2015autovetter}. AstroNet analyses the data in two views: a ``global view'', and ``local view'' \citep{Shallue_2018}. \par


% The global view shows the characteristics of the light curve over an orbital period, and a local view shows the moment at occurring the transit in detail

%different = space-based

Based on AstroNet, researchers have modified the original AstroNet model to rank candidates from different surveys, specifically for Kepler and TESS missions. \citet{ansdell2018scientific} developed a CNN trained on Kepler data, and included for the first time the information on the centroids, showing that the model improves performance considerably. Then, \citet{osborn2020rapid} and \citet{yu2019identifying} also included the centroids information, but in addition, \citet{osborn2020rapid} included information of the stellar and transit parameters. Finally, \citet{rao2021nigraha} proposed a pipeline that includes a new ``half-phase'' view of the transit signal. This half-phase view represents a transit view with a different time and phase. The purpose of this view is to recover any possible secondary eclipse (the object hiding behind the disk of the primary star).


%last pipeline applies a procedure after the prediction of the model to obtain new candidates, this process is carried out through a series of steps that include the evaluation with Discovery and Validation of Exoplanets (DAVE) \citet{kostov2019discovery} that was adapted for the TESS telescope.\par
%



\subsection{Attention mechanisms in astronomy}
\label{sec:relatedwork_attention}
Despite the remarkable success of attention mechanisms in sequential data, few papers have exploited their advantages in astronomy. In particular, there are no models based on attention mechanisms for detecting planets. Below we present a summary of the main applications of this modeling approach to astronomy, based on two points of view; performance and interpretability of the model.\par
%Attention mechanisms have not yet been explored in all sub-areas of astronomy. However, recent works show a successful application of the mechanism.
%performance

The application of attention mechanisms has shown improvements in the performance of some regression and classification tasks compared to previous approaches. One of the first implementations of the attention mechanism was to find gravitational lenses proposed by \citet{thuruthipilly2021finding}. They designed 21 self-attention-based encoder models, where each model was trained separately with 18,000 simulated images, demonstrating that the model based on the Transformer has a better performance and uses fewer trainable parameters compared to CNN. A novel application was proposed by \citet{lin2021galaxy} for the morphological classification of galaxies, who used an architecture derived from the Transformer, named Vision Transformer (VIT) \citep{dosovitskiy2020image}. \citet{lin2021galaxy} demonstrated competitive results compared to CNNs. Another application with successful results was proposed by \citet{zerveas2021transformer}; which first proposed a transformer-based framework for learning unsupervised representations of multivariate time series. Their methodology takes advantage of unlabeled data to train an encoder and extract dense vector representations of time series. Subsequently, they evaluate the model for regression and classification tasks, demonstrating better performance than other state-of-the-art supervised methods, even with data sets with limited samples.

%interpretation
Regarding the interpretability of the model, a recent contribution that analyses the attention maps was presented by \citet{bowles20212}, which explored the use of group-equivariant self-attention for radio astronomy classification. Compared to other approaches, this model analysed the attention maps of the predictions and showed that the mechanism extracts the brightest spots and jets of the radio source more clearly. This indicates that attention maps for prediction interpretation could help experts see patterns that the human eye often misses. \par

In the field of variable stars, \citet{allam2021paying} employed the mechanism for classifying multivariate time series in variable stars. And additionally, \citet{allam2021paying} showed that the activation weights are accommodated according to the variation in brightness of the star, achieving a more interpretable model. And finally, related to the TESS telescope, \citet{morvan2022don} proposed a model that removes the noise from the light curves through the distribution of attention weights. \citet{morvan2022don} showed that the use of the attention mechanism is excellent for removing noise and outliers in time series datasets compared with other approaches. In addition, the use of attention maps allowed them to show the representations learned from the model. \par

Recent attention mechanism approaches in astronomy demonstrate comparable results with earlier approaches, such as CNNs. At the same time, they offer interpretability of their results, which allows a post-prediction analysis. \par






% An example of a floating figure using the graphicx package.
% Note that \label must occur AFTER (or within) \caption.
% For figures, \caption should occur after the \includegraphics.
% Note that IEEEtran v1.7 and later has special internal code that
% is designed to preserve the operation of \label within \caption
% even when the captionsoff option is in effect. However, because
% of issues like this, it may be the safest practice to put all your
% \label just after \caption rather than within \caption{}.
%
% Reminder: the "draftcls" or "draftclsnofoot", not "draft", class
% option should be used if it is desired that the figures are to be
% displayed while in draft mode.
%
%\begin{figure}[!t]
%\centering
%\includegraphics[width=2.5in]{myfigure}
% where an .eps filename suffix will be assumed under latex, 
% and a .pdf suffix will be assumed for pdflatex; or what has been declared
% via \DeclareGraphicsExtensions.
%\caption{Simulation results for the network.}
%\label{fig_sim}
%\end{figure}

% Note that the IEEE typically puts floats only at the top, even when this
% results in a large percentage of a column being occupied by floats.

\begin{table}[t]
    \centering
    \caption{The performance of different pre-trained models on ImageNet and infrared semantic segmentation datasets. The \textit{Scratch} means the performance of randomly initialized models. The \textit{PT Epochs} denotes the pre-training epochs while the \textit{IN1K FT epochs} represents the fine-tuning epochs on ImageNet \citep{imagenet}. $^\dag$ denotes models reproduced using official codes. $^\star$ refers to the effective epochs used in \citet{iBOT}. The top two results are marked in \textbf{bold} and \underline{underlined} format. Supervised and CL methods, MIM methods, and UNIP models are colored in \colorbox{orange!15}{\rule[-0.2ex]{0pt}{1.5ex}orange}, \colorbox{gray!15}{\rule[-0.2ex]{0pt}{1.5ex}gray}, and \colorbox{cyan!15}{\rule[-0.2ex]{0pt}{1.5ex}cyan}, respectively.}
    \label{tab:benchmark}
    \centering
    \scriptsize
    \setlength{\tabcolsep}{1.0mm}{
    \scalebox{1.0}{
    \begin{tabular}{l c c c c  c c c c c c c c}
        \toprule
         \multirow{2}{*}{Methods} & \multirow{2}{*}{\makecell[c]{PT \\ Epochs}} & \multicolumn{2}{c}{IN1K FT} & \multicolumn{4}{c}{Fine-tuning (FT)} & \multicolumn{4}{c}{Linear Probing (LP)} \\
         \cmidrule{3-4} \cmidrule(lr){5-8} \cmidrule(lr){9-12} 
         & & Epochs & Acc & SODA & MFNet-T & SCUT-Seg & Mean & SODA & MFNet-T & SCUT-Seg & Mean \\
         \midrule
         \textcolor{gray}{ViT-Tiny/16} & & &  & & & & & & & & \\
         Scratch & - & - & - & 31.34 & 19.50 & 41.09 & 30.64 & - & - & - & - \\
         \rowcolor{gray!15} MAE$^\dag$ \citep{mae} & 800 & 200 & \underline{71.8} & 52.85 & 35.93 & 51.31 & 46.70 & 23.75 & 15.79 & 27.18 & 22.24 \\
         \rowcolor{orange!15} DeiT \citep{deit} & 300 & - & \textbf{72.2} & 63.14 & 44.60 & 61.36 & 56.37 & 42.29 & 21.78 & 31.96 & 32.01 \\
         \rowcolor{cyan!15} UNIP (MAE-L) & 100 & - & - & \underline{64.83} & \textbf{48.77} & \underline{67.22} & \underline{60.27} & \underline{44.12} & \underline{28.26} & \underline{35.09} & \underline{35.82} \\
         \rowcolor{cyan!15} UNIP (iBOT-L) & 100 & - & - & \textbf{65.54} & \underline{48.45} & \textbf{67.73} & \textbf{60.57} & \textbf{52.95} & \textbf{30.10} & \textbf{40.12} & \textbf{41.06}  \\
         \midrule
         \textcolor{gray}{ViT-Small/16} & & & & & & & & & & & \\
         Scratch & - & - & - & 41.70 & 22.49 & 46.28 & 36.82 & - & - & - & - \\
         \rowcolor{gray!15} MAE$^\dag$ \citep{mae} & 800 & 200 & 80.0 & 63.36 & 42.44 & 60.38 & 55.39 & 38.17 & 21.14 & 34.15 & 31.15 \\
         \rowcolor{gray!15} CrossMAE \citep{crossmae} & 800 & 200 & 80.5 & 63.95 & 43.99 & 63.53 & 57.16 & 39.40 & 23.87 & 34.01 & 32.43 \\
         \rowcolor{orange!15} DeiT \citep{deit} & 300 & - & 79.9 & 68.08 & 45.91 & 66.17 & 60.05 & 44.88 & 28.53 & 38.92 & 37.44 \\
         \rowcolor{orange!15} DeiT III \citep{deit3} & 800 & - & 81.4 & 69.35 & 47.73 & 67.32 & 61.47 & 54.17 & 32.01 & 43.54 & 43.24 \\
         \rowcolor{orange!15} DINO \citep{dino} & 3200$^\star$ & 200 & \underline{82.0} & 68.56 & 47.98 & 68.74 & 61.76 & 56.02 & 32.94 & 45.94 & 44.97 \\
         \rowcolor{orange!15} iBOT \citep{iBOT} & 3200$^\star$ & 200 & \textbf{82.3} & 69.33 & 47.15 & 69.80 & 62.09 & 57.10 & 33.87 & 45.82 & 45.60 \\
         \rowcolor{cyan!15} UNIP (DINO-B) & 100 & - & - & 69.35 & 49.95 & 69.70 & 63.00 & \underline{57.76} & \underline{34.15} & \underline{46.37} & \underline{46.09} \\
         \rowcolor{cyan!15} UNIP (MAE-L) & 100 & - & - & \textbf{70.99} & \underline{51.32} & \underline{70.79} & \underline{64.37} & 55.25 & 33.49 & 43.37 & 44.04 \\
         \rowcolor{cyan!15} UNIP (iBOT-L) & 100 & - & - & \underline{70.75} & \textbf{51.81} & \textbf{71.55} & \textbf{64.70} & \textbf{60.28} & \textbf{37.16} & \textbf{47.68} & \textbf{48.37} \\ 
        \midrule
        \textcolor{gray}{ViT-Base/16} & & & & & & & & & & & \\
        Scratch & - & - & - & 44.25 & 23.72 & 49.44 & 39.14 & - & - & - & - \\
        \rowcolor{gray!15} MAE \citep{mae} & 1600 & 100 & 83.6 & 68.18 & 46.78 & 67.86 & 60.94 & 43.01 & 23.42 & 37.48 & 34.64 \\
        \rowcolor{gray!15} CrossMAE \citep{crossmae} & 800 & 100 & 83.7 & 68.29 & 47.85 & 68.39 & 61.51 & 43.35 & 26.03 & 38.36 & 35.91 \\
        \rowcolor{orange!15} DeiT \citep{deit} & 300 & - & 81.8 & 69.73 & 48.59 & 69.35 & 62.56 & 57.40 & 34.82 & 46.44 & 46.22 \\
        \rowcolor{orange!15} DeiT III \citep{deit3} & 800 & 20 & \underline{83.8} & 71.09 & 49.62 & 70.19 & 63.63 & 59.01 & \underline{35.34} & 48.01 & 47.45 \\
        \rowcolor{orange!15} DINO \citep{dino} & 1600$^\star$ & 100 & 83.6 & 69.79 & 48.54 & 69.82 & 62.72 & 59.33 & 34.86 & 47.23 & 47.14 \\
        \rowcolor{orange!15} iBOT \citep{iBOT} & 1600$^\star$ & 100 & \textbf{84.0} & 71.15 & 48.98 & 71.26 & 63.80 & \underline{60.05} & 34.34 & \underline{49.12} & \underline{47.84} \\
        \rowcolor{cyan!15} UNIP (MAE-L) & 100 & - & - & \underline{71.47} & \textbf{52.55} & \underline{71.82} & \textbf{65.28} & 58.82 & 34.75 & 48.74 & 47.43 \\
        \rowcolor{cyan!15} UNIP (iBOT-L) & 100 & - & - & \textbf{71.75} & \underline{51.46} & \textbf{72.00} & \underline{65.07} & \textbf{63.14} & \textbf{39.08} & \textbf{52.53} & \textbf{51.58} \\
        \midrule
        \textcolor{gray}{ViT-Large/16} & & & & & & & & & & & \\
        Scratch & - & - & - & 44.70 & 23.68 & 49.55 & 39.31 & - & - & - & - \\
        \rowcolor{gray!15} MAE \citep{mae} & 1600 & 50 & \textbf{85.9} & 71.04 & \underline{51.17} & 70.83 & 64.35 & 52.20 & 31.21 & 43.71 & 42.37 \\
        \rowcolor{gray!15} CrossMAE \citep{crossmae} & 800 & 50 & 85.4 & 70.48 & 50.97 & 70.24 & 63.90 & 53.29 & 33.09 & 45.01 & 43.80 \\
        \rowcolor{orange!15} DeiT3 \citep{deit3} & 800 & 20 & \underline{84.9} & \underline{71.67} & 50.78 & \textbf{71.54} & \underline{64.66} & \underline{59.42} & \textbf{37.57} & \textbf{50.27} & \underline{49.09} \\
        \rowcolor{orange!15} iBOT \citep{iBOT} & 1000$^\star$ & 50 & 84.8 & \textbf{71.75} & \textbf{51.66} & \underline{71.49} & \textbf{64.97} & \textbf{61.73} & \underline{36.68} & \underline{50.12} & \textbf{49.51} \\
        \bottomrule
    \end{tabular}}}
    \vspace{-2mm}
\end{table}

\section{Method}\label{sec:method}
\begin{figure}
    \centering
    \includegraphics[width=0.85\textwidth]{imgs/heatmap_acc.pdf}
    \caption{\textbf{Visualization of the proposed periodic Bayesian flow with mean parameter $\mu$ and accumulated accuracy parameter $c$ which corresponds to the entropy/uncertainty}. For $x = 0.3, \beta(1) = 1000$ and $\alpha_i$ defined in \cref{appd:bfn_cir}, this figure plots three colored stochastic parameter trajectories for receiver mean parameter $m$ and accumulated accuracy parameter $c$, superimposed on a log-scale heatmap of the Bayesian flow distribution $p_F(m|x,\senderacc)$ and $p_F(c|x,\senderacc)$. Note the \emph{non-monotonicity} and \emph{non-additive} property of $c$ which could inform the network the entropy of the mean parameter $m$ as a condition and the \emph{periodicity} of $m$. %\jj{Shrink the figures to save space}\hanlin{Do we need to make this figure one-column?}
    }
    \label{fig:vmbf_vis}
    \vskip -0.1in
\end{figure}
% \begin{wrapfigure}{r}{0.5\textwidth}
%     \centering
%     \includegraphics[width=0.49\textwidth]{imgs/heatmap_acc.pdf}
%     \caption{\textbf{Visualization of hyper-torus Bayesian flow based on von Mises Distribution}. For $x = 0.3, \beta(1) = 1000$ and $\alpha_i$ defined in \cref{appd:bfn_cir}, this figure plots three colored stochastic parameter trajectories for receiver mean parameter $m$ and accumulated accuracy parameter $c$, superimposed on a log-scale heatmap of the Bayesian flow distribution $p_F(m|x,\senderacc)$ and $p_F(c|x,\senderacc)$. Note the \emph{non-monotonicity} and \emph{non-additive} property of $c$. \jj{Shrink the figures to save space}}
%     \label{fig:vmbf_vis}
%     \vspace{-30pt}
% \end{wrapfigure}


In this section, we explain the detailed design of CrysBFN tackling theoretical and practical challenges. First, we describe how to derive our new formulation of Bayesian Flow Networks over hyper-torus $\mathbb{T}^{D}$ from scratch. Next, we illustrate the two key differences between \modelname and the original form of BFN: $1)$ a meticulously designed novel base distribution with different Bayesian update rules; and $2)$ different properties over the accuracy scheduling resulted from the periodicity and the new Bayesian update rules. Then, we present in detail the overall framework of \modelname over each manifold of the crystal space (\textit{i.e.} fractional coordinates, lattice vectors, atom types) respecting \textit{periodic E(3) invariance}. 

% In this section, we first demonstrate how to build Bayesian flow on hyper-torus $\mathbb{T}^{D}$ by overcoming theoretical and practical problems to provide a low-noise parameter-space approach to fractional atom coordinate generation. Next, we present how \modelname models each manifold of crystal space respecting \textit{periodic E(3) invariance}. 

\subsection{Periodic Bayesian Flow on Hyper-torus \texorpdfstring{$\mathbb{T}^{D}$}{}} 
For generative modeling of fractional coordinates in crystal, we first construct a periodic Bayesian flow on \texorpdfstring{$\mathbb{T}^{D}$}{} by designing every component of the totally new Bayesian update process which we demonstrate to be distinct from the original Bayesian flow (please see \cref{fig:non_add}). 
 %:) 
 
 The fractional atom coordinate system \citep{jiao2023crystal} inherently distributes over a hyper-torus support $\mathbb{T}^{3\times N}$. Hence, the normal distribution support on $\R$ used in the original \citep{bfn} is not suitable for this scenario. 
% The key problem of generative modeling for crystal is the periodicity of Cartesian atom coordinates $\vX$ requiring:
% \begin{equation}\label{eq:periodcity}
% p(\vA,\vL,\vX)=p(\vA,\vL,\vX+\vec{LK}),\text{where}~\vec{K}=\vec{k}\vec{1}_{1\times N},\forall\vec{k}\in\mathbb{Z}^{3\times1}
% \end{equation}
% However, there does not exist such a distribution supporting on $\R$ to model such property because the integration of such distribution over $\R$ will not be finite and equal to 1. Therefore, the normal distribution used in \citet{bfn} can not meet this condition.

To tackle this problem, the circular distribution~\citep{mardia2009directional} over the finite interval $[-\pi,\pi)$ is a natural choice as the base distribution for deriving the BFN on $\mathbb{T}^D$. 
% one natural choice is to 
% we would like to consider the circular distribution over the finite interval as the base 
% we find that circular distributions \citep{mardia2009directional} defined on a finite interval with lengths of $2\pi$ can be used as the instantiation of input distribution for the BFN on $\mathbb{T}^D$.
Specifically, circular distributions enjoy desirable periodic properties: $1)$ the integration over any interval length of $2\pi$ equals 1; $2)$ the probability distribution function is periodic with period $2\pi$.  Sharing the same intrinsic with fractional coordinates, such periodic property of circular distribution makes it suitable for the instantiation of BFN's input distribution, in parameterizing the belief towards ground truth $\x$ on $\mathbb{T}^D$. 
% \yuxuan{this is very complicated from my perspective.} \hanlin{But this property is exactly beautiful and perfectly fit into the BFN.}

\textbf{von Mises Distribution and its Bayesian Update} We choose von Mises distribution \citep{mardia2009directional} from various circular distributions as the form of input distribution, based on the appealing conjugacy property required in the derivation of the BFN framework.
% to leverage the Bayesian conjugacy property of von Mises distribution which is required by the BFN framework. 
That is, the posterior of a von Mises distribution parameterized likelihood is still in the family of von Mises distributions. The probability density function of von Mises distribution with mean direction parameter $m$ and concentration parameter $c$ (describing the entropy/uncertainty of $m$) is defined as: 
\begin{equation}
f(x|m,c)=vM(x|m,c)=\frac{\exp(c\cos(x-m))}{2\pi I_0(c)}
\end{equation}
where $I_0(c)$ is zeroth order modified Bessel function of the first kind as the normalizing constant. Given the last univariate belief parameterized by von Mises distribution with parameter $\theta_{i-1}=\{m_{i-1},\ c_{i-1}\}$ and the sample $y$ from sender distribution with unknown data sample $x$ and known accuracy $\alpha$ describing the entropy/uncertainty of $y$,  Bayesian update for the receiver is deducted as:
\begin{equation}
 h(\{m_{i-1},c_{i-1}\},y,\alpha)=\{m_i,c_i \}, \text{where}
\end{equation}
\begin{equation}\label{eq:h_m}
m_i=\text{atan2}(\alpha\sin y+c_{i-1}\sin m_{i-1}, {\alpha\cos y+c_{i-1}\cos m_{i-1}})
\end{equation}
\begin{equation}\label{eq:h_c}
c_i =\sqrt{\alpha^2+c_{i-1}^2+2\alpha c_{i-1}\cos(y-m_{i-1})}
\end{equation}
The proof of the above equations can be found in \cref{apdx:bayesian_update_function}. The atan2 function refers to  2-argument arctangent. Independently conducting  Bayesian update for each dimension, we can obtain the Bayesian update distribution by marginalizing $\y$:
\begin{equation}
p_U(\vtheta'|\vtheta,\bold{x};\alpha)=\mathbb{E}_{p_S(\bold{y}|\bold{x};\alpha)}\delta(\vtheta'-h(\vtheta,\bold{y},\alpha))=\mathbb{E}_{vM(\bold{y}|\bold{x},\alpha)}\delta(\vtheta'-h(\vtheta,\bold{y},\alpha))
\end{equation} 
\begin{figure}
    \centering
    \vskip -0.15in
    \includegraphics[width=0.95\linewidth]{imgs/non_add.pdf}
    \caption{An intuitive illustration of non-additive accuracy Bayesian update on the torus. The lengths of arrows represent the uncertainty/entropy of the belief (\emph{e.g.}~$1/\sigma^2$ for Gaussian and $c$ for von Mises). The directions of the arrows represent the believed location (\emph{e.g.}~ $\mu$ for Gaussian and $m$ for von Mises).}
    \label{fig:non_add}
    \vskip -0.15in
\end{figure}
\textbf{Non-additive Accuracy} 
The additive accuracy is a nice property held with the Gaussian-formed sender distribution of the original BFN expressed as:
\begin{align}
\label{eq:standard_id}
    \update(\parsn{}'' \mid \parsn{}, \x; \alpha_a+\alpha_b) = \E_{\update(\parsn{}' \mid \parsn{}, \x; \alpha_a)} \update(\parsn{}'' \mid \parsn{}', \x; \alpha_b)
\end{align}
Such property is mainly derived based on the standard identity of Gaussian variable:
\begin{equation}
X \sim \mathcal{N}\left(\mu_X, \sigma_X^2\right), Y \sim \mathcal{N}\left(\mu_Y, \sigma_Y^2\right) \Longrightarrow X+Y \sim \mathcal{N}\left(\mu_X+\mu_Y, \sigma_X^2+\sigma_Y^2\right)
\end{equation}
The additive accuracy property makes it feasible to derive the Bayesian flow distribution $
p_F(\boldsymbol{\theta} \mid \mathbf{x} ; i)=p_U\left(\boldsymbol{\theta} \mid \boldsymbol{\theta}_0, \mathbf{x}, \sum_{k=1}^{i} \alpha_i \right)
$ for the simulation-free training of \cref{eq:loss_n}.
It should be noted that the standard identity in \cref{eq:standard_id} does not hold in the von Mises distribution. Hence there exists an important difference between the original Bayesian flow defined on Euclidean space and the Bayesian flow of circular data on $\mathbb{T}^D$ based on von Mises distribution. With prior $\btheta = \{\bold{0},\bold{0}\}$, we could formally represent the non-additive accuracy issue as:
% The additive accuracy property implies the fact that the "confidence" for the data sample after observing a series of the noisy samples with accuracy ${\alpha_1, \cdots, \alpha_i}$ could be  as the accuracy sum  which could be  
% Here we 
% Here we emphasize the specific property of BFN based on von Mises distribution.
% Note that 
% \begin{equation}
% \update(\parsn'' \mid \parsn, \x; \alpha_a+\alpha_b) \ne \E_{\update(\parsn' \mid \parsn, \x; \alpha_a)} \update(\parsn'' \mid \parsn', \x; \alpha_b)
% \end{equation}
% \oyyw{please check whether the below equation is better}
% \yuxuan{I fill somehow confusing on what is the update distribution with $\alpha$. }
% \begin{equation}
% \update(\parsn{}'' \mid \parsn{}, \x; \alpha_a+\alpha_b) \ne \E_{\update(\parsn{}' \mid \parsn{}, \x; \alpha_a)} \update(\parsn{}'' \mid \parsn{}', \x; \alpha_b)
% \end{equation}
% We give an intuitive visualization of such difference in \cref{fig:non_add}. The untenability of this property can materialize by considering the following case: with prior $\btheta = \{\bold{0},\bold{0}\}$, check the two-step Bayesian update distribution with $\alpha_a,\alpha_b$ and one-step Bayesian update with $\alpha=\alpha_a+\alpha_b$:
\begin{align}
\label{eq:nonadd}
     &\update(c'' \mid \parsn, \x; \alpha_a+\alpha_b)  = \delta(c-\alpha_a-\alpha_b)
     \ne  \mathbb{E}_{p_U(\parsn' \mid \parsn, \x; \alpha_a)}\update(c'' \mid \parsn', \x; \alpha_b) \nonumber \\&= \mathbb{E}_{vM(\bold{y}_b|\bold{x},\alpha_a)}\mathbb{E}_{vM(\bold{y}_a|\bold{x},\alpha_b)}\delta(c-||[\alpha_a \cos\y_a+\alpha_b\cos \y_b,\alpha_a \sin\y_a+\alpha_b\sin \y_b]^T||_2)
\end{align}
A more intuitive visualization could be found in \cref{fig:non_add}. This fundamental difference between periodic Bayesian flow and that of \citet{bfn} presents both theoretical and practical challenges, which we will explain and address in the following contents.

% This makes constructing Bayesian flow based on von Mises distribution intrinsically different from previous Bayesian flows (\citet{bfn}).

% Thus, we must reformulate the framework of Bayesian flow networks  accordingly. % and do necessary reformulations of BFN. 

% \yuxuan{overall I feel this part is complicated by using the language of update distribution. I would like to suggest simply use bayesian update, to provide intuitive explantion.}\hanlin{See the illustration in \cref{fig:non_add}}

% That introduces a cascade of problems, and we investigate the following issues: $(1)$ Accuracies between sender and receiver are not synchronized and need to be differentiated. $(2)$ There is no tractable Bayesian flow distribution for a one-step sample conditioned on a given time step $i$, and naively simulating the Bayesian flow results in computational overhead. $(3)$ It is difficult to control the entropy of the Bayesian flow. $(4)$ Accuracy is no longer a function of $t$ and becomes a distribution conditioned on $t$, which can be different across dimensions.
%\jj{Edited till here}

\textbf{Entropy Conditioning} As a common practice in generative models~\citep{ddpm,flowmatching,bfn}, timestep $t$ is widely used to distinguish among generation states by feeding the timestep information into the networks. However, this paper shows that for periodic Bayesian flow, the accumulated accuracy $\vc_i$ is more effective than time-based conditioning by informing the network about the entropy and certainty of the states $\parsnt{i}$. This stems from the intrinsic non-additive accuracy which makes the receiver's accumulated accuracy $c$ not bijective function of $t$, but a distribution conditioned on accumulated accuracies $\vc_i$ instead. Therefore, the entropy parameter $\vc$ is taken logarithm and fed into the network to describe the entropy of the input corrupted structure. We verify this consideration in \cref{sec:exp_ablation}. 
% \yuxuan{implement variant. traditionally, the timestep is widely used to distinguish the different states by putting the timestep embedding into the networks. citation of FM, diffusion, BFN. However, we find that conditioned on time in periodic flow could not provide extra benefits. To further boost the performance, we introduce a simple yet effective modification term entropy conditional. This is based on that the accumulated accuracy which represents the current uncertainty or entropy could be a better indicator to distinguish different states. + Describe how you do this. }



\textbf{Reformulations of BFN}. Recall the original update function with Gaussian sender distribution, after receiving noisy samples $\y_1,\y_2,\dots,\y_i$ with accuracies $\senderacc$, the accumulated accuracies of the receiver side could be analytically obtained by the additive property and it is consistent with the sender side.
% Since observing sample $\y$ with $\alpha_i$ can not result in exact accuracy increment $\alpha_i$ for receiver, the accuracies between sender and receiver are not synchronized which need to be differentiated. 
However, as previously mentioned, this does not apply to periodic Bayesian flow, and some of the notations in original BFN~\citep{bfn} need to be adjusted accordingly. We maintain the notations of sender side's one-step accuracy $\alpha$ and added accuracy $\beta$, and alter the notation of receiver's accuracy parameter as $c$, which is needed to be simulated by cascade of Bayesian updates. We emphasize that the receiver's accumulated accuracy $c$ is no longer a function of $t$ (differently from the Gaussian case), and it becomes a distribution conditioned on received accuracies $\senderacc$ from the sender. Therefore, we represent the Bayesian flow distribution of von Mises distribution as $p_F(\btheta|\x;\alpha_1,\alpha_2,\dots,\alpha_i)$. And the original simulation-free training with Bayesian flow distribution is no longer applicable in this scenario.
% Different from previous BFNs where the accumulated accuracy $\rho$ is not explicitly modeled, the accumulated accuracy parameter $c$ (visualized in \cref{fig:vmbf_vis}) needs to be explicitly modeled by feeding it to the network to avoid information loss.
% the randomaccuracy parameter $c$ (visualized in \cref{fig:vmbf_vis}) implies that there exists information in $c$ from the sender just like $m$, meaning that $c$ also should be fed into the network to avoid information loss. 
% We ablate this consideration in  \cref{sec:exp_ablation}. 

\textbf{Fast Sampling from Equivalent Bayesian Flow Distribution} Based on the above reformulations, the Bayesian flow distribution of von Mises distribution is reframed as: 
\begin{equation}\label{eq:flow_frac}
p_F(\btheta_i|\x;\alpha_1,\alpha_2,\dots,\alpha_i)=\E_{\update(\parsnt{1} \mid \parsnt{0}, \x ; \alphat{1})}\dots\E_{\update(\parsn_{i-1} \mid \parsnt{i-2}, \x; \alphat{i-1})} \update(\parsnt{i} | \parsnt{i-1},\x;\alphat{i} )
\end{equation}
Naively sampling from \cref{eq:flow_frac} requires slow auto-regressive iterated simulation, making training unaffordable. Noticing the mathematical properties of \cref{eq:h_m,eq:h_c}, we  transform \cref{eq:flow_frac} to the equivalent form:
\begin{equation}\label{eq:cirflow_equiv}
p_F(\vec{m}_i|\x;\alpha_1,\alpha_2,\dots,\alpha_i)=\E_{vM(\y_1|\x,\alpha_1)\dots vM(\y_i|\x,\alpha_i)} \delta(\vec{m}_i-\text{atan2}(\sum_{j=1}^i \alpha_j \cos \y_j,\sum_{j=1}^i \alpha_j \sin \y_j))
\end{equation}
\begin{equation}\label{eq:cirflow_equiv2}
p_F(\vec{c}_i|\x;\alpha_1,\alpha_2,\dots,\alpha_i)=\E_{vM(\y_1|\x,\alpha_1)\dots vM(\y_i|\x,\alpha_i)}  \delta(\vec{c}_i-||[\sum_{j=1}^i \alpha_j \cos \y_j,\sum_{j=1}^i \alpha_j \sin \y_j]^T||_2)
\end{equation}
which bypasses the computation of intermediate variables and allows pure tensor operations, with negligible computational overhead.
\begin{restatable}{proposition}{cirflowequiv}
The probability density function of Bayesian flow distribution defined by \cref{eq:cirflow_equiv,eq:cirflow_equiv2} is equivalent to the original definition in \cref{eq:flow_frac}. 
\end{restatable}
\textbf{Numerical Determination of Linear Entropy Sender Accuracy Schedule} ~Original BFN designs the accuracy schedule $\beta(t)$ to make the entropy of input distribution linearly decrease. As for crystal generation task, to ensure information coherence between modalities, we choose a sender accuracy schedule $\senderacc$ that makes the receiver's belief entropy $H(t_i)=H(p_I(\cdot|\vtheta_i))=H(p_I(\cdot|\vc_i))$ linearly decrease \emph{w.r.t.} time $t_i$, given the initial and final accuracy parameter $c(0)$ and $c(1)$. Due to the intractability of \cref{eq:vm_entropy}, we first use numerical binary search in $[0,c(1)]$ to determine the receiver's $c(t_i)$ for $i=1,\dots, n$ by solving the equation $H(c(t_i))=(1-t_i)H(c(0))+tH(c(1))$. Next, with $c(t_i)$, we conduct numerical binary search for each $\alpha_i$ in $[0,c(1)]$ by solving the equations $\E_{y\sim vM(x,\alpha_i)}[\sqrt{\alpha_i^2+c_{i-1}^2+2\alpha_i c_{i-1}\cos(y-m_{i-1})}]=c(t_i)$ from $i=1$ to $i=n$ for arbitrarily selected $x\in[-\pi,\pi)$.

After tackling all those issues, we have now arrived at a new BFN architecture for effectively modeling crystals. Such BFN can also be adapted to other type of data located in hyper-torus $\mathbb{T}^{D}$.

\subsection{Equivariant Bayesian Flow for Crystal}
With the above Bayesian flow designed for generative modeling of fractional coordinate $\vF$, we are able to build equivariant Bayesian flow for each modality of crystal. In this section, we first give an overview of the general training and sampling algorithm of \modelname (visualized in \cref{fig:framework}). Then, we describe the details of the Bayesian flow of every modality. The training and sampling algorithm can be found in \cref{alg:train} and \cref{alg:sampling}.

\textbf{Overview} Operating in the parameter space $\bthetaM=\{\bthetaA,\bthetaL,\bthetaF\}$, \modelname generates high-fidelity crystals through a joint BFN sampling process on the parameter of  atom type $\bthetaA$, lattice parameter $\vec{\theta}^L=\{\bmuL,\brhoL\}$, and the parameter of fractional coordinate matrix $\bthetaF=\{\bmF,\bcF\}$. We index the $n$-steps of the generation process in a discrete manner $i$, and denote the corresponding continuous notation $t_i=i/n$ from prior parameter $\thetaM_0$ to a considerably low variance parameter $\thetaM_n$ (\emph{i.e.} large $\vrho^L,\bmF$, and centered $\bthetaA$).

At training time, \modelname samples time $i\sim U\{1,n\}$ and $\bthetaM_{i-1}$ from the Bayesian flow distribution of each modality, serving as the input to the network. The network $\net$ outputs $\net(\parsnt{i-1}^\mathcal{M},t_{i-1})=\net(\parsnt{i-1}^A,\parsnt{i-1}^F,\parsnt{i-1}^L,t_{i-1})$ and conducts gradient descents on loss function \cref{eq:loss_n} for each modality. After proper training, the sender distribution $p_S$ can be approximated by the receiver distribution $p_R$. 

At inference time, from predefined $\thetaM_0$, we conduct transitions from $\thetaM_{i-1}$ to $\thetaM_{i}$ by: $(1)$ sampling $\y_i\sim p_R(\bold{y}|\thetaM_{i-1};t_i,\alpha_i)$ according to network prediction $\predM{i-1}$; and $(2)$ performing Bayesian update $h(\thetaM_{i-1},\y^\calM_{i-1},\alpha_i)$ for each dimension. 

% Alternatively, we complete this transition using the flow-back technique by sampling 
% $\thetaM_{i}$ from Bayesian flow distribution $\flow(\btheta^M_{i}|\predM{i-1};t_{i-1})$. 

% The training objective of $\net$ is to minimize the KL divergence between sender distribution and receiver distribution for every modality as defined in \cref{eq:loss_n} which is equivalent to optimizing the negative variational lower bound $\calL^{VLB}$ as discussed in \cref{sec:preliminaries}. 

%In the following part, we will present the Bayesian flow of each modality in detail.

\textbf{Bayesian Flow of Fractional Coordinate $\vF$}~The distribution of the prior parameter $\bthetaF_0$ is defined as:
\begin{equation}\label{eq:prior_frac}
    p(\bthetaF_0) \defeq \{vM(\vm_0^F|\vec{0}_{3\times N},\vec{0}_{3\times N}),\delta(\vc_0^F-\vec{0}_{3\times N})\} = \{U(\vec{0},\vec{1}),\delta(\vc_0^F-\vec{0}_{3\times N})\}
\end{equation}
Note that this prior distribution of $\vm_0^F$ is uniform over $[\vec{0},\vec{1})$, ensuring the periodic translation invariance property in \cref{De:pi}. The training objective is minimizing the KL divergence between sender and receiver distribution (deduction can be found in \cref{appd:cir_loss}): 
%\oyyw{replace $\vF$ with $\x$?} \hanlin{notations follow Preliminary?}
\begin{align}\label{loss_frac}
\calL_F = n \E_{i \sim \ui{n}, \flow(\parsn{}^F \mid \vF ; \senderacc)} \alpha_i\frac{I_1(\alpha_i)}{I_0(\alpha_i)}(1-\cos(\vF-\predF{i-1}))
\end{align}
where $I_0(x)$ and $I_1(x)$ are the zeroth and the first order of modified Bessel functions. The transition from $\bthetaF_{i-1}$ to $\bthetaF_{i}$ is the Bayesian update distribution based on network prediction:
\begin{equation}\label{eq:transi_frac}
    p(\btheta^F_{i}|\parsnt{i-1}^\calM)=\mathbb{E}_{vM(\bold{y}|\predF{i-1},\alpha_i)}\delta(\btheta^F_{i}-h(\btheta^F_{i-1},\bold{y},\alpha_i))
\end{equation}
\begin{restatable}{proposition}{fracinv}
With $\net_{F}$ as a periodic translation equivariant function namely $\net_F(\parsnt{}^A,w(\parsnt{}^F+\vt),\parsnt{}^L,t)=w(\net_F(\parsnt{}^A,\parsnt{}^F,\parsnt{}^L,t)+\vt), \forall\vt\in\R^3$, the marginal distribution of $p(\vF_n)$ defined by \cref{eq:prior_frac,eq:transi_frac} is periodic translation invariant. 
\end{restatable}
\textbf{Bayesian Flow of Lattice Parameter \texorpdfstring{$\boldsymbol{L}$}{}}   
Noting the lattice parameter $\bm{L}$ located in Euclidean space, we set prior as the parameter of a isotropic multivariate normal distribution $\btheta^L_0\defeq\{\vmu_0^L,\vrho_0^L\}=\{\bm{0}_{3\times3},\bm{1}_{3\times3}\}$
% \begin{equation}\label{eq:lattice_prior}
% \btheta^L_0\defeq\{\vmu_0^L,\vrho_0^L\}=\{\bm{0}_{3\times3},\bm{1}_{3\times3}\}
% \end{equation}
such that the prior distribution of the Markov process on $\vmu^L$ is the Dirac distribution $\delta(\vec{\mu_0}-\vec{0})$ and $\delta(\vec{\rho_0}-\vec{1})$, 
% \begin{equation}
%     p_I^L(\boldsymbol{L}|\btheta_0^L)=\mathcal{N}(\bm{L}|\bm{0},\bm{I})
% \end{equation}
which ensures O(3)-invariance of prior distribution of $\vL$. By Eq. 77 from \citet{bfn}, the Bayesian flow distribution of the lattice parameter $\bm{L}$ is: 
\begin{align}% =p_U(\bmuL|\btheta_0^L,\bm{L},\beta(t))
p_F^L(\bmuL|\bm{L};t) &=\mathcal{N}(\bmuL|\gamma(t)\bm{L},\gamma(t)(1-\gamma(t))\bm{I}) 
\end{align}
where $\gamma(t) = 1 - \sigma_1^{2t}$ and $\sigma_1$ is the predefined hyper-parameter controlling the variance of input distribution at $t=1$ under linear entropy accuracy schedule. The variance parameter $\vrho$ does not need to be modeled and fed to the network, since it is deterministic given the accuracy schedule. After sampling $\bmuL_i$ from $p_F^L$, the training objective is defined as minimizing KL divergence between sender and receiver distribution (based on Eq. 96 in \citet{bfn}):
\begin{align}
\mathcal{L}_{L} = \frac{n}{2}\left(1-\sigma_1^{2/n}\right)\E_{i \sim \ui{n}}\E_{\flow(\bmuL_{i-1} |\vL ; t_{i-1})}  \frac{\left\|\vL -\predL{i-1}\right\|^2}{\sigma_1^{2i/n}},\label{eq:lattice_loss}
\end{align}
where the prediction term $\predL{i-1}$ is the lattice parameter part of network output. After training, the generation process is defined as the Bayesian update distribution given network prediction:
\begin{equation}\label{eq:lattice_sampling}
    p(\bmuL_{i}|\parsnt{i-1}^\calM)=\update^L(\bmuL_{i}|\predL{i-1},\bmuL_{i-1};t_{i-1})
\end{equation}
    

% The final prediction of the lattice parameter is given by $\bmuL_n = \predL{n-1}$.
% \begin{equation}\label{eq:final_lattice}
%     \bmuL_n = \predL{n-1}
% \end{equation}

\begin{restatable}{proposition}{latticeinv}\label{prop:latticeinv}
With $\net_{L}$ as  O(3)-equivariant function namely $\net_L(\parsnt{}^A,\parsnt{}^F,\vQ\parsnt{}^L,t)=\vQ\net_L(\parsnt{}^A,\parsnt{}^F,\parsnt{}^L,t),\forall\vQ^T\vQ=\vI$, the marginal distribution of $p(\bmuL_n)$ defined by \cref{eq:lattice_sampling} is O(3)-invariant. 
\end{restatable}


\textbf{Bayesian Flow of Atom Types \texorpdfstring{$\boldsymbol{A}$}{}} 
Given that atom types are discrete random variables located in a simplex $\calS^K$, the prior parameter of $\boldsymbol{A}$ is the discrete uniform distribution over the vocabulary $\parsnt{0}^A \defeq \frac{1}{K}\vec{1}_{1\times N}$. 
% \begin{align}\label{eq:disc_input_prior}
% \parsnt{0}^A \defeq \frac{1}{K}\vec{1}_{1\times N}
% \end{align}
% \begin{align}
%     (\oh{j}{K})_k \defeq \delta_{j k}, \text{where }\oh{j}{K}\in \R^{K},\oh{\vA}{KD} \defeq \left(\oh{a_1}{K},\dots,\oh{a_N}{K}\right) \in \R^{K\times N}
% \end{align}
With the notation of the projection from the class index $j$ to the length $K$ one-hot vector $ (\oh{j}{K})_k \defeq \delta_{j k}, \text{where }\oh{j}{K}\in \R^{K},\oh{\vA}{KD} \defeq \left(\oh{a_1}{K},\dots,\oh{a_N}{K}\right) \in \R^{K\times N}$, the Bayesian flow distribution of atom types $\vA$ is derived in \citet{bfn}:
\begin{align}
\flow^{A}(\parsn^A \mid \vA; t) &= \E_{\N{\y \mid \beta^A(t)\left(K \oh{\vA}{K\times N} - \vec{1}_{K\times N}\right)}{\beta^A(t) K \vec{I}_{K\times N \times N}}} \delta\left(\parsn^A - \frac{e^{\y}\parsnt{0}^A}{\sum_{k=1}^K e^{\y_k}(\parsnt{0})_{k}^A}\right).
\end{align}
where $\beta^A(t)$ is the predefined accuracy schedule for atom types. Sampling $\btheta_i^A$ from $p_F^A$ as the training signal, the training objective is the $n$-step discrete-time loss for discrete variable \citep{bfn}: 
% \oyyw{can we simplify the next equation? Such as remove $K \times N, K \times N \times N$}
% \begin{align}
% &\calL_A = n\E_{i \sim U\{1,n\},\flow^A(\parsn^A \mid \vA ; t_{i-1}),\N{\y \mid \alphat{i}\left(K \oh{\vA}{KD} - \vec{1}_{K\times N}\right)}{\alphat{i} K \vec{I}_{K\times N \times N}}} \ln \N{\y \mid \alphat{i}\left(K \oh{\vA}{K\times N} - \vec{1}_{K\times N}\right)}{\alphat{i} K \vec{I}_{K\times N \times N}}\nonumber\\
% &\qquad\qquad\qquad-\sum_{d=1}^N \ln \left(\sum_{k=1}^K \out^{(d)}(k \mid \parsn^A; t_{i-1}) \N{\ydd{d} \mid \alphat{i}\left(K\oh{k}{K}- \vec{1}_{K\times N}\right)}{\alphat{i} K \vec{I}_{K\times N \times N}}\right)\label{discdisc_t_loss_exp}
% \end{align}
\begin{align}
&\calL_A = n\E_{i \sim U\{1,n\},\flow^A(\parsn^A \mid \vA ; t_{i-1}),\N{\y \mid \alphat{i}\left(K \oh{\vA}{KD} - \vec{1}\right)}{\alphat{i} K \vec{I}}} \ln \N{\y \mid \alphat{i}\left(K \oh{\vA}{K\times N} - \vec{1}\right)}{\alphat{i} K \vec{I}}\nonumber\\
&\qquad\qquad\qquad-\sum_{d=1}^N \ln \left(\sum_{k=1}^K \out^{(d)}(k \mid \parsn^A; t_{i-1}) \N{\ydd{d} \mid \alphat{i}\left(K\oh{k}{K}- \vec{1}\right)}{\alphat{i} K \vec{I}}\right)\label{discdisc_t_loss_exp}
\end{align}
where $\vec{I}\in \R^{K\times N \times N}$ and $\vec{1}\in\R^{K\times D}$. When sampling, the transition from $\bthetaA_{i-1}$ to $\bthetaA_{i}$ is derived as:
\begin{equation}
    p(\btheta^A_{i}|\parsnt{i-1}^\calM)=\update^A(\btheta^A_{i}|\btheta^A_{i-1},\predA{i-1};t_{i-1})
\end{equation}

The detailed training and sampling algorithm could be found in \cref{alg:train} and \cref{alg:sampling}.





\begin{table*}[b] 
    \centering
     \label{tab:GPT}
    \caption{Results of Time-Series Forecasing with GPT-series Models. }
\begin{subtable}{\textwidth}
\centering
\caption{Results of Unimodal Short-term Time Series Forecasting with GPT-series Models.}
\label{tab:Uni_S_GPT}
\begin{tabular}{c|c|ccc|c} 
\hline
\multirow{2}{*}{Dataset}           & System 1    & \multicolumn{3}{c|}{~System 1 with~Test-time Reasoning Enhancement} & System~~~~ 2  \\ 
\cline{2-6}
                                   & GPT-4o      & with CoT     & with Self-Consistency & with Self-Correction  & o1-mini       \\ 
\hline
Agriculture & 0.021$\pm$0.011 & \worse{0.909$\pm$1.275} & \better{0.021$\pm$0.003} & \worse{0.025$\pm$0.007} & \worse{0.069$\pm$0.013} \\
 Climate & 1.599$\pm$0.500 & \worse{1.704$\pm$0.164} & \better{1.517$\pm$0.263} & \worse{1.998$\pm$0.677} & \better{1.412$\pm$0.159} \\
 Economy & 0.631$\pm$0.135 & \worse{0.638$\pm$0.410} & \better{0.450$\pm$0.171} & \worse{1.018$\pm$0.184} & \better{0.583$\pm$0.001} \\
 Energy & 0.363$\pm$0.110 & \better{0.258$\pm$0.029} & \better{0.167$\pm$0.242} & \worse{0.396$\pm$0.086} & \worse{0.930$\pm$0.747} \\
 Flu & 0.568$\pm$0.425 & \worse{0.592$\pm$0.291} & \better{0.481$\pm$0.288} & \worse{0.663$\pm$0.078} & \worse{1.441$\pm$1.234} \\
 Security & 0.093$\pm$0.029 & \worse{0.259$\pm$0.001} & \better{0.084$\pm$0.028} & \worse{0.165$\pm$0.070} & \worse{0.225$\pm$0.048} \\
 Employment & 0.010$\pm$0.004 & \better{0.006$\pm$0.002} & \worse{0.012$\pm$0.001} & \worse{0.013$\pm$0.003} & \worse{0.021$\pm$0.003} \\
Traffic & 0.385$\pm$0.471 & \better{0.113$\pm$0.063} & \worse{0.0468$\pm$0.009} & \better{0.053$\pm$0.009} & \worse{0.566$\pm$0.731} \\
\hline
Win System 1 & NA & $3/8$ & $5/8$ & $1/8$ & $2/8$ \\
\hline
\end{tabular}
\end{subtable}
 % Add vertical space between subtables
\begin{subtable}{\textwidth}
\centering
\caption{Short-term Unimodal forecasting (Gemini))}
\label{tab:performance_comparison_unimodal}
\begin{tabular}{c|c|ccc|c} 
\hline
\multirow{2}{*}{Dataset}           & System 1    & \multicolumn{3}{c|}{~System 1 with~Test-time Reasoning Enhancement} & System~~~~ 2  \\ 
\cline{2-6}
                                   & Gemini-2.0-flash      & with CoT     & with Self-Consistency & with Self-Correction  & Gemini-2.0-flash-thinking       \\ 
\hline
 Agriculture & 0.011$\pm$0.001 & \better{0.010$\pm$0.004} & \better{0.009$\pm$0.004} & \worse{0.012$\pm$0.008} & \worse{0.017$\pm$0.004} \\
 Climate & 1.234$\pm$0.239 & \worse{1.800$\pm$0.326} & \worse{1.749$\pm$0.791} & \worse{1.703$\pm$0.280} & \worse{2.416$\pm$0.112} \\
 Economy & 0.113$\pm$0.007 & \worse{0.272$\pm$0.256} & \worse{0.229$\pm$0.145} & \worse{0.121$\pm$0.026} & \worse{0.172$\pm$0.049} \\
 Energy & 0.172$\pm$0.038 & \worse{0.181$\pm$0.048} & \better{0.132$\pm$0.047} & \worse{0.235$\pm$0.060} & \worse{0.327$\pm$0.054} \\
 Flu & 0.809$\pm$0.353 & \better{0.641$\pm$0.224} & \better{0.402$\pm$0.197} & \worse{1.854$\pm$1.271} & \worse{2.068$\pm$1.076} \\
 Security & 0.170$\pm$0.054 & \worse{0.252$\pm$0.104} & \worse{0.380$\pm$0.323} & \worse{0.191$\pm$0.095} & \worse{0.259$\pm$0.001} \\
 Employment & 0.002$\pm$0.001 & \worse{0.005$\pm$0.003} & \worse{0.004$\pm$0.004} & \worse{0.004$\pm$0.002} & \worse{0.311$\pm$0.001} \\
Traffic & 0.347$\pm$0.415 & \better{0.097$\pm$0.060} & \better{0.016$\pm$0.006} & \better{0.034$\pm$0.014} & \better{0.201$\pm$0.001} \\
\hline
Win System 1 & NA & $3/8$ & $4/8$ & $1/8$ & $1/8$ \\
\hline
\end{tabular}
\end{subtable}
 \begin{subtable}{\textwidth}
\centering
\caption{Short-term Unimodal forecasting (DeepSeek))}
\label{tab:performance_comparison_unimodal}
\begin{tabular}{c|c|ccc|c} 
\hline
\multirow{2}{*}{Dataset}           & System 1    & \multicolumn{3}{c|}{~System 1 with~Test-time Reasoning Enhancement} & System~~~~ 2  \\ 
\cline{2-6}
                                   & DeepSeek-V3      & with CoT     & with Self-Consistency & with Self-Correction  & DeepSeek-R1       \\ 
\hline
 Agriculture & 0.038$\pm$0.032 & \better{0.019$\pm$0.001} & \worse{0.046$\pm$0.015} & \better{0.013$\pm$0.003} & \better{0.016$\pm$0.010} \\
 Climate & 1.216$\pm$0.202 & \worse{2.650$\pm$0.905} & \better{1.207$\pm$0.197} & \worse{1.246$\pm$0.081} & \worse{1.541$\pm$0.397} \\
 Economy & 0.406$\pm$0.218 & \worse{0.433$\pm$0.031} & \better{0.284$\pm$0.227} & \worse{0.441$\pm$0.161} & \worse{0.583$\pm$0.001} \\
 Energy & 0.736$\pm$0.752 & \better{0.212$\pm$0.022} & \better{0.187$\pm$0.011} & \better{0.182$\pm$0.063} & \better{0.189$\pm$0.021} \\
 Flu & 1.464$\pm$1.031 & \worse{1.650$\pm$0.236} & \better{0.980$\pm$0.445} & \worse{1.682$\pm$0.292} & \better{1.298$\pm$1.330} \\
 Security & 0.283$\pm$0.140 & \better{0.218$\pm$0.093} & \better{0.185$\pm$0.052} & \better{0.116$\pm$0.012} & \better{0.247$\pm$0.017} \\
 Employment & 0.036$\pm$0.019 & \better{0.020$\pm$0.006} & \better{0.035$\pm$0.019} & \better{0.018$\pm$0.007} & \better{0.012$\pm$0.005} \\
Traffic & 0.066±0.031 & \worse{0.201±0.001} & \worse{0.109±0.028} & \worse{0.107±0.067} & \worse{0.113±0.073} \\
\hline
Win System 1 & NA & 4/84/8 & 6/86/8 & 4/84/8 & 5/85/8 \\
\hline
\end{tabular}
\end{subtable}
\end{table*}
 
\begin{table*}[b] % 让整个表格固定在页面底部
    \centering
     \label{tab:combined_UniLong}
    \caption{Main table title containing three subtables}
\begin{subtable}{\textwidth}
\centering
\caption{Results of Unimodal Long-term Time Series Forecasting with GPT-series Models.}
\label{tab:Uni_L_GPT}
\begin{tabular}{c|c|ccc|c} 
\hline
\multirow{2}{*}{Dataset}           & System 1    & \multicolumn{3}{c|}{~System 1 with~Test-time Reasoning Enhancement} & System~~~~ 2  \\ 
\cline{2-6}
                                   & GPT-4o      & with CoT     & with Self-Consistency & with Self-Correction  & o1-mini       \\ 
\hline
 Agriculture & 0.093±0.057 & \worse{0.920±1.134} & \better{0.057±0.011} & \better{0.068±0.018} & \worse{0.293±0.089} \\
 Climate & 0.754±0.051 & \worse{1.199±0.132} & \worse{0.811±0.081} & \worse{0.877±0.041} & \better{0.708±0.058} \\
 Economy & 0.463$\pm$0.146 & \worse{1.040$\pm$0.482} & \worse{0.620$\pm$0.116} & \worse{0.748$\pm$0.069} & \better{0.359$\pm$0.001} \\
 Energy & 0.197$\pm$0.038 & \worse{0.746$\pm$0.500} & \better{0.177$\pm$0.062} & \worse{0.296$\pm$0.153} & \worse{0.926$\pm$0.771} \\
 Flu & 0.219$\pm$0.053 & \worse{0.967$\pm$0.412} & \worse{0.230$\pm$0.077} & \worse{0.639$\pm$0.479} & \worse{0.862$\pm$0.597} \\
 Security & 0.183$\pm$0.044 & \better{0.162$\pm$0.038} & \better{0.135$\pm$0.011} & \better{0.165$\pm$0.017} & \worse{0.211$\pm$0.075} \\
 Employment & 0.011$\pm$0.006 & \worse{0.013$\pm$0.002} & \better{0.009$\pm$0.003} & \worse{0.013$\pm$0.004} & \worse{0.053$\pm$0.015} \\
Traffic & 0.066$\pm$0.046 & \worse{0.218$\pm$0.158} & \better{0.046$\pm$0.016} & \better{0.036$\pm$0.008} & \worse{0.091$\pm$0.042} \\
\hline
Win System 1 & NA & 1/8 & 5/8 & 3/8 & 2/8 \\
\hline
\end{tabular}
\end{subtable}
\begin{subtable}{\textwidth}
\centering
\caption{Results of Unimodal Long-term Time Series Forecasting with Gemini-series Models.}
\label{tab:Uni_L_Gemini}
\begin{tabular}{c|c|ccc|c} 
\hline
\multirow{2}{*}{Dataset}           & System 1    & \multicolumn{3}{c|}{~System 1 with~Test-time Reasoning Enhancement} & System~~~~ 2  \\ 
\cline{2-6}
                                   & Gemini-2.0-flash      & with CoT     & with Self-Consistency & with Self-Correction  & Gemini-2.0-flash-thinking       \\ 
\hline
 Agriculture & 0.032$\pm$0.007 & \worse{0.036$\pm$0.011} & \worse{0.035$\pm$0.007} & \worse{0.077$\pm$0.026} & \worse{0.093$\pm$0.018} \\
 Climate & 1.476$\pm$0.651 & \better{0.964$\pm$0.321} & \better{0.674$\pm$0.092} & \better{0.908$\pm$0.153} & \better{1.240$\pm$0.705} \\
 Economy & 0.092$\pm$0.038 & \worse{0.216$\pm$0.142} & \better{0.078$\pm$0.013} & \better{0.066$\pm$0.003} & \worse{0.244$\pm$0.035} \\
 Energy & 0.303$\pm$0.044 & \better{0.130$\pm$0.021} & \better{0.241$\pm$0.060} & \worse{0.489$\pm$0.134} & \better{0.241$\pm$0.148} \\
 Flu & 1.190$\pm$1.171 & \better{1.049$\pm$0.447} & \better{0.596$\pm$0.128} & \better{1.095$\pm$0.546} & \worse{1.920$\pm$0.001} \\
 Security & 0.196$\pm$0.052 & \worse{0.533$\pm$0.493} & \worse{0.955$\pm$0.389} & \better{0.154$\pm$0.031} & \worse{0.207$\pm$0.001} \\
 Employment & 0.011$\pm$0.001 & \worse{0.019$\pm$0.007} & \better{0.009$\pm$0.002} & \worse{0.013$\pm$0.005} & \worse{0.268$\pm$0.001} \\
Traffic & 0.068$\pm$0.063 & \worse{0.215$\pm$0.079} & \worse{0.074$\pm$0.048} & \better{0.050$\pm$0.013} & \worse{0.414$\pm$0.001} \\
\hline
Win System 1& NA & $3/8$ & $5/8$ & $5/8$ & $2/8$ \\
\hline
\end{tabular}
\end{subtable}
\begin{subtable}{\textwidth}
\centering
\caption{Long-term Unimodal forecasting (DeepSeek))}
\label{tab:performance_comparison_unimodal}
\begin{tabular}{c|c|ccc|c} 
\hline
\multirow{2}{*}{Dataset}           & System 1    & \multicolumn{3}{c|}{~System 1 with~Test-time Reasoning Enhancement} & System~~~~ 2  \\ 
\cline{2-6}
                                   & DeepSeek-V3      & with CoT     & with Self-Consistency & with Self-Correction  & DeepSeek-R1       \\ 
\hline
 Agriculture & 0.216$\pm$0.049 & \better{0.102$\pm$0.034} & \better{0.103$\pm$0.014} & \better{0.121$\pm$0.065} & \better{0.091$\pm$0.019} \\
 Climate & 0.902$\pm$0.001 & \worse{1.383$\pm$0.227} & \better{0.786$\pm$0.153} & \worse{0.913$\pm$0.078} & \better{0.662$\pm$0.051} \\
 Economy & 0.613$\pm$0.776 & \better{0.540$\pm$0.386} & \better{0.393$\pm$0.113} & \worse{0.948$\pm$0.589} & \better{0.359$\pm$0.001} \\
 Energy & 0.603$\pm$0.359 & \better{0.575$\pm$0.452} & \worse{0.923$\pm$0.265} & \better{0.332$\pm$0.150} & \worse{1.396$\pm$0.001} \\
 Flu & 0.841$\pm$0.215 & \better{0.658$\pm$0.227} & \better{0.538$\pm$0.021} & \worse{0.939$\pm$0.328} & \worse{0.972$\pm$0.533} \\
 Security & 0.275$\pm$0.060 & \better{0.245$\pm$0.039} & \worse{0.280$\pm$0.004} & \better{0.186$\pm$0.033} & \better{0.168$\pm$0.028} \\
 Employment & 0.051$\pm$0.013 & \better{0.021$\pm$0.002} & \better{0.039$\pm$0.006} & \better{0.023$\pm$0.003} & \better{0.021$\pm$0.001} \\
Traffic & 0.414$\pm$0.001 & \better{0.209$\pm$0.145} & \worse{94.305$\pm$66.620} & \better{0.306$\pm$0.153} & \better{0.158$\pm$0.181} \\
\hline
Win System 1 & NA & 7/8 & 5/8 & 5/8 & 6/8 \\
\hline
\end{tabular}
\end{subtable}
\end{table*}
 
\begin{table*}[b] % 让整个表格固定在页面底部
    \centering
     \label{tab:combined_mul_s}
    \caption{Main table title containing three subtables}
\begin{subtable}{\textwidth}
\centering
\caption{Results of Multimodal Short-term Time Series Forecasting with GPT-series Models.}
\label{tab:Multi_S_GPT}
\begin{tabular}{c|c|ccc|c} 
\hline
\multirow{2}{*}{Dataset}           & System 1    & \multicolumn{3}{c|}{~System 1 with~Test-time Reasoning Enhancement} & System~~~~ 2  \\ 
\cline{2-6}
                                   & GPT-4o      & with CoT     & with Self-Consistency & with Self-Correction  & o1-mini       \\ 
\hline
 Agriculture & 0.018$\pm$0.015 & \better{0.018$\pm$0.011} & \better{0.013$\pm$0.008} & \better{0.018$\pm$0.006} & \worse{0.045$\pm$0.056} \\
 Climate & 1.716$\pm$0.580 & \worse{1.920$\pm$0.505} & \better{1.712$\pm$0.191} & \worse{2.042$\pm$0.609} & \better{1.603$\pm$0.496} \\
 Economy & 0.569$\pm$0.162 & \worse{0.940$\pm$0.445} & \better{0.291$\pm$0.127} & \better{0.503$\pm$0.071} & \worse{0.583$\pm$0.001} \\
 Energy & 0.541$\pm$0.457 & \better{0.316$\pm$0.125} & \better{0.187$\pm$0.090} & \better{0.225$\pm$0.080} & \worse{0.718$\pm$0.786} \\
 Flu & 0.548$\pm$0.164 & \worse{1.071$\pm$0.643} & \better{0.288$\pm$0.071} & \worse{1.261$\pm$1.164} & \worse{0.983$\pm$1.177} \\
 Security & 0.076$\pm$0.052 & \worse{0.110$\pm$0.087} & \worse{0.146$\pm$0.025} & \worse{0.151$\pm$0.035} & \worse{0.244$\pm$0.020} \\
 Employment & 0.020$\pm$0.006 & \worse{0.020$\pm$0.003} & \better{0.019$\pm$0.003} & \worse{0.021$\pm$0.004} & \worse{0.028$\pm$0.008} \\
Traffic & 0.551$\pm$0.396 & \worse{1.577$\pm$1.421} & \better{0.030$\pm$0.010} & \better{0.347$\pm$0.349} & \worse{0.911$\pm$0.594} \\
\hline
Win System 1 & NA & $2/8$ & $7/8$ & $4/8$ & $1/8$ \\
\hline
\end{tabular}
\end{subtable}
\begin{subtable}{\textwidth}
\centering
\caption{Short-term Multimodal forecasting (Gemini))}
\label{tab:performance_comparison_multimodal}
\begin{tabular}{c|c|ccc|c} 
\hline
\multirow{2}{*}{Dataset}           & System 1    & \multicolumn{3}{c|}{~System 1 with~Test-time Reasoning Enhancement} & System~~~~ 2  \\ 
\cline{2-6}
                                   & Gemini-2.0-flash      & with CoT     & with Self-Consistency & with Self-Correction  & Gemini-2.0-flash-thinking       \\ 
\hline

 Agriculture & 0.010$\pm$0.003 & \better{0.006$\pm$0.001} & \better{0.009$\pm$0.002} & \worse{0.011$\pm$0.004} & \better{0.008$\pm$0.002} \\
 Climate & 2.115$\pm$0.660 & \better{1.725$\pm$0.227} & \better{1.980$\pm$0.760} & \better{1.529$\pm$0.290} & \better{2.106$\pm$0.294} \\
 Economy & 0.376$\pm$0.085 & \better{0.326$\pm$0.067} & \better{0.373$\pm$0.079} & \better{0.283$\pm$0.083} & \worse{0.509$\pm$0.109} \\
 Energy & 0.143$\pm$0.069 & \better{0.117$\pm$0.015} & \worse{0.143$\pm$0.027} & \better{0.091$\pm$0.065} & \worse{0.218$\pm$0.106} \\
 Flu & 0.594$\pm$0.219 & \worse{0.607$\pm$0.294} & \better{0.332$\pm$0.102} & \worse{1.422$\pm$0.542} & \worse{3.171$\pm$0.001} \\
 Security & 0.558$\pm$0.604 & \better{0.145$\pm$0.050} & \better{0.172$\pm$0.119} & \better{0.141$\pm$0.065} & \better{0.259$\pm$0.001} \\
 Employment & 0.013$\pm$0.002 & \worse{0.015$\pm$0.002} & \better{0.011$\pm$0.002} & \better{0.011$\pm$0.003} & \worse{0.311$\pm$0.001} \\
Traffic & 0.322$\pm$0.196 & \better{0.046$\pm$0.017} & \better{0.163$\pm$0.106} & \worse{0.425$\pm$0.235} & \better{0.201$\pm$0.001} \\
\hline
Win System 1 & NA & $6/8$ & $7/8$ & $5/8$ & $4/8$ \\
\hline
\end{tabular}
\end{subtable}
\begin{subtable}{\textwidth}
\centering
\caption{Short-term Multimodal forecasting (DeepSeek))}
\label{tab:performance_comparison_multimodal}
\begin{tabular}{c|c|ccc|c} 
\hline
\multirow{2}{*}{Dataset}           & System 1    & \multicolumn{3}{c|}{~System 1 with~Test-time Reasoning Enhancement} & System~~~~ 2  \\ 
\cline{2-6}
                                   & DeepSeek-V3      & with CoT     & with Self-Consistency & with Self-Correction  & DeepSeek-R1       \\ 
\hline
 Agriculture & 0.032$\pm$0.012 & \better{0.027$\pm$0.006} & \better{0.023$\pm$0.001} & \worse{0.042$\pm$0.025} & \worse{2.712$\pm$0.001} \\
 Climate & 1.428$\pm$0.432 & \worse{1.857$\pm$0.431} & \better{1.371$\pm$0.001} & \better{1.411$\pm$0.258} & \worse{2.235$\pm$0.850} \\
 Economy & 0.427$\pm$0.174 & \worse{0.598$\pm$0.069} & \better{0.306$\pm$0.005} & \better{0.369$\pm$0.128} & \worse{0.615$\pm$0.101} \\
 Energy & 0.253$\pm$0.089 & \worse{0.486$\pm$0.318} & \better{0.197$\pm$0.001} & \worse{0.505$\pm$0.339} & \worse{0.731$\pm$0.777} \\
 Flu & 1.073$\pm$0.447 & \worse{1.564$\pm$0.982} & \better{0.362$\pm$0.161} & \better{0.441$\pm$0.173} & \worse{1.329$\pm$1.306} \\
 Security & 0.186$\pm$0.001 & \worse{0.206$\pm$0.010} & \worse{0.187$\pm$0.001} & \better{0.130$\pm$0.018} & \better{0.161$\pm$0.051} \\
 Employment & 0.016$\pm$0.001 & \worse{0.022$\pm$0.003} & \worse{0.016$\pm$0.001} & \worse{0.016$\pm$0.001} & \worse{0.114$\pm$0.139} \\
Traffic & 0.201$\pm$0.001 & \worse{0.201$\pm$0.001} & \worse{0.201$\pm$0.001} & \better{0.114$\pm$0.063} & \better{0.153$\pm$0.069} \\
\hline
Win System 1 & NA & $1/8$ & $5/8$ & $5/8$ & $2/8$ \\
\hline
\end{tabular}
\end{subtable}
\end{table*}
 
\begin{table*}[b] % 让整个表格固定在页面底部
    \centering
     \label{tab:combined_mul_s}
    \caption{Main table title containing three subtables}
\begin{subtable}{\textwidth}
\centering
\caption{Results of Multimodal Long-term Time Series Forecasting with GPT-series Models.}
\label{tab:Multi_L_GPT}
\begin{tabular}{c|c|ccc|c} 
\hline
\multirow{2}{*}{Dataset}           & System 1    & \multicolumn{3}{c|}{~System 1 with~Test-time Reasoning Enhancement} & System~~~~ 2  \\ 
\cline{2-6}
                                   & GPT-4o      & with CoT     & with Self-Consistency & with Self-Correction  & o1-mini       \\ 
\hline
 Agriculture & 0.110$\pm$0.065 & \better{0.097$\pm$0.044} & \better{0.063$\pm$0.009} & \better{0.051$\pm$0.042} & \worse{0.210$\pm$0.022} \\
 Climate & 1.365$\pm$0.479 & \better{0.995$\pm$0.109} & \better{1.065$\pm$0.014} & \better{0.912$\pm$0.004} & \worse{1.549$\pm$0.566} \\
 Economy & 0.487$\pm$0.237 & \worse{1.027$\pm$0.321} & \worse{0.500$\pm$0.184} & \worse{0.543$\pm$0.074} & \worse{0.827$\pm$0.662} \\
 Energy & 0.365$\pm$0.185 & \better{0.254$\pm$0.122} & \worse{33.743$\pm$23.911} & \better{0.293$\pm$0.026} & \worse{0.707$\pm$0.499} \\
 Flu & 0.291$\pm$0.065 & \worse{0.369$\pm$0.058} & \worse{0.445$\pm$0.210} & \worse{0.529$\pm$0.365} & \worse{1.070$\pm$0.284} \\
 Security & 0.196$\pm$0.056 & \better{0.188$\pm$0.027} & \better{0.140$\pm$0.028} & \better{0.116$\pm$0.041} & \worse{0.207$\pm$0.001} \\
 Employment & 0.015$\pm$0.002 & \worse{0.021$\pm$0.007} & \worse{0.021$\pm$0.002} & \worse{0.106$\pm$0.115} & \worse{0.031$\pm$0.003} \\
Traffic & 0.207$\pm$0.205 & \worse{0.341$\pm$0.402} & \better{0.045$\pm$0.013} & \worse{0.377$\pm$0.504} & \worse{1.482$\pm$1.788} \\
\hline
Win System 1 & NA & 4/84/8 & 4/84/8 & 4/84/8 & 0/80/8 \\
\hline
\end{tabular}
\end{subtable}
\begin{subtable}{\textwidth}
\centering
\caption{Long-term Multimodal forecasting (Gemini))}
\label{tab:performance_comparison_multimodal}
\begin{tabular}{c|c|ccc|c} 
\hline
\multirow{2}{*}{Dataset}           & System 1    & \multicolumn{3}{c|}{~System 1 with~Test-time Reasoning Enhancement} & System~~~~ 2  \\ 
\cline{2-6}
                                   & Gemini-2.0-flash      & with CoT     & with Self-Consistency & with Self-Correction  & Gemini-2.0-flash-thinking       \\ 
\hline
 Agriculture & 0.052$\pm$0.026 & \better{0.034$\pm$0.009} & \better{0.034$\pm$0.006} & \better{0.024$\pm$0.007} & \worse{0.096$\pm$0.032} \\
 Climate & 1.644$\pm$0.398 & \better{1.452$\pm$0.461} & \better{1.318$\pm$0.079} & \better{1.292$\pm$0.401} & \better{1.006$\pm$0.327} \\
 Economy & 0.092$\pm$0.010 & \worse{0.234$\pm$0.049} & \worse{0.134$\pm$0.044} & \worse{10.357$\pm$14.475} & \worse{1.093$\pm$0.806} \\
 Energy & 0.138$\pm$0.106 & \worse{0.208$\pm$0.116} & \worse{0.159$\pm$0.077} & \worse{0.384$\pm$0.074} & \worse{0.713$\pm$0.513} \\
 Flu & 0.659$\pm$0.173 & \better{0.557$\pm$0.164} & \better{0.477$\pm$0.006} & \worse{0.785$\pm$0.064} & \worse{1.920$\pm$0.001} \\
 Security & 0.123$\pm$0.062 & \better{0.109$\pm$0.020} & \worse{0.142$\pm$0.043} & \worse{0.151$\pm$0.053} & \worse{0.207$\pm$0.001} \\
 Employment & 0.029$\pm$0.004 & \better{0.022$\pm$0.003} & \better{0.026$\pm$0.003} & \better{0.026$\pm$0.002} & \worse{0.268$\pm$0.001} \\
Traffic & 0.085$\pm$0.068 & \better{0.037$\pm$0.027} & \better{0.020$\pm$0.007} & \better{0.058$\pm$0.010} & \worse{0.414$\pm$0.001} \\
\hline
Win System 1 & NA & $6/8$ & $5/8$ & $4/8$ & $1/8$ \\
\hline
\end{tabular}
\end{subtable}
\begin{subtable}{\textwidth}
\centering
\caption{Long-term Multimodal forecasting (DeepSeek))}
\label{tab:performance_comparison_multimodal}
\begin{tabular}{c|c|ccc|c} 
\hline
\multirow{2}{*}{Dataset}           & System 1    & \multicolumn{3}{c|}{~System 1 with~Test-time Reasoning Enhancement} & System~~~~ 2  \\ 
\cline{2-6}
                                   & DeepSeek-V3      & with CoT     & with Self-Consistency & with Self-Correction  & DeepSeek-R1       \\ 
\hline
 Agriculture & 0.088$\pm$0.058 & \better{0.063$\pm$0.022} & \worse{0.136$\pm$0.080} & \worse{0.119$\pm$0.078} & \better{0.019$\pm$0.010} \\
 Climate & 0.897$\pm$0.001 & \worse{2.193$\pm$0.330} & \worse{0.897$\pm$0.001} & \worse{0.939$\pm$0.074} & \worse{1.849$\pm$0.570} \\
 Economy & 0.629$\pm$0.147 & \better{0.558$\pm$0.282} & \better{0.486$\pm$0.074} & \better{0.623$\pm$0.218} & \worse{0.806$\pm$0.354} \\
 Energy & 0.995$\pm$0.139 & \worse{1.286$\pm$0.568} & \better{0.809$\pm$0.241} & \better{0.493$\pm$0.112} & \better{0.746$\pm$0.459} \\
 Flu & 2.624$\pm$2.400 & \better{0.974$\pm$0.446} & \better{0.644$\pm$0.488} & \better{1.135$\pm$0.643} & \better{1.560$\pm$0.957} \\
 Security & 0.179$\pm$0.002 & \worse{0.250$\pm$0.024} & \better{0.156$\pm$0.027} & \worse{0.274$\pm$0.071} & \better{0.134$\pm$0.055} \\
 Employment & 0.034$\pm$0.001 & \better{0.029$\pm$0.008} & \worse{0.034$\pm$0.001} & \better{0.030$\pm$0.005} & \worse{0.105$\pm$0.115} \\
Traffic & 0.414$\pm$0.001 & \worse{0.414$\pm$0.001} & \worse{0.414$\pm$0.001} & \better{0.192$\pm$0.157} & \better{0.152$\pm$0.185} \\
\hline
Win System 1 & NA & $4/8$ & $4/8$ & $5/8$ & $5/8$ \\
\hline
\end{tabular}
\end{subtable}
\end{table*}
\newpage
% An example of a double column floating figure using two subfigures.
% (The subfig.sty package must be loaded for this to work.)
% The subfigure \label commands are set within each subfloat command,
% and the \label for the overall figure must come after \caption.
% \hfil is used as a separator to get equal spacing.
% Watch out that the combined width of all the subfigures on a 
% line do not exceed the text width or a line break will occur.
%
%\begin{figure*}[!t]
%\centering
%\subfloat[Case I]{\includegraphics[width=2.5in]{box}%
%\label{fig_first_case}}
%\hfil
%\subfloat[Case II]{\includegraphics[width=2.5in]{box}%
%\label{fig_second_case}}
%\caption{Simulation results for the network.}
%\label{fig_sim}
%\end{figure*}
%
% Note that often IEEE papers with subfigures do not employ subfigure
% captions (using the optional argument to \subfloat[]), but instead will
% reference/describe all of them (a), (b), etc., within the main caption.
% Be aware that for subfig.sty to generate the (a), (b), etc., subfigure
% labels, the optional argument to \subfloat must be present. If a
% subcaption is not desired, just leave its contents blank,
% e.g., \subfloat[].


% An example of a floating table. Note that, for IEEE style tables, the
% \caption command should come BEFORE the table and, given that table
% captions serve much like titles, are usually capitalized except for words
% such as a, an, and, as, at, but, by, for, in, nor, of, on, or, the, to
% and up, which are usually not capitalized unless they are the first or
% last word of the caption. Table text will default to \footnotesize as
% the IEEE normally uses this smaller font for tables.
% The \label must come after \caption as always.
%
%\begin{table}[!t]
%% increase table row spacing, adjust to taste
%\renewcommand{\arraystretch}{1.3}
% if using array.sty, it might be a good idea to tweak the value of
% \extrarowheight as needed to properly center the text within the cells
%\caption{An Example of a Table}
%\label{table_example}
%\centering
%% Some packages, such as MDW tools, offer better commands for making tables
%% than the plain LaTeX2e tabular which is used here.
%\begin{tabular}{|c||c|}
%\hline
%One & Two\\
%\hline
%Three & Four\\
%\hline
%\end{tabular}
%\end{table}


% Note that the IEEE does not put floats in the very first column
% - or typically anywhere on the first page for that matter. Also,
% in-text middle ("here") positioning is typically not used, but it
% is allowed and encouraged for Computer Society conferences (but
% not Computer Society journals). Most IEEE journals/conferences use
% top floats exclusively. 
% Note that, LaTeX2e, unlike IEEE journals/conferences, places
% footnotes above bottom floats. This can be corrected via the
% \fnbelowfloat command of the stfloats package.




\section{Conclusion}
In this work, we propose a simple yet effective approach, called SMILE, for graph few-shot learning with fewer tasks. Specifically, we introduce a novel dual-level mixup strategy, including within-task and across-task mixup, for enriching the diversity of nodes within each task and the diversity of tasks. Also, we incorporate the degree-based prior information to learn expressive node embeddings. Theoretically, we prove that SMILE effectively enhances the model's generalization performance. Empirically, we conduct extensive experiments on multiple benchmarks and the results suggest that SMILE significantly outperforms other baselines, including both in-domain and cross-domain few-shot settings.

{\small
\bibliographystyle{IEEEtran}
\bibliography{bib}
}



% that's all folks
\end{document}


