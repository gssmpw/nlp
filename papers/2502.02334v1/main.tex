%% bare_jrnl.tex
%% V1.4b
%% 2015/08/26
%% by Michael Shell
%% see http://www.michaelshell.org/
%% for current contact information.
%%
%% This is a skeleton file demonstrating the use of IEEEtran.cls
%% (requires IEEEtran.cls version 1.8b or later) with an IEEE
%% journal paper.
%%
%% Support sites:
%% http://www.michaelshell.org/tex/ieeetran/
%% http://www.ctan.org/pkg/ieeetran
%% and
%% http://www.ieee.org/

%%*************************************************************************
%% Legal Notice:
%% This code is offered as-is without any warranty either expressed or
%% implied; without even the implied warranty of MERCHANTABILITY or
%% FITNESS FOR A PARTICULAR PURPOSE! 
%% User assumes all risk.
%% In no event shall the IEEE or any contributor to this code be liable for
%% any damages or losses, including, but not limited to, incidental,
%% consequential, or any other damages, resulting from the use or misuse
%% of any information contained here.
%%
%% All comments are the opinions of their respective authors and are not
%% necessarily endorsed by the IEEE.
%%
%% This work is distributed under the LaTeX Project Public License (LPPL)
%% ( http://www.latex-project.org/ ) version 1.3, and may be freely used,
%% distributed and modified. A copy of the LPPL, version 1.3, is included
%% in the base LaTeX documentation of all distributions of LaTeX released
%% 2003/12/01 or later.
%% Retain all contribution notices and credits.
%% ** Modified files should be clearly indicated as such, including  **
%% ** renaming them and changing author support contact information. **
%%*************************************************************************


% *** Authors should verify (and, if needed, correct) their LaTeX system  ***
% *** with the testflow diagnostic prior to trusting their LaTeX platform ***
% *** with production work. The IEEE's font choices and paper sizes can   ***
% *** trigger bugs that do not appear when using other class files.       ***                          ***
% The testflow support page is at:
% http://www.michaelshell.org/tex/testflow/



\documentclass[journal]{IEEEtran}
%
% If IEEEtran.cls has not been installed into the LaTeX system files,
% manually specify the path to it like:
% \documentclass[journal]{../sty/IEEEtran}





% Some very useful LaTeX packages include:
% (uncomment the ones you want to load)


% *** MISC UTILITY PACKAGES ***
%
%\usepackage{ifpdf}
% Heiko Oberdiek's ifpdf.sty is very useful if you need conditional
% compilation based on whether the output is pdf or dvi.
% usage:
% \ifpdf
%   % pdf code
% \else
%   % dvi code
% \fi
% The latest version of ifpdf.sty can be obtained from:
% http://www.ctan.org/pkg/ifpdf
% Also, note that IEEEtran.cls V1.7 and later provides a builtin
% \ifCLASSINFOpdf conditional that works the same way.
% When switching from latex to pdflatex and vice-versa, the compiler may
% have to be run twice to clear warning/error messages.






% *** CITATION PACKAGES ***
%
%\usepackage{cite}
% cite.sty was written by Donald Arseneau
% V1.6 and later of IEEEtran pre-defines the format of the cite.sty package
% \cite{} output to follow that of the IEEE. Loading the cite package will
% result in citation numbers being automatically sorted and properly
% "compressed/ranged". e.g., [1], [9], [2], [7], [5], [6] without using
% cite.sty will become [1], [2], [5]--[7], [9] using cite.sty. cite.sty's
% \cite will automatically add leading space, if needed. Use cite.sty's
% noadjust option (cite.sty V3.8 and later) if you want to turn this off
% such as if a citation ever needs to be enclosed in parenthesis.
% cite.sty is already installed on most LaTeX systems. Be sure and use
% version 5.0 (2009-03-20) and later if using hyperref.sty.
% The latest version can be obtained at:
% http://www.ctan.org/pkg/cite
% The documentation is contained in the cite.sty file itself.






% *** GRAPHICS RELATED PACKAGES ***
%
\ifCLASSINFOpdf
  % \usepackage[pdftex]{graphicx}
  % declare the path(s) where your graphic files are
  % \graphicspath{{../pdf/}{../jpeg/}}
  % and their extensions so you won't have to specify these with
  % every instance of \includegraphics
  % \DeclareGraphicsExtensions{.pdf,.jpeg,.png}
\else
  % or other class option (dvipsone, dvipdf, if not using dvips). graphicx
  % will default to the driver specified in the system graphics.cfg if no
  % driver is specified.
  % \usepackage[dvips]{graphicx}
  % declare the path(s) where your graphic files are
  % \graphicspath{{../eps/}}
  % and their extensions so you won't have to specify these with
  % every instance of \includegraphics
  % \DeclareGraphicsExtensions{.eps}
\fi
% graphicx was written by David Carlisle and Sebastian Rahtz. It is
% required if you want graphics, photos, etc. graphicx.sty is already
% installed on most LaTeX systems. The latest version and documentation
% can be obtained at: 
% http://www.ctan.org/pkg/graphicx
% Another good source of documentation is "Using Imported Graphics in
% LaTeX2e" by Keith Reckdahl which can be found at:
% http://www.ctan.org/pkg/epslatex
%
% latex, and pdflatex in dvi mode, support graphics in encapsulated
% postscript (.eps) format. pdflatex in pdf mode supports graphics
% in .pdf, .jpeg, .png and .mps (metapost) formats. Users should ensure
% that all non-photo figures use a vector format (.eps, .pdf, .mps) and
% not a bitmapped formats (.jpeg, .png). The IEEE frowns on bitmapped formats
% which can result in "jaggedy"/blurry rendering of lines and letters as
% well as large increases in file sizes.
%
% You can find documentation about the pdfTeX application at:
% http://www.tug.org/applications/pdftex





% *** MATH PACKAGES ***
%
%\usepackage{amsmath}
% A popular package from the American Mathematical Society that provides
% many useful and powerful commands for dealing with mathematics.
%
% Note that the amsmath package sets \interdisplaylinepenalty to 10000
% thus preventing page breaks from occurring within multiline equations. Use:
%\interdisplaylinepenalty=2500
% after loading amsmath to restore such page breaks as IEEEtran.cls normally
% does. amsmath.sty is already installed on most LaTeX systems. The latest
% version and documentation can be obtained at:
% http://www.ctan.org/pkg/amsmath





% *** SPECIALIZED LIST PACKAGES ***
%
%\usepackage{algorithmic}
% algorithmic.sty was written by Peter Williams and Rogerio Brito.
% This package provides an algorithmic environment fo describing algorithms.
% You can use the algorithmic environment in-text or within a figure
% environment to provide for a floating algorithm. Do NOT use the algorithm
% floating environment provided by algorithm.sty (by the same authors) or
% algorithm2e.sty (by Christophe Fiorio) as the IEEE does not use dedicated
% algorithm float types and packages that provide these will not provide
% correct IEEE style captions. The latest version and documentation of
% algorithmic.sty can be obtained at:
% http://www.ctan.org/pkg/algorithms
% Also of interest may be the (relatively newer and more customizable)
% algorithmicx.sty package by Szasz Janos:
% http://www.ctan.org/pkg/algorithmicx




% *** ALIGNMENT PACKAGES ***
%
%\usepackage{array}
% Frank Mittelbach's and David Carlisle's array.sty patches and improves
% the standard LaTeX2e array and tabular environments to provide better
% appearance and additional user controls. As the default LaTeX2e table
% generation code is lacking to the point of almost being broken with
% respect to the quality of the end results, all users are strongly
% advised to use an enhanced (at the very least that provided by array.sty)
% set of table tools. array.sty is already installed on most systems. The
% latest version and documentation can be obtained at:
% http://www.ctan.org/pkg/array


% IEEEtran contains the IEEEeqnarray family of commands that can be used to
% generate multiline equations as well as matrices, tables, etc., of high
% quality.




% *** SUBFIGURE PACKAGES ***
%\ifCLASSOPTIONcompsoc
%  \usepackage[caption=false,font=normalsize,labelfont=sf,textfont=sf]{subfig}
%\else
%  \usepackage[caption=false,font=footnotesize]{subfig}
%\fi
% subfig.sty, written by Steven Douglas Cochran, is the modern replacement
% for subfigure.sty, the latter of which is no longer maintained and is
% incompatible with some LaTeX packages including fixltx2e. However,
% subfig.sty requires and automatically loads Axel Sommerfeldt's caption.sty
% which will override IEEEtran.cls' handling of captions and this will result
% in non-IEEE style figure/table captions. To prevent this problem, be sure
% and invoke subfig.sty's "caption=false" package option (available since
% subfig.sty version 1.3, 2005/06/28) as this is will preserve IEEEtran.cls
% handling of captions.
% Note that the Computer Society format requires a larger sans serif font
% than the serif footnote size font used in traditional IEEE formatting
% and thus the need to invoke different subfig.sty package options depending
% on whether compsoc mode has been enabled.
%
% The latest version and documentation of subfig.sty can be obtained at:
% http://www.ctan.org/pkg/subfig




% *** FLOAT PACKAGES ***
%
%\usepackage{fixltx2e}
% fixltx2e, the successor to the earlier fix2col.sty, was written by
% Frank Mittelbach and David Carlisle. This package corrects a few problems
% in the LaTeX2e kernel, the most notable of which is that in current
% LaTeX2e releases, the ordering of single and double column floats is not
% guaranteed to be preserved. Thus, an unpatched LaTeX2e can allow a
% single column figure to be placed prior to an earlier double column
% figure.
% Be aware that LaTeX2e kernels dated 2015 and later have fixltx2e.sty's
% corrections already built into the system in which case a warning will
% be issued if an attempt is made to load fixltx2e.sty as it is no longer
% needed.
% The latest version and documentation can be found at:
% http://www.ctan.org/pkg/fixltx2e


%\usepackage{stfloats}
% stfloats.sty was written by Sigitas Tolusis. This package gives LaTeX2e
% the ability to do double column floats at the bottom of the page as well
% as the top. (e.g., "\begin{figure*}[!b]" is not normally possible in
% LaTeX2e). It also provides a command:
%\fnbelowfloat
% to enable the placement of footnotes below bottom floats (the standard
% LaTeX2e kernel puts them above bottom floats). This is an invasive package
% which rewrites many portions of the LaTeX2e float routines. It may not work
% with other packages that modify the LaTeX2e float routines. The latest
% version and documentation can be obtained at:
% http://www.ctan.org/pkg/stfloats
% Do not use the stfloats baselinefloat ability as the IEEE does not allow
% \baselineskip to stretch. Authors submitting work to the IEEE should note
% that the IEEE rarely uses double column equations and that authors should try
% to avoid such use. Do not be tempted to use the cuted.sty or midfloat.sty
% packages (also by Sigitas Tolusis) as the IEEE does not format its papers in
% such ways.
% Do not attempt to use stfloats with fixltx2e as they are incompatible.
% Instead, use Morten Hogholm'a dblfloatfix which combines the features
% of both fixltx2e and stfloats:
%
% \usepackage{dblfloatfix}
% The latest version can be found at:
% http://www.ctan.org/pkg/dblfloatfix




%\ifCLASSOPTIONcaptionsoff
%  \usepackage[nomarkers]{endfloat}
% \let\MYoriglatexcaption\caption
% \renewcommand{\caption}[2][\relax]{\MYoriglatexcaption[#2]{#2}}
%\fi
% endfloat.sty was written by James Darrell McCauley, Jeff Goldberg and 
% Axel Sommerfeldt. This package may be useful when used in conjunction with 
% IEEEtran.cls'  captionsoff option. Some IEEE journals/societies require that
% submissions have lists of figures/tables at the end of the paper and that
% figures/tables without any captions are placed on a page by themselves at
% the end of the document. If needed, the draftcls IEEEtran class option or
% \CLASSINPUTbaselinestretch interface can be used to increase the line
% spacing as well. Be sure and use the nomarkers option of endfloat to
% prevent endfloat from "marking" where the figures would have been placed
% in the text. The two hack lines of code above are a slight modification of
% that suggested by in the endfloat docs (section 8.4.1) to ensure that
% the full captions always appear in the list of figures/tables - even if
% the user used the short optional argument of \caption[]{}.
% IEEE papers do not typically make use of \caption[]'s optional argument,
% so this should not be an issue. A similar trick can be used to disable
% captions of packages such as subfig.sty that lack options to turn off
% the subcaptions:
% For subfig.sty:
% \let\MYorigsubfloat\subfloat
% \renewcommand{\subfloat}[2][\relax]{\MYorigsubfloat[]{#2}}
% However, the above trick will not work if both optional arguments of
% the \subfloat command are used. Furthermore, there needs to be a
% description of each subfigure *somewhere* and endfloat does not add
% subfigure captions to its list of figures. Thus, the best approach is to
% avoid the use of subfigure captions (many IEEE journals avoid them anyway)
% and instead reference/explain all the subfigures within the main caption.
% The latest version of endfloat.sty and its documentation can obtained at:
% http://www.ctan.org/pkg/endfloat
%
% The IEEEtran \ifCLASSOPTIONcaptionsoff conditional can also be used
% later in the document, say, to conditionally put the References on a 
% page by themselves.




% *** PDF, URL AND HYPERLINK PACKAGES ***
%
%\usepackage{url}
% url.sty was written by Donald Arseneau. It provides better support for
% handling and breaking URLs. url.sty is already installed on most LaTeX
% systems. The latest version and documentation can be obtained at:
% http://www.ctan.org/pkg/url
% Basically, \url{my_url_here}.




% *** Do not adjust lengths that control margins, column widths, etc. ***
% *** Do not use packages that alter fonts (such as pslatex).         ***
% There should be no need to do such things with IEEEtran.cls V1.6 and later.
% (Unless specifically asked to do so by the journal or conference you plan
% to submit to, of course. )


% correct bad hyphenation here
\hyphenation{op-tical net-works semi-conduc-tor}

\usepackage[table,xcdraw]{xcolor}
\usepackage{amssymb}
\newcommand{\YKL}[1]{\textcolor{red}{#1}}
\definecolor{rblue}{rgb}{0,0.5,1}
\definecolor{awesome}{rgb}{1.0, 0.13, 0.32}
\definecolor{hollywoodcerise}{rgb}{0.96, 0.0, 0.63}
\definecolor{lasallegreen}{rgb}{0.03, 0.47, 0.19}
\definecolor{hanpurple}{rgb}{0.32, 0.09, 0.98}
\definecolor{green(pigment)}{rgb}{0.0, 0.65, 0.31}

\usepackage[pagebackref=false,breaklinks=true,colorlinks,bookmarks=false]{hyperref}
\hypersetup{colorlinks=true,linkcolor={red},citecolor={hanpurple},urlcolor={magenta}}
\usepackage{times}
\usepackage{epsfig}
\usepackage{graphicx}
\usepackage{amsmath}
\usepackage{amssymb}
\usepackage{array}
\usepackage{float}
\usepackage{placeins}
\usepackage{morewrites}
\usepackage{makecell}
% \usepackage{subtitle}



% borrowed packages from monoscene
%\usepackage{xcolor}
\usepackage{caption}
\usepackage{subcaption}
\usepackage{tabularx}
\usepackage{rotating}
\usepackage{verbatim}
%\usepackage{xcolor,colortbl}
\usepackage{etoolbox}
\usepackage[normalem]{ulem}
\usepackage{booktabs}
\usepackage{multirow}
\usepackage{lipsum}
\usepackage{stmaryrd}
\usepackage{stackengine}
\usepackage{makecell}
\usepackage{pifont}
\usepackage{cancel}
\usepackage{adjustbox}
\usepackage{dblfloatfix}
\usepackage[table]{xcolor}
%\usepackage[pagebackref=true,breaklinks=true,letterpaper=true,colorlinks,bookmarks=false]{hyperref}
%\hypersetup{colorlinks}

%\definecolor{cvprblue}{rgb}{0.21,0.49,0.74}
%\usepackage[pagebackref,breaklinks,colorlinks,allcolors=cvprblue]{hyperref}

\newcommand{\cmark}{\ding{51}}%
\newcommand{\xmark}{\ding{55}}%

\DeclareMathOperator*{\argmin}{argmin}
\DeclareMathOperator*{\argmax}{argmax}

\newcommand{\Cbb}{\ensuremath{\mathbb{C}}} 
\newcommand{\Rbb}{\ensuremath{\mathbb{R}}} 
\newcommand{\Nbb}{\ensuremath{\mathbb{N}}}

\newcommand{\camera}{\includegraphics[width=3.8mm]{icons/camera.png}}
\newcommand{\LiDAR}{\includegraphics[width=4mm]{icons/LiDAR.png}}
\newcommand{\event}{\includegraphics[width=4mm]{icons/event.png}}
\newcommand{\radar}{\includegraphics[width=4mm]{icons/radar.png}}
\newcommand{\tick}{\includegraphics[width=4mm]{icons/tick.png}}
\newcommand{\cross}{\includegraphics[width=4mm]{icons/cross.png}}
\newcommand{\cir}{\includegraphics[width=4mm]{icons/circle.png}}

\newcommand{\real}{\includegraphics[width=4mm]{icons/real.png}}
\newcommand{\syn}{\includegraphics[width=4mm]{icons/syn.png}}

\usepackage[capitalize]{cleveref}
\crefname{section}{Sec.}{Secs.}
\Crefname{section}{Section}{Sections}
\Crefname{table}{Table}{Tables}
\crefname{table}{Tab.}{Tabs.}
\usepackage{colortbl}
%\usepackage[table,xcdraw]{xcolor}
% \pdfminorversion=5


% define colormap for SemanticKITTI
\definecolor{car_s}{rgb}{0.39215686, 0.58823529, 0.96078431}
\definecolor{bicycle_s}{rgb}{0.39215686, 0.90196078, 0.96078431}
\definecolor{motorcycle_s}{rgb}{0.11764706, 0.23529412, 0.58823529}
\definecolor{truck_s}{rgb}{0.31372549, 0.11764706, 0.70588235}
\definecolor{other-vehicle_s}{rgb}{0.39215686, 0.31372549, 0.98039216}
\definecolor{person_s}{rgb}{1.        , 0.11764706, 0.11764706}
\definecolor{bicyclist_s}{rgb}{1.        , 0.15686275, 0.78431373}
\definecolor{motorcyclist_s}{rgb}{0.58823529, 0.11764706, 0.35294118}
\definecolor{road_s}{rgb}{1.        , 0.        , 1.        }
\definecolor{parking_s}{rgb}{1.        , 0.58823529, 1.        }
\definecolor{sidewalk_s}{rgb}{0.29411765, 0.        , 0.29411765}
\definecolor{other-ground_s}{rgb}{0.68627451, 0.        , 0.29411765}
\definecolor{building_s}{rgb}{1.        , 0.78431373, 0.        }
\definecolor{fence_s}{rgb}{1.        , 0.47058824, 0.19607843}
\definecolor{vegetation_s}{rgb}{0.        , 0.68627451, 0.        }
\definecolor{trunk_s}{rgb}{0.52941176, 0.23529412, 0.        }
\definecolor{terrain_s}{rgb}{0.58823529, 0.94117647, 0.31372549}
\definecolor{pole_s}{rgb}{1.        , 0.94117647, 0.58823529}
\definecolor{traffic-sign_s}{rgb}{1.        , 0.        , 0.    }

% define colormap for Dsec
\definecolor{car_d}{rgb}{0.0, 0.0, 0.55686275}
\definecolor{bicycle_d}{rgb}{0.46666667, 0.04313725, 0.12549020}
\definecolor{motorcycle_d}{rgb}{0.0, 0.0, 0.90196078}
\definecolor{truck_d}{rgb}{0.0, 0.0, 0.27450980}
\definecolor{other-vehicle_d}{rgb}{0.90196078, 0.58823529, 0.54901961}
\definecolor{person_d}{rgb}{0.86274510, 0.07843137, 0.23529412}
\definecolor{road_d}{rgb}{0.50196078, 0.25098039, 0.50196078}
\definecolor{sidewalk_d}{rgb}{0.95686275, 0.13725490, 0.90980392}
\definecolor{building_d}{rgb}{0.27450980, 0.27450980, 0.27450980}
\definecolor{fence_d}{rgb}{0.74509804, 0.60000000, 0.60000000}
\definecolor{vegetation_d}{rgb}{0.41960784, 0.55686275, 0.13725490}
\definecolor{terrain_d}{rgb}{0.59607843, 0.98431373, 0.59607843}
\definecolor{pole_d}{rgb}{0.60000000, 0.60000000, 0.60000000}
\definecolor{traffic-sign_d}{rgb}{0.86274510, 0.86274510, 0.0}
% define class-frequencies for SemanticKITTI
\makeatletter
\newcommand{\car@semkitfreq}{3.92}
\newcommand{\bicycle@semkitfreq}{0.03}
\newcommand{\motorcycle@semkitfreq}{0.03}
\newcommand{\truck@semkitfreq}{0.16}
\newcommand{\othervehicle@semkitfreq}{0.20}
\newcommand{\person@semkitfreq}{0.07}
\newcommand{\bicyclist@semkitfreq}{0.07}
\newcommand{\motorcyclist@semkitfreq}{0.05}
\newcommand{\road@semkitfreq}{15.30}  %
\newcommand{\parking@semkitfreq}{1.12}
\newcommand{\sidewalk@semkitfreq}{11.13}  %
\newcommand{\otherground@semkitfreq}{0.56}
\newcommand{\building@semkitfreq}{14.1}  %
\newcommand{\fence@semkitfreq}{3.90}
\newcommand{\vegetation@semkitfreq}{39.3}  %
\newcommand{\trunk@semkitfreq}{0.51}
\newcommand{\terrain@semkitfreq}{9.17} %
\newcommand{\pole@semkitfreq}{0.29}
\newcommand{\trafficsign@semkitfreq}{0.08}
\newcommand{\semkitfreq}[1]{{\csname #1@semkitfreq\endcsname}}
% define class-frequencies for Dsec
\makeatletter
\newcommand{\car@dsecfreq}{1.71}
\newcommand{\bicycle@dsecfreq}{0.02}
\newcommand{\motorcycle@dsecfreq}{0.01}
\newcommand{\truck@dsecfreq}{0.17}
\newcommand{\othervehicle@dsecfreq}{0.26}
\newcommand{\person@dsecfreq}{0.05}
\newcommand{\road@dsecfreq}{29.02}
\newcommand{\sidewalk@dsecfreq}{16.58}
\newcommand{\building@dsecfreq}{17.22}
\newcommand{\fence@dsecfreq}{5.96}
\newcommand{\vegetation@dsecfreq}{26.88}
\newcommand{\terrain@dsecfreq}{1.50}
\newcommand{\pole@dsecfreq}{0.54}
\newcommand{\trafficsign@dsecfreq}{0.09}
\newcommand{\dsecfreq}[1]{{\csname #1@dsecfreq\endcsname}}
\makeatother

\begin{document}
%
% paper title
% Titles are generally capitalized except for words such as a, an, and, as,
% at, but, by, for, in, nor, of, on, or, the, to and up, which are usually
% not capitalized unless they are the first or last word of the title.
% Linebreaks \\ can be used within to get better formatting as desired.
% Do not put math or special symbols in the title.
\title{Event-aided Semantic Scene Completion}
%
%
% author names and IEEE memberships
% note positions of commas and nonbreaking spaces ( ~ ) LaTeX will not break
% a structure at a ~ so this keeps an author's name from being broken across
% two lines.
% use \thanks{} to gain access to the first footnote area
% a separate \thanks must be used for each paragraph as LaTeX2e's \thanks
% was not built to handle multiple paragraphs
%

\author{Shangwei Guo$^{1,*}$, Hao Shi$^{1,*}$, Song Wang$^{3}$, Xiaoting Yin$^{1}$, Kailun Yang$^{2,\dag}$, and Kaiwei Wang$^{1,\dag}$% <-this % stops a space
\thanks{This work was supported in part by Zhejiang Provincial Natural Science Foundation of China (Grant No. LZ24F050003), the National Natural Science Foundation of China (Grant No. 12174341 and No. 62473139), and in part by Shanghai SUPREMIND Technology Company Ltd.
}
\thanks{$^{1}$The authors are with the State Key Laboratory of Extreme Photonics and Instrumentation, Zhejiang University, Hangzhou 310027, China (email: wangkaiwei@zju.edu.cn).}%
\thanks{$^{2}$The author is with the School of Robotics and the National Engineering Research Center of Robot Visual Perception and Control Technology, Hunan University, Changsha 410082, China (email: kailun.yang@hnu.edu.cn).}%
\thanks{$^{3}$The author is with the College of Computer Science and Technology, Zhejiang University, Hangzhou 310027, China.}%
\thanks{$^*$Equal contribution.}%
\thanks{$^\dag$Corresponding authors: Kaiwei Wang and Kailun Yang.}%
}



% The paper headers

% The only time the second header will appear is for the odd numbered pages
% after the title page when using the twoside option.
% 
% *** Note that you probably will NOT want to include the author's ***
% *** name in the headers of peer review papers.                   ***
% You can use \ifCLASSOPTIONpeerreview for conditional compilation here if
% you desire.




% If you want to put a publisher's ID mark on the page you can do it like
% this:
%\IEEEpubid{0000--0000/00\$00.00~\copyright~2015 IEEE}
% Remember, if you use this you must call \IEEEpubidadjcol in the second
% column for its text to clear the IEEEpubid mark.



% use for special paper notices
%\IEEEspecialpapernotice{(Invited Paper)}




% make the title area
\maketitle

% As a general rule, do not put math, special symbols or citations
% in the abstract or keywords.
\begin{abstract}
\begin{abstract}
Retrieval-Augmented Generation (RAG) is often used with Large Language Models (LLMs) to infuse domain knowledge or user-specific information. In RAG, given a user query, a retriever extracts chunks of relevant text from a knowledge base. These chunks are sent to an LLM as part of the input prompt. Typically, any given chunk is repeatedly retrieved across user questions. However, currently, for every question, attention-layers in LLMs fully compute the key values (KVs) repeatedly for the input chunks, as state-of-the-art methods cannot reuse KV-caches when chunks appear at arbitrary locations with arbitrary contexts. Naive reuse leads to output quality degradation.  This leads to potentially redundant computations on expensive GPUs and increases latency. In this work, we propose \sys, a system for managing and reusing precomputed KVs corresponding to the text chunks (we call \textit{chunk-caches}) in RAG-based systems. We present how to identify \hl{\textit{chunk-caches} that are reusable}, how to efficiently perform a small fraction of recomputation to \textit{fix} the cache to maintain output quality, and how to efficiently store and evict \textit{chunk-caches} in the hardware for maximizing reuse while masking any overheads. With real production workloads as well as synthetic datasets, we show that \sys reduces redundant computation by \textbf{51\%} over SOTA prefix-caching and \textbf{75\%} over full recomputation.
\hl{Additionally, with continuous batching on a real production workload, we get a \textbf{1.6$\times$} speedup in throughput and a \textbf{2$\times$} reduction in end-to-end response latency over prefix-caching while maintaining quality, for both the \llama-3-8B and \llama-3-70B models. 
}
\end{abstract}





\end{abstract}

% Note that keywords are not normally used for peerreview papers.
\begin{IEEEkeywords}
Semantic Occupancy Prediction, Event Camera, Multimodal Perception, Semantic Scene Completion
\end{IEEEkeywords}






% For peer review papers, you can put extra information on the cover
% page as needed:
% \ifCLASSOPTIONpeerreview
% \begin{center} \bfseries EDICS Category: 3-BBND \end{center}
% \fi
%
% For peerreview papers, this IEEEtran command inserts a page break and
% creates the second title. It will be ignored for other modes.
\IEEEpeerreviewmaketitle



\section{Introduction}
\label{sec:intro}

\begin{figure*}[tb]
    \centering
    \includegraphics[width=0.848\linewidth]{figs/circuitnn.pdf} 
    \caption{Illustration of differentiable CircuitNN. CircuitNN is designed based on differentiable NAND gates. After DAS is guided by PI and PO pairs of the truth table, CircuitNN can get the precise circuit architecture logic equivalent to the truth table.}
    \label{fig:circuitnn}
\end{figure*}

% 1. Describe the importance of logic synthesis
% 2. Existing Problems
% (a) Neural Architecture Search: Unstable, Predefined Setting, etc.
% (b) Circuit Generation: Probabilistic Model, Logic Equivalence

With the rapid advancement of technology, the scale of integrated circuits (ICs) has expanded exponentially. 
This expansion has introduced significant challenges in chip manufacturing, particularly concerning power and area metrics.
A primary objective in IC design is achieving the same circuit function with fewer transistors, thereby reducing power usage and area occupancy.

Logic synthesis~\cite{hachtel2005logicsynth}, a critical step in electronic design automation (EDA), transforms behavioral-level circuit designs into optimized gate-level circuits, ultimately yielding the final IC layout. 
The primary goal of logic synthesis is to identify the physical implementation with the fewest gates for a given circuit function. 
This task constitutes a challenging NP-hard combinatorial optimization problem. 
Current logic synthesis tools~\cite{brayton2010abc, wolf2013yosys} rely on human-designed heuristics, often leading to sub-optimal outcomes.

Differentiable architecture search (DAS) techniques~\cite{liu2018darts, chu2020darts} offer novel perspectives on addressing challenges in this problem.
Circuit functions can be represented through truth tables, which map binary inputs to their corresponding outputs. 
Truth tables provide a precise representation of input-output relationships, ensuring the design of functionally equivalent circuits.
Inspired by this, researchers~\cite{deepmind2024ai4sys, wang2024tnet} have begun exploring the application of DAS to synthesize circuits directly from truth tables.
Specifically, \citet{deepmind2024ai4sys} proposed CircuitNN, a framework that learns differentiable connection structures with logic gates, enabling the automatic generation of logic circuits from truth tables.
This approach significantly reduces the complexity of traditional circuit generation. 
Building on this, \citet{wang2024tnet} introduced T-Net, a triangle-shaped variant of CircuitNN, incorporating regularization techniques to enhance the efficiency of DAS.

Despite these advancements, several challenges remain. 
The computational complexity of DAS grows quadratically with the number of gates, posing scalability issues.
Although triangle-shaped architecture~\cite{wang2024tnet} partially mitigates this problem, redundancy persists. 
%Additionally, DAS is susceptible to converging to local optima, limiting the ability to search architectures that satisfy the given truth tables~\cite{liu2018darts}. 
%Furthermore, hyperparameters (network depth and layer width) require extensive searches, introducing complexity and prolonging the synthesis process. 
Additionally, DAS is susceptible to converging to local optima~\cite{liu2018darts} and hyperparameters (network depth and layer width) require extensive searches. 
The challenges arise from the vast search space in DAS. 
% Even with predefined settings for CircuitNN, finding a configuration that meets the truth table requires extensive trial and error during the DAS process. 
Intuitively, limiting the search space through predefined parameters (network depth, gates per layer, and connection probabilities) can significantly reduce the complexity.

Recent advances~\cite{openai2023gpt4, abramson2024alphafold3, esser2024sd3, li2024mar} in conditional generative models have demonstrated remarkable performance across language, vision, and graph generation tasks. 
Motivated by these developments, we propose a novel approach to circuit generation that generates preliminary circuit structures to guide DAS in generating refined circuits matching specified truth tables. 
Firstly, we introduce CircuitVQ, a tokenizer with a discrete codebook for circuit tokenization. 
Built upon our Circuit AutoEncoder framework~\cite{hou2022graphmae,li2023maskgae,wu2025mgvga}, CircuitVQ is trained through a circuit reconstruction task. 
Specifically, the CircuitVQ encoder encodes input circuits into discrete tokens using a learnable codebook, while the decoder reconstructs the circuit adjacency matrix based on these tokens.
Subsequently, the CircuitVQ encoder serves as a circuit tokenizer for CircuitAR pretraining, which employs a masked autoregressive modeling paradigm~\cite{chang2022maskgit, li2023mage}. 
In this process, the discrete codes function as supervision signals. 
After training, CircuitAR can generate discrete tokens progressively, which can be decoded into initial circuit structures by the decoder of the CircuitVQ. 
These prior insights can guide DAS in producing refined circuits that match the target truth tables precisely.

Our key contributions can be summarized as follows:
\begin{itemize}
\item We introduce CircuitVQ, a circuit tokenizer that facilitates graph autoregressive modeling for circuit generation, based on our Circuit AutoEncoder framework;
\item Develop CircuitAR, a model trained using masked autoregressive modeling, which generates initial circuit structures conditioned on given truth tables;
\item Propose a refinement framework that integrates differentiable architecture search to produce functionally equivalent circuits guided by target truth tables;
\item Comprehensive experiments demonstrating the scalability and capability emergence of our CircuitAR and the superior performance of the proposed circuit generation approach.
\end{itemize}

% Motivation
% (a) Diffusion (Vision, Graph), Autoregressive (Language, Vision)
% (b) Circuit Generation for Predefined Setting
% (c) Neural Architecture Search for Strict Logic Equivalence

% Contribution
% (a) Circuit Tokenizer (new transformer arch, training strategy)
% (b) CircuitAR (train and gen strategies, post-ar strategy)
% (c) Extensive Evaluation including BitD (Bit Distance) for Scalability



\section{Related Work}
% \subsection{Vision Language Model}
% 시각장애인에서 상황을 설명할 DB가 없으니 만들었다. 그리고 이를 VLM에 튜닝했다.
\subsection{Technical approaches for assisting the visually-impaired}


\subsection{Datasets for visual instruction tuning}





% An example of a floating figure using the graphicx package.
% Note that \label must occur AFTER (or within) \caption.
% For figures, \caption should occur after the \includegraphics.
% Note that IEEEtran v1.7 and later has special internal code that
% is designed to preserve the operation of \label within \caption
% even when the captionsoff option is in effect. However, because
% of issues like this, it may be the safest practice to put all your
% \label just after \caption rather than within \caption{}.
%
% Reminder: the "draftcls" or "draftclsnofoot", not "draft", class
% option should be used if it is desired that the figures are to be
% displayed while in draft mode.
%
%\begin{figure}[!t]
%\centering
%\includegraphics[width=2.5in]{myfigure}
% where an .eps filename suffix will be assumed under latex, 
% and a .pdf suffix will be assumed for pdflatex; or what has been declared
% via \DeclareGraphicsExtensions.
%\caption{Simulation results for the network.}
%\label{fig_sim}
%\end{figure}

% Note that the IEEE typically puts floats only at the top, even when this
% results in a large percentage of a column being occupied by floats.

Frontier language models demonstrate a remarkable mismatch between their problem-solving capabilities and poor out-of-box verification capabilities.
These limitations have largely been attributed to the inability of current language models to self-diagnose hallucinations or enforce rigour \citep{zhang_how_2023,orgad_llms_2024,snyder_early_2024,kamoi_evaluating_2024, tyen_llms_2024, DBLP:conf/iclr/0009CMZYSZ24}.
However, our findings that models can be directed to accurately perform verifications at scale suggest that these out-of-box limitations can be addressed with standard methods like instruction tuning.
We compiled a set of challenging reasoning problems and candidate solutions to provide a benchmark for these deficits.

Each entry in this benchmark consists of a question, a correct candidate response, and an incorrect candidate response, and is manually curated from the residuals of our sampling-based search experiments (Section~\ref{section:pipeline}).
An example entry from this benchmark can be found below (see Appendix~\ref{app:examplebenchmark} for more).

\vspace{0.4cm}
\begin{tcolorbox}[title=Question from LiveBench Reasoning (Web-of-Lies Puzzle), breakable]
In this question, assume each person either always tells the truth or always lies. The person at the campground thinks their friend is lying. Mateo is at the aquarium. The person at the restaurant says the person at the hotel lies. Farid is at the movie theater. The person at the movie theater says the person at the campground lies. Ryan is at the shopping mall. The person at the cafe says the person at the campground lies. The person at the observatory says the person at the museum lies. The person at the museum says the person at the restaurant tells the truth. The person at the campground says the person at the airport tells the truth. The person at the campground tells the truth. Luna is at the restaurant. The person at the movie theater saw a firetruck. Isabella is at the observatory. The person at the amusement park says the person at the museum tells the truth. The person at the restaurant says the person at the shopping mall tells the truth. Liam is at the beach. Ayaan is at the airport. The person at the restaurant says the person at the beach tells the truth. Maya is at the amusement park. The person at the restaurant says the person at the aquarium lies. The person at the restaurant says the person at the movie theater lies. The person at the library says the person at the campground tells the truth. Elowen is at the hotel. Priya is at the library. The person at the gym says the person at the restaurant tells the truth. Kehinde is at the cafe. Anika is at the museum. Quan is at the campground. Soren is at the gym. Does the person at the museum tell the truth? Does the person at the restaurant tell the truth? Does the person at the movie theater tell the truth? Think step by step, and then put your answer in **bold** as a list of three words, yes or no (for example, **yes, no, yes**). If you don't know, guess.
\end{tcolorbox}
\definecolor{mydarkgreen}{rgb}{0.0, 0.5, 0.0} %


\begin{paracol}{2}

\switchcolumn[0]
\begin{tcolorbox}[breakable, colframe=mydarkgreen!70, colback=mydarkgreen!10, title=Correct response]
Here's how we can break down this logic puzzle:

1. **Quan is at the campground and tells the truth.** This is a given fact.

...

4. **The person at the restaurant (Luna) says the person at the hotel (Elowen) lies.**  We don't know yet if Luna is lying or telling the truth.

...

10. **Since Luna tells the truth about Farid, Luna *always* tells the truth.**  This unlocks several other pieces of information.

...

13. **The person at the cafe (Kehinde) says the person at the campground lies.** Since Quan tells the truth, Kehinde is lying.

Therefore:
* **Anika (museum):** Tells the truth.
* **Luna (restaurant):** Tells the truth.
* **Farid (movie theater):** Lies.

**yes, yes, no**
\end{tcolorbox}

\switchcolumn
\begin{tcolorbox}[breakable, colframe=red!70, colback=red!10, title=Incorrect Response Due to Hallucination]
Here's how we can solve this logic puzzle:

1. **Quan is at the campground and tells the truth.** This is given.

2. **The person at the airport (Ayaan) tells the truth.** Quan (campground) says the person at the airport tells the truth, and Quan tells the truth.

...

4. **Luna (restaurant) lies.**  The person at the restaurant says the person at the hotel (Elowen) lies. \textbf{\color{red} Later, the person at the restaurant says the person at the campground (Quan) lies.} Since Quan tells the truth, Luna (restaurant) must be lying.

...

Therefore:
* **Anika (museum): lies**
* **Luna (restaurant): lies**
* **Farid (movie theater): lies**

**Answer: no, no, no** 
\end{tcolorbox}
\end{paracol}

\noindent\begin{minipage}{\textwidth}
\captionof{figure}{Example of an entry in our verification benchmark. The question is sourced from the LiveBench Reasoning benchmark, and the two responses are generated by Gemini v1.5 Pro. The green response has the correct final answer; the red response has the wrong final answer due to hallucinating a non-existent clause.}
\label{fig:example}
\vspace{0.4cm}
\end{minipage}

\noindent
On each entry, our benchmark studies verification accuracy on two tasks:
\begin{enumerate}
    \item \textbf{Scoring task.} When given only the question and one of the responses, is the model able to discern the correctness of the response?
    \item \textbf{Comparison task.} When provided the whole tuple with the correctness labels of the responses masked and a guarantee that at least one response is correct, is the model able to discern which response is correct and which is incorrect?
\end{enumerate}

\noindent
The scoring task is also evaluated over a separate set of (question, response) pairs where the response reaches the correct final answer by coincidence but contains fatal errors and should be labeled by a reasonable verifier as being incorrect; an example can be found in Appendix~\ref{app:examplebenchmark}.
In the scoring task, models are provided only with the task description; in the comparison task, models are provided only with the task description and a suggestion to identify disagreements between responses in its reasoning.

Table~\ref{tab:benchmark} lists the baseline performances of current commercial model offerings on this benchmark.
Gemini v1.5 Pro is omitted from the benchmark as the entries in the benchmark are curated from the residuals of Gemini v1.5 Pro.
The prompts used in Table~\ref{tab:benchmark} are provided in Appendix~\ref{app:benchmarkprompts}.

As we previously observed, and has been noted in prior works \citep{tyen_llms_2024, kamoi_evaluating_2024}, verification errors are typically due to low recall.
Even the easier comparison task, models perform only marginally better---and often worse---than random chance.
In many cases, Consistency@5 performs worse than one-shot inference because Consistency simply averages out noise from an output distribution, meaning that a model biased towards producing an incorrect answer will do so with higher probability under Consistency.
Addressing these deficits in verification capabilities---which we see as low-hanging fruit for post-training---would enable not only better sampling-based search, but also other downstream applications of verification including reinforcement learning \citep[e.g.][]{o1-preview,deepseekai2025deepseekr1incentivizingreasoningcapability}, data flywheeling \citep[e.g.,][]{welleck_generating_2022}, and end-user experience (see Section~\ref{sec:related} for further discussion).


\begin{table}[htbp]
\centering
\begin{tabular}{llcccccc}
\toprule
\textbf{Model} & \textbf{Metric} & \multicolumn{3}{c}{\textbf{Scoring Accuracy}} & \multicolumn{1}{c}{\textbf{Comparison Accuracy}} \\
\cmidrule(lr){3-5} \cmidrule(lr){6-6}
 &  & \textbf{Correct} & \textbf{Wrong} & \textbf{Flawed} &  \\
\midrule
\multirow{2}{*}{GPT-4o} & Pass@1    & 76.5\%  & 31.0\% & 22.2\% & 43.2\%\\
 & Consistency@5 & 77.4\% & 30.0\% & 11.1\% & 35.4\% \\
\midrule
\multirow{2}{*}{Claude 3.5 Sonnet} & Pass@1 & 89.6\% & 22.5\% & 33.3\% & 56.1\% \\
 & Consistency@5 & 90.3\% & 17.5\% & 33.3\% & 61.2\% \\
\midrule
\multirow{2}{*}{o1-preview} & Pass@1 & 100\% & 68.8\% & 80.0\% & 84.5\% \\
 & Consistency@5 & 100\% & 79.4\% & 88.8\% & 92\% \\
\midrule
\multirow{2}{*}{Gemini 2.0 Flash} & Pass@1 & 73.5\% & 44.5\% & 60\% & 58\%  \\
 & Consistency@5 & 77.4\% & 42.5\% & 66.6\% & 58.7\% \\
\midrule
\multirow{2}{*}{Gemini 2.0 Thinking Flash} & Pass@1 & 75.4\% & 56.5\% & 53.3\%  & 80\%  \\
 & Consistency@5 & 77.4\%  & 55\% & 55.5\%  & 89.1\% \\
\midrule
\multicolumn{2}{c}{Random guessing}  & 80\% & 20\% & 20\% & 50\% \\
\bottomrule
\end{tabular}
\caption{Accuracy rates of commercial language models on our verification benchmark. For the task of response scoring (Scoring Accuracy), accuracy rates are broken down for entries that require identifying a correct response as being correct (Correct), entries that require identifying a wrong response as being wrong (Wrong), and entries that require identifying a wrong response that coincidentally reaches the correct answer as being wrong (Flawed).
GPT-4o and Claude 3.5 Sonnet only perform marginally better than random guessing across all tasks. o1-Preview performs better, but still fails to identify 20-30\% of wrong responses.
}
\label{tab:benchmark}
\end{table}


% \begin{figure}
%     \centering
%     \includegraphics[width=0.5\linewidth]{Move_teaser.pdf}
%     \caption{Comparison of different dynamic compute approaches. length of arrow indicates residual transformation per token while width indicates velocity of transformation.}
%     \label{fig:enter-label}
% \end{figure}

\section{Method}
\label{sec:method}
Residual connections play a crucial role in shaping token representations, yet their dynamics remain underexplored in the context of efficient decoding. In this work, we delve deeper into transformer residual dynamics and investigate how modulating residual transformation velocity can improve inference efficiency in token-level processing, optimizing both dense and sparse MoE transformers.


\subsection{Residual Dynamics and Motivation for Multi-rate Residuals} \label{sec:motivation}

To analyze how hidden representations evolve across different layers of a transformer architecture, it's crucial to consider the effect of residual connections. Each transformer decoder layer typically has residual connections across attention and MLP submodules. As the residual stream $h_i$ traverses from interval $E_j$ to $E_{j+1}$, it undergoes a residual transformation given by:  
% \begin{equation}
% \label{eq:slow_residual_transformation}
% H_{E_{j+1}} = H_{E_j} \prod_{i=E_j}^{E_{j+1}} \left( I + \mathcal{A}_i \right) \left( I + \mathcal{M}_i \right) \quad \text{where} \quad \mathcal{A}_i = f(c_i, h_{i}), \mathcal{M}_i = g(h_i)
% \end{equation}

\begin{equation} \label{eq:slow_residual_transformation}
h_{E_{j+1}} = h_{E_j} + \sum_{i=E_j}^{E_{j+1}-1} \left( \mathcal{A}_i(h_i) + \mathcal{M}_i(h_i + \mathcal{A}_i(h_i)) \right) \quad \text{where} \quad \mathcal{A}_i = f(c_i, h_{i}), \mathcal{M}_i = g(h_i). 
\end{equation}

Here, \( \mathcal{A}_i \) denotes the non-linear transformation introduced by the multi-head attention mechanism at layer \( i \), while \( \mathcal{M}_i \) corresponds to the non-linear transformation of the MLP block at the same layer. These transformations depend on the input residual stream \( h_i \) and, in the case of \( \mathcal{A}_i \), the previous contextual representation \( c_i \).\footnote{Normalization layers are typically applied in practice but are omitted here for simplicity of the argument.}


% For easy tokens, the magnitude and direction of this delta transformation become progressively smaller with each successive layer as shown in \cref{fig:delta_transformation}. Consequently, it is feasible to predict these tokens after only a few residual connections, whereas harder tokens necessitate more extensive processing through additional layers.

\begin{figure}[ht]
    \centering
    \begin{subfigure}{0.48\textwidth}
        \centering
        \includegraphics[width=\textwidth]{sections/figures/residual_change.pdf}
        \caption{}
        \label{fig:residual_change}
    \end{subfigure}%
    \hfill
    \begin{subfigure}{0.48\textwidth}
        \centering
        \includegraphics[width=\textwidth]{sections/figures/alignment_wrt_dedicated_model.pdf}
        \caption{}
    \label{fig:alignment_wrt_dedicated_model}
    \end{subfigure}
    \caption{(a) As residual streams propagate through the model, the directional shifts in the residuals become progressively smaller. (b) A dedicated model with $k$ layers achieves a faster rate of change in residual streams and higher alignment than base model leveraging early exit mechanisms at layer $k$.}
    \label{fig}
\end{figure}


To examine whether residual transformations can be accelerated across layers, we conducted experiments using a diverse set of prompts on a pre-trained Phi3 model~\cite{phi3_report}. As illustrated in \cref{fig:residual_change}, we measured the directional shift in residual states as \( 1 - \mathcal{C}(h_{i-1}, h_i) \), where \(\mathcal{C}\) denotes normalized cosine similarity. This shift is notably higher in the initial layers, gradually decreasing in subsequent layers. This behavior allows traditional early exit approaches to effectively accelerate decoding by enabling earlier exits for simpler tokens. However, these approaches typically rely on a distance-based approximation, where the full residual transformation of the model is approximated by the residual transformations of the initial layers. To gain deeper insights into the distance versus velocity aspects of residual transformation, we conducted a comparative study. Specifically, we trained an early exit head at layer $k$ of the Phi3 model, which consists of 32 layers, restricting the distance traveled by each token. To accelerate the residual transformation relative to number of layers, we trained a smaller model consisting of only $k$ layers, while keeping all other hyperparameters consistent. We then compared the next-token prediction accuracy of the early exit head of the base model with that of the smaller model. To ensure an equal number of trainable parameters, we inserted low-rank adapters into the smaller model and trained only these adapters, whereas, in the distance-based approach, we trained solely the early exit head. In addition, to accelerate the residual transformation in smaller model, we distilled the residual streams from the larger model by incorporating a distillation loss ~\cite{sanh2019distilbert} between the residual state at layer \(i\) of the smaller model and the residual state at layer \(4 \times i\) of the larger model. As shown in ~\cref{fig:alignment_wrt_dedicated_model} the smaller model demonstrates a significantly faster rate of change in residual streams, leading to higher next token prediction accuracy after $k$ layers compared to the base model that employs traditional early exit mechanisms after $k$ layers \cite{schuster2022confident, chen2023eellm, varshney-etal-2024-investigating}. This experimental setup, which modifies only the rate of change in residual streams while keeping other factors constant, suggests that dense transformers, trained with a fixed number of layers, may inherently possess a slow residual transformation bias.

This observation raises an intriguing question: if the rate of change in residual streams could be accelerated relative to the number of layers, is it possible to facilitate earlier alignment for a greater proportion of tokens? Earlier alignment would be beneficial to not only facilitate dynamic computation but also for generating speculative tokens efficiently with high acceptance rates in speculative decoding setups ~\cite{leviathan2023fast, chen2023accelerating}. 

%thereby enhancing the efficiency of early exiting? 
 % This bias likely constrains the effectiveness of early exiting, particularly for easier tokens. By addressing this limitation through accelerated residual transformations, we hypothesize that it is possible to substantially improve the efficiency and accuracy of early exit strategies in transformer models.

\subsection{Multi-Rate Residual Transformation} \label{m2r2_method}

To address the slow residual transformation bias described in ~\cref{sec:motivation}, we introduce \textit{accelerated residual streams} that operate at rate $R$ relative to original slow residual stream. We pair slow residual stream, $h$ with an accelerated residual stream, $p$, which has an intrinsic bias towards earlier alignment. Relative to ~\cref{eq:slow_residual_transformation}, accelerated residual transformation from interval $E_j$ to $E_{j+1}$ can be represented as: 

% \begin{equation}
% \label{eq:fast_residual_transformation}
% P_{E_{j+1}} = P_{E_j} \prod_{i=E_j}^{E_{j+1}} \left( I + \hat{\mathcal{A}_i} \right) \left( I + \hat{\mathcal{M}_i} \right) \quad \text{where} \quad \hat{\mathcal{A}_i} = \hat{f}(c_i, P_{i}), \hat{\mathcal{M}_i} = \hat{g}(P_{i})
% \end{equation}


\begin{equation} \label{eq:fast_residual_transformation}
p_{E_{j+1}} = p_{E_j} + \sum_{i=E_j}^{E_{j+1}-1} \left( \hat{\mathcal{A}_i}(p_i) + \hat{\mathcal{M}_i}(p_i + \hat{\mathcal{A}_i}(p_i)) \right) \quad \text{where} \quad \hat{\mathcal{A}_i} = \hat{f}(c_i, p_{i}), \hat{\mathcal{M}_i} = \hat{g}(h_i), 
\end{equation}



where $\hat{\mathcal{A}_i}$ and $\hat{\mathcal{M}_i}$ denote non-linear transformation added by layer $i$ to previous accelerated residual $p_{i}$. Similar to $\mathcal{A}_i$, non-linear transformation $\hat{\mathcal{A}_i}$ attends to same context $c_i$ but uses a different transformation $\hat{f}$ for accelerating $p_{E_j}$ relative to $h_{E_j}$. 

We integrate accelerated residual transformation directly into the base network using parallel accelerator adapters such that rank of accelerator adapters $R_p << d$ where $d$ denotes base model hidden dimension. This setup allows the slow residual stream $h_{E_j}$ to pass through the base model layers while the accelerated residual stream $p_{E_j}$ utilizes these parallel adapters as shown in ~\cref{fig:m2r2_main}. Both slow and accelerated residuals are processed in same forward pass via attention masking and incur negligible additional inference latency in memory bound decoding setups, while in compute bound decoding setups where FLOPs optimization is essential, accelerated residual stream utilizes a fraction of attention heads that of slow residual (see ~\cref{sec:flops_optimization}). Additionally, to maximize the utility of accelerated residual transformations without introducing dedicated KV caches, we propose a shared caching mechanism between the slow and accelerated streams which minimally impact alignment benefits of our approach while offering substantial memory savings (see ~\cref{fig:koala_alignment}). Specifically, the attention operation on the slow residuals \( \text{MHA}(h_t, h_{\leq t}, h_{\leq t}) \) is redefined for accelerated residuals as 
\[
\hat{\mathcal{A}} = MHA(p_t, h_{<t} \oplus p_t, h_{<t} \oplus p_t),
\]
where the accelerated residual at time-step $t$, \( p_t \) attends to the slow residual’s KV cache, facilitating the reuse of contextual information across both residual streams without incurring additional caching costs. Here, \(MHA(q, k, v) \) represents multi-head attention between query \( q \), key \( k \), and value \( v \).

\begin{figure}
    \centering
    \includegraphics[width=0.8\linewidth]{sections//figures/m2r2_main2.pdf}
    \caption{Multi-rate Residuals Framework: Slow residual stream of base model is accompanied by a faster stream that operates at a $2-(J+1)\times$ rate relative to the slow stream, undergoing transformations via accelerator adapters as detailed in \cref{m2r2_method}, where J denotes number of early exit intervals. Colors within the slow and fast residual streams indicate similarity, with matching colors representing the most closely aligned residual states. At the beginning of the forward pass and at each exit point, the accelerated residual state is initialized from the corresponding slow residual state to avoid gradient conflict during training (see ~\cref{sec:grad_conflict}). Early exiting decisions are informed by the Accelerated Residual Latent Attention (ARLA) mechanism, described in \cref{method_arla}, which evaluates residual dynamics across consecutive exit gates.}
    \label{fig:m2r2_main}
\end{figure}

% Furthermore. to maximize the benefits of fast residual transformations without using dedicated KV caches, we propose sharing the fast network’s cache with the slow network. Formally speaking, We modify attention operation on slow residuals $MHA(H_t, H_{<=t}, H_{<=t})$ as $MHA(P_{t}, H_{<t} \oplus P_t, H_{<t}  \oplus P_t)$ such that accelerated residuals attend to previous slow context KV cache, where $MHA(q,k,v)$ denotes multi head attention between query, $q$, key $k$ and value $v$.


\subsection{Enhanced Early Residual Alignment}
Early residual alignment is instrumental in optimizing early exiting, speculative decoding, and Mixture-of-Experts (MoE) inference mechanisms. In this section, we provide a detailed analysis of how accelerated residuals enhance these inference setups.

% By aligning the residual states of intermediate layers with the final output representations, the model can maintain high prediction accuracy even when computations are truncated at earlier layers. This enables more reliable early exiting, reducing the overall computational cost while preserving performance. Additionally, in speculative decoding, early residual alignment allows the model to make confident predictions using faster, partial computations, thereby accelerating inference without sacrificing output quality.


\subsubsection{Early Exiting} \label{method_early_exiting}

A prevalent strategy for enabling early exiting at an intermediate layer $E_{j}$ involves approximating the residual transformation between $E_{j}$ and the final layer $N-1$ using a linear, context independent mapping, $\mathcal{T}$, such that $H_{N-1} \approx \mathcal{T}(H_{E_{j}})$. This approximation has been extensively employed in conventional approaches ~\cite{schuster2022confident, chen2023eellm, varshney-etal-2024-investigating}, providing a computationally efficient means to project the output of deeper layers from intermediate states. Specifically, residual state of layer $N-1$ with this approximation can be expressed as:


% \begin{equation}
% \label{eq: vanila_ea_assumption}
% \Phi(H_{E_{j}}) \sim H_{E_{j}} \prod_{i=E_{j}}^{N}\left( I + \mathcal{A}_i \right) \left( I + \mathcal{M}_i \right) \quad \text{where} \quad \Phi \perp C
% \end{equation}

\begin{equation} \label{eq:early_exiting}
h_{E_j} + \sum_{i=E_j}^{N-1} \left( \mathcal{A}_i(h_i) + \mathcal{M}_i(h_i + \mathcal{A}_i(h_i)) \right) \sim \mathcal{T}(h_{E_{j}})  \quad \text{where} \quad \mathcal{T} \perp c. 
\end{equation}


Here, $\mathcal{A}_i$ and $\mathcal{M}_i$ represent the residual contributions of the multi-head attention and MLP layers, respectively, while $\mathcal{T}$ remains independent of $c$, the preceding context.

This approach is inherently limited by two major factors: first, the assumption of linearity between $h_{E_{j}}$ and $h_{N-1}$ may not hold uniformly for all tokens, particularly when $E_j \ll N$. Second, the linear transformation $\mathcal{T}$ disregards the influence of the context $c$ and fails to account for the latent representations of previous contextual states. In contrast, M2R2 accelerated residual states mitigate both of these challenges by approximating the slow residual transformation of all layers via a faster residual transformation of fewer layers as:
% \begin{equation}
% H_{E_j} \prod_{i=E_j}^{N}\left( I + \mathcal{A}_i \right) \left( I + \mathcal{M}_i \right) \sim P_{E_j} \prod_{i=E_j}^{E_j+1}\left( I + \hat{\mathcal{A}_i} \right) \left( I + \hat{\mathcal{M}_i} \right)
% \end{equation}


\begin{equation} \label{eq:m2r2_approximating_ea}
h_{E_j} + \sum_{i=E_j}^{N-1} \left( \mathcal{A}_i(h_i) + \mathcal{M}_i(h_i + \mathcal{A}_i(h_i)) \right) \sim p_{E_j} + \sum_{i=E_j}^{E_{j+1}-1} \left( \hat{\mathcal{A}_i}(p_i) + \hat{\mathcal{M}_i}(p_i + \hat{\mathcal{A}_i}(p_i)) \right), 
\end{equation}

% \begin{equation} \label{eq:fast_residual_transformation}
% p_{E_{j+1}} = p_{E_j} + \sum_{i=E_j}^{E_{j+1}-1} \left( \hat{\mathcal{A}_i}(p_i) + \hat{\mathcal{M}_i}(p_i + \hat{\mathcal{A}_i}(p_i)) \right) \quad \text{where} \quad \hat{\mathcal{A}_i} = \hat{f}(c_i, p_{i}), \hat{\mathcal{M}_i} = \hat{g}(h_i) 
% \end{equation}






where $p_{E_j}$ is initialized from the slow residual state $h_{E_j}$ at each early exit interval $E_j$ using an identity transformation (see ~\cref{fig:m2r2_main}). As shown in ~\cref{fig:m2r2_residual_sim}, accelerated residuals offer a smoother, more consistent shift in residual direction across layers, in contrast to the abrupt changes typically seen at early exit points in standard early exit methods. Moreover, the normalized cosine similarity between accelerated states at early exit intervals and final residual states is substantially higher compared to traditional early exit techniques, highlighting improved alignment with final layer representations. Traditional adaptive compute methods are constrained by two principal factors: the number of tokens eligible for early exit at intermediate layers and the precision of early exit decision. If residual streams fail to saturate early, the majority of tokens remain ineligible for exit, thereby diminishing potential speedups. Additionally, imprecise delineations between tokens suitable for early exit can lead to underthinking (premature exits that adversely affect accuracy) or overthinking (unnecessary processing that compromises efficiency) ~\cite{zhou2020self, dai2020dynamic}. Enhanced early alignment using ~\cref{eq:m2r2_approximating_ea} helps to address  first issue. To address the second issue we introduce Accelerated Residual Latent Attention, which dynamically assesses the saturation of the residual stream, allowing for a more precise differentiation between tokens that can exit early and those requiring further processing.

% This results in uniform change in residual direction    
% % We keep $\mathcal{A} = \hat{\mathcal{A}}$, while $\hat{\mathcal{M}}$ is accelerated by a factor of $2 - (N_{E}+1)X$ relative to the slower residual transformation $\mathcal{M}$, where $N_E$ represents number of early exiting intervals.
% Figure~\cref{fig:rate_change_comparison} illustrates the comparative rate of change between these transformation streams.



% fig:rate_change_comparison
% - grid plot x axis -> layer id (0, 8) , y axis -> layer id -> dark color cell for max similarity , lighter for lower 
% 
-------------------------------------------------------
Let's consider residual stream $h_i$ traverses through interval $E_j$ to $E_{j+1}$ and undergoes residual transformation given by 
\begin{equation}
h_{E_{j+1}} = h_{E_j} \prod_{i=E_j}^{E_{j+1}} \left( 1 + \delta_i \right)    
\end{equation}

where $\delta_i$ denotes non-linear transformation added by layer $i$. Each non-linear transformation of layer $i$ is a function of previous contextual representation, $c_i$ and input residual stream $h_i-1$ as
$\delta_i = f(c_i, h_{i-1})$ 

One way to exit early at exit $E_j+1$ is to assume that residual transformation from $E_j+1$ to final layer $N-1$ can be approximated by a linear function $\phi$ as $h_{N-1} \sim \Phi(h_{E_j+1})$ and most conventional approaches such as \todo{cite EA papers} use this approach. In other words, 

\begin{equation}
\Phi(h_{E_j+1} \sim h_{E_j+1} \prod_{i=E_j+1}^{N} \left( 1 + \delta_i \right)   
\end{equation}

This approach suffers from two primary issues, linearity assumption from $h_E_j+1$ to $H_N-1$ if often incorrect, particularly when $E_j << N$. More importantly, linear transformation $\Phi$ doesn't consider effect of context $C_i$. M2R2  effectively addresses these issues as accelerated residual stream at interval $E_j+1$ can be represented as 

\begin{equation}
r_{E_{j+1}} = r_{E_j} \prod_{i=E_j}^{E_{j+1}} \left( 1 + \gamma_i \right)    
\end{equation}

where $\gamma_i$ denotes non-linear transformation added by layer $i$ to previous accelerated residual $r_i-1$. Similar to $\delta_i$, non-linear transformation $\gamma_i$ considers context $C_i$ as 
$\gamma_i = g(c_i, r_{i-1})$. So in summary, slow residual transformation is approximated by accelerated residual as: 

\begin{equation}
h_{E_j} \prod_{i=E_j}^{N} \left( 1 + \delta_i \right) \sim h_{E_j} \prod_{i=E_j}^{E_j+1} \left( 1 + \gamma_i \right)
\end{equation}

It's worth noting that accelerated residual $r_i$ and slow residual $h_i$ are processed concurrently at layer $i$ by constructing proper attention mask such as attention of slow residual is represented as 

$MHA(H_it, H_{i<=t}, H_{i<=t}$ while attention of fast residual is computed as 

$MHA(r_it, H_{i<=t}, H_{i<=t}$ where $MHA(q,k,v$ denotes multi head attention between query, $q$, key $k$ and value $v$.


------------------------------------------------------------------

Vertical latent attention on accelerated residual is computed as 
$MHA(S_mt, S(Ej<=i<=m)t, S(Ej<=i<=m)t)$ where $Smt$ denotes query/key/value projection in latent domain at layer $m$ at time $t$. 
------------------------------------------------------------------

Gradient conflict Avoidance: 

Let's consider $w_j$ is a trainable parameter that belongs to a layer between $E_j$ and $E_j+1$. Consider early exit loss at gate $E_j+1$, $L_j+1$, gradient propagation of $w_j$ at another trainable parameter $w_j-n$ can be gives as 

$\sum_{k=E_j-n}^{E_j} \beta_k \frac{\partial L_{E_k}}{\partial w_k}$

where $\beta_j$ denotes backward transformation coefficient for weight $w_j$ to reach gate $E_j$. 
 
On the other hand, gradient propagation in proposed approach can be represented as 

\[
\frac{\partial L_{E_j}}{\partial w_j} = 
\begin{cases} 
\beta_j \frac{\partial L_{E_j}}{\partial w_j} & \text{if } E_j \leq w_j \leq E_{j+1} \\
0 & \text{otherwise}
\end{cases}
\]







% \begin{figure}[ht]
%     \centering
%     \includegraphics[width=0.8\textwidth, height=5cm]{rate_change_comparison.png}
%     \caption{Rate of change comparison between fast and slow residual streams.}
%     \label{fig:rate_change_comparison}
% \end{figure}

%vary k and and plot EA accuracy for larger and smaller models. 

% \begin{figure}[ht]
%     \centering
%     \includegraphics[width=0.5\textwidth,height=5cm]{sections/figures/alignment_comparison_dialogsum.pdf}
%     \caption{Alignment of exited tokens for different early exit layers using traditional early exiting heads, dedicated faster networks, and faster residuals.}
%     \label{fig:small_model_early_exiting}
% \end{figure}


\textbf{Accelerated Residual Latent Attention} \label{method_arla}

In the context of residual streams, we observe that the decision to exit at a given layer can be more effectively informed by analyzing the dynamics of residual stream transformations, instead of solely relying on a classification head applied at the early exit interval $E_j$. To capture the subtle dynamics of residual acceleration, we propose a \textit{Accelerated Residual Latent Attention} (ARLA) mechanism. This approach involves making the exit decision at gate $E_j$ by attending to the residuals spanning from gate $E_{j-1}$ to $E_j$, rather than considering only the residual at gate $E_j$. To minimize the computational overhead associated with exit decision-making, the attention mechanism operates within the latent domain as depicted in ~\cref{fig:arla_arch}. Formally, for each interval $[E_j, E_{j+1}]$, the accelerated residuals are projected into Query ($Q^s_{E_j}, \ldots, Q^s_{E_{j+1}}$), Key ($K^s_{E_j}, \ldots, K^s_{E_{j+1}}$), and Value ($V^s_{E_j}, \ldots, V^s_{E_{j+1}}$) vectors, with latent dimension $d^s$ for $Q^s$, $K^s$, and $V^s$ being significantly smaller than hidden dimension of $p$.\footnote{We use $d^s = 64$ for experiments described in ~\cref{sec:experiments}.} Notably, when the router is allowed to make exit decisions at gate $E_j$ based on residual change dynamics, we observe that the attention is not confined to the residual state at $E_j$ but is distributed across residual states from $E_{j-1}$ to $E_j$, %as illustrated in Figure~\ref{fig:vertical_latent_attention_dynamics}. 
This broader focus on residual dynamics significantly reduces decision ambiguity in early exits, as demonstrated in Figure~\ref{fig:roc_arla}, which contrasts routers based on the last hidden state, and the proposed ARLA router.

%show R -> S transformation. 
%show parameter and flop overhead as compared to adapter on last hidden state.

% \begin{figure}[ht]
%     \centering
%     \includegraphics[width=0.5\textwidth,height=5cm]{sections/figures/roc_arla.pdf}
%     \caption{ROC curves of early exit decision strategies: confidence-based methods (CALM/LITE), routers based on the accelerated hidden state, and latent attention routers.}
%     \label{fig:decision_making_comparison}
% \end{figure}

% \begin{figure}[ht]
%     \centering
%     \includegraphics[width=0.5\textwidth,height=5cm]{vertical_latent_attention.png}
%     \caption{Vertical latent attention mechanism for optimizing early exit decisions by considering residuals from gate \(M\) through \(M-1\).}
%     \label{fig:vertical_latent_attention}
% \end{figure}

\begin{figure}[ht]
    \centering
    \begin{subfigure}{0.52\textwidth}
        \centering
        \includegraphics[width=\textwidth, height = 4cm]{sections/figures/arla_arch.pdf}
        \caption{Accelerated Residual Latent Attention (ARLA): Accelerated residuals between early exit gates are projected into latent domain and attention over residual states within the interval is computed to capture residual dynamics and exit decision is made based on residual saturation.}
        \label{fig:arla_arch}
    \end{subfigure}%
    \hfill
    \begin{subfigure}{0.45\textwidth}
        \centering
        \includegraphics[width=\textwidth, height = 4.5cm]{sections/figures/vla_roc.pdf}
        \caption{ROC classification curves of early exit decision strategies using a linear router used on last residual state ~\cite{schuster2022confident, varshney-etal-2024-investigating, chen2023eellm}  and using ARLA approach that considers residual dynamics. }
        \label{fig:roc_arla}
    \end{subfigure}
    \caption{Effectiveness of ARLA in capturing residual dynamics for early exiting decisions.}


\end{figure}



% \begin{figure}[ht]
%     \centering
%     \includegraphics[width=1\textwidth,height=5cm]{sections/figures/arla.pdf}
%     \caption{fig that plots 32 rows 2 cols heatmap showing attention at each gate}
%     \label{fig:vertical_latent_attention_dynamics}
% \end{figure}

\subsubsection{Self Speculative Decoding} \label{method_self_speculative_decoding}

An alternative means to exploit the early alignment properties of our approach is through the use of accelerated residual states for speculative token sampling to accelerate autoregressive decoding. Speculative decoding aims to speed up memory-bound transformer inference by employing a lightweight draft model to predict candidate tokens, while verifying speculated tokens in parallel and advancing token generation by more than one token per full model invocation \cite{leviathan2023fast, chen2023accelerating, xia2023speculative, miao2023specinfer}. Despite its effectiveness in accelerating large language models (LLMs), speculative decoding introduces substantial complexity in both deployment and training. A separate draft model must be specifically trained and aligned with the target model for each application, which increases the training load and operational complexity ~\cite{chen2023accelerating}. Additionally, this approach is resource-inefficient, as it requires both the draft and target models to be simultaneously maintained in memory during inference \cite{leviathan2023fast, chen2023accelerating}. 

One strategy to address this inefficiency is to leverage the initial layers of the target model itself to generate speculative candidates, as depicted in ~\cite{Tang2024}. While this method reduces the autoregressive overhead associated with speculation, it suffers from suboptimal acceptance rates. This occurs because the linear transformation employed for translating hidden states from layer $k$ to the final layer $N$ is typically a poor approximation, as discussed in ~\cref{sec:motivation} and ~\cref{method_early_exiting}. Our approach resolves this limitation by utilizing accelerated residuals, which demonstrate higher fidelity to their slower counterparts. By utilizing accelerated residuals operating at a rate of $N/k$, where $k$ denotes the number of layers used for candidate speculation, we are able to efficiently generate speculative tokens for decoding.\footnote{We typically set $k = 4$ to balance the trade-off between autoregressive drafting overhead and acceptance rate, as discussed in~\cref{sec:experiments}.}
 This technique not only obviates the need for multiple models during inference but also improves the overall efficiency and effectiveness of speculative decoding.

\begin{figure}
    \centering    \includegraphics[width=1\linewidth]{sections/figures/m2r2_aot_loading.pdf}
    \caption{Ahead-of-Time Expert Loading: M2R2 accelerated residual stream predicts experts required for future layers, reducing reliance on on-demand lazy loading. Speculative pre-loading is efficiently overlapped with computation of multi-head attention (MHA) and MLP transformations. Only incorrectly speculated experts are loaded lazily, resulting in faster inference steps and improved computational efficiency. Here, H indicates LBM Host while D indicates HBM Device.}
    \label{fig:moe_expert_aot_loading}
\end{figure}


\subsubsection{Ahead of Time Expert Loading:} \label{method_aot_expert_loading}

Recent advancements in sparse Mixture-of-Experts (MoE) architectures ~\cite{shazeer2017outrageously, fedus2022switch, artetxe2019massively, lepikhin2020gshard, zoph2022designing} have introduced a paradigm shift in token generation by dynamically activating only a subset of experts per input, achieving superior efficiency in comparison to dense models, particularly under memory-bound constraints of autoregressive decoding \cite{fedus2022switch, zoph2022designing}. This sparse activation approach enables MoE-based language models to generate tokens more swiftly, leveraging the efficiency of selective expert usage and avoiding the overhead of full dense layer invocation. In dense transformer models, pre-loading layers is a common strategy to enhance throughput, as computations of current layer can be overlapped with pre-loading of next layer parameters ~\cite{narayanan2021efficient, shoeybi2020megatron}. However, MoE models face a unique challenge: expert selection occurs dynamically based on previous layer’s output, making it infeasible to preload next layer’s experts in parallel. This limitation results in inherent latency, as expert loading becomes a sequential, on-demand process ~\cite{lepikhin2020gshard, fedus2022switch}.

To address this inefficiency, our method introduces a mechanism with \textit{accelerated residuals}, which not only captures key characteristics of base slower residual states but also exhibit high cosine similarity with their final counterparts (as illustrated in \cref{fig:m2r2_residual_sim}). By employing accelerated residual streams, we can effectively predict the necessary experts for future layers well in advance of their actual invocation. Specifically, using a $2\times$ accelerated residual, the experts needed for layers $2i+2$ and $2i+3$ can be identified while still computing in layer $i$, thus overcoming the bottleneck of sequential, on-demand expert selection and mitigating latency in the decoding pipeline, as shown in \cref{fig:moe_expert_aot_loading}. Note that, we use fixed set of accelerator adapters for transforming accelerated residuals (as discussed in ~\cref{m2r2_method}) while slow residual is transformed via expert routing mechanism. 

Furthermore, our approach integrates a Least Recently Used (LRU) caching strategy, which enhances memory efficiency by replacing the least recently used experts with speculated experts that are anticipated to be needed in upcoming layers. This hybrid approach of preemptive expert loading with LRU caching yields substantial improvements over traditional on-demand loading or standalone caching strategies. By minimizing cache misses and efficiently managing memory, this approach addresses both compute and memory bottlenecks, leading to faster, more resource-efficient token generation in MoE architectures. A comprehensive evaluation of this strategy, in relation to state-of-the-art methods, is provided in \cref{experiments_aot}, and the compute and memory traces on an A100 GPU are detailed in \cref{fig:moe_aot_cuda_trace}.



% Recent advancements in sparse Mixture-of-Experts (MoE) architectures have introduced the concept of utilizing distinct computational paths for different tokens \cite{shazeer2017outrageously}. This approach, wherein only a subset of experts are activated per input, enables MoE-based language models to generate tokens more swiftly compared to their dense counterparts due to memory-bound nature of auto-regressive decoding. In dense models, pre-loading layers in advance is a common strategy to enhance computational efficiency. However, this technique is not applicable to MoE models, where expert selection occurs dynamically based on the outputs of previous layers, preventing parallel pre-fetching of experts.

% Our proposed method addresses this inefficiency. Accelerated residuals, which are highly similar to their slower counterparts (see \cref{fig:similarity}), can reliably predict the necessary experts ahead of time. For instance, by utilizing $2X$ accelerated residual stream, we can predict the experts needed for the layer $2i+1$ and $2i+3$ while carrying out computation in layer $i$. This enables us to commence expert loading significantly earlier, as illustrated in \cref{expert_loading}, effectively mitigating the delays observed with the naive on-demand expert loading. Additionally, our method benefits from incorporating a Least Recently Used (LRU) strategy, where speculated experts replace those that are least recently utilized, resulting in improved performance compared to using either strategy alone. For a comprehensive evaluation, refer to \cref{moe_trace}, which provides a CUDA compute and memory trace of our approach executed on <>.



% A naive solution involves using the residual state of the previous layer along with the gating function of the next layer to predict which experts need to be loaded, and initiating the expert loading process in parallel with the attention computation of the next layer. Yet, as shown in \cref{fig:MOE_attn_vs_loading_time}, the attention computation for medium to long contexts is considerably faster than the expert loading time, making this approach inefficient.




\subsection{Training} \label{method_training}
% This approach is feasible due to the absence of gradient conflicts, as discussed in \cref{sec:grad_conflict}.

To accelerate residual streams, we employ parallel accelerator adapters as described in \cref{m2r2_method}.  For the early exiting use-case outlined in \cref{method_early_exiting}, we define the training objective for these adapters using the following loss function, which combines cross-entropy loss at each exit $E_j$ with distillation loss at each layer $i$. Loss weights coefficients $\alpha_0$ and $\alpha_1$ are employed to balance contribution of corresponding losses.

\begin{align} \label{eq:mr_loss}
L_{\text{m2r2}} = \underbrace{-\alpha_0 \sum_{j=1}^{J} \sum_{t=1}^{T} \log p_{\theta} \left( \hat{y}_t^{E_j} \mid y_{<t}, x \right)}_{\text{cross-entropy loss}} 
+ \underbrace{\alpha_1\sum_{i=1}^{E_{J-1}} \sum_{t=1}^{T} \| \mathbf{p}_{t}^{i} - \mathbf{h}_{t}^{((i - E_{j(i)}) \cdot R_i) + E_{j(i)})} \|^2}_{\text{distillation loss}}.
\end{align}

where $\hat{y}_t^{E_j}$ denotes the predictions from the accelerated residual stream at layer $E_j$ and time step $t$, $y_t$ represents the corresponding ground truth tokens, and $x$ indicates previous context tokens. The distillation loss at each layer $i$ is computed by comparing accelerated residuals at layer $i$ with slow residuals at layer $(i - E_{j(i)}) \cdot R_i + E_{j(i)}$, where $R_i$ denotes the rate of accelerated residuals at layer $i$ while $E_{j(i)}$ represents the most recent gate layer index such that $E_{j(i)} <= i$. \( J \) represents the total number of early exit gates, N denotes number of hidden layers and $E_j$ denotes layer index corresponding to gate index $j$ and \( T \) denotes the sequence length. 

In dynamic compute settings, after training of accelerator adapters, we optimize the query, key, and value parameters governing the ARLA routers (see ~\cref{method_arla}) across all exits in parallel on binary cross entropy loss between predicted decision and ground truth exiting decision. The ground truth labels for the router are determined based on whether the application of the final logit head on $\hat{y}_t^{E_j}$ yields the correct next-token prediction. 


% The objective for this optimization is defined by the following loss function:


%TODO are equations required ? 
% \begin{equation} \label{eq:arla_loss_combined}\small
%     L_{\text{arla}} = -\frac{1}{N} \sum_{t=1}^{T} \left( \sum_{j=1}^{E_n} \left[ O_t^{E_j} \log(\hat{O}_t^{E_j}) + (1 - O_t^{E_j}) \log(1 - \hat{O}_t^{E_j}) \right] \right), \quad \text{where} \quad 
%     O_t^{E_j} = \begin{cases} 
%     1, & \text{if } L(\hat{y}_t^{E_j}) = y_t^{E_j} \\
%     0, & \text{otherwise}
%     \end{cases}
% \end{equation}

% where $\hat{O}_t^{E_j}$ represents the binary predicted logits produced by the vertical latent attention router, as described in \cref{sec:arla}, at gate $E_j$ and time step $t$, and $O_t^{E_j}$ denotes the corresponding ground truth labels. The ground truth labels for the router are determined based on whether the application of the logit head on $\hat{y}_t^{E_j}$ yields the correct next-token prediction. The parameters controlling vertical latent attention are trained concurrently to ensure consistency and efficient use of computational resources.

For self-speculative decoding, as described in \cref{method_self_speculative_decoding}, the training objective remains the same as \cref{eq:mr_loss}, but with the number of intervals set to $J = 1$ and the rate of residual transformation set to $R_n = N/k$, where the first $k$ layers generate speculative candidate tokens. In the context of Ahead-of-Time Expert Loading for Mixture-of-Experts (MoE) models (see \cref{method_aot_expert_loading}), setting the rate of residual transformation to $R_n = 2$ typically offers a good trade-off between the accuracy of expert speculation and AoT pre-loading of experts. 

% Thus, we set $J = 1$ and $E_1 = 16$.


~\subsection{FLOPs Optimization} \label{sec:flops_optimization}

Naively implemented, M2R2 incurs higher FLOP overhead compared to traditional speculative decoding and early exiting approaches such as ~\cite{medusa, schuster2022confident, Tang2024}. However, modern accelerators demonstrate compute bandwidth that exceeds memory access bandwidth by an order of magnitude or more~\cite{databricksLLMInference2023, jouppi2021ten}, meaning increased FLOPs do not necessarily translate to increased decoding latency. Nevertheless, to ensure fair comparison and efficiency in compute bound scenarios, we introduce targeted optimizations.

~\textbf{Attention FLOPs Optimization} For medium-to-long context lengths, attention computation dominates FLOPs in the self-attention layer, surpassing the contribution from MLP layers. Specifically, matrix multiplications involving queries, cached keys, and cached values scale with $l_{kv} * l_{q}$ where $l_{kv}$ denotes previous context length and $l_q$ denotes current query length. Since M2R2 pairs accelerated residuals with slow residuals, a naive implementation results in twice the FLOPs consumption compared to a standard attention layer. To address this, we limit the attention of accelerated residual stream to selectively attend to the top-k most relevant tokens, identified by the slow residual stream based on top attention coefficients\footnote{We set to k = 64 and attend to top 64 tokens as identified by the slow residual stream.}. This is possible since slow and accelerated residual streams are processed in same forward pass and accelerated streams have access to attention coefficients of slow stream. Note that, the faster residual stream still retains the flexibility to assign distinct attention coefficients to these tokens. Furthermore, we design the faster residual stream to employ only 8 attention heads, compared to the 32 heads used in the slow residual stream of the Phi-3 model, reducing query, key, value, and output projection FLOPs by a factor of 1/4. ~\cref{fig:m2r2_num_heads_ablation} indicates effect of using a slicker stream on alignment. As depicted, using $\hat{n}_h = 8$ offers a good trade-off between alignment and FLOPs overhead. 

~\textbf{MLP FLOPs Optimization} The accelerator adapters operating on the accelerated residual stream are intentionally designed with lower rank than their counterparts in the base model. This reduces FLOP overhead by a factor proportional to $hiddenSize / rank$. Additionally, since the faster residual stream uses only 8 attention heads (compared to 32 in the slow residual stream of Phi-3), the subsequent MLP layers process a smaller set of activations, further reducing FLOPs by another factor of 1/4.

These optimizations significantly reduce the FLOP overhead per speculative draft generation, as illustrated in ~\cref{fig:flops_optmization}. Notably, while traditional early-exiting speculative approaches such as DEED require propagating the full slow residual state through the initial layers, incurring substantial computational costs, M2R2 achieves efficient token generation via slimmer, low-rank faster residual streams. In contrast, Medusa introduces considerable FLOP overhead due to per-head computations scaling with $d^2+dv$\footnote{Here $d$ denotes hidden state dimension while $v$ denotes vocab size.}, whereas M2R2 employs low-rank layers for both MLP and language modeling heads, maintaining computational efficiency. All experiments involving the M2R2 approach, as detailed in ~\cref{sec:experiments}, are conducted using these FLOPs optimizations.









% \[
% O_t^{E_j} = 
% \begin{cases} 
% 1, & \text{if } L(\hat{y}_t^{E_j}) = y_t^{E_j} \\
% 0, & \text{otherwise}
% \end{cases}
% \]




%add distillation
% We train accelerator adapters described in \cref{m2r2_method} to accelerate residual streams on next token prediction all in parallel since there are no gradient conflict issues as described in \cref{sec:grad_conflict}.

% \begin{align} \label{eq:mr_loss}
% L_{mr} =  & -\sum_{j = 1}^{E_n} (\sum_{t=1}^{T}\log p_{\theta} (\hat{y}_t^{E_j} | \hat{y}_{<t}, x)) \nonumber
% \end{align}

% where $\hat{y_t^{E_j}}$ denotes predicted logits obtained from accelerated residual stream at gate $E_j$ and time-step $t$ while $y_t^{E_j}$ denotes corresponding truth tokens. 

% Upon training of adapters responsible for accelerating residual streams, we train query, key, value parameters responsible for vertical latent attention of all gates in parallel as

% \begin{equation} \label{eq:arla_loss}
%     L_{arla} = -\frac{1}{N} (\sum_{t=1}^{T}(1\sum_{j=1}^{E_n} \left[ O_t^{E_j} \log(\hat{O}_t^{E_j}) + (1 - o_t^{E_j}) \log(1 - \hat{o_t}_{E_j}) \right]))
% \end{equation}

% where $\hat{O_t^{E_j}}$ denotes binary predicted logits obtained from vertical latent attention router described in \cref{sec:arla} at gate $E_j$ and timestep $t$ while $O_t^{E_j}$ denotes corresponding truth label. Truth labels for router are obtained by computing whether logit head application on $\hat{y}_t^j$ results in true next token prediction. Formally speaking, 

% $O_t^{E_j} = 1 if L(\hat{y_t^{E_j}}) == y_t^{E_j} , 0 otherwise$. 

% Parameters responsible for vertical latent attention are also trained in parallel as well. 

%todo: training slow and fast residuals together and distillation can be two training mdoes. 
%Distillation can be an ablation. 




% Although transformer decoding is memory bound on most mainstream accelerators, there could be scenarios where flop savings are crucial. For instance, on on-device settings power consumption is directly correlated with flops per decoding step and reducing flops does help with overall energy consumption. Vanilla early exiting methods help with flop reduction but suffer from mismatch between training and inference due to early exited tokens. If token at decoding step $t$, $T_t$ exited at layer $E_i$, while token $T_{t+k}$ exits at layer $E_j$ such that $E_i < E_j$, hidden state $H_{t+k}l$ does not have corresponding hidden state $H_tl$ to attend to where $E_i < l <= E_j$. One solution that's often used in literature is to rely on last hidden state available, $H_t{E_j}$, however it tends to be sub-optimal and does affect generation quality \cite{ref}.  To alleviate this mismatch while reducing flops, we train router such that attention mask between token $T_{t+k}$ and token $T_{<t+k}$ is given by: 

% \begin{equation}
%     a_{T_{{t+k}{T_{<t+k}}} = 1 if  E_{T_{<t+k}} >= E{T_{t+k}}
%     else 0
% \end{equation}

% This attention mask enables router to account for exited tokens and get trained accordingly. Since attention mechanism during decoding remains exactly same as that during training, impact on generation quality tends to be minimal as noted in \cref{fig:gen_auality_with_and_without_recompute_attention_show_flops}.  Although MoD does not suffer from training and inference mismatch, we observe that it suffers from discountinuity between pre-training and super-vised fine-tuning resulting in sub-optimal perplexity. On the other hand, our method doesn't not require pre-training , doesn't suffer from discountinuity, and achieves much better perplexity in super-vised fine-tuning and instruction tuning setups as shown in \cref{fig:Mod_vs_m2r2_loss_curves}.






% Our techniques are directly applicable in such scenarios.    




%expert loading with cuda streams in experiments

%
\begin{figure*}[!t]
\includegraphics[width=1.0\linewidth]{figures/corruption_image.pdf}
%\vskip-2ex
\caption{\textbf{Overview of degradation modes in the SemanticKITTI dataset.} 
The degradation modes are illustrated, where clean image (a) serves as the baseline. Fog (b), brightness (c), darkness (d), and motion blur (f) represent degradations influenced by environmental factors or the interaction between the environment and sensors. Shot noise (e), primarily caused by sensor limitations or camera malfunctions, is included to evaluate the ability of event cameras to compensate for the shortcomings of traditional cameras in adverse scenarios.
}
\label{fig:corruption}
%\vskip-3ex
\end{figure*}
\begin{table*}[h]
    % \scriptsize
    \setlength{\tabcolsep}{0.0035\linewidth}
    \caption{\textbf{Semantic scene completion results on the SemanticKITTI validation set}~\cite{behley2019semantickitti}\textbf{.}}
    % \vspace{-2mm}
    \newcommand{\classfreq}[1]{{~\tiny(\semkitfreq{#1}\%)}}  %
    \centering
    \begin{tabular}{l|c|c|c c c c c c c c c c c c c c c c c c c|c}
        \toprule
        % & & SC & \multicolumn{20}{c}{SSC} \\
        Method & SSC Input & IoU
        & \rotatebox{90}{\textcolor{road_s}{$\blacksquare$} \makecell[l]{road \vspace{-3pt} \\ \classfreq{road}}} 
        & \rotatebox{90}{\textcolor{sidewalk_s}{$\blacksquare$} \makecell[l]{sidewalk \vspace{-3pt} \\ \classfreq{sidewalk}}}
        & \rotatebox{90}{\textcolor{parking_s}{$\blacksquare$} \makecell[l]{parking \vspace{-3pt} \\ \classfreq{parking}}} 
        & \rotatebox{90}{\textcolor{other-ground_s}{$\blacksquare$} \makecell[l]{other-ground \vspace{-3pt} \\ \classfreq{otherground}}} 
        & \rotatebox{90}{\textcolor{building_s}{$\blacksquare$} \makecell[l]{building \vspace{-3pt} \\ \classfreq{building}}} 
        & \rotatebox{90}{\textcolor{car_s}{$\blacksquare$} \makecell[l]{car \vspace{-3pt} \\ \classfreq{car}}} 
        & \rotatebox{90}{\textcolor{truck_s}{$\blacksquare$} \makecell[l]{truck \vspace{-3pt} \\ \classfreq{truck}}} 
        & \rotatebox{90}{\textcolor{bicycle_s}{$\blacksquare$} \makecell[l]{bicycle \vspace{-3pt} \\ \classfreq{bicycle}}} 
        & \rotatebox{90}{\textcolor{motorcycle_s}{$\blacksquare$} \makecell[l]{motorcycle \vspace{-3pt} \\ \classfreq{motorcycle}}} 
        & \rotatebox{90}{\textcolor{other-vehicle_s}{$\blacksquare$} \makecell[l]{other-vehicle \vspace{-3pt} \\ \classfreq{othervehicle}}} 
        & \rotatebox{90}{\textcolor{vegetation_s}{$\blacksquare$} \makecell[l]{vegetation \vspace{-3pt} \\ \classfreq{vegetation}}} 
        & \rotatebox{90}{\textcolor{trunk_s}{$\blacksquare$} \makecell[l]{trunk \vspace{-3pt} \\ \classfreq{trunk}}} 
        & \rotatebox{90}{\textcolor{terrain_s}{$\blacksquare$} \makecell[l]{terrain \vspace{-3pt} \\ \classfreq{terrain}}} 
        & \rotatebox{90}{\textcolor{person_s}{$\blacksquare$} \makecell[l]{person \vspace{-3pt} \\ \classfreq{person}}} 
        & \rotatebox{90}{\textcolor{bicyclist_s}{$\blacksquare$} \makecell[l]{bicyclist \vspace{-3pt} \\ \classfreq{bicyclist}}} 
        & \rotatebox{90}{\textcolor{motorcyclist_s}{$\blacksquare$} \makecell[l]{motorcyclist \vspace{-3pt} \\ \classfreq{motorcyclist}}} 
        & \rotatebox{90}{\textcolor{fence_s}{$\blacksquare$} \makecell[l]{fence \vspace{-3pt} \\ \classfreq{fence}}} 
        & \rotatebox{90}{\textcolor{pole_s}{$\blacksquare$} \makecell[l]{pole \vspace{-3pt} \\ \classfreq{pole}}} 
        & \rotatebox{90}{\textcolor{traffic-sign_s}{$\blacksquare$} \makecell[l]{traffic-sign \vspace{-3pt} \\ \classfreq{trafficsign}}} 
        & mIoU \\
        \midrule\midrule
        MonoScene~\cite{cao2022monoscene} & $x^{\text{rgb}}$ & 36.86 & 56.52 & 26.72 & 14.27 & 0.46 & 14.09 & 23.26 & 6.98 & 0.61 & 0.45 & 1.48 & 17.89 & 2.81 & 29.64 & 1.86 & 1.20 & 0.00 & 5.84 & 4.14 & 2.25 & 11.08 \\  
        OccFormer~\cite{zhang2023occformer} & $x^{\text{rgb}}$ & 36.63 & \textbf{59.45} & 28.10 & \textbf{21.44} & 0.33 & 11.27 & 15.09 & \textbf{25.42} & \textbf{9.91} & 2.21 & 1.52 & 19.40 & 3.53 & 31.99 & 3.50 & \textbf{3.87} & 0.00 & 5.96 & 4.03 & 2.52 & 13.13 \\ 
        VoxFormer~\cite{li2023voxformer} & $x^{\text{rgb}}$ & 44.42 & 57.20 & 28.68 & 13.66 & 0.36 & 19.12 & 27.37 & 5.22 & 0.40 & 0.53 & 4.12 & 25.83 & 6.24 & 33.29 & 1.11 & 1.58 & 0.00 & 7.66 & 7.53 & 4.46 & 12.86 \\  
        % \multirow{2}{*}{(ours)} & $x^{\text{event}}$,$x^{\text{rgb}}$ & 45.01 & 57.68 & 28.74 & 16.43 & 0.48 & 19.43 & 27.51 & 11.27 & 0.59 & 0.78 & 5.05 & 25.95 & 6.82 & 34.92 & 1.61 & 1.96 & 0.00 & 7.68 & 7.46 & 4.21 & 13.61 \\
        % & $x^{\text{event}}$,$x^{\text{rgb}}$ & - & - & - & - & - & - & - & - & - & - & - & - & - & - & - & - & - & - & - & - & - \\
        SGN~\cite{mei2024sgn} & $x^{\text{rgb}}$ & 43.60 & 59.32 & \textbf{30.51} & 18.46 & 0.42 & 21.43 & \textbf{31.88} & 13.18 & 0.58 & 0.17 & 5.68 & 25.98 & 7.43 & 34.42 & 1.28 & 1.49 & 0.00 & 9.66 & 9.83 & 4.71 & 14.55 \\
        Symphonies~\cite{jiang2024symphonize} & $x^{\text{rgb}}$ & 41.92 & 56.37 & 27.58 & 15.28 & \textbf{0.95} & 21.64 & 28.68 & 20.44 & 2.54 & \textbf{2.82} & \textbf{13.89} & 25.72 & 6.60 & 30.87 & \textbf{3.52} & 2.24 & 0.00 & 8.40 & 9.57 & 5.76 & 14.89 \\ \midrule
        \rowcolor{gray!20}EvSSC (VoxFormer) & $x^{\text{event}}$,$x^{\text{rgb}}$ & \textbf{45.01} & 57.68 & 28.74 & 16.43 & 0.48 & 19.43 & 27.51 & 11.27 & 0.59 & 0.78 & 5.05 & 25.95 & 6.82 & \textbf{34.92} & 1.61 & 1.96 & 0.00 & 7.68 & 7.46 & 4.21 & 13.61 \\
        \rowcolor{gray!20}EvSSC (SGN-S) & $x^{\text{event}}$,$x^{\text{rgb}}$ & 43.17 & 58.24 & 30.50 & 19.93 & 0.52 & \textbf{21.67} & 31.80 & 18.34 & 0.62 & 0.07 & 4.67 & \textbf{26.79} & \textbf{7.69} & 34.48 & 2.35 & 2.76 & 0.00 & \textbf{9.93} & \textbf{11.27} & \textbf{6.25}& \textbf{15.15} \\
        \bottomrule
    \end{tabular}\\
    \label{table:kitti_e_ssc}
    % \vspace{-3mm}
\end{table*}

% \begin{table*}[h]
%     \scriptsize
%     \setlength{\tabcolsep}{0.0035\linewidth}
%     \caption{\textbf{Semantic scene completion results on the SemanticKITTI validation set}~\cite{behley2019semantickitti}\textbf{.}}
%     \vspace{-2mm}
%     \newcommand{\classfreq}[1]{{~\tiny(\semkitfreq{#1}\%)}}  %
%     \centering
%     \begin{tabular}{l|c|c|c c c c c c c c c c c c c c c c c c c|c}
%         \toprule
%         & & SC & \multicolumn{20}{c}{SSC} \\
%         Method & SSC Input & IoU
%         & \rotatebox{90}{\textcolor{road_s}{$\blacksquare$} road\classfreq{road}} 
%         & \rotatebox{90}{\textcolor{sidewalk_s}{$\blacksquare$} sidewalk\classfreq{sidewalk}}
%         & \rotatebox{90}{\textcolor{parking_s}{$\blacksquare$} parking\classfreq{parking}} 
%         & \rotatebox{90}{\textcolor{other-ground_s}{$\blacksquare$} other-ground\classfreq{otherground}} 
%         & \rotatebox{90}{\textcolor{building_s}{$\blacksquare$} building\classfreq{building}} 
%         & \rotatebox{90}{\textcolor{car_s}{$\blacksquare$} car\classfreq{car}} 
%         & \rotatebox{90}{\textcolor{truck_s}{$\blacksquare$} truck\classfreq{truck}} 
%         & \rotatebox{90}{\textcolor{bicycle_s}{$\blacksquare$} bicycle\classfreq{bicycle}} 
%         & \rotatebox{90}{\textcolor{motorcycle_s}{$\blacksquare$} motorcycle\classfreq{motorcycle}} 
%         & \rotatebox{90}{\textcolor{other-vehicle_s}{$\blacksquare$} other-vehicle\classfreq{othervehicle}} 
%         & \rotatebox{90}{\textcolor{vegetation_s}{$\blacksquare$} vegetation\classfreq{vegetation}} 
%         & \rotatebox{90}{\textcolor{trunk_s}{$\blacksquare$} trunk\classfreq{trunk}} 
%         & \rotatebox{90}{\textcolor{terrain_s}{$\blacksquare$} terrain\classfreq{terrain}} 
%         & \rotatebox{90}{\textcolor{person_s}{$\blacksquare$} person\classfreq{person}} 
%         & \rotatebox{90}{\textcolor{bicyclist_s}{$\blacksquare$} bicyclist\classfreq{bicyclist}} 
%         & \rotatebox{90}{\textcolor{motorcyclist_s}{$\blacksquare$} motorcyclist\classfreq{motorcyclist}} 
%         & \rotatebox{90}{\textcolor{fence_s}{$\blacksquare$} fence\classfreq{fence}} 
%         & \rotatebox{90}{\textcolor{pole_s}{$\blacksquare$} pole\classfreq{pole}} 
%         & \rotatebox{90}{\textcolor{traffic-sign_s}{$\blacksquare$} traffic-sign\classfreq{trafficsign}} 
%         & mIoU \\
%         \midrule
%         MonoScene~\cite{cao2022monoscene} & $x^{\text{rgb}}$ & 37.12 & \textcolor{blue}{57.47} & 27.05 & 15.72 & \textcolor{red}{0.87} & 14.24 & 23.55 & 7.83 & 0.20 & 0.77 & 3.59 & \textcolor{red}{30.76} & 2.57 & 30.76 & 1.03 & 0.00 & 0.00 & 6.39 & 4.11 & 2.48 & 11.50 \\  
%         OccFormer~\cite{zhang2023occformer} & $x^{\text{rgb}}$ & 36.63 & \textcolor{red}{59.45} & 28.10 & \textcolor{red}{21.44} & 0.33 & 11.27 & 15.09 & \textcolor{red}{25.42} & \textcolor{red}{9.91} & 2.21 & 1.52 & 19.40 & 3.53 & 31.99 & 3.50 & \textcolor{red}{3.87} & 0.00 & 5.96 & 4.03 & 2.52 & 13.13 \\ 
%         VoxFormer~\cite{li2023voxformer} & $x^{\text{rgb}}$ & \textcolor{yellow}{44.42} & 57.20 & 28.68 & 13.66 & 0.36 & 19.12 & 27.37 & 5.22 & 0.40 & 0.53 & 4.12 & 25.83 & 6.24 & 33.29 & 1.11 & 1.58 & 0.00 & 7.66 & 7.53 & 4.46 & 12.86 \\  
%         SGN~\cite{mei2024sgn} & $x^{\text{rgb}}$ & \textcolor{blue}{43.60} & \textcolor{yellow}{59.32} & \textcolor{red}{30.51} & \textcolor{yellow}{18.46} & 0.42 & \textcolor{yellow}{21.43} & \textcolor{yellow}{31.88} & \textcolor{yellow}{13.18} & 0.58 & 0.17 & \textcolor{yellow}{5.68} & 25.98 & \textcolor{red}{7.43} & \textcolor{yellow}{34.42} & 1.28 & 1.49 & 0.00 & \textcolor{yellow}{9.66} & \textcolor{yellow}{9.83} & \textcolor{yellow}{4.71} & \textcolor{blue}{14.55} \\
%         Symphonies~\cite{jiang2024symphonize} & $x^{\text{rgb}}$ & 41.92 & 56.37 & 27.58 & 15.28 & \textcolor{blue}{0.95} & \textcolor{red}{21.64} & 28.68 & \textcolor{blue}{14.75} & \textcolor{blue}{2.54} & \textcolor{yellow}{2.82} & \textcolor{red}{13.89} & 25.72 & \textcolor{yellow}{6.60} & 30.87 & \textcolor{red}{3.52} & \textcolor{yellow}{2.24} & 0.00 & 8.40 & 9.57 & \textcolor{red}{5.76} & \textcolor{red}{14.89} \\ \hline
%         \rowcolor{gray!20}EvSSC (VoxFormer) & $x^{\text{event}}$,$x^{\text{rgb}}$ & \textcolor{red}{45.01} & 57.68 & 28.74 & \textcolor{yellow}{16.43} & 0.48 & 19.43 & 27.51 & 11.27 & \textcolor{yellow}{0.59} & 0.78 & 5.05 & 25.95 & 6.82 & \textcolor{red}{34.92} & 1.61 & 1.96 & 0.00 & 7.68 & 7.46 & 4.21 & 13.61 \\
%         \rowcolor{gray!20}EvSSC (SGN-S) & $x^{\text{event}}$,$x^{\text{rgb}}$ & 43.17 & \textcolor{blue}{58.24} & \textcolor{blue}{30.05} & \textcolor{blue}{19.93} & \textcolor{yellow}{0.52} & 21.67 & \textcolor{blue}{31.80} & \textcolor{red}{18.34} & \textcolor{yellow}{0.62} & 0.07 & 4.67 & \textcolor{yellow}{26.79} & \textcolor{blue}{7.69} & 34.48 & \textcolor{blue}{2.35} & \textcolor{blue}{2.76} & 0.00 & \textcolor{red}{9.93} & \textcolor{red}{11.27} & \textcolor{blue}{6.25} & \textcolor{blue}{15.15} \\
%         \bottomrule
%     \end{tabular}\\
%     \label{table:kitti_e_ssc}
%     \vspace{-3mm}
% \end{table*}




\section{Experiment}
\label{sec:exp}
%
\begin{figure*}[t]
\centering
\includegraphics[width=1\linewidth]
{figures/DSEC-SSCvisual.pdf}
% \vskip-2ex
\caption{\textbf{Qualitative results of EvSSC and baseline on DSEC-SSC.} EvSSC better captures scene layouts in low-light scenes.}
\label{fig:dsecquality}
% \vskip-2ex
\end{figure*}


\subsection{Datasets}
\noindent\textbf{DSEC}, introduced by Gehrig~\textit{et al.}~\cite{gehrig2021dsec}, consists of $24$ sequences captured in real-world outdoor driving scenes using event cameras, RGB cameras, and LiDAR. 
The dataset includes $7,800$ training samples and $2,100$ testing samples, captured at a size of $640{\times}480$. It covers both daytime and nighttime scenarios, with small and large motions. As the dataset does not provide official occupancy ground truth, we generated semantic occupancy labels by utilizing the released 2D semantic segmentation annotations and depth data, followed by manual refinement, resulting in the derived DSEC-SSC dataset (see Sec.~\ref{sec:benchmark} for details). 
LiDAR scans were voxelized into a $128{\times}128{\times}16$ grid with voxel sizes of $0.4m$, labeled into $14$ common categories (see Tab.~\ref{table:dsec_ssc}), consistent with semantic labels provided by DSEC.

\noindent \textbf{SemanticKITTI}~\cite{behley2019semantickitti} consists of outdoor LiDAR scans voxelized into a $256{\times}256{\times}32$ grid with $0.2m$ voxels, labeled into $21$ classes ($19$ \emph{semantic}, $1$ \emph{free}, $1$ \emph{unknown}). 
We use RGB images of size $370{\times}1,220$. We adopt the official $3834/815$ train/val splits and consistently evaluate at full scale (\textit{i.e.}, $1{:}1$). 
To assess the impact of the event modality on occupancy prediction for this dataset, we generate event sequences for SemanticKITTI using the official DVS-Voltmeter codebase~\cite{lin2022dvs}, resulting in the derived SemanticKITTI-E dataset. 
Additionally, to evaluate SSC's robustness against camera corruption, we simulate five common real-world corruptions including \emph{motion blur}, \emph{fog}, \emph{brightness}, \emph{darkness}, and \emph{shot noise}.

\subsection{Quantitative Comparison}
\noindent \textbf{Analyses on the DSEC-SSC benchmark.} 
We first benchmark popular camera occupancy prediction methods~\cite{cao2022monoscene,zhang2023occformer,li2023voxformer,mei2024sgn} on the newly proposed DSEC-SSC dataset. 
For a fair comparison, we report training results using both 2D images ($x^{rgb}$) and event data ($x^{event}$) as inputs. 
Note that we did not modify the baseline model structures. 
As a framework approach, EvSSC integrates both image and event modalities, making it applicable to various baselines. 
As shown in Tab.~\ref{table:dsec_ssc}, using only the event modality decreases mIoU and IoU across all camera methods. 
For example, the event modality version of MonoScene~\cite{cao2022monoscene} shows a drop in mIoU compared to the RGB modality version by $1.86$ ($15.86$ \textit{vs.} $17.72$), which is expected as the event modality lacks the detailed textures of RGB. 
EvSSC proves effective across different RGB SSC baselines. 
EvSSC (VoxFormer) improves mIoU by $0.72$ compared to RGB-only VoxFormer ($26.34$ \textit{vs.} $25.62$) and by $10.26$ compared to its event modality counterpart ($26.34$ \textit{vs.} $16.08$). Similarly, EvSSC (SGN) boots mIoU by $0.31$ compared to its RGB baseline ($29.37$ \textit{vs.} $29.06$) and by $6.84$ compared to its event modality counterpart ($29.37$ \textit{vs.} $22.53$).
%

\noindent \textbf{Analyses on the SemanticKITTI-E benchmark.}
%
As shown in Tab.~\ref{table:kitti_e_ssc}, we compare EvSSC, which fuses event and RGB modalities, with RGB-only occupancy prediction methods on SemanticKITTI-E. 
Despite the simulated nature of event data on this dataset, EvSSC effectively shows the complementary strengths of event and RGB modalities. 
EvSSC (VoxFormer) improves mIoU by $5.8\%$ over VoxFormer~\cite{li2023voxformer} ($13.61$ \textit{vs.} $12.86$). 
Using SGN-S~\cite{mei2024sgn} as the baseline, EvSSC (SGN) achieves the highest accuracy among visual occupancy models, reaching mIoU of $15.15$, surpassing 
%
Symphonies~\cite{jiang2024symphonize} ($15.15$ \textit{vs.} $14.89$). 
This further demonstrates the effectiveness of incorporating event modality to aid RGB in occupancy prediction, which comes with only a slight increase in terms of latency.
Evaluated on a single NVIDIA RTX 3090 GPU, the inference latencies of VoxFormer and EvSSC are $0.996s$ and $1.005s$ per frame, respectively.

%

\subsection{Enhancing Robustness with Event Modality}
%

Traditional visual occupancy models are highly susceptible to image degradation challenges, such as low-light conditions and motion blur. 
As shown in Tab.~\ref{table:dsec_ssc}, DSEC-SSC includes numerous low-light scenes, where we quantitatively demonstrate the robustness that the event modality brings to occupancy prediction in real-world settings. 
To further evaluate the impact of event modality under various degradation conditions, we generated the SemanticKITTI-C dataset, featuring five common degradation scenarios: \emph{motion blur}, \emph{fog}, \emph{brightness}, \emph{darkness}, and \emph{shot noise} from image sensors. 
Specifically, we consider two experimental setups, reporting occupancy mIoU for each.

\begin{table}[t!]
\centering
    \scriptsize
    \setlength{\tabcolsep}{0.0035\linewidth}
    \caption{\textbf{Semantic scene completion results on the corrupted SemanticKITTI-C dataset.}}
    % \vskip-2ex
\setlength{\tabcolsep}{4pt} %5pt 设定列之间的宽度
\resizebox{\columnwidth}{!}{%
\begin{tabular}{l|>{\columncolor{gray!10}}l>{\columncolor{blue!8}}l|>{\columncolor{gray!10}}l>{\columncolor{blue!8}}l}
\toprule
\textbf{Corruption} & \textbf{VoxFormer-S} & \textbf{EvSSC (VoxFormer)} & \textbf{SGN-S} & \textbf{EvSSC (SGN)}\\
\midrule\midrule
\multicolumn{5}{c}{\textit{Out-of-domain}} \\ \midrule 
\textbf{Motion Blur} & 8.54 & 9.12~\textcolor{blue}{ (+0.58)} & 8.25 & 8.91~\textcolor{blue}{ (+0.66)}\\ 
\textbf{Fog} & 8.28 & 9.13~\textcolor{blue}{ (+0.85)} & 8.35 & 8.90~\textcolor{blue}{ (+0.55)} \\ 
\textbf{Brightness} & 9.78 & 10.75~\textcolor{blue}{ (+0.97)} & 10.00 & 11.05~\textcolor{blue}{ (+1.05)} \\ 
\textbf{Darkness} & 9.53 & 9.87~\textcolor{blue}{ (+0.34)} & 9.16 & 9.78~\textcolor{blue}{ (+0.62)} \\
\textbf{Shot noise} & 8.10 & 8.83~\textcolor{blue}{ (+0.73)} & 5.87 & 6.07~\textcolor{blue}{ (+0.20)} \\
\midrule
\multicolumn{5}{c}{\textit{In-domain}} \\ \midrule 
\textbf{Motion Blur} & 11.57 & 12.35~\textcolor{blue}{ (+0.78)} & 13.02 & 13.55~\textcolor{blue}{ (+0.53)} \\ 
\textbf{Fog} & 12.38 & 13.13~\textcolor{blue}{ (+0.75)} & 14.23 & 14.55~\textcolor{blue}{ (+0.32)} \\ 
\textbf{Brightness} & 12.33 & 13.21~\textcolor{blue}{ (+0.88)} & 14.08 & 14.17~\textcolor{blue}{ (+0.09)} \\ 
\textbf{Darkness} & 11.48 & 11.84~\textcolor{blue}{ (+0.36)} & 13.16 & 13.29~\textcolor{blue}{ (+0.13)} \\
\textbf{Shot noise} & 8.29 & 12.64~\textcolor{blue}{ (+4.35)} & 13.62 & 14.32~\textcolor{blue}{ (+0.70)} \\
\bottomrule
\end{tabular}
%
}
\label{table:kitti_c_ssc}
% \vskip-3ex
\end{table}



1) \textbf{Out-Of-Domain (OOD):} 
In this setting, models are trained on clean data and tested directly on degraded scenes without prior exposure to degradation. 
As shown in Tab.~\ref{table:kitti_c_ssc}, both VoxFormer~\cite{li2023voxformer} and SGN~\cite{mei2024sgn} show large performance drops across all degradation types, especially in motion blur ($12.86{\rightarrow}8.54$), darkness ($12.86{\rightarrow}9.53$), and shot noise ($12.86{\rightarrow}8.10$). 
Although EvSSC is also trained only on clean data, the inclusion of the event modality effectively enhances the robustness of both baselines across all degradations. 
For example, with motion blur, EvSSC improves mIoU by $6.8\%$ ($9.12$ \textit{vs.} $8.54$) and $8.0\%$ ($8.91$ \textit{vs.} $8.25$) over the baselines, demonstrating the value of temporal cues in events; for darkness, EvSSC improves by $3.6\%$ ($9.87$ \textit{vs.} $9.53$) and $6.8\%$ ($9.78$ \textit{vs.} $9.16$), showing complementary strengths of the high dynamic range in event sensors.
\begin{table}[t!]
    \scriptsize
    \centering
    \setlength{\tabcolsep}{0.0035\linewidth}
    \caption{\textbf{Ablation studies on fusion paradigms, RGB-Event fusion strategies, and event representation design on the SemanticKITTI-E dataset.}}
    \vskip-2ex
    \label{table:four_grid}
    \begin{subtable}[t]{0.48\linewidth}
        \caption{Ablations on fusion paradigms in EvSSC (VoxFormer-S).}
        \centering
        \begin{tabular}{p{2.5cm}|c|c}
            \toprule
            \rowcolor{gray!20}
            Aggregation & IoU & mIoU \\
            \midrule
            \midrule
            \textit{w/o} & 44.42 & 12.86 \\ \midrule
            Fusion then lifting & 45.05 & 12.86 \\
            \rowcolor{blue!8}\textbf{Fusion-based lifting} & 44.45 & \textbf{13.10}\\
            Decode then lifting & \textbf{45.54} & 12.68 \\  
            \bottomrule
        \end{tabular}
        \label{table:voxformer_add}
        % \vskip-2ex
    \end{subtable}%
    \hfill
    \begin{subtable}[t]{0.48\linewidth}
        \caption{Ablations on event representation in EvSSC (VoxFormer-S). }
        \centering
        \begin{tabular}{p{2.5cm}|c|c}
            \toprule
            \rowcolor{gray!20}
            Representation & IoU
            & mIoU \\
            \midrule
            \midrule
            Timesurface~\cite{delbruck2008frame} & 44.92 & 13.49 \\  
            2D event frame & \textbf{45.09}  & 13.40 \\  
            HATS~\cite{vonmarcard2018recovering}& 44.91 & 13.44 \\          
            % EST & - & - \\  
            \rowcolor{blue!8}\textbf{Rasterized event} & 45.01 & \textbf{13.61}   \\
            \bottomrule
        \end{tabular}
        \label{table:voxformer_event_representation}
        %\vskip-2ex
    \end{subtable}%

    \vspace{1mm} % 在行之间留出垂直间距

    \begin{subtable}[t]{0.48\linewidth}
        \caption{Ablations on RGB-Event fusion in EvSSC (VoxFormer-S).}
        \centering
        \begin{tabular}{l|c|c}
            \toprule
            \rowcolor{gray!20}
            Aggregation & IoU & mIoU \\
            \midrule
            \midrule
            \textit{w/o} & 44.42 & 12.86 \\ \midrule
            Add & 44.45 & 13.10 \\
            Concat & 44.51 & 13.08 \\
            CBAM~\cite{woo2018cbam} & 44.62 & 11.75 \\
            CMX (RGB-X)~\cite{zhang2023cmx} &44.41 & 12.58\\
            EFNet (RGB-Event)~\cite{sun2022event_deblurring} & 44.30 & 12.13 \\
            \rowcolor{blue!8}\textbf{ELM (ours)} & \textbf{45.01} & \textbf{13.61} \\ 
            \bottomrule
        \end{tabular}
        \label{table:voxformer_fbl}
    \end{subtable}%
    \hfill
    \begin{subtable}[t]{0.48\linewidth}
        \caption{Ablations on RGB-Event fusion in EvSSC (SGN-S).}
        \centering
        \begin{tabular}{l|c|c}
            \toprule
            \rowcolor{gray!20}
            Aggregation & IoU & mIoU \\
            \midrule
            \midrule
            \textit{w/o} & \textbf{44.60} & 14.55 \\ \midrule
            Add & 43.80 & 14.58 \\
            Concat & 43.41 & 14.66 \\ 
            CBAM~\cite{woo2018cbam} & 43.27 & 13.99 \\
            CMX (RGB-X)~\cite{zhang2023cmx} & 38.84 &11.52 \\
            EFNet (RGB-Event)~\cite{sun2022event_deblurring} & 43.63 & 14.64 \\
            \rowcolor{blue!8}\textbf{ELM (ours)} & 43.17 & \textbf{15.15} \\
            \bottomrule
        \end{tabular}
        \label{table:sgn_fbl}
    \end{subtable}%

\label{tab:ablations}

\end{table}



2) \textbf{In-Domain (ID):} Here, models are pre-trained on degraded data. 
As shown in Tab.~\ref{table:kitti_c_ssc}, while overall occupancy prediction improves under ID training compared to OOD, accuracy remains lower than on clean data. 
Including events significantly enhances the in-domain performance of both baselines across all degradations. 
%
Especially in shot noise, EvSSC further improves by $52.5\%$ ($12.64$ \textit{vs.} $8.29$) and $5.1\%$ ($14.32$ \textit{vs.} $13.62$), confirming the robustness of event-RGB fusion when the image sensor partially fails.
%

%
\begin{figure*}[t!]
\centering
%\vskip-2ex
\begin{minipage}{1\linewidth}
    \centering
    \includegraphics[width=1\linewidth]{figures/KITTI_visualization1.pdf}
    %\vskip-2ex
\end{minipage}

%\vskip-0.3ex

\begin{minipage}{1\linewidth}
    \centering
    \includegraphics[width=1\linewidth]{figures/KITTI_visualization2.pdf}
    %\vskip-2ex
\end{minipage}
%\vskip-1ex
\caption{\textbf{Qualitative results of EvSSC and baseline on SemanticKITTI-C.} More visual results of 3D occupancy on SemanticKITTI-C, showing predictions from VoxFormer-S~\cite{li2023voxformer} and EvSSC (VoxFormer). The first four rows correspond to the degradation mode of fog, while the last four rows correspond to the degradation mode of motion blur.}
\label{fig:corruption_v}
%\vskip-4ex
\end{figure*}
\subsection{Qualitative Comparison}
\noindent \textbf{Analyses on the DSEC-SSC benchmark.}
Based on the visualization results of EvSSC and the baseline SGN~\cite{mei2024sgn} on the DSEC-SSC validation set, as shown in Fig.~\ref{fig:dsecquality}, it can be observed that SGN often fails to fully reconstruct or detect traffic signs and cars in low-light conditions. 
In contrast, with the assistance of event data, EvSSC is capable of achieving complete reconstruction. Notably, in the fifth column, for vehicles with incomplete appearances in the image that SGN fails to detect, the integration of event data enables successful detection. 
Furthermore, as seen in the first and sixth columns, the incorporation of event data significantly improves the detection of lane markings and facilitates the reconstruction of detailed scene elements at the traffic intersections on low-contrast road surfaces.

\noindent \textbf{Analyses on the SemanticKITTI-C benchmark.}
As illustrated in Fig.~\ref{fig:corruption_v}, we visualize the prediction results of EvSSC and the baseline in scenarios with motion blur. It can be observed that due to the loss of texture information in the image, VoxFormer produces a significant number of mispredicted voxels. In contrast, with the assistance of event data, EvSSC achieves more accurate predictions, resulting in a cleaner and more refined visualization.


\subsection{Ablation Studies}
%

%
As shown in Tab.~\ref{tab:ablations}, we conduct ablations to validate design choices of EvSSC, reporting occupancy IoU and mIoU for various configurations.
(a) We examine the impact of three event-image fusion paradigms for 2D-to-3D occupancy models. 
To avoid bias introduced by complex fusion modules, we use simple additive fusion. 
Results show ``Fusion-Based Lifting'', which introduces event modality during the 2D-to-3D lifting process, provides the most significant mIoU gain ($13.10$ \textit{vs.} $12.86$).
Thus, EvSSC integrates the ELM module into the lifting stage.
(b) We explore the effects of different event representations on RGB-Event occupancy prediction.
The rasterized event representation used in this work encodes multiple statistics across channels, including counts and timestamps for positive and negative events. 
This approach effectively captures motion while aligning with RGB guidance, resulting in the highest mIoU gain over naive 2D event frames ($13.61$ \textit{vs.} $13.40$) and several representation methods~\cite{delbruck2008frame,vonmarcard2018recovering}.
(c) Building on (a), we investigate the impact of various attentional and RGB-X fusion methods~\cite{woo2018cbam,zhang2023cmx,sun2022event_deblurring} within a transformer-based lifting paradigm using VoxFormer~\cite{li2023voxformer}. 
Results show the dual-branch self-attention design in ELM provides the largest boost compared to simple addition ($13.61$ \textit{vs.} $13.10$).
(d) To assess the generalizability of ELM across architectures, we test ELM on LSS-style~\cite{philion2020lift} SGN~\cite{mei2024sgn}. 
Results confirm the effectiveness of dual-branch self-attention in this setup as well, with mIoU boosting from $14.58$ to $15.15$.

\subsection{Efficiency Analysis}
We further perform efficiency evaluations on the DSEC-SSC, SemanticKITTI-E, and SemanticKITTI-C datasets using a single NVIDIA RTX 3090 GPU. 
As shown in Tab.~\ref{table:efficiency}, our EvSSC consistently achieves great accuracy gains on all datasets and a remarkable $52.5\%$ improvement of mIoU in degraded scenarios while requiring only an additional $0.91$GB of GPU memory and a $0.9\%$ increase in latency.
Furthermore, on the non-degraded SemanticKITTI-E dataset, a mere millisecond-level latency is required to achieve a mIoU improvement of $0.75$. Efficiency analysis shows that minimal additional deployment costs can significantly improve prediction robustness, enhancing autonomous driving safety and practical application.

\setlength{\tabcolsep}{3pt}
\begin{table}
\centering
\caption{Efficiency of Different Methods}
\vspace{-0.1in}
\label{tab:efficiency}
% \resizebox{\textwidth}{!}{%
\scalebox{0.65}{%
\begin{tabular}{|c|ccc|ccc|ccc|}
  \hline
  \multirow{2}{*}{} & \multicolumn{3}{c|}{\textbf{memory size (MB)}} & \multicolumn{3}{c|}{\textbf{training time (s)}} & \multicolumn{3}{c|}{\textbf{matching time (s)}}\\
  % \cline{2-7}
  % {} & MByte & seconds/ep & seconds\\
  {} & \textbf{Beijing} & \textbf{Porto} & \textbf{Chengdu} & \textbf{Beijing} & \textbf{Porto} & \textbf{Chengdu} & \textbf{Beijing} & \textbf{Porto} & \textbf{Chengdu}\\
  % \cline{2-7}
  % {} & (MByte) & (minutes/ep) & (seconds/K) & (MByte) & (minutes/ep) & (seconds/K)\\
  \hline
  \textbf{MDP} & 1819MB & 2039MB & 2122MB & - & - & - & 389.14s & 361.15s & 599.51s  \\
  \textbf{HMM} & 1209MB & 1388MB & 1361MB & - & - & - & 427.97s & 380.05s & 589.08s \\
  % \hline
  \textbf{FMM} & 897MB & 931MB & 981MB & - & - & - & 1.13s & 1.02s & 1.87s \\
  % \hline
  \textbf{AMM} & 957MB & 1013MB & 1124MB & - & - & - & 3.42s & 3.05s & 5.16s \\
  % \hline
  \textbf{MTrajRec} & 9045MB & 12428MB & 11265MB & 182.4s & 2200.2s & 25672.4s & 51.22s & 42.27s & 73.68s\\
  % \hline
  \textbf{L2MM} & 9087MB & 11875MB & 10898MB & 189.1s & 2314.2s & 27032.2s & 6.71s & 5.26s & 9.10s\\
  % \hline
  \textbf{GraphMM} & 8537MB & 11752MB & 10378MB & 48.4s & 645.2s & 7311.4s & 8.06s & 6.96s & 11.18s\\
  % \hline
  \textbf{\modelName} & 2530MB & 2299MB & 2357MB & 11.9s & 126.4s & 1507.8s & 1.09s & 0.95s & 1.65s\\
  \hline
\end{tabular}}
\vspace{-0.15in}
\end{table}


% An example of a double column floating figure using two subfigures.
% (The subfig.sty package must be loaded for this to work.)
% The subfigure \label commands are set within each subfloat command,
% and the \label for the overall figure must come after \caption.
% \hfil is used as a separator to get equal spacing.
% Watch out that the combined width of all the subfigures on a 
% line do not exceed the text width or a line break will occur.
%
%\begin{figure*}[!t]
%\centering
%\subfloat[Case I]{\includegraphics[width=2.5in]{box}%
%\label{fig_first_case}}
%\hfil
%\subfloat[Case II]{\includegraphics[width=2.5in]{box}%
%\label{fig_second_case}}
%\caption{Simulation results for the network.}
%\label{fig_sim}
%\end{figure*}
%
% Note that often IEEE papers with subfigures do not employ subfigure
% captions (using the optional argument to \subfloat[]), but instead will
% reference/describe all of them (a), (b), etc., within the main caption.
% Be aware that for subfig.sty to generate the (a), (b), etc., subfigure
% labels, the optional argument to \subfloat must be present. If a
% subcaption is not desired, just leave its contents blank,
% e.g., \subfloat[].


% An example of a floating table. Note that, for IEEE style tables, the
% \caption command should come BEFORE the table and, given that table
% captions serve much like titles, are usually capitalized except for words
% such as a, an, and, as, at, but, by, for, in, nor, of, on, or, the, to
% and up, which are usually not capitalized unless they are the first or
% last word of the caption. Table text will default to \footnotesize as
% the IEEE normally uses this smaller font for tables.
% The \label must come after \caption as always.
%
%\begin{table}[!t]
%% increase table row spacing, adjust to taste
%\renewcommand{\arraystretch}{1.3}
% if using array.sty, it might be a good idea to tweak the value of
% \extrarowheight as needed to properly center the text within the cells
%\caption{An Example of a Table}
%\label{table_example}
%\centering
%% Some packages, such as MDW tools, offer better commands for making tables
%% than the plain LaTeX2e tabular which is used here.
%\begin{tabular}{|c||c|}
%\hline
%One & Two\\
%\hline
%Three & Four\\
%\hline
%\end{tabular}
%\end{table}


% Note that the IEEE does not put floats in the very first column
% - or typically anywhere on the first page for that matter. Also,
% in-text middle ("here") positioning is typically not used, but it
% is allowed and encouraged for Computer Society conferences (but
% not Computer Society journals). Most IEEE journals/conferences use
% top floats exclusively. 
% Note that, LaTeX2e, unlike IEEE journals/conferences, places
% footnotes above bottom floats. This can be corrected via the
% \fnbelowfloat command of the stfloats package.




\section*{Conclusion}
This paper aims to enhance our understanding of the computational complexity of computing various Shapley value variants. We found that for various ML models --- including decision trees, regression tree ensembles, weighted automata, and linear regression --- both local and global interventional and baseline SHAP can be computed in polynomial time under HMM modeled distributions. This extends popular algorithms, such as TreeSHAP, beyond their empirical distributional scope. We also establish strict complexity gaps between the various SHAP variants (baseline, interventional, and conditional) and prove the intractability of computing SHAP for tree ensembles and neural networks in simplified scenarios. Overall, we present SHAP as a versatile framework whose complexity depends on four key factors: \begin{inparaenum}[(i)] \item model type, \item SHAP variant, \item distribution modeling approach, \item and local vs. global explanations\end{inparaenum}. We believe this perspective provides deeper insight into the computational complexity of SHAP, paving the way for future work.




%We believe that our framework provides a more intricate understanding of SHAP computation complexity across different models, distributions, and variants, paving the way for further research.

Our work opens promising directions for future research. First, expanding our computational analysis to other SHAP-related metrics, such as asymmetric SHAP~\citep{frye20} and SAGE~\citep{covert2020understanding}, would be valuable. Additionally, we aim to explore more expressive distribution classes and relaxed assumptions beyond those in Section \ref{sec:tractable} while maintaining tractable SHAP computation. Finally, when exact computation is intractable (Section \ref{sec:intractable}), investigating the approximability of SHAP metrics through approximation and parameterized complexity theory~\citep{downey2012parameterized} is an important direction.

%Our work opens several promising avenues for future research on the computational properties of explainable AI methods, with a particular focus on SHAP. First, it would be interesting to broaden the computational analysis conducted in this work to include other popular SHAP-related metrics in the literature, such as asymmetric SHAP \cite{frye20} and SAGE \cite{covert2020understanding}. Also, in the future, we aim to explore more expressive distribution classes and relaxed distributional assumptions—extending beyond those examined in Section \ref{sec:tractable} —that still yield tractable SHAP computation. Finally, when exact computation proves intractable (Section \ref{sec:intractable}), it is worthwhile to theoretically investigate the question of the approximability of computing the SHAP metrics across various configurations, through the lens of approximation and parametrized complexity theory \cite{arora2009computational}.

%This paper aims to deepen our understanding of the computational complexity involved in obtaining different Shapley value variants. We found that for a variety of ML models, including decision trees, tree ensembles for regression, weighted automata, and linear regression models — computing both local and global interventional and baseline SHAP can be done in polynomial time when distributions are modeled by HMMs. This extends the distributional scope of popular algorithms like TreeSHAP, which is limited to empirical distributions. Additionally, we demonstrate a strict complexity gap between SHAP variants, showing that interventional and baseline SHAP can be strictly easier to compute than conditional SHAP. Despite these positive results, we uncovered intractability for various SHAP variants in neural networks and tree ensembles. Finally, we provided generalized complexity relations across SHAP variants. We believe that our framework offers a deeper understanding of the complexity involved in computing SHAP across various variants, models, distributions, as well as in both local and global computations, laying the groundwork for future research.

{\small
\bibliographystyle{IEEEtran}
\bibliography{bib}
}



% that's all folks
\end{document}


