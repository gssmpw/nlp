\section{Conclusion}

In this study, we have presented \Name{}, a novel hierarchical adaptive decoding system designed to enhance the efficiency and elasticity of edge computing for uplink decoding in vRANs. \Name{} capitalizes on the hierarchical nature of vRAN architecture, optimizing the use of MH bandwidth, and addressing the challenges posed by vRAN edge computing capacity constraints.

By innovatively splitting uplink decoding tasks into early and completion decoding phases, \Name{} adapts to the dynamic environment of edge computing and MH bandwidth resources. Our system’s flexibility in scheduling and offloading decisions has demonstrated significant improvements in computational efficiency and latency management. Through comprehensive evaluation, we have shown that \Name{} can maintain high decoding throughput and manage latency effectively even under stringent edge resource limitations.


The key insights from our evaluation indicate that \Name{} has virtually no decoding capacity decrease
with up to half of the edge decoding capacity reduction and \Name{} can scale up to 75\% of the maximum throughput capacity through effective offloading strategies. This is particularly notable when compared to existing systems like Nuberu and BaseLine, which do not support workload migration and thus, are bound by the available CPU cores at the edge. Furthermore, \Name{} proves its robustness in handling mixed traffic scenarios with varying delay budgets, showcasing its ability to prioritize tasks based on the stringent latency requirements of 5G networks.


%\section{Discussion}
%HARQ deadlines
%encoding 
%accelerator
%TDD