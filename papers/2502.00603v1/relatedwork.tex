\section{Related work}
\begin{table*}
  %\small
  \caption{\coloredtext{red}{Comparison of related work in vRAN }\\
  }
  \label{tab:related}
  \scalebox{1}{
      \begin{tabular}{ccccl}
        \toprule
        %\vtop{\hbox{\strut Approaches}\hbox{\strut (device)}} 
        \vtop{\hbox{\strut Approaches}} 
        &\thead{Offloading}
        &\thead{HARQ prediction}
        &\thead{PreParsing}
        &\thead{Resource pooling}\\\\        
        %&\thead{HARQ prediction}\\      
        %&\vtop{\hbox{\strut HARQ prediction}\hbox{\strut (relative)}}
        %&\vtop{\hbox{\strut distance}\hbox{\strut /latency}}
        %&\vtop{\hbox{\strut band-}}\\
        \midrule
        \Name{}     &Yes  &Proactive(~\ref{early_decoding})      &High accuracy &Yes\\
        Nuberu    &No   &Passive        &No &Yes\\
        RT-opex   &No   &No             &No &Yes\\
        CloudIQ   &No   &No             &No &Yes\\
        \bottomrule
      \end{tabular}}
\end{table*}

The concept of Virtualized RAN (vRAN) has garnered significant interest in recent years~\cite{CiscoReimaging, IntelvRAN}. The role and impact of virtualization in RAN systems have been thoroughly explored~\cite{ORAN}. A primary challenge in deploying compute-intensive RAN/vRAN processing over cloud-based platforms is meeting the real-time requirements of the PHY layer, especially for computationally demanding tasks like FEC decoding. Various strategies have been developed to manage the allocation of compute resources for these real-time constraints in RAN/vRAN functions.

In the commercial realm, dedicated accelerators are often employed to support RAN functions, notably FEC decoding. Examples include Intel FlexRAN~\cite{flexran} and NVIDIA Aerial~\cite{Aerial}, both of which offer Layer-1 (L1) solutions in commercial vRANs. These systems utilize specialized hardware, such as FPGAs in FlexRAN and GPUs in Aerial, to boost performance. However, implementations details on these platforms are limited as they are not publicly available as open-source.

Academic research has also been active in exploring methods to enhance the efficiency of real-time RAN processing to minimize resource usage. For instance, CloudIQ~\cite{CloudIQ} adopts a scheduling approach that assigns a set of base stations to a computing platform based on their processing demands to fulfill real-time processing needs. RT-OPEX~\cite{RT-OPEX} dynamically redistributes parallel tasks, including decoding, to idle computing resources in real-time, optimizing the use of edge computing. More recently, vrAIn~\cite{vrAIn} introduced machine-learning-based controllers to manage the distribution of radio and compute resources across sliced RAN instances, although it struggles to maintain RAN function reliability during computing resource shortages.
\iffalse
These works typically treat the entirety of the decoding process as constrained by real-time requirements and do not leverage the potential benefits of additional decoding time afforded by decoding result predictions (or HARQ predictions). In this context, Nuberu~\cite{Nuberu} emerged as a work closely related to \Name{}, albeit focusing on LTE. Nuberu passively employs HARQ prediction to prevent deadline misses in LTE, further limiting spectrum assignments as a form of computing congestion control to reduce load during edge resource scarcity. However, its approach still depends entirely on edge resources for decoding processes, thus limiting its capacity for decoding. As comparison in table~\ref{tab:related}, \Name{} introduces an innovative decoding scheduling framework that actively utilizes HARQ prediction, pre-parsing for packet subheaders and intelligently manages resource allocation between edge and remote servers, enhancing overall system efficiency and capability. 
\fi 

These works typically treat the entire decoding process as constrained by real-time requirements, overlooking the potential benefits of additional decoding time afforded by decoding result predictions (HARQ predictions). Nuberu~\cite{Nuberu}, a closely related work focused on LTE, passively uses HARQ prediction to prevent deadline misses and limits spectrum assignments to reduce load during edge resource scarcity. However, it still relies entirely on edge resources for decoding, thus restricting its capacity.
\coloredtext{red}{
In contrast, as shown in Table~\ref{tab:related}, \Name{} introduces an innovative decoding scheduling framework that actively uses HARQ prediction and pre-parsing for packet subheaders. It intelligently manages resource allocation between edge and remote servers, enhancing overall system efficiency and capability.
}