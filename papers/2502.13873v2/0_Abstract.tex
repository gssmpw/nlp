\begin{abstract}
Deep Neural Networks are increasingly leveraging sparsity to reduce the scaling up of model parameter size. However, reducing wall-clock time through sparsity and pruning remains challenging due to irregular memory access patterns, leading to frequent cache misses. In this paper, we present NPU Vector Runahead (NVR), a prefetching mechanism tailored for NPUs to address cache miss problems in sparse DNN workloads. Rather than optimising memory patterns with high overhead and poor portability, NVR adapts runahead execution to the unique architecture of NPUs. NVR provides a general micro-architectural solution for sparse DNN workloads without requiring compiler or algorithmic support, operating as a decoupled, speculative, lightweight hardware sub-thread alongside the NPU, with minimal hardware overhead (under 5\%). NVR achieves an average 90\% reduction in cache misses compared to SOTA prefetching in general-purpose processors, delivering 4x average speedup on sparse workloads versus NPUs without prefetching. Moreover, we investigate the advantages of incorporating a small cache (16KB) into the NPU combined with NVR. Our evaluation shows that expanding this modest cache delivers 5x higher performance benefits than increasing the L2 cache size by the same amount.

% 当计算资源被缩放,内存往往However, these applications typically feature extremely low computation-to-communication ratio and irregular memory accesses, meaning their performance is memory bound, and out-of-order cores provide significant performance advantages over their in-order counterparts.
\end{abstract}
