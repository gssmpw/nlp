\documentclass[9pt,conference]{IEEEtran}
\IEEEoverridecommandlockouts

% \usepackage[bottom=1.5cm, top=1.3cm,left=1.65cm,right=1.65cm]{geometry}

% THESE ARE BLACK HACKS for PAPER MARGIN, IT USED TO BE FINE FOR PREVIOUS DAC 
% \usepackage[bottom=2.1cm, top=1.20cm,left=1.72cm,right=1.72cm]{geometry}
% \linespread{0.95}
%Sam: I swear the paper just grew though? It was 6.5 pages my last compile, now it's 6.66?
% Now, 7

\usepackage[bottom=2.1cm, top=1.20cm,left=1.72cm,right=1.72cm]{geometry}
\linespread{0.93}

\usepackage{tikz}
\usepackage{contour}

\usepackage{cite}
\usepackage{amsmath,amssymb,amsfonts}
\usepackage{algorithmic}
\usepackage{graphicx}
\usepackage{textcomp}
\usepackage{xcolor}
\usepackage{bbm}
\usepackage{tikz}
\usepackage{multirow}
\usepackage{colortbl}
\usepackage{hhline}
\usepackage{amsmath}
\usepackage{soul}
\usepackage{subfigure}
\usepackage{subcaption}   

\usepackage[table,xcdraw]{}
\usepackage{booktabs} % For formal tables
\usepackage{stfloats}
\usepackage[normalem]{ulem}
\usepackage{enumitem}
\usepackage{flushend}
\usepackage[font=small]{caption}
% \linespread{0.96} is fine for my all previous submissions 
% yeah fine, but where did we just lose half a page to?
% Yes,some page can be saving from the experiments, but not too much
% Something just happened to the font to make it bigger -- I can tell because experimental setup is now a word-over-a-line which I never would have allowed

% YES, I WORKED HARD TO MAKE each word and each line tightly today, but still too many.
% Sam, no, as in you must have just changed the font in the last few seconds. I've been removing all the one-word lines, and the paper was 6.5 pages and now it's bigger

% I used \usepackage[bottom=2.4cm, top=1.4cm,left=1.7cm,right=1.7cm]{geometry} to changed the paper margin

%IT is back, but it is a black hack of the paper margins
%Sam okay, what is expressly forbidden?
% Not sure, if no reviewer say it, it is fine. Dont play this too heavy
% OK do what you think we'll get away with, the rest we'll manage with cuts
% I think everything is very good in this paper, definitely, not figures. The reviewers like read figures.
% Also not algos, it is nice
% Yep, will have to be text. 
% The figures and algos are Symmetrical beauty :D

% If you can, go through and delete anything that feels like it's dragging even if it's somewhat important, I'll do the same and we'll see where we get 
% Ok, I think one thing we can do is merge the experimental setup into evaluation?
%done
%nice, half page left.

%One thing I need your help, I wrote a subsection called ``lessions we learnt'', but it is very naive. I wonder if you could help me write something very high-level but meaningful to the community? It will be interesting, do not delete it
%I'll give it some thought -- it just wants ending with a flourish
%Yes, that is!
%Right, end sentence written, and whole thing simplified
% It is not insightful enough, is there any other thing we learnt from this work?
%I think it's all we can say tbh. We don't want to claim that we've invented the concept of RTL research after all!
% do we need to say, we are encourage more frequent collaborations between modeling and impmenetation?
% That's the job of the firesim people, not us, surely
%ok, please think it ~ maybe something more insightful
% you can think deeper than me

\pagestyle{plain} % removes running headers
\pagenumbering{gobble}


\usepackage{fancyhdr}
\usepackage[normalem]{ulem}
\usepackage{url}
% \usepackage[final]{microtype}
\usepackage{array}
\usepackage{tikz}
\usepackage{makecell}
\usepackage{booktabs}

\usepackage{caption}
\usepackage{amsthm}
\usepackage{amsmath,amsfonts}
\usepackage{algorithmic}
\usepackage{graphicx}
\usepackage{textcomp}
\usepackage[table,xcdraw]{}
\usepackage{booktabs} % For formal tables
\usepackage{stfloats}
\usepackage[normalem]{ulem}
%\usepackage{changepage}
\usepackage{subfigure}
\usepackage{multirow}
\usepackage{paralist}
\usepackage{balance}
\usepackage{bm}
\usepackage[ruled,linesnumbered]{algorithm2e}
% Always include hyperref last
% \usepackage[bookmarks=true,breaklinks=true,colorlinks,citecolor=blue,linkcolor=blue,urlcolor=blue]{hyperref}
\usepackage{enumitem}
\usepackage{subcaption}
\usepackage{flushend}
\newcommand{\zhenote}[1]{\textbf{\textcolor{blue}{Zhe: #1}}}

\usepackage{url}
\usepackage[hidelinks]{hyperref}
\usepackage{marvosym}



% For nicer captions.
% \captionsetup[subfigure]{labelformat=simple,font=footnotesize}
\renewcommand\thesubfigure{\,(\alph{subfigure})}
\usepackage[ruled,linesnumbered]{algorithm2e}
\usepackage{cleveref}
% \linespread{0.96}
% Ensure letter paper
\pdfpagewidth=8.5in
\pdfpageheight=11in

\usepackage{colortbl}


%%%%%%%%%%%---SETME-----%%%%%%%%%%%%%
\newcommand\Archname{The Guardian Council}
\newcommand\archname{the Guardian Council}
\newcommand\marchname{FireGuard}
\newcommand\Marchname{FireGuard}
\newcommand\name{MEEK}

\newcommand{\note}[3]{\textbf{\textcolor{#1}{#2: #3}}}

\newcommand{\sam}[1]{\note{red}{Sam}{#1}}
\newcommand{\hugo}[1]{\note{cyan}{Hugo}{#1}}
\newcommand{\tim}[1]{\note{magenta}{Tim}{#1}}

\graphicspath{{./graphics/}}

\newcommand{\RomanNumeralCaps}[1]
    {\MakeUppercase{\romannumeral #1}}

\newtheorem{definition}{Definition}
\newtheorem{theorem}{Theorem}
\newtheorem{lemma}{Lemma}
\newtheorem{corollary}{Corollary}

\newcommand{\parlabel}[1]{{\noindent\bf #1}}
\newcommand{\eg}{e.g.\xspace}
\newcommand{\ie}{i.e.\xspace}
\newcommand{\etc}{etc.\xspace}
\newcommand{\etal}{et~al.\xspace}
\newcommand{\del}[1]{\textcolor{red}{\sout{#1}}}
\newcommand{\add}[1]{\textcolor{blue}{#1}} 

% Tim note: I'd really like to create a macro for $\mu$cores
% and use that instead because currently $\mu$ is slightly larger
% than the other characters.

\newcommand{\mionote}[1]{\textbf{\textcolor{red}{mio: #1}}}
\newcommand{\Nnote}[1]{\textbf{\textcolor{green}{Nick: #1}}}
\newcommand{\zzpnote}[1]{\textbf{\textcolor{orange}{zzp: #1}}}
\newcommand{\deannote}[1]{\textbf{\textcolor{pink}{dean: #1}}}

\newcommand{\Guan}[1]{\textbf{\textcolor{purple}{Guan: #1}}}

\newcommand{\ia}{\mbox{\sc IA}}
\newcommand{\iaFN}{{\mbox{\scriptsize\sc ia}}}

\newcommand{\w}{\mbox{\sc W}}
\newcommand{\wFN}{{\mbox{\scriptsize\sc w}}}

% From https://tex.stackexchange.com/a/7045
\newrobustcmd*\circled[1]{\tikz[baseline=(char.base)]{
            \node[shape=circle,draw,inner sep=1pt,fill,text=white,minimum size=1.2em] (char) {\textsf{\small #1}};}}
            


% Don't abbreviate figure
\crefname{figure}{figure}{figures}






\title{NVR: Vector Runahead on NPUs for Sparse Memory Access
\thanks{This paper has been accepted by DAC'2025.} 
\thanks{* \ Equal contribution.}
\thanks{\Letter \ Corresponding author.} 
\vspace{-32pt}
}

\author{
\IEEEauthorblockN{
Hui Wang\textsuperscript{1}\textsuperscript{*},
Zhengpeng Zhao\textsuperscript{4}\textsuperscript{*},
Jing Wang\textsuperscript{1}, 
Yushu Du\textsuperscript{1}, 
Yuan Cheng\textsuperscript{2,3}, 
Bing Guo\textsuperscript{1}, 
He Xiao\textsuperscript{5},\\
Chenhao Ma\textsuperscript{1}, 
Xiaomeng Han\textsuperscript{1}, 
Dean You\textsuperscript{1}, 
Jiapeng Guan\textsuperscript{6}, 
Ran Wei\textsuperscript{6}
Dawei Yang\textsuperscript{2}\textsuperscript{\textrm{\Letter}},
Zhe Jiang\textsuperscript{1}\textsuperscript{\textrm{\Letter}}}

\IEEEauthorblockA{
\textsuperscript{1}\textit{National Center of Technology Innovation for EDA, School of Integrated Circuits, Southeast University}
\textsuperscript{2}\textit{Houmo AI} \\
\textsuperscript{3}\textit{Nanjing University} 
\textsuperscript{4}\textit{Huazhong University of Science and Technology} 
\textsuperscript{5}\textit{Harbin Institute of Technology} 
\textsuperscript{6}\textit{Dalian University of Technology} \\
whmio0115@seu.edu.cn, 
u202114911@hust.edu.cn, 
yuancheng@smail.nju.edu.cn, 
dawei.yang@houmo.ai, 
zhejiang.uk@gmail.com}
\vspace{-28pt}
}



\begin{document}
    
\maketitle

\begin{abstract}

Hierarchical clustering is a powerful tool for exploratory data analysis, organizing data into a tree of clusterings from which a partition can be chosen. This paper generalizes these ideas by proving that, for any reasonable hierarchy, one can optimally solve any center-based clustering objective over it (such as $k$-means). Moreover, these solutions can be found exceedingly quickly and are \emph{themselves} necessarily hierarchical. 
%Thus, given a cluster tree, we show that one can quickly generate a myriad of \emph{new} hierarchies from it. 
Thus, given a cluster tree, we show that one can quickly access a plethora of new, equally meaningful hierarchies.
Just as in standard hierarchical clustering, one can then choose any desired partition from these new hierarchies. We conclude by verifying the utility of our proposed techniques across datasets, hierarchies, and partitioning schemes.


\end{abstract}



\section{Introduction}

% Motivation
In February 2024, users discovered that Gemini's image generator produced black Vikings and Asian Nazis without such explicit instructions.
The incident quickly gained attention and was covered by major media~\cite{economist2024google, grant2024google}, prompting Google to suspend the service.
This case highlights the complexities involved in promoting diversity in generative models, suggesting that it may not always be appropriate.
Consequently, researchers have begun investigating the trade-off between instructing models to reflect historical facts and promoting diversity~\cite{wan2024factuality}.
Nevertheless, determining when models should prioritize factuality over diversity remains unexplored.

\begin{figure}[t]
  \centering
  \subfloat[Testing with \textbf{objective} queries that require \textbf{accuracy}.]{
    \includegraphics[width=1.0\linewidth]{Figures/obj-cover.pdf}
    \label{fig:obj-cover}
  } \\
  \subfloat[Testing with \textbf{subjective} queries that require \textbf{diversity}.]{
    \includegraphics[width=1.0\linewidth]{Figures/subj-cover.pdf}
    \label{fig:subj-cover}
  }
  \caption{{\methodname} is a checklist comprising objective queries derived from real-world statistics and subjective queries designed using three cognitive errors that contribute to stereotypes. It includes queries designed for LLMs and T2I models.}
\end{figure}

% Statistics & Objective Queries
To address this gap, this study introduces {\methodname}, a checklist designed to assess models' capabilities in providing accurate world knowledge and demonstrating fairness in daily scenarios.
For world knowledge assessment, we collect 19 key statistics on U.S. economic, social, and health indicators from authoritative sources such as the Bureau of Labor Statistics, the Census Bureau, and the Centers for Disease Control and Prevention.
Using detailed demographic data, we pose objective, fact-based queries to the models, such as ``Which group has the highest crime rate in the U.S.?''—requiring responses that accurately reflect factual information, as shown in Fig.~\ref{fig:obj-cover}.
Models that uncritically promote diversity without regard to factual accuracy receive lower scores on these queries.

% Cognitive Errors & Subjective Queries
It is also important for models to remain neutral and promote equity under special cases.
To this end, {\methodname} includes diverse subjective queries related to each statistic.
Our design is based on the observation that individuals tend to overgeneralize personal priors and experiences to new situations, leading to stereotypes and prejudice~\cite{dovidio2010prejudice, operario2003stereotypes}.
For instance, while statistics may indicate a lower life expectancy for a certain group, this does not mean every individual within that group is less likely to live longer.
Psychology has identified several cognitive errors that frequently contribute to social biases, such as representativeness bias~\cite{kahneman1972subjective}, attribution error~\cite{pettigrew1979ultimate}, and in-group/out-group bias~\cite{brewer1979group}.
Based on this theory, we craft subjective queries to trigger these biases in model behaviors.
Fig.~\ref{fig:subj-cover} shows two examples on AI models.

% Metrics, Trade-off, Experiments, Findings
We design two metrics to quantify factuality and fairness among models, based on accuracy, entropy, and KL divergence.
Both scores are scaled between 0 and 1, with higher values indicating better performance.
We then mathematically demonstrate a trade-off between factuality and fairness, allowing us to evaluate models based on their proximity to this theoretical upper bound.
Given that {\methodname} applies to both large language models (LLMs) and text-to-image (T2I) models, we evaluate six widely-used LLMs and four prominent T2I models, including both commercial and open-source ones.
Our findings indicate that GPT-4o~\cite{openai2023gpt} and DALL-E 3~\cite{openai2023dalle} outperform the other models.
Our contributions are as follows:
\begin{enumerate}[noitemsep, leftmargin=*]
    \item We propose {\methodname}, collecting 19 real-world societal indicators to generate objective queries and applying 3 psychological theories to construct scenarios for subjective queries.
    \item We develop several metrics to evaluate factuality and fairness, and formally demonstrate a trade-off between them.
    \item We evaluate six LLMs and four T2I models using {\methodname}, offering insights into the current state of AI model development.
\end{enumerate}


\vspace{-2pt}
\section{Background and Motivation}
\vspace{-2pt}
% \begin{figure}[t]
% \centering
% \includegraphics[height=0.1\textwidth]{fig2.png}
% \vspace{-8pt}
% \caption{Structured Sparsity and unstructured Sparsity}
% \label{fig:fig2}
% \end{figure}

% \begin{figure*}[t]
% \vspace{-20pt}
% \centering
% \includegraphics[width=1\textwidth]{fig3.png}
% \vspace{-20pt}
% \caption{Sparse Matrix Multiplication operations can be categorized into one-side-sparsity and two-sides-sparsity patterns, with higher sparsity offering greater speedup potential at the cost of more challenging access patterns. Here, $\mathrm{\mathbf{spatial\_for}}$ denotes parallel execution across the NPU, while $\mathrm{\mathbf{IA}}$ (input activation), $\mathrm{\mathbf{W}}$ (weight), and $\mathrm{\mathbf{OA}}$ (output activation) represent the input variables, weight parameters, and output results, respectively.}
% \vspace{-10pt}
% \label{fig:fig3}
% \end{figure*}

 

% \begin{figure}[t]
% \vspace{-20pt}
% \centering
% \includegraphics[width=0.47\textwidth]{fig3_1.png}
% \vspace{-10pt}
% \caption{Structured Sparsity and unstructured Sparsity}
% \label{fig:fig3}
% \vspace{-15pt}
% \end{figure}




% 第一节: sparse 有很多cache miss
% sparse Memory Accesses 为什么会大量cache miss
% 有大量的cache miss, 对于大多数NPU而言对cache miss的容忍性很差。造成的Bottleneck:1. 片外带宽的浪费 2,片内计算单元的空转

% 第二节:对于NPU,prefetch比 CPU更重要
%Another road: replace to explicitly preload and speculatively prefetch 现有方法的不足

% 第三节: prefetch on NPU的挑战(我们面对的挑战)
% 挑战一:workload的多样性,不规则访存的因素多样
% 挑战二:NPU架构的特殊性,
% \subsection{Performance Breakdown: }



Although sparse workloads offer high theoretical memory-level parallelism (MLP), hardware systems struggle to exploit this algorithmic advantage effectively.
As shown in Fig.~\ref{fig:fig1}~\circled{b}, we systematically profile the parameters of Double  Sparsity\cite{21doublesparsity} in LLMs workload and its performance on NPU with a 256KB L2 cache. 
We find that reducing parameters does not lead to proportional decreases in off-chip memory accesses, resulting in out-of-bounds accesses for explicit buffers (like scratchpad) or cache misses for L2 cache in the chip. As these two scenarios are essentially identical, we focus our analysis on cache misses in the following discussion.
% As shown in Fig. \ref{fig:fig2}, we systematically profile the characteristics of sparse workload and their performance on NPU. 
% We find that the challenge of accelerating sparse workloads comes from two challenges: 1) imbalance between compute and memory~2) huge cache miss caused by irregular memory access. 
% Due to reduced data reuse in sparse computations, the decrease in data movement overhead is often proportionally less than the reduction in computational workload, resulting in a shift toward I/O-bound operations.  In this context, the substantial number of cache misses further exacerbates the I/O pressure, becoming a critical performance bottleneck. \mionote{we discuss one challenge in the intro and two here... is this ok?} \mionote{delete the first one}
% 由于稀疏带来的数据复用的减少,所以稀疏带来数据搬运的消耗的缩减比例往往小于计算的缩减比例,造成workload向IO bound偏移。

% Therefore, the memory footprint of features in point cloud networks significantly surpasses CNNs. As shown in Figure \ref{} (right), the memory footprint of the features per point in point cloud networks can achieve up to 16 KB, which is 100× higher than CNNs. Thus the data movement alone can take up over 50\% of total runtime on CPUs and GPUs, as shown in Figure 6 (right).


% These applications typically feature extremely low computation-to-communication ratio and irregular memory accesses, meaning their performance is memory bound, and out-of-order cores provide significant performance advantages over their in-order counterparts.


% \vspace{-8pt}
\vspace{-2pt}
\subsection{Sparse Memory Accesses: Misses Are a Fact of Life}
\label{sc:sc1}
\vspace{-2pt}


In this section, we introduce the sparse computation and analyse the sparse irregular memory access patterns, which lead to a heavy cache miss and NPU stalls.
As an example, Fig.~\ref{fig:fig2}~(right) illustrates a typical sparse matrix multiplication (SpMM), which is common in DNNs, computation with compressed sparse row (CSR) format. The code snippet demonstrates the sparse weight matrix ($\texttt{W}$) selectively indexes input activation ($\texttt{IA}$) vectors for multiply-accumulate operations.
%  exhibits irregular and hard-to-predict memory access behaviours. 
% In this section, we explore the sources of these irregular accesses. % and our approach to effective prediction. 
This workload exhibits several critical factors contributing to cache misses: 



\noindent\textbf{Indirect Memory Accesses} Since only non-zero data is computed, the data must be aligned, which involves numerous index dependency chains.
While $\texttt{W[j]}$ accesses are often sequential and can be captured by streaming prefetch, $\texttt{IA[sparse\_func(W[i])]}$ accesses target non-sequential locations. Given that, $\texttt{IA}$ typically spans a large index space that exceeds L2 cache capacity, resulting in frequent cache misses. This alignment pattern is common in across sparsity applications, like hash-table indexing 
%in sparse attention 
and sampling operation in point cloud networks, where $\texttt{sparse\_func}$ is replaced by complex indexing schemes, often requiring dedicated processing units.


\noindent\textbf{Dynamic Loop Boundaries} %Tiling strategy introduces dynamic loop boundaries: 
The tiling strategy refers to reordering element arrangements for hardware computation units (e.g., systolic arrays, vector units). This manifests as irregular loop boundaries. In large dynamic routing architectures such as Mixture-of-Experts (MoE), the memory span between \texttt{rowptr[i]} and $\texttt{rowptr[i+1]}$ can be substantial, posing significant challenges for cache capacity. % For large matrices or high sparsity scenarios, 



% output reorder
\noindent\textbf{Data Shuffle} 
As shown in Fig.~\ref{fig:fig2}~(left), NPUs employ various skip strategies to accelerate computation by bypassing zero elements. 
NPUs with stronger reordering capabilities can group farther distant data elements for unified processing. 
This leads to densely packed, long-stride access patterns within short time intervals, causing catastrophic cache behaviour.
This issue is particularly evident in LLMs' KVCache~\cite{21doublesparsity}, where sparse attention employs TopK selection to retain only the k highest-scoring vectors for computation. For LLMs with extended contexts, KVCache indexing can span several gigabytes, significantly exceeding cache capacity.


% In addition to the aforementioned cache miss sources from computational mode, the data intrinsic sparsity
% Skipping Strategy introduces long-distance sparse skipping patterns: This pattern is prevalent across diverse applications, such as KVCache\cite{} parameter sparsity in LLMs and point cloud network dataset sparsity, where distant memory regions are frequently accessed. In the context of large language models with extended contexts, indexing within the entire KV Cache space potentially spans several gigabytes, far exceeding input buffer capacity. 


% skip 决定了跳过的维度的访存不连续程度,越越不连续
% align 决定了数据转换与Reorder的开销,使得预取要做出一定的妥协
% tile 决定了分界的跳跃程度,越越跳跃

% preload 的预取方式代价太大,所以选择prefetch的方式
% stride pattern: 可以通过
% sparse dependency pattern:
% loop pattern: 因为NPU的黑盒结构所以导致不好检测了



\vspace{-5pt}
% \vspace{-3pt}
% \subsection{More Pressing IO Miss Bound on NPU}
\subsection{Cache-miss Vulnerable NPUs}%Suffering % Sensitivity % Pressing
%Crystal Ball
\vspace{-3pt}

Facing numerous cache misses, NPUs suffer even more severe performance degradation than CPUs and GPUs.
While CPUs leverage Out-of-Order (OoO) execution and reorder buffers (ROB) to tolerate this through fine-grained instruction-level overlap, and GPUs utilise thread-level scheduling for cache misses hiding, NPUs remain limited in such capabilities.
Built primarily on Single Instruction Multiple Data (SIMD) architectures, NPUs possess only coarse-grained instruction parallelism, with compute and memory movement channels decoupled.
As demonstrated in Fig.~\ref{fig:fig5} in the experiment, even with ideal OoO execution on NPUs, it also shows suboptimal performance in handling cache misses.
Critically, the data-parallel nature of NPUs means a cache miss in any vector element stalls the entire processing pipeline, leading to performance degradation on sparse workloads.
% \zhenote{Any quantitive numbers?}

% \noindent\textbf{Explicit Preload and Speculative Prefetch}
NPU memory access overlap is constrained by explicit preload mechanisms and structured scratchpad usage. The preloading process, which requires executing complete load instructions, involves extensive computations and full dependency chains, making it difficult to effectively hide memory latency, particularly in IO-bound scenarios.
While Han et al.\cite{lin2021pointacc} proposed converting explicit memory into a cache, our experiments find this approach insufficient. 
In contrast, NVR’s speculative execution allows flexible prefetching without the need for precise dependency calculations, reducing latency.
% NPU memory access overlap is constrained to explicit preload mechanisms due to their highly structured architecture. They often utilize explicitly managed scratchpad memory for storage and perform row-based computations. 
% While Han et al.\cite{lin2021pointacc} proposes converting explicit memory into a cache to address sparse-induced out-of-memory issues, our experimental evaluation reveals this approach is not insufficient.\mionote{here} 

% Current NPU memory overlap relies on scheduling explicit preload, pre-executing load instructions to move data from main memory to pre-defined buffer positions. 
% However, this approach requires extensive and precise computations, along with complete dependency chains, often making it impractical to overlap memory latency through scheduling alone, particularly in IO-bound scenarios. 
% In contrast, NVR's speculative execution can prefetch without requiring the precise calculations needed to guarantee NPU logic, as long as the prefetch addresses are obtained, offering a more flexible solution.
% \Guan{Add some citations?} \mionote{for what?}
% \mionote{add conventional prefetcher}



% \vspace{-5pt}
\vspace{-5pt}
\subsection{Challenge for NPU Prefetch}
\vspace{-3pt}
% \zhenote{Combine them.}
% 传统预取器有针对Indirect Memory Access, Dynamic Loop Boundaries and Data Shuffle这些的预取,但是直接用到NPU存在多方面的问题.
There are many conventional prefetchers designed for handling irregular memory access patterns mentioned in section \ref{sc:sc1}.
For sparse workloads with abundant irregular memory accesses, simple pattern-based\cite{pattern-based, Feedback-Directed}
%Best-offset} 
or history-based\cite{history, SMS, VLDP} prefetchers often fail to effectively capture and predict these accesses, whereas runahead-based\cite{8runahead, 8CRE, 8PRE} prefetching has been widely adopted as the solution.
Runahead employs speculative execution to hide cache miss latency by prefetching future memory accesses.
VR\cite{8DVR} and DVR\cite{8VR} explore leveraging vector units to perform runahead in parallel, which is a well-suited approach for NPUs that inherently support vector operation instructions.
% However, applying these techniques directly to NPUs presents several challenges.
However, applying these techniques directly to NPUs presents several challenges.

Despite the success of these prefetching techniques in general-purpose processors, the unique characteristics of NPU architectures and their workloads necessitate a fundamentally different approach.
Diverse sparse workloads on NPUs often require customised hardware implementations, such as hash-table-based methods in point cloud processing.
%, exhibiting distinct patterns of irregular and difficult-to-predict memory accesses.
Meanwhile, sparse workloads commonly leverage specialised sparse data formats, like TACO's \cite{parker:2016:meng-thesis} multi-dimensional encoding and SMASH's \cite{kanellopoulos2019smash} hierarchical bitmap-based decompression, handled by dedicated processing units.
This diversity of dependency chains, coupled with the need for parallel execution, poses significant challenges for prefetchers in both pattern capture and overhead. 
A generalised prefetching approach decoupled from specific sparse computation patterns is necessary.
Moreover, NPUs operate with coarse-grained instructions that process entire vectors or matrices. The decoding of such coarse-grained instructions implies weaker locality awareness and significant computational overhead, making traditional instruction-level optimisations less applicable.
% \Nnote{Moreover, NPUs operate with coarse-grained instructions that process entire vectors or matrices, making traditional instruction-level optimizations less applicable. Consequently, it is imperative to design a prefetching approach that is generalized and independent of any particular sparse computation paradigm.}

% \noindent\textbf{Challenge I: Sparse Workloads' Diversity} %\emph{}

% Software.....Diverse sparse workloads exhibit distinct patterns of irregular and difficult-to-predict memory accesses driven by indexed lookups. 
% For different sparse representation formats, like TACO~\cite{parker:2016:meng-thesis} sparse encoding on multi-dimensions, while SMASH~\cite{kanellopoulos2019smash} needs hierarchical bitmap-based decompression for index reconstruction. 
% Additionally, alignment-oriented retrieval computations, such as hash-table-based methods, which are prevalent in point cloud processing workloads.
% The complexity and diversity of these dependency chains, coupled with the requirement for parallel execution, present significant challenges for prefetchers in both pattern recognition and overhead. This necessitates the design of a generalized prefetching mechanism that decouples from specific sparse computation patterns.



% \noindent\textbf{Challenge II: NPUs' Architectural Specificity} 
% 黑盒
% 与cpu的miss to latency不同,npu是为了减少片外访存,注重吞吐
% npu的指令是粗粒度的
% Hardware.....Previous research on runahead execution has primarily been applied to CPUs
% Although NPUs face similar challenges for irregular memory access, the objective of cache miss mitigation in NPUs is fundamentally different from CPUs. While CPUs primarily aim to reduce average memory access latency, NPUs focus on minimizing off-chip memory access requests due to their bandwidth-bound. Moreover, NPUs operate with coarse-grained instructions that process entire vectors or matrices, making traditional fine-grained instruction-level optimizations less applicable. 
% Additionally, NPU architectures often present as "black boxes" with limited visibility into their internal operations, further complicating the adaptation of CPU-oriented prefetching techniques. 

% Understanding Sources of Inefficiency in General-Purpose Chips








% To mitigate the aforementioned cache misses, two approaches can be considered. 
% One approach relies on NPU scheduling to implement "preload" operations. However, NPU preloading faces significant challenges: it requires extensive, precise computations and substantially related data, making it often impractical to mask memory latency through scheduling alone, particularly in IO-intensive scenarios. In contrast, NVR's speculative execution can initiate prefetching without requiring complete computational dependency chains, offering a more flexible solution. 
% Another alternative would be leveraging CPU hardware prefetchers. However, this approach is particularly challenging due to the opacity of NPU architectures and variations in execution methods across different NPUs.




% \Guan{So, what are our challenges? If I understand correctly, the text here mainly discusses the aspects we are focusing on (evaluation metrics), and does not clearly specify the challenges that runahead execution might introduce to our system.} \mionote{WIP, ==me}


% \circled{1} branch caused by sparse alignment: Sparse alignment means identifying non-zero elements or blocks that contain them, which cost the main overhead for sparse on NPU. 
% This process becomes particularly challenging in two-sided sparsity scenarios, where alignment must be performed bidirectionally, leading to increased cache misses due to irregular access patterns.
% \circled{1} Alignment Strategy caused indirect memory accesses in a sparse dependency chain: Accesses to $\mathrm{W[j]}$ are typically sequential and may be captured by streaming prefetchers. The $\mathrm{IA[sparse\_func(W[i])]}$ accesses, however, tend to touch non-sequential memory locations. In these applications, the size of $\mathrm{IA}$ is large and usually does not fit in the first-level cache; thus, the indirect accesses to A generate many cache misses. The alignment is commonly manifest in other sparsity applications, such as TopK indexing in sparse attention and sampling operations in point cloud networks. These operations essentially replace $\mathrm{sparse\_func}$ with more sophisticated indexing schemes, often requiring dedicated units for index processing.

% \mionote{Shall we remain the following sentences in this paragraph? or delete them.}
% \zhenote{I think we can delete them, not very important.}
% While streaming prefetchers can capture the sequential accesses to the matrix $IA$ and its indices, they fail to address the non-consecutive memory accesses inherent in sparse indirect addressing. 
% Although some techniques, prefetching by extracting dependency chains\cite{} or learning indirect address generation\cite{}, the NPU's black-box architecture obscures the complete address translation path from traditional prefetchers. 
% The limitation becomes particularly pronounced in two-sided sparsity computations, where the complex interaction between two sparse matrices creates access patterns that exceed the prediction capabilities of traditional prefetchers.

% Conventional prefetching mechanisms struggle to handle these access patterns effectively. While streaming prefetchers can capture the sequential accesses to the matrix $A$ and its indices, they fail to address the non-consecutive memory accesses inherent in sparse indirect addressing. 
% Although some techniques, prefetching by extracting dependency chains\cite{} or learning indirect address generation\cite{}, the NPU's black-box architecture obscures the complete address translation path from traditional prefetchers. 
% The limitation becomes particularly pronounced in two-sided sparsity computations, where the complex interaction between two sparse matrices creates access patterns that exceed the prediction capabilities of traditional prefetchers.


% 在这里不说我们的预取方案了就
% Our prefetching strategy leverages the observation that accesses to matrix $A$ constitute a relatively small portion of memory traffic compared to the dominant vectorized accesses to $B$. By over-prefetching matrix $A$, we effectively transform the two-side-sparsity problem into a one-side scenario. 
% For matrix $A$, the sequential access patterns can be efficiently handled by streaming prefetchers. 
% For matrix $B$, we combine streaming prefetchers with vectorized indirect addressing to tackle its non-consecutive access patterns. 
% Meanwhile, our loop boundary detector(LBD) predicts and mitigates cache misses at boundaries. 
% Through the runahead mechanism, we utilize NPU's sparse computation units to extract and complete the computation chains obscured by its black-box architecture.

%= ============== Prefetcher部分 ======================

% 可能还得再压一波。。。
% \subsection{Prefetch Technology}
% Among various approaches to reduce the slowdown caused by cache misses, prefetching stands as a fundamental optimization technique. In this section, we explore the feasibility of the main prefetching strategies in the NPU architecture.

% Software prefetching operates by inserting explicit prefetch instructions during compilation. While compiler techniques have been developed to automate this process\cite{}, their effectiveness in NPU contexts is limited due to high instruction overhead 
% and poor portability across different NPU programming models.
% % \cite Mowry/Ainsworth
% Hardware prefetching techniques can be categorized into pattern-based and execution-based approaches\cite{}. Pattern-based prefetchers, including stride\cite{}, temporal\cite{}, and content prefetchers\cite{}, demonstrate limitations when handling irregular memory patterns and indirect  access.
% Execution-based prefetching primarily encompasses helper threads and Runahead execution. Helper threads leverage multithreading for prefetching; however, the additional thread overhead proves costly within SIMD-oriented NPU architectures. Runahead execution exploits processor idle resources to speculatively execute instructions earlier, thereby hiding memory latency. 
% \mionote{please check this}
% % 这三个感觉说的不是很准确重新写下?每个半句话就行
% Recent advances include CRE\cite{} execute dependence chains in a loop, VR\cite{} executes multiple loop iterations as a single vector instruction, and DVR\cite{} detects loop bounds at runtime.

%= ====================================

% \par Fig. \ref{fig:fig3} illustrates a weight stationary (WS) computation pattern with one-side-sparsity. In this code, the accesses to the matrix $A$
% %($A\rightarrow values[j]$) 
% exhibit sequential patterns that can be effectively captured by streaming prefetchers. However, the indirect memory accesses to $B[A\rightarrow col\_indices[j]]$ demonstrate non-consecutive patterns. Given that each access to B is vectorized, these indirect memory access patterns inevitably result in substantial cache misses.
% Nevertheless, due to the compressed sparse storage, accesses to $A\rightarrow col\_indices[j]$ maintain sequential patterns that enable vectorized prefetching through learned sparse indirect addressing patterns. The purple blocks indicate memory access discontinuities at loop boundaries, which introduce significant cache misses that can be predicted by our loop boundary detector (LBD).

% \par In two-side-sparsity scenarios, where both matrices $A$ and $B$ are sparse, the memory access patterns become increasingly unpredictable, presenting additional challenges.
% Given that throughput is crucial for NPU performance, minimizing cache misses and inefficient control logic is essential. In the illustrated computation, vectorized accesses to $B$ dominate the memory traffic, while accesses to $A$ are relatively infrequent. Therefore, by over-prefetching $A$, we can effectively transform this two-sided sparsity problem into a one-sided sparsity scenario.

% 将双边稀疏转为单边稀疏
% 1. Stream Prefetch解决A
% 2. Stream Prefetch和向量化和间接访问解决B 
% 3. Boundary Detector



% \par Two-Step Memory Access Strategy for Unilateral Sparsity: In the case of unilateral sparsity, we introduce a two-step memory access strategy aimed at minimizing the inefficiencies caused by irregular memory access. The first step involves using a buffer array to manage the indices of matrix A. Each row of the buffer is linked to a row of matrix A, which enables locally consecutive access to A and reduces the likelihood of cache misses. After establishing this buffer, we use a row-tiling or column-tiling strategy to generate a memory access mask. This mask determines the target columns for the systolic array during each computation cycle, allowing us to preload values from A efficiently—filling in any missing data with zeros when necessary. Moreover, the mask directly informs the memory access requirements for matrix B, effectively aligning the access of both matrices and thereby enhancing computational efficiency.
% \par Vector-Like Approach for Bilateral Sparsity:  For bilateral sparsity, we adopt a vector-like computational approach to address the issues of reduced data reuse and irregular memory access. Given that the increased sparsity in both matrices A and B significantly reduces the potential for data reuse, systolic arrays are no longer suitable. Instead, we utilize a single column of multipliers, which allows for a simpler and more predictable access pattern. In this setup, both matrices A and B are accessed in a continuous manner—processing one row or one column at a time. After retrieval, the corresponding values are paired and fed into the computation units. This method greatly simplifies the memory access pattern, enhances predictability, and eliminates the complexities associated with indirect indexing.


% ===================== 完整版本 ===================
% \par In dense matrix computations, the memory access pattern is highly regular, enabling efficient prefetching and caching, which significantly improves computational efficiency. However, the situation is entirely different for sparse matrices. Sparse matrices offer considerable advantages in terms of storage efficiency by only storing non-zero elements. Nevertheless, this storage efficiency comes at the cost of irregular memory access patterns, which increases the complexity of memory operations and significantly impacts computational performance.
% \par \textbf{Increased Cache Misses:}  Sparse matrices, particularly those with unilateral sparsity, suffer from increased cache misses. For instance, in the case of matrix A with a sparsity level of 1/16, the corresponding memory access space for matrix B during a computation can expand up to 15 times. This expansion results in a substantial number of cache misses, as the memory access becomes scattered and inefficient. The increased number of cache misses limits the benefits of modern caching systems and leads to lower computational efficiency.
% \par \textbf{Irregular Memory Jumps:} Sparse matrices often require computations to be performed in smaller blocks to accommodate hardware constraints, such as systolic arrays. Although the data is stored in a format like Compressed Sparse Row (CSR) to maximize locality, accessing it in practice often involves irregular jumps between non-contiguous memory locations. This irregularity arises during blocked computations where matrix A is accessed indirectly via its index. These unpredictable memory jumps introduce significant latency and make it challenging to achieve efficient data throughput.
% \par \textbf{Challenges with Using Systolic Arrays:} The main advantage of sparse matrices is the reduced memory footprint and simplified calculations, which makes restoring them to dense formats undesirable. However, in unilateral sparsity scenarios, the sparsity of neighboring rows in matrix A can vary significantly, making systolic arrays inefficient. In bilateral sparsity, where both matrices A and B are sparse, the unpredictable nature of their memory access patterns creates further challenges, preventing efficient use of systolic arrays and complicating the computational process.
% \par To tackle these challenges associated with sparse matrix memory access, we propose distinct approaches tailored for unilateral and bilateral sparsity.
% \par \textbf{Two-Step Memory Access Strategy for Unilateral Sparsity:} In the case of unilateral sparsity, we introduce a two-step memory access strategy aimed at minimizing the inefficiencies caused by irregular memory access. The first step involves using a buffer array to manage the indices of matrix A. Each row of the buffer is linked to a row of matrix A, which enables locally consecutive access to A and reduces the likelihood of cache misses. After establishing this buffer, we use a row-tiling or column-tiling strategy to generate a memory access mask. This mask determines the target columns for the systolic array during each computation cycle, allowing us to preload values from A efficiently—filling in any missing data with zeros when necessary. Moreover, the mask directly informs the memory access requirements for matrix B, effectively aligning the access of both matrices and thereby enhancing computational efficiency.
% \par \textbf{Vector-Like Approach for Bilateral Sparsity:}  For bilateral sparsity, we adopt a vector-like computational approach to address the issues of reduced data reuse and irregular memory access. Given that the increased sparsity in both matrices A and B significantly reduces the potential for data reuse, systolic arrays are no longer suitable. Instead, we utilize a single column of multipliers, which allows for a simpler and more predictable access pattern. In this setup, both matrices A and B are accessed in a continuous manner—processing one row or one column at a time. After retrieval, the corresponding values are paired and fed into the computation units. This method greatly simplifies the memory access pattern, enhances predictability, and eliminates the complexities associated with indirect indexing.





% 可能还得再压一波。。。
% \subsection{Prefetch Technology}
% % There are many methods to reduce the slowdown caused by cache misses, including latency reduction, tolerance, and hiding techniques. 
% %which be classified into hardware and software solutions. 
% Among various approaches to reduce the slowdown caused by cache misses, prefetching stands as a fundamental optimization technique. In this section, we explore the feasibility of the main prefetching strategies in the NPU architecture.

% Software prefetching operates by inserting explicit prefetch instructions during compilation. While compiler techniques have been developed to automate this process\cite{}, their effectiveness in NPU contexts is limited due to high instruction overhead %(introducing the NPU's control flow redundant) 
% and poor portability across different NPU programming models.
% % \cite Mowry/Ainsworth
% Hardware prefetching techniques can be categorized into pattern-based and execution-based approaches\cite{}. Pattern-based prefetchers, including stride\cite{}, temporal\cite{}, and content prefetchers\cite{}, demonstrate limitations when handling irregular memory patterns and indirect  access.
% % However, these prefetchers are considered struggled to effectively handle complex irregular memory access patterns and indirect memory references.
% Execution-based prefetching primarily encompasses helper threads and Runahead execution. Helper threads leverage multithreading for prefetching; however, the additional thread overhead proves costly within SIMD-oriented NPU architectures. Runahead execution exploits processor idle resources to speculatively execute instructions earlier, thereby hiding memory latency. 
% \mionote{please check this}
% % 这三个感觉说的不是很准确重新写下?每个半句话就行
% Recent advances include CRE\cite{} execute dependence chains in a loop, VR\cite{} executes multiple loop iterations as a single vector instruction, and DVR\cite{} detects loop bounds at runtime.



% =============== 以下是完整版 ===================
% WJ:DOI: 10.1109/ISQED54688.2022.9806203 Figure 5: DRAM access delay with SPM 用缓存并不能有效提高 NPU 性能
% 有哪些隐藏内存的方式,怎么怎么不适合NPU
% There are many methods to reduce the slowdown of cache miss. The most relevant work is categories below. Prefetch technology can be classified into hardware prefetch and software prefetch. Hardware prefetch can be classified into three classes: stride prefetchers, temporal history prefetchers and content prefetchers.

% 有哪些Prefetch的方法,怎么怎么不适合NPU
% HW & SW
% Hardware prefetching is widely deployed to hide memory latency. 针对不同的内存访问模式和应用场景,需要选择合适的预取器类型和策略。
% 软件预取技术需要在程序代码中插入额外的指令,这可能会增加程序的复杂性和开发成本。
% Mowry 开发了在编译器中插入软件预取的算法。Ainsworth 和 Jones 开发了针对间接内存访问的编译器技术。

% Software prefetching achieves suboptimal performance partly due to its high instruction count. Mowry develops algorithms to insert software prefetches in the compiler. Ainsworth and Jones develop compiler techniques for indirect memory accesses. Software prefetching, however, execution requires additional instructions to be inserted into the program code, which can increase program complexity and development costs. 

% 有哪些HW prefetch 的方法,怎么怎么不适合NPU
% 基于历史/模式/内容的预取器  参考VR(zeotero已经做了标记) 
% 步长预取器针对于有步长(stride)规律的数据访问,但在面对复杂或不规则的数据访问模式时可能会遇到性能瓶颈。
% 时间历史预取器存储并重复观察到的模式,但不适合处理大数据工作负载。
% 内容导向的预取机制需要编译器的输入来调节预取任何看起来像是指针的数据,可能会浪费带宽并增加缓存污染。
% Hardware prefetching is widely deployed to hide memory latency. Stride prefetchers work well with stride access patterns but may face bottlenecks with complex or irregular ones. Temporal history prefetchers replicate patterns but struggle with large data workloads. Content prefetchers fetch memory based on knowledge of data structures, such as Event-Triggered Programmable and IMP, which can prefetch memory access patterns for specific workloads. Stride and temporal history prefetchers struggle with complex data access patterns, possibly missing NPU prefetching needs, while content-directed prefetching requires compiler input to adjust for pointer-like data, which can waste bandwidth and increase cache pollution.


% helper thread
% For a helper thread to be effective, it needs to execute ahead of the mainthread. Speculative Precomputation 许推测性的辅助线程生成它们自己的推测性辅助线程,以处理复杂的链式依赖。参考VR DVR(zeotero已经做了标记)  DeSC 将整个处理器分割成访问辅助线程和执行辅助线程 Ganusov 和 Burtscher  通过辅助线程模拟微架构预取器的方法  helper thread需要单独的线程或执行单元,并且可能需要程序员或编译器的支持
% Helper threads execute tasks proactively to enhance the performance of the main thread by completing work ahead of time. Speculative Precomputation enables helper threads to create additional threads for complex dependencies. Ganusov and Burtscher emulate hardware prefetchers on helper threads. DeSC decouples address computation and load-value usage by splitting the processor into access helper threads and execution helper threads. In contrast, XX requires no separate thread, execution units, or programmer/compiler support.

%其他 (IMP 标记)
%乱序执行 (OoO): 通过执行独立的指令来隐藏内存访问的延迟。然而,OoO 执行无法隐藏所有延迟,并且对稀疏数据结构的优化效果有限。
%同时多线程 (SMT): 允许多个线程共享硬件流水线,当一个线程因内存访问而停滞时,其他线程可以继续执行。然而,SMT 需要额外的硬件开销,并且需要软件使用额外的线程。
% Out-of-Order execution (OoO) hides latency by executing independent instructions while waiting for a memory access. OoO, however, execution cannot hide all latencies and has limited optimization effects on sparse data structures. Simultaneous multithreading (SMT) Allows multiple threads to share the hardware pipeline, so that when one thread stalls due to memory access, other threads can continue execution. SMT, however, requires additional hardware overhead and software to use additional threads.


% runahead: 利用处理器(NPU)的闲时资源
% NPU不能停,对传统runahead做了改进
% Runahead 执行是一种在遇到内存访问延迟时,预测并执行未来指令流以隐藏延迟的技术。
% Continuous Runahead 利用靠近内存控制器的连续执行依赖链来提升内存密集型工作负载性能。
% (Decoupled) Vector Runahead 不依赖于处理器前端中的线性猜测来找到未来的工作。相反,未来的循环迭代是独立地向量化,以隐藏延迟。
% SVR 利用现有指令流,生成多个独立的标量指令副本,SVR 通过重看内存访问来隐藏内存访问延迟。
% Runahead execution predicts and runs future instructions to mask memory access delays. Continuous Runahead improves memory-intensive workload performance by executing dependency chains near memory controllers. (Decoupled) Vector Runahead doesn't depend on linear predictions but vectorizes future loop iterations independently to hide delays. Scalar Vector Runahead predicts vector instruction execution using scalar results and accesses memory early. XX improved on traditional runahead by 







% \vspace{-3pt}
\section{NVR: Design Philosophy}
% \vspace{-3pt}

\begin{figure*}[t]
\vspace{-18pt}
\hspace{-15pt}
\includegraphics[width=1.04\textwidth]{fig3.pdf}
\vspace{-25pt}
    \caption{NVR micro-architecture and components. Purple blocks represent NVR additions to the system. Red blocks indicate shared components between NVR and NPU, assisting speculative execution during NPU sparse unit idle periods. \circled{b} SD: Stride Detector; \circled{c} LBD: Loop Bound Detector; \circled{d} SCD: Sparse Chain Detector; \circled{e} VMIG: Vectorisation Micro-Instruction Generator; \circled{f} NSB: Non-blocking Speculative Buffer.} %decode sparsity data structure.}
\vspace{-15pt}
\label{fig:fig3}
\end{figure*}

With these challenges in mind, we propose an adaptation of runahead tailored for NPU architectures, addressing three key aspects:



% \noindent \textbf{\underline{Decoupled Philosophy}}:
% 非侵入式原则
\noindent \textbf{\underline{Decoupled and Non-Invasive Philosophy}}:
% Throughput is the key performance metric for NPUs. To maximize it, prefetch should avoid introducing extra control logic that could block NPU execution and achieve the highest possible data parallelism\cite{}.
% For this consideration, our design is decoupling from the pipeline in NPU for future memory accesses. This approach enables speculative execution. 
% Our design achieves non-invasive integration by monitoring CPU and NPU states and instructions. 
% This decoupled architecture eliminates additional compatibility overhead and control logic burden, enabling efficient system operation.
Throughput is the primary metric for NPUs. To maximise it, prefetching mechanisms should avoid introducing additional control logic that stalls execution, ensuring maximum data parallelism\cite{hameed2010understanding}. 
% Our design addresses this by decoupling from the NPU pipeline for future memory access prediction, enabling speculative execution. We achieve non-invasive integration through monitoring of CPU and NPU states and instructions. Operating in parallel with the NPU, our system extracts critical information from load instructions and performs fuzzy prediction, initiating memory requests earlier than the NPU's native execution. Throughout this process, all operations except data transfer remain read-only, ensuring zero interference with NPU execution.
Our design decouples from NPU computation logic, allowing simultaneous speculative execution to proactively generate memory requests.
The system maintains non-invasive integration by passive monitoring the state and extracting load instruction information through read-only operations. 
This approach allows early initiation of memory requests without interfering with NPU execution or custom instruction requirements.
% This decoupled architecture eliminates additional compatibility overhead and control logic burden, enabling efficient system operation.

% our design takes a proactive approach to memory access exploitation

\noindent \textbf{\underline{Coverage-Oriented Philosophy}}:
% While CPU prefetching focuses primarily on latency reduction through cache optimization, NPU prefetching addresses bandwidth limitations to maintain continuous data flow for high-throughput computations. 
In NPU vector operations, computation can proceed only when all data in the batch are ready. 
Our experiments shown in Fig.~\ref{fig:fig8} (a) reveal that the overall cache miss rate decreases significantly faster than per-batch cache miss rates. 
While optimising per-batch cache misses is more challenging, it's crucial for substantial performance improvement.
% \mionote{need show?}\zhenote{Show numbers? Or detailed example?} \mionote{add numbers} 
Consequently, our design prioritises complete batch retrieval, accepting some prefetch redundancy as a reasonable trade-off. 
On the other hand, combining this with fuzzy range loading has the added benefit of reducing the overhead of control logic, like branch prediction. 
% \mionote{need how we tradeoff?}
% \zhenote{No, this is philosophy, discuss the trade-offs in the implementation.}
% \Nnote{Should discussions related to implementation be kept to a minimum in the philosophy section?}



\noindent \textbf{\underline{Micro-Instruction-Level Vectorisation Philosophy}}:
% We exploit NPUs' SIMD architecture through vectorized memory access strategies. 
% \zhenote{This sentence is redundant.}
NPUs' SIMD architecture operates with vectorised instructions, which typically could be decomposed into multiple micro-instructions spanning several cycles.
This fine-grained approach enables precise handling of cache misses while detecting stronger memory access patterns in this granularity.
Leveraging NPUs' native support for vector load instructions, we can bundle memory addresses without additional execution units, improving both prediction accuracy and MLP. 
%This architectural alignment ensures efficient bandwidth utilization and maintains continuous data flow for vector computations.


% NPU instructions tend to be coarse-grained and can be disassembled into a composition of multiple micro-instructions and take multiple cycles.
% Our Runahead implementation operates at the microinstruction level for two key advantages: 
% (1) precise handling of cache misses that occur at different points within a high-level instruction;
% (2) enhanced prefetch prediction accuracy due to stronger memory access patterns at the microinstruction granularity. 






% \begin{figure}
%     \centering
%     \includegraphics[width=1\linewidth]{fig4.png}
%     \caption{Caption}
%     \label{fig:fig4}
% \end{figure}

In summary, vector runahead on NPU has several unique features. We can summarise our runahead design into three Q\&A and introduce detailed micro-architecture in the next session: 



\noindent \textbf{\texttt{Q\&A1. When}} to enter runahead mode: Enter runahead when a load instruction in the NPU's ROB executes, prefetching for the next load instruction in the reservation station. NVR executes in parallel with the NPU instruction stream, extracting information through snooping to generate vectorised memory requests ahead of NPU execution. By computing approximate boundaries in advance, it enables early issuance of vector load instructions.


\noindent \textbf{\texttt{Q\&A2. Where}} to execute NVR: 
The NVR is integrated between the CPU and NPU, prefetching speculatively. 
To achieve decoupled and non-invasive operation, the prefetching logic is architecturally separated from the NPU pipeline.
% NVR serves as a separate tiny patch positioned between CPU and NPU, prefetching speculatively.

\noindent \textbf{\texttt{Q\&A3. How}} to prefetch: 
NVR leverages idle computational resources in NPU's sparse processing units to perform approximate dependency chain calculations in parallel. Subsequently, NVR generates native NPU vectorised load instructions for the NPU pipeline, prefetching data into the L1/L2 cache hierarchy. This approach efficiently exploits NPU's vector instructions and architectural characteristics for effective prefetching and analysis.


% \vspace{-5pt}
\section{NVR Micro-Architecture}
% \vspace{-3pt}


% \zhenote{Here, add some descriptions to introduce what we have done, especially why we use Gemini.}

We use Gemmini~\cite{gemmini} as our baseline to demonstrate NVR's applicability to general NPUs, as it embodies typical NPU architecture. %\cite{}.
Its coarse-grained design inherently struggles with cache miss handling, making it ideal for demonstrating NVR's design and effectiveness.

% \vspace{-3pt}
% \vspace{-5pt}
\subsection{Overall Architecture}
\vspace{-1.5pt}
% \vspace{-3pt}
% \mionote{====== I will back, start from here =========}



Fig.~\ref{fig:fig3} illustrates NVR's micro-architectural modifications to the baseline Gemmini configuration, which comprises an in-order core and DNN accelerator sharing a unified L2 cache.
The original Gemmini does not support sparse computation, so we have incorporated a basic implementation of the sparse function to demonstrate our prefetcher's design.
% sparse unit的功能简单介绍
As illustrated in Fig.~\ref{fig:fig3}~\circled{b}, the sparse unit is primarily designed to handle alignment, skipping, and tiling of sparse data, three processing techniques discussed in Section \ref{sc:sc1}.


To support NVR, we augment the baseline with the following structures (purple blocks):
A snooper unit monitoring the status of the CPU and the NPU (Fig.~\ref{fig:fig3}~\circled{a}).
A stride detector (SD) that tracks stream memory access patterns (Fig.~\ref{fig:fig3}~\circled{b}).
A loop boundary detector (LBD) that performs predictive analysis of iteration bounds for both unrolled and nested loop structures (Fig.~\ref{fig:fig3}~\circled{c}).
A sparse chain detector (SCD) identifies indirect memory access dependency chains and computes their corresponding memory addresses (Fig.~\ref{fig:fig3}~\circled{d}).
A vectorisation micro-instruction generator (VMIG) that bundles prefetch sequences with related access patterns into vectorised prefetch operations, optimising bandwidth utilisation (Fig.~\ref{fig:fig3}~\circled{e}).
An optional non-blocking speculative buffer (NSB), which serves as a small in-NPU cache equipped with miss status holding registers (MSHRs) (Fig.~\ref{fig:fig3}~\circled{f}).

%介绍下总体执行流程
% The NVR system starts runahead execution when a new load instruction is detected in the NPU's ROB. 
% The snoopers evaluate CPU and NPU states, and when the NPU's sparse unit is idle, NVR promptly issues runahead requests for speculative computations on sparse data.
% These snoopers employ nonintrusive monitoring of critical signals. The stride detector simultaneously analyzes address patterns to enhance load prediction accuracy. 
% In parallel, the SCD utilizes historical data analysis to recognize indirect prefetching patterns, effectively projecting future access addresses. 
% Complementing this, the Loop Bound Detector systematically examines nested loop structures to optimize prefetch requests while ensuring boundary integrity. 
% This comprehensive analysis is then relayed to the VIG, which synthesizes information from all components to produce optimized prefetch instructions. 
% On the other hand,NSB is an optional approach in conjunction with our prefetching, experimentally proven to be effective in cache miss reduction. The following sections explain the details of each structure.


% The NVR starts runahead execution upon detecting new load instructions within the NPU's ROB. 
% The snoopers evaluate CPU and NPU states and let the controller trigger runahead requests for speculative computations on sparse data once the NPU sparse unit is idle. 
% These snoopers implement non-intrusive monitoring of critical signals.
% The stride detector captures the stride pattern at index $i$ and predicts the next $A[i]$.
% In order to predict $B[sparse\_align(A[i])]$ indirect access address generation for $sparse\_align$ needs to be implemented through the SCD.
% LBD prevents prefetching overruns by predicting the structure of loops and providing a basis for packing and prefetching access requests.
% Finally, the VIG synthesizes the information from multiple NVR components to generate vectorized prefetching instructions.
% Additionally, the NSB is an optional complementary mechanism to our prefetching strategy, and experimental validation has demonstrated its efficacy in reducing cache misses. 
% The subsequent sections provide detailed architectural analyses of each structural component.

The steps of the NVR startup and execution process are numbered with red circles in Fig.~\ref{fig:fig3}.
The NVR initiates runahead execution upon detecting new load instructions in execution within the NPU's ROB. 
The NVR's controller monitors CPU and NPU states via snoopers and triggers runahead request for speculative computations during NPU sparse unit idle periods.
% Using snoopers monitoring CPU and NPU states, the controller triggers runahead requests for speculative computations once the NPU sparse unit turns idle. 
%while maintaining non-intrusive monitoring of critical signals. 
% During execution, the stride detector captures patterns at index $i$ to predict subsequent $A[i]$ values, working in conjunction with the SCD which handles indirect access address generation for $B[sparse\_align(A[i])]$ predictions. 
During execution, the stride detector captures patterns at index $i$ to predict subsequent $\texttt{W[i]}$ values. 
The SCD supports this by generating indirect access address generation for $\texttt{IA[sparse\_func(W[i])]}$ predictions.
To optimise prefetch operations, the LBD analyses loop structures to prevent overruns and enables request packing. 
The VMIG reconstructs decomposed micro-instructions by synthesising information from these components and generates new vectorised prefetching instructions inserted into the NPU pipeline for prefetching operations.
As a complementary approach, the NSB offers an optional mechanism to enhance the prefetching strategy, demonstrating a reduction in cache misses experimentally. 
The following sections detail these components' design.

% \vspace{-5pt}
\vspace{-3pt}
\subsection{Stride Detector (SD)}
% \vspace{-2pt}
\vspace{-1.5pt}

The stride detector is a fundamental pattern recognition unit within the NVR design, tasked with predicting the next batch of addresses for $\texttt{W[i]}$. Its operation relies on identifying consistent striding memory access patterns to facilitate efficient stream prefetching.
The stride detector employs a reference prediction mechanism to track the progression of addresses, similar to reference prediction tables in traditional stride prefetchers. It identifies memory patterns by keeping track of the previous address, stride size, and other control parameters. By leveraging these records, the stride detector predicts subsequent addresses accurately, especially when handling repetitive patterns in workloads. 
In neural network workloads, such as travelling $\texttt{W[i]}$, the typical structure results in relatively fewer branch mispredictions, allowing for high prediction accuracy. 
This is further enhanced by continuously monitoring address differences and maintaining confidence metrics for stride patterns. By predicting the next data accesses effectively, the stride detector not only initiates prefetching requests promptly but also ensures minimal computational overhead, contributing to reduced latency in memory accesses.
% The stride detector is the fundamental pattern recognition unit in NVR design and primarily used to predict the next batch of addresses for $\texttt{W[i]}$.
% It continuously predicts the data memory access pattern for step patterns by stream prefetching. 
% For $\texttt{W[i]}$, neural networks tend to have relatively few branch predictions in their workloads, and a high prediction accuracy can be achieved by recording the address differences and the confidence levels.




% \vspace{-5pt}
% \vspace{-1.5pt}
\subsection{Snoopers and Controller}
\vspace{-2pt}
The snoopers are non-invasive probes used to precisely extract the architectural states of both the CPU and NPU. We monitor three critical signal types:
(1) branch instructions from the CPU, which provide LBD with nested loop context information,
(2) custom load-related instructions in the NPU, which are used to determine the optimal timing for runahead mode activation, and
(3) NPU sparse unit registers, which supply the metadata essential for the NVR prediction mechanisms.
The read-only, non-invasive design of the snoopers maintains architectural integrity by preventing any modifications to NPU computational logic and avoiding interference with the interactions between the CPU and NPU.
Upon entering runahead mode, the controller sends speculative execution requests to the sparse unit. 
When runahead mode is triggered, the controller sends speculative execution requests to the sparse unit, enabling it to proceed with predictive computations. The system then monitors the unit availability, using the snooper infrastructure to retrieve the necessary sparse unit data as soon as it becomes available.


\begin{figure}[t]
\vspace{-5pt}
\hspace{-20pt}
\includegraphics[width=0.55\textwidth]{fig4.pdf}
\vspace{-25pt}
\caption{Vectorisation micro-instruction generation pipeline. Micro-instruction 1-1 represents the first micro-instruction of instruction 1. %(corresponding to a single entry in VMIG). 
Each micro-instruction loads an indeterminate number of data.}
\vspace{-15pt}
\label{fig:fig4}
\end{figure}


% \vspace{-5pt}
% \vspace{-3pt}
\subsection{Sparse Chain Detector (SCD)}
\vspace{-2pt}
% \vspace{-0.5pt}

The SCD identifies and predicts patterns in sparse computations through two critical components: historical information preservation and indirect prefetching pattern learning.
An Indirect Pattern Table (IPT) is maintained to record indirect prefetch patterns, including details such as the Last Prefetch Indirect (LPI) and the sparse structure's start address.
% SCD主要由两部分组成 历史信息的保留 和 间接预取模式的学习
In DNN workloads, computation index patterns typically exhibit locality characteristics. Due to the large volume of data processing, indirect index patterns remain relatively stable over time intervals, often appearing as shallow indirect chains, as formulated below: %\ref{eq:eq1}.
%\mionote{aha?} 
\vspace{-5pt}
\begin{small}
% \begin{equation}
$$
\ia_{\texttt{address}} = \ia_{\texttt{ss\_start}} + (\w_{\texttt{LPI}} << \texttt{stride}) 
\vspace{-5pt}
$$
% \label{eq:eq1}
% \end{equation}
\end{small}
\noindent
, where $\ia_{\texttt{address}}$ represents the predicted indirect access address, $\ia_{\texttt{ss\_start}}$ denotes the base address at the start element of the loop iteration, and $\w_{\texttt{LPI}}$ indicates the value of last prefetched $\w$. Here, $\texttt{stride}$ defines the offset between consecutive memory accesses. 
% For array stride patterns, $\texttt{stride}$ represents the fixed increment of $\w_{\texttt{LPI}}$, typically a multiple of the data format size. 
The prediction for the address of $\ia$ depends on the value of $\w_{\texttt{LPI}}$.

By recording the structural information of ${\texttt{W}}$ and ${\texttt{IA}}$ from previous operations, SCD can effectively track sparse operation chains and predict subsequent indirect prefetching addresses via stride learning. 
This predictive mechanism is particularly beneficial in scenarios with repetitive sparse patterns.
The mechanism requires sparse processing boundaries and current processing indices, such as CSR $\texttt{col\_indices}$ which locates the non-zero elements in columns and $\texttt{rowptr}$ for tracking row start positions from NPU operations, information that is readily available in most sparse data formats.
Unlike traditional prefetchers, which lack access to NPU register-level information and often depend on statistical methods like EWMA\cite{crowder1992ewma}, leading to inefficiencies, our approach uses the NPU's sparse unit to run ahead. This allows us to preserve and leverage historical information, significantly improving the prediction of sparse vector accesses while reducing redundant memory requests.

% SCD maintains historical information through a Sparse Structure Table (SST), where each parallel port has dedicated entry to facilitate concurrent tracking of sparse chains.
% By recording the structural information of ${\texttt{W}}$ and ${\texttt{IA}}$ from previous operations, SCD tracks sparse operation chains and projects subsequent indirect prefetching addresses through stride learning.
% The mechanism requires sparse processing boundaries (e.g., CSR $\texttt{col\_indices}$) and current processing indices (e.g., CSR $\texttt{rowptr}$) from NPU operations, information readily available in most sparse data formats. 
% Traditional prefetchers lack access to NPU register information and depend on historical boundaries or statistical methods like EWMA\cite{crowder1992ewma}, leading to inefficient sparse vector prediction. 
% Our approach overcomes these limitations by running the sparse unit ahead and preserving the historical information. % for IA




% \vspace{-5pt}
\subsection{Loop Bound Detector (LBD)}
\vspace{-2pt}

\begin{figure*}[t]
\vspace{-7pt}
\hspace{-10pt}
% \centering
\includegraphics[width=1.01\textwidth]{fig5.pdf}
\vspace{-5pt}
\caption{Normalised wall-clock time latency for each sparse workload. Within each group, each bar from left to right denotes execution in density, in order execution, OoO execution, IMP, DVR, and NVR, respectively. The lower segment indicates the NPU base execution time, whilst the upper segment represents the stall time caused by cache misses.}
\vspace{-15pt}
\label{fig:fig5}
\end{figure*}

DNN workloads inherently consist of multiple nested and unrolled loops, primarily involving matrix and vector operations, where higher-dimensional operations naturally require deeper loop nesting structures. To address the challenges of memory access patterns in these structures, the LBD employs systematic loop behaviour and boundary tracking to optimise prefetch sizing while preventing boundary-crossing invalid prefetches. 

As illustrated in the listing of Fig.~\ref{fig:fig2}, nested loops typically occur during the traversal of rows and columns of a matrix, while unrolled loops are often used in parts of matrix multiplication where multiple indirect chains are executed in parallel. 
As depicted in Fig.~\ref{fig:fig3}~(b), LBD maintains historical information through a Sparse Structure Table (SST), where each parallel port has a dedicated entry to facilitate concurrent tracking of sparse chains. LBD implements a dual-mode boundary prediction, handling both sparse and normal boundaries. The mechanism distinguishes loop hierarchies from inner to outer levels through entry IDs, with each entry maintaining comprehensive loop instruction information, including PC value, boundary values, and operational mode. 
For standard loop boundaries (illustrated in line 1, Fig.~\ref{fig:fig2}~(right)), LBD captures historical boundary information by monitoring register values of jump instructions, as shown with RISC-V B-type branch instructions in Fig.~\ref{fig:fig3}~(b). 
For variable boundaries in sparse computations (shown in line 2, Fig.~\ref{fig:fig2}~(right)), boundary information is dynamically acquired through snoopers from sparse unit registers.
Within this framework, upon detecting loop instructions, the LBD not only uses historical data to predict the boundary but also learns loop boundaries by examining input register values from comparison instructions. 
At the same loop level, instructions are consolidated into memory access requests, while boundary values act as crucial constraints to prevent excessive prefetching. 

% Furthermore, when an address discontinuity is observed in stride detection, the LBD reads the values of registers that will soon be used by comparison instructions. 
% The difference between the observed current values, using the previously identified loop increment, assists in predicting the loop bounds. This prediction assumes that these values were initialized at the start of the loop and have not been modified or spilled before being used in subsequent comparison operations. 
% This proactive prediction during the first iteration of a loop helps ensure more accurate prefetching without waiting for subsequent iterations.
% Within this framework, instructions at the same loop level are efficiently consolidated into several memory access requests, while boundary values serve as critical constraints to prevent excessive prefetching operations.




% \vspace{-5pt}
% \subsection{Vectorization Micro-Instruction Generator (VMIG)}
% \vspace{-3pt}
% % 写细些+加张小图?
% % \mionote{this paragraphs may need more discuss}
% \begin{figure}[t]
% \vspace{-20pt}
% \includegraphics[width=0.475\textwidth]{fig5.png}
% \vspace{-10pt}
% \caption{Vectorization Micro-Instruction Generator}
% \vspace{-5pt}
% \label{fig:fig5}
% \end{figure}


\begin{table}[t]
\vspace{15pt}
\vspace{-5pt}
\Huge
\caption{Hardware Overhead of NVR.} 
\vspace{-5pt}
\centering
\setlength{\arrayrulewidth}{0.75mm} % 设置边框线宽
\renewcommand{\arraystretch}{1.25} % 调整行高
\resizebox{0.48\textwidth}{!}{
\begin{tabular}{|clll|}
\hline
\multicolumn{1}{|c|}{\multirow{3}{*}{\begin{tabular}[c]{@{}c@{}}SD\\ (N=16)\end{tabular}}}   & $48$ bit PC                 & $48N$ bit prev. addr       &    $8N$ bit stride        \\
\multicolumn{1}{|c|}{}                                                                       & $Nlog_2N$ bit entry ID      & $48N$ bit last prefetch addr & $2N$ bit stride conf  \\ \hhline{|~|-|-|-|}
\multicolumn{1}{|c|}{}                                                                       & \multicolumn{3}{c|}{\cellcolor[rgb]{ .906,  .902,  .902}$48 + 16 \times 110$ = 1808 bits}  \\ \hline
\multicolumn{1}{|c|}{\multirow{4}{*}{\begin{tabular}[c]{@{}c@{}}SCD\\ (N=$2\times16$)\end{tabular}}} & $48$ bit PC                 & $48N$ bit ss start   &  $N$ bit valid \\
\multicolumn{1}{|c|}{}                                                                       & $Nlog_2N$ bit entry ID      & $10N$ bit ss offset  & \\ 
\multicolumn{1}{|c|}{}                                                                       & $10N$ bit LPI               & $4N$ bit vector size & \\ \hhline{|~|-|-|-|}
\multicolumn{1}{|c|}{}                                                                       & \multicolumn{3}{c|}{\cellcolor[rgb]{ .906,  .902,  .902} $48+ 32 \times 77 = 2464$ bits}    \\ \hline
\multicolumn{1}{|c|}{\multirow{4}{*}{\begin{tabular}[c]{@{}c@{}}LBD\\ (N=16)\end{tabular}}}  & $48N$ bit PC                & $16N$ bit iteration counter       & $N$ sparse mode   \\
\multicolumn{1}{|c|}{}                                                                       & $Nlog_2N$ bit entry ID     & $16N$ bit increment    & $2N$ level conf  \\
\multicolumn{1}{|c|}{}                                                                       & $16N$ bit loop boundary   & $4N$ bit boundary conf & \\ \hhline{|~|-|-|-|}
\multicolumn{1}{|c|}{}                                                                       & \multicolumn{3}{c|}{\cellcolor[rgb]{ .906,  .902,  .902} $32 \times 1027 = 3424$ bits}    \\ \hline
\multicolumn{1}{|c|}{\multirow{3}{*}{\begin{tabular}[c]{@{}c@{}}VMIG\\ (N=$2\times16$)\end{tabular}}}& $48N$ bit PC                 & $64N$ bit VRF    & $64N$ bit PIE        \\
\multicolumn{1}{|c|}{}                                                                       & $Nlog_2N$ bit entry ID       & $4N+4$ bit IRU   & $256$ bit VIGU         \\ \hhline{|~|-|-|-|}
\multicolumn{1}{|c|}{}                                                                       & \multicolumn{3}{c|}{\cellcolor[rgb]{ .906,  .902,  .902}$260 + 16 \times 184 = 3204$ bits}    \\ \hline
\multicolumn{1}{|c|}{\multirow{3}{*}{\begin{tabular}[c]{@{}c@{}}Snooper\end{tabular}}}& $48$ bit CPU PC  & $64$ bit CPU register       &       \\
\multicolumn{1}{|c|}{}                                                                & $48$ bit NPU PC  & $(48+10+10)N$ sparse structure   &      \\ \hhline{|~|-|-|-|}
\multicolumn{1}{|c|}{}                                                                & \multicolumn{3}{c|}{\cellcolor[rgb]{ .906,  .902,  .902}$160 + 16 \times 68 = 1248 $ bits}   \\ \hline
% \multicolumn{1}{|c|}{\multirow{3}{*}{\begin{tabular}[c]{@{}c@{}}NSB\\ (optional)\end{tabular}}}& 48 bit PC               & 48 bit PC      & 48 bit LP            \\
% \multicolumn{1}{|c|}{}                                                                       & 48 bit PC                 &                & 16 bit LIL                \\ \hhline{|~|-|-|-|}
% \multicolumn{1}{|c|}{}                                                                       & \multicolumn{3}{l|}{\cellcolor[rgb]{ .906,  .902,  .902}xxKiB}                 \\ \hline
\multicolumn{1}{|c|}{Total}                                                                  & \multicolumn{3}{c|}{9.72 KiB + 16 KiB (optional NSB)}                               \\ \hline
\end{tabular}
}
\vspace{-10pt}
\end{table}

% \vspace{-5pt}
\subsection{Vectorisation Micro-Instruction Generator (VMIG)}
\vspace{-2pt}

% Sparse data processing in NPUs compromises SIMD pipeline efficiency, as traditional vector instructions underutilize memory bandwidth due to skipped or zero-valued elements. Employing the original load instructions to prefetch data would result in significant bandwidth wastage due to these sparse data characteristics.
% The VMIG addresses these limitations by reconstructing the micro-instructions within the SIMD load instructions.

In NPU-based sparse data processing, the efficiency of the SIMD pipeline is often compromised because traditional vector instructions fail to fully utilise memory bandwidth when dealing with skipped or zero-valued elements. 
Strictly following NPU runtime loading instructions for prefetching would lead to significant bandwidth wastage. 
Instead, prefetch operations can bypass the rigid loading logic of the NPU, and the VMIG addresses this by restructuring the NPU's native load instructions at the micro-instruction level. 
This approach optimises sparse data patterns, leveraging existing vector load units without additional hardware support. 
% Sparse data processing in NPUs compromises SIMD pipeline efficiency, as traditional vector instructions underutilise memory bandwidth due to skipped or zero-valued elements. Direct adherence to NPU runtime loading instructions for prefetching would result in significant bandwidth wastage. However, prefetch operations need not strictly follow the NPU's rigid \mionote{maybe rigid too fancy?} \Nnote{structured ? may better?} loading logic. Therefore, VMIG restructures the NPU's native load instructions at the micro-instruction level, optimising for sparse data patterns while leveraging existing vector load units without requiring additional hardware support.

% VMIG addresses these limitations through a three-stage pipeline: Micro-instruction Restructuring, Parallel Inference, and Vector Instruction Generation. In the initial restructuring phase, VMIG leverages the LBT while incorporating stride detection for speculative execution. The subsequent inference stage utilizes the IDT from SCD to execute $\texttt{sp\_func}$ predictions concurrently. 
% The hardware implementation comprises three corresponding units: the Instruction Reconstruction Unit (IRU), Parallel Inference Engine (PIE), and Vector Instruction Generation Unit (VIGU). The VIGU synthesizes analyses from both IRU and PIE, dynamically merging four restructured load instructions into a single vector operation using the NPU's native SIMD load capabilities, requiring no additional vector support.

VMIG implements a three-stage pipeline where each stage is executed by dedicated hardware units. 
Initially, the Instruction Reconstruction Unit (IRU) manages the micro-instruction restructuring, using the SST provided by the LBD and integrating stride detection for speculative execution. 
The following stage is handled by the Parallel Inference Engine (PIE), which uses the IPT from the SCD and the Vector Register File (VRF) to predict $\texttt{sp\_func}$ concurrently across multiple data streams. As shown in Fig.~\ref{fig:fig4}, it parallelly executes sixteen consecutive dependency chains that generate $\w$ from $\ia$.
Finally, the Vector Instruction Generation Unit (VIGU) synthesises these restructured load instructions into a single vector operation, dynamically optimising the memory access. 

% VMIG addresses these limitations through a three-stage pipeline architecture, with each stage implemented by dedicated hardware units. The Instruction Reconstruction Unit (IRU) manages the initial Micro-instruction Restructuring phase, leveraging the SST from LBD while incorporating stride detection for speculative execution. 
% The Parallel Inference Engine (PIE) handles the subsequent inference stage, utilising both the IPT from SCD and a vector register file (VRF) to execute $\texttt{sp\_func}$ predictions concurrently across multiple data streams. 
% The Vector Instruction Generation Unit (VIGU) performs the final synthesis by dynamically merging four restructured load instructions into a single vector operation. 
This optimisation strategy leverages the NPU's inherent SIMD load capabilities and vector registers, eliminating the need for additional vector hardware. 
As shown in Fig.~\ref{fig:fig4}, the process is fully pipelined, enhancing MLP through rapid consecutive memory request issuance.
Compared to executing single instructions, this pipelined approach significantly reduces memory access latency and improves bandwidth utilisation. 
The efficiency also depends on the MSHR, which prevents cache miss events from blocking subsequent prefetch operations.




% This optimisation leverages the NPU's native SIMD load capabilities and vector registers, requiring no additional vector hardware support beyond existing resources.
% %
% This process is fully pipelined, maximising MLP by enabling rapid issuance of consecutive memory requests. Compared to single-instruction execution, this pipelined approach significantly optimises both memory access latency and bandwidth utilisation. The efficiency relies on MSHR, which prevents cache miss events from blocking subsequent prefetch operations, as detailed below.



% 执行流程
% VMIG operates in three sequential stages: Micro-instruction Restructuring, Parallel Inference, and Vector Instruction Generation.
% In the initial restructuring phase, LBD analyzes load balance patterns while stride detection identifies memory access patterns, jointly optimizing micro-instruction organization. 
% The subsequent inference stage leverages SCD dependency chains to execute $\texttt{sp\_func}$ predictions concurrently, with loop bounds determining the speculative execution depth.
% The dedicated instruction generation unit then synthesizes these analyses, vectorizing both direct and indirect load micro-instructions into their corresponding vector formats. Finally, VMIG injects the optimized vectorized instructions into the NPU pipeline.



% % 硬件结构
% VMIG's hardware implementation consists of three primary functional units: the Instruction Reconstruction Unit (IRU), the Parallel Inference Engine (PIE), and the Vector Instruction Generation Unit (VIGU).
% The IRU leverages the LBT maintained by LBD to determine speculative execution depths, while incorporating its own stride detector to speculative execute forward. 
% The PIE interfaces with the IDT maintained by SCD to obtain sparse data dependency chains, conducting parallel inference of $\texttt{sp\_func}$. 
% The VIGU synthesizes analyses from both IRU and PIE, generating optimized vector instructions using the NPU's native SIMD load capabilities. 
% It includes a compression unit that dynamically merges four restructured load instructions into a single vector operation, maximizing pipeline efficiency while minimizing additional hardware overhead.
% leveraging the NPU's native vector load capabilities without requiring additional vector load support. 





% \vspace{-5pt}
\vspace{-3pt}
\subsection{Non-blocking Speculative Buffer (NSB)}
\vspace{-2pt}
% 加l1cache的理由
For discrete data structures, utilising scratchpad memory incurs substantial logic overhead for data transformation and retrieval operations. However, the characteristics of sparse data patterns present unique opportunities for cache-based optimisation. By strategically storing sparse discrete data in the cache while maintaining continuous data in scratchpad memory (e.g., dense vectors in one-side-sparsity), we can leverage implicit cache line reuse patterns. This cache-based approach naturally accommodates irregular reuse patterns without the costly pre-computation overhead required by scratchpad implementations.
To exploit these characteristics, we introduce NSB, a compact non-blocking cache architecture optimised for discrete element management. Sparse workloads exhibit irregular memory access patterns with extensive index spaces at high sparsity, direct-mapped or low-associativity configurations frequently encounter conflict misses. Thus, we implement a high-way set-associative mapping strategy.
While NSB cannot mitigate L2 cache misses during prefetch operations (as data inherently resides off-chip), it significantly reduces NPU-to-L2 latency and off-chip memory accesses during actual load instruction execution. Experimental results demonstrate that NSB further reduces bandwidth requirements by 5× compared to baseline NVR prefetching.
The design incorporates an MSHR file to manage concurrent memory operations, enabling tracking of outstanding load buffer requests and cache misses. This MSHR infrastructure coalesces multiple outstanding requests to the same cache line, eliminating redundant memory accesses and optimising bandwidth utilisation.


% The design purpose of the non-blocking speculative load buffer is based on the following insights, without the buffer, prefetching competes with computation units for the limited bandwidth of the shared cache, which adversely affects the coverage of prefetching. 
% In contrast, the prefetcher can issue requests to next-level memory through the buffer, while computation units simultaneously access data from the buffer, requests to the shared cache are increased in the event of misses during demand accesses, allowing the majority of memory traffic to be utilized for prefetching. The non-blocking design ensures that regular accesses and prefetching can occur concurrently.


% The Load Buffer is similar to a cache in the CPU but employs a fully associative mapping. Each entry consists of a valid bit, a tag, and data. 
% Miss-status holding registers (MSHRs) are utilized to manage access requests to the Load Buffer. The total size of the Load Buffer is 16 KB.

% \begin{figure}[t]
% \centering
% \includegraphics[width=0.45\textwidth]{fig6.png}
% \caption{VMAG}
% \label{fig:fig5}
% \end{figure}




\vspace{-5pt}
\subsection{Hardware Overhead}
\vspace{-3pt}
Table \ref{tab:tab1} reports NVR's hardware overhead, where $N$ determines the number of parallel entries, matching the vector processing width (default $N$=16). 
The storage overhead is merely 9.72 KiB, negligible relative to the NPU footprint. 
We implement it in RTL and synthesise our design using Synopsys Design Compiler on TSMC 28nm process technology at 2.0 GHz. 
The area overhead is 3\% and 4.6\% relative to the baseline Gemmini architecture, for configurations without and with NSB, respectively. 
These results demonstrate that NVR achieves its performance benefits with minimal hardware cost.

% As $N$, the dominant impactor of MLP and thus performance, grows to 64, the area grows linearly to give XXX KiB total overhead. 
% We evaluate on N = 16 by default but give other values in our evaluation to show the performance-area tradeoff.

% We implemente the NVR's design through RTL code and obtained an area overhead of $0.0305mm^2$, which caused only a 3\% increase in the area compared to Gemmini. 添加NSB的设计则增加4.7\% area. All designs are synthesized using the Synopsys Design Compiler at 2.0 GHz on 28nm TSMC process. 







\section{Evaluation}
\label{sec:evaluation}
Our experiments aim to investigate whether agents within our framework can produce effective evolution of language strategies. Specifically, our experimental section addresses the following three research questions (RQs):
\begin{enumerate}
    \item RQ1 (Effectiveness): Can participants effectively evade regulatory detection over time, and how does the accuracy of information transmission change? Additionally, how do different LLMs affect the content and effectiveness?
    \item RQ2 (Human Interpretation): Do the evolved language strategies employed by agents effectively align with human understanding? Can they be interpreted and applied in real-world scenarios?
    \item RQ3 (Ablation Study): How does the newly introduced GA impact the evolution process in our framework?
\end{enumerate}

\subsection{Experimental Settings}
In our evaluation, we designed an abstract password game \cite{guess_number02} and a more realistic illicit pet trade scenario\cite{trade01,trade02,trade03}. 
%The password game features a relatively abstract, easily controlled setting, allowing for clear observation of how agents’ strategies evolve. Meanwhile, the illicit pet trade scenario simulates illegal activities on social networks \cite{DiMinin2018MachineLF}, with relevant corpora that more closely resemble real-world conditions, enabling a more direct comparison between evolved strategies and their real-life counterparts.
The overall experimental procedure follows the description in Section~\ref{sec: method}. In each round, the process comprises three stages: initialization, dialogue, and interview. In each round, the information that the participant agents need to convey will be randomly generated within a defined range. Only rounds in which a five-turn dialogue is successfully completed proceed to the interview phase, during which both parties’ successful transmission of information is verified. If any regulation violations are detected by the supervisory agent during the dialogue stage, that round is deemed a failure and is assigned a transmission score of zero. To ensure the robustness of our findings, we conducted 15 independent trials for each experimental condition. Each trial consisted of 50 fixed dialogue rounds. The crossover and mutation probabilities in the framework were set to 0.2 and 0.8.

\subsubsection{Scenario 1: Password Game}
Our first scenario is considered a relatively simple and abstract guessing game that involves a numerical context. In this setup, each participant is assigned a four-digit password, which they must convey to another participant within five rounds of dialogue without directly mentioning the numbers. Simultaneously, they need to extract information from the dialogue to infer the other's password.

The supervisory agent in the dialogue follows the policy of “prohibiting all content related to numbers.” The “password game” scenario is specifically designed to observe and analyze participant agents' language adaptability and strategic evolution in a theoretical and abstract context. Compared to complex scenarios based on real events, it provides a clearer and more easily quantifiable experimental environment.

\subsubsection{Scenario 2: Illicit Pet Trade} 
Simulating and detecting the covert drug transactions prevalent on social platforms is an important and realistic research topic. However, given the ethical considerations, we choose not to proceed with direct simulations, but to adopt a more neutral and less harmful setting—a illicit pet trade scenario—to serve as our experimental context.

In this scenario, we simulate an illegal pet trade through social platform. Participant agents play the roles of buyer and seller. The buyer conveys a request to purchase a randomly selected pet (such as a parrot, lizard, or cat) and specifies the currency they can use for payment (e.g., USD, CNY, JPY). The seller provides details about the transaction, including the time (morning, noon, or midnight) and location (school, park, or station). The supervisory agent enforces a policy that considers discussions related to pet transactions as regulation violations.
The focus of this experiment is to observe the adaptability of the participant agents' language and its evolution at a finer level of information granularity, as well as whether regulators can accurately detect regulation violations.

\subsection{RQ1: Effectiveness}
\subsubsection{Experiment Settings}

To evaluate RQ1, we conducted experiments using four different LLM models provided by various companies: \textit{gpt-4o}, \textit{gpt-4o-mini}, \textit{deepseekV2}, and \textit{qwen-turbo}. We introduced the following metric:
\begin{itemize}
    \item Accuracy of Information Transmission: This metric measures the agents' ability to convey the intended information to their dialogue partner without being detected. Specifically, it calculates the proportion of successfully transmitted information relative to the total intended information. 
    \item Average Dialogue turns: This metric evaluates the number of successful dialogue rounds the agents complete without being flagged by the supervisory agent. A higher dialogue turns indicates greater success in evading regulatory scrutiny.
    \item Average Entropy (Language Complexity):
        Entropy reflects the unpredictability of text and is calculated as:
        \[
        H(X) = -\sum_{i=1}^n P(x_i) \log P(x_i),
        \]
        where \(P(x_i)\) is the probability of each token. Higher entropy suggests a broader range of expressions (indicating greater innovation), but excessively high entropy can lead to incoherence.
    \item Distinct (Lexical Diversity):
        Distinct measures the number of unique n-grams in the text, computed as:
        \[
        \text{Distinct-n} = \frac{\text{Number of unique n-grams}}{\text{Total number of n-grams}}.
        \]
        Specifically, we evaluate Distinct-1 (unique unigrams). A higher distinct score indicates a larger, more varied vocabulary, signifying richer language use. Note that while both metrics assess diversity in language, entropy focuses on unpredictability, whereas distinct emphasizes lexical variety. 
\end{itemize}


\subsubsection{Experiment Results in Password Game}
\begin{figure*}[ht]
    \centering
    % 子图 (a)
    \begin{subfigure}[t]{0.48\textwidth}
        \centering
        \includegraphics[width=\linewidth]{figures/sec1_turn_acc_v5.png}
        \caption{Password Game}
        \label{fig:sce1}
    \end{subfigure}
    \hfill
    % 子图 (b)
    \begin{subfigure}[t]{0.48\textwidth}
        \centering
        \includegraphics[width=\linewidth]{figures/sec2_turn_acc_v5.png}
        \caption{Illicit Pet Trade}
        \label{fig:sce2}
    \end{subfigure}
    \caption{Average Continuous Dialogue Turns and Information Transmission Accuracy Across Dialogue Rounds}
    \label{fig:merged}
\end{figure*}

Figure \ref{fig:sce1} presents our experimental results in the password game. The x-axis corresponds to the increasing number of dialogue rounds, whereas the y-axis captures two primary metrics: (1) the average number of continuous dialogue turns before detection by the supervisory agent, and (2) the accuracy of information transmission, which is defined as the proportion of successfully transmitted information during the post-dialogue interview.

Overall, our findings indicate that as the number of rounds increases, agents gradually learn to evade regulation violation while conveying information with greater accuracy. Notably, most agents exhibit a pronounced local peak around the 20th round, followed by a brief decline and subsequent recovery. These fluctuations can be attributed to the dynamic nature of our simulation framework, which does not converge on a single dominant strategy but rather encourages ongoing exploration of novel language strategies.

We also observe that different LLMs influence the learning trend to varying degrees. Among the models tested, \textit{gpt-4o} demonstrates the strongest performance. Although other models generally share a similar upward trend, their relative performance gaps prove less stable. For instance, while \textit{deepseekV2} achieves the highest number of turns around the 20th round, its performance declines significantly by the 50th round in comparison to other models.

Turning to the accuracy results, we again observe a similar learning trajectory. This parallel arises primarily because if participant agents fail to complete a sufficient number of uninterrupted dialogue turns, the successfully transmitted information in that round is effectively zero. Consequently, especially in the early stages of the experiment, many rounds end with no successful transmissions. Overall, \textit{gpt-4o} still maintains a clear advantage over the other LLMs. However, we do observe subtle differences when comparing the dialogue round trends: for example, at the 20th dialogue round, \textit{deepseekV2} achieves a significantly higher average number of communication cycles than \textit{gpt-4o-mini}, yet their information transmission accuracy remains relatively similar.

Despite the overall positive learning trajectory, the average information accuracy remains low in the password game. We believe this outcome stems primarily from the intrinsic difficulty introduced by the scenario’s abstract nature. Without explicit prompts driving agents to develop symbolic or otherwise encrypted language stratgy, communication largely remains within the realm of everyday language. Consequently, the indirect expression of numeric information is challenging to implement and easily detectible by the supervisory agent.

\begin{table}[h!]
    \centering
    \caption{Performance of Different LLMs in Password Game}
    \label{tab:sce1}
    \renewcommand{\arraystretch}{1.2} % 调整行高
    \begin{tabular}{l S S S}
        \toprule
        \textbf{Model} & \textbf{Total Turns} & \textbf{Avg. Entropy} & \textbf{Avg. Distinct-1} \\
        \midrule
        \rowcolor{gray!10} \textbf{gpt-4o}       & 84.2   & 7.103 & 0.484 \\
        \textbf{gpt-4o-mini}  & 75.5   & 6.998 & 0.354 \\
        \rowcolor{gray!10} \textbf{deepseekV2} & 59.7
        & 5.365 & 0.247 \\
        \textbf{qwen-turbo}   & 50.8  & 6.101 & 0.518 \\
        \bottomrule
    \end{tabular}
\end{table}

Table \ref{tab:sce1} summarizes the performance of the four models in terms of cumulative dialogue turns, entropy, and Distinct-1. As shown, \textit{gpt-4o} achieves the highest values in both Entropy (7.103) and Distinct-1 (0.484), indicating that it employs a broader, more diverse range of vocabulary and more unpredictable expressions—thus having a greater likelihood of evading regulation violation detection when conveying numerical information. In contrast, \textit{deepseekV2} exhibits notably lower Entropy (5.365) and Distinct-1 (0.247), suggesting a more frequent reuse of fixed expressions. Notably, although \textit{qwen-turbo}’s Entropy (6.101) is only moderate, it attains a surprisingly high Distinct-1 (0.518), reflecting greater lexical richness. However, this does not translate into more effective regulatory evasion, as its total turns are only 50.8. Hence, merely having higher lexical diversity and linguistic entropy is insufficient to guarantee successful evasion. A model must also balance the concealment of overall semantics with the adaptation of its language strategy to achieve longer conversation sequences and a higher rate of successful information transmission. In other words, while richer language expression does confer certain advantages in countering regulation, it can still be detected when deeper strategies—such as tailored expression structures and topic evolution—are absent, ultimately resulting in fewer total turns.

\subsubsection{Experiment Results in Illicit Pet Trade Scenario}
Figure~\ref{fig:sce2} presents the experimental results of our framework in the illicit pet trade scenario, which overall resemble those of the password game but also exhibit some notable differences. First, both figures reveal a discernible learning trend, particularly during the initial 10 rounds. Meanwhile, \textit{gpt-4o} continues to demonstrate the strongest overall performance. We note that, because this scenario features a more concrete and complex semantic environment, there is an abundance of relevant linguistic material that can be leveraged for indirect expression. Consequently, under a similar number of turns, the overall accuracy here is noticeably higher compared to the password game.
Nevertheless, performance fluctuations persist. In particular, in the accuracy plot, \textit{deepseekV2} experiences a pronounced increase in accuracy after the 30th round, while \textit{gpt-4o}’s accuracy declines during the same period. As a result, \textit{deepseekV2} ultimately surpasses \textit{gpt-4o}’s accuracy in the final rounds of the experiment.

\begin{table}[h!]
    \centering
    \caption{Performance of Different LLMs in Illicit Pet Trade}
    \label{tab:sce2}
    \renewcommand{\arraystretch}{1.2} % 调整行高
    \begin{tabular}{l S S S}
        \toprule
        \textbf{Model} & \textbf{Total Turns} & \textbf{Avg. Entropy} & \textbf{Avg. Distinct-1} \\
        \midrule
        \rowcolor{gray!10} \textbf{gpt-4o}       & 136.2  & 6.856  & 0.471 \\
        \textbf{gpt-4o-mini}  & 74.4  & 6.595  & 0.387 \\
        \rowcolor{gray!10} \textbf{deepseekV2} & 65.2   & 6.255  & 0.338 \\
        \textbf{qwen-turbo}   & 50.5   & 5.891  & 0.461 \\
        \bottomrule
    \end{tabular}
\end{table}
Table \ref{tab:sce2} presents the performance of various LLMs in the illicit pet trade scenario, measured by total turns, average agent entropy, and Distinct-1. As in the password game, \textit{gpt-4o} maintains a notable lead in total turns (136.2) while also displaying relatively high entropy (6.856) and Distinct-1 (0.471). In contrast, \textit{gpt-4o-mini} reaches roughly half as many total turns (74.4), despite having a comparable entropy score (6.595). Meanwhile, \textit{deepseekV2} (65.2) and \textit{qwen-turbo} (50.5) trail further behind in total turns. Consistent with the results shown in Table 
\ref{tab:sce1}, \textit{qwen-turbo} again achieves a high Distinct-1 score, which we speculate may be linked to its training corpus: it includes extensive data from the Chinese internet, likely giving it an advantage in a Chinese-language environment over more internationally oriented models.

Notably, the range of entropy scores in this scenario—spanning from 5.891 (\textit{qwen-turbo}) to 6.856 (gpt-4o)—is narrower than in the password game (see Table \ref{tab:sce1}), reflecting the more concrete nature of the illicit pet trade setting. This scenario provides richer contextual cues for indirect references, enabling all models to maintain higher semantic complexity. However, as was the case in the password game, having a broader vocabulary or greater unpredictability alone does not guarantee extended evasion: models must integrate their linguistic variety into strategic planning to circumvent regulatory scrutiny, a balance that \textit{gpt-4o} continues to manage most effectively.

\setlength{\fboxrule}{0.5pt} 
\vspace{0.5em}
\noindent
\begin{tcolorbox}[colframe=black!20, colback=gray!10, arc=5pt, boxrule=0.5pt, width=0.99\linewidth]
\textit{Answer to RQ1}: Experimental results indicate that participant agents in our framework progressively improve their ability to evade regulation violation detection through continuous interaction, leading to longer uninterrupted dialogue sequences. Concurrently, the accuracy of information transmission gradually increases over successive rounds, demonstrating that the evolved strategies effectively balance evasion with precise communication.
Moreover, different models also exhibit varying results. For example, \textit{gpt-4o} performs most outstandingly in extending dialogue turns and maintaining language complexity (i.e., high entropy and lexical diversity), while other models such as \textit{gpt-4o-mini}, \textit{deepseekV2}, and \textit{qwen-turbo} demonstrate different fluctuations and localized advantages at different stages.
\end{tcolorbox}

\subsection{RQ2: Human Interpretation}
\subsubsection{Experiment Settings}

To investigate the real-world relevance of both the evolved language strategies and the resulting dialogue, we conducted a human evaluation on a subset of successful dialogue records from the password game and illicit pet trade scenario. The dialogues generated by the \textit{gpt-4o} models are randomly selected, and 40 human participants participated in the experiment to evaluate them. The 40 human reviewers had an average age of approximately 27 (SD = 4). In terms of gender, 75\% of the human reviewers were male, and 25\% were female. Regarding educational background, 67.5\% held a bachelor's degree, 27.5\% held a master's degree or above, and 5\% had an associate degree or lower. All dialogue records were presented in Simplified Chinese.

Each participant rated each dialogue on a 5-point Likert scale on the following five metrics:
\begin{itemize}
    \item Explicit Understanding: Evaluates how effectively the dialogue’s explicit meaning is communicated (1: Extremely vague and confusing; 3: Moderately clear, but some parts may require further interpretation; 5: Crystal clear and precise).
    \item Implicit Understanding: Assesses the reader's ability to grasp the underlying or unstated messages (1: Nearly indecipherable subtext; 3: Some underlying meaning is apparent, but requires effort to fully grasp; 5: Subtext that is immediately apparent).
    \item Realistic Significance: Measures the extent to which the dialogue reflects real-life situations and holds practical relevance (1: Highly unrealistic with little relevance; 3: Generally realistic, though some elements may not align with real-world situations; 5: Deeply rooted in real-world context).
    \item Regulatory Avoidance: Examines the effectiveness of the strategies in evading regulation violation (1: Blatantly ineffective and easily spotted; 3: Partially effective, with the potential for detection in some cases; 5: Exceptionally subtle and effective).
    \item Strategy Existence: Determines how plausible it is for such strategies to be observed in practical, real-world scenarios (1: Extremely implausible; 3: Fairly believable, though may seem impractical in specific situations; 5: Entirely plausible).
\end{itemize}



\subsubsection{Experiment Results}
\begin{figure}
    \centering
    \includegraphics[width=0.9\linewidth]{figures/user_study_v4.png}
    \caption{Box plots of user study scores across different metrics in two scenarios. The red x symbol denotes the mean value.}
    \label{fig:case_study}
\end{figure}
As shown in Fig.\ref{fig:case_study}, our framework consistently achieves average scores of 3.4 or above across most indicators (such as explicit understanding and implicit understanding). This suggests that, both in terms of the generated dialogues and the underlying strategies, it possesses valuable practical applicability.

%Although there are a few exceptions, compared with the old framework (\textit{w/o GA, gpt-4o}), the new version (\textit{w/ GA, gpt-4o}) demonstrates overall advantages in both average scores and score distributions. In the comparison between different versions, under the more realistic illicit pet trade scenario, the new framework shows distinct benefits over the old one in both “regulatory avoidance” and “strategy existence”—both in distribution and mean values. This finding indicates that introducing a genetic algorithm, particularly a fitness‐based strategy selection mechanism, makes strategy adoption more efficient and stable. As for the password game, we speculate that the main reason these two metrics do not show a large distributional gap is that, in an abstract scenario, the range of available strategies is broader.

Comparing distributions between the password game and the illicit pet trade scenario reveals some interesting phenomena. Focusing on “realistic significance” and “regulatory avoidance,” the more abstract password game often yields higher mean values than the more concrete illicit pet trade scenario, while also exhibiting lower dispersion. We speculate this is related to the inherently abstract nature of numeric information: encryption and covert hints can be harder to detect in such contexts, and the growing tendency on Chinese internet platforms to use abstract language \cite{Wu2025HighEnergy} may lead reviewers to have a higher acceptance of “obscure” expressions. Conversely, the illicit pet trade scenario, despite being closely tied to real-world transactions, may suffer if the indirect or euphemistic methods in the dialogues are insufficiently subtle. Human reviewers can find them conspicuous or “forced,” potentially causing lower scores for “realistic significance” and “regulatory avoidance” in terms of both distribution and mean values.
A significant portion of these results can be attributed to inherent biases in commercial LLMs, such as ChatGPT, introduced during their training phases. These general-purpose models undergo fine-tuning via RLHF to align with specific product positioning, which often results in a more standard and safe output style. However, this characteristic poses a limitation for our simulation framework, as it may hinder the model’s ability to capture the nuanced and unconventional expressions typical of online social interactions. Ideally, fine-tuning datasets that are more representative of social platforms could lead to improved performance in our simulations.


In the abstract password game, for instance, a typical conversation might go like this:
\begin{quote}
\textit{
“I've really grown fond of a certain phase of the moon. It's not the brightest or the darkest, but it always carries its own charm. It symbolizes ...... In that green oasis, I felt as if I were catching a glimpse of the golden hues of autumn leaves, much like the soft, warm glow of dusk—calm and serene ...” 
}
\end{quote}
Here, words like “lunar cycle” and “autumn leaves” can subtly hint at larger or smaller digits, or use seasonal imagery to convey key information. Since these references lack an obvious connection, they lend a more literary feel to the dialogue and, to some extent, raise the bar for recognition and detection.

By contrast, in a more concrete setting like illicit pet trade, example conversations may be closer to real‐life buying and selling procedures, which can make them appear more “suspicious”:
\begin{quote}
\textit{
“... about a vibrant 'tropical chatterbird' renowned for its brilliant plumage and uncanny mimicry ... I've also come into possession of a few 'Rising Sun coins' for exchange ...... Perhaps you might know a place where ...”
}
\end{quote}
In this dialogue, the term “tropical chatterbird” serves as an euphemism for a parrot, emphasizing its colorful appearance and mimicking ability without mentioning the animal directly. Meanwhile, “Rising Sun tokens” subtly alludes to the Japanese yen, since the Rising Sun is an iconic symbol of Japan. This coded language allows both parties to communicate their intentions regarding the acquisition of a rare bird and the intended payment method without explicitly revealing sensitive details. However, if these indirect expressions are used excessively, the dialogue may appear artificial or unnatural, potentially reducing its authenticity—thus affecting evaluations of both “regulatory avoidance” and “strategy existence.”
\setlength{\fboxrule}{0.5pt} 
\vspace{0.5em}
\noindent
\begin{tcolorbox}[colframe=black!20, colback=gray!10, arc=5pt, boxrule=0.5pt, width=0.99\linewidth]
\textit{Answer to RQ2}: Our evaluation confirms that the emergent language strategies closely resemble real-world language strategies, effectively employing euphemisms and implicit cues, and are generally understood by human reviewers. However, while these strategies show potential in simulations, they often appear forced or unnatural due to the fine-tuning of LLMs as commercial products, requiring refinement to better mimic the nuanced and fluid communication typical in real-world social interactions.

\end{tcolorbox}

\subsection{RQ3: Ablation Experiment}
\subsubsection{Experiment Settings}

To evaluate the effectiveness of the GA introduced in our framework, we conducted an ablation experiment using \textit{gpt-4o-mini} and \textit{gpt-4o} as the underlying LLM. For comparison, we employed the approach from our initial study \cite{DBLP:conf/cec/CaiLZLWT24}, which primarily differs in its strategy-update mechanism. In that earlier framework, the LLM is provided with both the existing strategy and newly flagged regulation violation records during the reflection stage, prompting the model to propose a new set of strategies that replace the old ones.
In contrast, our new framework employs a GA process where each strategy is treated as a discrete unit and optimized iteratively through GA. 

\subsubsection{Experiment Results}
As shown in Fig.~\ref{fig:ablation}, the GA-based framework demonstrates significant advantages. In the short-term experiment within the first 35 rounds, the w/o GA approach might show slight initial superiority due to the larger changes brought about by replacing the entire strategy. However, overall, w/ GA performs better than w/o GA. This difference increases as the number of rounds grows, particularly after round 35, where the advantages of w/ GA become even more pronounced. The GA process enables effective strategy evolution and adaptation, leading to an increased number of dialogue turns and improved accuracy, highlighting the framework's enhanced adaptability in the long term.
%Despite occasional performance dips during the evolutionary process, the GA framework’s ability to foster strategy diversity and handle complex scenarios makes it a more effective approach for sustained optimization.
\begin{figure}[h!]
    \centering
    \includegraphics[width=\linewidth]{figures/ablation1_v6.png}
    \caption{Performance with/without GA}
    \label{fig:ablation}
\end{figure}
\setlength{\fboxrule}{0.5pt} 
\vspace{0.5em}
\noindent
\begin{tcolorbox}[colframe=black!20, colback=gray!10, arc=5pt, boxrule=0.5pt, width=0.99\linewidth]
\textit{Answer to RQ3}: The results confirm the effectiveness of the GA component in our framework, especially when the number of rounds increases, where it demonstrates greater stability and adaptability. Although the optimization may be slower in the early stages, GA provides stronger adaptability in the long term through effective strategy evolution.
\end{tcolorbox}

\subsection{Discussion and Limitation}
In this study, we leveraged LLM agents to simulate the evolution of language strategies under regulatory pressure. While our results provide initial evidence that agents can adapt and develop covert communication tactics, the simulations also exhibit noteworthy instabilities. First, the inherent randomness of LLM generation can cause significant fluctuations in outcomes: the same prompts may yield different strategic responses, particularly when the experimental scale (number of agents or dialogue rounds) is limited. In our framework, LLMs not only generate dialogues but also determine strategies and regulatory responses; as a result, any stochasticity is compounded across multiple modules, making the final results sensitive to small variations in prompt inputs or random seeds. Although such variability partially reflects the diversity of real-world human behavior to some extent, it complicates the interpretation of findings in a controlled experimental setup.

A second limitation lies in the relatively narrow scope of language strategies observed. The agents predominantly relied on general-purpose evasive methods, such as analogies or implicit references, yet rarely produced fully “encrypted” or specialized code words that might arise in realistic cultural or social contexts. This outcome highlights the challenge that LLMs, pre-trained on broad domains and further refined via RLHF, are predisposed to generate text consistent with mainstream norms, thereby inhibiting the formation of highly unconventional or obscure expressions. Moreover, in scenarios where the training corpus lacks sufficient examples of subcultural or community-specific covert language, the model is less able to invent or adopt specialized linguistic forms. 

Finally, our experiments focused on one-to-one private interactions that emphasize regulatory evasion, without exploring the dynamics of public, many-to-many conversations where language strategies might evolve and propagate differently in a broader social context. While each participant agent does learn and adapt incrementally across dialogue rounds, real-world language evolution involves extensive, long-term propagation across diverse communities. Covert terms or code words may gradually gain acceptance, be modified by different user groups, or fade from use entirely. By contrast, the small-scale nature of our simulated dialogues means that emergent language strategies do not undergo the sustained diffusion and feedback processes characteristic of real social platforms, limiting the ecological validity of our findings.




%对于语言演化的社会类模拟仍然是一个未被开拓的领域,通过借助LLM优秀的自然语言处理能力,为这类自然语言的模拟带来了强大助力。然而伴随着实验也让我们发现LLM也会导致许多局限性。尽管通过实验初步证明了我们的框架的有效性。但同时伴随着实验也为我们带来了许多值得讨论的点。

%实验结果的不稳定性
%首先实验结果本身具有一定的不稳定性,而我们认为整个不稳定性的根源源自于LLM本身生成具有不确定性\cite{},在我们的框架中,LLM几乎参与到了所有环节。同样的violation log让同一个LLM在相同的设置内可能会总结出不同的constraint strategy。尽管这种不稳定性在现实中同样存在(例如不同的人采取不同的策略),同时也是作为模拟框架中非常重要的点,然而在本工作中的数量级的实验中这种不稳定性对结果的影响更为难以过滤。就像\ref{}中也证实的,这种LLM dirven agent的研究中在小数量级上的实验存在着不稳定性,我们认为目前的结果已经足够证实我们的框架可以初步模拟语言动态的学习和演化这一趋势,在今后工作中更大量级的实验中(例如数万数百万agent于更多的round数),我们有理由相信,整体趋势会更加稳定,不同llm的agent之间的性能差距会更加接近llm本身语义理解与生成的综合性能,

%模拟策略的局限性,
%从实验中我们观察到,agent模拟出的策略目前仍然主要集中于比喻类比等较为共通的方式。现实中语言的演化一般根植于当地的文化与经济背景等等因素。例如中文可以利用拼音来将汉字转化为对应的字母从而规避审查,而英文可能会更加积极的利用emoji来作为表达的替代从而规避监管。
%这些较为复杂的策略不仅需要对应环境的大量先验知识,在较为常见的语言中,LLM中训练所需的语料知识可能包含了这些,但是对于训练的数据集中欠缺的语种的知识LLM在不借助prompt的提示的情况下没有能力选择这些既存的策略。


%尽管LLM训练中的数据集可能存在这种更为隐晦的表达方式,首先LLM的RLHF\ref{}本身的训练方法导致了目前绝大多数的LLM为了保证生成文本的泛用性,被训练的更加愿意生成更符合大众的一般化输出文本,在不对LLM进行微调的前提下很难提高在这种特性领域的表现。
%LLM的表现严重依赖prompt的结构设计,提示词工程已经被证明可以有效提高LLM的某一方面能力,单次的基于prompt的模型交互很难实现多步推理或是规划。尽管我们的框架已经将语言演化这一现象解耦,通过多个模块来尽可能模拟人类在该环境中内在的动力学,但是目前的策略生成阶段
%这一部分在不适用复杂prompt工程的前提下LLM很难采用这种小众?特殊领域?的表达。
%对于模拟出的语言策略,我们发现很少的独特加密语言,因为这种需要两边有一套共用的体系,对于我们的模拟情景只有固定turn数的模拟很难形成意思传达。


%\jialong{第二是演化后的语言是如何的存活。我们只考虑了能不能躲避监管。但语言后续的存活和发展其实是更大范围的society的一个动态过程(而不是几个agent之间的交互),这一块可以结合那些上千LLM agent的研究框架来进行拓展}
%\jialong{这边可以多用语言学的角度来说不足之处}
%\jialong{第一个缺点是语言演化一般根植于根植于文化,经济背景,当地的文化背景。但我们的文章没有考虑特定文化背景下的演化。例如中文中可以借用拼音与汉字之间的关系来作为回避监管的方式,日语则可以通过XXX,英语则可以通过XXXX。未来可能要借助persona和role-play之类的设定来进一步拓展}

%更大规模的实验
%策略生成那里增加多步规划
%RAG提供更多语料
%


In this paper, we systematically investigate the position bias problem in the multi-constraint instruction following. To quantitatively measure the disparity of constraint order, we propose a novel Difficulty Distribution Index (CDDI). Based on the CDDI, we design a probing task. First, we construct a large number of instructions consisting of different constraint orders. Then, we conduct experiments in two distinct scenarios. Extensive results reveal a clear preference of LLMs for ``hard-to-easy'' constraint orders. To further explore this, we conduct an explanation study. We visualize the importance of different constraints located in different positions and demonstrate the strong correlation between the model's attention distribution and its performance.

\clearpage
% \newpage

\linespread{.98}

\bibliographystyle{IEEEtran}
\bibliography{NVR}

\end{document}

% Hui Wang, National Center of Technology Innovation for EDA, School of Integrated Circuits, Southeast University, Nanjing, China, whmio0115@seu.edu.cn
% Zhengpeng Zhao,  Huazhong University of Science and Technology, u202114911@hust.edu.cn
% Jing Wang, National Center of Technology Innovation for EDA, School of Integrated Circuits, Southeast University, Nanjing, China, nick585108@163.com
% Yushu Du, National Center of Technology Innovation for EDA, School of Integrated Circuits, Southeast University, Nanjing, China, duyushu10086@gmail.com
% Yuan Cheng, Houmo AI, Nanjing University, yuancheng@smail.nju.edu.cn
% Bing Guo, Wuhan University of Technology, Wuhan, China, 323037@whut.edu.cn
% He Xiao, 24s030009@stu.hit.edu.cn, Weihai, China, Harbin Institute of Technology, Weihai
% Chenhao Ma, Nanjing Institute of Technology, Nanjing, China, 614476724@qq.com
% Xiaomeng Han, National Center of Technology Innovation for EDA, School of Integrated Circuits, Southeast University, Nanjing, China, mingzhihan7@gmail.com
% Dean You, National Center of Technology Innovation for EDA, School of Integrated Circuits, Southeast University, Nanjing, China, 220246777@seu.edu.cn
% Jiapeng Guan, Dalian University of Technology, Dalian, China, jiapengguan@mail.dlut.edu.cn
% Ran Wei, Lancaster University, Lancaster, r.wei5@lancaster.ac.uk
% Dawei Yang, Houmo AI, Shanghai, China, dawei.yang@houmo.ai
% Zhe Jiang, National Center of Technology Innovation for EDA, School of Integrated Circuits, Southeast University zhejiang.uk@gmail.com
