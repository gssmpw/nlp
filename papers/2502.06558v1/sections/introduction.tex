This work studies the \emph{online open/closed unit-length interval coloring} problem, which is a variant of \emph{online interval coloring} problem.
% Open and Closed variant
The input consists of a sequence of unit-length intervals, where each interval is either an \emph{open} interval $(r, r+1) = \{ i \mid r < i < r+1 \}$ or a \emph{closed} interval $[r, r+1] = \{ i \mid r \leq i \leq r+1 \}$.
The intervals are released to the algorithm one by one.
Once an interval is released, the online algorithm has to irrevocably assign the interval a color that is not assigned to any previously released interval that overlaps with the released interval.
The aim is to minimize the colors used.
In this work, we study the performance of the \ff algorithm for coloring the open/closed unit-length intervals.

As a fundamental problem in graph theory and scheduling areas, intensive research is being conducted on the online interval coloring problem.
The optimal online algorithm for the general case where intervals have arbitrary lengths uses at most $3\omega - 2$ colors where $\omega$ is the optimal number of colors~\cite{DBLP:journals/ita/ChrobakS88, kierstead1981extremal, DBLP:conf/mfcs/Slusarek89}.
Recently, interest in coloring bounded-length intervals arose, and better algorithms were found for special families of instances. 
For intervals with the length within some fixed range between $1$ and $\sigma \geq 1$, a $(\sigma + 1)$-competitive algorithm was proposed~\cite{DBLP:journals/ejc/ChybowskaSokol24}, which improved the performance of online algorithms for the cases where $\sigma<2$.

On the other hand, the \ff algorithm, a classical greedy algorithm, is interesting for its elegance. 
The \ff algorithm is currently the best algorithm for (closed) unit-length intervals, which uses $2\omega-1$ colors for any $\omega$-colorable set of unit-length intervals~\cite{DBLP:journals/ita/ChrobakS88, DBLP:conf/icalp/EpsteinL05}.
For arbitrary-length intervals, the competitive ratio is between $5$ and $8$~\cite{DBLP:journals/ejc/KiersteadST16, DBLP:journals/order/NarayanaswamyB08}, and it is open for a long time what the actual competitive ratio is.

%% Why we are interested in Open and Closed variant 
The motivation to look into the performance of the \ff algorithm on open/closed unit-length intervals coloring is two-fold.
First, the performance of the \ff algorithm is not fully understood. 
It is known that for unit-length closed intervals, the \ff algorithm is exactly $(2-\frac{1}{\omega})$-competitive, where $\omega$ is the optimal number of colors needed. However, for general instances, the \ff algorithm is at least $5$-competitive.
The performance of the \ff algorithm is unclear for the instances between these cases. 
As the open/closed unit-length intervals case is the smallest possible instance of the coloring problem on bounded length intervals, this case is key to knowing the actual competitive ratio of the \ff algorithm.

Moreover, this research helps us understand the impact of ``$\varepsilon$-uncertainty''.
It can be considered that closed unit-length intervals are $\varepsilon \approx 0$ longer than the open unit-length intervals. 
This difference of $\varepsilon$ increases the complexity of the problem as an open interval can be a proper subset of a closed interval, which was previously not the case when only closed unit-length intervals were considered.
It was shown that the \ff algorithm is at least $2$-competitive~\cite{Curbelo}, meaning that the open/closed unit-length intervals case is strictly ``harder'' for the \ff algorithm than the closed unit-length intervals case. 
Finding the competitive ratio of the \ff algorithm on open/closed unit-length intervals coloring quantifies the power of the uncertainty of $\varepsilon$-difference by understanding how much the adversary can use this uncertainty to trap the \ff algorithm.

% Previous work
% Optimal algorithm
\paragraph*{Related work} For the online interval coloring problem, Kierstead and Trotter designed an optimal online algorithm that uses at most $3\omega - 2$ colors on an $\omega$-colorable interval graph~\cite{kierstead1981extremal}.
Independently, Chrobak and \'{S}lusarek found the same upper and lower bounds~\cite{DBLP:journals/ita/ChrobakS88, DBLP:conf/mfcs/Slusarek89}.

% Bounded length
Recently, interest in coloring bounded-length intervals arose.
Chybowska-Sok\'{o}\l{} et al.~\cite{DBLP:journals/ejc/ChybowskaSokol24} studied the online interval coloring problem where the size of the intervals is within some fixed range $[1, \sigma_{\geq 1}]$.
When $\sigma = \infty$, it is the previously mentioned general case, and when $\sigma = 1$, it is the unit-length case.
They proposed a $(\sigma + 1)$-competitive algorithm, which improved the performance of the online algorithm for $1<\sigma<2$.
% New result Curbelo
Curbelo~\cite{DBLP:journals/corr/abs-2401-05648} then shows that, in the setting where the interval representation is unknown for the algorithm, for any $\varepsilon > 0$, there exists an $\sigma > 1$, such that any algorithm is at least $(3 - \varepsilon)$-competitive. Matching the upper bound from Kierstead and Trotter for coloring intervals of general length.

%FirstFit
\runtitle{\ff.}
On the other hand, despite the aforementioned results that included sophisticated algorithms, people are also interested in the performance of the naive but easy-to-implement \ff{} algorithm.
The exact competitive ratio of the \ff{} algorithm is a long standing open problem.
As early as 1976, Witsenhausen~\cite{DBLP:journals/jct/Witsenhausen76}, and independently, in 1988, Chrobak and \'{S}lusarek~\cite{DBLP:journals/ita/ChrobakS88}, proved that \ff{} is at least 4 competitive.
Later, \'{S}lusarek~\cite{DBLP:conf/mfcs/Slusarek89} improved this bound to 4.45.
The best lower bound known to date is from Kierstead et al.~\cite{DBLP:journals/ejc/KiersteadST16}.
They prove that for every $\varepsilon > 0$, there exist an input such that the \ff{} algorithm uses strictly more than $(5 - \varepsilon)\omega$ colors.

For the upper bounds of the competitive ratio of $\ff$, Kierstead~\cite{DBLP:journals/siamdm/Kierstead88} first proved a constant competitive of $40$ in 1988.
Later, this result was improved to $25.72$ by Kierstead and Qin~\cite{DBLP:journals/dm/KiersteadQ95}.
The next breakthrough was in 2003, when Pemmaraju et al.~\cite{DBLP:journals/talg/PemmarajuRV11} showed that the \ff{} algorithm is at most $10$-competitive.
The current best analysis of $\ff$ was by Narayanaswamy and Subhash Babu~\cite{DBLP:journals/order/NarayanaswamyB08}, who showed that \ff{} uses at most $8\omega - 3$ colors on a set of $\omega$-colorable intervals.

% Unit interval coloring
The performance of the \ff{} algorithm has also been of interest in the special cases.
Chybowska-Sok\'{o}\l{} et al.~\cite{DBLP:journals/ejc/ChybowskaSokol24} studied the \ff{} algorithm where the size of the intervals is within some fixed range $[1, \sigma_{\geq 1}]$.
The authors first show a trivial bound of $\lceil \sigma + 1 \rceil\omega$ on the number of colors \ff uses, where $\omega$ is the optimal coloring number.

In the case where only unit-length (closed) intervals are considered, Chrobak and \'{S}lusarek~\cite{DBLP:journals/ita/ChrobakS88}, and independently also Epstein and Levy~\cite{DBLP:conf/icalp/EpsteinL05} proved that the \ff{} algorithm uses exactly $2\omega - 1$ colors for a set of $\omega$-colorable unit-length intervals.

In 2022, instead of only closed intervals as is typically considered, Bir\'{o} and Curbelo proved that \ff{} uses at least $2\omega$ and at most $3\omega - 3$ colors when the optimal number of colors is $\omega$~\cite[Theorem 6.0.1]{Curbelo}.
Furthermore, Curbelo claimed that \ff{} uses at most $3\omega - f(\omega)$ colors for a non-specified function $f$, which increases with $\omega$.
In this paper, we find a better bound for this case.

\runtitle{\ff and counting.}
The analysis of the \ff{} algorithm is generally done via careful counting intersections a critical interval can have.
The elegance of the analysis lies in determining from which interval to count its number of intersecting intervals. 
The first analysis by Kierstead and Qin~\cite{DBLP:journals/siamdm/Kierstead88, DBLP:journals/dm/KiersteadQ95} identify a large set of mutual-intersecting intervals and then carefully count structural properties of the graph, such as the number of overlapping intervals.
The later results by Pemmaraju et al.~\cite{DBLP:journals/talg/PemmarajuRV11} and by Narayanaswamy and Subhash Babu~\cite{DBLP:journals/order/NarayanaswamyB08} first represent the overlapping structure of intervals using ``columns''.
The columns are assigned different labels that encrypt the information of the intervals structure and colors assigned.
Then, the analysis carefully counts the occurrences of each label to bound the total number of colors used by the \ff{} algorithm.
% Unit interval coloring
Also, for unit-length (closed) intervals, the analysis carefully counts for each interval at what positions an interval could intersect it and how many of such intervals there could be~\cite{DBLP:journals/ita/ChrobakS88, DBLP:conf/icalp/EpsteinL05}.

% Our results
\paragraph*{Our contribution}
Following the traditional counting approach, we analyze the $\ff$ algorithm on open/closed unit-length intervals by counting how many intersections an interval can have.

Our main technical contribution is a generalization of the traditional neighborhood bound of the $\ff$ algorithm (Lemma~\ref{lem:prop_ff_equiv}). 
Informally, this lemma shows that the color assigned to interval $\x$ by \ff can be bounded via another interval intersecting $\x$ that is critical in the sense that it is a relatively high-colored interval but with relatively few intersections.
Our results are built on this newly-observed property of \ff. 
In a nutshell, for any interval $\x$, by the fact that the input instance consists only of unit-length open or closed intervals, our analysis identifies the critical interval intersecting $\x$. It shows that the number of intersections is bounded by sophisticated counting. 

For a special case where all intervals have integral endpoints, we prove that $\ff$ is $2$-competitive, which is tight as it matches the $\ff$ algorithm lower bound by Bir\'{o} and Curbelo~\cite{Curbelo}.
Formally, for any instance that can be colored by $\omega$ colors, we show that
\begin{theorem}
    \label{thm:integral}
    For any unit interval graph with open and closed intervals with integral endpoints, the \ff{} algorithm for online coloring uses at most $2\omega$ colors.
\end{theorem}

Then, for unit-length intervals with arbitrary endpoints, we prove the following.
\begin{theorem}
    \label{thm:any}
    For any unit interval graph with open and closed intervals with arbitrary endpoints, the \ff{} algorithm for online coloring uses at most $\result$ colors. 
\end{theorem}

\paragraph*{Paper organization}
Section~\ref{sec:prelim} defines the problem formally. We also introduce the key lemmas, the framework of our analysis, and important concepts that are used heavily throughout the paper.
As a starter, Section~\ref{section:integral_endpoints} shows the proof of $2$-competitiveness (Theorem~\ref{thm:integral}) of \ff on open/closed unit-length intervals with integral endpoints.
We also prove some interesting facts that hold in both the restricted and the more general setting where intervals have arbitrary endpoints.
Section~\ref{section:any_endpoints} then analyze the case where the unit-length intervals have arbitrary endpoints and prove Theorem~\ref{thm:any}.
Finally, we end in Section~\ref{section:conclusion} with concluding remarks.
Due to the page limit, we leave most of the proofs in Section~\ref{section:any_endpoints} to the full version in the appendix.