Given that the optimal coloring uses $\omega$ colors on $\orderedInput$. 
For any interval $\x \in \orderedInput$, we bound its color from \ff by using the Neighborhood bound (Lemma~\ref{lem:prop_ff}) and the Pivot bound (Lemma~\ref{lem:prop_ff_equiv}).
More specifically, we pick a pivot $\pivot$ with special structural property. 
Using the property, we bound the size of the corresponding $\pivotset$. 
On the other hand, we bound the color $\ffc(\pivot)$ using the Neighborhood bound. 
Finally, we use the Pivot bound to bound the color of $\x$ using the pivot $\pivot$.

Practically, we first identify the ``easy cases'' where the $\ffc(\x)$ is at most~$2\omega$ as desired.
Then, after peeling off these easy cases, we focus on the tough kernel of the analysis where there is no trivial choice for $\pivot$ such that the sum of $\ffc(\pivot)$ and $|\pivotset|$ is small.
For this tough kernel of analysis, we identify two mediocre choices for $\pivot$ for which the sum of $\ffc(\pivot)$ and $|\pivotset|$ cannot both be large at the same time.

\medskip

\runtitle{Intervals $\x$ with small $\nrows{3}{\x}$.}
These intervals have relatively small neighborhoods. Formally, \shortversion{}{by the Neighborhood bound (Lemma~\ref{lem:prop_ff}) and Lemma~\ref{lem:ris} (\ref{Obs:sumrrr}), we can show the following lemma.}
\begin{lemma}
    \label{lem:integral_open_and_closed_assumtion_not_hold}
    For each interval $\x \in \orderedInput$, 
    if $\nrows{3}{\x} < \nrows{1}{\x} + 2$, then $\ffc(\x) \leq 2\omega$.
\end{lemma}

\fullversion{
\begin{proof}
    By the Neighborhood bound (Lemma~\ref{lem:prop_ff}), $\ffc(\x) \leq 1 + |\nb_{\x}(\x)|\leq 1 + |\nb(\x)|$. 
    By Lemma~\ref{lem:ris} (\ref{Obs:sumrrr}), $\ffc(\x)\leq 1 + \nrows{1}{\x} + 2\nrows{2}{\x} + 3\nrows{3}{\x}$.
    The given condition $\nrows{3}{\x} < \nrows{1}{\x} + 2$ implies that $\ffc(\x) \leq 2 + 2\nrows{1}{\x} + 2\nrows{2}{\x} + 2\nrows{3}{\x}$. 
    Since $\nrows{0}{\x} \geq 1$, it follows that $\nrows{1}{\x} + \nrows{2}{\x} + \nrows{3}{\x} \leq \omega - 1$.
    Thus, $\ffc(\x) \leq 2\omega$.  
\end{proof}
}

Since any interval in $\rows{3}{\x}$ must be in $\lmr(\x)$, an empty $\lmr(\x)$ implies $\nrows{3}{\x} = 0$. 
Hence, we have a useful corollary:

\begin{corollary}
    \label{cor:y_exists}
    If $\lmr(\x) = \emptyset$, then $\ffc(\x) \leq 2\omega$.
\end{corollary}

\medskip

\runtitle{Open intervals.\footnote{This is a generalization from the results of Chrobak and {\'{S}}lusarek~\cite{DBLP:journals/ita/ChrobakS88} and Epstein and Levy~\cite{DBLP:conf/icalp/EpsteinL05}, which showed for closed unit-length intervals, \ff uses exactly $2\omega - 1$ colors.}}
By Lemma~\ref{lem:ris} (\ref{lem:open_intersects_2}), open intervals cannot have too many intersections per row. More specifically,

\begin{theorem}
    \label{lem:open_interval_small_intersection}
    For each open interval $\x \in \orderedInput$, $\ffc(\x) \leq 2\omega - 1$.
\end{theorem}

\fullversion{
\begin{proof}
    By Lemma~\ref{lem:ris} (\ref{lem:open_intersects_2}), an open interval $\x$ intersects at most $2$ intervals per row.
    Furthermore, since there are $\omega$ rows and $\x \in \rows{0}{\x}$, at most $\omega - 1$ rows remain to have any intervals intersecting $\x$.
    It follows that $|\nb_\x(\x)| \leq |\nb(\x)| \leq 2\omega - 2$.
    By the Neighborhood bound, $\ffc(\x) \leq 2\omega - 1$.
\end{proof}
}

\runtitle{Closed intervals $\x$ with large $\nrows{3}{\x}$.}
Intuitively, because the intervals in $\twins(\x)$ have an identical neighborhood to interval $\x$, these intervals are tricky to deal with.
%If it would be simple to bound the color of an interval in $\twins(\x)$, it should not be that much harder to bound the color of interval $\x$ itself.
This leaves us to consider the intervals in the sets $\lmr(\x)$ and $\nlmr(\x)$ as potential pivot intervals.
In the ideal scenario, by picking as a pivot the interval with the largest color in the sets $\lmr(\x)$ and $\nlmr(\x)$, $\pivotset \subseteq \twins(\x)$.
Using the Neighborhood bound, we bound the color of this largest interval.
However, it is not always possible to pick such a pivot such that $\pivotset \subseteq \twins(\x)$.
In the case that $\pivotset \not\subseteq \twins(\x)$, we look more closely at what additional intervals are part of this set.
Finally, we apply the Pivot bound in order to prove an upper bound on the color of interval $\x$.
\hide{
First, we use the Neighborhood bound to bound the colors of intervals in $\lmr(\x)$. In \bk{Lemma~\ref{lem:integral_uvw}}, we show that as these intervals have an aligned structure, they have relatively small neighborhoods.
Once we bound the color of an interval in $\lmr(\x)$, it can be used as a pivot in the Pivot bound.

\bk{Second, we look into intervals in the set $\nlmr(\x)$.}
The intervals in $\nlmr(\x)$ have misaligned positions with $\x$ but can have higher colors than the intervals in $\lmr(\x)$ and cause a higher color of $\x$.
\bk{Therefore, they can also be considered as a pivot.
However, it requires some extra effort to bound the colors of intervals in $\nlmr(\x)$.
In Lemma~\ref{lem:any_z}, we show that if some structural properties are met, the color of intervals in $\nlmr(\x)$ is small.}
However, \bk{in Lemma~\ref{lem:any_z}}, we show that the size of $\nlmr(\x)$ is small (otherwise, we get the desired bound directly). 
Then, by using intervals in $\lmr(\x)$ as a pivot, we show that $\pivotset \subseteq \nlmr(\x)$ and bound the color of $\x$ by the Pivot bound.
}
\medskip


In Sections~\ref{section:integral_endpoints} and~\ref{section:any_endpoints}, we will focus on the last case where $\x$ is a closed interval with large~$\nrows{3}{\x}$.

\hide{
We are going to prove upper bounds on the color of any interval $\x \in \orderedInput$.
By Lemma~\ref{lem:prop_ff}, the Neighborhood bound, the size of the neighborhood can be used as an upper bound.
Then, it follows that intervals with a small neighborhood, say less than $2\omega$, have inherently a color of at most $2\omega$.
Therefore, it is more interesting to look into intervals with large neighborhoods, i.e., intervals that intersect a large number of other intervals.

From the pigeonhole principle it follows that when an interval $\x$ has more than $2\omega$ intersections, there must exist at least one $\rows{3}{\x}$ row.
Furthermore, as will be shown in Lemma~\ref{lem:integral_open_and_closed_assumtion_not_hold}, it must be the case that the number of $\rows{3}{\x}$ rows is larger than the number of $\rows{1}{\x}$ rows.
And, of course, a larger number of $\rows{3}{\x}$ rows implies that $\x$ intersects a larger number of intervals.
This observation gives us the following paradox; although interval $\x$ intersects a large number of intervals, the number of colors used is not as many, which means that interval $\x$, despite intersecting a large number of intervals will be assigned a relatively small color.
This is due to the fact that the intervals in $\rows{3}{\x}$ rows are perfectly aligned, which means that on these $\rows{3}{\x}$ rows, the intervals in the $\rows{3}{\x}$ rows intersect only a single interval, i.e., the intervals identical to themselves.

\paragraph*{A better bound.}
This observation can be used to find a better upper bound for the color of interval $\x$.
Using the Neighborhood bound, a strict upper bound on the color of any interval in the sets $L(\x), M(\x)$ and $R(\x)$ can be computed.
We are interested in a bound on any of the intervals in these sets because this upper bound will be relatively low, as the neighborhood of any interval in these sets will be relatively small.
Then, we need to consider the intervals that were not in the sets $L(\x), M(\x)$ or $R(\x)$.
Any of these intervals might be assigned a color which is greater than the upper bound previously computed.
Using Lemma~\ref{lem:prop_ff_equiv}, the Pivot bound, with $\pivot$ being the interval in $L(\x), M(\x)$ or $R(\x)$ with the largest color, and with $\pivotset \subseteq \orderedInput \setminus \left( L(\x) \cup M(\x) \cup R(\x) \right)$, an upper bound on the color of interval $\x$ can be computed.
}

