
In this section, we consider the instance $\orderedInput$, where intervals have arbitrary endpoints. 
More specifically, by Lemma~\ref{lem:integral_open_and_closed_assumtion_not_hold} and Lemma~\ref{lem:open_interval_small_intersection}, we focus on $\x \in \orderedInput$ that are closed and $\nrows{3}{\x} \geq \nrows{1}{\x} + 2$.
Recall from Section~\ref{section:integral_endpoints} that we partition $\nb(\x)$ into $\twins(\x)$, $\lmr(\x)$, and $\nlmr(\x)$, that is, the set of twins, the set of intervals that are twins of intervals in $\rows{3}{\x}$, and the rest of intervals with endpoints ``misaligned'' with the endpoints of $\x$.

\shortversion{Primarily, we pick a pivot interval $\pivot$ and bound its color using the Neighborhood bound. 
The choice of pivot should provide a plausible way to bound the size of the corresponding~$\pivotset$.
However, applying the Pivot bound to bound the color $\ffc(\x)$ in this general case is more challenging than the integral-endpoints case due to the misaligned structure of intervals.
Directly applying the same technique as in the intergral-endpoints case results in a ratio of~$3\omega$ (see the full version).}
\fullversion{
Applying the Pivot bound to bound the color $\ffc(\x)$ in this general case is more challenging than the integral-endpoints case. 
First, unlike in the integral-endpoints case where all intervals that intersect $\x$ must be either in $\twins(\x)$ or in $\lmr(\x)$, the candidates in $\pivotset$ can also be in $\nlmr(\x)$. 
These intervals in $\nlmr(\x)$ can cross-interact with other intervals, making it difficult to bound the size of $\pivotset$.
Second, selecting a good pivot interval $\pivot$ in the general case is more challenging. }
\fullversion{We first show that using the same strategy as we used for the integral-endpoints case only guarantees a bound of $\ffc(\x) \leq 3\omega$.

% All intervals in R_1(x) and R_2(x) can be in set S
\paragraph*{A naive attempt}

As in the integral-endpoints case, let $\pivot$ be $\y$, the interval with the highest color in~$\lmr(\x)$. 
In the general case, any interval in the sets $\rows{1}{\x}$ and $\rows{2}{\x}$ can be in $\nlmr(\x)$ and can therefore be a candidate of $\pivotset$.
If we now apply the same strategy as in the integral-endpoints setting, it is no longer the case that the set $\pivotset$ is a subset of $\rows{1}{\x}$.
Then, in each row $\mathcal{R} \in \rows{2}{\x}$, it can be the case that two intervals (of which one intersects $\x$ and the other does not) are in the neighborhood of the pivot $\pivot$.
Furthermore, the remaining interval on row $\mathcal{R}$ that also intersects with $\x$ can be considered for the set $\pivotset$, resulting in us counting $3$ intervals per row in $\rows{2}{\x}$.
Thus, the bound tends to $3\omega$ when $\nrows{1}{\x}$ and $\nrows{3}{\x}$ are both small.
To work around this, we need to closely look at the color of the intervals in $\rows{2}{\x}$. 
}

% Define interval z
\fullversion{
\subsection{Roadmap}
Primarily, we pick a pivot interval $\pivot$ and bound its color using the Neighborhood bound. 
The choice of pivot should provide a plausible way to bound the size of the corresponding $\pivotset$. }

In the general case, we have another possible candidate of $\pivot$.
Symmetric to the dominating aligned interval $\y$, we define interval $\z$ as the dominating interval in rows in $\rows{2}{\x}$ that is misaligned with $\x$ and assigned the highest color by $\ff$.
Formally,

\begin{definition}
    \shortversion{\label{def:z_short}}
    \fullversion{\label{def:z}}
    \emph{\textbf{(Dominating interval $\z$ in $\rows{2}{\x}\cap\nlmr(\x)$)}}
    Let interval $\z$ be the interval in $\rows{2}{\x}\cap\nlmr(\x)$ with the highest color by $\ff$. That is, for all $I^\prime \in \rows{2}{\x}$, $\ffc(I^\prime) \leq \ffc(\z)$.
\end{definition}

A special property of the selection of $\z$ is that since $\z \in \nlmr(\x)$, any row in $\rows{0}{\x}, \rows{1}{\x}$ or $\rows{3}{\x}$, cannot be in $\rows{3}{\z}$\fullversion{ (Lemma~\ref{lem:R3z_in_R2x_short})}.
Therefore, all rows in $\rows{3}{\z}$ must also be in $\rows{2}{\x}$.

\shortversion{
\begin{lemma}
    \label{lem:R3z_in_R2x_short}
    $\rows{3}{\z} \subseteq \rows{2}{\x}$.
    \end{lemma}
}

% Case distinction
In the general case, we pick one of $\y$ and $\z$ as the pivot and use the Pivot bound to bound the color $\ffc(\x)$.
More specifically, we consider the following two cases.

% CASE 1: c(\y) >= c(\z)

\shortversion{
\paragraph*{Case (1): $\ffc(\z) \leq \ffc(\y)$}
}
\fullversion{
\subparagraph*{(1)}
$\ffc(\z) \leq \ffc(\y)$.}

In this case, the misaligned intervals in $\rows{2}{\x}$ are not in $\pivotset$ since they all have colors of at most $\ffc(\z) \leq \ffc(\y)$.
Using $\y$ as the pivot and following a strategy\shortversion{: }
\fullversion{similar to the case of integral-endpoints, we can show that $\ffc(\x) \leq 2\omega$ (Lemma~\ref{lem:c_u>=c_z}).}

\shortversion{

\smallskip

\begin{lemma}
    \label{lem:c_u>=c_z_short}
    For any interval $\x \in \orderedInput$, if $\ffc(\y) \geq \ffc(\z)$, then $\ffc(\x) \leq 2\omega$.
\end{lemma}
}


% CASE 2: c(\y) < c(\y)
\shortversion{\paragraph*{Case (2): $\ffc(\z) > \ffc(\y)$}}
\fullversion{\subparagraph*{(2)}
% z in S
$\ffc(\z) > \ffc(\y)$.}

% What is interesting about z
In this case, applying the Pivot bound with $\pivot = \y$ leads to a bound of $\ffc(\x) \leq 3 \omega$ as shown in the naive attempt.
To be precise, it is no longer true that $\pivotset \subseteq \rows{1}{\x}$, and any interval in $\rows{2}{\x}$ with a color greater than $\ffc(\y)$ is also a candidate of $\pivotset$.

% Why we use Z
An alternative strategy is to pick the interval $\z$ as the pivot in this case.
However, in the general case, the intervals in $\rows{1}{\z}$ are not particularly structured,
More precisely, we do not know which intervals also intersect with $\x$, and the neighborhood of $\z$ can be huge. 
When~$\z$ has a large number of intersections, using the Neighborhood bound to bound the color $\ffc(\z)$ can lead to a high bound and further lead to a high bound of $\ffc(\x)$ by the Pivot bound.
To deal with this situation, we first identify the rows containing intervals in $\rows{1}{\z}$ that we know their locations, namely, the intervals that are twins with $\z$.

\smallskip

\begin{definition}
    \shortversion{\label{def:Z_short}}
    \fullversion{\label{def:Z}}
    Let the set of rows in $\rows{2}{\x}$ with intervals identical to interval $\z$ be 
    \[ \Z (\x ) = \{ \text{row } \mathcal{R} \in \rows{2}{\x} \mid \mathcal{R} \text{ contains at least one interval that is in } \twins(\z)  \}.\]
\end{definition}

\shortversion{
    In combination with the set $\Z(\x)$ we are interested in the set $\rows{3}{\z}$.       
    Similar to the definition of $\alp$, we define fractions of specific $\rows{2}{\x}$ rows, that it, $\gam$ and $\delt$:
    
    \begin{definition}
    \label{def:gamma_delta_short}
    Given $\z \in \rows{2}{\x}$, 
        \begin{enumerate}[(a)]
            \item $\bet \in [0,1]$ denotes the fraction of rows $\mathcal{R} \in \rows{2}{\x}$ where the interval $I^\prime$ in $\mathcal{R} \cap \nb(\x)$ is in $\Z(\x)$, and
            \item $\gam \in [0,1]$ denotes the fraction of rows $\mathcal{R} \in \rows{2}{\x}$ where the interval $I^\prime$ in $\mathcal{R} \cap \nb(\x)$ is in $\rows{3}{\z}$.
        \end{enumerate}
    \end{definition}
}

% CASE 2.a: |\Z(x)| >= |Z bar(x)|

Note that $\Z(\x) \subseteq \rows{1}{\z}$, and thus $\nrows{3}{\z} \leq |\Z(\x)|$ implies that $\nrows{3}{\z} \leq \nrows{1}{\z}$. On the contrary, $\z$ potentially has a large neighborhood if $\nrows{3}{\z} > |\Z(\x)|$.
According to the size of $|\Z(\x)|$, we consider the following cases \textbf{2.a} and \textbf{2.b}.

\shortversion{\paragraph*{Case (2.a): $\nrows{3}{\z} \leq |\Z(\x)|$}}
\fullversion{\subparagraph*{(2.a)}
$\nrows{3}{\z} \leq |\Z(\x)|$.}

In this case, we pick $\z$ as the pivot $\pivot$.
By the selection of $\z$ and $\ffc(\z) > \ffc(\y)$, the only intervals that can contribute to $\pivotset$ are those in the set $\rows{1}{\x}$.
Furthermore, by $|\Z(\x)| \geq \nrows{3}{\z}$, the neighborhood of $\z$ is small, which enables us to use the Neighborhood bound to prove an upper bound the color of interval $\z$\shortversion{:} \fullversion{(Lemma~\ref{lem:any_z}).}
\shortversion{

\smallskip

\begin{lemma}
    \label{lem:any_z_short}
    The color of interval $\z$ is at most $c(\z) \leq \omega + \alp(\z) \cdot \nrows{1}{\x} - \bet  \cdot \nrows{2}{\x} + \gam  \cdot \nrows{2}{\x}  + \nrows{2}{\x} + \nrows{3}{\x}$. 
\end{lemma}

By selecting $\z$ as the pivot $\pivot$, the following bound of $|\pivotset|$ can be shown by the Neighborhood bound together with bounding the intersection of intervals in $\nlmr(\x)$.
\begin{lemma}
    \label{lem:pivotset_z_short}
    When $\ffc(\z) > \ffc(\y)$, by selecting $\z$ as pivot $\pivot$, $ |\pivotset| = (1 - \alp(\z)) \cdot \nrows{1}{\x}$.
\end{lemma}
}
Then, by the Pivot bound\shortversion{:} \fullversion{($\ffc(\x)$ is bounded by $2\omega$ Lemma~\ref{lem:gamma<delta}).}
\shortversion{
\begin{lemma}
    \shortversion{\label{lem:gamma<delta_short}}
    If $|\Z(\x)| \geq \nrows{3}{\z}$, then $\ffc(\x) \leq 2\omega$.
\end{lemma}
}


% CASE 2.b: |\Z(x)| < |Z bar(x)|
\shortversion{
\paragraph*{Case (2.b): $\nrows{3}{\z} > |\Z(\x)|$}
}
\fullversion{\subparagraph*{(2.b)}
$\nrows{3}{\z} > |\Z(\x)|$.}


This is the most technical part of our analysis.
In this case, $\z$ may intersect more than~$2$ intervals per row on average and have a large neighborhood, and using the Neighborhood bound on $\ffc(\z)$ can lead to a large bound.
We apply our framework recursively on $\z$ by first showing that the intervals in $\rows{3}{\z}$ rows do not intersect many intervals when $\nrows{3}{\z}$ is large\fullversion{ (Lemma~\ref{lem:any_z_bar})}.
Specifically, we let $\zbar$ be the highest colored interval in $\rows{3}{\z}$ that intersects with interval $x$.

\smallskip

\begin{definition}
    \shortversion{\label{def:z_bar_short}}
    \fullversion{\label{def:z_bar}}
    Let interval $\zbar \in \rows{3}{\z}$ be the dominating interval that has the highest color such that there exists no $I^\prime \in \rows{3}{\z}$ that intersects $\x$ such that $\ffc(I^\prime) > \ffc(\zbar)$.\footnote{Recall that we say an interval $I^\prime$ is in $\rows{i}{\x}$ if it is in $\nb(\x)$ and is colored by one of the colors in $\rows{i}{\x}$ by the optimal solution.}
\end{definition}

It is critical to use $\y$, $\z$, or $\zbar$ as the pivot according to whether $\ffc(\zbar)$ is larger than $\ffc(\y)$.

% CASE 2.b.i: c(z bar) >= c(\y)
\shortversion{
\paragraph*{Case (2.b.ii): $\ffc(\zbar) \geq \ffc(\y)$}
}
\fullversion{
\subparagraph*{(2.b.i)}
$\ffc(\zbar) \geq \ffc(\y)$.}

In this case, we apply the Pivot bound by selecting $\zbar$ as the pivot $\pivot$.
Recall that~$\y$ is the dominating interval in $\rows{3}{\x}$.
\shortversion{
Observe that by the definition of interval $\z$, $\ffc(\zbar) < \ffc(\z)$.
Hence, analogous to parameter $\alp$, we define $\delt$.

\smallskip

\begin{definition}
    \label{def:delta_short}
    Given any interval $\hat{I}$, $\delt(\hat{I}) \in [0, 1]$ denotes the fraction of rows $\mathcal{R} \in \rows{2}{\x} \setminus \rows{3}{\z}$ where interval $I^\prime$ in $\mathcal{R} \cap \nb(\x) \cap \nb(\z)$ satisfies $\ffc(I^\prime) \leq \ffc(\hat{I})$.
\end{definition}
}
The condition $\ffc(\zbar) \geq \ffc(\y)$ and the selection of $\zbar$ ensure that no interval in $\rows{3}{\x}$ can be in $\pivotset$.
Furthermore, by the definition of $\zbar$, no interval in $\rows{3}{\z}$ can contribute to $\pivotset$.
Therefore, $\pivotset$ is a subset of intervals in $\rows{1}{\x}$ and $\rows{2}{\x}\setminus \rows{3}{\z}$.

\shortversion{
\begin{lemma}
    \label{lem:pivotset_zbar_short}
    By selecting $\zbar$ as pivot $\pivot$, $|\pivotset| = (1-\alp(\zbar)) \cdot \nrows{1}{\x} + (1 - \delt(\zbar) \cdot (1 - \gam) \cdot 2\nrows{2}{\x}$.
\end{lemma}
}

The definition of $\rows{3}{\z}$ and the selection of $\zbar$ guarantee that a row in the set $\rows{3}{\zbar}$ must contain an interval identical to the interval $\z$ \fullversion{(see Figure~\ref{fig:y_intersections})}.
Thus, $\rows{3}{\zbar} \subseteq \Z(\x)$, and $\nrows{3}{\zbar} \leq |\Z(\x)|$.
Moreover, since $\twins(\zbar) \subseteq \rows{1}{\zbar}$, and every $\rows{3}{\z}$ contains a twin of $\zbar$, $\nrows{1}{\zbar} \geq \nrows{3}{\z}$. 
Together with the condition $\nrows{3}{\z} > |\Z(\x)|$, it guarantees that 
$\nrows{1}{\zbar} \geq \nrows{3}{\z} > |\Z(\x)| \geq \nrows{3}{\zbar}$.
Therefore, $\zbar$ intersects at most $2$ intervals per row on average, and $\ffc(\zbar)$ can be bounded by the Neighborhood bound effectively.
\shortversion{
\begin{lemma}
    \label{lem:any_z_bar_short}
    The color of interval $\zbar$, $\ffc(\zbar) \leq \nrows{0}{\x} + (1 + \alp(\zbar)) \cdot \nrows{1}{\x} + \delt(\zbar) \cdot (1 - \gam ) \cdot 2\nrows{2}{\x} + \bet  \cdot \nrows{2}{\x} + \gam  \cdot \nrows{2}{\x} + 2\nrows{3}{\x}$.
\end{lemma}
}
\fullversion{With the bound $\pivotset \subseteq \rows{1}{\x} \cup (\rows{2}{\x} \setminus \rows{3}{\z})$, the upper bound $\ffc(\x) \leq 2\omega$ (Lemma~\ref{lem:cy>=cu}).}
\shortversion{
Then, by the Pivot bound:
\begin{lemma}
    \label{lem:cy>=cu_short}
    If $\ffc(\zbar) \geq \ffc(\y)$, then $\ffc(\x) \leq 2\omega$   
\end{lemma}
}

% CASE 2.b.ii: c(z bar) < c(\y)
\shortversion{
\paragraph*{Case (2.b.ii): $\ffc(\zbar) < \ffc(\y)$}
}
\fullversion{
\subparagraph*{(2.b.ii)}
$\ffc(\zbar) < \ffc(\y)$.}

% Why we don't use z bar

This case is the tough kernel of the analysis.
In this case, considering using $\y$, $\z$, or $\zbar$ as the pivot only is not sufficient to have an upper bound of $\ffc(\x)$ that is smaller than $3\omega$.
% Why not \y
Recall that this case is a subcase of case $\ffc(\y) < \ffc(\z)$, picking $\y$ as the pivot does not stop any interval in $\rows{2}{\x}$ from contributing to $\pivotset$. 
% Why not \z
On the other hand, as $\nrows{3}{\z} > |\Z(\x)|$, $\nrows{3}{\z}$ may be much larger than $\nrows{1}{\z}$.
Applying the Neighborhood bound yields a huge upper bound of $\ffc(\pivot)$.
% Why not \zbar
Finally, condition $\ffc(\zbar) < \ffc(\y)$ implies that when $\zbar$ is selected as the pivot, $\pivotset$ could contain any interval in $\rows{3}{\x}$ and become too large.


% Why we can use z and y
To deal with this tough case, we express the upper bounds obtained by using $\y$ as the pivot and using $\z$ as the pivot in terms of $|\Z(\x)|$ and $\nrows{2}{\z}$\fullversion{ (Lemmas~\ref{lem:any_z} and~\ref{lem:any_uvw})}. 
\shortversion{
For the color of interval $\z$, Lemma~\ref{lem:any_z_short} still suffices.
However, for interval $\y$ we take a closer look at the neighborhood of interval $\y$ given the current assumptions.
In order to be more precise, let us first make additional observations regarding the intersections of interval $\y$ with intervals in $\rows{2}{\x}$ rows (also see Figure~\ref{fig:y_intersections_short}).

\begin{figure}[t]
\centering
\begin{minipage}{0.5\textwidth}
\centering
\includegraphics[width=\textwidth]{pictures/Brick_Rock_y.pdf}
\end{minipage}%
\caption{
An exemplary non-exhaustive list of possible rows per type of $\rows{2}{\x}$.
Yellow intervals intersect only interval $\y$, purple intervals intersect only $\x$ and orange intervals intersect both $\y$ and $\x$.}
\label{fig:y_intersections_short}
\end{figure}

\medskip

\begin{observation}
    \label{obs:any_y_short}
    Interval $\y$ intersects
    \begin{enumerate}[(a)]
        \item at most $1$ interval that is not in $\nb(\x)$ per row contained in $\rows{2}{\x}$, and
        \item at most $1$ interval that is in $\nb(\x)$ per row contained in $\rows{3}{\z}$,
        \item at most $1$ intervals that is in $\nb(\x)$ per row contained in $\Z(\x)$, and
        \item at most $1$ intervals that is in $\nb(\x)$ per row contained in $\rows{2}{\x} \setminus (\Z(\x) \cup \rows{3}{\z} )$.
    \end{enumerate}
\end{observation}

\begin{lemma}
    \label{lem:any_uvw_short}
    The color of interval $\y$,
    $\ffc(\y) \leq 2\nrows{0}{\x} + 2\nrows{1}{\x} + \alp(\y) \cdot \nrows{1}{\x} + \gam  \cdot \nrows{2}{\x} + \nrows{2}{\x} + \delt(\y) \cdot (1 - \gam ) \cdot \nrows{2}{\x} + \nrows{3}{\x}$. 
\end{lemma}
\begin{proof}
    By the Neighborhood bound, $\ffc(\y) \leq 1 + |\mathcal{N}(\y)|$.
    By Observation~\ref{obs:integral}, there are at most $ 2\nrows{0}{\x} + 2\nrows{1}{\x} + \nrows{3}{\x}$ intervals in $\rows{0}{\x}, \rows{1}{\x}$ and $\rows{3}{\x}$ that are in $\nb(\y)$.
    Together with the at most $\alp(\y) \cdot \nrows{1}{\x}$ intervals that were assigned a color below $\ffc(\y)$ which are either in $\twins(\x)$ or in $\nlmr(\x)$.
    
    By Observation~\ref{obs:any_y_short}, there are at most $\nrows{2}{\x} + \gam \cdot \nrows{2}{\x} + \bet \cdot \nrows{2}{\x} + (1 - \gam - \bet) \cdot \nrows{2}{\x}$ intervals that are in $\rows{2}{\x}$ and in $\nb(\y)$.
    By our assumption that $\ffc(\zbar) < \ffc(\y)$, the $\gam  \cdot \nrows{2}{\x}$ intervals in the set $\rows{3}{\z} \cap \nb(\x)$, must have a color smaller than $\ffc(\y)$.
    For the remaining $(1 - \gam) \cdot \nrows{2}{\x}$ intervals that are in $\nb(\x)$, by the definition of $\delt$,  $(1 - \delt(\y)) \cdot (1 - \gam) \cdot \nrows{2}{\x}$ are assigned a color larger than $\ffc(\z)$ and hence we do not consider them for this bound.
    
    Observe that by this bound we count the row that contains interval $\y$ itself.
    Since interval $\y$ cannot intersect any interval on this row, we should subtract at least $1$ from this bound. 
    Thus,
    \begin{align*}
            \ffc(\y) &\leq |\nb(\y)| + 1\\
            &\leq 2\nrows{0}{\x} + 2\nrows{1}{\x} + \alp(\y) \cdot \nrows{1}{\x} + \gam  \cdot \nrows{2}{\x} \\&\hspace{1cm} + \nrows{2}{\x} + \delt(\y) \cdot (1 - \gam ) \cdot \nrows{2}{\x} + \nrows{3}{\x}
        \end{align*}    
\end{proof}

When we pick interval $\z$ as pivot, the size of $|\pivotset|$ proven in Lemma~\ref{lem:pivotset_z_short} remains true.
For the case where we pick interval $\y$, we show the following bound.

\begin{lemma}
    \label{lem:pivotset_y_exact_short}
    When $\ffc(\y) \geq \ffc(\zbar)$, by selecting $\y$ as pivot $\pivot$, $|\pivotset| \leq (1 - \alp(\y)) \cdot \nrows{1}{\x} + (1 - \delt(\z)) \cdot (1 - \gam ) \cdot \nrows{2}{\x} + (1 - \gam ) \cdot \nrows{2}{\x}$
\end{lemma}
\begin{proof}
    First, by the definition of $\rows{0}{\x}$, $\x$ does not intersect any interval in $\rows{0}{\x}$.
    Thus, no interval in $\rows{0}{\x}$ can be in $\pivotset$.
    By the definition of $\alp$, only $(1 - \alp(\y) ) \cdot \nrows{1}{\x}$ intervals in $\rows{1}{\x}$ are assigned a color strictly larger than $\ffc(\y)$.
    Thus the only intervals in an $\rows{1}{\x}$ that are eligible for $\pivotset$ are those $(1 - \alp(\y) ) \cdot \nrows{1}{\x}$ intervals.
    Since $\ffc(\y) > \ffc(\zbar)$, no interval in an $\rows{3}{\z}$ row is assigned a color larger than $\ffc(\y)$.
    Thus, none of the $\gam \cdot 2\nrows{2}{\x}$ intervals in $\rows{3}{\z} \cap \nb(\x)$ can contribute to $\pivotset$.
    By the definition of $\delt$, only $(1 - \delt(\y) ) \cdot (1 - \gam) \cdot \nrows{2}{\x}$ intervals both in $\nb(\y)$ and $\rows{2}{\x}$ are assigned a color strictly larger than $\ffc(\y)$.
    Furthermore, all $(1 - \gam) \cdot \nrows{2}{\x}$ intervals which are in $\rows{2}{\x}$ but not in $\nb(\y)$ and not in $\rows{3}{\z}$ could potentially be assigned a color larger than $\ffc(\y)$ and therefore contribute to $\pivotset$.
    Finally, as all intervals in $\rows{3}{\x}$ are in $\lmr(\x)$, all intervals in $\rows{3}{\x}$ have a color below $\ffc(\y)$ and cannot contribute to $\pivotset$.
    Thus, $|\pivotset| \leq (1 - \alp(\y)) \cdot \nrows{1}{\x} + (1 - \delt(\z)) \cdot (1 - \gam ) \cdot \nrows{2}{\x} + (1 - \gam ) \cdot \nrows{2}{\x}$.
\end{proof}`}
These formulations show that the upper bounds from applying the Pivot bound using interval $\y$ as $\pivot$ and using interval $\z$ as $\pivot$ form a trade-off.
More specifically, the upper bound using $\y$ is maximized when $|\Z(\x)| = \emptyset$, while upper bound using $\z$ is maximized when $|\Z(\x)| = \rows{2}{\x}$.
Then, we use that in any circumstance, the upper bound is at most $2\omega + \frac{|\Z(\x)|}{2}$, which is strictly less than $2\omega + \frac{\nrows{2}{\x}}{2}\cdot\omega$ 
(Definition~\ref{def:Z}). 
\fullversion{Since $\nrows{2}{\x}$ can be upper bounded by $\frac{2}{3}\omega - 1$ (Corollary~\ref{cor:r2_bound}), the color assigned to interval $\x$ is bounded by $\result$ (Theorem~\ref{thm:any}).}
\shortversion{We next prove a bound on the number of $\rows{2}{\x}$ rows.
\begin{lemma}
    \label{lem:r2_bound_short}
    For any interval $\x \in \orderedInput$, if $\nrows{2}{\x} \geq \frac{2}{3} \omega$, then $\ffc(\x) \leq \frac{7}{3} \omega - 2$. 
\end{lemma}

\begin{proof}
    We show this using the Neighborhood bound.
    Observe that $\nrows{0}{\x} \geq 1$, since interval $\x$ does not intersect any intervals on its own row.
    Furthermore, note that the Neighborhood bound is maximized when we maximize the number of intervals that intersect interval $\x$.
    That is, we maximize the bound when we maximize $\nrows{3}{\x}$.
    Then, it follows by $\nrows{0}{\x} \geq 1$ and $\nrows{2}{\x} \geq \frac{2}{3}\omega$, that $\nrows{3}{\x} = \frac{1}{3}\omega - 1$.
    For the remaining rows, we maximize the bound by maximizing $\nrows{2}{\x}$, and hence $\nrows{2}{\x} = \frac{2}{3}\omega$.
    Then, $\nrows{1}{\x} = 0$, and we get the following upper bound on the color of interval $\x$,
    \begin{align*}
        \ffc(\x) &\leq \nrows{1}{\x} + 2\nrows{2}{\x} + 3\nrows{3}{\x} + 1 \\
            &\leq 0 + 2 \cdot \frac{2}{3} \omega + 3 \cdot (\frac{1}{3}\omega - 1) + 1\\
            &= \frac{7}{3} \omega - 2
    \end{align*}
\end{proof}

Then, we can bound the color assigned to interval $\x$ by $\result$.
\paragraph*{Proof of Theorem~\ref{thm:any}.}
\emph{For any interval $\x \in \orderedInput$, the color of $\x$ is at most $\ffc(\x) \leq \result$.}

\begin{proof}
    By Lemma~\ref{lem:open_interval_small_intersection},~\ref{lem:integral_open_and_closed_assumtion_not_hold}, \ref{lem:c_u>=c_z_short}, \ref{lem:gamma<delta_short} and \ref{lem:cy>=cu_short}, $\ffc(\x) \leq 2\omega$ if $\x$ is open, $\nrows{1}{\x} \leq \nrows{3}{\x} - 2$, $\ffc(\y) < \ffc(\z)$, $\bet  < \gam $ or $\ffc(\zbar) < \ffc(\y)$.
    Thus in the following we focus on closed intervals with $\nrows{1}{\x} \leq \nrows{3}{\x} - 2$, $\ffc(\zbar) < \ffc(\y) < \ffc(\z)$ and $\bet  < \gam $.

    Now we can compute the first bound on the color of interval $\x$, using the Pivot bound with interval $\y$ as pivot $\pivot$.
    According to Lemma~\ref{lem:any_uvw_short}, $\ffc(\y) \leq  2\nrows{0}{\x} + 2\nrows{1}{\x} + \alp(\y) \cdot \nrows{1}{\x} + \gam  \cdot \nrows{2}{\x} + \nrows{2}{\x} + \delt(\z) \cdot (1 - \gam ) \cdot \nrows{2}{\x} + \nrows{3}{\x}$ and, by taking $\y$ as pivot $\pivot$,  Lemma~\ref{lem:pivotset_y_exact_short}, $|\pivotset| \leq (1 - \alp(\y)) \cdot \nrows{1}{\x} + (1 - \delt(\z)) \cdot (1 - \gam ) \cdot \nrows{2}{\x} + (1 - \gam ) \cdot \nrows{2}{\x}$.
    Thus,

    \begin{align*}
        \ffc(\x) &\leq \ffc(\pivot) + |\pivotset| + 1\\
        &\leq \ffc(\y) + (1 - \alp(\y)) \cdot \nrows{1}{\x} + (1 - \delt(\z)) \cdot (1 - \gam ) \cdot \nrows{2}{\x} \\ &\hspace{1cm} + (1 - \gam ) \cdot \nrows{2}{\x} + 1 \\
        &\leq 2\nrows{0}{\x} + 3\nrows{1}{\x} + 2\nrows{2}{\x} + (1 - \gam ) \cdot \nrows{2}{\x} + \nrows{3}{\x} + 1 \\
        &\leq 2\nrows{0}{\x} + 2\nrows{1}{\x} + (1 - \gam ) \cdot \nrows{2}{\x} + 2\nrows{2}{\x} + 2\nrows{3}{\x} - 1\\
        &= 2\omega + (1 - \gam ) \cdot \nrows{2}{\x} - 1
    \end{align*}

    Next we move to the other bound.
    By taking $\z$ as pivot $\pivot$, according to Lemma~\ref{lem:any_z_short}, $\ffc(\z) \leq \omega + \alp(\z) \cdot \nrows{1}{\x} - \bet  \cdot \nrows{2}{\x} + \gam  \cdot \nrows{2}{\x} + \nrows{2}{\x} + \nrows{3}{\x}$, and Lemma~\ref{lem:pivotset_z_short}, $|\pivotset| = (1 - \alp(\z))\cdot\nrows{1}{\x}$.
    Then, it follows that,

    \begin{align*}
        \ffc(\x) &\leq \ffc(\pivot) + |\pivotset| + 1\\
        &\leq \ffc(\z) + (1 - \alp(\z)) \cdot \nrows{1}{\x} + 1 \\
        &\leq \omega + \nrows{1}{\x} - \bet  \cdot \nrows{2}{\x} + \gam  \cdot \nrows{2}{\x} + \nrows{2}{\x} + \nrows{3}{\x} + 1\\ 
        &\leq 2\omega + \gam  \cdot \nrows{2}{\x} - 1 % beta cannot be 0
    \end{align*}

    We have proven that simultaneously, the color of interval $\x$ is at most $\ffc(\x) \leq 2\omega + \gam  \cdot \nrows{2}{\x} - 1$ and at most $\ffc(\x) \leq 2\omega + (1 - \gam ) \cdot \nrows{2}{\x} - 1$.
    Then, it follows that for any value of $\gam $ the color of interval $\x$ is at most $\ffc(\x) \leq 2\omega + \frac{1}{2}\cdot \nrows{2}{\x} - 1$
    Then, it follows from Lemma~\ref{lem:r2_bound_short} that the number of $\rows{2}{\x}$ rows must be less than $\nrows{2}{\x}  < \frac{2}{3}\omega$, as otherwise the color of interval $\x$ is bounded by $\ffc(\x) \leq \frac{7}{3} \omega - 2$.
    Thus, the color of interval $\x$ is at most $\ffc(\x) < 2\omega + \frac{1}{3}\cdot \omega - 1 = \frac{7}{3}\omega - 1$.
    Which, by the integrality of $\ffc(\x)$, is at most $\ffc(\x) \leq \result$  
\end{proof}
}

%%%%%%%%%%%%%%%%%%%%%%%%%%%%%%%%%%%%%%%%%%%%%%%%%%%%%%%%%%%%%%%%%%%%%%%%%%%%%%%%%%%%%%
%% END OF SHORTVERSION
%%%%%%%%%%%%%%%%%%%%%%%%%%%%%%%%%%%%%%%%%%%%%%%%%%%%%%%%%%%%%%%%%%%%%%%%%%%%%%%%%%%%%%
\fullversion{
% Why lemma 11 still holds for any endpoints
\subsection{Analysis}

\paragraph*{Case (1): $\ffc(\y) \geq \ffc(\z)$}

We start by proving that in case (1), that is, when $\ffc(\y) \geq \ffc(\z)$, the color of interval $\x$ can be bounded by $2\omega$.

\medskip

\runtitle{Pick a pivot $\pivot$}.
    Given the condition where interval $\y$ has a color larger or equal to interval $\z$, picking $\y$ as a pivot provides a smaller bound of $|\pivotset|$.
    Therefore, we pick $\y$ as the pivot $\pivot$.

\medskip

\runtitle{Bound the color of $\ffc(\pivot)$.}
    Next, we bound on the color of the pivot $\ffc(\pivot)$.
    By Corollary~\ref{cor:integral_LMR}, $\ffc(\pivot) = \ffc(\y) \leq 2\omega + \alp (\y) \cdot \nrows{1}{\x} - \nrows{3}{\x}$, where $\alp(\y)$ is the fraction of rows in $\rows{1}{\x}$ that contains intervals in $\twins(\x) \cup \nlmr(\x)$ and having color not higher than $\y$.

\medskip

\runtitle{Bound the size of $\pivotset$}.
Given that we use $\y$ as pivot $\pivot$, we bound the size of $\pivotset$.
Recall that $\pivotset$ is the set of intervals in $\nb(\x)$ which are assigned colors strictly higher than $\ffc(\pivot)$.
Since $\ffc(\y) \geq \ffc(\z)$, only intervals in $\twins(\x)$ are possible to contribute to~$\pivotset$.


\begin{lemma}
    \label{lem:pivotset_y}
    When $\ffc(\y) \geq \ffc(\z)$, by selecting $\y$ as pivot $\pivot$, $|\pivotset| = (1 - \alp(\y)) \cdot \nrows{1}{\x}$.
\end{lemma}

\begin{proof}
    First, by the definition of $\rows{0}{\x}$, $\x$ does not intersect any interval in $\rows{0}{\x}$.
    Thus, no interval in $\rows{0}{\x}$ can be in $\pivotset$.
    Next, the intervals in $\rows{2}{\x}$ are either in $\lmr(\x)$ or in $\nlmr(\x)$.
    Moreover, since $\ffc(\y) \geq \ffc(\z)$, all intervals in $\rows{2}{\x}$ have a color below $\ffc(\y)$ and cannot contribute to $\pivotset$.
    Similarly, as all intervals in $\rows{3}{\x}$ are in $\lmr(\x)$, all intervals in $\rows{3}{\x}$ have a color below $\ffc(\y)$ and cannot contribute to $\pivotset$.
    The remaining intervals are the intervals in $\rows{1}{\x}$.
    By definition, $\alp(\y) \cdot \nrows{1}{\x}$ of these intervals have a color at most $\ffc(\y)$.
    Hence, the remaining $(1 - \alp(\y) ) \cdot \nrows{1}{\x}$ are the only intervals intersecting $\x$ that have a color greater than $\ffc(\y)$. That is, $|\pivotset| = (1 - \alp(\y)) \cdot \nrows{1}{\x}$.
\end{proof}

\medskip

\runtitle{Bound the color of $\ffc(\x)$.}
Now, we wrap up the analysis of the case where $\ffc(\y) \geq \ffc(\z)$ using the bounds of $\ffc(\pivot)$ and $|\pivotset|$.

\begin{lemma}
    \fullversion{\label{lem:c_u>=c_z}}
    For any interval $\x \in \orderedInput$, if $\ffc(\y) \geq \ffc(\z)$, then $\ffc(\x) \leq 2\omega$.
\end{lemma}

\begin{proof}
    By Lemma~\ref{lem:integral_open_and_closed_assumtion_not_hold} and Lemma~\ref{lem:open_interval_small_intersection}, $\ffc(\x) \leq 2\omega$ if $\x$ is open or $\nrows{3}{\x} < \nrows{1}{\x} + 2$.
    Thus, in the following, we focus on closed intervals $\x$ with $\nrows{1}{\x} \leq \nrows{3}{\x} - 2$.
    
    By taking $\y$ as pivot $\pivot$, according to Corollary~\ref{cor:integral_LMR}, $\ffc(\y) \leq 2\omega + \alp(\y) \cdot \nrows{1}{\x} - \nrows{3}{\x}$.
    On the other hand, by  Lemma~\ref{lem:pivotset_y}, $|\pivotset| = (1 - \alp(\y)) \cdot \nrows{1}{\x}$.
    Thus,
    \begin{align*}
        \ffc(\x) &\leq \ffc(\pivot) + |\pivotset| + 1\\
        &\leq \ffc(\y) + (1 - \alp(\y) ) \cdot \nrows{1}{\x} + 1\\
        &\leq 2\omega + \nrows{1}{\x} - \nrows{3}{\x} + 1\\
        &\leq 2\omega - 1
    \end{align*}
\end{proof}

\hide{
In the case where interval $\z$ is in $\lmr(\x)$, then, from the definition of $\y$, it must be the case that $\ffc(\y) \geq \ffc(\z)$.
Therefore, the next result follows directly from Lemma~\ref{lem:c_u>=c_z}.

\begin{corollary}
    \label{cor:z_in_pos_u_v}
    If $\z \in \lmr(\x)$, then $\ffc(\x) \leq 2\omega$.
\end{corollary}
}

\medskip

\paragraph*{Case (2): $\ffc(\y) < \ffc(\z)$}

% Intro to next case
From now on, we focus on the case where $\ffc(\y) < \ffc(\z)$.

\medskip

\runtitle{Pick a pivot $\pivot$.}
    The condition where $\ffc(\y) < \ffc(\z)$ guarantees that all intervals in the set $\lmr(\x)$ have colors at most $\ffc(\z)$.
    Therefore, for the sake of the size of the $\pivotset$, it is more efficient to pick the interval $\z$ as a pivot than picking the interval~$\y$.
    Therefore, we pick interval $\z$ as pivot $\pivot$.

%when $\ffc(\z) > \ffc(\y)$. 

Recall from Definition~\ref{def:z} that $\z$ is the interval with the highest color in the rows in~$\rows{2}{\x}$. 
The selection of $\z$ attains a good structural property on $\rows{3}{\z}$ with regards to $\rows{2}{\x}$:

\begin{lemma}
    \label{lem:R3z_in_R2x}
    $\rows{3}{\z} \subseteq \rows{2}{\x}$.
\end{lemma}

\begin{proof}
    By case distinction on $\rows{i}{\x}$, we prove by contradiction that there is no intersection between $\rows{3}{\z} \cap \rows{i}{\x}$.
    
    \textbf{(1)} Suppose on the contrary that $\rows{3}{\z} \cap \rows{0}{\x} \neq \emptyset$.
    If there exists a row with $3$ intervals that intersect $\z$, but none of them intersects $\x$, then the intervals $\x$ and $\z$ must not intersect.
    Thus, $\z \notin \nlmr(\x)$, which is a contradiction by the definition of $\z$.

    \medskip
    
    \textbf{(2)} Suppose on the contrary that $\rows{3}{\z} \cap \rows{1}{\x} \neq \emptyset$.
    If there exists a row with $3$ intervals that intersect $\z$, but only one of those intervals intersects $\x$. 
    Then, either the interval~$\z$ is in the set $\lmr(\x)$ or $\x$ and $\z$ do not intersect.
    In either case, $\z \not \in \nlmr(\x)$, which is a contradiction by the definition of $\z$.
    
    \medskip
    
    \textbf{(3)} Suppose on the contrary that $\rows{3}{\z} \cap \rows{3}{\x} \neq \emptyset$. 
    If there exists a row with $3$ intervals that intersect both $\x$ and $\z$, then interval $\z$ must be in the set $\twins(\x)$.
    Then, $\z \not \in \nlmr(\x)$, which is a contradiction by the definition of $\z$.
    
    \medskip
    
    By \textbf{(1)} to \textbf{(3)}, $\rows{3}{\z} \cap \left( \rows{0}{\x} \cup \rows{1}{\x} \cup \rows{3}{\x} \right) = \emptyset$.
    Therefore, it follows that $\rows{3}{\z} \subseteq \rows{2}{\x}$.
\end{proof}

% Redefine Z Zbar and zbar
Recall from Definition~\ref{def:Z} that the set $\Z(\x) \subset \rows{2}{\x}$ contains all intervals in an~$\rows{2}{\x}$ row that are twins of interval $\z$. 
\hide{And the set $\rows{3}{\z}$ contains all rows in which~$3$ intervals intersect interval $\z$.
This set $\rows{3}{\z}$ must also be a subset of $\rows{2}{\x}$.}
Furthermore, recall from Definition~\ref{def:z_bar} that interval $\zbar$ is the highest colored interval in $\rows{3}{\z}$ that intersects with $\x$.
Observe that the intersection of the sets $\Z(\x)$ and $\rows{3}{\z}$ must be empty, as a row cannot have both $3$ intervals intersecting $\z$ and an interval identical to $\z$, and therefore $|\Z(\x)| + \nrows{3}{\z} \leq \nrows{2}{\x}$.

% Figures
\begin{figure}[t]
\begin{minipage}{0.35\textwidth}
\centering
\includegraphics[width=\textwidth]{pictures/Brick_Rock_r2.pdf}
\end{minipage}%
\hspace{0.1\textwidth}%
\begin{minipage}{0.55\textwidth}
\centering
\includegraphics[width=\textwidth]{pictures/Brick_Rock_z.pdf}
\end{minipage}
\caption{
An exemplary non-exhaustive list of possible rows per type of row.
Yellow intervals intersect only interval $\z$, purple intervals intersect only $\x$ and orange intervals intersect both $\z$ and $\x$.
}
\label{fig:r2_and_z_intersections}
\end{figure}

\medskip

\runtitle{Bound the color of $\ffc(\pivot)$.}
% Observations
We first provide observations on the maximum number of intersections intervals in $\nlmr(\x)$ and specifically $\z$ have in each type of row (also see Figure~\ref{fig:r2_and_z_intersections}).

\begin{observation}
    \label{obs:any_r2}
    Any interval in the set $\nlmr(\x)$ intersects
    \begin{enumerate}[(a)]
        \item at most $1$ interval, other than $\x$, per row in $\rows{0}{\x}$,
        \item at most $1$ interval, which does not intersect interval $\x$, per row in $\rows{1}{\x}$, and
        \item exactly $2$ intervals per row in $\rows{3}{\x}$.
    \end{enumerate}
\end{observation}

\begin{observation}
    \label{obs:any_z}
    Interval $\z$ intersects
    \begin{enumerate}[(a)]
        \item exactly $3$ intervals per row in $\rows{3}{\z}$,
        \item exactly $1$ interval per row in $\Z(\x)$, and
        \item at most $2$ intervals per row in $\rows{2}{\x} \setminus \left( \Z(\x) \cup \rows{3}{\z} \right)$.
    \end{enumerate}
\end{observation}

% Definition of gamma and delta
Furthermore, we define two variables with respect to the sets $\Z(\x)$ and $\rows{3}{\z}$.
Similarly to $\alp(i)$ being a fraction of the $\rows{1}{\x}$ rows, we define $\bet $ and $\gam $ as fractions of the set~$\rows{2}{\x}$ with respect to interval $\z$.

\begin{definition}
    \label{def:gamma_delta}
    Given $\z \in \rows{2}{\x}$, 
    \begin{enumerate}[(a)]
        \item $\bet \in [0,1]$ denotes the fraction of rows $\mathcal{R} \in \rows{2}{\x}$ where the interval $I^\prime$ in $\mathcal{R} \cap \nb(\x)$ is in $\zbar$, and
        \item $\gam \in [0,1]$ denotes the fraction of rows $\mathcal{R} \in \rows{2}{\x}$ where the interval $I^\prime$ in $\mathcal{R} \cap \nb(\x)$ is in $\rows{3}{\z}$.
    \end{enumerate}
\end{definition}

% A sentence introducing the lemma
Now we are equipped to compute an upper bound on the color of interval $\z$.

\begin{lemma}
    \fullversion{\label{lem:any_z}}
    The color of interval $\z$ is at most $c(\z) \leq \omega + \alp(\z) \cdot \nrows{1}{\x} - \bet  \cdot \nrows{2}{\x} + \gam  \cdot \nrows{2}{\x}  + \nrows{2}{\x} + \nrows{3}{\x}$. 
\end{lemma}

\begin{proof}
    Using the Neighborhood bound and Observations~\ref{obs:any_r2} and~\ref{obs:any_z}, we can describe the maximum color of interval $\z$ as $1$ plus the number of intersections per type of row.
    Observe that by doing this, we count the row that contains interval $\z$ itself, and since $\z$ cannot intersect any interval in this row, we may subtract at least $1$.
    Then it follows that
    \begin{align*}
            \ffc(\z) &\leq \nrows{0}{\x} + (1 + \alp(\z)) \cdot \nrows{1}{\x} + \bet \cdot \nrows{2}{\x} + \gam \cdot 3\nrows{2}{\x} \\&\hspace{2cm}+ (1 - \bet - \gam) \cdot 2\nrows{2}{\x} + 2\nrows{3}{\x} + 1 - 1\\
            &= \omega + \alp(\z) \cdot \nrows{1}{\x} - \bet  \cdot \nrows{2}{\x} + \gam \cdot \nrows{2}{\x} + \nrows{2}{\x} + \nrows{3}{\x}
        \end{align*}
\end{proof}

% Case distinction

Next, we make a case distinction based on the relative sizes of the sets $\Z(\x)$ and $\rows{3}{\z}$.
%%%%%%%%%%%%%%%%%%%%%%%%%%%%
% CASE 2.a
%%%%%%%%%%%%%%%%%%%%%%%%%%%%
\paragraph*{Case (2.a): $|\Z(\x)| \geq \nrows{3}{\z}$}
We start with the case where the number of intervals in the set $\Z(\x)$ is at least as large as the number of intervals in the set $\rows{3}{\z}$.

\medskip 

\runtitle{Bound the size of $\pivotset$.}

\begin{lemma}
    \fullversion{\label{lem:pivotset_z}}
    When $\ffc(\z) > \ffc(\y)$, by selecting $\z$ as pivot $\pivot$, $ |\pivotset| = (1 - \alp(\z)) \cdot \nrows{1}{\x}$.
\end{lemma}

\begin{proof}
    First, by the definition of $\rows{0}{\x}$, $\x$ does not intersect any interval in $\rows{0}{\x}$.
    Thus, no interval in $\rows{0}{\x}$ can be in $\pivotset$.
    Next, all intervals in $\rows{2}{\x}$ are either in $\lmr(\x)$ or $\nlmr(\x)$.
    Since $\ffc(\z) > \ffc(\y)$, it follows that all intervals in $\rows{2}{\x}$ have a color below $\ffc(\z)$ and cannot contribute to $\pivotset$.
    Similarly, as all intervals in $\rows{3}{\x}$ are in $\lmr(\x)$, all intervals in $\rows{3}{\x}$ have a color below $\ffc(\z)$ and cannot contribute to $\pivotset$.
    The remaining intervals are those in $\rows{1}{\x}$.
    By definition, $\alp(\x) \cdot \nrows{1}{\x}$ of these intervals have a color at most $\ffc(\z)$.
    Hence, the remaining $(1 - \alp(\z) ) \cdot \nrows{1}{\x}$ are the only intervals intersecting $\x$ that have a color greater than $\ffc(\z)$. That is, $|\pivotset| = (1 - \alp(\z)) \cdot \nrows{1}{\x}$.
\end{proof}

\medskip

\runtitle{Bound the color of $\ffc(\x)$.}

\fullversion{\begin{lemma}
    \fullversion{\label{lem:gamma<delta}}
    If $|\Z(\x)| \geq \nrows{3}{\z}$, then $\ffc(\x) \leq 2\omega$.
\end{lemma}}

\begin{proof}
    Assume that $\ffc(\z) > \ffc(\y)$ and interval $\x$ is a closed interval, as otherwise by Lemma~\ref{lem:c_u>=c_z} and Lemma~\ref{lem:open_interval_small_intersection} it follows that $\ffc(\x) \leq 2\omega$.
    %From Lemma~\ref{lem:any_z} we have a bound for the color of interval $\z$, i.e., $\ffc(\z) \leq \omega + \alp(\z) \cdot \nrows{1}{\x} - \bet  \cdot \nrows{2}{\x} + \gam  \cdot \nrows{2}{\x} + \nrows{2}{\x} + \nrows{3}{\x}$.
    From the definition of $\bet $ and $\gam $ it follows that since $|\Z(\x)| \geq \nrows{3}{\z}$, also $\bet  \geq \gam $.
    
    Now we can compute a bound on the color of interval $\x$, using the Pivot bound where we take interval $\z$ as $\pivot$.
    By Lemma~\ref{lem:any_z}, the color of interval $\z$ is bounded by $\ffc(\z) \leq \omega + \alp(\z) \cdot \nrows{1}{\x} - \bet  \cdot \nrows{2}{\x} + \gam  \cdot \nrows{2}{\x} + \nrows{2}{\x} + \nrows{3}{\x}$.
    And, by Lemma~\ref{lem:pivotset_z}, the size of $\pivotset$ equals $(1 - \alp(\z)) \cdot \nrows{1}{\x}$.
    Then it follows that,

    \begin{align*}
        \ffc(\x) &\leq \ffc(\pivot) + |\pivotset| + 1\\
        &= \ffc(\z) + (1 - \alp(\z)) \cdot \nrows{1}{\x} + 1 \\
        &\leq \omega + \nrows{1}{\x} - \bet  \cdot \nrows{2}{\x} + \gam  \cdot \nrows{2}{\x} + \nrows{2}{\x} + \nrows{3}{\x} + 1 \\
        &\leq \omega + \nrows{1}{\x} + \nrows{2}{\x} + \nrows{3}{\x} + 1 \\
        &\leq 2\omega
    \end{align*}
\end{proof}
%%%%%%%%%%%%%%%%%%%%%%%%%%%%
% CASE 2.b
%%%%%%%%%%%%%%%%%%%%%%%%%%%%
% Continue with the second case of the case distinction. And introduce new one
\paragraph*{Case (2.b): $|\Z(\x)| < \nrows{3}{\z}$}
We continue with the case where the number of intervals in the set $\Z(\x)$ is strictly smaller than the number of intervals in the set $\rows{3}{\z}$.
This relation between $\Z(\x)$ and $\rows{3}{\z}$ indicates that the average number of intersections interval $\z$ has per row can be greater than $2$.
Then, it might be interesting to explore other options as a pivot.
For this we make another case distinction on the relation between the colors of interval $\y$ and interval $\zbar$.

%%%%%%%%%%%%%%%%%%%%%%%%%%%%
% CASE 2.b.i
%%%%%%%%%%%%%%%%%%%%%%%%%%%%
\paragraph*{Case (2.b.i): $\ffc(\zbar) \geq \ffc(\y)$}
First, let us explore the case where the color of interval $\zbar$ is at least as large as the color of interval $\y$.

\medskip

\runtitle{Pick a pivot $\pivot$}
The condition that $\ffc(\zbar) \geq \ffc(\y)$ guarantees that all intervals in the set $\lmr(\x)$ have colors at most $\ffc(\zbar)$.
Therefore, for the sake of the size of the size of the set $\pivotset$, it is more efficient to pick interval $\zbar$ as a pivot compared to picking interval $\y$.
Therefore, we pick interval $\zbar$ as pivot $\pivot$.

\medskip

\runtitle{Bound the color $\ffc(\pivot)$}

% zbar is not in LMR
As we now want to compute an upper bound on the color of interval $\zbar$, it is of importance to know where interval $\zbar$ might be located.

\begin{lemma}
    \label{lem:zbar_not_in_LMR}
    For any interval $I^\prime \in \rows{3}{\z}$, $I^\prime \not \in \lmr(\x)$
\end{lemma}

\begin{proof}
    Assume aiming towards a contradiction that $I^\prime \in \lmr(\x)$.
    \medskip

    \runtitle{(1) $I^\prime$ is a closed interval.} %$i \in L(\x)$. 
    %$L(\x)$ contains only closed intervals.
    Assume without loss of generality that interval $I^\prime$ intersects interval $\x$ on the left-hand side of interval $\x$.
    Since $I^\prime \in \rows{3}{\z}$, and $I^\prime$ is a closed interval, it follows that there are precisely two positions where interval $\z$ could lie.
    The first possible position is to the left of interval $I^\prime$.
    In this case, interval $\z$ and interval $\x$ cannot intersect.
    Then it follows that $\z \not \in \nlmr(\x)$, which is a contradiction by the definition of $\z$.
    The other possible position is to the right of interval $I^\prime$.
    In this case, interval $\z$ must be identical to interval $\x$, i.e., $\z \in \twins(\x)$.
    Then it follows that $\z \not \in \nlmr(\x)$, which is a contradiction by the definition of $\z$.
    
    \medskip
    
    \runtitle{(2) $I^\prime$ is an open interval.} %$i \in M(\x)$.
    %$M(\x)$ contains only open intervals.
    Since $I^\prime \in \lmr(\x)$ and $I^\prime$ is an open interval, it follows that $I^\prime \in \twins(\x)$.
    Similarly, since $I^\prime \in \rows{3}{\z}$, and $I^\prime$ is an open interval, it follows that $I^\prime \in \twins(\z)$.
    An interval can only be in both $\twins(\x)$ and $\twins(\z)$ if interval $\x$ and interval $\z$ are identical, i.e., $\z \in \twins(\x)$.
    Then it follows that $\z \not \in \nlmr(\x)$, which is a contradiction by the definition of $\z$.
\end{proof}

% Figures
\begin{figure}%[ht]
\centering
\begin{minipage}{0.49\textwidth}
\centering
\includegraphics[width=\textwidth]{pictures/Brick_Rock_zbar.pdf}
\end{minipage}%
\hspace{0.01\textwidth}%
\begin{minipage}{0.49\textwidth}
\centering
\includegraphics[width=\textwidth]{pictures/Brick_Rock_y.pdf}
\end{minipage}%
\caption{
An exemplary non-exhaustive list of possible rows per type of $\rows{2}{\x}$.
Yellow intervals intersect only interval $\zbar$ (resp. $\y$), purple intervals intersect only $\x$ and orange intervals intersect both $\zbar$ (resp. $\y$) and $\x$.}
\label{fig:y_intersections}
\end{figure}

% Recall observations from previous proof
Note that Observation~\ref{obs:any_r2} remains true for interval $\zbar$.
% Introduce new observations
Now, let us make an additional observation regarding the intersections interval $\zbar$ has with intervals in $\rows{2}{\x}$ rows (also see Figure~\ref{fig:y_intersections}):
%Exemplary rows are depicted in Figure~\ref{fig:r3z_intersections} to give some intuition.

\begin{observation}
    \label{obs:any_z_bar}
    Interval $\zbar$ intersects
    \begin{enumerate}[(a)]
        \item exactly $1$ interval per row contained in $\rows{3}{\z}$,
        \item exactly $3$ intervals per row contained in $\Z(\x)$, and
        \item at most $2$ intervals per row contained in $\rows{2}{\x} \setminus (\Z(\x) \cup \rows{3}{\z} )$.
    \end{enumerate}
\end{observation}


% Not all intervals are below or above zbar
Furthermore, it is important to specify how many of the intervals in $\rows{2}{\x} \setminus \rows{3}{\z}$ are assigned a color larger than $\ffc(\zbar)$ or smaller than $\ffc(\zbar)$.
This was previously not necessary, when we considered interval $\z$, which is, by definition, the interval in $\rows{2}{\x}$ assigned the largest color.
Analogously, we do not need to consider the intervals in $\rows{3}{\z}$ as per definition interval $\zbar$ is the largest colored interval in this set.

\begin{definition}
    \label{def:delta}
    Given any interval $\hat{I}$, $\delt(\hat{I}) \in [0, 1]$ denotes the fraction of rows $\mathcal{R} \in \rows{2}{\x} \setminus \rows{3}{\z}$ where interval $I^\prime$ in $\mathcal{R} \cap \nb(\x) \cap \nb(\z)$ satisfies $\ffc(I^\prime) \leq \ffc(\hat{I})$.
\end{definition}

% Introduce the next lemma
Now we are equipped to compute a bound on the color of interval $\zbar$.

\begin{lemma}
    \label{lem:any_z_bar}
    The color of interval $\zbar$, $\ffc(\zbar) \leq \nrows{0}{\x} + (1 + \alp(\zbar)) \cdot \nrows{1}{\x} + \delt(\zbar) \cdot (1 - \gam ) \cdot 2\nrows{2}{\x} + \bet  \cdot \nrows{2}{\x} + \gam  \cdot \nrows{2}{\x} + 2\nrows{3}{\x}$.
\end{lemma}

\begin{proof}
    By the Neighborhood bound $\ffc(\zbar) \leq 1 + |\mathcal{N}(\zbar)|$.
    By Observation~\ref{obs:any_r2}, there are at most $\nrows{0}{\x} + 2\nrows{1}{\x} + 2\nrows{3}{\x}$ intervals in $\rows{0}{\x}, \rows{1}{\x}$ and $\rows{3}{\x}$ that are in $\mathcal{N}(\zbar)$.
    However, by Definition~\ref{def:alpha}, $(1 - \alp(\zbar))$ of those intervals have a color larger than $\ffc(\zbar)$ and hence we do not consider them for this bound.
    By Observation~\ref{obs:any_z_bar}, there are at most $\nrows{3}{\z} + 3\cdot|\Z(\x)| + 2\cdot|\rows{2}{\x}\setminus(\Z(\x)\cup\rows{3}{\z})| = \gam \cdot \nrows{2}{\x} + \bet \cdot 3\nrows{2}{\x} + (1 - \bet - \gam) \cdot 2\nrows{2}{\x}$ intervals in $\rows{2}{\x}$ that are in $\mathcal{N}(\zbar)$.
    However, by Definition~\ref{def:delta}, $(1 - \delt(\zbar)) \cdot (\bet \cdot 2\nrows{2}{\x} + (1 - \bet - \gam) \cdot 2\nrows{2}{\x})$ of those intervals have a color larger than $\ffc(\zbar)$ and hence we do not consider them for this bound.
    Observe that by this bound, we count the row that contains interval $\zbar$ itself.
    Since $\zbar$ cannot intersect any interval on this row, we should subtract at least 1 from this bound.
    Thus,

    \begin{align*}
            \ffc(\zbar) &\leq \nrows{0}{\x} + 2\nrows{1}{\x} - (1 - \alp(\zbar)) \cdot \nrows{1}{\x} + 2\nrows{3}{\x}  
            \\ &\hspace{1cm} + \bet  \cdot 3\nrows{2}{\x} + \gam  \cdot \nrows{2}{\x} + (1 - \bet  - \gam ) \cdot 2\nrows{2}{\x}
            \\ &\hspace{1cm} - (1 - \delt(\zbar)) \cdot \left( \bet  \cdot 2\nrows{2}{\x} + (1 - \bet  - \gam ) \cdot 2\nrows{2}{\x} \right) -1 + 1\\
            %
            &= \nrows{0}{\x} + (1 + \alp(\zbar)) \cdot \nrows{1}{\x} + \delt(\zbar) \cdot (1 - \gam ) \cdot 2\nrows{2}{\x} + \bet  \cdot \nrows{2}{\x} \\&\hspace{1cm} + \gam  \cdot \nrows{2}{\x} + 2\nrows{3}{\x}
        \end{align*}
\end{proof}

\medskip

\runtitle{Bound the size of $\pivotset$.}

\begin{lemma}
    \label{lem:pivotset_zbar}
    By selecting $\zbar$ as pivot $\pivot$, $|\pivotset| = (1-\alp(\zbar)) \cdot \nrows{1}{\x} + (1 - \delt(\zbar) \cdot (1 - \gam) \cdot 2\nrows{2}{\x}$.
\end{lemma}

\begin{proof}
    First, by the definition of $\rows{0}{\x}$, $\x$ does not intersect any interval in $\rows{0}{\x}$.
    Thus, no interval in $\rows{0}{\x}$ can be in $\pivotset$.
    Next, all intervals in $\rows{3}{\x}$ are in the set $\lmr(\x)$.
    Thus, all these intervals are assigned a color below $\ffc(\zbar)$ and cannot contribute to $\pivotset$.
    Of the intervals in $\rows{1}{\x}$, by definition of $\alp$, only $(1 - \alp(\zbar))\cdot \nrows{1}{\x}$ intervals have a color larger than $\ffc(\zbar)$, and can contribute to set $\pivotset$.
    Of the $\rows{2}{\x}$ rows, out of the intervals intersecting $\x$ that are in the rows that are also in the set $\rows{3}{\z}$, none have a color larger than $\ffc(\zbar)$ by the definition of $\zbar$.
    Then, of the remaining $(1 - \gam) \cdot \nrows{2}{\x}$ $\rows{2}{\x}$ rows, by the definition of $\delt$, only $(1 - \delt(\zbar)) \cdot (1 - \bet - \gam) \cdot 2\nrows{2}{\x}$ intervals have a color larger than $\ffc(\zbar)$ and can contribute to the set $\pivotset$.
    Then it follows that $|\pivotset| = (1 - \alp(\zbar))\cdot \nrows{1}{\x} + (1 - \delt(\zbar)) \cdot (1 - \gam) \cdot 2\nrows{2}{\x} $
\end{proof}

\medskip

\runtitle{Bound the color $\ffc(\x)$.}

% Something to connect the lemma's?
Now we can use this result to prove an upper bound on the color of interval $\x$, when the color of interval $\zbar$ is at least as large as the color of interval $\y$.

\begin{lemma}
    \label{lem:cy>=cu}
    If $\ffc(\zbar) \geq \ffc(\y)$, then $\ffc(\x) \leq 2\omega$   
\end{lemma}

\begin{proof}
    By Lemma~\ref{lem:gamma<delta}, we assume that $|\Z(\x)| < \nrows{3}{\z}$, and thus $\bet  < \gam$.
    By Lemma~\ref{lem:any_z_bar}, $\ffc(\zbar) \leq \nrows{0}{\x} + (1 + \alp(\zbar)) \cdot \nrows{1}{\x} + \delt(\zbar) \cdot (1 - \gam) \cdot 2\nrows{2}{\x} + \bet  \cdot \nrows{2}{\x} + \gam  \cdot \nrows{2}{\x} + 2\nrows{3}{\x}$.
    By Lemma~\ref{lem:pivotset_zbar}, $|\pivotset| = (1-\alp(\zbar)) \cdot \nrows{1}{\x} + (1 - \delt(\zbar) \cdot (1 - \gam) \cdot 2\nrows{2}{\x}$.
    It follows that,
    
    \begin{align*}
        \ffc(\x) &\leq \ffc(\pivot) + |\pivotset| + 1\\
        &\leq \ffc(\zbar) + (1 - \alp(\zbar)) \cdot \nrows{1}{\x} + ( 1 -  \delt(\zbar) ) \cdot (1 - \gam ) \cdot 2\nrows{2}{\x} + 1\\
        &\leq \nrows{0}{\x} + 2\nrows{1}{\x}  + \bet  \cdot \nrows{2}{\x} + \gam  \cdot \nrows{2}{\x} + (1 - \gam ) \cdot 2\nrows{2}{\x} + 2\nrows{3}{\x} + 1\\
        &< \nrows{0}{\x} + 2\nrows{1}{\x}  + \bet  \cdot \nrows{2}{\x} + (1 - \bet ) \cdot \nrows{2}{\x} + \nrows{2}{\x} + 2\nrows{3}{\x} + 1\\
        &\leq 2\omega
    \end{align*}
\end{proof}

%%%%%%%%%%%%%%%%%%%%%%%%%%%%
% CASE 2.b.ii
%%%%%%%%%%%%%%%%%%%%%%%%%%%%
\paragraph*{Case (2.b.ii): $\ffc(\y) > \ffc(\zbar)$ }

In order to prove Theorem~\ref{thm:any} we still need to consider the tough kernel of the analysis.
That is, the case where $\ffc(\zbar) < \ffc(\y) < \ffc(\z)$ and $\bet  < \gam $.
For this case we abandon our usual mechanism slightly, where instead of focusing on a single pivot $\pivot$ and a single set $\pivotset$, we explore the option where two distinct pairs of a pivot and a set cannot both admit a large solution.

\medskip

\runtitle{Pick the pivots $\pivot$.}
Although we have reasoned earlier that picking interval $\y$ and interval $\z$ as the pivot is not sufficient to have an upper bound on $\ffc(\x)$ that is smaller than $3\omega$, we can deal with this tough case by expressing the bound obtained by using $\y$ or $\z$ as the pivot in terms of $|\Z(\x)|$.

\medskip


\runtitle{Bound the color of $\pivot$}
Let us first take a closer look at the color of interval $\y$.
Although Lemma~\ref{lem:integral_uvw} is technically still correct, knowing that $\ffc(\zbar) < \ffc(\y)$, we can be slightly more precise about which intersections are assigned a color greater than $\ffc(\y)$.
In order to be more precise, let us first make additional observations regarding the intersections of interval $\y$ with interval in $\rows{2}{\x}$ rows (also see Figure~\ref{fig:y_intersections}).

\begin{observation}
    \label{obs:any_y}
    Interval $\y$ intersects
    \begin{enumerate}[(a)]
        \item at most $1$ interval that is not in $\nb(\x)$ per row contained in $\rows{2}{\x}$, and
        \item at most $1$ interval that is in $\nb(\x)$ per row contained in $\rows{3}{\z}$,
        \item at most $1$ intervals that is in $\nb(\x)$ per row contained in $\Z(\x)$, and
        \item at most $1$ intervals that is in $\nb(\x)$ per row contained in $\rows{2}{\x} \setminus (\Z(\x) \cup \rows{3}{\z} )$.
    \end{enumerate}
\end{observation}

\begin{lemma}
    \label{lem:any_uvw}
    The color of interval $\y$,
    $\ffc(\y) \leq 2\nrows{0}{\x} + 2\nrows{1}{\x} + \alp(\y) \cdot \nrows{1}{\x} + \gam  \cdot \nrows{2}{\x} + \nrows{2}{\x} + \delt(\y) \cdot (1 - \gam ) \cdot \nrows{2}{\x} + \nrows{3}{\x}$. 
\end{lemma}

\begin{proof}
    By the Neighborhood bound, $\ffc(\y) \leq 1 + |\mathcal{N}(\y)|$.
    By Observation~\ref{obs:integral}, there are at most $ 2\nrows{0}{\x} + 2\nrows{1}{\x} + \nrows{3}{\x}$ intervals in $\rows{0}{\x}, \rows{1}{\x}$ and $\rows{3}{\x}$ that are in $\nb(\y)$.
    Together with the at most $\alp(\y) \cdot \nrows{1}{\x}$ intervals that were assigned a color below $\ffc(\y)$ which are either in $\twins(\x)$ or do not intersect $\y$ and are in $\nlmr(\x)$.
    
    By Observation~\ref{obs:any_y}, there are at most $\nrows{2}{\x} + \gam \cdot \nrows{2}{\x} + \bet \cdot \nrows{2}{\x} + (1 - \gam - \bet) \cdot \nrows{2}{\x}$ intervals that are in $\rows{2}{\x}$ and in $\nb(\y)$.
    By our assumption that $\ffc(\zbar) < \ffc(\y)$, the $\gam  \cdot \nrows{2}{\x}$ intervals in the set $\rows{3}{\z} \cap \nb(\x)$, must have a color smaller than $\ffc(\y)$.
    For the remaining $(1 - \gam) \cdot \nrows{2}{\x}$ intervals that are in $\nb(\x)$, by the definition of $\delt$,  $(1 - \delt(\y)) \cdot (1 - \gam) \cdot \nrows{2}{\x}$ are assigned a color larger than $\ffc(\z)$ and hence we do not consider them for this bound.
    
    Observe that by this bound we count the row that contains interval $\y$ itself.
    Since interval $\y$ cannot intersect any interval on this row, we should subtract at least 1 from this bound. 
    Thus,

    \begin{align*}
            \ffc(\y) &\leq |\nb(\y)| + 1\\
            &\leq 2\nrows{0}{\x} + 2\nrows{1}{\x} + \alp(\y) \cdot \nrows{1}{\x} + \gam  \cdot \nrows{2}{\x} \\&\hspace{1cm} + \nrows{2}{\x} + \delt(\y) \cdot (1 - \gam ) \cdot \nrows{2}{\x} + \nrows{3}{\x}
        \end{align*}    
\end{proof}

As for the bound on the color of interval $\z$, Lemma~\ref{lem:any_z} still suffices.

\medskip


\runtitle{Bound the sizes of $\pivotset$.}
Given that we use $\y$ as pivot $\pivot$, we bound the size of $\pivotset$.

\begin{lemma}
    \label{lem:pivotset_y_exact}
    When $\ffc(\y) \geq \ffc(\zbar)$, by selecting $\y$ as pivot $\pivot$, $|\pivotset| \leq (1 - \alp(\y)) \cdot \nrows{1}{\x} + (1 - \delt(\z)) \cdot (1 - \gam ) \cdot \nrows{2}{\x} + (1 - \gam ) \cdot \nrows{2}{\x}$
\end{lemma}

\begin{proof}
    First, by the definition of $\rows{0}{\x}$, $\x$ does not intersect any interval in $\rows{0}{\x}$.
    Thus, no interval in $\rows{0}{\x}$ can be in $\pivotset$.
    By the definition of $\alp$, only $(1 - \alp(\y) ) \cdot \nrows{1}{\x}$ intervals in $\rows{1}{\x}$ are assigned a color strictly larger than $\ffc(\y)$.
    Thus the only intervals in an $\rows{1}{\x}$ that are eligible for $\pivotset$ are those $(1 - \alp(\y) ) \cdot \nrows{1}{\x}$ intervals.
    Since $\ffc(\y) > \ffc(\zbar)$, no interval in an $\rows{3}{\z}$ row is assigned a color larger than $\ffc(\y)$.
    Thus, none of the $\gam \cdot 2\nrows{2}{\x}$ intervals in $\rows{3}{\z} \cap \nb(\x)$ can contribute to $\pivotset$.
    By the definition of $\delt$, only $(1 - \delt(\y) ) \cdot (1 - \gam) \cdot \nrows{2}{\x}$ intervals both in $\nb(\y)$ and $\rows{2}{\x}$ are assigned a color strictly larger than $\ffc(\y)$.
    Furthermore, all $(1 - \gam) \cdot \nrows{2}{\x}$ intervals which are in $\rows{2}{\x}$ but not in $\nb(\y)$ and not in $\rows{3}{\z}$ could potentially be assigned a color larger than $\ffc(\y)$ and therefore contribute to $\pivotset$.
    Finally, as all intervals in $\rows{3}{\x}$ are in $\lmr(\x)$, all intervals in $\rows{3}{\x}$ have a color below $\ffc(\y)$ and cannot contribute to $\pivotset$.
    Thus, $|\pivotset| \leq (1 - \alp(\y)) \cdot \nrows{1}{\x} + (1 - \delt(\z)) \cdot (1 - \gam ) \cdot \nrows{2}{\x} + (1 - \gam ) \cdot \nrows{2}{\x}$.
\end{proof}

As for the set $\pivotset$ when we pick interval $\z$ as pivot, Lemma~\ref{lem:pivotset_z} still suffices.

\medskip


\runtitle{Bound the size of $\ffc(\x)$.}

% The number of R_2(x) rows is bounded
As the bound we are going to prove in the proof of Theorem~\ref{thm:any} is based on the size of set $\Z(\x)$, which in turn is strictly smaller than the size of set $\rows{2}{\x}$, an upper bound on $\nrows{2}{\x}$ improves the bound on the color of $\x$ we would otherwise obtain.
This upper bound on the number of rows in $\rows{2}{\x}$ is obtained by a lower bound on the number of rows in $\rows{3}{\x}$.

\begin{lemma}
    \label{lem:r3_lb}
    For any interval $\x \in \orderedInput$, if $\nrows{3}{\x} < \frac{1}{3}\omega$, then $\ffc(\x) \leq \frac{7}{3} \omega - 2$.
\end{lemma}

\begin{proof}
    We show this using the Neighborhood bound.
    Observe that $\nrows{0}{\x} \geq 1$, since interval $\x$ does not intersect any intervals on its own row.
    Furthermore, note that the Neighborhood bound is maximized when we maximize the number of intervals that intersect interval $\x$.
    That is, we maximize the bound when we maximize $\nrows{3}{\x}$.
    Then, $\nrows{3}{\x} = \frac{1}{3}\omega - 1$.
    For the remaining rows, we maximize the bound by maximizing $\nrows{2}{\x}$, and hence $\nrows{2}{\x} = \frac{2}{3}\omega$.
    Then, $\nrows{1}{\x} = 0$, and we get the following upper bound on the color of interval $\x$,
    \begin{align*}
        \ffc(\x) &\leq \nrows{1}{\x} + 2\nrows{2}{\x} + 3\nrows{3}{\x} + 1 \\
            &\leq 0 + 2 \cdot \frac{2}{3} \omega + 3 \cdot (\frac{1}{3}\omega - 1) + 1\\
            &= \frac{7}{3} \omega - 2
    \end{align*}
\end{proof}

Now we have shown that $\nrows{3}{\x} \geq \frac{1}{3}\omega$, it simply follows that $\nrows{2}{\x} < \frac{2}{3}\omega$.

\begin{corollary}
    \label{cor:r2_bound}
    For any interval $\x \in \orderedInput$, if $\nrows{2}{\x} \geq \frac{2}{3} \omega$, then $\ffc(\x) \leq \frac{7}{3} \omega - 2$. 
\end{corollary}

% Intro to main thm
We have now shown everything we need to show in order to compute a bound on the color of interval $\x$.

\paragraph*{Proof of Theorem~\ref{thm:any}.}
\emph{For any interval $\x \in \orderedInput$, the color of $\x$ is at most $\ffc(\x) \leq \result$.}

\begin{proof}
    By Lemma~\ref{lem:open_interval_small_intersection},~\ref{lem:integral_open_and_closed_assumtion_not_hold}, \ref{lem:c_u>=c_z}, \ref{lem:gamma<delta} and \ref{lem:cy>=cu}, $\ffc(\x) \leq 2\omega$ if $\x$ is open, $\nrows{1}{\x} \leq \nrows{3}{\x} - 2$, $\ffc(\y) < \ffc(\z)$, $\bet  < \gam $ or $\ffc(\zbar) < \ffc(\y)$.
    Thus in the following we focus on closed intervals with $\nrows{1}{\x} \leq \nrows{3}{\x} - 2$, $\ffc(\zbar) < \ffc(\y) < \ffc(\z)$ and $\bet  < \gam $.

    Now we can compute the first bound on the color of interval $\x$, using the Pivot bound with interval $\y$ as pivot $\pivot$.
    According to Lemma~\ref{lem:any_uvw}, $\ffc(\y) \leq  2\nrows{0}{\x} + 2\nrows{1}{\x} + \alp(\y) \cdot \nrows{1}{\x} + \gam  \cdot \nrows{2}{\x} + \nrows{2}{\x} + \delt(\z) \cdot (1 - \gam ) \cdot \nrows{2}{\x} + \nrows{3}{\x}$ and, by taking $\y$ as pivot $\pivot$,  Lemma~\ref{lem:pivotset_y_exact}, $|\pivotset| \leq (1 - \alp(\y)) \cdot \nrows{1}{\x} + (1 - \delt(\z)) \cdot (1 - \gam ) \cdot \nrows{2}{\x} + (1 - \gam ) \cdot \nrows{2}{\x}$.
    Thus,

    \begin{align*}
        \ffc(\x) &\leq \ffc(\pivot) + |\pivotset| + 1\\
        &\leq \ffc(\y) + (1 - \alp(\y)) \cdot \nrows{1}{\x} + (1 - \delt(\z)) \cdot (1 - \gam ) \cdot \nrows{2}{\x} \\ &\hspace{1cm} + (1 - \gam ) \cdot \nrows{2}{\x} + 1 \\
        &\leq 2\nrows{0}{\x} + 3\nrows{1}{\x} + 2\nrows{2}{\x} + (1 - \gam ) \cdot \nrows{2}{\x} + \nrows{3}{\x} + 1 \\
        &\leq 2\nrows{0}{\x} + 2\nrows{1}{\x} + (1 - \gam ) \cdot \nrows{2}{\x} + 2\nrows{2}{\x} + 2\nrows{3}{\x} - 1\\
        &= 2\omega + (1 - \gam ) \cdot \nrows{2}{\x} - 1
    \end{align*}

    Next we move to the other bound.
    By taking $\z$ as pivot $\pivot$, according to Lemma~\ref{lem:any_z}, $\ffc(\z) \leq \omega + \alp(\z) \cdot \nrows{1}{\x} - \bet  \cdot \nrows{2}{\x} + \gam  \cdot \nrows{2}{\x} + \nrows{2}{\x} + \nrows{3}{\x}$, and Lemma~\ref{lem:pivotset_z}, $|\pivotset| = (1 - \alp(\z))\cdot\nrows{1}{\x}$.
    Then, it follows that,

    \begin{align*}
        \ffc(\x) &\leq \ffc(\pivot) + |\pivotset| + 1\\
        &\leq \ffc(\z) + (1 - \alp(\z)) \cdot \nrows{1}{\x} + 1 \\
        &\leq \omega + \nrows{1}{\x} - \bet  \cdot \nrows{2}{\x} + \gam  \cdot \nrows{2}{\x} + \nrows{2}{\x} + \nrows{3}{\x} + 1\\ 
        &\leq 2\omega + \gam  \cdot \nrows{2}{\x} - 1 % beta cannot be 0
    \end{align*}

    We have proven that simultaneously, the color of interval $\x$ is at most $\ffc(\x) \leq 2\omega + \gam  \cdot \nrows{2}{\x} - 1$ and at most $\ffc(\x) \leq 2\omega + (1 - \gam ) \cdot \nrows{2}{\x} - 1$.
    Then, it follows that for any value of $\gam $ the color of interval $\x$ is at most $\ffc(\x) \leq 2\omega + \frac{1}{2}\cdot \nrows{2}{\x} - 1$
    Then, it follows from Corollary~\ref{cor:r2_bound} that the number of $\rows{2}{\x}$ rows must be less than $\nrows{2}{\x}  < \frac{2}{3}\omega$, as otherwise the color of interval $\x$ is bounded by $\ffc(\x) \leq \frac{7}{3} \omega - 2$.
    Thus, the color of interval $\x$ is at most $\ffc(\x) < 2\omega + \frac{1}{3}\cdot \omega - 1 = \frac{7}{3}\omega - 1$.
    Which, by the integrality of $\ffc(\x)$, is at most $\ffc(\x) \leq \result$  
\end{proof}
}