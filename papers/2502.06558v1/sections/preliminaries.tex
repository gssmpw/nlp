%Terminologies

% input is sigma, ordered set
The input to our online coloring problem is an ordered set of open and closed unit-length intervals~$\orderedInput = \{I_1, I_2, \cdots\}$, where the ordering of the set of intervals is the order in which they are revealed to the online algorithm. 
That is, $I_j$ is revealed earlier than $I_k$ if $j<k$.
Each interval $I_j\in \orderedInput$ is either an \emph{open} unit-length interval $(r_j, r_j+1)$ or a \emph{closed} unit-length interval $[r_j, r_j+1]$ for some real number $r$.
In either case, we say $r_j$ and $r_j+1$ are the \emph{endpoints} of the interval. 
Note that the value of $r_j$ is unrelated to the order of $I_j$ in $\orderedInput$.
Namely, $r_j$ can be smaller than, equal to, or larger than $r_k$ when $j<k$.
We ignore the suffix of intervals when the reveal ordering is not necessarily to be emphasized.


% Identical interval sets
Two intervals are referred to as \emph{twins} when they are identical in terms of the exact location.
That is, for interval $I = [r, r+1]$, all intervals $I^\prime = [r, r+1]$ are its twins. 
Similarly, for interval $I = (r, r+1)$, all intervals $I^\prime = (r, r+1)$ are its twins.
However, an interval $(r,r+1)$ is not a twin of $[r, r+1]$, and vice versa.
When intervals $I$ and $I^\prime$ are twins, we write $I \equiv I^\prime$.
We further denote the set of twins of $I$ by $\twins(I)$.
Note that we define $I \not \in \twins(I)$ for later usage.


% formal definition of coloring?
A \emph{proper coloring} of a set of intervals is a function $c : \orderedInput \rightarrow \mathbb{N}$, such that $c(I) \neq c(I^\prime)$ if $I \cap I^\prime \neq \emptyset$.
The algorithm aims to find a proper coloring of the intervals using the least number of colors.


% definition neighborhood
We use terms from graph algorithms. 
The intervals in $\orderedInput$ form an \emph{interval graph}, where each $I\in \orderedInput$ is a vertex in the interval graph, and there is an edge between the vertices corresponding to $I$ and $I^\prime$ if and only if $I \cap I^\prime \neq \emptyset$.
We denote the \emph{neighborhood} of $I$ by $\nb(I)$, which is the set of all the intervals interesting with $I$.

As the coloring depends on intersecting intervals as well as the ordering of the input, we define the $\ell$-\emph{neighborhood} of an interval as its neighborhood at the moment when $I_\ell$ is revealed to the algorithm.
Formally, $\nb_\ell(I_j) = \{ I_k \mid I_j \cap I_k \neq \emptyset \text{ and } k < \ell \}$.\footnote{We only use $\nb_\ell(I_j)$ for $\ell \geq j$.}
Naturally, $\nb(I_j) = \nb_{|\orderedInput|}(I_j) \geq \nb_\ell(I_j)$ for any $\ell \in [j,|\orderedInput|]$.
Note that $\nb_j(I_j)$ is the set of all intervals that intersect with $I_j$ when $I_j$ is revealed.

Interval graphs are \emph{perfect graphs}~\cite{golumbic2004algorithmic}, and the chromatic number of the graph is equal to the maximum clique size $\omega$.
This clique corresponds to the maximum set of mutually intersecting intervals for the corresponding intervals.
We denote $\omega$ as the number of colors an optimal solution uses for the given input $\orderedInput$.

\hide{
\begin{definition}
    \label{def:neighbourhood}
    The $\ell$-\emph{neighborhood} of the interval $I_i \in \orderedInput$ is the set of intervals intersecting with $I_i$, which arrived before $I_\ell$ in the input.
    That is, \\
    $\nb_\ell(I_i) = \{ I_k \mid I_i \cap I_k \neq \emptyset \text{ and } k < \ell \}$.
\end{definition}
}

\medskip

% First-Fit
\runtitle{\ff algorithm.}
The \ff algorithm assigns each interval the least available color, that is, using $\ffc(I_j)$ to denote the color assigned to $I_j$ by \ff,
$\ffc(I_j) \leftarrow \min \{ \mathbb{N} \setminus \{ \ffc(I) \mid I \in N_j(I_j) \} \}$.
%
This greedy strategy provides an upper bound for the color of an interval.
That is, the color of an interval can never be larger than $1$ color above the number of intervals it intersects.
More specifically, the color of the interval will be at most $1$ greater than the size of the neighborhood when it is revealed. 
Formally,
\hide{This gives us the following upper bound for a color assigned by the \ff{} algorithm.}

\begin{lemma}\emph{\textbf{(Neighborhood bound)}}
    \label{lem:prop_ff}
    For any interval $I_j \in \orderedInput$, its color assigned by \ff is at most $1 + |N_j(I_j)|$.
\end{lemma}
\fullversion{
\begin{proof}
    Assume aiming towards contradiction that $\ffc(I_j) \geq 2 + |N_j(I_j)|$.
    Then, according to the pigeonhole principle, there must exist at least one color $1 \leq c < \ffc(I_j)$ that is not assigned to any of the intervals in $N_j(I_j)$.
    Then $\ffc(I_j) = \min \{ \mathbb{N} \setminus \{ \ffc(I) \mid I \in N_j(I_j) \} \} \leq c$, and it leads to a contradiction.
\end{proof}
}

\runtitle{Pivot interval and ideas of further bonding the color.}
In this work, we further generalize Lemma~\ref{lem:prop_ff} by considering a \emph{pivot} interval in the neighborhood of the interval $I$ and use the pivot interval to bound the color $\ffc(I)$ more carefully.
Informally, given any interval~$I$ and a pivot interval $\pivot$ that can be any interval intersecting with $I$, $\ffc(I)$ is no more than~$1$ plus the number of intervals in $I$'s neighborhood with color greater than the color of $\pivot$. %\todo{(It is still a bit like something revealed to you in a dream...)}
More formally,

\begin{lemma}\emph{\textbf{(Pivot bound)}}
    \label{lem:prop_ff_equiv}
    For any interval $I_j \in \orderedInput$, let $\pivot \in N(I_j)$ be any interval in $\nb(I_j)$ and let $\pivotset \subset \nb(I_j)$ be the set of intervals in the neighborhood of $I_j$ such that for all $I^\prime\in \pivotset$, $\texttt{FF}(I^\prime)>\texttt{FF}(\pivot)$.
    Then, $\texttt{FF}(I_j) \leq \texttt{FF}(\pivot) + |\pivotset| + 1$.
\end{lemma}

\begin{proof}
    Assume aiming toward contradiction that $\ffc(I_j) \geq 2 + \ffc(\pivot) + |\pivotset|$.
    Then, according to the pigeonhole principle, there must exist at least one color $\ffc(\pivot) < c < \ffc(I_j)$ that is not assigned to any of the intervals in $\pivotset$.
    Then $\ffc(I_j) = \min \{ \mathbb{N} \setminus \{ \ffc(I) \mid I \in N_j(I_j) \} \} \leq c$, and it leads to a contradiction.
\end{proof}

Note that the Neighborhood bound is a special case of the Pivot bound by selecting an empty set as its pivot. In this extreme case, the pivot is colored by $0$ by \ff, and $\pivotset \subseteq \nb(I)$.

Intuitively, for a smaller upper bound of $\ffc(I_j)$, by the Pivot bound, we can choose a~pivot~$\pivot$ with a small color $\ffc(\pivot)$ or choose a~$\pivot$ with a small $|\pivotset|$. 
However, we have a~two-fold challenge. 
First, it is not trivial to find the best choice between minimizing $\ffc(\pivot)$ or minimizing $|\pivotset|$. 
Second, even when a good pivot $\pivot$ is given, it does not necessarily have non-trivial upper bounds of $\ffc(\pivot)$ and $|\pivotset|$.
To cope with the difficulties, we construct a mechanism to balance the two choices of minimizing $\ffc(\pivot)$ or $|\pivotset|$ and different manners to bound $\ffc(\pivot)$ and $|\pivotset|$.
\hide{Ideally, a good selection of pivot interval $\pivot$ with a small $|\pivotset|$ provides a better bound of the color assigned by \ff.
However, if we try to find a pivot $\pivot$ with very small color $\ffc(\pivot)$, it might be the case that $|\pivotset|$ is very large, and vice versa.}


% Visualization of rows
\begin{figure}[t]
    \centering
    \includegraphics[width=0.35\linewidth]{pictures/Brick_Rock_example.pdf}
    \caption{Rows denote the color assigned by \opt, where actual colors denote the color assigned by \ff, where the order is based on the index.}
    \label{fig:example_of_a_row}
\end{figure}

\medskip

% Rows vs colors
\runtitle{Optimal coloring and the image of walls of rocks and bricks.}
In the analysis of \ff algorithm, we compare the solution of the \ff algorithm to an \emph{optimal offline algorithm} \opt that knows the complete input in advance.

We first imagine the intervals in $\orderedInput$ as rocks (open intervals) and bricks (closed intervals) with fixed horizontal positions. 
That is, the rocks and bricks can be shifted vertically but not horizontally. 
We imagine that the optimal solution orients these rocks and bricks into a wall row by row using the fewest rows (without violating the fixed horizontal position constraints).
More specifically, the intervals on the $i$-th row (from the bottom of the wall) are the intervals colored by $i$ according to the optimal solution \opt.
Since \opt uses precisely~$\omega$ colors, the number of rows is exactly $\omega$.
Furthermore, it naturally follows that no two intervals drawn in the same row will intersect with each other, as otherwise \opt admits no proper coloring.


To avoid confusion between the color assigned by the optimal offline algorithm and the color assigned by the \ff algorithm, from now on, we distinguish the color of an interval by \ff or by \opt by its \emph{color} or its \emph{row}. 
More specifically, the color of an interval is the color assigned by \ff algorithm, and the row refers to the color assigned by \opt.
An example of the distinction between colors and rows is shown in Figure~\ref{fig:example_of_a_row}.

\medskip

\runtitle{Relate the intervals and their structure in the wall.}
% definition of R_i etcetera
With the image of rock-and-brick wall in mind, given any interval (which may be a rock or a brick), we partition the rows according to their alignment with the interval $\x$.

\begin{definition}
    For any interval $\x \in \orderedInput$, let $\rows{i}{\x}$ be the set of rows in the optimal solution \opt{}$(\orderedInput)$ that contain $i$ intervals intersecting $\x$.
    And let $\nrows{\texttt{i}}{I}$ denote $|\rows{i}{\x}|$. 
    \hide{, i.e., the number of $\rows{i}{\x}$ rows in the optimal solution \opt{}$(\orderedInput)$ for interval $\x$.}
\end{definition}

Strictly speaking, the set $\rows{i}{\x}$ contains rows instead of containing intervals.
However, by slight abuse of notation, we say that interval $I^\prime \in \rows{i}{\x}$ if $I^\prime$ is in $\nb(\x)$ and colored by one of the colors in  $R_i(\x)$ by the optimal solution.
In other words, $I^\prime \in \rows{i}{\x}$ if $I^\prime$ is an interval overlapping with $\x$ and colored by one of the colors in $\rows{i}{\x}$ by \opt.

% Closed intervals can intersect at most 3 per row, open at most 2 per row.
Since the intervals are either open unit intervals or closed-unit intervals,
it is clear that not for all $i$ the set $\rows{i}{\x}$ is non-empty.
Formally,
\hide{
Given either an open or a closed unit-length interval $\x$, a row can only be contained in the set $R_i(\x)$ if it is possible that $\x$ intersects with $i$ intervals in that row.
Therefore we shall prove upper bounds on the number of intersections an interval can have per row.
}
\begin{lemma}\label{lem:ris}
    Consider any $I\in \orderedInput$.
    \begin{enumerate}[(a)]
        \item \label{lem:r0_not_empty} $\rows{0}{I}$ contains at least the row where $I$ is, and $\nrows{\texttt{0}}{I} \geq 1$.
        \item \label{lem:closed_intersects_3}
        If $I$ is closed, $\nrows{\texttt{i}}{I} = 0$ for all $\texttt{i}\geq 4$.
        \item \label{lem:open_intersects_2}
        If $I$ is open, $\nrows{\texttt{i}}{I} = 0$ for all $\texttt{i}\geq 3$.
        \item \label{Obs:sumrrr}
        Since \opt uses $\omega$ rows, $\nrows{\texttt{0}}{I} + \nrows{\texttt{1}}{I} + \nrows{\texttt{2}}{I} + \nrows{\texttt{3}}{I} = \omega$.
    \end{enumerate}
\end{lemma}

\hide{
\begin{lemma}
    \label{lem:closed_intersects_3}
    A closed unit-length interval intersects at most $3$ intervals per row.
\end{lemma}
}

\fullversion{
\begin{proof}

\runtitle{(\ref{lem:r0_not_empty})}
    By definition, the rows $\rows{i}{\x}$ correspond to the rows of an optimal solution \opt.
    Therefore, as \opt admits a proper coloring, there is no interval on the same row where $\x$ is contained and intersects $\x$.
    Hence, on the row that contains $\x$, no interval intersects $\x$. Thus, this row is contained in $\rows{0}{\x}$. 
    It follows that $\nrows{0}{\x} \geq 1$.

\runtitle{(\ref{lem:closed_intersects_3})}
    Given a closed unit-length interval $\x = [a, a + 1]$, consider the following 3 points; $a, a + \frac{1}{2}$ and $a + 1$.
    Observe that the distance between these points is strictly smaller than the length of any (open or closed) unit-length interval, and that the endpoints of $\x$ are within these points.
    Then it follows that any interval that intersects interval $\x$ must intersect at least one of the points $a, a + \frac{1}{2}$, or $a + 1$.
    Thus since intervals which are drawn on the same row cannot intersect each other, at most $3$ intervals per row can intersect interval $\x$.

\runtitle{(\ref{lem:open_intersects_2})}
    Given an open unit-length interval $\x = (a, a + 1)$ and an infinitesimal number $\varepsilon$, consider the following 2 points; $a + \varepsilon$ and $a + 1 - \varepsilon$.
    Observe that the distance between these points is strictly smaller than the length of any (open or closed) unit-length interval, and that there exist no points between $a$ and $a + \varepsilon$ and similarly there exist no points between $a + 1$ and $a + 1 - \varepsilon$.
    Then it follows that any interval that intersects interval $\x$ must intersect at least one of the points $a + \varepsilon$ or $a + 1 - \varepsilon$.
    Thus, since intervals that are drawn on the same row cannot intersect each other, at most $2$ intervals per row can intersect interval $\x$.

\runtitle{(\ref{Obs:sumrrr})}
    By definition, the rows in $\rows{i}{\x}$ correspond to the rows of an optimal solution \opt, of which there are $\omega$ such rows.
    We know from (\ref{lem:closed_intersects_3}) that for all $i \geq 4$, $\nrows{i}{\x} = 0$.
    Then, it follows that $\nrows{\texttt{0}}{I} + \nrows{\texttt{1}}{I} + \nrows{\texttt{2}}{I} + \nrows{\texttt{3}}{I} = \omega$.
    Note that this is for any interval $\x$ an exact equality because of the inclusion of $\nrows{0}{\x}$.
\end{proof}
}



% neighbourhood is upper bounded by i r_i
By the definition of $\rows{i}{\x}$, the number of intervals that intersect with $\x$ can be upper bounded in terms of $\nrows{i}{\x}$. More specifically,

\begin{observation}
    \label{obs:r_i=N}
    For any interval $\x \in \orderedInput$, $|\nb_\x(\x)| \leq |\nb(\x)| \leq \nrows{\texttt{1}}{\x} + 2\cdot \nrows{\texttt{2}}{\x} + 3\cdot \nrows{\texttt{3}}{\x}$.
\end{observation}

Recall from Lemma~\ref{lem:ris} (\ref{lem:closed_intersects_3}) and (\ref{lem:open_intersects_2}) that only closed intervals $I$ can have non-empty $\rows{3}{I}$. 
Further, there is only a single way for a closed interval to intersect three intervals in the same row. 
Therefore, all rows in $\rows{3}{I}$ contain three sets of intervals, where the intervals in each set are twins.
Formally, given any interval $I = [r, r+1]$ where $r$ is some real number, we partition the intervals in $\nb(I)$ into three sets:
\begin{itemize}
    \item $\twins(I)$: Intervals $I^\prime \equiv [r, r+1]$.
    
    \item $\lmr(I)$: Intervals that are twins of intervals in $\rows{3}{I}$. More specifically, $\lmr(I)$ consists of the intervals $I^\prime \equiv [r-1, r]$, $I^\prime \equiv (r, r+1)$, and $I^\prime \equiv [r+1, r+2]$.
    
    \item $\nlmr(I)$: All other intervals in $\nb(I)$.
\end{itemize}
