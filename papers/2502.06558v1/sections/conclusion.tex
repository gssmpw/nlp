In this work, we develop a sophisticated counting method based on the Pivot bound and show that \ff{} uses at most $2\omega$ colors in the case that all open and closed unit-length intervals have integral endpoints, which matches the lower bound.
We also show that \ff{} uses at most $\result$ colors when the input open or closed unit-length intervals have arbitrary endpoints.
It remains open to find a tight bound for this problem.

In an attempt to improve the current upper bound for the number of colors used by the \ff{} algorithm for the general case, it may be of interest to investigate the number of colors used by the \ff algorithm for the so-called bounded length intervals as introduced by Chybowska-Sok\'{o}l et al.~\cite{DBLP:journals/ejc/ChybowskaSokol24}.
The technique and the results on open/closed unit-length intervals developed in this work serve as a first step towards investigating \ff's performance via gradually relaxing the lengths bound.
%These authors already provide upper and lower bounds for specific values of the bounded length $\sigma$. However, it may be interesting to find a full characterization of the number of colors used by the \ff{} algorithm.