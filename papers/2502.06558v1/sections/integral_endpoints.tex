% -Basic idea-
% Key observation that makes integral case easy
As a starter, we consider a special case where all intervals have integral endpoints. 
More formally, for any interval $\x \in \orderedInput$, $I = (i, i+1)$ for some integer $i$ if $I$ is open, and $I = [i, i+1]$ for some integer $i$ if $I$ is closed.
In this section, we show that any $\x \in \orderedInput$ has color $\ffc(\x)\leq 2\omega$, where $\omega$ is the number of colors used by the optimal solution.
Note that in the following, we focus on the closed intervals $\x \in \orderedInput$ with $\nrows{3}{\x} \geq \nrows{1}{\x}+2$ (by Lemma~\ref{lem:open_interval_small_intersection} and Lemma~\ref{lem:integral_open_and_closed_assumtion_not_hold}).

For such a set of integral-endpoints intervals, it is an important property that $\nlmr(\x) = \emptyset$ for any $\x \in \orderedInput$.
That is, the intervals in $\nb(\x)$ are either in $\twins(\x)$ (twins of $\x$) or in $\lmr(\x)$ (twins of the intervals in $\rows{3}{\x}$).
An example is shown in Figure~\ref{fig:column_structure_in_rows}.

\paragraph*{Pick a pivot $\pivot$}

For any closed intervals $\x \in \orderedInput$ with $\nrows{3}{\x} \geq \nrows{1}{\x}+2$, we first identify the interval in $\lmr(\x)$ with the highest color by \ff. 

\begin{definition}\emph{\textbf{(Dominating interval $\y$ in $\lmr(\x)$)}}
    \label{def:y}
    Let interval $\y \in \lmr(\x)$ be the highest colored interval. That is, there is no $I^\prime \in \lmr(\x)$ such that $\ffc(I^\prime) > \ffc(\y)$.
\end{definition}

Note that $\y$ must exist as $|\lmr(\x)| \geq \nrows{3}{\x} \geq \nrows{1}{\x} + 2 \geq 2$.

In this analysis, we want to bound $\ffc(\x)$ by using $\y$ as a pivot $\pivot$ and applying the Pivot bound (Lemma~\ref{lem:prop_ff_equiv}).


\paragraph*{Bound the size of $\pivotset$}
Given that we use $\y$ as the pivot $\pivot$, we bound the size of $\pivotset$.
Recall that $\pivotset$ is the set of intervals in $\nb(\x)$ which are assigned higher color than $\ffc(\pivot)$.
% How interval x can have a higher color
Since all intervals in $\orderedInput$ have integral endpoints, any interval in $\nb(\x)$ is either in $\lmr(\x)$ or in $\twins(\x)$.
Therefore, by the definition of $\y$, only intervals in $\twins(\x)$ are possible to contribute to $\pivotset$.
% Not too many intervals identical to x
Moreover, since we focus on $\x$ that is closed, all intervals in $\twins(\x)$ must be in the rows in $\rows{1}{\x}$. 
We use the following definition to represent the fraction of rows in $\rows{1}{\x}$ that cannot contribute to $\pivotset$.



\begin{definition}
    \label{def:alpha}
    Given any $\hat{I} \in \rows{1}{\x}$, $\alp(\hat{I}) \in [0,1]$ denotes the fraction of rows $\mathcal{R} \in \rows{1}{\x}$ where the interval $I^\prime$ in $\mathcal{R} \cap \nb(\x)$ satisfies \emph{1)} $I^\prime \notin \lmr(\x)$ and \emph{2)} $\ffc(I^\prime) \leq \ffc(\hat{I})$.
    %such that for all $I^\prime \in \rows{1}{\x}$,\footnote{Recall that we say that an interval $I^\prime \in \rows{i}{\x}$ if $I^\prime$ is overlapping with $\x$ and colored by one of the colors in $\nrows{i}{\x}$ by \opt.} $I^\prime \not \in \lmr(\x)$ and $\ffc(I^\prime) \leq \ffc(\hat{I})$.
\end{definition}

Note that we make Definition~\ref{def:alpha} more general for later usage. 
For the integral-endpoints case, it is sufficient to set $\hat{I}$ as $\y$, and $\alp(\y) \cdot \nrows{1}{\x}$ is the number of intervals in $\twins(\x)$ that are not in $\pivotset$.

\begin{lemma}
    \label{lem:integral_S}
    If all intervals in $\orderedInput$ have integral endpoints, by selecting $\y$ as a pivot~$\pivot$, $|\pivotset| = (1-\alp(\y)) \cdot \nrows{1}{\x}$.
\end{lemma}

\begin{proof}
    First, by the definition of $\rows{0}{\x}$, $\x$ does not intersect any interval in $\rows{0}{\x}$. Thus, no interval in $\rows{0}{\x}$ can be in $\pivotset$.
    Next, as all intervals have integral endpoints, all intervals in $\rows{2}{\x}$, $\rows{3}{\x}$, or $\rows{1}{\x} \setminus \twins(\x)$ are in the sets $\lmr(\x)$.
    Thus, all these intervals are assigned a color below $\ffc(\y)$ and cannot contribute to $\pivotset$.
    The remaining intervals are those in $\twins(\x)$.
    By definition, $\alp(\y) \cdot \nrows{1}{\x}$ of these intervals have a color at most $\ffc(\y)$.
    Hence, the $(1 - \alp(\y)) \cdot \nrows{1}{\x}$ intervals are the only intervals intersecting $\x$ that have a color greater than $\ffc(\y)$.
    That is, $|\pivotset| = (1 - \alp(\y)) \cdot \nrows{1}{\x}$.
\end{proof}


\paragraph*{Bound the color $\ffc(\pivot)$}

We bound the color of $\y$ using the Neighborhood bound (Lemma~\ref{lem:prop_ff}).
In general, apart from intervals in $\twins(\x)$, the number of intervals any interval in $\lmr(\x)$ can intersect is shown in the following observation (also see Figure~\ref{fig:intersections_with_v}):

\begin{observation}
    \label{obs:integral}
    Any interval in the set $\lmr(\x)$ intersects
    \begin{enumerate}[(a)]
        \item at most $2$ intervals per row in $\rows{0}{\x}$,
        \item at most $2$ intervals which are not twins with $\x$ per row in $\rows{1}{\x}$,
        \item at most $2$ intervals per row in $\rows{2}{\x}$, and
        \item exactly $1$ interval per row in $\rows{3}{\x}$.
    \end{enumerate}
\end{observation}

% Figures
\begin{figure}[t]
\begin{minipage}{0.4\textwidth}
\centering
\includegraphics[scale=0.4]{pictures/Brick_Rock_integral_rows.pdf}
\caption{
All possible positions of intervals intersecting $\x$ per type of row.
The intervals in $\lmr(\x)$ are drawn in orange, green and blue, the intervals in $\twins(\x)$ in purple.
}
\label{fig:column_structure_in_rows}
\end{minipage}%
\hspace{0.2\textwidth}%
\begin{minipage}{0.4\textwidth}
\centering
\includegraphics[scale=0.4]{pictures/Brick_Rock_integral_intersections.pdf}
\caption{
All possible intersections with interval $\y$ per type of row.
The intervals from $\lmr(\x)$ that intersect with $\y$ are drawn in orange, the intervals in $\twins(\x)$ in purple and the remaining intervals intersecting $\y$ are drawn in yellow.}
\label{fig:intersections_with_v}
\end{minipage}
\end{figure}

% -Start of analysis-
According to the definition of $\y$, Observation~\ref{obs:integral} applies to $\y$. 


\begin{lemma}
    \label{lem:integral_uvw}
    The color of interval $\y$, $\ffc(\y) \leq 2 \omega + \alp(\y) \cdot \nrows{1}{\x} - \nrows{3}{\x}$.
\end{lemma}


\begin{proof}
    By the Neighborhood bound, $\ffc(\y) \leq 1+|N(\y)|$.
    By Observation~\ref{obs:integral}, there are at most $2\nrows{0}{\x} + 2\nrows{1}{\x} + 2\nrows{2}{\x} + \nrows{3}{\x}$ intervals in $\lmr(\x)$ that are in $\nb(\y)$. 
    Together with the $\alp(\y) \cdot \nrows{1}{\x}$ intervals that ``occupied'' the smaller colors from the choice of $\y$ at the moment when $\y$ arrives, $|\nb_{\y}(\y)| \leq 2\nrows{0}{\x} + 2\nrows{1}{\x} + \alp(\y) \cdot \nrows{1}{\x} + 2\nrows{2}{\x} + \nrows{3}{\x}$.
    Observe that by this bound, we count the row that contains interval $\y$ itself. 
    Since $\y$ cannot intersect any interval on this row, we should subtract at least $1$ from this bound.
    Thus,
    \begin{align*}
        \ffc(\y) &\leq 1 + 2\nrows{0}{\x} + 2\nrows{1}{\x} + \alp(\y)\cdot \nrows{1}{\x} + 2\nrows{2}{\x} + \nrows{3}{\x} - 1\\
        &= 2 \omega + \alp(\y)\cdot \nrows{1}{\x} - \nrows{3}{\x}
    \end{align*}
\end{proof}

Note that due to the abstraction of $\alp(\y)$, this proof does not rely on the property of integral endpoints.
Therefore, Observation~\ref{obs:integral} also holds for the general case where the intervals in $\orderedInput$ have arbitrary endpoints.

\begin{corollary}
    \label{cor:integral_LMR}
    Given instance $\orderedInput$ with arbitrary endpoints, the color of interval $\y$ is at most $2\omega + \alp(\y) \cdot \nrows{1}{\x} - \nrows{3}{\x}$.
\end{corollary}

\paragraph*{Proof of Theorem~\ref{thm:integral}}
Now, we are ready to prove Theorem~\ref{thm:integral} that for any $\orderedInput$ with open or closed unit-length intervals that have integral endpoints, $\ffc(\x) \leq 2\omega$ for all $\x \in \orderedInput$, where $\omega$ is the optimal color needed for properly coloring $\orderedInput$.

\begin{proof}
    By Lemma~\ref{lem:integral_open_and_closed_assumtion_not_hold}, we assume that $\nrows{1}{\x} \leq \nrows{3}{\x}-2$. 
    Moreover, by Lemma~\ref{lem:open_interval_small_intersection}, we focus on closed $\x$.
    According to Corollary~\ref{cor:y_exists}, we assume that there exists a dominating interval $\y$ in $\lmr(\x)$ that has the highest color by \ff.
    By Lemma~\ref{lem:integral_uvw}, $\ffc(\y) \leq 2 \omega + \alp(\y) \cdot \nrows{1}{\x} - \nrows{3}{\x}$.

    We now use the Pivot bound to bound the color of interval $\x$ from above, using $\y$ as the role of $\pivot$.
    By Lemma~\ref{lem:integral_S}, $|\pivotset| = (1 - \alp(\y)) \cdot \nrows{1}{\x}$.
    \shortversion{Altogether, we have $\ffc(\x) \leq \ffc(\pivot) + |\pivotset| + 1 \leq 2 \omega - 1$}
    \fullversion{
    It follows that
    \begin{align*}
        \ffc(\x) &\leq \ffc(\pivot) + |\pivotset| + 1 \\
        &= \ffc(\y) + (1 - \alp(\y)) \cdot \nrows{1}{\x} + 1\\
        &\leq 2\omega + \alp(\y)\cdot \nrows{1}{\x} - \nrows{3}{\x} + (1 - \alp(\y)) \cdot \nrows{1}{\x} + 1\\
        &\leq 2\omega + \nrows{1}{\x} - \nrows{3}{\x} + 1 \\
        &\leq 2 \omega - 1
    \end{align*}
    }
\end{proof}