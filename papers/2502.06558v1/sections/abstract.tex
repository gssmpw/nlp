Online interval coloring is a fundamental problem in graph algorithms and scheduling. 
Although the optimal online algorithm for coloring arbitrary-length intervals is known as $3$-competitive (Kierstead and Trotter, 1981), the interest in coloring bounded-length intervals arose recently (Chybowska-Sok\'{o}\l{} et al., 2024). 
On the other hand, people are also interested in the FirstFit algorithm's performance because of its elegance.
The competitive ratio of FirstFit on unit-length intervals is exact $2-\frac{1}{\omega}$, where $\omega$ is the optimal number of colors needed (Epstein and Levy, 2005).
However, for arbitrary-length intervals, the competitive ratio is only known to be between $5$ (Kierstead et al., 2005) and~$8$ (Narayanaswamy and Subhash Babu, 2008). 
It has been open for a long time what the actual competitive ratio of FirstFit is for arbitrary-length intervals.

In this paper, we study the performance of the FirstFit algorithm for the online unit-length intervals coloring problem where the intervals can be either open or closed, which serves a further investigation towards the actual performance of FirstFit.
We develop a sophisticated counting method by generalizing the classic neighborhood bound, which limits the color FirstFit can assign an interval by counting the potential intersections. 
In the generalization, we show that for any interval, there is a critical interval intersecting it that can help reduce the overestimation of the number of intersections, and it further helps bound the color an interval can be assigned. 
The technical challenge then falls on identifying these critical intervals that guarantee the effectiveness of counting. 
Using this new mechanism for bounding the color that FirstFit can assign an interval, we provide a tight analysis of $2\omega$ colors when all intervals have integral endpoints and an upper bound of $\result$ colors for the general case, where $\omega$ is the optimal number of colors needed for the input set of intervals.
